% !TEX encoding = UTF-8 Unicode
% !TEX TS-program = xelatex

\documentclass[french]{report}

%\usepackage[utf8]{inputenc}
%\usepackage[T1]{fontenc}
\usepackage{babel}


\newif\ifcomment
%\commenttrue # Show comments

\usepackage{physics}
\usepackage{amssymb}


\usepackage{amsthm}
% \usepackage{thmtools}
\usepackage{mathtools}
\usepackage{amsfonts}

\usepackage{color}

\usepackage{tikz}

\usepackage{geometry}
\geometry{a5paper, margin=0.1in, right=1cm}

\usepackage{dsfont}

\usepackage{graphicx}
\graphicspath{ {images/} }

\usepackage{faktor}

\usepackage{IEEEtrantools}
\usepackage{enumerate}   
\usepackage[PostScript=dvips]{"/Users/aware/Documents/Courses/diagrams"}


\newtheorem{theorem}{Théorème}[section]
\renewcommand{\thetheorem}{\arabic{theorem}}
\newtheorem{lemme}{Lemme}[section]
\renewcommand{\thelemme}{\arabic{lemme}}
\newtheorem{proposition}{Proposition}[section]
\renewcommand{\theproposition}{\arabic{proposition}}
\newtheorem{notations}{Notations}[section]
\newtheorem{problem}{Problème}[section]
\newtheorem{corollary}{Corollaire}[theorem]
\renewcommand{\thecorollary}{\arabic{corollary}}
\newtheorem{property}{Propriété}[section]
\newtheorem{objective}{Objectif}[section]

\theoremstyle{definition}
\newtheorem{definition}{Définition}[section]
\renewcommand{\thedefinition}{\arabic{definition}}
\newtheorem{exercise}{Exercice}[chapter]
\renewcommand{\theexercise}{\arabic{exercise}}
\newtheorem{example}{Exemple}[chapter]
\renewcommand{\theexample}{\arabic{example}}
\newtheorem*{solution}{Solution}
\newtheorem*{application}{Application}
\newtheorem*{notation}{Notation}
\newtheorem*{vocabulary}{Vocabulaire}
\newtheorem*{properties}{Propriétés}



\theoremstyle{remark}
\newtheorem*{remark}{Remarque}
\newtheorem*{rappel}{Rappel}


\usepackage{etoolbox}
\AtBeginEnvironment{exercise}{\small}
\AtBeginEnvironment{example}{\small}

\usepackage{cases}
\usepackage[red]{mypack}

\usepackage[framemethod=TikZ]{mdframed}

\definecolor{bg}{rgb}{0.4,0.25,0.95}
\definecolor{pagebg}{rgb}{0,0,0.5}
\surroundwithmdframed[
   topline=false,
   rightline=false,
   bottomline=false,
   leftmargin=\parindent,
   skipabove=8pt,
   skipbelow=8pt,
   linecolor=blue,
   innerbottommargin=10pt,
   % backgroundcolor=bg,font=\color{orange}\sffamily, fontcolor=white
]{definition}

\usepackage{empheq}
\usepackage[most]{tcolorbox}

\newtcbox{\mymath}[1][]{%
    nobeforeafter, math upper, tcbox raise base,
    enhanced, colframe=blue!30!black,
    colback=red!10, boxrule=1pt,
    #1}

\usepackage{unixode}


\DeclareMathOperator{\ord}{ord}
\DeclareMathOperator{\orb}{orb}
\DeclareMathOperator{\stab}{stab}
\DeclareMathOperator{\Stab}{stab}
\DeclareMathOperator{\ppcm}{ppcm}
\DeclareMathOperator{\conj}{Conj}
\DeclareMathOperator{\End}{End}
\DeclareMathOperator{\rot}{rot}
\DeclareMathOperator{\trs}{trace}
\DeclareMathOperator{\Ind}{Ind}
\DeclareMathOperator{\mat}{Mat}
\DeclareMathOperator{\id}{Id}
\DeclareMathOperator{\vect}{vect}
\DeclareMathOperator{\img}{img}
\DeclareMathOperator{\cov}{Cov}
\DeclareMathOperator{\dist}{dist}
\DeclareMathOperator{\irr}{Irr}
\DeclareMathOperator{\image}{Im}
\DeclareMathOperator{\pd}{\partial}
\DeclareMathOperator{\epi}{epi}
\DeclareMathOperator{\Argmin}{Argmin}
\DeclareMathOperator{\dom}{dom}
\DeclareMathOperator{\proj}{proj}
\DeclareMathOperator{\ctg}{ctg}
\DeclareMathOperator{\supp}{supp}
\DeclareMathOperator{\argmin}{argmin}
\DeclareMathOperator{\mult}{mult}
\DeclareMathOperator{\ch}{ch}
\DeclareMathOperator{\sh}{sh}
\DeclareMathOperator{\rang}{rang}
\DeclareMathOperator{\diam}{diam}
\DeclareMathOperator{\Epigraphe}{Epigraphe}




\usepackage{xcolor}
\everymath{\color{blue}}
%\everymath{\color[rgb]{0,1,1}}
%\pagecolor[rgb]{0,0,0.5}


\newcommand*{\pdtest}[3][]{\ensuremath{\frac{\partial^{#1} #2}{\partial #3}}}

\newcommand*{\deffunc}[6][]{\ensuremath{
\begin{array}{rcl}
#2 : #3 &\rightarrow& #4\\
#5 &\mapsto& #6
\end{array}
}}

\newcommand{\eqcolon}{\mathrel{\resizebox{\widthof{$\mathord{=}$}}{\height}{ $\!\!=\!\!\resizebox{1.2\width}{0.8\height}{\raisebox{0.23ex}{$\mathop{:}$}}\!\!$ }}}
\newcommand{\coloneq}{\mathrel{\resizebox{\widthof{$\mathord{=}$}}{\height}{ $\!\!\resizebox{1.2\width}{0.8\height}{\raisebox{0.23ex}{$\mathop{:}$}}\!\!=\!\!$ }}}
\newcommand{\eqcolonl}{\ensuremath{\mathrel{=\!\!\mathop{:}}}}
\newcommand{\coloneql}{\ensuremath{\mathrel{\mathop{:} \!\! =}}}
\newcommand{\vc}[1]{% inline column vector
  \left(\begin{smallmatrix}#1\end{smallmatrix}\right)%
}
\newcommand{\vr}[1]{% inline row vector
  \begin{smallmatrix}(\,#1\,)\end{smallmatrix}%
}
\makeatletter
\newcommand*{\defeq}{\ =\mathrel{\rlap{%
                     \raisebox{0.3ex}{$\m@th\cdot$}}%
                     \raisebox{-0.3ex}{$\m@th\cdot$}}%
                     }
\makeatother

\newcommand{\mathcircle}[1]{% inline row vector
 \overset{\circ}{#1}
}
\newcommand{\ulim}{% low limit
 \underline{\lim}
}
\newcommand{\ssi}{% iff
\iff
}
\newcommand{\ps}[2]{
\expval{#1 | #2}
}
\newcommand{\df}[1]{
\mqty{#1}
}
\newcommand{\n}[1]{
\norm{#1}
}
\newcommand{\sys}[1]{
\left\{\smqty{#1}\right.
}


\newcommand{\eqdef}{\ensuremath{\overset{\text{def}}=}}


\def\Circlearrowright{\ensuremath{%
  \rotatebox[origin=c]{230}{$\circlearrowright$}}}

\newcommand\ct[1]{\text{\rmfamily\upshape #1}}
\newcommand\question[1]{ {\color{red} ...!? \small #1}}
\newcommand\caz[1]{\left\{\begin{array} #1 \end{array}\right.}
\newcommand\const{\text{\rmfamily\upshape const}}
\newcommand\toP{ \overset{\pro}{\to}}
\newcommand\toPP{ \overset{\text{PP}}{\to}}
\newcommand{\oeq}{\mathrel{\text{\textcircled{$=$}}}}





\usepackage{xcolor}
% \usepackage[normalem]{ulem}
\usepackage{lipsum}
\makeatletter
% \newcommand\colorwave[1][blue]{\bgroup \markoverwith{\lower3.5\p@\hbox{\sixly \textcolor{#1}{\char58}}}\ULon}
%\font\sixly=lasy6 % does not re-load if already loaded, so no memory problem.

\newmdtheoremenv[
linewidth= 1pt,linecolor= blue,%
leftmargin=20,rightmargin=20,innertopmargin=0pt, innerrightmargin=40,%
tikzsetting = { draw=lightgray, line width = 0.3pt,dashed,%
dash pattern = on 15pt off 3pt},%
splittopskip=\topskip,skipbelow=\baselineskip,%
skipabove=\baselineskip,ntheorem,roundcorner=0pt,
% backgroundcolor=pagebg,font=\color{orange}\sffamily, fontcolor=white
]{examplebox}{Exemple}[section]



\newcommand\R{\mathbb{R}}
\newcommand\Z{\mathbb{Z}}
\newcommand\N{\mathbb{N}}
\newcommand\E{\mathbb{E}}
\newcommand\F{\mathcal{F}}
\newcommand\cH{\mathcal{H}}
\newcommand\V{\mathbb{V}}
\newcommand\dmo{ ^{-1} }
\newcommand\kapa{\kappa}
\newcommand\im{Im}
\newcommand\hs{\mathcal{H}}





\usepackage{soul}

\makeatletter
\newcommand*{\whiten}[1]{\llap{\textcolor{white}{{\the\SOUL@token}}\hspace{#1pt}}}
\DeclareRobustCommand*\myul{%
    \def\SOUL@everyspace{\underline{\space}\kern\z@}%
    \def\SOUL@everytoken{%
     \setbox0=\hbox{\the\SOUL@token}%
     \ifdim\dp0>\z@
        \raisebox{\dp0}{\underline{\phantom{\the\SOUL@token}}}%
        \whiten{1}\whiten{0}%
        \whiten{-1}\whiten{-2}%
        \llap{\the\SOUL@token}%
     \else
        \underline{\the\SOUL@token}%
     \fi}%
\SOUL@}
\makeatother

\newcommand*{\demp}{\fontfamily{lmtt}\selectfont}

\DeclareTextFontCommand{\textdemp}{\demp}

\begin{document}

\ifcomment
Multiline
comment
\fi
\ifcomment
\myul{Typesetting test}
% \color[rgb]{1,1,1}
$∑_i^n≠ 60º±∞π∆¬≈√j∫h≤≥µ$

$\CR \R\pro\ind\pro\gS\pro
\mqty[a&b\\c&d]$
$\pro\mathbb{P}$
$\dd{x}$

  \[
    \alpha(x)=\left\{
                \begin{array}{ll}
                  x\\
                  \frac{1}{1+e^{-kx}}\\
                  \frac{e^x-e^{-x}}{e^x+e^{-x}}
                \end{array}
              \right.
  \]

  $\expval{x}$
  
  $\chi_\rho(ghg\dmo)=\Tr(\rho_{ghg\dmo})=\Tr(\rho_g\circ\rho_h\circ\rho\dmo_g)=\Tr(\rho_h)\overset{\mbox{\scalebox{0.5}{$\Tr(AB)=\Tr(BA)$}}}{=}\chi_\rho(h)$
  	$\mathop{\oplus}_{\substack{x\in X}}$

$\mat(\rho_g)=(a_{ij}(g))_{\scriptsize \substack{1\leq i\leq d \\ 1\leq j\leq d}}$ et $\mat(\rho'_g)=(a'_{ij}(g))_{\scriptsize \substack{1\leq i'\leq d' \\ 1\leq j'\leq d'}}$



\[\int_a^b{\mathbb{R}^2}g(u, v)\dd{P_{XY}}(u, v)=\iint g(u,v) f_{XY}(u, v)\dd \lambda(u) \dd \lambda(v)\]
$$\lim_{x\to\infty} f(x)$$	
$$\iiiint_V \mu(t,u,v,w) \,dt\,du\,dv\,dw$$
$$\sum_{n=1}^{\infty} 2^{-n} = 1$$	
\begin{definition}
	Si $X$ et $Y$ sont 2 v.a. ou definit la \textsc{Covariance} entre $X$ et $Y$ comme
	$\cov(X,Y)\overset{\text{def}}{=}\E\left[(X-\E(X))(Y-\E(Y))\right]=\E(XY)-\E(X)\E(Y)$.
\end{definition}
\fi
\pagebreak

% \tableofcontents

% insert your code here
%% !TEX encoding = UTF-8 Unicode
% !TEX TS-program = xelatex

\documentclass[french]{report}

%\usepackage[utf8]{inputenc}
%\usepackage[T1]{fontenc}
\usepackage{babel}


\newif\ifcomment
%\commenttrue # Show comments

\usepackage{physics}
\usepackage{amssymb}


\usepackage{amsthm}
% \usepackage{thmtools}
\usepackage{mathtools}
\usepackage{amsfonts}

\usepackage{color}

\usepackage{tikz}

\usepackage{geometry}
\geometry{a5paper, margin=0.1in, right=1cm}

\usepackage{dsfont}

\usepackage{graphicx}
\graphicspath{ {images/} }

\usepackage{faktor}

\usepackage{IEEEtrantools}
\usepackage{enumerate}   
\usepackage[PostScript=dvips]{"/Users/aware/Documents/Courses/diagrams"}


\newtheorem{theorem}{Théorème}[section]
\renewcommand{\thetheorem}{\arabic{theorem}}
\newtheorem{lemme}{Lemme}[section]
\renewcommand{\thelemme}{\arabic{lemme}}
\newtheorem{proposition}{Proposition}[section]
\renewcommand{\theproposition}{\arabic{proposition}}
\newtheorem{notations}{Notations}[section]
\newtheorem{problem}{Problème}[section]
\newtheorem{corollary}{Corollaire}[theorem]
\renewcommand{\thecorollary}{\arabic{corollary}}
\newtheorem{property}{Propriété}[section]
\newtheorem{objective}{Objectif}[section]

\theoremstyle{definition}
\newtheorem{definition}{Définition}[section]
\renewcommand{\thedefinition}{\arabic{definition}}
\newtheorem{exercise}{Exercice}[chapter]
\renewcommand{\theexercise}{\arabic{exercise}}
\newtheorem{example}{Exemple}[chapter]
\renewcommand{\theexample}{\arabic{example}}
\newtheorem*{solution}{Solution}
\newtheorem*{application}{Application}
\newtheorem*{notation}{Notation}
\newtheorem*{vocabulary}{Vocabulaire}
\newtheorem*{properties}{Propriétés}



\theoremstyle{remark}
\newtheorem*{remark}{Remarque}
\newtheorem*{rappel}{Rappel}


\usepackage{etoolbox}
\AtBeginEnvironment{exercise}{\small}
\AtBeginEnvironment{example}{\small}

\usepackage{cases}
\usepackage[red]{mypack}

\usepackage[framemethod=TikZ]{mdframed}

\definecolor{bg}{rgb}{0.4,0.25,0.95}
\definecolor{pagebg}{rgb}{0,0,0.5}
\surroundwithmdframed[
   topline=false,
   rightline=false,
   bottomline=false,
   leftmargin=\parindent,
   skipabove=8pt,
   skipbelow=8pt,
   linecolor=blue,
   innerbottommargin=10pt,
   % backgroundcolor=bg,font=\color{orange}\sffamily, fontcolor=white
]{definition}

\usepackage{empheq}
\usepackage[most]{tcolorbox}

\newtcbox{\mymath}[1][]{%
    nobeforeafter, math upper, tcbox raise base,
    enhanced, colframe=blue!30!black,
    colback=red!10, boxrule=1pt,
    #1}

\usepackage{unixode}


\DeclareMathOperator{\ord}{ord}
\DeclareMathOperator{\orb}{orb}
\DeclareMathOperator{\stab}{stab}
\DeclareMathOperator{\Stab}{stab}
\DeclareMathOperator{\ppcm}{ppcm}
\DeclareMathOperator{\conj}{Conj}
\DeclareMathOperator{\End}{End}
\DeclareMathOperator{\rot}{rot}
\DeclareMathOperator{\trs}{trace}
\DeclareMathOperator{\Ind}{Ind}
\DeclareMathOperator{\mat}{Mat}
\DeclareMathOperator{\id}{Id}
\DeclareMathOperator{\vect}{vect}
\DeclareMathOperator{\img}{img}
\DeclareMathOperator{\cov}{Cov}
\DeclareMathOperator{\dist}{dist}
\DeclareMathOperator{\irr}{Irr}
\DeclareMathOperator{\image}{Im}
\DeclareMathOperator{\pd}{\partial}
\DeclareMathOperator{\epi}{epi}
\DeclareMathOperator{\Argmin}{Argmin}
\DeclareMathOperator{\dom}{dom}
\DeclareMathOperator{\proj}{proj}
\DeclareMathOperator{\ctg}{ctg}
\DeclareMathOperator{\supp}{supp}
\DeclareMathOperator{\argmin}{argmin}
\DeclareMathOperator{\mult}{mult}
\DeclareMathOperator{\ch}{ch}
\DeclareMathOperator{\sh}{sh}
\DeclareMathOperator{\rang}{rang}
\DeclareMathOperator{\diam}{diam}
\DeclareMathOperator{\Epigraphe}{Epigraphe}




\usepackage{xcolor}
\everymath{\color{blue}}
%\everymath{\color[rgb]{0,1,1}}
%\pagecolor[rgb]{0,0,0.5}


\newcommand*{\pdtest}[3][]{\ensuremath{\frac{\partial^{#1} #2}{\partial #3}}}

\newcommand*{\deffunc}[6][]{\ensuremath{
\begin{array}{rcl}
#2 : #3 &\rightarrow& #4\\
#5 &\mapsto& #6
\end{array}
}}

\newcommand{\eqcolon}{\mathrel{\resizebox{\widthof{$\mathord{=}$}}{\height}{ $\!\!=\!\!\resizebox{1.2\width}{0.8\height}{\raisebox{0.23ex}{$\mathop{:}$}}\!\!$ }}}
\newcommand{\coloneq}{\mathrel{\resizebox{\widthof{$\mathord{=}$}}{\height}{ $\!\!\resizebox{1.2\width}{0.8\height}{\raisebox{0.23ex}{$\mathop{:}$}}\!\!=\!\!$ }}}
\newcommand{\eqcolonl}{\ensuremath{\mathrel{=\!\!\mathop{:}}}}
\newcommand{\coloneql}{\ensuremath{\mathrel{\mathop{:} \!\! =}}}
\newcommand{\vc}[1]{% inline column vector
  \left(\begin{smallmatrix}#1\end{smallmatrix}\right)%
}
\newcommand{\vr}[1]{% inline row vector
  \begin{smallmatrix}(\,#1\,)\end{smallmatrix}%
}
\makeatletter
\newcommand*{\defeq}{\ =\mathrel{\rlap{%
                     \raisebox{0.3ex}{$\m@th\cdot$}}%
                     \raisebox{-0.3ex}{$\m@th\cdot$}}%
                     }
\makeatother

\newcommand{\mathcircle}[1]{% inline row vector
 \overset{\circ}{#1}
}
\newcommand{\ulim}{% low limit
 \underline{\lim}
}
\newcommand{\ssi}{% iff
\iff
}
\newcommand{\ps}[2]{
\expval{#1 | #2}
}
\newcommand{\df}[1]{
\mqty{#1}
}
\newcommand{\n}[1]{
\norm{#1}
}
\newcommand{\sys}[1]{
\left\{\smqty{#1}\right.
}


\newcommand{\eqdef}{\ensuremath{\overset{\text{def}}=}}


\def\Circlearrowright{\ensuremath{%
  \rotatebox[origin=c]{230}{$\circlearrowright$}}}

\newcommand\ct[1]{\text{\rmfamily\upshape #1}}
\newcommand\question[1]{ {\color{red} ...!? \small #1}}
\newcommand\caz[1]{\left\{\begin{array} #1 \end{array}\right.}
\newcommand\const{\text{\rmfamily\upshape const}}
\newcommand\toP{ \overset{\pro}{\to}}
\newcommand\toPP{ \overset{\text{PP}}{\to}}
\newcommand{\oeq}{\mathrel{\text{\textcircled{$=$}}}}





\usepackage{xcolor}
% \usepackage[normalem]{ulem}
\usepackage{lipsum}
\makeatletter
% \newcommand\colorwave[1][blue]{\bgroup \markoverwith{\lower3.5\p@\hbox{\sixly \textcolor{#1}{\char58}}}\ULon}
%\font\sixly=lasy6 % does not re-load if already loaded, so no memory problem.

\newmdtheoremenv[
linewidth= 1pt,linecolor= blue,%
leftmargin=20,rightmargin=20,innertopmargin=0pt, innerrightmargin=40,%
tikzsetting = { draw=lightgray, line width = 0.3pt,dashed,%
dash pattern = on 15pt off 3pt},%
splittopskip=\topskip,skipbelow=\baselineskip,%
skipabove=\baselineskip,ntheorem,roundcorner=0pt,
% backgroundcolor=pagebg,font=\color{orange}\sffamily, fontcolor=white
]{examplebox}{Exemple}[section]



\newcommand\R{\mathbb{R}}
\newcommand\Z{\mathbb{Z}}
\newcommand\N{\mathbb{N}}
\newcommand\E{\mathbb{E}}
\newcommand\F{\mathcal{F}}
\newcommand\cH{\mathcal{H}}
\newcommand\V{\mathbb{V}}
\newcommand\dmo{ ^{-1} }
\newcommand\kapa{\kappa}
\newcommand\im{Im}
\newcommand\hs{\mathcal{H}}





\usepackage{soul}

\makeatletter
\newcommand*{\whiten}[1]{\llap{\textcolor{white}{{\the\SOUL@token}}\hspace{#1pt}}}
\DeclareRobustCommand*\myul{%
    \def\SOUL@everyspace{\underline{\space}\kern\z@}%
    \def\SOUL@everytoken{%
     \setbox0=\hbox{\the\SOUL@token}%
     \ifdim\dp0>\z@
        \raisebox{\dp0}{\underline{\phantom{\the\SOUL@token}}}%
        \whiten{1}\whiten{0}%
        \whiten{-1}\whiten{-2}%
        \llap{\the\SOUL@token}%
     \else
        \underline{\the\SOUL@token}%
     \fi}%
\SOUL@}
\makeatother

\newcommand*{\demp}{\fontfamily{lmtt}\selectfont}

\DeclareTextFontCommand{\textdemp}{\demp}

\begin{document}

\ifcomment
Multiline
comment
\fi
\ifcomment
\myul{Typesetting test}
% \color[rgb]{1,1,1}
$∑_i^n≠ 60º±∞π∆¬≈√j∫h≤≥µ$

$\CR \R\pro\ind\pro\gS\pro
\mqty[a&b\\c&d]$
$\pro\mathbb{P}$
$\dd{x}$

  \[
    \alpha(x)=\left\{
                \begin{array}{ll}
                  x\\
                  \frac{1}{1+e^{-kx}}\\
                  \frac{e^x-e^{-x}}{e^x+e^{-x}}
                \end{array}
              \right.
  \]

  $\expval{x}$
  
  $\chi_\rho(ghg\dmo)=\Tr(\rho_{ghg\dmo})=\Tr(\rho_g\circ\rho_h\circ\rho\dmo_g)=\Tr(\rho_h)\overset{\mbox{\scalebox{0.5}{$\Tr(AB)=\Tr(BA)$}}}{=}\chi_\rho(h)$
  	$\mathop{\oplus}_{\substack{x\in X}}$

$\mat(\rho_g)=(a_{ij}(g))_{\scriptsize \substack{1\leq i\leq d \\ 1\leq j\leq d}}$ et $\mat(\rho'_g)=(a'_{ij}(g))_{\scriptsize \substack{1\leq i'\leq d' \\ 1\leq j'\leq d'}}$



\[\int_a^b{\mathbb{R}^2}g(u, v)\dd{P_{XY}}(u, v)=\iint g(u,v) f_{XY}(u, v)\dd \lambda(u) \dd \lambda(v)\]
$$\lim_{x\to\infty} f(x)$$	
$$\iiiint_V \mu(t,u,v,w) \,dt\,du\,dv\,dw$$
$$\sum_{n=1}^{\infty} 2^{-n} = 1$$	
\begin{definition}
	Si $X$ et $Y$ sont 2 v.a. ou definit la \textsc{Covariance} entre $X$ et $Y$ comme
	$\cov(X,Y)\overset{\text{def}}{=}\E\left[(X-\E(X))(Y-\E(Y))\right]=\E(XY)-\E(X)\E(Y)$.
\end{definition}
\fi
\pagebreak

% \tableofcontents

% insert your code here
%% !TEX encoding = UTF-8 Unicode
% !TEX TS-program = xelatex

\documentclass[french]{report}

%\usepackage[utf8]{inputenc}
%\usepackage[T1]{fontenc}
\usepackage{babel}


\newif\ifcomment
%\commenttrue # Show comments

\usepackage{physics}
\usepackage{amssymb}


\usepackage{amsthm}
% \usepackage{thmtools}
\usepackage{mathtools}
\usepackage{amsfonts}

\usepackage{color}

\usepackage{tikz}

\usepackage{geometry}
\geometry{a5paper, margin=0.1in, right=1cm}

\usepackage{dsfont}

\usepackage{graphicx}
\graphicspath{ {images/} }

\usepackage{faktor}

\usepackage{IEEEtrantools}
\usepackage{enumerate}   
\usepackage[PostScript=dvips]{"/Users/aware/Documents/Courses/diagrams"}


\newtheorem{theorem}{Théorème}[section]
\renewcommand{\thetheorem}{\arabic{theorem}}
\newtheorem{lemme}{Lemme}[section]
\renewcommand{\thelemme}{\arabic{lemme}}
\newtheorem{proposition}{Proposition}[section]
\renewcommand{\theproposition}{\arabic{proposition}}
\newtheorem{notations}{Notations}[section]
\newtheorem{problem}{Problème}[section]
\newtheorem{corollary}{Corollaire}[theorem]
\renewcommand{\thecorollary}{\arabic{corollary}}
\newtheorem{property}{Propriété}[section]
\newtheorem{objective}{Objectif}[section]

\theoremstyle{definition}
\newtheorem{definition}{Définition}[section]
\renewcommand{\thedefinition}{\arabic{definition}}
\newtheorem{exercise}{Exercice}[chapter]
\renewcommand{\theexercise}{\arabic{exercise}}
\newtheorem{example}{Exemple}[chapter]
\renewcommand{\theexample}{\arabic{example}}
\newtheorem*{solution}{Solution}
\newtheorem*{application}{Application}
\newtheorem*{notation}{Notation}
\newtheorem*{vocabulary}{Vocabulaire}
\newtheorem*{properties}{Propriétés}



\theoremstyle{remark}
\newtheorem*{remark}{Remarque}
\newtheorem*{rappel}{Rappel}


\usepackage{etoolbox}
\AtBeginEnvironment{exercise}{\small}
\AtBeginEnvironment{example}{\small}

\usepackage{cases}
\usepackage[red]{mypack}

\usepackage[framemethod=TikZ]{mdframed}

\definecolor{bg}{rgb}{0.4,0.25,0.95}
\definecolor{pagebg}{rgb}{0,0,0.5}
\surroundwithmdframed[
   topline=false,
   rightline=false,
   bottomline=false,
   leftmargin=\parindent,
   skipabove=8pt,
   skipbelow=8pt,
   linecolor=blue,
   innerbottommargin=10pt,
   % backgroundcolor=bg,font=\color{orange}\sffamily, fontcolor=white
]{definition}

\usepackage{empheq}
\usepackage[most]{tcolorbox}

\newtcbox{\mymath}[1][]{%
    nobeforeafter, math upper, tcbox raise base,
    enhanced, colframe=blue!30!black,
    colback=red!10, boxrule=1pt,
    #1}

\usepackage{unixode}


\DeclareMathOperator{\ord}{ord}
\DeclareMathOperator{\orb}{orb}
\DeclareMathOperator{\stab}{stab}
\DeclareMathOperator{\Stab}{stab}
\DeclareMathOperator{\ppcm}{ppcm}
\DeclareMathOperator{\conj}{Conj}
\DeclareMathOperator{\End}{End}
\DeclareMathOperator{\rot}{rot}
\DeclareMathOperator{\trs}{trace}
\DeclareMathOperator{\Ind}{Ind}
\DeclareMathOperator{\mat}{Mat}
\DeclareMathOperator{\id}{Id}
\DeclareMathOperator{\vect}{vect}
\DeclareMathOperator{\img}{img}
\DeclareMathOperator{\cov}{Cov}
\DeclareMathOperator{\dist}{dist}
\DeclareMathOperator{\irr}{Irr}
\DeclareMathOperator{\image}{Im}
\DeclareMathOperator{\pd}{\partial}
\DeclareMathOperator{\epi}{epi}
\DeclareMathOperator{\Argmin}{Argmin}
\DeclareMathOperator{\dom}{dom}
\DeclareMathOperator{\proj}{proj}
\DeclareMathOperator{\ctg}{ctg}
\DeclareMathOperator{\supp}{supp}
\DeclareMathOperator{\argmin}{argmin}
\DeclareMathOperator{\mult}{mult}
\DeclareMathOperator{\ch}{ch}
\DeclareMathOperator{\sh}{sh}
\DeclareMathOperator{\rang}{rang}
\DeclareMathOperator{\diam}{diam}
\DeclareMathOperator{\Epigraphe}{Epigraphe}




\usepackage{xcolor}
\everymath{\color{blue}}
%\everymath{\color[rgb]{0,1,1}}
%\pagecolor[rgb]{0,0,0.5}


\newcommand*{\pdtest}[3][]{\ensuremath{\frac{\partial^{#1} #2}{\partial #3}}}

\newcommand*{\deffunc}[6][]{\ensuremath{
\begin{array}{rcl}
#2 : #3 &\rightarrow& #4\\
#5 &\mapsto& #6
\end{array}
}}

\newcommand{\eqcolon}{\mathrel{\resizebox{\widthof{$\mathord{=}$}}{\height}{ $\!\!=\!\!\resizebox{1.2\width}{0.8\height}{\raisebox{0.23ex}{$\mathop{:}$}}\!\!$ }}}
\newcommand{\coloneq}{\mathrel{\resizebox{\widthof{$\mathord{=}$}}{\height}{ $\!\!\resizebox{1.2\width}{0.8\height}{\raisebox{0.23ex}{$\mathop{:}$}}\!\!=\!\!$ }}}
\newcommand{\eqcolonl}{\ensuremath{\mathrel{=\!\!\mathop{:}}}}
\newcommand{\coloneql}{\ensuremath{\mathrel{\mathop{:} \!\! =}}}
\newcommand{\vc}[1]{% inline column vector
  \left(\begin{smallmatrix}#1\end{smallmatrix}\right)%
}
\newcommand{\vr}[1]{% inline row vector
  \begin{smallmatrix}(\,#1\,)\end{smallmatrix}%
}
\makeatletter
\newcommand*{\defeq}{\ =\mathrel{\rlap{%
                     \raisebox{0.3ex}{$\m@th\cdot$}}%
                     \raisebox{-0.3ex}{$\m@th\cdot$}}%
                     }
\makeatother

\newcommand{\mathcircle}[1]{% inline row vector
 \overset{\circ}{#1}
}
\newcommand{\ulim}{% low limit
 \underline{\lim}
}
\newcommand{\ssi}{% iff
\iff
}
\newcommand{\ps}[2]{
\expval{#1 | #2}
}
\newcommand{\df}[1]{
\mqty{#1}
}
\newcommand{\n}[1]{
\norm{#1}
}
\newcommand{\sys}[1]{
\left\{\smqty{#1}\right.
}


\newcommand{\eqdef}{\ensuremath{\overset{\text{def}}=}}


\def\Circlearrowright{\ensuremath{%
  \rotatebox[origin=c]{230}{$\circlearrowright$}}}

\newcommand\ct[1]{\text{\rmfamily\upshape #1}}
\newcommand\question[1]{ {\color{red} ...!? \small #1}}
\newcommand\caz[1]{\left\{\begin{array} #1 \end{array}\right.}
\newcommand\const{\text{\rmfamily\upshape const}}
\newcommand\toP{ \overset{\pro}{\to}}
\newcommand\toPP{ \overset{\text{PP}}{\to}}
\newcommand{\oeq}{\mathrel{\text{\textcircled{$=$}}}}





\usepackage{xcolor}
% \usepackage[normalem]{ulem}
\usepackage{lipsum}
\makeatletter
% \newcommand\colorwave[1][blue]{\bgroup \markoverwith{\lower3.5\p@\hbox{\sixly \textcolor{#1}{\char58}}}\ULon}
%\font\sixly=lasy6 % does not re-load if already loaded, so no memory problem.

\newmdtheoremenv[
linewidth= 1pt,linecolor= blue,%
leftmargin=20,rightmargin=20,innertopmargin=0pt, innerrightmargin=40,%
tikzsetting = { draw=lightgray, line width = 0.3pt,dashed,%
dash pattern = on 15pt off 3pt},%
splittopskip=\topskip,skipbelow=\baselineskip,%
skipabove=\baselineskip,ntheorem,roundcorner=0pt,
% backgroundcolor=pagebg,font=\color{orange}\sffamily, fontcolor=white
]{examplebox}{Exemple}[section]



\newcommand\R{\mathbb{R}}
\newcommand\Z{\mathbb{Z}}
\newcommand\N{\mathbb{N}}
\newcommand\E{\mathbb{E}}
\newcommand\F{\mathcal{F}}
\newcommand\cH{\mathcal{H}}
\newcommand\V{\mathbb{V}}
\newcommand\dmo{ ^{-1} }
\newcommand\kapa{\kappa}
\newcommand\im{Im}
\newcommand\hs{\mathcal{H}}





\usepackage{soul}

\makeatletter
\newcommand*{\whiten}[1]{\llap{\textcolor{white}{{\the\SOUL@token}}\hspace{#1pt}}}
\DeclareRobustCommand*\myul{%
    \def\SOUL@everyspace{\underline{\space}\kern\z@}%
    \def\SOUL@everytoken{%
     \setbox0=\hbox{\the\SOUL@token}%
     \ifdim\dp0>\z@
        \raisebox{\dp0}{\underline{\phantom{\the\SOUL@token}}}%
        \whiten{1}\whiten{0}%
        \whiten{-1}\whiten{-2}%
        \llap{\the\SOUL@token}%
     \else
        \underline{\the\SOUL@token}%
     \fi}%
\SOUL@}
\makeatother

\newcommand*{\demp}{\fontfamily{lmtt}\selectfont}

\DeclareTextFontCommand{\textdemp}{\demp}

\begin{document}

\ifcomment
Multiline
comment
\fi
\ifcomment
\myul{Typesetting test}
% \color[rgb]{1,1,1}
$∑_i^n≠ 60º±∞π∆¬≈√j∫h≤≥µ$

$\CR \R\pro\ind\pro\gS\pro
\mqty[a&b\\c&d]$
$\pro\mathbb{P}$
$\dd{x}$

  \[
    \alpha(x)=\left\{
                \begin{array}{ll}
                  x\\
                  \frac{1}{1+e^{-kx}}\\
                  \frac{e^x-e^{-x}}{e^x+e^{-x}}
                \end{array}
              \right.
  \]

  $\expval{x}$
  
  $\chi_\rho(ghg\dmo)=\Tr(\rho_{ghg\dmo})=\Tr(\rho_g\circ\rho_h\circ\rho\dmo_g)=\Tr(\rho_h)\overset{\mbox{\scalebox{0.5}{$\Tr(AB)=\Tr(BA)$}}}{=}\chi_\rho(h)$
  	$\mathop{\oplus}_{\substack{x\in X}}$

$\mat(\rho_g)=(a_{ij}(g))_{\scriptsize \substack{1\leq i\leq d \\ 1\leq j\leq d}}$ et $\mat(\rho'_g)=(a'_{ij}(g))_{\scriptsize \substack{1\leq i'\leq d' \\ 1\leq j'\leq d'}}$



\[\int_a^b{\mathbb{R}^2}g(u, v)\dd{P_{XY}}(u, v)=\iint g(u,v) f_{XY}(u, v)\dd \lambda(u) \dd \lambda(v)\]
$$\lim_{x\to\infty} f(x)$$	
$$\iiiint_V \mu(t,u,v,w) \,dt\,du\,dv\,dw$$
$$\sum_{n=1}^{\infty} 2^{-n} = 1$$	
\begin{definition}
	Si $X$ et $Y$ sont 2 v.a. ou definit la \textsc{Covariance} entre $X$ et $Y$ comme
	$\cov(X,Y)\overset{\text{def}}{=}\E\left[(X-\E(X))(Y-\E(Y))\right]=\E(XY)-\E(X)\E(Y)$.
\end{definition}
\fi
\pagebreak

% \tableofcontents

% insert your code here
%% !TEX encoding = UTF-8 Unicode
% !TEX TS-program = xelatex

\documentclass[french]{report}

%\usepackage[utf8]{inputenc}
%\usepackage[T1]{fontenc}
\usepackage{babel}


\newif\ifcomment
%\commenttrue # Show comments

\usepackage{physics}
\usepackage{amssymb}


\usepackage{amsthm}
% \usepackage{thmtools}
\usepackage{mathtools}
\usepackage{amsfonts}

\usepackage{color}

\usepackage{tikz}

\usepackage{geometry}
\geometry{a5paper, margin=0.1in, right=1cm}

\usepackage{dsfont}

\usepackage{graphicx}
\graphicspath{ {images/} }

\usepackage{faktor}

\usepackage{IEEEtrantools}
\usepackage{enumerate}   
\usepackage[PostScript=dvips]{"/Users/aware/Documents/Courses/diagrams"}


\newtheorem{theorem}{Théorème}[section]
\renewcommand{\thetheorem}{\arabic{theorem}}
\newtheorem{lemme}{Lemme}[section]
\renewcommand{\thelemme}{\arabic{lemme}}
\newtheorem{proposition}{Proposition}[section]
\renewcommand{\theproposition}{\arabic{proposition}}
\newtheorem{notations}{Notations}[section]
\newtheorem{problem}{Problème}[section]
\newtheorem{corollary}{Corollaire}[theorem]
\renewcommand{\thecorollary}{\arabic{corollary}}
\newtheorem{property}{Propriété}[section]
\newtheorem{objective}{Objectif}[section]

\theoremstyle{definition}
\newtheorem{definition}{Définition}[section]
\renewcommand{\thedefinition}{\arabic{definition}}
\newtheorem{exercise}{Exercice}[chapter]
\renewcommand{\theexercise}{\arabic{exercise}}
\newtheorem{example}{Exemple}[chapter]
\renewcommand{\theexample}{\arabic{example}}
\newtheorem*{solution}{Solution}
\newtheorem*{application}{Application}
\newtheorem*{notation}{Notation}
\newtheorem*{vocabulary}{Vocabulaire}
\newtheorem*{properties}{Propriétés}



\theoremstyle{remark}
\newtheorem*{remark}{Remarque}
\newtheorem*{rappel}{Rappel}


\usepackage{etoolbox}
\AtBeginEnvironment{exercise}{\small}
\AtBeginEnvironment{example}{\small}

\usepackage{cases}
\usepackage[red]{mypack}

\usepackage[framemethod=TikZ]{mdframed}

\definecolor{bg}{rgb}{0.4,0.25,0.95}
\definecolor{pagebg}{rgb}{0,0,0.5}
\surroundwithmdframed[
   topline=false,
   rightline=false,
   bottomline=false,
   leftmargin=\parindent,
   skipabove=8pt,
   skipbelow=8pt,
   linecolor=blue,
   innerbottommargin=10pt,
   % backgroundcolor=bg,font=\color{orange}\sffamily, fontcolor=white
]{definition}

\usepackage{empheq}
\usepackage[most]{tcolorbox}

\newtcbox{\mymath}[1][]{%
    nobeforeafter, math upper, tcbox raise base,
    enhanced, colframe=blue!30!black,
    colback=red!10, boxrule=1pt,
    #1}

\usepackage{unixode}


\DeclareMathOperator{\ord}{ord}
\DeclareMathOperator{\orb}{orb}
\DeclareMathOperator{\stab}{stab}
\DeclareMathOperator{\Stab}{stab}
\DeclareMathOperator{\ppcm}{ppcm}
\DeclareMathOperator{\conj}{Conj}
\DeclareMathOperator{\End}{End}
\DeclareMathOperator{\rot}{rot}
\DeclareMathOperator{\trs}{trace}
\DeclareMathOperator{\Ind}{Ind}
\DeclareMathOperator{\mat}{Mat}
\DeclareMathOperator{\id}{Id}
\DeclareMathOperator{\vect}{vect}
\DeclareMathOperator{\img}{img}
\DeclareMathOperator{\cov}{Cov}
\DeclareMathOperator{\dist}{dist}
\DeclareMathOperator{\irr}{Irr}
\DeclareMathOperator{\image}{Im}
\DeclareMathOperator{\pd}{\partial}
\DeclareMathOperator{\epi}{epi}
\DeclareMathOperator{\Argmin}{Argmin}
\DeclareMathOperator{\dom}{dom}
\DeclareMathOperator{\proj}{proj}
\DeclareMathOperator{\ctg}{ctg}
\DeclareMathOperator{\supp}{supp}
\DeclareMathOperator{\argmin}{argmin}
\DeclareMathOperator{\mult}{mult}
\DeclareMathOperator{\ch}{ch}
\DeclareMathOperator{\sh}{sh}
\DeclareMathOperator{\rang}{rang}
\DeclareMathOperator{\diam}{diam}
\DeclareMathOperator{\Epigraphe}{Epigraphe}




\usepackage{xcolor}
\everymath{\color{blue}}
%\everymath{\color[rgb]{0,1,1}}
%\pagecolor[rgb]{0,0,0.5}


\newcommand*{\pdtest}[3][]{\ensuremath{\frac{\partial^{#1} #2}{\partial #3}}}

\newcommand*{\deffunc}[6][]{\ensuremath{
\begin{array}{rcl}
#2 : #3 &\rightarrow& #4\\
#5 &\mapsto& #6
\end{array}
}}

\newcommand{\eqcolon}{\mathrel{\resizebox{\widthof{$\mathord{=}$}}{\height}{ $\!\!=\!\!\resizebox{1.2\width}{0.8\height}{\raisebox{0.23ex}{$\mathop{:}$}}\!\!$ }}}
\newcommand{\coloneq}{\mathrel{\resizebox{\widthof{$\mathord{=}$}}{\height}{ $\!\!\resizebox{1.2\width}{0.8\height}{\raisebox{0.23ex}{$\mathop{:}$}}\!\!=\!\!$ }}}
\newcommand{\eqcolonl}{\ensuremath{\mathrel{=\!\!\mathop{:}}}}
\newcommand{\coloneql}{\ensuremath{\mathrel{\mathop{:} \!\! =}}}
\newcommand{\vc}[1]{% inline column vector
  \left(\begin{smallmatrix}#1\end{smallmatrix}\right)%
}
\newcommand{\vr}[1]{% inline row vector
  \begin{smallmatrix}(\,#1\,)\end{smallmatrix}%
}
\makeatletter
\newcommand*{\defeq}{\ =\mathrel{\rlap{%
                     \raisebox{0.3ex}{$\m@th\cdot$}}%
                     \raisebox{-0.3ex}{$\m@th\cdot$}}%
                     }
\makeatother

\newcommand{\mathcircle}[1]{% inline row vector
 \overset{\circ}{#1}
}
\newcommand{\ulim}{% low limit
 \underline{\lim}
}
\newcommand{\ssi}{% iff
\iff
}
\newcommand{\ps}[2]{
\expval{#1 | #2}
}
\newcommand{\df}[1]{
\mqty{#1}
}
\newcommand{\n}[1]{
\norm{#1}
}
\newcommand{\sys}[1]{
\left\{\smqty{#1}\right.
}


\newcommand{\eqdef}{\ensuremath{\overset{\text{def}}=}}


\def\Circlearrowright{\ensuremath{%
  \rotatebox[origin=c]{230}{$\circlearrowright$}}}

\newcommand\ct[1]{\text{\rmfamily\upshape #1}}
\newcommand\question[1]{ {\color{red} ...!? \small #1}}
\newcommand\caz[1]{\left\{\begin{array} #1 \end{array}\right.}
\newcommand\const{\text{\rmfamily\upshape const}}
\newcommand\toP{ \overset{\pro}{\to}}
\newcommand\toPP{ \overset{\text{PP}}{\to}}
\newcommand{\oeq}{\mathrel{\text{\textcircled{$=$}}}}





\usepackage{xcolor}
% \usepackage[normalem]{ulem}
\usepackage{lipsum}
\makeatletter
% \newcommand\colorwave[1][blue]{\bgroup \markoverwith{\lower3.5\p@\hbox{\sixly \textcolor{#1}{\char58}}}\ULon}
%\font\sixly=lasy6 % does not re-load if already loaded, so no memory problem.

\newmdtheoremenv[
linewidth= 1pt,linecolor= blue,%
leftmargin=20,rightmargin=20,innertopmargin=0pt, innerrightmargin=40,%
tikzsetting = { draw=lightgray, line width = 0.3pt,dashed,%
dash pattern = on 15pt off 3pt},%
splittopskip=\topskip,skipbelow=\baselineskip,%
skipabove=\baselineskip,ntheorem,roundcorner=0pt,
% backgroundcolor=pagebg,font=\color{orange}\sffamily, fontcolor=white
]{examplebox}{Exemple}[section]



\newcommand\R{\mathbb{R}}
\newcommand\Z{\mathbb{Z}}
\newcommand\N{\mathbb{N}}
\newcommand\E{\mathbb{E}}
\newcommand\F{\mathcal{F}}
\newcommand\cH{\mathcal{H}}
\newcommand\V{\mathbb{V}}
\newcommand\dmo{ ^{-1} }
\newcommand\kapa{\kappa}
\newcommand\im{Im}
\newcommand\hs{\mathcal{H}}





\usepackage{soul}

\makeatletter
\newcommand*{\whiten}[1]{\llap{\textcolor{white}{{\the\SOUL@token}}\hspace{#1pt}}}
\DeclareRobustCommand*\myul{%
    \def\SOUL@everyspace{\underline{\space}\kern\z@}%
    \def\SOUL@everytoken{%
     \setbox0=\hbox{\the\SOUL@token}%
     \ifdim\dp0>\z@
        \raisebox{\dp0}{\underline{\phantom{\the\SOUL@token}}}%
        \whiten{1}\whiten{0}%
        \whiten{-1}\whiten{-2}%
        \llap{\the\SOUL@token}%
     \else
        \underline{\the\SOUL@token}%
     \fi}%
\SOUL@}
\makeatother

\newcommand*{\demp}{\fontfamily{lmtt}\selectfont}

\DeclareTextFontCommand{\textdemp}{\demp}

\begin{document}

\ifcomment
Multiline
comment
\fi
\ifcomment
\myul{Typesetting test}
% \color[rgb]{1,1,1}
$∑_i^n≠ 60º±∞π∆¬≈√j∫h≤≥µ$

$\CR \R\pro\ind\pro\gS\pro
\mqty[a&b\\c&d]$
$\pro\mathbb{P}$
$\dd{x}$

  \[
    \alpha(x)=\left\{
                \begin{array}{ll}
                  x\\
                  \frac{1}{1+e^{-kx}}\\
                  \frac{e^x-e^{-x}}{e^x+e^{-x}}
                \end{array}
              \right.
  \]

  $\expval{x}$
  
  $\chi_\rho(ghg\dmo)=\Tr(\rho_{ghg\dmo})=\Tr(\rho_g\circ\rho_h\circ\rho\dmo_g)=\Tr(\rho_h)\overset{\mbox{\scalebox{0.5}{$\Tr(AB)=\Tr(BA)$}}}{=}\chi_\rho(h)$
  	$\mathop{\oplus}_{\substack{x\in X}}$

$\mat(\rho_g)=(a_{ij}(g))_{\scriptsize \substack{1\leq i\leq d \\ 1\leq j\leq d}}$ et $\mat(\rho'_g)=(a'_{ij}(g))_{\scriptsize \substack{1\leq i'\leq d' \\ 1\leq j'\leq d'}}$



\[\int_a^b{\mathbb{R}^2}g(u, v)\dd{P_{XY}}(u, v)=\iint g(u,v) f_{XY}(u, v)\dd \lambda(u) \dd \lambda(v)\]
$$\lim_{x\to\infty} f(x)$$	
$$\iiiint_V \mu(t,u,v,w) \,dt\,du\,dv\,dw$$
$$\sum_{n=1}^{\infty} 2^{-n} = 1$$	
\begin{definition}
	Si $X$ et $Y$ sont 2 v.a. ou definit la \textsc{Covariance} entre $X$ et $Y$ comme
	$\cov(X,Y)\overset{\text{def}}{=}\E\left[(X-\E(X))(Y-\E(Y))\right]=\E(XY)-\E(X)\E(Y)$.
\end{definition}
\fi
\pagebreak

% \tableofcontents

% insert your code here
%\input{./algebra/main.tex}
%\input{./geometrie-differentielle/main.tex}
%\input{./probabilite/main.tex}
%\input{./analyse-fonctionnelle/main.tex}
% \input{./Analyse-convexe-et-dualite-en-optimisation/main.tex}
%\input{./tikz/main.tex}
%\input{./Theorie-du-distributions/main.tex}
%\input{./optimisation/mine.tex}
 \input{./modelisation/main.tex}

% yves.aubry@univ-tln.fr : algebra

\end{document}

%% !TEX encoding = UTF-8 Unicode
% !TEX TS-program = xelatex

\documentclass[french]{report}

%\usepackage[utf8]{inputenc}
%\usepackage[T1]{fontenc}
\usepackage{babel}


\newif\ifcomment
%\commenttrue # Show comments

\usepackage{physics}
\usepackage{amssymb}


\usepackage{amsthm}
% \usepackage{thmtools}
\usepackage{mathtools}
\usepackage{amsfonts}

\usepackage{color}

\usepackage{tikz}

\usepackage{geometry}
\geometry{a5paper, margin=0.1in, right=1cm}

\usepackage{dsfont}

\usepackage{graphicx}
\graphicspath{ {images/} }

\usepackage{faktor}

\usepackage{IEEEtrantools}
\usepackage{enumerate}   
\usepackage[PostScript=dvips]{"/Users/aware/Documents/Courses/diagrams"}


\newtheorem{theorem}{Théorème}[section]
\renewcommand{\thetheorem}{\arabic{theorem}}
\newtheorem{lemme}{Lemme}[section]
\renewcommand{\thelemme}{\arabic{lemme}}
\newtheorem{proposition}{Proposition}[section]
\renewcommand{\theproposition}{\arabic{proposition}}
\newtheorem{notations}{Notations}[section]
\newtheorem{problem}{Problème}[section]
\newtheorem{corollary}{Corollaire}[theorem]
\renewcommand{\thecorollary}{\arabic{corollary}}
\newtheorem{property}{Propriété}[section]
\newtheorem{objective}{Objectif}[section]

\theoremstyle{definition}
\newtheorem{definition}{Définition}[section]
\renewcommand{\thedefinition}{\arabic{definition}}
\newtheorem{exercise}{Exercice}[chapter]
\renewcommand{\theexercise}{\arabic{exercise}}
\newtheorem{example}{Exemple}[chapter]
\renewcommand{\theexample}{\arabic{example}}
\newtheorem*{solution}{Solution}
\newtheorem*{application}{Application}
\newtheorem*{notation}{Notation}
\newtheorem*{vocabulary}{Vocabulaire}
\newtheorem*{properties}{Propriétés}



\theoremstyle{remark}
\newtheorem*{remark}{Remarque}
\newtheorem*{rappel}{Rappel}


\usepackage{etoolbox}
\AtBeginEnvironment{exercise}{\small}
\AtBeginEnvironment{example}{\small}

\usepackage{cases}
\usepackage[red]{mypack}

\usepackage[framemethod=TikZ]{mdframed}

\definecolor{bg}{rgb}{0.4,0.25,0.95}
\definecolor{pagebg}{rgb}{0,0,0.5}
\surroundwithmdframed[
   topline=false,
   rightline=false,
   bottomline=false,
   leftmargin=\parindent,
   skipabove=8pt,
   skipbelow=8pt,
   linecolor=blue,
   innerbottommargin=10pt,
   % backgroundcolor=bg,font=\color{orange}\sffamily, fontcolor=white
]{definition}

\usepackage{empheq}
\usepackage[most]{tcolorbox}

\newtcbox{\mymath}[1][]{%
    nobeforeafter, math upper, tcbox raise base,
    enhanced, colframe=blue!30!black,
    colback=red!10, boxrule=1pt,
    #1}

\usepackage{unixode}


\DeclareMathOperator{\ord}{ord}
\DeclareMathOperator{\orb}{orb}
\DeclareMathOperator{\stab}{stab}
\DeclareMathOperator{\Stab}{stab}
\DeclareMathOperator{\ppcm}{ppcm}
\DeclareMathOperator{\conj}{Conj}
\DeclareMathOperator{\End}{End}
\DeclareMathOperator{\rot}{rot}
\DeclareMathOperator{\trs}{trace}
\DeclareMathOperator{\Ind}{Ind}
\DeclareMathOperator{\mat}{Mat}
\DeclareMathOperator{\id}{Id}
\DeclareMathOperator{\vect}{vect}
\DeclareMathOperator{\img}{img}
\DeclareMathOperator{\cov}{Cov}
\DeclareMathOperator{\dist}{dist}
\DeclareMathOperator{\irr}{Irr}
\DeclareMathOperator{\image}{Im}
\DeclareMathOperator{\pd}{\partial}
\DeclareMathOperator{\epi}{epi}
\DeclareMathOperator{\Argmin}{Argmin}
\DeclareMathOperator{\dom}{dom}
\DeclareMathOperator{\proj}{proj}
\DeclareMathOperator{\ctg}{ctg}
\DeclareMathOperator{\supp}{supp}
\DeclareMathOperator{\argmin}{argmin}
\DeclareMathOperator{\mult}{mult}
\DeclareMathOperator{\ch}{ch}
\DeclareMathOperator{\sh}{sh}
\DeclareMathOperator{\rang}{rang}
\DeclareMathOperator{\diam}{diam}
\DeclareMathOperator{\Epigraphe}{Epigraphe}




\usepackage{xcolor}
\everymath{\color{blue}}
%\everymath{\color[rgb]{0,1,1}}
%\pagecolor[rgb]{0,0,0.5}


\newcommand*{\pdtest}[3][]{\ensuremath{\frac{\partial^{#1} #2}{\partial #3}}}

\newcommand*{\deffunc}[6][]{\ensuremath{
\begin{array}{rcl}
#2 : #3 &\rightarrow& #4\\
#5 &\mapsto& #6
\end{array}
}}

\newcommand{\eqcolon}{\mathrel{\resizebox{\widthof{$\mathord{=}$}}{\height}{ $\!\!=\!\!\resizebox{1.2\width}{0.8\height}{\raisebox{0.23ex}{$\mathop{:}$}}\!\!$ }}}
\newcommand{\coloneq}{\mathrel{\resizebox{\widthof{$\mathord{=}$}}{\height}{ $\!\!\resizebox{1.2\width}{0.8\height}{\raisebox{0.23ex}{$\mathop{:}$}}\!\!=\!\!$ }}}
\newcommand{\eqcolonl}{\ensuremath{\mathrel{=\!\!\mathop{:}}}}
\newcommand{\coloneql}{\ensuremath{\mathrel{\mathop{:} \!\! =}}}
\newcommand{\vc}[1]{% inline column vector
  \left(\begin{smallmatrix}#1\end{smallmatrix}\right)%
}
\newcommand{\vr}[1]{% inline row vector
  \begin{smallmatrix}(\,#1\,)\end{smallmatrix}%
}
\makeatletter
\newcommand*{\defeq}{\ =\mathrel{\rlap{%
                     \raisebox{0.3ex}{$\m@th\cdot$}}%
                     \raisebox{-0.3ex}{$\m@th\cdot$}}%
                     }
\makeatother

\newcommand{\mathcircle}[1]{% inline row vector
 \overset{\circ}{#1}
}
\newcommand{\ulim}{% low limit
 \underline{\lim}
}
\newcommand{\ssi}{% iff
\iff
}
\newcommand{\ps}[2]{
\expval{#1 | #2}
}
\newcommand{\df}[1]{
\mqty{#1}
}
\newcommand{\n}[1]{
\norm{#1}
}
\newcommand{\sys}[1]{
\left\{\smqty{#1}\right.
}


\newcommand{\eqdef}{\ensuremath{\overset{\text{def}}=}}


\def\Circlearrowright{\ensuremath{%
  \rotatebox[origin=c]{230}{$\circlearrowright$}}}

\newcommand\ct[1]{\text{\rmfamily\upshape #1}}
\newcommand\question[1]{ {\color{red} ...!? \small #1}}
\newcommand\caz[1]{\left\{\begin{array} #1 \end{array}\right.}
\newcommand\const{\text{\rmfamily\upshape const}}
\newcommand\toP{ \overset{\pro}{\to}}
\newcommand\toPP{ \overset{\text{PP}}{\to}}
\newcommand{\oeq}{\mathrel{\text{\textcircled{$=$}}}}





\usepackage{xcolor}
% \usepackage[normalem]{ulem}
\usepackage{lipsum}
\makeatletter
% \newcommand\colorwave[1][blue]{\bgroup \markoverwith{\lower3.5\p@\hbox{\sixly \textcolor{#1}{\char58}}}\ULon}
%\font\sixly=lasy6 % does not re-load if already loaded, so no memory problem.

\newmdtheoremenv[
linewidth= 1pt,linecolor= blue,%
leftmargin=20,rightmargin=20,innertopmargin=0pt, innerrightmargin=40,%
tikzsetting = { draw=lightgray, line width = 0.3pt,dashed,%
dash pattern = on 15pt off 3pt},%
splittopskip=\topskip,skipbelow=\baselineskip,%
skipabove=\baselineskip,ntheorem,roundcorner=0pt,
% backgroundcolor=pagebg,font=\color{orange}\sffamily, fontcolor=white
]{examplebox}{Exemple}[section]



\newcommand\R{\mathbb{R}}
\newcommand\Z{\mathbb{Z}}
\newcommand\N{\mathbb{N}}
\newcommand\E{\mathbb{E}}
\newcommand\F{\mathcal{F}}
\newcommand\cH{\mathcal{H}}
\newcommand\V{\mathbb{V}}
\newcommand\dmo{ ^{-1} }
\newcommand\kapa{\kappa}
\newcommand\im{Im}
\newcommand\hs{\mathcal{H}}





\usepackage{soul}

\makeatletter
\newcommand*{\whiten}[1]{\llap{\textcolor{white}{{\the\SOUL@token}}\hspace{#1pt}}}
\DeclareRobustCommand*\myul{%
    \def\SOUL@everyspace{\underline{\space}\kern\z@}%
    \def\SOUL@everytoken{%
     \setbox0=\hbox{\the\SOUL@token}%
     \ifdim\dp0>\z@
        \raisebox{\dp0}{\underline{\phantom{\the\SOUL@token}}}%
        \whiten{1}\whiten{0}%
        \whiten{-1}\whiten{-2}%
        \llap{\the\SOUL@token}%
     \else
        \underline{\the\SOUL@token}%
     \fi}%
\SOUL@}
\makeatother

\newcommand*{\demp}{\fontfamily{lmtt}\selectfont}

\DeclareTextFontCommand{\textdemp}{\demp}

\begin{document}

\ifcomment
Multiline
comment
\fi
\ifcomment
\myul{Typesetting test}
% \color[rgb]{1,1,1}
$∑_i^n≠ 60º±∞π∆¬≈√j∫h≤≥µ$

$\CR \R\pro\ind\pro\gS\pro
\mqty[a&b\\c&d]$
$\pro\mathbb{P}$
$\dd{x}$

  \[
    \alpha(x)=\left\{
                \begin{array}{ll}
                  x\\
                  \frac{1}{1+e^{-kx}}\\
                  \frac{e^x-e^{-x}}{e^x+e^{-x}}
                \end{array}
              \right.
  \]

  $\expval{x}$
  
  $\chi_\rho(ghg\dmo)=\Tr(\rho_{ghg\dmo})=\Tr(\rho_g\circ\rho_h\circ\rho\dmo_g)=\Tr(\rho_h)\overset{\mbox{\scalebox{0.5}{$\Tr(AB)=\Tr(BA)$}}}{=}\chi_\rho(h)$
  	$\mathop{\oplus}_{\substack{x\in X}}$

$\mat(\rho_g)=(a_{ij}(g))_{\scriptsize \substack{1\leq i\leq d \\ 1\leq j\leq d}}$ et $\mat(\rho'_g)=(a'_{ij}(g))_{\scriptsize \substack{1\leq i'\leq d' \\ 1\leq j'\leq d'}}$



\[\int_a^b{\mathbb{R}^2}g(u, v)\dd{P_{XY}}(u, v)=\iint g(u,v) f_{XY}(u, v)\dd \lambda(u) \dd \lambda(v)\]
$$\lim_{x\to\infty} f(x)$$	
$$\iiiint_V \mu(t,u,v,w) \,dt\,du\,dv\,dw$$
$$\sum_{n=1}^{\infty} 2^{-n} = 1$$	
\begin{definition}
	Si $X$ et $Y$ sont 2 v.a. ou definit la \textsc{Covariance} entre $X$ et $Y$ comme
	$\cov(X,Y)\overset{\text{def}}{=}\E\left[(X-\E(X))(Y-\E(Y))\right]=\E(XY)-\E(X)\E(Y)$.
\end{definition}
\fi
\pagebreak

% \tableofcontents

% insert your code here
%\input{./algebra/main.tex}
%\input{./geometrie-differentielle/main.tex}
%\input{./probabilite/main.tex}
%\input{./analyse-fonctionnelle/main.tex}
% \input{./Analyse-convexe-et-dualite-en-optimisation/main.tex}
%\input{./tikz/main.tex}
%\input{./Theorie-du-distributions/main.tex}
%\input{./optimisation/mine.tex}
 \input{./modelisation/main.tex}

% yves.aubry@univ-tln.fr : algebra

\end{document}

%% !TEX encoding = UTF-8 Unicode
% !TEX TS-program = xelatex

\documentclass[french]{report}

%\usepackage[utf8]{inputenc}
%\usepackage[T1]{fontenc}
\usepackage{babel}


\newif\ifcomment
%\commenttrue # Show comments

\usepackage{physics}
\usepackage{amssymb}


\usepackage{amsthm}
% \usepackage{thmtools}
\usepackage{mathtools}
\usepackage{amsfonts}

\usepackage{color}

\usepackage{tikz}

\usepackage{geometry}
\geometry{a5paper, margin=0.1in, right=1cm}

\usepackage{dsfont}

\usepackage{graphicx}
\graphicspath{ {images/} }

\usepackage{faktor}

\usepackage{IEEEtrantools}
\usepackage{enumerate}   
\usepackage[PostScript=dvips]{"/Users/aware/Documents/Courses/diagrams"}


\newtheorem{theorem}{Théorème}[section]
\renewcommand{\thetheorem}{\arabic{theorem}}
\newtheorem{lemme}{Lemme}[section]
\renewcommand{\thelemme}{\arabic{lemme}}
\newtheorem{proposition}{Proposition}[section]
\renewcommand{\theproposition}{\arabic{proposition}}
\newtheorem{notations}{Notations}[section]
\newtheorem{problem}{Problème}[section]
\newtheorem{corollary}{Corollaire}[theorem]
\renewcommand{\thecorollary}{\arabic{corollary}}
\newtheorem{property}{Propriété}[section]
\newtheorem{objective}{Objectif}[section]

\theoremstyle{definition}
\newtheorem{definition}{Définition}[section]
\renewcommand{\thedefinition}{\arabic{definition}}
\newtheorem{exercise}{Exercice}[chapter]
\renewcommand{\theexercise}{\arabic{exercise}}
\newtheorem{example}{Exemple}[chapter]
\renewcommand{\theexample}{\arabic{example}}
\newtheorem*{solution}{Solution}
\newtheorem*{application}{Application}
\newtheorem*{notation}{Notation}
\newtheorem*{vocabulary}{Vocabulaire}
\newtheorem*{properties}{Propriétés}



\theoremstyle{remark}
\newtheorem*{remark}{Remarque}
\newtheorem*{rappel}{Rappel}


\usepackage{etoolbox}
\AtBeginEnvironment{exercise}{\small}
\AtBeginEnvironment{example}{\small}

\usepackage{cases}
\usepackage[red]{mypack}

\usepackage[framemethod=TikZ]{mdframed}

\definecolor{bg}{rgb}{0.4,0.25,0.95}
\definecolor{pagebg}{rgb}{0,0,0.5}
\surroundwithmdframed[
   topline=false,
   rightline=false,
   bottomline=false,
   leftmargin=\parindent,
   skipabove=8pt,
   skipbelow=8pt,
   linecolor=blue,
   innerbottommargin=10pt,
   % backgroundcolor=bg,font=\color{orange}\sffamily, fontcolor=white
]{definition}

\usepackage{empheq}
\usepackage[most]{tcolorbox}

\newtcbox{\mymath}[1][]{%
    nobeforeafter, math upper, tcbox raise base,
    enhanced, colframe=blue!30!black,
    colback=red!10, boxrule=1pt,
    #1}

\usepackage{unixode}


\DeclareMathOperator{\ord}{ord}
\DeclareMathOperator{\orb}{orb}
\DeclareMathOperator{\stab}{stab}
\DeclareMathOperator{\Stab}{stab}
\DeclareMathOperator{\ppcm}{ppcm}
\DeclareMathOperator{\conj}{Conj}
\DeclareMathOperator{\End}{End}
\DeclareMathOperator{\rot}{rot}
\DeclareMathOperator{\trs}{trace}
\DeclareMathOperator{\Ind}{Ind}
\DeclareMathOperator{\mat}{Mat}
\DeclareMathOperator{\id}{Id}
\DeclareMathOperator{\vect}{vect}
\DeclareMathOperator{\img}{img}
\DeclareMathOperator{\cov}{Cov}
\DeclareMathOperator{\dist}{dist}
\DeclareMathOperator{\irr}{Irr}
\DeclareMathOperator{\image}{Im}
\DeclareMathOperator{\pd}{\partial}
\DeclareMathOperator{\epi}{epi}
\DeclareMathOperator{\Argmin}{Argmin}
\DeclareMathOperator{\dom}{dom}
\DeclareMathOperator{\proj}{proj}
\DeclareMathOperator{\ctg}{ctg}
\DeclareMathOperator{\supp}{supp}
\DeclareMathOperator{\argmin}{argmin}
\DeclareMathOperator{\mult}{mult}
\DeclareMathOperator{\ch}{ch}
\DeclareMathOperator{\sh}{sh}
\DeclareMathOperator{\rang}{rang}
\DeclareMathOperator{\diam}{diam}
\DeclareMathOperator{\Epigraphe}{Epigraphe}




\usepackage{xcolor}
\everymath{\color{blue}}
%\everymath{\color[rgb]{0,1,1}}
%\pagecolor[rgb]{0,0,0.5}


\newcommand*{\pdtest}[3][]{\ensuremath{\frac{\partial^{#1} #2}{\partial #3}}}

\newcommand*{\deffunc}[6][]{\ensuremath{
\begin{array}{rcl}
#2 : #3 &\rightarrow& #4\\
#5 &\mapsto& #6
\end{array}
}}

\newcommand{\eqcolon}{\mathrel{\resizebox{\widthof{$\mathord{=}$}}{\height}{ $\!\!=\!\!\resizebox{1.2\width}{0.8\height}{\raisebox{0.23ex}{$\mathop{:}$}}\!\!$ }}}
\newcommand{\coloneq}{\mathrel{\resizebox{\widthof{$\mathord{=}$}}{\height}{ $\!\!\resizebox{1.2\width}{0.8\height}{\raisebox{0.23ex}{$\mathop{:}$}}\!\!=\!\!$ }}}
\newcommand{\eqcolonl}{\ensuremath{\mathrel{=\!\!\mathop{:}}}}
\newcommand{\coloneql}{\ensuremath{\mathrel{\mathop{:} \!\! =}}}
\newcommand{\vc}[1]{% inline column vector
  \left(\begin{smallmatrix}#1\end{smallmatrix}\right)%
}
\newcommand{\vr}[1]{% inline row vector
  \begin{smallmatrix}(\,#1\,)\end{smallmatrix}%
}
\makeatletter
\newcommand*{\defeq}{\ =\mathrel{\rlap{%
                     \raisebox{0.3ex}{$\m@th\cdot$}}%
                     \raisebox{-0.3ex}{$\m@th\cdot$}}%
                     }
\makeatother

\newcommand{\mathcircle}[1]{% inline row vector
 \overset{\circ}{#1}
}
\newcommand{\ulim}{% low limit
 \underline{\lim}
}
\newcommand{\ssi}{% iff
\iff
}
\newcommand{\ps}[2]{
\expval{#1 | #2}
}
\newcommand{\df}[1]{
\mqty{#1}
}
\newcommand{\n}[1]{
\norm{#1}
}
\newcommand{\sys}[1]{
\left\{\smqty{#1}\right.
}


\newcommand{\eqdef}{\ensuremath{\overset{\text{def}}=}}


\def\Circlearrowright{\ensuremath{%
  \rotatebox[origin=c]{230}{$\circlearrowright$}}}

\newcommand\ct[1]{\text{\rmfamily\upshape #1}}
\newcommand\question[1]{ {\color{red} ...!? \small #1}}
\newcommand\caz[1]{\left\{\begin{array} #1 \end{array}\right.}
\newcommand\const{\text{\rmfamily\upshape const}}
\newcommand\toP{ \overset{\pro}{\to}}
\newcommand\toPP{ \overset{\text{PP}}{\to}}
\newcommand{\oeq}{\mathrel{\text{\textcircled{$=$}}}}





\usepackage{xcolor}
% \usepackage[normalem]{ulem}
\usepackage{lipsum}
\makeatletter
% \newcommand\colorwave[1][blue]{\bgroup \markoverwith{\lower3.5\p@\hbox{\sixly \textcolor{#1}{\char58}}}\ULon}
%\font\sixly=lasy6 % does not re-load if already loaded, so no memory problem.

\newmdtheoremenv[
linewidth= 1pt,linecolor= blue,%
leftmargin=20,rightmargin=20,innertopmargin=0pt, innerrightmargin=40,%
tikzsetting = { draw=lightgray, line width = 0.3pt,dashed,%
dash pattern = on 15pt off 3pt},%
splittopskip=\topskip,skipbelow=\baselineskip,%
skipabove=\baselineskip,ntheorem,roundcorner=0pt,
% backgroundcolor=pagebg,font=\color{orange}\sffamily, fontcolor=white
]{examplebox}{Exemple}[section]



\newcommand\R{\mathbb{R}}
\newcommand\Z{\mathbb{Z}}
\newcommand\N{\mathbb{N}}
\newcommand\E{\mathbb{E}}
\newcommand\F{\mathcal{F}}
\newcommand\cH{\mathcal{H}}
\newcommand\V{\mathbb{V}}
\newcommand\dmo{ ^{-1} }
\newcommand\kapa{\kappa}
\newcommand\im{Im}
\newcommand\hs{\mathcal{H}}





\usepackage{soul}

\makeatletter
\newcommand*{\whiten}[1]{\llap{\textcolor{white}{{\the\SOUL@token}}\hspace{#1pt}}}
\DeclareRobustCommand*\myul{%
    \def\SOUL@everyspace{\underline{\space}\kern\z@}%
    \def\SOUL@everytoken{%
     \setbox0=\hbox{\the\SOUL@token}%
     \ifdim\dp0>\z@
        \raisebox{\dp0}{\underline{\phantom{\the\SOUL@token}}}%
        \whiten{1}\whiten{0}%
        \whiten{-1}\whiten{-2}%
        \llap{\the\SOUL@token}%
     \else
        \underline{\the\SOUL@token}%
     \fi}%
\SOUL@}
\makeatother

\newcommand*{\demp}{\fontfamily{lmtt}\selectfont}

\DeclareTextFontCommand{\textdemp}{\demp}

\begin{document}

\ifcomment
Multiline
comment
\fi
\ifcomment
\myul{Typesetting test}
% \color[rgb]{1,1,1}
$∑_i^n≠ 60º±∞π∆¬≈√j∫h≤≥µ$

$\CR \R\pro\ind\pro\gS\pro
\mqty[a&b\\c&d]$
$\pro\mathbb{P}$
$\dd{x}$

  \[
    \alpha(x)=\left\{
                \begin{array}{ll}
                  x\\
                  \frac{1}{1+e^{-kx}}\\
                  \frac{e^x-e^{-x}}{e^x+e^{-x}}
                \end{array}
              \right.
  \]

  $\expval{x}$
  
  $\chi_\rho(ghg\dmo)=\Tr(\rho_{ghg\dmo})=\Tr(\rho_g\circ\rho_h\circ\rho\dmo_g)=\Tr(\rho_h)\overset{\mbox{\scalebox{0.5}{$\Tr(AB)=\Tr(BA)$}}}{=}\chi_\rho(h)$
  	$\mathop{\oplus}_{\substack{x\in X}}$

$\mat(\rho_g)=(a_{ij}(g))_{\scriptsize \substack{1\leq i\leq d \\ 1\leq j\leq d}}$ et $\mat(\rho'_g)=(a'_{ij}(g))_{\scriptsize \substack{1\leq i'\leq d' \\ 1\leq j'\leq d'}}$



\[\int_a^b{\mathbb{R}^2}g(u, v)\dd{P_{XY}}(u, v)=\iint g(u,v) f_{XY}(u, v)\dd \lambda(u) \dd \lambda(v)\]
$$\lim_{x\to\infty} f(x)$$	
$$\iiiint_V \mu(t,u,v,w) \,dt\,du\,dv\,dw$$
$$\sum_{n=1}^{\infty} 2^{-n} = 1$$	
\begin{definition}
	Si $X$ et $Y$ sont 2 v.a. ou definit la \textsc{Covariance} entre $X$ et $Y$ comme
	$\cov(X,Y)\overset{\text{def}}{=}\E\left[(X-\E(X))(Y-\E(Y))\right]=\E(XY)-\E(X)\E(Y)$.
\end{definition}
\fi
\pagebreak

% \tableofcontents

% insert your code here
%\input{./algebra/main.tex}
%\input{./geometrie-differentielle/main.tex}
%\input{./probabilite/main.tex}
%\input{./analyse-fonctionnelle/main.tex}
% \input{./Analyse-convexe-et-dualite-en-optimisation/main.tex}
%\input{./tikz/main.tex}
%\input{./Theorie-du-distributions/main.tex}
%\input{./optimisation/mine.tex}
 \input{./modelisation/main.tex}

% yves.aubry@univ-tln.fr : algebra

\end{document}

%% !TEX encoding = UTF-8 Unicode
% !TEX TS-program = xelatex

\documentclass[french]{report}

%\usepackage[utf8]{inputenc}
%\usepackage[T1]{fontenc}
\usepackage{babel}


\newif\ifcomment
%\commenttrue # Show comments

\usepackage{physics}
\usepackage{amssymb}


\usepackage{amsthm}
% \usepackage{thmtools}
\usepackage{mathtools}
\usepackage{amsfonts}

\usepackage{color}

\usepackage{tikz}

\usepackage{geometry}
\geometry{a5paper, margin=0.1in, right=1cm}

\usepackage{dsfont}

\usepackage{graphicx}
\graphicspath{ {images/} }

\usepackage{faktor}

\usepackage{IEEEtrantools}
\usepackage{enumerate}   
\usepackage[PostScript=dvips]{"/Users/aware/Documents/Courses/diagrams"}


\newtheorem{theorem}{Théorème}[section]
\renewcommand{\thetheorem}{\arabic{theorem}}
\newtheorem{lemme}{Lemme}[section]
\renewcommand{\thelemme}{\arabic{lemme}}
\newtheorem{proposition}{Proposition}[section]
\renewcommand{\theproposition}{\arabic{proposition}}
\newtheorem{notations}{Notations}[section]
\newtheorem{problem}{Problème}[section]
\newtheorem{corollary}{Corollaire}[theorem]
\renewcommand{\thecorollary}{\arabic{corollary}}
\newtheorem{property}{Propriété}[section]
\newtheorem{objective}{Objectif}[section]

\theoremstyle{definition}
\newtheorem{definition}{Définition}[section]
\renewcommand{\thedefinition}{\arabic{definition}}
\newtheorem{exercise}{Exercice}[chapter]
\renewcommand{\theexercise}{\arabic{exercise}}
\newtheorem{example}{Exemple}[chapter]
\renewcommand{\theexample}{\arabic{example}}
\newtheorem*{solution}{Solution}
\newtheorem*{application}{Application}
\newtheorem*{notation}{Notation}
\newtheorem*{vocabulary}{Vocabulaire}
\newtheorem*{properties}{Propriétés}



\theoremstyle{remark}
\newtheorem*{remark}{Remarque}
\newtheorem*{rappel}{Rappel}


\usepackage{etoolbox}
\AtBeginEnvironment{exercise}{\small}
\AtBeginEnvironment{example}{\small}

\usepackage{cases}
\usepackage[red]{mypack}

\usepackage[framemethod=TikZ]{mdframed}

\definecolor{bg}{rgb}{0.4,0.25,0.95}
\definecolor{pagebg}{rgb}{0,0,0.5}
\surroundwithmdframed[
   topline=false,
   rightline=false,
   bottomline=false,
   leftmargin=\parindent,
   skipabove=8pt,
   skipbelow=8pt,
   linecolor=blue,
   innerbottommargin=10pt,
   % backgroundcolor=bg,font=\color{orange}\sffamily, fontcolor=white
]{definition}

\usepackage{empheq}
\usepackage[most]{tcolorbox}

\newtcbox{\mymath}[1][]{%
    nobeforeafter, math upper, tcbox raise base,
    enhanced, colframe=blue!30!black,
    colback=red!10, boxrule=1pt,
    #1}

\usepackage{unixode}


\DeclareMathOperator{\ord}{ord}
\DeclareMathOperator{\orb}{orb}
\DeclareMathOperator{\stab}{stab}
\DeclareMathOperator{\Stab}{stab}
\DeclareMathOperator{\ppcm}{ppcm}
\DeclareMathOperator{\conj}{Conj}
\DeclareMathOperator{\End}{End}
\DeclareMathOperator{\rot}{rot}
\DeclareMathOperator{\trs}{trace}
\DeclareMathOperator{\Ind}{Ind}
\DeclareMathOperator{\mat}{Mat}
\DeclareMathOperator{\id}{Id}
\DeclareMathOperator{\vect}{vect}
\DeclareMathOperator{\img}{img}
\DeclareMathOperator{\cov}{Cov}
\DeclareMathOperator{\dist}{dist}
\DeclareMathOperator{\irr}{Irr}
\DeclareMathOperator{\image}{Im}
\DeclareMathOperator{\pd}{\partial}
\DeclareMathOperator{\epi}{epi}
\DeclareMathOperator{\Argmin}{Argmin}
\DeclareMathOperator{\dom}{dom}
\DeclareMathOperator{\proj}{proj}
\DeclareMathOperator{\ctg}{ctg}
\DeclareMathOperator{\supp}{supp}
\DeclareMathOperator{\argmin}{argmin}
\DeclareMathOperator{\mult}{mult}
\DeclareMathOperator{\ch}{ch}
\DeclareMathOperator{\sh}{sh}
\DeclareMathOperator{\rang}{rang}
\DeclareMathOperator{\diam}{diam}
\DeclareMathOperator{\Epigraphe}{Epigraphe}




\usepackage{xcolor}
\everymath{\color{blue}}
%\everymath{\color[rgb]{0,1,1}}
%\pagecolor[rgb]{0,0,0.5}


\newcommand*{\pdtest}[3][]{\ensuremath{\frac{\partial^{#1} #2}{\partial #3}}}

\newcommand*{\deffunc}[6][]{\ensuremath{
\begin{array}{rcl}
#2 : #3 &\rightarrow& #4\\
#5 &\mapsto& #6
\end{array}
}}

\newcommand{\eqcolon}{\mathrel{\resizebox{\widthof{$\mathord{=}$}}{\height}{ $\!\!=\!\!\resizebox{1.2\width}{0.8\height}{\raisebox{0.23ex}{$\mathop{:}$}}\!\!$ }}}
\newcommand{\coloneq}{\mathrel{\resizebox{\widthof{$\mathord{=}$}}{\height}{ $\!\!\resizebox{1.2\width}{0.8\height}{\raisebox{0.23ex}{$\mathop{:}$}}\!\!=\!\!$ }}}
\newcommand{\eqcolonl}{\ensuremath{\mathrel{=\!\!\mathop{:}}}}
\newcommand{\coloneql}{\ensuremath{\mathrel{\mathop{:} \!\! =}}}
\newcommand{\vc}[1]{% inline column vector
  \left(\begin{smallmatrix}#1\end{smallmatrix}\right)%
}
\newcommand{\vr}[1]{% inline row vector
  \begin{smallmatrix}(\,#1\,)\end{smallmatrix}%
}
\makeatletter
\newcommand*{\defeq}{\ =\mathrel{\rlap{%
                     \raisebox{0.3ex}{$\m@th\cdot$}}%
                     \raisebox{-0.3ex}{$\m@th\cdot$}}%
                     }
\makeatother

\newcommand{\mathcircle}[1]{% inline row vector
 \overset{\circ}{#1}
}
\newcommand{\ulim}{% low limit
 \underline{\lim}
}
\newcommand{\ssi}{% iff
\iff
}
\newcommand{\ps}[2]{
\expval{#1 | #2}
}
\newcommand{\df}[1]{
\mqty{#1}
}
\newcommand{\n}[1]{
\norm{#1}
}
\newcommand{\sys}[1]{
\left\{\smqty{#1}\right.
}


\newcommand{\eqdef}{\ensuremath{\overset{\text{def}}=}}


\def\Circlearrowright{\ensuremath{%
  \rotatebox[origin=c]{230}{$\circlearrowright$}}}

\newcommand\ct[1]{\text{\rmfamily\upshape #1}}
\newcommand\question[1]{ {\color{red} ...!? \small #1}}
\newcommand\caz[1]{\left\{\begin{array} #1 \end{array}\right.}
\newcommand\const{\text{\rmfamily\upshape const}}
\newcommand\toP{ \overset{\pro}{\to}}
\newcommand\toPP{ \overset{\text{PP}}{\to}}
\newcommand{\oeq}{\mathrel{\text{\textcircled{$=$}}}}





\usepackage{xcolor}
% \usepackage[normalem]{ulem}
\usepackage{lipsum}
\makeatletter
% \newcommand\colorwave[1][blue]{\bgroup \markoverwith{\lower3.5\p@\hbox{\sixly \textcolor{#1}{\char58}}}\ULon}
%\font\sixly=lasy6 % does not re-load if already loaded, so no memory problem.

\newmdtheoremenv[
linewidth= 1pt,linecolor= blue,%
leftmargin=20,rightmargin=20,innertopmargin=0pt, innerrightmargin=40,%
tikzsetting = { draw=lightgray, line width = 0.3pt,dashed,%
dash pattern = on 15pt off 3pt},%
splittopskip=\topskip,skipbelow=\baselineskip,%
skipabove=\baselineskip,ntheorem,roundcorner=0pt,
% backgroundcolor=pagebg,font=\color{orange}\sffamily, fontcolor=white
]{examplebox}{Exemple}[section]



\newcommand\R{\mathbb{R}}
\newcommand\Z{\mathbb{Z}}
\newcommand\N{\mathbb{N}}
\newcommand\E{\mathbb{E}}
\newcommand\F{\mathcal{F}}
\newcommand\cH{\mathcal{H}}
\newcommand\V{\mathbb{V}}
\newcommand\dmo{ ^{-1} }
\newcommand\kapa{\kappa}
\newcommand\im{Im}
\newcommand\hs{\mathcal{H}}





\usepackage{soul}

\makeatletter
\newcommand*{\whiten}[1]{\llap{\textcolor{white}{{\the\SOUL@token}}\hspace{#1pt}}}
\DeclareRobustCommand*\myul{%
    \def\SOUL@everyspace{\underline{\space}\kern\z@}%
    \def\SOUL@everytoken{%
     \setbox0=\hbox{\the\SOUL@token}%
     \ifdim\dp0>\z@
        \raisebox{\dp0}{\underline{\phantom{\the\SOUL@token}}}%
        \whiten{1}\whiten{0}%
        \whiten{-1}\whiten{-2}%
        \llap{\the\SOUL@token}%
     \else
        \underline{\the\SOUL@token}%
     \fi}%
\SOUL@}
\makeatother

\newcommand*{\demp}{\fontfamily{lmtt}\selectfont}

\DeclareTextFontCommand{\textdemp}{\demp}

\begin{document}

\ifcomment
Multiline
comment
\fi
\ifcomment
\myul{Typesetting test}
% \color[rgb]{1,1,1}
$∑_i^n≠ 60º±∞π∆¬≈√j∫h≤≥µ$

$\CR \R\pro\ind\pro\gS\pro
\mqty[a&b\\c&d]$
$\pro\mathbb{P}$
$\dd{x}$

  \[
    \alpha(x)=\left\{
                \begin{array}{ll}
                  x\\
                  \frac{1}{1+e^{-kx}}\\
                  \frac{e^x-e^{-x}}{e^x+e^{-x}}
                \end{array}
              \right.
  \]

  $\expval{x}$
  
  $\chi_\rho(ghg\dmo)=\Tr(\rho_{ghg\dmo})=\Tr(\rho_g\circ\rho_h\circ\rho\dmo_g)=\Tr(\rho_h)\overset{\mbox{\scalebox{0.5}{$\Tr(AB)=\Tr(BA)$}}}{=}\chi_\rho(h)$
  	$\mathop{\oplus}_{\substack{x\in X}}$

$\mat(\rho_g)=(a_{ij}(g))_{\scriptsize \substack{1\leq i\leq d \\ 1\leq j\leq d}}$ et $\mat(\rho'_g)=(a'_{ij}(g))_{\scriptsize \substack{1\leq i'\leq d' \\ 1\leq j'\leq d'}}$



\[\int_a^b{\mathbb{R}^2}g(u, v)\dd{P_{XY}}(u, v)=\iint g(u,v) f_{XY}(u, v)\dd \lambda(u) \dd \lambda(v)\]
$$\lim_{x\to\infty} f(x)$$	
$$\iiiint_V \mu(t,u,v,w) \,dt\,du\,dv\,dw$$
$$\sum_{n=1}^{\infty} 2^{-n} = 1$$	
\begin{definition}
	Si $X$ et $Y$ sont 2 v.a. ou definit la \textsc{Covariance} entre $X$ et $Y$ comme
	$\cov(X,Y)\overset{\text{def}}{=}\E\left[(X-\E(X))(Y-\E(Y))\right]=\E(XY)-\E(X)\E(Y)$.
\end{definition}
\fi
\pagebreak

% \tableofcontents

% insert your code here
%\input{./algebra/main.tex}
%\input{./geometrie-differentielle/main.tex}
%\input{./probabilite/main.tex}
%\input{./analyse-fonctionnelle/main.tex}
% \input{./Analyse-convexe-et-dualite-en-optimisation/main.tex}
%\input{./tikz/main.tex}
%\input{./Theorie-du-distributions/main.tex}
%\input{./optimisation/mine.tex}
 \input{./modelisation/main.tex}

% yves.aubry@univ-tln.fr : algebra

\end{document}

% % !TEX encoding = UTF-8 Unicode
% !TEX TS-program = xelatex

\documentclass[french]{report}

%\usepackage[utf8]{inputenc}
%\usepackage[T1]{fontenc}
\usepackage{babel}


\newif\ifcomment
%\commenttrue # Show comments

\usepackage{physics}
\usepackage{amssymb}


\usepackage{amsthm}
% \usepackage{thmtools}
\usepackage{mathtools}
\usepackage{amsfonts}

\usepackage{color}

\usepackage{tikz}

\usepackage{geometry}
\geometry{a5paper, margin=0.1in, right=1cm}

\usepackage{dsfont}

\usepackage{graphicx}
\graphicspath{ {images/} }

\usepackage{faktor}

\usepackage{IEEEtrantools}
\usepackage{enumerate}   
\usepackage[PostScript=dvips]{"/Users/aware/Documents/Courses/diagrams"}


\newtheorem{theorem}{Théorème}[section]
\renewcommand{\thetheorem}{\arabic{theorem}}
\newtheorem{lemme}{Lemme}[section]
\renewcommand{\thelemme}{\arabic{lemme}}
\newtheorem{proposition}{Proposition}[section]
\renewcommand{\theproposition}{\arabic{proposition}}
\newtheorem{notations}{Notations}[section]
\newtheorem{problem}{Problème}[section]
\newtheorem{corollary}{Corollaire}[theorem]
\renewcommand{\thecorollary}{\arabic{corollary}}
\newtheorem{property}{Propriété}[section]
\newtheorem{objective}{Objectif}[section]

\theoremstyle{definition}
\newtheorem{definition}{Définition}[section]
\renewcommand{\thedefinition}{\arabic{definition}}
\newtheorem{exercise}{Exercice}[chapter]
\renewcommand{\theexercise}{\arabic{exercise}}
\newtheorem{example}{Exemple}[chapter]
\renewcommand{\theexample}{\arabic{example}}
\newtheorem*{solution}{Solution}
\newtheorem*{application}{Application}
\newtheorem*{notation}{Notation}
\newtheorem*{vocabulary}{Vocabulaire}
\newtheorem*{properties}{Propriétés}



\theoremstyle{remark}
\newtheorem*{remark}{Remarque}
\newtheorem*{rappel}{Rappel}


\usepackage{etoolbox}
\AtBeginEnvironment{exercise}{\small}
\AtBeginEnvironment{example}{\small}

\usepackage{cases}
\usepackage[red]{mypack}

\usepackage[framemethod=TikZ]{mdframed}

\definecolor{bg}{rgb}{0.4,0.25,0.95}
\definecolor{pagebg}{rgb}{0,0,0.5}
\surroundwithmdframed[
   topline=false,
   rightline=false,
   bottomline=false,
   leftmargin=\parindent,
   skipabove=8pt,
   skipbelow=8pt,
   linecolor=blue,
   innerbottommargin=10pt,
   % backgroundcolor=bg,font=\color{orange}\sffamily, fontcolor=white
]{definition}

\usepackage{empheq}
\usepackage[most]{tcolorbox}

\newtcbox{\mymath}[1][]{%
    nobeforeafter, math upper, tcbox raise base,
    enhanced, colframe=blue!30!black,
    colback=red!10, boxrule=1pt,
    #1}

\usepackage{unixode}


\DeclareMathOperator{\ord}{ord}
\DeclareMathOperator{\orb}{orb}
\DeclareMathOperator{\stab}{stab}
\DeclareMathOperator{\Stab}{stab}
\DeclareMathOperator{\ppcm}{ppcm}
\DeclareMathOperator{\conj}{Conj}
\DeclareMathOperator{\End}{End}
\DeclareMathOperator{\rot}{rot}
\DeclareMathOperator{\trs}{trace}
\DeclareMathOperator{\Ind}{Ind}
\DeclareMathOperator{\mat}{Mat}
\DeclareMathOperator{\id}{Id}
\DeclareMathOperator{\vect}{vect}
\DeclareMathOperator{\img}{img}
\DeclareMathOperator{\cov}{Cov}
\DeclareMathOperator{\dist}{dist}
\DeclareMathOperator{\irr}{Irr}
\DeclareMathOperator{\image}{Im}
\DeclareMathOperator{\pd}{\partial}
\DeclareMathOperator{\epi}{epi}
\DeclareMathOperator{\Argmin}{Argmin}
\DeclareMathOperator{\dom}{dom}
\DeclareMathOperator{\proj}{proj}
\DeclareMathOperator{\ctg}{ctg}
\DeclareMathOperator{\supp}{supp}
\DeclareMathOperator{\argmin}{argmin}
\DeclareMathOperator{\mult}{mult}
\DeclareMathOperator{\ch}{ch}
\DeclareMathOperator{\sh}{sh}
\DeclareMathOperator{\rang}{rang}
\DeclareMathOperator{\diam}{diam}
\DeclareMathOperator{\Epigraphe}{Epigraphe}




\usepackage{xcolor}
\everymath{\color{blue}}
%\everymath{\color[rgb]{0,1,1}}
%\pagecolor[rgb]{0,0,0.5}


\newcommand*{\pdtest}[3][]{\ensuremath{\frac{\partial^{#1} #2}{\partial #3}}}

\newcommand*{\deffunc}[6][]{\ensuremath{
\begin{array}{rcl}
#2 : #3 &\rightarrow& #4\\
#5 &\mapsto& #6
\end{array}
}}

\newcommand{\eqcolon}{\mathrel{\resizebox{\widthof{$\mathord{=}$}}{\height}{ $\!\!=\!\!\resizebox{1.2\width}{0.8\height}{\raisebox{0.23ex}{$\mathop{:}$}}\!\!$ }}}
\newcommand{\coloneq}{\mathrel{\resizebox{\widthof{$\mathord{=}$}}{\height}{ $\!\!\resizebox{1.2\width}{0.8\height}{\raisebox{0.23ex}{$\mathop{:}$}}\!\!=\!\!$ }}}
\newcommand{\eqcolonl}{\ensuremath{\mathrel{=\!\!\mathop{:}}}}
\newcommand{\coloneql}{\ensuremath{\mathrel{\mathop{:} \!\! =}}}
\newcommand{\vc}[1]{% inline column vector
  \left(\begin{smallmatrix}#1\end{smallmatrix}\right)%
}
\newcommand{\vr}[1]{% inline row vector
  \begin{smallmatrix}(\,#1\,)\end{smallmatrix}%
}
\makeatletter
\newcommand*{\defeq}{\ =\mathrel{\rlap{%
                     \raisebox{0.3ex}{$\m@th\cdot$}}%
                     \raisebox{-0.3ex}{$\m@th\cdot$}}%
                     }
\makeatother

\newcommand{\mathcircle}[1]{% inline row vector
 \overset{\circ}{#1}
}
\newcommand{\ulim}{% low limit
 \underline{\lim}
}
\newcommand{\ssi}{% iff
\iff
}
\newcommand{\ps}[2]{
\expval{#1 | #2}
}
\newcommand{\df}[1]{
\mqty{#1}
}
\newcommand{\n}[1]{
\norm{#1}
}
\newcommand{\sys}[1]{
\left\{\smqty{#1}\right.
}


\newcommand{\eqdef}{\ensuremath{\overset{\text{def}}=}}


\def\Circlearrowright{\ensuremath{%
  \rotatebox[origin=c]{230}{$\circlearrowright$}}}

\newcommand\ct[1]{\text{\rmfamily\upshape #1}}
\newcommand\question[1]{ {\color{red} ...!? \small #1}}
\newcommand\caz[1]{\left\{\begin{array} #1 \end{array}\right.}
\newcommand\const{\text{\rmfamily\upshape const}}
\newcommand\toP{ \overset{\pro}{\to}}
\newcommand\toPP{ \overset{\text{PP}}{\to}}
\newcommand{\oeq}{\mathrel{\text{\textcircled{$=$}}}}





\usepackage{xcolor}
% \usepackage[normalem]{ulem}
\usepackage{lipsum}
\makeatletter
% \newcommand\colorwave[1][blue]{\bgroup \markoverwith{\lower3.5\p@\hbox{\sixly \textcolor{#1}{\char58}}}\ULon}
%\font\sixly=lasy6 % does not re-load if already loaded, so no memory problem.

\newmdtheoremenv[
linewidth= 1pt,linecolor= blue,%
leftmargin=20,rightmargin=20,innertopmargin=0pt, innerrightmargin=40,%
tikzsetting = { draw=lightgray, line width = 0.3pt,dashed,%
dash pattern = on 15pt off 3pt},%
splittopskip=\topskip,skipbelow=\baselineskip,%
skipabove=\baselineskip,ntheorem,roundcorner=0pt,
% backgroundcolor=pagebg,font=\color{orange}\sffamily, fontcolor=white
]{examplebox}{Exemple}[section]



\newcommand\R{\mathbb{R}}
\newcommand\Z{\mathbb{Z}}
\newcommand\N{\mathbb{N}}
\newcommand\E{\mathbb{E}}
\newcommand\F{\mathcal{F}}
\newcommand\cH{\mathcal{H}}
\newcommand\V{\mathbb{V}}
\newcommand\dmo{ ^{-1} }
\newcommand\kapa{\kappa}
\newcommand\im{Im}
\newcommand\hs{\mathcal{H}}





\usepackage{soul}

\makeatletter
\newcommand*{\whiten}[1]{\llap{\textcolor{white}{{\the\SOUL@token}}\hspace{#1pt}}}
\DeclareRobustCommand*\myul{%
    \def\SOUL@everyspace{\underline{\space}\kern\z@}%
    \def\SOUL@everytoken{%
     \setbox0=\hbox{\the\SOUL@token}%
     \ifdim\dp0>\z@
        \raisebox{\dp0}{\underline{\phantom{\the\SOUL@token}}}%
        \whiten{1}\whiten{0}%
        \whiten{-1}\whiten{-2}%
        \llap{\the\SOUL@token}%
     \else
        \underline{\the\SOUL@token}%
     \fi}%
\SOUL@}
\makeatother

\newcommand*{\demp}{\fontfamily{lmtt}\selectfont}

\DeclareTextFontCommand{\textdemp}{\demp}

\begin{document}

\ifcomment
Multiline
comment
\fi
\ifcomment
\myul{Typesetting test}
% \color[rgb]{1,1,1}
$∑_i^n≠ 60º±∞π∆¬≈√j∫h≤≥µ$

$\CR \R\pro\ind\pro\gS\pro
\mqty[a&b\\c&d]$
$\pro\mathbb{P}$
$\dd{x}$

  \[
    \alpha(x)=\left\{
                \begin{array}{ll}
                  x\\
                  \frac{1}{1+e^{-kx}}\\
                  \frac{e^x-e^{-x}}{e^x+e^{-x}}
                \end{array}
              \right.
  \]

  $\expval{x}$
  
  $\chi_\rho(ghg\dmo)=\Tr(\rho_{ghg\dmo})=\Tr(\rho_g\circ\rho_h\circ\rho\dmo_g)=\Tr(\rho_h)\overset{\mbox{\scalebox{0.5}{$\Tr(AB)=\Tr(BA)$}}}{=}\chi_\rho(h)$
  	$\mathop{\oplus}_{\substack{x\in X}}$

$\mat(\rho_g)=(a_{ij}(g))_{\scriptsize \substack{1\leq i\leq d \\ 1\leq j\leq d}}$ et $\mat(\rho'_g)=(a'_{ij}(g))_{\scriptsize \substack{1\leq i'\leq d' \\ 1\leq j'\leq d'}}$



\[\int_a^b{\mathbb{R}^2}g(u, v)\dd{P_{XY}}(u, v)=\iint g(u,v) f_{XY}(u, v)\dd \lambda(u) \dd \lambda(v)\]
$$\lim_{x\to\infty} f(x)$$	
$$\iiiint_V \mu(t,u,v,w) \,dt\,du\,dv\,dw$$
$$\sum_{n=1}^{\infty} 2^{-n} = 1$$	
\begin{definition}
	Si $X$ et $Y$ sont 2 v.a. ou definit la \textsc{Covariance} entre $X$ et $Y$ comme
	$\cov(X,Y)\overset{\text{def}}{=}\E\left[(X-\E(X))(Y-\E(Y))\right]=\E(XY)-\E(X)\E(Y)$.
\end{definition}
\fi
\pagebreak

% \tableofcontents

% insert your code here
%\input{./algebra/main.tex}
%\input{./geometrie-differentielle/main.tex}
%\input{./probabilite/main.tex}
%\input{./analyse-fonctionnelle/main.tex}
% \input{./Analyse-convexe-et-dualite-en-optimisation/main.tex}
%\input{./tikz/main.tex}
%\input{./Theorie-du-distributions/main.tex}
%\input{./optimisation/mine.tex}
 \input{./modelisation/main.tex}

% yves.aubry@univ-tln.fr : algebra

\end{document}

%% !TEX encoding = UTF-8 Unicode
% !TEX TS-program = xelatex

\documentclass[french]{report}

%\usepackage[utf8]{inputenc}
%\usepackage[T1]{fontenc}
\usepackage{babel}


\newif\ifcomment
%\commenttrue # Show comments

\usepackage{physics}
\usepackage{amssymb}


\usepackage{amsthm}
% \usepackage{thmtools}
\usepackage{mathtools}
\usepackage{amsfonts}

\usepackage{color}

\usepackage{tikz}

\usepackage{geometry}
\geometry{a5paper, margin=0.1in, right=1cm}

\usepackage{dsfont}

\usepackage{graphicx}
\graphicspath{ {images/} }

\usepackage{faktor}

\usepackage{IEEEtrantools}
\usepackage{enumerate}   
\usepackage[PostScript=dvips]{"/Users/aware/Documents/Courses/diagrams"}


\newtheorem{theorem}{Théorème}[section]
\renewcommand{\thetheorem}{\arabic{theorem}}
\newtheorem{lemme}{Lemme}[section]
\renewcommand{\thelemme}{\arabic{lemme}}
\newtheorem{proposition}{Proposition}[section]
\renewcommand{\theproposition}{\arabic{proposition}}
\newtheorem{notations}{Notations}[section]
\newtheorem{problem}{Problème}[section]
\newtheorem{corollary}{Corollaire}[theorem]
\renewcommand{\thecorollary}{\arabic{corollary}}
\newtheorem{property}{Propriété}[section]
\newtheorem{objective}{Objectif}[section]

\theoremstyle{definition}
\newtheorem{definition}{Définition}[section]
\renewcommand{\thedefinition}{\arabic{definition}}
\newtheorem{exercise}{Exercice}[chapter]
\renewcommand{\theexercise}{\arabic{exercise}}
\newtheorem{example}{Exemple}[chapter]
\renewcommand{\theexample}{\arabic{example}}
\newtheorem*{solution}{Solution}
\newtheorem*{application}{Application}
\newtheorem*{notation}{Notation}
\newtheorem*{vocabulary}{Vocabulaire}
\newtheorem*{properties}{Propriétés}



\theoremstyle{remark}
\newtheorem*{remark}{Remarque}
\newtheorem*{rappel}{Rappel}


\usepackage{etoolbox}
\AtBeginEnvironment{exercise}{\small}
\AtBeginEnvironment{example}{\small}

\usepackage{cases}
\usepackage[red]{mypack}

\usepackage[framemethod=TikZ]{mdframed}

\definecolor{bg}{rgb}{0.4,0.25,0.95}
\definecolor{pagebg}{rgb}{0,0,0.5}
\surroundwithmdframed[
   topline=false,
   rightline=false,
   bottomline=false,
   leftmargin=\parindent,
   skipabove=8pt,
   skipbelow=8pt,
   linecolor=blue,
   innerbottommargin=10pt,
   % backgroundcolor=bg,font=\color{orange}\sffamily, fontcolor=white
]{definition}

\usepackage{empheq}
\usepackage[most]{tcolorbox}

\newtcbox{\mymath}[1][]{%
    nobeforeafter, math upper, tcbox raise base,
    enhanced, colframe=blue!30!black,
    colback=red!10, boxrule=1pt,
    #1}

\usepackage{unixode}


\DeclareMathOperator{\ord}{ord}
\DeclareMathOperator{\orb}{orb}
\DeclareMathOperator{\stab}{stab}
\DeclareMathOperator{\Stab}{stab}
\DeclareMathOperator{\ppcm}{ppcm}
\DeclareMathOperator{\conj}{Conj}
\DeclareMathOperator{\End}{End}
\DeclareMathOperator{\rot}{rot}
\DeclareMathOperator{\trs}{trace}
\DeclareMathOperator{\Ind}{Ind}
\DeclareMathOperator{\mat}{Mat}
\DeclareMathOperator{\id}{Id}
\DeclareMathOperator{\vect}{vect}
\DeclareMathOperator{\img}{img}
\DeclareMathOperator{\cov}{Cov}
\DeclareMathOperator{\dist}{dist}
\DeclareMathOperator{\irr}{Irr}
\DeclareMathOperator{\image}{Im}
\DeclareMathOperator{\pd}{\partial}
\DeclareMathOperator{\epi}{epi}
\DeclareMathOperator{\Argmin}{Argmin}
\DeclareMathOperator{\dom}{dom}
\DeclareMathOperator{\proj}{proj}
\DeclareMathOperator{\ctg}{ctg}
\DeclareMathOperator{\supp}{supp}
\DeclareMathOperator{\argmin}{argmin}
\DeclareMathOperator{\mult}{mult}
\DeclareMathOperator{\ch}{ch}
\DeclareMathOperator{\sh}{sh}
\DeclareMathOperator{\rang}{rang}
\DeclareMathOperator{\diam}{diam}
\DeclareMathOperator{\Epigraphe}{Epigraphe}




\usepackage{xcolor}
\everymath{\color{blue}}
%\everymath{\color[rgb]{0,1,1}}
%\pagecolor[rgb]{0,0,0.5}


\newcommand*{\pdtest}[3][]{\ensuremath{\frac{\partial^{#1} #2}{\partial #3}}}

\newcommand*{\deffunc}[6][]{\ensuremath{
\begin{array}{rcl}
#2 : #3 &\rightarrow& #4\\
#5 &\mapsto& #6
\end{array}
}}

\newcommand{\eqcolon}{\mathrel{\resizebox{\widthof{$\mathord{=}$}}{\height}{ $\!\!=\!\!\resizebox{1.2\width}{0.8\height}{\raisebox{0.23ex}{$\mathop{:}$}}\!\!$ }}}
\newcommand{\coloneq}{\mathrel{\resizebox{\widthof{$\mathord{=}$}}{\height}{ $\!\!\resizebox{1.2\width}{0.8\height}{\raisebox{0.23ex}{$\mathop{:}$}}\!\!=\!\!$ }}}
\newcommand{\eqcolonl}{\ensuremath{\mathrel{=\!\!\mathop{:}}}}
\newcommand{\coloneql}{\ensuremath{\mathrel{\mathop{:} \!\! =}}}
\newcommand{\vc}[1]{% inline column vector
  \left(\begin{smallmatrix}#1\end{smallmatrix}\right)%
}
\newcommand{\vr}[1]{% inline row vector
  \begin{smallmatrix}(\,#1\,)\end{smallmatrix}%
}
\makeatletter
\newcommand*{\defeq}{\ =\mathrel{\rlap{%
                     \raisebox{0.3ex}{$\m@th\cdot$}}%
                     \raisebox{-0.3ex}{$\m@th\cdot$}}%
                     }
\makeatother

\newcommand{\mathcircle}[1]{% inline row vector
 \overset{\circ}{#1}
}
\newcommand{\ulim}{% low limit
 \underline{\lim}
}
\newcommand{\ssi}{% iff
\iff
}
\newcommand{\ps}[2]{
\expval{#1 | #2}
}
\newcommand{\df}[1]{
\mqty{#1}
}
\newcommand{\n}[1]{
\norm{#1}
}
\newcommand{\sys}[1]{
\left\{\smqty{#1}\right.
}


\newcommand{\eqdef}{\ensuremath{\overset{\text{def}}=}}


\def\Circlearrowright{\ensuremath{%
  \rotatebox[origin=c]{230}{$\circlearrowright$}}}

\newcommand\ct[1]{\text{\rmfamily\upshape #1}}
\newcommand\question[1]{ {\color{red} ...!? \small #1}}
\newcommand\caz[1]{\left\{\begin{array} #1 \end{array}\right.}
\newcommand\const{\text{\rmfamily\upshape const}}
\newcommand\toP{ \overset{\pro}{\to}}
\newcommand\toPP{ \overset{\text{PP}}{\to}}
\newcommand{\oeq}{\mathrel{\text{\textcircled{$=$}}}}





\usepackage{xcolor}
% \usepackage[normalem]{ulem}
\usepackage{lipsum}
\makeatletter
% \newcommand\colorwave[1][blue]{\bgroup \markoverwith{\lower3.5\p@\hbox{\sixly \textcolor{#1}{\char58}}}\ULon}
%\font\sixly=lasy6 % does not re-load if already loaded, so no memory problem.

\newmdtheoremenv[
linewidth= 1pt,linecolor= blue,%
leftmargin=20,rightmargin=20,innertopmargin=0pt, innerrightmargin=40,%
tikzsetting = { draw=lightgray, line width = 0.3pt,dashed,%
dash pattern = on 15pt off 3pt},%
splittopskip=\topskip,skipbelow=\baselineskip,%
skipabove=\baselineskip,ntheorem,roundcorner=0pt,
% backgroundcolor=pagebg,font=\color{orange}\sffamily, fontcolor=white
]{examplebox}{Exemple}[section]



\newcommand\R{\mathbb{R}}
\newcommand\Z{\mathbb{Z}}
\newcommand\N{\mathbb{N}}
\newcommand\E{\mathbb{E}}
\newcommand\F{\mathcal{F}}
\newcommand\cH{\mathcal{H}}
\newcommand\V{\mathbb{V}}
\newcommand\dmo{ ^{-1} }
\newcommand\kapa{\kappa}
\newcommand\im{Im}
\newcommand\hs{\mathcal{H}}





\usepackage{soul}

\makeatletter
\newcommand*{\whiten}[1]{\llap{\textcolor{white}{{\the\SOUL@token}}\hspace{#1pt}}}
\DeclareRobustCommand*\myul{%
    \def\SOUL@everyspace{\underline{\space}\kern\z@}%
    \def\SOUL@everytoken{%
     \setbox0=\hbox{\the\SOUL@token}%
     \ifdim\dp0>\z@
        \raisebox{\dp0}{\underline{\phantom{\the\SOUL@token}}}%
        \whiten{1}\whiten{0}%
        \whiten{-1}\whiten{-2}%
        \llap{\the\SOUL@token}%
     \else
        \underline{\the\SOUL@token}%
     \fi}%
\SOUL@}
\makeatother

\newcommand*{\demp}{\fontfamily{lmtt}\selectfont}

\DeclareTextFontCommand{\textdemp}{\demp}

\begin{document}

\ifcomment
Multiline
comment
\fi
\ifcomment
\myul{Typesetting test}
% \color[rgb]{1,1,1}
$∑_i^n≠ 60º±∞π∆¬≈√j∫h≤≥µ$

$\CR \R\pro\ind\pro\gS\pro
\mqty[a&b\\c&d]$
$\pro\mathbb{P}$
$\dd{x}$

  \[
    \alpha(x)=\left\{
                \begin{array}{ll}
                  x\\
                  \frac{1}{1+e^{-kx}}\\
                  \frac{e^x-e^{-x}}{e^x+e^{-x}}
                \end{array}
              \right.
  \]

  $\expval{x}$
  
  $\chi_\rho(ghg\dmo)=\Tr(\rho_{ghg\dmo})=\Tr(\rho_g\circ\rho_h\circ\rho\dmo_g)=\Tr(\rho_h)\overset{\mbox{\scalebox{0.5}{$\Tr(AB)=\Tr(BA)$}}}{=}\chi_\rho(h)$
  	$\mathop{\oplus}_{\substack{x\in X}}$

$\mat(\rho_g)=(a_{ij}(g))_{\scriptsize \substack{1\leq i\leq d \\ 1\leq j\leq d}}$ et $\mat(\rho'_g)=(a'_{ij}(g))_{\scriptsize \substack{1\leq i'\leq d' \\ 1\leq j'\leq d'}}$



\[\int_a^b{\mathbb{R}^2}g(u, v)\dd{P_{XY}}(u, v)=\iint g(u,v) f_{XY}(u, v)\dd \lambda(u) \dd \lambda(v)\]
$$\lim_{x\to\infty} f(x)$$	
$$\iiiint_V \mu(t,u,v,w) \,dt\,du\,dv\,dw$$
$$\sum_{n=1}^{\infty} 2^{-n} = 1$$	
\begin{definition}
	Si $X$ et $Y$ sont 2 v.a. ou definit la \textsc{Covariance} entre $X$ et $Y$ comme
	$\cov(X,Y)\overset{\text{def}}{=}\E\left[(X-\E(X))(Y-\E(Y))\right]=\E(XY)-\E(X)\E(Y)$.
\end{definition}
\fi
\pagebreak

% \tableofcontents

% insert your code here
%\input{./algebra/main.tex}
%\input{./geometrie-differentielle/main.tex}
%\input{./probabilite/main.tex}
%\input{./analyse-fonctionnelle/main.tex}
% \input{./Analyse-convexe-et-dualite-en-optimisation/main.tex}
%\input{./tikz/main.tex}
%\input{./Theorie-du-distributions/main.tex}
%\input{./optimisation/mine.tex}
 \input{./modelisation/main.tex}

% yves.aubry@univ-tln.fr : algebra

\end{document}

%% !TEX encoding = UTF-8 Unicode
% !TEX TS-program = xelatex

\documentclass[french]{report}

%\usepackage[utf8]{inputenc}
%\usepackage[T1]{fontenc}
\usepackage{babel}


\newif\ifcomment
%\commenttrue # Show comments

\usepackage{physics}
\usepackage{amssymb}


\usepackage{amsthm}
% \usepackage{thmtools}
\usepackage{mathtools}
\usepackage{amsfonts}

\usepackage{color}

\usepackage{tikz}

\usepackage{geometry}
\geometry{a5paper, margin=0.1in, right=1cm}

\usepackage{dsfont}

\usepackage{graphicx}
\graphicspath{ {images/} }

\usepackage{faktor}

\usepackage{IEEEtrantools}
\usepackage{enumerate}   
\usepackage[PostScript=dvips]{"/Users/aware/Documents/Courses/diagrams"}


\newtheorem{theorem}{Théorème}[section]
\renewcommand{\thetheorem}{\arabic{theorem}}
\newtheorem{lemme}{Lemme}[section]
\renewcommand{\thelemme}{\arabic{lemme}}
\newtheorem{proposition}{Proposition}[section]
\renewcommand{\theproposition}{\arabic{proposition}}
\newtheorem{notations}{Notations}[section]
\newtheorem{problem}{Problème}[section]
\newtheorem{corollary}{Corollaire}[theorem]
\renewcommand{\thecorollary}{\arabic{corollary}}
\newtheorem{property}{Propriété}[section]
\newtheorem{objective}{Objectif}[section]

\theoremstyle{definition}
\newtheorem{definition}{Définition}[section]
\renewcommand{\thedefinition}{\arabic{definition}}
\newtheorem{exercise}{Exercice}[chapter]
\renewcommand{\theexercise}{\arabic{exercise}}
\newtheorem{example}{Exemple}[chapter]
\renewcommand{\theexample}{\arabic{example}}
\newtheorem*{solution}{Solution}
\newtheorem*{application}{Application}
\newtheorem*{notation}{Notation}
\newtheorem*{vocabulary}{Vocabulaire}
\newtheorem*{properties}{Propriétés}



\theoremstyle{remark}
\newtheorem*{remark}{Remarque}
\newtheorem*{rappel}{Rappel}


\usepackage{etoolbox}
\AtBeginEnvironment{exercise}{\small}
\AtBeginEnvironment{example}{\small}

\usepackage{cases}
\usepackage[red]{mypack}

\usepackage[framemethod=TikZ]{mdframed}

\definecolor{bg}{rgb}{0.4,0.25,0.95}
\definecolor{pagebg}{rgb}{0,0,0.5}
\surroundwithmdframed[
   topline=false,
   rightline=false,
   bottomline=false,
   leftmargin=\parindent,
   skipabove=8pt,
   skipbelow=8pt,
   linecolor=blue,
   innerbottommargin=10pt,
   % backgroundcolor=bg,font=\color{orange}\sffamily, fontcolor=white
]{definition}

\usepackage{empheq}
\usepackage[most]{tcolorbox}

\newtcbox{\mymath}[1][]{%
    nobeforeafter, math upper, tcbox raise base,
    enhanced, colframe=blue!30!black,
    colback=red!10, boxrule=1pt,
    #1}

\usepackage{unixode}


\DeclareMathOperator{\ord}{ord}
\DeclareMathOperator{\orb}{orb}
\DeclareMathOperator{\stab}{stab}
\DeclareMathOperator{\Stab}{stab}
\DeclareMathOperator{\ppcm}{ppcm}
\DeclareMathOperator{\conj}{Conj}
\DeclareMathOperator{\End}{End}
\DeclareMathOperator{\rot}{rot}
\DeclareMathOperator{\trs}{trace}
\DeclareMathOperator{\Ind}{Ind}
\DeclareMathOperator{\mat}{Mat}
\DeclareMathOperator{\id}{Id}
\DeclareMathOperator{\vect}{vect}
\DeclareMathOperator{\img}{img}
\DeclareMathOperator{\cov}{Cov}
\DeclareMathOperator{\dist}{dist}
\DeclareMathOperator{\irr}{Irr}
\DeclareMathOperator{\image}{Im}
\DeclareMathOperator{\pd}{\partial}
\DeclareMathOperator{\epi}{epi}
\DeclareMathOperator{\Argmin}{Argmin}
\DeclareMathOperator{\dom}{dom}
\DeclareMathOperator{\proj}{proj}
\DeclareMathOperator{\ctg}{ctg}
\DeclareMathOperator{\supp}{supp}
\DeclareMathOperator{\argmin}{argmin}
\DeclareMathOperator{\mult}{mult}
\DeclareMathOperator{\ch}{ch}
\DeclareMathOperator{\sh}{sh}
\DeclareMathOperator{\rang}{rang}
\DeclareMathOperator{\diam}{diam}
\DeclareMathOperator{\Epigraphe}{Epigraphe}




\usepackage{xcolor}
\everymath{\color{blue}}
%\everymath{\color[rgb]{0,1,1}}
%\pagecolor[rgb]{0,0,0.5}


\newcommand*{\pdtest}[3][]{\ensuremath{\frac{\partial^{#1} #2}{\partial #3}}}

\newcommand*{\deffunc}[6][]{\ensuremath{
\begin{array}{rcl}
#2 : #3 &\rightarrow& #4\\
#5 &\mapsto& #6
\end{array}
}}

\newcommand{\eqcolon}{\mathrel{\resizebox{\widthof{$\mathord{=}$}}{\height}{ $\!\!=\!\!\resizebox{1.2\width}{0.8\height}{\raisebox{0.23ex}{$\mathop{:}$}}\!\!$ }}}
\newcommand{\coloneq}{\mathrel{\resizebox{\widthof{$\mathord{=}$}}{\height}{ $\!\!\resizebox{1.2\width}{0.8\height}{\raisebox{0.23ex}{$\mathop{:}$}}\!\!=\!\!$ }}}
\newcommand{\eqcolonl}{\ensuremath{\mathrel{=\!\!\mathop{:}}}}
\newcommand{\coloneql}{\ensuremath{\mathrel{\mathop{:} \!\! =}}}
\newcommand{\vc}[1]{% inline column vector
  \left(\begin{smallmatrix}#1\end{smallmatrix}\right)%
}
\newcommand{\vr}[1]{% inline row vector
  \begin{smallmatrix}(\,#1\,)\end{smallmatrix}%
}
\makeatletter
\newcommand*{\defeq}{\ =\mathrel{\rlap{%
                     \raisebox{0.3ex}{$\m@th\cdot$}}%
                     \raisebox{-0.3ex}{$\m@th\cdot$}}%
                     }
\makeatother

\newcommand{\mathcircle}[1]{% inline row vector
 \overset{\circ}{#1}
}
\newcommand{\ulim}{% low limit
 \underline{\lim}
}
\newcommand{\ssi}{% iff
\iff
}
\newcommand{\ps}[2]{
\expval{#1 | #2}
}
\newcommand{\df}[1]{
\mqty{#1}
}
\newcommand{\n}[1]{
\norm{#1}
}
\newcommand{\sys}[1]{
\left\{\smqty{#1}\right.
}


\newcommand{\eqdef}{\ensuremath{\overset{\text{def}}=}}


\def\Circlearrowright{\ensuremath{%
  \rotatebox[origin=c]{230}{$\circlearrowright$}}}

\newcommand\ct[1]{\text{\rmfamily\upshape #1}}
\newcommand\question[1]{ {\color{red} ...!? \small #1}}
\newcommand\caz[1]{\left\{\begin{array} #1 \end{array}\right.}
\newcommand\const{\text{\rmfamily\upshape const}}
\newcommand\toP{ \overset{\pro}{\to}}
\newcommand\toPP{ \overset{\text{PP}}{\to}}
\newcommand{\oeq}{\mathrel{\text{\textcircled{$=$}}}}





\usepackage{xcolor}
% \usepackage[normalem]{ulem}
\usepackage{lipsum}
\makeatletter
% \newcommand\colorwave[1][blue]{\bgroup \markoverwith{\lower3.5\p@\hbox{\sixly \textcolor{#1}{\char58}}}\ULon}
%\font\sixly=lasy6 % does not re-load if already loaded, so no memory problem.

\newmdtheoremenv[
linewidth= 1pt,linecolor= blue,%
leftmargin=20,rightmargin=20,innertopmargin=0pt, innerrightmargin=40,%
tikzsetting = { draw=lightgray, line width = 0.3pt,dashed,%
dash pattern = on 15pt off 3pt},%
splittopskip=\topskip,skipbelow=\baselineskip,%
skipabove=\baselineskip,ntheorem,roundcorner=0pt,
% backgroundcolor=pagebg,font=\color{orange}\sffamily, fontcolor=white
]{examplebox}{Exemple}[section]



\newcommand\R{\mathbb{R}}
\newcommand\Z{\mathbb{Z}}
\newcommand\N{\mathbb{N}}
\newcommand\E{\mathbb{E}}
\newcommand\F{\mathcal{F}}
\newcommand\cH{\mathcal{H}}
\newcommand\V{\mathbb{V}}
\newcommand\dmo{ ^{-1} }
\newcommand\kapa{\kappa}
\newcommand\im{Im}
\newcommand\hs{\mathcal{H}}





\usepackage{soul}

\makeatletter
\newcommand*{\whiten}[1]{\llap{\textcolor{white}{{\the\SOUL@token}}\hspace{#1pt}}}
\DeclareRobustCommand*\myul{%
    \def\SOUL@everyspace{\underline{\space}\kern\z@}%
    \def\SOUL@everytoken{%
     \setbox0=\hbox{\the\SOUL@token}%
     \ifdim\dp0>\z@
        \raisebox{\dp0}{\underline{\phantom{\the\SOUL@token}}}%
        \whiten{1}\whiten{0}%
        \whiten{-1}\whiten{-2}%
        \llap{\the\SOUL@token}%
     \else
        \underline{\the\SOUL@token}%
     \fi}%
\SOUL@}
\makeatother

\newcommand*{\demp}{\fontfamily{lmtt}\selectfont}

\DeclareTextFontCommand{\textdemp}{\demp}

\begin{document}

\ifcomment
Multiline
comment
\fi
\ifcomment
\myul{Typesetting test}
% \color[rgb]{1,1,1}
$∑_i^n≠ 60º±∞π∆¬≈√j∫h≤≥µ$

$\CR \R\pro\ind\pro\gS\pro
\mqty[a&b\\c&d]$
$\pro\mathbb{P}$
$\dd{x}$

  \[
    \alpha(x)=\left\{
                \begin{array}{ll}
                  x\\
                  \frac{1}{1+e^{-kx}}\\
                  \frac{e^x-e^{-x}}{e^x+e^{-x}}
                \end{array}
              \right.
  \]

  $\expval{x}$
  
  $\chi_\rho(ghg\dmo)=\Tr(\rho_{ghg\dmo})=\Tr(\rho_g\circ\rho_h\circ\rho\dmo_g)=\Tr(\rho_h)\overset{\mbox{\scalebox{0.5}{$\Tr(AB)=\Tr(BA)$}}}{=}\chi_\rho(h)$
  	$\mathop{\oplus}_{\substack{x\in X}}$

$\mat(\rho_g)=(a_{ij}(g))_{\scriptsize \substack{1\leq i\leq d \\ 1\leq j\leq d}}$ et $\mat(\rho'_g)=(a'_{ij}(g))_{\scriptsize \substack{1\leq i'\leq d' \\ 1\leq j'\leq d'}}$



\[\int_a^b{\mathbb{R}^2}g(u, v)\dd{P_{XY}}(u, v)=\iint g(u,v) f_{XY}(u, v)\dd \lambda(u) \dd \lambda(v)\]
$$\lim_{x\to\infty} f(x)$$	
$$\iiiint_V \mu(t,u,v,w) \,dt\,du\,dv\,dw$$
$$\sum_{n=1}^{\infty} 2^{-n} = 1$$	
\begin{definition}
	Si $X$ et $Y$ sont 2 v.a. ou definit la \textsc{Covariance} entre $X$ et $Y$ comme
	$\cov(X,Y)\overset{\text{def}}{=}\E\left[(X-\E(X))(Y-\E(Y))\right]=\E(XY)-\E(X)\E(Y)$.
\end{definition}
\fi
\pagebreak

% \tableofcontents

% insert your code here
%\input{./algebra/main.tex}
%\input{./geometrie-differentielle/main.tex}
%\input{./probabilite/main.tex}
%\input{./analyse-fonctionnelle/main.tex}
% \input{./Analyse-convexe-et-dualite-en-optimisation/main.tex}
%\input{./tikz/main.tex}
%\input{./Theorie-du-distributions/main.tex}
%\input{./optimisation/mine.tex}
 \input{./modelisation/main.tex}

% yves.aubry@univ-tln.fr : algebra

\end{document}

%\input{./optimisation/mine.tex}
 % !TEX encoding = UTF-8 Unicode
% !TEX TS-program = xelatex

\documentclass[french]{report}

%\usepackage[utf8]{inputenc}
%\usepackage[T1]{fontenc}
\usepackage{babel}


\newif\ifcomment
%\commenttrue # Show comments

\usepackage{physics}
\usepackage{amssymb}


\usepackage{amsthm}
% \usepackage{thmtools}
\usepackage{mathtools}
\usepackage{amsfonts}

\usepackage{color}

\usepackage{tikz}

\usepackage{geometry}
\geometry{a5paper, margin=0.1in, right=1cm}

\usepackage{dsfont}

\usepackage{graphicx}
\graphicspath{ {images/} }

\usepackage{faktor}

\usepackage{IEEEtrantools}
\usepackage{enumerate}   
\usepackage[PostScript=dvips]{"/Users/aware/Documents/Courses/diagrams"}


\newtheorem{theorem}{Théorème}[section]
\renewcommand{\thetheorem}{\arabic{theorem}}
\newtheorem{lemme}{Lemme}[section]
\renewcommand{\thelemme}{\arabic{lemme}}
\newtheorem{proposition}{Proposition}[section]
\renewcommand{\theproposition}{\arabic{proposition}}
\newtheorem{notations}{Notations}[section]
\newtheorem{problem}{Problème}[section]
\newtheorem{corollary}{Corollaire}[theorem]
\renewcommand{\thecorollary}{\arabic{corollary}}
\newtheorem{property}{Propriété}[section]
\newtheorem{objective}{Objectif}[section]

\theoremstyle{definition}
\newtheorem{definition}{Définition}[section]
\renewcommand{\thedefinition}{\arabic{definition}}
\newtheorem{exercise}{Exercice}[chapter]
\renewcommand{\theexercise}{\arabic{exercise}}
\newtheorem{example}{Exemple}[chapter]
\renewcommand{\theexample}{\arabic{example}}
\newtheorem*{solution}{Solution}
\newtheorem*{application}{Application}
\newtheorem*{notation}{Notation}
\newtheorem*{vocabulary}{Vocabulaire}
\newtheorem*{properties}{Propriétés}



\theoremstyle{remark}
\newtheorem*{remark}{Remarque}
\newtheorem*{rappel}{Rappel}


\usepackage{etoolbox}
\AtBeginEnvironment{exercise}{\small}
\AtBeginEnvironment{example}{\small}

\usepackage{cases}
\usepackage[red]{mypack}

\usepackage[framemethod=TikZ]{mdframed}

\definecolor{bg}{rgb}{0.4,0.25,0.95}
\definecolor{pagebg}{rgb}{0,0,0.5}
\surroundwithmdframed[
   topline=false,
   rightline=false,
   bottomline=false,
   leftmargin=\parindent,
   skipabove=8pt,
   skipbelow=8pt,
   linecolor=blue,
   innerbottommargin=10pt,
   % backgroundcolor=bg,font=\color{orange}\sffamily, fontcolor=white
]{definition}

\usepackage{empheq}
\usepackage[most]{tcolorbox}

\newtcbox{\mymath}[1][]{%
    nobeforeafter, math upper, tcbox raise base,
    enhanced, colframe=blue!30!black,
    colback=red!10, boxrule=1pt,
    #1}

\usepackage{unixode}


\DeclareMathOperator{\ord}{ord}
\DeclareMathOperator{\orb}{orb}
\DeclareMathOperator{\stab}{stab}
\DeclareMathOperator{\Stab}{stab}
\DeclareMathOperator{\ppcm}{ppcm}
\DeclareMathOperator{\conj}{Conj}
\DeclareMathOperator{\End}{End}
\DeclareMathOperator{\rot}{rot}
\DeclareMathOperator{\trs}{trace}
\DeclareMathOperator{\Ind}{Ind}
\DeclareMathOperator{\mat}{Mat}
\DeclareMathOperator{\id}{Id}
\DeclareMathOperator{\vect}{vect}
\DeclareMathOperator{\img}{img}
\DeclareMathOperator{\cov}{Cov}
\DeclareMathOperator{\dist}{dist}
\DeclareMathOperator{\irr}{Irr}
\DeclareMathOperator{\image}{Im}
\DeclareMathOperator{\pd}{\partial}
\DeclareMathOperator{\epi}{epi}
\DeclareMathOperator{\Argmin}{Argmin}
\DeclareMathOperator{\dom}{dom}
\DeclareMathOperator{\proj}{proj}
\DeclareMathOperator{\ctg}{ctg}
\DeclareMathOperator{\supp}{supp}
\DeclareMathOperator{\argmin}{argmin}
\DeclareMathOperator{\mult}{mult}
\DeclareMathOperator{\ch}{ch}
\DeclareMathOperator{\sh}{sh}
\DeclareMathOperator{\rang}{rang}
\DeclareMathOperator{\diam}{diam}
\DeclareMathOperator{\Epigraphe}{Epigraphe}




\usepackage{xcolor}
\everymath{\color{blue}}
%\everymath{\color[rgb]{0,1,1}}
%\pagecolor[rgb]{0,0,0.5}


\newcommand*{\pdtest}[3][]{\ensuremath{\frac{\partial^{#1} #2}{\partial #3}}}

\newcommand*{\deffunc}[6][]{\ensuremath{
\begin{array}{rcl}
#2 : #3 &\rightarrow& #4\\
#5 &\mapsto& #6
\end{array}
}}

\newcommand{\eqcolon}{\mathrel{\resizebox{\widthof{$\mathord{=}$}}{\height}{ $\!\!=\!\!\resizebox{1.2\width}{0.8\height}{\raisebox{0.23ex}{$\mathop{:}$}}\!\!$ }}}
\newcommand{\coloneq}{\mathrel{\resizebox{\widthof{$\mathord{=}$}}{\height}{ $\!\!\resizebox{1.2\width}{0.8\height}{\raisebox{0.23ex}{$\mathop{:}$}}\!\!=\!\!$ }}}
\newcommand{\eqcolonl}{\ensuremath{\mathrel{=\!\!\mathop{:}}}}
\newcommand{\coloneql}{\ensuremath{\mathrel{\mathop{:} \!\! =}}}
\newcommand{\vc}[1]{% inline column vector
  \left(\begin{smallmatrix}#1\end{smallmatrix}\right)%
}
\newcommand{\vr}[1]{% inline row vector
  \begin{smallmatrix}(\,#1\,)\end{smallmatrix}%
}
\makeatletter
\newcommand*{\defeq}{\ =\mathrel{\rlap{%
                     \raisebox{0.3ex}{$\m@th\cdot$}}%
                     \raisebox{-0.3ex}{$\m@th\cdot$}}%
                     }
\makeatother

\newcommand{\mathcircle}[1]{% inline row vector
 \overset{\circ}{#1}
}
\newcommand{\ulim}{% low limit
 \underline{\lim}
}
\newcommand{\ssi}{% iff
\iff
}
\newcommand{\ps}[2]{
\expval{#1 | #2}
}
\newcommand{\df}[1]{
\mqty{#1}
}
\newcommand{\n}[1]{
\norm{#1}
}
\newcommand{\sys}[1]{
\left\{\smqty{#1}\right.
}


\newcommand{\eqdef}{\ensuremath{\overset{\text{def}}=}}


\def\Circlearrowright{\ensuremath{%
  \rotatebox[origin=c]{230}{$\circlearrowright$}}}

\newcommand\ct[1]{\text{\rmfamily\upshape #1}}
\newcommand\question[1]{ {\color{red} ...!? \small #1}}
\newcommand\caz[1]{\left\{\begin{array} #1 \end{array}\right.}
\newcommand\const{\text{\rmfamily\upshape const}}
\newcommand\toP{ \overset{\pro}{\to}}
\newcommand\toPP{ \overset{\text{PP}}{\to}}
\newcommand{\oeq}{\mathrel{\text{\textcircled{$=$}}}}





\usepackage{xcolor}
% \usepackage[normalem]{ulem}
\usepackage{lipsum}
\makeatletter
% \newcommand\colorwave[1][blue]{\bgroup \markoverwith{\lower3.5\p@\hbox{\sixly \textcolor{#1}{\char58}}}\ULon}
%\font\sixly=lasy6 % does not re-load if already loaded, so no memory problem.

\newmdtheoremenv[
linewidth= 1pt,linecolor= blue,%
leftmargin=20,rightmargin=20,innertopmargin=0pt, innerrightmargin=40,%
tikzsetting = { draw=lightgray, line width = 0.3pt,dashed,%
dash pattern = on 15pt off 3pt},%
splittopskip=\topskip,skipbelow=\baselineskip,%
skipabove=\baselineskip,ntheorem,roundcorner=0pt,
% backgroundcolor=pagebg,font=\color{orange}\sffamily, fontcolor=white
]{examplebox}{Exemple}[section]



\newcommand\R{\mathbb{R}}
\newcommand\Z{\mathbb{Z}}
\newcommand\N{\mathbb{N}}
\newcommand\E{\mathbb{E}}
\newcommand\F{\mathcal{F}}
\newcommand\cH{\mathcal{H}}
\newcommand\V{\mathbb{V}}
\newcommand\dmo{ ^{-1} }
\newcommand\kapa{\kappa}
\newcommand\im{Im}
\newcommand\hs{\mathcal{H}}





\usepackage{soul}

\makeatletter
\newcommand*{\whiten}[1]{\llap{\textcolor{white}{{\the\SOUL@token}}\hspace{#1pt}}}
\DeclareRobustCommand*\myul{%
    \def\SOUL@everyspace{\underline{\space}\kern\z@}%
    \def\SOUL@everytoken{%
     \setbox0=\hbox{\the\SOUL@token}%
     \ifdim\dp0>\z@
        \raisebox{\dp0}{\underline{\phantom{\the\SOUL@token}}}%
        \whiten{1}\whiten{0}%
        \whiten{-1}\whiten{-2}%
        \llap{\the\SOUL@token}%
     \else
        \underline{\the\SOUL@token}%
     \fi}%
\SOUL@}
\makeatother

\newcommand*{\demp}{\fontfamily{lmtt}\selectfont}

\DeclareTextFontCommand{\textdemp}{\demp}

\begin{document}

\ifcomment
Multiline
comment
\fi
\ifcomment
\myul{Typesetting test}
% \color[rgb]{1,1,1}
$∑_i^n≠ 60º±∞π∆¬≈√j∫h≤≥µ$

$\CR \R\pro\ind\pro\gS\pro
\mqty[a&b\\c&d]$
$\pro\mathbb{P}$
$\dd{x}$

  \[
    \alpha(x)=\left\{
                \begin{array}{ll}
                  x\\
                  \frac{1}{1+e^{-kx}}\\
                  \frac{e^x-e^{-x}}{e^x+e^{-x}}
                \end{array}
              \right.
  \]

  $\expval{x}$
  
  $\chi_\rho(ghg\dmo)=\Tr(\rho_{ghg\dmo})=\Tr(\rho_g\circ\rho_h\circ\rho\dmo_g)=\Tr(\rho_h)\overset{\mbox{\scalebox{0.5}{$\Tr(AB)=\Tr(BA)$}}}{=}\chi_\rho(h)$
  	$\mathop{\oplus}_{\substack{x\in X}}$

$\mat(\rho_g)=(a_{ij}(g))_{\scriptsize \substack{1\leq i\leq d \\ 1\leq j\leq d}}$ et $\mat(\rho'_g)=(a'_{ij}(g))_{\scriptsize \substack{1\leq i'\leq d' \\ 1\leq j'\leq d'}}$



\[\int_a^b{\mathbb{R}^2}g(u, v)\dd{P_{XY}}(u, v)=\iint g(u,v) f_{XY}(u, v)\dd \lambda(u) \dd \lambda(v)\]
$$\lim_{x\to\infty} f(x)$$	
$$\iiiint_V \mu(t,u,v,w) \,dt\,du\,dv\,dw$$
$$\sum_{n=1}^{\infty} 2^{-n} = 1$$	
\begin{definition}
	Si $X$ et $Y$ sont 2 v.a. ou definit la \textsc{Covariance} entre $X$ et $Y$ comme
	$\cov(X,Y)\overset{\text{def}}{=}\E\left[(X-\E(X))(Y-\E(Y))\right]=\E(XY)-\E(X)\E(Y)$.
\end{definition}
\fi
\pagebreak

% \tableofcontents

% insert your code here
%\input{./algebra/main.tex}
%\input{./geometrie-differentielle/main.tex}
%\input{./probabilite/main.tex}
%\input{./analyse-fonctionnelle/main.tex}
% \input{./Analyse-convexe-et-dualite-en-optimisation/main.tex}
%\input{./tikz/main.tex}
%\input{./Theorie-du-distributions/main.tex}
%\input{./optimisation/mine.tex}
 \input{./modelisation/main.tex}

% yves.aubry@univ-tln.fr : algebra

\end{document}


% yves.aubry@univ-tln.fr : algebra

\end{document}

%% !TEX encoding = UTF-8 Unicode
% !TEX TS-program = xelatex

\documentclass[french]{report}

%\usepackage[utf8]{inputenc}
%\usepackage[T1]{fontenc}
\usepackage{babel}


\newif\ifcomment
%\commenttrue # Show comments

\usepackage{physics}
\usepackage{amssymb}


\usepackage{amsthm}
% \usepackage{thmtools}
\usepackage{mathtools}
\usepackage{amsfonts}

\usepackage{color}

\usepackage{tikz}

\usepackage{geometry}
\geometry{a5paper, margin=0.1in, right=1cm}

\usepackage{dsfont}

\usepackage{graphicx}
\graphicspath{ {images/} }

\usepackage{faktor}

\usepackage{IEEEtrantools}
\usepackage{enumerate}   
\usepackage[PostScript=dvips]{"/Users/aware/Documents/Courses/diagrams"}


\newtheorem{theorem}{Théorème}[section]
\renewcommand{\thetheorem}{\arabic{theorem}}
\newtheorem{lemme}{Lemme}[section]
\renewcommand{\thelemme}{\arabic{lemme}}
\newtheorem{proposition}{Proposition}[section]
\renewcommand{\theproposition}{\arabic{proposition}}
\newtheorem{notations}{Notations}[section]
\newtheorem{problem}{Problème}[section]
\newtheorem{corollary}{Corollaire}[theorem]
\renewcommand{\thecorollary}{\arabic{corollary}}
\newtheorem{property}{Propriété}[section]
\newtheorem{objective}{Objectif}[section]

\theoremstyle{definition}
\newtheorem{definition}{Définition}[section]
\renewcommand{\thedefinition}{\arabic{definition}}
\newtheorem{exercise}{Exercice}[chapter]
\renewcommand{\theexercise}{\arabic{exercise}}
\newtheorem{example}{Exemple}[chapter]
\renewcommand{\theexample}{\arabic{example}}
\newtheorem*{solution}{Solution}
\newtheorem*{application}{Application}
\newtheorem*{notation}{Notation}
\newtheorem*{vocabulary}{Vocabulaire}
\newtheorem*{properties}{Propriétés}



\theoremstyle{remark}
\newtheorem*{remark}{Remarque}
\newtheorem*{rappel}{Rappel}


\usepackage{etoolbox}
\AtBeginEnvironment{exercise}{\small}
\AtBeginEnvironment{example}{\small}

\usepackage{cases}
\usepackage[red]{mypack}

\usepackage[framemethod=TikZ]{mdframed}

\definecolor{bg}{rgb}{0.4,0.25,0.95}
\definecolor{pagebg}{rgb}{0,0,0.5}
\surroundwithmdframed[
   topline=false,
   rightline=false,
   bottomline=false,
   leftmargin=\parindent,
   skipabove=8pt,
   skipbelow=8pt,
   linecolor=blue,
   innerbottommargin=10pt,
   % backgroundcolor=bg,font=\color{orange}\sffamily, fontcolor=white
]{definition}

\usepackage{empheq}
\usepackage[most]{tcolorbox}

\newtcbox{\mymath}[1][]{%
    nobeforeafter, math upper, tcbox raise base,
    enhanced, colframe=blue!30!black,
    colback=red!10, boxrule=1pt,
    #1}

\usepackage{unixode}


\DeclareMathOperator{\ord}{ord}
\DeclareMathOperator{\orb}{orb}
\DeclareMathOperator{\stab}{stab}
\DeclareMathOperator{\Stab}{stab}
\DeclareMathOperator{\ppcm}{ppcm}
\DeclareMathOperator{\conj}{Conj}
\DeclareMathOperator{\End}{End}
\DeclareMathOperator{\rot}{rot}
\DeclareMathOperator{\trs}{trace}
\DeclareMathOperator{\Ind}{Ind}
\DeclareMathOperator{\mat}{Mat}
\DeclareMathOperator{\id}{Id}
\DeclareMathOperator{\vect}{vect}
\DeclareMathOperator{\img}{img}
\DeclareMathOperator{\cov}{Cov}
\DeclareMathOperator{\dist}{dist}
\DeclareMathOperator{\irr}{Irr}
\DeclareMathOperator{\image}{Im}
\DeclareMathOperator{\pd}{\partial}
\DeclareMathOperator{\epi}{epi}
\DeclareMathOperator{\Argmin}{Argmin}
\DeclareMathOperator{\dom}{dom}
\DeclareMathOperator{\proj}{proj}
\DeclareMathOperator{\ctg}{ctg}
\DeclareMathOperator{\supp}{supp}
\DeclareMathOperator{\argmin}{argmin}
\DeclareMathOperator{\mult}{mult}
\DeclareMathOperator{\ch}{ch}
\DeclareMathOperator{\sh}{sh}
\DeclareMathOperator{\rang}{rang}
\DeclareMathOperator{\diam}{diam}
\DeclareMathOperator{\Epigraphe}{Epigraphe}




\usepackage{xcolor}
\everymath{\color{blue}}
%\everymath{\color[rgb]{0,1,1}}
%\pagecolor[rgb]{0,0,0.5}


\newcommand*{\pdtest}[3][]{\ensuremath{\frac{\partial^{#1} #2}{\partial #3}}}

\newcommand*{\deffunc}[6][]{\ensuremath{
\begin{array}{rcl}
#2 : #3 &\rightarrow& #4\\
#5 &\mapsto& #6
\end{array}
}}

\newcommand{\eqcolon}{\mathrel{\resizebox{\widthof{$\mathord{=}$}}{\height}{ $\!\!=\!\!\resizebox{1.2\width}{0.8\height}{\raisebox{0.23ex}{$\mathop{:}$}}\!\!$ }}}
\newcommand{\coloneq}{\mathrel{\resizebox{\widthof{$\mathord{=}$}}{\height}{ $\!\!\resizebox{1.2\width}{0.8\height}{\raisebox{0.23ex}{$\mathop{:}$}}\!\!=\!\!$ }}}
\newcommand{\eqcolonl}{\ensuremath{\mathrel{=\!\!\mathop{:}}}}
\newcommand{\coloneql}{\ensuremath{\mathrel{\mathop{:} \!\! =}}}
\newcommand{\vc}[1]{% inline column vector
  \left(\begin{smallmatrix}#1\end{smallmatrix}\right)%
}
\newcommand{\vr}[1]{% inline row vector
  \begin{smallmatrix}(\,#1\,)\end{smallmatrix}%
}
\makeatletter
\newcommand*{\defeq}{\ =\mathrel{\rlap{%
                     \raisebox{0.3ex}{$\m@th\cdot$}}%
                     \raisebox{-0.3ex}{$\m@th\cdot$}}%
                     }
\makeatother

\newcommand{\mathcircle}[1]{% inline row vector
 \overset{\circ}{#1}
}
\newcommand{\ulim}{% low limit
 \underline{\lim}
}
\newcommand{\ssi}{% iff
\iff
}
\newcommand{\ps}[2]{
\expval{#1 | #2}
}
\newcommand{\df}[1]{
\mqty{#1}
}
\newcommand{\n}[1]{
\norm{#1}
}
\newcommand{\sys}[1]{
\left\{\smqty{#1}\right.
}


\newcommand{\eqdef}{\ensuremath{\overset{\text{def}}=}}


\def\Circlearrowright{\ensuremath{%
  \rotatebox[origin=c]{230}{$\circlearrowright$}}}

\newcommand\ct[1]{\text{\rmfamily\upshape #1}}
\newcommand\question[1]{ {\color{red} ...!? \small #1}}
\newcommand\caz[1]{\left\{\begin{array} #1 \end{array}\right.}
\newcommand\const{\text{\rmfamily\upshape const}}
\newcommand\toP{ \overset{\pro}{\to}}
\newcommand\toPP{ \overset{\text{PP}}{\to}}
\newcommand{\oeq}{\mathrel{\text{\textcircled{$=$}}}}





\usepackage{xcolor}
% \usepackage[normalem]{ulem}
\usepackage{lipsum}
\makeatletter
% \newcommand\colorwave[1][blue]{\bgroup \markoverwith{\lower3.5\p@\hbox{\sixly \textcolor{#1}{\char58}}}\ULon}
%\font\sixly=lasy6 % does not re-load if already loaded, so no memory problem.

\newmdtheoremenv[
linewidth= 1pt,linecolor= blue,%
leftmargin=20,rightmargin=20,innertopmargin=0pt, innerrightmargin=40,%
tikzsetting = { draw=lightgray, line width = 0.3pt,dashed,%
dash pattern = on 15pt off 3pt},%
splittopskip=\topskip,skipbelow=\baselineskip,%
skipabove=\baselineskip,ntheorem,roundcorner=0pt,
% backgroundcolor=pagebg,font=\color{orange}\sffamily, fontcolor=white
]{examplebox}{Exemple}[section]



\newcommand\R{\mathbb{R}}
\newcommand\Z{\mathbb{Z}}
\newcommand\N{\mathbb{N}}
\newcommand\E{\mathbb{E}}
\newcommand\F{\mathcal{F}}
\newcommand\cH{\mathcal{H}}
\newcommand\V{\mathbb{V}}
\newcommand\dmo{ ^{-1} }
\newcommand\kapa{\kappa}
\newcommand\im{Im}
\newcommand\hs{\mathcal{H}}





\usepackage{soul}

\makeatletter
\newcommand*{\whiten}[1]{\llap{\textcolor{white}{{\the\SOUL@token}}\hspace{#1pt}}}
\DeclareRobustCommand*\myul{%
    \def\SOUL@everyspace{\underline{\space}\kern\z@}%
    \def\SOUL@everytoken{%
     \setbox0=\hbox{\the\SOUL@token}%
     \ifdim\dp0>\z@
        \raisebox{\dp0}{\underline{\phantom{\the\SOUL@token}}}%
        \whiten{1}\whiten{0}%
        \whiten{-1}\whiten{-2}%
        \llap{\the\SOUL@token}%
     \else
        \underline{\the\SOUL@token}%
     \fi}%
\SOUL@}
\makeatother

\newcommand*{\demp}{\fontfamily{lmtt}\selectfont}

\DeclareTextFontCommand{\textdemp}{\demp}

\begin{document}

\ifcomment
Multiline
comment
\fi
\ifcomment
\myul{Typesetting test}
% \color[rgb]{1,1,1}
$∑_i^n≠ 60º±∞π∆¬≈√j∫h≤≥µ$

$\CR \R\pro\ind\pro\gS\pro
\mqty[a&b\\c&d]$
$\pro\mathbb{P}$
$\dd{x}$

  \[
    \alpha(x)=\left\{
                \begin{array}{ll}
                  x\\
                  \frac{1}{1+e^{-kx}}\\
                  \frac{e^x-e^{-x}}{e^x+e^{-x}}
                \end{array}
              \right.
  \]

  $\expval{x}$
  
  $\chi_\rho(ghg\dmo)=\Tr(\rho_{ghg\dmo})=\Tr(\rho_g\circ\rho_h\circ\rho\dmo_g)=\Tr(\rho_h)\overset{\mbox{\scalebox{0.5}{$\Tr(AB)=\Tr(BA)$}}}{=}\chi_\rho(h)$
  	$\mathop{\oplus}_{\substack{x\in X}}$

$\mat(\rho_g)=(a_{ij}(g))_{\scriptsize \substack{1\leq i\leq d \\ 1\leq j\leq d}}$ et $\mat(\rho'_g)=(a'_{ij}(g))_{\scriptsize \substack{1\leq i'\leq d' \\ 1\leq j'\leq d'}}$



\[\int_a^b{\mathbb{R}^2}g(u, v)\dd{P_{XY}}(u, v)=\iint g(u,v) f_{XY}(u, v)\dd \lambda(u) \dd \lambda(v)\]
$$\lim_{x\to\infty} f(x)$$	
$$\iiiint_V \mu(t,u,v,w) \,dt\,du\,dv\,dw$$
$$\sum_{n=1}^{\infty} 2^{-n} = 1$$	
\begin{definition}
	Si $X$ et $Y$ sont 2 v.a. ou definit la \textsc{Covariance} entre $X$ et $Y$ comme
	$\cov(X,Y)\overset{\text{def}}{=}\E\left[(X-\E(X))(Y-\E(Y))\right]=\E(XY)-\E(X)\E(Y)$.
\end{definition}
\fi
\pagebreak

% \tableofcontents

% insert your code here
%% !TEX encoding = UTF-8 Unicode
% !TEX TS-program = xelatex

\documentclass[french]{report}

%\usepackage[utf8]{inputenc}
%\usepackage[T1]{fontenc}
\usepackage{babel}


\newif\ifcomment
%\commenttrue # Show comments

\usepackage{physics}
\usepackage{amssymb}


\usepackage{amsthm}
% \usepackage{thmtools}
\usepackage{mathtools}
\usepackage{amsfonts}

\usepackage{color}

\usepackage{tikz}

\usepackage{geometry}
\geometry{a5paper, margin=0.1in, right=1cm}

\usepackage{dsfont}

\usepackage{graphicx}
\graphicspath{ {images/} }

\usepackage{faktor}

\usepackage{IEEEtrantools}
\usepackage{enumerate}   
\usepackage[PostScript=dvips]{"/Users/aware/Documents/Courses/diagrams"}


\newtheorem{theorem}{Théorème}[section]
\renewcommand{\thetheorem}{\arabic{theorem}}
\newtheorem{lemme}{Lemme}[section]
\renewcommand{\thelemme}{\arabic{lemme}}
\newtheorem{proposition}{Proposition}[section]
\renewcommand{\theproposition}{\arabic{proposition}}
\newtheorem{notations}{Notations}[section]
\newtheorem{problem}{Problème}[section]
\newtheorem{corollary}{Corollaire}[theorem]
\renewcommand{\thecorollary}{\arabic{corollary}}
\newtheorem{property}{Propriété}[section]
\newtheorem{objective}{Objectif}[section]

\theoremstyle{definition}
\newtheorem{definition}{Définition}[section]
\renewcommand{\thedefinition}{\arabic{definition}}
\newtheorem{exercise}{Exercice}[chapter]
\renewcommand{\theexercise}{\arabic{exercise}}
\newtheorem{example}{Exemple}[chapter]
\renewcommand{\theexample}{\arabic{example}}
\newtheorem*{solution}{Solution}
\newtheorem*{application}{Application}
\newtheorem*{notation}{Notation}
\newtheorem*{vocabulary}{Vocabulaire}
\newtheorem*{properties}{Propriétés}



\theoremstyle{remark}
\newtheorem*{remark}{Remarque}
\newtheorem*{rappel}{Rappel}


\usepackage{etoolbox}
\AtBeginEnvironment{exercise}{\small}
\AtBeginEnvironment{example}{\small}

\usepackage{cases}
\usepackage[red]{mypack}

\usepackage[framemethod=TikZ]{mdframed}

\definecolor{bg}{rgb}{0.4,0.25,0.95}
\definecolor{pagebg}{rgb}{0,0,0.5}
\surroundwithmdframed[
   topline=false,
   rightline=false,
   bottomline=false,
   leftmargin=\parindent,
   skipabove=8pt,
   skipbelow=8pt,
   linecolor=blue,
   innerbottommargin=10pt,
   % backgroundcolor=bg,font=\color{orange}\sffamily, fontcolor=white
]{definition}

\usepackage{empheq}
\usepackage[most]{tcolorbox}

\newtcbox{\mymath}[1][]{%
    nobeforeafter, math upper, tcbox raise base,
    enhanced, colframe=blue!30!black,
    colback=red!10, boxrule=1pt,
    #1}

\usepackage{unixode}


\DeclareMathOperator{\ord}{ord}
\DeclareMathOperator{\orb}{orb}
\DeclareMathOperator{\stab}{stab}
\DeclareMathOperator{\Stab}{stab}
\DeclareMathOperator{\ppcm}{ppcm}
\DeclareMathOperator{\conj}{Conj}
\DeclareMathOperator{\End}{End}
\DeclareMathOperator{\rot}{rot}
\DeclareMathOperator{\trs}{trace}
\DeclareMathOperator{\Ind}{Ind}
\DeclareMathOperator{\mat}{Mat}
\DeclareMathOperator{\id}{Id}
\DeclareMathOperator{\vect}{vect}
\DeclareMathOperator{\img}{img}
\DeclareMathOperator{\cov}{Cov}
\DeclareMathOperator{\dist}{dist}
\DeclareMathOperator{\irr}{Irr}
\DeclareMathOperator{\image}{Im}
\DeclareMathOperator{\pd}{\partial}
\DeclareMathOperator{\epi}{epi}
\DeclareMathOperator{\Argmin}{Argmin}
\DeclareMathOperator{\dom}{dom}
\DeclareMathOperator{\proj}{proj}
\DeclareMathOperator{\ctg}{ctg}
\DeclareMathOperator{\supp}{supp}
\DeclareMathOperator{\argmin}{argmin}
\DeclareMathOperator{\mult}{mult}
\DeclareMathOperator{\ch}{ch}
\DeclareMathOperator{\sh}{sh}
\DeclareMathOperator{\rang}{rang}
\DeclareMathOperator{\diam}{diam}
\DeclareMathOperator{\Epigraphe}{Epigraphe}




\usepackage{xcolor}
\everymath{\color{blue}}
%\everymath{\color[rgb]{0,1,1}}
%\pagecolor[rgb]{0,0,0.5}


\newcommand*{\pdtest}[3][]{\ensuremath{\frac{\partial^{#1} #2}{\partial #3}}}

\newcommand*{\deffunc}[6][]{\ensuremath{
\begin{array}{rcl}
#2 : #3 &\rightarrow& #4\\
#5 &\mapsto& #6
\end{array}
}}

\newcommand{\eqcolon}{\mathrel{\resizebox{\widthof{$\mathord{=}$}}{\height}{ $\!\!=\!\!\resizebox{1.2\width}{0.8\height}{\raisebox{0.23ex}{$\mathop{:}$}}\!\!$ }}}
\newcommand{\coloneq}{\mathrel{\resizebox{\widthof{$\mathord{=}$}}{\height}{ $\!\!\resizebox{1.2\width}{0.8\height}{\raisebox{0.23ex}{$\mathop{:}$}}\!\!=\!\!$ }}}
\newcommand{\eqcolonl}{\ensuremath{\mathrel{=\!\!\mathop{:}}}}
\newcommand{\coloneql}{\ensuremath{\mathrel{\mathop{:} \!\! =}}}
\newcommand{\vc}[1]{% inline column vector
  \left(\begin{smallmatrix}#1\end{smallmatrix}\right)%
}
\newcommand{\vr}[1]{% inline row vector
  \begin{smallmatrix}(\,#1\,)\end{smallmatrix}%
}
\makeatletter
\newcommand*{\defeq}{\ =\mathrel{\rlap{%
                     \raisebox{0.3ex}{$\m@th\cdot$}}%
                     \raisebox{-0.3ex}{$\m@th\cdot$}}%
                     }
\makeatother

\newcommand{\mathcircle}[1]{% inline row vector
 \overset{\circ}{#1}
}
\newcommand{\ulim}{% low limit
 \underline{\lim}
}
\newcommand{\ssi}{% iff
\iff
}
\newcommand{\ps}[2]{
\expval{#1 | #2}
}
\newcommand{\df}[1]{
\mqty{#1}
}
\newcommand{\n}[1]{
\norm{#1}
}
\newcommand{\sys}[1]{
\left\{\smqty{#1}\right.
}


\newcommand{\eqdef}{\ensuremath{\overset{\text{def}}=}}


\def\Circlearrowright{\ensuremath{%
  \rotatebox[origin=c]{230}{$\circlearrowright$}}}

\newcommand\ct[1]{\text{\rmfamily\upshape #1}}
\newcommand\question[1]{ {\color{red} ...!? \small #1}}
\newcommand\caz[1]{\left\{\begin{array} #1 \end{array}\right.}
\newcommand\const{\text{\rmfamily\upshape const}}
\newcommand\toP{ \overset{\pro}{\to}}
\newcommand\toPP{ \overset{\text{PP}}{\to}}
\newcommand{\oeq}{\mathrel{\text{\textcircled{$=$}}}}





\usepackage{xcolor}
% \usepackage[normalem]{ulem}
\usepackage{lipsum}
\makeatletter
% \newcommand\colorwave[1][blue]{\bgroup \markoverwith{\lower3.5\p@\hbox{\sixly \textcolor{#1}{\char58}}}\ULon}
%\font\sixly=lasy6 % does not re-load if already loaded, so no memory problem.

\newmdtheoremenv[
linewidth= 1pt,linecolor= blue,%
leftmargin=20,rightmargin=20,innertopmargin=0pt, innerrightmargin=40,%
tikzsetting = { draw=lightgray, line width = 0.3pt,dashed,%
dash pattern = on 15pt off 3pt},%
splittopskip=\topskip,skipbelow=\baselineskip,%
skipabove=\baselineskip,ntheorem,roundcorner=0pt,
% backgroundcolor=pagebg,font=\color{orange}\sffamily, fontcolor=white
]{examplebox}{Exemple}[section]



\newcommand\R{\mathbb{R}}
\newcommand\Z{\mathbb{Z}}
\newcommand\N{\mathbb{N}}
\newcommand\E{\mathbb{E}}
\newcommand\F{\mathcal{F}}
\newcommand\cH{\mathcal{H}}
\newcommand\V{\mathbb{V}}
\newcommand\dmo{ ^{-1} }
\newcommand\kapa{\kappa}
\newcommand\im{Im}
\newcommand\hs{\mathcal{H}}





\usepackage{soul}

\makeatletter
\newcommand*{\whiten}[1]{\llap{\textcolor{white}{{\the\SOUL@token}}\hspace{#1pt}}}
\DeclareRobustCommand*\myul{%
    \def\SOUL@everyspace{\underline{\space}\kern\z@}%
    \def\SOUL@everytoken{%
     \setbox0=\hbox{\the\SOUL@token}%
     \ifdim\dp0>\z@
        \raisebox{\dp0}{\underline{\phantom{\the\SOUL@token}}}%
        \whiten{1}\whiten{0}%
        \whiten{-1}\whiten{-2}%
        \llap{\the\SOUL@token}%
     \else
        \underline{\the\SOUL@token}%
     \fi}%
\SOUL@}
\makeatother

\newcommand*{\demp}{\fontfamily{lmtt}\selectfont}

\DeclareTextFontCommand{\textdemp}{\demp}

\begin{document}

\ifcomment
Multiline
comment
\fi
\ifcomment
\myul{Typesetting test}
% \color[rgb]{1,1,1}
$∑_i^n≠ 60º±∞π∆¬≈√j∫h≤≥µ$

$\CR \R\pro\ind\pro\gS\pro
\mqty[a&b\\c&d]$
$\pro\mathbb{P}$
$\dd{x}$

  \[
    \alpha(x)=\left\{
                \begin{array}{ll}
                  x\\
                  \frac{1}{1+e^{-kx}}\\
                  \frac{e^x-e^{-x}}{e^x+e^{-x}}
                \end{array}
              \right.
  \]

  $\expval{x}$
  
  $\chi_\rho(ghg\dmo)=\Tr(\rho_{ghg\dmo})=\Tr(\rho_g\circ\rho_h\circ\rho\dmo_g)=\Tr(\rho_h)\overset{\mbox{\scalebox{0.5}{$\Tr(AB)=\Tr(BA)$}}}{=}\chi_\rho(h)$
  	$\mathop{\oplus}_{\substack{x\in X}}$

$\mat(\rho_g)=(a_{ij}(g))_{\scriptsize \substack{1\leq i\leq d \\ 1\leq j\leq d}}$ et $\mat(\rho'_g)=(a'_{ij}(g))_{\scriptsize \substack{1\leq i'\leq d' \\ 1\leq j'\leq d'}}$



\[\int_a^b{\mathbb{R}^2}g(u, v)\dd{P_{XY}}(u, v)=\iint g(u,v) f_{XY}(u, v)\dd \lambda(u) \dd \lambda(v)\]
$$\lim_{x\to\infty} f(x)$$	
$$\iiiint_V \mu(t,u,v,w) \,dt\,du\,dv\,dw$$
$$\sum_{n=1}^{\infty} 2^{-n} = 1$$	
\begin{definition}
	Si $X$ et $Y$ sont 2 v.a. ou definit la \textsc{Covariance} entre $X$ et $Y$ comme
	$\cov(X,Y)\overset{\text{def}}{=}\E\left[(X-\E(X))(Y-\E(Y))\right]=\E(XY)-\E(X)\E(Y)$.
\end{definition}
\fi
\pagebreak

% \tableofcontents

% insert your code here
%\input{./algebra/main.tex}
%\input{./geometrie-differentielle/main.tex}
%\input{./probabilite/main.tex}
%\input{./analyse-fonctionnelle/main.tex}
% \input{./Analyse-convexe-et-dualite-en-optimisation/main.tex}
%\input{./tikz/main.tex}
%\input{./Theorie-du-distributions/main.tex}
%\input{./optimisation/mine.tex}
 \input{./modelisation/main.tex}

% yves.aubry@univ-tln.fr : algebra

\end{document}

%% !TEX encoding = UTF-8 Unicode
% !TEX TS-program = xelatex

\documentclass[french]{report}

%\usepackage[utf8]{inputenc}
%\usepackage[T1]{fontenc}
\usepackage{babel}


\newif\ifcomment
%\commenttrue # Show comments

\usepackage{physics}
\usepackage{amssymb}


\usepackage{amsthm}
% \usepackage{thmtools}
\usepackage{mathtools}
\usepackage{amsfonts}

\usepackage{color}

\usepackage{tikz}

\usepackage{geometry}
\geometry{a5paper, margin=0.1in, right=1cm}

\usepackage{dsfont}

\usepackage{graphicx}
\graphicspath{ {images/} }

\usepackage{faktor}

\usepackage{IEEEtrantools}
\usepackage{enumerate}   
\usepackage[PostScript=dvips]{"/Users/aware/Documents/Courses/diagrams"}


\newtheorem{theorem}{Théorème}[section]
\renewcommand{\thetheorem}{\arabic{theorem}}
\newtheorem{lemme}{Lemme}[section]
\renewcommand{\thelemme}{\arabic{lemme}}
\newtheorem{proposition}{Proposition}[section]
\renewcommand{\theproposition}{\arabic{proposition}}
\newtheorem{notations}{Notations}[section]
\newtheorem{problem}{Problème}[section]
\newtheorem{corollary}{Corollaire}[theorem]
\renewcommand{\thecorollary}{\arabic{corollary}}
\newtheorem{property}{Propriété}[section]
\newtheorem{objective}{Objectif}[section]

\theoremstyle{definition}
\newtheorem{definition}{Définition}[section]
\renewcommand{\thedefinition}{\arabic{definition}}
\newtheorem{exercise}{Exercice}[chapter]
\renewcommand{\theexercise}{\arabic{exercise}}
\newtheorem{example}{Exemple}[chapter]
\renewcommand{\theexample}{\arabic{example}}
\newtheorem*{solution}{Solution}
\newtheorem*{application}{Application}
\newtheorem*{notation}{Notation}
\newtheorem*{vocabulary}{Vocabulaire}
\newtheorem*{properties}{Propriétés}



\theoremstyle{remark}
\newtheorem*{remark}{Remarque}
\newtheorem*{rappel}{Rappel}


\usepackage{etoolbox}
\AtBeginEnvironment{exercise}{\small}
\AtBeginEnvironment{example}{\small}

\usepackage{cases}
\usepackage[red]{mypack}

\usepackage[framemethod=TikZ]{mdframed}

\definecolor{bg}{rgb}{0.4,0.25,0.95}
\definecolor{pagebg}{rgb}{0,0,0.5}
\surroundwithmdframed[
   topline=false,
   rightline=false,
   bottomline=false,
   leftmargin=\parindent,
   skipabove=8pt,
   skipbelow=8pt,
   linecolor=blue,
   innerbottommargin=10pt,
   % backgroundcolor=bg,font=\color{orange}\sffamily, fontcolor=white
]{definition}

\usepackage{empheq}
\usepackage[most]{tcolorbox}

\newtcbox{\mymath}[1][]{%
    nobeforeafter, math upper, tcbox raise base,
    enhanced, colframe=blue!30!black,
    colback=red!10, boxrule=1pt,
    #1}

\usepackage{unixode}


\DeclareMathOperator{\ord}{ord}
\DeclareMathOperator{\orb}{orb}
\DeclareMathOperator{\stab}{stab}
\DeclareMathOperator{\Stab}{stab}
\DeclareMathOperator{\ppcm}{ppcm}
\DeclareMathOperator{\conj}{Conj}
\DeclareMathOperator{\End}{End}
\DeclareMathOperator{\rot}{rot}
\DeclareMathOperator{\trs}{trace}
\DeclareMathOperator{\Ind}{Ind}
\DeclareMathOperator{\mat}{Mat}
\DeclareMathOperator{\id}{Id}
\DeclareMathOperator{\vect}{vect}
\DeclareMathOperator{\img}{img}
\DeclareMathOperator{\cov}{Cov}
\DeclareMathOperator{\dist}{dist}
\DeclareMathOperator{\irr}{Irr}
\DeclareMathOperator{\image}{Im}
\DeclareMathOperator{\pd}{\partial}
\DeclareMathOperator{\epi}{epi}
\DeclareMathOperator{\Argmin}{Argmin}
\DeclareMathOperator{\dom}{dom}
\DeclareMathOperator{\proj}{proj}
\DeclareMathOperator{\ctg}{ctg}
\DeclareMathOperator{\supp}{supp}
\DeclareMathOperator{\argmin}{argmin}
\DeclareMathOperator{\mult}{mult}
\DeclareMathOperator{\ch}{ch}
\DeclareMathOperator{\sh}{sh}
\DeclareMathOperator{\rang}{rang}
\DeclareMathOperator{\diam}{diam}
\DeclareMathOperator{\Epigraphe}{Epigraphe}




\usepackage{xcolor}
\everymath{\color{blue}}
%\everymath{\color[rgb]{0,1,1}}
%\pagecolor[rgb]{0,0,0.5}


\newcommand*{\pdtest}[3][]{\ensuremath{\frac{\partial^{#1} #2}{\partial #3}}}

\newcommand*{\deffunc}[6][]{\ensuremath{
\begin{array}{rcl}
#2 : #3 &\rightarrow& #4\\
#5 &\mapsto& #6
\end{array}
}}

\newcommand{\eqcolon}{\mathrel{\resizebox{\widthof{$\mathord{=}$}}{\height}{ $\!\!=\!\!\resizebox{1.2\width}{0.8\height}{\raisebox{0.23ex}{$\mathop{:}$}}\!\!$ }}}
\newcommand{\coloneq}{\mathrel{\resizebox{\widthof{$\mathord{=}$}}{\height}{ $\!\!\resizebox{1.2\width}{0.8\height}{\raisebox{0.23ex}{$\mathop{:}$}}\!\!=\!\!$ }}}
\newcommand{\eqcolonl}{\ensuremath{\mathrel{=\!\!\mathop{:}}}}
\newcommand{\coloneql}{\ensuremath{\mathrel{\mathop{:} \!\! =}}}
\newcommand{\vc}[1]{% inline column vector
  \left(\begin{smallmatrix}#1\end{smallmatrix}\right)%
}
\newcommand{\vr}[1]{% inline row vector
  \begin{smallmatrix}(\,#1\,)\end{smallmatrix}%
}
\makeatletter
\newcommand*{\defeq}{\ =\mathrel{\rlap{%
                     \raisebox{0.3ex}{$\m@th\cdot$}}%
                     \raisebox{-0.3ex}{$\m@th\cdot$}}%
                     }
\makeatother

\newcommand{\mathcircle}[1]{% inline row vector
 \overset{\circ}{#1}
}
\newcommand{\ulim}{% low limit
 \underline{\lim}
}
\newcommand{\ssi}{% iff
\iff
}
\newcommand{\ps}[2]{
\expval{#1 | #2}
}
\newcommand{\df}[1]{
\mqty{#1}
}
\newcommand{\n}[1]{
\norm{#1}
}
\newcommand{\sys}[1]{
\left\{\smqty{#1}\right.
}


\newcommand{\eqdef}{\ensuremath{\overset{\text{def}}=}}


\def\Circlearrowright{\ensuremath{%
  \rotatebox[origin=c]{230}{$\circlearrowright$}}}

\newcommand\ct[1]{\text{\rmfamily\upshape #1}}
\newcommand\question[1]{ {\color{red} ...!? \small #1}}
\newcommand\caz[1]{\left\{\begin{array} #1 \end{array}\right.}
\newcommand\const{\text{\rmfamily\upshape const}}
\newcommand\toP{ \overset{\pro}{\to}}
\newcommand\toPP{ \overset{\text{PP}}{\to}}
\newcommand{\oeq}{\mathrel{\text{\textcircled{$=$}}}}





\usepackage{xcolor}
% \usepackage[normalem]{ulem}
\usepackage{lipsum}
\makeatletter
% \newcommand\colorwave[1][blue]{\bgroup \markoverwith{\lower3.5\p@\hbox{\sixly \textcolor{#1}{\char58}}}\ULon}
%\font\sixly=lasy6 % does not re-load if already loaded, so no memory problem.

\newmdtheoremenv[
linewidth= 1pt,linecolor= blue,%
leftmargin=20,rightmargin=20,innertopmargin=0pt, innerrightmargin=40,%
tikzsetting = { draw=lightgray, line width = 0.3pt,dashed,%
dash pattern = on 15pt off 3pt},%
splittopskip=\topskip,skipbelow=\baselineskip,%
skipabove=\baselineskip,ntheorem,roundcorner=0pt,
% backgroundcolor=pagebg,font=\color{orange}\sffamily, fontcolor=white
]{examplebox}{Exemple}[section]



\newcommand\R{\mathbb{R}}
\newcommand\Z{\mathbb{Z}}
\newcommand\N{\mathbb{N}}
\newcommand\E{\mathbb{E}}
\newcommand\F{\mathcal{F}}
\newcommand\cH{\mathcal{H}}
\newcommand\V{\mathbb{V}}
\newcommand\dmo{ ^{-1} }
\newcommand\kapa{\kappa}
\newcommand\im{Im}
\newcommand\hs{\mathcal{H}}





\usepackage{soul}

\makeatletter
\newcommand*{\whiten}[1]{\llap{\textcolor{white}{{\the\SOUL@token}}\hspace{#1pt}}}
\DeclareRobustCommand*\myul{%
    \def\SOUL@everyspace{\underline{\space}\kern\z@}%
    \def\SOUL@everytoken{%
     \setbox0=\hbox{\the\SOUL@token}%
     \ifdim\dp0>\z@
        \raisebox{\dp0}{\underline{\phantom{\the\SOUL@token}}}%
        \whiten{1}\whiten{0}%
        \whiten{-1}\whiten{-2}%
        \llap{\the\SOUL@token}%
     \else
        \underline{\the\SOUL@token}%
     \fi}%
\SOUL@}
\makeatother

\newcommand*{\demp}{\fontfamily{lmtt}\selectfont}

\DeclareTextFontCommand{\textdemp}{\demp}

\begin{document}

\ifcomment
Multiline
comment
\fi
\ifcomment
\myul{Typesetting test}
% \color[rgb]{1,1,1}
$∑_i^n≠ 60º±∞π∆¬≈√j∫h≤≥µ$

$\CR \R\pro\ind\pro\gS\pro
\mqty[a&b\\c&d]$
$\pro\mathbb{P}$
$\dd{x}$

  \[
    \alpha(x)=\left\{
                \begin{array}{ll}
                  x\\
                  \frac{1}{1+e^{-kx}}\\
                  \frac{e^x-e^{-x}}{e^x+e^{-x}}
                \end{array}
              \right.
  \]

  $\expval{x}$
  
  $\chi_\rho(ghg\dmo)=\Tr(\rho_{ghg\dmo})=\Tr(\rho_g\circ\rho_h\circ\rho\dmo_g)=\Tr(\rho_h)\overset{\mbox{\scalebox{0.5}{$\Tr(AB)=\Tr(BA)$}}}{=}\chi_\rho(h)$
  	$\mathop{\oplus}_{\substack{x\in X}}$

$\mat(\rho_g)=(a_{ij}(g))_{\scriptsize \substack{1\leq i\leq d \\ 1\leq j\leq d}}$ et $\mat(\rho'_g)=(a'_{ij}(g))_{\scriptsize \substack{1\leq i'\leq d' \\ 1\leq j'\leq d'}}$



\[\int_a^b{\mathbb{R}^2}g(u, v)\dd{P_{XY}}(u, v)=\iint g(u,v) f_{XY}(u, v)\dd \lambda(u) \dd \lambda(v)\]
$$\lim_{x\to\infty} f(x)$$	
$$\iiiint_V \mu(t,u,v,w) \,dt\,du\,dv\,dw$$
$$\sum_{n=1}^{\infty} 2^{-n} = 1$$	
\begin{definition}
	Si $X$ et $Y$ sont 2 v.a. ou definit la \textsc{Covariance} entre $X$ et $Y$ comme
	$\cov(X,Y)\overset{\text{def}}{=}\E\left[(X-\E(X))(Y-\E(Y))\right]=\E(XY)-\E(X)\E(Y)$.
\end{definition}
\fi
\pagebreak

% \tableofcontents

% insert your code here
%\input{./algebra/main.tex}
%\input{./geometrie-differentielle/main.tex}
%\input{./probabilite/main.tex}
%\input{./analyse-fonctionnelle/main.tex}
% \input{./Analyse-convexe-et-dualite-en-optimisation/main.tex}
%\input{./tikz/main.tex}
%\input{./Theorie-du-distributions/main.tex}
%\input{./optimisation/mine.tex}
 \input{./modelisation/main.tex}

% yves.aubry@univ-tln.fr : algebra

\end{document}

%% !TEX encoding = UTF-8 Unicode
% !TEX TS-program = xelatex

\documentclass[french]{report}

%\usepackage[utf8]{inputenc}
%\usepackage[T1]{fontenc}
\usepackage{babel}


\newif\ifcomment
%\commenttrue # Show comments

\usepackage{physics}
\usepackage{amssymb}


\usepackage{amsthm}
% \usepackage{thmtools}
\usepackage{mathtools}
\usepackage{amsfonts}

\usepackage{color}

\usepackage{tikz}

\usepackage{geometry}
\geometry{a5paper, margin=0.1in, right=1cm}

\usepackage{dsfont}

\usepackage{graphicx}
\graphicspath{ {images/} }

\usepackage{faktor}

\usepackage{IEEEtrantools}
\usepackage{enumerate}   
\usepackage[PostScript=dvips]{"/Users/aware/Documents/Courses/diagrams"}


\newtheorem{theorem}{Théorème}[section]
\renewcommand{\thetheorem}{\arabic{theorem}}
\newtheorem{lemme}{Lemme}[section]
\renewcommand{\thelemme}{\arabic{lemme}}
\newtheorem{proposition}{Proposition}[section]
\renewcommand{\theproposition}{\arabic{proposition}}
\newtheorem{notations}{Notations}[section]
\newtheorem{problem}{Problème}[section]
\newtheorem{corollary}{Corollaire}[theorem]
\renewcommand{\thecorollary}{\arabic{corollary}}
\newtheorem{property}{Propriété}[section]
\newtheorem{objective}{Objectif}[section]

\theoremstyle{definition}
\newtheorem{definition}{Définition}[section]
\renewcommand{\thedefinition}{\arabic{definition}}
\newtheorem{exercise}{Exercice}[chapter]
\renewcommand{\theexercise}{\arabic{exercise}}
\newtheorem{example}{Exemple}[chapter]
\renewcommand{\theexample}{\arabic{example}}
\newtheorem*{solution}{Solution}
\newtheorem*{application}{Application}
\newtheorem*{notation}{Notation}
\newtheorem*{vocabulary}{Vocabulaire}
\newtheorem*{properties}{Propriétés}



\theoremstyle{remark}
\newtheorem*{remark}{Remarque}
\newtheorem*{rappel}{Rappel}


\usepackage{etoolbox}
\AtBeginEnvironment{exercise}{\small}
\AtBeginEnvironment{example}{\small}

\usepackage{cases}
\usepackage[red]{mypack}

\usepackage[framemethod=TikZ]{mdframed}

\definecolor{bg}{rgb}{0.4,0.25,0.95}
\definecolor{pagebg}{rgb}{0,0,0.5}
\surroundwithmdframed[
   topline=false,
   rightline=false,
   bottomline=false,
   leftmargin=\parindent,
   skipabove=8pt,
   skipbelow=8pt,
   linecolor=blue,
   innerbottommargin=10pt,
   % backgroundcolor=bg,font=\color{orange}\sffamily, fontcolor=white
]{definition}

\usepackage{empheq}
\usepackage[most]{tcolorbox}

\newtcbox{\mymath}[1][]{%
    nobeforeafter, math upper, tcbox raise base,
    enhanced, colframe=blue!30!black,
    colback=red!10, boxrule=1pt,
    #1}

\usepackage{unixode}


\DeclareMathOperator{\ord}{ord}
\DeclareMathOperator{\orb}{orb}
\DeclareMathOperator{\stab}{stab}
\DeclareMathOperator{\Stab}{stab}
\DeclareMathOperator{\ppcm}{ppcm}
\DeclareMathOperator{\conj}{Conj}
\DeclareMathOperator{\End}{End}
\DeclareMathOperator{\rot}{rot}
\DeclareMathOperator{\trs}{trace}
\DeclareMathOperator{\Ind}{Ind}
\DeclareMathOperator{\mat}{Mat}
\DeclareMathOperator{\id}{Id}
\DeclareMathOperator{\vect}{vect}
\DeclareMathOperator{\img}{img}
\DeclareMathOperator{\cov}{Cov}
\DeclareMathOperator{\dist}{dist}
\DeclareMathOperator{\irr}{Irr}
\DeclareMathOperator{\image}{Im}
\DeclareMathOperator{\pd}{\partial}
\DeclareMathOperator{\epi}{epi}
\DeclareMathOperator{\Argmin}{Argmin}
\DeclareMathOperator{\dom}{dom}
\DeclareMathOperator{\proj}{proj}
\DeclareMathOperator{\ctg}{ctg}
\DeclareMathOperator{\supp}{supp}
\DeclareMathOperator{\argmin}{argmin}
\DeclareMathOperator{\mult}{mult}
\DeclareMathOperator{\ch}{ch}
\DeclareMathOperator{\sh}{sh}
\DeclareMathOperator{\rang}{rang}
\DeclareMathOperator{\diam}{diam}
\DeclareMathOperator{\Epigraphe}{Epigraphe}




\usepackage{xcolor}
\everymath{\color{blue}}
%\everymath{\color[rgb]{0,1,1}}
%\pagecolor[rgb]{0,0,0.5}


\newcommand*{\pdtest}[3][]{\ensuremath{\frac{\partial^{#1} #2}{\partial #3}}}

\newcommand*{\deffunc}[6][]{\ensuremath{
\begin{array}{rcl}
#2 : #3 &\rightarrow& #4\\
#5 &\mapsto& #6
\end{array}
}}

\newcommand{\eqcolon}{\mathrel{\resizebox{\widthof{$\mathord{=}$}}{\height}{ $\!\!=\!\!\resizebox{1.2\width}{0.8\height}{\raisebox{0.23ex}{$\mathop{:}$}}\!\!$ }}}
\newcommand{\coloneq}{\mathrel{\resizebox{\widthof{$\mathord{=}$}}{\height}{ $\!\!\resizebox{1.2\width}{0.8\height}{\raisebox{0.23ex}{$\mathop{:}$}}\!\!=\!\!$ }}}
\newcommand{\eqcolonl}{\ensuremath{\mathrel{=\!\!\mathop{:}}}}
\newcommand{\coloneql}{\ensuremath{\mathrel{\mathop{:} \!\! =}}}
\newcommand{\vc}[1]{% inline column vector
  \left(\begin{smallmatrix}#1\end{smallmatrix}\right)%
}
\newcommand{\vr}[1]{% inline row vector
  \begin{smallmatrix}(\,#1\,)\end{smallmatrix}%
}
\makeatletter
\newcommand*{\defeq}{\ =\mathrel{\rlap{%
                     \raisebox{0.3ex}{$\m@th\cdot$}}%
                     \raisebox{-0.3ex}{$\m@th\cdot$}}%
                     }
\makeatother

\newcommand{\mathcircle}[1]{% inline row vector
 \overset{\circ}{#1}
}
\newcommand{\ulim}{% low limit
 \underline{\lim}
}
\newcommand{\ssi}{% iff
\iff
}
\newcommand{\ps}[2]{
\expval{#1 | #2}
}
\newcommand{\df}[1]{
\mqty{#1}
}
\newcommand{\n}[1]{
\norm{#1}
}
\newcommand{\sys}[1]{
\left\{\smqty{#1}\right.
}


\newcommand{\eqdef}{\ensuremath{\overset{\text{def}}=}}


\def\Circlearrowright{\ensuremath{%
  \rotatebox[origin=c]{230}{$\circlearrowright$}}}

\newcommand\ct[1]{\text{\rmfamily\upshape #1}}
\newcommand\question[1]{ {\color{red} ...!? \small #1}}
\newcommand\caz[1]{\left\{\begin{array} #1 \end{array}\right.}
\newcommand\const{\text{\rmfamily\upshape const}}
\newcommand\toP{ \overset{\pro}{\to}}
\newcommand\toPP{ \overset{\text{PP}}{\to}}
\newcommand{\oeq}{\mathrel{\text{\textcircled{$=$}}}}





\usepackage{xcolor}
% \usepackage[normalem]{ulem}
\usepackage{lipsum}
\makeatletter
% \newcommand\colorwave[1][blue]{\bgroup \markoverwith{\lower3.5\p@\hbox{\sixly \textcolor{#1}{\char58}}}\ULon}
%\font\sixly=lasy6 % does not re-load if already loaded, so no memory problem.

\newmdtheoremenv[
linewidth= 1pt,linecolor= blue,%
leftmargin=20,rightmargin=20,innertopmargin=0pt, innerrightmargin=40,%
tikzsetting = { draw=lightgray, line width = 0.3pt,dashed,%
dash pattern = on 15pt off 3pt},%
splittopskip=\topskip,skipbelow=\baselineskip,%
skipabove=\baselineskip,ntheorem,roundcorner=0pt,
% backgroundcolor=pagebg,font=\color{orange}\sffamily, fontcolor=white
]{examplebox}{Exemple}[section]



\newcommand\R{\mathbb{R}}
\newcommand\Z{\mathbb{Z}}
\newcommand\N{\mathbb{N}}
\newcommand\E{\mathbb{E}}
\newcommand\F{\mathcal{F}}
\newcommand\cH{\mathcal{H}}
\newcommand\V{\mathbb{V}}
\newcommand\dmo{ ^{-1} }
\newcommand\kapa{\kappa}
\newcommand\im{Im}
\newcommand\hs{\mathcal{H}}





\usepackage{soul}

\makeatletter
\newcommand*{\whiten}[1]{\llap{\textcolor{white}{{\the\SOUL@token}}\hspace{#1pt}}}
\DeclareRobustCommand*\myul{%
    \def\SOUL@everyspace{\underline{\space}\kern\z@}%
    \def\SOUL@everytoken{%
     \setbox0=\hbox{\the\SOUL@token}%
     \ifdim\dp0>\z@
        \raisebox{\dp0}{\underline{\phantom{\the\SOUL@token}}}%
        \whiten{1}\whiten{0}%
        \whiten{-1}\whiten{-2}%
        \llap{\the\SOUL@token}%
     \else
        \underline{\the\SOUL@token}%
     \fi}%
\SOUL@}
\makeatother

\newcommand*{\demp}{\fontfamily{lmtt}\selectfont}

\DeclareTextFontCommand{\textdemp}{\demp}

\begin{document}

\ifcomment
Multiline
comment
\fi
\ifcomment
\myul{Typesetting test}
% \color[rgb]{1,1,1}
$∑_i^n≠ 60º±∞π∆¬≈√j∫h≤≥µ$

$\CR \R\pro\ind\pro\gS\pro
\mqty[a&b\\c&d]$
$\pro\mathbb{P}$
$\dd{x}$

  \[
    \alpha(x)=\left\{
                \begin{array}{ll}
                  x\\
                  \frac{1}{1+e^{-kx}}\\
                  \frac{e^x-e^{-x}}{e^x+e^{-x}}
                \end{array}
              \right.
  \]

  $\expval{x}$
  
  $\chi_\rho(ghg\dmo)=\Tr(\rho_{ghg\dmo})=\Tr(\rho_g\circ\rho_h\circ\rho\dmo_g)=\Tr(\rho_h)\overset{\mbox{\scalebox{0.5}{$\Tr(AB)=\Tr(BA)$}}}{=}\chi_\rho(h)$
  	$\mathop{\oplus}_{\substack{x\in X}}$

$\mat(\rho_g)=(a_{ij}(g))_{\scriptsize \substack{1\leq i\leq d \\ 1\leq j\leq d}}$ et $\mat(\rho'_g)=(a'_{ij}(g))_{\scriptsize \substack{1\leq i'\leq d' \\ 1\leq j'\leq d'}}$



\[\int_a^b{\mathbb{R}^2}g(u, v)\dd{P_{XY}}(u, v)=\iint g(u,v) f_{XY}(u, v)\dd \lambda(u) \dd \lambda(v)\]
$$\lim_{x\to\infty} f(x)$$	
$$\iiiint_V \mu(t,u,v,w) \,dt\,du\,dv\,dw$$
$$\sum_{n=1}^{\infty} 2^{-n} = 1$$	
\begin{definition}
	Si $X$ et $Y$ sont 2 v.a. ou definit la \textsc{Covariance} entre $X$ et $Y$ comme
	$\cov(X,Y)\overset{\text{def}}{=}\E\left[(X-\E(X))(Y-\E(Y))\right]=\E(XY)-\E(X)\E(Y)$.
\end{definition}
\fi
\pagebreak

% \tableofcontents

% insert your code here
%\input{./algebra/main.tex}
%\input{./geometrie-differentielle/main.tex}
%\input{./probabilite/main.tex}
%\input{./analyse-fonctionnelle/main.tex}
% \input{./Analyse-convexe-et-dualite-en-optimisation/main.tex}
%\input{./tikz/main.tex}
%\input{./Theorie-du-distributions/main.tex}
%\input{./optimisation/mine.tex}
 \input{./modelisation/main.tex}

% yves.aubry@univ-tln.fr : algebra

\end{document}

%% !TEX encoding = UTF-8 Unicode
% !TEX TS-program = xelatex

\documentclass[french]{report}

%\usepackage[utf8]{inputenc}
%\usepackage[T1]{fontenc}
\usepackage{babel}


\newif\ifcomment
%\commenttrue # Show comments

\usepackage{physics}
\usepackage{amssymb}


\usepackage{amsthm}
% \usepackage{thmtools}
\usepackage{mathtools}
\usepackage{amsfonts}

\usepackage{color}

\usepackage{tikz}

\usepackage{geometry}
\geometry{a5paper, margin=0.1in, right=1cm}

\usepackage{dsfont}

\usepackage{graphicx}
\graphicspath{ {images/} }

\usepackage{faktor}

\usepackage{IEEEtrantools}
\usepackage{enumerate}   
\usepackage[PostScript=dvips]{"/Users/aware/Documents/Courses/diagrams"}


\newtheorem{theorem}{Théorème}[section]
\renewcommand{\thetheorem}{\arabic{theorem}}
\newtheorem{lemme}{Lemme}[section]
\renewcommand{\thelemme}{\arabic{lemme}}
\newtheorem{proposition}{Proposition}[section]
\renewcommand{\theproposition}{\arabic{proposition}}
\newtheorem{notations}{Notations}[section]
\newtheorem{problem}{Problème}[section]
\newtheorem{corollary}{Corollaire}[theorem]
\renewcommand{\thecorollary}{\arabic{corollary}}
\newtheorem{property}{Propriété}[section]
\newtheorem{objective}{Objectif}[section]

\theoremstyle{definition}
\newtheorem{definition}{Définition}[section]
\renewcommand{\thedefinition}{\arabic{definition}}
\newtheorem{exercise}{Exercice}[chapter]
\renewcommand{\theexercise}{\arabic{exercise}}
\newtheorem{example}{Exemple}[chapter]
\renewcommand{\theexample}{\arabic{example}}
\newtheorem*{solution}{Solution}
\newtheorem*{application}{Application}
\newtheorem*{notation}{Notation}
\newtheorem*{vocabulary}{Vocabulaire}
\newtheorem*{properties}{Propriétés}



\theoremstyle{remark}
\newtheorem*{remark}{Remarque}
\newtheorem*{rappel}{Rappel}


\usepackage{etoolbox}
\AtBeginEnvironment{exercise}{\small}
\AtBeginEnvironment{example}{\small}

\usepackage{cases}
\usepackage[red]{mypack}

\usepackage[framemethod=TikZ]{mdframed}

\definecolor{bg}{rgb}{0.4,0.25,0.95}
\definecolor{pagebg}{rgb}{0,0,0.5}
\surroundwithmdframed[
   topline=false,
   rightline=false,
   bottomline=false,
   leftmargin=\parindent,
   skipabove=8pt,
   skipbelow=8pt,
   linecolor=blue,
   innerbottommargin=10pt,
   % backgroundcolor=bg,font=\color{orange}\sffamily, fontcolor=white
]{definition}

\usepackage{empheq}
\usepackage[most]{tcolorbox}

\newtcbox{\mymath}[1][]{%
    nobeforeafter, math upper, tcbox raise base,
    enhanced, colframe=blue!30!black,
    colback=red!10, boxrule=1pt,
    #1}

\usepackage{unixode}


\DeclareMathOperator{\ord}{ord}
\DeclareMathOperator{\orb}{orb}
\DeclareMathOperator{\stab}{stab}
\DeclareMathOperator{\Stab}{stab}
\DeclareMathOperator{\ppcm}{ppcm}
\DeclareMathOperator{\conj}{Conj}
\DeclareMathOperator{\End}{End}
\DeclareMathOperator{\rot}{rot}
\DeclareMathOperator{\trs}{trace}
\DeclareMathOperator{\Ind}{Ind}
\DeclareMathOperator{\mat}{Mat}
\DeclareMathOperator{\id}{Id}
\DeclareMathOperator{\vect}{vect}
\DeclareMathOperator{\img}{img}
\DeclareMathOperator{\cov}{Cov}
\DeclareMathOperator{\dist}{dist}
\DeclareMathOperator{\irr}{Irr}
\DeclareMathOperator{\image}{Im}
\DeclareMathOperator{\pd}{\partial}
\DeclareMathOperator{\epi}{epi}
\DeclareMathOperator{\Argmin}{Argmin}
\DeclareMathOperator{\dom}{dom}
\DeclareMathOperator{\proj}{proj}
\DeclareMathOperator{\ctg}{ctg}
\DeclareMathOperator{\supp}{supp}
\DeclareMathOperator{\argmin}{argmin}
\DeclareMathOperator{\mult}{mult}
\DeclareMathOperator{\ch}{ch}
\DeclareMathOperator{\sh}{sh}
\DeclareMathOperator{\rang}{rang}
\DeclareMathOperator{\diam}{diam}
\DeclareMathOperator{\Epigraphe}{Epigraphe}




\usepackage{xcolor}
\everymath{\color{blue}}
%\everymath{\color[rgb]{0,1,1}}
%\pagecolor[rgb]{0,0,0.5}


\newcommand*{\pdtest}[3][]{\ensuremath{\frac{\partial^{#1} #2}{\partial #3}}}

\newcommand*{\deffunc}[6][]{\ensuremath{
\begin{array}{rcl}
#2 : #3 &\rightarrow& #4\\
#5 &\mapsto& #6
\end{array}
}}

\newcommand{\eqcolon}{\mathrel{\resizebox{\widthof{$\mathord{=}$}}{\height}{ $\!\!=\!\!\resizebox{1.2\width}{0.8\height}{\raisebox{0.23ex}{$\mathop{:}$}}\!\!$ }}}
\newcommand{\coloneq}{\mathrel{\resizebox{\widthof{$\mathord{=}$}}{\height}{ $\!\!\resizebox{1.2\width}{0.8\height}{\raisebox{0.23ex}{$\mathop{:}$}}\!\!=\!\!$ }}}
\newcommand{\eqcolonl}{\ensuremath{\mathrel{=\!\!\mathop{:}}}}
\newcommand{\coloneql}{\ensuremath{\mathrel{\mathop{:} \!\! =}}}
\newcommand{\vc}[1]{% inline column vector
  \left(\begin{smallmatrix}#1\end{smallmatrix}\right)%
}
\newcommand{\vr}[1]{% inline row vector
  \begin{smallmatrix}(\,#1\,)\end{smallmatrix}%
}
\makeatletter
\newcommand*{\defeq}{\ =\mathrel{\rlap{%
                     \raisebox{0.3ex}{$\m@th\cdot$}}%
                     \raisebox{-0.3ex}{$\m@th\cdot$}}%
                     }
\makeatother

\newcommand{\mathcircle}[1]{% inline row vector
 \overset{\circ}{#1}
}
\newcommand{\ulim}{% low limit
 \underline{\lim}
}
\newcommand{\ssi}{% iff
\iff
}
\newcommand{\ps}[2]{
\expval{#1 | #2}
}
\newcommand{\df}[1]{
\mqty{#1}
}
\newcommand{\n}[1]{
\norm{#1}
}
\newcommand{\sys}[1]{
\left\{\smqty{#1}\right.
}


\newcommand{\eqdef}{\ensuremath{\overset{\text{def}}=}}


\def\Circlearrowright{\ensuremath{%
  \rotatebox[origin=c]{230}{$\circlearrowright$}}}

\newcommand\ct[1]{\text{\rmfamily\upshape #1}}
\newcommand\question[1]{ {\color{red} ...!? \small #1}}
\newcommand\caz[1]{\left\{\begin{array} #1 \end{array}\right.}
\newcommand\const{\text{\rmfamily\upshape const}}
\newcommand\toP{ \overset{\pro}{\to}}
\newcommand\toPP{ \overset{\text{PP}}{\to}}
\newcommand{\oeq}{\mathrel{\text{\textcircled{$=$}}}}





\usepackage{xcolor}
% \usepackage[normalem]{ulem}
\usepackage{lipsum}
\makeatletter
% \newcommand\colorwave[1][blue]{\bgroup \markoverwith{\lower3.5\p@\hbox{\sixly \textcolor{#1}{\char58}}}\ULon}
%\font\sixly=lasy6 % does not re-load if already loaded, so no memory problem.

\newmdtheoremenv[
linewidth= 1pt,linecolor= blue,%
leftmargin=20,rightmargin=20,innertopmargin=0pt, innerrightmargin=40,%
tikzsetting = { draw=lightgray, line width = 0.3pt,dashed,%
dash pattern = on 15pt off 3pt},%
splittopskip=\topskip,skipbelow=\baselineskip,%
skipabove=\baselineskip,ntheorem,roundcorner=0pt,
% backgroundcolor=pagebg,font=\color{orange}\sffamily, fontcolor=white
]{examplebox}{Exemple}[section]



\newcommand\R{\mathbb{R}}
\newcommand\Z{\mathbb{Z}}
\newcommand\N{\mathbb{N}}
\newcommand\E{\mathbb{E}}
\newcommand\F{\mathcal{F}}
\newcommand\cH{\mathcal{H}}
\newcommand\V{\mathbb{V}}
\newcommand\dmo{ ^{-1} }
\newcommand\kapa{\kappa}
\newcommand\im{Im}
\newcommand\hs{\mathcal{H}}





\usepackage{soul}

\makeatletter
\newcommand*{\whiten}[1]{\llap{\textcolor{white}{{\the\SOUL@token}}\hspace{#1pt}}}
\DeclareRobustCommand*\myul{%
    \def\SOUL@everyspace{\underline{\space}\kern\z@}%
    \def\SOUL@everytoken{%
     \setbox0=\hbox{\the\SOUL@token}%
     \ifdim\dp0>\z@
        \raisebox{\dp0}{\underline{\phantom{\the\SOUL@token}}}%
        \whiten{1}\whiten{0}%
        \whiten{-1}\whiten{-2}%
        \llap{\the\SOUL@token}%
     \else
        \underline{\the\SOUL@token}%
     \fi}%
\SOUL@}
\makeatother

\newcommand*{\demp}{\fontfamily{lmtt}\selectfont}

\DeclareTextFontCommand{\textdemp}{\demp}

\begin{document}

\ifcomment
Multiline
comment
\fi
\ifcomment
\myul{Typesetting test}
% \color[rgb]{1,1,1}
$∑_i^n≠ 60º±∞π∆¬≈√j∫h≤≥µ$

$\CR \R\pro\ind\pro\gS\pro
\mqty[a&b\\c&d]$
$\pro\mathbb{P}$
$\dd{x}$

  \[
    \alpha(x)=\left\{
                \begin{array}{ll}
                  x\\
                  \frac{1}{1+e^{-kx}}\\
                  \frac{e^x-e^{-x}}{e^x+e^{-x}}
                \end{array}
              \right.
  \]

  $\expval{x}$
  
  $\chi_\rho(ghg\dmo)=\Tr(\rho_{ghg\dmo})=\Tr(\rho_g\circ\rho_h\circ\rho\dmo_g)=\Tr(\rho_h)\overset{\mbox{\scalebox{0.5}{$\Tr(AB)=\Tr(BA)$}}}{=}\chi_\rho(h)$
  	$\mathop{\oplus}_{\substack{x\in X}}$

$\mat(\rho_g)=(a_{ij}(g))_{\scriptsize \substack{1\leq i\leq d \\ 1\leq j\leq d}}$ et $\mat(\rho'_g)=(a'_{ij}(g))_{\scriptsize \substack{1\leq i'\leq d' \\ 1\leq j'\leq d'}}$



\[\int_a^b{\mathbb{R}^2}g(u, v)\dd{P_{XY}}(u, v)=\iint g(u,v) f_{XY}(u, v)\dd \lambda(u) \dd \lambda(v)\]
$$\lim_{x\to\infty} f(x)$$	
$$\iiiint_V \mu(t,u,v,w) \,dt\,du\,dv\,dw$$
$$\sum_{n=1}^{\infty} 2^{-n} = 1$$	
\begin{definition}
	Si $X$ et $Y$ sont 2 v.a. ou definit la \textsc{Covariance} entre $X$ et $Y$ comme
	$\cov(X,Y)\overset{\text{def}}{=}\E\left[(X-\E(X))(Y-\E(Y))\right]=\E(XY)-\E(X)\E(Y)$.
\end{definition}
\fi
\pagebreak

% \tableofcontents

% insert your code here
%\input{./algebra/main.tex}
%\input{./geometrie-differentielle/main.tex}
%\input{./probabilite/main.tex}
%\input{./analyse-fonctionnelle/main.tex}
% \input{./Analyse-convexe-et-dualite-en-optimisation/main.tex}
%\input{./tikz/main.tex}
%\input{./Theorie-du-distributions/main.tex}
%\input{./optimisation/mine.tex}
 \input{./modelisation/main.tex}

% yves.aubry@univ-tln.fr : algebra

\end{document}

% % !TEX encoding = UTF-8 Unicode
% !TEX TS-program = xelatex

\documentclass[french]{report}

%\usepackage[utf8]{inputenc}
%\usepackage[T1]{fontenc}
\usepackage{babel}


\newif\ifcomment
%\commenttrue # Show comments

\usepackage{physics}
\usepackage{amssymb}


\usepackage{amsthm}
% \usepackage{thmtools}
\usepackage{mathtools}
\usepackage{amsfonts}

\usepackage{color}

\usepackage{tikz}

\usepackage{geometry}
\geometry{a5paper, margin=0.1in, right=1cm}

\usepackage{dsfont}

\usepackage{graphicx}
\graphicspath{ {images/} }

\usepackage{faktor}

\usepackage{IEEEtrantools}
\usepackage{enumerate}   
\usepackage[PostScript=dvips]{"/Users/aware/Documents/Courses/diagrams"}


\newtheorem{theorem}{Théorème}[section]
\renewcommand{\thetheorem}{\arabic{theorem}}
\newtheorem{lemme}{Lemme}[section]
\renewcommand{\thelemme}{\arabic{lemme}}
\newtheorem{proposition}{Proposition}[section]
\renewcommand{\theproposition}{\arabic{proposition}}
\newtheorem{notations}{Notations}[section]
\newtheorem{problem}{Problème}[section]
\newtheorem{corollary}{Corollaire}[theorem]
\renewcommand{\thecorollary}{\arabic{corollary}}
\newtheorem{property}{Propriété}[section]
\newtheorem{objective}{Objectif}[section]

\theoremstyle{definition}
\newtheorem{definition}{Définition}[section]
\renewcommand{\thedefinition}{\arabic{definition}}
\newtheorem{exercise}{Exercice}[chapter]
\renewcommand{\theexercise}{\arabic{exercise}}
\newtheorem{example}{Exemple}[chapter]
\renewcommand{\theexample}{\arabic{example}}
\newtheorem*{solution}{Solution}
\newtheorem*{application}{Application}
\newtheorem*{notation}{Notation}
\newtheorem*{vocabulary}{Vocabulaire}
\newtheorem*{properties}{Propriétés}



\theoremstyle{remark}
\newtheorem*{remark}{Remarque}
\newtheorem*{rappel}{Rappel}


\usepackage{etoolbox}
\AtBeginEnvironment{exercise}{\small}
\AtBeginEnvironment{example}{\small}

\usepackage{cases}
\usepackage[red]{mypack}

\usepackage[framemethod=TikZ]{mdframed}

\definecolor{bg}{rgb}{0.4,0.25,0.95}
\definecolor{pagebg}{rgb}{0,0,0.5}
\surroundwithmdframed[
   topline=false,
   rightline=false,
   bottomline=false,
   leftmargin=\parindent,
   skipabove=8pt,
   skipbelow=8pt,
   linecolor=blue,
   innerbottommargin=10pt,
   % backgroundcolor=bg,font=\color{orange}\sffamily, fontcolor=white
]{definition}

\usepackage{empheq}
\usepackage[most]{tcolorbox}

\newtcbox{\mymath}[1][]{%
    nobeforeafter, math upper, tcbox raise base,
    enhanced, colframe=blue!30!black,
    colback=red!10, boxrule=1pt,
    #1}

\usepackage{unixode}


\DeclareMathOperator{\ord}{ord}
\DeclareMathOperator{\orb}{orb}
\DeclareMathOperator{\stab}{stab}
\DeclareMathOperator{\Stab}{stab}
\DeclareMathOperator{\ppcm}{ppcm}
\DeclareMathOperator{\conj}{Conj}
\DeclareMathOperator{\End}{End}
\DeclareMathOperator{\rot}{rot}
\DeclareMathOperator{\trs}{trace}
\DeclareMathOperator{\Ind}{Ind}
\DeclareMathOperator{\mat}{Mat}
\DeclareMathOperator{\id}{Id}
\DeclareMathOperator{\vect}{vect}
\DeclareMathOperator{\img}{img}
\DeclareMathOperator{\cov}{Cov}
\DeclareMathOperator{\dist}{dist}
\DeclareMathOperator{\irr}{Irr}
\DeclareMathOperator{\image}{Im}
\DeclareMathOperator{\pd}{\partial}
\DeclareMathOperator{\epi}{epi}
\DeclareMathOperator{\Argmin}{Argmin}
\DeclareMathOperator{\dom}{dom}
\DeclareMathOperator{\proj}{proj}
\DeclareMathOperator{\ctg}{ctg}
\DeclareMathOperator{\supp}{supp}
\DeclareMathOperator{\argmin}{argmin}
\DeclareMathOperator{\mult}{mult}
\DeclareMathOperator{\ch}{ch}
\DeclareMathOperator{\sh}{sh}
\DeclareMathOperator{\rang}{rang}
\DeclareMathOperator{\diam}{diam}
\DeclareMathOperator{\Epigraphe}{Epigraphe}




\usepackage{xcolor}
\everymath{\color{blue}}
%\everymath{\color[rgb]{0,1,1}}
%\pagecolor[rgb]{0,0,0.5}


\newcommand*{\pdtest}[3][]{\ensuremath{\frac{\partial^{#1} #2}{\partial #3}}}

\newcommand*{\deffunc}[6][]{\ensuremath{
\begin{array}{rcl}
#2 : #3 &\rightarrow& #4\\
#5 &\mapsto& #6
\end{array}
}}

\newcommand{\eqcolon}{\mathrel{\resizebox{\widthof{$\mathord{=}$}}{\height}{ $\!\!=\!\!\resizebox{1.2\width}{0.8\height}{\raisebox{0.23ex}{$\mathop{:}$}}\!\!$ }}}
\newcommand{\coloneq}{\mathrel{\resizebox{\widthof{$\mathord{=}$}}{\height}{ $\!\!\resizebox{1.2\width}{0.8\height}{\raisebox{0.23ex}{$\mathop{:}$}}\!\!=\!\!$ }}}
\newcommand{\eqcolonl}{\ensuremath{\mathrel{=\!\!\mathop{:}}}}
\newcommand{\coloneql}{\ensuremath{\mathrel{\mathop{:} \!\! =}}}
\newcommand{\vc}[1]{% inline column vector
  \left(\begin{smallmatrix}#1\end{smallmatrix}\right)%
}
\newcommand{\vr}[1]{% inline row vector
  \begin{smallmatrix}(\,#1\,)\end{smallmatrix}%
}
\makeatletter
\newcommand*{\defeq}{\ =\mathrel{\rlap{%
                     \raisebox{0.3ex}{$\m@th\cdot$}}%
                     \raisebox{-0.3ex}{$\m@th\cdot$}}%
                     }
\makeatother

\newcommand{\mathcircle}[1]{% inline row vector
 \overset{\circ}{#1}
}
\newcommand{\ulim}{% low limit
 \underline{\lim}
}
\newcommand{\ssi}{% iff
\iff
}
\newcommand{\ps}[2]{
\expval{#1 | #2}
}
\newcommand{\df}[1]{
\mqty{#1}
}
\newcommand{\n}[1]{
\norm{#1}
}
\newcommand{\sys}[1]{
\left\{\smqty{#1}\right.
}


\newcommand{\eqdef}{\ensuremath{\overset{\text{def}}=}}


\def\Circlearrowright{\ensuremath{%
  \rotatebox[origin=c]{230}{$\circlearrowright$}}}

\newcommand\ct[1]{\text{\rmfamily\upshape #1}}
\newcommand\question[1]{ {\color{red} ...!? \small #1}}
\newcommand\caz[1]{\left\{\begin{array} #1 \end{array}\right.}
\newcommand\const{\text{\rmfamily\upshape const}}
\newcommand\toP{ \overset{\pro}{\to}}
\newcommand\toPP{ \overset{\text{PP}}{\to}}
\newcommand{\oeq}{\mathrel{\text{\textcircled{$=$}}}}





\usepackage{xcolor}
% \usepackage[normalem]{ulem}
\usepackage{lipsum}
\makeatletter
% \newcommand\colorwave[1][blue]{\bgroup \markoverwith{\lower3.5\p@\hbox{\sixly \textcolor{#1}{\char58}}}\ULon}
%\font\sixly=lasy6 % does not re-load if already loaded, so no memory problem.

\newmdtheoremenv[
linewidth= 1pt,linecolor= blue,%
leftmargin=20,rightmargin=20,innertopmargin=0pt, innerrightmargin=40,%
tikzsetting = { draw=lightgray, line width = 0.3pt,dashed,%
dash pattern = on 15pt off 3pt},%
splittopskip=\topskip,skipbelow=\baselineskip,%
skipabove=\baselineskip,ntheorem,roundcorner=0pt,
% backgroundcolor=pagebg,font=\color{orange}\sffamily, fontcolor=white
]{examplebox}{Exemple}[section]



\newcommand\R{\mathbb{R}}
\newcommand\Z{\mathbb{Z}}
\newcommand\N{\mathbb{N}}
\newcommand\E{\mathbb{E}}
\newcommand\F{\mathcal{F}}
\newcommand\cH{\mathcal{H}}
\newcommand\V{\mathbb{V}}
\newcommand\dmo{ ^{-1} }
\newcommand\kapa{\kappa}
\newcommand\im{Im}
\newcommand\hs{\mathcal{H}}





\usepackage{soul}

\makeatletter
\newcommand*{\whiten}[1]{\llap{\textcolor{white}{{\the\SOUL@token}}\hspace{#1pt}}}
\DeclareRobustCommand*\myul{%
    \def\SOUL@everyspace{\underline{\space}\kern\z@}%
    \def\SOUL@everytoken{%
     \setbox0=\hbox{\the\SOUL@token}%
     \ifdim\dp0>\z@
        \raisebox{\dp0}{\underline{\phantom{\the\SOUL@token}}}%
        \whiten{1}\whiten{0}%
        \whiten{-1}\whiten{-2}%
        \llap{\the\SOUL@token}%
     \else
        \underline{\the\SOUL@token}%
     \fi}%
\SOUL@}
\makeatother

\newcommand*{\demp}{\fontfamily{lmtt}\selectfont}

\DeclareTextFontCommand{\textdemp}{\demp}

\begin{document}

\ifcomment
Multiline
comment
\fi
\ifcomment
\myul{Typesetting test}
% \color[rgb]{1,1,1}
$∑_i^n≠ 60º±∞π∆¬≈√j∫h≤≥µ$

$\CR \R\pro\ind\pro\gS\pro
\mqty[a&b\\c&d]$
$\pro\mathbb{P}$
$\dd{x}$

  \[
    \alpha(x)=\left\{
                \begin{array}{ll}
                  x\\
                  \frac{1}{1+e^{-kx}}\\
                  \frac{e^x-e^{-x}}{e^x+e^{-x}}
                \end{array}
              \right.
  \]

  $\expval{x}$
  
  $\chi_\rho(ghg\dmo)=\Tr(\rho_{ghg\dmo})=\Tr(\rho_g\circ\rho_h\circ\rho\dmo_g)=\Tr(\rho_h)\overset{\mbox{\scalebox{0.5}{$\Tr(AB)=\Tr(BA)$}}}{=}\chi_\rho(h)$
  	$\mathop{\oplus}_{\substack{x\in X}}$

$\mat(\rho_g)=(a_{ij}(g))_{\scriptsize \substack{1\leq i\leq d \\ 1\leq j\leq d}}$ et $\mat(\rho'_g)=(a'_{ij}(g))_{\scriptsize \substack{1\leq i'\leq d' \\ 1\leq j'\leq d'}}$



\[\int_a^b{\mathbb{R}^2}g(u, v)\dd{P_{XY}}(u, v)=\iint g(u,v) f_{XY}(u, v)\dd \lambda(u) \dd \lambda(v)\]
$$\lim_{x\to\infty} f(x)$$	
$$\iiiint_V \mu(t,u,v,w) \,dt\,du\,dv\,dw$$
$$\sum_{n=1}^{\infty} 2^{-n} = 1$$	
\begin{definition}
	Si $X$ et $Y$ sont 2 v.a. ou definit la \textsc{Covariance} entre $X$ et $Y$ comme
	$\cov(X,Y)\overset{\text{def}}{=}\E\left[(X-\E(X))(Y-\E(Y))\right]=\E(XY)-\E(X)\E(Y)$.
\end{definition}
\fi
\pagebreak

% \tableofcontents

% insert your code here
%\input{./algebra/main.tex}
%\input{./geometrie-differentielle/main.tex}
%\input{./probabilite/main.tex}
%\input{./analyse-fonctionnelle/main.tex}
% \input{./Analyse-convexe-et-dualite-en-optimisation/main.tex}
%\input{./tikz/main.tex}
%\input{./Theorie-du-distributions/main.tex}
%\input{./optimisation/mine.tex}
 \input{./modelisation/main.tex}

% yves.aubry@univ-tln.fr : algebra

\end{document}

%% !TEX encoding = UTF-8 Unicode
% !TEX TS-program = xelatex

\documentclass[french]{report}

%\usepackage[utf8]{inputenc}
%\usepackage[T1]{fontenc}
\usepackage{babel}


\newif\ifcomment
%\commenttrue # Show comments

\usepackage{physics}
\usepackage{amssymb}


\usepackage{amsthm}
% \usepackage{thmtools}
\usepackage{mathtools}
\usepackage{amsfonts}

\usepackage{color}

\usepackage{tikz}

\usepackage{geometry}
\geometry{a5paper, margin=0.1in, right=1cm}

\usepackage{dsfont}

\usepackage{graphicx}
\graphicspath{ {images/} }

\usepackage{faktor}

\usepackage{IEEEtrantools}
\usepackage{enumerate}   
\usepackage[PostScript=dvips]{"/Users/aware/Documents/Courses/diagrams"}


\newtheorem{theorem}{Théorème}[section]
\renewcommand{\thetheorem}{\arabic{theorem}}
\newtheorem{lemme}{Lemme}[section]
\renewcommand{\thelemme}{\arabic{lemme}}
\newtheorem{proposition}{Proposition}[section]
\renewcommand{\theproposition}{\arabic{proposition}}
\newtheorem{notations}{Notations}[section]
\newtheorem{problem}{Problème}[section]
\newtheorem{corollary}{Corollaire}[theorem]
\renewcommand{\thecorollary}{\arabic{corollary}}
\newtheorem{property}{Propriété}[section]
\newtheorem{objective}{Objectif}[section]

\theoremstyle{definition}
\newtheorem{definition}{Définition}[section]
\renewcommand{\thedefinition}{\arabic{definition}}
\newtheorem{exercise}{Exercice}[chapter]
\renewcommand{\theexercise}{\arabic{exercise}}
\newtheorem{example}{Exemple}[chapter]
\renewcommand{\theexample}{\arabic{example}}
\newtheorem*{solution}{Solution}
\newtheorem*{application}{Application}
\newtheorem*{notation}{Notation}
\newtheorem*{vocabulary}{Vocabulaire}
\newtheorem*{properties}{Propriétés}



\theoremstyle{remark}
\newtheorem*{remark}{Remarque}
\newtheorem*{rappel}{Rappel}


\usepackage{etoolbox}
\AtBeginEnvironment{exercise}{\small}
\AtBeginEnvironment{example}{\small}

\usepackage{cases}
\usepackage[red]{mypack}

\usepackage[framemethod=TikZ]{mdframed}

\definecolor{bg}{rgb}{0.4,0.25,0.95}
\definecolor{pagebg}{rgb}{0,0,0.5}
\surroundwithmdframed[
   topline=false,
   rightline=false,
   bottomline=false,
   leftmargin=\parindent,
   skipabove=8pt,
   skipbelow=8pt,
   linecolor=blue,
   innerbottommargin=10pt,
   % backgroundcolor=bg,font=\color{orange}\sffamily, fontcolor=white
]{definition}

\usepackage{empheq}
\usepackage[most]{tcolorbox}

\newtcbox{\mymath}[1][]{%
    nobeforeafter, math upper, tcbox raise base,
    enhanced, colframe=blue!30!black,
    colback=red!10, boxrule=1pt,
    #1}

\usepackage{unixode}


\DeclareMathOperator{\ord}{ord}
\DeclareMathOperator{\orb}{orb}
\DeclareMathOperator{\stab}{stab}
\DeclareMathOperator{\Stab}{stab}
\DeclareMathOperator{\ppcm}{ppcm}
\DeclareMathOperator{\conj}{Conj}
\DeclareMathOperator{\End}{End}
\DeclareMathOperator{\rot}{rot}
\DeclareMathOperator{\trs}{trace}
\DeclareMathOperator{\Ind}{Ind}
\DeclareMathOperator{\mat}{Mat}
\DeclareMathOperator{\id}{Id}
\DeclareMathOperator{\vect}{vect}
\DeclareMathOperator{\img}{img}
\DeclareMathOperator{\cov}{Cov}
\DeclareMathOperator{\dist}{dist}
\DeclareMathOperator{\irr}{Irr}
\DeclareMathOperator{\image}{Im}
\DeclareMathOperator{\pd}{\partial}
\DeclareMathOperator{\epi}{epi}
\DeclareMathOperator{\Argmin}{Argmin}
\DeclareMathOperator{\dom}{dom}
\DeclareMathOperator{\proj}{proj}
\DeclareMathOperator{\ctg}{ctg}
\DeclareMathOperator{\supp}{supp}
\DeclareMathOperator{\argmin}{argmin}
\DeclareMathOperator{\mult}{mult}
\DeclareMathOperator{\ch}{ch}
\DeclareMathOperator{\sh}{sh}
\DeclareMathOperator{\rang}{rang}
\DeclareMathOperator{\diam}{diam}
\DeclareMathOperator{\Epigraphe}{Epigraphe}




\usepackage{xcolor}
\everymath{\color{blue}}
%\everymath{\color[rgb]{0,1,1}}
%\pagecolor[rgb]{0,0,0.5}


\newcommand*{\pdtest}[3][]{\ensuremath{\frac{\partial^{#1} #2}{\partial #3}}}

\newcommand*{\deffunc}[6][]{\ensuremath{
\begin{array}{rcl}
#2 : #3 &\rightarrow& #4\\
#5 &\mapsto& #6
\end{array}
}}

\newcommand{\eqcolon}{\mathrel{\resizebox{\widthof{$\mathord{=}$}}{\height}{ $\!\!=\!\!\resizebox{1.2\width}{0.8\height}{\raisebox{0.23ex}{$\mathop{:}$}}\!\!$ }}}
\newcommand{\coloneq}{\mathrel{\resizebox{\widthof{$\mathord{=}$}}{\height}{ $\!\!\resizebox{1.2\width}{0.8\height}{\raisebox{0.23ex}{$\mathop{:}$}}\!\!=\!\!$ }}}
\newcommand{\eqcolonl}{\ensuremath{\mathrel{=\!\!\mathop{:}}}}
\newcommand{\coloneql}{\ensuremath{\mathrel{\mathop{:} \!\! =}}}
\newcommand{\vc}[1]{% inline column vector
  \left(\begin{smallmatrix}#1\end{smallmatrix}\right)%
}
\newcommand{\vr}[1]{% inline row vector
  \begin{smallmatrix}(\,#1\,)\end{smallmatrix}%
}
\makeatletter
\newcommand*{\defeq}{\ =\mathrel{\rlap{%
                     \raisebox{0.3ex}{$\m@th\cdot$}}%
                     \raisebox{-0.3ex}{$\m@th\cdot$}}%
                     }
\makeatother

\newcommand{\mathcircle}[1]{% inline row vector
 \overset{\circ}{#1}
}
\newcommand{\ulim}{% low limit
 \underline{\lim}
}
\newcommand{\ssi}{% iff
\iff
}
\newcommand{\ps}[2]{
\expval{#1 | #2}
}
\newcommand{\df}[1]{
\mqty{#1}
}
\newcommand{\n}[1]{
\norm{#1}
}
\newcommand{\sys}[1]{
\left\{\smqty{#1}\right.
}


\newcommand{\eqdef}{\ensuremath{\overset{\text{def}}=}}


\def\Circlearrowright{\ensuremath{%
  \rotatebox[origin=c]{230}{$\circlearrowright$}}}

\newcommand\ct[1]{\text{\rmfamily\upshape #1}}
\newcommand\question[1]{ {\color{red} ...!? \small #1}}
\newcommand\caz[1]{\left\{\begin{array} #1 \end{array}\right.}
\newcommand\const{\text{\rmfamily\upshape const}}
\newcommand\toP{ \overset{\pro}{\to}}
\newcommand\toPP{ \overset{\text{PP}}{\to}}
\newcommand{\oeq}{\mathrel{\text{\textcircled{$=$}}}}





\usepackage{xcolor}
% \usepackage[normalem]{ulem}
\usepackage{lipsum}
\makeatletter
% \newcommand\colorwave[1][blue]{\bgroup \markoverwith{\lower3.5\p@\hbox{\sixly \textcolor{#1}{\char58}}}\ULon}
%\font\sixly=lasy6 % does not re-load if already loaded, so no memory problem.

\newmdtheoremenv[
linewidth= 1pt,linecolor= blue,%
leftmargin=20,rightmargin=20,innertopmargin=0pt, innerrightmargin=40,%
tikzsetting = { draw=lightgray, line width = 0.3pt,dashed,%
dash pattern = on 15pt off 3pt},%
splittopskip=\topskip,skipbelow=\baselineskip,%
skipabove=\baselineskip,ntheorem,roundcorner=0pt,
% backgroundcolor=pagebg,font=\color{orange}\sffamily, fontcolor=white
]{examplebox}{Exemple}[section]



\newcommand\R{\mathbb{R}}
\newcommand\Z{\mathbb{Z}}
\newcommand\N{\mathbb{N}}
\newcommand\E{\mathbb{E}}
\newcommand\F{\mathcal{F}}
\newcommand\cH{\mathcal{H}}
\newcommand\V{\mathbb{V}}
\newcommand\dmo{ ^{-1} }
\newcommand\kapa{\kappa}
\newcommand\im{Im}
\newcommand\hs{\mathcal{H}}





\usepackage{soul}

\makeatletter
\newcommand*{\whiten}[1]{\llap{\textcolor{white}{{\the\SOUL@token}}\hspace{#1pt}}}
\DeclareRobustCommand*\myul{%
    \def\SOUL@everyspace{\underline{\space}\kern\z@}%
    \def\SOUL@everytoken{%
     \setbox0=\hbox{\the\SOUL@token}%
     \ifdim\dp0>\z@
        \raisebox{\dp0}{\underline{\phantom{\the\SOUL@token}}}%
        \whiten{1}\whiten{0}%
        \whiten{-1}\whiten{-2}%
        \llap{\the\SOUL@token}%
     \else
        \underline{\the\SOUL@token}%
     \fi}%
\SOUL@}
\makeatother

\newcommand*{\demp}{\fontfamily{lmtt}\selectfont}

\DeclareTextFontCommand{\textdemp}{\demp}

\begin{document}

\ifcomment
Multiline
comment
\fi
\ifcomment
\myul{Typesetting test}
% \color[rgb]{1,1,1}
$∑_i^n≠ 60º±∞π∆¬≈√j∫h≤≥µ$

$\CR \R\pro\ind\pro\gS\pro
\mqty[a&b\\c&d]$
$\pro\mathbb{P}$
$\dd{x}$

  \[
    \alpha(x)=\left\{
                \begin{array}{ll}
                  x\\
                  \frac{1}{1+e^{-kx}}\\
                  \frac{e^x-e^{-x}}{e^x+e^{-x}}
                \end{array}
              \right.
  \]

  $\expval{x}$
  
  $\chi_\rho(ghg\dmo)=\Tr(\rho_{ghg\dmo})=\Tr(\rho_g\circ\rho_h\circ\rho\dmo_g)=\Tr(\rho_h)\overset{\mbox{\scalebox{0.5}{$\Tr(AB)=\Tr(BA)$}}}{=}\chi_\rho(h)$
  	$\mathop{\oplus}_{\substack{x\in X}}$

$\mat(\rho_g)=(a_{ij}(g))_{\scriptsize \substack{1\leq i\leq d \\ 1\leq j\leq d}}$ et $\mat(\rho'_g)=(a'_{ij}(g))_{\scriptsize \substack{1\leq i'\leq d' \\ 1\leq j'\leq d'}}$



\[\int_a^b{\mathbb{R}^2}g(u, v)\dd{P_{XY}}(u, v)=\iint g(u,v) f_{XY}(u, v)\dd \lambda(u) \dd \lambda(v)\]
$$\lim_{x\to\infty} f(x)$$	
$$\iiiint_V \mu(t,u,v,w) \,dt\,du\,dv\,dw$$
$$\sum_{n=1}^{\infty} 2^{-n} = 1$$	
\begin{definition}
	Si $X$ et $Y$ sont 2 v.a. ou definit la \textsc{Covariance} entre $X$ et $Y$ comme
	$\cov(X,Y)\overset{\text{def}}{=}\E\left[(X-\E(X))(Y-\E(Y))\right]=\E(XY)-\E(X)\E(Y)$.
\end{definition}
\fi
\pagebreak

% \tableofcontents

% insert your code here
%\input{./algebra/main.tex}
%\input{./geometrie-differentielle/main.tex}
%\input{./probabilite/main.tex}
%\input{./analyse-fonctionnelle/main.tex}
% \input{./Analyse-convexe-et-dualite-en-optimisation/main.tex}
%\input{./tikz/main.tex}
%\input{./Theorie-du-distributions/main.tex}
%\input{./optimisation/mine.tex}
 \input{./modelisation/main.tex}

% yves.aubry@univ-tln.fr : algebra

\end{document}

%% !TEX encoding = UTF-8 Unicode
% !TEX TS-program = xelatex

\documentclass[french]{report}

%\usepackage[utf8]{inputenc}
%\usepackage[T1]{fontenc}
\usepackage{babel}


\newif\ifcomment
%\commenttrue # Show comments

\usepackage{physics}
\usepackage{amssymb}


\usepackage{amsthm}
% \usepackage{thmtools}
\usepackage{mathtools}
\usepackage{amsfonts}

\usepackage{color}

\usepackage{tikz}

\usepackage{geometry}
\geometry{a5paper, margin=0.1in, right=1cm}

\usepackage{dsfont}

\usepackage{graphicx}
\graphicspath{ {images/} }

\usepackage{faktor}

\usepackage{IEEEtrantools}
\usepackage{enumerate}   
\usepackage[PostScript=dvips]{"/Users/aware/Documents/Courses/diagrams"}


\newtheorem{theorem}{Théorème}[section]
\renewcommand{\thetheorem}{\arabic{theorem}}
\newtheorem{lemme}{Lemme}[section]
\renewcommand{\thelemme}{\arabic{lemme}}
\newtheorem{proposition}{Proposition}[section]
\renewcommand{\theproposition}{\arabic{proposition}}
\newtheorem{notations}{Notations}[section]
\newtheorem{problem}{Problème}[section]
\newtheorem{corollary}{Corollaire}[theorem]
\renewcommand{\thecorollary}{\arabic{corollary}}
\newtheorem{property}{Propriété}[section]
\newtheorem{objective}{Objectif}[section]

\theoremstyle{definition}
\newtheorem{definition}{Définition}[section]
\renewcommand{\thedefinition}{\arabic{definition}}
\newtheorem{exercise}{Exercice}[chapter]
\renewcommand{\theexercise}{\arabic{exercise}}
\newtheorem{example}{Exemple}[chapter]
\renewcommand{\theexample}{\arabic{example}}
\newtheorem*{solution}{Solution}
\newtheorem*{application}{Application}
\newtheorem*{notation}{Notation}
\newtheorem*{vocabulary}{Vocabulaire}
\newtheorem*{properties}{Propriétés}



\theoremstyle{remark}
\newtheorem*{remark}{Remarque}
\newtheorem*{rappel}{Rappel}


\usepackage{etoolbox}
\AtBeginEnvironment{exercise}{\small}
\AtBeginEnvironment{example}{\small}

\usepackage{cases}
\usepackage[red]{mypack}

\usepackage[framemethod=TikZ]{mdframed}

\definecolor{bg}{rgb}{0.4,0.25,0.95}
\definecolor{pagebg}{rgb}{0,0,0.5}
\surroundwithmdframed[
   topline=false,
   rightline=false,
   bottomline=false,
   leftmargin=\parindent,
   skipabove=8pt,
   skipbelow=8pt,
   linecolor=blue,
   innerbottommargin=10pt,
   % backgroundcolor=bg,font=\color{orange}\sffamily, fontcolor=white
]{definition}

\usepackage{empheq}
\usepackage[most]{tcolorbox}

\newtcbox{\mymath}[1][]{%
    nobeforeafter, math upper, tcbox raise base,
    enhanced, colframe=blue!30!black,
    colback=red!10, boxrule=1pt,
    #1}

\usepackage{unixode}


\DeclareMathOperator{\ord}{ord}
\DeclareMathOperator{\orb}{orb}
\DeclareMathOperator{\stab}{stab}
\DeclareMathOperator{\Stab}{stab}
\DeclareMathOperator{\ppcm}{ppcm}
\DeclareMathOperator{\conj}{Conj}
\DeclareMathOperator{\End}{End}
\DeclareMathOperator{\rot}{rot}
\DeclareMathOperator{\trs}{trace}
\DeclareMathOperator{\Ind}{Ind}
\DeclareMathOperator{\mat}{Mat}
\DeclareMathOperator{\id}{Id}
\DeclareMathOperator{\vect}{vect}
\DeclareMathOperator{\img}{img}
\DeclareMathOperator{\cov}{Cov}
\DeclareMathOperator{\dist}{dist}
\DeclareMathOperator{\irr}{Irr}
\DeclareMathOperator{\image}{Im}
\DeclareMathOperator{\pd}{\partial}
\DeclareMathOperator{\epi}{epi}
\DeclareMathOperator{\Argmin}{Argmin}
\DeclareMathOperator{\dom}{dom}
\DeclareMathOperator{\proj}{proj}
\DeclareMathOperator{\ctg}{ctg}
\DeclareMathOperator{\supp}{supp}
\DeclareMathOperator{\argmin}{argmin}
\DeclareMathOperator{\mult}{mult}
\DeclareMathOperator{\ch}{ch}
\DeclareMathOperator{\sh}{sh}
\DeclareMathOperator{\rang}{rang}
\DeclareMathOperator{\diam}{diam}
\DeclareMathOperator{\Epigraphe}{Epigraphe}




\usepackage{xcolor}
\everymath{\color{blue}}
%\everymath{\color[rgb]{0,1,1}}
%\pagecolor[rgb]{0,0,0.5}


\newcommand*{\pdtest}[3][]{\ensuremath{\frac{\partial^{#1} #2}{\partial #3}}}

\newcommand*{\deffunc}[6][]{\ensuremath{
\begin{array}{rcl}
#2 : #3 &\rightarrow& #4\\
#5 &\mapsto& #6
\end{array}
}}

\newcommand{\eqcolon}{\mathrel{\resizebox{\widthof{$\mathord{=}$}}{\height}{ $\!\!=\!\!\resizebox{1.2\width}{0.8\height}{\raisebox{0.23ex}{$\mathop{:}$}}\!\!$ }}}
\newcommand{\coloneq}{\mathrel{\resizebox{\widthof{$\mathord{=}$}}{\height}{ $\!\!\resizebox{1.2\width}{0.8\height}{\raisebox{0.23ex}{$\mathop{:}$}}\!\!=\!\!$ }}}
\newcommand{\eqcolonl}{\ensuremath{\mathrel{=\!\!\mathop{:}}}}
\newcommand{\coloneql}{\ensuremath{\mathrel{\mathop{:} \!\! =}}}
\newcommand{\vc}[1]{% inline column vector
  \left(\begin{smallmatrix}#1\end{smallmatrix}\right)%
}
\newcommand{\vr}[1]{% inline row vector
  \begin{smallmatrix}(\,#1\,)\end{smallmatrix}%
}
\makeatletter
\newcommand*{\defeq}{\ =\mathrel{\rlap{%
                     \raisebox{0.3ex}{$\m@th\cdot$}}%
                     \raisebox{-0.3ex}{$\m@th\cdot$}}%
                     }
\makeatother

\newcommand{\mathcircle}[1]{% inline row vector
 \overset{\circ}{#1}
}
\newcommand{\ulim}{% low limit
 \underline{\lim}
}
\newcommand{\ssi}{% iff
\iff
}
\newcommand{\ps}[2]{
\expval{#1 | #2}
}
\newcommand{\df}[1]{
\mqty{#1}
}
\newcommand{\n}[1]{
\norm{#1}
}
\newcommand{\sys}[1]{
\left\{\smqty{#1}\right.
}


\newcommand{\eqdef}{\ensuremath{\overset{\text{def}}=}}


\def\Circlearrowright{\ensuremath{%
  \rotatebox[origin=c]{230}{$\circlearrowright$}}}

\newcommand\ct[1]{\text{\rmfamily\upshape #1}}
\newcommand\question[1]{ {\color{red} ...!? \small #1}}
\newcommand\caz[1]{\left\{\begin{array} #1 \end{array}\right.}
\newcommand\const{\text{\rmfamily\upshape const}}
\newcommand\toP{ \overset{\pro}{\to}}
\newcommand\toPP{ \overset{\text{PP}}{\to}}
\newcommand{\oeq}{\mathrel{\text{\textcircled{$=$}}}}





\usepackage{xcolor}
% \usepackage[normalem]{ulem}
\usepackage{lipsum}
\makeatletter
% \newcommand\colorwave[1][blue]{\bgroup \markoverwith{\lower3.5\p@\hbox{\sixly \textcolor{#1}{\char58}}}\ULon}
%\font\sixly=lasy6 % does not re-load if already loaded, so no memory problem.

\newmdtheoremenv[
linewidth= 1pt,linecolor= blue,%
leftmargin=20,rightmargin=20,innertopmargin=0pt, innerrightmargin=40,%
tikzsetting = { draw=lightgray, line width = 0.3pt,dashed,%
dash pattern = on 15pt off 3pt},%
splittopskip=\topskip,skipbelow=\baselineskip,%
skipabove=\baselineskip,ntheorem,roundcorner=0pt,
% backgroundcolor=pagebg,font=\color{orange}\sffamily, fontcolor=white
]{examplebox}{Exemple}[section]



\newcommand\R{\mathbb{R}}
\newcommand\Z{\mathbb{Z}}
\newcommand\N{\mathbb{N}}
\newcommand\E{\mathbb{E}}
\newcommand\F{\mathcal{F}}
\newcommand\cH{\mathcal{H}}
\newcommand\V{\mathbb{V}}
\newcommand\dmo{ ^{-1} }
\newcommand\kapa{\kappa}
\newcommand\im{Im}
\newcommand\hs{\mathcal{H}}





\usepackage{soul}

\makeatletter
\newcommand*{\whiten}[1]{\llap{\textcolor{white}{{\the\SOUL@token}}\hspace{#1pt}}}
\DeclareRobustCommand*\myul{%
    \def\SOUL@everyspace{\underline{\space}\kern\z@}%
    \def\SOUL@everytoken{%
     \setbox0=\hbox{\the\SOUL@token}%
     \ifdim\dp0>\z@
        \raisebox{\dp0}{\underline{\phantom{\the\SOUL@token}}}%
        \whiten{1}\whiten{0}%
        \whiten{-1}\whiten{-2}%
        \llap{\the\SOUL@token}%
     \else
        \underline{\the\SOUL@token}%
     \fi}%
\SOUL@}
\makeatother

\newcommand*{\demp}{\fontfamily{lmtt}\selectfont}

\DeclareTextFontCommand{\textdemp}{\demp}

\begin{document}

\ifcomment
Multiline
comment
\fi
\ifcomment
\myul{Typesetting test}
% \color[rgb]{1,1,1}
$∑_i^n≠ 60º±∞π∆¬≈√j∫h≤≥µ$

$\CR \R\pro\ind\pro\gS\pro
\mqty[a&b\\c&d]$
$\pro\mathbb{P}$
$\dd{x}$

  \[
    \alpha(x)=\left\{
                \begin{array}{ll}
                  x\\
                  \frac{1}{1+e^{-kx}}\\
                  \frac{e^x-e^{-x}}{e^x+e^{-x}}
                \end{array}
              \right.
  \]

  $\expval{x}$
  
  $\chi_\rho(ghg\dmo)=\Tr(\rho_{ghg\dmo})=\Tr(\rho_g\circ\rho_h\circ\rho\dmo_g)=\Tr(\rho_h)\overset{\mbox{\scalebox{0.5}{$\Tr(AB)=\Tr(BA)$}}}{=}\chi_\rho(h)$
  	$\mathop{\oplus}_{\substack{x\in X}}$

$\mat(\rho_g)=(a_{ij}(g))_{\scriptsize \substack{1\leq i\leq d \\ 1\leq j\leq d}}$ et $\mat(\rho'_g)=(a'_{ij}(g))_{\scriptsize \substack{1\leq i'\leq d' \\ 1\leq j'\leq d'}}$



\[\int_a^b{\mathbb{R}^2}g(u, v)\dd{P_{XY}}(u, v)=\iint g(u,v) f_{XY}(u, v)\dd \lambda(u) \dd \lambda(v)\]
$$\lim_{x\to\infty} f(x)$$	
$$\iiiint_V \mu(t,u,v,w) \,dt\,du\,dv\,dw$$
$$\sum_{n=1}^{\infty} 2^{-n} = 1$$	
\begin{definition}
	Si $X$ et $Y$ sont 2 v.a. ou definit la \textsc{Covariance} entre $X$ et $Y$ comme
	$\cov(X,Y)\overset{\text{def}}{=}\E\left[(X-\E(X))(Y-\E(Y))\right]=\E(XY)-\E(X)\E(Y)$.
\end{definition}
\fi
\pagebreak

% \tableofcontents

% insert your code here
%\input{./algebra/main.tex}
%\input{./geometrie-differentielle/main.tex}
%\input{./probabilite/main.tex}
%\input{./analyse-fonctionnelle/main.tex}
% \input{./Analyse-convexe-et-dualite-en-optimisation/main.tex}
%\input{./tikz/main.tex}
%\input{./Theorie-du-distributions/main.tex}
%\input{./optimisation/mine.tex}
 \input{./modelisation/main.tex}

% yves.aubry@univ-tln.fr : algebra

\end{document}

%\input{./optimisation/mine.tex}
 % !TEX encoding = UTF-8 Unicode
% !TEX TS-program = xelatex

\documentclass[french]{report}

%\usepackage[utf8]{inputenc}
%\usepackage[T1]{fontenc}
\usepackage{babel}


\newif\ifcomment
%\commenttrue # Show comments

\usepackage{physics}
\usepackage{amssymb}


\usepackage{amsthm}
% \usepackage{thmtools}
\usepackage{mathtools}
\usepackage{amsfonts}

\usepackage{color}

\usepackage{tikz}

\usepackage{geometry}
\geometry{a5paper, margin=0.1in, right=1cm}

\usepackage{dsfont}

\usepackage{graphicx}
\graphicspath{ {images/} }

\usepackage{faktor}

\usepackage{IEEEtrantools}
\usepackage{enumerate}   
\usepackage[PostScript=dvips]{"/Users/aware/Documents/Courses/diagrams"}


\newtheorem{theorem}{Théorème}[section]
\renewcommand{\thetheorem}{\arabic{theorem}}
\newtheorem{lemme}{Lemme}[section]
\renewcommand{\thelemme}{\arabic{lemme}}
\newtheorem{proposition}{Proposition}[section]
\renewcommand{\theproposition}{\arabic{proposition}}
\newtheorem{notations}{Notations}[section]
\newtheorem{problem}{Problème}[section]
\newtheorem{corollary}{Corollaire}[theorem]
\renewcommand{\thecorollary}{\arabic{corollary}}
\newtheorem{property}{Propriété}[section]
\newtheorem{objective}{Objectif}[section]

\theoremstyle{definition}
\newtheorem{definition}{Définition}[section]
\renewcommand{\thedefinition}{\arabic{definition}}
\newtheorem{exercise}{Exercice}[chapter]
\renewcommand{\theexercise}{\arabic{exercise}}
\newtheorem{example}{Exemple}[chapter]
\renewcommand{\theexample}{\arabic{example}}
\newtheorem*{solution}{Solution}
\newtheorem*{application}{Application}
\newtheorem*{notation}{Notation}
\newtheorem*{vocabulary}{Vocabulaire}
\newtheorem*{properties}{Propriétés}



\theoremstyle{remark}
\newtheorem*{remark}{Remarque}
\newtheorem*{rappel}{Rappel}


\usepackage{etoolbox}
\AtBeginEnvironment{exercise}{\small}
\AtBeginEnvironment{example}{\small}

\usepackage{cases}
\usepackage[red]{mypack}

\usepackage[framemethod=TikZ]{mdframed}

\definecolor{bg}{rgb}{0.4,0.25,0.95}
\definecolor{pagebg}{rgb}{0,0,0.5}
\surroundwithmdframed[
   topline=false,
   rightline=false,
   bottomline=false,
   leftmargin=\parindent,
   skipabove=8pt,
   skipbelow=8pt,
   linecolor=blue,
   innerbottommargin=10pt,
   % backgroundcolor=bg,font=\color{orange}\sffamily, fontcolor=white
]{definition}

\usepackage{empheq}
\usepackage[most]{tcolorbox}

\newtcbox{\mymath}[1][]{%
    nobeforeafter, math upper, tcbox raise base,
    enhanced, colframe=blue!30!black,
    colback=red!10, boxrule=1pt,
    #1}

\usepackage{unixode}


\DeclareMathOperator{\ord}{ord}
\DeclareMathOperator{\orb}{orb}
\DeclareMathOperator{\stab}{stab}
\DeclareMathOperator{\Stab}{stab}
\DeclareMathOperator{\ppcm}{ppcm}
\DeclareMathOperator{\conj}{Conj}
\DeclareMathOperator{\End}{End}
\DeclareMathOperator{\rot}{rot}
\DeclareMathOperator{\trs}{trace}
\DeclareMathOperator{\Ind}{Ind}
\DeclareMathOperator{\mat}{Mat}
\DeclareMathOperator{\id}{Id}
\DeclareMathOperator{\vect}{vect}
\DeclareMathOperator{\img}{img}
\DeclareMathOperator{\cov}{Cov}
\DeclareMathOperator{\dist}{dist}
\DeclareMathOperator{\irr}{Irr}
\DeclareMathOperator{\image}{Im}
\DeclareMathOperator{\pd}{\partial}
\DeclareMathOperator{\epi}{epi}
\DeclareMathOperator{\Argmin}{Argmin}
\DeclareMathOperator{\dom}{dom}
\DeclareMathOperator{\proj}{proj}
\DeclareMathOperator{\ctg}{ctg}
\DeclareMathOperator{\supp}{supp}
\DeclareMathOperator{\argmin}{argmin}
\DeclareMathOperator{\mult}{mult}
\DeclareMathOperator{\ch}{ch}
\DeclareMathOperator{\sh}{sh}
\DeclareMathOperator{\rang}{rang}
\DeclareMathOperator{\diam}{diam}
\DeclareMathOperator{\Epigraphe}{Epigraphe}




\usepackage{xcolor}
\everymath{\color{blue}}
%\everymath{\color[rgb]{0,1,1}}
%\pagecolor[rgb]{0,0,0.5}


\newcommand*{\pdtest}[3][]{\ensuremath{\frac{\partial^{#1} #2}{\partial #3}}}

\newcommand*{\deffunc}[6][]{\ensuremath{
\begin{array}{rcl}
#2 : #3 &\rightarrow& #4\\
#5 &\mapsto& #6
\end{array}
}}

\newcommand{\eqcolon}{\mathrel{\resizebox{\widthof{$\mathord{=}$}}{\height}{ $\!\!=\!\!\resizebox{1.2\width}{0.8\height}{\raisebox{0.23ex}{$\mathop{:}$}}\!\!$ }}}
\newcommand{\coloneq}{\mathrel{\resizebox{\widthof{$\mathord{=}$}}{\height}{ $\!\!\resizebox{1.2\width}{0.8\height}{\raisebox{0.23ex}{$\mathop{:}$}}\!\!=\!\!$ }}}
\newcommand{\eqcolonl}{\ensuremath{\mathrel{=\!\!\mathop{:}}}}
\newcommand{\coloneql}{\ensuremath{\mathrel{\mathop{:} \!\! =}}}
\newcommand{\vc}[1]{% inline column vector
  \left(\begin{smallmatrix}#1\end{smallmatrix}\right)%
}
\newcommand{\vr}[1]{% inline row vector
  \begin{smallmatrix}(\,#1\,)\end{smallmatrix}%
}
\makeatletter
\newcommand*{\defeq}{\ =\mathrel{\rlap{%
                     \raisebox{0.3ex}{$\m@th\cdot$}}%
                     \raisebox{-0.3ex}{$\m@th\cdot$}}%
                     }
\makeatother

\newcommand{\mathcircle}[1]{% inline row vector
 \overset{\circ}{#1}
}
\newcommand{\ulim}{% low limit
 \underline{\lim}
}
\newcommand{\ssi}{% iff
\iff
}
\newcommand{\ps}[2]{
\expval{#1 | #2}
}
\newcommand{\df}[1]{
\mqty{#1}
}
\newcommand{\n}[1]{
\norm{#1}
}
\newcommand{\sys}[1]{
\left\{\smqty{#1}\right.
}


\newcommand{\eqdef}{\ensuremath{\overset{\text{def}}=}}


\def\Circlearrowright{\ensuremath{%
  \rotatebox[origin=c]{230}{$\circlearrowright$}}}

\newcommand\ct[1]{\text{\rmfamily\upshape #1}}
\newcommand\question[1]{ {\color{red} ...!? \small #1}}
\newcommand\caz[1]{\left\{\begin{array} #1 \end{array}\right.}
\newcommand\const{\text{\rmfamily\upshape const}}
\newcommand\toP{ \overset{\pro}{\to}}
\newcommand\toPP{ \overset{\text{PP}}{\to}}
\newcommand{\oeq}{\mathrel{\text{\textcircled{$=$}}}}





\usepackage{xcolor}
% \usepackage[normalem]{ulem}
\usepackage{lipsum}
\makeatletter
% \newcommand\colorwave[1][blue]{\bgroup \markoverwith{\lower3.5\p@\hbox{\sixly \textcolor{#1}{\char58}}}\ULon}
%\font\sixly=lasy6 % does not re-load if already loaded, so no memory problem.

\newmdtheoremenv[
linewidth= 1pt,linecolor= blue,%
leftmargin=20,rightmargin=20,innertopmargin=0pt, innerrightmargin=40,%
tikzsetting = { draw=lightgray, line width = 0.3pt,dashed,%
dash pattern = on 15pt off 3pt},%
splittopskip=\topskip,skipbelow=\baselineskip,%
skipabove=\baselineskip,ntheorem,roundcorner=0pt,
% backgroundcolor=pagebg,font=\color{orange}\sffamily, fontcolor=white
]{examplebox}{Exemple}[section]



\newcommand\R{\mathbb{R}}
\newcommand\Z{\mathbb{Z}}
\newcommand\N{\mathbb{N}}
\newcommand\E{\mathbb{E}}
\newcommand\F{\mathcal{F}}
\newcommand\cH{\mathcal{H}}
\newcommand\V{\mathbb{V}}
\newcommand\dmo{ ^{-1} }
\newcommand\kapa{\kappa}
\newcommand\im{Im}
\newcommand\hs{\mathcal{H}}





\usepackage{soul}

\makeatletter
\newcommand*{\whiten}[1]{\llap{\textcolor{white}{{\the\SOUL@token}}\hspace{#1pt}}}
\DeclareRobustCommand*\myul{%
    \def\SOUL@everyspace{\underline{\space}\kern\z@}%
    \def\SOUL@everytoken{%
     \setbox0=\hbox{\the\SOUL@token}%
     \ifdim\dp0>\z@
        \raisebox{\dp0}{\underline{\phantom{\the\SOUL@token}}}%
        \whiten{1}\whiten{0}%
        \whiten{-1}\whiten{-2}%
        \llap{\the\SOUL@token}%
     \else
        \underline{\the\SOUL@token}%
     \fi}%
\SOUL@}
\makeatother

\newcommand*{\demp}{\fontfamily{lmtt}\selectfont}

\DeclareTextFontCommand{\textdemp}{\demp}

\begin{document}

\ifcomment
Multiline
comment
\fi
\ifcomment
\myul{Typesetting test}
% \color[rgb]{1,1,1}
$∑_i^n≠ 60º±∞π∆¬≈√j∫h≤≥µ$

$\CR \R\pro\ind\pro\gS\pro
\mqty[a&b\\c&d]$
$\pro\mathbb{P}$
$\dd{x}$

  \[
    \alpha(x)=\left\{
                \begin{array}{ll}
                  x\\
                  \frac{1}{1+e^{-kx}}\\
                  \frac{e^x-e^{-x}}{e^x+e^{-x}}
                \end{array}
              \right.
  \]

  $\expval{x}$
  
  $\chi_\rho(ghg\dmo)=\Tr(\rho_{ghg\dmo})=\Tr(\rho_g\circ\rho_h\circ\rho\dmo_g)=\Tr(\rho_h)\overset{\mbox{\scalebox{0.5}{$\Tr(AB)=\Tr(BA)$}}}{=}\chi_\rho(h)$
  	$\mathop{\oplus}_{\substack{x\in X}}$

$\mat(\rho_g)=(a_{ij}(g))_{\scriptsize \substack{1\leq i\leq d \\ 1\leq j\leq d}}$ et $\mat(\rho'_g)=(a'_{ij}(g))_{\scriptsize \substack{1\leq i'\leq d' \\ 1\leq j'\leq d'}}$



\[\int_a^b{\mathbb{R}^2}g(u, v)\dd{P_{XY}}(u, v)=\iint g(u,v) f_{XY}(u, v)\dd \lambda(u) \dd \lambda(v)\]
$$\lim_{x\to\infty} f(x)$$	
$$\iiiint_V \mu(t,u,v,w) \,dt\,du\,dv\,dw$$
$$\sum_{n=1}^{\infty} 2^{-n} = 1$$	
\begin{definition}
	Si $X$ et $Y$ sont 2 v.a. ou definit la \textsc{Covariance} entre $X$ et $Y$ comme
	$\cov(X,Y)\overset{\text{def}}{=}\E\left[(X-\E(X))(Y-\E(Y))\right]=\E(XY)-\E(X)\E(Y)$.
\end{definition}
\fi
\pagebreak

% \tableofcontents

% insert your code here
%\input{./algebra/main.tex}
%\input{./geometrie-differentielle/main.tex}
%\input{./probabilite/main.tex}
%\input{./analyse-fonctionnelle/main.tex}
% \input{./Analyse-convexe-et-dualite-en-optimisation/main.tex}
%\input{./tikz/main.tex}
%\input{./Theorie-du-distributions/main.tex}
%\input{./optimisation/mine.tex}
 \input{./modelisation/main.tex}

% yves.aubry@univ-tln.fr : algebra

\end{document}


% yves.aubry@univ-tln.fr : algebra

\end{document}

%% !TEX encoding = UTF-8 Unicode
% !TEX TS-program = xelatex

\documentclass[french]{report}

%\usepackage[utf8]{inputenc}
%\usepackage[T1]{fontenc}
\usepackage{babel}


\newif\ifcomment
%\commenttrue # Show comments

\usepackage{physics}
\usepackage{amssymb}


\usepackage{amsthm}
% \usepackage{thmtools}
\usepackage{mathtools}
\usepackage{amsfonts}

\usepackage{color}

\usepackage{tikz}

\usepackage{geometry}
\geometry{a5paper, margin=0.1in, right=1cm}

\usepackage{dsfont}

\usepackage{graphicx}
\graphicspath{ {images/} }

\usepackage{faktor}

\usepackage{IEEEtrantools}
\usepackage{enumerate}   
\usepackage[PostScript=dvips]{"/Users/aware/Documents/Courses/diagrams"}


\newtheorem{theorem}{Théorème}[section]
\renewcommand{\thetheorem}{\arabic{theorem}}
\newtheorem{lemme}{Lemme}[section]
\renewcommand{\thelemme}{\arabic{lemme}}
\newtheorem{proposition}{Proposition}[section]
\renewcommand{\theproposition}{\arabic{proposition}}
\newtheorem{notations}{Notations}[section]
\newtheorem{problem}{Problème}[section]
\newtheorem{corollary}{Corollaire}[theorem]
\renewcommand{\thecorollary}{\arabic{corollary}}
\newtheorem{property}{Propriété}[section]
\newtheorem{objective}{Objectif}[section]

\theoremstyle{definition}
\newtheorem{definition}{Définition}[section]
\renewcommand{\thedefinition}{\arabic{definition}}
\newtheorem{exercise}{Exercice}[chapter]
\renewcommand{\theexercise}{\arabic{exercise}}
\newtheorem{example}{Exemple}[chapter]
\renewcommand{\theexample}{\arabic{example}}
\newtheorem*{solution}{Solution}
\newtheorem*{application}{Application}
\newtheorem*{notation}{Notation}
\newtheorem*{vocabulary}{Vocabulaire}
\newtheorem*{properties}{Propriétés}



\theoremstyle{remark}
\newtheorem*{remark}{Remarque}
\newtheorem*{rappel}{Rappel}


\usepackage{etoolbox}
\AtBeginEnvironment{exercise}{\small}
\AtBeginEnvironment{example}{\small}

\usepackage{cases}
\usepackage[red]{mypack}

\usepackage[framemethod=TikZ]{mdframed}

\definecolor{bg}{rgb}{0.4,0.25,0.95}
\definecolor{pagebg}{rgb}{0,0,0.5}
\surroundwithmdframed[
   topline=false,
   rightline=false,
   bottomline=false,
   leftmargin=\parindent,
   skipabove=8pt,
   skipbelow=8pt,
   linecolor=blue,
   innerbottommargin=10pt,
   % backgroundcolor=bg,font=\color{orange}\sffamily, fontcolor=white
]{definition}

\usepackage{empheq}
\usepackage[most]{tcolorbox}

\newtcbox{\mymath}[1][]{%
    nobeforeafter, math upper, tcbox raise base,
    enhanced, colframe=blue!30!black,
    colback=red!10, boxrule=1pt,
    #1}

\usepackage{unixode}


\DeclareMathOperator{\ord}{ord}
\DeclareMathOperator{\orb}{orb}
\DeclareMathOperator{\stab}{stab}
\DeclareMathOperator{\Stab}{stab}
\DeclareMathOperator{\ppcm}{ppcm}
\DeclareMathOperator{\conj}{Conj}
\DeclareMathOperator{\End}{End}
\DeclareMathOperator{\rot}{rot}
\DeclareMathOperator{\trs}{trace}
\DeclareMathOperator{\Ind}{Ind}
\DeclareMathOperator{\mat}{Mat}
\DeclareMathOperator{\id}{Id}
\DeclareMathOperator{\vect}{vect}
\DeclareMathOperator{\img}{img}
\DeclareMathOperator{\cov}{Cov}
\DeclareMathOperator{\dist}{dist}
\DeclareMathOperator{\irr}{Irr}
\DeclareMathOperator{\image}{Im}
\DeclareMathOperator{\pd}{\partial}
\DeclareMathOperator{\epi}{epi}
\DeclareMathOperator{\Argmin}{Argmin}
\DeclareMathOperator{\dom}{dom}
\DeclareMathOperator{\proj}{proj}
\DeclareMathOperator{\ctg}{ctg}
\DeclareMathOperator{\supp}{supp}
\DeclareMathOperator{\argmin}{argmin}
\DeclareMathOperator{\mult}{mult}
\DeclareMathOperator{\ch}{ch}
\DeclareMathOperator{\sh}{sh}
\DeclareMathOperator{\rang}{rang}
\DeclareMathOperator{\diam}{diam}
\DeclareMathOperator{\Epigraphe}{Epigraphe}




\usepackage{xcolor}
\everymath{\color{blue}}
%\everymath{\color[rgb]{0,1,1}}
%\pagecolor[rgb]{0,0,0.5}


\newcommand*{\pdtest}[3][]{\ensuremath{\frac{\partial^{#1} #2}{\partial #3}}}

\newcommand*{\deffunc}[6][]{\ensuremath{
\begin{array}{rcl}
#2 : #3 &\rightarrow& #4\\
#5 &\mapsto& #6
\end{array}
}}

\newcommand{\eqcolon}{\mathrel{\resizebox{\widthof{$\mathord{=}$}}{\height}{ $\!\!=\!\!\resizebox{1.2\width}{0.8\height}{\raisebox{0.23ex}{$\mathop{:}$}}\!\!$ }}}
\newcommand{\coloneq}{\mathrel{\resizebox{\widthof{$\mathord{=}$}}{\height}{ $\!\!\resizebox{1.2\width}{0.8\height}{\raisebox{0.23ex}{$\mathop{:}$}}\!\!=\!\!$ }}}
\newcommand{\eqcolonl}{\ensuremath{\mathrel{=\!\!\mathop{:}}}}
\newcommand{\coloneql}{\ensuremath{\mathrel{\mathop{:} \!\! =}}}
\newcommand{\vc}[1]{% inline column vector
  \left(\begin{smallmatrix}#1\end{smallmatrix}\right)%
}
\newcommand{\vr}[1]{% inline row vector
  \begin{smallmatrix}(\,#1\,)\end{smallmatrix}%
}
\makeatletter
\newcommand*{\defeq}{\ =\mathrel{\rlap{%
                     \raisebox{0.3ex}{$\m@th\cdot$}}%
                     \raisebox{-0.3ex}{$\m@th\cdot$}}%
                     }
\makeatother

\newcommand{\mathcircle}[1]{% inline row vector
 \overset{\circ}{#1}
}
\newcommand{\ulim}{% low limit
 \underline{\lim}
}
\newcommand{\ssi}{% iff
\iff
}
\newcommand{\ps}[2]{
\expval{#1 | #2}
}
\newcommand{\df}[1]{
\mqty{#1}
}
\newcommand{\n}[1]{
\norm{#1}
}
\newcommand{\sys}[1]{
\left\{\smqty{#1}\right.
}


\newcommand{\eqdef}{\ensuremath{\overset{\text{def}}=}}


\def\Circlearrowright{\ensuremath{%
  \rotatebox[origin=c]{230}{$\circlearrowright$}}}

\newcommand\ct[1]{\text{\rmfamily\upshape #1}}
\newcommand\question[1]{ {\color{red} ...!? \small #1}}
\newcommand\caz[1]{\left\{\begin{array} #1 \end{array}\right.}
\newcommand\const{\text{\rmfamily\upshape const}}
\newcommand\toP{ \overset{\pro}{\to}}
\newcommand\toPP{ \overset{\text{PP}}{\to}}
\newcommand{\oeq}{\mathrel{\text{\textcircled{$=$}}}}





\usepackage{xcolor}
% \usepackage[normalem]{ulem}
\usepackage{lipsum}
\makeatletter
% \newcommand\colorwave[1][blue]{\bgroup \markoverwith{\lower3.5\p@\hbox{\sixly \textcolor{#1}{\char58}}}\ULon}
%\font\sixly=lasy6 % does not re-load if already loaded, so no memory problem.

\newmdtheoremenv[
linewidth= 1pt,linecolor= blue,%
leftmargin=20,rightmargin=20,innertopmargin=0pt, innerrightmargin=40,%
tikzsetting = { draw=lightgray, line width = 0.3pt,dashed,%
dash pattern = on 15pt off 3pt},%
splittopskip=\topskip,skipbelow=\baselineskip,%
skipabove=\baselineskip,ntheorem,roundcorner=0pt,
% backgroundcolor=pagebg,font=\color{orange}\sffamily, fontcolor=white
]{examplebox}{Exemple}[section]



\newcommand\R{\mathbb{R}}
\newcommand\Z{\mathbb{Z}}
\newcommand\N{\mathbb{N}}
\newcommand\E{\mathbb{E}}
\newcommand\F{\mathcal{F}}
\newcommand\cH{\mathcal{H}}
\newcommand\V{\mathbb{V}}
\newcommand\dmo{ ^{-1} }
\newcommand\kapa{\kappa}
\newcommand\im{Im}
\newcommand\hs{\mathcal{H}}





\usepackage{soul}

\makeatletter
\newcommand*{\whiten}[1]{\llap{\textcolor{white}{{\the\SOUL@token}}\hspace{#1pt}}}
\DeclareRobustCommand*\myul{%
    \def\SOUL@everyspace{\underline{\space}\kern\z@}%
    \def\SOUL@everytoken{%
     \setbox0=\hbox{\the\SOUL@token}%
     \ifdim\dp0>\z@
        \raisebox{\dp0}{\underline{\phantom{\the\SOUL@token}}}%
        \whiten{1}\whiten{0}%
        \whiten{-1}\whiten{-2}%
        \llap{\the\SOUL@token}%
     \else
        \underline{\the\SOUL@token}%
     \fi}%
\SOUL@}
\makeatother

\newcommand*{\demp}{\fontfamily{lmtt}\selectfont}

\DeclareTextFontCommand{\textdemp}{\demp}

\begin{document}

\ifcomment
Multiline
comment
\fi
\ifcomment
\myul{Typesetting test}
% \color[rgb]{1,1,1}
$∑_i^n≠ 60º±∞π∆¬≈√j∫h≤≥µ$

$\CR \R\pro\ind\pro\gS\pro
\mqty[a&b\\c&d]$
$\pro\mathbb{P}$
$\dd{x}$

  \[
    \alpha(x)=\left\{
                \begin{array}{ll}
                  x\\
                  \frac{1}{1+e^{-kx}}\\
                  \frac{e^x-e^{-x}}{e^x+e^{-x}}
                \end{array}
              \right.
  \]

  $\expval{x}$
  
  $\chi_\rho(ghg\dmo)=\Tr(\rho_{ghg\dmo})=\Tr(\rho_g\circ\rho_h\circ\rho\dmo_g)=\Tr(\rho_h)\overset{\mbox{\scalebox{0.5}{$\Tr(AB)=\Tr(BA)$}}}{=}\chi_\rho(h)$
  	$\mathop{\oplus}_{\substack{x\in X}}$

$\mat(\rho_g)=(a_{ij}(g))_{\scriptsize \substack{1\leq i\leq d \\ 1\leq j\leq d}}$ et $\mat(\rho'_g)=(a'_{ij}(g))_{\scriptsize \substack{1\leq i'\leq d' \\ 1\leq j'\leq d'}}$



\[\int_a^b{\mathbb{R}^2}g(u, v)\dd{P_{XY}}(u, v)=\iint g(u,v) f_{XY}(u, v)\dd \lambda(u) \dd \lambda(v)\]
$$\lim_{x\to\infty} f(x)$$	
$$\iiiint_V \mu(t,u,v,w) \,dt\,du\,dv\,dw$$
$$\sum_{n=1}^{\infty} 2^{-n} = 1$$	
\begin{definition}
	Si $X$ et $Y$ sont 2 v.a. ou definit la \textsc{Covariance} entre $X$ et $Y$ comme
	$\cov(X,Y)\overset{\text{def}}{=}\E\left[(X-\E(X))(Y-\E(Y))\right]=\E(XY)-\E(X)\E(Y)$.
\end{definition}
\fi
\pagebreak

% \tableofcontents

% insert your code here
%% !TEX encoding = UTF-8 Unicode
% !TEX TS-program = xelatex

\documentclass[french]{report}

%\usepackage[utf8]{inputenc}
%\usepackage[T1]{fontenc}
\usepackage{babel}


\newif\ifcomment
%\commenttrue # Show comments

\usepackage{physics}
\usepackage{amssymb}


\usepackage{amsthm}
% \usepackage{thmtools}
\usepackage{mathtools}
\usepackage{amsfonts}

\usepackage{color}

\usepackage{tikz}

\usepackage{geometry}
\geometry{a5paper, margin=0.1in, right=1cm}

\usepackage{dsfont}

\usepackage{graphicx}
\graphicspath{ {images/} }

\usepackage{faktor}

\usepackage{IEEEtrantools}
\usepackage{enumerate}   
\usepackage[PostScript=dvips]{"/Users/aware/Documents/Courses/diagrams"}


\newtheorem{theorem}{Théorème}[section]
\renewcommand{\thetheorem}{\arabic{theorem}}
\newtheorem{lemme}{Lemme}[section]
\renewcommand{\thelemme}{\arabic{lemme}}
\newtheorem{proposition}{Proposition}[section]
\renewcommand{\theproposition}{\arabic{proposition}}
\newtheorem{notations}{Notations}[section]
\newtheorem{problem}{Problème}[section]
\newtheorem{corollary}{Corollaire}[theorem]
\renewcommand{\thecorollary}{\arabic{corollary}}
\newtheorem{property}{Propriété}[section]
\newtheorem{objective}{Objectif}[section]

\theoremstyle{definition}
\newtheorem{definition}{Définition}[section]
\renewcommand{\thedefinition}{\arabic{definition}}
\newtheorem{exercise}{Exercice}[chapter]
\renewcommand{\theexercise}{\arabic{exercise}}
\newtheorem{example}{Exemple}[chapter]
\renewcommand{\theexample}{\arabic{example}}
\newtheorem*{solution}{Solution}
\newtheorem*{application}{Application}
\newtheorem*{notation}{Notation}
\newtheorem*{vocabulary}{Vocabulaire}
\newtheorem*{properties}{Propriétés}



\theoremstyle{remark}
\newtheorem*{remark}{Remarque}
\newtheorem*{rappel}{Rappel}


\usepackage{etoolbox}
\AtBeginEnvironment{exercise}{\small}
\AtBeginEnvironment{example}{\small}

\usepackage{cases}
\usepackage[red]{mypack}

\usepackage[framemethod=TikZ]{mdframed}

\definecolor{bg}{rgb}{0.4,0.25,0.95}
\definecolor{pagebg}{rgb}{0,0,0.5}
\surroundwithmdframed[
   topline=false,
   rightline=false,
   bottomline=false,
   leftmargin=\parindent,
   skipabove=8pt,
   skipbelow=8pt,
   linecolor=blue,
   innerbottommargin=10pt,
   % backgroundcolor=bg,font=\color{orange}\sffamily, fontcolor=white
]{definition}

\usepackage{empheq}
\usepackage[most]{tcolorbox}

\newtcbox{\mymath}[1][]{%
    nobeforeafter, math upper, tcbox raise base,
    enhanced, colframe=blue!30!black,
    colback=red!10, boxrule=1pt,
    #1}

\usepackage{unixode}


\DeclareMathOperator{\ord}{ord}
\DeclareMathOperator{\orb}{orb}
\DeclareMathOperator{\stab}{stab}
\DeclareMathOperator{\Stab}{stab}
\DeclareMathOperator{\ppcm}{ppcm}
\DeclareMathOperator{\conj}{Conj}
\DeclareMathOperator{\End}{End}
\DeclareMathOperator{\rot}{rot}
\DeclareMathOperator{\trs}{trace}
\DeclareMathOperator{\Ind}{Ind}
\DeclareMathOperator{\mat}{Mat}
\DeclareMathOperator{\id}{Id}
\DeclareMathOperator{\vect}{vect}
\DeclareMathOperator{\img}{img}
\DeclareMathOperator{\cov}{Cov}
\DeclareMathOperator{\dist}{dist}
\DeclareMathOperator{\irr}{Irr}
\DeclareMathOperator{\image}{Im}
\DeclareMathOperator{\pd}{\partial}
\DeclareMathOperator{\epi}{epi}
\DeclareMathOperator{\Argmin}{Argmin}
\DeclareMathOperator{\dom}{dom}
\DeclareMathOperator{\proj}{proj}
\DeclareMathOperator{\ctg}{ctg}
\DeclareMathOperator{\supp}{supp}
\DeclareMathOperator{\argmin}{argmin}
\DeclareMathOperator{\mult}{mult}
\DeclareMathOperator{\ch}{ch}
\DeclareMathOperator{\sh}{sh}
\DeclareMathOperator{\rang}{rang}
\DeclareMathOperator{\diam}{diam}
\DeclareMathOperator{\Epigraphe}{Epigraphe}




\usepackage{xcolor}
\everymath{\color{blue}}
%\everymath{\color[rgb]{0,1,1}}
%\pagecolor[rgb]{0,0,0.5}


\newcommand*{\pdtest}[3][]{\ensuremath{\frac{\partial^{#1} #2}{\partial #3}}}

\newcommand*{\deffunc}[6][]{\ensuremath{
\begin{array}{rcl}
#2 : #3 &\rightarrow& #4\\
#5 &\mapsto& #6
\end{array}
}}

\newcommand{\eqcolon}{\mathrel{\resizebox{\widthof{$\mathord{=}$}}{\height}{ $\!\!=\!\!\resizebox{1.2\width}{0.8\height}{\raisebox{0.23ex}{$\mathop{:}$}}\!\!$ }}}
\newcommand{\coloneq}{\mathrel{\resizebox{\widthof{$\mathord{=}$}}{\height}{ $\!\!\resizebox{1.2\width}{0.8\height}{\raisebox{0.23ex}{$\mathop{:}$}}\!\!=\!\!$ }}}
\newcommand{\eqcolonl}{\ensuremath{\mathrel{=\!\!\mathop{:}}}}
\newcommand{\coloneql}{\ensuremath{\mathrel{\mathop{:} \!\! =}}}
\newcommand{\vc}[1]{% inline column vector
  \left(\begin{smallmatrix}#1\end{smallmatrix}\right)%
}
\newcommand{\vr}[1]{% inline row vector
  \begin{smallmatrix}(\,#1\,)\end{smallmatrix}%
}
\makeatletter
\newcommand*{\defeq}{\ =\mathrel{\rlap{%
                     \raisebox{0.3ex}{$\m@th\cdot$}}%
                     \raisebox{-0.3ex}{$\m@th\cdot$}}%
                     }
\makeatother

\newcommand{\mathcircle}[1]{% inline row vector
 \overset{\circ}{#1}
}
\newcommand{\ulim}{% low limit
 \underline{\lim}
}
\newcommand{\ssi}{% iff
\iff
}
\newcommand{\ps}[2]{
\expval{#1 | #2}
}
\newcommand{\df}[1]{
\mqty{#1}
}
\newcommand{\n}[1]{
\norm{#1}
}
\newcommand{\sys}[1]{
\left\{\smqty{#1}\right.
}


\newcommand{\eqdef}{\ensuremath{\overset{\text{def}}=}}


\def\Circlearrowright{\ensuremath{%
  \rotatebox[origin=c]{230}{$\circlearrowright$}}}

\newcommand\ct[1]{\text{\rmfamily\upshape #1}}
\newcommand\question[1]{ {\color{red} ...!? \small #1}}
\newcommand\caz[1]{\left\{\begin{array} #1 \end{array}\right.}
\newcommand\const{\text{\rmfamily\upshape const}}
\newcommand\toP{ \overset{\pro}{\to}}
\newcommand\toPP{ \overset{\text{PP}}{\to}}
\newcommand{\oeq}{\mathrel{\text{\textcircled{$=$}}}}





\usepackage{xcolor}
% \usepackage[normalem]{ulem}
\usepackage{lipsum}
\makeatletter
% \newcommand\colorwave[1][blue]{\bgroup \markoverwith{\lower3.5\p@\hbox{\sixly \textcolor{#1}{\char58}}}\ULon}
%\font\sixly=lasy6 % does not re-load if already loaded, so no memory problem.

\newmdtheoremenv[
linewidth= 1pt,linecolor= blue,%
leftmargin=20,rightmargin=20,innertopmargin=0pt, innerrightmargin=40,%
tikzsetting = { draw=lightgray, line width = 0.3pt,dashed,%
dash pattern = on 15pt off 3pt},%
splittopskip=\topskip,skipbelow=\baselineskip,%
skipabove=\baselineskip,ntheorem,roundcorner=0pt,
% backgroundcolor=pagebg,font=\color{orange}\sffamily, fontcolor=white
]{examplebox}{Exemple}[section]



\newcommand\R{\mathbb{R}}
\newcommand\Z{\mathbb{Z}}
\newcommand\N{\mathbb{N}}
\newcommand\E{\mathbb{E}}
\newcommand\F{\mathcal{F}}
\newcommand\cH{\mathcal{H}}
\newcommand\V{\mathbb{V}}
\newcommand\dmo{ ^{-1} }
\newcommand\kapa{\kappa}
\newcommand\im{Im}
\newcommand\hs{\mathcal{H}}





\usepackage{soul}

\makeatletter
\newcommand*{\whiten}[1]{\llap{\textcolor{white}{{\the\SOUL@token}}\hspace{#1pt}}}
\DeclareRobustCommand*\myul{%
    \def\SOUL@everyspace{\underline{\space}\kern\z@}%
    \def\SOUL@everytoken{%
     \setbox0=\hbox{\the\SOUL@token}%
     \ifdim\dp0>\z@
        \raisebox{\dp0}{\underline{\phantom{\the\SOUL@token}}}%
        \whiten{1}\whiten{0}%
        \whiten{-1}\whiten{-2}%
        \llap{\the\SOUL@token}%
     \else
        \underline{\the\SOUL@token}%
     \fi}%
\SOUL@}
\makeatother

\newcommand*{\demp}{\fontfamily{lmtt}\selectfont}

\DeclareTextFontCommand{\textdemp}{\demp}

\begin{document}

\ifcomment
Multiline
comment
\fi
\ifcomment
\myul{Typesetting test}
% \color[rgb]{1,1,1}
$∑_i^n≠ 60º±∞π∆¬≈√j∫h≤≥µ$

$\CR \R\pro\ind\pro\gS\pro
\mqty[a&b\\c&d]$
$\pro\mathbb{P}$
$\dd{x}$

  \[
    \alpha(x)=\left\{
                \begin{array}{ll}
                  x\\
                  \frac{1}{1+e^{-kx}}\\
                  \frac{e^x-e^{-x}}{e^x+e^{-x}}
                \end{array}
              \right.
  \]

  $\expval{x}$
  
  $\chi_\rho(ghg\dmo)=\Tr(\rho_{ghg\dmo})=\Tr(\rho_g\circ\rho_h\circ\rho\dmo_g)=\Tr(\rho_h)\overset{\mbox{\scalebox{0.5}{$\Tr(AB)=\Tr(BA)$}}}{=}\chi_\rho(h)$
  	$\mathop{\oplus}_{\substack{x\in X}}$

$\mat(\rho_g)=(a_{ij}(g))_{\scriptsize \substack{1\leq i\leq d \\ 1\leq j\leq d}}$ et $\mat(\rho'_g)=(a'_{ij}(g))_{\scriptsize \substack{1\leq i'\leq d' \\ 1\leq j'\leq d'}}$



\[\int_a^b{\mathbb{R}^2}g(u, v)\dd{P_{XY}}(u, v)=\iint g(u,v) f_{XY}(u, v)\dd \lambda(u) \dd \lambda(v)\]
$$\lim_{x\to\infty} f(x)$$	
$$\iiiint_V \mu(t,u,v,w) \,dt\,du\,dv\,dw$$
$$\sum_{n=1}^{\infty} 2^{-n} = 1$$	
\begin{definition}
	Si $X$ et $Y$ sont 2 v.a. ou definit la \textsc{Covariance} entre $X$ et $Y$ comme
	$\cov(X,Y)\overset{\text{def}}{=}\E\left[(X-\E(X))(Y-\E(Y))\right]=\E(XY)-\E(X)\E(Y)$.
\end{definition}
\fi
\pagebreak

% \tableofcontents

% insert your code here
%\input{./algebra/main.tex}
%\input{./geometrie-differentielle/main.tex}
%\input{./probabilite/main.tex}
%\input{./analyse-fonctionnelle/main.tex}
% \input{./Analyse-convexe-et-dualite-en-optimisation/main.tex}
%\input{./tikz/main.tex}
%\input{./Theorie-du-distributions/main.tex}
%\input{./optimisation/mine.tex}
 \input{./modelisation/main.tex}

% yves.aubry@univ-tln.fr : algebra

\end{document}

%% !TEX encoding = UTF-8 Unicode
% !TEX TS-program = xelatex

\documentclass[french]{report}

%\usepackage[utf8]{inputenc}
%\usepackage[T1]{fontenc}
\usepackage{babel}


\newif\ifcomment
%\commenttrue # Show comments

\usepackage{physics}
\usepackage{amssymb}


\usepackage{amsthm}
% \usepackage{thmtools}
\usepackage{mathtools}
\usepackage{amsfonts}

\usepackage{color}

\usepackage{tikz}

\usepackage{geometry}
\geometry{a5paper, margin=0.1in, right=1cm}

\usepackage{dsfont}

\usepackage{graphicx}
\graphicspath{ {images/} }

\usepackage{faktor}

\usepackage{IEEEtrantools}
\usepackage{enumerate}   
\usepackage[PostScript=dvips]{"/Users/aware/Documents/Courses/diagrams"}


\newtheorem{theorem}{Théorème}[section]
\renewcommand{\thetheorem}{\arabic{theorem}}
\newtheorem{lemme}{Lemme}[section]
\renewcommand{\thelemme}{\arabic{lemme}}
\newtheorem{proposition}{Proposition}[section]
\renewcommand{\theproposition}{\arabic{proposition}}
\newtheorem{notations}{Notations}[section]
\newtheorem{problem}{Problème}[section]
\newtheorem{corollary}{Corollaire}[theorem]
\renewcommand{\thecorollary}{\arabic{corollary}}
\newtheorem{property}{Propriété}[section]
\newtheorem{objective}{Objectif}[section]

\theoremstyle{definition}
\newtheorem{definition}{Définition}[section]
\renewcommand{\thedefinition}{\arabic{definition}}
\newtheorem{exercise}{Exercice}[chapter]
\renewcommand{\theexercise}{\arabic{exercise}}
\newtheorem{example}{Exemple}[chapter]
\renewcommand{\theexample}{\arabic{example}}
\newtheorem*{solution}{Solution}
\newtheorem*{application}{Application}
\newtheorem*{notation}{Notation}
\newtheorem*{vocabulary}{Vocabulaire}
\newtheorem*{properties}{Propriétés}



\theoremstyle{remark}
\newtheorem*{remark}{Remarque}
\newtheorem*{rappel}{Rappel}


\usepackage{etoolbox}
\AtBeginEnvironment{exercise}{\small}
\AtBeginEnvironment{example}{\small}

\usepackage{cases}
\usepackage[red]{mypack}

\usepackage[framemethod=TikZ]{mdframed}

\definecolor{bg}{rgb}{0.4,0.25,0.95}
\definecolor{pagebg}{rgb}{0,0,0.5}
\surroundwithmdframed[
   topline=false,
   rightline=false,
   bottomline=false,
   leftmargin=\parindent,
   skipabove=8pt,
   skipbelow=8pt,
   linecolor=blue,
   innerbottommargin=10pt,
   % backgroundcolor=bg,font=\color{orange}\sffamily, fontcolor=white
]{definition}

\usepackage{empheq}
\usepackage[most]{tcolorbox}

\newtcbox{\mymath}[1][]{%
    nobeforeafter, math upper, tcbox raise base,
    enhanced, colframe=blue!30!black,
    colback=red!10, boxrule=1pt,
    #1}

\usepackage{unixode}


\DeclareMathOperator{\ord}{ord}
\DeclareMathOperator{\orb}{orb}
\DeclareMathOperator{\stab}{stab}
\DeclareMathOperator{\Stab}{stab}
\DeclareMathOperator{\ppcm}{ppcm}
\DeclareMathOperator{\conj}{Conj}
\DeclareMathOperator{\End}{End}
\DeclareMathOperator{\rot}{rot}
\DeclareMathOperator{\trs}{trace}
\DeclareMathOperator{\Ind}{Ind}
\DeclareMathOperator{\mat}{Mat}
\DeclareMathOperator{\id}{Id}
\DeclareMathOperator{\vect}{vect}
\DeclareMathOperator{\img}{img}
\DeclareMathOperator{\cov}{Cov}
\DeclareMathOperator{\dist}{dist}
\DeclareMathOperator{\irr}{Irr}
\DeclareMathOperator{\image}{Im}
\DeclareMathOperator{\pd}{\partial}
\DeclareMathOperator{\epi}{epi}
\DeclareMathOperator{\Argmin}{Argmin}
\DeclareMathOperator{\dom}{dom}
\DeclareMathOperator{\proj}{proj}
\DeclareMathOperator{\ctg}{ctg}
\DeclareMathOperator{\supp}{supp}
\DeclareMathOperator{\argmin}{argmin}
\DeclareMathOperator{\mult}{mult}
\DeclareMathOperator{\ch}{ch}
\DeclareMathOperator{\sh}{sh}
\DeclareMathOperator{\rang}{rang}
\DeclareMathOperator{\diam}{diam}
\DeclareMathOperator{\Epigraphe}{Epigraphe}




\usepackage{xcolor}
\everymath{\color{blue}}
%\everymath{\color[rgb]{0,1,1}}
%\pagecolor[rgb]{0,0,0.5}


\newcommand*{\pdtest}[3][]{\ensuremath{\frac{\partial^{#1} #2}{\partial #3}}}

\newcommand*{\deffunc}[6][]{\ensuremath{
\begin{array}{rcl}
#2 : #3 &\rightarrow& #4\\
#5 &\mapsto& #6
\end{array}
}}

\newcommand{\eqcolon}{\mathrel{\resizebox{\widthof{$\mathord{=}$}}{\height}{ $\!\!=\!\!\resizebox{1.2\width}{0.8\height}{\raisebox{0.23ex}{$\mathop{:}$}}\!\!$ }}}
\newcommand{\coloneq}{\mathrel{\resizebox{\widthof{$\mathord{=}$}}{\height}{ $\!\!\resizebox{1.2\width}{0.8\height}{\raisebox{0.23ex}{$\mathop{:}$}}\!\!=\!\!$ }}}
\newcommand{\eqcolonl}{\ensuremath{\mathrel{=\!\!\mathop{:}}}}
\newcommand{\coloneql}{\ensuremath{\mathrel{\mathop{:} \!\! =}}}
\newcommand{\vc}[1]{% inline column vector
  \left(\begin{smallmatrix}#1\end{smallmatrix}\right)%
}
\newcommand{\vr}[1]{% inline row vector
  \begin{smallmatrix}(\,#1\,)\end{smallmatrix}%
}
\makeatletter
\newcommand*{\defeq}{\ =\mathrel{\rlap{%
                     \raisebox{0.3ex}{$\m@th\cdot$}}%
                     \raisebox{-0.3ex}{$\m@th\cdot$}}%
                     }
\makeatother

\newcommand{\mathcircle}[1]{% inline row vector
 \overset{\circ}{#1}
}
\newcommand{\ulim}{% low limit
 \underline{\lim}
}
\newcommand{\ssi}{% iff
\iff
}
\newcommand{\ps}[2]{
\expval{#1 | #2}
}
\newcommand{\df}[1]{
\mqty{#1}
}
\newcommand{\n}[1]{
\norm{#1}
}
\newcommand{\sys}[1]{
\left\{\smqty{#1}\right.
}


\newcommand{\eqdef}{\ensuremath{\overset{\text{def}}=}}


\def\Circlearrowright{\ensuremath{%
  \rotatebox[origin=c]{230}{$\circlearrowright$}}}

\newcommand\ct[1]{\text{\rmfamily\upshape #1}}
\newcommand\question[1]{ {\color{red} ...!? \small #1}}
\newcommand\caz[1]{\left\{\begin{array} #1 \end{array}\right.}
\newcommand\const{\text{\rmfamily\upshape const}}
\newcommand\toP{ \overset{\pro}{\to}}
\newcommand\toPP{ \overset{\text{PP}}{\to}}
\newcommand{\oeq}{\mathrel{\text{\textcircled{$=$}}}}





\usepackage{xcolor}
% \usepackage[normalem]{ulem}
\usepackage{lipsum}
\makeatletter
% \newcommand\colorwave[1][blue]{\bgroup \markoverwith{\lower3.5\p@\hbox{\sixly \textcolor{#1}{\char58}}}\ULon}
%\font\sixly=lasy6 % does not re-load if already loaded, so no memory problem.

\newmdtheoremenv[
linewidth= 1pt,linecolor= blue,%
leftmargin=20,rightmargin=20,innertopmargin=0pt, innerrightmargin=40,%
tikzsetting = { draw=lightgray, line width = 0.3pt,dashed,%
dash pattern = on 15pt off 3pt},%
splittopskip=\topskip,skipbelow=\baselineskip,%
skipabove=\baselineskip,ntheorem,roundcorner=0pt,
% backgroundcolor=pagebg,font=\color{orange}\sffamily, fontcolor=white
]{examplebox}{Exemple}[section]



\newcommand\R{\mathbb{R}}
\newcommand\Z{\mathbb{Z}}
\newcommand\N{\mathbb{N}}
\newcommand\E{\mathbb{E}}
\newcommand\F{\mathcal{F}}
\newcommand\cH{\mathcal{H}}
\newcommand\V{\mathbb{V}}
\newcommand\dmo{ ^{-1} }
\newcommand\kapa{\kappa}
\newcommand\im{Im}
\newcommand\hs{\mathcal{H}}





\usepackage{soul}

\makeatletter
\newcommand*{\whiten}[1]{\llap{\textcolor{white}{{\the\SOUL@token}}\hspace{#1pt}}}
\DeclareRobustCommand*\myul{%
    \def\SOUL@everyspace{\underline{\space}\kern\z@}%
    \def\SOUL@everytoken{%
     \setbox0=\hbox{\the\SOUL@token}%
     \ifdim\dp0>\z@
        \raisebox{\dp0}{\underline{\phantom{\the\SOUL@token}}}%
        \whiten{1}\whiten{0}%
        \whiten{-1}\whiten{-2}%
        \llap{\the\SOUL@token}%
     \else
        \underline{\the\SOUL@token}%
     \fi}%
\SOUL@}
\makeatother

\newcommand*{\demp}{\fontfamily{lmtt}\selectfont}

\DeclareTextFontCommand{\textdemp}{\demp}

\begin{document}

\ifcomment
Multiline
comment
\fi
\ifcomment
\myul{Typesetting test}
% \color[rgb]{1,1,1}
$∑_i^n≠ 60º±∞π∆¬≈√j∫h≤≥µ$

$\CR \R\pro\ind\pro\gS\pro
\mqty[a&b\\c&d]$
$\pro\mathbb{P}$
$\dd{x}$

  \[
    \alpha(x)=\left\{
                \begin{array}{ll}
                  x\\
                  \frac{1}{1+e^{-kx}}\\
                  \frac{e^x-e^{-x}}{e^x+e^{-x}}
                \end{array}
              \right.
  \]

  $\expval{x}$
  
  $\chi_\rho(ghg\dmo)=\Tr(\rho_{ghg\dmo})=\Tr(\rho_g\circ\rho_h\circ\rho\dmo_g)=\Tr(\rho_h)\overset{\mbox{\scalebox{0.5}{$\Tr(AB)=\Tr(BA)$}}}{=}\chi_\rho(h)$
  	$\mathop{\oplus}_{\substack{x\in X}}$

$\mat(\rho_g)=(a_{ij}(g))_{\scriptsize \substack{1\leq i\leq d \\ 1\leq j\leq d}}$ et $\mat(\rho'_g)=(a'_{ij}(g))_{\scriptsize \substack{1\leq i'\leq d' \\ 1\leq j'\leq d'}}$



\[\int_a^b{\mathbb{R}^2}g(u, v)\dd{P_{XY}}(u, v)=\iint g(u,v) f_{XY}(u, v)\dd \lambda(u) \dd \lambda(v)\]
$$\lim_{x\to\infty} f(x)$$	
$$\iiiint_V \mu(t,u,v,w) \,dt\,du\,dv\,dw$$
$$\sum_{n=1}^{\infty} 2^{-n} = 1$$	
\begin{definition}
	Si $X$ et $Y$ sont 2 v.a. ou definit la \textsc{Covariance} entre $X$ et $Y$ comme
	$\cov(X,Y)\overset{\text{def}}{=}\E\left[(X-\E(X))(Y-\E(Y))\right]=\E(XY)-\E(X)\E(Y)$.
\end{definition}
\fi
\pagebreak

% \tableofcontents

% insert your code here
%\input{./algebra/main.tex}
%\input{./geometrie-differentielle/main.tex}
%\input{./probabilite/main.tex}
%\input{./analyse-fonctionnelle/main.tex}
% \input{./Analyse-convexe-et-dualite-en-optimisation/main.tex}
%\input{./tikz/main.tex}
%\input{./Theorie-du-distributions/main.tex}
%\input{./optimisation/mine.tex}
 \input{./modelisation/main.tex}

% yves.aubry@univ-tln.fr : algebra

\end{document}

%% !TEX encoding = UTF-8 Unicode
% !TEX TS-program = xelatex

\documentclass[french]{report}

%\usepackage[utf8]{inputenc}
%\usepackage[T1]{fontenc}
\usepackage{babel}


\newif\ifcomment
%\commenttrue # Show comments

\usepackage{physics}
\usepackage{amssymb}


\usepackage{amsthm}
% \usepackage{thmtools}
\usepackage{mathtools}
\usepackage{amsfonts}

\usepackage{color}

\usepackage{tikz}

\usepackage{geometry}
\geometry{a5paper, margin=0.1in, right=1cm}

\usepackage{dsfont}

\usepackage{graphicx}
\graphicspath{ {images/} }

\usepackage{faktor}

\usepackage{IEEEtrantools}
\usepackage{enumerate}   
\usepackage[PostScript=dvips]{"/Users/aware/Documents/Courses/diagrams"}


\newtheorem{theorem}{Théorème}[section]
\renewcommand{\thetheorem}{\arabic{theorem}}
\newtheorem{lemme}{Lemme}[section]
\renewcommand{\thelemme}{\arabic{lemme}}
\newtheorem{proposition}{Proposition}[section]
\renewcommand{\theproposition}{\arabic{proposition}}
\newtheorem{notations}{Notations}[section]
\newtheorem{problem}{Problème}[section]
\newtheorem{corollary}{Corollaire}[theorem]
\renewcommand{\thecorollary}{\arabic{corollary}}
\newtheorem{property}{Propriété}[section]
\newtheorem{objective}{Objectif}[section]

\theoremstyle{definition}
\newtheorem{definition}{Définition}[section]
\renewcommand{\thedefinition}{\arabic{definition}}
\newtheorem{exercise}{Exercice}[chapter]
\renewcommand{\theexercise}{\arabic{exercise}}
\newtheorem{example}{Exemple}[chapter]
\renewcommand{\theexample}{\arabic{example}}
\newtheorem*{solution}{Solution}
\newtheorem*{application}{Application}
\newtheorem*{notation}{Notation}
\newtheorem*{vocabulary}{Vocabulaire}
\newtheorem*{properties}{Propriétés}



\theoremstyle{remark}
\newtheorem*{remark}{Remarque}
\newtheorem*{rappel}{Rappel}


\usepackage{etoolbox}
\AtBeginEnvironment{exercise}{\small}
\AtBeginEnvironment{example}{\small}

\usepackage{cases}
\usepackage[red]{mypack}

\usepackage[framemethod=TikZ]{mdframed}

\definecolor{bg}{rgb}{0.4,0.25,0.95}
\definecolor{pagebg}{rgb}{0,0,0.5}
\surroundwithmdframed[
   topline=false,
   rightline=false,
   bottomline=false,
   leftmargin=\parindent,
   skipabove=8pt,
   skipbelow=8pt,
   linecolor=blue,
   innerbottommargin=10pt,
   % backgroundcolor=bg,font=\color{orange}\sffamily, fontcolor=white
]{definition}

\usepackage{empheq}
\usepackage[most]{tcolorbox}

\newtcbox{\mymath}[1][]{%
    nobeforeafter, math upper, tcbox raise base,
    enhanced, colframe=blue!30!black,
    colback=red!10, boxrule=1pt,
    #1}

\usepackage{unixode}


\DeclareMathOperator{\ord}{ord}
\DeclareMathOperator{\orb}{orb}
\DeclareMathOperator{\stab}{stab}
\DeclareMathOperator{\Stab}{stab}
\DeclareMathOperator{\ppcm}{ppcm}
\DeclareMathOperator{\conj}{Conj}
\DeclareMathOperator{\End}{End}
\DeclareMathOperator{\rot}{rot}
\DeclareMathOperator{\trs}{trace}
\DeclareMathOperator{\Ind}{Ind}
\DeclareMathOperator{\mat}{Mat}
\DeclareMathOperator{\id}{Id}
\DeclareMathOperator{\vect}{vect}
\DeclareMathOperator{\img}{img}
\DeclareMathOperator{\cov}{Cov}
\DeclareMathOperator{\dist}{dist}
\DeclareMathOperator{\irr}{Irr}
\DeclareMathOperator{\image}{Im}
\DeclareMathOperator{\pd}{\partial}
\DeclareMathOperator{\epi}{epi}
\DeclareMathOperator{\Argmin}{Argmin}
\DeclareMathOperator{\dom}{dom}
\DeclareMathOperator{\proj}{proj}
\DeclareMathOperator{\ctg}{ctg}
\DeclareMathOperator{\supp}{supp}
\DeclareMathOperator{\argmin}{argmin}
\DeclareMathOperator{\mult}{mult}
\DeclareMathOperator{\ch}{ch}
\DeclareMathOperator{\sh}{sh}
\DeclareMathOperator{\rang}{rang}
\DeclareMathOperator{\diam}{diam}
\DeclareMathOperator{\Epigraphe}{Epigraphe}




\usepackage{xcolor}
\everymath{\color{blue}}
%\everymath{\color[rgb]{0,1,1}}
%\pagecolor[rgb]{0,0,0.5}


\newcommand*{\pdtest}[3][]{\ensuremath{\frac{\partial^{#1} #2}{\partial #3}}}

\newcommand*{\deffunc}[6][]{\ensuremath{
\begin{array}{rcl}
#2 : #3 &\rightarrow& #4\\
#5 &\mapsto& #6
\end{array}
}}

\newcommand{\eqcolon}{\mathrel{\resizebox{\widthof{$\mathord{=}$}}{\height}{ $\!\!=\!\!\resizebox{1.2\width}{0.8\height}{\raisebox{0.23ex}{$\mathop{:}$}}\!\!$ }}}
\newcommand{\coloneq}{\mathrel{\resizebox{\widthof{$\mathord{=}$}}{\height}{ $\!\!\resizebox{1.2\width}{0.8\height}{\raisebox{0.23ex}{$\mathop{:}$}}\!\!=\!\!$ }}}
\newcommand{\eqcolonl}{\ensuremath{\mathrel{=\!\!\mathop{:}}}}
\newcommand{\coloneql}{\ensuremath{\mathrel{\mathop{:} \!\! =}}}
\newcommand{\vc}[1]{% inline column vector
  \left(\begin{smallmatrix}#1\end{smallmatrix}\right)%
}
\newcommand{\vr}[1]{% inline row vector
  \begin{smallmatrix}(\,#1\,)\end{smallmatrix}%
}
\makeatletter
\newcommand*{\defeq}{\ =\mathrel{\rlap{%
                     \raisebox{0.3ex}{$\m@th\cdot$}}%
                     \raisebox{-0.3ex}{$\m@th\cdot$}}%
                     }
\makeatother

\newcommand{\mathcircle}[1]{% inline row vector
 \overset{\circ}{#1}
}
\newcommand{\ulim}{% low limit
 \underline{\lim}
}
\newcommand{\ssi}{% iff
\iff
}
\newcommand{\ps}[2]{
\expval{#1 | #2}
}
\newcommand{\df}[1]{
\mqty{#1}
}
\newcommand{\n}[1]{
\norm{#1}
}
\newcommand{\sys}[1]{
\left\{\smqty{#1}\right.
}


\newcommand{\eqdef}{\ensuremath{\overset{\text{def}}=}}


\def\Circlearrowright{\ensuremath{%
  \rotatebox[origin=c]{230}{$\circlearrowright$}}}

\newcommand\ct[1]{\text{\rmfamily\upshape #1}}
\newcommand\question[1]{ {\color{red} ...!? \small #1}}
\newcommand\caz[1]{\left\{\begin{array} #1 \end{array}\right.}
\newcommand\const{\text{\rmfamily\upshape const}}
\newcommand\toP{ \overset{\pro}{\to}}
\newcommand\toPP{ \overset{\text{PP}}{\to}}
\newcommand{\oeq}{\mathrel{\text{\textcircled{$=$}}}}





\usepackage{xcolor}
% \usepackage[normalem]{ulem}
\usepackage{lipsum}
\makeatletter
% \newcommand\colorwave[1][blue]{\bgroup \markoverwith{\lower3.5\p@\hbox{\sixly \textcolor{#1}{\char58}}}\ULon}
%\font\sixly=lasy6 % does not re-load if already loaded, so no memory problem.

\newmdtheoremenv[
linewidth= 1pt,linecolor= blue,%
leftmargin=20,rightmargin=20,innertopmargin=0pt, innerrightmargin=40,%
tikzsetting = { draw=lightgray, line width = 0.3pt,dashed,%
dash pattern = on 15pt off 3pt},%
splittopskip=\topskip,skipbelow=\baselineskip,%
skipabove=\baselineskip,ntheorem,roundcorner=0pt,
% backgroundcolor=pagebg,font=\color{orange}\sffamily, fontcolor=white
]{examplebox}{Exemple}[section]



\newcommand\R{\mathbb{R}}
\newcommand\Z{\mathbb{Z}}
\newcommand\N{\mathbb{N}}
\newcommand\E{\mathbb{E}}
\newcommand\F{\mathcal{F}}
\newcommand\cH{\mathcal{H}}
\newcommand\V{\mathbb{V}}
\newcommand\dmo{ ^{-1} }
\newcommand\kapa{\kappa}
\newcommand\im{Im}
\newcommand\hs{\mathcal{H}}





\usepackage{soul}

\makeatletter
\newcommand*{\whiten}[1]{\llap{\textcolor{white}{{\the\SOUL@token}}\hspace{#1pt}}}
\DeclareRobustCommand*\myul{%
    \def\SOUL@everyspace{\underline{\space}\kern\z@}%
    \def\SOUL@everytoken{%
     \setbox0=\hbox{\the\SOUL@token}%
     \ifdim\dp0>\z@
        \raisebox{\dp0}{\underline{\phantom{\the\SOUL@token}}}%
        \whiten{1}\whiten{0}%
        \whiten{-1}\whiten{-2}%
        \llap{\the\SOUL@token}%
     \else
        \underline{\the\SOUL@token}%
     \fi}%
\SOUL@}
\makeatother

\newcommand*{\demp}{\fontfamily{lmtt}\selectfont}

\DeclareTextFontCommand{\textdemp}{\demp}

\begin{document}

\ifcomment
Multiline
comment
\fi
\ifcomment
\myul{Typesetting test}
% \color[rgb]{1,1,1}
$∑_i^n≠ 60º±∞π∆¬≈√j∫h≤≥µ$

$\CR \R\pro\ind\pro\gS\pro
\mqty[a&b\\c&d]$
$\pro\mathbb{P}$
$\dd{x}$

  \[
    \alpha(x)=\left\{
                \begin{array}{ll}
                  x\\
                  \frac{1}{1+e^{-kx}}\\
                  \frac{e^x-e^{-x}}{e^x+e^{-x}}
                \end{array}
              \right.
  \]

  $\expval{x}$
  
  $\chi_\rho(ghg\dmo)=\Tr(\rho_{ghg\dmo})=\Tr(\rho_g\circ\rho_h\circ\rho\dmo_g)=\Tr(\rho_h)\overset{\mbox{\scalebox{0.5}{$\Tr(AB)=\Tr(BA)$}}}{=}\chi_\rho(h)$
  	$\mathop{\oplus}_{\substack{x\in X}}$

$\mat(\rho_g)=(a_{ij}(g))_{\scriptsize \substack{1\leq i\leq d \\ 1\leq j\leq d}}$ et $\mat(\rho'_g)=(a'_{ij}(g))_{\scriptsize \substack{1\leq i'\leq d' \\ 1\leq j'\leq d'}}$



\[\int_a^b{\mathbb{R}^2}g(u, v)\dd{P_{XY}}(u, v)=\iint g(u,v) f_{XY}(u, v)\dd \lambda(u) \dd \lambda(v)\]
$$\lim_{x\to\infty} f(x)$$	
$$\iiiint_V \mu(t,u,v,w) \,dt\,du\,dv\,dw$$
$$\sum_{n=1}^{\infty} 2^{-n} = 1$$	
\begin{definition}
	Si $X$ et $Y$ sont 2 v.a. ou definit la \textsc{Covariance} entre $X$ et $Y$ comme
	$\cov(X,Y)\overset{\text{def}}{=}\E\left[(X-\E(X))(Y-\E(Y))\right]=\E(XY)-\E(X)\E(Y)$.
\end{definition}
\fi
\pagebreak

% \tableofcontents

% insert your code here
%\input{./algebra/main.tex}
%\input{./geometrie-differentielle/main.tex}
%\input{./probabilite/main.tex}
%\input{./analyse-fonctionnelle/main.tex}
% \input{./Analyse-convexe-et-dualite-en-optimisation/main.tex}
%\input{./tikz/main.tex}
%\input{./Theorie-du-distributions/main.tex}
%\input{./optimisation/mine.tex}
 \input{./modelisation/main.tex}

% yves.aubry@univ-tln.fr : algebra

\end{document}

%% !TEX encoding = UTF-8 Unicode
% !TEX TS-program = xelatex

\documentclass[french]{report}

%\usepackage[utf8]{inputenc}
%\usepackage[T1]{fontenc}
\usepackage{babel}


\newif\ifcomment
%\commenttrue # Show comments

\usepackage{physics}
\usepackage{amssymb}


\usepackage{amsthm}
% \usepackage{thmtools}
\usepackage{mathtools}
\usepackage{amsfonts}

\usepackage{color}

\usepackage{tikz}

\usepackage{geometry}
\geometry{a5paper, margin=0.1in, right=1cm}

\usepackage{dsfont}

\usepackage{graphicx}
\graphicspath{ {images/} }

\usepackage{faktor}

\usepackage{IEEEtrantools}
\usepackage{enumerate}   
\usepackage[PostScript=dvips]{"/Users/aware/Documents/Courses/diagrams"}


\newtheorem{theorem}{Théorème}[section]
\renewcommand{\thetheorem}{\arabic{theorem}}
\newtheorem{lemme}{Lemme}[section]
\renewcommand{\thelemme}{\arabic{lemme}}
\newtheorem{proposition}{Proposition}[section]
\renewcommand{\theproposition}{\arabic{proposition}}
\newtheorem{notations}{Notations}[section]
\newtheorem{problem}{Problème}[section]
\newtheorem{corollary}{Corollaire}[theorem]
\renewcommand{\thecorollary}{\arabic{corollary}}
\newtheorem{property}{Propriété}[section]
\newtheorem{objective}{Objectif}[section]

\theoremstyle{definition}
\newtheorem{definition}{Définition}[section]
\renewcommand{\thedefinition}{\arabic{definition}}
\newtheorem{exercise}{Exercice}[chapter]
\renewcommand{\theexercise}{\arabic{exercise}}
\newtheorem{example}{Exemple}[chapter]
\renewcommand{\theexample}{\arabic{example}}
\newtheorem*{solution}{Solution}
\newtheorem*{application}{Application}
\newtheorem*{notation}{Notation}
\newtheorem*{vocabulary}{Vocabulaire}
\newtheorem*{properties}{Propriétés}



\theoremstyle{remark}
\newtheorem*{remark}{Remarque}
\newtheorem*{rappel}{Rappel}


\usepackage{etoolbox}
\AtBeginEnvironment{exercise}{\small}
\AtBeginEnvironment{example}{\small}

\usepackage{cases}
\usepackage[red]{mypack}

\usepackage[framemethod=TikZ]{mdframed}

\definecolor{bg}{rgb}{0.4,0.25,0.95}
\definecolor{pagebg}{rgb}{0,0,0.5}
\surroundwithmdframed[
   topline=false,
   rightline=false,
   bottomline=false,
   leftmargin=\parindent,
   skipabove=8pt,
   skipbelow=8pt,
   linecolor=blue,
   innerbottommargin=10pt,
   % backgroundcolor=bg,font=\color{orange}\sffamily, fontcolor=white
]{definition}

\usepackage{empheq}
\usepackage[most]{tcolorbox}

\newtcbox{\mymath}[1][]{%
    nobeforeafter, math upper, tcbox raise base,
    enhanced, colframe=blue!30!black,
    colback=red!10, boxrule=1pt,
    #1}

\usepackage{unixode}


\DeclareMathOperator{\ord}{ord}
\DeclareMathOperator{\orb}{orb}
\DeclareMathOperator{\stab}{stab}
\DeclareMathOperator{\Stab}{stab}
\DeclareMathOperator{\ppcm}{ppcm}
\DeclareMathOperator{\conj}{Conj}
\DeclareMathOperator{\End}{End}
\DeclareMathOperator{\rot}{rot}
\DeclareMathOperator{\trs}{trace}
\DeclareMathOperator{\Ind}{Ind}
\DeclareMathOperator{\mat}{Mat}
\DeclareMathOperator{\id}{Id}
\DeclareMathOperator{\vect}{vect}
\DeclareMathOperator{\img}{img}
\DeclareMathOperator{\cov}{Cov}
\DeclareMathOperator{\dist}{dist}
\DeclareMathOperator{\irr}{Irr}
\DeclareMathOperator{\image}{Im}
\DeclareMathOperator{\pd}{\partial}
\DeclareMathOperator{\epi}{epi}
\DeclareMathOperator{\Argmin}{Argmin}
\DeclareMathOperator{\dom}{dom}
\DeclareMathOperator{\proj}{proj}
\DeclareMathOperator{\ctg}{ctg}
\DeclareMathOperator{\supp}{supp}
\DeclareMathOperator{\argmin}{argmin}
\DeclareMathOperator{\mult}{mult}
\DeclareMathOperator{\ch}{ch}
\DeclareMathOperator{\sh}{sh}
\DeclareMathOperator{\rang}{rang}
\DeclareMathOperator{\diam}{diam}
\DeclareMathOperator{\Epigraphe}{Epigraphe}




\usepackage{xcolor}
\everymath{\color{blue}}
%\everymath{\color[rgb]{0,1,1}}
%\pagecolor[rgb]{0,0,0.5}


\newcommand*{\pdtest}[3][]{\ensuremath{\frac{\partial^{#1} #2}{\partial #3}}}

\newcommand*{\deffunc}[6][]{\ensuremath{
\begin{array}{rcl}
#2 : #3 &\rightarrow& #4\\
#5 &\mapsto& #6
\end{array}
}}

\newcommand{\eqcolon}{\mathrel{\resizebox{\widthof{$\mathord{=}$}}{\height}{ $\!\!=\!\!\resizebox{1.2\width}{0.8\height}{\raisebox{0.23ex}{$\mathop{:}$}}\!\!$ }}}
\newcommand{\coloneq}{\mathrel{\resizebox{\widthof{$\mathord{=}$}}{\height}{ $\!\!\resizebox{1.2\width}{0.8\height}{\raisebox{0.23ex}{$\mathop{:}$}}\!\!=\!\!$ }}}
\newcommand{\eqcolonl}{\ensuremath{\mathrel{=\!\!\mathop{:}}}}
\newcommand{\coloneql}{\ensuremath{\mathrel{\mathop{:} \!\! =}}}
\newcommand{\vc}[1]{% inline column vector
  \left(\begin{smallmatrix}#1\end{smallmatrix}\right)%
}
\newcommand{\vr}[1]{% inline row vector
  \begin{smallmatrix}(\,#1\,)\end{smallmatrix}%
}
\makeatletter
\newcommand*{\defeq}{\ =\mathrel{\rlap{%
                     \raisebox{0.3ex}{$\m@th\cdot$}}%
                     \raisebox{-0.3ex}{$\m@th\cdot$}}%
                     }
\makeatother

\newcommand{\mathcircle}[1]{% inline row vector
 \overset{\circ}{#1}
}
\newcommand{\ulim}{% low limit
 \underline{\lim}
}
\newcommand{\ssi}{% iff
\iff
}
\newcommand{\ps}[2]{
\expval{#1 | #2}
}
\newcommand{\df}[1]{
\mqty{#1}
}
\newcommand{\n}[1]{
\norm{#1}
}
\newcommand{\sys}[1]{
\left\{\smqty{#1}\right.
}


\newcommand{\eqdef}{\ensuremath{\overset{\text{def}}=}}


\def\Circlearrowright{\ensuremath{%
  \rotatebox[origin=c]{230}{$\circlearrowright$}}}

\newcommand\ct[1]{\text{\rmfamily\upshape #1}}
\newcommand\question[1]{ {\color{red} ...!? \small #1}}
\newcommand\caz[1]{\left\{\begin{array} #1 \end{array}\right.}
\newcommand\const{\text{\rmfamily\upshape const}}
\newcommand\toP{ \overset{\pro}{\to}}
\newcommand\toPP{ \overset{\text{PP}}{\to}}
\newcommand{\oeq}{\mathrel{\text{\textcircled{$=$}}}}





\usepackage{xcolor}
% \usepackage[normalem]{ulem}
\usepackage{lipsum}
\makeatletter
% \newcommand\colorwave[1][blue]{\bgroup \markoverwith{\lower3.5\p@\hbox{\sixly \textcolor{#1}{\char58}}}\ULon}
%\font\sixly=lasy6 % does not re-load if already loaded, so no memory problem.

\newmdtheoremenv[
linewidth= 1pt,linecolor= blue,%
leftmargin=20,rightmargin=20,innertopmargin=0pt, innerrightmargin=40,%
tikzsetting = { draw=lightgray, line width = 0.3pt,dashed,%
dash pattern = on 15pt off 3pt},%
splittopskip=\topskip,skipbelow=\baselineskip,%
skipabove=\baselineskip,ntheorem,roundcorner=0pt,
% backgroundcolor=pagebg,font=\color{orange}\sffamily, fontcolor=white
]{examplebox}{Exemple}[section]



\newcommand\R{\mathbb{R}}
\newcommand\Z{\mathbb{Z}}
\newcommand\N{\mathbb{N}}
\newcommand\E{\mathbb{E}}
\newcommand\F{\mathcal{F}}
\newcommand\cH{\mathcal{H}}
\newcommand\V{\mathbb{V}}
\newcommand\dmo{ ^{-1} }
\newcommand\kapa{\kappa}
\newcommand\im{Im}
\newcommand\hs{\mathcal{H}}





\usepackage{soul}

\makeatletter
\newcommand*{\whiten}[1]{\llap{\textcolor{white}{{\the\SOUL@token}}\hspace{#1pt}}}
\DeclareRobustCommand*\myul{%
    \def\SOUL@everyspace{\underline{\space}\kern\z@}%
    \def\SOUL@everytoken{%
     \setbox0=\hbox{\the\SOUL@token}%
     \ifdim\dp0>\z@
        \raisebox{\dp0}{\underline{\phantom{\the\SOUL@token}}}%
        \whiten{1}\whiten{0}%
        \whiten{-1}\whiten{-2}%
        \llap{\the\SOUL@token}%
     \else
        \underline{\the\SOUL@token}%
     \fi}%
\SOUL@}
\makeatother

\newcommand*{\demp}{\fontfamily{lmtt}\selectfont}

\DeclareTextFontCommand{\textdemp}{\demp}

\begin{document}

\ifcomment
Multiline
comment
\fi
\ifcomment
\myul{Typesetting test}
% \color[rgb]{1,1,1}
$∑_i^n≠ 60º±∞π∆¬≈√j∫h≤≥µ$

$\CR \R\pro\ind\pro\gS\pro
\mqty[a&b\\c&d]$
$\pro\mathbb{P}$
$\dd{x}$

  \[
    \alpha(x)=\left\{
                \begin{array}{ll}
                  x\\
                  \frac{1}{1+e^{-kx}}\\
                  \frac{e^x-e^{-x}}{e^x+e^{-x}}
                \end{array}
              \right.
  \]

  $\expval{x}$
  
  $\chi_\rho(ghg\dmo)=\Tr(\rho_{ghg\dmo})=\Tr(\rho_g\circ\rho_h\circ\rho\dmo_g)=\Tr(\rho_h)\overset{\mbox{\scalebox{0.5}{$\Tr(AB)=\Tr(BA)$}}}{=}\chi_\rho(h)$
  	$\mathop{\oplus}_{\substack{x\in X}}$

$\mat(\rho_g)=(a_{ij}(g))_{\scriptsize \substack{1\leq i\leq d \\ 1\leq j\leq d}}$ et $\mat(\rho'_g)=(a'_{ij}(g))_{\scriptsize \substack{1\leq i'\leq d' \\ 1\leq j'\leq d'}}$



\[\int_a^b{\mathbb{R}^2}g(u, v)\dd{P_{XY}}(u, v)=\iint g(u,v) f_{XY}(u, v)\dd \lambda(u) \dd \lambda(v)\]
$$\lim_{x\to\infty} f(x)$$	
$$\iiiint_V \mu(t,u,v,w) \,dt\,du\,dv\,dw$$
$$\sum_{n=1}^{\infty} 2^{-n} = 1$$	
\begin{definition}
	Si $X$ et $Y$ sont 2 v.a. ou definit la \textsc{Covariance} entre $X$ et $Y$ comme
	$\cov(X,Y)\overset{\text{def}}{=}\E\left[(X-\E(X))(Y-\E(Y))\right]=\E(XY)-\E(X)\E(Y)$.
\end{definition}
\fi
\pagebreak

% \tableofcontents

% insert your code here
%\input{./algebra/main.tex}
%\input{./geometrie-differentielle/main.tex}
%\input{./probabilite/main.tex}
%\input{./analyse-fonctionnelle/main.tex}
% \input{./Analyse-convexe-et-dualite-en-optimisation/main.tex}
%\input{./tikz/main.tex}
%\input{./Theorie-du-distributions/main.tex}
%\input{./optimisation/mine.tex}
 \input{./modelisation/main.tex}

% yves.aubry@univ-tln.fr : algebra

\end{document}

% % !TEX encoding = UTF-8 Unicode
% !TEX TS-program = xelatex

\documentclass[french]{report}

%\usepackage[utf8]{inputenc}
%\usepackage[T1]{fontenc}
\usepackage{babel}


\newif\ifcomment
%\commenttrue # Show comments

\usepackage{physics}
\usepackage{amssymb}


\usepackage{amsthm}
% \usepackage{thmtools}
\usepackage{mathtools}
\usepackage{amsfonts}

\usepackage{color}

\usepackage{tikz}

\usepackage{geometry}
\geometry{a5paper, margin=0.1in, right=1cm}

\usepackage{dsfont}

\usepackage{graphicx}
\graphicspath{ {images/} }

\usepackage{faktor}

\usepackage{IEEEtrantools}
\usepackage{enumerate}   
\usepackage[PostScript=dvips]{"/Users/aware/Documents/Courses/diagrams"}


\newtheorem{theorem}{Théorème}[section]
\renewcommand{\thetheorem}{\arabic{theorem}}
\newtheorem{lemme}{Lemme}[section]
\renewcommand{\thelemme}{\arabic{lemme}}
\newtheorem{proposition}{Proposition}[section]
\renewcommand{\theproposition}{\arabic{proposition}}
\newtheorem{notations}{Notations}[section]
\newtheorem{problem}{Problème}[section]
\newtheorem{corollary}{Corollaire}[theorem]
\renewcommand{\thecorollary}{\arabic{corollary}}
\newtheorem{property}{Propriété}[section]
\newtheorem{objective}{Objectif}[section]

\theoremstyle{definition}
\newtheorem{definition}{Définition}[section]
\renewcommand{\thedefinition}{\arabic{definition}}
\newtheorem{exercise}{Exercice}[chapter]
\renewcommand{\theexercise}{\arabic{exercise}}
\newtheorem{example}{Exemple}[chapter]
\renewcommand{\theexample}{\arabic{example}}
\newtheorem*{solution}{Solution}
\newtheorem*{application}{Application}
\newtheorem*{notation}{Notation}
\newtheorem*{vocabulary}{Vocabulaire}
\newtheorem*{properties}{Propriétés}



\theoremstyle{remark}
\newtheorem*{remark}{Remarque}
\newtheorem*{rappel}{Rappel}


\usepackage{etoolbox}
\AtBeginEnvironment{exercise}{\small}
\AtBeginEnvironment{example}{\small}

\usepackage{cases}
\usepackage[red]{mypack}

\usepackage[framemethod=TikZ]{mdframed}

\definecolor{bg}{rgb}{0.4,0.25,0.95}
\definecolor{pagebg}{rgb}{0,0,0.5}
\surroundwithmdframed[
   topline=false,
   rightline=false,
   bottomline=false,
   leftmargin=\parindent,
   skipabove=8pt,
   skipbelow=8pt,
   linecolor=blue,
   innerbottommargin=10pt,
   % backgroundcolor=bg,font=\color{orange}\sffamily, fontcolor=white
]{definition}

\usepackage{empheq}
\usepackage[most]{tcolorbox}

\newtcbox{\mymath}[1][]{%
    nobeforeafter, math upper, tcbox raise base,
    enhanced, colframe=blue!30!black,
    colback=red!10, boxrule=1pt,
    #1}

\usepackage{unixode}


\DeclareMathOperator{\ord}{ord}
\DeclareMathOperator{\orb}{orb}
\DeclareMathOperator{\stab}{stab}
\DeclareMathOperator{\Stab}{stab}
\DeclareMathOperator{\ppcm}{ppcm}
\DeclareMathOperator{\conj}{Conj}
\DeclareMathOperator{\End}{End}
\DeclareMathOperator{\rot}{rot}
\DeclareMathOperator{\trs}{trace}
\DeclareMathOperator{\Ind}{Ind}
\DeclareMathOperator{\mat}{Mat}
\DeclareMathOperator{\id}{Id}
\DeclareMathOperator{\vect}{vect}
\DeclareMathOperator{\img}{img}
\DeclareMathOperator{\cov}{Cov}
\DeclareMathOperator{\dist}{dist}
\DeclareMathOperator{\irr}{Irr}
\DeclareMathOperator{\image}{Im}
\DeclareMathOperator{\pd}{\partial}
\DeclareMathOperator{\epi}{epi}
\DeclareMathOperator{\Argmin}{Argmin}
\DeclareMathOperator{\dom}{dom}
\DeclareMathOperator{\proj}{proj}
\DeclareMathOperator{\ctg}{ctg}
\DeclareMathOperator{\supp}{supp}
\DeclareMathOperator{\argmin}{argmin}
\DeclareMathOperator{\mult}{mult}
\DeclareMathOperator{\ch}{ch}
\DeclareMathOperator{\sh}{sh}
\DeclareMathOperator{\rang}{rang}
\DeclareMathOperator{\diam}{diam}
\DeclareMathOperator{\Epigraphe}{Epigraphe}




\usepackage{xcolor}
\everymath{\color{blue}}
%\everymath{\color[rgb]{0,1,1}}
%\pagecolor[rgb]{0,0,0.5}


\newcommand*{\pdtest}[3][]{\ensuremath{\frac{\partial^{#1} #2}{\partial #3}}}

\newcommand*{\deffunc}[6][]{\ensuremath{
\begin{array}{rcl}
#2 : #3 &\rightarrow& #4\\
#5 &\mapsto& #6
\end{array}
}}

\newcommand{\eqcolon}{\mathrel{\resizebox{\widthof{$\mathord{=}$}}{\height}{ $\!\!=\!\!\resizebox{1.2\width}{0.8\height}{\raisebox{0.23ex}{$\mathop{:}$}}\!\!$ }}}
\newcommand{\coloneq}{\mathrel{\resizebox{\widthof{$\mathord{=}$}}{\height}{ $\!\!\resizebox{1.2\width}{0.8\height}{\raisebox{0.23ex}{$\mathop{:}$}}\!\!=\!\!$ }}}
\newcommand{\eqcolonl}{\ensuremath{\mathrel{=\!\!\mathop{:}}}}
\newcommand{\coloneql}{\ensuremath{\mathrel{\mathop{:} \!\! =}}}
\newcommand{\vc}[1]{% inline column vector
  \left(\begin{smallmatrix}#1\end{smallmatrix}\right)%
}
\newcommand{\vr}[1]{% inline row vector
  \begin{smallmatrix}(\,#1\,)\end{smallmatrix}%
}
\makeatletter
\newcommand*{\defeq}{\ =\mathrel{\rlap{%
                     \raisebox{0.3ex}{$\m@th\cdot$}}%
                     \raisebox{-0.3ex}{$\m@th\cdot$}}%
                     }
\makeatother

\newcommand{\mathcircle}[1]{% inline row vector
 \overset{\circ}{#1}
}
\newcommand{\ulim}{% low limit
 \underline{\lim}
}
\newcommand{\ssi}{% iff
\iff
}
\newcommand{\ps}[2]{
\expval{#1 | #2}
}
\newcommand{\df}[1]{
\mqty{#1}
}
\newcommand{\n}[1]{
\norm{#1}
}
\newcommand{\sys}[1]{
\left\{\smqty{#1}\right.
}


\newcommand{\eqdef}{\ensuremath{\overset{\text{def}}=}}


\def\Circlearrowright{\ensuremath{%
  \rotatebox[origin=c]{230}{$\circlearrowright$}}}

\newcommand\ct[1]{\text{\rmfamily\upshape #1}}
\newcommand\question[1]{ {\color{red} ...!? \small #1}}
\newcommand\caz[1]{\left\{\begin{array} #1 \end{array}\right.}
\newcommand\const{\text{\rmfamily\upshape const}}
\newcommand\toP{ \overset{\pro}{\to}}
\newcommand\toPP{ \overset{\text{PP}}{\to}}
\newcommand{\oeq}{\mathrel{\text{\textcircled{$=$}}}}





\usepackage{xcolor}
% \usepackage[normalem]{ulem}
\usepackage{lipsum}
\makeatletter
% \newcommand\colorwave[1][blue]{\bgroup \markoverwith{\lower3.5\p@\hbox{\sixly \textcolor{#1}{\char58}}}\ULon}
%\font\sixly=lasy6 % does not re-load if already loaded, so no memory problem.

\newmdtheoremenv[
linewidth= 1pt,linecolor= blue,%
leftmargin=20,rightmargin=20,innertopmargin=0pt, innerrightmargin=40,%
tikzsetting = { draw=lightgray, line width = 0.3pt,dashed,%
dash pattern = on 15pt off 3pt},%
splittopskip=\topskip,skipbelow=\baselineskip,%
skipabove=\baselineskip,ntheorem,roundcorner=0pt,
% backgroundcolor=pagebg,font=\color{orange}\sffamily, fontcolor=white
]{examplebox}{Exemple}[section]



\newcommand\R{\mathbb{R}}
\newcommand\Z{\mathbb{Z}}
\newcommand\N{\mathbb{N}}
\newcommand\E{\mathbb{E}}
\newcommand\F{\mathcal{F}}
\newcommand\cH{\mathcal{H}}
\newcommand\V{\mathbb{V}}
\newcommand\dmo{ ^{-1} }
\newcommand\kapa{\kappa}
\newcommand\im{Im}
\newcommand\hs{\mathcal{H}}





\usepackage{soul}

\makeatletter
\newcommand*{\whiten}[1]{\llap{\textcolor{white}{{\the\SOUL@token}}\hspace{#1pt}}}
\DeclareRobustCommand*\myul{%
    \def\SOUL@everyspace{\underline{\space}\kern\z@}%
    \def\SOUL@everytoken{%
     \setbox0=\hbox{\the\SOUL@token}%
     \ifdim\dp0>\z@
        \raisebox{\dp0}{\underline{\phantom{\the\SOUL@token}}}%
        \whiten{1}\whiten{0}%
        \whiten{-1}\whiten{-2}%
        \llap{\the\SOUL@token}%
     \else
        \underline{\the\SOUL@token}%
     \fi}%
\SOUL@}
\makeatother

\newcommand*{\demp}{\fontfamily{lmtt}\selectfont}

\DeclareTextFontCommand{\textdemp}{\demp}

\begin{document}

\ifcomment
Multiline
comment
\fi
\ifcomment
\myul{Typesetting test}
% \color[rgb]{1,1,1}
$∑_i^n≠ 60º±∞π∆¬≈√j∫h≤≥µ$

$\CR \R\pro\ind\pro\gS\pro
\mqty[a&b\\c&d]$
$\pro\mathbb{P}$
$\dd{x}$

  \[
    \alpha(x)=\left\{
                \begin{array}{ll}
                  x\\
                  \frac{1}{1+e^{-kx}}\\
                  \frac{e^x-e^{-x}}{e^x+e^{-x}}
                \end{array}
              \right.
  \]

  $\expval{x}$
  
  $\chi_\rho(ghg\dmo)=\Tr(\rho_{ghg\dmo})=\Tr(\rho_g\circ\rho_h\circ\rho\dmo_g)=\Tr(\rho_h)\overset{\mbox{\scalebox{0.5}{$\Tr(AB)=\Tr(BA)$}}}{=}\chi_\rho(h)$
  	$\mathop{\oplus}_{\substack{x\in X}}$

$\mat(\rho_g)=(a_{ij}(g))_{\scriptsize \substack{1\leq i\leq d \\ 1\leq j\leq d}}$ et $\mat(\rho'_g)=(a'_{ij}(g))_{\scriptsize \substack{1\leq i'\leq d' \\ 1\leq j'\leq d'}}$



\[\int_a^b{\mathbb{R}^2}g(u, v)\dd{P_{XY}}(u, v)=\iint g(u,v) f_{XY}(u, v)\dd \lambda(u) \dd \lambda(v)\]
$$\lim_{x\to\infty} f(x)$$	
$$\iiiint_V \mu(t,u,v,w) \,dt\,du\,dv\,dw$$
$$\sum_{n=1}^{\infty} 2^{-n} = 1$$	
\begin{definition}
	Si $X$ et $Y$ sont 2 v.a. ou definit la \textsc{Covariance} entre $X$ et $Y$ comme
	$\cov(X,Y)\overset{\text{def}}{=}\E\left[(X-\E(X))(Y-\E(Y))\right]=\E(XY)-\E(X)\E(Y)$.
\end{definition}
\fi
\pagebreak

% \tableofcontents

% insert your code here
%\input{./algebra/main.tex}
%\input{./geometrie-differentielle/main.tex}
%\input{./probabilite/main.tex}
%\input{./analyse-fonctionnelle/main.tex}
% \input{./Analyse-convexe-et-dualite-en-optimisation/main.tex}
%\input{./tikz/main.tex}
%\input{./Theorie-du-distributions/main.tex}
%\input{./optimisation/mine.tex}
 \input{./modelisation/main.tex}

% yves.aubry@univ-tln.fr : algebra

\end{document}

%% !TEX encoding = UTF-8 Unicode
% !TEX TS-program = xelatex

\documentclass[french]{report}

%\usepackage[utf8]{inputenc}
%\usepackage[T1]{fontenc}
\usepackage{babel}


\newif\ifcomment
%\commenttrue # Show comments

\usepackage{physics}
\usepackage{amssymb}


\usepackage{amsthm}
% \usepackage{thmtools}
\usepackage{mathtools}
\usepackage{amsfonts}

\usepackage{color}

\usepackage{tikz}

\usepackage{geometry}
\geometry{a5paper, margin=0.1in, right=1cm}

\usepackage{dsfont}

\usepackage{graphicx}
\graphicspath{ {images/} }

\usepackage{faktor}

\usepackage{IEEEtrantools}
\usepackage{enumerate}   
\usepackage[PostScript=dvips]{"/Users/aware/Documents/Courses/diagrams"}


\newtheorem{theorem}{Théorème}[section]
\renewcommand{\thetheorem}{\arabic{theorem}}
\newtheorem{lemme}{Lemme}[section]
\renewcommand{\thelemme}{\arabic{lemme}}
\newtheorem{proposition}{Proposition}[section]
\renewcommand{\theproposition}{\arabic{proposition}}
\newtheorem{notations}{Notations}[section]
\newtheorem{problem}{Problème}[section]
\newtheorem{corollary}{Corollaire}[theorem]
\renewcommand{\thecorollary}{\arabic{corollary}}
\newtheorem{property}{Propriété}[section]
\newtheorem{objective}{Objectif}[section]

\theoremstyle{definition}
\newtheorem{definition}{Définition}[section]
\renewcommand{\thedefinition}{\arabic{definition}}
\newtheorem{exercise}{Exercice}[chapter]
\renewcommand{\theexercise}{\arabic{exercise}}
\newtheorem{example}{Exemple}[chapter]
\renewcommand{\theexample}{\arabic{example}}
\newtheorem*{solution}{Solution}
\newtheorem*{application}{Application}
\newtheorem*{notation}{Notation}
\newtheorem*{vocabulary}{Vocabulaire}
\newtheorem*{properties}{Propriétés}



\theoremstyle{remark}
\newtheorem*{remark}{Remarque}
\newtheorem*{rappel}{Rappel}


\usepackage{etoolbox}
\AtBeginEnvironment{exercise}{\small}
\AtBeginEnvironment{example}{\small}

\usepackage{cases}
\usepackage[red]{mypack}

\usepackage[framemethod=TikZ]{mdframed}

\definecolor{bg}{rgb}{0.4,0.25,0.95}
\definecolor{pagebg}{rgb}{0,0,0.5}
\surroundwithmdframed[
   topline=false,
   rightline=false,
   bottomline=false,
   leftmargin=\parindent,
   skipabove=8pt,
   skipbelow=8pt,
   linecolor=blue,
   innerbottommargin=10pt,
   % backgroundcolor=bg,font=\color{orange}\sffamily, fontcolor=white
]{definition}

\usepackage{empheq}
\usepackage[most]{tcolorbox}

\newtcbox{\mymath}[1][]{%
    nobeforeafter, math upper, tcbox raise base,
    enhanced, colframe=blue!30!black,
    colback=red!10, boxrule=1pt,
    #1}

\usepackage{unixode}


\DeclareMathOperator{\ord}{ord}
\DeclareMathOperator{\orb}{orb}
\DeclareMathOperator{\stab}{stab}
\DeclareMathOperator{\Stab}{stab}
\DeclareMathOperator{\ppcm}{ppcm}
\DeclareMathOperator{\conj}{Conj}
\DeclareMathOperator{\End}{End}
\DeclareMathOperator{\rot}{rot}
\DeclareMathOperator{\trs}{trace}
\DeclareMathOperator{\Ind}{Ind}
\DeclareMathOperator{\mat}{Mat}
\DeclareMathOperator{\id}{Id}
\DeclareMathOperator{\vect}{vect}
\DeclareMathOperator{\img}{img}
\DeclareMathOperator{\cov}{Cov}
\DeclareMathOperator{\dist}{dist}
\DeclareMathOperator{\irr}{Irr}
\DeclareMathOperator{\image}{Im}
\DeclareMathOperator{\pd}{\partial}
\DeclareMathOperator{\epi}{epi}
\DeclareMathOperator{\Argmin}{Argmin}
\DeclareMathOperator{\dom}{dom}
\DeclareMathOperator{\proj}{proj}
\DeclareMathOperator{\ctg}{ctg}
\DeclareMathOperator{\supp}{supp}
\DeclareMathOperator{\argmin}{argmin}
\DeclareMathOperator{\mult}{mult}
\DeclareMathOperator{\ch}{ch}
\DeclareMathOperator{\sh}{sh}
\DeclareMathOperator{\rang}{rang}
\DeclareMathOperator{\diam}{diam}
\DeclareMathOperator{\Epigraphe}{Epigraphe}




\usepackage{xcolor}
\everymath{\color{blue}}
%\everymath{\color[rgb]{0,1,1}}
%\pagecolor[rgb]{0,0,0.5}


\newcommand*{\pdtest}[3][]{\ensuremath{\frac{\partial^{#1} #2}{\partial #3}}}

\newcommand*{\deffunc}[6][]{\ensuremath{
\begin{array}{rcl}
#2 : #3 &\rightarrow& #4\\
#5 &\mapsto& #6
\end{array}
}}

\newcommand{\eqcolon}{\mathrel{\resizebox{\widthof{$\mathord{=}$}}{\height}{ $\!\!=\!\!\resizebox{1.2\width}{0.8\height}{\raisebox{0.23ex}{$\mathop{:}$}}\!\!$ }}}
\newcommand{\coloneq}{\mathrel{\resizebox{\widthof{$\mathord{=}$}}{\height}{ $\!\!\resizebox{1.2\width}{0.8\height}{\raisebox{0.23ex}{$\mathop{:}$}}\!\!=\!\!$ }}}
\newcommand{\eqcolonl}{\ensuremath{\mathrel{=\!\!\mathop{:}}}}
\newcommand{\coloneql}{\ensuremath{\mathrel{\mathop{:} \!\! =}}}
\newcommand{\vc}[1]{% inline column vector
  \left(\begin{smallmatrix}#1\end{smallmatrix}\right)%
}
\newcommand{\vr}[1]{% inline row vector
  \begin{smallmatrix}(\,#1\,)\end{smallmatrix}%
}
\makeatletter
\newcommand*{\defeq}{\ =\mathrel{\rlap{%
                     \raisebox{0.3ex}{$\m@th\cdot$}}%
                     \raisebox{-0.3ex}{$\m@th\cdot$}}%
                     }
\makeatother

\newcommand{\mathcircle}[1]{% inline row vector
 \overset{\circ}{#1}
}
\newcommand{\ulim}{% low limit
 \underline{\lim}
}
\newcommand{\ssi}{% iff
\iff
}
\newcommand{\ps}[2]{
\expval{#1 | #2}
}
\newcommand{\df}[1]{
\mqty{#1}
}
\newcommand{\n}[1]{
\norm{#1}
}
\newcommand{\sys}[1]{
\left\{\smqty{#1}\right.
}


\newcommand{\eqdef}{\ensuremath{\overset{\text{def}}=}}


\def\Circlearrowright{\ensuremath{%
  \rotatebox[origin=c]{230}{$\circlearrowright$}}}

\newcommand\ct[1]{\text{\rmfamily\upshape #1}}
\newcommand\question[1]{ {\color{red} ...!? \small #1}}
\newcommand\caz[1]{\left\{\begin{array} #1 \end{array}\right.}
\newcommand\const{\text{\rmfamily\upshape const}}
\newcommand\toP{ \overset{\pro}{\to}}
\newcommand\toPP{ \overset{\text{PP}}{\to}}
\newcommand{\oeq}{\mathrel{\text{\textcircled{$=$}}}}





\usepackage{xcolor}
% \usepackage[normalem]{ulem}
\usepackage{lipsum}
\makeatletter
% \newcommand\colorwave[1][blue]{\bgroup \markoverwith{\lower3.5\p@\hbox{\sixly \textcolor{#1}{\char58}}}\ULon}
%\font\sixly=lasy6 % does not re-load if already loaded, so no memory problem.

\newmdtheoremenv[
linewidth= 1pt,linecolor= blue,%
leftmargin=20,rightmargin=20,innertopmargin=0pt, innerrightmargin=40,%
tikzsetting = { draw=lightgray, line width = 0.3pt,dashed,%
dash pattern = on 15pt off 3pt},%
splittopskip=\topskip,skipbelow=\baselineskip,%
skipabove=\baselineskip,ntheorem,roundcorner=0pt,
% backgroundcolor=pagebg,font=\color{orange}\sffamily, fontcolor=white
]{examplebox}{Exemple}[section]



\newcommand\R{\mathbb{R}}
\newcommand\Z{\mathbb{Z}}
\newcommand\N{\mathbb{N}}
\newcommand\E{\mathbb{E}}
\newcommand\F{\mathcal{F}}
\newcommand\cH{\mathcal{H}}
\newcommand\V{\mathbb{V}}
\newcommand\dmo{ ^{-1} }
\newcommand\kapa{\kappa}
\newcommand\im{Im}
\newcommand\hs{\mathcal{H}}





\usepackage{soul}

\makeatletter
\newcommand*{\whiten}[1]{\llap{\textcolor{white}{{\the\SOUL@token}}\hspace{#1pt}}}
\DeclareRobustCommand*\myul{%
    \def\SOUL@everyspace{\underline{\space}\kern\z@}%
    \def\SOUL@everytoken{%
     \setbox0=\hbox{\the\SOUL@token}%
     \ifdim\dp0>\z@
        \raisebox{\dp0}{\underline{\phantom{\the\SOUL@token}}}%
        \whiten{1}\whiten{0}%
        \whiten{-1}\whiten{-2}%
        \llap{\the\SOUL@token}%
     \else
        \underline{\the\SOUL@token}%
     \fi}%
\SOUL@}
\makeatother

\newcommand*{\demp}{\fontfamily{lmtt}\selectfont}

\DeclareTextFontCommand{\textdemp}{\demp}

\begin{document}

\ifcomment
Multiline
comment
\fi
\ifcomment
\myul{Typesetting test}
% \color[rgb]{1,1,1}
$∑_i^n≠ 60º±∞π∆¬≈√j∫h≤≥µ$

$\CR \R\pro\ind\pro\gS\pro
\mqty[a&b\\c&d]$
$\pro\mathbb{P}$
$\dd{x}$

  \[
    \alpha(x)=\left\{
                \begin{array}{ll}
                  x\\
                  \frac{1}{1+e^{-kx}}\\
                  \frac{e^x-e^{-x}}{e^x+e^{-x}}
                \end{array}
              \right.
  \]

  $\expval{x}$
  
  $\chi_\rho(ghg\dmo)=\Tr(\rho_{ghg\dmo})=\Tr(\rho_g\circ\rho_h\circ\rho\dmo_g)=\Tr(\rho_h)\overset{\mbox{\scalebox{0.5}{$\Tr(AB)=\Tr(BA)$}}}{=}\chi_\rho(h)$
  	$\mathop{\oplus}_{\substack{x\in X}}$

$\mat(\rho_g)=(a_{ij}(g))_{\scriptsize \substack{1\leq i\leq d \\ 1\leq j\leq d}}$ et $\mat(\rho'_g)=(a'_{ij}(g))_{\scriptsize \substack{1\leq i'\leq d' \\ 1\leq j'\leq d'}}$



\[\int_a^b{\mathbb{R}^2}g(u, v)\dd{P_{XY}}(u, v)=\iint g(u,v) f_{XY}(u, v)\dd \lambda(u) \dd \lambda(v)\]
$$\lim_{x\to\infty} f(x)$$	
$$\iiiint_V \mu(t,u,v,w) \,dt\,du\,dv\,dw$$
$$\sum_{n=1}^{\infty} 2^{-n} = 1$$	
\begin{definition}
	Si $X$ et $Y$ sont 2 v.a. ou definit la \textsc{Covariance} entre $X$ et $Y$ comme
	$\cov(X,Y)\overset{\text{def}}{=}\E\left[(X-\E(X))(Y-\E(Y))\right]=\E(XY)-\E(X)\E(Y)$.
\end{definition}
\fi
\pagebreak

% \tableofcontents

% insert your code here
%\input{./algebra/main.tex}
%\input{./geometrie-differentielle/main.tex}
%\input{./probabilite/main.tex}
%\input{./analyse-fonctionnelle/main.tex}
% \input{./Analyse-convexe-et-dualite-en-optimisation/main.tex}
%\input{./tikz/main.tex}
%\input{./Theorie-du-distributions/main.tex}
%\input{./optimisation/mine.tex}
 \input{./modelisation/main.tex}

% yves.aubry@univ-tln.fr : algebra

\end{document}

%% !TEX encoding = UTF-8 Unicode
% !TEX TS-program = xelatex

\documentclass[french]{report}

%\usepackage[utf8]{inputenc}
%\usepackage[T1]{fontenc}
\usepackage{babel}


\newif\ifcomment
%\commenttrue # Show comments

\usepackage{physics}
\usepackage{amssymb}


\usepackage{amsthm}
% \usepackage{thmtools}
\usepackage{mathtools}
\usepackage{amsfonts}

\usepackage{color}

\usepackage{tikz}

\usepackage{geometry}
\geometry{a5paper, margin=0.1in, right=1cm}

\usepackage{dsfont}

\usepackage{graphicx}
\graphicspath{ {images/} }

\usepackage{faktor}

\usepackage{IEEEtrantools}
\usepackage{enumerate}   
\usepackage[PostScript=dvips]{"/Users/aware/Documents/Courses/diagrams"}


\newtheorem{theorem}{Théorème}[section]
\renewcommand{\thetheorem}{\arabic{theorem}}
\newtheorem{lemme}{Lemme}[section]
\renewcommand{\thelemme}{\arabic{lemme}}
\newtheorem{proposition}{Proposition}[section]
\renewcommand{\theproposition}{\arabic{proposition}}
\newtheorem{notations}{Notations}[section]
\newtheorem{problem}{Problème}[section]
\newtheorem{corollary}{Corollaire}[theorem]
\renewcommand{\thecorollary}{\arabic{corollary}}
\newtheorem{property}{Propriété}[section]
\newtheorem{objective}{Objectif}[section]

\theoremstyle{definition}
\newtheorem{definition}{Définition}[section]
\renewcommand{\thedefinition}{\arabic{definition}}
\newtheorem{exercise}{Exercice}[chapter]
\renewcommand{\theexercise}{\arabic{exercise}}
\newtheorem{example}{Exemple}[chapter]
\renewcommand{\theexample}{\arabic{example}}
\newtheorem*{solution}{Solution}
\newtheorem*{application}{Application}
\newtheorem*{notation}{Notation}
\newtheorem*{vocabulary}{Vocabulaire}
\newtheorem*{properties}{Propriétés}



\theoremstyle{remark}
\newtheorem*{remark}{Remarque}
\newtheorem*{rappel}{Rappel}


\usepackage{etoolbox}
\AtBeginEnvironment{exercise}{\small}
\AtBeginEnvironment{example}{\small}

\usepackage{cases}
\usepackage[red]{mypack}

\usepackage[framemethod=TikZ]{mdframed}

\definecolor{bg}{rgb}{0.4,0.25,0.95}
\definecolor{pagebg}{rgb}{0,0,0.5}
\surroundwithmdframed[
   topline=false,
   rightline=false,
   bottomline=false,
   leftmargin=\parindent,
   skipabove=8pt,
   skipbelow=8pt,
   linecolor=blue,
   innerbottommargin=10pt,
   % backgroundcolor=bg,font=\color{orange}\sffamily, fontcolor=white
]{definition}

\usepackage{empheq}
\usepackage[most]{tcolorbox}

\newtcbox{\mymath}[1][]{%
    nobeforeafter, math upper, tcbox raise base,
    enhanced, colframe=blue!30!black,
    colback=red!10, boxrule=1pt,
    #1}

\usepackage{unixode}


\DeclareMathOperator{\ord}{ord}
\DeclareMathOperator{\orb}{orb}
\DeclareMathOperator{\stab}{stab}
\DeclareMathOperator{\Stab}{stab}
\DeclareMathOperator{\ppcm}{ppcm}
\DeclareMathOperator{\conj}{Conj}
\DeclareMathOperator{\End}{End}
\DeclareMathOperator{\rot}{rot}
\DeclareMathOperator{\trs}{trace}
\DeclareMathOperator{\Ind}{Ind}
\DeclareMathOperator{\mat}{Mat}
\DeclareMathOperator{\id}{Id}
\DeclareMathOperator{\vect}{vect}
\DeclareMathOperator{\img}{img}
\DeclareMathOperator{\cov}{Cov}
\DeclareMathOperator{\dist}{dist}
\DeclareMathOperator{\irr}{Irr}
\DeclareMathOperator{\image}{Im}
\DeclareMathOperator{\pd}{\partial}
\DeclareMathOperator{\epi}{epi}
\DeclareMathOperator{\Argmin}{Argmin}
\DeclareMathOperator{\dom}{dom}
\DeclareMathOperator{\proj}{proj}
\DeclareMathOperator{\ctg}{ctg}
\DeclareMathOperator{\supp}{supp}
\DeclareMathOperator{\argmin}{argmin}
\DeclareMathOperator{\mult}{mult}
\DeclareMathOperator{\ch}{ch}
\DeclareMathOperator{\sh}{sh}
\DeclareMathOperator{\rang}{rang}
\DeclareMathOperator{\diam}{diam}
\DeclareMathOperator{\Epigraphe}{Epigraphe}




\usepackage{xcolor}
\everymath{\color{blue}}
%\everymath{\color[rgb]{0,1,1}}
%\pagecolor[rgb]{0,0,0.5}


\newcommand*{\pdtest}[3][]{\ensuremath{\frac{\partial^{#1} #2}{\partial #3}}}

\newcommand*{\deffunc}[6][]{\ensuremath{
\begin{array}{rcl}
#2 : #3 &\rightarrow& #4\\
#5 &\mapsto& #6
\end{array}
}}

\newcommand{\eqcolon}{\mathrel{\resizebox{\widthof{$\mathord{=}$}}{\height}{ $\!\!=\!\!\resizebox{1.2\width}{0.8\height}{\raisebox{0.23ex}{$\mathop{:}$}}\!\!$ }}}
\newcommand{\coloneq}{\mathrel{\resizebox{\widthof{$\mathord{=}$}}{\height}{ $\!\!\resizebox{1.2\width}{0.8\height}{\raisebox{0.23ex}{$\mathop{:}$}}\!\!=\!\!$ }}}
\newcommand{\eqcolonl}{\ensuremath{\mathrel{=\!\!\mathop{:}}}}
\newcommand{\coloneql}{\ensuremath{\mathrel{\mathop{:} \!\! =}}}
\newcommand{\vc}[1]{% inline column vector
  \left(\begin{smallmatrix}#1\end{smallmatrix}\right)%
}
\newcommand{\vr}[1]{% inline row vector
  \begin{smallmatrix}(\,#1\,)\end{smallmatrix}%
}
\makeatletter
\newcommand*{\defeq}{\ =\mathrel{\rlap{%
                     \raisebox{0.3ex}{$\m@th\cdot$}}%
                     \raisebox{-0.3ex}{$\m@th\cdot$}}%
                     }
\makeatother

\newcommand{\mathcircle}[1]{% inline row vector
 \overset{\circ}{#1}
}
\newcommand{\ulim}{% low limit
 \underline{\lim}
}
\newcommand{\ssi}{% iff
\iff
}
\newcommand{\ps}[2]{
\expval{#1 | #2}
}
\newcommand{\df}[1]{
\mqty{#1}
}
\newcommand{\n}[1]{
\norm{#1}
}
\newcommand{\sys}[1]{
\left\{\smqty{#1}\right.
}


\newcommand{\eqdef}{\ensuremath{\overset{\text{def}}=}}


\def\Circlearrowright{\ensuremath{%
  \rotatebox[origin=c]{230}{$\circlearrowright$}}}

\newcommand\ct[1]{\text{\rmfamily\upshape #1}}
\newcommand\question[1]{ {\color{red} ...!? \small #1}}
\newcommand\caz[1]{\left\{\begin{array} #1 \end{array}\right.}
\newcommand\const{\text{\rmfamily\upshape const}}
\newcommand\toP{ \overset{\pro}{\to}}
\newcommand\toPP{ \overset{\text{PP}}{\to}}
\newcommand{\oeq}{\mathrel{\text{\textcircled{$=$}}}}





\usepackage{xcolor}
% \usepackage[normalem]{ulem}
\usepackage{lipsum}
\makeatletter
% \newcommand\colorwave[1][blue]{\bgroup \markoverwith{\lower3.5\p@\hbox{\sixly \textcolor{#1}{\char58}}}\ULon}
%\font\sixly=lasy6 % does not re-load if already loaded, so no memory problem.

\newmdtheoremenv[
linewidth= 1pt,linecolor= blue,%
leftmargin=20,rightmargin=20,innertopmargin=0pt, innerrightmargin=40,%
tikzsetting = { draw=lightgray, line width = 0.3pt,dashed,%
dash pattern = on 15pt off 3pt},%
splittopskip=\topskip,skipbelow=\baselineskip,%
skipabove=\baselineskip,ntheorem,roundcorner=0pt,
% backgroundcolor=pagebg,font=\color{orange}\sffamily, fontcolor=white
]{examplebox}{Exemple}[section]



\newcommand\R{\mathbb{R}}
\newcommand\Z{\mathbb{Z}}
\newcommand\N{\mathbb{N}}
\newcommand\E{\mathbb{E}}
\newcommand\F{\mathcal{F}}
\newcommand\cH{\mathcal{H}}
\newcommand\V{\mathbb{V}}
\newcommand\dmo{ ^{-1} }
\newcommand\kapa{\kappa}
\newcommand\im{Im}
\newcommand\hs{\mathcal{H}}





\usepackage{soul}

\makeatletter
\newcommand*{\whiten}[1]{\llap{\textcolor{white}{{\the\SOUL@token}}\hspace{#1pt}}}
\DeclareRobustCommand*\myul{%
    \def\SOUL@everyspace{\underline{\space}\kern\z@}%
    \def\SOUL@everytoken{%
     \setbox0=\hbox{\the\SOUL@token}%
     \ifdim\dp0>\z@
        \raisebox{\dp0}{\underline{\phantom{\the\SOUL@token}}}%
        \whiten{1}\whiten{0}%
        \whiten{-1}\whiten{-2}%
        \llap{\the\SOUL@token}%
     \else
        \underline{\the\SOUL@token}%
     \fi}%
\SOUL@}
\makeatother

\newcommand*{\demp}{\fontfamily{lmtt}\selectfont}

\DeclareTextFontCommand{\textdemp}{\demp}

\begin{document}

\ifcomment
Multiline
comment
\fi
\ifcomment
\myul{Typesetting test}
% \color[rgb]{1,1,1}
$∑_i^n≠ 60º±∞π∆¬≈√j∫h≤≥µ$

$\CR \R\pro\ind\pro\gS\pro
\mqty[a&b\\c&d]$
$\pro\mathbb{P}$
$\dd{x}$

  \[
    \alpha(x)=\left\{
                \begin{array}{ll}
                  x\\
                  \frac{1}{1+e^{-kx}}\\
                  \frac{e^x-e^{-x}}{e^x+e^{-x}}
                \end{array}
              \right.
  \]

  $\expval{x}$
  
  $\chi_\rho(ghg\dmo)=\Tr(\rho_{ghg\dmo})=\Tr(\rho_g\circ\rho_h\circ\rho\dmo_g)=\Tr(\rho_h)\overset{\mbox{\scalebox{0.5}{$\Tr(AB)=\Tr(BA)$}}}{=}\chi_\rho(h)$
  	$\mathop{\oplus}_{\substack{x\in X}}$

$\mat(\rho_g)=(a_{ij}(g))_{\scriptsize \substack{1\leq i\leq d \\ 1\leq j\leq d}}$ et $\mat(\rho'_g)=(a'_{ij}(g))_{\scriptsize \substack{1\leq i'\leq d' \\ 1\leq j'\leq d'}}$



\[\int_a^b{\mathbb{R}^2}g(u, v)\dd{P_{XY}}(u, v)=\iint g(u,v) f_{XY}(u, v)\dd \lambda(u) \dd \lambda(v)\]
$$\lim_{x\to\infty} f(x)$$	
$$\iiiint_V \mu(t,u,v,w) \,dt\,du\,dv\,dw$$
$$\sum_{n=1}^{\infty} 2^{-n} = 1$$	
\begin{definition}
	Si $X$ et $Y$ sont 2 v.a. ou definit la \textsc{Covariance} entre $X$ et $Y$ comme
	$\cov(X,Y)\overset{\text{def}}{=}\E\left[(X-\E(X))(Y-\E(Y))\right]=\E(XY)-\E(X)\E(Y)$.
\end{definition}
\fi
\pagebreak

% \tableofcontents

% insert your code here
%\input{./algebra/main.tex}
%\input{./geometrie-differentielle/main.tex}
%\input{./probabilite/main.tex}
%\input{./analyse-fonctionnelle/main.tex}
% \input{./Analyse-convexe-et-dualite-en-optimisation/main.tex}
%\input{./tikz/main.tex}
%\input{./Theorie-du-distributions/main.tex}
%\input{./optimisation/mine.tex}
 \input{./modelisation/main.tex}

% yves.aubry@univ-tln.fr : algebra

\end{document}

%\input{./optimisation/mine.tex}
 % !TEX encoding = UTF-8 Unicode
% !TEX TS-program = xelatex

\documentclass[french]{report}

%\usepackage[utf8]{inputenc}
%\usepackage[T1]{fontenc}
\usepackage{babel}


\newif\ifcomment
%\commenttrue # Show comments

\usepackage{physics}
\usepackage{amssymb}


\usepackage{amsthm}
% \usepackage{thmtools}
\usepackage{mathtools}
\usepackage{amsfonts}

\usepackage{color}

\usepackage{tikz}

\usepackage{geometry}
\geometry{a5paper, margin=0.1in, right=1cm}

\usepackage{dsfont}

\usepackage{graphicx}
\graphicspath{ {images/} }

\usepackage{faktor}

\usepackage{IEEEtrantools}
\usepackage{enumerate}   
\usepackage[PostScript=dvips]{"/Users/aware/Documents/Courses/diagrams"}


\newtheorem{theorem}{Théorème}[section]
\renewcommand{\thetheorem}{\arabic{theorem}}
\newtheorem{lemme}{Lemme}[section]
\renewcommand{\thelemme}{\arabic{lemme}}
\newtheorem{proposition}{Proposition}[section]
\renewcommand{\theproposition}{\arabic{proposition}}
\newtheorem{notations}{Notations}[section]
\newtheorem{problem}{Problème}[section]
\newtheorem{corollary}{Corollaire}[theorem]
\renewcommand{\thecorollary}{\arabic{corollary}}
\newtheorem{property}{Propriété}[section]
\newtheorem{objective}{Objectif}[section]

\theoremstyle{definition}
\newtheorem{definition}{Définition}[section]
\renewcommand{\thedefinition}{\arabic{definition}}
\newtheorem{exercise}{Exercice}[chapter]
\renewcommand{\theexercise}{\arabic{exercise}}
\newtheorem{example}{Exemple}[chapter]
\renewcommand{\theexample}{\arabic{example}}
\newtheorem*{solution}{Solution}
\newtheorem*{application}{Application}
\newtheorem*{notation}{Notation}
\newtheorem*{vocabulary}{Vocabulaire}
\newtheorem*{properties}{Propriétés}



\theoremstyle{remark}
\newtheorem*{remark}{Remarque}
\newtheorem*{rappel}{Rappel}


\usepackage{etoolbox}
\AtBeginEnvironment{exercise}{\small}
\AtBeginEnvironment{example}{\small}

\usepackage{cases}
\usepackage[red]{mypack}

\usepackage[framemethod=TikZ]{mdframed}

\definecolor{bg}{rgb}{0.4,0.25,0.95}
\definecolor{pagebg}{rgb}{0,0,0.5}
\surroundwithmdframed[
   topline=false,
   rightline=false,
   bottomline=false,
   leftmargin=\parindent,
   skipabove=8pt,
   skipbelow=8pt,
   linecolor=blue,
   innerbottommargin=10pt,
   % backgroundcolor=bg,font=\color{orange}\sffamily, fontcolor=white
]{definition}

\usepackage{empheq}
\usepackage[most]{tcolorbox}

\newtcbox{\mymath}[1][]{%
    nobeforeafter, math upper, tcbox raise base,
    enhanced, colframe=blue!30!black,
    colback=red!10, boxrule=1pt,
    #1}

\usepackage{unixode}


\DeclareMathOperator{\ord}{ord}
\DeclareMathOperator{\orb}{orb}
\DeclareMathOperator{\stab}{stab}
\DeclareMathOperator{\Stab}{stab}
\DeclareMathOperator{\ppcm}{ppcm}
\DeclareMathOperator{\conj}{Conj}
\DeclareMathOperator{\End}{End}
\DeclareMathOperator{\rot}{rot}
\DeclareMathOperator{\trs}{trace}
\DeclareMathOperator{\Ind}{Ind}
\DeclareMathOperator{\mat}{Mat}
\DeclareMathOperator{\id}{Id}
\DeclareMathOperator{\vect}{vect}
\DeclareMathOperator{\img}{img}
\DeclareMathOperator{\cov}{Cov}
\DeclareMathOperator{\dist}{dist}
\DeclareMathOperator{\irr}{Irr}
\DeclareMathOperator{\image}{Im}
\DeclareMathOperator{\pd}{\partial}
\DeclareMathOperator{\epi}{epi}
\DeclareMathOperator{\Argmin}{Argmin}
\DeclareMathOperator{\dom}{dom}
\DeclareMathOperator{\proj}{proj}
\DeclareMathOperator{\ctg}{ctg}
\DeclareMathOperator{\supp}{supp}
\DeclareMathOperator{\argmin}{argmin}
\DeclareMathOperator{\mult}{mult}
\DeclareMathOperator{\ch}{ch}
\DeclareMathOperator{\sh}{sh}
\DeclareMathOperator{\rang}{rang}
\DeclareMathOperator{\diam}{diam}
\DeclareMathOperator{\Epigraphe}{Epigraphe}




\usepackage{xcolor}
\everymath{\color{blue}}
%\everymath{\color[rgb]{0,1,1}}
%\pagecolor[rgb]{0,0,0.5}


\newcommand*{\pdtest}[3][]{\ensuremath{\frac{\partial^{#1} #2}{\partial #3}}}

\newcommand*{\deffunc}[6][]{\ensuremath{
\begin{array}{rcl}
#2 : #3 &\rightarrow& #4\\
#5 &\mapsto& #6
\end{array}
}}

\newcommand{\eqcolon}{\mathrel{\resizebox{\widthof{$\mathord{=}$}}{\height}{ $\!\!=\!\!\resizebox{1.2\width}{0.8\height}{\raisebox{0.23ex}{$\mathop{:}$}}\!\!$ }}}
\newcommand{\coloneq}{\mathrel{\resizebox{\widthof{$\mathord{=}$}}{\height}{ $\!\!\resizebox{1.2\width}{0.8\height}{\raisebox{0.23ex}{$\mathop{:}$}}\!\!=\!\!$ }}}
\newcommand{\eqcolonl}{\ensuremath{\mathrel{=\!\!\mathop{:}}}}
\newcommand{\coloneql}{\ensuremath{\mathrel{\mathop{:} \!\! =}}}
\newcommand{\vc}[1]{% inline column vector
  \left(\begin{smallmatrix}#1\end{smallmatrix}\right)%
}
\newcommand{\vr}[1]{% inline row vector
  \begin{smallmatrix}(\,#1\,)\end{smallmatrix}%
}
\makeatletter
\newcommand*{\defeq}{\ =\mathrel{\rlap{%
                     \raisebox{0.3ex}{$\m@th\cdot$}}%
                     \raisebox{-0.3ex}{$\m@th\cdot$}}%
                     }
\makeatother

\newcommand{\mathcircle}[1]{% inline row vector
 \overset{\circ}{#1}
}
\newcommand{\ulim}{% low limit
 \underline{\lim}
}
\newcommand{\ssi}{% iff
\iff
}
\newcommand{\ps}[2]{
\expval{#1 | #2}
}
\newcommand{\df}[1]{
\mqty{#1}
}
\newcommand{\n}[1]{
\norm{#1}
}
\newcommand{\sys}[1]{
\left\{\smqty{#1}\right.
}


\newcommand{\eqdef}{\ensuremath{\overset{\text{def}}=}}


\def\Circlearrowright{\ensuremath{%
  \rotatebox[origin=c]{230}{$\circlearrowright$}}}

\newcommand\ct[1]{\text{\rmfamily\upshape #1}}
\newcommand\question[1]{ {\color{red} ...!? \small #1}}
\newcommand\caz[1]{\left\{\begin{array} #1 \end{array}\right.}
\newcommand\const{\text{\rmfamily\upshape const}}
\newcommand\toP{ \overset{\pro}{\to}}
\newcommand\toPP{ \overset{\text{PP}}{\to}}
\newcommand{\oeq}{\mathrel{\text{\textcircled{$=$}}}}





\usepackage{xcolor}
% \usepackage[normalem]{ulem}
\usepackage{lipsum}
\makeatletter
% \newcommand\colorwave[1][blue]{\bgroup \markoverwith{\lower3.5\p@\hbox{\sixly \textcolor{#1}{\char58}}}\ULon}
%\font\sixly=lasy6 % does not re-load if already loaded, so no memory problem.

\newmdtheoremenv[
linewidth= 1pt,linecolor= blue,%
leftmargin=20,rightmargin=20,innertopmargin=0pt, innerrightmargin=40,%
tikzsetting = { draw=lightgray, line width = 0.3pt,dashed,%
dash pattern = on 15pt off 3pt},%
splittopskip=\topskip,skipbelow=\baselineskip,%
skipabove=\baselineskip,ntheorem,roundcorner=0pt,
% backgroundcolor=pagebg,font=\color{orange}\sffamily, fontcolor=white
]{examplebox}{Exemple}[section]



\newcommand\R{\mathbb{R}}
\newcommand\Z{\mathbb{Z}}
\newcommand\N{\mathbb{N}}
\newcommand\E{\mathbb{E}}
\newcommand\F{\mathcal{F}}
\newcommand\cH{\mathcal{H}}
\newcommand\V{\mathbb{V}}
\newcommand\dmo{ ^{-1} }
\newcommand\kapa{\kappa}
\newcommand\im{Im}
\newcommand\hs{\mathcal{H}}





\usepackage{soul}

\makeatletter
\newcommand*{\whiten}[1]{\llap{\textcolor{white}{{\the\SOUL@token}}\hspace{#1pt}}}
\DeclareRobustCommand*\myul{%
    \def\SOUL@everyspace{\underline{\space}\kern\z@}%
    \def\SOUL@everytoken{%
     \setbox0=\hbox{\the\SOUL@token}%
     \ifdim\dp0>\z@
        \raisebox{\dp0}{\underline{\phantom{\the\SOUL@token}}}%
        \whiten{1}\whiten{0}%
        \whiten{-1}\whiten{-2}%
        \llap{\the\SOUL@token}%
     \else
        \underline{\the\SOUL@token}%
     \fi}%
\SOUL@}
\makeatother

\newcommand*{\demp}{\fontfamily{lmtt}\selectfont}

\DeclareTextFontCommand{\textdemp}{\demp}

\begin{document}

\ifcomment
Multiline
comment
\fi
\ifcomment
\myul{Typesetting test}
% \color[rgb]{1,1,1}
$∑_i^n≠ 60º±∞π∆¬≈√j∫h≤≥µ$

$\CR \R\pro\ind\pro\gS\pro
\mqty[a&b\\c&d]$
$\pro\mathbb{P}$
$\dd{x}$

  \[
    \alpha(x)=\left\{
                \begin{array}{ll}
                  x\\
                  \frac{1}{1+e^{-kx}}\\
                  \frac{e^x-e^{-x}}{e^x+e^{-x}}
                \end{array}
              \right.
  \]

  $\expval{x}$
  
  $\chi_\rho(ghg\dmo)=\Tr(\rho_{ghg\dmo})=\Tr(\rho_g\circ\rho_h\circ\rho\dmo_g)=\Tr(\rho_h)\overset{\mbox{\scalebox{0.5}{$\Tr(AB)=\Tr(BA)$}}}{=}\chi_\rho(h)$
  	$\mathop{\oplus}_{\substack{x\in X}}$

$\mat(\rho_g)=(a_{ij}(g))_{\scriptsize \substack{1\leq i\leq d \\ 1\leq j\leq d}}$ et $\mat(\rho'_g)=(a'_{ij}(g))_{\scriptsize \substack{1\leq i'\leq d' \\ 1\leq j'\leq d'}}$



\[\int_a^b{\mathbb{R}^2}g(u, v)\dd{P_{XY}}(u, v)=\iint g(u,v) f_{XY}(u, v)\dd \lambda(u) \dd \lambda(v)\]
$$\lim_{x\to\infty} f(x)$$	
$$\iiiint_V \mu(t,u,v,w) \,dt\,du\,dv\,dw$$
$$\sum_{n=1}^{\infty} 2^{-n} = 1$$	
\begin{definition}
	Si $X$ et $Y$ sont 2 v.a. ou definit la \textsc{Covariance} entre $X$ et $Y$ comme
	$\cov(X,Y)\overset{\text{def}}{=}\E\left[(X-\E(X))(Y-\E(Y))\right]=\E(XY)-\E(X)\E(Y)$.
\end{definition}
\fi
\pagebreak

% \tableofcontents

% insert your code here
%\input{./algebra/main.tex}
%\input{./geometrie-differentielle/main.tex}
%\input{./probabilite/main.tex}
%\input{./analyse-fonctionnelle/main.tex}
% \input{./Analyse-convexe-et-dualite-en-optimisation/main.tex}
%\input{./tikz/main.tex}
%\input{./Theorie-du-distributions/main.tex}
%\input{./optimisation/mine.tex}
 \input{./modelisation/main.tex}

% yves.aubry@univ-tln.fr : algebra

\end{document}


% yves.aubry@univ-tln.fr : algebra

\end{document}

%% !TEX encoding = UTF-8 Unicode
% !TEX TS-program = xelatex

\documentclass[french]{report}

%\usepackage[utf8]{inputenc}
%\usepackage[T1]{fontenc}
\usepackage{babel}


\newif\ifcomment
%\commenttrue # Show comments

\usepackage{physics}
\usepackage{amssymb}


\usepackage{amsthm}
% \usepackage{thmtools}
\usepackage{mathtools}
\usepackage{amsfonts}

\usepackage{color}

\usepackage{tikz}

\usepackage{geometry}
\geometry{a5paper, margin=0.1in, right=1cm}

\usepackage{dsfont}

\usepackage{graphicx}
\graphicspath{ {images/} }

\usepackage{faktor}

\usepackage{IEEEtrantools}
\usepackage{enumerate}   
\usepackage[PostScript=dvips]{"/Users/aware/Documents/Courses/diagrams"}


\newtheorem{theorem}{Théorème}[section]
\renewcommand{\thetheorem}{\arabic{theorem}}
\newtheorem{lemme}{Lemme}[section]
\renewcommand{\thelemme}{\arabic{lemme}}
\newtheorem{proposition}{Proposition}[section]
\renewcommand{\theproposition}{\arabic{proposition}}
\newtheorem{notations}{Notations}[section]
\newtheorem{problem}{Problème}[section]
\newtheorem{corollary}{Corollaire}[theorem]
\renewcommand{\thecorollary}{\arabic{corollary}}
\newtheorem{property}{Propriété}[section]
\newtheorem{objective}{Objectif}[section]

\theoremstyle{definition}
\newtheorem{definition}{Définition}[section]
\renewcommand{\thedefinition}{\arabic{definition}}
\newtheorem{exercise}{Exercice}[chapter]
\renewcommand{\theexercise}{\arabic{exercise}}
\newtheorem{example}{Exemple}[chapter]
\renewcommand{\theexample}{\arabic{example}}
\newtheorem*{solution}{Solution}
\newtheorem*{application}{Application}
\newtheorem*{notation}{Notation}
\newtheorem*{vocabulary}{Vocabulaire}
\newtheorem*{properties}{Propriétés}



\theoremstyle{remark}
\newtheorem*{remark}{Remarque}
\newtheorem*{rappel}{Rappel}


\usepackage{etoolbox}
\AtBeginEnvironment{exercise}{\small}
\AtBeginEnvironment{example}{\small}

\usepackage{cases}
\usepackage[red]{mypack}

\usepackage[framemethod=TikZ]{mdframed}

\definecolor{bg}{rgb}{0.4,0.25,0.95}
\definecolor{pagebg}{rgb}{0,0,0.5}
\surroundwithmdframed[
   topline=false,
   rightline=false,
   bottomline=false,
   leftmargin=\parindent,
   skipabove=8pt,
   skipbelow=8pt,
   linecolor=blue,
   innerbottommargin=10pt,
   % backgroundcolor=bg,font=\color{orange}\sffamily, fontcolor=white
]{definition}

\usepackage{empheq}
\usepackage[most]{tcolorbox}

\newtcbox{\mymath}[1][]{%
    nobeforeafter, math upper, tcbox raise base,
    enhanced, colframe=blue!30!black,
    colback=red!10, boxrule=1pt,
    #1}

\usepackage{unixode}


\DeclareMathOperator{\ord}{ord}
\DeclareMathOperator{\orb}{orb}
\DeclareMathOperator{\stab}{stab}
\DeclareMathOperator{\Stab}{stab}
\DeclareMathOperator{\ppcm}{ppcm}
\DeclareMathOperator{\conj}{Conj}
\DeclareMathOperator{\End}{End}
\DeclareMathOperator{\rot}{rot}
\DeclareMathOperator{\trs}{trace}
\DeclareMathOperator{\Ind}{Ind}
\DeclareMathOperator{\mat}{Mat}
\DeclareMathOperator{\id}{Id}
\DeclareMathOperator{\vect}{vect}
\DeclareMathOperator{\img}{img}
\DeclareMathOperator{\cov}{Cov}
\DeclareMathOperator{\dist}{dist}
\DeclareMathOperator{\irr}{Irr}
\DeclareMathOperator{\image}{Im}
\DeclareMathOperator{\pd}{\partial}
\DeclareMathOperator{\epi}{epi}
\DeclareMathOperator{\Argmin}{Argmin}
\DeclareMathOperator{\dom}{dom}
\DeclareMathOperator{\proj}{proj}
\DeclareMathOperator{\ctg}{ctg}
\DeclareMathOperator{\supp}{supp}
\DeclareMathOperator{\argmin}{argmin}
\DeclareMathOperator{\mult}{mult}
\DeclareMathOperator{\ch}{ch}
\DeclareMathOperator{\sh}{sh}
\DeclareMathOperator{\rang}{rang}
\DeclareMathOperator{\diam}{diam}
\DeclareMathOperator{\Epigraphe}{Epigraphe}




\usepackage{xcolor}
\everymath{\color{blue}}
%\everymath{\color[rgb]{0,1,1}}
%\pagecolor[rgb]{0,0,0.5}


\newcommand*{\pdtest}[3][]{\ensuremath{\frac{\partial^{#1} #2}{\partial #3}}}

\newcommand*{\deffunc}[6][]{\ensuremath{
\begin{array}{rcl}
#2 : #3 &\rightarrow& #4\\
#5 &\mapsto& #6
\end{array}
}}

\newcommand{\eqcolon}{\mathrel{\resizebox{\widthof{$\mathord{=}$}}{\height}{ $\!\!=\!\!\resizebox{1.2\width}{0.8\height}{\raisebox{0.23ex}{$\mathop{:}$}}\!\!$ }}}
\newcommand{\coloneq}{\mathrel{\resizebox{\widthof{$\mathord{=}$}}{\height}{ $\!\!\resizebox{1.2\width}{0.8\height}{\raisebox{0.23ex}{$\mathop{:}$}}\!\!=\!\!$ }}}
\newcommand{\eqcolonl}{\ensuremath{\mathrel{=\!\!\mathop{:}}}}
\newcommand{\coloneql}{\ensuremath{\mathrel{\mathop{:} \!\! =}}}
\newcommand{\vc}[1]{% inline column vector
  \left(\begin{smallmatrix}#1\end{smallmatrix}\right)%
}
\newcommand{\vr}[1]{% inline row vector
  \begin{smallmatrix}(\,#1\,)\end{smallmatrix}%
}
\makeatletter
\newcommand*{\defeq}{\ =\mathrel{\rlap{%
                     \raisebox{0.3ex}{$\m@th\cdot$}}%
                     \raisebox{-0.3ex}{$\m@th\cdot$}}%
                     }
\makeatother

\newcommand{\mathcircle}[1]{% inline row vector
 \overset{\circ}{#1}
}
\newcommand{\ulim}{% low limit
 \underline{\lim}
}
\newcommand{\ssi}{% iff
\iff
}
\newcommand{\ps}[2]{
\expval{#1 | #2}
}
\newcommand{\df}[1]{
\mqty{#1}
}
\newcommand{\n}[1]{
\norm{#1}
}
\newcommand{\sys}[1]{
\left\{\smqty{#1}\right.
}


\newcommand{\eqdef}{\ensuremath{\overset{\text{def}}=}}


\def\Circlearrowright{\ensuremath{%
  \rotatebox[origin=c]{230}{$\circlearrowright$}}}

\newcommand\ct[1]{\text{\rmfamily\upshape #1}}
\newcommand\question[1]{ {\color{red} ...!? \small #1}}
\newcommand\caz[1]{\left\{\begin{array} #1 \end{array}\right.}
\newcommand\const{\text{\rmfamily\upshape const}}
\newcommand\toP{ \overset{\pro}{\to}}
\newcommand\toPP{ \overset{\text{PP}}{\to}}
\newcommand{\oeq}{\mathrel{\text{\textcircled{$=$}}}}





\usepackage{xcolor}
% \usepackage[normalem]{ulem}
\usepackage{lipsum}
\makeatletter
% \newcommand\colorwave[1][blue]{\bgroup \markoverwith{\lower3.5\p@\hbox{\sixly \textcolor{#1}{\char58}}}\ULon}
%\font\sixly=lasy6 % does not re-load if already loaded, so no memory problem.

\newmdtheoremenv[
linewidth= 1pt,linecolor= blue,%
leftmargin=20,rightmargin=20,innertopmargin=0pt, innerrightmargin=40,%
tikzsetting = { draw=lightgray, line width = 0.3pt,dashed,%
dash pattern = on 15pt off 3pt},%
splittopskip=\topskip,skipbelow=\baselineskip,%
skipabove=\baselineskip,ntheorem,roundcorner=0pt,
% backgroundcolor=pagebg,font=\color{orange}\sffamily, fontcolor=white
]{examplebox}{Exemple}[section]



\newcommand\R{\mathbb{R}}
\newcommand\Z{\mathbb{Z}}
\newcommand\N{\mathbb{N}}
\newcommand\E{\mathbb{E}}
\newcommand\F{\mathcal{F}}
\newcommand\cH{\mathcal{H}}
\newcommand\V{\mathbb{V}}
\newcommand\dmo{ ^{-1} }
\newcommand\kapa{\kappa}
\newcommand\im{Im}
\newcommand\hs{\mathcal{H}}





\usepackage{soul}

\makeatletter
\newcommand*{\whiten}[1]{\llap{\textcolor{white}{{\the\SOUL@token}}\hspace{#1pt}}}
\DeclareRobustCommand*\myul{%
    \def\SOUL@everyspace{\underline{\space}\kern\z@}%
    \def\SOUL@everytoken{%
     \setbox0=\hbox{\the\SOUL@token}%
     \ifdim\dp0>\z@
        \raisebox{\dp0}{\underline{\phantom{\the\SOUL@token}}}%
        \whiten{1}\whiten{0}%
        \whiten{-1}\whiten{-2}%
        \llap{\the\SOUL@token}%
     \else
        \underline{\the\SOUL@token}%
     \fi}%
\SOUL@}
\makeatother

\newcommand*{\demp}{\fontfamily{lmtt}\selectfont}

\DeclareTextFontCommand{\textdemp}{\demp}

\begin{document}

\ifcomment
Multiline
comment
\fi
\ifcomment
\myul{Typesetting test}
% \color[rgb]{1,1,1}
$∑_i^n≠ 60º±∞π∆¬≈√j∫h≤≥µ$

$\CR \R\pro\ind\pro\gS\pro
\mqty[a&b\\c&d]$
$\pro\mathbb{P}$
$\dd{x}$

  \[
    \alpha(x)=\left\{
                \begin{array}{ll}
                  x\\
                  \frac{1}{1+e^{-kx}}\\
                  \frac{e^x-e^{-x}}{e^x+e^{-x}}
                \end{array}
              \right.
  \]

  $\expval{x}$
  
  $\chi_\rho(ghg\dmo)=\Tr(\rho_{ghg\dmo})=\Tr(\rho_g\circ\rho_h\circ\rho\dmo_g)=\Tr(\rho_h)\overset{\mbox{\scalebox{0.5}{$\Tr(AB)=\Tr(BA)$}}}{=}\chi_\rho(h)$
  	$\mathop{\oplus}_{\substack{x\in X}}$

$\mat(\rho_g)=(a_{ij}(g))_{\scriptsize \substack{1\leq i\leq d \\ 1\leq j\leq d}}$ et $\mat(\rho'_g)=(a'_{ij}(g))_{\scriptsize \substack{1\leq i'\leq d' \\ 1\leq j'\leq d'}}$



\[\int_a^b{\mathbb{R}^2}g(u, v)\dd{P_{XY}}(u, v)=\iint g(u,v) f_{XY}(u, v)\dd \lambda(u) \dd \lambda(v)\]
$$\lim_{x\to\infty} f(x)$$	
$$\iiiint_V \mu(t,u,v,w) \,dt\,du\,dv\,dw$$
$$\sum_{n=1}^{\infty} 2^{-n} = 1$$	
\begin{definition}
	Si $X$ et $Y$ sont 2 v.a. ou definit la \textsc{Covariance} entre $X$ et $Y$ comme
	$\cov(X,Y)\overset{\text{def}}{=}\E\left[(X-\E(X))(Y-\E(Y))\right]=\E(XY)-\E(X)\E(Y)$.
\end{definition}
\fi
\pagebreak

% \tableofcontents

% insert your code here
%% !TEX encoding = UTF-8 Unicode
% !TEX TS-program = xelatex

\documentclass[french]{report}

%\usepackage[utf8]{inputenc}
%\usepackage[T1]{fontenc}
\usepackage{babel}


\newif\ifcomment
%\commenttrue # Show comments

\usepackage{physics}
\usepackage{amssymb}


\usepackage{amsthm}
% \usepackage{thmtools}
\usepackage{mathtools}
\usepackage{amsfonts}

\usepackage{color}

\usepackage{tikz}

\usepackage{geometry}
\geometry{a5paper, margin=0.1in, right=1cm}

\usepackage{dsfont}

\usepackage{graphicx}
\graphicspath{ {images/} }

\usepackage{faktor}

\usepackage{IEEEtrantools}
\usepackage{enumerate}   
\usepackage[PostScript=dvips]{"/Users/aware/Documents/Courses/diagrams"}


\newtheorem{theorem}{Théorème}[section]
\renewcommand{\thetheorem}{\arabic{theorem}}
\newtheorem{lemme}{Lemme}[section]
\renewcommand{\thelemme}{\arabic{lemme}}
\newtheorem{proposition}{Proposition}[section]
\renewcommand{\theproposition}{\arabic{proposition}}
\newtheorem{notations}{Notations}[section]
\newtheorem{problem}{Problème}[section]
\newtheorem{corollary}{Corollaire}[theorem]
\renewcommand{\thecorollary}{\arabic{corollary}}
\newtheorem{property}{Propriété}[section]
\newtheorem{objective}{Objectif}[section]

\theoremstyle{definition}
\newtheorem{definition}{Définition}[section]
\renewcommand{\thedefinition}{\arabic{definition}}
\newtheorem{exercise}{Exercice}[chapter]
\renewcommand{\theexercise}{\arabic{exercise}}
\newtheorem{example}{Exemple}[chapter]
\renewcommand{\theexample}{\arabic{example}}
\newtheorem*{solution}{Solution}
\newtheorem*{application}{Application}
\newtheorem*{notation}{Notation}
\newtheorem*{vocabulary}{Vocabulaire}
\newtheorem*{properties}{Propriétés}



\theoremstyle{remark}
\newtheorem*{remark}{Remarque}
\newtheorem*{rappel}{Rappel}


\usepackage{etoolbox}
\AtBeginEnvironment{exercise}{\small}
\AtBeginEnvironment{example}{\small}

\usepackage{cases}
\usepackage[red]{mypack}

\usepackage[framemethod=TikZ]{mdframed}

\definecolor{bg}{rgb}{0.4,0.25,0.95}
\definecolor{pagebg}{rgb}{0,0,0.5}
\surroundwithmdframed[
   topline=false,
   rightline=false,
   bottomline=false,
   leftmargin=\parindent,
   skipabove=8pt,
   skipbelow=8pt,
   linecolor=blue,
   innerbottommargin=10pt,
   % backgroundcolor=bg,font=\color{orange}\sffamily, fontcolor=white
]{definition}

\usepackage{empheq}
\usepackage[most]{tcolorbox}

\newtcbox{\mymath}[1][]{%
    nobeforeafter, math upper, tcbox raise base,
    enhanced, colframe=blue!30!black,
    colback=red!10, boxrule=1pt,
    #1}

\usepackage{unixode}


\DeclareMathOperator{\ord}{ord}
\DeclareMathOperator{\orb}{orb}
\DeclareMathOperator{\stab}{stab}
\DeclareMathOperator{\Stab}{stab}
\DeclareMathOperator{\ppcm}{ppcm}
\DeclareMathOperator{\conj}{Conj}
\DeclareMathOperator{\End}{End}
\DeclareMathOperator{\rot}{rot}
\DeclareMathOperator{\trs}{trace}
\DeclareMathOperator{\Ind}{Ind}
\DeclareMathOperator{\mat}{Mat}
\DeclareMathOperator{\id}{Id}
\DeclareMathOperator{\vect}{vect}
\DeclareMathOperator{\img}{img}
\DeclareMathOperator{\cov}{Cov}
\DeclareMathOperator{\dist}{dist}
\DeclareMathOperator{\irr}{Irr}
\DeclareMathOperator{\image}{Im}
\DeclareMathOperator{\pd}{\partial}
\DeclareMathOperator{\epi}{epi}
\DeclareMathOperator{\Argmin}{Argmin}
\DeclareMathOperator{\dom}{dom}
\DeclareMathOperator{\proj}{proj}
\DeclareMathOperator{\ctg}{ctg}
\DeclareMathOperator{\supp}{supp}
\DeclareMathOperator{\argmin}{argmin}
\DeclareMathOperator{\mult}{mult}
\DeclareMathOperator{\ch}{ch}
\DeclareMathOperator{\sh}{sh}
\DeclareMathOperator{\rang}{rang}
\DeclareMathOperator{\diam}{diam}
\DeclareMathOperator{\Epigraphe}{Epigraphe}




\usepackage{xcolor}
\everymath{\color{blue}}
%\everymath{\color[rgb]{0,1,1}}
%\pagecolor[rgb]{0,0,0.5}


\newcommand*{\pdtest}[3][]{\ensuremath{\frac{\partial^{#1} #2}{\partial #3}}}

\newcommand*{\deffunc}[6][]{\ensuremath{
\begin{array}{rcl}
#2 : #3 &\rightarrow& #4\\
#5 &\mapsto& #6
\end{array}
}}

\newcommand{\eqcolon}{\mathrel{\resizebox{\widthof{$\mathord{=}$}}{\height}{ $\!\!=\!\!\resizebox{1.2\width}{0.8\height}{\raisebox{0.23ex}{$\mathop{:}$}}\!\!$ }}}
\newcommand{\coloneq}{\mathrel{\resizebox{\widthof{$\mathord{=}$}}{\height}{ $\!\!\resizebox{1.2\width}{0.8\height}{\raisebox{0.23ex}{$\mathop{:}$}}\!\!=\!\!$ }}}
\newcommand{\eqcolonl}{\ensuremath{\mathrel{=\!\!\mathop{:}}}}
\newcommand{\coloneql}{\ensuremath{\mathrel{\mathop{:} \!\! =}}}
\newcommand{\vc}[1]{% inline column vector
  \left(\begin{smallmatrix}#1\end{smallmatrix}\right)%
}
\newcommand{\vr}[1]{% inline row vector
  \begin{smallmatrix}(\,#1\,)\end{smallmatrix}%
}
\makeatletter
\newcommand*{\defeq}{\ =\mathrel{\rlap{%
                     \raisebox{0.3ex}{$\m@th\cdot$}}%
                     \raisebox{-0.3ex}{$\m@th\cdot$}}%
                     }
\makeatother

\newcommand{\mathcircle}[1]{% inline row vector
 \overset{\circ}{#1}
}
\newcommand{\ulim}{% low limit
 \underline{\lim}
}
\newcommand{\ssi}{% iff
\iff
}
\newcommand{\ps}[2]{
\expval{#1 | #2}
}
\newcommand{\df}[1]{
\mqty{#1}
}
\newcommand{\n}[1]{
\norm{#1}
}
\newcommand{\sys}[1]{
\left\{\smqty{#1}\right.
}


\newcommand{\eqdef}{\ensuremath{\overset{\text{def}}=}}


\def\Circlearrowright{\ensuremath{%
  \rotatebox[origin=c]{230}{$\circlearrowright$}}}

\newcommand\ct[1]{\text{\rmfamily\upshape #1}}
\newcommand\question[1]{ {\color{red} ...!? \small #1}}
\newcommand\caz[1]{\left\{\begin{array} #1 \end{array}\right.}
\newcommand\const{\text{\rmfamily\upshape const}}
\newcommand\toP{ \overset{\pro}{\to}}
\newcommand\toPP{ \overset{\text{PP}}{\to}}
\newcommand{\oeq}{\mathrel{\text{\textcircled{$=$}}}}





\usepackage{xcolor}
% \usepackage[normalem]{ulem}
\usepackage{lipsum}
\makeatletter
% \newcommand\colorwave[1][blue]{\bgroup \markoverwith{\lower3.5\p@\hbox{\sixly \textcolor{#1}{\char58}}}\ULon}
%\font\sixly=lasy6 % does not re-load if already loaded, so no memory problem.

\newmdtheoremenv[
linewidth= 1pt,linecolor= blue,%
leftmargin=20,rightmargin=20,innertopmargin=0pt, innerrightmargin=40,%
tikzsetting = { draw=lightgray, line width = 0.3pt,dashed,%
dash pattern = on 15pt off 3pt},%
splittopskip=\topskip,skipbelow=\baselineskip,%
skipabove=\baselineskip,ntheorem,roundcorner=0pt,
% backgroundcolor=pagebg,font=\color{orange}\sffamily, fontcolor=white
]{examplebox}{Exemple}[section]



\newcommand\R{\mathbb{R}}
\newcommand\Z{\mathbb{Z}}
\newcommand\N{\mathbb{N}}
\newcommand\E{\mathbb{E}}
\newcommand\F{\mathcal{F}}
\newcommand\cH{\mathcal{H}}
\newcommand\V{\mathbb{V}}
\newcommand\dmo{ ^{-1} }
\newcommand\kapa{\kappa}
\newcommand\im{Im}
\newcommand\hs{\mathcal{H}}





\usepackage{soul}

\makeatletter
\newcommand*{\whiten}[1]{\llap{\textcolor{white}{{\the\SOUL@token}}\hspace{#1pt}}}
\DeclareRobustCommand*\myul{%
    \def\SOUL@everyspace{\underline{\space}\kern\z@}%
    \def\SOUL@everytoken{%
     \setbox0=\hbox{\the\SOUL@token}%
     \ifdim\dp0>\z@
        \raisebox{\dp0}{\underline{\phantom{\the\SOUL@token}}}%
        \whiten{1}\whiten{0}%
        \whiten{-1}\whiten{-2}%
        \llap{\the\SOUL@token}%
     \else
        \underline{\the\SOUL@token}%
     \fi}%
\SOUL@}
\makeatother

\newcommand*{\demp}{\fontfamily{lmtt}\selectfont}

\DeclareTextFontCommand{\textdemp}{\demp}

\begin{document}

\ifcomment
Multiline
comment
\fi
\ifcomment
\myul{Typesetting test}
% \color[rgb]{1,1,1}
$∑_i^n≠ 60º±∞π∆¬≈√j∫h≤≥µ$

$\CR \R\pro\ind\pro\gS\pro
\mqty[a&b\\c&d]$
$\pro\mathbb{P}$
$\dd{x}$

  \[
    \alpha(x)=\left\{
                \begin{array}{ll}
                  x\\
                  \frac{1}{1+e^{-kx}}\\
                  \frac{e^x-e^{-x}}{e^x+e^{-x}}
                \end{array}
              \right.
  \]

  $\expval{x}$
  
  $\chi_\rho(ghg\dmo)=\Tr(\rho_{ghg\dmo})=\Tr(\rho_g\circ\rho_h\circ\rho\dmo_g)=\Tr(\rho_h)\overset{\mbox{\scalebox{0.5}{$\Tr(AB)=\Tr(BA)$}}}{=}\chi_\rho(h)$
  	$\mathop{\oplus}_{\substack{x\in X}}$

$\mat(\rho_g)=(a_{ij}(g))_{\scriptsize \substack{1\leq i\leq d \\ 1\leq j\leq d}}$ et $\mat(\rho'_g)=(a'_{ij}(g))_{\scriptsize \substack{1\leq i'\leq d' \\ 1\leq j'\leq d'}}$



\[\int_a^b{\mathbb{R}^2}g(u, v)\dd{P_{XY}}(u, v)=\iint g(u,v) f_{XY}(u, v)\dd \lambda(u) \dd \lambda(v)\]
$$\lim_{x\to\infty} f(x)$$	
$$\iiiint_V \mu(t,u,v,w) \,dt\,du\,dv\,dw$$
$$\sum_{n=1}^{\infty} 2^{-n} = 1$$	
\begin{definition}
	Si $X$ et $Y$ sont 2 v.a. ou definit la \textsc{Covariance} entre $X$ et $Y$ comme
	$\cov(X,Y)\overset{\text{def}}{=}\E\left[(X-\E(X))(Y-\E(Y))\right]=\E(XY)-\E(X)\E(Y)$.
\end{definition}
\fi
\pagebreak

% \tableofcontents

% insert your code here
%\input{./algebra/main.tex}
%\input{./geometrie-differentielle/main.tex}
%\input{./probabilite/main.tex}
%\input{./analyse-fonctionnelle/main.tex}
% \input{./Analyse-convexe-et-dualite-en-optimisation/main.tex}
%\input{./tikz/main.tex}
%\input{./Theorie-du-distributions/main.tex}
%\input{./optimisation/mine.tex}
 \input{./modelisation/main.tex}

% yves.aubry@univ-tln.fr : algebra

\end{document}

%% !TEX encoding = UTF-8 Unicode
% !TEX TS-program = xelatex

\documentclass[french]{report}

%\usepackage[utf8]{inputenc}
%\usepackage[T1]{fontenc}
\usepackage{babel}


\newif\ifcomment
%\commenttrue # Show comments

\usepackage{physics}
\usepackage{amssymb}


\usepackage{amsthm}
% \usepackage{thmtools}
\usepackage{mathtools}
\usepackage{amsfonts}

\usepackage{color}

\usepackage{tikz}

\usepackage{geometry}
\geometry{a5paper, margin=0.1in, right=1cm}

\usepackage{dsfont}

\usepackage{graphicx}
\graphicspath{ {images/} }

\usepackage{faktor}

\usepackage{IEEEtrantools}
\usepackage{enumerate}   
\usepackage[PostScript=dvips]{"/Users/aware/Documents/Courses/diagrams"}


\newtheorem{theorem}{Théorème}[section]
\renewcommand{\thetheorem}{\arabic{theorem}}
\newtheorem{lemme}{Lemme}[section]
\renewcommand{\thelemme}{\arabic{lemme}}
\newtheorem{proposition}{Proposition}[section]
\renewcommand{\theproposition}{\arabic{proposition}}
\newtheorem{notations}{Notations}[section]
\newtheorem{problem}{Problème}[section]
\newtheorem{corollary}{Corollaire}[theorem]
\renewcommand{\thecorollary}{\arabic{corollary}}
\newtheorem{property}{Propriété}[section]
\newtheorem{objective}{Objectif}[section]

\theoremstyle{definition}
\newtheorem{definition}{Définition}[section]
\renewcommand{\thedefinition}{\arabic{definition}}
\newtheorem{exercise}{Exercice}[chapter]
\renewcommand{\theexercise}{\arabic{exercise}}
\newtheorem{example}{Exemple}[chapter]
\renewcommand{\theexample}{\arabic{example}}
\newtheorem*{solution}{Solution}
\newtheorem*{application}{Application}
\newtheorem*{notation}{Notation}
\newtheorem*{vocabulary}{Vocabulaire}
\newtheorem*{properties}{Propriétés}



\theoremstyle{remark}
\newtheorem*{remark}{Remarque}
\newtheorem*{rappel}{Rappel}


\usepackage{etoolbox}
\AtBeginEnvironment{exercise}{\small}
\AtBeginEnvironment{example}{\small}

\usepackage{cases}
\usepackage[red]{mypack}

\usepackage[framemethod=TikZ]{mdframed}

\definecolor{bg}{rgb}{0.4,0.25,0.95}
\definecolor{pagebg}{rgb}{0,0,0.5}
\surroundwithmdframed[
   topline=false,
   rightline=false,
   bottomline=false,
   leftmargin=\parindent,
   skipabove=8pt,
   skipbelow=8pt,
   linecolor=blue,
   innerbottommargin=10pt,
   % backgroundcolor=bg,font=\color{orange}\sffamily, fontcolor=white
]{definition}

\usepackage{empheq}
\usepackage[most]{tcolorbox}

\newtcbox{\mymath}[1][]{%
    nobeforeafter, math upper, tcbox raise base,
    enhanced, colframe=blue!30!black,
    colback=red!10, boxrule=1pt,
    #1}

\usepackage{unixode}


\DeclareMathOperator{\ord}{ord}
\DeclareMathOperator{\orb}{orb}
\DeclareMathOperator{\stab}{stab}
\DeclareMathOperator{\Stab}{stab}
\DeclareMathOperator{\ppcm}{ppcm}
\DeclareMathOperator{\conj}{Conj}
\DeclareMathOperator{\End}{End}
\DeclareMathOperator{\rot}{rot}
\DeclareMathOperator{\trs}{trace}
\DeclareMathOperator{\Ind}{Ind}
\DeclareMathOperator{\mat}{Mat}
\DeclareMathOperator{\id}{Id}
\DeclareMathOperator{\vect}{vect}
\DeclareMathOperator{\img}{img}
\DeclareMathOperator{\cov}{Cov}
\DeclareMathOperator{\dist}{dist}
\DeclareMathOperator{\irr}{Irr}
\DeclareMathOperator{\image}{Im}
\DeclareMathOperator{\pd}{\partial}
\DeclareMathOperator{\epi}{epi}
\DeclareMathOperator{\Argmin}{Argmin}
\DeclareMathOperator{\dom}{dom}
\DeclareMathOperator{\proj}{proj}
\DeclareMathOperator{\ctg}{ctg}
\DeclareMathOperator{\supp}{supp}
\DeclareMathOperator{\argmin}{argmin}
\DeclareMathOperator{\mult}{mult}
\DeclareMathOperator{\ch}{ch}
\DeclareMathOperator{\sh}{sh}
\DeclareMathOperator{\rang}{rang}
\DeclareMathOperator{\diam}{diam}
\DeclareMathOperator{\Epigraphe}{Epigraphe}




\usepackage{xcolor}
\everymath{\color{blue}}
%\everymath{\color[rgb]{0,1,1}}
%\pagecolor[rgb]{0,0,0.5}


\newcommand*{\pdtest}[3][]{\ensuremath{\frac{\partial^{#1} #2}{\partial #3}}}

\newcommand*{\deffunc}[6][]{\ensuremath{
\begin{array}{rcl}
#2 : #3 &\rightarrow& #4\\
#5 &\mapsto& #6
\end{array}
}}

\newcommand{\eqcolon}{\mathrel{\resizebox{\widthof{$\mathord{=}$}}{\height}{ $\!\!=\!\!\resizebox{1.2\width}{0.8\height}{\raisebox{0.23ex}{$\mathop{:}$}}\!\!$ }}}
\newcommand{\coloneq}{\mathrel{\resizebox{\widthof{$\mathord{=}$}}{\height}{ $\!\!\resizebox{1.2\width}{0.8\height}{\raisebox{0.23ex}{$\mathop{:}$}}\!\!=\!\!$ }}}
\newcommand{\eqcolonl}{\ensuremath{\mathrel{=\!\!\mathop{:}}}}
\newcommand{\coloneql}{\ensuremath{\mathrel{\mathop{:} \!\! =}}}
\newcommand{\vc}[1]{% inline column vector
  \left(\begin{smallmatrix}#1\end{smallmatrix}\right)%
}
\newcommand{\vr}[1]{% inline row vector
  \begin{smallmatrix}(\,#1\,)\end{smallmatrix}%
}
\makeatletter
\newcommand*{\defeq}{\ =\mathrel{\rlap{%
                     \raisebox{0.3ex}{$\m@th\cdot$}}%
                     \raisebox{-0.3ex}{$\m@th\cdot$}}%
                     }
\makeatother

\newcommand{\mathcircle}[1]{% inline row vector
 \overset{\circ}{#1}
}
\newcommand{\ulim}{% low limit
 \underline{\lim}
}
\newcommand{\ssi}{% iff
\iff
}
\newcommand{\ps}[2]{
\expval{#1 | #2}
}
\newcommand{\df}[1]{
\mqty{#1}
}
\newcommand{\n}[1]{
\norm{#1}
}
\newcommand{\sys}[1]{
\left\{\smqty{#1}\right.
}


\newcommand{\eqdef}{\ensuremath{\overset{\text{def}}=}}


\def\Circlearrowright{\ensuremath{%
  \rotatebox[origin=c]{230}{$\circlearrowright$}}}

\newcommand\ct[1]{\text{\rmfamily\upshape #1}}
\newcommand\question[1]{ {\color{red} ...!? \small #1}}
\newcommand\caz[1]{\left\{\begin{array} #1 \end{array}\right.}
\newcommand\const{\text{\rmfamily\upshape const}}
\newcommand\toP{ \overset{\pro}{\to}}
\newcommand\toPP{ \overset{\text{PP}}{\to}}
\newcommand{\oeq}{\mathrel{\text{\textcircled{$=$}}}}





\usepackage{xcolor}
% \usepackage[normalem]{ulem}
\usepackage{lipsum}
\makeatletter
% \newcommand\colorwave[1][blue]{\bgroup \markoverwith{\lower3.5\p@\hbox{\sixly \textcolor{#1}{\char58}}}\ULon}
%\font\sixly=lasy6 % does not re-load if already loaded, so no memory problem.

\newmdtheoremenv[
linewidth= 1pt,linecolor= blue,%
leftmargin=20,rightmargin=20,innertopmargin=0pt, innerrightmargin=40,%
tikzsetting = { draw=lightgray, line width = 0.3pt,dashed,%
dash pattern = on 15pt off 3pt},%
splittopskip=\topskip,skipbelow=\baselineskip,%
skipabove=\baselineskip,ntheorem,roundcorner=0pt,
% backgroundcolor=pagebg,font=\color{orange}\sffamily, fontcolor=white
]{examplebox}{Exemple}[section]



\newcommand\R{\mathbb{R}}
\newcommand\Z{\mathbb{Z}}
\newcommand\N{\mathbb{N}}
\newcommand\E{\mathbb{E}}
\newcommand\F{\mathcal{F}}
\newcommand\cH{\mathcal{H}}
\newcommand\V{\mathbb{V}}
\newcommand\dmo{ ^{-1} }
\newcommand\kapa{\kappa}
\newcommand\im{Im}
\newcommand\hs{\mathcal{H}}





\usepackage{soul}

\makeatletter
\newcommand*{\whiten}[1]{\llap{\textcolor{white}{{\the\SOUL@token}}\hspace{#1pt}}}
\DeclareRobustCommand*\myul{%
    \def\SOUL@everyspace{\underline{\space}\kern\z@}%
    \def\SOUL@everytoken{%
     \setbox0=\hbox{\the\SOUL@token}%
     \ifdim\dp0>\z@
        \raisebox{\dp0}{\underline{\phantom{\the\SOUL@token}}}%
        \whiten{1}\whiten{0}%
        \whiten{-1}\whiten{-2}%
        \llap{\the\SOUL@token}%
     \else
        \underline{\the\SOUL@token}%
     \fi}%
\SOUL@}
\makeatother

\newcommand*{\demp}{\fontfamily{lmtt}\selectfont}

\DeclareTextFontCommand{\textdemp}{\demp}

\begin{document}

\ifcomment
Multiline
comment
\fi
\ifcomment
\myul{Typesetting test}
% \color[rgb]{1,1,1}
$∑_i^n≠ 60º±∞π∆¬≈√j∫h≤≥µ$

$\CR \R\pro\ind\pro\gS\pro
\mqty[a&b\\c&d]$
$\pro\mathbb{P}$
$\dd{x}$

  \[
    \alpha(x)=\left\{
                \begin{array}{ll}
                  x\\
                  \frac{1}{1+e^{-kx}}\\
                  \frac{e^x-e^{-x}}{e^x+e^{-x}}
                \end{array}
              \right.
  \]

  $\expval{x}$
  
  $\chi_\rho(ghg\dmo)=\Tr(\rho_{ghg\dmo})=\Tr(\rho_g\circ\rho_h\circ\rho\dmo_g)=\Tr(\rho_h)\overset{\mbox{\scalebox{0.5}{$\Tr(AB)=\Tr(BA)$}}}{=}\chi_\rho(h)$
  	$\mathop{\oplus}_{\substack{x\in X}}$

$\mat(\rho_g)=(a_{ij}(g))_{\scriptsize \substack{1\leq i\leq d \\ 1\leq j\leq d}}$ et $\mat(\rho'_g)=(a'_{ij}(g))_{\scriptsize \substack{1\leq i'\leq d' \\ 1\leq j'\leq d'}}$



\[\int_a^b{\mathbb{R}^2}g(u, v)\dd{P_{XY}}(u, v)=\iint g(u,v) f_{XY}(u, v)\dd \lambda(u) \dd \lambda(v)\]
$$\lim_{x\to\infty} f(x)$$	
$$\iiiint_V \mu(t,u,v,w) \,dt\,du\,dv\,dw$$
$$\sum_{n=1}^{\infty} 2^{-n} = 1$$	
\begin{definition}
	Si $X$ et $Y$ sont 2 v.a. ou definit la \textsc{Covariance} entre $X$ et $Y$ comme
	$\cov(X,Y)\overset{\text{def}}{=}\E\left[(X-\E(X))(Y-\E(Y))\right]=\E(XY)-\E(X)\E(Y)$.
\end{definition}
\fi
\pagebreak

% \tableofcontents

% insert your code here
%\input{./algebra/main.tex}
%\input{./geometrie-differentielle/main.tex}
%\input{./probabilite/main.tex}
%\input{./analyse-fonctionnelle/main.tex}
% \input{./Analyse-convexe-et-dualite-en-optimisation/main.tex}
%\input{./tikz/main.tex}
%\input{./Theorie-du-distributions/main.tex}
%\input{./optimisation/mine.tex}
 \input{./modelisation/main.tex}

% yves.aubry@univ-tln.fr : algebra

\end{document}

%% !TEX encoding = UTF-8 Unicode
% !TEX TS-program = xelatex

\documentclass[french]{report}

%\usepackage[utf8]{inputenc}
%\usepackage[T1]{fontenc}
\usepackage{babel}


\newif\ifcomment
%\commenttrue # Show comments

\usepackage{physics}
\usepackage{amssymb}


\usepackage{amsthm}
% \usepackage{thmtools}
\usepackage{mathtools}
\usepackage{amsfonts}

\usepackage{color}

\usepackage{tikz}

\usepackage{geometry}
\geometry{a5paper, margin=0.1in, right=1cm}

\usepackage{dsfont}

\usepackage{graphicx}
\graphicspath{ {images/} }

\usepackage{faktor}

\usepackage{IEEEtrantools}
\usepackage{enumerate}   
\usepackage[PostScript=dvips]{"/Users/aware/Documents/Courses/diagrams"}


\newtheorem{theorem}{Théorème}[section]
\renewcommand{\thetheorem}{\arabic{theorem}}
\newtheorem{lemme}{Lemme}[section]
\renewcommand{\thelemme}{\arabic{lemme}}
\newtheorem{proposition}{Proposition}[section]
\renewcommand{\theproposition}{\arabic{proposition}}
\newtheorem{notations}{Notations}[section]
\newtheorem{problem}{Problème}[section]
\newtheorem{corollary}{Corollaire}[theorem]
\renewcommand{\thecorollary}{\arabic{corollary}}
\newtheorem{property}{Propriété}[section]
\newtheorem{objective}{Objectif}[section]

\theoremstyle{definition}
\newtheorem{definition}{Définition}[section]
\renewcommand{\thedefinition}{\arabic{definition}}
\newtheorem{exercise}{Exercice}[chapter]
\renewcommand{\theexercise}{\arabic{exercise}}
\newtheorem{example}{Exemple}[chapter]
\renewcommand{\theexample}{\arabic{example}}
\newtheorem*{solution}{Solution}
\newtheorem*{application}{Application}
\newtheorem*{notation}{Notation}
\newtheorem*{vocabulary}{Vocabulaire}
\newtheorem*{properties}{Propriétés}



\theoremstyle{remark}
\newtheorem*{remark}{Remarque}
\newtheorem*{rappel}{Rappel}


\usepackage{etoolbox}
\AtBeginEnvironment{exercise}{\small}
\AtBeginEnvironment{example}{\small}

\usepackage{cases}
\usepackage[red]{mypack}

\usepackage[framemethod=TikZ]{mdframed}

\definecolor{bg}{rgb}{0.4,0.25,0.95}
\definecolor{pagebg}{rgb}{0,0,0.5}
\surroundwithmdframed[
   topline=false,
   rightline=false,
   bottomline=false,
   leftmargin=\parindent,
   skipabove=8pt,
   skipbelow=8pt,
   linecolor=blue,
   innerbottommargin=10pt,
   % backgroundcolor=bg,font=\color{orange}\sffamily, fontcolor=white
]{definition}

\usepackage{empheq}
\usepackage[most]{tcolorbox}

\newtcbox{\mymath}[1][]{%
    nobeforeafter, math upper, tcbox raise base,
    enhanced, colframe=blue!30!black,
    colback=red!10, boxrule=1pt,
    #1}

\usepackage{unixode}


\DeclareMathOperator{\ord}{ord}
\DeclareMathOperator{\orb}{orb}
\DeclareMathOperator{\stab}{stab}
\DeclareMathOperator{\Stab}{stab}
\DeclareMathOperator{\ppcm}{ppcm}
\DeclareMathOperator{\conj}{Conj}
\DeclareMathOperator{\End}{End}
\DeclareMathOperator{\rot}{rot}
\DeclareMathOperator{\trs}{trace}
\DeclareMathOperator{\Ind}{Ind}
\DeclareMathOperator{\mat}{Mat}
\DeclareMathOperator{\id}{Id}
\DeclareMathOperator{\vect}{vect}
\DeclareMathOperator{\img}{img}
\DeclareMathOperator{\cov}{Cov}
\DeclareMathOperator{\dist}{dist}
\DeclareMathOperator{\irr}{Irr}
\DeclareMathOperator{\image}{Im}
\DeclareMathOperator{\pd}{\partial}
\DeclareMathOperator{\epi}{epi}
\DeclareMathOperator{\Argmin}{Argmin}
\DeclareMathOperator{\dom}{dom}
\DeclareMathOperator{\proj}{proj}
\DeclareMathOperator{\ctg}{ctg}
\DeclareMathOperator{\supp}{supp}
\DeclareMathOperator{\argmin}{argmin}
\DeclareMathOperator{\mult}{mult}
\DeclareMathOperator{\ch}{ch}
\DeclareMathOperator{\sh}{sh}
\DeclareMathOperator{\rang}{rang}
\DeclareMathOperator{\diam}{diam}
\DeclareMathOperator{\Epigraphe}{Epigraphe}




\usepackage{xcolor}
\everymath{\color{blue}}
%\everymath{\color[rgb]{0,1,1}}
%\pagecolor[rgb]{0,0,0.5}


\newcommand*{\pdtest}[3][]{\ensuremath{\frac{\partial^{#1} #2}{\partial #3}}}

\newcommand*{\deffunc}[6][]{\ensuremath{
\begin{array}{rcl}
#2 : #3 &\rightarrow& #4\\
#5 &\mapsto& #6
\end{array}
}}

\newcommand{\eqcolon}{\mathrel{\resizebox{\widthof{$\mathord{=}$}}{\height}{ $\!\!=\!\!\resizebox{1.2\width}{0.8\height}{\raisebox{0.23ex}{$\mathop{:}$}}\!\!$ }}}
\newcommand{\coloneq}{\mathrel{\resizebox{\widthof{$\mathord{=}$}}{\height}{ $\!\!\resizebox{1.2\width}{0.8\height}{\raisebox{0.23ex}{$\mathop{:}$}}\!\!=\!\!$ }}}
\newcommand{\eqcolonl}{\ensuremath{\mathrel{=\!\!\mathop{:}}}}
\newcommand{\coloneql}{\ensuremath{\mathrel{\mathop{:} \!\! =}}}
\newcommand{\vc}[1]{% inline column vector
  \left(\begin{smallmatrix}#1\end{smallmatrix}\right)%
}
\newcommand{\vr}[1]{% inline row vector
  \begin{smallmatrix}(\,#1\,)\end{smallmatrix}%
}
\makeatletter
\newcommand*{\defeq}{\ =\mathrel{\rlap{%
                     \raisebox{0.3ex}{$\m@th\cdot$}}%
                     \raisebox{-0.3ex}{$\m@th\cdot$}}%
                     }
\makeatother

\newcommand{\mathcircle}[1]{% inline row vector
 \overset{\circ}{#1}
}
\newcommand{\ulim}{% low limit
 \underline{\lim}
}
\newcommand{\ssi}{% iff
\iff
}
\newcommand{\ps}[2]{
\expval{#1 | #2}
}
\newcommand{\df}[1]{
\mqty{#1}
}
\newcommand{\n}[1]{
\norm{#1}
}
\newcommand{\sys}[1]{
\left\{\smqty{#1}\right.
}


\newcommand{\eqdef}{\ensuremath{\overset{\text{def}}=}}


\def\Circlearrowright{\ensuremath{%
  \rotatebox[origin=c]{230}{$\circlearrowright$}}}

\newcommand\ct[1]{\text{\rmfamily\upshape #1}}
\newcommand\question[1]{ {\color{red} ...!? \small #1}}
\newcommand\caz[1]{\left\{\begin{array} #1 \end{array}\right.}
\newcommand\const{\text{\rmfamily\upshape const}}
\newcommand\toP{ \overset{\pro}{\to}}
\newcommand\toPP{ \overset{\text{PP}}{\to}}
\newcommand{\oeq}{\mathrel{\text{\textcircled{$=$}}}}





\usepackage{xcolor}
% \usepackage[normalem]{ulem}
\usepackage{lipsum}
\makeatletter
% \newcommand\colorwave[1][blue]{\bgroup \markoverwith{\lower3.5\p@\hbox{\sixly \textcolor{#1}{\char58}}}\ULon}
%\font\sixly=lasy6 % does not re-load if already loaded, so no memory problem.

\newmdtheoremenv[
linewidth= 1pt,linecolor= blue,%
leftmargin=20,rightmargin=20,innertopmargin=0pt, innerrightmargin=40,%
tikzsetting = { draw=lightgray, line width = 0.3pt,dashed,%
dash pattern = on 15pt off 3pt},%
splittopskip=\topskip,skipbelow=\baselineskip,%
skipabove=\baselineskip,ntheorem,roundcorner=0pt,
% backgroundcolor=pagebg,font=\color{orange}\sffamily, fontcolor=white
]{examplebox}{Exemple}[section]



\newcommand\R{\mathbb{R}}
\newcommand\Z{\mathbb{Z}}
\newcommand\N{\mathbb{N}}
\newcommand\E{\mathbb{E}}
\newcommand\F{\mathcal{F}}
\newcommand\cH{\mathcal{H}}
\newcommand\V{\mathbb{V}}
\newcommand\dmo{ ^{-1} }
\newcommand\kapa{\kappa}
\newcommand\im{Im}
\newcommand\hs{\mathcal{H}}





\usepackage{soul}

\makeatletter
\newcommand*{\whiten}[1]{\llap{\textcolor{white}{{\the\SOUL@token}}\hspace{#1pt}}}
\DeclareRobustCommand*\myul{%
    \def\SOUL@everyspace{\underline{\space}\kern\z@}%
    \def\SOUL@everytoken{%
     \setbox0=\hbox{\the\SOUL@token}%
     \ifdim\dp0>\z@
        \raisebox{\dp0}{\underline{\phantom{\the\SOUL@token}}}%
        \whiten{1}\whiten{0}%
        \whiten{-1}\whiten{-2}%
        \llap{\the\SOUL@token}%
     \else
        \underline{\the\SOUL@token}%
     \fi}%
\SOUL@}
\makeatother

\newcommand*{\demp}{\fontfamily{lmtt}\selectfont}

\DeclareTextFontCommand{\textdemp}{\demp}

\begin{document}

\ifcomment
Multiline
comment
\fi
\ifcomment
\myul{Typesetting test}
% \color[rgb]{1,1,1}
$∑_i^n≠ 60º±∞π∆¬≈√j∫h≤≥µ$

$\CR \R\pro\ind\pro\gS\pro
\mqty[a&b\\c&d]$
$\pro\mathbb{P}$
$\dd{x}$

  \[
    \alpha(x)=\left\{
                \begin{array}{ll}
                  x\\
                  \frac{1}{1+e^{-kx}}\\
                  \frac{e^x-e^{-x}}{e^x+e^{-x}}
                \end{array}
              \right.
  \]

  $\expval{x}$
  
  $\chi_\rho(ghg\dmo)=\Tr(\rho_{ghg\dmo})=\Tr(\rho_g\circ\rho_h\circ\rho\dmo_g)=\Tr(\rho_h)\overset{\mbox{\scalebox{0.5}{$\Tr(AB)=\Tr(BA)$}}}{=}\chi_\rho(h)$
  	$\mathop{\oplus}_{\substack{x\in X}}$

$\mat(\rho_g)=(a_{ij}(g))_{\scriptsize \substack{1\leq i\leq d \\ 1\leq j\leq d}}$ et $\mat(\rho'_g)=(a'_{ij}(g))_{\scriptsize \substack{1\leq i'\leq d' \\ 1\leq j'\leq d'}}$



\[\int_a^b{\mathbb{R}^2}g(u, v)\dd{P_{XY}}(u, v)=\iint g(u,v) f_{XY}(u, v)\dd \lambda(u) \dd \lambda(v)\]
$$\lim_{x\to\infty} f(x)$$	
$$\iiiint_V \mu(t,u,v,w) \,dt\,du\,dv\,dw$$
$$\sum_{n=1}^{\infty} 2^{-n} = 1$$	
\begin{definition}
	Si $X$ et $Y$ sont 2 v.a. ou definit la \textsc{Covariance} entre $X$ et $Y$ comme
	$\cov(X,Y)\overset{\text{def}}{=}\E\left[(X-\E(X))(Y-\E(Y))\right]=\E(XY)-\E(X)\E(Y)$.
\end{definition}
\fi
\pagebreak

% \tableofcontents

% insert your code here
%\input{./algebra/main.tex}
%\input{./geometrie-differentielle/main.tex}
%\input{./probabilite/main.tex}
%\input{./analyse-fonctionnelle/main.tex}
% \input{./Analyse-convexe-et-dualite-en-optimisation/main.tex}
%\input{./tikz/main.tex}
%\input{./Theorie-du-distributions/main.tex}
%\input{./optimisation/mine.tex}
 \input{./modelisation/main.tex}

% yves.aubry@univ-tln.fr : algebra

\end{document}

%% !TEX encoding = UTF-8 Unicode
% !TEX TS-program = xelatex

\documentclass[french]{report}

%\usepackage[utf8]{inputenc}
%\usepackage[T1]{fontenc}
\usepackage{babel}


\newif\ifcomment
%\commenttrue # Show comments

\usepackage{physics}
\usepackage{amssymb}


\usepackage{amsthm}
% \usepackage{thmtools}
\usepackage{mathtools}
\usepackage{amsfonts}

\usepackage{color}

\usepackage{tikz}

\usepackage{geometry}
\geometry{a5paper, margin=0.1in, right=1cm}

\usepackage{dsfont}

\usepackage{graphicx}
\graphicspath{ {images/} }

\usepackage{faktor}

\usepackage{IEEEtrantools}
\usepackage{enumerate}   
\usepackage[PostScript=dvips]{"/Users/aware/Documents/Courses/diagrams"}


\newtheorem{theorem}{Théorème}[section]
\renewcommand{\thetheorem}{\arabic{theorem}}
\newtheorem{lemme}{Lemme}[section]
\renewcommand{\thelemme}{\arabic{lemme}}
\newtheorem{proposition}{Proposition}[section]
\renewcommand{\theproposition}{\arabic{proposition}}
\newtheorem{notations}{Notations}[section]
\newtheorem{problem}{Problème}[section]
\newtheorem{corollary}{Corollaire}[theorem]
\renewcommand{\thecorollary}{\arabic{corollary}}
\newtheorem{property}{Propriété}[section]
\newtheorem{objective}{Objectif}[section]

\theoremstyle{definition}
\newtheorem{definition}{Définition}[section]
\renewcommand{\thedefinition}{\arabic{definition}}
\newtheorem{exercise}{Exercice}[chapter]
\renewcommand{\theexercise}{\arabic{exercise}}
\newtheorem{example}{Exemple}[chapter]
\renewcommand{\theexample}{\arabic{example}}
\newtheorem*{solution}{Solution}
\newtheorem*{application}{Application}
\newtheorem*{notation}{Notation}
\newtheorem*{vocabulary}{Vocabulaire}
\newtheorem*{properties}{Propriétés}



\theoremstyle{remark}
\newtheorem*{remark}{Remarque}
\newtheorem*{rappel}{Rappel}


\usepackage{etoolbox}
\AtBeginEnvironment{exercise}{\small}
\AtBeginEnvironment{example}{\small}

\usepackage{cases}
\usepackage[red]{mypack}

\usepackage[framemethod=TikZ]{mdframed}

\definecolor{bg}{rgb}{0.4,0.25,0.95}
\definecolor{pagebg}{rgb}{0,0,0.5}
\surroundwithmdframed[
   topline=false,
   rightline=false,
   bottomline=false,
   leftmargin=\parindent,
   skipabove=8pt,
   skipbelow=8pt,
   linecolor=blue,
   innerbottommargin=10pt,
   % backgroundcolor=bg,font=\color{orange}\sffamily, fontcolor=white
]{definition}

\usepackage{empheq}
\usepackage[most]{tcolorbox}

\newtcbox{\mymath}[1][]{%
    nobeforeafter, math upper, tcbox raise base,
    enhanced, colframe=blue!30!black,
    colback=red!10, boxrule=1pt,
    #1}

\usepackage{unixode}


\DeclareMathOperator{\ord}{ord}
\DeclareMathOperator{\orb}{orb}
\DeclareMathOperator{\stab}{stab}
\DeclareMathOperator{\Stab}{stab}
\DeclareMathOperator{\ppcm}{ppcm}
\DeclareMathOperator{\conj}{Conj}
\DeclareMathOperator{\End}{End}
\DeclareMathOperator{\rot}{rot}
\DeclareMathOperator{\trs}{trace}
\DeclareMathOperator{\Ind}{Ind}
\DeclareMathOperator{\mat}{Mat}
\DeclareMathOperator{\id}{Id}
\DeclareMathOperator{\vect}{vect}
\DeclareMathOperator{\img}{img}
\DeclareMathOperator{\cov}{Cov}
\DeclareMathOperator{\dist}{dist}
\DeclareMathOperator{\irr}{Irr}
\DeclareMathOperator{\image}{Im}
\DeclareMathOperator{\pd}{\partial}
\DeclareMathOperator{\epi}{epi}
\DeclareMathOperator{\Argmin}{Argmin}
\DeclareMathOperator{\dom}{dom}
\DeclareMathOperator{\proj}{proj}
\DeclareMathOperator{\ctg}{ctg}
\DeclareMathOperator{\supp}{supp}
\DeclareMathOperator{\argmin}{argmin}
\DeclareMathOperator{\mult}{mult}
\DeclareMathOperator{\ch}{ch}
\DeclareMathOperator{\sh}{sh}
\DeclareMathOperator{\rang}{rang}
\DeclareMathOperator{\diam}{diam}
\DeclareMathOperator{\Epigraphe}{Epigraphe}




\usepackage{xcolor}
\everymath{\color{blue}}
%\everymath{\color[rgb]{0,1,1}}
%\pagecolor[rgb]{0,0,0.5}


\newcommand*{\pdtest}[3][]{\ensuremath{\frac{\partial^{#1} #2}{\partial #3}}}

\newcommand*{\deffunc}[6][]{\ensuremath{
\begin{array}{rcl}
#2 : #3 &\rightarrow& #4\\
#5 &\mapsto& #6
\end{array}
}}

\newcommand{\eqcolon}{\mathrel{\resizebox{\widthof{$\mathord{=}$}}{\height}{ $\!\!=\!\!\resizebox{1.2\width}{0.8\height}{\raisebox{0.23ex}{$\mathop{:}$}}\!\!$ }}}
\newcommand{\coloneq}{\mathrel{\resizebox{\widthof{$\mathord{=}$}}{\height}{ $\!\!\resizebox{1.2\width}{0.8\height}{\raisebox{0.23ex}{$\mathop{:}$}}\!\!=\!\!$ }}}
\newcommand{\eqcolonl}{\ensuremath{\mathrel{=\!\!\mathop{:}}}}
\newcommand{\coloneql}{\ensuremath{\mathrel{\mathop{:} \!\! =}}}
\newcommand{\vc}[1]{% inline column vector
  \left(\begin{smallmatrix}#1\end{smallmatrix}\right)%
}
\newcommand{\vr}[1]{% inline row vector
  \begin{smallmatrix}(\,#1\,)\end{smallmatrix}%
}
\makeatletter
\newcommand*{\defeq}{\ =\mathrel{\rlap{%
                     \raisebox{0.3ex}{$\m@th\cdot$}}%
                     \raisebox{-0.3ex}{$\m@th\cdot$}}%
                     }
\makeatother

\newcommand{\mathcircle}[1]{% inline row vector
 \overset{\circ}{#1}
}
\newcommand{\ulim}{% low limit
 \underline{\lim}
}
\newcommand{\ssi}{% iff
\iff
}
\newcommand{\ps}[2]{
\expval{#1 | #2}
}
\newcommand{\df}[1]{
\mqty{#1}
}
\newcommand{\n}[1]{
\norm{#1}
}
\newcommand{\sys}[1]{
\left\{\smqty{#1}\right.
}


\newcommand{\eqdef}{\ensuremath{\overset{\text{def}}=}}


\def\Circlearrowright{\ensuremath{%
  \rotatebox[origin=c]{230}{$\circlearrowright$}}}

\newcommand\ct[1]{\text{\rmfamily\upshape #1}}
\newcommand\question[1]{ {\color{red} ...!? \small #1}}
\newcommand\caz[1]{\left\{\begin{array} #1 \end{array}\right.}
\newcommand\const{\text{\rmfamily\upshape const}}
\newcommand\toP{ \overset{\pro}{\to}}
\newcommand\toPP{ \overset{\text{PP}}{\to}}
\newcommand{\oeq}{\mathrel{\text{\textcircled{$=$}}}}





\usepackage{xcolor}
% \usepackage[normalem]{ulem}
\usepackage{lipsum}
\makeatletter
% \newcommand\colorwave[1][blue]{\bgroup \markoverwith{\lower3.5\p@\hbox{\sixly \textcolor{#1}{\char58}}}\ULon}
%\font\sixly=lasy6 % does not re-load if already loaded, so no memory problem.

\newmdtheoremenv[
linewidth= 1pt,linecolor= blue,%
leftmargin=20,rightmargin=20,innertopmargin=0pt, innerrightmargin=40,%
tikzsetting = { draw=lightgray, line width = 0.3pt,dashed,%
dash pattern = on 15pt off 3pt},%
splittopskip=\topskip,skipbelow=\baselineskip,%
skipabove=\baselineskip,ntheorem,roundcorner=0pt,
% backgroundcolor=pagebg,font=\color{orange}\sffamily, fontcolor=white
]{examplebox}{Exemple}[section]



\newcommand\R{\mathbb{R}}
\newcommand\Z{\mathbb{Z}}
\newcommand\N{\mathbb{N}}
\newcommand\E{\mathbb{E}}
\newcommand\F{\mathcal{F}}
\newcommand\cH{\mathcal{H}}
\newcommand\V{\mathbb{V}}
\newcommand\dmo{ ^{-1} }
\newcommand\kapa{\kappa}
\newcommand\im{Im}
\newcommand\hs{\mathcal{H}}





\usepackage{soul}

\makeatletter
\newcommand*{\whiten}[1]{\llap{\textcolor{white}{{\the\SOUL@token}}\hspace{#1pt}}}
\DeclareRobustCommand*\myul{%
    \def\SOUL@everyspace{\underline{\space}\kern\z@}%
    \def\SOUL@everytoken{%
     \setbox0=\hbox{\the\SOUL@token}%
     \ifdim\dp0>\z@
        \raisebox{\dp0}{\underline{\phantom{\the\SOUL@token}}}%
        \whiten{1}\whiten{0}%
        \whiten{-1}\whiten{-2}%
        \llap{\the\SOUL@token}%
     \else
        \underline{\the\SOUL@token}%
     \fi}%
\SOUL@}
\makeatother

\newcommand*{\demp}{\fontfamily{lmtt}\selectfont}

\DeclareTextFontCommand{\textdemp}{\demp}

\begin{document}

\ifcomment
Multiline
comment
\fi
\ifcomment
\myul{Typesetting test}
% \color[rgb]{1,1,1}
$∑_i^n≠ 60º±∞π∆¬≈√j∫h≤≥µ$

$\CR \R\pro\ind\pro\gS\pro
\mqty[a&b\\c&d]$
$\pro\mathbb{P}$
$\dd{x}$

  \[
    \alpha(x)=\left\{
                \begin{array}{ll}
                  x\\
                  \frac{1}{1+e^{-kx}}\\
                  \frac{e^x-e^{-x}}{e^x+e^{-x}}
                \end{array}
              \right.
  \]

  $\expval{x}$
  
  $\chi_\rho(ghg\dmo)=\Tr(\rho_{ghg\dmo})=\Tr(\rho_g\circ\rho_h\circ\rho\dmo_g)=\Tr(\rho_h)\overset{\mbox{\scalebox{0.5}{$\Tr(AB)=\Tr(BA)$}}}{=}\chi_\rho(h)$
  	$\mathop{\oplus}_{\substack{x\in X}}$

$\mat(\rho_g)=(a_{ij}(g))_{\scriptsize \substack{1\leq i\leq d \\ 1\leq j\leq d}}$ et $\mat(\rho'_g)=(a'_{ij}(g))_{\scriptsize \substack{1\leq i'\leq d' \\ 1\leq j'\leq d'}}$



\[\int_a^b{\mathbb{R}^2}g(u, v)\dd{P_{XY}}(u, v)=\iint g(u,v) f_{XY}(u, v)\dd \lambda(u) \dd \lambda(v)\]
$$\lim_{x\to\infty} f(x)$$	
$$\iiiint_V \mu(t,u,v,w) \,dt\,du\,dv\,dw$$
$$\sum_{n=1}^{\infty} 2^{-n} = 1$$	
\begin{definition}
	Si $X$ et $Y$ sont 2 v.a. ou definit la \textsc{Covariance} entre $X$ et $Y$ comme
	$\cov(X,Y)\overset{\text{def}}{=}\E\left[(X-\E(X))(Y-\E(Y))\right]=\E(XY)-\E(X)\E(Y)$.
\end{definition}
\fi
\pagebreak

% \tableofcontents

% insert your code here
%\input{./algebra/main.tex}
%\input{./geometrie-differentielle/main.tex}
%\input{./probabilite/main.tex}
%\input{./analyse-fonctionnelle/main.tex}
% \input{./Analyse-convexe-et-dualite-en-optimisation/main.tex}
%\input{./tikz/main.tex}
%\input{./Theorie-du-distributions/main.tex}
%\input{./optimisation/mine.tex}
 \input{./modelisation/main.tex}

% yves.aubry@univ-tln.fr : algebra

\end{document}

% % !TEX encoding = UTF-8 Unicode
% !TEX TS-program = xelatex

\documentclass[french]{report}

%\usepackage[utf8]{inputenc}
%\usepackage[T1]{fontenc}
\usepackage{babel}


\newif\ifcomment
%\commenttrue # Show comments

\usepackage{physics}
\usepackage{amssymb}


\usepackage{amsthm}
% \usepackage{thmtools}
\usepackage{mathtools}
\usepackage{amsfonts}

\usepackage{color}

\usepackage{tikz}

\usepackage{geometry}
\geometry{a5paper, margin=0.1in, right=1cm}

\usepackage{dsfont}

\usepackage{graphicx}
\graphicspath{ {images/} }

\usepackage{faktor}

\usepackage{IEEEtrantools}
\usepackage{enumerate}   
\usepackage[PostScript=dvips]{"/Users/aware/Documents/Courses/diagrams"}


\newtheorem{theorem}{Théorème}[section]
\renewcommand{\thetheorem}{\arabic{theorem}}
\newtheorem{lemme}{Lemme}[section]
\renewcommand{\thelemme}{\arabic{lemme}}
\newtheorem{proposition}{Proposition}[section]
\renewcommand{\theproposition}{\arabic{proposition}}
\newtheorem{notations}{Notations}[section]
\newtheorem{problem}{Problème}[section]
\newtheorem{corollary}{Corollaire}[theorem]
\renewcommand{\thecorollary}{\arabic{corollary}}
\newtheorem{property}{Propriété}[section]
\newtheorem{objective}{Objectif}[section]

\theoremstyle{definition}
\newtheorem{definition}{Définition}[section]
\renewcommand{\thedefinition}{\arabic{definition}}
\newtheorem{exercise}{Exercice}[chapter]
\renewcommand{\theexercise}{\arabic{exercise}}
\newtheorem{example}{Exemple}[chapter]
\renewcommand{\theexample}{\arabic{example}}
\newtheorem*{solution}{Solution}
\newtheorem*{application}{Application}
\newtheorem*{notation}{Notation}
\newtheorem*{vocabulary}{Vocabulaire}
\newtheorem*{properties}{Propriétés}



\theoremstyle{remark}
\newtheorem*{remark}{Remarque}
\newtheorem*{rappel}{Rappel}


\usepackage{etoolbox}
\AtBeginEnvironment{exercise}{\small}
\AtBeginEnvironment{example}{\small}

\usepackage{cases}
\usepackage[red]{mypack}

\usepackage[framemethod=TikZ]{mdframed}

\definecolor{bg}{rgb}{0.4,0.25,0.95}
\definecolor{pagebg}{rgb}{0,0,0.5}
\surroundwithmdframed[
   topline=false,
   rightline=false,
   bottomline=false,
   leftmargin=\parindent,
   skipabove=8pt,
   skipbelow=8pt,
   linecolor=blue,
   innerbottommargin=10pt,
   % backgroundcolor=bg,font=\color{orange}\sffamily, fontcolor=white
]{definition}

\usepackage{empheq}
\usepackage[most]{tcolorbox}

\newtcbox{\mymath}[1][]{%
    nobeforeafter, math upper, tcbox raise base,
    enhanced, colframe=blue!30!black,
    colback=red!10, boxrule=1pt,
    #1}

\usepackage{unixode}


\DeclareMathOperator{\ord}{ord}
\DeclareMathOperator{\orb}{orb}
\DeclareMathOperator{\stab}{stab}
\DeclareMathOperator{\Stab}{stab}
\DeclareMathOperator{\ppcm}{ppcm}
\DeclareMathOperator{\conj}{Conj}
\DeclareMathOperator{\End}{End}
\DeclareMathOperator{\rot}{rot}
\DeclareMathOperator{\trs}{trace}
\DeclareMathOperator{\Ind}{Ind}
\DeclareMathOperator{\mat}{Mat}
\DeclareMathOperator{\id}{Id}
\DeclareMathOperator{\vect}{vect}
\DeclareMathOperator{\img}{img}
\DeclareMathOperator{\cov}{Cov}
\DeclareMathOperator{\dist}{dist}
\DeclareMathOperator{\irr}{Irr}
\DeclareMathOperator{\image}{Im}
\DeclareMathOperator{\pd}{\partial}
\DeclareMathOperator{\epi}{epi}
\DeclareMathOperator{\Argmin}{Argmin}
\DeclareMathOperator{\dom}{dom}
\DeclareMathOperator{\proj}{proj}
\DeclareMathOperator{\ctg}{ctg}
\DeclareMathOperator{\supp}{supp}
\DeclareMathOperator{\argmin}{argmin}
\DeclareMathOperator{\mult}{mult}
\DeclareMathOperator{\ch}{ch}
\DeclareMathOperator{\sh}{sh}
\DeclareMathOperator{\rang}{rang}
\DeclareMathOperator{\diam}{diam}
\DeclareMathOperator{\Epigraphe}{Epigraphe}




\usepackage{xcolor}
\everymath{\color{blue}}
%\everymath{\color[rgb]{0,1,1}}
%\pagecolor[rgb]{0,0,0.5}


\newcommand*{\pdtest}[3][]{\ensuremath{\frac{\partial^{#1} #2}{\partial #3}}}

\newcommand*{\deffunc}[6][]{\ensuremath{
\begin{array}{rcl}
#2 : #3 &\rightarrow& #4\\
#5 &\mapsto& #6
\end{array}
}}

\newcommand{\eqcolon}{\mathrel{\resizebox{\widthof{$\mathord{=}$}}{\height}{ $\!\!=\!\!\resizebox{1.2\width}{0.8\height}{\raisebox{0.23ex}{$\mathop{:}$}}\!\!$ }}}
\newcommand{\coloneq}{\mathrel{\resizebox{\widthof{$\mathord{=}$}}{\height}{ $\!\!\resizebox{1.2\width}{0.8\height}{\raisebox{0.23ex}{$\mathop{:}$}}\!\!=\!\!$ }}}
\newcommand{\eqcolonl}{\ensuremath{\mathrel{=\!\!\mathop{:}}}}
\newcommand{\coloneql}{\ensuremath{\mathrel{\mathop{:} \!\! =}}}
\newcommand{\vc}[1]{% inline column vector
  \left(\begin{smallmatrix}#1\end{smallmatrix}\right)%
}
\newcommand{\vr}[1]{% inline row vector
  \begin{smallmatrix}(\,#1\,)\end{smallmatrix}%
}
\makeatletter
\newcommand*{\defeq}{\ =\mathrel{\rlap{%
                     \raisebox{0.3ex}{$\m@th\cdot$}}%
                     \raisebox{-0.3ex}{$\m@th\cdot$}}%
                     }
\makeatother

\newcommand{\mathcircle}[1]{% inline row vector
 \overset{\circ}{#1}
}
\newcommand{\ulim}{% low limit
 \underline{\lim}
}
\newcommand{\ssi}{% iff
\iff
}
\newcommand{\ps}[2]{
\expval{#1 | #2}
}
\newcommand{\df}[1]{
\mqty{#1}
}
\newcommand{\n}[1]{
\norm{#1}
}
\newcommand{\sys}[1]{
\left\{\smqty{#1}\right.
}


\newcommand{\eqdef}{\ensuremath{\overset{\text{def}}=}}


\def\Circlearrowright{\ensuremath{%
  \rotatebox[origin=c]{230}{$\circlearrowright$}}}

\newcommand\ct[1]{\text{\rmfamily\upshape #1}}
\newcommand\question[1]{ {\color{red} ...!? \small #1}}
\newcommand\caz[1]{\left\{\begin{array} #1 \end{array}\right.}
\newcommand\const{\text{\rmfamily\upshape const}}
\newcommand\toP{ \overset{\pro}{\to}}
\newcommand\toPP{ \overset{\text{PP}}{\to}}
\newcommand{\oeq}{\mathrel{\text{\textcircled{$=$}}}}





\usepackage{xcolor}
% \usepackage[normalem]{ulem}
\usepackage{lipsum}
\makeatletter
% \newcommand\colorwave[1][blue]{\bgroup \markoverwith{\lower3.5\p@\hbox{\sixly \textcolor{#1}{\char58}}}\ULon}
%\font\sixly=lasy6 % does not re-load if already loaded, so no memory problem.

\newmdtheoremenv[
linewidth= 1pt,linecolor= blue,%
leftmargin=20,rightmargin=20,innertopmargin=0pt, innerrightmargin=40,%
tikzsetting = { draw=lightgray, line width = 0.3pt,dashed,%
dash pattern = on 15pt off 3pt},%
splittopskip=\topskip,skipbelow=\baselineskip,%
skipabove=\baselineskip,ntheorem,roundcorner=0pt,
% backgroundcolor=pagebg,font=\color{orange}\sffamily, fontcolor=white
]{examplebox}{Exemple}[section]



\newcommand\R{\mathbb{R}}
\newcommand\Z{\mathbb{Z}}
\newcommand\N{\mathbb{N}}
\newcommand\E{\mathbb{E}}
\newcommand\F{\mathcal{F}}
\newcommand\cH{\mathcal{H}}
\newcommand\V{\mathbb{V}}
\newcommand\dmo{ ^{-1} }
\newcommand\kapa{\kappa}
\newcommand\im{Im}
\newcommand\hs{\mathcal{H}}





\usepackage{soul}

\makeatletter
\newcommand*{\whiten}[1]{\llap{\textcolor{white}{{\the\SOUL@token}}\hspace{#1pt}}}
\DeclareRobustCommand*\myul{%
    \def\SOUL@everyspace{\underline{\space}\kern\z@}%
    \def\SOUL@everytoken{%
     \setbox0=\hbox{\the\SOUL@token}%
     \ifdim\dp0>\z@
        \raisebox{\dp0}{\underline{\phantom{\the\SOUL@token}}}%
        \whiten{1}\whiten{0}%
        \whiten{-1}\whiten{-2}%
        \llap{\the\SOUL@token}%
     \else
        \underline{\the\SOUL@token}%
     \fi}%
\SOUL@}
\makeatother

\newcommand*{\demp}{\fontfamily{lmtt}\selectfont}

\DeclareTextFontCommand{\textdemp}{\demp}

\begin{document}

\ifcomment
Multiline
comment
\fi
\ifcomment
\myul{Typesetting test}
% \color[rgb]{1,1,1}
$∑_i^n≠ 60º±∞π∆¬≈√j∫h≤≥µ$

$\CR \R\pro\ind\pro\gS\pro
\mqty[a&b\\c&d]$
$\pro\mathbb{P}$
$\dd{x}$

  \[
    \alpha(x)=\left\{
                \begin{array}{ll}
                  x\\
                  \frac{1}{1+e^{-kx}}\\
                  \frac{e^x-e^{-x}}{e^x+e^{-x}}
                \end{array}
              \right.
  \]

  $\expval{x}$
  
  $\chi_\rho(ghg\dmo)=\Tr(\rho_{ghg\dmo})=\Tr(\rho_g\circ\rho_h\circ\rho\dmo_g)=\Tr(\rho_h)\overset{\mbox{\scalebox{0.5}{$\Tr(AB)=\Tr(BA)$}}}{=}\chi_\rho(h)$
  	$\mathop{\oplus}_{\substack{x\in X}}$

$\mat(\rho_g)=(a_{ij}(g))_{\scriptsize \substack{1\leq i\leq d \\ 1\leq j\leq d}}$ et $\mat(\rho'_g)=(a'_{ij}(g))_{\scriptsize \substack{1\leq i'\leq d' \\ 1\leq j'\leq d'}}$



\[\int_a^b{\mathbb{R}^2}g(u, v)\dd{P_{XY}}(u, v)=\iint g(u,v) f_{XY}(u, v)\dd \lambda(u) \dd \lambda(v)\]
$$\lim_{x\to\infty} f(x)$$	
$$\iiiint_V \mu(t,u,v,w) \,dt\,du\,dv\,dw$$
$$\sum_{n=1}^{\infty} 2^{-n} = 1$$	
\begin{definition}
	Si $X$ et $Y$ sont 2 v.a. ou definit la \textsc{Covariance} entre $X$ et $Y$ comme
	$\cov(X,Y)\overset{\text{def}}{=}\E\left[(X-\E(X))(Y-\E(Y))\right]=\E(XY)-\E(X)\E(Y)$.
\end{definition}
\fi
\pagebreak

% \tableofcontents

% insert your code here
%\input{./algebra/main.tex}
%\input{./geometrie-differentielle/main.tex}
%\input{./probabilite/main.tex}
%\input{./analyse-fonctionnelle/main.tex}
% \input{./Analyse-convexe-et-dualite-en-optimisation/main.tex}
%\input{./tikz/main.tex}
%\input{./Theorie-du-distributions/main.tex}
%\input{./optimisation/mine.tex}
 \input{./modelisation/main.tex}

% yves.aubry@univ-tln.fr : algebra

\end{document}

%% !TEX encoding = UTF-8 Unicode
% !TEX TS-program = xelatex

\documentclass[french]{report}

%\usepackage[utf8]{inputenc}
%\usepackage[T1]{fontenc}
\usepackage{babel}


\newif\ifcomment
%\commenttrue # Show comments

\usepackage{physics}
\usepackage{amssymb}


\usepackage{amsthm}
% \usepackage{thmtools}
\usepackage{mathtools}
\usepackage{amsfonts}

\usepackage{color}

\usepackage{tikz}

\usepackage{geometry}
\geometry{a5paper, margin=0.1in, right=1cm}

\usepackage{dsfont}

\usepackage{graphicx}
\graphicspath{ {images/} }

\usepackage{faktor}

\usepackage{IEEEtrantools}
\usepackage{enumerate}   
\usepackage[PostScript=dvips]{"/Users/aware/Documents/Courses/diagrams"}


\newtheorem{theorem}{Théorème}[section]
\renewcommand{\thetheorem}{\arabic{theorem}}
\newtheorem{lemme}{Lemme}[section]
\renewcommand{\thelemme}{\arabic{lemme}}
\newtheorem{proposition}{Proposition}[section]
\renewcommand{\theproposition}{\arabic{proposition}}
\newtheorem{notations}{Notations}[section]
\newtheorem{problem}{Problème}[section]
\newtheorem{corollary}{Corollaire}[theorem]
\renewcommand{\thecorollary}{\arabic{corollary}}
\newtheorem{property}{Propriété}[section]
\newtheorem{objective}{Objectif}[section]

\theoremstyle{definition}
\newtheorem{definition}{Définition}[section]
\renewcommand{\thedefinition}{\arabic{definition}}
\newtheorem{exercise}{Exercice}[chapter]
\renewcommand{\theexercise}{\arabic{exercise}}
\newtheorem{example}{Exemple}[chapter]
\renewcommand{\theexample}{\arabic{example}}
\newtheorem*{solution}{Solution}
\newtheorem*{application}{Application}
\newtheorem*{notation}{Notation}
\newtheorem*{vocabulary}{Vocabulaire}
\newtheorem*{properties}{Propriétés}



\theoremstyle{remark}
\newtheorem*{remark}{Remarque}
\newtheorem*{rappel}{Rappel}


\usepackage{etoolbox}
\AtBeginEnvironment{exercise}{\small}
\AtBeginEnvironment{example}{\small}

\usepackage{cases}
\usepackage[red]{mypack}

\usepackage[framemethod=TikZ]{mdframed}

\definecolor{bg}{rgb}{0.4,0.25,0.95}
\definecolor{pagebg}{rgb}{0,0,0.5}
\surroundwithmdframed[
   topline=false,
   rightline=false,
   bottomline=false,
   leftmargin=\parindent,
   skipabove=8pt,
   skipbelow=8pt,
   linecolor=blue,
   innerbottommargin=10pt,
   % backgroundcolor=bg,font=\color{orange}\sffamily, fontcolor=white
]{definition}

\usepackage{empheq}
\usepackage[most]{tcolorbox}

\newtcbox{\mymath}[1][]{%
    nobeforeafter, math upper, tcbox raise base,
    enhanced, colframe=blue!30!black,
    colback=red!10, boxrule=1pt,
    #1}

\usepackage{unixode}


\DeclareMathOperator{\ord}{ord}
\DeclareMathOperator{\orb}{orb}
\DeclareMathOperator{\stab}{stab}
\DeclareMathOperator{\Stab}{stab}
\DeclareMathOperator{\ppcm}{ppcm}
\DeclareMathOperator{\conj}{Conj}
\DeclareMathOperator{\End}{End}
\DeclareMathOperator{\rot}{rot}
\DeclareMathOperator{\trs}{trace}
\DeclareMathOperator{\Ind}{Ind}
\DeclareMathOperator{\mat}{Mat}
\DeclareMathOperator{\id}{Id}
\DeclareMathOperator{\vect}{vect}
\DeclareMathOperator{\img}{img}
\DeclareMathOperator{\cov}{Cov}
\DeclareMathOperator{\dist}{dist}
\DeclareMathOperator{\irr}{Irr}
\DeclareMathOperator{\image}{Im}
\DeclareMathOperator{\pd}{\partial}
\DeclareMathOperator{\epi}{epi}
\DeclareMathOperator{\Argmin}{Argmin}
\DeclareMathOperator{\dom}{dom}
\DeclareMathOperator{\proj}{proj}
\DeclareMathOperator{\ctg}{ctg}
\DeclareMathOperator{\supp}{supp}
\DeclareMathOperator{\argmin}{argmin}
\DeclareMathOperator{\mult}{mult}
\DeclareMathOperator{\ch}{ch}
\DeclareMathOperator{\sh}{sh}
\DeclareMathOperator{\rang}{rang}
\DeclareMathOperator{\diam}{diam}
\DeclareMathOperator{\Epigraphe}{Epigraphe}




\usepackage{xcolor}
\everymath{\color{blue}}
%\everymath{\color[rgb]{0,1,1}}
%\pagecolor[rgb]{0,0,0.5}


\newcommand*{\pdtest}[3][]{\ensuremath{\frac{\partial^{#1} #2}{\partial #3}}}

\newcommand*{\deffunc}[6][]{\ensuremath{
\begin{array}{rcl}
#2 : #3 &\rightarrow& #4\\
#5 &\mapsto& #6
\end{array}
}}

\newcommand{\eqcolon}{\mathrel{\resizebox{\widthof{$\mathord{=}$}}{\height}{ $\!\!=\!\!\resizebox{1.2\width}{0.8\height}{\raisebox{0.23ex}{$\mathop{:}$}}\!\!$ }}}
\newcommand{\coloneq}{\mathrel{\resizebox{\widthof{$\mathord{=}$}}{\height}{ $\!\!\resizebox{1.2\width}{0.8\height}{\raisebox{0.23ex}{$\mathop{:}$}}\!\!=\!\!$ }}}
\newcommand{\eqcolonl}{\ensuremath{\mathrel{=\!\!\mathop{:}}}}
\newcommand{\coloneql}{\ensuremath{\mathrel{\mathop{:} \!\! =}}}
\newcommand{\vc}[1]{% inline column vector
  \left(\begin{smallmatrix}#1\end{smallmatrix}\right)%
}
\newcommand{\vr}[1]{% inline row vector
  \begin{smallmatrix}(\,#1\,)\end{smallmatrix}%
}
\makeatletter
\newcommand*{\defeq}{\ =\mathrel{\rlap{%
                     \raisebox{0.3ex}{$\m@th\cdot$}}%
                     \raisebox{-0.3ex}{$\m@th\cdot$}}%
                     }
\makeatother

\newcommand{\mathcircle}[1]{% inline row vector
 \overset{\circ}{#1}
}
\newcommand{\ulim}{% low limit
 \underline{\lim}
}
\newcommand{\ssi}{% iff
\iff
}
\newcommand{\ps}[2]{
\expval{#1 | #2}
}
\newcommand{\df}[1]{
\mqty{#1}
}
\newcommand{\n}[1]{
\norm{#1}
}
\newcommand{\sys}[1]{
\left\{\smqty{#1}\right.
}


\newcommand{\eqdef}{\ensuremath{\overset{\text{def}}=}}


\def\Circlearrowright{\ensuremath{%
  \rotatebox[origin=c]{230}{$\circlearrowright$}}}

\newcommand\ct[1]{\text{\rmfamily\upshape #1}}
\newcommand\question[1]{ {\color{red} ...!? \small #1}}
\newcommand\caz[1]{\left\{\begin{array} #1 \end{array}\right.}
\newcommand\const{\text{\rmfamily\upshape const}}
\newcommand\toP{ \overset{\pro}{\to}}
\newcommand\toPP{ \overset{\text{PP}}{\to}}
\newcommand{\oeq}{\mathrel{\text{\textcircled{$=$}}}}





\usepackage{xcolor}
% \usepackage[normalem]{ulem}
\usepackage{lipsum}
\makeatletter
% \newcommand\colorwave[1][blue]{\bgroup \markoverwith{\lower3.5\p@\hbox{\sixly \textcolor{#1}{\char58}}}\ULon}
%\font\sixly=lasy6 % does not re-load if already loaded, so no memory problem.

\newmdtheoremenv[
linewidth= 1pt,linecolor= blue,%
leftmargin=20,rightmargin=20,innertopmargin=0pt, innerrightmargin=40,%
tikzsetting = { draw=lightgray, line width = 0.3pt,dashed,%
dash pattern = on 15pt off 3pt},%
splittopskip=\topskip,skipbelow=\baselineskip,%
skipabove=\baselineskip,ntheorem,roundcorner=0pt,
% backgroundcolor=pagebg,font=\color{orange}\sffamily, fontcolor=white
]{examplebox}{Exemple}[section]



\newcommand\R{\mathbb{R}}
\newcommand\Z{\mathbb{Z}}
\newcommand\N{\mathbb{N}}
\newcommand\E{\mathbb{E}}
\newcommand\F{\mathcal{F}}
\newcommand\cH{\mathcal{H}}
\newcommand\V{\mathbb{V}}
\newcommand\dmo{ ^{-1} }
\newcommand\kapa{\kappa}
\newcommand\im{Im}
\newcommand\hs{\mathcal{H}}





\usepackage{soul}

\makeatletter
\newcommand*{\whiten}[1]{\llap{\textcolor{white}{{\the\SOUL@token}}\hspace{#1pt}}}
\DeclareRobustCommand*\myul{%
    \def\SOUL@everyspace{\underline{\space}\kern\z@}%
    \def\SOUL@everytoken{%
     \setbox0=\hbox{\the\SOUL@token}%
     \ifdim\dp0>\z@
        \raisebox{\dp0}{\underline{\phantom{\the\SOUL@token}}}%
        \whiten{1}\whiten{0}%
        \whiten{-1}\whiten{-2}%
        \llap{\the\SOUL@token}%
     \else
        \underline{\the\SOUL@token}%
     \fi}%
\SOUL@}
\makeatother

\newcommand*{\demp}{\fontfamily{lmtt}\selectfont}

\DeclareTextFontCommand{\textdemp}{\demp}

\begin{document}

\ifcomment
Multiline
comment
\fi
\ifcomment
\myul{Typesetting test}
% \color[rgb]{1,1,1}
$∑_i^n≠ 60º±∞π∆¬≈√j∫h≤≥µ$

$\CR \R\pro\ind\pro\gS\pro
\mqty[a&b\\c&d]$
$\pro\mathbb{P}$
$\dd{x}$

  \[
    \alpha(x)=\left\{
                \begin{array}{ll}
                  x\\
                  \frac{1}{1+e^{-kx}}\\
                  \frac{e^x-e^{-x}}{e^x+e^{-x}}
                \end{array}
              \right.
  \]

  $\expval{x}$
  
  $\chi_\rho(ghg\dmo)=\Tr(\rho_{ghg\dmo})=\Tr(\rho_g\circ\rho_h\circ\rho\dmo_g)=\Tr(\rho_h)\overset{\mbox{\scalebox{0.5}{$\Tr(AB)=\Tr(BA)$}}}{=}\chi_\rho(h)$
  	$\mathop{\oplus}_{\substack{x\in X}}$

$\mat(\rho_g)=(a_{ij}(g))_{\scriptsize \substack{1\leq i\leq d \\ 1\leq j\leq d}}$ et $\mat(\rho'_g)=(a'_{ij}(g))_{\scriptsize \substack{1\leq i'\leq d' \\ 1\leq j'\leq d'}}$



\[\int_a^b{\mathbb{R}^2}g(u, v)\dd{P_{XY}}(u, v)=\iint g(u,v) f_{XY}(u, v)\dd \lambda(u) \dd \lambda(v)\]
$$\lim_{x\to\infty} f(x)$$	
$$\iiiint_V \mu(t,u,v,w) \,dt\,du\,dv\,dw$$
$$\sum_{n=1}^{\infty} 2^{-n} = 1$$	
\begin{definition}
	Si $X$ et $Y$ sont 2 v.a. ou definit la \textsc{Covariance} entre $X$ et $Y$ comme
	$\cov(X,Y)\overset{\text{def}}{=}\E\left[(X-\E(X))(Y-\E(Y))\right]=\E(XY)-\E(X)\E(Y)$.
\end{definition}
\fi
\pagebreak

% \tableofcontents

% insert your code here
%\input{./algebra/main.tex}
%\input{./geometrie-differentielle/main.tex}
%\input{./probabilite/main.tex}
%\input{./analyse-fonctionnelle/main.tex}
% \input{./Analyse-convexe-et-dualite-en-optimisation/main.tex}
%\input{./tikz/main.tex}
%\input{./Theorie-du-distributions/main.tex}
%\input{./optimisation/mine.tex}
 \input{./modelisation/main.tex}

% yves.aubry@univ-tln.fr : algebra

\end{document}

%% !TEX encoding = UTF-8 Unicode
% !TEX TS-program = xelatex

\documentclass[french]{report}

%\usepackage[utf8]{inputenc}
%\usepackage[T1]{fontenc}
\usepackage{babel}


\newif\ifcomment
%\commenttrue # Show comments

\usepackage{physics}
\usepackage{amssymb}


\usepackage{amsthm}
% \usepackage{thmtools}
\usepackage{mathtools}
\usepackage{amsfonts}

\usepackage{color}

\usepackage{tikz}

\usepackage{geometry}
\geometry{a5paper, margin=0.1in, right=1cm}

\usepackage{dsfont}

\usepackage{graphicx}
\graphicspath{ {images/} }

\usepackage{faktor}

\usepackage{IEEEtrantools}
\usepackage{enumerate}   
\usepackage[PostScript=dvips]{"/Users/aware/Documents/Courses/diagrams"}


\newtheorem{theorem}{Théorème}[section]
\renewcommand{\thetheorem}{\arabic{theorem}}
\newtheorem{lemme}{Lemme}[section]
\renewcommand{\thelemme}{\arabic{lemme}}
\newtheorem{proposition}{Proposition}[section]
\renewcommand{\theproposition}{\arabic{proposition}}
\newtheorem{notations}{Notations}[section]
\newtheorem{problem}{Problème}[section]
\newtheorem{corollary}{Corollaire}[theorem]
\renewcommand{\thecorollary}{\arabic{corollary}}
\newtheorem{property}{Propriété}[section]
\newtheorem{objective}{Objectif}[section]

\theoremstyle{definition}
\newtheorem{definition}{Définition}[section]
\renewcommand{\thedefinition}{\arabic{definition}}
\newtheorem{exercise}{Exercice}[chapter]
\renewcommand{\theexercise}{\arabic{exercise}}
\newtheorem{example}{Exemple}[chapter]
\renewcommand{\theexample}{\arabic{example}}
\newtheorem*{solution}{Solution}
\newtheorem*{application}{Application}
\newtheorem*{notation}{Notation}
\newtheorem*{vocabulary}{Vocabulaire}
\newtheorem*{properties}{Propriétés}



\theoremstyle{remark}
\newtheorem*{remark}{Remarque}
\newtheorem*{rappel}{Rappel}


\usepackage{etoolbox}
\AtBeginEnvironment{exercise}{\small}
\AtBeginEnvironment{example}{\small}

\usepackage{cases}
\usepackage[red]{mypack}

\usepackage[framemethod=TikZ]{mdframed}

\definecolor{bg}{rgb}{0.4,0.25,0.95}
\definecolor{pagebg}{rgb}{0,0,0.5}
\surroundwithmdframed[
   topline=false,
   rightline=false,
   bottomline=false,
   leftmargin=\parindent,
   skipabove=8pt,
   skipbelow=8pt,
   linecolor=blue,
   innerbottommargin=10pt,
   % backgroundcolor=bg,font=\color{orange}\sffamily, fontcolor=white
]{definition}

\usepackage{empheq}
\usepackage[most]{tcolorbox}

\newtcbox{\mymath}[1][]{%
    nobeforeafter, math upper, tcbox raise base,
    enhanced, colframe=blue!30!black,
    colback=red!10, boxrule=1pt,
    #1}

\usepackage{unixode}


\DeclareMathOperator{\ord}{ord}
\DeclareMathOperator{\orb}{orb}
\DeclareMathOperator{\stab}{stab}
\DeclareMathOperator{\Stab}{stab}
\DeclareMathOperator{\ppcm}{ppcm}
\DeclareMathOperator{\conj}{Conj}
\DeclareMathOperator{\End}{End}
\DeclareMathOperator{\rot}{rot}
\DeclareMathOperator{\trs}{trace}
\DeclareMathOperator{\Ind}{Ind}
\DeclareMathOperator{\mat}{Mat}
\DeclareMathOperator{\id}{Id}
\DeclareMathOperator{\vect}{vect}
\DeclareMathOperator{\img}{img}
\DeclareMathOperator{\cov}{Cov}
\DeclareMathOperator{\dist}{dist}
\DeclareMathOperator{\irr}{Irr}
\DeclareMathOperator{\image}{Im}
\DeclareMathOperator{\pd}{\partial}
\DeclareMathOperator{\epi}{epi}
\DeclareMathOperator{\Argmin}{Argmin}
\DeclareMathOperator{\dom}{dom}
\DeclareMathOperator{\proj}{proj}
\DeclareMathOperator{\ctg}{ctg}
\DeclareMathOperator{\supp}{supp}
\DeclareMathOperator{\argmin}{argmin}
\DeclareMathOperator{\mult}{mult}
\DeclareMathOperator{\ch}{ch}
\DeclareMathOperator{\sh}{sh}
\DeclareMathOperator{\rang}{rang}
\DeclareMathOperator{\diam}{diam}
\DeclareMathOperator{\Epigraphe}{Epigraphe}




\usepackage{xcolor}
\everymath{\color{blue}}
%\everymath{\color[rgb]{0,1,1}}
%\pagecolor[rgb]{0,0,0.5}


\newcommand*{\pdtest}[3][]{\ensuremath{\frac{\partial^{#1} #2}{\partial #3}}}

\newcommand*{\deffunc}[6][]{\ensuremath{
\begin{array}{rcl}
#2 : #3 &\rightarrow& #4\\
#5 &\mapsto& #6
\end{array}
}}

\newcommand{\eqcolon}{\mathrel{\resizebox{\widthof{$\mathord{=}$}}{\height}{ $\!\!=\!\!\resizebox{1.2\width}{0.8\height}{\raisebox{0.23ex}{$\mathop{:}$}}\!\!$ }}}
\newcommand{\coloneq}{\mathrel{\resizebox{\widthof{$\mathord{=}$}}{\height}{ $\!\!\resizebox{1.2\width}{0.8\height}{\raisebox{0.23ex}{$\mathop{:}$}}\!\!=\!\!$ }}}
\newcommand{\eqcolonl}{\ensuremath{\mathrel{=\!\!\mathop{:}}}}
\newcommand{\coloneql}{\ensuremath{\mathrel{\mathop{:} \!\! =}}}
\newcommand{\vc}[1]{% inline column vector
  \left(\begin{smallmatrix}#1\end{smallmatrix}\right)%
}
\newcommand{\vr}[1]{% inline row vector
  \begin{smallmatrix}(\,#1\,)\end{smallmatrix}%
}
\makeatletter
\newcommand*{\defeq}{\ =\mathrel{\rlap{%
                     \raisebox{0.3ex}{$\m@th\cdot$}}%
                     \raisebox{-0.3ex}{$\m@th\cdot$}}%
                     }
\makeatother

\newcommand{\mathcircle}[1]{% inline row vector
 \overset{\circ}{#1}
}
\newcommand{\ulim}{% low limit
 \underline{\lim}
}
\newcommand{\ssi}{% iff
\iff
}
\newcommand{\ps}[2]{
\expval{#1 | #2}
}
\newcommand{\df}[1]{
\mqty{#1}
}
\newcommand{\n}[1]{
\norm{#1}
}
\newcommand{\sys}[1]{
\left\{\smqty{#1}\right.
}


\newcommand{\eqdef}{\ensuremath{\overset{\text{def}}=}}


\def\Circlearrowright{\ensuremath{%
  \rotatebox[origin=c]{230}{$\circlearrowright$}}}

\newcommand\ct[1]{\text{\rmfamily\upshape #1}}
\newcommand\question[1]{ {\color{red} ...!? \small #1}}
\newcommand\caz[1]{\left\{\begin{array} #1 \end{array}\right.}
\newcommand\const{\text{\rmfamily\upshape const}}
\newcommand\toP{ \overset{\pro}{\to}}
\newcommand\toPP{ \overset{\text{PP}}{\to}}
\newcommand{\oeq}{\mathrel{\text{\textcircled{$=$}}}}





\usepackage{xcolor}
% \usepackage[normalem]{ulem}
\usepackage{lipsum}
\makeatletter
% \newcommand\colorwave[1][blue]{\bgroup \markoverwith{\lower3.5\p@\hbox{\sixly \textcolor{#1}{\char58}}}\ULon}
%\font\sixly=lasy6 % does not re-load if already loaded, so no memory problem.

\newmdtheoremenv[
linewidth= 1pt,linecolor= blue,%
leftmargin=20,rightmargin=20,innertopmargin=0pt, innerrightmargin=40,%
tikzsetting = { draw=lightgray, line width = 0.3pt,dashed,%
dash pattern = on 15pt off 3pt},%
splittopskip=\topskip,skipbelow=\baselineskip,%
skipabove=\baselineskip,ntheorem,roundcorner=0pt,
% backgroundcolor=pagebg,font=\color{orange}\sffamily, fontcolor=white
]{examplebox}{Exemple}[section]



\newcommand\R{\mathbb{R}}
\newcommand\Z{\mathbb{Z}}
\newcommand\N{\mathbb{N}}
\newcommand\E{\mathbb{E}}
\newcommand\F{\mathcal{F}}
\newcommand\cH{\mathcal{H}}
\newcommand\V{\mathbb{V}}
\newcommand\dmo{ ^{-1} }
\newcommand\kapa{\kappa}
\newcommand\im{Im}
\newcommand\hs{\mathcal{H}}





\usepackage{soul}

\makeatletter
\newcommand*{\whiten}[1]{\llap{\textcolor{white}{{\the\SOUL@token}}\hspace{#1pt}}}
\DeclareRobustCommand*\myul{%
    \def\SOUL@everyspace{\underline{\space}\kern\z@}%
    \def\SOUL@everytoken{%
     \setbox0=\hbox{\the\SOUL@token}%
     \ifdim\dp0>\z@
        \raisebox{\dp0}{\underline{\phantom{\the\SOUL@token}}}%
        \whiten{1}\whiten{0}%
        \whiten{-1}\whiten{-2}%
        \llap{\the\SOUL@token}%
     \else
        \underline{\the\SOUL@token}%
     \fi}%
\SOUL@}
\makeatother

\newcommand*{\demp}{\fontfamily{lmtt}\selectfont}

\DeclareTextFontCommand{\textdemp}{\demp}

\begin{document}

\ifcomment
Multiline
comment
\fi
\ifcomment
\myul{Typesetting test}
% \color[rgb]{1,1,1}
$∑_i^n≠ 60º±∞π∆¬≈√j∫h≤≥µ$

$\CR \R\pro\ind\pro\gS\pro
\mqty[a&b\\c&d]$
$\pro\mathbb{P}$
$\dd{x}$

  \[
    \alpha(x)=\left\{
                \begin{array}{ll}
                  x\\
                  \frac{1}{1+e^{-kx}}\\
                  \frac{e^x-e^{-x}}{e^x+e^{-x}}
                \end{array}
              \right.
  \]

  $\expval{x}$
  
  $\chi_\rho(ghg\dmo)=\Tr(\rho_{ghg\dmo})=\Tr(\rho_g\circ\rho_h\circ\rho\dmo_g)=\Tr(\rho_h)\overset{\mbox{\scalebox{0.5}{$\Tr(AB)=\Tr(BA)$}}}{=}\chi_\rho(h)$
  	$\mathop{\oplus}_{\substack{x\in X}}$

$\mat(\rho_g)=(a_{ij}(g))_{\scriptsize \substack{1\leq i\leq d \\ 1\leq j\leq d}}$ et $\mat(\rho'_g)=(a'_{ij}(g))_{\scriptsize \substack{1\leq i'\leq d' \\ 1\leq j'\leq d'}}$



\[\int_a^b{\mathbb{R}^2}g(u, v)\dd{P_{XY}}(u, v)=\iint g(u,v) f_{XY}(u, v)\dd \lambda(u) \dd \lambda(v)\]
$$\lim_{x\to\infty} f(x)$$	
$$\iiiint_V \mu(t,u,v,w) \,dt\,du\,dv\,dw$$
$$\sum_{n=1}^{\infty} 2^{-n} = 1$$	
\begin{definition}
	Si $X$ et $Y$ sont 2 v.a. ou definit la \textsc{Covariance} entre $X$ et $Y$ comme
	$\cov(X,Y)\overset{\text{def}}{=}\E\left[(X-\E(X))(Y-\E(Y))\right]=\E(XY)-\E(X)\E(Y)$.
\end{definition}
\fi
\pagebreak

% \tableofcontents

% insert your code here
%\input{./algebra/main.tex}
%\input{./geometrie-differentielle/main.tex}
%\input{./probabilite/main.tex}
%\input{./analyse-fonctionnelle/main.tex}
% \input{./Analyse-convexe-et-dualite-en-optimisation/main.tex}
%\input{./tikz/main.tex}
%\input{./Theorie-du-distributions/main.tex}
%\input{./optimisation/mine.tex}
 \input{./modelisation/main.tex}

% yves.aubry@univ-tln.fr : algebra

\end{document}

%\input{./optimisation/mine.tex}
 % !TEX encoding = UTF-8 Unicode
% !TEX TS-program = xelatex

\documentclass[french]{report}

%\usepackage[utf8]{inputenc}
%\usepackage[T1]{fontenc}
\usepackage{babel}


\newif\ifcomment
%\commenttrue # Show comments

\usepackage{physics}
\usepackage{amssymb}


\usepackage{amsthm}
% \usepackage{thmtools}
\usepackage{mathtools}
\usepackage{amsfonts}

\usepackage{color}

\usepackage{tikz}

\usepackage{geometry}
\geometry{a5paper, margin=0.1in, right=1cm}

\usepackage{dsfont}

\usepackage{graphicx}
\graphicspath{ {images/} }

\usepackage{faktor}

\usepackage{IEEEtrantools}
\usepackage{enumerate}   
\usepackage[PostScript=dvips]{"/Users/aware/Documents/Courses/diagrams"}


\newtheorem{theorem}{Théorème}[section]
\renewcommand{\thetheorem}{\arabic{theorem}}
\newtheorem{lemme}{Lemme}[section]
\renewcommand{\thelemme}{\arabic{lemme}}
\newtheorem{proposition}{Proposition}[section]
\renewcommand{\theproposition}{\arabic{proposition}}
\newtheorem{notations}{Notations}[section]
\newtheorem{problem}{Problème}[section]
\newtheorem{corollary}{Corollaire}[theorem]
\renewcommand{\thecorollary}{\arabic{corollary}}
\newtheorem{property}{Propriété}[section]
\newtheorem{objective}{Objectif}[section]

\theoremstyle{definition}
\newtheorem{definition}{Définition}[section]
\renewcommand{\thedefinition}{\arabic{definition}}
\newtheorem{exercise}{Exercice}[chapter]
\renewcommand{\theexercise}{\arabic{exercise}}
\newtheorem{example}{Exemple}[chapter]
\renewcommand{\theexample}{\arabic{example}}
\newtheorem*{solution}{Solution}
\newtheorem*{application}{Application}
\newtheorem*{notation}{Notation}
\newtheorem*{vocabulary}{Vocabulaire}
\newtheorem*{properties}{Propriétés}



\theoremstyle{remark}
\newtheorem*{remark}{Remarque}
\newtheorem*{rappel}{Rappel}


\usepackage{etoolbox}
\AtBeginEnvironment{exercise}{\small}
\AtBeginEnvironment{example}{\small}

\usepackage{cases}
\usepackage[red]{mypack}

\usepackage[framemethod=TikZ]{mdframed}

\definecolor{bg}{rgb}{0.4,0.25,0.95}
\definecolor{pagebg}{rgb}{0,0,0.5}
\surroundwithmdframed[
   topline=false,
   rightline=false,
   bottomline=false,
   leftmargin=\parindent,
   skipabove=8pt,
   skipbelow=8pt,
   linecolor=blue,
   innerbottommargin=10pt,
   % backgroundcolor=bg,font=\color{orange}\sffamily, fontcolor=white
]{definition}

\usepackage{empheq}
\usepackage[most]{tcolorbox}

\newtcbox{\mymath}[1][]{%
    nobeforeafter, math upper, tcbox raise base,
    enhanced, colframe=blue!30!black,
    colback=red!10, boxrule=1pt,
    #1}

\usepackage{unixode}


\DeclareMathOperator{\ord}{ord}
\DeclareMathOperator{\orb}{orb}
\DeclareMathOperator{\stab}{stab}
\DeclareMathOperator{\Stab}{stab}
\DeclareMathOperator{\ppcm}{ppcm}
\DeclareMathOperator{\conj}{Conj}
\DeclareMathOperator{\End}{End}
\DeclareMathOperator{\rot}{rot}
\DeclareMathOperator{\trs}{trace}
\DeclareMathOperator{\Ind}{Ind}
\DeclareMathOperator{\mat}{Mat}
\DeclareMathOperator{\id}{Id}
\DeclareMathOperator{\vect}{vect}
\DeclareMathOperator{\img}{img}
\DeclareMathOperator{\cov}{Cov}
\DeclareMathOperator{\dist}{dist}
\DeclareMathOperator{\irr}{Irr}
\DeclareMathOperator{\image}{Im}
\DeclareMathOperator{\pd}{\partial}
\DeclareMathOperator{\epi}{epi}
\DeclareMathOperator{\Argmin}{Argmin}
\DeclareMathOperator{\dom}{dom}
\DeclareMathOperator{\proj}{proj}
\DeclareMathOperator{\ctg}{ctg}
\DeclareMathOperator{\supp}{supp}
\DeclareMathOperator{\argmin}{argmin}
\DeclareMathOperator{\mult}{mult}
\DeclareMathOperator{\ch}{ch}
\DeclareMathOperator{\sh}{sh}
\DeclareMathOperator{\rang}{rang}
\DeclareMathOperator{\diam}{diam}
\DeclareMathOperator{\Epigraphe}{Epigraphe}




\usepackage{xcolor}
\everymath{\color{blue}}
%\everymath{\color[rgb]{0,1,1}}
%\pagecolor[rgb]{0,0,0.5}


\newcommand*{\pdtest}[3][]{\ensuremath{\frac{\partial^{#1} #2}{\partial #3}}}

\newcommand*{\deffunc}[6][]{\ensuremath{
\begin{array}{rcl}
#2 : #3 &\rightarrow& #4\\
#5 &\mapsto& #6
\end{array}
}}

\newcommand{\eqcolon}{\mathrel{\resizebox{\widthof{$\mathord{=}$}}{\height}{ $\!\!=\!\!\resizebox{1.2\width}{0.8\height}{\raisebox{0.23ex}{$\mathop{:}$}}\!\!$ }}}
\newcommand{\coloneq}{\mathrel{\resizebox{\widthof{$\mathord{=}$}}{\height}{ $\!\!\resizebox{1.2\width}{0.8\height}{\raisebox{0.23ex}{$\mathop{:}$}}\!\!=\!\!$ }}}
\newcommand{\eqcolonl}{\ensuremath{\mathrel{=\!\!\mathop{:}}}}
\newcommand{\coloneql}{\ensuremath{\mathrel{\mathop{:} \!\! =}}}
\newcommand{\vc}[1]{% inline column vector
  \left(\begin{smallmatrix}#1\end{smallmatrix}\right)%
}
\newcommand{\vr}[1]{% inline row vector
  \begin{smallmatrix}(\,#1\,)\end{smallmatrix}%
}
\makeatletter
\newcommand*{\defeq}{\ =\mathrel{\rlap{%
                     \raisebox{0.3ex}{$\m@th\cdot$}}%
                     \raisebox{-0.3ex}{$\m@th\cdot$}}%
                     }
\makeatother

\newcommand{\mathcircle}[1]{% inline row vector
 \overset{\circ}{#1}
}
\newcommand{\ulim}{% low limit
 \underline{\lim}
}
\newcommand{\ssi}{% iff
\iff
}
\newcommand{\ps}[2]{
\expval{#1 | #2}
}
\newcommand{\df}[1]{
\mqty{#1}
}
\newcommand{\n}[1]{
\norm{#1}
}
\newcommand{\sys}[1]{
\left\{\smqty{#1}\right.
}


\newcommand{\eqdef}{\ensuremath{\overset{\text{def}}=}}


\def\Circlearrowright{\ensuremath{%
  \rotatebox[origin=c]{230}{$\circlearrowright$}}}

\newcommand\ct[1]{\text{\rmfamily\upshape #1}}
\newcommand\question[1]{ {\color{red} ...!? \small #1}}
\newcommand\caz[1]{\left\{\begin{array} #1 \end{array}\right.}
\newcommand\const{\text{\rmfamily\upshape const}}
\newcommand\toP{ \overset{\pro}{\to}}
\newcommand\toPP{ \overset{\text{PP}}{\to}}
\newcommand{\oeq}{\mathrel{\text{\textcircled{$=$}}}}





\usepackage{xcolor}
% \usepackage[normalem]{ulem}
\usepackage{lipsum}
\makeatletter
% \newcommand\colorwave[1][blue]{\bgroup \markoverwith{\lower3.5\p@\hbox{\sixly \textcolor{#1}{\char58}}}\ULon}
%\font\sixly=lasy6 % does not re-load if already loaded, so no memory problem.

\newmdtheoremenv[
linewidth= 1pt,linecolor= blue,%
leftmargin=20,rightmargin=20,innertopmargin=0pt, innerrightmargin=40,%
tikzsetting = { draw=lightgray, line width = 0.3pt,dashed,%
dash pattern = on 15pt off 3pt},%
splittopskip=\topskip,skipbelow=\baselineskip,%
skipabove=\baselineskip,ntheorem,roundcorner=0pt,
% backgroundcolor=pagebg,font=\color{orange}\sffamily, fontcolor=white
]{examplebox}{Exemple}[section]



\newcommand\R{\mathbb{R}}
\newcommand\Z{\mathbb{Z}}
\newcommand\N{\mathbb{N}}
\newcommand\E{\mathbb{E}}
\newcommand\F{\mathcal{F}}
\newcommand\cH{\mathcal{H}}
\newcommand\V{\mathbb{V}}
\newcommand\dmo{ ^{-1} }
\newcommand\kapa{\kappa}
\newcommand\im{Im}
\newcommand\hs{\mathcal{H}}





\usepackage{soul}

\makeatletter
\newcommand*{\whiten}[1]{\llap{\textcolor{white}{{\the\SOUL@token}}\hspace{#1pt}}}
\DeclareRobustCommand*\myul{%
    \def\SOUL@everyspace{\underline{\space}\kern\z@}%
    \def\SOUL@everytoken{%
     \setbox0=\hbox{\the\SOUL@token}%
     \ifdim\dp0>\z@
        \raisebox{\dp0}{\underline{\phantom{\the\SOUL@token}}}%
        \whiten{1}\whiten{0}%
        \whiten{-1}\whiten{-2}%
        \llap{\the\SOUL@token}%
     \else
        \underline{\the\SOUL@token}%
     \fi}%
\SOUL@}
\makeatother

\newcommand*{\demp}{\fontfamily{lmtt}\selectfont}

\DeclareTextFontCommand{\textdemp}{\demp}

\begin{document}

\ifcomment
Multiline
comment
\fi
\ifcomment
\myul{Typesetting test}
% \color[rgb]{1,1,1}
$∑_i^n≠ 60º±∞π∆¬≈√j∫h≤≥µ$

$\CR \R\pro\ind\pro\gS\pro
\mqty[a&b\\c&d]$
$\pro\mathbb{P}$
$\dd{x}$

  \[
    \alpha(x)=\left\{
                \begin{array}{ll}
                  x\\
                  \frac{1}{1+e^{-kx}}\\
                  \frac{e^x-e^{-x}}{e^x+e^{-x}}
                \end{array}
              \right.
  \]

  $\expval{x}$
  
  $\chi_\rho(ghg\dmo)=\Tr(\rho_{ghg\dmo})=\Tr(\rho_g\circ\rho_h\circ\rho\dmo_g)=\Tr(\rho_h)\overset{\mbox{\scalebox{0.5}{$\Tr(AB)=\Tr(BA)$}}}{=}\chi_\rho(h)$
  	$\mathop{\oplus}_{\substack{x\in X}}$

$\mat(\rho_g)=(a_{ij}(g))_{\scriptsize \substack{1\leq i\leq d \\ 1\leq j\leq d}}$ et $\mat(\rho'_g)=(a'_{ij}(g))_{\scriptsize \substack{1\leq i'\leq d' \\ 1\leq j'\leq d'}}$



\[\int_a^b{\mathbb{R}^2}g(u, v)\dd{P_{XY}}(u, v)=\iint g(u,v) f_{XY}(u, v)\dd \lambda(u) \dd \lambda(v)\]
$$\lim_{x\to\infty} f(x)$$	
$$\iiiint_V \mu(t,u,v,w) \,dt\,du\,dv\,dw$$
$$\sum_{n=1}^{\infty} 2^{-n} = 1$$	
\begin{definition}
	Si $X$ et $Y$ sont 2 v.a. ou definit la \textsc{Covariance} entre $X$ et $Y$ comme
	$\cov(X,Y)\overset{\text{def}}{=}\E\left[(X-\E(X))(Y-\E(Y))\right]=\E(XY)-\E(X)\E(Y)$.
\end{definition}
\fi
\pagebreak

% \tableofcontents

% insert your code here
%\input{./algebra/main.tex}
%\input{./geometrie-differentielle/main.tex}
%\input{./probabilite/main.tex}
%\input{./analyse-fonctionnelle/main.tex}
% \input{./Analyse-convexe-et-dualite-en-optimisation/main.tex}
%\input{./tikz/main.tex}
%\input{./Theorie-du-distributions/main.tex}
%\input{./optimisation/mine.tex}
 \input{./modelisation/main.tex}

% yves.aubry@univ-tln.fr : algebra

\end{document}


% yves.aubry@univ-tln.fr : algebra

\end{document}

% % !TEX encoding = UTF-8 Unicode
% !TEX TS-program = xelatex

\documentclass[french]{report}

%\usepackage[utf8]{inputenc}
%\usepackage[T1]{fontenc}
\usepackage{babel}


\newif\ifcomment
%\commenttrue # Show comments

\usepackage{physics}
\usepackage{amssymb}


\usepackage{amsthm}
% \usepackage{thmtools}
\usepackage{mathtools}
\usepackage{amsfonts}

\usepackage{color}

\usepackage{tikz}

\usepackage{geometry}
\geometry{a5paper, margin=0.1in, right=1cm}

\usepackage{dsfont}

\usepackage{graphicx}
\graphicspath{ {images/} }

\usepackage{faktor}

\usepackage{IEEEtrantools}
\usepackage{enumerate}   
\usepackage[PostScript=dvips]{"/Users/aware/Documents/Courses/diagrams"}


\newtheorem{theorem}{Théorème}[section]
\renewcommand{\thetheorem}{\arabic{theorem}}
\newtheorem{lemme}{Lemme}[section]
\renewcommand{\thelemme}{\arabic{lemme}}
\newtheorem{proposition}{Proposition}[section]
\renewcommand{\theproposition}{\arabic{proposition}}
\newtheorem{notations}{Notations}[section]
\newtheorem{problem}{Problème}[section]
\newtheorem{corollary}{Corollaire}[theorem]
\renewcommand{\thecorollary}{\arabic{corollary}}
\newtheorem{property}{Propriété}[section]
\newtheorem{objective}{Objectif}[section]

\theoremstyle{definition}
\newtheorem{definition}{Définition}[section]
\renewcommand{\thedefinition}{\arabic{definition}}
\newtheorem{exercise}{Exercice}[chapter]
\renewcommand{\theexercise}{\arabic{exercise}}
\newtheorem{example}{Exemple}[chapter]
\renewcommand{\theexample}{\arabic{example}}
\newtheorem*{solution}{Solution}
\newtheorem*{application}{Application}
\newtheorem*{notation}{Notation}
\newtheorem*{vocabulary}{Vocabulaire}
\newtheorem*{properties}{Propriétés}



\theoremstyle{remark}
\newtheorem*{remark}{Remarque}
\newtheorem*{rappel}{Rappel}


\usepackage{etoolbox}
\AtBeginEnvironment{exercise}{\small}
\AtBeginEnvironment{example}{\small}

\usepackage{cases}
\usepackage[red]{mypack}

\usepackage[framemethod=TikZ]{mdframed}

\definecolor{bg}{rgb}{0.4,0.25,0.95}
\definecolor{pagebg}{rgb}{0,0,0.5}
\surroundwithmdframed[
   topline=false,
   rightline=false,
   bottomline=false,
   leftmargin=\parindent,
   skipabove=8pt,
   skipbelow=8pt,
   linecolor=blue,
   innerbottommargin=10pt,
   % backgroundcolor=bg,font=\color{orange}\sffamily, fontcolor=white
]{definition}

\usepackage{empheq}
\usepackage[most]{tcolorbox}

\newtcbox{\mymath}[1][]{%
    nobeforeafter, math upper, tcbox raise base,
    enhanced, colframe=blue!30!black,
    colback=red!10, boxrule=1pt,
    #1}

\usepackage{unixode}


\DeclareMathOperator{\ord}{ord}
\DeclareMathOperator{\orb}{orb}
\DeclareMathOperator{\stab}{stab}
\DeclareMathOperator{\Stab}{stab}
\DeclareMathOperator{\ppcm}{ppcm}
\DeclareMathOperator{\conj}{Conj}
\DeclareMathOperator{\End}{End}
\DeclareMathOperator{\rot}{rot}
\DeclareMathOperator{\trs}{trace}
\DeclareMathOperator{\Ind}{Ind}
\DeclareMathOperator{\mat}{Mat}
\DeclareMathOperator{\id}{Id}
\DeclareMathOperator{\vect}{vect}
\DeclareMathOperator{\img}{img}
\DeclareMathOperator{\cov}{Cov}
\DeclareMathOperator{\dist}{dist}
\DeclareMathOperator{\irr}{Irr}
\DeclareMathOperator{\image}{Im}
\DeclareMathOperator{\pd}{\partial}
\DeclareMathOperator{\epi}{epi}
\DeclareMathOperator{\Argmin}{Argmin}
\DeclareMathOperator{\dom}{dom}
\DeclareMathOperator{\proj}{proj}
\DeclareMathOperator{\ctg}{ctg}
\DeclareMathOperator{\supp}{supp}
\DeclareMathOperator{\argmin}{argmin}
\DeclareMathOperator{\mult}{mult}
\DeclareMathOperator{\ch}{ch}
\DeclareMathOperator{\sh}{sh}
\DeclareMathOperator{\rang}{rang}
\DeclareMathOperator{\diam}{diam}
\DeclareMathOperator{\Epigraphe}{Epigraphe}




\usepackage{xcolor}
\everymath{\color{blue}}
%\everymath{\color[rgb]{0,1,1}}
%\pagecolor[rgb]{0,0,0.5}


\newcommand*{\pdtest}[3][]{\ensuremath{\frac{\partial^{#1} #2}{\partial #3}}}

\newcommand*{\deffunc}[6][]{\ensuremath{
\begin{array}{rcl}
#2 : #3 &\rightarrow& #4\\
#5 &\mapsto& #6
\end{array}
}}

\newcommand{\eqcolon}{\mathrel{\resizebox{\widthof{$\mathord{=}$}}{\height}{ $\!\!=\!\!\resizebox{1.2\width}{0.8\height}{\raisebox{0.23ex}{$\mathop{:}$}}\!\!$ }}}
\newcommand{\coloneq}{\mathrel{\resizebox{\widthof{$\mathord{=}$}}{\height}{ $\!\!\resizebox{1.2\width}{0.8\height}{\raisebox{0.23ex}{$\mathop{:}$}}\!\!=\!\!$ }}}
\newcommand{\eqcolonl}{\ensuremath{\mathrel{=\!\!\mathop{:}}}}
\newcommand{\coloneql}{\ensuremath{\mathrel{\mathop{:} \!\! =}}}
\newcommand{\vc}[1]{% inline column vector
  \left(\begin{smallmatrix}#1\end{smallmatrix}\right)%
}
\newcommand{\vr}[1]{% inline row vector
  \begin{smallmatrix}(\,#1\,)\end{smallmatrix}%
}
\makeatletter
\newcommand*{\defeq}{\ =\mathrel{\rlap{%
                     \raisebox{0.3ex}{$\m@th\cdot$}}%
                     \raisebox{-0.3ex}{$\m@th\cdot$}}%
                     }
\makeatother

\newcommand{\mathcircle}[1]{% inline row vector
 \overset{\circ}{#1}
}
\newcommand{\ulim}{% low limit
 \underline{\lim}
}
\newcommand{\ssi}{% iff
\iff
}
\newcommand{\ps}[2]{
\expval{#1 | #2}
}
\newcommand{\df}[1]{
\mqty{#1}
}
\newcommand{\n}[1]{
\norm{#1}
}
\newcommand{\sys}[1]{
\left\{\smqty{#1}\right.
}


\newcommand{\eqdef}{\ensuremath{\overset{\text{def}}=}}


\def\Circlearrowright{\ensuremath{%
  \rotatebox[origin=c]{230}{$\circlearrowright$}}}

\newcommand\ct[1]{\text{\rmfamily\upshape #1}}
\newcommand\question[1]{ {\color{red} ...!? \small #1}}
\newcommand\caz[1]{\left\{\begin{array} #1 \end{array}\right.}
\newcommand\const{\text{\rmfamily\upshape const}}
\newcommand\toP{ \overset{\pro}{\to}}
\newcommand\toPP{ \overset{\text{PP}}{\to}}
\newcommand{\oeq}{\mathrel{\text{\textcircled{$=$}}}}





\usepackage{xcolor}
% \usepackage[normalem]{ulem}
\usepackage{lipsum}
\makeatletter
% \newcommand\colorwave[1][blue]{\bgroup \markoverwith{\lower3.5\p@\hbox{\sixly \textcolor{#1}{\char58}}}\ULon}
%\font\sixly=lasy6 % does not re-load if already loaded, so no memory problem.

\newmdtheoremenv[
linewidth= 1pt,linecolor= blue,%
leftmargin=20,rightmargin=20,innertopmargin=0pt, innerrightmargin=40,%
tikzsetting = { draw=lightgray, line width = 0.3pt,dashed,%
dash pattern = on 15pt off 3pt},%
splittopskip=\topskip,skipbelow=\baselineskip,%
skipabove=\baselineskip,ntheorem,roundcorner=0pt,
% backgroundcolor=pagebg,font=\color{orange}\sffamily, fontcolor=white
]{examplebox}{Exemple}[section]



\newcommand\R{\mathbb{R}}
\newcommand\Z{\mathbb{Z}}
\newcommand\N{\mathbb{N}}
\newcommand\E{\mathbb{E}}
\newcommand\F{\mathcal{F}}
\newcommand\cH{\mathcal{H}}
\newcommand\V{\mathbb{V}}
\newcommand\dmo{ ^{-1} }
\newcommand\kapa{\kappa}
\newcommand\im{Im}
\newcommand\hs{\mathcal{H}}





\usepackage{soul}

\makeatletter
\newcommand*{\whiten}[1]{\llap{\textcolor{white}{{\the\SOUL@token}}\hspace{#1pt}}}
\DeclareRobustCommand*\myul{%
    \def\SOUL@everyspace{\underline{\space}\kern\z@}%
    \def\SOUL@everytoken{%
     \setbox0=\hbox{\the\SOUL@token}%
     \ifdim\dp0>\z@
        \raisebox{\dp0}{\underline{\phantom{\the\SOUL@token}}}%
        \whiten{1}\whiten{0}%
        \whiten{-1}\whiten{-2}%
        \llap{\the\SOUL@token}%
     \else
        \underline{\the\SOUL@token}%
     \fi}%
\SOUL@}
\makeatother

\newcommand*{\demp}{\fontfamily{lmtt}\selectfont}

\DeclareTextFontCommand{\textdemp}{\demp}

\begin{document}

\ifcomment
Multiline
comment
\fi
\ifcomment
\myul{Typesetting test}
% \color[rgb]{1,1,1}
$∑_i^n≠ 60º±∞π∆¬≈√j∫h≤≥µ$

$\CR \R\pro\ind\pro\gS\pro
\mqty[a&b\\c&d]$
$\pro\mathbb{P}$
$\dd{x}$

  \[
    \alpha(x)=\left\{
                \begin{array}{ll}
                  x\\
                  \frac{1}{1+e^{-kx}}\\
                  \frac{e^x-e^{-x}}{e^x+e^{-x}}
                \end{array}
              \right.
  \]

  $\expval{x}$
  
  $\chi_\rho(ghg\dmo)=\Tr(\rho_{ghg\dmo})=\Tr(\rho_g\circ\rho_h\circ\rho\dmo_g)=\Tr(\rho_h)\overset{\mbox{\scalebox{0.5}{$\Tr(AB)=\Tr(BA)$}}}{=}\chi_\rho(h)$
  	$\mathop{\oplus}_{\substack{x\in X}}$

$\mat(\rho_g)=(a_{ij}(g))_{\scriptsize \substack{1\leq i\leq d \\ 1\leq j\leq d}}$ et $\mat(\rho'_g)=(a'_{ij}(g))_{\scriptsize \substack{1\leq i'\leq d' \\ 1\leq j'\leq d'}}$



\[\int_a^b{\mathbb{R}^2}g(u, v)\dd{P_{XY}}(u, v)=\iint g(u,v) f_{XY}(u, v)\dd \lambda(u) \dd \lambda(v)\]
$$\lim_{x\to\infty} f(x)$$	
$$\iiiint_V \mu(t,u,v,w) \,dt\,du\,dv\,dw$$
$$\sum_{n=1}^{\infty} 2^{-n} = 1$$	
\begin{definition}
	Si $X$ et $Y$ sont 2 v.a. ou definit la \textsc{Covariance} entre $X$ et $Y$ comme
	$\cov(X,Y)\overset{\text{def}}{=}\E\left[(X-\E(X))(Y-\E(Y))\right]=\E(XY)-\E(X)\E(Y)$.
\end{definition}
\fi
\pagebreak

% \tableofcontents

% insert your code here
%% !TEX encoding = UTF-8 Unicode
% !TEX TS-program = xelatex

\documentclass[french]{report}

%\usepackage[utf8]{inputenc}
%\usepackage[T1]{fontenc}
\usepackage{babel}


\newif\ifcomment
%\commenttrue # Show comments

\usepackage{physics}
\usepackage{amssymb}


\usepackage{amsthm}
% \usepackage{thmtools}
\usepackage{mathtools}
\usepackage{amsfonts}

\usepackage{color}

\usepackage{tikz}

\usepackage{geometry}
\geometry{a5paper, margin=0.1in, right=1cm}

\usepackage{dsfont}

\usepackage{graphicx}
\graphicspath{ {images/} }

\usepackage{faktor}

\usepackage{IEEEtrantools}
\usepackage{enumerate}   
\usepackage[PostScript=dvips]{"/Users/aware/Documents/Courses/diagrams"}


\newtheorem{theorem}{Théorème}[section]
\renewcommand{\thetheorem}{\arabic{theorem}}
\newtheorem{lemme}{Lemme}[section]
\renewcommand{\thelemme}{\arabic{lemme}}
\newtheorem{proposition}{Proposition}[section]
\renewcommand{\theproposition}{\arabic{proposition}}
\newtheorem{notations}{Notations}[section]
\newtheorem{problem}{Problème}[section]
\newtheorem{corollary}{Corollaire}[theorem]
\renewcommand{\thecorollary}{\arabic{corollary}}
\newtheorem{property}{Propriété}[section]
\newtheorem{objective}{Objectif}[section]

\theoremstyle{definition}
\newtheorem{definition}{Définition}[section]
\renewcommand{\thedefinition}{\arabic{definition}}
\newtheorem{exercise}{Exercice}[chapter]
\renewcommand{\theexercise}{\arabic{exercise}}
\newtheorem{example}{Exemple}[chapter]
\renewcommand{\theexample}{\arabic{example}}
\newtheorem*{solution}{Solution}
\newtheorem*{application}{Application}
\newtheorem*{notation}{Notation}
\newtheorem*{vocabulary}{Vocabulaire}
\newtheorem*{properties}{Propriétés}



\theoremstyle{remark}
\newtheorem*{remark}{Remarque}
\newtheorem*{rappel}{Rappel}


\usepackage{etoolbox}
\AtBeginEnvironment{exercise}{\small}
\AtBeginEnvironment{example}{\small}

\usepackage{cases}
\usepackage[red]{mypack}

\usepackage[framemethod=TikZ]{mdframed}

\definecolor{bg}{rgb}{0.4,0.25,0.95}
\definecolor{pagebg}{rgb}{0,0,0.5}
\surroundwithmdframed[
   topline=false,
   rightline=false,
   bottomline=false,
   leftmargin=\parindent,
   skipabove=8pt,
   skipbelow=8pt,
   linecolor=blue,
   innerbottommargin=10pt,
   % backgroundcolor=bg,font=\color{orange}\sffamily, fontcolor=white
]{definition}

\usepackage{empheq}
\usepackage[most]{tcolorbox}

\newtcbox{\mymath}[1][]{%
    nobeforeafter, math upper, tcbox raise base,
    enhanced, colframe=blue!30!black,
    colback=red!10, boxrule=1pt,
    #1}

\usepackage{unixode}


\DeclareMathOperator{\ord}{ord}
\DeclareMathOperator{\orb}{orb}
\DeclareMathOperator{\stab}{stab}
\DeclareMathOperator{\Stab}{stab}
\DeclareMathOperator{\ppcm}{ppcm}
\DeclareMathOperator{\conj}{Conj}
\DeclareMathOperator{\End}{End}
\DeclareMathOperator{\rot}{rot}
\DeclareMathOperator{\trs}{trace}
\DeclareMathOperator{\Ind}{Ind}
\DeclareMathOperator{\mat}{Mat}
\DeclareMathOperator{\id}{Id}
\DeclareMathOperator{\vect}{vect}
\DeclareMathOperator{\img}{img}
\DeclareMathOperator{\cov}{Cov}
\DeclareMathOperator{\dist}{dist}
\DeclareMathOperator{\irr}{Irr}
\DeclareMathOperator{\image}{Im}
\DeclareMathOperator{\pd}{\partial}
\DeclareMathOperator{\epi}{epi}
\DeclareMathOperator{\Argmin}{Argmin}
\DeclareMathOperator{\dom}{dom}
\DeclareMathOperator{\proj}{proj}
\DeclareMathOperator{\ctg}{ctg}
\DeclareMathOperator{\supp}{supp}
\DeclareMathOperator{\argmin}{argmin}
\DeclareMathOperator{\mult}{mult}
\DeclareMathOperator{\ch}{ch}
\DeclareMathOperator{\sh}{sh}
\DeclareMathOperator{\rang}{rang}
\DeclareMathOperator{\diam}{diam}
\DeclareMathOperator{\Epigraphe}{Epigraphe}




\usepackage{xcolor}
\everymath{\color{blue}}
%\everymath{\color[rgb]{0,1,1}}
%\pagecolor[rgb]{0,0,0.5}


\newcommand*{\pdtest}[3][]{\ensuremath{\frac{\partial^{#1} #2}{\partial #3}}}

\newcommand*{\deffunc}[6][]{\ensuremath{
\begin{array}{rcl}
#2 : #3 &\rightarrow& #4\\
#5 &\mapsto& #6
\end{array}
}}

\newcommand{\eqcolon}{\mathrel{\resizebox{\widthof{$\mathord{=}$}}{\height}{ $\!\!=\!\!\resizebox{1.2\width}{0.8\height}{\raisebox{0.23ex}{$\mathop{:}$}}\!\!$ }}}
\newcommand{\coloneq}{\mathrel{\resizebox{\widthof{$\mathord{=}$}}{\height}{ $\!\!\resizebox{1.2\width}{0.8\height}{\raisebox{0.23ex}{$\mathop{:}$}}\!\!=\!\!$ }}}
\newcommand{\eqcolonl}{\ensuremath{\mathrel{=\!\!\mathop{:}}}}
\newcommand{\coloneql}{\ensuremath{\mathrel{\mathop{:} \!\! =}}}
\newcommand{\vc}[1]{% inline column vector
  \left(\begin{smallmatrix}#1\end{smallmatrix}\right)%
}
\newcommand{\vr}[1]{% inline row vector
  \begin{smallmatrix}(\,#1\,)\end{smallmatrix}%
}
\makeatletter
\newcommand*{\defeq}{\ =\mathrel{\rlap{%
                     \raisebox{0.3ex}{$\m@th\cdot$}}%
                     \raisebox{-0.3ex}{$\m@th\cdot$}}%
                     }
\makeatother

\newcommand{\mathcircle}[1]{% inline row vector
 \overset{\circ}{#1}
}
\newcommand{\ulim}{% low limit
 \underline{\lim}
}
\newcommand{\ssi}{% iff
\iff
}
\newcommand{\ps}[2]{
\expval{#1 | #2}
}
\newcommand{\df}[1]{
\mqty{#1}
}
\newcommand{\n}[1]{
\norm{#1}
}
\newcommand{\sys}[1]{
\left\{\smqty{#1}\right.
}


\newcommand{\eqdef}{\ensuremath{\overset{\text{def}}=}}


\def\Circlearrowright{\ensuremath{%
  \rotatebox[origin=c]{230}{$\circlearrowright$}}}

\newcommand\ct[1]{\text{\rmfamily\upshape #1}}
\newcommand\question[1]{ {\color{red} ...!? \small #1}}
\newcommand\caz[1]{\left\{\begin{array} #1 \end{array}\right.}
\newcommand\const{\text{\rmfamily\upshape const}}
\newcommand\toP{ \overset{\pro}{\to}}
\newcommand\toPP{ \overset{\text{PP}}{\to}}
\newcommand{\oeq}{\mathrel{\text{\textcircled{$=$}}}}





\usepackage{xcolor}
% \usepackage[normalem]{ulem}
\usepackage{lipsum}
\makeatletter
% \newcommand\colorwave[1][blue]{\bgroup \markoverwith{\lower3.5\p@\hbox{\sixly \textcolor{#1}{\char58}}}\ULon}
%\font\sixly=lasy6 % does not re-load if already loaded, so no memory problem.

\newmdtheoremenv[
linewidth= 1pt,linecolor= blue,%
leftmargin=20,rightmargin=20,innertopmargin=0pt, innerrightmargin=40,%
tikzsetting = { draw=lightgray, line width = 0.3pt,dashed,%
dash pattern = on 15pt off 3pt},%
splittopskip=\topskip,skipbelow=\baselineskip,%
skipabove=\baselineskip,ntheorem,roundcorner=0pt,
% backgroundcolor=pagebg,font=\color{orange}\sffamily, fontcolor=white
]{examplebox}{Exemple}[section]



\newcommand\R{\mathbb{R}}
\newcommand\Z{\mathbb{Z}}
\newcommand\N{\mathbb{N}}
\newcommand\E{\mathbb{E}}
\newcommand\F{\mathcal{F}}
\newcommand\cH{\mathcal{H}}
\newcommand\V{\mathbb{V}}
\newcommand\dmo{ ^{-1} }
\newcommand\kapa{\kappa}
\newcommand\im{Im}
\newcommand\hs{\mathcal{H}}





\usepackage{soul}

\makeatletter
\newcommand*{\whiten}[1]{\llap{\textcolor{white}{{\the\SOUL@token}}\hspace{#1pt}}}
\DeclareRobustCommand*\myul{%
    \def\SOUL@everyspace{\underline{\space}\kern\z@}%
    \def\SOUL@everytoken{%
     \setbox0=\hbox{\the\SOUL@token}%
     \ifdim\dp0>\z@
        \raisebox{\dp0}{\underline{\phantom{\the\SOUL@token}}}%
        \whiten{1}\whiten{0}%
        \whiten{-1}\whiten{-2}%
        \llap{\the\SOUL@token}%
     \else
        \underline{\the\SOUL@token}%
     \fi}%
\SOUL@}
\makeatother

\newcommand*{\demp}{\fontfamily{lmtt}\selectfont}

\DeclareTextFontCommand{\textdemp}{\demp}

\begin{document}

\ifcomment
Multiline
comment
\fi
\ifcomment
\myul{Typesetting test}
% \color[rgb]{1,1,1}
$∑_i^n≠ 60º±∞π∆¬≈√j∫h≤≥µ$

$\CR \R\pro\ind\pro\gS\pro
\mqty[a&b\\c&d]$
$\pro\mathbb{P}$
$\dd{x}$

  \[
    \alpha(x)=\left\{
                \begin{array}{ll}
                  x\\
                  \frac{1}{1+e^{-kx}}\\
                  \frac{e^x-e^{-x}}{e^x+e^{-x}}
                \end{array}
              \right.
  \]

  $\expval{x}$
  
  $\chi_\rho(ghg\dmo)=\Tr(\rho_{ghg\dmo})=\Tr(\rho_g\circ\rho_h\circ\rho\dmo_g)=\Tr(\rho_h)\overset{\mbox{\scalebox{0.5}{$\Tr(AB)=\Tr(BA)$}}}{=}\chi_\rho(h)$
  	$\mathop{\oplus}_{\substack{x\in X}}$

$\mat(\rho_g)=(a_{ij}(g))_{\scriptsize \substack{1\leq i\leq d \\ 1\leq j\leq d}}$ et $\mat(\rho'_g)=(a'_{ij}(g))_{\scriptsize \substack{1\leq i'\leq d' \\ 1\leq j'\leq d'}}$



\[\int_a^b{\mathbb{R}^2}g(u, v)\dd{P_{XY}}(u, v)=\iint g(u,v) f_{XY}(u, v)\dd \lambda(u) \dd \lambda(v)\]
$$\lim_{x\to\infty} f(x)$$	
$$\iiiint_V \mu(t,u,v,w) \,dt\,du\,dv\,dw$$
$$\sum_{n=1}^{\infty} 2^{-n} = 1$$	
\begin{definition}
	Si $X$ et $Y$ sont 2 v.a. ou definit la \textsc{Covariance} entre $X$ et $Y$ comme
	$\cov(X,Y)\overset{\text{def}}{=}\E\left[(X-\E(X))(Y-\E(Y))\right]=\E(XY)-\E(X)\E(Y)$.
\end{definition}
\fi
\pagebreak

% \tableofcontents

% insert your code here
%\input{./algebra/main.tex}
%\input{./geometrie-differentielle/main.tex}
%\input{./probabilite/main.tex}
%\input{./analyse-fonctionnelle/main.tex}
% \input{./Analyse-convexe-et-dualite-en-optimisation/main.tex}
%\input{./tikz/main.tex}
%\input{./Theorie-du-distributions/main.tex}
%\input{./optimisation/mine.tex}
 \input{./modelisation/main.tex}

% yves.aubry@univ-tln.fr : algebra

\end{document}

%% !TEX encoding = UTF-8 Unicode
% !TEX TS-program = xelatex

\documentclass[french]{report}

%\usepackage[utf8]{inputenc}
%\usepackage[T1]{fontenc}
\usepackage{babel}


\newif\ifcomment
%\commenttrue # Show comments

\usepackage{physics}
\usepackage{amssymb}


\usepackage{amsthm}
% \usepackage{thmtools}
\usepackage{mathtools}
\usepackage{amsfonts}

\usepackage{color}

\usepackage{tikz}

\usepackage{geometry}
\geometry{a5paper, margin=0.1in, right=1cm}

\usepackage{dsfont}

\usepackage{graphicx}
\graphicspath{ {images/} }

\usepackage{faktor}

\usepackage{IEEEtrantools}
\usepackage{enumerate}   
\usepackage[PostScript=dvips]{"/Users/aware/Documents/Courses/diagrams"}


\newtheorem{theorem}{Théorème}[section]
\renewcommand{\thetheorem}{\arabic{theorem}}
\newtheorem{lemme}{Lemme}[section]
\renewcommand{\thelemme}{\arabic{lemme}}
\newtheorem{proposition}{Proposition}[section]
\renewcommand{\theproposition}{\arabic{proposition}}
\newtheorem{notations}{Notations}[section]
\newtheorem{problem}{Problème}[section]
\newtheorem{corollary}{Corollaire}[theorem]
\renewcommand{\thecorollary}{\arabic{corollary}}
\newtheorem{property}{Propriété}[section]
\newtheorem{objective}{Objectif}[section]

\theoremstyle{definition}
\newtheorem{definition}{Définition}[section]
\renewcommand{\thedefinition}{\arabic{definition}}
\newtheorem{exercise}{Exercice}[chapter]
\renewcommand{\theexercise}{\arabic{exercise}}
\newtheorem{example}{Exemple}[chapter]
\renewcommand{\theexample}{\arabic{example}}
\newtheorem*{solution}{Solution}
\newtheorem*{application}{Application}
\newtheorem*{notation}{Notation}
\newtheorem*{vocabulary}{Vocabulaire}
\newtheorem*{properties}{Propriétés}



\theoremstyle{remark}
\newtheorem*{remark}{Remarque}
\newtheorem*{rappel}{Rappel}


\usepackage{etoolbox}
\AtBeginEnvironment{exercise}{\small}
\AtBeginEnvironment{example}{\small}

\usepackage{cases}
\usepackage[red]{mypack}

\usepackage[framemethod=TikZ]{mdframed}

\definecolor{bg}{rgb}{0.4,0.25,0.95}
\definecolor{pagebg}{rgb}{0,0,0.5}
\surroundwithmdframed[
   topline=false,
   rightline=false,
   bottomline=false,
   leftmargin=\parindent,
   skipabove=8pt,
   skipbelow=8pt,
   linecolor=blue,
   innerbottommargin=10pt,
   % backgroundcolor=bg,font=\color{orange}\sffamily, fontcolor=white
]{definition}

\usepackage{empheq}
\usepackage[most]{tcolorbox}

\newtcbox{\mymath}[1][]{%
    nobeforeafter, math upper, tcbox raise base,
    enhanced, colframe=blue!30!black,
    colback=red!10, boxrule=1pt,
    #1}

\usepackage{unixode}


\DeclareMathOperator{\ord}{ord}
\DeclareMathOperator{\orb}{orb}
\DeclareMathOperator{\stab}{stab}
\DeclareMathOperator{\Stab}{stab}
\DeclareMathOperator{\ppcm}{ppcm}
\DeclareMathOperator{\conj}{Conj}
\DeclareMathOperator{\End}{End}
\DeclareMathOperator{\rot}{rot}
\DeclareMathOperator{\trs}{trace}
\DeclareMathOperator{\Ind}{Ind}
\DeclareMathOperator{\mat}{Mat}
\DeclareMathOperator{\id}{Id}
\DeclareMathOperator{\vect}{vect}
\DeclareMathOperator{\img}{img}
\DeclareMathOperator{\cov}{Cov}
\DeclareMathOperator{\dist}{dist}
\DeclareMathOperator{\irr}{Irr}
\DeclareMathOperator{\image}{Im}
\DeclareMathOperator{\pd}{\partial}
\DeclareMathOperator{\epi}{epi}
\DeclareMathOperator{\Argmin}{Argmin}
\DeclareMathOperator{\dom}{dom}
\DeclareMathOperator{\proj}{proj}
\DeclareMathOperator{\ctg}{ctg}
\DeclareMathOperator{\supp}{supp}
\DeclareMathOperator{\argmin}{argmin}
\DeclareMathOperator{\mult}{mult}
\DeclareMathOperator{\ch}{ch}
\DeclareMathOperator{\sh}{sh}
\DeclareMathOperator{\rang}{rang}
\DeclareMathOperator{\diam}{diam}
\DeclareMathOperator{\Epigraphe}{Epigraphe}




\usepackage{xcolor}
\everymath{\color{blue}}
%\everymath{\color[rgb]{0,1,1}}
%\pagecolor[rgb]{0,0,0.5}


\newcommand*{\pdtest}[3][]{\ensuremath{\frac{\partial^{#1} #2}{\partial #3}}}

\newcommand*{\deffunc}[6][]{\ensuremath{
\begin{array}{rcl}
#2 : #3 &\rightarrow& #4\\
#5 &\mapsto& #6
\end{array}
}}

\newcommand{\eqcolon}{\mathrel{\resizebox{\widthof{$\mathord{=}$}}{\height}{ $\!\!=\!\!\resizebox{1.2\width}{0.8\height}{\raisebox{0.23ex}{$\mathop{:}$}}\!\!$ }}}
\newcommand{\coloneq}{\mathrel{\resizebox{\widthof{$\mathord{=}$}}{\height}{ $\!\!\resizebox{1.2\width}{0.8\height}{\raisebox{0.23ex}{$\mathop{:}$}}\!\!=\!\!$ }}}
\newcommand{\eqcolonl}{\ensuremath{\mathrel{=\!\!\mathop{:}}}}
\newcommand{\coloneql}{\ensuremath{\mathrel{\mathop{:} \!\! =}}}
\newcommand{\vc}[1]{% inline column vector
  \left(\begin{smallmatrix}#1\end{smallmatrix}\right)%
}
\newcommand{\vr}[1]{% inline row vector
  \begin{smallmatrix}(\,#1\,)\end{smallmatrix}%
}
\makeatletter
\newcommand*{\defeq}{\ =\mathrel{\rlap{%
                     \raisebox{0.3ex}{$\m@th\cdot$}}%
                     \raisebox{-0.3ex}{$\m@th\cdot$}}%
                     }
\makeatother

\newcommand{\mathcircle}[1]{% inline row vector
 \overset{\circ}{#1}
}
\newcommand{\ulim}{% low limit
 \underline{\lim}
}
\newcommand{\ssi}{% iff
\iff
}
\newcommand{\ps}[2]{
\expval{#1 | #2}
}
\newcommand{\df}[1]{
\mqty{#1}
}
\newcommand{\n}[1]{
\norm{#1}
}
\newcommand{\sys}[1]{
\left\{\smqty{#1}\right.
}


\newcommand{\eqdef}{\ensuremath{\overset{\text{def}}=}}


\def\Circlearrowright{\ensuremath{%
  \rotatebox[origin=c]{230}{$\circlearrowright$}}}

\newcommand\ct[1]{\text{\rmfamily\upshape #1}}
\newcommand\question[1]{ {\color{red} ...!? \small #1}}
\newcommand\caz[1]{\left\{\begin{array} #1 \end{array}\right.}
\newcommand\const{\text{\rmfamily\upshape const}}
\newcommand\toP{ \overset{\pro}{\to}}
\newcommand\toPP{ \overset{\text{PP}}{\to}}
\newcommand{\oeq}{\mathrel{\text{\textcircled{$=$}}}}





\usepackage{xcolor}
% \usepackage[normalem]{ulem}
\usepackage{lipsum}
\makeatletter
% \newcommand\colorwave[1][blue]{\bgroup \markoverwith{\lower3.5\p@\hbox{\sixly \textcolor{#1}{\char58}}}\ULon}
%\font\sixly=lasy6 % does not re-load if already loaded, so no memory problem.

\newmdtheoremenv[
linewidth= 1pt,linecolor= blue,%
leftmargin=20,rightmargin=20,innertopmargin=0pt, innerrightmargin=40,%
tikzsetting = { draw=lightgray, line width = 0.3pt,dashed,%
dash pattern = on 15pt off 3pt},%
splittopskip=\topskip,skipbelow=\baselineskip,%
skipabove=\baselineskip,ntheorem,roundcorner=0pt,
% backgroundcolor=pagebg,font=\color{orange}\sffamily, fontcolor=white
]{examplebox}{Exemple}[section]



\newcommand\R{\mathbb{R}}
\newcommand\Z{\mathbb{Z}}
\newcommand\N{\mathbb{N}}
\newcommand\E{\mathbb{E}}
\newcommand\F{\mathcal{F}}
\newcommand\cH{\mathcal{H}}
\newcommand\V{\mathbb{V}}
\newcommand\dmo{ ^{-1} }
\newcommand\kapa{\kappa}
\newcommand\im{Im}
\newcommand\hs{\mathcal{H}}





\usepackage{soul}

\makeatletter
\newcommand*{\whiten}[1]{\llap{\textcolor{white}{{\the\SOUL@token}}\hspace{#1pt}}}
\DeclareRobustCommand*\myul{%
    \def\SOUL@everyspace{\underline{\space}\kern\z@}%
    \def\SOUL@everytoken{%
     \setbox0=\hbox{\the\SOUL@token}%
     \ifdim\dp0>\z@
        \raisebox{\dp0}{\underline{\phantom{\the\SOUL@token}}}%
        \whiten{1}\whiten{0}%
        \whiten{-1}\whiten{-2}%
        \llap{\the\SOUL@token}%
     \else
        \underline{\the\SOUL@token}%
     \fi}%
\SOUL@}
\makeatother

\newcommand*{\demp}{\fontfamily{lmtt}\selectfont}

\DeclareTextFontCommand{\textdemp}{\demp}

\begin{document}

\ifcomment
Multiline
comment
\fi
\ifcomment
\myul{Typesetting test}
% \color[rgb]{1,1,1}
$∑_i^n≠ 60º±∞π∆¬≈√j∫h≤≥µ$

$\CR \R\pro\ind\pro\gS\pro
\mqty[a&b\\c&d]$
$\pro\mathbb{P}$
$\dd{x}$

  \[
    \alpha(x)=\left\{
                \begin{array}{ll}
                  x\\
                  \frac{1}{1+e^{-kx}}\\
                  \frac{e^x-e^{-x}}{e^x+e^{-x}}
                \end{array}
              \right.
  \]

  $\expval{x}$
  
  $\chi_\rho(ghg\dmo)=\Tr(\rho_{ghg\dmo})=\Tr(\rho_g\circ\rho_h\circ\rho\dmo_g)=\Tr(\rho_h)\overset{\mbox{\scalebox{0.5}{$\Tr(AB)=\Tr(BA)$}}}{=}\chi_\rho(h)$
  	$\mathop{\oplus}_{\substack{x\in X}}$

$\mat(\rho_g)=(a_{ij}(g))_{\scriptsize \substack{1\leq i\leq d \\ 1\leq j\leq d}}$ et $\mat(\rho'_g)=(a'_{ij}(g))_{\scriptsize \substack{1\leq i'\leq d' \\ 1\leq j'\leq d'}}$



\[\int_a^b{\mathbb{R}^2}g(u, v)\dd{P_{XY}}(u, v)=\iint g(u,v) f_{XY}(u, v)\dd \lambda(u) \dd \lambda(v)\]
$$\lim_{x\to\infty} f(x)$$	
$$\iiiint_V \mu(t,u,v,w) \,dt\,du\,dv\,dw$$
$$\sum_{n=1}^{\infty} 2^{-n} = 1$$	
\begin{definition}
	Si $X$ et $Y$ sont 2 v.a. ou definit la \textsc{Covariance} entre $X$ et $Y$ comme
	$\cov(X,Y)\overset{\text{def}}{=}\E\left[(X-\E(X))(Y-\E(Y))\right]=\E(XY)-\E(X)\E(Y)$.
\end{definition}
\fi
\pagebreak

% \tableofcontents

% insert your code here
%\input{./algebra/main.tex}
%\input{./geometrie-differentielle/main.tex}
%\input{./probabilite/main.tex}
%\input{./analyse-fonctionnelle/main.tex}
% \input{./Analyse-convexe-et-dualite-en-optimisation/main.tex}
%\input{./tikz/main.tex}
%\input{./Theorie-du-distributions/main.tex}
%\input{./optimisation/mine.tex}
 \input{./modelisation/main.tex}

% yves.aubry@univ-tln.fr : algebra

\end{document}

%% !TEX encoding = UTF-8 Unicode
% !TEX TS-program = xelatex

\documentclass[french]{report}

%\usepackage[utf8]{inputenc}
%\usepackage[T1]{fontenc}
\usepackage{babel}


\newif\ifcomment
%\commenttrue # Show comments

\usepackage{physics}
\usepackage{amssymb}


\usepackage{amsthm}
% \usepackage{thmtools}
\usepackage{mathtools}
\usepackage{amsfonts}

\usepackage{color}

\usepackage{tikz}

\usepackage{geometry}
\geometry{a5paper, margin=0.1in, right=1cm}

\usepackage{dsfont}

\usepackage{graphicx}
\graphicspath{ {images/} }

\usepackage{faktor}

\usepackage{IEEEtrantools}
\usepackage{enumerate}   
\usepackage[PostScript=dvips]{"/Users/aware/Documents/Courses/diagrams"}


\newtheorem{theorem}{Théorème}[section]
\renewcommand{\thetheorem}{\arabic{theorem}}
\newtheorem{lemme}{Lemme}[section]
\renewcommand{\thelemme}{\arabic{lemme}}
\newtheorem{proposition}{Proposition}[section]
\renewcommand{\theproposition}{\arabic{proposition}}
\newtheorem{notations}{Notations}[section]
\newtheorem{problem}{Problème}[section]
\newtheorem{corollary}{Corollaire}[theorem]
\renewcommand{\thecorollary}{\arabic{corollary}}
\newtheorem{property}{Propriété}[section]
\newtheorem{objective}{Objectif}[section]

\theoremstyle{definition}
\newtheorem{definition}{Définition}[section]
\renewcommand{\thedefinition}{\arabic{definition}}
\newtheorem{exercise}{Exercice}[chapter]
\renewcommand{\theexercise}{\arabic{exercise}}
\newtheorem{example}{Exemple}[chapter]
\renewcommand{\theexample}{\arabic{example}}
\newtheorem*{solution}{Solution}
\newtheorem*{application}{Application}
\newtheorem*{notation}{Notation}
\newtheorem*{vocabulary}{Vocabulaire}
\newtheorem*{properties}{Propriétés}



\theoremstyle{remark}
\newtheorem*{remark}{Remarque}
\newtheorem*{rappel}{Rappel}


\usepackage{etoolbox}
\AtBeginEnvironment{exercise}{\small}
\AtBeginEnvironment{example}{\small}

\usepackage{cases}
\usepackage[red]{mypack}

\usepackage[framemethod=TikZ]{mdframed}

\definecolor{bg}{rgb}{0.4,0.25,0.95}
\definecolor{pagebg}{rgb}{0,0,0.5}
\surroundwithmdframed[
   topline=false,
   rightline=false,
   bottomline=false,
   leftmargin=\parindent,
   skipabove=8pt,
   skipbelow=8pt,
   linecolor=blue,
   innerbottommargin=10pt,
   % backgroundcolor=bg,font=\color{orange}\sffamily, fontcolor=white
]{definition}

\usepackage{empheq}
\usepackage[most]{tcolorbox}

\newtcbox{\mymath}[1][]{%
    nobeforeafter, math upper, tcbox raise base,
    enhanced, colframe=blue!30!black,
    colback=red!10, boxrule=1pt,
    #1}

\usepackage{unixode}


\DeclareMathOperator{\ord}{ord}
\DeclareMathOperator{\orb}{orb}
\DeclareMathOperator{\stab}{stab}
\DeclareMathOperator{\Stab}{stab}
\DeclareMathOperator{\ppcm}{ppcm}
\DeclareMathOperator{\conj}{Conj}
\DeclareMathOperator{\End}{End}
\DeclareMathOperator{\rot}{rot}
\DeclareMathOperator{\trs}{trace}
\DeclareMathOperator{\Ind}{Ind}
\DeclareMathOperator{\mat}{Mat}
\DeclareMathOperator{\id}{Id}
\DeclareMathOperator{\vect}{vect}
\DeclareMathOperator{\img}{img}
\DeclareMathOperator{\cov}{Cov}
\DeclareMathOperator{\dist}{dist}
\DeclareMathOperator{\irr}{Irr}
\DeclareMathOperator{\image}{Im}
\DeclareMathOperator{\pd}{\partial}
\DeclareMathOperator{\epi}{epi}
\DeclareMathOperator{\Argmin}{Argmin}
\DeclareMathOperator{\dom}{dom}
\DeclareMathOperator{\proj}{proj}
\DeclareMathOperator{\ctg}{ctg}
\DeclareMathOperator{\supp}{supp}
\DeclareMathOperator{\argmin}{argmin}
\DeclareMathOperator{\mult}{mult}
\DeclareMathOperator{\ch}{ch}
\DeclareMathOperator{\sh}{sh}
\DeclareMathOperator{\rang}{rang}
\DeclareMathOperator{\diam}{diam}
\DeclareMathOperator{\Epigraphe}{Epigraphe}




\usepackage{xcolor}
\everymath{\color{blue}}
%\everymath{\color[rgb]{0,1,1}}
%\pagecolor[rgb]{0,0,0.5}


\newcommand*{\pdtest}[3][]{\ensuremath{\frac{\partial^{#1} #2}{\partial #3}}}

\newcommand*{\deffunc}[6][]{\ensuremath{
\begin{array}{rcl}
#2 : #3 &\rightarrow& #4\\
#5 &\mapsto& #6
\end{array}
}}

\newcommand{\eqcolon}{\mathrel{\resizebox{\widthof{$\mathord{=}$}}{\height}{ $\!\!=\!\!\resizebox{1.2\width}{0.8\height}{\raisebox{0.23ex}{$\mathop{:}$}}\!\!$ }}}
\newcommand{\coloneq}{\mathrel{\resizebox{\widthof{$\mathord{=}$}}{\height}{ $\!\!\resizebox{1.2\width}{0.8\height}{\raisebox{0.23ex}{$\mathop{:}$}}\!\!=\!\!$ }}}
\newcommand{\eqcolonl}{\ensuremath{\mathrel{=\!\!\mathop{:}}}}
\newcommand{\coloneql}{\ensuremath{\mathrel{\mathop{:} \!\! =}}}
\newcommand{\vc}[1]{% inline column vector
  \left(\begin{smallmatrix}#1\end{smallmatrix}\right)%
}
\newcommand{\vr}[1]{% inline row vector
  \begin{smallmatrix}(\,#1\,)\end{smallmatrix}%
}
\makeatletter
\newcommand*{\defeq}{\ =\mathrel{\rlap{%
                     \raisebox{0.3ex}{$\m@th\cdot$}}%
                     \raisebox{-0.3ex}{$\m@th\cdot$}}%
                     }
\makeatother

\newcommand{\mathcircle}[1]{% inline row vector
 \overset{\circ}{#1}
}
\newcommand{\ulim}{% low limit
 \underline{\lim}
}
\newcommand{\ssi}{% iff
\iff
}
\newcommand{\ps}[2]{
\expval{#1 | #2}
}
\newcommand{\df}[1]{
\mqty{#1}
}
\newcommand{\n}[1]{
\norm{#1}
}
\newcommand{\sys}[1]{
\left\{\smqty{#1}\right.
}


\newcommand{\eqdef}{\ensuremath{\overset{\text{def}}=}}


\def\Circlearrowright{\ensuremath{%
  \rotatebox[origin=c]{230}{$\circlearrowright$}}}

\newcommand\ct[1]{\text{\rmfamily\upshape #1}}
\newcommand\question[1]{ {\color{red} ...!? \small #1}}
\newcommand\caz[1]{\left\{\begin{array} #1 \end{array}\right.}
\newcommand\const{\text{\rmfamily\upshape const}}
\newcommand\toP{ \overset{\pro}{\to}}
\newcommand\toPP{ \overset{\text{PP}}{\to}}
\newcommand{\oeq}{\mathrel{\text{\textcircled{$=$}}}}





\usepackage{xcolor}
% \usepackage[normalem]{ulem}
\usepackage{lipsum}
\makeatletter
% \newcommand\colorwave[1][blue]{\bgroup \markoverwith{\lower3.5\p@\hbox{\sixly \textcolor{#1}{\char58}}}\ULon}
%\font\sixly=lasy6 % does not re-load if already loaded, so no memory problem.

\newmdtheoremenv[
linewidth= 1pt,linecolor= blue,%
leftmargin=20,rightmargin=20,innertopmargin=0pt, innerrightmargin=40,%
tikzsetting = { draw=lightgray, line width = 0.3pt,dashed,%
dash pattern = on 15pt off 3pt},%
splittopskip=\topskip,skipbelow=\baselineskip,%
skipabove=\baselineskip,ntheorem,roundcorner=0pt,
% backgroundcolor=pagebg,font=\color{orange}\sffamily, fontcolor=white
]{examplebox}{Exemple}[section]



\newcommand\R{\mathbb{R}}
\newcommand\Z{\mathbb{Z}}
\newcommand\N{\mathbb{N}}
\newcommand\E{\mathbb{E}}
\newcommand\F{\mathcal{F}}
\newcommand\cH{\mathcal{H}}
\newcommand\V{\mathbb{V}}
\newcommand\dmo{ ^{-1} }
\newcommand\kapa{\kappa}
\newcommand\im{Im}
\newcommand\hs{\mathcal{H}}





\usepackage{soul}

\makeatletter
\newcommand*{\whiten}[1]{\llap{\textcolor{white}{{\the\SOUL@token}}\hspace{#1pt}}}
\DeclareRobustCommand*\myul{%
    \def\SOUL@everyspace{\underline{\space}\kern\z@}%
    \def\SOUL@everytoken{%
     \setbox0=\hbox{\the\SOUL@token}%
     \ifdim\dp0>\z@
        \raisebox{\dp0}{\underline{\phantom{\the\SOUL@token}}}%
        \whiten{1}\whiten{0}%
        \whiten{-1}\whiten{-2}%
        \llap{\the\SOUL@token}%
     \else
        \underline{\the\SOUL@token}%
     \fi}%
\SOUL@}
\makeatother

\newcommand*{\demp}{\fontfamily{lmtt}\selectfont}

\DeclareTextFontCommand{\textdemp}{\demp}

\begin{document}

\ifcomment
Multiline
comment
\fi
\ifcomment
\myul{Typesetting test}
% \color[rgb]{1,1,1}
$∑_i^n≠ 60º±∞π∆¬≈√j∫h≤≥µ$

$\CR \R\pro\ind\pro\gS\pro
\mqty[a&b\\c&d]$
$\pro\mathbb{P}$
$\dd{x}$

  \[
    \alpha(x)=\left\{
                \begin{array}{ll}
                  x\\
                  \frac{1}{1+e^{-kx}}\\
                  \frac{e^x-e^{-x}}{e^x+e^{-x}}
                \end{array}
              \right.
  \]

  $\expval{x}$
  
  $\chi_\rho(ghg\dmo)=\Tr(\rho_{ghg\dmo})=\Tr(\rho_g\circ\rho_h\circ\rho\dmo_g)=\Tr(\rho_h)\overset{\mbox{\scalebox{0.5}{$\Tr(AB)=\Tr(BA)$}}}{=}\chi_\rho(h)$
  	$\mathop{\oplus}_{\substack{x\in X}}$

$\mat(\rho_g)=(a_{ij}(g))_{\scriptsize \substack{1\leq i\leq d \\ 1\leq j\leq d}}$ et $\mat(\rho'_g)=(a'_{ij}(g))_{\scriptsize \substack{1\leq i'\leq d' \\ 1\leq j'\leq d'}}$



\[\int_a^b{\mathbb{R}^2}g(u, v)\dd{P_{XY}}(u, v)=\iint g(u,v) f_{XY}(u, v)\dd \lambda(u) \dd \lambda(v)\]
$$\lim_{x\to\infty} f(x)$$	
$$\iiiint_V \mu(t,u,v,w) \,dt\,du\,dv\,dw$$
$$\sum_{n=1}^{\infty} 2^{-n} = 1$$	
\begin{definition}
	Si $X$ et $Y$ sont 2 v.a. ou definit la \textsc{Covariance} entre $X$ et $Y$ comme
	$\cov(X,Y)\overset{\text{def}}{=}\E\left[(X-\E(X))(Y-\E(Y))\right]=\E(XY)-\E(X)\E(Y)$.
\end{definition}
\fi
\pagebreak

% \tableofcontents

% insert your code here
%\input{./algebra/main.tex}
%\input{./geometrie-differentielle/main.tex}
%\input{./probabilite/main.tex}
%\input{./analyse-fonctionnelle/main.tex}
% \input{./Analyse-convexe-et-dualite-en-optimisation/main.tex}
%\input{./tikz/main.tex}
%\input{./Theorie-du-distributions/main.tex}
%\input{./optimisation/mine.tex}
 \input{./modelisation/main.tex}

% yves.aubry@univ-tln.fr : algebra

\end{document}

%% !TEX encoding = UTF-8 Unicode
% !TEX TS-program = xelatex

\documentclass[french]{report}

%\usepackage[utf8]{inputenc}
%\usepackage[T1]{fontenc}
\usepackage{babel}


\newif\ifcomment
%\commenttrue # Show comments

\usepackage{physics}
\usepackage{amssymb}


\usepackage{amsthm}
% \usepackage{thmtools}
\usepackage{mathtools}
\usepackage{amsfonts}

\usepackage{color}

\usepackage{tikz}

\usepackage{geometry}
\geometry{a5paper, margin=0.1in, right=1cm}

\usepackage{dsfont}

\usepackage{graphicx}
\graphicspath{ {images/} }

\usepackage{faktor}

\usepackage{IEEEtrantools}
\usepackage{enumerate}   
\usepackage[PostScript=dvips]{"/Users/aware/Documents/Courses/diagrams"}


\newtheorem{theorem}{Théorème}[section]
\renewcommand{\thetheorem}{\arabic{theorem}}
\newtheorem{lemme}{Lemme}[section]
\renewcommand{\thelemme}{\arabic{lemme}}
\newtheorem{proposition}{Proposition}[section]
\renewcommand{\theproposition}{\arabic{proposition}}
\newtheorem{notations}{Notations}[section]
\newtheorem{problem}{Problème}[section]
\newtheorem{corollary}{Corollaire}[theorem]
\renewcommand{\thecorollary}{\arabic{corollary}}
\newtheorem{property}{Propriété}[section]
\newtheorem{objective}{Objectif}[section]

\theoremstyle{definition}
\newtheorem{definition}{Définition}[section]
\renewcommand{\thedefinition}{\arabic{definition}}
\newtheorem{exercise}{Exercice}[chapter]
\renewcommand{\theexercise}{\arabic{exercise}}
\newtheorem{example}{Exemple}[chapter]
\renewcommand{\theexample}{\arabic{example}}
\newtheorem*{solution}{Solution}
\newtheorem*{application}{Application}
\newtheorem*{notation}{Notation}
\newtheorem*{vocabulary}{Vocabulaire}
\newtheorem*{properties}{Propriétés}



\theoremstyle{remark}
\newtheorem*{remark}{Remarque}
\newtheorem*{rappel}{Rappel}


\usepackage{etoolbox}
\AtBeginEnvironment{exercise}{\small}
\AtBeginEnvironment{example}{\small}

\usepackage{cases}
\usepackage[red]{mypack}

\usepackage[framemethod=TikZ]{mdframed}

\definecolor{bg}{rgb}{0.4,0.25,0.95}
\definecolor{pagebg}{rgb}{0,0,0.5}
\surroundwithmdframed[
   topline=false,
   rightline=false,
   bottomline=false,
   leftmargin=\parindent,
   skipabove=8pt,
   skipbelow=8pt,
   linecolor=blue,
   innerbottommargin=10pt,
   % backgroundcolor=bg,font=\color{orange}\sffamily, fontcolor=white
]{definition}

\usepackage{empheq}
\usepackage[most]{tcolorbox}

\newtcbox{\mymath}[1][]{%
    nobeforeafter, math upper, tcbox raise base,
    enhanced, colframe=blue!30!black,
    colback=red!10, boxrule=1pt,
    #1}

\usepackage{unixode}


\DeclareMathOperator{\ord}{ord}
\DeclareMathOperator{\orb}{orb}
\DeclareMathOperator{\stab}{stab}
\DeclareMathOperator{\Stab}{stab}
\DeclareMathOperator{\ppcm}{ppcm}
\DeclareMathOperator{\conj}{Conj}
\DeclareMathOperator{\End}{End}
\DeclareMathOperator{\rot}{rot}
\DeclareMathOperator{\trs}{trace}
\DeclareMathOperator{\Ind}{Ind}
\DeclareMathOperator{\mat}{Mat}
\DeclareMathOperator{\id}{Id}
\DeclareMathOperator{\vect}{vect}
\DeclareMathOperator{\img}{img}
\DeclareMathOperator{\cov}{Cov}
\DeclareMathOperator{\dist}{dist}
\DeclareMathOperator{\irr}{Irr}
\DeclareMathOperator{\image}{Im}
\DeclareMathOperator{\pd}{\partial}
\DeclareMathOperator{\epi}{epi}
\DeclareMathOperator{\Argmin}{Argmin}
\DeclareMathOperator{\dom}{dom}
\DeclareMathOperator{\proj}{proj}
\DeclareMathOperator{\ctg}{ctg}
\DeclareMathOperator{\supp}{supp}
\DeclareMathOperator{\argmin}{argmin}
\DeclareMathOperator{\mult}{mult}
\DeclareMathOperator{\ch}{ch}
\DeclareMathOperator{\sh}{sh}
\DeclareMathOperator{\rang}{rang}
\DeclareMathOperator{\diam}{diam}
\DeclareMathOperator{\Epigraphe}{Epigraphe}




\usepackage{xcolor}
\everymath{\color{blue}}
%\everymath{\color[rgb]{0,1,1}}
%\pagecolor[rgb]{0,0,0.5}


\newcommand*{\pdtest}[3][]{\ensuremath{\frac{\partial^{#1} #2}{\partial #3}}}

\newcommand*{\deffunc}[6][]{\ensuremath{
\begin{array}{rcl}
#2 : #3 &\rightarrow& #4\\
#5 &\mapsto& #6
\end{array}
}}

\newcommand{\eqcolon}{\mathrel{\resizebox{\widthof{$\mathord{=}$}}{\height}{ $\!\!=\!\!\resizebox{1.2\width}{0.8\height}{\raisebox{0.23ex}{$\mathop{:}$}}\!\!$ }}}
\newcommand{\coloneq}{\mathrel{\resizebox{\widthof{$\mathord{=}$}}{\height}{ $\!\!\resizebox{1.2\width}{0.8\height}{\raisebox{0.23ex}{$\mathop{:}$}}\!\!=\!\!$ }}}
\newcommand{\eqcolonl}{\ensuremath{\mathrel{=\!\!\mathop{:}}}}
\newcommand{\coloneql}{\ensuremath{\mathrel{\mathop{:} \!\! =}}}
\newcommand{\vc}[1]{% inline column vector
  \left(\begin{smallmatrix}#1\end{smallmatrix}\right)%
}
\newcommand{\vr}[1]{% inline row vector
  \begin{smallmatrix}(\,#1\,)\end{smallmatrix}%
}
\makeatletter
\newcommand*{\defeq}{\ =\mathrel{\rlap{%
                     \raisebox{0.3ex}{$\m@th\cdot$}}%
                     \raisebox{-0.3ex}{$\m@th\cdot$}}%
                     }
\makeatother

\newcommand{\mathcircle}[1]{% inline row vector
 \overset{\circ}{#1}
}
\newcommand{\ulim}{% low limit
 \underline{\lim}
}
\newcommand{\ssi}{% iff
\iff
}
\newcommand{\ps}[2]{
\expval{#1 | #2}
}
\newcommand{\df}[1]{
\mqty{#1}
}
\newcommand{\n}[1]{
\norm{#1}
}
\newcommand{\sys}[1]{
\left\{\smqty{#1}\right.
}


\newcommand{\eqdef}{\ensuremath{\overset{\text{def}}=}}


\def\Circlearrowright{\ensuremath{%
  \rotatebox[origin=c]{230}{$\circlearrowright$}}}

\newcommand\ct[1]{\text{\rmfamily\upshape #1}}
\newcommand\question[1]{ {\color{red} ...!? \small #1}}
\newcommand\caz[1]{\left\{\begin{array} #1 \end{array}\right.}
\newcommand\const{\text{\rmfamily\upshape const}}
\newcommand\toP{ \overset{\pro}{\to}}
\newcommand\toPP{ \overset{\text{PP}}{\to}}
\newcommand{\oeq}{\mathrel{\text{\textcircled{$=$}}}}





\usepackage{xcolor}
% \usepackage[normalem]{ulem}
\usepackage{lipsum}
\makeatletter
% \newcommand\colorwave[1][blue]{\bgroup \markoverwith{\lower3.5\p@\hbox{\sixly \textcolor{#1}{\char58}}}\ULon}
%\font\sixly=lasy6 % does not re-load if already loaded, so no memory problem.

\newmdtheoremenv[
linewidth= 1pt,linecolor= blue,%
leftmargin=20,rightmargin=20,innertopmargin=0pt, innerrightmargin=40,%
tikzsetting = { draw=lightgray, line width = 0.3pt,dashed,%
dash pattern = on 15pt off 3pt},%
splittopskip=\topskip,skipbelow=\baselineskip,%
skipabove=\baselineskip,ntheorem,roundcorner=0pt,
% backgroundcolor=pagebg,font=\color{orange}\sffamily, fontcolor=white
]{examplebox}{Exemple}[section]



\newcommand\R{\mathbb{R}}
\newcommand\Z{\mathbb{Z}}
\newcommand\N{\mathbb{N}}
\newcommand\E{\mathbb{E}}
\newcommand\F{\mathcal{F}}
\newcommand\cH{\mathcal{H}}
\newcommand\V{\mathbb{V}}
\newcommand\dmo{ ^{-1} }
\newcommand\kapa{\kappa}
\newcommand\im{Im}
\newcommand\hs{\mathcal{H}}





\usepackage{soul}

\makeatletter
\newcommand*{\whiten}[1]{\llap{\textcolor{white}{{\the\SOUL@token}}\hspace{#1pt}}}
\DeclareRobustCommand*\myul{%
    \def\SOUL@everyspace{\underline{\space}\kern\z@}%
    \def\SOUL@everytoken{%
     \setbox0=\hbox{\the\SOUL@token}%
     \ifdim\dp0>\z@
        \raisebox{\dp0}{\underline{\phantom{\the\SOUL@token}}}%
        \whiten{1}\whiten{0}%
        \whiten{-1}\whiten{-2}%
        \llap{\the\SOUL@token}%
     \else
        \underline{\the\SOUL@token}%
     \fi}%
\SOUL@}
\makeatother

\newcommand*{\demp}{\fontfamily{lmtt}\selectfont}

\DeclareTextFontCommand{\textdemp}{\demp}

\begin{document}

\ifcomment
Multiline
comment
\fi
\ifcomment
\myul{Typesetting test}
% \color[rgb]{1,1,1}
$∑_i^n≠ 60º±∞π∆¬≈√j∫h≤≥µ$

$\CR \R\pro\ind\pro\gS\pro
\mqty[a&b\\c&d]$
$\pro\mathbb{P}$
$\dd{x}$

  \[
    \alpha(x)=\left\{
                \begin{array}{ll}
                  x\\
                  \frac{1}{1+e^{-kx}}\\
                  \frac{e^x-e^{-x}}{e^x+e^{-x}}
                \end{array}
              \right.
  \]

  $\expval{x}$
  
  $\chi_\rho(ghg\dmo)=\Tr(\rho_{ghg\dmo})=\Tr(\rho_g\circ\rho_h\circ\rho\dmo_g)=\Tr(\rho_h)\overset{\mbox{\scalebox{0.5}{$\Tr(AB)=\Tr(BA)$}}}{=}\chi_\rho(h)$
  	$\mathop{\oplus}_{\substack{x\in X}}$

$\mat(\rho_g)=(a_{ij}(g))_{\scriptsize \substack{1\leq i\leq d \\ 1\leq j\leq d}}$ et $\mat(\rho'_g)=(a'_{ij}(g))_{\scriptsize \substack{1\leq i'\leq d' \\ 1\leq j'\leq d'}}$



\[\int_a^b{\mathbb{R}^2}g(u, v)\dd{P_{XY}}(u, v)=\iint g(u,v) f_{XY}(u, v)\dd \lambda(u) \dd \lambda(v)\]
$$\lim_{x\to\infty} f(x)$$	
$$\iiiint_V \mu(t,u,v,w) \,dt\,du\,dv\,dw$$
$$\sum_{n=1}^{\infty} 2^{-n} = 1$$	
\begin{definition}
	Si $X$ et $Y$ sont 2 v.a. ou definit la \textsc{Covariance} entre $X$ et $Y$ comme
	$\cov(X,Y)\overset{\text{def}}{=}\E\left[(X-\E(X))(Y-\E(Y))\right]=\E(XY)-\E(X)\E(Y)$.
\end{definition}
\fi
\pagebreak

% \tableofcontents

% insert your code here
%\input{./algebra/main.tex}
%\input{./geometrie-differentielle/main.tex}
%\input{./probabilite/main.tex}
%\input{./analyse-fonctionnelle/main.tex}
% \input{./Analyse-convexe-et-dualite-en-optimisation/main.tex}
%\input{./tikz/main.tex}
%\input{./Theorie-du-distributions/main.tex}
%\input{./optimisation/mine.tex}
 \input{./modelisation/main.tex}

% yves.aubry@univ-tln.fr : algebra

\end{document}

% % !TEX encoding = UTF-8 Unicode
% !TEX TS-program = xelatex

\documentclass[french]{report}

%\usepackage[utf8]{inputenc}
%\usepackage[T1]{fontenc}
\usepackage{babel}


\newif\ifcomment
%\commenttrue # Show comments

\usepackage{physics}
\usepackage{amssymb}


\usepackage{amsthm}
% \usepackage{thmtools}
\usepackage{mathtools}
\usepackage{amsfonts}

\usepackage{color}

\usepackage{tikz}

\usepackage{geometry}
\geometry{a5paper, margin=0.1in, right=1cm}

\usepackage{dsfont}

\usepackage{graphicx}
\graphicspath{ {images/} }

\usepackage{faktor}

\usepackage{IEEEtrantools}
\usepackage{enumerate}   
\usepackage[PostScript=dvips]{"/Users/aware/Documents/Courses/diagrams"}


\newtheorem{theorem}{Théorème}[section]
\renewcommand{\thetheorem}{\arabic{theorem}}
\newtheorem{lemme}{Lemme}[section]
\renewcommand{\thelemme}{\arabic{lemme}}
\newtheorem{proposition}{Proposition}[section]
\renewcommand{\theproposition}{\arabic{proposition}}
\newtheorem{notations}{Notations}[section]
\newtheorem{problem}{Problème}[section]
\newtheorem{corollary}{Corollaire}[theorem]
\renewcommand{\thecorollary}{\arabic{corollary}}
\newtheorem{property}{Propriété}[section]
\newtheorem{objective}{Objectif}[section]

\theoremstyle{definition}
\newtheorem{definition}{Définition}[section]
\renewcommand{\thedefinition}{\arabic{definition}}
\newtheorem{exercise}{Exercice}[chapter]
\renewcommand{\theexercise}{\arabic{exercise}}
\newtheorem{example}{Exemple}[chapter]
\renewcommand{\theexample}{\arabic{example}}
\newtheorem*{solution}{Solution}
\newtheorem*{application}{Application}
\newtheorem*{notation}{Notation}
\newtheorem*{vocabulary}{Vocabulaire}
\newtheorem*{properties}{Propriétés}



\theoremstyle{remark}
\newtheorem*{remark}{Remarque}
\newtheorem*{rappel}{Rappel}


\usepackage{etoolbox}
\AtBeginEnvironment{exercise}{\small}
\AtBeginEnvironment{example}{\small}

\usepackage{cases}
\usepackage[red]{mypack}

\usepackage[framemethod=TikZ]{mdframed}

\definecolor{bg}{rgb}{0.4,0.25,0.95}
\definecolor{pagebg}{rgb}{0,0,0.5}
\surroundwithmdframed[
   topline=false,
   rightline=false,
   bottomline=false,
   leftmargin=\parindent,
   skipabove=8pt,
   skipbelow=8pt,
   linecolor=blue,
   innerbottommargin=10pt,
   % backgroundcolor=bg,font=\color{orange}\sffamily, fontcolor=white
]{definition}

\usepackage{empheq}
\usepackage[most]{tcolorbox}

\newtcbox{\mymath}[1][]{%
    nobeforeafter, math upper, tcbox raise base,
    enhanced, colframe=blue!30!black,
    colback=red!10, boxrule=1pt,
    #1}

\usepackage{unixode}


\DeclareMathOperator{\ord}{ord}
\DeclareMathOperator{\orb}{orb}
\DeclareMathOperator{\stab}{stab}
\DeclareMathOperator{\Stab}{stab}
\DeclareMathOperator{\ppcm}{ppcm}
\DeclareMathOperator{\conj}{Conj}
\DeclareMathOperator{\End}{End}
\DeclareMathOperator{\rot}{rot}
\DeclareMathOperator{\trs}{trace}
\DeclareMathOperator{\Ind}{Ind}
\DeclareMathOperator{\mat}{Mat}
\DeclareMathOperator{\id}{Id}
\DeclareMathOperator{\vect}{vect}
\DeclareMathOperator{\img}{img}
\DeclareMathOperator{\cov}{Cov}
\DeclareMathOperator{\dist}{dist}
\DeclareMathOperator{\irr}{Irr}
\DeclareMathOperator{\image}{Im}
\DeclareMathOperator{\pd}{\partial}
\DeclareMathOperator{\epi}{epi}
\DeclareMathOperator{\Argmin}{Argmin}
\DeclareMathOperator{\dom}{dom}
\DeclareMathOperator{\proj}{proj}
\DeclareMathOperator{\ctg}{ctg}
\DeclareMathOperator{\supp}{supp}
\DeclareMathOperator{\argmin}{argmin}
\DeclareMathOperator{\mult}{mult}
\DeclareMathOperator{\ch}{ch}
\DeclareMathOperator{\sh}{sh}
\DeclareMathOperator{\rang}{rang}
\DeclareMathOperator{\diam}{diam}
\DeclareMathOperator{\Epigraphe}{Epigraphe}




\usepackage{xcolor}
\everymath{\color{blue}}
%\everymath{\color[rgb]{0,1,1}}
%\pagecolor[rgb]{0,0,0.5}


\newcommand*{\pdtest}[3][]{\ensuremath{\frac{\partial^{#1} #2}{\partial #3}}}

\newcommand*{\deffunc}[6][]{\ensuremath{
\begin{array}{rcl}
#2 : #3 &\rightarrow& #4\\
#5 &\mapsto& #6
\end{array}
}}

\newcommand{\eqcolon}{\mathrel{\resizebox{\widthof{$\mathord{=}$}}{\height}{ $\!\!=\!\!\resizebox{1.2\width}{0.8\height}{\raisebox{0.23ex}{$\mathop{:}$}}\!\!$ }}}
\newcommand{\coloneq}{\mathrel{\resizebox{\widthof{$\mathord{=}$}}{\height}{ $\!\!\resizebox{1.2\width}{0.8\height}{\raisebox{0.23ex}{$\mathop{:}$}}\!\!=\!\!$ }}}
\newcommand{\eqcolonl}{\ensuremath{\mathrel{=\!\!\mathop{:}}}}
\newcommand{\coloneql}{\ensuremath{\mathrel{\mathop{:} \!\! =}}}
\newcommand{\vc}[1]{% inline column vector
  \left(\begin{smallmatrix}#1\end{smallmatrix}\right)%
}
\newcommand{\vr}[1]{% inline row vector
  \begin{smallmatrix}(\,#1\,)\end{smallmatrix}%
}
\makeatletter
\newcommand*{\defeq}{\ =\mathrel{\rlap{%
                     \raisebox{0.3ex}{$\m@th\cdot$}}%
                     \raisebox{-0.3ex}{$\m@th\cdot$}}%
                     }
\makeatother

\newcommand{\mathcircle}[1]{% inline row vector
 \overset{\circ}{#1}
}
\newcommand{\ulim}{% low limit
 \underline{\lim}
}
\newcommand{\ssi}{% iff
\iff
}
\newcommand{\ps}[2]{
\expval{#1 | #2}
}
\newcommand{\df}[1]{
\mqty{#1}
}
\newcommand{\n}[1]{
\norm{#1}
}
\newcommand{\sys}[1]{
\left\{\smqty{#1}\right.
}


\newcommand{\eqdef}{\ensuremath{\overset{\text{def}}=}}


\def\Circlearrowright{\ensuremath{%
  \rotatebox[origin=c]{230}{$\circlearrowright$}}}

\newcommand\ct[1]{\text{\rmfamily\upshape #1}}
\newcommand\question[1]{ {\color{red} ...!? \small #1}}
\newcommand\caz[1]{\left\{\begin{array} #1 \end{array}\right.}
\newcommand\const{\text{\rmfamily\upshape const}}
\newcommand\toP{ \overset{\pro}{\to}}
\newcommand\toPP{ \overset{\text{PP}}{\to}}
\newcommand{\oeq}{\mathrel{\text{\textcircled{$=$}}}}





\usepackage{xcolor}
% \usepackage[normalem]{ulem}
\usepackage{lipsum}
\makeatletter
% \newcommand\colorwave[1][blue]{\bgroup \markoverwith{\lower3.5\p@\hbox{\sixly \textcolor{#1}{\char58}}}\ULon}
%\font\sixly=lasy6 % does not re-load if already loaded, so no memory problem.

\newmdtheoremenv[
linewidth= 1pt,linecolor= blue,%
leftmargin=20,rightmargin=20,innertopmargin=0pt, innerrightmargin=40,%
tikzsetting = { draw=lightgray, line width = 0.3pt,dashed,%
dash pattern = on 15pt off 3pt},%
splittopskip=\topskip,skipbelow=\baselineskip,%
skipabove=\baselineskip,ntheorem,roundcorner=0pt,
% backgroundcolor=pagebg,font=\color{orange}\sffamily, fontcolor=white
]{examplebox}{Exemple}[section]



\newcommand\R{\mathbb{R}}
\newcommand\Z{\mathbb{Z}}
\newcommand\N{\mathbb{N}}
\newcommand\E{\mathbb{E}}
\newcommand\F{\mathcal{F}}
\newcommand\cH{\mathcal{H}}
\newcommand\V{\mathbb{V}}
\newcommand\dmo{ ^{-1} }
\newcommand\kapa{\kappa}
\newcommand\im{Im}
\newcommand\hs{\mathcal{H}}





\usepackage{soul}

\makeatletter
\newcommand*{\whiten}[1]{\llap{\textcolor{white}{{\the\SOUL@token}}\hspace{#1pt}}}
\DeclareRobustCommand*\myul{%
    \def\SOUL@everyspace{\underline{\space}\kern\z@}%
    \def\SOUL@everytoken{%
     \setbox0=\hbox{\the\SOUL@token}%
     \ifdim\dp0>\z@
        \raisebox{\dp0}{\underline{\phantom{\the\SOUL@token}}}%
        \whiten{1}\whiten{0}%
        \whiten{-1}\whiten{-2}%
        \llap{\the\SOUL@token}%
     \else
        \underline{\the\SOUL@token}%
     \fi}%
\SOUL@}
\makeatother

\newcommand*{\demp}{\fontfamily{lmtt}\selectfont}

\DeclareTextFontCommand{\textdemp}{\demp}

\begin{document}

\ifcomment
Multiline
comment
\fi
\ifcomment
\myul{Typesetting test}
% \color[rgb]{1,1,1}
$∑_i^n≠ 60º±∞π∆¬≈√j∫h≤≥µ$

$\CR \R\pro\ind\pro\gS\pro
\mqty[a&b\\c&d]$
$\pro\mathbb{P}$
$\dd{x}$

  \[
    \alpha(x)=\left\{
                \begin{array}{ll}
                  x\\
                  \frac{1}{1+e^{-kx}}\\
                  \frac{e^x-e^{-x}}{e^x+e^{-x}}
                \end{array}
              \right.
  \]

  $\expval{x}$
  
  $\chi_\rho(ghg\dmo)=\Tr(\rho_{ghg\dmo})=\Tr(\rho_g\circ\rho_h\circ\rho\dmo_g)=\Tr(\rho_h)\overset{\mbox{\scalebox{0.5}{$\Tr(AB)=\Tr(BA)$}}}{=}\chi_\rho(h)$
  	$\mathop{\oplus}_{\substack{x\in X}}$

$\mat(\rho_g)=(a_{ij}(g))_{\scriptsize \substack{1\leq i\leq d \\ 1\leq j\leq d}}$ et $\mat(\rho'_g)=(a'_{ij}(g))_{\scriptsize \substack{1\leq i'\leq d' \\ 1\leq j'\leq d'}}$



\[\int_a^b{\mathbb{R}^2}g(u, v)\dd{P_{XY}}(u, v)=\iint g(u,v) f_{XY}(u, v)\dd \lambda(u) \dd \lambda(v)\]
$$\lim_{x\to\infty} f(x)$$	
$$\iiiint_V \mu(t,u,v,w) \,dt\,du\,dv\,dw$$
$$\sum_{n=1}^{\infty} 2^{-n} = 1$$	
\begin{definition}
	Si $X$ et $Y$ sont 2 v.a. ou definit la \textsc{Covariance} entre $X$ et $Y$ comme
	$\cov(X,Y)\overset{\text{def}}{=}\E\left[(X-\E(X))(Y-\E(Y))\right]=\E(XY)-\E(X)\E(Y)$.
\end{definition}
\fi
\pagebreak

% \tableofcontents

% insert your code here
%\input{./algebra/main.tex}
%\input{./geometrie-differentielle/main.tex}
%\input{./probabilite/main.tex}
%\input{./analyse-fonctionnelle/main.tex}
% \input{./Analyse-convexe-et-dualite-en-optimisation/main.tex}
%\input{./tikz/main.tex}
%\input{./Theorie-du-distributions/main.tex}
%\input{./optimisation/mine.tex}
 \input{./modelisation/main.tex}

% yves.aubry@univ-tln.fr : algebra

\end{document}

%% !TEX encoding = UTF-8 Unicode
% !TEX TS-program = xelatex

\documentclass[french]{report}

%\usepackage[utf8]{inputenc}
%\usepackage[T1]{fontenc}
\usepackage{babel}


\newif\ifcomment
%\commenttrue # Show comments

\usepackage{physics}
\usepackage{amssymb}


\usepackage{amsthm}
% \usepackage{thmtools}
\usepackage{mathtools}
\usepackage{amsfonts}

\usepackage{color}

\usepackage{tikz}

\usepackage{geometry}
\geometry{a5paper, margin=0.1in, right=1cm}

\usepackage{dsfont}

\usepackage{graphicx}
\graphicspath{ {images/} }

\usepackage{faktor}

\usepackage{IEEEtrantools}
\usepackage{enumerate}   
\usepackage[PostScript=dvips]{"/Users/aware/Documents/Courses/diagrams"}


\newtheorem{theorem}{Théorème}[section]
\renewcommand{\thetheorem}{\arabic{theorem}}
\newtheorem{lemme}{Lemme}[section]
\renewcommand{\thelemme}{\arabic{lemme}}
\newtheorem{proposition}{Proposition}[section]
\renewcommand{\theproposition}{\arabic{proposition}}
\newtheorem{notations}{Notations}[section]
\newtheorem{problem}{Problème}[section]
\newtheorem{corollary}{Corollaire}[theorem]
\renewcommand{\thecorollary}{\arabic{corollary}}
\newtheorem{property}{Propriété}[section]
\newtheorem{objective}{Objectif}[section]

\theoremstyle{definition}
\newtheorem{definition}{Définition}[section]
\renewcommand{\thedefinition}{\arabic{definition}}
\newtheorem{exercise}{Exercice}[chapter]
\renewcommand{\theexercise}{\arabic{exercise}}
\newtheorem{example}{Exemple}[chapter]
\renewcommand{\theexample}{\arabic{example}}
\newtheorem*{solution}{Solution}
\newtheorem*{application}{Application}
\newtheorem*{notation}{Notation}
\newtheorem*{vocabulary}{Vocabulaire}
\newtheorem*{properties}{Propriétés}



\theoremstyle{remark}
\newtheorem*{remark}{Remarque}
\newtheorem*{rappel}{Rappel}


\usepackage{etoolbox}
\AtBeginEnvironment{exercise}{\small}
\AtBeginEnvironment{example}{\small}

\usepackage{cases}
\usepackage[red]{mypack}

\usepackage[framemethod=TikZ]{mdframed}

\definecolor{bg}{rgb}{0.4,0.25,0.95}
\definecolor{pagebg}{rgb}{0,0,0.5}
\surroundwithmdframed[
   topline=false,
   rightline=false,
   bottomline=false,
   leftmargin=\parindent,
   skipabove=8pt,
   skipbelow=8pt,
   linecolor=blue,
   innerbottommargin=10pt,
   % backgroundcolor=bg,font=\color{orange}\sffamily, fontcolor=white
]{definition}

\usepackage{empheq}
\usepackage[most]{tcolorbox}

\newtcbox{\mymath}[1][]{%
    nobeforeafter, math upper, tcbox raise base,
    enhanced, colframe=blue!30!black,
    colback=red!10, boxrule=1pt,
    #1}

\usepackage{unixode}


\DeclareMathOperator{\ord}{ord}
\DeclareMathOperator{\orb}{orb}
\DeclareMathOperator{\stab}{stab}
\DeclareMathOperator{\Stab}{stab}
\DeclareMathOperator{\ppcm}{ppcm}
\DeclareMathOperator{\conj}{Conj}
\DeclareMathOperator{\End}{End}
\DeclareMathOperator{\rot}{rot}
\DeclareMathOperator{\trs}{trace}
\DeclareMathOperator{\Ind}{Ind}
\DeclareMathOperator{\mat}{Mat}
\DeclareMathOperator{\id}{Id}
\DeclareMathOperator{\vect}{vect}
\DeclareMathOperator{\img}{img}
\DeclareMathOperator{\cov}{Cov}
\DeclareMathOperator{\dist}{dist}
\DeclareMathOperator{\irr}{Irr}
\DeclareMathOperator{\image}{Im}
\DeclareMathOperator{\pd}{\partial}
\DeclareMathOperator{\epi}{epi}
\DeclareMathOperator{\Argmin}{Argmin}
\DeclareMathOperator{\dom}{dom}
\DeclareMathOperator{\proj}{proj}
\DeclareMathOperator{\ctg}{ctg}
\DeclareMathOperator{\supp}{supp}
\DeclareMathOperator{\argmin}{argmin}
\DeclareMathOperator{\mult}{mult}
\DeclareMathOperator{\ch}{ch}
\DeclareMathOperator{\sh}{sh}
\DeclareMathOperator{\rang}{rang}
\DeclareMathOperator{\diam}{diam}
\DeclareMathOperator{\Epigraphe}{Epigraphe}




\usepackage{xcolor}
\everymath{\color{blue}}
%\everymath{\color[rgb]{0,1,1}}
%\pagecolor[rgb]{0,0,0.5}


\newcommand*{\pdtest}[3][]{\ensuremath{\frac{\partial^{#1} #2}{\partial #3}}}

\newcommand*{\deffunc}[6][]{\ensuremath{
\begin{array}{rcl}
#2 : #3 &\rightarrow& #4\\
#5 &\mapsto& #6
\end{array}
}}

\newcommand{\eqcolon}{\mathrel{\resizebox{\widthof{$\mathord{=}$}}{\height}{ $\!\!=\!\!\resizebox{1.2\width}{0.8\height}{\raisebox{0.23ex}{$\mathop{:}$}}\!\!$ }}}
\newcommand{\coloneq}{\mathrel{\resizebox{\widthof{$\mathord{=}$}}{\height}{ $\!\!\resizebox{1.2\width}{0.8\height}{\raisebox{0.23ex}{$\mathop{:}$}}\!\!=\!\!$ }}}
\newcommand{\eqcolonl}{\ensuremath{\mathrel{=\!\!\mathop{:}}}}
\newcommand{\coloneql}{\ensuremath{\mathrel{\mathop{:} \!\! =}}}
\newcommand{\vc}[1]{% inline column vector
  \left(\begin{smallmatrix}#1\end{smallmatrix}\right)%
}
\newcommand{\vr}[1]{% inline row vector
  \begin{smallmatrix}(\,#1\,)\end{smallmatrix}%
}
\makeatletter
\newcommand*{\defeq}{\ =\mathrel{\rlap{%
                     \raisebox{0.3ex}{$\m@th\cdot$}}%
                     \raisebox{-0.3ex}{$\m@th\cdot$}}%
                     }
\makeatother

\newcommand{\mathcircle}[1]{% inline row vector
 \overset{\circ}{#1}
}
\newcommand{\ulim}{% low limit
 \underline{\lim}
}
\newcommand{\ssi}{% iff
\iff
}
\newcommand{\ps}[2]{
\expval{#1 | #2}
}
\newcommand{\df}[1]{
\mqty{#1}
}
\newcommand{\n}[1]{
\norm{#1}
}
\newcommand{\sys}[1]{
\left\{\smqty{#1}\right.
}


\newcommand{\eqdef}{\ensuremath{\overset{\text{def}}=}}


\def\Circlearrowright{\ensuremath{%
  \rotatebox[origin=c]{230}{$\circlearrowright$}}}

\newcommand\ct[1]{\text{\rmfamily\upshape #1}}
\newcommand\question[1]{ {\color{red} ...!? \small #1}}
\newcommand\caz[1]{\left\{\begin{array} #1 \end{array}\right.}
\newcommand\const{\text{\rmfamily\upshape const}}
\newcommand\toP{ \overset{\pro}{\to}}
\newcommand\toPP{ \overset{\text{PP}}{\to}}
\newcommand{\oeq}{\mathrel{\text{\textcircled{$=$}}}}





\usepackage{xcolor}
% \usepackage[normalem]{ulem}
\usepackage{lipsum}
\makeatletter
% \newcommand\colorwave[1][blue]{\bgroup \markoverwith{\lower3.5\p@\hbox{\sixly \textcolor{#1}{\char58}}}\ULon}
%\font\sixly=lasy6 % does not re-load if already loaded, so no memory problem.

\newmdtheoremenv[
linewidth= 1pt,linecolor= blue,%
leftmargin=20,rightmargin=20,innertopmargin=0pt, innerrightmargin=40,%
tikzsetting = { draw=lightgray, line width = 0.3pt,dashed,%
dash pattern = on 15pt off 3pt},%
splittopskip=\topskip,skipbelow=\baselineskip,%
skipabove=\baselineskip,ntheorem,roundcorner=0pt,
% backgroundcolor=pagebg,font=\color{orange}\sffamily, fontcolor=white
]{examplebox}{Exemple}[section]



\newcommand\R{\mathbb{R}}
\newcommand\Z{\mathbb{Z}}
\newcommand\N{\mathbb{N}}
\newcommand\E{\mathbb{E}}
\newcommand\F{\mathcal{F}}
\newcommand\cH{\mathcal{H}}
\newcommand\V{\mathbb{V}}
\newcommand\dmo{ ^{-1} }
\newcommand\kapa{\kappa}
\newcommand\im{Im}
\newcommand\hs{\mathcal{H}}





\usepackage{soul}

\makeatletter
\newcommand*{\whiten}[1]{\llap{\textcolor{white}{{\the\SOUL@token}}\hspace{#1pt}}}
\DeclareRobustCommand*\myul{%
    \def\SOUL@everyspace{\underline{\space}\kern\z@}%
    \def\SOUL@everytoken{%
     \setbox0=\hbox{\the\SOUL@token}%
     \ifdim\dp0>\z@
        \raisebox{\dp0}{\underline{\phantom{\the\SOUL@token}}}%
        \whiten{1}\whiten{0}%
        \whiten{-1}\whiten{-2}%
        \llap{\the\SOUL@token}%
     \else
        \underline{\the\SOUL@token}%
     \fi}%
\SOUL@}
\makeatother

\newcommand*{\demp}{\fontfamily{lmtt}\selectfont}

\DeclareTextFontCommand{\textdemp}{\demp}

\begin{document}

\ifcomment
Multiline
comment
\fi
\ifcomment
\myul{Typesetting test}
% \color[rgb]{1,1,1}
$∑_i^n≠ 60º±∞π∆¬≈√j∫h≤≥µ$

$\CR \R\pro\ind\pro\gS\pro
\mqty[a&b\\c&d]$
$\pro\mathbb{P}$
$\dd{x}$

  \[
    \alpha(x)=\left\{
                \begin{array}{ll}
                  x\\
                  \frac{1}{1+e^{-kx}}\\
                  \frac{e^x-e^{-x}}{e^x+e^{-x}}
                \end{array}
              \right.
  \]

  $\expval{x}$
  
  $\chi_\rho(ghg\dmo)=\Tr(\rho_{ghg\dmo})=\Tr(\rho_g\circ\rho_h\circ\rho\dmo_g)=\Tr(\rho_h)\overset{\mbox{\scalebox{0.5}{$\Tr(AB)=\Tr(BA)$}}}{=}\chi_\rho(h)$
  	$\mathop{\oplus}_{\substack{x\in X}}$

$\mat(\rho_g)=(a_{ij}(g))_{\scriptsize \substack{1\leq i\leq d \\ 1\leq j\leq d}}$ et $\mat(\rho'_g)=(a'_{ij}(g))_{\scriptsize \substack{1\leq i'\leq d' \\ 1\leq j'\leq d'}}$



\[\int_a^b{\mathbb{R}^2}g(u, v)\dd{P_{XY}}(u, v)=\iint g(u,v) f_{XY}(u, v)\dd \lambda(u) \dd \lambda(v)\]
$$\lim_{x\to\infty} f(x)$$	
$$\iiiint_V \mu(t,u,v,w) \,dt\,du\,dv\,dw$$
$$\sum_{n=1}^{\infty} 2^{-n} = 1$$	
\begin{definition}
	Si $X$ et $Y$ sont 2 v.a. ou definit la \textsc{Covariance} entre $X$ et $Y$ comme
	$\cov(X,Y)\overset{\text{def}}{=}\E\left[(X-\E(X))(Y-\E(Y))\right]=\E(XY)-\E(X)\E(Y)$.
\end{definition}
\fi
\pagebreak

% \tableofcontents

% insert your code here
%\input{./algebra/main.tex}
%\input{./geometrie-differentielle/main.tex}
%\input{./probabilite/main.tex}
%\input{./analyse-fonctionnelle/main.tex}
% \input{./Analyse-convexe-et-dualite-en-optimisation/main.tex}
%\input{./tikz/main.tex}
%\input{./Theorie-du-distributions/main.tex}
%\input{./optimisation/mine.tex}
 \input{./modelisation/main.tex}

% yves.aubry@univ-tln.fr : algebra

\end{document}

%% !TEX encoding = UTF-8 Unicode
% !TEX TS-program = xelatex

\documentclass[french]{report}

%\usepackage[utf8]{inputenc}
%\usepackage[T1]{fontenc}
\usepackage{babel}


\newif\ifcomment
%\commenttrue # Show comments

\usepackage{physics}
\usepackage{amssymb}


\usepackage{amsthm}
% \usepackage{thmtools}
\usepackage{mathtools}
\usepackage{amsfonts}

\usepackage{color}

\usepackage{tikz}

\usepackage{geometry}
\geometry{a5paper, margin=0.1in, right=1cm}

\usepackage{dsfont}

\usepackage{graphicx}
\graphicspath{ {images/} }

\usepackage{faktor}

\usepackage{IEEEtrantools}
\usepackage{enumerate}   
\usepackage[PostScript=dvips]{"/Users/aware/Documents/Courses/diagrams"}


\newtheorem{theorem}{Théorème}[section]
\renewcommand{\thetheorem}{\arabic{theorem}}
\newtheorem{lemme}{Lemme}[section]
\renewcommand{\thelemme}{\arabic{lemme}}
\newtheorem{proposition}{Proposition}[section]
\renewcommand{\theproposition}{\arabic{proposition}}
\newtheorem{notations}{Notations}[section]
\newtheorem{problem}{Problème}[section]
\newtheorem{corollary}{Corollaire}[theorem]
\renewcommand{\thecorollary}{\arabic{corollary}}
\newtheorem{property}{Propriété}[section]
\newtheorem{objective}{Objectif}[section]

\theoremstyle{definition}
\newtheorem{definition}{Définition}[section]
\renewcommand{\thedefinition}{\arabic{definition}}
\newtheorem{exercise}{Exercice}[chapter]
\renewcommand{\theexercise}{\arabic{exercise}}
\newtheorem{example}{Exemple}[chapter]
\renewcommand{\theexample}{\arabic{example}}
\newtheorem*{solution}{Solution}
\newtheorem*{application}{Application}
\newtheorem*{notation}{Notation}
\newtheorem*{vocabulary}{Vocabulaire}
\newtheorem*{properties}{Propriétés}



\theoremstyle{remark}
\newtheorem*{remark}{Remarque}
\newtheorem*{rappel}{Rappel}


\usepackage{etoolbox}
\AtBeginEnvironment{exercise}{\small}
\AtBeginEnvironment{example}{\small}

\usepackage{cases}
\usepackage[red]{mypack}

\usepackage[framemethod=TikZ]{mdframed}

\definecolor{bg}{rgb}{0.4,0.25,0.95}
\definecolor{pagebg}{rgb}{0,0,0.5}
\surroundwithmdframed[
   topline=false,
   rightline=false,
   bottomline=false,
   leftmargin=\parindent,
   skipabove=8pt,
   skipbelow=8pt,
   linecolor=blue,
   innerbottommargin=10pt,
   % backgroundcolor=bg,font=\color{orange}\sffamily, fontcolor=white
]{definition}

\usepackage{empheq}
\usepackage[most]{tcolorbox}

\newtcbox{\mymath}[1][]{%
    nobeforeafter, math upper, tcbox raise base,
    enhanced, colframe=blue!30!black,
    colback=red!10, boxrule=1pt,
    #1}

\usepackage{unixode}


\DeclareMathOperator{\ord}{ord}
\DeclareMathOperator{\orb}{orb}
\DeclareMathOperator{\stab}{stab}
\DeclareMathOperator{\Stab}{stab}
\DeclareMathOperator{\ppcm}{ppcm}
\DeclareMathOperator{\conj}{Conj}
\DeclareMathOperator{\End}{End}
\DeclareMathOperator{\rot}{rot}
\DeclareMathOperator{\trs}{trace}
\DeclareMathOperator{\Ind}{Ind}
\DeclareMathOperator{\mat}{Mat}
\DeclareMathOperator{\id}{Id}
\DeclareMathOperator{\vect}{vect}
\DeclareMathOperator{\img}{img}
\DeclareMathOperator{\cov}{Cov}
\DeclareMathOperator{\dist}{dist}
\DeclareMathOperator{\irr}{Irr}
\DeclareMathOperator{\image}{Im}
\DeclareMathOperator{\pd}{\partial}
\DeclareMathOperator{\epi}{epi}
\DeclareMathOperator{\Argmin}{Argmin}
\DeclareMathOperator{\dom}{dom}
\DeclareMathOperator{\proj}{proj}
\DeclareMathOperator{\ctg}{ctg}
\DeclareMathOperator{\supp}{supp}
\DeclareMathOperator{\argmin}{argmin}
\DeclareMathOperator{\mult}{mult}
\DeclareMathOperator{\ch}{ch}
\DeclareMathOperator{\sh}{sh}
\DeclareMathOperator{\rang}{rang}
\DeclareMathOperator{\diam}{diam}
\DeclareMathOperator{\Epigraphe}{Epigraphe}




\usepackage{xcolor}
\everymath{\color{blue}}
%\everymath{\color[rgb]{0,1,1}}
%\pagecolor[rgb]{0,0,0.5}


\newcommand*{\pdtest}[3][]{\ensuremath{\frac{\partial^{#1} #2}{\partial #3}}}

\newcommand*{\deffunc}[6][]{\ensuremath{
\begin{array}{rcl}
#2 : #3 &\rightarrow& #4\\
#5 &\mapsto& #6
\end{array}
}}

\newcommand{\eqcolon}{\mathrel{\resizebox{\widthof{$\mathord{=}$}}{\height}{ $\!\!=\!\!\resizebox{1.2\width}{0.8\height}{\raisebox{0.23ex}{$\mathop{:}$}}\!\!$ }}}
\newcommand{\coloneq}{\mathrel{\resizebox{\widthof{$\mathord{=}$}}{\height}{ $\!\!\resizebox{1.2\width}{0.8\height}{\raisebox{0.23ex}{$\mathop{:}$}}\!\!=\!\!$ }}}
\newcommand{\eqcolonl}{\ensuremath{\mathrel{=\!\!\mathop{:}}}}
\newcommand{\coloneql}{\ensuremath{\mathrel{\mathop{:} \!\! =}}}
\newcommand{\vc}[1]{% inline column vector
  \left(\begin{smallmatrix}#1\end{smallmatrix}\right)%
}
\newcommand{\vr}[1]{% inline row vector
  \begin{smallmatrix}(\,#1\,)\end{smallmatrix}%
}
\makeatletter
\newcommand*{\defeq}{\ =\mathrel{\rlap{%
                     \raisebox{0.3ex}{$\m@th\cdot$}}%
                     \raisebox{-0.3ex}{$\m@th\cdot$}}%
                     }
\makeatother

\newcommand{\mathcircle}[1]{% inline row vector
 \overset{\circ}{#1}
}
\newcommand{\ulim}{% low limit
 \underline{\lim}
}
\newcommand{\ssi}{% iff
\iff
}
\newcommand{\ps}[2]{
\expval{#1 | #2}
}
\newcommand{\df}[1]{
\mqty{#1}
}
\newcommand{\n}[1]{
\norm{#1}
}
\newcommand{\sys}[1]{
\left\{\smqty{#1}\right.
}


\newcommand{\eqdef}{\ensuremath{\overset{\text{def}}=}}


\def\Circlearrowright{\ensuremath{%
  \rotatebox[origin=c]{230}{$\circlearrowright$}}}

\newcommand\ct[1]{\text{\rmfamily\upshape #1}}
\newcommand\question[1]{ {\color{red} ...!? \small #1}}
\newcommand\caz[1]{\left\{\begin{array} #1 \end{array}\right.}
\newcommand\const{\text{\rmfamily\upshape const}}
\newcommand\toP{ \overset{\pro}{\to}}
\newcommand\toPP{ \overset{\text{PP}}{\to}}
\newcommand{\oeq}{\mathrel{\text{\textcircled{$=$}}}}





\usepackage{xcolor}
% \usepackage[normalem]{ulem}
\usepackage{lipsum}
\makeatletter
% \newcommand\colorwave[1][blue]{\bgroup \markoverwith{\lower3.5\p@\hbox{\sixly \textcolor{#1}{\char58}}}\ULon}
%\font\sixly=lasy6 % does not re-load if already loaded, so no memory problem.

\newmdtheoremenv[
linewidth= 1pt,linecolor= blue,%
leftmargin=20,rightmargin=20,innertopmargin=0pt, innerrightmargin=40,%
tikzsetting = { draw=lightgray, line width = 0.3pt,dashed,%
dash pattern = on 15pt off 3pt},%
splittopskip=\topskip,skipbelow=\baselineskip,%
skipabove=\baselineskip,ntheorem,roundcorner=0pt,
% backgroundcolor=pagebg,font=\color{orange}\sffamily, fontcolor=white
]{examplebox}{Exemple}[section]



\newcommand\R{\mathbb{R}}
\newcommand\Z{\mathbb{Z}}
\newcommand\N{\mathbb{N}}
\newcommand\E{\mathbb{E}}
\newcommand\F{\mathcal{F}}
\newcommand\cH{\mathcal{H}}
\newcommand\V{\mathbb{V}}
\newcommand\dmo{ ^{-1} }
\newcommand\kapa{\kappa}
\newcommand\im{Im}
\newcommand\hs{\mathcal{H}}





\usepackage{soul}

\makeatletter
\newcommand*{\whiten}[1]{\llap{\textcolor{white}{{\the\SOUL@token}}\hspace{#1pt}}}
\DeclareRobustCommand*\myul{%
    \def\SOUL@everyspace{\underline{\space}\kern\z@}%
    \def\SOUL@everytoken{%
     \setbox0=\hbox{\the\SOUL@token}%
     \ifdim\dp0>\z@
        \raisebox{\dp0}{\underline{\phantom{\the\SOUL@token}}}%
        \whiten{1}\whiten{0}%
        \whiten{-1}\whiten{-2}%
        \llap{\the\SOUL@token}%
     \else
        \underline{\the\SOUL@token}%
     \fi}%
\SOUL@}
\makeatother

\newcommand*{\demp}{\fontfamily{lmtt}\selectfont}

\DeclareTextFontCommand{\textdemp}{\demp}

\begin{document}

\ifcomment
Multiline
comment
\fi
\ifcomment
\myul{Typesetting test}
% \color[rgb]{1,1,1}
$∑_i^n≠ 60º±∞π∆¬≈√j∫h≤≥µ$

$\CR \R\pro\ind\pro\gS\pro
\mqty[a&b\\c&d]$
$\pro\mathbb{P}$
$\dd{x}$

  \[
    \alpha(x)=\left\{
                \begin{array}{ll}
                  x\\
                  \frac{1}{1+e^{-kx}}\\
                  \frac{e^x-e^{-x}}{e^x+e^{-x}}
                \end{array}
              \right.
  \]

  $\expval{x}$
  
  $\chi_\rho(ghg\dmo)=\Tr(\rho_{ghg\dmo})=\Tr(\rho_g\circ\rho_h\circ\rho\dmo_g)=\Tr(\rho_h)\overset{\mbox{\scalebox{0.5}{$\Tr(AB)=\Tr(BA)$}}}{=}\chi_\rho(h)$
  	$\mathop{\oplus}_{\substack{x\in X}}$

$\mat(\rho_g)=(a_{ij}(g))_{\scriptsize \substack{1\leq i\leq d \\ 1\leq j\leq d}}$ et $\mat(\rho'_g)=(a'_{ij}(g))_{\scriptsize \substack{1\leq i'\leq d' \\ 1\leq j'\leq d'}}$



\[\int_a^b{\mathbb{R}^2}g(u, v)\dd{P_{XY}}(u, v)=\iint g(u,v) f_{XY}(u, v)\dd \lambda(u) \dd \lambda(v)\]
$$\lim_{x\to\infty} f(x)$$	
$$\iiiint_V \mu(t,u,v,w) \,dt\,du\,dv\,dw$$
$$\sum_{n=1}^{\infty} 2^{-n} = 1$$	
\begin{definition}
	Si $X$ et $Y$ sont 2 v.a. ou definit la \textsc{Covariance} entre $X$ et $Y$ comme
	$\cov(X,Y)\overset{\text{def}}{=}\E\left[(X-\E(X))(Y-\E(Y))\right]=\E(XY)-\E(X)\E(Y)$.
\end{definition}
\fi
\pagebreak

% \tableofcontents

% insert your code here
%\input{./algebra/main.tex}
%\input{./geometrie-differentielle/main.tex}
%\input{./probabilite/main.tex}
%\input{./analyse-fonctionnelle/main.tex}
% \input{./Analyse-convexe-et-dualite-en-optimisation/main.tex}
%\input{./tikz/main.tex}
%\input{./Theorie-du-distributions/main.tex}
%\input{./optimisation/mine.tex}
 \input{./modelisation/main.tex}

% yves.aubry@univ-tln.fr : algebra

\end{document}

%\input{./optimisation/mine.tex}
 % !TEX encoding = UTF-8 Unicode
% !TEX TS-program = xelatex

\documentclass[french]{report}

%\usepackage[utf8]{inputenc}
%\usepackage[T1]{fontenc}
\usepackage{babel}


\newif\ifcomment
%\commenttrue # Show comments

\usepackage{physics}
\usepackage{amssymb}


\usepackage{amsthm}
% \usepackage{thmtools}
\usepackage{mathtools}
\usepackage{amsfonts}

\usepackage{color}

\usepackage{tikz}

\usepackage{geometry}
\geometry{a5paper, margin=0.1in, right=1cm}

\usepackage{dsfont}

\usepackage{graphicx}
\graphicspath{ {images/} }

\usepackage{faktor}

\usepackage{IEEEtrantools}
\usepackage{enumerate}   
\usepackage[PostScript=dvips]{"/Users/aware/Documents/Courses/diagrams"}


\newtheorem{theorem}{Théorème}[section]
\renewcommand{\thetheorem}{\arabic{theorem}}
\newtheorem{lemme}{Lemme}[section]
\renewcommand{\thelemme}{\arabic{lemme}}
\newtheorem{proposition}{Proposition}[section]
\renewcommand{\theproposition}{\arabic{proposition}}
\newtheorem{notations}{Notations}[section]
\newtheorem{problem}{Problème}[section]
\newtheorem{corollary}{Corollaire}[theorem]
\renewcommand{\thecorollary}{\arabic{corollary}}
\newtheorem{property}{Propriété}[section]
\newtheorem{objective}{Objectif}[section]

\theoremstyle{definition}
\newtheorem{definition}{Définition}[section]
\renewcommand{\thedefinition}{\arabic{definition}}
\newtheorem{exercise}{Exercice}[chapter]
\renewcommand{\theexercise}{\arabic{exercise}}
\newtheorem{example}{Exemple}[chapter]
\renewcommand{\theexample}{\arabic{example}}
\newtheorem*{solution}{Solution}
\newtheorem*{application}{Application}
\newtheorem*{notation}{Notation}
\newtheorem*{vocabulary}{Vocabulaire}
\newtheorem*{properties}{Propriétés}



\theoremstyle{remark}
\newtheorem*{remark}{Remarque}
\newtheorem*{rappel}{Rappel}


\usepackage{etoolbox}
\AtBeginEnvironment{exercise}{\small}
\AtBeginEnvironment{example}{\small}

\usepackage{cases}
\usepackage[red]{mypack}

\usepackage[framemethod=TikZ]{mdframed}

\definecolor{bg}{rgb}{0.4,0.25,0.95}
\definecolor{pagebg}{rgb}{0,0,0.5}
\surroundwithmdframed[
   topline=false,
   rightline=false,
   bottomline=false,
   leftmargin=\parindent,
   skipabove=8pt,
   skipbelow=8pt,
   linecolor=blue,
   innerbottommargin=10pt,
   % backgroundcolor=bg,font=\color{orange}\sffamily, fontcolor=white
]{definition}

\usepackage{empheq}
\usepackage[most]{tcolorbox}

\newtcbox{\mymath}[1][]{%
    nobeforeafter, math upper, tcbox raise base,
    enhanced, colframe=blue!30!black,
    colback=red!10, boxrule=1pt,
    #1}

\usepackage{unixode}


\DeclareMathOperator{\ord}{ord}
\DeclareMathOperator{\orb}{orb}
\DeclareMathOperator{\stab}{stab}
\DeclareMathOperator{\Stab}{stab}
\DeclareMathOperator{\ppcm}{ppcm}
\DeclareMathOperator{\conj}{Conj}
\DeclareMathOperator{\End}{End}
\DeclareMathOperator{\rot}{rot}
\DeclareMathOperator{\trs}{trace}
\DeclareMathOperator{\Ind}{Ind}
\DeclareMathOperator{\mat}{Mat}
\DeclareMathOperator{\id}{Id}
\DeclareMathOperator{\vect}{vect}
\DeclareMathOperator{\img}{img}
\DeclareMathOperator{\cov}{Cov}
\DeclareMathOperator{\dist}{dist}
\DeclareMathOperator{\irr}{Irr}
\DeclareMathOperator{\image}{Im}
\DeclareMathOperator{\pd}{\partial}
\DeclareMathOperator{\epi}{epi}
\DeclareMathOperator{\Argmin}{Argmin}
\DeclareMathOperator{\dom}{dom}
\DeclareMathOperator{\proj}{proj}
\DeclareMathOperator{\ctg}{ctg}
\DeclareMathOperator{\supp}{supp}
\DeclareMathOperator{\argmin}{argmin}
\DeclareMathOperator{\mult}{mult}
\DeclareMathOperator{\ch}{ch}
\DeclareMathOperator{\sh}{sh}
\DeclareMathOperator{\rang}{rang}
\DeclareMathOperator{\diam}{diam}
\DeclareMathOperator{\Epigraphe}{Epigraphe}




\usepackage{xcolor}
\everymath{\color{blue}}
%\everymath{\color[rgb]{0,1,1}}
%\pagecolor[rgb]{0,0,0.5}


\newcommand*{\pdtest}[3][]{\ensuremath{\frac{\partial^{#1} #2}{\partial #3}}}

\newcommand*{\deffunc}[6][]{\ensuremath{
\begin{array}{rcl}
#2 : #3 &\rightarrow& #4\\
#5 &\mapsto& #6
\end{array}
}}

\newcommand{\eqcolon}{\mathrel{\resizebox{\widthof{$\mathord{=}$}}{\height}{ $\!\!=\!\!\resizebox{1.2\width}{0.8\height}{\raisebox{0.23ex}{$\mathop{:}$}}\!\!$ }}}
\newcommand{\coloneq}{\mathrel{\resizebox{\widthof{$\mathord{=}$}}{\height}{ $\!\!\resizebox{1.2\width}{0.8\height}{\raisebox{0.23ex}{$\mathop{:}$}}\!\!=\!\!$ }}}
\newcommand{\eqcolonl}{\ensuremath{\mathrel{=\!\!\mathop{:}}}}
\newcommand{\coloneql}{\ensuremath{\mathrel{\mathop{:} \!\! =}}}
\newcommand{\vc}[1]{% inline column vector
  \left(\begin{smallmatrix}#1\end{smallmatrix}\right)%
}
\newcommand{\vr}[1]{% inline row vector
  \begin{smallmatrix}(\,#1\,)\end{smallmatrix}%
}
\makeatletter
\newcommand*{\defeq}{\ =\mathrel{\rlap{%
                     \raisebox{0.3ex}{$\m@th\cdot$}}%
                     \raisebox{-0.3ex}{$\m@th\cdot$}}%
                     }
\makeatother

\newcommand{\mathcircle}[1]{% inline row vector
 \overset{\circ}{#1}
}
\newcommand{\ulim}{% low limit
 \underline{\lim}
}
\newcommand{\ssi}{% iff
\iff
}
\newcommand{\ps}[2]{
\expval{#1 | #2}
}
\newcommand{\df}[1]{
\mqty{#1}
}
\newcommand{\n}[1]{
\norm{#1}
}
\newcommand{\sys}[1]{
\left\{\smqty{#1}\right.
}


\newcommand{\eqdef}{\ensuremath{\overset{\text{def}}=}}


\def\Circlearrowright{\ensuremath{%
  \rotatebox[origin=c]{230}{$\circlearrowright$}}}

\newcommand\ct[1]{\text{\rmfamily\upshape #1}}
\newcommand\question[1]{ {\color{red} ...!? \small #1}}
\newcommand\caz[1]{\left\{\begin{array} #1 \end{array}\right.}
\newcommand\const{\text{\rmfamily\upshape const}}
\newcommand\toP{ \overset{\pro}{\to}}
\newcommand\toPP{ \overset{\text{PP}}{\to}}
\newcommand{\oeq}{\mathrel{\text{\textcircled{$=$}}}}





\usepackage{xcolor}
% \usepackage[normalem]{ulem}
\usepackage{lipsum}
\makeatletter
% \newcommand\colorwave[1][blue]{\bgroup \markoverwith{\lower3.5\p@\hbox{\sixly \textcolor{#1}{\char58}}}\ULon}
%\font\sixly=lasy6 % does not re-load if already loaded, so no memory problem.

\newmdtheoremenv[
linewidth= 1pt,linecolor= blue,%
leftmargin=20,rightmargin=20,innertopmargin=0pt, innerrightmargin=40,%
tikzsetting = { draw=lightgray, line width = 0.3pt,dashed,%
dash pattern = on 15pt off 3pt},%
splittopskip=\topskip,skipbelow=\baselineskip,%
skipabove=\baselineskip,ntheorem,roundcorner=0pt,
% backgroundcolor=pagebg,font=\color{orange}\sffamily, fontcolor=white
]{examplebox}{Exemple}[section]



\newcommand\R{\mathbb{R}}
\newcommand\Z{\mathbb{Z}}
\newcommand\N{\mathbb{N}}
\newcommand\E{\mathbb{E}}
\newcommand\F{\mathcal{F}}
\newcommand\cH{\mathcal{H}}
\newcommand\V{\mathbb{V}}
\newcommand\dmo{ ^{-1} }
\newcommand\kapa{\kappa}
\newcommand\im{Im}
\newcommand\hs{\mathcal{H}}





\usepackage{soul}

\makeatletter
\newcommand*{\whiten}[1]{\llap{\textcolor{white}{{\the\SOUL@token}}\hspace{#1pt}}}
\DeclareRobustCommand*\myul{%
    \def\SOUL@everyspace{\underline{\space}\kern\z@}%
    \def\SOUL@everytoken{%
     \setbox0=\hbox{\the\SOUL@token}%
     \ifdim\dp0>\z@
        \raisebox{\dp0}{\underline{\phantom{\the\SOUL@token}}}%
        \whiten{1}\whiten{0}%
        \whiten{-1}\whiten{-2}%
        \llap{\the\SOUL@token}%
     \else
        \underline{\the\SOUL@token}%
     \fi}%
\SOUL@}
\makeatother

\newcommand*{\demp}{\fontfamily{lmtt}\selectfont}

\DeclareTextFontCommand{\textdemp}{\demp}

\begin{document}

\ifcomment
Multiline
comment
\fi
\ifcomment
\myul{Typesetting test}
% \color[rgb]{1,1,1}
$∑_i^n≠ 60º±∞π∆¬≈√j∫h≤≥µ$

$\CR \R\pro\ind\pro\gS\pro
\mqty[a&b\\c&d]$
$\pro\mathbb{P}$
$\dd{x}$

  \[
    \alpha(x)=\left\{
                \begin{array}{ll}
                  x\\
                  \frac{1}{1+e^{-kx}}\\
                  \frac{e^x-e^{-x}}{e^x+e^{-x}}
                \end{array}
              \right.
  \]

  $\expval{x}$
  
  $\chi_\rho(ghg\dmo)=\Tr(\rho_{ghg\dmo})=\Tr(\rho_g\circ\rho_h\circ\rho\dmo_g)=\Tr(\rho_h)\overset{\mbox{\scalebox{0.5}{$\Tr(AB)=\Tr(BA)$}}}{=}\chi_\rho(h)$
  	$\mathop{\oplus}_{\substack{x\in X}}$

$\mat(\rho_g)=(a_{ij}(g))_{\scriptsize \substack{1\leq i\leq d \\ 1\leq j\leq d}}$ et $\mat(\rho'_g)=(a'_{ij}(g))_{\scriptsize \substack{1\leq i'\leq d' \\ 1\leq j'\leq d'}}$



\[\int_a^b{\mathbb{R}^2}g(u, v)\dd{P_{XY}}(u, v)=\iint g(u,v) f_{XY}(u, v)\dd \lambda(u) \dd \lambda(v)\]
$$\lim_{x\to\infty} f(x)$$	
$$\iiiint_V \mu(t,u,v,w) \,dt\,du\,dv\,dw$$
$$\sum_{n=1}^{\infty} 2^{-n} = 1$$	
\begin{definition}
	Si $X$ et $Y$ sont 2 v.a. ou definit la \textsc{Covariance} entre $X$ et $Y$ comme
	$\cov(X,Y)\overset{\text{def}}{=}\E\left[(X-\E(X))(Y-\E(Y))\right]=\E(XY)-\E(X)\E(Y)$.
\end{definition}
\fi
\pagebreak

% \tableofcontents

% insert your code here
%\input{./algebra/main.tex}
%\input{./geometrie-differentielle/main.tex}
%\input{./probabilite/main.tex}
%\input{./analyse-fonctionnelle/main.tex}
% \input{./Analyse-convexe-et-dualite-en-optimisation/main.tex}
%\input{./tikz/main.tex}
%\input{./Theorie-du-distributions/main.tex}
%\input{./optimisation/mine.tex}
 \input{./modelisation/main.tex}

% yves.aubry@univ-tln.fr : algebra

\end{document}


% yves.aubry@univ-tln.fr : algebra

\end{document}

%% !TEX encoding = UTF-8 Unicode
% !TEX TS-program = xelatex

\documentclass[french]{report}

%\usepackage[utf8]{inputenc}
%\usepackage[T1]{fontenc}
\usepackage{babel}


\newif\ifcomment
%\commenttrue # Show comments

\usepackage{physics}
\usepackage{amssymb}


\usepackage{amsthm}
% \usepackage{thmtools}
\usepackage{mathtools}
\usepackage{amsfonts}

\usepackage{color}

\usepackage{tikz}

\usepackage{geometry}
\geometry{a5paper, margin=0.1in, right=1cm}

\usepackage{dsfont}

\usepackage{graphicx}
\graphicspath{ {images/} }

\usepackage{faktor}

\usepackage{IEEEtrantools}
\usepackage{enumerate}   
\usepackage[PostScript=dvips]{"/Users/aware/Documents/Courses/diagrams"}


\newtheorem{theorem}{Théorème}[section]
\renewcommand{\thetheorem}{\arabic{theorem}}
\newtheorem{lemme}{Lemme}[section]
\renewcommand{\thelemme}{\arabic{lemme}}
\newtheorem{proposition}{Proposition}[section]
\renewcommand{\theproposition}{\arabic{proposition}}
\newtheorem{notations}{Notations}[section]
\newtheorem{problem}{Problème}[section]
\newtheorem{corollary}{Corollaire}[theorem]
\renewcommand{\thecorollary}{\arabic{corollary}}
\newtheorem{property}{Propriété}[section]
\newtheorem{objective}{Objectif}[section]

\theoremstyle{definition}
\newtheorem{definition}{Définition}[section]
\renewcommand{\thedefinition}{\arabic{definition}}
\newtheorem{exercise}{Exercice}[chapter]
\renewcommand{\theexercise}{\arabic{exercise}}
\newtheorem{example}{Exemple}[chapter]
\renewcommand{\theexample}{\arabic{example}}
\newtheorem*{solution}{Solution}
\newtheorem*{application}{Application}
\newtheorem*{notation}{Notation}
\newtheorem*{vocabulary}{Vocabulaire}
\newtheorem*{properties}{Propriétés}



\theoremstyle{remark}
\newtheorem*{remark}{Remarque}
\newtheorem*{rappel}{Rappel}


\usepackage{etoolbox}
\AtBeginEnvironment{exercise}{\small}
\AtBeginEnvironment{example}{\small}

\usepackage{cases}
\usepackage[red]{mypack}

\usepackage[framemethod=TikZ]{mdframed}

\definecolor{bg}{rgb}{0.4,0.25,0.95}
\definecolor{pagebg}{rgb}{0,0,0.5}
\surroundwithmdframed[
   topline=false,
   rightline=false,
   bottomline=false,
   leftmargin=\parindent,
   skipabove=8pt,
   skipbelow=8pt,
   linecolor=blue,
   innerbottommargin=10pt,
   % backgroundcolor=bg,font=\color{orange}\sffamily, fontcolor=white
]{definition}

\usepackage{empheq}
\usepackage[most]{tcolorbox}

\newtcbox{\mymath}[1][]{%
    nobeforeafter, math upper, tcbox raise base,
    enhanced, colframe=blue!30!black,
    colback=red!10, boxrule=1pt,
    #1}

\usepackage{unixode}


\DeclareMathOperator{\ord}{ord}
\DeclareMathOperator{\orb}{orb}
\DeclareMathOperator{\stab}{stab}
\DeclareMathOperator{\Stab}{stab}
\DeclareMathOperator{\ppcm}{ppcm}
\DeclareMathOperator{\conj}{Conj}
\DeclareMathOperator{\End}{End}
\DeclareMathOperator{\rot}{rot}
\DeclareMathOperator{\trs}{trace}
\DeclareMathOperator{\Ind}{Ind}
\DeclareMathOperator{\mat}{Mat}
\DeclareMathOperator{\id}{Id}
\DeclareMathOperator{\vect}{vect}
\DeclareMathOperator{\img}{img}
\DeclareMathOperator{\cov}{Cov}
\DeclareMathOperator{\dist}{dist}
\DeclareMathOperator{\irr}{Irr}
\DeclareMathOperator{\image}{Im}
\DeclareMathOperator{\pd}{\partial}
\DeclareMathOperator{\epi}{epi}
\DeclareMathOperator{\Argmin}{Argmin}
\DeclareMathOperator{\dom}{dom}
\DeclareMathOperator{\proj}{proj}
\DeclareMathOperator{\ctg}{ctg}
\DeclareMathOperator{\supp}{supp}
\DeclareMathOperator{\argmin}{argmin}
\DeclareMathOperator{\mult}{mult}
\DeclareMathOperator{\ch}{ch}
\DeclareMathOperator{\sh}{sh}
\DeclareMathOperator{\rang}{rang}
\DeclareMathOperator{\diam}{diam}
\DeclareMathOperator{\Epigraphe}{Epigraphe}




\usepackage{xcolor}
\everymath{\color{blue}}
%\everymath{\color[rgb]{0,1,1}}
%\pagecolor[rgb]{0,0,0.5}


\newcommand*{\pdtest}[3][]{\ensuremath{\frac{\partial^{#1} #2}{\partial #3}}}

\newcommand*{\deffunc}[6][]{\ensuremath{
\begin{array}{rcl}
#2 : #3 &\rightarrow& #4\\
#5 &\mapsto& #6
\end{array}
}}

\newcommand{\eqcolon}{\mathrel{\resizebox{\widthof{$\mathord{=}$}}{\height}{ $\!\!=\!\!\resizebox{1.2\width}{0.8\height}{\raisebox{0.23ex}{$\mathop{:}$}}\!\!$ }}}
\newcommand{\coloneq}{\mathrel{\resizebox{\widthof{$\mathord{=}$}}{\height}{ $\!\!\resizebox{1.2\width}{0.8\height}{\raisebox{0.23ex}{$\mathop{:}$}}\!\!=\!\!$ }}}
\newcommand{\eqcolonl}{\ensuremath{\mathrel{=\!\!\mathop{:}}}}
\newcommand{\coloneql}{\ensuremath{\mathrel{\mathop{:} \!\! =}}}
\newcommand{\vc}[1]{% inline column vector
  \left(\begin{smallmatrix}#1\end{smallmatrix}\right)%
}
\newcommand{\vr}[1]{% inline row vector
  \begin{smallmatrix}(\,#1\,)\end{smallmatrix}%
}
\makeatletter
\newcommand*{\defeq}{\ =\mathrel{\rlap{%
                     \raisebox{0.3ex}{$\m@th\cdot$}}%
                     \raisebox{-0.3ex}{$\m@th\cdot$}}%
                     }
\makeatother

\newcommand{\mathcircle}[1]{% inline row vector
 \overset{\circ}{#1}
}
\newcommand{\ulim}{% low limit
 \underline{\lim}
}
\newcommand{\ssi}{% iff
\iff
}
\newcommand{\ps}[2]{
\expval{#1 | #2}
}
\newcommand{\df}[1]{
\mqty{#1}
}
\newcommand{\n}[1]{
\norm{#1}
}
\newcommand{\sys}[1]{
\left\{\smqty{#1}\right.
}


\newcommand{\eqdef}{\ensuremath{\overset{\text{def}}=}}


\def\Circlearrowright{\ensuremath{%
  \rotatebox[origin=c]{230}{$\circlearrowright$}}}

\newcommand\ct[1]{\text{\rmfamily\upshape #1}}
\newcommand\question[1]{ {\color{red} ...!? \small #1}}
\newcommand\caz[1]{\left\{\begin{array} #1 \end{array}\right.}
\newcommand\const{\text{\rmfamily\upshape const}}
\newcommand\toP{ \overset{\pro}{\to}}
\newcommand\toPP{ \overset{\text{PP}}{\to}}
\newcommand{\oeq}{\mathrel{\text{\textcircled{$=$}}}}





\usepackage{xcolor}
% \usepackage[normalem]{ulem}
\usepackage{lipsum}
\makeatletter
% \newcommand\colorwave[1][blue]{\bgroup \markoverwith{\lower3.5\p@\hbox{\sixly \textcolor{#1}{\char58}}}\ULon}
%\font\sixly=lasy6 % does not re-load if already loaded, so no memory problem.

\newmdtheoremenv[
linewidth= 1pt,linecolor= blue,%
leftmargin=20,rightmargin=20,innertopmargin=0pt, innerrightmargin=40,%
tikzsetting = { draw=lightgray, line width = 0.3pt,dashed,%
dash pattern = on 15pt off 3pt},%
splittopskip=\topskip,skipbelow=\baselineskip,%
skipabove=\baselineskip,ntheorem,roundcorner=0pt,
% backgroundcolor=pagebg,font=\color{orange}\sffamily, fontcolor=white
]{examplebox}{Exemple}[section]



\newcommand\R{\mathbb{R}}
\newcommand\Z{\mathbb{Z}}
\newcommand\N{\mathbb{N}}
\newcommand\E{\mathbb{E}}
\newcommand\F{\mathcal{F}}
\newcommand\cH{\mathcal{H}}
\newcommand\V{\mathbb{V}}
\newcommand\dmo{ ^{-1} }
\newcommand\kapa{\kappa}
\newcommand\im{Im}
\newcommand\hs{\mathcal{H}}





\usepackage{soul}

\makeatletter
\newcommand*{\whiten}[1]{\llap{\textcolor{white}{{\the\SOUL@token}}\hspace{#1pt}}}
\DeclareRobustCommand*\myul{%
    \def\SOUL@everyspace{\underline{\space}\kern\z@}%
    \def\SOUL@everytoken{%
     \setbox0=\hbox{\the\SOUL@token}%
     \ifdim\dp0>\z@
        \raisebox{\dp0}{\underline{\phantom{\the\SOUL@token}}}%
        \whiten{1}\whiten{0}%
        \whiten{-1}\whiten{-2}%
        \llap{\the\SOUL@token}%
     \else
        \underline{\the\SOUL@token}%
     \fi}%
\SOUL@}
\makeatother

\newcommand*{\demp}{\fontfamily{lmtt}\selectfont}

\DeclareTextFontCommand{\textdemp}{\demp}

\begin{document}

\ifcomment
Multiline
comment
\fi
\ifcomment
\myul{Typesetting test}
% \color[rgb]{1,1,1}
$∑_i^n≠ 60º±∞π∆¬≈√j∫h≤≥µ$

$\CR \R\pro\ind\pro\gS\pro
\mqty[a&b\\c&d]$
$\pro\mathbb{P}$
$\dd{x}$

  \[
    \alpha(x)=\left\{
                \begin{array}{ll}
                  x\\
                  \frac{1}{1+e^{-kx}}\\
                  \frac{e^x-e^{-x}}{e^x+e^{-x}}
                \end{array}
              \right.
  \]

  $\expval{x}$
  
  $\chi_\rho(ghg\dmo)=\Tr(\rho_{ghg\dmo})=\Tr(\rho_g\circ\rho_h\circ\rho\dmo_g)=\Tr(\rho_h)\overset{\mbox{\scalebox{0.5}{$\Tr(AB)=\Tr(BA)$}}}{=}\chi_\rho(h)$
  	$\mathop{\oplus}_{\substack{x\in X}}$

$\mat(\rho_g)=(a_{ij}(g))_{\scriptsize \substack{1\leq i\leq d \\ 1\leq j\leq d}}$ et $\mat(\rho'_g)=(a'_{ij}(g))_{\scriptsize \substack{1\leq i'\leq d' \\ 1\leq j'\leq d'}}$



\[\int_a^b{\mathbb{R}^2}g(u, v)\dd{P_{XY}}(u, v)=\iint g(u,v) f_{XY}(u, v)\dd \lambda(u) \dd \lambda(v)\]
$$\lim_{x\to\infty} f(x)$$	
$$\iiiint_V \mu(t,u,v,w) \,dt\,du\,dv\,dw$$
$$\sum_{n=1}^{\infty} 2^{-n} = 1$$	
\begin{definition}
	Si $X$ et $Y$ sont 2 v.a. ou definit la \textsc{Covariance} entre $X$ et $Y$ comme
	$\cov(X,Y)\overset{\text{def}}{=}\E\left[(X-\E(X))(Y-\E(Y))\right]=\E(XY)-\E(X)\E(Y)$.
\end{definition}
\fi
\pagebreak

% \tableofcontents

% insert your code here
%% !TEX encoding = UTF-8 Unicode
% !TEX TS-program = xelatex

\documentclass[french]{report}

%\usepackage[utf8]{inputenc}
%\usepackage[T1]{fontenc}
\usepackage{babel}


\newif\ifcomment
%\commenttrue # Show comments

\usepackage{physics}
\usepackage{amssymb}


\usepackage{amsthm}
% \usepackage{thmtools}
\usepackage{mathtools}
\usepackage{amsfonts}

\usepackage{color}

\usepackage{tikz}

\usepackage{geometry}
\geometry{a5paper, margin=0.1in, right=1cm}

\usepackage{dsfont}

\usepackage{graphicx}
\graphicspath{ {images/} }

\usepackage{faktor}

\usepackage{IEEEtrantools}
\usepackage{enumerate}   
\usepackage[PostScript=dvips]{"/Users/aware/Documents/Courses/diagrams"}


\newtheorem{theorem}{Théorème}[section]
\renewcommand{\thetheorem}{\arabic{theorem}}
\newtheorem{lemme}{Lemme}[section]
\renewcommand{\thelemme}{\arabic{lemme}}
\newtheorem{proposition}{Proposition}[section]
\renewcommand{\theproposition}{\arabic{proposition}}
\newtheorem{notations}{Notations}[section]
\newtheorem{problem}{Problème}[section]
\newtheorem{corollary}{Corollaire}[theorem]
\renewcommand{\thecorollary}{\arabic{corollary}}
\newtheorem{property}{Propriété}[section]
\newtheorem{objective}{Objectif}[section]

\theoremstyle{definition}
\newtheorem{definition}{Définition}[section]
\renewcommand{\thedefinition}{\arabic{definition}}
\newtheorem{exercise}{Exercice}[chapter]
\renewcommand{\theexercise}{\arabic{exercise}}
\newtheorem{example}{Exemple}[chapter]
\renewcommand{\theexample}{\arabic{example}}
\newtheorem*{solution}{Solution}
\newtheorem*{application}{Application}
\newtheorem*{notation}{Notation}
\newtheorem*{vocabulary}{Vocabulaire}
\newtheorem*{properties}{Propriétés}



\theoremstyle{remark}
\newtheorem*{remark}{Remarque}
\newtheorem*{rappel}{Rappel}


\usepackage{etoolbox}
\AtBeginEnvironment{exercise}{\small}
\AtBeginEnvironment{example}{\small}

\usepackage{cases}
\usepackage[red]{mypack}

\usepackage[framemethod=TikZ]{mdframed}

\definecolor{bg}{rgb}{0.4,0.25,0.95}
\definecolor{pagebg}{rgb}{0,0,0.5}
\surroundwithmdframed[
   topline=false,
   rightline=false,
   bottomline=false,
   leftmargin=\parindent,
   skipabove=8pt,
   skipbelow=8pt,
   linecolor=blue,
   innerbottommargin=10pt,
   % backgroundcolor=bg,font=\color{orange}\sffamily, fontcolor=white
]{definition}

\usepackage{empheq}
\usepackage[most]{tcolorbox}

\newtcbox{\mymath}[1][]{%
    nobeforeafter, math upper, tcbox raise base,
    enhanced, colframe=blue!30!black,
    colback=red!10, boxrule=1pt,
    #1}

\usepackage{unixode}


\DeclareMathOperator{\ord}{ord}
\DeclareMathOperator{\orb}{orb}
\DeclareMathOperator{\stab}{stab}
\DeclareMathOperator{\Stab}{stab}
\DeclareMathOperator{\ppcm}{ppcm}
\DeclareMathOperator{\conj}{Conj}
\DeclareMathOperator{\End}{End}
\DeclareMathOperator{\rot}{rot}
\DeclareMathOperator{\trs}{trace}
\DeclareMathOperator{\Ind}{Ind}
\DeclareMathOperator{\mat}{Mat}
\DeclareMathOperator{\id}{Id}
\DeclareMathOperator{\vect}{vect}
\DeclareMathOperator{\img}{img}
\DeclareMathOperator{\cov}{Cov}
\DeclareMathOperator{\dist}{dist}
\DeclareMathOperator{\irr}{Irr}
\DeclareMathOperator{\image}{Im}
\DeclareMathOperator{\pd}{\partial}
\DeclareMathOperator{\epi}{epi}
\DeclareMathOperator{\Argmin}{Argmin}
\DeclareMathOperator{\dom}{dom}
\DeclareMathOperator{\proj}{proj}
\DeclareMathOperator{\ctg}{ctg}
\DeclareMathOperator{\supp}{supp}
\DeclareMathOperator{\argmin}{argmin}
\DeclareMathOperator{\mult}{mult}
\DeclareMathOperator{\ch}{ch}
\DeclareMathOperator{\sh}{sh}
\DeclareMathOperator{\rang}{rang}
\DeclareMathOperator{\diam}{diam}
\DeclareMathOperator{\Epigraphe}{Epigraphe}




\usepackage{xcolor}
\everymath{\color{blue}}
%\everymath{\color[rgb]{0,1,1}}
%\pagecolor[rgb]{0,0,0.5}


\newcommand*{\pdtest}[3][]{\ensuremath{\frac{\partial^{#1} #2}{\partial #3}}}

\newcommand*{\deffunc}[6][]{\ensuremath{
\begin{array}{rcl}
#2 : #3 &\rightarrow& #4\\
#5 &\mapsto& #6
\end{array}
}}

\newcommand{\eqcolon}{\mathrel{\resizebox{\widthof{$\mathord{=}$}}{\height}{ $\!\!=\!\!\resizebox{1.2\width}{0.8\height}{\raisebox{0.23ex}{$\mathop{:}$}}\!\!$ }}}
\newcommand{\coloneq}{\mathrel{\resizebox{\widthof{$\mathord{=}$}}{\height}{ $\!\!\resizebox{1.2\width}{0.8\height}{\raisebox{0.23ex}{$\mathop{:}$}}\!\!=\!\!$ }}}
\newcommand{\eqcolonl}{\ensuremath{\mathrel{=\!\!\mathop{:}}}}
\newcommand{\coloneql}{\ensuremath{\mathrel{\mathop{:} \!\! =}}}
\newcommand{\vc}[1]{% inline column vector
  \left(\begin{smallmatrix}#1\end{smallmatrix}\right)%
}
\newcommand{\vr}[1]{% inline row vector
  \begin{smallmatrix}(\,#1\,)\end{smallmatrix}%
}
\makeatletter
\newcommand*{\defeq}{\ =\mathrel{\rlap{%
                     \raisebox{0.3ex}{$\m@th\cdot$}}%
                     \raisebox{-0.3ex}{$\m@th\cdot$}}%
                     }
\makeatother

\newcommand{\mathcircle}[1]{% inline row vector
 \overset{\circ}{#1}
}
\newcommand{\ulim}{% low limit
 \underline{\lim}
}
\newcommand{\ssi}{% iff
\iff
}
\newcommand{\ps}[2]{
\expval{#1 | #2}
}
\newcommand{\df}[1]{
\mqty{#1}
}
\newcommand{\n}[1]{
\norm{#1}
}
\newcommand{\sys}[1]{
\left\{\smqty{#1}\right.
}


\newcommand{\eqdef}{\ensuremath{\overset{\text{def}}=}}


\def\Circlearrowright{\ensuremath{%
  \rotatebox[origin=c]{230}{$\circlearrowright$}}}

\newcommand\ct[1]{\text{\rmfamily\upshape #1}}
\newcommand\question[1]{ {\color{red} ...!? \small #1}}
\newcommand\caz[1]{\left\{\begin{array} #1 \end{array}\right.}
\newcommand\const{\text{\rmfamily\upshape const}}
\newcommand\toP{ \overset{\pro}{\to}}
\newcommand\toPP{ \overset{\text{PP}}{\to}}
\newcommand{\oeq}{\mathrel{\text{\textcircled{$=$}}}}





\usepackage{xcolor}
% \usepackage[normalem]{ulem}
\usepackage{lipsum}
\makeatletter
% \newcommand\colorwave[1][blue]{\bgroup \markoverwith{\lower3.5\p@\hbox{\sixly \textcolor{#1}{\char58}}}\ULon}
%\font\sixly=lasy6 % does not re-load if already loaded, so no memory problem.

\newmdtheoremenv[
linewidth= 1pt,linecolor= blue,%
leftmargin=20,rightmargin=20,innertopmargin=0pt, innerrightmargin=40,%
tikzsetting = { draw=lightgray, line width = 0.3pt,dashed,%
dash pattern = on 15pt off 3pt},%
splittopskip=\topskip,skipbelow=\baselineskip,%
skipabove=\baselineskip,ntheorem,roundcorner=0pt,
% backgroundcolor=pagebg,font=\color{orange}\sffamily, fontcolor=white
]{examplebox}{Exemple}[section]



\newcommand\R{\mathbb{R}}
\newcommand\Z{\mathbb{Z}}
\newcommand\N{\mathbb{N}}
\newcommand\E{\mathbb{E}}
\newcommand\F{\mathcal{F}}
\newcommand\cH{\mathcal{H}}
\newcommand\V{\mathbb{V}}
\newcommand\dmo{ ^{-1} }
\newcommand\kapa{\kappa}
\newcommand\im{Im}
\newcommand\hs{\mathcal{H}}





\usepackage{soul}

\makeatletter
\newcommand*{\whiten}[1]{\llap{\textcolor{white}{{\the\SOUL@token}}\hspace{#1pt}}}
\DeclareRobustCommand*\myul{%
    \def\SOUL@everyspace{\underline{\space}\kern\z@}%
    \def\SOUL@everytoken{%
     \setbox0=\hbox{\the\SOUL@token}%
     \ifdim\dp0>\z@
        \raisebox{\dp0}{\underline{\phantom{\the\SOUL@token}}}%
        \whiten{1}\whiten{0}%
        \whiten{-1}\whiten{-2}%
        \llap{\the\SOUL@token}%
     \else
        \underline{\the\SOUL@token}%
     \fi}%
\SOUL@}
\makeatother

\newcommand*{\demp}{\fontfamily{lmtt}\selectfont}

\DeclareTextFontCommand{\textdemp}{\demp}

\begin{document}

\ifcomment
Multiline
comment
\fi
\ifcomment
\myul{Typesetting test}
% \color[rgb]{1,1,1}
$∑_i^n≠ 60º±∞π∆¬≈√j∫h≤≥µ$

$\CR \R\pro\ind\pro\gS\pro
\mqty[a&b\\c&d]$
$\pro\mathbb{P}$
$\dd{x}$

  \[
    \alpha(x)=\left\{
                \begin{array}{ll}
                  x\\
                  \frac{1}{1+e^{-kx}}\\
                  \frac{e^x-e^{-x}}{e^x+e^{-x}}
                \end{array}
              \right.
  \]

  $\expval{x}$
  
  $\chi_\rho(ghg\dmo)=\Tr(\rho_{ghg\dmo})=\Tr(\rho_g\circ\rho_h\circ\rho\dmo_g)=\Tr(\rho_h)\overset{\mbox{\scalebox{0.5}{$\Tr(AB)=\Tr(BA)$}}}{=}\chi_\rho(h)$
  	$\mathop{\oplus}_{\substack{x\in X}}$

$\mat(\rho_g)=(a_{ij}(g))_{\scriptsize \substack{1\leq i\leq d \\ 1\leq j\leq d}}$ et $\mat(\rho'_g)=(a'_{ij}(g))_{\scriptsize \substack{1\leq i'\leq d' \\ 1\leq j'\leq d'}}$



\[\int_a^b{\mathbb{R}^2}g(u, v)\dd{P_{XY}}(u, v)=\iint g(u,v) f_{XY}(u, v)\dd \lambda(u) \dd \lambda(v)\]
$$\lim_{x\to\infty} f(x)$$	
$$\iiiint_V \mu(t,u,v,w) \,dt\,du\,dv\,dw$$
$$\sum_{n=1}^{\infty} 2^{-n} = 1$$	
\begin{definition}
	Si $X$ et $Y$ sont 2 v.a. ou definit la \textsc{Covariance} entre $X$ et $Y$ comme
	$\cov(X,Y)\overset{\text{def}}{=}\E\left[(X-\E(X))(Y-\E(Y))\right]=\E(XY)-\E(X)\E(Y)$.
\end{definition}
\fi
\pagebreak

% \tableofcontents

% insert your code here
%\input{./algebra/main.tex}
%\input{./geometrie-differentielle/main.tex}
%\input{./probabilite/main.tex}
%\input{./analyse-fonctionnelle/main.tex}
% \input{./Analyse-convexe-et-dualite-en-optimisation/main.tex}
%\input{./tikz/main.tex}
%\input{./Theorie-du-distributions/main.tex}
%\input{./optimisation/mine.tex}
 \input{./modelisation/main.tex}

% yves.aubry@univ-tln.fr : algebra

\end{document}

%% !TEX encoding = UTF-8 Unicode
% !TEX TS-program = xelatex

\documentclass[french]{report}

%\usepackage[utf8]{inputenc}
%\usepackage[T1]{fontenc}
\usepackage{babel}


\newif\ifcomment
%\commenttrue # Show comments

\usepackage{physics}
\usepackage{amssymb}


\usepackage{amsthm}
% \usepackage{thmtools}
\usepackage{mathtools}
\usepackage{amsfonts}

\usepackage{color}

\usepackage{tikz}

\usepackage{geometry}
\geometry{a5paper, margin=0.1in, right=1cm}

\usepackage{dsfont}

\usepackage{graphicx}
\graphicspath{ {images/} }

\usepackage{faktor}

\usepackage{IEEEtrantools}
\usepackage{enumerate}   
\usepackage[PostScript=dvips]{"/Users/aware/Documents/Courses/diagrams"}


\newtheorem{theorem}{Théorème}[section]
\renewcommand{\thetheorem}{\arabic{theorem}}
\newtheorem{lemme}{Lemme}[section]
\renewcommand{\thelemme}{\arabic{lemme}}
\newtheorem{proposition}{Proposition}[section]
\renewcommand{\theproposition}{\arabic{proposition}}
\newtheorem{notations}{Notations}[section]
\newtheorem{problem}{Problème}[section]
\newtheorem{corollary}{Corollaire}[theorem]
\renewcommand{\thecorollary}{\arabic{corollary}}
\newtheorem{property}{Propriété}[section]
\newtheorem{objective}{Objectif}[section]

\theoremstyle{definition}
\newtheorem{definition}{Définition}[section]
\renewcommand{\thedefinition}{\arabic{definition}}
\newtheorem{exercise}{Exercice}[chapter]
\renewcommand{\theexercise}{\arabic{exercise}}
\newtheorem{example}{Exemple}[chapter]
\renewcommand{\theexample}{\arabic{example}}
\newtheorem*{solution}{Solution}
\newtheorem*{application}{Application}
\newtheorem*{notation}{Notation}
\newtheorem*{vocabulary}{Vocabulaire}
\newtheorem*{properties}{Propriétés}



\theoremstyle{remark}
\newtheorem*{remark}{Remarque}
\newtheorem*{rappel}{Rappel}


\usepackage{etoolbox}
\AtBeginEnvironment{exercise}{\small}
\AtBeginEnvironment{example}{\small}

\usepackage{cases}
\usepackage[red]{mypack}

\usepackage[framemethod=TikZ]{mdframed}

\definecolor{bg}{rgb}{0.4,0.25,0.95}
\definecolor{pagebg}{rgb}{0,0,0.5}
\surroundwithmdframed[
   topline=false,
   rightline=false,
   bottomline=false,
   leftmargin=\parindent,
   skipabove=8pt,
   skipbelow=8pt,
   linecolor=blue,
   innerbottommargin=10pt,
   % backgroundcolor=bg,font=\color{orange}\sffamily, fontcolor=white
]{definition}

\usepackage{empheq}
\usepackage[most]{tcolorbox}

\newtcbox{\mymath}[1][]{%
    nobeforeafter, math upper, tcbox raise base,
    enhanced, colframe=blue!30!black,
    colback=red!10, boxrule=1pt,
    #1}

\usepackage{unixode}


\DeclareMathOperator{\ord}{ord}
\DeclareMathOperator{\orb}{orb}
\DeclareMathOperator{\stab}{stab}
\DeclareMathOperator{\Stab}{stab}
\DeclareMathOperator{\ppcm}{ppcm}
\DeclareMathOperator{\conj}{Conj}
\DeclareMathOperator{\End}{End}
\DeclareMathOperator{\rot}{rot}
\DeclareMathOperator{\trs}{trace}
\DeclareMathOperator{\Ind}{Ind}
\DeclareMathOperator{\mat}{Mat}
\DeclareMathOperator{\id}{Id}
\DeclareMathOperator{\vect}{vect}
\DeclareMathOperator{\img}{img}
\DeclareMathOperator{\cov}{Cov}
\DeclareMathOperator{\dist}{dist}
\DeclareMathOperator{\irr}{Irr}
\DeclareMathOperator{\image}{Im}
\DeclareMathOperator{\pd}{\partial}
\DeclareMathOperator{\epi}{epi}
\DeclareMathOperator{\Argmin}{Argmin}
\DeclareMathOperator{\dom}{dom}
\DeclareMathOperator{\proj}{proj}
\DeclareMathOperator{\ctg}{ctg}
\DeclareMathOperator{\supp}{supp}
\DeclareMathOperator{\argmin}{argmin}
\DeclareMathOperator{\mult}{mult}
\DeclareMathOperator{\ch}{ch}
\DeclareMathOperator{\sh}{sh}
\DeclareMathOperator{\rang}{rang}
\DeclareMathOperator{\diam}{diam}
\DeclareMathOperator{\Epigraphe}{Epigraphe}




\usepackage{xcolor}
\everymath{\color{blue}}
%\everymath{\color[rgb]{0,1,1}}
%\pagecolor[rgb]{0,0,0.5}


\newcommand*{\pdtest}[3][]{\ensuremath{\frac{\partial^{#1} #2}{\partial #3}}}

\newcommand*{\deffunc}[6][]{\ensuremath{
\begin{array}{rcl}
#2 : #3 &\rightarrow& #4\\
#5 &\mapsto& #6
\end{array}
}}

\newcommand{\eqcolon}{\mathrel{\resizebox{\widthof{$\mathord{=}$}}{\height}{ $\!\!=\!\!\resizebox{1.2\width}{0.8\height}{\raisebox{0.23ex}{$\mathop{:}$}}\!\!$ }}}
\newcommand{\coloneq}{\mathrel{\resizebox{\widthof{$\mathord{=}$}}{\height}{ $\!\!\resizebox{1.2\width}{0.8\height}{\raisebox{0.23ex}{$\mathop{:}$}}\!\!=\!\!$ }}}
\newcommand{\eqcolonl}{\ensuremath{\mathrel{=\!\!\mathop{:}}}}
\newcommand{\coloneql}{\ensuremath{\mathrel{\mathop{:} \!\! =}}}
\newcommand{\vc}[1]{% inline column vector
  \left(\begin{smallmatrix}#1\end{smallmatrix}\right)%
}
\newcommand{\vr}[1]{% inline row vector
  \begin{smallmatrix}(\,#1\,)\end{smallmatrix}%
}
\makeatletter
\newcommand*{\defeq}{\ =\mathrel{\rlap{%
                     \raisebox{0.3ex}{$\m@th\cdot$}}%
                     \raisebox{-0.3ex}{$\m@th\cdot$}}%
                     }
\makeatother

\newcommand{\mathcircle}[1]{% inline row vector
 \overset{\circ}{#1}
}
\newcommand{\ulim}{% low limit
 \underline{\lim}
}
\newcommand{\ssi}{% iff
\iff
}
\newcommand{\ps}[2]{
\expval{#1 | #2}
}
\newcommand{\df}[1]{
\mqty{#1}
}
\newcommand{\n}[1]{
\norm{#1}
}
\newcommand{\sys}[1]{
\left\{\smqty{#1}\right.
}


\newcommand{\eqdef}{\ensuremath{\overset{\text{def}}=}}


\def\Circlearrowright{\ensuremath{%
  \rotatebox[origin=c]{230}{$\circlearrowright$}}}

\newcommand\ct[1]{\text{\rmfamily\upshape #1}}
\newcommand\question[1]{ {\color{red} ...!? \small #1}}
\newcommand\caz[1]{\left\{\begin{array} #1 \end{array}\right.}
\newcommand\const{\text{\rmfamily\upshape const}}
\newcommand\toP{ \overset{\pro}{\to}}
\newcommand\toPP{ \overset{\text{PP}}{\to}}
\newcommand{\oeq}{\mathrel{\text{\textcircled{$=$}}}}





\usepackage{xcolor}
% \usepackage[normalem]{ulem}
\usepackage{lipsum}
\makeatletter
% \newcommand\colorwave[1][blue]{\bgroup \markoverwith{\lower3.5\p@\hbox{\sixly \textcolor{#1}{\char58}}}\ULon}
%\font\sixly=lasy6 % does not re-load if already loaded, so no memory problem.

\newmdtheoremenv[
linewidth= 1pt,linecolor= blue,%
leftmargin=20,rightmargin=20,innertopmargin=0pt, innerrightmargin=40,%
tikzsetting = { draw=lightgray, line width = 0.3pt,dashed,%
dash pattern = on 15pt off 3pt},%
splittopskip=\topskip,skipbelow=\baselineskip,%
skipabove=\baselineskip,ntheorem,roundcorner=0pt,
% backgroundcolor=pagebg,font=\color{orange}\sffamily, fontcolor=white
]{examplebox}{Exemple}[section]



\newcommand\R{\mathbb{R}}
\newcommand\Z{\mathbb{Z}}
\newcommand\N{\mathbb{N}}
\newcommand\E{\mathbb{E}}
\newcommand\F{\mathcal{F}}
\newcommand\cH{\mathcal{H}}
\newcommand\V{\mathbb{V}}
\newcommand\dmo{ ^{-1} }
\newcommand\kapa{\kappa}
\newcommand\im{Im}
\newcommand\hs{\mathcal{H}}





\usepackage{soul}

\makeatletter
\newcommand*{\whiten}[1]{\llap{\textcolor{white}{{\the\SOUL@token}}\hspace{#1pt}}}
\DeclareRobustCommand*\myul{%
    \def\SOUL@everyspace{\underline{\space}\kern\z@}%
    \def\SOUL@everytoken{%
     \setbox0=\hbox{\the\SOUL@token}%
     \ifdim\dp0>\z@
        \raisebox{\dp0}{\underline{\phantom{\the\SOUL@token}}}%
        \whiten{1}\whiten{0}%
        \whiten{-1}\whiten{-2}%
        \llap{\the\SOUL@token}%
     \else
        \underline{\the\SOUL@token}%
     \fi}%
\SOUL@}
\makeatother

\newcommand*{\demp}{\fontfamily{lmtt}\selectfont}

\DeclareTextFontCommand{\textdemp}{\demp}

\begin{document}

\ifcomment
Multiline
comment
\fi
\ifcomment
\myul{Typesetting test}
% \color[rgb]{1,1,1}
$∑_i^n≠ 60º±∞π∆¬≈√j∫h≤≥µ$

$\CR \R\pro\ind\pro\gS\pro
\mqty[a&b\\c&d]$
$\pro\mathbb{P}$
$\dd{x}$

  \[
    \alpha(x)=\left\{
                \begin{array}{ll}
                  x\\
                  \frac{1}{1+e^{-kx}}\\
                  \frac{e^x-e^{-x}}{e^x+e^{-x}}
                \end{array}
              \right.
  \]

  $\expval{x}$
  
  $\chi_\rho(ghg\dmo)=\Tr(\rho_{ghg\dmo})=\Tr(\rho_g\circ\rho_h\circ\rho\dmo_g)=\Tr(\rho_h)\overset{\mbox{\scalebox{0.5}{$\Tr(AB)=\Tr(BA)$}}}{=}\chi_\rho(h)$
  	$\mathop{\oplus}_{\substack{x\in X}}$

$\mat(\rho_g)=(a_{ij}(g))_{\scriptsize \substack{1\leq i\leq d \\ 1\leq j\leq d}}$ et $\mat(\rho'_g)=(a'_{ij}(g))_{\scriptsize \substack{1\leq i'\leq d' \\ 1\leq j'\leq d'}}$



\[\int_a^b{\mathbb{R}^2}g(u, v)\dd{P_{XY}}(u, v)=\iint g(u,v) f_{XY}(u, v)\dd \lambda(u) \dd \lambda(v)\]
$$\lim_{x\to\infty} f(x)$$	
$$\iiiint_V \mu(t,u,v,w) \,dt\,du\,dv\,dw$$
$$\sum_{n=1}^{\infty} 2^{-n} = 1$$	
\begin{definition}
	Si $X$ et $Y$ sont 2 v.a. ou definit la \textsc{Covariance} entre $X$ et $Y$ comme
	$\cov(X,Y)\overset{\text{def}}{=}\E\left[(X-\E(X))(Y-\E(Y))\right]=\E(XY)-\E(X)\E(Y)$.
\end{definition}
\fi
\pagebreak

% \tableofcontents

% insert your code here
%\input{./algebra/main.tex}
%\input{./geometrie-differentielle/main.tex}
%\input{./probabilite/main.tex}
%\input{./analyse-fonctionnelle/main.tex}
% \input{./Analyse-convexe-et-dualite-en-optimisation/main.tex}
%\input{./tikz/main.tex}
%\input{./Theorie-du-distributions/main.tex}
%\input{./optimisation/mine.tex}
 \input{./modelisation/main.tex}

% yves.aubry@univ-tln.fr : algebra

\end{document}

%% !TEX encoding = UTF-8 Unicode
% !TEX TS-program = xelatex

\documentclass[french]{report}

%\usepackage[utf8]{inputenc}
%\usepackage[T1]{fontenc}
\usepackage{babel}


\newif\ifcomment
%\commenttrue # Show comments

\usepackage{physics}
\usepackage{amssymb}


\usepackage{amsthm}
% \usepackage{thmtools}
\usepackage{mathtools}
\usepackage{amsfonts}

\usepackage{color}

\usepackage{tikz}

\usepackage{geometry}
\geometry{a5paper, margin=0.1in, right=1cm}

\usepackage{dsfont}

\usepackage{graphicx}
\graphicspath{ {images/} }

\usepackage{faktor}

\usepackage{IEEEtrantools}
\usepackage{enumerate}   
\usepackage[PostScript=dvips]{"/Users/aware/Documents/Courses/diagrams"}


\newtheorem{theorem}{Théorème}[section]
\renewcommand{\thetheorem}{\arabic{theorem}}
\newtheorem{lemme}{Lemme}[section]
\renewcommand{\thelemme}{\arabic{lemme}}
\newtheorem{proposition}{Proposition}[section]
\renewcommand{\theproposition}{\arabic{proposition}}
\newtheorem{notations}{Notations}[section]
\newtheorem{problem}{Problème}[section]
\newtheorem{corollary}{Corollaire}[theorem]
\renewcommand{\thecorollary}{\arabic{corollary}}
\newtheorem{property}{Propriété}[section]
\newtheorem{objective}{Objectif}[section]

\theoremstyle{definition}
\newtheorem{definition}{Définition}[section]
\renewcommand{\thedefinition}{\arabic{definition}}
\newtheorem{exercise}{Exercice}[chapter]
\renewcommand{\theexercise}{\arabic{exercise}}
\newtheorem{example}{Exemple}[chapter]
\renewcommand{\theexample}{\arabic{example}}
\newtheorem*{solution}{Solution}
\newtheorem*{application}{Application}
\newtheorem*{notation}{Notation}
\newtheorem*{vocabulary}{Vocabulaire}
\newtheorem*{properties}{Propriétés}



\theoremstyle{remark}
\newtheorem*{remark}{Remarque}
\newtheorem*{rappel}{Rappel}


\usepackage{etoolbox}
\AtBeginEnvironment{exercise}{\small}
\AtBeginEnvironment{example}{\small}

\usepackage{cases}
\usepackage[red]{mypack}

\usepackage[framemethod=TikZ]{mdframed}

\definecolor{bg}{rgb}{0.4,0.25,0.95}
\definecolor{pagebg}{rgb}{0,0,0.5}
\surroundwithmdframed[
   topline=false,
   rightline=false,
   bottomline=false,
   leftmargin=\parindent,
   skipabove=8pt,
   skipbelow=8pt,
   linecolor=blue,
   innerbottommargin=10pt,
   % backgroundcolor=bg,font=\color{orange}\sffamily, fontcolor=white
]{definition}

\usepackage{empheq}
\usepackage[most]{tcolorbox}

\newtcbox{\mymath}[1][]{%
    nobeforeafter, math upper, tcbox raise base,
    enhanced, colframe=blue!30!black,
    colback=red!10, boxrule=1pt,
    #1}

\usepackage{unixode}


\DeclareMathOperator{\ord}{ord}
\DeclareMathOperator{\orb}{orb}
\DeclareMathOperator{\stab}{stab}
\DeclareMathOperator{\Stab}{stab}
\DeclareMathOperator{\ppcm}{ppcm}
\DeclareMathOperator{\conj}{Conj}
\DeclareMathOperator{\End}{End}
\DeclareMathOperator{\rot}{rot}
\DeclareMathOperator{\trs}{trace}
\DeclareMathOperator{\Ind}{Ind}
\DeclareMathOperator{\mat}{Mat}
\DeclareMathOperator{\id}{Id}
\DeclareMathOperator{\vect}{vect}
\DeclareMathOperator{\img}{img}
\DeclareMathOperator{\cov}{Cov}
\DeclareMathOperator{\dist}{dist}
\DeclareMathOperator{\irr}{Irr}
\DeclareMathOperator{\image}{Im}
\DeclareMathOperator{\pd}{\partial}
\DeclareMathOperator{\epi}{epi}
\DeclareMathOperator{\Argmin}{Argmin}
\DeclareMathOperator{\dom}{dom}
\DeclareMathOperator{\proj}{proj}
\DeclareMathOperator{\ctg}{ctg}
\DeclareMathOperator{\supp}{supp}
\DeclareMathOperator{\argmin}{argmin}
\DeclareMathOperator{\mult}{mult}
\DeclareMathOperator{\ch}{ch}
\DeclareMathOperator{\sh}{sh}
\DeclareMathOperator{\rang}{rang}
\DeclareMathOperator{\diam}{diam}
\DeclareMathOperator{\Epigraphe}{Epigraphe}




\usepackage{xcolor}
\everymath{\color{blue}}
%\everymath{\color[rgb]{0,1,1}}
%\pagecolor[rgb]{0,0,0.5}


\newcommand*{\pdtest}[3][]{\ensuremath{\frac{\partial^{#1} #2}{\partial #3}}}

\newcommand*{\deffunc}[6][]{\ensuremath{
\begin{array}{rcl}
#2 : #3 &\rightarrow& #4\\
#5 &\mapsto& #6
\end{array}
}}

\newcommand{\eqcolon}{\mathrel{\resizebox{\widthof{$\mathord{=}$}}{\height}{ $\!\!=\!\!\resizebox{1.2\width}{0.8\height}{\raisebox{0.23ex}{$\mathop{:}$}}\!\!$ }}}
\newcommand{\coloneq}{\mathrel{\resizebox{\widthof{$\mathord{=}$}}{\height}{ $\!\!\resizebox{1.2\width}{0.8\height}{\raisebox{0.23ex}{$\mathop{:}$}}\!\!=\!\!$ }}}
\newcommand{\eqcolonl}{\ensuremath{\mathrel{=\!\!\mathop{:}}}}
\newcommand{\coloneql}{\ensuremath{\mathrel{\mathop{:} \!\! =}}}
\newcommand{\vc}[1]{% inline column vector
  \left(\begin{smallmatrix}#1\end{smallmatrix}\right)%
}
\newcommand{\vr}[1]{% inline row vector
  \begin{smallmatrix}(\,#1\,)\end{smallmatrix}%
}
\makeatletter
\newcommand*{\defeq}{\ =\mathrel{\rlap{%
                     \raisebox{0.3ex}{$\m@th\cdot$}}%
                     \raisebox{-0.3ex}{$\m@th\cdot$}}%
                     }
\makeatother

\newcommand{\mathcircle}[1]{% inline row vector
 \overset{\circ}{#1}
}
\newcommand{\ulim}{% low limit
 \underline{\lim}
}
\newcommand{\ssi}{% iff
\iff
}
\newcommand{\ps}[2]{
\expval{#1 | #2}
}
\newcommand{\df}[1]{
\mqty{#1}
}
\newcommand{\n}[1]{
\norm{#1}
}
\newcommand{\sys}[1]{
\left\{\smqty{#1}\right.
}


\newcommand{\eqdef}{\ensuremath{\overset{\text{def}}=}}


\def\Circlearrowright{\ensuremath{%
  \rotatebox[origin=c]{230}{$\circlearrowright$}}}

\newcommand\ct[1]{\text{\rmfamily\upshape #1}}
\newcommand\question[1]{ {\color{red} ...!? \small #1}}
\newcommand\caz[1]{\left\{\begin{array} #1 \end{array}\right.}
\newcommand\const{\text{\rmfamily\upshape const}}
\newcommand\toP{ \overset{\pro}{\to}}
\newcommand\toPP{ \overset{\text{PP}}{\to}}
\newcommand{\oeq}{\mathrel{\text{\textcircled{$=$}}}}





\usepackage{xcolor}
% \usepackage[normalem]{ulem}
\usepackage{lipsum}
\makeatletter
% \newcommand\colorwave[1][blue]{\bgroup \markoverwith{\lower3.5\p@\hbox{\sixly \textcolor{#1}{\char58}}}\ULon}
%\font\sixly=lasy6 % does not re-load if already loaded, so no memory problem.

\newmdtheoremenv[
linewidth= 1pt,linecolor= blue,%
leftmargin=20,rightmargin=20,innertopmargin=0pt, innerrightmargin=40,%
tikzsetting = { draw=lightgray, line width = 0.3pt,dashed,%
dash pattern = on 15pt off 3pt},%
splittopskip=\topskip,skipbelow=\baselineskip,%
skipabove=\baselineskip,ntheorem,roundcorner=0pt,
% backgroundcolor=pagebg,font=\color{orange}\sffamily, fontcolor=white
]{examplebox}{Exemple}[section]



\newcommand\R{\mathbb{R}}
\newcommand\Z{\mathbb{Z}}
\newcommand\N{\mathbb{N}}
\newcommand\E{\mathbb{E}}
\newcommand\F{\mathcal{F}}
\newcommand\cH{\mathcal{H}}
\newcommand\V{\mathbb{V}}
\newcommand\dmo{ ^{-1} }
\newcommand\kapa{\kappa}
\newcommand\im{Im}
\newcommand\hs{\mathcal{H}}





\usepackage{soul}

\makeatletter
\newcommand*{\whiten}[1]{\llap{\textcolor{white}{{\the\SOUL@token}}\hspace{#1pt}}}
\DeclareRobustCommand*\myul{%
    \def\SOUL@everyspace{\underline{\space}\kern\z@}%
    \def\SOUL@everytoken{%
     \setbox0=\hbox{\the\SOUL@token}%
     \ifdim\dp0>\z@
        \raisebox{\dp0}{\underline{\phantom{\the\SOUL@token}}}%
        \whiten{1}\whiten{0}%
        \whiten{-1}\whiten{-2}%
        \llap{\the\SOUL@token}%
     \else
        \underline{\the\SOUL@token}%
     \fi}%
\SOUL@}
\makeatother

\newcommand*{\demp}{\fontfamily{lmtt}\selectfont}

\DeclareTextFontCommand{\textdemp}{\demp}

\begin{document}

\ifcomment
Multiline
comment
\fi
\ifcomment
\myul{Typesetting test}
% \color[rgb]{1,1,1}
$∑_i^n≠ 60º±∞π∆¬≈√j∫h≤≥µ$

$\CR \R\pro\ind\pro\gS\pro
\mqty[a&b\\c&d]$
$\pro\mathbb{P}$
$\dd{x}$

  \[
    \alpha(x)=\left\{
                \begin{array}{ll}
                  x\\
                  \frac{1}{1+e^{-kx}}\\
                  \frac{e^x-e^{-x}}{e^x+e^{-x}}
                \end{array}
              \right.
  \]

  $\expval{x}$
  
  $\chi_\rho(ghg\dmo)=\Tr(\rho_{ghg\dmo})=\Tr(\rho_g\circ\rho_h\circ\rho\dmo_g)=\Tr(\rho_h)\overset{\mbox{\scalebox{0.5}{$\Tr(AB)=\Tr(BA)$}}}{=}\chi_\rho(h)$
  	$\mathop{\oplus}_{\substack{x\in X}}$

$\mat(\rho_g)=(a_{ij}(g))_{\scriptsize \substack{1\leq i\leq d \\ 1\leq j\leq d}}$ et $\mat(\rho'_g)=(a'_{ij}(g))_{\scriptsize \substack{1\leq i'\leq d' \\ 1\leq j'\leq d'}}$



\[\int_a^b{\mathbb{R}^2}g(u, v)\dd{P_{XY}}(u, v)=\iint g(u,v) f_{XY}(u, v)\dd \lambda(u) \dd \lambda(v)\]
$$\lim_{x\to\infty} f(x)$$	
$$\iiiint_V \mu(t,u,v,w) \,dt\,du\,dv\,dw$$
$$\sum_{n=1}^{\infty} 2^{-n} = 1$$	
\begin{definition}
	Si $X$ et $Y$ sont 2 v.a. ou definit la \textsc{Covariance} entre $X$ et $Y$ comme
	$\cov(X,Y)\overset{\text{def}}{=}\E\left[(X-\E(X))(Y-\E(Y))\right]=\E(XY)-\E(X)\E(Y)$.
\end{definition}
\fi
\pagebreak

% \tableofcontents

% insert your code here
%\input{./algebra/main.tex}
%\input{./geometrie-differentielle/main.tex}
%\input{./probabilite/main.tex}
%\input{./analyse-fonctionnelle/main.tex}
% \input{./Analyse-convexe-et-dualite-en-optimisation/main.tex}
%\input{./tikz/main.tex}
%\input{./Theorie-du-distributions/main.tex}
%\input{./optimisation/mine.tex}
 \input{./modelisation/main.tex}

% yves.aubry@univ-tln.fr : algebra

\end{document}

%% !TEX encoding = UTF-8 Unicode
% !TEX TS-program = xelatex

\documentclass[french]{report}

%\usepackage[utf8]{inputenc}
%\usepackage[T1]{fontenc}
\usepackage{babel}


\newif\ifcomment
%\commenttrue # Show comments

\usepackage{physics}
\usepackage{amssymb}


\usepackage{amsthm}
% \usepackage{thmtools}
\usepackage{mathtools}
\usepackage{amsfonts}

\usepackage{color}

\usepackage{tikz}

\usepackage{geometry}
\geometry{a5paper, margin=0.1in, right=1cm}

\usepackage{dsfont}

\usepackage{graphicx}
\graphicspath{ {images/} }

\usepackage{faktor}

\usepackage{IEEEtrantools}
\usepackage{enumerate}   
\usepackage[PostScript=dvips]{"/Users/aware/Documents/Courses/diagrams"}


\newtheorem{theorem}{Théorème}[section]
\renewcommand{\thetheorem}{\arabic{theorem}}
\newtheorem{lemme}{Lemme}[section]
\renewcommand{\thelemme}{\arabic{lemme}}
\newtheorem{proposition}{Proposition}[section]
\renewcommand{\theproposition}{\arabic{proposition}}
\newtheorem{notations}{Notations}[section]
\newtheorem{problem}{Problème}[section]
\newtheorem{corollary}{Corollaire}[theorem]
\renewcommand{\thecorollary}{\arabic{corollary}}
\newtheorem{property}{Propriété}[section]
\newtheorem{objective}{Objectif}[section]

\theoremstyle{definition}
\newtheorem{definition}{Définition}[section]
\renewcommand{\thedefinition}{\arabic{definition}}
\newtheorem{exercise}{Exercice}[chapter]
\renewcommand{\theexercise}{\arabic{exercise}}
\newtheorem{example}{Exemple}[chapter]
\renewcommand{\theexample}{\arabic{example}}
\newtheorem*{solution}{Solution}
\newtheorem*{application}{Application}
\newtheorem*{notation}{Notation}
\newtheorem*{vocabulary}{Vocabulaire}
\newtheorem*{properties}{Propriétés}



\theoremstyle{remark}
\newtheorem*{remark}{Remarque}
\newtheorem*{rappel}{Rappel}


\usepackage{etoolbox}
\AtBeginEnvironment{exercise}{\small}
\AtBeginEnvironment{example}{\small}

\usepackage{cases}
\usepackage[red]{mypack}

\usepackage[framemethod=TikZ]{mdframed}

\definecolor{bg}{rgb}{0.4,0.25,0.95}
\definecolor{pagebg}{rgb}{0,0,0.5}
\surroundwithmdframed[
   topline=false,
   rightline=false,
   bottomline=false,
   leftmargin=\parindent,
   skipabove=8pt,
   skipbelow=8pt,
   linecolor=blue,
   innerbottommargin=10pt,
   % backgroundcolor=bg,font=\color{orange}\sffamily, fontcolor=white
]{definition}

\usepackage{empheq}
\usepackage[most]{tcolorbox}

\newtcbox{\mymath}[1][]{%
    nobeforeafter, math upper, tcbox raise base,
    enhanced, colframe=blue!30!black,
    colback=red!10, boxrule=1pt,
    #1}

\usepackage{unixode}


\DeclareMathOperator{\ord}{ord}
\DeclareMathOperator{\orb}{orb}
\DeclareMathOperator{\stab}{stab}
\DeclareMathOperator{\Stab}{stab}
\DeclareMathOperator{\ppcm}{ppcm}
\DeclareMathOperator{\conj}{Conj}
\DeclareMathOperator{\End}{End}
\DeclareMathOperator{\rot}{rot}
\DeclareMathOperator{\trs}{trace}
\DeclareMathOperator{\Ind}{Ind}
\DeclareMathOperator{\mat}{Mat}
\DeclareMathOperator{\id}{Id}
\DeclareMathOperator{\vect}{vect}
\DeclareMathOperator{\img}{img}
\DeclareMathOperator{\cov}{Cov}
\DeclareMathOperator{\dist}{dist}
\DeclareMathOperator{\irr}{Irr}
\DeclareMathOperator{\image}{Im}
\DeclareMathOperator{\pd}{\partial}
\DeclareMathOperator{\epi}{epi}
\DeclareMathOperator{\Argmin}{Argmin}
\DeclareMathOperator{\dom}{dom}
\DeclareMathOperator{\proj}{proj}
\DeclareMathOperator{\ctg}{ctg}
\DeclareMathOperator{\supp}{supp}
\DeclareMathOperator{\argmin}{argmin}
\DeclareMathOperator{\mult}{mult}
\DeclareMathOperator{\ch}{ch}
\DeclareMathOperator{\sh}{sh}
\DeclareMathOperator{\rang}{rang}
\DeclareMathOperator{\diam}{diam}
\DeclareMathOperator{\Epigraphe}{Epigraphe}




\usepackage{xcolor}
\everymath{\color{blue}}
%\everymath{\color[rgb]{0,1,1}}
%\pagecolor[rgb]{0,0,0.5}


\newcommand*{\pdtest}[3][]{\ensuremath{\frac{\partial^{#1} #2}{\partial #3}}}

\newcommand*{\deffunc}[6][]{\ensuremath{
\begin{array}{rcl}
#2 : #3 &\rightarrow& #4\\
#5 &\mapsto& #6
\end{array}
}}

\newcommand{\eqcolon}{\mathrel{\resizebox{\widthof{$\mathord{=}$}}{\height}{ $\!\!=\!\!\resizebox{1.2\width}{0.8\height}{\raisebox{0.23ex}{$\mathop{:}$}}\!\!$ }}}
\newcommand{\coloneq}{\mathrel{\resizebox{\widthof{$\mathord{=}$}}{\height}{ $\!\!\resizebox{1.2\width}{0.8\height}{\raisebox{0.23ex}{$\mathop{:}$}}\!\!=\!\!$ }}}
\newcommand{\eqcolonl}{\ensuremath{\mathrel{=\!\!\mathop{:}}}}
\newcommand{\coloneql}{\ensuremath{\mathrel{\mathop{:} \!\! =}}}
\newcommand{\vc}[1]{% inline column vector
  \left(\begin{smallmatrix}#1\end{smallmatrix}\right)%
}
\newcommand{\vr}[1]{% inline row vector
  \begin{smallmatrix}(\,#1\,)\end{smallmatrix}%
}
\makeatletter
\newcommand*{\defeq}{\ =\mathrel{\rlap{%
                     \raisebox{0.3ex}{$\m@th\cdot$}}%
                     \raisebox{-0.3ex}{$\m@th\cdot$}}%
                     }
\makeatother

\newcommand{\mathcircle}[1]{% inline row vector
 \overset{\circ}{#1}
}
\newcommand{\ulim}{% low limit
 \underline{\lim}
}
\newcommand{\ssi}{% iff
\iff
}
\newcommand{\ps}[2]{
\expval{#1 | #2}
}
\newcommand{\df}[1]{
\mqty{#1}
}
\newcommand{\n}[1]{
\norm{#1}
}
\newcommand{\sys}[1]{
\left\{\smqty{#1}\right.
}


\newcommand{\eqdef}{\ensuremath{\overset{\text{def}}=}}


\def\Circlearrowright{\ensuremath{%
  \rotatebox[origin=c]{230}{$\circlearrowright$}}}

\newcommand\ct[1]{\text{\rmfamily\upshape #1}}
\newcommand\question[1]{ {\color{red} ...!? \small #1}}
\newcommand\caz[1]{\left\{\begin{array} #1 \end{array}\right.}
\newcommand\const{\text{\rmfamily\upshape const}}
\newcommand\toP{ \overset{\pro}{\to}}
\newcommand\toPP{ \overset{\text{PP}}{\to}}
\newcommand{\oeq}{\mathrel{\text{\textcircled{$=$}}}}





\usepackage{xcolor}
% \usepackage[normalem]{ulem}
\usepackage{lipsum}
\makeatletter
% \newcommand\colorwave[1][blue]{\bgroup \markoverwith{\lower3.5\p@\hbox{\sixly \textcolor{#1}{\char58}}}\ULon}
%\font\sixly=lasy6 % does not re-load if already loaded, so no memory problem.

\newmdtheoremenv[
linewidth= 1pt,linecolor= blue,%
leftmargin=20,rightmargin=20,innertopmargin=0pt, innerrightmargin=40,%
tikzsetting = { draw=lightgray, line width = 0.3pt,dashed,%
dash pattern = on 15pt off 3pt},%
splittopskip=\topskip,skipbelow=\baselineskip,%
skipabove=\baselineskip,ntheorem,roundcorner=0pt,
% backgroundcolor=pagebg,font=\color{orange}\sffamily, fontcolor=white
]{examplebox}{Exemple}[section]



\newcommand\R{\mathbb{R}}
\newcommand\Z{\mathbb{Z}}
\newcommand\N{\mathbb{N}}
\newcommand\E{\mathbb{E}}
\newcommand\F{\mathcal{F}}
\newcommand\cH{\mathcal{H}}
\newcommand\V{\mathbb{V}}
\newcommand\dmo{ ^{-1} }
\newcommand\kapa{\kappa}
\newcommand\im{Im}
\newcommand\hs{\mathcal{H}}





\usepackage{soul}

\makeatletter
\newcommand*{\whiten}[1]{\llap{\textcolor{white}{{\the\SOUL@token}}\hspace{#1pt}}}
\DeclareRobustCommand*\myul{%
    \def\SOUL@everyspace{\underline{\space}\kern\z@}%
    \def\SOUL@everytoken{%
     \setbox0=\hbox{\the\SOUL@token}%
     \ifdim\dp0>\z@
        \raisebox{\dp0}{\underline{\phantom{\the\SOUL@token}}}%
        \whiten{1}\whiten{0}%
        \whiten{-1}\whiten{-2}%
        \llap{\the\SOUL@token}%
     \else
        \underline{\the\SOUL@token}%
     \fi}%
\SOUL@}
\makeatother

\newcommand*{\demp}{\fontfamily{lmtt}\selectfont}

\DeclareTextFontCommand{\textdemp}{\demp}

\begin{document}

\ifcomment
Multiline
comment
\fi
\ifcomment
\myul{Typesetting test}
% \color[rgb]{1,1,1}
$∑_i^n≠ 60º±∞π∆¬≈√j∫h≤≥µ$

$\CR \R\pro\ind\pro\gS\pro
\mqty[a&b\\c&d]$
$\pro\mathbb{P}$
$\dd{x}$

  \[
    \alpha(x)=\left\{
                \begin{array}{ll}
                  x\\
                  \frac{1}{1+e^{-kx}}\\
                  \frac{e^x-e^{-x}}{e^x+e^{-x}}
                \end{array}
              \right.
  \]

  $\expval{x}$
  
  $\chi_\rho(ghg\dmo)=\Tr(\rho_{ghg\dmo})=\Tr(\rho_g\circ\rho_h\circ\rho\dmo_g)=\Tr(\rho_h)\overset{\mbox{\scalebox{0.5}{$\Tr(AB)=\Tr(BA)$}}}{=}\chi_\rho(h)$
  	$\mathop{\oplus}_{\substack{x\in X}}$

$\mat(\rho_g)=(a_{ij}(g))_{\scriptsize \substack{1\leq i\leq d \\ 1\leq j\leq d}}$ et $\mat(\rho'_g)=(a'_{ij}(g))_{\scriptsize \substack{1\leq i'\leq d' \\ 1\leq j'\leq d'}}$



\[\int_a^b{\mathbb{R}^2}g(u, v)\dd{P_{XY}}(u, v)=\iint g(u,v) f_{XY}(u, v)\dd \lambda(u) \dd \lambda(v)\]
$$\lim_{x\to\infty} f(x)$$	
$$\iiiint_V \mu(t,u,v,w) \,dt\,du\,dv\,dw$$
$$\sum_{n=1}^{\infty} 2^{-n} = 1$$	
\begin{definition}
	Si $X$ et $Y$ sont 2 v.a. ou definit la \textsc{Covariance} entre $X$ et $Y$ comme
	$\cov(X,Y)\overset{\text{def}}{=}\E\left[(X-\E(X))(Y-\E(Y))\right]=\E(XY)-\E(X)\E(Y)$.
\end{definition}
\fi
\pagebreak

% \tableofcontents

% insert your code here
%\input{./algebra/main.tex}
%\input{./geometrie-differentielle/main.tex}
%\input{./probabilite/main.tex}
%\input{./analyse-fonctionnelle/main.tex}
% \input{./Analyse-convexe-et-dualite-en-optimisation/main.tex}
%\input{./tikz/main.tex}
%\input{./Theorie-du-distributions/main.tex}
%\input{./optimisation/mine.tex}
 \input{./modelisation/main.tex}

% yves.aubry@univ-tln.fr : algebra

\end{document}

% % !TEX encoding = UTF-8 Unicode
% !TEX TS-program = xelatex

\documentclass[french]{report}

%\usepackage[utf8]{inputenc}
%\usepackage[T1]{fontenc}
\usepackage{babel}


\newif\ifcomment
%\commenttrue # Show comments

\usepackage{physics}
\usepackage{amssymb}


\usepackage{amsthm}
% \usepackage{thmtools}
\usepackage{mathtools}
\usepackage{amsfonts}

\usepackage{color}

\usepackage{tikz}

\usepackage{geometry}
\geometry{a5paper, margin=0.1in, right=1cm}

\usepackage{dsfont}

\usepackage{graphicx}
\graphicspath{ {images/} }

\usepackage{faktor}

\usepackage{IEEEtrantools}
\usepackage{enumerate}   
\usepackage[PostScript=dvips]{"/Users/aware/Documents/Courses/diagrams"}


\newtheorem{theorem}{Théorème}[section]
\renewcommand{\thetheorem}{\arabic{theorem}}
\newtheorem{lemme}{Lemme}[section]
\renewcommand{\thelemme}{\arabic{lemme}}
\newtheorem{proposition}{Proposition}[section]
\renewcommand{\theproposition}{\arabic{proposition}}
\newtheorem{notations}{Notations}[section]
\newtheorem{problem}{Problème}[section]
\newtheorem{corollary}{Corollaire}[theorem]
\renewcommand{\thecorollary}{\arabic{corollary}}
\newtheorem{property}{Propriété}[section]
\newtheorem{objective}{Objectif}[section]

\theoremstyle{definition}
\newtheorem{definition}{Définition}[section]
\renewcommand{\thedefinition}{\arabic{definition}}
\newtheorem{exercise}{Exercice}[chapter]
\renewcommand{\theexercise}{\arabic{exercise}}
\newtheorem{example}{Exemple}[chapter]
\renewcommand{\theexample}{\arabic{example}}
\newtheorem*{solution}{Solution}
\newtheorem*{application}{Application}
\newtheorem*{notation}{Notation}
\newtheorem*{vocabulary}{Vocabulaire}
\newtheorem*{properties}{Propriétés}



\theoremstyle{remark}
\newtheorem*{remark}{Remarque}
\newtheorem*{rappel}{Rappel}


\usepackage{etoolbox}
\AtBeginEnvironment{exercise}{\small}
\AtBeginEnvironment{example}{\small}

\usepackage{cases}
\usepackage[red]{mypack}

\usepackage[framemethod=TikZ]{mdframed}

\definecolor{bg}{rgb}{0.4,0.25,0.95}
\definecolor{pagebg}{rgb}{0,0,0.5}
\surroundwithmdframed[
   topline=false,
   rightline=false,
   bottomline=false,
   leftmargin=\parindent,
   skipabove=8pt,
   skipbelow=8pt,
   linecolor=blue,
   innerbottommargin=10pt,
   % backgroundcolor=bg,font=\color{orange}\sffamily, fontcolor=white
]{definition}

\usepackage{empheq}
\usepackage[most]{tcolorbox}

\newtcbox{\mymath}[1][]{%
    nobeforeafter, math upper, tcbox raise base,
    enhanced, colframe=blue!30!black,
    colback=red!10, boxrule=1pt,
    #1}

\usepackage{unixode}


\DeclareMathOperator{\ord}{ord}
\DeclareMathOperator{\orb}{orb}
\DeclareMathOperator{\stab}{stab}
\DeclareMathOperator{\Stab}{stab}
\DeclareMathOperator{\ppcm}{ppcm}
\DeclareMathOperator{\conj}{Conj}
\DeclareMathOperator{\End}{End}
\DeclareMathOperator{\rot}{rot}
\DeclareMathOperator{\trs}{trace}
\DeclareMathOperator{\Ind}{Ind}
\DeclareMathOperator{\mat}{Mat}
\DeclareMathOperator{\id}{Id}
\DeclareMathOperator{\vect}{vect}
\DeclareMathOperator{\img}{img}
\DeclareMathOperator{\cov}{Cov}
\DeclareMathOperator{\dist}{dist}
\DeclareMathOperator{\irr}{Irr}
\DeclareMathOperator{\image}{Im}
\DeclareMathOperator{\pd}{\partial}
\DeclareMathOperator{\epi}{epi}
\DeclareMathOperator{\Argmin}{Argmin}
\DeclareMathOperator{\dom}{dom}
\DeclareMathOperator{\proj}{proj}
\DeclareMathOperator{\ctg}{ctg}
\DeclareMathOperator{\supp}{supp}
\DeclareMathOperator{\argmin}{argmin}
\DeclareMathOperator{\mult}{mult}
\DeclareMathOperator{\ch}{ch}
\DeclareMathOperator{\sh}{sh}
\DeclareMathOperator{\rang}{rang}
\DeclareMathOperator{\diam}{diam}
\DeclareMathOperator{\Epigraphe}{Epigraphe}




\usepackage{xcolor}
\everymath{\color{blue}}
%\everymath{\color[rgb]{0,1,1}}
%\pagecolor[rgb]{0,0,0.5}


\newcommand*{\pdtest}[3][]{\ensuremath{\frac{\partial^{#1} #2}{\partial #3}}}

\newcommand*{\deffunc}[6][]{\ensuremath{
\begin{array}{rcl}
#2 : #3 &\rightarrow& #4\\
#5 &\mapsto& #6
\end{array}
}}

\newcommand{\eqcolon}{\mathrel{\resizebox{\widthof{$\mathord{=}$}}{\height}{ $\!\!=\!\!\resizebox{1.2\width}{0.8\height}{\raisebox{0.23ex}{$\mathop{:}$}}\!\!$ }}}
\newcommand{\coloneq}{\mathrel{\resizebox{\widthof{$\mathord{=}$}}{\height}{ $\!\!\resizebox{1.2\width}{0.8\height}{\raisebox{0.23ex}{$\mathop{:}$}}\!\!=\!\!$ }}}
\newcommand{\eqcolonl}{\ensuremath{\mathrel{=\!\!\mathop{:}}}}
\newcommand{\coloneql}{\ensuremath{\mathrel{\mathop{:} \!\! =}}}
\newcommand{\vc}[1]{% inline column vector
  \left(\begin{smallmatrix}#1\end{smallmatrix}\right)%
}
\newcommand{\vr}[1]{% inline row vector
  \begin{smallmatrix}(\,#1\,)\end{smallmatrix}%
}
\makeatletter
\newcommand*{\defeq}{\ =\mathrel{\rlap{%
                     \raisebox{0.3ex}{$\m@th\cdot$}}%
                     \raisebox{-0.3ex}{$\m@th\cdot$}}%
                     }
\makeatother

\newcommand{\mathcircle}[1]{% inline row vector
 \overset{\circ}{#1}
}
\newcommand{\ulim}{% low limit
 \underline{\lim}
}
\newcommand{\ssi}{% iff
\iff
}
\newcommand{\ps}[2]{
\expval{#1 | #2}
}
\newcommand{\df}[1]{
\mqty{#1}
}
\newcommand{\n}[1]{
\norm{#1}
}
\newcommand{\sys}[1]{
\left\{\smqty{#1}\right.
}


\newcommand{\eqdef}{\ensuremath{\overset{\text{def}}=}}


\def\Circlearrowright{\ensuremath{%
  \rotatebox[origin=c]{230}{$\circlearrowright$}}}

\newcommand\ct[1]{\text{\rmfamily\upshape #1}}
\newcommand\question[1]{ {\color{red} ...!? \small #1}}
\newcommand\caz[1]{\left\{\begin{array} #1 \end{array}\right.}
\newcommand\const{\text{\rmfamily\upshape const}}
\newcommand\toP{ \overset{\pro}{\to}}
\newcommand\toPP{ \overset{\text{PP}}{\to}}
\newcommand{\oeq}{\mathrel{\text{\textcircled{$=$}}}}





\usepackage{xcolor}
% \usepackage[normalem]{ulem}
\usepackage{lipsum}
\makeatletter
% \newcommand\colorwave[1][blue]{\bgroup \markoverwith{\lower3.5\p@\hbox{\sixly \textcolor{#1}{\char58}}}\ULon}
%\font\sixly=lasy6 % does not re-load if already loaded, so no memory problem.

\newmdtheoremenv[
linewidth= 1pt,linecolor= blue,%
leftmargin=20,rightmargin=20,innertopmargin=0pt, innerrightmargin=40,%
tikzsetting = { draw=lightgray, line width = 0.3pt,dashed,%
dash pattern = on 15pt off 3pt},%
splittopskip=\topskip,skipbelow=\baselineskip,%
skipabove=\baselineskip,ntheorem,roundcorner=0pt,
% backgroundcolor=pagebg,font=\color{orange}\sffamily, fontcolor=white
]{examplebox}{Exemple}[section]



\newcommand\R{\mathbb{R}}
\newcommand\Z{\mathbb{Z}}
\newcommand\N{\mathbb{N}}
\newcommand\E{\mathbb{E}}
\newcommand\F{\mathcal{F}}
\newcommand\cH{\mathcal{H}}
\newcommand\V{\mathbb{V}}
\newcommand\dmo{ ^{-1} }
\newcommand\kapa{\kappa}
\newcommand\im{Im}
\newcommand\hs{\mathcal{H}}





\usepackage{soul}

\makeatletter
\newcommand*{\whiten}[1]{\llap{\textcolor{white}{{\the\SOUL@token}}\hspace{#1pt}}}
\DeclareRobustCommand*\myul{%
    \def\SOUL@everyspace{\underline{\space}\kern\z@}%
    \def\SOUL@everytoken{%
     \setbox0=\hbox{\the\SOUL@token}%
     \ifdim\dp0>\z@
        \raisebox{\dp0}{\underline{\phantom{\the\SOUL@token}}}%
        \whiten{1}\whiten{0}%
        \whiten{-1}\whiten{-2}%
        \llap{\the\SOUL@token}%
     \else
        \underline{\the\SOUL@token}%
     \fi}%
\SOUL@}
\makeatother

\newcommand*{\demp}{\fontfamily{lmtt}\selectfont}

\DeclareTextFontCommand{\textdemp}{\demp}

\begin{document}

\ifcomment
Multiline
comment
\fi
\ifcomment
\myul{Typesetting test}
% \color[rgb]{1,1,1}
$∑_i^n≠ 60º±∞π∆¬≈√j∫h≤≥µ$

$\CR \R\pro\ind\pro\gS\pro
\mqty[a&b\\c&d]$
$\pro\mathbb{P}$
$\dd{x}$

  \[
    \alpha(x)=\left\{
                \begin{array}{ll}
                  x\\
                  \frac{1}{1+e^{-kx}}\\
                  \frac{e^x-e^{-x}}{e^x+e^{-x}}
                \end{array}
              \right.
  \]

  $\expval{x}$
  
  $\chi_\rho(ghg\dmo)=\Tr(\rho_{ghg\dmo})=\Tr(\rho_g\circ\rho_h\circ\rho\dmo_g)=\Tr(\rho_h)\overset{\mbox{\scalebox{0.5}{$\Tr(AB)=\Tr(BA)$}}}{=}\chi_\rho(h)$
  	$\mathop{\oplus}_{\substack{x\in X}}$

$\mat(\rho_g)=(a_{ij}(g))_{\scriptsize \substack{1\leq i\leq d \\ 1\leq j\leq d}}$ et $\mat(\rho'_g)=(a'_{ij}(g))_{\scriptsize \substack{1\leq i'\leq d' \\ 1\leq j'\leq d'}}$



\[\int_a^b{\mathbb{R}^2}g(u, v)\dd{P_{XY}}(u, v)=\iint g(u,v) f_{XY}(u, v)\dd \lambda(u) \dd \lambda(v)\]
$$\lim_{x\to\infty} f(x)$$	
$$\iiiint_V \mu(t,u,v,w) \,dt\,du\,dv\,dw$$
$$\sum_{n=1}^{\infty} 2^{-n} = 1$$	
\begin{definition}
	Si $X$ et $Y$ sont 2 v.a. ou definit la \textsc{Covariance} entre $X$ et $Y$ comme
	$\cov(X,Y)\overset{\text{def}}{=}\E\left[(X-\E(X))(Y-\E(Y))\right]=\E(XY)-\E(X)\E(Y)$.
\end{definition}
\fi
\pagebreak

% \tableofcontents

% insert your code here
%\input{./algebra/main.tex}
%\input{./geometrie-differentielle/main.tex}
%\input{./probabilite/main.tex}
%\input{./analyse-fonctionnelle/main.tex}
% \input{./Analyse-convexe-et-dualite-en-optimisation/main.tex}
%\input{./tikz/main.tex}
%\input{./Theorie-du-distributions/main.tex}
%\input{./optimisation/mine.tex}
 \input{./modelisation/main.tex}

% yves.aubry@univ-tln.fr : algebra

\end{document}

%% !TEX encoding = UTF-8 Unicode
% !TEX TS-program = xelatex

\documentclass[french]{report}

%\usepackage[utf8]{inputenc}
%\usepackage[T1]{fontenc}
\usepackage{babel}


\newif\ifcomment
%\commenttrue # Show comments

\usepackage{physics}
\usepackage{amssymb}


\usepackage{amsthm}
% \usepackage{thmtools}
\usepackage{mathtools}
\usepackage{amsfonts}

\usepackage{color}

\usepackage{tikz}

\usepackage{geometry}
\geometry{a5paper, margin=0.1in, right=1cm}

\usepackage{dsfont}

\usepackage{graphicx}
\graphicspath{ {images/} }

\usepackage{faktor}

\usepackage{IEEEtrantools}
\usepackage{enumerate}   
\usepackage[PostScript=dvips]{"/Users/aware/Documents/Courses/diagrams"}


\newtheorem{theorem}{Théorème}[section]
\renewcommand{\thetheorem}{\arabic{theorem}}
\newtheorem{lemme}{Lemme}[section]
\renewcommand{\thelemme}{\arabic{lemme}}
\newtheorem{proposition}{Proposition}[section]
\renewcommand{\theproposition}{\arabic{proposition}}
\newtheorem{notations}{Notations}[section]
\newtheorem{problem}{Problème}[section]
\newtheorem{corollary}{Corollaire}[theorem]
\renewcommand{\thecorollary}{\arabic{corollary}}
\newtheorem{property}{Propriété}[section]
\newtheorem{objective}{Objectif}[section]

\theoremstyle{definition}
\newtheorem{definition}{Définition}[section]
\renewcommand{\thedefinition}{\arabic{definition}}
\newtheorem{exercise}{Exercice}[chapter]
\renewcommand{\theexercise}{\arabic{exercise}}
\newtheorem{example}{Exemple}[chapter]
\renewcommand{\theexample}{\arabic{example}}
\newtheorem*{solution}{Solution}
\newtheorem*{application}{Application}
\newtheorem*{notation}{Notation}
\newtheorem*{vocabulary}{Vocabulaire}
\newtheorem*{properties}{Propriétés}



\theoremstyle{remark}
\newtheorem*{remark}{Remarque}
\newtheorem*{rappel}{Rappel}


\usepackage{etoolbox}
\AtBeginEnvironment{exercise}{\small}
\AtBeginEnvironment{example}{\small}

\usepackage{cases}
\usepackage[red]{mypack}

\usepackage[framemethod=TikZ]{mdframed}

\definecolor{bg}{rgb}{0.4,0.25,0.95}
\definecolor{pagebg}{rgb}{0,0,0.5}
\surroundwithmdframed[
   topline=false,
   rightline=false,
   bottomline=false,
   leftmargin=\parindent,
   skipabove=8pt,
   skipbelow=8pt,
   linecolor=blue,
   innerbottommargin=10pt,
   % backgroundcolor=bg,font=\color{orange}\sffamily, fontcolor=white
]{definition}

\usepackage{empheq}
\usepackage[most]{tcolorbox}

\newtcbox{\mymath}[1][]{%
    nobeforeafter, math upper, tcbox raise base,
    enhanced, colframe=blue!30!black,
    colback=red!10, boxrule=1pt,
    #1}

\usepackage{unixode}


\DeclareMathOperator{\ord}{ord}
\DeclareMathOperator{\orb}{orb}
\DeclareMathOperator{\stab}{stab}
\DeclareMathOperator{\Stab}{stab}
\DeclareMathOperator{\ppcm}{ppcm}
\DeclareMathOperator{\conj}{Conj}
\DeclareMathOperator{\End}{End}
\DeclareMathOperator{\rot}{rot}
\DeclareMathOperator{\trs}{trace}
\DeclareMathOperator{\Ind}{Ind}
\DeclareMathOperator{\mat}{Mat}
\DeclareMathOperator{\id}{Id}
\DeclareMathOperator{\vect}{vect}
\DeclareMathOperator{\img}{img}
\DeclareMathOperator{\cov}{Cov}
\DeclareMathOperator{\dist}{dist}
\DeclareMathOperator{\irr}{Irr}
\DeclareMathOperator{\image}{Im}
\DeclareMathOperator{\pd}{\partial}
\DeclareMathOperator{\epi}{epi}
\DeclareMathOperator{\Argmin}{Argmin}
\DeclareMathOperator{\dom}{dom}
\DeclareMathOperator{\proj}{proj}
\DeclareMathOperator{\ctg}{ctg}
\DeclareMathOperator{\supp}{supp}
\DeclareMathOperator{\argmin}{argmin}
\DeclareMathOperator{\mult}{mult}
\DeclareMathOperator{\ch}{ch}
\DeclareMathOperator{\sh}{sh}
\DeclareMathOperator{\rang}{rang}
\DeclareMathOperator{\diam}{diam}
\DeclareMathOperator{\Epigraphe}{Epigraphe}




\usepackage{xcolor}
\everymath{\color{blue}}
%\everymath{\color[rgb]{0,1,1}}
%\pagecolor[rgb]{0,0,0.5}


\newcommand*{\pdtest}[3][]{\ensuremath{\frac{\partial^{#1} #2}{\partial #3}}}

\newcommand*{\deffunc}[6][]{\ensuremath{
\begin{array}{rcl}
#2 : #3 &\rightarrow& #4\\
#5 &\mapsto& #6
\end{array}
}}

\newcommand{\eqcolon}{\mathrel{\resizebox{\widthof{$\mathord{=}$}}{\height}{ $\!\!=\!\!\resizebox{1.2\width}{0.8\height}{\raisebox{0.23ex}{$\mathop{:}$}}\!\!$ }}}
\newcommand{\coloneq}{\mathrel{\resizebox{\widthof{$\mathord{=}$}}{\height}{ $\!\!\resizebox{1.2\width}{0.8\height}{\raisebox{0.23ex}{$\mathop{:}$}}\!\!=\!\!$ }}}
\newcommand{\eqcolonl}{\ensuremath{\mathrel{=\!\!\mathop{:}}}}
\newcommand{\coloneql}{\ensuremath{\mathrel{\mathop{:} \!\! =}}}
\newcommand{\vc}[1]{% inline column vector
  \left(\begin{smallmatrix}#1\end{smallmatrix}\right)%
}
\newcommand{\vr}[1]{% inline row vector
  \begin{smallmatrix}(\,#1\,)\end{smallmatrix}%
}
\makeatletter
\newcommand*{\defeq}{\ =\mathrel{\rlap{%
                     \raisebox{0.3ex}{$\m@th\cdot$}}%
                     \raisebox{-0.3ex}{$\m@th\cdot$}}%
                     }
\makeatother

\newcommand{\mathcircle}[1]{% inline row vector
 \overset{\circ}{#1}
}
\newcommand{\ulim}{% low limit
 \underline{\lim}
}
\newcommand{\ssi}{% iff
\iff
}
\newcommand{\ps}[2]{
\expval{#1 | #2}
}
\newcommand{\df}[1]{
\mqty{#1}
}
\newcommand{\n}[1]{
\norm{#1}
}
\newcommand{\sys}[1]{
\left\{\smqty{#1}\right.
}


\newcommand{\eqdef}{\ensuremath{\overset{\text{def}}=}}


\def\Circlearrowright{\ensuremath{%
  \rotatebox[origin=c]{230}{$\circlearrowright$}}}

\newcommand\ct[1]{\text{\rmfamily\upshape #1}}
\newcommand\question[1]{ {\color{red} ...!? \small #1}}
\newcommand\caz[1]{\left\{\begin{array} #1 \end{array}\right.}
\newcommand\const{\text{\rmfamily\upshape const}}
\newcommand\toP{ \overset{\pro}{\to}}
\newcommand\toPP{ \overset{\text{PP}}{\to}}
\newcommand{\oeq}{\mathrel{\text{\textcircled{$=$}}}}





\usepackage{xcolor}
% \usepackage[normalem]{ulem}
\usepackage{lipsum}
\makeatletter
% \newcommand\colorwave[1][blue]{\bgroup \markoverwith{\lower3.5\p@\hbox{\sixly \textcolor{#1}{\char58}}}\ULon}
%\font\sixly=lasy6 % does not re-load if already loaded, so no memory problem.

\newmdtheoremenv[
linewidth= 1pt,linecolor= blue,%
leftmargin=20,rightmargin=20,innertopmargin=0pt, innerrightmargin=40,%
tikzsetting = { draw=lightgray, line width = 0.3pt,dashed,%
dash pattern = on 15pt off 3pt},%
splittopskip=\topskip,skipbelow=\baselineskip,%
skipabove=\baselineskip,ntheorem,roundcorner=0pt,
% backgroundcolor=pagebg,font=\color{orange}\sffamily, fontcolor=white
]{examplebox}{Exemple}[section]



\newcommand\R{\mathbb{R}}
\newcommand\Z{\mathbb{Z}}
\newcommand\N{\mathbb{N}}
\newcommand\E{\mathbb{E}}
\newcommand\F{\mathcal{F}}
\newcommand\cH{\mathcal{H}}
\newcommand\V{\mathbb{V}}
\newcommand\dmo{ ^{-1} }
\newcommand\kapa{\kappa}
\newcommand\im{Im}
\newcommand\hs{\mathcal{H}}





\usepackage{soul}

\makeatletter
\newcommand*{\whiten}[1]{\llap{\textcolor{white}{{\the\SOUL@token}}\hspace{#1pt}}}
\DeclareRobustCommand*\myul{%
    \def\SOUL@everyspace{\underline{\space}\kern\z@}%
    \def\SOUL@everytoken{%
     \setbox0=\hbox{\the\SOUL@token}%
     \ifdim\dp0>\z@
        \raisebox{\dp0}{\underline{\phantom{\the\SOUL@token}}}%
        \whiten{1}\whiten{0}%
        \whiten{-1}\whiten{-2}%
        \llap{\the\SOUL@token}%
     \else
        \underline{\the\SOUL@token}%
     \fi}%
\SOUL@}
\makeatother

\newcommand*{\demp}{\fontfamily{lmtt}\selectfont}

\DeclareTextFontCommand{\textdemp}{\demp}

\begin{document}

\ifcomment
Multiline
comment
\fi
\ifcomment
\myul{Typesetting test}
% \color[rgb]{1,1,1}
$∑_i^n≠ 60º±∞π∆¬≈√j∫h≤≥µ$

$\CR \R\pro\ind\pro\gS\pro
\mqty[a&b\\c&d]$
$\pro\mathbb{P}$
$\dd{x}$

  \[
    \alpha(x)=\left\{
                \begin{array}{ll}
                  x\\
                  \frac{1}{1+e^{-kx}}\\
                  \frac{e^x-e^{-x}}{e^x+e^{-x}}
                \end{array}
              \right.
  \]

  $\expval{x}$
  
  $\chi_\rho(ghg\dmo)=\Tr(\rho_{ghg\dmo})=\Tr(\rho_g\circ\rho_h\circ\rho\dmo_g)=\Tr(\rho_h)\overset{\mbox{\scalebox{0.5}{$\Tr(AB)=\Tr(BA)$}}}{=}\chi_\rho(h)$
  	$\mathop{\oplus}_{\substack{x\in X}}$

$\mat(\rho_g)=(a_{ij}(g))_{\scriptsize \substack{1\leq i\leq d \\ 1\leq j\leq d}}$ et $\mat(\rho'_g)=(a'_{ij}(g))_{\scriptsize \substack{1\leq i'\leq d' \\ 1\leq j'\leq d'}}$



\[\int_a^b{\mathbb{R}^2}g(u, v)\dd{P_{XY}}(u, v)=\iint g(u,v) f_{XY}(u, v)\dd \lambda(u) \dd \lambda(v)\]
$$\lim_{x\to\infty} f(x)$$	
$$\iiiint_V \mu(t,u,v,w) \,dt\,du\,dv\,dw$$
$$\sum_{n=1}^{\infty} 2^{-n} = 1$$	
\begin{definition}
	Si $X$ et $Y$ sont 2 v.a. ou definit la \textsc{Covariance} entre $X$ et $Y$ comme
	$\cov(X,Y)\overset{\text{def}}{=}\E\left[(X-\E(X))(Y-\E(Y))\right]=\E(XY)-\E(X)\E(Y)$.
\end{definition}
\fi
\pagebreak

% \tableofcontents

% insert your code here
%\input{./algebra/main.tex}
%\input{./geometrie-differentielle/main.tex}
%\input{./probabilite/main.tex}
%\input{./analyse-fonctionnelle/main.tex}
% \input{./Analyse-convexe-et-dualite-en-optimisation/main.tex}
%\input{./tikz/main.tex}
%\input{./Theorie-du-distributions/main.tex}
%\input{./optimisation/mine.tex}
 \input{./modelisation/main.tex}

% yves.aubry@univ-tln.fr : algebra

\end{document}

%% !TEX encoding = UTF-8 Unicode
% !TEX TS-program = xelatex

\documentclass[french]{report}

%\usepackage[utf8]{inputenc}
%\usepackage[T1]{fontenc}
\usepackage{babel}


\newif\ifcomment
%\commenttrue # Show comments

\usepackage{physics}
\usepackage{amssymb}


\usepackage{amsthm}
% \usepackage{thmtools}
\usepackage{mathtools}
\usepackage{amsfonts}

\usepackage{color}

\usepackage{tikz}

\usepackage{geometry}
\geometry{a5paper, margin=0.1in, right=1cm}

\usepackage{dsfont}

\usepackage{graphicx}
\graphicspath{ {images/} }

\usepackage{faktor}

\usepackage{IEEEtrantools}
\usepackage{enumerate}   
\usepackage[PostScript=dvips]{"/Users/aware/Documents/Courses/diagrams"}


\newtheorem{theorem}{Théorème}[section]
\renewcommand{\thetheorem}{\arabic{theorem}}
\newtheorem{lemme}{Lemme}[section]
\renewcommand{\thelemme}{\arabic{lemme}}
\newtheorem{proposition}{Proposition}[section]
\renewcommand{\theproposition}{\arabic{proposition}}
\newtheorem{notations}{Notations}[section]
\newtheorem{problem}{Problème}[section]
\newtheorem{corollary}{Corollaire}[theorem]
\renewcommand{\thecorollary}{\arabic{corollary}}
\newtheorem{property}{Propriété}[section]
\newtheorem{objective}{Objectif}[section]

\theoremstyle{definition}
\newtheorem{definition}{Définition}[section]
\renewcommand{\thedefinition}{\arabic{definition}}
\newtheorem{exercise}{Exercice}[chapter]
\renewcommand{\theexercise}{\arabic{exercise}}
\newtheorem{example}{Exemple}[chapter]
\renewcommand{\theexample}{\arabic{example}}
\newtheorem*{solution}{Solution}
\newtheorem*{application}{Application}
\newtheorem*{notation}{Notation}
\newtheorem*{vocabulary}{Vocabulaire}
\newtheorem*{properties}{Propriétés}



\theoremstyle{remark}
\newtheorem*{remark}{Remarque}
\newtheorem*{rappel}{Rappel}


\usepackage{etoolbox}
\AtBeginEnvironment{exercise}{\small}
\AtBeginEnvironment{example}{\small}

\usepackage{cases}
\usepackage[red]{mypack}

\usepackage[framemethod=TikZ]{mdframed}

\definecolor{bg}{rgb}{0.4,0.25,0.95}
\definecolor{pagebg}{rgb}{0,0,0.5}
\surroundwithmdframed[
   topline=false,
   rightline=false,
   bottomline=false,
   leftmargin=\parindent,
   skipabove=8pt,
   skipbelow=8pt,
   linecolor=blue,
   innerbottommargin=10pt,
   % backgroundcolor=bg,font=\color{orange}\sffamily, fontcolor=white
]{definition}

\usepackage{empheq}
\usepackage[most]{tcolorbox}

\newtcbox{\mymath}[1][]{%
    nobeforeafter, math upper, tcbox raise base,
    enhanced, colframe=blue!30!black,
    colback=red!10, boxrule=1pt,
    #1}

\usepackage{unixode}


\DeclareMathOperator{\ord}{ord}
\DeclareMathOperator{\orb}{orb}
\DeclareMathOperator{\stab}{stab}
\DeclareMathOperator{\Stab}{stab}
\DeclareMathOperator{\ppcm}{ppcm}
\DeclareMathOperator{\conj}{Conj}
\DeclareMathOperator{\End}{End}
\DeclareMathOperator{\rot}{rot}
\DeclareMathOperator{\trs}{trace}
\DeclareMathOperator{\Ind}{Ind}
\DeclareMathOperator{\mat}{Mat}
\DeclareMathOperator{\id}{Id}
\DeclareMathOperator{\vect}{vect}
\DeclareMathOperator{\img}{img}
\DeclareMathOperator{\cov}{Cov}
\DeclareMathOperator{\dist}{dist}
\DeclareMathOperator{\irr}{Irr}
\DeclareMathOperator{\image}{Im}
\DeclareMathOperator{\pd}{\partial}
\DeclareMathOperator{\epi}{epi}
\DeclareMathOperator{\Argmin}{Argmin}
\DeclareMathOperator{\dom}{dom}
\DeclareMathOperator{\proj}{proj}
\DeclareMathOperator{\ctg}{ctg}
\DeclareMathOperator{\supp}{supp}
\DeclareMathOperator{\argmin}{argmin}
\DeclareMathOperator{\mult}{mult}
\DeclareMathOperator{\ch}{ch}
\DeclareMathOperator{\sh}{sh}
\DeclareMathOperator{\rang}{rang}
\DeclareMathOperator{\diam}{diam}
\DeclareMathOperator{\Epigraphe}{Epigraphe}




\usepackage{xcolor}
\everymath{\color{blue}}
%\everymath{\color[rgb]{0,1,1}}
%\pagecolor[rgb]{0,0,0.5}


\newcommand*{\pdtest}[3][]{\ensuremath{\frac{\partial^{#1} #2}{\partial #3}}}

\newcommand*{\deffunc}[6][]{\ensuremath{
\begin{array}{rcl}
#2 : #3 &\rightarrow& #4\\
#5 &\mapsto& #6
\end{array}
}}

\newcommand{\eqcolon}{\mathrel{\resizebox{\widthof{$\mathord{=}$}}{\height}{ $\!\!=\!\!\resizebox{1.2\width}{0.8\height}{\raisebox{0.23ex}{$\mathop{:}$}}\!\!$ }}}
\newcommand{\coloneq}{\mathrel{\resizebox{\widthof{$\mathord{=}$}}{\height}{ $\!\!\resizebox{1.2\width}{0.8\height}{\raisebox{0.23ex}{$\mathop{:}$}}\!\!=\!\!$ }}}
\newcommand{\eqcolonl}{\ensuremath{\mathrel{=\!\!\mathop{:}}}}
\newcommand{\coloneql}{\ensuremath{\mathrel{\mathop{:} \!\! =}}}
\newcommand{\vc}[1]{% inline column vector
  \left(\begin{smallmatrix}#1\end{smallmatrix}\right)%
}
\newcommand{\vr}[1]{% inline row vector
  \begin{smallmatrix}(\,#1\,)\end{smallmatrix}%
}
\makeatletter
\newcommand*{\defeq}{\ =\mathrel{\rlap{%
                     \raisebox{0.3ex}{$\m@th\cdot$}}%
                     \raisebox{-0.3ex}{$\m@th\cdot$}}%
                     }
\makeatother

\newcommand{\mathcircle}[1]{% inline row vector
 \overset{\circ}{#1}
}
\newcommand{\ulim}{% low limit
 \underline{\lim}
}
\newcommand{\ssi}{% iff
\iff
}
\newcommand{\ps}[2]{
\expval{#1 | #2}
}
\newcommand{\df}[1]{
\mqty{#1}
}
\newcommand{\n}[1]{
\norm{#1}
}
\newcommand{\sys}[1]{
\left\{\smqty{#1}\right.
}


\newcommand{\eqdef}{\ensuremath{\overset{\text{def}}=}}


\def\Circlearrowright{\ensuremath{%
  \rotatebox[origin=c]{230}{$\circlearrowright$}}}

\newcommand\ct[1]{\text{\rmfamily\upshape #1}}
\newcommand\question[1]{ {\color{red} ...!? \small #1}}
\newcommand\caz[1]{\left\{\begin{array} #1 \end{array}\right.}
\newcommand\const{\text{\rmfamily\upshape const}}
\newcommand\toP{ \overset{\pro}{\to}}
\newcommand\toPP{ \overset{\text{PP}}{\to}}
\newcommand{\oeq}{\mathrel{\text{\textcircled{$=$}}}}





\usepackage{xcolor}
% \usepackage[normalem]{ulem}
\usepackage{lipsum}
\makeatletter
% \newcommand\colorwave[1][blue]{\bgroup \markoverwith{\lower3.5\p@\hbox{\sixly \textcolor{#1}{\char58}}}\ULon}
%\font\sixly=lasy6 % does not re-load if already loaded, so no memory problem.

\newmdtheoremenv[
linewidth= 1pt,linecolor= blue,%
leftmargin=20,rightmargin=20,innertopmargin=0pt, innerrightmargin=40,%
tikzsetting = { draw=lightgray, line width = 0.3pt,dashed,%
dash pattern = on 15pt off 3pt},%
splittopskip=\topskip,skipbelow=\baselineskip,%
skipabove=\baselineskip,ntheorem,roundcorner=0pt,
% backgroundcolor=pagebg,font=\color{orange}\sffamily, fontcolor=white
]{examplebox}{Exemple}[section]



\newcommand\R{\mathbb{R}}
\newcommand\Z{\mathbb{Z}}
\newcommand\N{\mathbb{N}}
\newcommand\E{\mathbb{E}}
\newcommand\F{\mathcal{F}}
\newcommand\cH{\mathcal{H}}
\newcommand\V{\mathbb{V}}
\newcommand\dmo{ ^{-1} }
\newcommand\kapa{\kappa}
\newcommand\im{Im}
\newcommand\hs{\mathcal{H}}





\usepackage{soul}

\makeatletter
\newcommand*{\whiten}[1]{\llap{\textcolor{white}{{\the\SOUL@token}}\hspace{#1pt}}}
\DeclareRobustCommand*\myul{%
    \def\SOUL@everyspace{\underline{\space}\kern\z@}%
    \def\SOUL@everytoken{%
     \setbox0=\hbox{\the\SOUL@token}%
     \ifdim\dp0>\z@
        \raisebox{\dp0}{\underline{\phantom{\the\SOUL@token}}}%
        \whiten{1}\whiten{0}%
        \whiten{-1}\whiten{-2}%
        \llap{\the\SOUL@token}%
     \else
        \underline{\the\SOUL@token}%
     \fi}%
\SOUL@}
\makeatother

\newcommand*{\demp}{\fontfamily{lmtt}\selectfont}

\DeclareTextFontCommand{\textdemp}{\demp}

\begin{document}

\ifcomment
Multiline
comment
\fi
\ifcomment
\myul{Typesetting test}
% \color[rgb]{1,1,1}
$∑_i^n≠ 60º±∞π∆¬≈√j∫h≤≥µ$

$\CR \R\pro\ind\pro\gS\pro
\mqty[a&b\\c&d]$
$\pro\mathbb{P}$
$\dd{x}$

  \[
    \alpha(x)=\left\{
                \begin{array}{ll}
                  x\\
                  \frac{1}{1+e^{-kx}}\\
                  \frac{e^x-e^{-x}}{e^x+e^{-x}}
                \end{array}
              \right.
  \]

  $\expval{x}$
  
  $\chi_\rho(ghg\dmo)=\Tr(\rho_{ghg\dmo})=\Tr(\rho_g\circ\rho_h\circ\rho\dmo_g)=\Tr(\rho_h)\overset{\mbox{\scalebox{0.5}{$\Tr(AB)=\Tr(BA)$}}}{=}\chi_\rho(h)$
  	$\mathop{\oplus}_{\substack{x\in X}}$

$\mat(\rho_g)=(a_{ij}(g))_{\scriptsize \substack{1\leq i\leq d \\ 1\leq j\leq d}}$ et $\mat(\rho'_g)=(a'_{ij}(g))_{\scriptsize \substack{1\leq i'\leq d' \\ 1\leq j'\leq d'}}$



\[\int_a^b{\mathbb{R}^2}g(u, v)\dd{P_{XY}}(u, v)=\iint g(u,v) f_{XY}(u, v)\dd \lambda(u) \dd \lambda(v)\]
$$\lim_{x\to\infty} f(x)$$	
$$\iiiint_V \mu(t,u,v,w) \,dt\,du\,dv\,dw$$
$$\sum_{n=1}^{\infty} 2^{-n} = 1$$	
\begin{definition}
	Si $X$ et $Y$ sont 2 v.a. ou definit la \textsc{Covariance} entre $X$ et $Y$ comme
	$\cov(X,Y)\overset{\text{def}}{=}\E\left[(X-\E(X))(Y-\E(Y))\right]=\E(XY)-\E(X)\E(Y)$.
\end{definition}
\fi
\pagebreak

% \tableofcontents

% insert your code here
%\input{./algebra/main.tex}
%\input{./geometrie-differentielle/main.tex}
%\input{./probabilite/main.tex}
%\input{./analyse-fonctionnelle/main.tex}
% \input{./Analyse-convexe-et-dualite-en-optimisation/main.tex}
%\input{./tikz/main.tex}
%\input{./Theorie-du-distributions/main.tex}
%\input{./optimisation/mine.tex}
 \input{./modelisation/main.tex}

% yves.aubry@univ-tln.fr : algebra

\end{document}

%\input{./optimisation/mine.tex}
 % !TEX encoding = UTF-8 Unicode
% !TEX TS-program = xelatex

\documentclass[french]{report}

%\usepackage[utf8]{inputenc}
%\usepackage[T1]{fontenc}
\usepackage{babel}


\newif\ifcomment
%\commenttrue # Show comments

\usepackage{physics}
\usepackage{amssymb}


\usepackage{amsthm}
% \usepackage{thmtools}
\usepackage{mathtools}
\usepackage{amsfonts}

\usepackage{color}

\usepackage{tikz}

\usepackage{geometry}
\geometry{a5paper, margin=0.1in, right=1cm}

\usepackage{dsfont}

\usepackage{graphicx}
\graphicspath{ {images/} }

\usepackage{faktor}

\usepackage{IEEEtrantools}
\usepackage{enumerate}   
\usepackage[PostScript=dvips]{"/Users/aware/Documents/Courses/diagrams"}


\newtheorem{theorem}{Théorème}[section]
\renewcommand{\thetheorem}{\arabic{theorem}}
\newtheorem{lemme}{Lemme}[section]
\renewcommand{\thelemme}{\arabic{lemme}}
\newtheorem{proposition}{Proposition}[section]
\renewcommand{\theproposition}{\arabic{proposition}}
\newtheorem{notations}{Notations}[section]
\newtheorem{problem}{Problème}[section]
\newtheorem{corollary}{Corollaire}[theorem]
\renewcommand{\thecorollary}{\arabic{corollary}}
\newtheorem{property}{Propriété}[section]
\newtheorem{objective}{Objectif}[section]

\theoremstyle{definition}
\newtheorem{definition}{Définition}[section]
\renewcommand{\thedefinition}{\arabic{definition}}
\newtheorem{exercise}{Exercice}[chapter]
\renewcommand{\theexercise}{\arabic{exercise}}
\newtheorem{example}{Exemple}[chapter]
\renewcommand{\theexample}{\arabic{example}}
\newtheorem*{solution}{Solution}
\newtheorem*{application}{Application}
\newtheorem*{notation}{Notation}
\newtheorem*{vocabulary}{Vocabulaire}
\newtheorem*{properties}{Propriétés}



\theoremstyle{remark}
\newtheorem*{remark}{Remarque}
\newtheorem*{rappel}{Rappel}


\usepackage{etoolbox}
\AtBeginEnvironment{exercise}{\small}
\AtBeginEnvironment{example}{\small}

\usepackage{cases}
\usepackage[red]{mypack}

\usepackage[framemethod=TikZ]{mdframed}

\definecolor{bg}{rgb}{0.4,0.25,0.95}
\definecolor{pagebg}{rgb}{0,0,0.5}
\surroundwithmdframed[
   topline=false,
   rightline=false,
   bottomline=false,
   leftmargin=\parindent,
   skipabove=8pt,
   skipbelow=8pt,
   linecolor=blue,
   innerbottommargin=10pt,
   % backgroundcolor=bg,font=\color{orange}\sffamily, fontcolor=white
]{definition}

\usepackage{empheq}
\usepackage[most]{tcolorbox}

\newtcbox{\mymath}[1][]{%
    nobeforeafter, math upper, tcbox raise base,
    enhanced, colframe=blue!30!black,
    colback=red!10, boxrule=1pt,
    #1}

\usepackage{unixode}


\DeclareMathOperator{\ord}{ord}
\DeclareMathOperator{\orb}{orb}
\DeclareMathOperator{\stab}{stab}
\DeclareMathOperator{\Stab}{stab}
\DeclareMathOperator{\ppcm}{ppcm}
\DeclareMathOperator{\conj}{Conj}
\DeclareMathOperator{\End}{End}
\DeclareMathOperator{\rot}{rot}
\DeclareMathOperator{\trs}{trace}
\DeclareMathOperator{\Ind}{Ind}
\DeclareMathOperator{\mat}{Mat}
\DeclareMathOperator{\id}{Id}
\DeclareMathOperator{\vect}{vect}
\DeclareMathOperator{\img}{img}
\DeclareMathOperator{\cov}{Cov}
\DeclareMathOperator{\dist}{dist}
\DeclareMathOperator{\irr}{Irr}
\DeclareMathOperator{\image}{Im}
\DeclareMathOperator{\pd}{\partial}
\DeclareMathOperator{\epi}{epi}
\DeclareMathOperator{\Argmin}{Argmin}
\DeclareMathOperator{\dom}{dom}
\DeclareMathOperator{\proj}{proj}
\DeclareMathOperator{\ctg}{ctg}
\DeclareMathOperator{\supp}{supp}
\DeclareMathOperator{\argmin}{argmin}
\DeclareMathOperator{\mult}{mult}
\DeclareMathOperator{\ch}{ch}
\DeclareMathOperator{\sh}{sh}
\DeclareMathOperator{\rang}{rang}
\DeclareMathOperator{\diam}{diam}
\DeclareMathOperator{\Epigraphe}{Epigraphe}




\usepackage{xcolor}
\everymath{\color{blue}}
%\everymath{\color[rgb]{0,1,1}}
%\pagecolor[rgb]{0,0,0.5}


\newcommand*{\pdtest}[3][]{\ensuremath{\frac{\partial^{#1} #2}{\partial #3}}}

\newcommand*{\deffunc}[6][]{\ensuremath{
\begin{array}{rcl}
#2 : #3 &\rightarrow& #4\\
#5 &\mapsto& #6
\end{array}
}}

\newcommand{\eqcolon}{\mathrel{\resizebox{\widthof{$\mathord{=}$}}{\height}{ $\!\!=\!\!\resizebox{1.2\width}{0.8\height}{\raisebox{0.23ex}{$\mathop{:}$}}\!\!$ }}}
\newcommand{\coloneq}{\mathrel{\resizebox{\widthof{$\mathord{=}$}}{\height}{ $\!\!\resizebox{1.2\width}{0.8\height}{\raisebox{0.23ex}{$\mathop{:}$}}\!\!=\!\!$ }}}
\newcommand{\eqcolonl}{\ensuremath{\mathrel{=\!\!\mathop{:}}}}
\newcommand{\coloneql}{\ensuremath{\mathrel{\mathop{:} \!\! =}}}
\newcommand{\vc}[1]{% inline column vector
  \left(\begin{smallmatrix}#1\end{smallmatrix}\right)%
}
\newcommand{\vr}[1]{% inline row vector
  \begin{smallmatrix}(\,#1\,)\end{smallmatrix}%
}
\makeatletter
\newcommand*{\defeq}{\ =\mathrel{\rlap{%
                     \raisebox{0.3ex}{$\m@th\cdot$}}%
                     \raisebox{-0.3ex}{$\m@th\cdot$}}%
                     }
\makeatother

\newcommand{\mathcircle}[1]{% inline row vector
 \overset{\circ}{#1}
}
\newcommand{\ulim}{% low limit
 \underline{\lim}
}
\newcommand{\ssi}{% iff
\iff
}
\newcommand{\ps}[2]{
\expval{#1 | #2}
}
\newcommand{\df}[1]{
\mqty{#1}
}
\newcommand{\n}[1]{
\norm{#1}
}
\newcommand{\sys}[1]{
\left\{\smqty{#1}\right.
}


\newcommand{\eqdef}{\ensuremath{\overset{\text{def}}=}}


\def\Circlearrowright{\ensuremath{%
  \rotatebox[origin=c]{230}{$\circlearrowright$}}}

\newcommand\ct[1]{\text{\rmfamily\upshape #1}}
\newcommand\question[1]{ {\color{red} ...!? \small #1}}
\newcommand\caz[1]{\left\{\begin{array} #1 \end{array}\right.}
\newcommand\const{\text{\rmfamily\upshape const}}
\newcommand\toP{ \overset{\pro}{\to}}
\newcommand\toPP{ \overset{\text{PP}}{\to}}
\newcommand{\oeq}{\mathrel{\text{\textcircled{$=$}}}}





\usepackage{xcolor}
% \usepackage[normalem]{ulem}
\usepackage{lipsum}
\makeatletter
% \newcommand\colorwave[1][blue]{\bgroup \markoverwith{\lower3.5\p@\hbox{\sixly \textcolor{#1}{\char58}}}\ULon}
%\font\sixly=lasy6 % does not re-load if already loaded, so no memory problem.

\newmdtheoremenv[
linewidth= 1pt,linecolor= blue,%
leftmargin=20,rightmargin=20,innertopmargin=0pt, innerrightmargin=40,%
tikzsetting = { draw=lightgray, line width = 0.3pt,dashed,%
dash pattern = on 15pt off 3pt},%
splittopskip=\topskip,skipbelow=\baselineskip,%
skipabove=\baselineskip,ntheorem,roundcorner=0pt,
% backgroundcolor=pagebg,font=\color{orange}\sffamily, fontcolor=white
]{examplebox}{Exemple}[section]



\newcommand\R{\mathbb{R}}
\newcommand\Z{\mathbb{Z}}
\newcommand\N{\mathbb{N}}
\newcommand\E{\mathbb{E}}
\newcommand\F{\mathcal{F}}
\newcommand\cH{\mathcal{H}}
\newcommand\V{\mathbb{V}}
\newcommand\dmo{ ^{-1} }
\newcommand\kapa{\kappa}
\newcommand\im{Im}
\newcommand\hs{\mathcal{H}}





\usepackage{soul}

\makeatletter
\newcommand*{\whiten}[1]{\llap{\textcolor{white}{{\the\SOUL@token}}\hspace{#1pt}}}
\DeclareRobustCommand*\myul{%
    \def\SOUL@everyspace{\underline{\space}\kern\z@}%
    \def\SOUL@everytoken{%
     \setbox0=\hbox{\the\SOUL@token}%
     \ifdim\dp0>\z@
        \raisebox{\dp0}{\underline{\phantom{\the\SOUL@token}}}%
        \whiten{1}\whiten{0}%
        \whiten{-1}\whiten{-2}%
        \llap{\the\SOUL@token}%
     \else
        \underline{\the\SOUL@token}%
     \fi}%
\SOUL@}
\makeatother

\newcommand*{\demp}{\fontfamily{lmtt}\selectfont}

\DeclareTextFontCommand{\textdemp}{\demp}

\begin{document}

\ifcomment
Multiline
comment
\fi
\ifcomment
\myul{Typesetting test}
% \color[rgb]{1,1,1}
$∑_i^n≠ 60º±∞π∆¬≈√j∫h≤≥µ$

$\CR \R\pro\ind\pro\gS\pro
\mqty[a&b\\c&d]$
$\pro\mathbb{P}$
$\dd{x}$

  \[
    \alpha(x)=\left\{
                \begin{array}{ll}
                  x\\
                  \frac{1}{1+e^{-kx}}\\
                  \frac{e^x-e^{-x}}{e^x+e^{-x}}
                \end{array}
              \right.
  \]

  $\expval{x}$
  
  $\chi_\rho(ghg\dmo)=\Tr(\rho_{ghg\dmo})=\Tr(\rho_g\circ\rho_h\circ\rho\dmo_g)=\Tr(\rho_h)\overset{\mbox{\scalebox{0.5}{$\Tr(AB)=\Tr(BA)$}}}{=}\chi_\rho(h)$
  	$\mathop{\oplus}_{\substack{x\in X}}$

$\mat(\rho_g)=(a_{ij}(g))_{\scriptsize \substack{1\leq i\leq d \\ 1\leq j\leq d}}$ et $\mat(\rho'_g)=(a'_{ij}(g))_{\scriptsize \substack{1\leq i'\leq d' \\ 1\leq j'\leq d'}}$



\[\int_a^b{\mathbb{R}^2}g(u, v)\dd{P_{XY}}(u, v)=\iint g(u,v) f_{XY}(u, v)\dd \lambda(u) \dd \lambda(v)\]
$$\lim_{x\to\infty} f(x)$$	
$$\iiiint_V \mu(t,u,v,w) \,dt\,du\,dv\,dw$$
$$\sum_{n=1}^{\infty} 2^{-n} = 1$$	
\begin{definition}
	Si $X$ et $Y$ sont 2 v.a. ou definit la \textsc{Covariance} entre $X$ et $Y$ comme
	$\cov(X,Y)\overset{\text{def}}{=}\E\left[(X-\E(X))(Y-\E(Y))\right]=\E(XY)-\E(X)\E(Y)$.
\end{definition}
\fi
\pagebreak

% \tableofcontents

% insert your code here
%\input{./algebra/main.tex}
%\input{./geometrie-differentielle/main.tex}
%\input{./probabilite/main.tex}
%\input{./analyse-fonctionnelle/main.tex}
% \input{./Analyse-convexe-et-dualite-en-optimisation/main.tex}
%\input{./tikz/main.tex}
%\input{./Theorie-du-distributions/main.tex}
%\input{./optimisation/mine.tex}
 \input{./modelisation/main.tex}

% yves.aubry@univ-tln.fr : algebra

\end{document}


% yves.aubry@univ-tln.fr : algebra

\end{document}

%% !TEX encoding = UTF-8 Unicode
% !TEX TS-program = xelatex

\documentclass[french]{report}

%\usepackage[utf8]{inputenc}
%\usepackage[T1]{fontenc}
\usepackage{babel}


\newif\ifcomment
%\commenttrue # Show comments

\usepackage{physics}
\usepackage{amssymb}


\usepackage{amsthm}
% \usepackage{thmtools}
\usepackage{mathtools}
\usepackage{amsfonts}

\usepackage{color}

\usepackage{tikz}

\usepackage{geometry}
\geometry{a5paper, margin=0.1in, right=1cm}

\usepackage{dsfont}

\usepackage{graphicx}
\graphicspath{ {images/} }

\usepackage{faktor}

\usepackage{IEEEtrantools}
\usepackage{enumerate}   
\usepackage[PostScript=dvips]{"/Users/aware/Documents/Courses/diagrams"}


\newtheorem{theorem}{Théorème}[section]
\renewcommand{\thetheorem}{\arabic{theorem}}
\newtheorem{lemme}{Lemme}[section]
\renewcommand{\thelemme}{\arabic{lemme}}
\newtheorem{proposition}{Proposition}[section]
\renewcommand{\theproposition}{\arabic{proposition}}
\newtheorem{notations}{Notations}[section]
\newtheorem{problem}{Problème}[section]
\newtheorem{corollary}{Corollaire}[theorem]
\renewcommand{\thecorollary}{\arabic{corollary}}
\newtheorem{property}{Propriété}[section]
\newtheorem{objective}{Objectif}[section]

\theoremstyle{definition}
\newtheorem{definition}{Définition}[section]
\renewcommand{\thedefinition}{\arabic{definition}}
\newtheorem{exercise}{Exercice}[chapter]
\renewcommand{\theexercise}{\arabic{exercise}}
\newtheorem{example}{Exemple}[chapter]
\renewcommand{\theexample}{\arabic{example}}
\newtheorem*{solution}{Solution}
\newtheorem*{application}{Application}
\newtheorem*{notation}{Notation}
\newtheorem*{vocabulary}{Vocabulaire}
\newtheorem*{properties}{Propriétés}



\theoremstyle{remark}
\newtheorem*{remark}{Remarque}
\newtheorem*{rappel}{Rappel}


\usepackage{etoolbox}
\AtBeginEnvironment{exercise}{\small}
\AtBeginEnvironment{example}{\small}

\usepackage{cases}
\usepackage[red]{mypack}

\usepackage[framemethod=TikZ]{mdframed}

\definecolor{bg}{rgb}{0.4,0.25,0.95}
\definecolor{pagebg}{rgb}{0,0,0.5}
\surroundwithmdframed[
   topline=false,
   rightline=false,
   bottomline=false,
   leftmargin=\parindent,
   skipabove=8pt,
   skipbelow=8pt,
   linecolor=blue,
   innerbottommargin=10pt,
   % backgroundcolor=bg,font=\color{orange}\sffamily, fontcolor=white
]{definition}

\usepackage{empheq}
\usepackage[most]{tcolorbox}

\newtcbox{\mymath}[1][]{%
    nobeforeafter, math upper, tcbox raise base,
    enhanced, colframe=blue!30!black,
    colback=red!10, boxrule=1pt,
    #1}

\usepackage{unixode}


\DeclareMathOperator{\ord}{ord}
\DeclareMathOperator{\orb}{orb}
\DeclareMathOperator{\stab}{stab}
\DeclareMathOperator{\Stab}{stab}
\DeclareMathOperator{\ppcm}{ppcm}
\DeclareMathOperator{\conj}{Conj}
\DeclareMathOperator{\End}{End}
\DeclareMathOperator{\rot}{rot}
\DeclareMathOperator{\trs}{trace}
\DeclareMathOperator{\Ind}{Ind}
\DeclareMathOperator{\mat}{Mat}
\DeclareMathOperator{\id}{Id}
\DeclareMathOperator{\vect}{vect}
\DeclareMathOperator{\img}{img}
\DeclareMathOperator{\cov}{Cov}
\DeclareMathOperator{\dist}{dist}
\DeclareMathOperator{\irr}{Irr}
\DeclareMathOperator{\image}{Im}
\DeclareMathOperator{\pd}{\partial}
\DeclareMathOperator{\epi}{epi}
\DeclareMathOperator{\Argmin}{Argmin}
\DeclareMathOperator{\dom}{dom}
\DeclareMathOperator{\proj}{proj}
\DeclareMathOperator{\ctg}{ctg}
\DeclareMathOperator{\supp}{supp}
\DeclareMathOperator{\argmin}{argmin}
\DeclareMathOperator{\mult}{mult}
\DeclareMathOperator{\ch}{ch}
\DeclareMathOperator{\sh}{sh}
\DeclareMathOperator{\rang}{rang}
\DeclareMathOperator{\diam}{diam}
\DeclareMathOperator{\Epigraphe}{Epigraphe}




\usepackage{xcolor}
\everymath{\color{blue}}
%\everymath{\color[rgb]{0,1,1}}
%\pagecolor[rgb]{0,0,0.5}


\newcommand*{\pdtest}[3][]{\ensuremath{\frac{\partial^{#1} #2}{\partial #3}}}

\newcommand*{\deffunc}[6][]{\ensuremath{
\begin{array}{rcl}
#2 : #3 &\rightarrow& #4\\
#5 &\mapsto& #6
\end{array}
}}

\newcommand{\eqcolon}{\mathrel{\resizebox{\widthof{$\mathord{=}$}}{\height}{ $\!\!=\!\!\resizebox{1.2\width}{0.8\height}{\raisebox{0.23ex}{$\mathop{:}$}}\!\!$ }}}
\newcommand{\coloneq}{\mathrel{\resizebox{\widthof{$\mathord{=}$}}{\height}{ $\!\!\resizebox{1.2\width}{0.8\height}{\raisebox{0.23ex}{$\mathop{:}$}}\!\!=\!\!$ }}}
\newcommand{\eqcolonl}{\ensuremath{\mathrel{=\!\!\mathop{:}}}}
\newcommand{\coloneql}{\ensuremath{\mathrel{\mathop{:} \!\! =}}}
\newcommand{\vc}[1]{% inline column vector
  \left(\begin{smallmatrix}#1\end{smallmatrix}\right)%
}
\newcommand{\vr}[1]{% inline row vector
  \begin{smallmatrix}(\,#1\,)\end{smallmatrix}%
}
\makeatletter
\newcommand*{\defeq}{\ =\mathrel{\rlap{%
                     \raisebox{0.3ex}{$\m@th\cdot$}}%
                     \raisebox{-0.3ex}{$\m@th\cdot$}}%
                     }
\makeatother

\newcommand{\mathcircle}[1]{% inline row vector
 \overset{\circ}{#1}
}
\newcommand{\ulim}{% low limit
 \underline{\lim}
}
\newcommand{\ssi}{% iff
\iff
}
\newcommand{\ps}[2]{
\expval{#1 | #2}
}
\newcommand{\df}[1]{
\mqty{#1}
}
\newcommand{\n}[1]{
\norm{#1}
}
\newcommand{\sys}[1]{
\left\{\smqty{#1}\right.
}


\newcommand{\eqdef}{\ensuremath{\overset{\text{def}}=}}


\def\Circlearrowright{\ensuremath{%
  \rotatebox[origin=c]{230}{$\circlearrowright$}}}

\newcommand\ct[1]{\text{\rmfamily\upshape #1}}
\newcommand\question[1]{ {\color{red} ...!? \small #1}}
\newcommand\caz[1]{\left\{\begin{array} #1 \end{array}\right.}
\newcommand\const{\text{\rmfamily\upshape const}}
\newcommand\toP{ \overset{\pro}{\to}}
\newcommand\toPP{ \overset{\text{PP}}{\to}}
\newcommand{\oeq}{\mathrel{\text{\textcircled{$=$}}}}





\usepackage{xcolor}
% \usepackage[normalem]{ulem}
\usepackage{lipsum}
\makeatletter
% \newcommand\colorwave[1][blue]{\bgroup \markoverwith{\lower3.5\p@\hbox{\sixly \textcolor{#1}{\char58}}}\ULon}
%\font\sixly=lasy6 % does not re-load if already loaded, so no memory problem.

\newmdtheoremenv[
linewidth= 1pt,linecolor= blue,%
leftmargin=20,rightmargin=20,innertopmargin=0pt, innerrightmargin=40,%
tikzsetting = { draw=lightgray, line width = 0.3pt,dashed,%
dash pattern = on 15pt off 3pt},%
splittopskip=\topskip,skipbelow=\baselineskip,%
skipabove=\baselineskip,ntheorem,roundcorner=0pt,
% backgroundcolor=pagebg,font=\color{orange}\sffamily, fontcolor=white
]{examplebox}{Exemple}[section]



\newcommand\R{\mathbb{R}}
\newcommand\Z{\mathbb{Z}}
\newcommand\N{\mathbb{N}}
\newcommand\E{\mathbb{E}}
\newcommand\F{\mathcal{F}}
\newcommand\cH{\mathcal{H}}
\newcommand\V{\mathbb{V}}
\newcommand\dmo{ ^{-1} }
\newcommand\kapa{\kappa}
\newcommand\im{Im}
\newcommand\hs{\mathcal{H}}





\usepackage{soul}

\makeatletter
\newcommand*{\whiten}[1]{\llap{\textcolor{white}{{\the\SOUL@token}}\hspace{#1pt}}}
\DeclareRobustCommand*\myul{%
    \def\SOUL@everyspace{\underline{\space}\kern\z@}%
    \def\SOUL@everytoken{%
     \setbox0=\hbox{\the\SOUL@token}%
     \ifdim\dp0>\z@
        \raisebox{\dp0}{\underline{\phantom{\the\SOUL@token}}}%
        \whiten{1}\whiten{0}%
        \whiten{-1}\whiten{-2}%
        \llap{\the\SOUL@token}%
     \else
        \underline{\the\SOUL@token}%
     \fi}%
\SOUL@}
\makeatother

\newcommand*{\demp}{\fontfamily{lmtt}\selectfont}

\DeclareTextFontCommand{\textdemp}{\demp}

\begin{document}

\ifcomment
Multiline
comment
\fi
\ifcomment
\myul{Typesetting test}
% \color[rgb]{1,1,1}
$∑_i^n≠ 60º±∞π∆¬≈√j∫h≤≥µ$

$\CR \R\pro\ind\pro\gS\pro
\mqty[a&b\\c&d]$
$\pro\mathbb{P}$
$\dd{x}$

  \[
    \alpha(x)=\left\{
                \begin{array}{ll}
                  x\\
                  \frac{1}{1+e^{-kx}}\\
                  \frac{e^x-e^{-x}}{e^x+e^{-x}}
                \end{array}
              \right.
  \]

  $\expval{x}$
  
  $\chi_\rho(ghg\dmo)=\Tr(\rho_{ghg\dmo})=\Tr(\rho_g\circ\rho_h\circ\rho\dmo_g)=\Tr(\rho_h)\overset{\mbox{\scalebox{0.5}{$\Tr(AB)=\Tr(BA)$}}}{=}\chi_\rho(h)$
  	$\mathop{\oplus}_{\substack{x\in X}}$

$\mat(\rho_g)=(a_{ij}(g))_{\scriptsize \substack{1\leq i\leq d \\ 1\leq j\leq d}}$ et $\mat(\rho'_g)=(a'_{ij}(g))_{\scriptsize \substack{1\leq i'\leq d' \\ 1\leq j'\leq d'}}$



\[\int_a^b{\mathbb{R}^2}g(u, v)\dd{P_{XY}}(u, v)=\iint g(u,v) f_{XY}(u, v)\dd \lambda(u) \dd \lambda(v)\]
$$\lim_{x\to\infty} f(x)$$	
$$\iiiint_V \mu(t,u,v,w) \,dt\,du\,dv\,dw$$
$$\sum_{n=1}^{\infty} 2^{-n} = 1$$	
\begin{definition}
	Si $X$ et $Y$ sont 2 v.a. ou definit la \textsc{Covariance} entre $X$ et $Y$ comme
	$\cov(X,Y)\overset{\text{def}}{=}\E\left[(X-\E(X))(Y-\E(Y))\right]=\E(XY)-\E(X)\E(Y)$.
\end{definition}
\fi
\pagebreak

% \tableofcontents

% insert your code here
%% !TEX encoding = UTF-8 Unicode
% !TEX TS-program = xelatex

\documentclass[french]{report}

%\usepackage[utf8]{inputenc}
%\usepackage[T1]{fontenc}
\usepackage{babel}


\newif\ifcomment
%\commenttrue # Show comments

\usepackage{physics}
\usepackage{amssymb}


\usepackage{amsthm}
% \usepackage{thmtools}
\usepackage{mathtools}
\usepackage{amsfonts}

\usepackage{color}

\usepackage{tikz}

\usepackage{geometry}
\geometry{a5paper, margin=0.1in, right=1cm}

\usepackage{dsfont}

\usepackage{graphicx}
\graphicspath{ {images/} }

\usepackage{faktor}

\usepackage{IEEEtrantools}
\usepackage{enumerate}   
\usepackage[PostScript=dvips]{"/Users/aware/Documents/Courses/diagrams"}


\newtheorem{theorem}{Théorème}[section]
\renewcommand{\thetheorem}{\arabic{theorem}}
\newtheorem{lemme}{Lemme}[section]
\renewcommand{\thelemme}{\arabic{lemme}}
\newtheorem{proposition}{Proposition}[section]
\renewcommand{\theproposition}{\arabic{proposition}}
\newtheorem{notations}{Notations}[section]
\newtheorem{problem}{Problème}[section]
\newtheorem{corollary}{Corollaire}[theorem]
\renewcommand{\thecorollary}{\arabic{corollary}}
\newtheorem{property}{Propriété}[section]
\newtheorem{objective}{Objectif}[section]

\theoremstyle{definition}
\newtheorem{definition}{Définition}[section]
\renewcommand{\thedefinition}{\arabic{definition}}
\newtheorem{exercise}{Exercice}[chapter]
\renewcommand{\theexercise}{\arabic{exercise}}
\newtheorem{example}{Exemple}[chapter]
\renewcommand{\theexample}{\arabic{example}}
\newtheorem*{solution}{Solution}
\newtheorem*{application}{Application}
\newtheorem*{notation}{Notation}
\newtheorem*{vocabulary}{Vocabulaire}
\newtheorem*{properties}{Propriétés}



\theoremstyle{remark}
\newtheorem*{remark}{Remarque}
\newtheorem*{rappel}{Rappel}


\usepackage{etoolbox}
\AtBeginEnvironment{exercise}{\small}
\AtBeginEnvironment{example}{\small}

\usepackage{cases}
\usepackage[red]{mypack}

\usepackage[framemethod=TikZ]{mdframed}

\definecolor{bg}{rgb}{0.4,0.25,0.95}
\definecolor{pagebg}{rgb}{0,0,0.5}
\surroundwithmdframed[
   topline=false,
   rightline=false,
   bottomline=false,
   leftmargin=\parindent,
   skipabove=8pt,
   skipbelow=8pt,
   linecolor=blue,
   innerbottommargin=10pt,
   % backgroundcolor=bg,font=\color{orange}\sffamily, fontcolor=white
]{definition}

\usepackage{empheq}
\usepackage[most]{tcolorbox}

\newtcbox{\mymath}[1][]{%
    nobeforeafter, math upper, tcbox raise base,
    enhanced, colframe=blue!30!black,
    colback=red!10, boxrule=1pt,
    #1}

\usepackage{unixode}


\DeclareMathOperator{\ord}{ord}
\DeclareMathOperator{\orb}{orb}
\DeclareMathOperator{\stab}{stab}
\DeclareMathOperator{\Stab}{stab}
\DeclareMathOperator{\ppcm}{ppcm}
\DeclareMathOperator{\conj}{Conj}
\DeclareMathOperator{\End}{End}
\DeclareMathOperator{\rot}{rot}
\DeclareMathOperator{\trs}{trace}
\DeclareMathOperator{\Ind}{Ind}
\DeclareMathOperator{\mat}{Mat}
\DeclareMathOperator{\id}{Id}
\DeclareMathOperator{\vect}{vect}
\DeclareMathOperator{\img}{img}
\DeclareMathOperator{\cov}{Cov}
\DeclareMathOperator{\dist}{dist}
\DeclareMathOperator{\irr}{Irr}
\DeclareMathOperator{\image}{Im}
\DeclareMathOperator{\pd}{\partial}
\DeclareMathOperator{\epi}{epi}
\DeclareMathOperator{\Argmin}{Argmin}
\DeclareMathOperator{\dom}{dom}
\DeclareMathOperator{\proj}{proj}
\DeclareMathOperator{\ctg}{ctg}
\DeclareMathOperator{\supp}{supp}
\DeclareMathOperator{\argmin}{argmin}
\DeclareMathOperator{\mult}{mult}
\DeclareMathOperator{\ch}{ch}
\DeclareMathOperator{\sh}{sh}
\DeclareMathOperator{\rang}{rang}
\DeclareMathOperator{\diam}{diam}
\DeclareMathOperator{\Epigraphe}{Epigraphe}




\usepackage{xcolor}
\everymath{\color{blue}}
%\everymath{\color[rgb]{0,1,1}}
%\pagecolor[rgb]{0,0,0.5}


\newcommand*{\pdtest}[3][]{\ensuremath{\frac{\partial^{#1} #2}{\partial #3}}}

\newcommand*{\deffunc}[6][]{\ensuremath{
\begin{array}{rcl}
#2 : #3 &\rightarrow& #4\\
#5 &\mapsto& #6
\end{array}
}}

\newcommand{\eqcolon}{\mathrel{\resizebox{\widthof{$\mathord{=}$}}{\height}{ $\!\!=\!\!\resizebox{1.2\width}{0.8\height}{\raisebox{0.23ex}{$\mathop{:}$}}\!\!$ }}}
\newcommand{\coloneq}{\mathrel{\resizebox{\widthof{$\mathord{=}$}}{\height}{ $\!\!\resizebox{1.2\width}{0.8\height}{\raisebox{0.23ex}{$\mathop{:}$}}\!\!=\!\!$ }}}
\newcommand{\eqcolonl}{\ensuremath{\mathrel{=\!\!\mathop{:}}}}
\newcommand{\coloneql}{\ensuremath{\mathrel{\mathop{:} \!\! =}}}
\newcommand{\vc}[1]{% inline column vector
  \left(\begin{smallmatrix}#1\end{smallmatrix}\right)%
}
\newcommand{\vr}[1]{% inline row vector
  \begin{smallmatrix}(\,#1\,)\end{smallmatrix}%
}
\makeatletter
\newcommand*{\defeq}{\ =\mathrel{\rlap{%
                     \raisebox{0.3ex}{$\m@th\cdot$}}%
                     \raisebox{-0.3ex}{$\m@th\cdot$}}%
                     }
\makeatother

\newcommand{\mathcircle}[1]{% inline row vector
 \overset{\circ}{#1}
}
\newcommand{\ulim}{% low limit
 \underline{\lim}
}
\newcommand{\ssi}{% iff
\iff
}
\newcommand{\ps}[2]{
\expval{#1 | #2}
}
\newcommand{\df}[1]{
\mqty{#1}
}
\newcommand{\n}[1]{
\norm{#1}
}
\newcommand{\sys}[1]{
\left\{\smqty{#1}\right.
}


\newcommand{\eqdef}{\ensuremath{\overset{\text{def}}=}}


\def\Circlearrowright{\ensuremath{%
  \rotatebox[origin=c]{230}{$\circlearrowright$}}}

\newcommand\ct[1]{\text{\rmfamily\upshape #1}}
\newcommand\question[1]{ {\color{red} ...!? \small #1}}
\newcommand\caz[1]{\left\{\begin{array} #1 \end{array}\right.}
\newcommand\const{\text{\rmfamily\upshape const}}
\newcommand\toP{ \overset{\pro}{\to}}
\newcommand\toPP{ \overset{\text{PP}}{\to}}
\newcommand{\oeq}{\mathrel{\text{\textcircled{$=$}}}}





\usepackage{xcolor}
% \usepackage[normalem]{ulem}
\usepackage{lipsum}
\makeatletter
% \newcommand\colorwave[1][blue]{\bgroup \markoverwith{\lower3.5\p@\hbox{\sixly \textcolor{#1}{\char58}}}\ULon}
%\font\sixly=lasy6 % does not re-load if already loaded, so no memory problem.

\newmdtheoremenv[
linewidth= 1pt,linecolor= blue,%
leftmargin=20,rightmargin=20,innertopmargin=0pt, innerrightmargin=40,%
tikzsetting = { draw=lightgray, line width = 0.3pt,dashed,%
dash pattern = on 15pt off 3pt},%
splittopskip=\topskip,skipbelow=\baselineskip,%
skipabove=\baselineskip,ntheorem,roundcorner=0pt,
% backgroundcolor=pagebg,font=\color{orange}\sffamily, fontcolor=white
]{examplebox}{Exemple}[section]



\newcommand\R{\mathbb{R}}
\newcommand\Z{\mathbb{Z}}
\newcommand\N{\mathbb{N}}
\newcommand\E{\mathbb{E}}
\newcommand\F{\mathcal{F}}
\newcommand\cH{\mathcal{H}}
\newcommand\V{\mathbb{V}}
\newcommand\dmo{ ^{-1} }
\newcommand\kapa{\kappa}
\newcommand\im{Im}
\newcommand\hs{\mathcal{H}}





\usepackage{soul}

\makeatletter
\newcommand*{\whiten}[1]{\llap{\textcolor{white}{{\the\SOUL@token}}\hspace{#1pt}}}
\DeclareRobustCommand*\myul{%
    \def\SOUL@everyspace{\underline{\space}\kern\z@}%
    \def\SOUL@everytoken{%
     \setbox0=\hbox{\the\SOUL@token}%
     \ifdim\dp0>\z@
        \raisebox{\dp0}{\underline{\phantom{\the\SOUL@token}}}%
        \whiten{1}\whiten{0}%
        \whiten{-1}\whiten{-2}%
        \llap{\the\SOUL@token}%
     \else
        \underline{\the\SOUL@token}%
     \fi}%
\SOUL@}
\makeatother

\newcommand*{\demp}{\fontfamily{lmtt}\selectfont}

\DeclareTextFontCommand{\textdemp}{\demp}

\begin{document}

\ifcomment
Multiline
comment
\fi
\ifcomment
\myul{Typesetting test}
% \color[rgb]{1,1,1}
$∑_i^n≠ 60º±∞π∆¬≈√j∫h≤≥µ$

$\CR \R\pro\ind\pro\gS\pro
\mqty[a&b\\c&d]$
$\pro\mathbb{P}$
$\dd{x}$

  \[
    \alpha(x)=\left\{
                \begin{array}{ll}
                  x\\
                  \frac{1}{1+e^{-kx}}\\
                  \frac{e^x-e^{-x}}{e^x+e^{-x}}
                \end{array}
              \right.
  \]

  $\expval{x}$
  
  $\chi_\rho(ghg\dmo)=\Tr(\rho_{ghg\dmo})=\Tr(\rho_g\circ\rho_h\circ\rho\dmo_g)=\Tr(\rho_h)\overset{\mbox{\scalebox{0.5}{$\Tr(AB)=\Tr(BA)$}}}{=}\chi_\rho(h)$
  	$\mathop{\oplus}_{\substack{x\in X}}$

$\mat(\rho_g)=(a_{ij}(g))_{\scriptsize \substack{1\leq i\leq d \\ 1\leq j\leq d}}$ et $\mat(\rho'_g)=(a'_{ij}(g))_{\scriptsize \substack{1\leq i'\leq d' \\ 1\leq j'\leq d'}}$



\[\int_a^b{\mathbb{R}^2}g(u, v)\dd{P_{XY}}(u, v)=\iint g(u,v) f_{XY}(u, v)\dd \lambda(u) \dd \lambda(v)\]
$$\lim_{x\to\infty} f(x)$$	
$$\iiiint_V \mu(t,u,v,w) \,dt\,du\,dv\,dw$$
$$\sum_{n=1}^{\infty} 2^{-n} = 1$$	
\begin{definition}
	Si $X$ et $Y$ sont 2 v.a. ou definit la \textsc{Covariance} entre $X$ et $Y$ comme
	$\cov(X,Y)\overset{\text{def}}{=}\E\left[(X-\E(X))(Y-\E(Y))\right]=\E(XY)-\E(X)\E(Y)$.
\end{definition}
\fi
\pagebreak

% \tableofcontents

% insert your code here
%\input{./algebra/main.tex}
%\input{./geometrie-differentielle/main.tex}
%\input{./probabilite/main.tex}
%\input{./analyse-fonctionnelle/main.tex}
% \input{./Analyse-convexe-et-dualite-en-optimisation/main.tex}
%\input{./tikz/main.tex}
%\input{./Theorie-du-distributions/main.tex}
%\input{./optimisation/mine.tex}
 \input{./modelisation/main.tex}

% yves.aubry@univ-tln.fr : algebra

\end{document}

%% !TEX encoding = UTF-8 Unicode
% !TEX TS-program = xelatex

\documentclass[french]{report}

%\usepackage[utf8]{inputenc}
%\usepackage[T1]{fontenc}
\usepackage{babel}


\newif\ifcomment
%\commenttrue # Show comments

\usepackage{physics}
\usepackage{amssymb}


\usepackage{amsthm}
% \usepackage{thmtools}
\usepackage{mathtools}
\usepackage{amsfonts}

\usepackage{color}

\usepackage{tikz}

\usepackage{geometry}
\geometry{a5paper, margin=0.1in, right=1cm}

\usepackage{dsfont}

\usepackage{graphicx}
\graphicspath{ {images/} }

\usepackage{faktor}

\usepackage{IEEEtrantools}
\usepackage{enumerate}   
\usepackage[PostScript=dvips]{"/Users/aware/Documents/Courses/diagrams"}


\newtheorem{theorem}{Théorème}[section]
\renewcommand{\thetheorem}{\arabic{theorem}}
\newtheorem{lemme}{Lemme}[section]
\renewcommand{\thelemme}{\arabic{lemme}}
\newtheorem{proposition}{Proposition}[section]
\renewcommand{\theproposition}{\arabic{proposition}}
\newtheorem{notations}{Notations}[section]
\newtheorem{problem}{Problème}[section]
\newtheorem{corollary}{Corollaire}[theorem]
\renewcommand{\thecorollary}{\arabic{corollary}}
\newtheorem{property}{Propriété}[section]
\newtheorem{objective}{Objectif}[section]

\theoremstyle{definition}
\newtheorem{definition}{Définition}[section]
\renewcommand{\thedefinition}{\arabic{definition}}
\newtheorem{exercise}{Exercice}[chapter]
\renewcommand{\theexercise}{\arabic{exercise}}
\newtheorem{example}{Exemple}[chapter]
\renewcommand{\theexample}{\arabic{example}}
\newtheorem*{solution}{Solution}
\newtheorem*{application}{Application}
\newtheorem*{notation}{Notation}
\newtheorem*{vocabulary}{Vocabulaire}
\newtheorem*{properties}{Propriétés}



\theoremstyle{remark}
\newtheorem*{remark}{Remarque}
\newtheorem*{rappel}{Rappel}


\usepackage{etoolbox}
\AtBeginEnvironment{exercise}{\small}
\AtBeginEnvironment{example}{\small}

\usepackage{cases}
\usepackage[red]{mypack}

\usepackage[framemethod=TikZ]{mdframed}

\definecolor{bg}{rgb}{0.4,0.25,0.95}
\definecolor{pagebg}{rgb}{0,0,0.5}
\surroundwithmdframed[
   topline=false,
   rightline=false,
   bottomline=false,
   leftmargin=\parindent,
   skipabove=8pt,
   skipbelow=8pt,
   linecolor=blue,
   innerbottommargin=10pt,
   % backgroundcolor=bg,font=\color{orange}\sffamily, fontcolor=white
]{definition}

\usepackage{empheq}
\usepackage[most]{tcolorbox}

\newtcbox{\mymath}[1][]{%
    nobeforeafter, math upper, tcbox raise base,
    enhanced, colframe=blue!30!black,
    colback=red!10, boxrule=1pt,
    #1}

\usepackage{unixode}


\DeclareMathOperator{\ord}{ord}
\DeclareMathOperator{\orb}{orb}
\DeclareMathOperator{\stab}{stab}
\DeclareMathOperator{\Stab}{stab}
\DeclareMathOperator{\ppcm}{ppcm}
\DeclareMathOperator{\conj}{Conj}
\DeclareMathOperator{\End}{End}
\DeclareMathOperator{\rot}{rot}
\DeclareMathOperator{\trs}{trace}
\DeclareMathOperator{\Ind}{Ind}
\DeclareMathOperator{\mat}{Mat}
\DeclareMathOperator{\id}{Id}
\DeclareMathOperator{\vect}{vect}
\DeclareMathOperator{\img}{img}
\DeclareMathOperator{\cov}{Cov}
\DeclareMathOperator{\dist}{dist}
\DeclareMathOperator{\irr}{Irr}
\DeclareMathOperator{\image}{Im}
\DeclareMathOperator{\pd}{\partial}
\DeclareMathOperator{\epi}{epi}
\DeclareMathOperator{\Argmin}{Argmin}
\DeclareMathOperator{\dom}{dom}
\DeclareMathOperator{\proj}{proj}
\DeclareMathOperator{\ctg}{ctg}
\DeclareMathOperator{\supp}{supp}
\DeclareMathOperator{\argmin}{argmin}
\DeclareMathOperator{\mult}{mult}
\DeclareMathOperator{\ch}{ch}
\DeclareMathOperator{\sh}{sh}
\DeclareMathOperator{\rang}{rang}
\DeclareMathOperator{\diam}{diam}
\DeclareMathOperator{\Epigraphe}{Epigraphe}




\usepackage{xcolor}
\everymath{\color{blue}}
%\everymath{\color[rgb]{0,1,1}}
%\pagecolor[rgb]{0,0,0.5}


\newcommand*{\pdtest}[3][]{\ensuremath{\frac{\partial^{#1} #2}{\partial #3}}}

\newcommand*{\deffunc}[6][]{\ensuremath{
\begin{array}{rcl}
#2 : #3 &\rightarrow& #4\\
#5 &\mapsto& #6
\end{array}
}}

\newcommand{\eqcolon}{\mathrel{\resizebox{\widthof{$\mathord{=}$}}{\height}{ $\!\!=\!\!\resizebox{1.2\width}{0.8\height}{\raisebox{0.23ex}{$\mathop{:}$}}\!\!$ }}}
\newcommand{\coloneq}{\mathrel{\resizebox{\widthof{$\mathord{=}$}}{\height}{ $\!\!\resizebox{1.2\width}{0.8\height}{\raisebox{0.23ex}{$\mathop{:}$}}\!\!=\!\!$ }}}
\newcommand{\eqcolonl}{\ensuremath{\mathrel{=\!\!\mathop{:}}}}
\newcommand{\coloneql}{\ensuremath{\mathrel{\mathop{:} \!\! =}}}
\newcommand{\vc}[1]{% inline column vector
  \left(\begin{smallmatrix}#1\end{smallmatrix}\right)%
}
\newcommand{\vr}[1]{% inline row vector
  \begin{smallmatrix}(\,#1\,)\end{smallmatrix}%
}
\makeatletter
\newcommand*{\defeq}{\ =\mathrel{\rlap{%
                     \raisebox{0.3ex}{$\m@th\cdot$}}%
                     \raisebox{-0.3ex}{$\m@th\cdot$}}%
                     }
\makeatother

\newcommand{\mathcircle}[1]{% inline row vector
 \overset{\circ}{#1}
}
\newcommand{\ulim}{% low limit
 \underline{\lim}
}
\newcommand{\ssi}{% iff
\iff
}
\newcommand{\ps}[2]{
\expval{#1 | #2}
}
\newcommand{\df}[1]{
\mqty{#1}
}
\newcommand{\n}[1]{
\norm{#1}
}
\newcommand{\sys}[1]{
\left\{\smqty{#1}\right.
}


\newcommand{\eqdef}{\ensuremath{\overset{\text{def}}=}}


\def\Circlearrowright{\ensuremath{%
  \rotatebox[origin=c]{230}{$\circlearrowright$}}}

\newcommand\ct[1]{\text{\rmfamily\upshape #1}}
\newcommand\question[1]{ {\color{red} ...!? \small #1}}
\newcommand\caz[1]{\left\{\begin{array} #1 \end{array}\right.}
\newcommand\const{\text{\rmfamily\upshape const}}
\newcommand\toP{ \overset{\pro}{\to}}
\newcommand\toPP{ \overset{\text{PP}}{\to}}
\newcommand{\oeq}{\mathrel{\text{\textcircled{$=$}}}}





\usepackage{xcolor}
% \usepackage[normalem]{ulem}
\usepackage{lipsum}
\makeatletter
% \newcommand\colorwave[1][blue]{\bgroup \markoverwith{\lower3.5\p@\hbox{\sixly \textcolor{#1}{\char58}}}\ULon}
%\font\sixly=lasy6 % does not re-load if already loaded, so no memory problem.

\newmdtheoremenv[
linewidth= 1pt,linecolor= blue,%
leftmargin=20,rightmargin=20,innertopmargin=0pt, innerrightmargin=40,%
tikzsetting = { draw=lightgray, line width = 0.3pt,dashed,%
dash pattern = on 15pt off 3pt},%
splittopskip=\topskip,skipbelow=\baselineskip,%
skipabove=\baselineskip,ntheorem,roundcorner=0pt,
% backgroundcolor=pagebg,font=\color{orange}\sffamily, fontcolor=white
]{examplebox}{Exemple}[section]



\newcommand\R{\mathbb{R}}
\newcommand\Z{\mathbb{Z}}
\newcommand\N{\mathbb{N}}
\newcommand\E{\mathbb{E}}
\newcommand\F{\mathcal{F}}
\newcommand\cH{\mathcal{H}}
\newcommand\V{\mathbb{V}}
\newcommand\dmo{ ^{-1} }
\newcommand\kapa{\kappa}
\newcommand\im{Im}
\newcommand\hs{\mathcal{H}}





\usepackage{soul}

\makeatletter
\newcommand*{\whiten}[1]{\llap{\textcolor{white}{{\the\SOUL@token}}\hspace{#1pt}}}
\DeclareRobustCommand*\myul{%
    \def\SOUL@everyspace{\underline{\space}\kern\z@}%
    \def\SOUL@everytoken{%
     \setbox0=\hbox{\the\SOUL@token}%
     \ifdim\dp0>\z@
        \raisebox{\dp0}{\underline{\phantom{\the\SOUL@token}}}%
        \whiten{1}\whiten{0}%
        \whiten{-1}\whiten{-2}%
        \llap{\the\SOUL@token}%
     \else
        \underline{\the\SOUL@token}%
     \fi}%
\SOUL@}
\makeatother

\newcommand*{\demp}{\fontfamily{lmtt}\selectfont}

\DeclareTextFontCommand{\textdemp}{\demp}

\begin{document}

\ifcomment
Multiline
comment
\fi
\ifcomment
\myul{Typesetting test}
% \color[rgb]{1,1,1}
$∑_i^n≠ 60º±∞π∆¬≈√j∫h≤≥µ$

$\CR \R\pro\ind\pro\gS\pro
\mqty[a&b\\c&d]$
$\pro\mathbb{P}$
$\dd{x}$

  \[
    \alpha(x)=\left\{
                \begin{array}{ll}
                  x\\
                  \frac{1}{1+e^{-kx}}\\
                  \frac{e^x-e^{-x}}{e^x+e^{-x}}
                \end{array}
              \right.
  \]

  $\expval{x}$
  
  $\chi_\rho(ghg\dmo)=\Tr(\rho_{ghg\dmo})=\Tr(\rho_g\circ\rho_h\circ\rho\dmo_g)=\Tr(\rho_h)\overset{\mbox{\scalebox{0.5}{$\Tr(AB)=\Tr(BA)$}}}{=}\chi_\rho(h)$
  	$\mathop{\oplus}_{\substack{x\in X}}$

$\mat(\rho_g)=(a_{ij}(g))_{\scriptsize \substack{1\leq i\leq d \\ 1\leq j\leq d}}$ et $\mat(\rho'_g)=(a'_{ij}(g))_{\scriptsize \substack{1\leq i'\leq d' \\ 1\leq j'\leq d'}}$



\[\int_a^b{\mathbb{R}^2}g(u, v)\dd{P_{XY}}(u, v)=\iint g(u,v) f_{XY}(u, v)\dd \lambda(u) \dd \lambda(v)\]
$$\lim_{x\to\infty} f(x)$$	
$$\iiiint_V \mu(t,u,v,w) \,dt\,du\,dv\,dw$$
$$\sum_{n=1}^{\infty} 2^{-n} = 1$$	
\begin{definition}
	Si $X$ et $Y$ sont 2 v.a. ou definit la \textsc{Covariance} entre $X$ et $Y$ comme
	$\cov(X,Y)\overset{\text{def}}{=}\E\left[(X-\E(X))(Y-\E(Y))\right]=\E(XY)-\E(X)\E(Y)$.
\end{definition}
\fi
\pagebreak

% \tableofcontents

% insert your code here
%\input{./algebra/main.tex}
%\input{./geometrie-differentielle/main.tex}
%\input{./probabilite/main.tex}
%\input{./analyse-fonctionnelle/main.tex}
% \input{./Analyse-convexe-et-dualite-en-optimisation/main.tex}
%\input{./tikz/main.tex}
%\input{./Theorie-du-distributions/main.tex}
%\input{./optimisation/mine.tex}
 \input{./modelisation/main.tex}

% yves.aubry@univ-tln.fr : algebra

\end{document}

%% !TEX encoding = UTF-8 Unicode
% !TEX TS-program = xelatex

\documentclass[french]{report}

%\usepackage[utf8]{inputenc}
%\usepackage[T1]{fontenc}
\usepackage{babel}


\newif\ifcomment
%\commenttrue # Show comments

\usepackage{physics}
\usepackage{amssymb}


\usepackage{amsthm}
% \usepackage{thmtools}
\usepackage{mathtools}
\usepackage{amsfonts}

\usepackage{color}

\usepackage{tikz}

\usepackage{geometry}
\geometry{a5paper, margin=0.1in, right=1cm}

\usepackage{dsfont}

\usepackage{graphicx}
\graphicspath{ {images/} }

\usepackage{faktor}

\usepackage{IEEEtrantools}
\usepackage{enumerate}   
\usepackage[PostScript=dvips]{"/Users/aware/Documents/Courses/diagrams"}


\newtheorem{theorem}{Théorème}[section]
\renewcommand{\thetheorem}{\arabic{theorem}}
\newtheorem{lemme}{Lemme}[section]
\renewcommand{\thelemme}{\arabic{lemme}}
\newtheorem{proposition}{Proposition}[section]
\renewcommand{\theproposition}{\arabic{proposition}}
\newtheorem{notations}{Notations}[section]
\newtheorem{problem}{Problème}[section]
\newtheorem{corollary}{Corollaire}[theorem]
\renewcommand{\thecorollary}{\arabic{corollary}}
\newtheorem{property}{Propriété}[section]
\newtheorem{objective}{Objectif}[section]

\theoremstyle{definition}
\newtheorem{definition}{Définition}[section]
\renewcommand{\thedefinition}{\arabic{definition}}
\newtheorem{exercise}{Exercice}[chapter]
\renewcommand{\theexercise}{\arabic{exercise}}
\newtheorem{example}{Exemple}[chapter]
\renewcommand{\theexample}{\arabic{example}}
\newtheorem*{solution}{Solution}
\newtheorem*{application}{Application}
\newtheorem*{notation}{Notation}
\newtheorem*{vocabulary}{Vocabulaire}
\newtheorem*{properties}{Propriétés}



\theoremstyle{remark}
\newtheorem*{remark}{Remarque}
\newtheorem*{rappel}{Rappel}


\usepackage{etoolbox}
\AtBeginEnvironment{exercise}{\small}
\AtBeginEnvironment{example}{\small}

\usepackage{cases}
\usepackage[red]{mypack}

\usepackage[framemethod=TikZ]{mdframed}

\definecolor{bg}{rgb}{0.4,0.25,0.95}
\definecolor{pagebg}{rgb}{0,0,0.5}
\surroundwithmdframed[
   topline=false,
   rightline=false,
   bottomline=false,
   leftmargin=\parindent,
   skipabove=8pt,
   skipbelow=8pt,
   linecolor=blue,
   innerbottommargin=10pt,
   % backgroundcolor=bg,font=\color{orange}\sffamily, fontcolor=white
]{definition}

\usepackage{empheq}
\usepackage[most]{tcolorbox}

\newtcbox{\mymath}[1][]{%
    nobeforeafter, math upper, tcbox raise base,
    enhanced, colframe=blue!30!black,
    colback=red!10, boxrule=1pt,
    #1}

\usepackage{unixode}


\DeclareMathOperator{\ord}{ord}
\DeclareMathOperator{\orb}{orb}
\DeclareMathOperator{\stab}{stab}
\DeclareMathOperator{\Stab}{stab}
\DeclareMathOperator{\ppcm}{ppcm}
\DeclareMathOperator{\conj}{Conj}
\DeclareMathOperator{\End}{End}
\DeclareMathOperator{\rot}{rot}
\DeclareMathOperator{\trs}{trace}
\DeclareMathOperator{\Ind}{Ind}
\DeclareMathOperator{\mat}{Mat}
\DeclareMathOperator{\id}{Id}
\DeclareMathOperator{\vect}{vect}
\DeclareMathOperator{\img}{img}
\DeclareMathOperator{\cov}{Cov}
\DeclareMathOperator{\dist}{dist}
\DeclareMathOperator{\irr}{Irr}
\DeclareMathOperator{\image}{Im}
\DeclareMathOperator{\pd}{\partial}
\DeclareMathOperator{\epi}{epi}
\DeclareMathOperator{\Argmin}{Argmin}
\DeclareMathOperator{\dom}{dom}
\DeclareMathOperator{\proj}{proj}
\DeclareMathOperator{\ctg}{ctg}
\DeclareMathOperator{\supp}{supp}
\DeclareMathOperator{\argmin}{argmin}
\DeclareMathOperator{\mult}{mult}
\DeclareMathOperator{\ch}{ch}
\DeclareMathOperator{\sh}{sh}
\DeclareMathOperator{\rang}{rang}
\DeclareMathOperator{\diam}{diam}
\DeclareMathOperator{\Epigraphe}{Epigraphe}




\usepackage{xcolor}
\everymath{\color{blue}}
%\everymath{\color[rgb]{0,1,1}}
%\pagecolor[rgb]{0,0,0.5}


\newcommand*{\pdtest}[3][]{\ensuremath{\frac{\partial^{#1} #2}{\partial #3}}}

\newcommand*{\deffunc}[6][]{\ensuremath{
\begin{array}{rcl}
#2 : #3 &\rightarrow& #4\\
#5 &\mapsto& #6
\end{array}
}}

\newcommand{\eqcolon}{\mathrel{\resizebox{\widthof{$\mathord{=}$}}{\height}{ $\!\!=\!\!\resizebox{1.2\width}{0.8\height}{\raisebox{0.23ex}{$\mathop{:}$}}\!\!$ }}}
\newcommand{\coloneq}{\mathrel{\resizebox{\widthof{$\mathord{=}$}}{\height}{ $\!\!\resizebox{1.2\width}{0.8\height}{\raisebox{0.23ex}{$\mathop{:}$}}\!\!=\!\!$ }}}
\newcommand{\eqcolonl}{\ensuremath{\mathrel{=\!\!\mathop{:}}}}
\newcommand{\coloneql}{\ensuremath{\mathrel{\mathop{:} \!\! =}}}
\newcommand{\vc}[1]{% inline column vector
  \left(\begin{smallmatrix}#1\end{smallmatrix}\right)%
}
\newcommand{\vr}[1]{% inline row vector
  \begin{smallmatrix}(\,#1\,)\end{smallmatrix}%
}
\makeatletter
\newcommand*{\defeq}{\ =\mathrel{\rlap{%
                     \raisebox{0.3ex}{$\m@th\cdot$}}%
                     \raisebox{-0.3ex}{$\m@th\cdot$}}%
                     }
\makeatother

\newcommand{\mathcircle}[1]{% inline row vector
 \overset{\circ}{#1}
}
\newcommand{\ulim}{% low limit
 \underline{\lim}
}
\newcommand{\ssi}{% iff
\iff
}
\newcommand{\ps}[2]{
\expval{#1 | #2}
}
\newcommand{\df}[1]{
\mqty{#1}
}
\newcommand{\n}[1]{
\norm{#1}
}
\newcommand{\sys}[1]{
\left\{\smqty{#1}\right.
}


\newcommand{\eqdef}{\ensuremath{\overset{\text{def}}=}}


\def\Circlearrowright{\ensuremath{%
  \rotatebox[origin=c]{230}{$\circlearrowright$}}}

\newcommand\ct[1]{\text{\rmfamily\upshape #1}}
\newcommand\question[1]{ {\color{red} ...!? \small #1}}
\newcommand\caz[1]{\left\{\begin{array} #1 \end{array}\right.}
\newcommand\const{\text{\rmfamily\upshape const}}
\newcommand\toP{ \overset{\pro}{\to}}
\newcommand\toPP{ \overset{\text{PP}}{\to}}
\newcommand{\oeq}{\mathrel{\text{\textcircled{$=$}}}}





\usepackage{xcolor}
% \usepackage[normalem]{ulem}
\usepackage{lipsum}
\makeatletter
% \newcommand\colorwave[1][blue]{\bgroup \markoverwith{\lower3.5\p@\hbox{\sixly \textcolor{#1}{\char58}}}\ULon}
%\font\sixly=lasy6 % does not re-load if already loaded, so no memory problem.

\newmdtheoremenv[
linewidth= 1pt,linecolor= blue,%
leftmargin=20,rightmargin=20,innertopmargin=0pt, innerrightmargin=40,%
tikzsetting = { draw=lightgray, line width = 0.3pt,dashed,%
dash pattern = on 15pt off 3pt},%
splittopskip=\topskip,skipbelow=\baselineskip,%
skipabove=\baselineskip,ntheorem,roundcorner=0pt,
% backgroundcolor=pagebg,font=\color{orange}\sffamily, fontcolor=white
]{examplebox}{Exemple}[section]



\newcommand\R{\mathbb{R}}
\newcommand\Z{\mathbb{Z}}
\newcommand\N{\mathbb{N}}
\newcommand\E{\mathbb{E}}
\newcommand\F{\mathcal{F}}
\newcommand\cH{\mathcal{H}}
\newcommand\V{\mathbb{V}}
\newcommand\dmo{ ^{-1} }
\newcommand\kapa{\kappa}
\newcommand\im{Im}
\newcommand\hs{\mathcal{H}}





\usepackage{soul}

\makeatletter
\newcommand*{\whiten}[1]{\llap{\textcolor{white}{{\the\SOUL@token}}\hspace{#1pt}}}
\DeclareRobustCommand*\myul{%
    \def\SOUL@everyspace{\underline{\space}\kern\z@}%
    \def\SOUL@everytoken{%
     \setbox0=\hbox{\the\SOUL@token}%
     \ifdim\dp0>\z@
        \raisebox{\dp0}{\underline{\phantom{\the\SOUL@token}}}%
        \whiten{1}\whiten{0}%
        \whiten{-1}\whiten{-2}%
        \llap{\the\SOUL@token}%
     \else
        \underline{\the\SOUL@token}%
     \fi}%
\SOUL@}
\makeatother

\newcommand*{\demp}{\fontfamily{lmtt}\selectfont}

\DeclareTextFontCommand{\textdemp}{\demp}

\begin{document}

\ifcomment
Multiline
comment
\fi
\ifcomment
\myul{Typesetting test}
% \color[rgb]{1,1,1}
$∑_i^n≠ 60º±∞π∆¬≈√j∫h≤≥µ$

$\CR \R\pro\ind\pro\gS\pro
\mqty[a&b\\c&d]$
$\pro\mathbb{P}$
$\dd{x}$

  \[
    \alpha(x)=\left\{
                \begin{array}{ll}
                  x\\
                  \frac{1}{1+e^{-kx}}\\
                  \frac{e^x-e^{-x}}{e^x+e^{-x}}
                \end{array}
              \right.
  \]

  $\expval{x}$
  
  $\chi_\rho(ghg\dmo)=\Tr(\rho_{ghg\dmo})=\Tr(\rho_g\circ\rho_h\circ\rho\dmo_g)=\Tr(\rho_h)\overset{\mbox{\scalebox{0.5}{$\Tr(AB)=\Tr(BA)$}}}{=}\chi_\rho(h)$
  	$\mathop{\oplus}_{\substack{x\in X}}$

$\mat(\rho_g)=(a_{ij}(g))_{\scriptsize \substack{1\leq i\leq d \\ 1\leq j\leq d}}$ et $\mat(\rho'_g)=(a'_{ij}(g))_{\scriptsize \substack{1\leq i'\leq d' \\ 1\leq j'\leq d'}}$



\[\int_a^b{\mathbb{R}^2}g(u, v)\dd{P_{XY}}(u, v)=\iint g(u,v) f_{XY}(u, v)\dd \lambda(u) \dd \lambda(v)\]
$$\lim_{x\to\infty} f(x)$$	
$$\iiiint_V \mu(t,u,v,w) \,dt\,du\,dv\,dw$$
$$\sum_{n=1}^{\infty} 2^{-n} = 1$$	
\begin{definition}
	Si $X$ et $Y$ sont 2 v.a. ou definit la \textsc{Covariance} entre $X$ et $Y$ comme
	$\cov(X,Y)\overset{\text{def}}{=}\E\left[(X-\E(X))(Y-\E(Y))\right]=\E(XY)-\E(X)\E(Y)$.
\end{definition}
\fi
\pagebreak

% \tableofcontents

% insert your code here
%\input{./algebra/main.tex}
%\input{./geometrie-differentielle/main.tex}
%\input{./probabilite/main.tex}
%\input{./analyse-fonctionnelle/main.tex}
% \input{./Analyse-convexe-et-dualite-en-optimisation/main.tex}
%\input{./tikz/main.tex}
%\input{./Theorie-du-distributions/main.tex}
%\input{./optimisation/mine.tex}
 \input{./modelisation/main.tex}

% yves.aubry@univ-tln.fr : algebra

\end{document}

%% !TEX encoding = UTF-8 Unicode
% !TEX TS-program = xelatex

\documentclass[french]{report}

%\usepackage[utf8]{inputenc}
%\usepackage[T1]{fontenc}
\usepackage{babel}


\newif\ifcomment
%\commenttrue # Show comments

\usepackage{physics}
\usepackage{amssymb}


\usepackage{amsthm}
% \usepackage{thmtools}
\usepackage{mathtools}
\usepackage{amsfonts}

\usepackage{color}

\usepackage{tikz}

\usepackage{geometry}
\geometry{a5paper, margin=0.1in, right=1cm}

\usepackage{dsfont}

\usepackage{graphicx}
\graphicspath{ {images/} }

\usepackage{faktor}

\usepackage{IEEEtrantools}
\usepackage{enumerate}   
\usepackage[PostScript=dvips]{"/Users/aware/Documents/Courses/diagrams"}


\newtheorem{theorem}{Théorème}[section]
\renewcommand{\thetheorem}{\arabic{theorem}}
\newtheorem{lemme}{Lemme}[section]
\renewcommand{\thelemme}{\arabic{lemme}}
\newtheorem{proposition}{Proposition}[section]
\renewcommand{\theproposition}{\arabic{proposition}}
\newtheorem{notations}{Notations}[section]
\newtheorem{problem}{Problème}[section]
\newtheorem{corollary}{Corollaire}[theorem]
\renewcommand{\thecorollary}{\arabic{corollary}}
\newtheorem{property}{Propriété}[section]
\newtheorem{objective}{Objectif}[section]

\theoremstyle{definition}
\newtheorem{definition}{Définition}[section]
\renewcommand{\thedefinition}{\arabic{definition}}
\newtheorem{exercise}{Exercice}[chapter]
\renewcommand{\theexercise}{\arabic{exercise}}
\newtheorem{example}{Exemple}[chapter]
\renewcommand{\theexample}{\arabic{example}}
\newtheorem*{solution}{Solution}
\newtheorem*{application}{Application}
\newtheorem*{notation}{Notation}
\newtheorem*{vocabulary}{Vocabulaire}
\newtheorem*{properties}{Propriétés}



\theoremstyle{remark}
\newtheorem*{remark}{Remarque}
\newtheorem*{rappel}{Rappel}


\usepackage{etoolbox}
\AtBeginEnvironment{exercise}{\small}
\AtBeginEnvironment{example}{\small}

\usepackage{cases}
\usepackage[red]{mypack}

\usepackage[framemethod=TikZ]{mdframed}

\definecolor{bg}{rgb}{0.4,0.25,0.95}
\definecolor{pagebg}{rgb}{0,0,0.5}
\surroundwithmdframed[
   topline=false,
   rightline=false,
   bottomline=false,
   leftmargin=\parindent,
   skipabove=8pt,
   skipbelow=8pt,
   linecolor=blue,
   innerbottommargin=10pt,
   % backgroundcolor=bg,font=\color{orange}\sffamily, fontcolor=white
]{definition}

\usepackage{empheq}
\usepackage[most]{tcolorbox}

\newtcbox{\mymath}[1][]{%
    nobeforeafter, math upper, tcbox raise base,
    enhanced, colframe=blue!30!black,
    colback=red!10, boxrule=1pt,
    #1}

\usepackage{unixode}


\DeclareMathOperator{\ord}{ord}
\DeclareMathOperator{\orb}{orb}
\DeclareMathOperator{\stab}{stab}
\DeclareMathOperator{\Stab}{stab}
\DeclareMathOperator{\ppcm}{ppcm}
\DeclareMathOperator{\conj}{Conj}
\DeclareMathOperator{\End}{End}
\DeclareMathOperator{\rot}{rot}
\DeclareMathOperator{\trs}{trace}
\DeclareMathOperator{\Ind}{Ind}
\DeclareMathOperator{\mat}{Mat}
\DeclareMathOperator{\id}{Id}
\DeclareMathOperator{\vect}{vect}
\DeclareMathOperator{\img}{img}
\DeclareMathOperator{\cov}{Cov}
\DeclareMathOperator{\dist}{dist}
\DeclareMathOperator{\irr}{Irr}
\DeclareMathOperator{\image}{Im}
\DeclareMathOperator{\pd}{\partial}
\DeclareMathOperator{\epi}{epi}
\DeclareMathOperator{\Argmin}{Argmin}
\DeclareMathOperator{\dom}{dom}
\DeclareMathOperator{\proj}{proj}
\DeclareMathOperator{\ctg}{ctg}
\DeclareMathOperator{\supp}{supp}
\DeclareMathOperator{\argmin}{argmin}
\DeclareMathOperator{\mult}{mult}
\DeclareMathOperator{\ch}{ch}
\DeclareMathOperator{\sh}{sh}
\DeclareMathOperator{\rang}{rang}
\DeclareMathOperator{\diam}{diam}
\DeclareMathOperator{\Epigraphe}{Epigraphe}




\usepackage{xcolor}
\everymath{\color{blue}}
%\everymath{\color[rgb]{0,1,1}}
%\pagecolor[rgb]{0,0,0.5}


\newcommand*{\pdtest}[3][]{\ensuremath{\frac{\partial^{#1} #2}{\partial #3}}}

\newcommand*{\deffunc}[6][]{\ensuremath{
\begin{array}{rcl}
#2 : #3 &\rightarrow& #4\\
#5 &\mapsto& #6
\end{array}
}}

\newcommand{\eqcolon}{\mathrel{\resizebox{\widthof{$\mathord{=}$}}{\height}{ $\!\!=\!\!\resizebox{1.2\width}{0.8\height}{\raisebox{0.23ex}{$\mathop{:}$}}\!\!$ }}}
\newcommand{\coloneq}{\mathrel{\resizebox{\widthof{$\mathord{=}$}}{\height}{ $\!\!\resizebox{1.2\width}{0.8\height}{\raisebox{0.23ex}{$\mathop{:}$}}\!\!=\!\!$ }}}
\newcommand{\eqcolonl}{\ensuremath{\mathrel{=\!\!\mathop{:}}}}
\newcommand{\coloneql}{\ensuremath{\mathrel{\mathop{:} \!\! =}}}
\newcommand{\vc}[1]{% inline column vector
  \left(\begin{smallmatrix}#1\end{smallmatrix}\right)%
}
\newcommand{\vr}[1]{% inline row vector
  \begin{smallmatrix}(\,#1\,)\end{smallmatrix}%
}
\makeatletter
\newcommand*{\defeq}{\ =\mathrel{\rlap{%
                     \raisebox{0.3ex}{$\m@th\cdot$}}%
                     \raisebox{-0.3ex}{$\m@th\cdot$}}%
                     }
\makeatother

\newcommand{\mathcircle}[1]{% inline row vector
 \overset{\circ}{#1}
}
\newcommand{\ulim}{% low limit
 \underline{\lim}
}
\newcommand{\ssi}{% iff
\iff
}
\newcommand{\ps}[2]{
\expval{#1 | #2}
}
\newcommand{\df}[1]{
\mqty{#1}
}
\newcommand{\n}[1]{
\norm{#1}
}
\newcommand{\sys}[1]{
\left\{\smqty{#1}\right.
}


\newcommand{\eqdef}{\ensuremath{\overset{\text{def}}=}}


\def\Circlearrowright{\ensuremath{%
  \rotatebox[origin=c]{230}{$\circlearrowright$}}}

\newcommand\ct[1]{\text{\rmfamily\upshape #1}}
\newcommand\question[1]{ {\color{red} ...!? \small #1}}
\newcommand\caz[1]{\left\{\begin{array} #1 \end{array}\right.}
\newcommand\const{\text{\rmfamily\upshape const}}
\newcommand\toP{ \overset{\pro}{\to}}
\newcommand\toPP{ \overset{\text{PP}}{\to}}
\newcommand{\oeq}{\mathrel{\text{\textcircled{$=$}}}}





\usepackage{xcolor}
% \usepackage[normalem]{ulem}
\usepackage{lipsum}
\makeatletter
% \newcommand\colorwave[1][blue]{\bgroup \markoverwith{\lower3.5\p@\hbox{\sixly \textcolor{#1}{\char58}}}\ULon}
%\font\sixly=lasy6 % does not re-load if already loaded, so no memory problem.

\newmdtheoremenv[
linewidth= 1pt,linecolor= blue,%
leftmargin=20,rightmargin=20,innertopmargin=0pt, innerrightmargin=40,%
tikzsetting = { draw=lightgray, line width = 0.3pt,dashed,%
dash pattern = on 15pt off 3pt},%
splittopskip=\topskip,skipbelow=\baselineskip,%
skipabove=\baselineskip,ntheorem,roundcorner=0pt,
% backgroundcolor=pagebg,font=\color{orange}\sffamily, fontcolor=white
]{examplebox}{Exemple}[section]



\newcommand\R{\mathbb{R}}
\newcommand\Z{\mathbb{Z}}
\newcommand\N{\mathbb{N}}
\newcommand\E{\mathbb{E}}
\newcommand\F{\mathcal{F}}
\newcommand\cH{\mathcal{H}}
\newcommand\V{\mathbb{V}}
\newcommand\dmo{ ^{-1} }
\newcommand\kapa{\kappa}
\newcommand\im{Im}
\newcommand\hs{\mathcal{H}}





\usepackage{soul}

\makeatletter
\newcommand*{\whiten}[1]{\llap{\textcolor{white}{{\the\SOUL@token}}\hspace{#1pt}}}
\DeclareRobustCommand*\myul{%
    \def\SOUL@everyspace{\underline{\space}\kern\z@}%
    \def\SOUL@everytoken{%
     \setbox0=\hbox{\the\SOUL@token}%
     \ifdim\dp0>\z@
        \raisebox{\dp0}{\underline{\phantom{\the\SOUL@token}}}%
        \whiten{1}\whiten{0}%
        \whiten{-1}\whiten{-2}%
        \llap{\the\SOUL@token}%
     \else
        \underline{\the\SOUL@token}%
     \fi}%
\SOUL@}
\makeatother

\newcommand*{\demp}{\fontfamily{lmtt}\selectfont}

\DeclareTextFontCommand{\textdemp}{\demp}

\begin{document}

\ifcomment
Multiline
comment
\fi
\ifcomment
\myul{Typesetting test}
% \color[rgb]{1,1,1}
$∑_i^n≠ 60º±∞π∆¬≈√j∫h≤≥µ$

$\CR \R\pro\ind\pro\gS\pro
\mqty[a&b\\c&d]$
$\pro\mathbb{P}$
$\dd{x}$

  \[
    \alpha(x)=\left\{
                \begin{array}{ll}
                  x\\
                  \frac{1}{1+e^{-kx}}\\
                  \frac{e^x-e^{-x}}{e^x+e^{-x}}
                \end{array}
              \right.
  \]

  $\expval{x}$
  
  $\chi_\rho(ghg\dmo)=\Tr(\rho_{ghg\dmo})=\Tr(\rho_g\circ\rho_h\circ\rho\dmo_g)=\Tr(\rho_h)\overset{\mbox{\scalebox{0.5}{$\Tr(AB)=\Tr(BA)$}}}{=}\chi_\rho(h)$
  	$\mathop{\oplus}_{\substack{x\in X}}$

$\mat(\rho_g)=(a_{ij}(g))_{\scriptsize \substack{1\leq i\leq d \\ 1\leq j\leq d}}$ et $\mat(\rho'_g)=(a'_{ij}(g))_{\scriptsize \substack{1\leq i'\leq d' \\ 1\leq j'\leq d'}}$



\[\int_a^b{\mathbb{R}^2}g(u, v)\dd{P_{XY}}(u, v)=\iint g(u,v) f_{XY}(u, v)\dd \lambda(u) \dd \lambda(v)\]
$$\lim_{x\to\infty} f(x)$$	
$$\iiiint_V \mu(t,u,v,w) \,dt\,du\,dv\,dw$$
$$\sum_{n=1}^{\infty} 2^{-n} = 1$$	
\begin{definition}
	Si $X$ et $Y$ sont 2 v.a. ou definit la \textsc{Covariance} entre $X$ et $Y$ comme
	$\cov(X,Y)\overset{\text{def}}{=}\E\left[(X-\E(X))(Y-\E(Y))\right]=\E(XY)-\E(X)\E(Y)$.
\end{definition}
\fi
\pagebreak

% \tableofcontents

% insert your code here
%\input{./algebra/main.tex}
%\input{./geometrie-differentielle/main.tex}
%\input{./probabilite/main.tex}
%\input{./analyse-fonctionnelle/main.tex}
% \input{./Analyse-convexe-et-dualite-en-optimisation/main.tex}
%\input{./tikz/main.tex}
%\input{./Theorie-du-distributions/main.tex}
%\input{./optimisation/mine.tex}
 \input{./modelisation/main.tex}

% yves.aubry@univ-tln.fr : algebra

\end{document}

% % !TEX encoding = UTF-8 Unicode
% !TEX TS-program = xelatex

\documentclass[french]{report}

%\usepackage[utf8]{inputenc}
%\usepackage[T1]{fontenc}
\usepackage{babel}


\newif\ifcomment
%\commenttrue # Show comments

\usepackage{physics}
\usepackage{amssymb}


\usepackage{amsthm}
% \usepackage{thmtools}
\usepackage{mathtools}
\usepackage{amsfonts}

\usepackage{color}

\usepackage{tikz}

\usepackage{geometry}
\geometry{a5paper, margin=0.1in, right=1cm}

\usepackage{dsfont}

\usepackage{graphicx}
\graphicspath{ {images/} }

\usepackage{faktor}

\usepackage{IEEEtrantools}
\usepackage{enumerate}   
\usepackage[PostScript=dvips]{"/Users/aware/Documents/Courses/diagrams"}


\newtheorem{theorem}{Théorème}[section]
\renewcommand{\thetheorem}{\arabic{theorem}}
\newtheorem{lemme}{Lemme}[section]
\renewcommand{\thelemme}{\arabic{lemme}}
\newtheorem{proposition}{Proposition}[section]
\renewcommand{\theproposition}{\arabic{proposition}}
\newtheorem{notations}{Notations}[section]
\newtheorem{problem}{Problème}[section]
\newtheorem{corollary}{Corollaire}[theorem]
\renewcommand{\thecorollary}{\arabic{corollary}}
\newtheorem{property}{Propriété}[section]
\newtheorem{objective}{Objectif}[section]

\theoremstyle{definition}
\newtheorem{definition}{Définition}[section]
\renewcommand{\thedefinition}{\arabic{definition}}
\newtheorem{exercise}{Exercice}[chapter]
\renewcommand{\theexercise}{\arabic{exercise}}
\newtheorem{example}{Exemple}[chapter]
\renewcommand{\theexample}{\arabic{example}}
\newtheorem*{solution}{Solution}
\newtheorem*{application}{Application}
\newtheorem*{notation}{Notation}
\newtheorem*{vocabulary}{Vocabulaire}
\newtheorem*{properties}{Propriétés}



\theoremstyle{remark}
\newtheorem*{remark}{Remarque}
\newtheorem*{rappel}{Rappel}


\usepackage{etoolbox}
\AtBeginEnvironment{exercise}{\small}
\AtBeginEnvironment{example}{\small}

\usepackage{cases}
\usepackage[red]{mypack}

\usepackage[framemethod=TikZ]{mdframed}

\definecolor{bg}{rgb}{0.4,0.25,0.95}
\definecolor{pagebg}{rgb}{0,0,0.5}
\surroundwithmdframed[
   topline=false,
   rightline=false,
   bottomline=false,
   leftmargin=\parindent,
   skipabove=8pt,
   skipbelow=8pt,
   linecolor=blue,
   innerbottommargin=10pt,
   % backgroundcolor=bg,font=\color{orange}\sffamily, fontcolor=white
]{definition}

\usepackage{empheq}
\usepackage[most]{tcolorbox}

\newtcbox{\mymath}[1][]{%
    nobeforeafter, math upper, tcbox raise base,
    enhanced, colframe=blue!30!black,
    colback=red!10, boxrule=1pt,
    #1}

\usepackage{unixode}


\DeclareMathOperator{\ord}{ord}
\DeclareMathOperator{\orb}{orb}
\DeclareMathOperator{\stab}{stab}
\DeclareMathOperator{\Stab}{stab}
\DeclareMathOperator{\ppcm}{ppcm}
\DeclareMathOperator{\conj}{Conj}
\DeclareMathOperator{\End}{End}
\DeclareMathOperator{\rot}{rot}
\DeclareMathOperator{\trs}{trace}
\DeclareMathOperator{\Ind}{Ind}
\DeclareMathOperator{\mat}{Mat}
\DeclareMathOperator{\id}{Id}
\DeclareMathOperator{\vect}{vect}
\DeclareMathOperator{\img}{img}
\DeclareMathOperator{\cov}{Cov}
\DeclareMathOperator{\dist}{dist}
\DeclareMathOperator{\irr}{Irr}
\DeclareMathOperator{\image}{Im}
\DeclareMathOperator{\pd}{\partial}
\DeclareMathOperator{\epi}{epi}
\DeclareMathOperator{\Argmin}{Argmin}
\DeclareMathOperator{\dom}{dom}
\DeclareMathOperator{\proj}{proj}
\DeclareMathOperator{\ctg}{ctg}
\DeclareMathOperator{\supp}{supp}
\DeclareMathOperator{\argmin}{argmin}
\DeclareMathOperator{\mult}{mult}
\DeclareMathOperator{\ch}{ch}
\DeclareMathOperator{\sh}{sh}
\DeclareMathOperator{\rang}{rang}
\DeclareMathOperator{\diam}{diam}
\DeclareMathOperator{\Epigraphe}{Epigraphe}




\usepackage{xcolor}
\everymath{\color{blue}}
%\everymath{\color[rgb]{0,1,1}}
%\pagecolor[rgb]{0,0,0.5}


\newcommand*{\pdtest}[3][]{\ensuremath{\frac{\partial^{#1} #2}{\partial #3}}}

\newcommand*{\deffunc}[6][]{\ensuremath{
\begin{array}{rcl}
#2 : #3 &\rightarrow& #4\\
#5 &\mapsto& #6
\end{array}
}}

\newcommand{\eqcolon}{\mathrel{\resizebox{\widthof{$\mathord{=}$}}{\height}{ $\!\!=\!\!\resizebox{1.2\width}{0.8\height}{\raisebox{0.23ex}{$\mathop{:}$}}\!\!$ }}}
\newcommand{\coloneq}{\mathrel{\resizebox{\widthof{$\mathord{=}$}}{\height}{ $\!\!\resizebox{1.2\width}{0.8\height}{\raisebox{0.23ex}{$\mathop{:}$}}\!\!=\!\!$ }}}
\newcommand{\eqcolonl}{\ensuremath{\mathrel{=\!\!\mathop{:}}}}
\newcommand{\coloneql}{\ensuremath{\mathrel{\mathop{:} \!\! =}}}
\newcommand{\vc}[1]{% inline column vector
  \left(\begin{smallmatrix}#1\end{smallmatrix}\right)%
}
\newcommand{\vr}[1]{% inline row vector
  \begin{smallmatrix}(\,#1\,)\end{smallmatrix}%
}
\makeatletter
\newcommand*{\defeq}{\ =\mathrel{\rlap{%
                     \raisebox{0.3ex}{$\m@th\cdot$}}%
                     \raisebox{-0.3ex}{$\m@th\cdot$}}%
                     }
\makeatother

\newcommand{\mathcircle}[1]{% inline row vector
 \overset{\circ}{#1}
}
\newcommand{\ulim}{% low limit
 \underline{\lim}
}
\newcommand{\ssi}{% iff
\iff
}
\newcommand{\ps}[2]{
\expval{#1 | #2}
}
\newcommand{\df}[1]{
\mqty{#1}
}
\newcommand{\n}[1]{
\norm{#1}
}
\newcommand{\sys}[1]{
\left\{\smqty{#1}\right.
}


\newcommand{\eqdef}{\ensuremath{\overset{\text{def}}=}}


\def\Circlearrowright{\ensuremath{%
  \rotatebox[origin=c]{230}{$\circlearrowright$}}}

\newcommand\ct[1]{\text{\rmfamily\upshape #1}}
\newcommand\question[1]{ {\color{red} ...!? \small #1}}
\newcommand\caz[1]{\left\{\begin{array} #1 \end{array}\right.}
\newcommand\const{\text{\rmfamily\upshape const}}
\newcommand\toP{ \overset{\pro}{\to}}
\newcommand\toPP{ \overset{\text{PP}}{\to}}
\newcommand{\oeq}{\mathrel{\text{\textcircled{$=$}}}}





\usepackage{xcolor}
% \usepackage[normalem]{ulem}
\usepackage{lipsum}
\makeatletter
% \newcommand\colorwave[1][blue]{\bgroup \markoverwith{\lower3.5\p@\hbox{\sixly \textcolor{#1}{\char58}}}\ULon}
%\font\sixly=lasy6 % does not re-load if already loaded, so no memory problem.

\newmdtheoremenv[
linewidth= 1pt,linecolor= blue,%
leftmargin=20,rightmargin=20,innertopmargin=0pt, innerrightmargin=40,%
tikzsetting = { draw=lightgray, line width = 0.3pt,dashed,%
dash pattern = on 15pt off 3pt},%
splittopskip=\topskip,skipbelow=\baselineskip,%
skipabove=\baselineskip,ntheorem,roundcorner=0pt,
% backgroundcolor=pagebg,font=\color{orange}\sffamily, fontcolor=white
]{examplebox}{Exemple}[section]



\newcommand\R{\mathbb{R}}
\newcommand\Z{\mathbb{Z}}
\newcommand\N{\mathbb{N}}
\newcommand\E{\mathbb{E}}
\newcommand\F{\mathcal{F}}
\newcommand\cH{\mathcal{H}}
\newcommand\V{\mathbb{V}}
\newcommand\dmo{ ^{-1} }
\newcommand\kapa{\kappa}
\newcommand\im{Im}
\newcommand\hs{\mathcal{H}}





\usepackage{soul}

\makeatletter
\newcommand*{\whiten}[1]{\llap{\textcolor{white}{{\the\SOUL@token}}\hspace{#1pt}}}
\DeclareRobustCommand*\myul{%
    \def\SOUL@everyspace{\underline{\space}\kern\z@}%
    \def\SOUL@everytoken{%
     \setbox0=\hbox{\the\SOUL@token}%
     \ifdim\dp0>\z@
        \raisebox{\dp0}{\underline{\phantom{\the\SOUL@token}}}%
        \whiten{1}\whiten{0}%
        \whiten{-1}\whiten{-2}%
        \llap{\the\SOUL@token}%
     \else
        \underline{\the\SOUL@token}%
     \fi}%
\SOUL@}
\makeatother

\newcommand*{\demp}{\fontfamily{lmtt}\selectfont}

\DeclareTextFontCommand{\textdemp}{\demp}

\begin{document}

\ifcomment
Multiline
comment
\fi
\ifcomment
\myul{Typesetting test}
% \color[rgb]{1,1,1}
$∑_i^n≠ 60º±∞π∆¬≈√j∫h≤≥µ$

$\CR \R\pro\ind\pro\gS\pro
\mqty[a&b\\c&d]$
$\pro\mathbb{P}$
$\dd{x}$

  \[
    \alpha(x)=\left\{
                \begin{array}{ll}
                  x\\
                  \frac{1}{1+e^{-kx}}\\
                  \frac{e^x-e^{-x}}{e^x+e^{-x}}
                \end{array}
              \right.
  \]

  $\expval{x}$
  
  $\chi_\rho(ghg\dmo)=\Tr(\rho_{ghg\dmo})=\Tr(\rho_g\circ\rho_h\circ\rho\dmo_g)=\Tr(\rho_h)\overset{\mbox{\scalebox{0.5}{$\Tr(AB)=\Tr(BA)$}}}{=}\chi_\rho(h)$
  	$\mathop{\oplus}_{\substack{x\in X}}$

$\mat(\rho_g)=(a_{ij}(g))_{\scriptsize \substack{1\leq i\leq d \\ 1\leq j\leq d}}$ et $\mat(\rho'_g)=(a'_{ij}(g))_{\scriptsize \substack{1\leq i'\leq d' \\ 1\leq j'\leq d'}}$



\[\int_a^b{\mathbb{R}^2}g(u, v)\dd{P_{XY}}(u, v)=\iint g(u,v) f_{XY}(u, v)\dd \lambda(u) \dd \lambda(v)\]
$$\lim_{x\to\infty} f(x)$$	
$$\iiiint_V \mu(t,u,v,w) \,dt\,du\,dv\,dw$$
$$\sum_{n=1}^{\infty} 2^{-n} = 1$$	
\begin{definition}
	Si $X$ et $Y$ sont 2 v.a. ou definit la \textsc{Covariance} entre $X$ et $Y$ comme
	$\cov(X,Y)\overset{\text{def}}{=}\E\left[(X-\E(X))(Y-\E(Y))\right]=\E(XY)-\E(X)\E(Y)$.
\end{definition}
\fi
\pagebreak

% \tableofcontents

% insert your code here
%\input{./algebra/main.tex}
%\input{./geometrie-differentielle/main.tex}
%\input{./probabilite/main.tex}
%\input{./analyse-fonctionnelle/main.tex}
% \input{./Analyse-convexe-et-dualite-en-optimisation/main.tex}
%\input{./tikz/main.tex}
%\input{./Theorie-du-distributions/main.tex}
%\input{./optimisation/mine.tex}
 \input{./modelisation/main.tex}

% yves.aubry@univ-tln.fr : algebra

\end{document}

%% !TEX encoding = UTF-8 Unicode
% !TEX TS-program = xelatex

\documentclass[french]{report}

%\usepackage[utf8]{inputenc}
%\usepackage[T1]{fontenc}
\usepackage{babel}


\newif\ifcomment
%\commenttrue # Show comments

\usepackage{physics}
\usepackage{amssymb}


\usepackage{amsthm}
% \usepackage{thmtools}
\usepackage{mathtools}
\usepackage{amsfonts}

\usepackage{color}

\usepackage{tikz}

\usepackage{geometry}
\geometry{a5paper, margin=0.1in, right=1cm}

\usepackage{dsfont}

\usepackage{graphicx}
\graphicspath{ {images/} }

\usepackage{faktor}

\usepackage{IEEEtrantools}
\usepackage{enumerate}   
\usepackage[PostScript=dvips]{"/Users/aware/Documents/Courses/diagrams"}


\newtheorem{theorem}{Théorème}[section]
\renewcommand{\thetheorem}{\arabic{theorem}}
\newtheorem{lemme}{Lemme}[section]
\renewcommand{\thelemme}{\arabic{lemme}}
\newtheorem{proposition}{Proposition}[section]
\renewcommand{\theproposition}{\arabic{proposition}}
\newtheorem{notations}{Notations}[section]
\newtheorem{problem}{Problème}[section]
\newtheorem{corollary}{Corollaire}[theorem]
\renewcommand{\thecorollary}{\arabic{corollary}}
\newtheorem{property}{Propriété}[section]
\newtheorem{objective}{Objectif}[section]

\theoremstyle{definition}
\newtheorem{definition}{Définition}[section]
\renewcommand{\thedefinition}{\arabic{definition}}
\newtheorem{exercise}{Exercice}[chapter]
\renewcommand{\theexercise}{\arabic{exercise}}
\newtheorem{example}{Exemple}[chapter]
\renewcommand{\theexample}{\arabic{example}}
\newtheorem*{solution}{Solution}
\newtheorem*{application}{Application}
\newtheorem*{notation}{Notation}
\newtheorem*{vocabulary}{Vocabulaire}
\newtheorem*{properties}{Propriétés}



\theoremstyle{remark}
\newtheorem*{remark}{Remarque}
\newtheorem*{rappel}{Rappel}


\usepackage{etoolbox}
\AtBeginEnvironment{exercise}{\small}
\AtBeginEnvironment{example}{\small}

\usepackage{cases}
\usepackage[red]{mypack}

\usepackage[framemethod=TikZ]{mdframed}

\definecolor{bg}{rgb}{0.4,0.25,0.95}
\definecolor{pagebg}{rgb}{0,0,0.5}
\surroundwithmdframed[
   topline=false,
   rightline=false,
   bottomline=false,
   leftmargin=\parindent,
   skipabove=8pt,
   skipbelow=8pt,
   linecolor=blue,
   innerbottommargin=10pt,
   % backgroundcolor=bg,font=\color{orange}\sffamily, fontcolor=white
]{definition}

\usepackage{empheq}
\usepackage[most]{tcolorbox}

\newtcbox{\mymath}[1][]{%
    nobeforeafter, math upper, tcbox raise base,
    enhanced, colframe=blue!30!black,
    colback=red!10, boxrule=1pt,
    #1}

\usepackage{unixode}


\DeclareMathOperator{\ord}{ord}
\DeclareMathOperator{\orb}{orb}
\DeclareMathOperator{\stab}{stab}
\DeclareMathOperator{\Stab}{stab}
\DeclareMathOperator{\ppcm}{ppcm}
\DeclareMathOperator{\conj}{Conj}
\DeclareMathOperator{\End}{End}
\DeclareMathOperator{\rot}{rot}
\DeclareMathOperator{\trs}{trace}
\DeclareMathOperator{\Ind}{Ind}
\DeclareMathOperator{\mat}{Mat}
\DeclareMathOperator{\id}{Id}
\DeclareMathOperator{\vect}{vect}
\DeclareMathOperator{\img}{img}
\DeclareMathOperator{\cov}{Cov}
\DeclareMathOperator{\dist}{dist}
\DeclareMathOperator{\irr}{Irr}
\DeclareMathOperator{\image}{Im}
\DeclareMathOperator{\pd}{\partial}
\DeclareMathOperator{\epi}{epi}
\DeclareMathOperator{\Argmin}{Argmin}
\DeclareMathOperator{\dom}{dom}
\DeclareMathOperator{\proj}{proj}
\DeclareMathOperator{\ctg}{ctg}
\DeclareMathOperator{\supp}{supp}
\DeclareMathOperator{\argmin}{argmin}
\DeclareMathOperator{\mult}{mult}
\DeclareMathOperator{\ch}{ch}
\DeclareMathOperator{\sh}{sh}
\DeclareMathOperator{\rang}{rang}
\DeclareMathOperator{\diam}{diam}
\DeclareMathOperator{\Epigraphe}{Epigraphe}




\usepackage{xcolor}
\everymath{\color{blue}}
%\everymath{\color[rgb]{0,1,1}}
%\pagecolor[rgb]{0,0,0.5}


\newcommand*{\pdtest}[3][]{\ensuremath{\frac{\partial^{#1} #2}{\partial #3}}}

\newcommand*{\deffunc}[6][]{\ensuremath{
\begin{array}{rcl}
#2 : #3 &\rightarrow& #4\\
#5 &\mapsto& #6
\end{array}
}}

\newcommand{\eqcolon}{\mathrel{\resizebox{\widthof{$\mathord{=}$}}{\height}{ $\!\!=\!\!\resizebox{1.2\width}{0.8\height}{\raisebox{0.23ex}{$\mathop{:}$}}\!\!$ }}}
\newcommand{\coloneq}{\mathrel{\resizebox{\widthof{$\mathord{=}$}}{\height}{ $\!\!\resizebox{1.2\width}{0.8\height}{\raisebox{0.23ex}{$\mathop{:}$}}\!\!=\!\!$ }}}
\newcommand{\eqcolonl}{\ensuremath{\mathrel{=\!\!\mathop{:}}}}
\newcommand{\coloneql}{\ensuremath{\mathrel{\mathop{:} \!\! =}}}
\newcommand{\vc}[1]{% inline column vector
  \left(\begin{smallmatrix}#1\end{smallmatrix}\right)%
}
\newcommand{\vr}[1]{% inline row vector
  \begin{smallmatrix}(\,#1\,)\end{smallmatrix}%
}
\makeatletter
\newcommand*{\defeq}{\ =\mathrel{\rlap{%
                     \raisebox{0.3ex}{$\m@th\cdot$}}%
                     \raisebox{-0.3ex}{$\m@th\cdot$}}%
                     }
\makeatother

\newcommand{\mathcircle}[1]{% inline row vector
 \overset{\circ}{#1}
}
\newcommand{\ulim}{% low limit
 \underline{\lim}
}
\newcommand{\ssi}{% iff
\iff
}
\newcommand{\ps}[2]{
\expval{#1 | #2}
}
\newcommand{\df}[1]{
\mqty{#1}
}
\newcommand{\n}[1]{
\norm{#1}
}
\newcommand{\sys}[1]{
\left\{\smqty{#1}\right.
}


\newcommand{\eqdef}{\ensuremath{\overset{\text{def}}=}}


\def\Circlearrowright{\ensuremath{%
  \rotatebox[origin=c]{230}{$\circlearrowright$}}}

\newcommand\ct[1]{\text{\rmfamily\upshape #1}}
\newcommand\question[1]{ {\color{red} ...!? \small #1}}
\newcommand\caz[1]{\left\{\begin{array} #1 \end{array}\right.}
\newcommand\const{\text{\rmfamily\upshape const}}
\newcommand\toP{ \overset{\pro}{\to}}
\newcommand\toPP{ \overset{\text{PP}}{\to}}
\newcommand{\oeq}{\mathrel{\text{\textcircled{$=$}}}}





\usepackage{xcolor}
% \usepackage[normalem]{ulem}
\usepackage{lipsum}
\makeatletter
% \newcommand\colorwave[1][blue]{\bgroup \markoverwith{\lower3.5\p@\hbox{\sixly \textcolor{#1}{\char58}}}\ULon}
%\font\sixly=lasy6 % does not re-load if already loaded, so no memory problem.

\newmdtheoremenv[
linewidth= 1pt,linecolor= blue,%
leftmargin=20,rightmargin=20,innertopmargin=0pt, innerrightmargin=40,%
tikzsetting = { draw=lightgray, line width = 0.3pt,dashed,%
dash pattern = on 15pt off 3pt},%
splittopskip=\topskip,skipbelow=\baselineskip,%
skipabove=\baselineskip,ntheorem,roundcorner=0pt,
% backgroundcolor=pagebg,font=\color{orange}\sffamily, fontcolor=white
]{examplebox}{Exemple}[section]



\newcommand\R{\mathbb{R}}
\newcommand\Z{\mathbb{Z}}
\newcommand\N{\mathbb{N}}
\newcommand\E{\mathbb{E}}
\newcommand\F{\mathcal{F}}
\newcommand\cH{\mathcal{H}}
\newcommand\V{\mathbb{V}}
\newcommand\dmo{ ^{-1} }
\newcommand\kapa{\kappa}
\newcommand\im{Im}
\newcommand\hs{\mathcal{H}}





\usepackage{soul}

\makeatletter
\newcommand*{\whiten}[1]{\llap{\textcolor{white}{{\the\SOUL@token}}\hspace{#1pt}}}
\DeclareRobustCommand*\myul{%
    \def\SOUL@everyspace{\underline{\space}\kern\z@}%
    \def\SOUL@everytoken{%
     \setbox0=\hbox{\the\SOUL@token}%
     \ifdim\dp0>\z@
        \raisebox{\dp0}{\underline{\phantom{\the\SOUL@token}}}%
        \whiten{1}\whiten{0}%
        \whiten{-1}\whiten{-2}%
        \llap{\the\SOUL@token}%
     \else
        \underline{\the\SOUL@token}%
     \fi}%
\SOUL@}
\makeatother

\newcommand*{\demp}{\fontfamily{lmtt}\selectfont}

\DeclareTextFontCommand{\textdemp}{\demp}

\begin{document}

\ifcomment
Multiline
comment
\fi
\ifcomment
\myul{Typesetting test}
% \color[rgb]{1,1,1}
$∑_i^n≠ 60º±∞π∆¬≈√j∫h≤≥µ$

$\CR \R\pro\ind\pro\gS\pro
\mqty[a&b\\c&d]$
$\pro\mathbb{P}$
$\dd{x}$

  \[
    \alpha(x)=\left\{
                \begin{array}{ll}
                  x\\
                  \frac{1}{1+e^{-kx}}\\
                  \frac{e^x-e^{-x}}{e^x+e^{-x}}
                \end{array}
              \right.
  \]

  $\expval{x}$
  
  $\chi_\rho(ghg\dmo)=\Tr(\rho_{ghg\dmo})=\Tr(\rho_g\circ\rho_h\circ\rho\dmo_g)=\Tr(\rho_h)\overset{\mbox{\scalebox{0.5}{$\Tr(AB)=\Tr(BA)$}}}{=}\chi_\rho(h)$
  	$\mathop{\oplus}_{\substack{x\in X}}$

$\mat(\rho_g)=(a_{ij}(g))_{\scriptsize \substack{1\leq i\leq d \\ 1\leq j\leq d}}$ et $\mat(\rho'_g)=(a'_{ij}(g))_{\scriptsize \substack{1\leq i'\leq d' \\ 1\leq j'\leq d'}}$



\[\int_a^b{\mathbb{R}^2}g(u, v)\dd{P_{XY}}(u, v)=\iint g(u,v) f_{XY}(u, v)\dd \lambda(u) \dd \lambda(v)\]
$$\lim_{x\to\infty} f(x)$$	
$$\iiiint_V \mu(t,u,v,w) \,dt\,du\,dv\,dw$$
$$\sum_{n=1}^{\infty} 2^{-n} = 1$$	
\begin{definition}
	Si $X$ et $Y$ sont 2 v.a. ou definit la \textsc{Covariance} entre $X$ et $Y$ comme
	$\cov(X,Y)\overset{\text{def}}{=}\E\left[(X-\E(X))(Y-\E(Y))\right]=\E(XY)-\E(X)\E(Y)$.
\end{definition}
\fi
\pagebreak

% \tableofcontents

% insert your code here
%\input{./algebra/main.tex}
%\input{./geometrie-differentielle/main.tex}
%\input{./probabilite/main.tex}
%\input{./analyse-fonctionnelle/main.tex}
% \input{./Analyse-convexe-et-dualite-en-optimisation/main.tex}
%\input{./tikz/main.tex}
%\input{./Theorie-du-distributions/main.tex}
%\input{./optimisation/mine.tex}
 \input{./modelisation/main.tex}

% yves.aubry@univ-tln.fr : algebra

\end{document}

%% !TEX encoding = UTF-8 Unicode
% !TEX TS-program = xelatex

\documentclass[french]{report}

%\usepackage[utf8]{inputenc}
%\usepackage[T1]{fontenc}
\usepackage{babel}


\newif\ifcomment
%\commenttrue # Show comments

\usepackage{physics}
\usepackage{amssymb}


\usepackage{amsthm}
% \usepackage{thmtools}
\usepackage{mathtools}
\usepackage{amsfonts}

\usepackage{color}

\usepackage{tikz}

\usepackage{geometry}
\geometry{a5paper, margin=0.1in, right=1cm}

\usepackage{dsfont}

\usepackage{graphicx}
\graphicspath{ {images/} }

\usepackage{faktor}

\usepackage{IEEEtrantools}
\usepackage{enumerate}   
\usepackage[PostScript=dvips]{"/Users/aware/Documents/Courses/diagrams"}


\newtheorem{theorem}{Théorème}[section]
\renewcommand{\thetheorem}{\arabic{theorem}}
\newtheorem{lemme}{Lemme}[section]
\renewcommand{\thelemme}{\arabic{lemme}}
\newtheorem{proposition}{Proposition}[section]
\renewcommand{\theproposition}{\arabic{proposition}}
\newtheorem{notations}{Notations}[section]
\newtheorem{problem}{Problème}[section]
\newtheorem{corollary}{Corollaire}[theorem]
\renewcommand{\thecorollary}{\arabic{corollary}}
\newtheorem{property}{Propriété}[section]
\newtheorem{objective}{Objectif}[section]

\theoremstyle{definition}
\newtheorem{definition}{Définition}[section]
\renewcommand{\thedefinition}{\arabic{definition}}
\newtheorem{exercise}{Exercice}[chapter]
\renewcommand{\theexercise}{\arabic{exercise}}
\newtheorem{example}{Exemple}[chapter]
\renewcommand{\theexample}{\arabic{example}}
\newtheorem*{solution}{Solution}
\newtheorem*{application}{Application}
\newtheorem*{notation}{Notation}
\newtheorem*{vocabulary}{Vocabulaire}
\newtheorem*{properties}{Propriétés}



\theoremstyle{remark}
\newtheorem*{remark}{Remarque}
\newtheorem*{rappel}{Rappel}


\usepackage{etoolbox}
\AtBeginEnvironment{exercise}{\small}
\AtBeginEnvironment{example}{\small}

\usepackage{cases}
\usepackage[red]{mypack}

\usepackage[framemethod=TikZ]{mdframed}

\definecolor{bg}{rgb}{0.4,0.25,0.95}
\definecolor{pagebg}{rgb}{0,0,0.5}
\surroundwithmdframed[
   topline=false,
   rightline=false,
   bottomline=false,
   leftmargin=\parindent,
   skipabove=8pt,
   skipbelow=8pt,
   linecolor=blue,
   innerbottommargin=10pt,
   % backgroundcolor=bg,font=\color{orange}\sffamily, fontcolor=white
]{definition}

\usepackage{empheq}
\usepackage[most]{tcolorbox}

\newtcbox{\mymath}[1][]{%
    nobeforeafter, math upper, tcbox raise base,
    enhanced, colframe=blue!30!black,
    colback=red!10, boxrule=1pt,
    #1}

\usepackage{unixode}


\DeclareMathOperator{\ord}{ord}
\DeclareMathOperator{\orb}{orb}
\DeclareMathOperator{\stab}{stab}
\DeclareMathOperator{\Stab}{stab}
\DeclareMathOperator{\ppcm}{ppcm}
\DeclareMathOperator{\conj}{Conj}
\DeclareMathOperator{\End}{End}
\DeclareMathOperator{\rot}{rot}
\DeclareMathOperator{\trs}{trace}
\DeclareMathOperator{\Ind}{Ind}
\DeclareMathOperator{\mat}{Mat}
\DeclareMathOperator{\id}{Id}
\DeclareMathOperator{\vect}{vect}
\DeclareMathOperator{\img}{img}
\DeclareMathOperator{\cov}{Cov}
\DeclareMathOperator{\dist}{dist}
\DeclareMathOperator{\irr}{Irr}
\DeclareMathOperator{\image}{Im}
\DeclareMathOperator{\pd}{\partial}
\DeclareMathOperator{\epi}{epi}
\DeclareMathOperator{\Argmin}{Argmin}
\DeclareMathOperator{\dom}{dom}
\DeclareMathOperator{\proj}{proj}
\DeclareMathOperator{\ctg}{ctg}
\DeclareMathOperator{\supp}{supp}
\DeclareMathOperator{\argmin}{argmin}
\DeclareMathOperator{\mult}{mult}
\DeclareMathOperator{\ch}{ch}
\DeclareMathOperator{\sh}{sh}
\DeclareMathOperator{\rang}{rang}
\DeclareMathOperator{\diam}{diam}
\DeclareMathOperator{\Epigraphe}{Epigraphe}




\usepackage{xcolor}
\everymath{\color{blue}}
%\everymath{\color[rgb]{0,1,1}}
%\pagecolor[rgb]{0,0,0.5}


\newcommand*{\pdtest}[3][]{\ensuremath{\frac{\partial^{#1} #2}{\partial #3}}}

\newcommand*{\deffunc}[6][]{\ensuremath{
\begin{array}{rcl}
#2 : #3 &\rightarrow& #4\\
#5 &\mapsto& #6
\end{array}
}}

\newcommand{\eqcolon}{\mathrel{\resizebox{\widthof{$\mathord{=}$}}{\height}{ $\!\!=\!\!\resizebox{1.2\width}{0.8\height}{\raisebox{0.23ex}{$\mathop{:}$}}\!\!$ }}}
\newcommand{\coloneq}{\mathrel{\resizebox{\widthof{$\mathord{=}$}}{\height}{ $\!\!\resizebox{1.2\width}{0.8\height}{\raisebox{0.23ex}{$\mathop{:}$}}\!\!=\!\!$ }}}
\newcommand{\eqcolonl}{\ensuremath{\mathrel{=\!\!\mathop{:}}}}
\newcommand{\coloneql}{\ensuremath{\mathrel{\mathop{:} \!\! =}}}
\newcommand{\vc}[1]{% inline column vector
  \left(\begin{smallmatrix}#1\end{smallmatrix}\right)%
}
\newcommand{\vr}[1]{% inline row vector
  \begin{smallmatrix}(\,#1\,)\end{smallmatrix}%
}
\makeatletter
\newcommand*{\defeq}{\ =\mathrel{\rlap{%
                     \raisebox{0.3ex}{$\m@th\cdot$}}%
                     \raisebox{-0.3ex}{$\m@th\cdot$}}%
                     }
\makeatother

\newcommand{\mathcircle}[1]{% inline row vector
 \overset{\circ}{#1}
}
\newcommand{\ulim}{% low limit
 \underline{\lim}
}
\newcommand{\ssi}{% iff
\iff
}
\newcommand{\ps}[2]{
\expval{#1 | #2}
}
\newcommand{\df}[1]{
\mqty{#1}
}
\newcommand{\n}[1]{
\norm{#1}
}
\newcommand{\sys}[1]{
\left\{\smqty{#1}\right.
}


\newcommand{\eqdef}{\ensuremath{\overset{\text{def}}=}}


\def\Circlearrowright{\ensuremath{%
  \rotatebox[origin=c]{230}{$\circlearrowright$}}}

\newcommand\ct[1]{\text{\rmfamily\upshape #1}}
\newcommand\question[1]{ {\color{red} ...!? \small #1}}
\newcommand\caz[1]{\left\{\begin{array} #1 \end{array}\right.}
\newcommand\const{\text{\rmfamily\upshape const}}
\newcommand\toP{ \overset{\pro}{\to}}
\newcommand\toPP{ \overset{\text{PP}}{\to}}
\newcommand{\oeq}{\mathrel{\text{\textcircled{$=$}}}}





\usepackage{xcolor}
% \usepackage[normalem]{ulem}
\usepackage{lipsum}
\makeatletter
% \newcommand\colorwave[1][blue]{\bgroup \markoverwith{\lower3.5\p@\hbox{\sixly \textcolor{#1}{\char58}}}\ULon}
%\font\sixly=lasy6 % does not re-load if already loaded, so no memory problem.

\newmdtheoremenv[
linewidth= 1pt,linecolor= blue,%
leftmargin=20,rightmargin=20,innertopmargin=0pt, innerrightmargin=40,%
tikzsetting = { draw=lightgray, line width = 0.3pt,dashed,%
dash pattern = on 15pt off 3pt},%
splittopskip=\topskip,skipbelow=\baselineskip,%
skipabove=\baselineskip,ntheorem,roundcorner=0pt,
% backgroundcolor=pagebg,font=\color{orange}\sffamily, fontcolor=white
]{examplebox}{Exemple}[section]



\newcommand\R{\mathbb{R}}
\newcommand\Z{\mathbb{Z}}
\newcommand\N{\mathbb{N}}
\newcommand\E{\mathbb{E}}
\newcommand\F{\mathcal{F}}
\newcommand\cH{\mathcal{H}}
\newcommand\V{\mathbb{V}}
\newcommand\dmo{ ^{-1} }
\newcommand\kapa{\kappa}
\newcommand\im{Im}
\newcommand\hs{\mathcal{H}}





\usepackage{soul}

\makeatletter
\newcommand*{\whiten}[1]{\llap{\textcolor{white}{{\the\SOUL@token}}\hspace{#1pt}}}
\DeclareRobustCommand*\myul{%
    \def\SOUL@everyspace{\underline{\space}\kern\z@}%
    \def\SOUL@everytoken{%
     \setbox0=\hbox{\the\SOUL@token}%
     \ifdim\dp0>\z@
        \raisebox{\dp0}{\underline{\phantom{\the\SOUL@token}}}%
        \whiten{1}\whiten{0}%
        \whiten{-1}\whiten{-2}%
        \llap{\the\SOUL@token}%
     \else
        \underline{\the\SOUL@token}%
     \fi}%
\SOUL@}
\makeatother

\newcommand*{\demp}{\fontfamily{lmtt}\selectfont}

\DeclareTextFontCommand{\textdemp}{\demp}

\begin{document}

\ifcomment
Multiline
comment
\fi
\ifcomment
\myul{Typesetting test}
% \color[rgb]{1,1,1}
$∑_i^n≠ 60º±∞π∆¬≈√j∫h≤≥µ$

$\CR \R\pro\ind\pro\gS\pro
\mqty[a&b\\c&d]$
$\pro\mathbb{P}$
$\dd{x}$

  \[
    \alpha(x)=\left\{
                \begin{array}{ll}
                  x\\
                  \frac{1}{1+e^{-kx}}\\
                  \frac{e^x-e^{-x}}{e^x+e^{-x}}
                \end{array}
              \right.
  \]

  $\expval{x}$
  
  $\chi_\rho(ghg\dmo)=\Tr(\rho_{ghg\dmo})=\Tr(\rho_g\circ\rho_h\circ\rho\dmo_g)=\Tr(\rho_h)\overset{\mbox{\scalebox{0.5}{$\Tr(AB)=\Tr(BA)$}}}{=}\chi_\rho(h)$
  	$\mathop{\oplus}_{\substack{x\in X}}$

$\mat(\rho_g)=(a_{ij}(g))_{\scriptsize \substack{1\leq i\leq d \\ 1\leq j\leq d}}$ et $\mat(\rho'_g)=(a'_{ij}(g))_{\scriptsize \substack{1\leq i'\leq d' \\ 1\leq j'\leq d'}}$



\[\int_a^b{\mathbb{R}^2}g(u, v)\dd{P_{XY}}(u, v)=\iint g(u,v) f_{XY}(u, v)\dd \lambda(u) \dd \lambda(v)\]
$$\lim_{x\to\infty} f(x)$$	
$$\iiiint_V \mu(t,u,v,w) \,dt\,du\,dv\,dw$$
$$\sum_{n=1}^{\infty} 2^{-n} = 1$$	
\begin{definition}
	Si $X$ et $Y$ sont 2 v.a. ou definit la \textsc{Covariance} entre $X$ et $Y$ comme
	$\cov(X,Y)\overset{\text{def}}{=}\E\left[(X-\E(X))(Y-\E(Y))\right]=\E(XY)-\E(X)\E(Y)$.
\end{definition}
\fi
\pagebreak

% \tableofcontents

% insert your code here
%\input{./algebra/main.tex}
%\input{./geometrie-differentielle/main.tex}
%\input{./probabilite/main.tex}
%\input{./analyse-fonctionnelle/main.tex}
% \input{./Analyse-convexe-et-dualite-en-optimisation/main.tex}
%\input{./tikz/main.tex}
%\input{./Theorie-du-distributions/main.tex}
%\input{./optimisation/mine.tex}
 \input{./modelisation/main.tex}

% yves.aubry@univ-tln.fr : algebra

\end{document}

%\input{./optimisation/mine.tex}
 % !TEX encoding = UTF-8 Unicode
% !TEX TS-program = xelatex

\documentclass[french]{report}

%\usepackage[utf8]{inputenc}
%\usepackage[T1]{fontenc}
\usepackage{babel}


\newif\ifcomment
%\commenttrue # Show comments

\usepackage{physics}
\usepackage{amssymb}


\usepackage{amsthm}
% \usepackage{thmtools}
\usepackage{mathtools}
\usepackage{amsfonts}

\usepackage{color}

\usepackage{tikz}

\usepackage{geometry}
\geometry{a5paper, margin=0.1in, right=1cm}

\usepackage{dsfont}

\usepackage{graphicx}
\graphicspath{ {images/} }

\usepackage{faktor}

\usepackage{IEEEtrantools}
\usepackage{enumerate}   
\usepackage[PostScript=dvips]{"/Users/aware/Documents/Courses/diagrams"}


\newtheorem{theorem}{Théorème}[section]
\renewcommand{\thetheorem}{\arabic{theorem}}
\newtheorem{lemme}{Lemme}[section]
\renewcommand{\thelemme}{\arabic{lemme}}
\newtheorem{proposition}{Proposition}[section]
\renewcommand{\theproposition}{\arabic{proposition}}
\newtheorem{notations}{Notations}[section]
\newtheorem{problem}{Problème}[section]
\newtheorem{corollary}{Corollaire}[theorem]
\renewcommand{\thecorollary}{\arabic{corollary}}
\newtheorem{property}{Propriété}[section]
\newtheorem{objective}{Objectif}[section]

\theoremstyle{definition}
\newtheorem{definition}{Définition}[section]
\renewcommand{\thedefinition}{\arabic{definition}}
\newtheorem{exercise}{Exercice}[chapter]
\renewcommand{\theexercise}{\arabic{exercise}}
\newtheorem{example}{Exemple}[chapter]
\renewcommand{\theexample}{\arabic{example}}
\newtheorem*{solution}{Solution}
\newtheorem*{application}{Application}
\newtheorem*{notation}{Notation}
\newtheorem*{vocabulary}{Vocabulaire}
\newtheorem*{properties}{Propriétés}



\theoremstyle{remark}
\newtheorem*{remark}{Remarque}
\newtheorem*{rappel}{Rappel}


\usepackage{etoolbox}
\AtBeginEnvironment{exercise}{\small}
\AtBeginEnvironment{example}{\small}

\usepackage{cases}
\usepackage[red]{mypack}

\usepackage[framemethod=TikZ]{mdframed}

\definecolor{bg}{rgb}{0.4,0.25,0.95}
\definecolor{pagebg}{rgb}{0,0,0.5}
\surroundwithmdframed[
   topline=false,
   rightline=false,
   bottomline=false,
   leftmargin=\parindent,
   skipabove=8pt,
   skipbelow=8pt,
   linecolor=blue,
   innerbottommargin=10pt,
   % backgroundcolor=bg,font=\color{orange}\sffamily, fontcolor=white
]{definition}

\usepackage{empheq}
\usepackage[most]{tcolorbox}

\newtcbox{\mymath}[1][]{%
    nobeforeafter, math upper, tcbox raise base,
    enhanced, colframe=blue!30!black,
    colback=red!10, boxrule=1pt,
    #1}

\usepackage{unixode}


\DeclareMathOperator{\ord}{ord}
\DeclareMathOperator{\orb}{orb}
\DeclareMathOperator{\stab}{stab}
\DeclareMathOperator{\Stab}{stab}
\DeclareMathOperator{\ppcm}{ppcm}
\DeclareMathOperator{\conj}{Conj}
\DeclareMathOperator{\End}{End}
\DeclareMathOperator{\rot}{rot}
\DeclareMathOperator{\trs}{trace}
\DeclareMathOperator{\Ind}{Ind}
\DeclareMathOperator{\mat}{Mat}
\DeclareMathOperator{\id}{Id}
\DeclareMathOperator{\vect}{vect}
\DeclareMathOperator{\img}{img}
\DeclareMathOperator{\cov}{Cov}
\DeclareMathOperator{\dist}{dist}
\DeclareMathOperator{\irr}{Irr}
\DeclareMathOperator{\image}{Im}
\DeclareMathOperator{\pd}{\partial}
\DeclareMathOperator{\epi}{epi}
\DeclareMathOperator{\Argmin}{Argmin}
\DeclareMathOperator{\dom}{dom}
\DeclareMathOperator{\proj}{proj}
\DeclareMathOperator{\ctg}{ctg}
\DeclareMathOperator{\supp}{supp}
\DeclareMathOperator{\argmin}{argmin}
\DeclareMathOperator{\mult}{mult}
\DeclareMathOperator{\ch}{ch}
\DeclareMathOperator{\sh}{sh}
\DeclareMathOperator{\rang}{rang}
\DeclareMathOperator{\diam}{diam}
\DeclareMathOperator{\Epigraphe}{Epigraphe}




\usepackage{xcolor}
\everymath{\color{blue}}
%\everymath{\color[rgb]{0,1,1}}
%\pagecolor[rgb]{0,0,0.5}


\newcommand*{\pdtest}[3][]{\ensuremath{\frac{\partial^{#1} #2}{\partial #3}}}

\newcommand*{\deffunc}[6][]{\ensuremath{
\begin{array}{rcl}
#2 : #3 &\rightarrow& #4\\
#5 &\mapsto& #6
\end{array}
}}

\newcommand{\eqcolon}{\mathrel{\resizebox{\widthof{$\mathord{=}$}}{\height}{ $\!\!=\!\!\resizebox{1.2\width}{0.8\height}{\raisebox{0.23ex}{$\mathop{:}$}}\!\!$ }}}
\newcommand{\coloneq}{\mathrel{\resizebox{\widthof{$\mathord{=}$}}{\height}{ $\!\!\resizebox{1.2\width}{0.8\height}{\raisebox{0.23ex}{$\mathop{:}$}}\!\!=\!\!$ }}}
\newcommand{\eqcolonl}{\ensuremath{\mathrel{=\!\!\mathop{:}}}}
\newcommand{\coloneql}{\ensuremath{\mathrel{\mathop{:} \!\! =}}}
\newcommand{\vc}[1]{% inline column vector
  \left(\begin{smallmatrix}#1\end{smallmatrix}\right)%
}
\newcommand{\vr}[1]{% inline row vector
  \begin{smallmatrix}(\,#1\,)\end{smallmatrix}%
}
\makeatletter
\newcommand*{\defeq}{\ =\mathrel{\rlap{%
                     \raisebox{0.3ex}{$\m@th\cdot$}}%
                     \raisebox{-0.3ex}{$\m@th\cdot$}}%
                     }
\makeatother

\newcommand{\mathcircle}[1]{% inline row vector
 \overset{\circ}{#1}
}
\newcommand{\ulim}{% low limit
 \underline{\lim}
}
\newcommand{\ssi}{% iff
\iff
}
\newcommand{\ps}[2]{
\expval{#1 | #2}
}
\newcommand{\df}[1]{
\mqty{#1}
}
\newcommand{\n}[1]{
\norm{#1}
}
\newcommand{\sys}[1]{
\left\{\smqty{#1}\right.
}


\newcommand{\eqdef}{\ensuremath{\overset{\text{def}}=}}


\def\Circlearrowright{\ensuremath{%
  \rotatebox[origin=c]{230}{$\circlearrowright$}}}

\newcommand\ct[1]{\text{\rmfamily\upshape #1}}
\newcommand\question[1]{ {\color{red} ...!? \small #1}}
\newcommand\caz[1]{\left\{\begin{array} #1 \end{array}\right.}
\newcommand\const{\text{\rmfamily\upshape const}}
\newcommand\toP{ \overset{\pro}{\to}}
\newcommand\toPP{ \overset{\text{PP}}{\to}}
\newcommand{\oeq}{\mathrel{\text{\textcircled{$=$}}}}





\usepackage{xcolor}
% \usepackage[normalem]{ulem}
\usepackage{lipsum}
\makeatletter
% \newcommand\colorwave[1][blue]{\bgroup \markoverwith{\lower3.5\p@\hbox{\sixly \textcolor{#1}{\char58}}}\ULon}
%\font\sixly=lasy6 % does not re-load if already loaded, so no memory problem.

\newmdtheoremenv[
linewidth= 1pt,linecolor= blue,%
leftmargin=20,rightmargin=20,innertopmargin=0pt, innerrightmargin=40,%
tikzsetting = { draw=lightgray, line width = 0.3pt,dashed,%
dash pattern = on 15pt off 3pt},%
splittopskip=\topskip,skipbelow=\baselineskip,%
skipabove=\baselineskip,ntheorem,roundcorner=0pt,
% backgroundcolor=pagebg,font=\color{orange}\sffamily, fontcolor=white
]{examplebox}{Exemple}[section]



\newcommand\R{\mathbb{R}}
\newcommand\Z{\mathbb{Z}}
\newcommand\N{\mathbb{N}}
\newcommand\E{\mathbb{E}}
\newcommand\F{\mathcal{F}}
\newcommand\cH{\mathcal{H}}
\newcommand\V{\mathbb{V}}
\newcommand\dmo{ ^{-1} }
\newcommand\kapa{\kappa}
\newcommand\im{Im}
\newcommand\hs{\mathcal{H}}





\usepackage{soul}

\makeatletter
\newcommand*{\whiten}[1]{\llap{\textcolor{white}{{\the\SOUL@token}}\hspace{#1pt}}}
\DeclareRobustCommand*\myul{%
    \def\SOUL@everyspace{\underline{\space}\kern\z@}%
    \def\SOUL@everytoken{%
     \setbox0=\hbox{\the\SOUL@token}%
     \ifdim\dp0>\z@
        \raisebox{\dp0}{\underline{\phantom{\the\SOUL@token}}}%
        \whiten{1}\whiten{0}%
        \whiten{-1}\whiten{-2}%
        \llap{\the\SOUL@token}%
     \else
        \underline{\the\SOUL@token}%
     \fi}%
\SOUL@}
\makeatother

\newcommand*{\demp}{\fontfamily{lmtt}\selectfont}

\DeclareTextFontCommand{\textdemp}{\demp}

\begin{document}

\ifcomment
Multiline
comment
\fi
\ifcomment
\myul{Typesetting test}
% \color[rgb]{1,1,1}
$∑_i^n≠ 60º±∞π∆¬≈√j∫h≤≥µ$

$\CR \R\pro\ind\pro\gS\pro
\mqty[a&b\\c&d]$
$\pro\mathbb{P}$
$\dd{x}$

  \[
    \alpha(x)=\left\{
                \begin{array}{ll}
                  x\\
                  \frac{1}{1+e^{-kx}}\\
                  \frac{e^x-e^{-x}}{e^x+e^{-x}}
                \end{array}
              \right.
  \]

  $\expval{x}$
  
  $\chi_\rho(ghg\dmo)=\Tr(\rho_{ghg\dmo})=\Tr(\rho_g\circ\rho_h\circ\rho\dmo_g)=\Tr(\rho_h)\overset{\mbox{\scalebox{0.5}{$\Tr(AB)=\Tr(BA)$}}}{=}\chi_\rho(h)$
  	$\mathop{\oplus}_{\substack{x\in X}}$

$\mat(\rho_g)=(a_{ij}(g))_{\scriptsize \substack{1\leq i\leq d \\ 1\leq j\leq d}}$ et $\mat(\rho'_g)=(a'_{ij}(g))_{\scriptsize \substack{1\leq i'\leq d' \\ 1\leq j'\leq d'}}$



\[\int_a^b{\mathbb{R}^2}g(u, v)\dd{P_{XY}}(u, v)=\iint g(u,v) f_{XY}(u, v)\dd \lambda(u) \dd \lambda(v)\]
$$\lim_{x\to\infty} f(x)$$	
$$\iiiint_V \mu(t,u,v,w) \,dt\,du\,dv\,dw$$
$$\sum_{n=1}^{\infty} 2^{-n} = 1$$	
\begin{definition}
	Si $X$ et $Y$ sont 2 v.a. ou definit la \textsc{Covariance} entre $X$ et $Y$ comme
	$\cov(X,Y)\overset{\text{def}}{=}\E\left[(X-\E(X))(Y-\E(Y))\right]=\E(XY)-\E(X)\E(Y)$.
\end{definition}
\fi
\pagebreak

% \tableofcontents

% insert your code here
%\input{./algebra/main.tex}
%\input{./geometrie-differentielle/main.tex}
%\input{./probabilite/main.tex}
%\input{./analyse-fonctionnelle/main.tex}
% \input{./Analyse-convexe-et-dualite-en-optimisation/main.tex}
%\input{./tikz/main.tex}
%\input{./Theorie-du-distributions/main.tex}
%\input{./optimisation/mine.tex}
 \input{./modelisation/main.tex}

% yves.aubry@univ-tln.fr : algebra

\end{document}


% yves.aubry@univ-tln.fr : algebra

\end{document}

%\input{./optimisation/mine.tex}
 % !TEX encoding = UTF-8 Unicode
% !TEX TS-program = xelatex

\documentclass[french]{report}

%\usepackage[utf8]{inputenc}
%\usepackage[T1]{fontenc}
\usepackage{babel}


\newif\ifcomment
%\commenttrue # Show comments

\usepackage{physics}
\usepackage{amssymb}


\usepackage{amsthm}
% \usepackage{thmtools}
\usepackage{mathtools}
\usepackage{amsfonts}

\usepackage{color}

\usepackage{tikz}

\usepackage{geometry}
\geometry{a5paper, margin=0.1in, right=1cm}

\usepackage{dsfont}

\usepackage{graphicx}
\graphicspath{ {images/} }

\usepackage{faktor}

\usepackage{IEEEtrantools}
\usepackage{enumerate}   
\usepackage[PostScript=dvips]{"/Users/aware/Documents/Courses/diagrams"}


\newtheorem{theorem}{Théorème}[section]
\renewcommand{\thetheorem}{\arabic{theorem}}
\newtheorem{lemme}{Lemme}[section]
\renewcommand{\thelemme}{\arabic{lemme}}
\newtheorem{proposition}{Proposition}[section]
\renewcommand{\theproposition}{\arabic{proposition}}
\newtheorem{notations}{Notations}[section]
\newtheorem{problem}{Problème}[section]
\newtheorem{corollary}{Corollaire}[theorem]
\renewcommand{\thecorollary}{\arabic{corollary}}
\newtheorem{property}{Propriété}[section]
\newtheorem{objective}{Objectif}[section]

\theoremstyle{definition}
\newtheorem{definition}{Définition}[section]
\renewcommand{\thedefinition}{\arabic{definition}}
\newtheorem{exercise}{Exercice}[chapter]
\renewcommand{\theexercise}{\arabic{exercise}}
\newtheorem{example}{Exemple}[chapter]
\renewcommand{\theexample}{\arabic{example}}
\newtheorem*{solution}{Solution}
\newtheorem*{application}{Application}
\newtheorem*{notation}{Notation}
\newtheorem*{vocabulary}{Vocabulaire}
\newtheorem*{properties}{Propriétés}



\theoremstyle{remark}
\newtheorem*{remark}{Remarque}
\newtheorem*{rappel}{Rappel}


\usepackage{etoolbox}
\AtBeginEnvironment{exercise}{\small}
\AtBeginEnvironment{example}{\small}

\usepackage{cases}
\usepackage[red]{mypack}

\usepackage[framemethod=TikZ]{mdframed}

\definecolor{bg}{rgb}{0.4,0.25,0.95}
\definecolor{pagebg}{rgb}{0,0,0.5}
\surroundwithmdframed[
   topline=false,
   rightline=false,
   bottomline=false,
   leftmargin=\parindent,
   skipabove=8pt,
   skipbelow=8pt,
   linecolor=blue,
   innerbottommargin=10pt,
   % backgroundcolor=bg,font=\color{orange}\sffamily, fontcolor=white
]{definition}

\usepackage{empheq}
\usepackage[most]{tcolorbox}

\newtcbox{\mymath}[1][]{%
    nobeforeafter, math upper, tcbox raise base,
    enhanced, colframe=blue!30!black,
    colback=red!10, boxrule=1pt,
    #1}

\usepackage{unixode}


\DeclareMathOperator{\ord}{ord}
\DeclareMathOperator{\orb}{orb}
\DeclareMathOperator{\stab}{stab}
\DeclareMathOperator{\Stab}{stab}
\DeclareMathOperator{\ppcm}{ppcm}
\DeclareMathOperator{\conj}{Conj}
\DeclareMathOperator{\End}{End}
\DeclareMathOperator{\rot}{rot}
\DeclareMathOperator{\trs}{trace}
\DeclareMathOperator{\Ind}{Ind}
\DeclareMathOperator{\mat}{Mat}
\DeclareMathOperator{\id}{Id}
\DeclareMathOperator{\vect}{vect}
\DeclareMathOperator{\img}{img}
\DeclareMathOperator{\cov}{Cov}
\DeclareMathOperator{\dist}{dist}
\DeclareMathOperator{\irr}{Irr}
\DeclareMathOperator{\image}{Im}
\DeclareMathOperator{\pd}{\partial}
\DeclareMathOperator{\epi}{epi}
\DeclareMathOperator{\Argmin}{Argmin}
\DeclareMathOperator{\dom}{dom}
\DeclareMathOperator{\proj}{proj}
\DeclareMathOperator{\ctg}{ctg}
\DeclareMathOperator{\supp}{supp}
\DeclareMathOperator{\argmin}{argmin}
\DeclareMathOperator{\mult}{mult}
\DeclareMathOperator{\ch}{ch}
\DeclareMathOperator{\sh}{sh}
\DeclareMathOperator{\rang}{rang}
\DeclareMathOperator{\diam}{diam}
\DeclareMathOperator{\Epigraphe}{Epigraphe}




\usepackage{xcolor}
\everymath{\color{blue}}
%\everymath{\color[rgb]{0,1,1}}
%\pagecolor[rgb]{0,0,0.5}


\newcommand*{\pdtest}[3][]{\ensuremath{\frac{\partial^{#1} #2}{\partial #3}}}

\newcommand*{\deffunc}[6][]{\ensuremath{
\begin{array}{rcl}
#2 : #3 &\rightarrow& #4\\
#5 &\mapsto& #6
\end{array}
}}

\newcommand{\eqcolon}{\mathrel{\resizebox{\widthof{$\mathord{=}$}}{\height}{ $\!\!=\!\!\resizebox{1.2\width}{0.8\height}{\raisebox{0.23ex}{$\mathop{:}$}}\!\!$ }}}
\newcommand{\coloneq}{\mathrel{\resizebox{\widthof{$\mathord{=}$}}{\height}{ $\!\!\resizebox{1.2\width}{0.8\height}{\raisebox{0.23ex}{$\mathop{:}$}}\!\!=\!\!$ }}}
\newcommand{\eqcolonl}{\ensuremath{\mathrel{=\!\!\mathop{:}}}}
\newcommand{\coloneql}{\ensuremath{\mathrel{\mathop{:} \!\! =}}}
\newcommand{\vc}[1]{% inline column vector
  \left(\begin{smallmatrix}#1\end{smallmatrix}\right)%
}
\newcommand{\vr}[1]{% inline row vector
  \begin{smallmatrix}(\,#1\,)\end{smallmatrix}%
}
\makeatletter
\newcommand*{\defeq}{\ =\mathrel{\rlap{%
                     \raisebox{0.3ex}{$\m@th\cdot$}}%
                     \raisebox{-0.3ex}{$\m@th\cdot$}}%
                     }
\makeatother

\newcommand{\mathcircle}[1]{% inline row vector
 \overset{\circ}{#1}
}
\newcommand{\ulim}{% low limit
 \underline{\lim}
}
\newcommand{\ssi}{% iff
\iff
}
\newcommand{\ps}[2]{
\expval{#1 | #2}
}
\newcommand{\df}[1]{
\mqty{#1}
}
\newcommand{\n}[1]{
\norm{#1}
}
\newcommand{\sys}[1]{
\left\{\smqty{#1}\right.
}


\newcommand{\eqdef}{\ensuremath{\overset{\text{def}}=}}


\def\Circlearrowright{\ensuremath{%
  \rotatebox[origin=c]{230}{$\circlearrowright$}}}

\newcommand\ct[1]{\text{\rmfamily\upshape #1}}
\newcommand\question[1]{ {\color{red} ...!? \small #1}}
\newcommand\caz[1]{\left\{\begin{array} #1 \end{array}\right.}
\newcommand\const{\text{\rmfamily\upshape const}}
\newcommand\toP{ \overset{\pro}{\to}}
\newcommand\toPP{ \overset{\text{PP}}{\to}}
\newcommand{\oeq}{\mathrel{\text{\textcircled{$=$}}}}





\usepackage{xcolor}
% \usepackage[normalem]{ulem}
\usepackage{lipsum}
\makeatletter
% \newcommand\colorwave[1][blue]{\bgroup \markoverwith{\lower3.5\p@\hbox{\sixly \textcolor{#1}{\char58}}}\ULon}
%\font\sixly=lasy6 % does not re-load if already loaded, so no memory problem.

\newmdtheoremenv[
linewidth= 1pt,linecolor= blue,%
leftmargin=20,rightmargin=20,innertopmargin=0pt, innerrightmargin=40,%
tikzsetting = { draw=lightgray, line width = 0.3pt,dashed,%
dash pattern = on 15pt off 3pt},%
splittopskip=\topskip,skipbelow=\baselineskip,%
skipabove=\baselineskip,ntheorem,roundcorner=0pt,
% backgroundcolor=pagebg,font=\color{orange}\sffamily, fontcolor=white
]{examplebox}{Exemple}[section]



\newcommand\R{\mathbb{R}}
\newcommand\Z{\mathbb{Z}}
\newcommand\N{\mathbb{N}}
\newcommand\E{\mathbb{E}}
\newcommand\F{\mathcal{F}}
\newcommand\cH{\mathcal{H}}
\newcommand\V{\mathbb{V}}
\newcommand\dmo{ ^{-1} }
\newcommand\kapa{\kappa}
\newcommand\im{Im}
\newcommand\hs{\mathcal{H}}





\usepackage{soul}

\makeatletter
\newcommand*{\whiten}[1]{\llap{\textcolor{white}{{\the\SOUL@token}}\hspace{#1pt}}}
\DeclareRobustCommand*\myul{%
    \def\SOUL@everyspace{\underline{\space}\kern\z@}%
    \def\SOUL@everytoken{%
     \setbox0=\hbox{\the\SOUL@token}%
     \ifdim\dp0>\z@
        \raisebox{\dp0}{\underline{\phantom{\the\SOUL@token}}}%
        \whiten{1}\whiten{0}%
        \whiten{-1}\whiten{-2}%
        \llap{\the\SOUL@token}%
     \else
        \underline{\the\SOUL@token}%
     \fi}%
\SOUL@}
\makeatother

\newcommand*{\demp}{\fontfamily{lmtt}\selectfont}

\DeclareTextFontCommand{\textdemp}{\demp}

\begin{document}

\ifcomment
Multiline
comment
\fi
\ifcomment
\myul{Typesetting test}
% \color[rgb]{1,1,1}
$∑_i^n≠ 60º±∞π∆¬≈√j∫h≤≥µ$

$\CR \R\pro\ind\pro\gS\pro
\mqty[a&b\\c&d]$
$\pro\mathbb{P}$
$\dd{x}$

  \[
    \alpha(x)=\left\{
                \begin{array}{ll}
                  x\\
                  \frac{1}{1+e^{-kx}}\\
                  \frac{e^x-e^{-x}}{e^x+e^{-x}}
                \end{array}
              \right.
  \]

  $\expval{x}$
  
  $\chi_\rho(ghg\dmo)=\Tr(\rho_{ghg\dmo})=\Tr(\rho_g\circ\rho_h\circ\rho\dmo_g)=\Tr(\rho_h)\overset{\mbox{\scalebox{0.5}{$\Tr(AB)=\Tr(BA)$}}}{=}\chi_\rho(h)$
  	$\mathop{\oplus}_{\substack{x\in X}}$

$\mat(\rho_g)=(a_{ij}(g))_{\scriptsize \substack{1\leq i\leq d \\ 1\leq j\leq d}}$ et $\mat(\rho'_g)=(a'_{ij}(g))_{\scriptsize \substack{1\leq i'\leq d' \\ 1\leq j'\leq d'}}$



\[\int_a^b{\mathbb{R}^2}g(u, v)\dd{P_{XY}}(u, v)=\iint g(u,v) f_{XY}(u, v)\dd \lambda(u) \dd \lambda(v)\]
$$\lim_{x\to\infty} f(x)$$	
$$\iiiint_V \mu(t,u,v,w) \,dt\,du\,dv\,dw$$
$$\sum_{n=1}^{\infty} 2^{-n} = 1$$	
\begin{definition}
	Si $X$ et $Y$ sont 2 v.a. ou definit la \textsc{Covariance} entre $X$ et $Y$ comme
	$\cov(X,Y)\overset{\text{def}}{=}\E\left[(X-\E(X))(Y-\E(Y))\right]=\E(XY)-\E(X)\E(Y)$.
\end{definition}
\fi
\pagebreak

% \tableofcontents

% insert your code here
%% !TEX encoding = UTF-8 Unicode
% !TEX TS-program = xelatex

\documentclass[french]{report}

%\usepackage[utf8]{inputenc}
%\usepackage[T1]{fontenc}
\usepackage{babel}


\newif\ifcomment
%\commenttrue # Show comments

\usepackage{physics}
\usepackage{amssymb}


\usepackage{amsthm}
% \usepackage{thmtools}
\usepackage{mathtools}
\usepackage{amsfonts}

\usepackage{color}

\usepackage{tikz}

\usepackage{geometry}
\geometry{a5paper, margin=0.1in, right=1cm}

\usepackage{dsfont}

\usepackage{graphicx}
\graphicspath{ {images/} }

\usepackage{faktor}

\usepackage{IEEEtrantools}
\usepackage{enumerate}   
\usepackage[PostScript=dvips]{"/Users/aware/Documents/Courses/diagrams"}


\newtheorem{theorem}{Théorème}[section]
\renewcommand{\thetheorem}{\arabic{theorem}}
\newtheorem{lemme}{Lemme}[section]
\renewcommand{\thelemme}{\arabic{lemme}}
\newtheorem{proposition}{Proposition}[section]
\renewcommand{\theproposition}{\arabic{proposition}}
\newtheorem{notations}{Notations}[section]
\newtheorem{problem}{Problème}[section]
\newtheorem{corollary}{Corollaire}[theorem]
\renewcommand{\thecorollary}{\arabic{corollary}}
\newtheorem{property}{Propriété}[section]
\newtheorem{objective}{Objectif}[section]

\theoremstyle{definition}
\newtheorem{definition}{Définition}[section]
\renewcommand{\thedefinition}{\arabic{definition}}
\newtheorem{exercise}{Exercice}[chapter]
\renewcommand{\theexercise}{\arabic{exercise}}
\newtheorem{example}{Exemple}[chapter]
\renewcommand{\theexample}{\arabic{example}}
\newtheorem*{solution}{Solution}
\newtheorem*{application}{Application}
\newtheorem*{notation}{Notation}
\newtheorem*{vocabulary}{Vocabulaire}
\newtheorem*{properties}{Propriétés}



\theoremstyle{remark}
\newtheorem*{remark}{Remarque}
\newtheorem*{rappel}{Rappel}


\usepackage{etoolbox}
\AtBeginEnvironment{exercise}{\small}
\AtBeginEnvironment{example}{\small}

\usepackage{cases}
\usepackage[red]{mypack}

\usepackage[framemethod=TikZ]{mdframed}

\definecolor{bg}{rgb}{0.4,0.25,0.95}
\definecolor{pagebg}{rgb}{0,0,0.5}
\surroundwithmdframed[
   topline=false,
   rightline=false,
   bottomline=false,
   leftmargin=\parindent,
   skipabove=8pt,
   skipbelow=8pt,
   linecolor=blue,
   innerbottommargin=10pt,
   % backgroundcolor=bg,font=\color{orange}\sffamily, fontcolor=white
]{definition}

\usepackage{empheq}
\usepackage[most]{tcolorbox}

\newtcbox{\mymath}[1][]{%
    nobeforeafter, math upper, tcbox raise base,
    enhanced, colframe=blue!30!black,
    colback=red!10, boxrule=1pt,
    #1}

\usepackage{unixode}


\DeclareMathOperator{\ord}{ord}
\DeclareMathOperator{\orb}{orb}
\DeclareMathOperator{\stab}{stab}
\DeclareMathOperator{\Stab}{stab}
\DeclareMathOperator{\ppcm}{ppcm}
\DeclareMathOperator{\conj}{Conj}
\DeclareMathOperator{\End}{End}
\DeclareMathOperator{\rot}{rot}
\DeclareMathOperator{\trs}{trace}
\DeclareMathOperator{\Ind}{Ind}
\DeclareMathOperator{\mat}{Mat}
\DeclareMathOperator{\id}{Id}
\DeclareMathOperator{\vect}{vect}
\DeclareMathOperator{\img}{img}
\DeclareMathOperator{\cov}{Cov}
\DeclareMathOperator{\dist}{dist}
\DeclareMathOperator{\irr}{Irr}
\DeclareMathOperator{\image}{Im}
\DeclareMathOperator{\pd}{\partial}
\DeclareMathOperator{\epi}{epi}
\DeclareMathOperator{\Argmin}{Argmin}
\DeclareMathOperator{\dom}{dom}
\DeclareMathOperator{\proj}{proj}
\DeclareMathOperator{\ctg}{ctg}
\DeclareMathOperator{\supp}{supp}
\DeclareMathOperator{\argmin}{argmin}
\DeclareMathOperator{\mult}{mult}
\DeclareMathOperator{\ch}{ch}
\DeclareMathOperator{\sh}{sh}
\DeclareMathOperator{\rang}{rang}
\DeclareMathOperator{\diam}{diam}
\DeclareMathOperator{\Epigraphe}{Epigraphe}




\usepackage{xcolor}
\everymath{\color{blue}}
%\everymath{\color[rgb]{0,1,1}}
%\pagecolor[rgb]{0,0,0.5}


\newcommand*{\pdtest}[3][]{\ensuremath{\frac{\partial^{#1} #2}{\partial #3}}}

\newcommand*{\deffunc}[6][]{\ensuremath{
\begin{array}{rcl}
#2 : #3 &\rightarrow& #4\\
#5 &\mapsto& #6
\end{array}
}}

\newcommand{\eqcolon}{\mathrel{\resizebox{\widthof{$\mathord{=}$}}{\height}{ $\!\!=\!\!\resizebox{1.2\width}{0.8\height}{\raisebox{0.23ex}{$\mathop{:}$}}\!\!$ }}}
\newcommand{\coloneq}{\mathrel{\resizebox{\widthof{$\mathord{=}$}}{\height}{ $\!\!\resizebox{1.2\width}{0.8\height}{\raisebox{0.23ex}{$\mathop{:}$}}\!\!=\!\!$ }}}
\newcommand{\eqcolonl}{\ensuremath{\mathrel{=\!\!\mathop{:}}}}
\newcommand{\coloneql}{\ensuremath{\mathrel{\mathop{:} \!\! =}}}
\newcommand{\vc}[1]{% inline column vector
  \left(\begin{smallmatrix}#1\end{smallmatrix}\right)%
}
\newcommand{\vr}[1]{% inline row vector
  \begin{smallmatrix}(\,#1\,)\end{smallmatrix}%
}
\makeatletter
\newcommand*{\defeq}{\ =\mathrel{\rlap{%
                     \raisebox{0.3ex}{$\m@th\cdot$}}%
                     \raisebox{-0.3ex}{$\m@th\cdot$}}%
                     }
\makeatother

\newcommand{\mathcircle}[1]{% inline row vector
 \overset{\circ}{#1}
}
\newcommand{\ulim}{% low limit
 \underline{\lim}
}
\newcommand{\ssi}{% iff
\iff
}
\newcommand{\ps}[2]{
\expval{#1 | #2}
}
\newcommand{\df}[1]{
\mqty{#1}
}
\newcommand{\n}[1]{
\norm{#1}
}
\newcommand{\sys}[1]{
\left\{\smqty{#1}\right.
}


\newcommand{\eqdef}{\ensuremath{\overset{\text{def}}=}}


\def\Circlearrowright{\ensuremath{%
  \rotatebox[origin=c]{230}{$\circlearrowright$}}}

\newcommand\ct[1]{\text{\rmfamily\upshape #1}}
\newcommand\question[1]{ {\color{red} ...!? \small #1}}
\newcommand\caz[1]{\left\{\begin{array} #1 \end{array}\right.}
\newcommand\const{\text{\rmfamily\upshape const}}
\newcommand\toP{ \overset{\pro}{\to}}
\newcommand\toPP{ \overset{\text{PP}}{\to}}
\newcommand{\oeq}{\mathrel{\text{\textcircled{$=$}}}}





\usepackage{xcolor}
% \usepackage[normalem]{ulem}
\usepackage{lipsum}
\makeatletter
% \newcommand\colorwave[1][blue]{\bgroup \markoverwith{\lower3.5\p@\hbox{\sixly \textcolor{#1}{\char58}}}\ULon}
%\font\sixly=lasy6 % does not re-load if already loaded, so no memory problem.

\newmdtheoremenv[
linewidth= 1pt,linecolor= blue,%
leftmargin=20,rightmargin=20,innertopmargin=0pt, innerrightmargin=40,%
tikzsetting = { draw=lightgray, line width = 0.3pt,dashed,%
dash pattern = on 15pt off 3pt},%
splittopskip=\topskip,skipbelow=\baselineskip,%
skipabove=\baselineskip,ntheorem,roundcorner=0pt,
% backgroundcolor=pagebg,font=\color{orange}\sffamily, fontcolor=white
]{examplebox}{Exemple}[section]



\newcommand\R{\mathbb{R}}
\newcommand\Z{\mathbb{Z}}
\newcommand\N{\mathbb{N}}
\newcommand\E{\mathbb{E}}
\newcommand\F{\mathcal{F}}
\newcommand\cH{\mathcal{H}}
\newcommand\V{\mathbb{V}}
\newcommand\dmo{ ^{-1} }
\newcommand\kapa{\kappa}
\newcommand\im{Im}
\newcommand\hs{\mathcal{H}}





\usepackage{soul}

\makeatletter
\newcommand*{\whiten}[1]{\llap{\textcolor{white}{{\the\SOUL@token}}\hspace{#1pt}}}
\DeclareRobustCommand*\myul{%
    \def\SOUL@everyspace{\underline{\space}\kern\z@}%
    \def\SOUL@everytoken{%
     \setbox0=\hbox{\the\SOUL@token}%
     \ifdim\dp0>\z@
        \raisebox{\dp0}{\underline{\phantom{\the\SOUL@token}}}%
        \whiten{1}\whiten{0}%
        \whiten{-1}\whiten{-2}%
        \llap{\the\SOUL@token}%
     \else
        \underline{\the\SOUL@token}%
     \fi}%
\SOUL@}
\makeatother

\newcommand*{\demp}{\fontfamily{lmtt}\selectfont}

\DeclareTextFontCommand{\textdemp}{\demp}

\begin{document}

\ifcomment
Multiline
comment
\fi
\ifcomment
\myul{Typesetting test}
% \color[rgb]{1,1,1}
$∑_i^n≠ 60º±∞π∆¬≈√j∫h≤≥µ$

$\CR \R\pro\ind\pro\gS\pro
\mqty[a&b\\c&d]$
$\pro\mathbb{P}$
$\dd{x}$

  \[
    \alpha(x)=\left\{
                \begin{array}{ll}
                  x\\
                  \frac{1}{1+e^{-kx}}\\
                  \frac{e^x-e^{-x}}{e^x+e^{-x}}
                \end{array}
              \right.
  \]

  $\expval{x}$
  
  $\chi_\rho(ghg\dmo)=\Tr(\rho_{ghg\dmo})=\Tr(\rho_g\circ\rho_h\circ\rho\dmo_g)=\Tr(\rho_h)\overset{\mbox{\scalebox{0.5}{$\Tr(AB)=\Tr(BA)$}}}{=}\chi_\rho(h)$
  	$\mathop{\oplus}_{\substack{x\in X}}$

$\mat(\rho_g)=(a_{ij}(g))_{\scriptsize \substack{1\leq i\leq d \\ 1\leq j\leq d}}$ et $\mat(\rho'_g)=(a'_{ij}(g))_{\scriptsize \substack{1\leq i'\leq d' \\ 1\leq j'\leq d'}}$



\[\int_a^b{\mathbb{R}^2}g(u, v)\dd{P_{XY}}(u, v)=\iint g(u,v) f_{XY}(u, v)\dd \lambda(u) \dd \lambda(v)\]
$$\lim_{x\to\infty} f(x)$$	
$$\iiiint_V \mu(t,u,v,w) \,dt\,du\,dv\,dw$$
$$\sum_{n=1}^{\infty} 2^{-n} = 1$$	
\begin{definition}
	Si $X$ et $Y$ sont 2 v.a. ou definit la \textsc{Covariance} entre $X$ et $Y$ comme
	$\cov(X,Y)\overset{\text{def}}{=}\E\left[(X-\E(X))(Y-\E(Y))\right]=\E(XY)-\E(X)\E(Y)$.
\end{definition}
\fi
\pagebreak

% \tableofcontents

% insert your code here
%\input{./algebra/main.tex}
%\input{./geometrie-differentielle/main.tex}
%\input{./probabilite/main.tex}
%\input{./analyse-fonctionnelle/main.tex}
% \input{./Analyse-convexe-et-dualite-en-optimisation/main.tex}
%\input{./tikz/main.tex}
%\input{./Theorie-du-distributions/main.tex}
%\input{./optimisation/mine.tex}
 \input{./modelisation/main.tex}

% yves.aubry@univ-tln.fr : algebra

\end{document}

%% !TEX encoding = UTF-8 Unicode
% !TEX TS-program = xelatex

\documentclass[french]{report}

%\usepackage[utf8]{inputenc}
%\usepackage[T1]{fontenc}
\usepackage{babel}


\newif\ifcomment
%\commenttrue # Show comments

\usepackage{physics}
\usepackage{amssymb}


\usepackage{amsthm}
% \usepackage{thmtools}
\usepackage{mathtools}
\usepackage{amsfonts}

\usepackage{color}

\usepackage{tikz}

\usepackage{geometry}
\geometry{a5paper, margin=0.1in, right=1cm}

\usepackage{dsfont}

\usepackage{graphicx}
\graphicspath{ {images/} }

\usepackage{faktor}

\usepackage{IEEEtrantools}
\usepackage{enumerate}   
\usepackage[PostScript=dvips]{"/Users/aware/Documents/Courses/diagrams"}


\newtheorem{theorem}{Théorème}[section]
\renewcommand{\thetheorem}{\arabic{theorem}}
\newtheorem{lemme}{Lemme}[section]
\renewcommand{\thelemme}{\arabic{lemme}}
\newtheorem{proposition}{Proposition}[section]
\renewcommand{\theproposition}{\arabic{proposition}}
\newtheorem{notations}{Notations}[section]
\newtheorem{problem}{Problème}[section]
\newtheorem{corollary}{Corollaire}[theorem]
\renewcommand{\thecorollary}{\arabic{corollary}}
\newtheorem{property}{Propriété}[section]
\newtheorem{objective}{Objectif}[section]

\theoremstyle{definition}
\newtheorem{definition}{Définition}[section]
\renewcommand{\thedefinition}{\arabic{definition}}
\newtheorem{exercise}{Exercice}[chapter]
\renewcommand{\theexercise}{\arabic{exercise}}
\newtheorem{example}{Exemple}[chapter]
\renewcommand{\theexample}{\arabic{example}}
\newtheorem*{solution}{Solution}
\newtheorem*{application}{Application}
\newtheorem*{notation}{Notation}
\newtheorem*{vocabulary}{Vocabulaire}
\newtheorem*{properties}{Propriétés}



\theoremstyle{remark}
\newtheorem*{remark}{Remarque}
\newtheorem*{rappel}{Rappel}


\usepackage{etoolbox}
\AtBeginEnvironment{exercise}{\small}
\AtBeginEnvironment{example}{\small}

\usepackage{cases}
\usepackage[red]{mypack}

\usepackage[framemethod=TikZ]{mdframed}

\definecolor{bg}{rgb}{0.4,0.25,0.95}
\definecolor{pagebg}{rgb}{0,0,0.5}
\surroundwithmdframed[
   topline=false,
   rightline=false,
   bottomline=false,
   leftmargin=\parindent,
   skipabove=8pt,
   skipbelow=8pt,
   linecolor=blue,
   innerbottommargin=10pt,
   % backgroundcolor=bg,font=\color{orange}\sffamily, fontcolor=white
]{definition}

\usepackage{empheq}
\usepackage[most]{tcolorbox}

\newtcbox{\mymath}[1][]{%
    nobeforeafter, math upper, tcbox raise base,
    enhanced, colframe=blue!30!black,
    colback=red!10, boxrule=1pt,
    #1}

\usepackage{unixode}


\DeclareMathOperator{\ord}{ord}
\DeclareMathOperator{\orb}{orb}
\DeclareMathOperator{\stab}{stab}
\DeclareMathOperator{\Stab}{stab}
\DeclareMathOperator{\ppcm}{ppcm}
\DeclareMathOperator{\conj}{Conj}
\DeclareMathOperator{\End}{End}
\DeclareMathOperator{\rot}{rot}
\DeclareMathOperator{\trs}{trace}
\DeclareMathOperator{\Ind}{Ind}
\DeclareMathOperator{\mat}{Mat}
\DeclareMathOperator{\id}{Id}
\DeclareMathOperator{\vect}{vect}
\DeclareMathOperator{\img}{img}
\DeclareMathOperator{\cov}{Cov}
\DeclareMathOperator{\dist}{dist}
\DeclareMathOperator{\irr}{Irr}
\DeclareMathOperator{\image}{Im}
\DeclareMathOperator{\pd}{\partial}
\DeclareMathOperator{\epi}{epi}
\DeclareMathOperator{\Argmin}{Argmin}
\DeclareMathOperator{\dom}{dom}
\DeclareMathOperator{\proj}{proj}
\DeclareMathOperator{\ctg}{ctg}
\DeclareMathOperator{\supp}{supp}
\DeclareMathOperator{\argmin}{argmin}
\DeclareMathOperator{\mult}{mult}
\DeclareMathOperator{\ch}{ch}
\DeclareMathOperator{\sh}{sh}
\DeclareMathOperator{\rang}{rang}
\DeclareMathOperator{\diam}{diam}
\DeclareMathOperator{\Epigraphe}{Epigraphe}




\usepackage{xcolor}
\everymath{\color{blue}}
%\everymath{\color[rgb]{0,1,1}}
%\pagecolor[rgb]{0,0,0.5}


\newcommand*{\pdtest}[3][]{\ensuremath{\frac{\partial^{#1} #2}{\partial #3}}}

\newcommand*{\deffunc}[6][]{\ensuremath{
\begin{array}{rcl}
#2 : #3 &\rightarrow& #4\\
#5 &\mapsto& #6
\end{array}
}}

\newcommand{\eqcolon}{\mathrel{\resizebox{\widthof{$\mathord{=}$}}{\height}{ $\!\!=\!\!\resizebox{1.2\width}{0.8\height}{\raisebox{0.23ex}{$\mathop{:}$}}\!\!$ }}}
\newcommand{\coloneq}{\mathrel{\resizebox{\widthof{$\mathord{=}$}}{\height}{ $\!\!\resizebox{1.2\width}{0.8\height}{\raisebox{0.23ex}{$\mathop{:}$}}\!\!=\!\!$ }}}
\newcommand{\eqcolonl}{\ensuremath{\mathrel{=\!\!\mathop{:}}}}
\newcommand{\coloneql}{\ensuremath{\mathrel{\mathop{:} \!\! =}}}
\newcommand{\vc}[1]{% inline column vector
  \left(\begin{smallmatrix}#1\end{smallmatrix}\right)%
}
\newcommand{\vr}[1]{% inline row vector
  \begin{smallmatrix}(\,#1\,)\end{smallmatrix}%
}
\makeatletter
\newcommand*{\defeq}{\ =\mathrel{\rlap{%
                     \raisebox{0.3ex}{$\m@th\cdot$}}%
                     \raisebox{-0.3ex}{$\m@th\cdot$}}%
                     }
\makeatother

\newcommand{\mathcircle}[1]{% inline row vector
 \overset{\circ}{#1}
}
\newcommand{\ulim}{% low limit
 \underline{\lim}
}
\newcommand{\ssi}{% iff
\iff
}
\newcommand{\ps}[2]{
\expval{#1 | #2}
}
\newcommand{\df}[1]{
\mqty{#1}
}
\newcommand{\n}[1]{
\norm{#1}
}
\newcommand{\sys}[1]{
\left\{\smqty{#1}\right.
}


\newcommand{\eqdef}{\ensuremath{\overset{\text{def}}=}}


\def\Circlearrowright{\ensuremath{%
  \rotatebox[origin=c]{230}{$\circlearrowright$}}}

\newcommand\ct[1]{\text{\rmfamily\upshape #1}}
\newcommand\question[1]{ {\color{red} ...!? \small #1}}
\newcommand\caz[1]{\left\{\begin{array} #1 \end{array}\right.}
\newcommand\const{\text{\rmfamily\upshape const}}
\newcommand\toP{ \overset{\pro}{\to}}
\newcommand\toPP{ \overset{\text{PP}}{\to}}
\newcommand{\oeq}{\mathrel{\text{\textcircled{$=$}}}}





\usepackage{xcolor}
% \usepackage[normalem]{ulem}
\usepackage{lipsum}
\makeatletter
% \newcommand\colorwave[1][blue]{\bgroup \markoverwith{\lower3.5\p@\hbox{\sixly \textcolor{#1}{\char58}}}\ULon}
%\font\sixly=lasy6 % does not re-load if already loaded, so no memory problem.

\newmdtheoremenv[
linewidth= 1pt,linecolor= blue,%
leftmargin=20,rightmargin=20,innertopmargin=0pt, innerrightmargin=40,%
tikzsetting = { draw=lightgray, line width = 0.3pt,dashed,%
dash pattern = on 15pt off 3pt},%
splittopskip=\topskip,skipbelow=\baselineskip,%
skipabove=\baselineskip,ntheorem,roundcorner=0pt,
% backgroundcolor=pagebg,font=\color{orange}\sffamily, fontcolor=white
]{examplebox}{Exemple}[section]



\newcommand\R{\mathbb{R}}
\newcommand\Z{\mathbb{Z}}
\newcommand\N{\mathbb{N}}
\newcommand\E{\mathbb{E}}
\newcommand\F{\mathcal{F}}
\newcommand\cH{\mathcal{H}}
\newcommand\V{\mathbb{V}}
\newcommand\dmo{ ^{-1} }
\newcommand\kapa{\kappa}
\newcommand\im{Im}
\newcommand\hs{\mathcal{H}}





\usepackage{soul}

\makeatletter
\newcommand*{\whiten}[1]{\llap{\textcolor{white}{{\the\SOUL@token}}\hspace{#1pt}}}
\DeclareRobustCommand*\myul{%
    \def\SOUL@everyspace{\underline{\space}\kern\z@}%
    \def\SOUL@everytoken{%
     \setbox0=\hbox{\the\SOUL@token}%
     \ifdim\dp0>\z@
        \raisebox{\dp0}{\underline{\phantom{\the\SOUL@token}}}%
        \whiten{1}\whiten{0}%
        \whiten{-1}\whiten{-2}%
        \llap{\the\SOUL@token}%
     \else
        \underline{\the\SOUL@token}%
     \fi}%
\SOUL@}
\makeatother

\newcommand*{\demp}{\fontfamily{lmtt}\selectfont}

\DeclareTextFontCommand{\textdemp}{\demp}

\begin{document}

\ifcomment
Multiline
comment
\fi
\ifcomment
\myul{Typesetting test}
% \color[rgb]{1,1,1}
$∑_i^n≠ 60º±∞π∆¬≈√j∫h≤≥µ$

$\CR \R\pro\ind\pro\gS\pro
\mqty[a&b\\c&d]$
$\pro\mathbb{P}$
$\dd{x}$

  \[
    \alpha(x)=\left\{
                \begin{array}{ll}
                  x\\
                  \frac{1}{1+e^{-kx}}\\
                  \frac{e^x-e^{-x}}{e^x+e^{-x}}
                \end{array}
              \right.
  \]

  $\expval{x}$
  
  $\chi_\rho(ghg\dmo)=\Tr(\rho_{ghg\dmo})=\Tr(\rho_g\circ\rho_h\circ\rho\dmo_g)=\Tr(\rho_h)\overset{\mbox{\scalebox{0.5}{$\Tr(AB)=\Tr(BA)$}}}{=}\chi_\rho(h)$
  	$\mathop{\oplus}_{\substack{x\in X}}$

$\mat(\rho_g)=(a_{ij}(g))_{\scriptsize \substack{1\leq i\leq d \\ 1\leq j\leq d}}$ et $\mat(\rho'_g)=(a'_{ij}(g))_{\scriptsize \substack{1\leq i'\leq d' \\ 1\leq j'\leq d'}}$



\[\int_a^b{\mathbb{R}^2}g(u, v)\dd{P_{XY}}(u, v)=\iint g(u,v) f_{XY}(u, v)\dd \lambda(u) \dd \lambda(v)\]
$$\lim_{x\to\infty} f(x)$$	
$$\iiiint_V \mu(t,u,v,w) \,dt\,du\,dv\,dw$$
$$\sum_{n=1}^{\infty} 2^{-n} = 1$$	
\begin{definition}
	Si $X$ et $Y$ sont 2 v.a. ou definit la \textsc{Covariance} entre $X$ et $Y$ comme
	$\cov(X,Y)\overset{\text{def}}{=}\E\left[(X-\E(X))(Y-\E(Y))\right]=\E(XY)-\E(X)\E(Y)$.
\end{definition}
\fi
\pagebreak

% \tableofcontents

% insert your code here
%\input{./algebra/main.tex}
%\input{./geometrie-differentielle/main.tex}
%\input{./probabilite/main.tex}
%\input{./analyse-fonctionnelle/main.tex}
% \input{./Analyse-convexe-et-dualite-en-optimisation/main.tex}
%\input{./tikz/main.tex}
%\input{./Theorie-du-distributions/main.tex}
%\input{./optimisation/mine.tex}
 \input{./modelisation/main.tex}

% yves.aubry@univ-tln.fr : algebra

\end{document}

%% !TEX encoding = UTF-8 Unicode
% !TEX TS-program = xelatex

\documentclass[french]{report}

%\usepackage[utf8]{inputenc}
%\usepackage[T1]{fontenc}
\usepackage{babel}


\newif\ifcomment
%\commenttrue # Show comments

\usepackage{physics}
\usepackage{amssymb}


\usepackage{amsthm}
% \usepackage{thmtools}
\usepackage{mathtools}
\usepackage{amsfonts}

\usepackage{color}

\usepackage{tikz}

\usepackage{geometry}
\geometry{a5paper, margin=0.1in, right=1cm}

\usepackage{dsfont}

\usepackage{graphicx}
\graphicspath{ {images/} }

\usepackage{faktor}

\usepackage{IEEEtrantools}
\usepackage{enumerate}   
\usepackage[PostScript=dvips]{"/Users/aware/Documents/Courses/diagrams"}


\newtheorem{theorem}{Théorème}[section]
\renewcommand{\thetheorem}{\arabic{theorem}}
\newtheorem{lemme}{Lemme}[section]
\renewcommand{\thelemme}{\arabic{lemme}}
\newtheorem{proposition}{Proposition}[section]
\renewcommand{\theproposition}{\arabic{proposition}}
\newtheorem{notations}{Notations}[section]
\newtheorem{problem}{Problème}[section]
\newtheorem{corollary}{Corollaire}[theorem]
\renewcommand{\thecorollary}{\arabic{corollary}}
\newtheorem{property}{Propriété}[section]
\newtheorem{objective}{Objectif}[section]

\theoremstyle{definition}
\newtheorem{definition}{Définition}[section]
\renewcommand{\thedefinition}{\arabic{definition}}
\newtheorem{exercise}{Exercice}[chapter]
\renewcommand{\theexercise}{\arabic{exercise}}
\newtheorem{example}{Exemple}[chapter]
\renewcommand{\theexample}{\arabic{example}}
\newtheorem*{solution}{Solution}
\newtheorem*{application}{Application}
\newtheorem*{notation}{Notation}
\newtheorem*{vocabulary}{Vocabulaire}
\newtheorem*{properties}{Propriétés}



\theoremstyle{remark}
\newtheorem*{remark}{Remarque}
\newtheorem*{rappel}{Rappel}


\usepackage{etoolbox}
\AtBeginEnvironment{exercise}{\small}
\AtBeginEnvironment{example}{\small}

\usepackage{cases}
\usepackage[red]{mypack}

\usepackage[framemethod=TikZ]{mdframed}

\definecolor{bg}{rgb}{0.4,0.25,0.95}
\definecolor{pagebg}{rgb}{0,0,0.5}
\surroundwithmdframed[
   topline=false,
   rightline=false,
   bottomline=false,
   leftmargin=\parindent,
   skipabove=8pt,
   skipbelow=8pt,
   linecolor=blue,
   innerbottommargin=10pt,
   % backgroundcolor=bg,font=\color{orange}\sffamily, fontcolor=white
]{definition}

\usepackage{empheq}
\usepackage[most]{tcolorbox}

\newtcbox{\mymath}[1][]{%
    nobeforeafter, math upper, tcbox raise base,
    enhanced, colframe=blue!30!black,
    colback=red!10, boxrule=1pt,
    #1}

\usepackage{unixode}


\DeclareMathOperator{\ord}{ord}
\DeclareMathOperator{\orb}{orb}
\DeclareMathOperator{\stab}{stab}
\DeclareMathOperator{\Stab}{stab}
\DeclareMathOperator{\ppcm}{ppcm}
\DeclareMathOperator{\conj}{Conj}
\DeclareMathOperator{\End}{End}
\DeclareMathOperator{\rot}{rot}
\DeclareMathOperator{\trs}{trace}
\DeclareMathOperator{\Ind}{Ind}
\DeclareMathOperator{\mat}{Mat}
\DeclareMathOperator{\id}{Id}
\DeclareMathOperator{\vect}{vect}
\DeclareMathOperator{\img}{img}
\DeclareMathOperator{\cov}{Cov}
\DeclareMathOperator{\dist}{dist}
\DeclareMathOperator{\irr}{Irr}
\DeclareMathOperator{\image}{Im}
\DeclareMathOperator{\pd}{\partial}
\DeclareMathOperator{\epi}{epi}
\DeclareMathOperator{\Argmin}{Argmin}
\DeclareMathOperator{\dom}{dom}
\DeclareMathOperator{\proj}{proj}
\DeclareMathOperator{\ctg}{ctg}
\DeclareMathOperator{\supp}{supp}
\DeclareMathOperator{\argmin}{argmin}
\DeclareMathOperator{\mult}{mult}
\DeclareMathOperator{\ch}{ch}
\DeclareMathOperator{\sh}{sh}
\DeclareMathOperator{\rang}{rang}
\DeclareMathOperator{\diam}{diam}
\DeclareMathOperator{\Epigraphe}{Epigraphe}




\usepackage{xcolor}
\everymath{\color{blue}}
%\everymath{\color[rgb]{0,1,1}}
%\pagecolor[rgb]{0,0,0.5}


\newcommand*{\pdtest}[3][]{\ensuremath{\frac{\partial^{#1} #2}{\partial #3}}}

\newcommand*{\deffunc}[6][]{\ensuremath{
\begin{array}{rcl}
#2 : #3 &\rightarrow& #4\\
#5 &\mapsto& #6
\end{array}
}}

\newcommand{\eqcolon}{\mathrel{\resizebox{\widthof{$\mathord{=}$}}{\height}{ $\!\!=\!\!\resizebox{1.2\width}{0.8\height}{\raisebox{0.23ex}{$\mathop{:}$}}\!\!$ }}}
\newcommand{\coloneq}{\mathrel{\resizebox{\widthof{$\mathord{=}$}}{\height}{ $\!\!\resizebox{1.2\width}{0.8\height}{\raisebox{0.23ex}{$\mathop{:}$}}\!\!=\!\!$ }}}
\newcommand{\eqcolonl}{\ensuremath{\mathrel{=\!\!\mathop{:}}}}
\newcommand{\coloneql}{\ensuremath{\mathrel{\mathop{:} \!\! =}}}
\newcommand{\vc}[1]{% inline column vector
  \left(\begin{smallmatrix}#1\end{smallmatrix}\right)%
}
\newcommand{\vr}[1]{% inline row vector
  \begin{smallmatrix}(\,#1\,)\end{smallmatrix}%
}
\makeatletter
\newcommand*{\defeq}{\ =\mathrel{\rlap{%
                     \raisebox{0.3ex}{$\m@th\cdot$}}%
                     \raisebox{-0.3ex}{$\m@th\cdot$}}%
                     }
\makeatother

\newcommand{\mathcircle}[1]{% inline row vector
 \overset{\circ}{#1}
}
\newcommand{\ulim}{% low limit
 \underline{\lim}
}
\newcommand{\ssi}{% iff
\iff
}
\newcommand{\ps}[2]{
\expval{#1 | #2}
}
\newcommand{\df}[1]{
\mqty{#1}
}
\newcommand{\n}[1]{
\norm{#1}
}
\newcommand{\sys}[1]{
\left\{\smqty{#1}\right.
}


\newcommand{\eqdef}{\ensuremath{\overset{\text{def}}=}}


\def\Circlearrowright{\ensuremath{%
  \rotatebox[origin=c]{230}{$\circlearrowright$}}}

\newcommand\ct[1]{\text{\rmfamily\upshape #1}}
\newcommand\question[1]{ {\color{red} ...!? \small #1}}
\newcommand\caz[1]{\left\{\begin{array} #1 \end{array}\right.}
\newcommand\const{\text{\rmfamily\upshape const}}
\newcommand\toP{ \overset{\pro}{\to}}
\newcommand\toPP{ \overset{\text{PP}}{\to}}
\newcommand{\oeq}{\mathrel{\text{\textcircled{$=$}}}}





\usepackage{xcolor}
% \usepackage[normalem]{ulem}
\usepackage{lipsum}
\makeatletter
% \newcommand\colorwave[1][blue]{\bgroup \markoverwith{\lower3.5\p@\hbox{\sixly \textcolor{#1}{\char58}}}\ULon}
%\font\sixly=lasy6 % does not re-load if already loaded, so no memory problem.

\newmdtheoremenv[
linewidth= 1pt,linecolor= blue,%
leftmargin=20,rightmargin=20,innertopmargin=0pt, innerrightmargin=40,%
tikzsetting = { draw=lightgray, line width = 0.3pt,dashed,%
dash pattern = on 15pt off 3pt},%
splittopskip=\topskip,skipbelow=\baselineskip,%
skipabove=\baselineskip,ntheorem,roundcorner=0pt,
% backgroundcolor=pagebg,font=\color{orange}\sffamily, fontcolor=white
]{examplebox}{Exemple}[section]



\newcommand\R{\mathbb{R}}
\newcommand\Z{\mathbb{Z}}
\newcommand\N{\mathbb{N}}
\newcommand\E{\mathbb{E}}
\newcommand\F{\mathcal{F}}
\newcommand\cH{\mathcal{H}}
\newcommand\V{\mathbb{V}}
\newcommand\dmo{ ^{-1} }
\newcommand\kapa{\kappa}
\newcommand\im{Im}
\newcommand\hs{\mathcal{H}}





\usepackage{soul}

\makeatletter
\newcommand*{\whiten}[1]{\llap{\textcolor{white}{{\the\SOUL@token}}\hspace{#1pt}}}
\DeclareRobustCommand*\myul{%
    \def\SOUL@everyspace{\underline{\space}\kern\z@}%
    \def\SOUL@everytoken{%
     \setbox0=\hbox{\the\SOUL@token}%
     \ifdim\dp0>\z@
        \raisebox{\dp0}{\underline{\phantom{\the\SOUL@token}}}%
        \whiten{1}\whiten{0}%
        \whiten{-1}\whiten{-2}%
        \llap{\the\SOUL@token}%
     \else
        \underline{\the\SOUL@token}%
     \fi}%
\SOUL@}
\makeatother

\newcommand*{\demp}{\fontfamily{lmtt}\selectfont}

\DeclareTextFontCommand{\textdemp}{\demp}

\begin{document}

\ifcomment
Multiline
comment
\fi
\ifcomment
\myul{Typesetting test}
% \color[rgb]{1,1,1}
$∑_i^n≠ 60º±∞π∆¬≈√j∫h≤≥µ$

$\CR \R\pro\ind\pro\gS\pro
\mqty[a&b\\c&d]$
$\pro\mathbb{P}$
$\dd{x}$

  \[
    \alpha(x)=\left\{
                \begin{array}{ll}
                  x\\
                  \frac{1}{1+e^{-kx}}\\
                  \frac{e^x-e^{-x}}{e^x+e^{-x}}
                \end{array}
              \right.
  \]

  $\expval{x}$
  
  $\chi_\rho(ghg\dmo)=\Tr(\rho_{ghg\dmo})=\Tr(\rho_g\circ\rho_h\circ\rho\dmo_g)=\Tr(\rho_h)\overset{\mbox{\scalebox{0.5}{$\Tr(AB)=\Tr(BA)$}}}{=}\chi_\rho(h)$
  	$\mathop{\oplus}_{\substack{x\in X}}$

$\mat(\rho_g)=(a_{ij}(g))_{\scriptsize \substack{1\leq i\leq d \\ 1\leq j\leq d}}$ et $\mat(\rho'_g)=(a'_{ij}(g))_{\scriptsize \substack{1\leq i'\leq d' \\ 1\leq j'\leq d'}}$



\[\int_a^b{\mathbb{R}^2}g(u, v)\dd{P_{XY}}(u, v)=\iint g(u,v) f_{XY}(u, v)\dd \lambda(u) \dd \lambda(v)\]
$$\lim_{x\to\infty} f(x)$$	
$$\iiiint_V \mu(t,u,v,w) \,dt\,du\,dv\,dw$$
$$\sum_{n=1}^{\infty} 2^{-n} = 1$$	
\begin{definition}
	Si $X$ et $Y$ sont 2 v.a. ou definit la \textsc{Covariance} entre $X$ et $Y$ comme
	$\cov(X,Y)\overset{\text{def}}{=}\E\left[(X-\E(X))(Y-\E(Y))\right]=\E(XY)-\E(X)\E(Y)$.
\end{definition}
\fi
\pagebreak

% \tableofcontents

% insert your code here
%\input{./algebra/main.tex}
%\input{./geometrie-differentielle/main.tex}
%\input{./probabilite/main.tex}
%\input{./analyse-fonctionnelle/main.tex}
% \input{./Analyse-convexe-et-dualite-en-optimisation/main.tex}
%\input{./tikz/main.tex}
%\input{./Theorie-du-distributions/main.tex}
%\input{./optimisation/mine.tex}
 \input{./modelisation/main.tex}

% yves.aubry@univ-tln.fr : algebra

\end{document}

%% !TEX encoding = UTF-8 Unicode
% !TEX TS-program = xelatex

\documentclass[french]{report}

%\usepackage[utf8]{inputenc}
%\usepackage[T1]{fontenc}
\usepackage{babel}


\newif\ifcomment
%\commenttrue # Show comments

\usepackage{physics}
\usepackage{amssymb}


\usepackage{amsthm}
% \usepackage{thmtools}
\usepackage{mathtools}
\usepackage{amsfonts}

\usepackage{color}

\usepackage{tikz}

\usepackage{geometry}
\geometry{a5paper, margin=0.1in, right=1cm}

\usepackage{dsfont}

\usepackage{graphicx}
\graphicspath{ {images/} }

\usepackage{faktor}

\usepackage{IEEEtrantools}
\usepackage{enumerate}   
\usepackage[PostScript=dvips]{"/Users/aware/Documents/Courses/diagrams"}


\newtheorem{theorem}{Théorème}[section]
\renewcommand{\thetheorem}{\arabic{theorem}}
\newtheorem{lemme}{Lemme}[section]
\renewcommand{\thelemme}{\arabic{lemme}}
\newtheorem{proposition}{Proposition}[section]
\renewcommand{\theproposition}{\arabic{proposition}}
\newtheorem{notations}{Notations}[section]
\newtheorem{problem}{Problème}[section]
\newtheorem{corollary}{Corollaire}[theorem]
\renewcommand{\thecorollary}{\arabic{corollary}}
\newtheorem{property}{Propriété}[section]
\newtheorem{objective}{Objectif}[section]

\theoremstyle{definition}
\newtheorem{definition}{Définition}[section]
\renewcommand{\thedefinition}{\arabic{definition}}
\newtheorem{exercise}{Exercice}[chapter]
\renewcommand{\theexercise}{\arabic{exercise}}
\newtheorem{example}{Exemple}[chapter]
\renewcommand{\theexample}{\arabic{example}}
\newtheorem*{solution}{Solution}
\newtheorem*{application}{Application}
\newtheorem*{notation}{Notation}
\newtheorem*{vocabulary}{Vocabulaire}
\newtheorem*{properties}{Propriétés}



\theoremstyle{remark}
\newtheorem*{remark}{Remarque}
\newtheorem*{rappel}{Rappel}


\usepackage{etoolbox}
\AtBeginEnvironment{exercise}{\small}
\AtBeginEnvironment{example}{\small}

\usepackage{cases}
\usepackage[red]{mypack}

\usepackage[framemethod=TikZ]{mdframed}

\definecolor{bg}{rgb}{0.4,0.25,0.95}
\definecolor{pagebg}{rgb}{0,0,0.5}
\surroundwithmdframed[
   topline=false,
   rightline=false,
   bottomline=false,
   leftmargin=\parindent,
   skipabove=8pt,
   skipbelow=8pt,
   linecolor=blue,
   innerbottommargin=10pt,
   % backgroundcolor=bg,font=\color{orange}\sffamily, fontcolor=white
]{definition}

\usepackage{empheq}
\usepackage[most]{tcolorbox}

\newtcbox{\mymath}[1][]{%
    nobeforeafter, math upper, tcbox raise base,
    enhanced, colframe=blue!30!black,
    colback=red!10, boxrule=1pt,
    #1}

\usepackage{unixode}


\DeclareMathOperator{\ord}{ord}
\DeclareMathOperator{\orb}{orb}
\DeclareMathOperator{\stab}{stab}
\DeclareMathOperator{\Stab}{stab}
\DeclareMathOperator{\ppcm}{ppcm}
\DeclareMathOperator{\conj}{Conj}
\DeclareMathOperator{\End}{End}
\DeclareMathOperator{\rot}{rot}
\DeclareMathOperator{\trs}{trace}
\DeclareMathOperator{\Ind}{Ind}
\DeclareMathOperator{\mat}{Mat}
\DeclareMathOperator{\id}{Id}
\DeclareMathOperator{\vect}{vect}
\DeclareMathOperator{\img}{img}
\DeclareMathOperator{\cov}{Cov}
\DeclareMathOperator{\dist}{dist}
\DeclareMathOperator{\irr}{Irr}
\DeclareMathOperator{\image}{Im}
\DeclareMathOperator{\pd}{\partial}
\DeclareMathOperator{\epi}{epi}
\DeclareMathOperator{\Argmin}{Argmin}
\DeclareMathOperator{\dom}{dom}
\DeclareMathOperator{\proj}{proj}
\DeclareMathOperator{\ctg}{ctg}
\DeclareMathOperator{\supp}{supp}
\DeclareMathOperator{\argmin}{argmin}
\DeclareMathOperator{\mult}{mult}
\DeclareMathOperator{\ch}{ch}
\DeclareMathOperator{\sh}{sh}
\DeclareMathOperator{\rang}{rang}
\DeclareMathOperator{\diam}{diam}
\DeclareMathOperator{\Epigraphe}{Epigraphe}




\usepackage{xcolor}
\everymath{\color{blue}}
%\everymath{\color[rgb]{0,1,1}}
%\pagecolor[rgb]{0,0,0.5}


\newcommand*{\pdtest}[3][]{\ensuremath{\frac{\partial^{#1} #2}{\partial #3}}}

\newcommand*{\deffunc}[6][]{\ensuremath{
\begin{array}{rcl}
#2 : #3 &\rightarrow& #4\\
#5 &\mapsto& #6
\end{array}
}}

\newcommand{\eqcolon}{\mathrel{\resizebox{\widthof{$\mathord{=}$}}{\height}{ $\!\!=\!\!\resizebox{1.2\width}{0.8\height}{\raisebox{0.23ex}{$\mathop{:}$}}\!\!$ }}}
\newcommand{\coloneq}{\mathrel{\resizebox{\widthof{$\mathord{=}$}}{\height}{ $\!\!\resizebox{1.2\width}{0.8\height}{\raisebox{0.23ex}{$\mathop{:}$}}\!\!=\!\!$ }}}
\newcommand{\eqcolonl}{\ensuremath{\mathrel{=\!\!\mathop{:}}}}
\newcommand{\coloneql}{\ensuremath{\mathrel{\mathop{:} \!\! =}}}
\newcommand{\vc}[1]{% inline column vector
  \left(\begin{smallmatrix}#1\end{smallmatrix}\right)%
}
\newcommand{\vr}[1]{% inline row vector
  \begin{smallmatrix}(\,#1\,)\end{smallmatrix}%
}
\makeatletter
\newcommand*{\defeq}{\ =\mathrel{\rlap{%
                     \raisebox{0.3ex}{$\m@th\cdot$}}%
                     \raisebox{-0.3ex}{$\m@th\cdot$}}%
                     }
\makeatother

\newcommand{\mathcircle}[1]{% inline row vector
 \overset{\circ}{#1}
}
\newcommand{\ulim}{% low limit
 \underline{\lim}
}
\newcommand{\ssi}{% iff
\iff
}
\newcommand{\ps}[2]{
\expval{#1 | #2}
}
\newcommand{\df}[1]{
\mqty{#1}
}
\newcommand{\n}[1]{
\norm{#1}
}
\newcommand{\sys}[1]{
\left\{\smqty{#1}\right.
}


\newcommand{\eqdef}{\ensuremath{\overset{\text{def}}=}}


\def\Circlearrowright{\ensuremath{%
  \rotatebox[origin=c]{230}{$\circlearrowright$}}}

\newcommand\ct[1]{\text{\rmfamily\upshape #1}}
\newcommand\question[1]{ {\color{red} ...!? \small #1}}
\newcommand\caz[1]{\left\{\begin{array} #1 \end{array}\right.}
\newcommand\const{\text{\rmfamily\upshape const}}
\newcommand\toP{ \overset{\pro}{\to}}
\newcommand\toPP{ \overset{\text{PP}}{\to}}
\newcommand{\oeq}{\mathrel{\text{\textcircled{$=$}}}}





\usepackage{xcolor}
% \usepackage[normalem]{ulem}
\usepackage{lipsum}
\makeatletter
% \newcommand\colorwave[1][blue]{\bgroup \markoverwith{\lower3.5\p@\hbox{\sixly \textcolor{#1}{\char58}}}\ULon}
%\font\sixly=lasy6 % does not re-load if already loaded, so no memory problem.

\newmdtheoremenv[
linewidth= 1pt,linecolor= blue,%
leftmargin=20,rightmargin=20,innertopmargin=0pt, innerrightmargin=40,%
tikzsetting = { draw=lightgray, line width = 0.3pt,dashed,%
dash pattern = on 15pt off 3pt},%
splittopskip=\topskip,skipbelow=\baselineskip,%
skipabove=\baselineskip,ntheorem,roundcorner=0pt,
% backgroundcolor=pagebg,font=\color{orange}\sffamily, fontcolor=white
]{examplebox}{Exemple}[section]



\newcommand\R{\mathbb{R}}
\newcommand\Z{\mathbb{Z}}
\newcommand\N{\mathbb{N}}
\newcommand\E{\mathbb{E}}
\newcommand\F{\mathcal{F}}
\newcommand\cH{\mathcal{H}}
\newcommand\V{\mathbb{V}}
\newcommand\dmo{ ^{-1} }
\newcommand\kapa{\kappa}
\newcommand\im{Im}
\newcommand\hs{\mathcal{H}}





\usepackage{soul}

\makeatletter
\newcommand*{\whiten}[1]{\llap{\textcolor{white}{{\the\SOUL@token}}\hspace{#1pt}}}
\DeclareRobustCommand*\myul{%
    \def\SOUL@everyspace{\underline{\space}\kern\z@}%
    \def\SOUL@everytoken{%
     \setbox0=\hbox{\the\SOUL@token}%
     \ifdim\dp0>\z@
        \raisebox{\dp0}{\underline{\phantom{\the\SOUL@token}}}%
        \whiten{1}\whiten{0}%
        \whiten{-1}\whiten{-2}%
        \llap{\the\SOUL@token}%
     \else
        \underline{\the\SOUL@token}%
     \fi}%
\SOUL@}
\makeatother

\newcommand*{\demp}{\fontfamily{lmtt}\selectfont}

\DeclareTextFontCommand{\textdemp}{\demp}

\begin{document}

\ifcomment
Multiline
comment
\fi
\ifcomment
\myul{Typesetting test}
% \color[rgb]{1,1,1}
$∑_i^n≠ 60º±∞π∆¬≈√j∫h≤≥µ$

$\CR \R\pro\ind\pro\gS\pro
\mqty[a&b\\c&d]$
$\pro\mathbb{P}$
$\dd{x}$

  \[
    \alpha(x)=\left\{
                \begin{array}{ll}
                  x\\
                  \frac{1}{1+e^{-kx}}\\
                  \frac{e^x-e^{-x}}{e^x+e^{-x}}
                \end{array}
              \right.
  \]

  $\expval{x}$
  
  $\chi_\rho(ghg\dmo)=\Tr(\rho_{ghg\dmo})=\Tr(\rho_g\circ\rho_h\circ\rho\dmo_g)=\Tr(\rho_h)\overset{\mbox{\scalebox{0.5}{$\Tr(AB)=\Tr(BA)$}}}{=}\chi_\rho(h)$
  	$\mathop{\oplus}_{\substack{x\in X}}$

$\mat(\rho_g)=(a_{ij}(g))_{\scriptsize \substack{1\leq i\leq d \\ 1\leq j\leq d}}$ et $\mat(\rho'_g)=(a'_{ij}(g))_{\scriptsize \substack{1\leq i'\leq d' \\ 1\leq j'\leq d'}}$



\[\int_a^b{\mathbb{R}^2}g(u, v)\dd{P_{XY}}(u, v)=\iint g(u,v) f_{XY}(u, v)\dd \lambda(u) \dd \lambda(v)\]
$$\lim_{x\to\infty} f(x)$$	
$$\iiiint_V \mu(t,u,v,w) \,dt\,du\,dv\,dw$$
$$\sum_{n=1}^{\infty} 2^{-n} = 1$$	
\begin{definition}
	Si $X$ et $Y$ sont 2 v.a. ou definit la \textsc{Covariance} entre $X$ et $Y$ comme
	$\cov(X,Y)\overset{\text{def}}{=}\E\left[(X-\E(X))(Y-\E(Y))\right]=\E(XY)-\E(X)\E(Y)$.
\end{definition}
\fi
\pagebreak

% \tableofcontents

% insert your code here
%\input{./algebra/main.tex}
%\input{./geometrie-differentielle/main.tex}
%\input{./probabilite/main.tex}
%\input{./analyse-fonctionnelle/main.tex}
% \input{./Analyse-convexe-et-dualite-en-optimisation/main.tex}
%\input{./tikz/main.tex}
%\input{./Theorie-du-distributions/main.tex}
%\input{./optimisation/mine.tex}
 \input{./modelisation/main.tex}

% yves.aubry@univ-tln.fr : algebra

\end{document}

% % !TEX encoding = UTF-8 Unicode
% !TEX TS-program = xelatex

\documentclass[french]{report}

%\usepackage[utf8]{inputenc}
%\usepackage[T1]{fontenc}
\usepackage{babel}


\newif\ifcomment
%\commenttrue # Show comments

\usepackage{physics}
\usepackage{amssymb}


\usepackage{amsthm}
% \usepackage{thmtools}
\usepackage{mathtools}
\usepackage{amsfonts}

\usepackage{color}

\usepackage{tikz}

\usepackage{geometry}
\geometry{a5paper, margin=0.1in, right=1cm}

\usepackage{dsfont}

\usepackage{graphicx}
\graphicspath{ {images/} }

\usepackage{faktor}

\usepackage{IEEEtrantools}
\usepackage{enumerate}   
\usepackage[PostScript=dvips]{"/Users/aware/Documents/Courses/diagrams"}


\newtheorem{theorem}{Théorème}[section]
\renewcommand{\thetheorem}{\arabic{theorem}}
\newtheorem{lemme}{Lemme}[section]
\renewcommand{\thelemme}{\arabic{lemme}}
\newtheorem{proposition}{Proposition}[section]
\renewcommand{\theproposition}{\arabic{proposition}}
\newtheorem{notations}{Notations}[section]
\newtheorem{problem}{Problème}[section]
\newtheorem{corollary}{Corollaire}[theorem]
\renewcommand{\thecorollary}{\arabic{corollary}}
\newtheorem{property}{Propriété}[section]
\newtheorem{objective}{Objectif}[section]

\theoremstyle{definition}
\newtheorem{definition}{Définition}[section]
\renewcommand{\thedefinition}{\arabic{definition}}
\newtheorem{exercise}{Exercice}[chapter]
\renewcommand{\theexercise}{\arabic{exercise}}
\newtheorem{example}{Exemple}[chapter]
\renewcommand{\theexample}{\arabic{example}}
\newtheorem*{solution}{Solution}
\newtheorem*{application}{Application}
\newtheorem*{notation}{Notation}
\newtheorem*{vocabulary}{Vocabulaire}
\newtheorem*{properties}{Propriétés}



\theoremstyle{remark}
\newtheorem*{remark}{Remarque}
\newtheorem*{rappel}{Rappel}


\usepackage{etoolbox}
\AtBeginEnvironment{exercise}{\small}
\AtBeginEnvironment{example}{\small}

\usepackage{cases}
\usepackage[red]{mypack}

\usepackage[framemethod=TikZ]{mdframed}

\definecolor{bg}{rgb}{0.4,0.25,0.95}
\definecolor{pagebg}{rgb}{0,0,0.5}
\surroundwithmdframed[
   topline=false,
   rightline=false,
   bottomline=false,
   leftmargin=\parindent,
   skipabove=8pt,
   skipbelow=8pt,
   linecolor=blue,
   innerbottommargin=10pt,
   % backgroundcolor=bg,font=\color{orange}\sffamily, fontcolor=white
]{definition}

\usepackage{empheq}
\usepackage[most]{tcolorbox}

\newtcbox{\mymath}[1][]{%
    nobeforeafter, math upper, tcbox raise base,
    enhanced, colframe=blue!30!black,
    colback=red!10, boxrule=1pt,
    #1}

\usepackage{unixode}


\DeclareMathOperator{\ord}{ord}
\DeclareMathOperator{\orb}{orb}
\DeclareMathOperator{\stab}{stab}
\DeclareMathOperator{\Stab}{stab}
\DeclareMathOperator{\ppcm}{ppcm}
\DeclareMathOperator{\conj}{Conj}
\DeclareMathOperator{\End}{End}
\DeclareMathOperator{\rot}{rot}
\DeclareMathOperator{\trs}{trace}
\DeclareMathOperator{\Ind}{Ind}
\DeclareMathOperator{\mat}{Mat}
\DeclareMathOperator{\id}{Id}
\DeclareMathOperator{\vect}{vect}
\DeclareMathOperator{\img}{img}
\DeclareMathOperator{\cov}{Cov}
\DeclareMathOperator{\dist}{dist}
\DeclareMathOperator{\irr}{Irr}
\DeclareMathOperator{\image}{Im}
\DeclareMathOperator{\pd}{\partial}
\DeclareMathOperator{\epi}{epi}
\DeclareMathOperator{\Argmin}{Argmin}
\DeclareMathOperator{\dom}{dom}
\DeclareMathOperator{\proj}{proj}
\DeclareMathOperator{\ctg}{ctg}
\DeclareMathOperator{\supp}{supp}
\DeclareMathOperator{\argmin}{argmin}
\DeclareMathOperator{\mult}{mult}
\DeclareMathOperator{\ch}{ch}
\DeclareMathOperator{\sh}{sh}
\DeclareMathOperator{\rang}{rang}
\DeclareMathOperator{\diam}{diam}
\DeclareMathOperator{\Epigraphe}{Epigraphe}




\usepackage{xcolor}
\everymath{\color{blue}}
%\everymath{\color[rgb]{0,1,1}}
%\pagecolor[rgb]{0,0,0.5}


\newcommand*{\pdtest}[3][]{\ensuremath{\frac{\partial^{#1} #2}{\partial #3}}}

\newcommand*{\deffunc}[6][]{\ensuremath{
\begin{array}{rcl}
#2 : #3 &\rightarrow& #4\\
#5 &\mapsto& #6
\end{array}
}}

\newcommand{\eqcolon}{\mathrel{\resizebox{\widthof{$\mathord{=}$}}{\height}{ $\!\!=\!\!\resizebox{1.2\width}{0.8\height}{\raisebox{0.23ex}{$\mathop{:}$}}\!\!$ }}}
\newcommand{\coloneq}{\mathrel{\resizebox{\widthof{$\mathord{=}$}}{\height}{ $\!\!\resizebox{1.2\width}{0.8\height}{\raisebox{0.23ex}{$\mathop{:}$}}\!\!=\!\!$ }}}
\newcommand{\eqcolonl}{\ensuremath{\mathrel{=\!\!\mathop{:}}}}
\newcommand{\coloneql}{\ensuremath{\mathrel{\mathop{:} \!\! =}}}
\newcommand{\vc}[1]{% inline column vector
  \left(\begin{smallmatrix}#1\end{smallmatrix}\right)%
}
\newcommand{\vr}[1]{% inline row vector
  \begin{smallmatrix}(\,#1\,)\end{smallmatrix}%
}
\makeatletter
\newcommand*{\defeq}{\ =\mathrel{\rlap{%
                     \raisebox{0.3ex}{$\m@th\cdot$}}%
                     \raisebox{-0.3ex}{$\m@th\cdot$}}%
                     }
\makeatother

\newcommand{\mathcircle}[1]{% inline row vector
 \overset{\circ}{#1}
}
\newcommand{\ulim}{% low limit
 \underline{\lim}
}
\newcommand{\ssi}{% iff
\iff
}
\newcommand{\ps}[2]{
\expval{#1 | #2}
}
\newcommand{\df}[1]{
\mqty{#1}
}
\newcommand{\n}[1]{
\norm{#1}
}
\newcommand{\sys}[1]{
\left\{\smqty{#1}\right.
}


\newcommand{\eqdef}{\ensuremath{\overset{\text{def}}=}}


\def\Circlearrowright{\ensuremath{%
  \rotatebox[origin=c]{230}{$\circlearrowright$}}}

\newcommand\ct[1]{\text{\rmfamily\upshape #1}}
\newcommand\question[1]{ {\color{red} ...!? \small #1}}
\newcommand\caz[1]{\left\{\begin{array} #1 \end{array}\right.}
\newcommand\const{\text{\rmfamily\upshape const}}
\newcommand\toP{ \overset{\pro}{\to}}
\newcommand\toPP{ \overset{\text{PP}}{\to}}
\newcommand{\oeq}{\mathrel{\text{\textcircled{$=$}}}}





\usepackage{xcolor}
% \usepackage[normalem]{ulem}
\usepackage{lipsum}
\makeatletter
% \newcommand\colorwave[1][blue]{\bgroup \markoverwith{\lower3.5\p@\hbox{\sixly \textcolor{#1}{\char58}}}\ULon}
%\font\sixly=lasy6 % does not re-load if already loaded, so no memory problem.

\newmdtheoremenv[
linewidth= 1pt,linecolor= blue,%
leftmargin=20,rightmargin=20,innertopmargin=0pt, innerrightmargin=40,%
tikzsetting = { draw=lightgray, line width = 0.3pt,dashed,%
dash pattern = on 15pt off 3pt},%
splittopskip=\topskip,skipbelow=\baselineskip,%
skipabove=\baselineskip,ntheorem,roundcorner=0pt,
% backgroundcolor=pagebg,font=\color{orange}\sffamily, fontcolor=white
]{examplebox}{Exemple}[section]



\newcommand\R{\mathbb{R}}
\newcommand\Z{\mathbb{Z}}
\newcommand\N{\mathbb{N}}
\newcommand\E{\mathbb{E}}
\newcommand\F{\mathcal{F}}
\newcommand\cH{\mathcal{H}}
\newcommand\V{\mathbb{V}}
\newcommand\dmo{ ^{-1} }
\newcommand\kapa{\kappa}
\newcommand\im{Im}
\newcommand\hs{\mathcal{H}}





\usepackage{soul}

\makeatletter
\newcommand*{\whiten}[1]{\llap{\textcolor{white}{{\the\SOUL@token}}\hspace{#1pt}}}
\DeclareRobustCommand*\myul{%
    \def\SOUL@everyspace{\underline{\space}\kern\z@}%
    \def\SOUL@everytoken{%
     \setbox0=\hbox{\the\SOUL@token}%
     \ifdim\dp0>\z@
        \raisebox{\dp0}{\underline{\phantom{\the\SOUL@token}}}%
        \whiten{1}\whiten{0}%
        \whiten{-1}\whiten{-2}%
        \llap{\the\SOUL@token}%
     \else
        \underline{\the\SOUL@token}%
     \fi}%
\SOUL@}
\makeatother

\newcommand*{\demp}{\fontfamily{lmtt}\selectfont}

\DeclareTextFontCommand{\textdemp}{\demp}

\begin{document}

\ifcomment
Multiline
comment
\fi
\ifcomment
\myul{Typesetting test}
% \color[rgb]{1,1,1}
$∑_i^n≠ 60º±∞π∆¬≈√j∫h≤≥µ$

$\CR \R\pro\ind\pro\gS\pro
\mqty[a&b\\c&d]$
$\pro\mathbb{P}$
$\dd{x}$

  \[
    \alpha(x)=\left\{
                \begin{array}{ll}
                  x\\
                  \frac{1}{1+e^{-kx}}\\
                  \frac{e^x-e^{-x}}{e^x+e^{-x}}
                \end{array}
              \right.
  \]

  $\expval{x}$
  
  $\chi_\rho(ghg\dmo)=\Tr(\rho_{ghg\dmo})=\Tr(\rho_g\circ\rho_h\circ\rho\dmo_g)=\Tr(\rho_h)\overset{\mbox{\scalebox{0.5}{$\Tr(AB)=\Tr(BA)$}}}{=}\chi_\rho(h)$
  	$\mathop{\oplus}_{\substack{x\in X}}$

$\mat(\rho_g)=(a_{ij}(g))_{\scriptsize \substack{1\leq i\leq d \\ 1\leq j\leq d}}$ et $\mat(\rho'_g)=(a'_{ij}(g))_{\scriptsize \substack{1\leq i'\leq d' \\ 1\leq j'\leq d'}}$



\[\int_a^b{\mathbb{R}^2}g(u, v)\dd{P_{XY}}(u, v)=\iint g(u,v) f_{XY}(u, v)\dd \lambda(u) \dd \lambda(v)\]
$$\lim_{x\to\infty} f(x)$$	
$$\iiiint_V \mu(t,u,v,w) \,dt\,du\,dv\,dw$$
$$\sum_{n=1}^{\infty} 2^{-n} = 1$$	
\begin{definition}
	Si $X$ et $Y$ sont 2 v.a. ou definit la \textsc{Covariance} entre $X$ et $Y$ comme
	$\cov(X,Y)\overset{\text{def}}{=}\E\left[(X-\E(X))(Y-\E(Y))\right]=\E(XY)-\E(X)\E(Y)$.
\end{definition}
\fi
\pagebreak

% \tableofcontents

% insert your code here
%\input{./algebra/main.tex}
%\input{./geometrie-differentielle/main.tex}
%\input{./probabilite/main.tex}
%\input{./analyse-fonctionnelle/main.tex}
% \input{./Analyse-convexe-et-dualite-en-optimisation/main.tex}
%\input{./tikz/main.tex}
%\input{./Theorie-du-distributions/main.tex}
%\input{./optimisation/mine.tex}
 \input{./modelisation/main.tex}

% yves.aubry@univ-tln.fr : algebra

\end{document}

%% !TEX encoding = UTF-8 Unicode
% !TEX TS-program = xelatex

\documentclass[french]{report}

%\usepackage[utf8]{inputenc}
%\usepackage[T1]{fontenc}
\usepackage{babel}


\newif\ifcomment
%\commenttrue # Show comments

\usepackage{physics}
\usepackage{amssymb}


\usepackage{amsthm}
% \usepackage{thmtools}
\usepackage{mathtools}
\usepackage{amsfonts}

\usepackage{color}

\usepackage{tikz}

\usepackage{geometry}
\geometry{a5paper, margin=0.1in, right=1cm}

\usepackage{dsfont}

\usepackage{graphicx}
\graphicspath{ {images/} }

\usepackage{faktor}

\usepackage{IEEEtrantools}
\usepackage{enumerate}   
\usepackage[PostScript=dvips]{"/Users/aware/Documents/Courses/diagrams"}


\newtheorem{theorem}{Théorème}[section]
\renewcommand{\thetheorem}{\arabic{theorem}}
\newtheorem{lemme}{Lemme}[section]
\renewcommand{\thelemme}{\arabic{lemme}}
\newtheorem{proposition}{Proposition}[section]
\renewcommand{\theproposition}{\arabic{proposition}}
\newtheorem{notations}{Notations}[section]
\newtheorem{problem}{Problème}[section]
\newtheorem{corollary}{Corollaire}[theorem]
\renewcommand{\thecorollary}{\arabic{corollary}}
\newtheorem{property}{Propriété}[section]
\newtheorem{objective}{Objectif}[section]

\theoremstyle{definition}
\newtheorem{definition}{Définition}[section]
\renewcommand{\thedefinition}{\arabic{definition}}
\newtheorem{exercise}{Exercice}[chapter]
\renewcommand{\theexercise}{\arabic{exercise}}
\newtheorem{example}{Exemple}[chapter]
\renewcommand{\theexample}{\arabic{example}}
\newtheorem*{solution}{Solution}
\newtheorem*{application}{Application}
\newtheorem*{notation}{Notation}
\newtheorem*{vocabulary}{Vocabulaire}
\newtheorem*{properties}{Propriétés}



\theoremstyle{remark}
\newtheorem*{remark}{Remarque}
\newtheorem*{rappel}{Rappel}


\usepackage{etoolbox}
\AtBeginEnvironment{exercise}{\small}
\AtBeginEnvironment{example}{\small}

\usepackage{cases}
\usepackage[red]{mypack}

\usepackage[framemethod=TikZ]{mdframed}

\definecolor{bg}{rgb}{0.4,0.25,0.95}
\definecolor{pagebg}{rgb}{0,0,0.5}
\surroundwithmdframed[
   topline=false,
   rightline=false,
   bottomline=false,
   leftmargin=\parindent,
   skipabove=8pt,
   skipbelow=8pt,
   linecolor=blue,
   innerbottommargin=10pt,
   % backgroundcolor=bg,font=\color{orange}\sffamily, fontcolor=white
]{definition}

\usepackage{empheq}
\usepackage[most]{tcolorbox}

\newtcbox{\mymath}[1][]{%
    nobeforeafter, math upper, tcbox raise base,
    enhanced, colframe=blue!30!black,
    colback=red!10, boxrule=1pt,
    #1}

\usepackage{unixode}


\DeclareMathOperator{\ord}{ord}
\DeclareMathOperator{\orb}{orb}
\DeclareMathOperator{\stab}{stab}
\DeclareMathOperator{\Stab}{stab}
\DeclareMathOperator{\ppcm}{ppcm}
\DeclareMathOperator{\conj}{Conj}
\DeclareMathOperator{\End}{End}
\DeclareMathOperator{\rot}{rot}
\DeclareMathOperator{\trs}{trace}
\DeclareMathOperator{\Ind}{Ind}
\DeclareMathOperator{\mat}{Mat}
\DeclareMathOperator{\id}{Id}
\DeclareMathOperator{\vect}{vect}
\DeclareMathOperator{\img}{img}
\DeclareMathOperator{\cov}{Cov}
\DeclareMathOperator{\dist}{dist}
\DeclareMathOperator{\irr}{Irr}
\DeclareMathOperator{\image}{Im}
\DeclareMathOperator{\pd}{\partial}
\DeclareMathOperator{\epi}{epi}
\DeclareMathOperator{\Argmin}{Argmin}
\DeclareMathOperator{\dom}{dom}
\DeclareMathOperator{\proj}{proj}
\DeclareMathOperator{\ctg}{ctg}
\DeclareMathOperator{\supp}{supp}
\DeclareMathOperator{\argmin}{argmin}
\DeclareMathOperator{\mult}{mult}
\DeclareMathOperator{\ch}{ch}
\DeclareMathOperator{\sh}{sh}
\DeclareMathOperator{\rang}{rang}
\DeclareMathOperator{\diam}{diam}
\DeclareMathOperator{\Epigraphe}{Epigraphe}




\usepackage{xcolor}
\everymath{\color{blue}}
%\everymath{\color[rgb]{0,1,1}}
%\pagecolor[rgb]{0,0,0.5}


\newcommand*{\pdtest}[3][]{\ensuremath{\frac{\partial^{#1} #2}{\partial #3}}}

\newcommand*{\deffunc}[6][]{\ensuremath{
\begin{array}{rcl}
#2 : #3 &\rightarrow& #4\\
#5 &\mapsto& #6
\end{array}
}}

\newcommand{\eqcolon}{\mathrel{\resizebox{\widthof{$\mathord{=}$}}{\height}{ $\!\!=\!\!\resizebox{1.2\width}{0.8\height}{\raisebox{0.23ex}{$\mathop{:}$}}\!\!$ }}}
\newcommand{\coloneq}{\mathrel{\resizebox{\widthof{$\mathord{=}$}}{\height}{ $\!\!\resizebox{1.2\width}{0.8\height}{\raisebox{0.23ex}{$\mathop{:}$}}\!\!=\!\!$ }}}
\newcommand{\eqcolonl}{\ensuremath{\mathrel{=\!\!\mathop{:}}}}
\newcommand{\coloneql}{\ensuremath{\mathrel{\mathop{:} \!\! =}}}
\newcommand{\vc}[1]{% inline column vector
  \left(\begin{smallmatrix}#1\end{smallmatrix}\right)%
}
\newcommand{\vr}[1]{% inline row vector
  \begin{smallmatrix}(\,#1\,)\end{smallmatrix}%
}
\makeatletter
\newcommand*{\defeq}{\ =\mathrel{\rlap{%
                     \raisebox{0.3ex}{$\m@th\cdot$}}%
                     \raisebox{-0.3ex}{$\m@th\cdot$}}%
                     }
\makeatother

\newcommand{\mathcircle}[1]{% inline row vector
 \overset{\circ}{#1}
}
\newcommand{\ulim}{% low limit
 \underline{\lim}
}
\newcommand{\ssi}{% iff
\iff
}
\newcommand{\ps}[2]{
\expval{#1 | #2}
}
\newcommand{\df}[1]{
\mqty{#1}
}
\newcommand{\n}[1]{
\norm{#1}
}
\newcommand{\sys}[1]{
\left\{\smqty{#1}\right.
}


\newcommand{\eqdef}{\ensuremath{\overset{\text{def}}=}}


\def\Circlearrowright{\ensuremath{%
  \rotatebox[origin=c]{230}{$\circlearrowright$}}}

\newcommand\ct[1]{\text{\rmfamily\upshape #1}}
\newcommand\question[1]{ {\color{red} ...!? \small #1}}
\newcommand\caz[1]{\left\{\begin{array} #1 \end{array}\right.}
\newcommand\const{\text{\rmfamily\upshape const}}
\newcommand\toP{ \overset{\pro}{\to}}
\newcommand\toPP{ \overset{\text{PP}}{\to}}
\newcommand{\oeq}{\mathrel{\text{\textcircled{$=$}}}}





\usepackage{xcolor}
% \usepackage[normalem]{ulem}
\usepackage{lipsum}
\makeatletter
% \newcommand\colorwave[1][blue]{\bgroup \markoverwith{\lower3.5\p@\hbox{\sixly \textcolor{#1}{\char58}}}\ULon}
%\font\sixly=lasy6 % does not re-load if already loaded, so no memory problem.

\newmdtheoremenv[
linewidth= 1pt,linecolor= blue,%
leftmargin=20,rightmargin=20,innertopmargin=0pt, innerrightmargin=40,%
tikzsetting = { draw=lightgray, line width = 0.3pt,dashed,%
dash pattern = on 15pt off 3pt},%
splittopskip=\topskip,skipbelow=\baselineskip,%
skipabove=\baselineskip,ntheorem,roundcorner=0pt,
% backgroundcolor=pagebg,font=\color{orange}\sffamily, fontcolor=white
]{examplebox}{Exemple}[section]



\newcommand\R{\mathbb{R}}
\newcommand\Z{\mathbb{Z}}
\newcommand\N{\mathbb{N}}
\newcommand\E{\mathbb{E}}
\newcommand\F{\mathcal{F}}
\newcommand\cH{\mathcal{H}}
\newcommand\V{\mathbb{V}}
\newcommand\dmo{ ^{-1} }
\newcommand\kapa{\kappa}
\newcommand\im{Im}
\newcommand\hs{\mathcal{H}}





\usepackage{soul}

\makeatletter
\newcommand*{\whiten}[1]{\llap{\textcolor{white}{{\the\SOUL@token}}\hspace{#1pt}}}
\DeclareRobustCommand*\myul{%
    \def\SOUL@everyspace{\underline{\space}\kern\z@}%
    \def\SOUL@everytoken{%
     \setbox0=\hbox{\the\SOUL@token}%
     \ifdim\dp0>\z@
        \raisebox{\dp0}{\underline{\phantom{\the\SOUL@token}}}%
        \whiten{1}\whiten{0}%
        \whiten{-1}\whiten{-2}%
        \llap{\the\SOUL@token}%
     \else
        \underline{\the\SOUL@token}%
     \fi}%
\SOUL@}
\makeatother

\newcommand*{\demp}{\fontfamily{lmtt}\selectfont}

\DeclareTextFontCommand{\textdemp}{\demp}

\begin{document}

\ifcomment
Multiline
comment
\fi
\ifcomment
\myul{Typesetting test}
% \color[rgb]{1,1,1}
$∑_i^n≠ 60º±∞π∆¬≈√j∫h≤≥µ$

$\CR \R\pro\ind\pro\gS\pro
\mqty[a&b\\c&d]$
$\pro\mathbb{P}$
$\dd{x}$

  \[
    \alpha(x)=\left\{
                \begin{array}{ll}
                  x\\
                  \frac{1}{1+e^{-kx}}\\
                  \frac{e^x-e^{-x}}{e^x+e^{-x}}
                \end{array}
              \right.
  \]

  $\expval{x}$
  
  $\chi_\rho(ghg\dmo)=\Tr(\rho_{ghg\dmo})=\Tr(\rho_g\circ\rho_h\circ\rho\dmo_g)=\Tr(\rho_h)\overset{\mbox{\scalebox{0.5}{$\Tr(AB)=\Tr(BA)$}}}{=}\chi_\rho(h)$
  	$\mathop{\oplus}_{\substack{x\in X}}$

$\mat(\rho_g)=(a_{ij}(g))_{\scriptsize \substack{1\leq i\leq d \\ 1\leq j\leq d}}$ et $\mat(\rho'_g)=(a'_{ij}(g))_{\scriptsize \substack{1\leq i'\leq d' \\ 1\leq j'\leq d'}}$



\[\int_a^b{\mathbb{R}^2}g(u, v)\dd{P_{XY}}(u, v)=\iint g(u,v) f_{XY}(u, v)\dd \lambda(u) \dd \lambda(v)\]
$$\lim_{x\to\infty} f(x)$$	
$$\iiiint_V \mu(t,u,v,w) \,dt\,du\,dv\,dw$$
$$\sum_{n=1}^{\infty} 2^{-n} = 1$$	
\begin{definition}
	Si $X$ et $Y$ sont 2 v.a. ou definit la \textsc{Covariance} entre $X$ et $Y$ comme
	$\cov(X,Y)\overset{\text{def}}{=}\E\left[(X-\E(X))(Y-\E(Y))\right]=\E(XY)-\E(X)\E(Y)$.
\end{definition}
\fi
\pagebreak

% \tableofcontents

% insert your code here
%\input{./algebra/main.tex}
%\input{./geometrie-differentielle/main.tex}
%\input{./probabilite/main.tex}
%\input{./analyse-fonctionnelle/main.tex}
% \input{./Analyse-convexe-et-dualite-en-optimisation/main.tex}
%\input{./tikz/main.tex}
%\input{./Theorie-du-distributions/main.tex}
%\input{./optimisation/mine.tex}
 \input{./modelisation/main.tex}

% yves.aubry@univ-tln.fr : algebra

\end{document}

%% !TEX encoding = UTF-8 Unicode
% !TEX TS-program = xelatex

\documentclass[french]{report}

%\usepackage[utf8]{inputenc}
%\usepackage[T1]{fontenc}
\usepackage{babel}


\newif\ifcomment
%\commenttrue # Show comments

\usepackage{physics}
\usepackage{amssymb}


\usepackage{amsthm}
% \usepackage{thmtools}
\usepackage{mathtools}
\usepackage{amsfonts}

\usepackage{color}

\usepackage{tikz}

\usepackage{geometry}
\geometry{a5paper, margin=0.1in, right=1cm}

\usepackage{dsfont}

\usepackage{graphicx}
\graphicspath{ {images/} }

\usepackage{faktor}

\usepackage{IEEEtrantools}
\usepackage{enumerate}   
\usepackage[PostScript=dvips]{"/Users/aware/Documents/Courses/diagrams"}


\newtheorem{theorem}{Théorème}[section]
\renewcommand{\thetheorem}{\arabic{theorem}}
\newtheorem{lemme}{Lemme}[section]
\renewcommand{\thelemme}{\arabic{lemme}}
\newtheorem{proposition}{Proposition}[section]
\renewcommand{\theproposition}{\arabic{proposition}}
\newtheorem{notations}{Notations}[section]
\newtheorem{problem}{Problème}[section]
\newtheorem{corollary}{Corollaire}[theorem]
\renewcommand{\thecorollary}{\arabic{corollary}}
\newtheorem{property}{Propriété}[section]
\newtheorem{objective}{Objectif}[section]

\theoremstyle{definition}
\newtheorem{definition}{Définition}[section]
\renewcommand{\thedefinition}{\arabic{definition}}
\newtheorem{exercise}{Exercice}[chapter]
\renewcommand{\theexercise}{\arabic{exercise}}
\newtheorem{example}{Exemple}[chapter]
\renewcommand{\theexample}{\arabic{example}}
\newtheorem*{solution}{Solution}
\newtheorem*{application}{Application}
\newtheorem*{notation}{Notation}
\newtheorem*{vocabulary}{Vocabulaire}
\newtheorem*{properties}{Propriétés}



\theoremstyle{remark}
\newtheorem*{remark}{Remarque}
\newtheorem*{rappel}{Rappel}


\usepackage{etoolbox}
\AtBeginEnvironment{exercise}{\small}
\AtBeginEnvironment{example}{\small}

\usepackage{cases}
\usepackage[red]{mypack}

\usepackage[framemethod=TikZ]{mdframed}

\definecolor{bg}{rgb}{0.4,0.25,0.95}
\definecolor{pagebg}{rgb}{0,0,0.5}
\surroundwithmdframed[
   topline=false,
   rightline=false,
   bottomline=false,
   leftmargin=\parindent,
   skipabove=8pt,
   skipbelow=8pt,
   linecolor=blue,
   innerbottommargin=10pt,
   % backgroundcolor=bg,font=\color{orange}\sffamily, fontcolor=white
]{definition}

\usepackage{empheq}
\usepackage[most]{tcolorbox}

\newtcbox{\mymath}[1][]{%
    nobeforeafter, math upper, tcbox raise base,
    enhanced, colframe=blue!30!black,
    colback=red!10, boxrule=1pt,
    #1}

\usepackage{unixode}


\DeclareMathOperator{\ord}{ord}
\DeclareMathOperator{\orb}{orb}
\DeclareMathOperator{\stab}{stab}
\DeclareMathOperator{\Stab}{stab}
\DeclareMathOperator{\ppcm}{ppcm}
\DeclareMathOperator{\conj}{Conj}
\DeclareMathOperator{\End}{End}
\DeclareMathOperator{\rot}{rot}
\DeclareMathOperator{\trs}{trace}
\DeclareMathOperator{\Ind}{Ind}
\DeclareMathOperator{\mat}{Mat}
\DeclareMathOperator{\id}{Id}
\DeclareMathOperator{\vect}{vect}
\DeclareMathOperator{\img}{img}
\DeclareMathOperator{\cov}{Cov}
\DeclareMathOperator{\dist}{dist}
\DeclareMathOperator{\irr}{Irr}
\DeclareMathOperator{\image}{Im}
\DeclareMathOperator{\pd}{\partial}
\DeclareMathOperator{\epi}{epi}
\DeclareMathOperator{\Argmin}{Argmin}
\DeclareMathOperator{\dom}{dom}
\DeclareMathOperator{\proj}{proj}
\DeclareMathOperator{\ctg}{ctg}
\DeclareMathOperator{\supp}{supp}
\DeclareMathOperator{\argmin}{argmin}
\DeclareMathOperator{\mult}{mult}
\DeclareMathOperator{\ch}{ch}
\DeclareMathOperator{\sh}{sh}
\DeclareMathOperator{\rang}{rang}
\DeclareMathOperator{\diam}{diam}
\DeclareMathOperator{\Epigraphe}{Epigraphe}




\usepackage{xcolor}
\everymath{\color{blue}}
%\everymath{\color[rgb]{0,1,1}}
%\pagecolor[rgb]{0,0,0.5}


\newcommand*{\pdtest}[3][]{\ensuremath{\frac{\partial^{#1} #2}{\partial #3}}}

\newcommand*{\deffunc}[6][]{\ensuremath{
\begin{array}{rcl}
#2 : #3 &\rightarrow& #4\\
#5 &\mapsto& #6
\end{array}
}}

\newcommand{\eqcolon}{\mathrel{\resizebox{\widthof{$\mathord{=}$}}{\height}{ $\!\!=\!\!\resizebox{1.2\width}{0.8\height}{\raisebox{0.23ex}{$\mathop{:}$}}\!\!$ }}}
\newcommand{\coloneq}{\mathrel{\resizebox{\widthof{$\mathord{=}$}}{\height}{ $\!\!\resizebox{1.2\width}{0.8\height}{\raisebox{0.23ex}{$\mathop{:}$}}\!\!=\!\!$ }}}
\newcommand{\eqcolonl}{\ensuremath{\mathrel{=\!\!\mathop{:}}}}
\newcommand{\coloneql}{\ensuremath{\mathrel{\mathop{:} \!\! =}}}
\newcommand{\vc}[1]{% inline column vector
  \left(\begin{smallmatrix}#1\end{smallmatrix}\right)%
}
\newcommand{\vr}[1]{% inline row vector
  \begin{smallmatrix}(\,#1\,)\end{smallmatrix}%
}
\makeatletter
\newcommand*{\defeq}{\ =\mathrel{\rlap{%
                     \raisebox{0.3ex}{$\m@th\cdot$}}%
                     \raisebox{-0.3ex}{$\m@th\cdot$}}%
                     }
\makeatother

\newcommand{\mathcircle}[1]{% inline row vector
 \overset{\circ}{#1}
}
\newcommand{\ulim}{% low limit
 \underline{\lim}
}
\newcommand{\ssi}{% iff
\iff
}
\newcommand{\ps}[2]{
\expval{#1 | #2}
}
\newcommand{\df}[1]{
\mqty{#1}
}
\newcommand{\n}[1]{
\norm{#1}
}
\newcommand{\sys}[1]{
\left\{\smqty{#1}\right.
}


\newcommand{\eqdef}{\ensuremath{\overset{\text{def}}=}}


\def\Circlearrowright{\ensuremath{%
  \rotatebox[origin=c]{230}{$\circlearrowright$}}}

\newcommand\ct[1]{\text{\rmfamily\upshape #1}}
\newcommand\question[1]{ {\color{red} ...!? \small #1}}
\newcommand\caz[1]{\left\{\begin{array} #1 \end{array}\right.}
\newcommand\const{\text{\rmfamily\upshape const}}
\newcommand\toP{ \overset{\pro}{\to}}
\newcommand\toPP{ \overset{\text{PP}}{\to}}
\newcommand{\oeq}{\mathrel{\text{\textcircled{$=$}}}}





\usepackage{xcolor}
% \usepackage[normalem]{ulem}
\usepackage{lipsum}
\makeatletter
% \newcommand\colorwave[1][blue]{\bgroup \markoverwith{\lower3.5\p@\hbox{\sixly \textcolor{#1}{\char58}}}\ULon}
%\font\sixly=lasy6 % does not re-load if already loaded, so no memory problem.

\newmdtheoremenv[
linewidth= 1pt,linecolor= blue,%
leftmargin=20,rightmargin=20,innertopmargin=0pt, innerrightmargin=40,%
tikzsetting = { draw=lightgray, line width = 0.3pt,dashed,%
dash pattern = on 15pt off 3pt},%
splittopskip=\topskip,skipbelow=\baselineskip,%
skipabove=\baselineskip,ntheorem,roundcorner=0pt,
% backgroundcolor=pagebg,font=\color{orange}\sffamily, fontcolor=white
]{examplebox}{Exemple}[section]



\newcommand\R{\mathbb{R}}
\newcommand\Z{\mathbb{Z}}
\newcommand\N{\mathbb{N}}
\newcommand\E{\mathbb{E}}
\newcommand\F{\mathcal{F}}
\newcommand\cH{\mathcal{H}}
\newcommand\V{\mathbb{V}}
\newcommand\dmo{ ^{-1} }
\newcommand\kapa{\kappa}
\newcommand\im{Im}
\newcommand\hs{\mathcal{H}}





\usepackage{soul}

\makeatletter
\newcommand*{\whiten}[1]{\llap{\textcolor{white}{{\the\SOUL@token}}\hspace{#1pt}}}
\DeclareRobustCommand*\myul{%
    \def\SOUL@everyspace{\underline{\space}\kern\z@}%
    \def\SOUL@everytoken{%
     \setbox0=\hbox{\the\SOUL@token}%
     \ifdim\dp0>\z@
        \raisebox{\dp0}{\underline{\phantom{\the\SOUL@token}}}%
        \whiten{1}\whiten{0}%
        \whiten{-1}\whiten{-2}%
        \llap{\the\SOUL@token}%
     \else
        \underline{\the\SOUL@token}%
     \fi}%
\SOUL@}
\makeatother

\newcommand*{\demp}{\fontfamily{lmtt}\selectfont}

\DeclareTextFontCommand{\textdemp}{\demp}

\begin{document}

\ifcomment
Multiline
comment
\fi
\ifcomment
\myul{Typesetting test}
% \color[rgb]{1,1,1}
$∑_i^n≠ 60º±∞π∆¬≈√j∫h≤≥µ$

$\CR \R\pro\ind\pro\gS\pro
\mqty[a&b\\c&d]$
$\pro\mathbb{P}$
$\dd{x}$

  \[
    \alpha(x)=\left\{
                \begin{array}{ll}
                  x\\
                  \frac{1}{1+e^{-kx}}\\
                  \frac{e^x-e^{-x}}{e^x+e^{-x}}
                \end{array}
              \right.
  \]

  $\expval{x}$
  
  $\chi_\rho(ghg\dmo)=\Tr(\rho_{ghg\dmo})=\Tr(\rho_g\circ\rho_h\circ\rho\dmo_g)=\Tr(\rho_h)\overset{\mbox{\scalebox{0.5}{$\Tr(AB)=\Tr(BA)$}}}{=}\chi_\rho(h)$
  	$\mathop{\oplus}_{\substack{x\in X}}$

$\mat(\rho_g)=(a_{ij}(g))_{\scriptsize \substack{1\leq i\leq d \\ 1\leq j\leq d}}$ et $\mat(\rho'_g)=(a'_{ij}(g))_{\scriptsize \substack{1\leq i'\leq d' \\ 1\leq j'\leq d'}}$



\[\int_a^b{\mathbb{R}^2}g(u, v)\dd{P_{XY}}(u, v)=\iint g(u,v) f_{XY}(u, v)\dd \lambda(u) \dd \lambda(v)\]
$$\lim_{x\to\infty} f(x)$$	
$$\iiiint_V \mu(t,u,v,w) \,dt\,du\,dv\,dw$$
$$\sum_{n=1}^{\infty} 2^{-n} = 1$$	
\begin{definition}
	Si $X$ et $Y$ sont 2 v.a. ou definit la \textsc{Covariance} entre $X$ et $Y$ comme
	$\cov(X,Y)\overset{\text{def}}{=}\E\left[(X-\E(X))(Y-\E(Y))\right]=\E(XY)-\E(X)\E(Y)$.
\end{definition}
\fi
\pagebreak

% \tableofcontents

% insert your code here
%\input{./algebra/main.tex}
%\input{./geometrie-differentielle/main.tex}
%\input{./probabilite/main.tex}
%\input{./analyse-fonctionnelle/main.tex}
% \input{./Analyse-convexe-et-dualite-en-optimisation/main.tex}
%\input{./tikz/main.tex}
%\input{./Theorie-du-distributions/main.tex}
%\input{./optimisation/mine.tex}
 \input{./modelisation/main.tex}

% yves.aubry@univ-tln.fr : algebra

\end{document}

%\input{./optimisation/mine.tex}
 % !TEX encoding = UTF-8 Unicode
% !TEX TS-program = xelatex

\documentclass[french]{report}

%\usepackage[utf8]{inputenc}
%\usepackage[T1]{fontenc}
\usepackage{babel}


\newif\ifcomment
%\commenttrue # Show comments

\usepackage{physics}
\usepackage{amssymb}


\usepackage{amsthm}
% \usepackage{thmtools}
\usepackage{mathtools}
\usepackage{amsfonts}

\usepackage{color}

\usepackage{tikz}

\usepackage{geometry}
\geometry{a5paper, margin=0.1in, right=1cm}

\usepackage{dsfont}

\usepackage{graphicx}
\graphicspath{ {images/} }

\usepackage{faktor}

\usepackage{IEEEtrantools}
\usepackage{enumerate}   
\usepackage[PostScript=dvips]{"/Users/aware/Documents/Courses/diagrams"}


\newtheorem{theorem}{Théorème}[section]
\renewcommand{\thetheorem}{\arabic{theorem}}
\newtheorem{lemme}{Lemme}[section]
\renewcommand{\thelemme}{\arabic{lemme}}
\newtheorem{proposition}{Proposition}[section]
\renewcommand{\theproposition}{\arabic{proposition}}
\newtheorem{notations}{Notations}[section]
\newtheorem{problem}{Problème}[section]
\newtheorem{corollary}{Corollaire}[theorem]
\renewcommand{\thecorollary}{\arabic{corollary}}
\newtheorem{property}{Propriété}[section]
\newtheorem{objective}{Objectif}[section]

\theoremstyle{definition}
\newtheorem{definition}{Définition}[section]
\renewcommand{\thedefinition}{\arabic{definition}}
\newtheorem{exercise}{Exercice}[chapter]
\renewcommand{\theexercise}{\arabic{exercise}}
\newtheorem{example}{Exemple}[chapter]
\renewcommand{\theexample}{\arabic{example}}
\newtheorem*{solution}{Solution}
\newtheorem*{application}{Application}
\newtheorem*{notation}{Notation}
\newtheorem*{vocabulary}{Vocabulaire}
\newtheorem*{properties}{Propriétés}



\theoremstyle{remark}
\newtheorem*{remark}{Remarque}
\newtheorem*{rappel}{Rappel}


\usepackage{etoolbox}
\AtBeginEnvironment{exercise}{\small}
\AtBeginEnvironment{example}{\small}

\usepackage{cases}
\usepackage[red]{mypack}

\usepackage[framemethod=TikZ]{mdframed}

\definecolor{bg}{rgb}{0.4,0.25,0.95}
\definecolor{pagebg}{rgb}{0,0,0.5}
\surroundwithmdframed[
   topline=false,
   rightline=false,
   bottomline=false,
   leftmargin=\parindent,
   skipabove=8pt,
   skipbelow=8pt,
   linecolor=blue,
   innerbottommargin=10pt,
   % backgroundcolor=bg,font=\color{orange}\sffamily, fontcolor=white
]{definition}

\usepackage{empheq}
\usepackage[most]{tcolorbox}

\newtcbox{\mymath}[1][]{%
    nobeforeafter, math upper, tcbox raise base,
    enhanced, colframe=blue!30!black,
    colback=red!10, boxrule=1pt,
    #1}

\usepackage{unixode}


\DeclareMathOperator{\ord}{ord}
\DeclareMathOperator{\orb}{orb}
\DeclareMathOperator{\stab}{stab}
\DeclareMathOperator{\Stab}{stab}
\DeclareMathOperator{\ppcm}{ppcm}
\DeclareMathOperator{\conj}{Conj}
\DeclareMathOperator{\End}{End}
\DeclareMathOperator{\rot}{rot}
\DeclareMathOperator{\trs}{trace}
\DeclareMathOperator{\Ind}{Ind}
\DeclareMathOperator{\mat}{Mat}
\DeclareMathOperator{\id}{Id}
\DeclareMathOperator{\vect}{vect}
\DeclareMathOperator{\img}{img}
\DeclareMathOperator{\cov}{Cov}
\DeclareMathOperator{\dist}{dist}
\DeclareMathOperator{\irr}{Irr}
\DeclareMathOperator{\image}{Im}
\DeclareMathOperator{\pd}{\partial}
\DeclareMathOperator{\epi}{epi}
\DeclareMathOperator{\Argmin}{Argmin}
\DeclareMathOperator{\dom}{dom}
\DeclareMathOperator{\proj}{proj}
\DeclareMathOperator{\ctg}{ctg}
\DeclareMathOperator{\supp}{supp}
\DeclareMathOperator{\argmin}{argmin}
\DeclareMathOperator{\mult}{mult}
\DeclareMathOperator{\ch}{ch}
\DeclareMathOperator{\sh}{sh}
\DeclareMathOperator{\rang}{rang}
\DeclareMathOperator{\diam}{diam}
\DeclareMathOperator{\Epigraphe}{Epigraphe}




\usepackage{xcolor}
\everymath{\color{blue}}
%\everymath{\color[rgb]{0,1,1}}
%\pagecolor[rgb]{0,0,0.5}


\newcommand*{\pdtest}[3][]{\ensuremath{\frac{\partial^{#1} #2}{\partial #3}}}

\newcommand*{\deffunc}[6][]{\ensuremath{
\begin{array}{rcl}
#2 : #3 &\rightarrow& #4\\
#5 &\mapsto& #6
\end{array}
}}

\newcommand{\eqcolon}{\mathrel{\resizebox{\widthof{$\mathord{=}$}}{\height}{ $\!\!=\!\!\resizebox{1.2\width}{0.8\height}{\raisebox{0.23ex}{$\mathop{:}$}}\!\!$ }}}
\newcommand{\coloneq}{\mathrel{\resizebox{\widthof{$\mathord{=}$}}{\height}{ $\!\!\resizebox{1.2\width}{0.8\height}{\raisebox{0.23ex}{$\mathop{:}$}}\!\!=\!\!$ }}}
\newcommand{\eqcolonl}{\ensuremath{\mathrel{=\!\!\mathop{:}}}}
\newcommand{\coloneql}{\ensuremath{\mathrel{\mathop{:} \!\! =}}}
\newcommand{\vc}[1]{% inline column vector
  \left(\begin{smallmatrix}#1\end{smallmatrix}\right)%
}
\newcommand{\vr}[1]{% inline row vector
  \begin{smallmatrix}(\,#1\,)\end{smallmatrix}%
}
\makeatletter
\newcommand*{\defeq}{\ =\mathrel{\rlap{%
                     \raisebox{0.3ex}{$\m@th\cdot$}}%
                     \raisebox{-0.3ex}{$\m@th\cdot$}}%
                     }
\makeatother

\newcommand{\mathcircle}[1]{% inline row vector
 \overset{\circ}{#1}
}
\newcommand{\ulim}{% low limit
 \underline{\lim}
}
\newcommand{\ssi}{% iff
\iff
}
\newcommand{\ps}[2]{
\expval{#1 | #2}
}
\newcommand{\df}[1]{
\mqty{#1}
}
\newcommand{\n}[1]{
\norm{#1}
}
\newcommand{\sys}[1]{
\left\{\smqty{#1}\right.
}


\newcommand{\eqdef}{\ensuremath{\overset{\text{def}}=}}


\def\Circlearrowright{\ensuremath{%
  \rotatebox[origin=c]{230}{$\circlearrowright$}}}

\newcommand\ct[1]{\text{\rmfamily\upshape #1}}
\newcommand\question[1]{ {\color{red} ...!? \small #1}}
\newcommand\caz[1]{\left\{\begin{array} #1 \end{array}\right.}
\newcommand\const{\text{\rmfamily\upshape const}}
\newcommand\toP{ \overset{\pro}{\to}}
\newcommand\toPP{ \overset{\text{PP}}{\to}}
\newcommand{\oeq}{\mathrel{\text{\textcircled{$=$}}}}





\usepackage{xcolor}
% \usepackage[normalem]{ulem}
\usepackage{lipsum}
\makeatletter
% \newcommand\colorwave[1][blue]{\bgroup \markoverwith{\lower3.5\p@\hbox{\sixly \textcolor{#1}{\char58}}}\ULon}
%\font\sixly=lasy6 % does not re-load if already loaded, so no memory problem.

\newmdtheoremenv[
linewidth= 1pt,linecolor= blue,%
leftmargin=20,rightmargin=20,innertopmargin=0pt, innerrightmargin=40,%
tikzsetting = { draw=lightgray, line width = 0.3pt,dashed,%
dash pattern = on 15pt off 3pt},%
splittopskip=\topskip,skipbelow=\baselineskip,%
skipabove=\baselineskip,ntheorem,roundcorner=0pt,
% backgroundcolor=pagebg,font=\color{orange}\sffamily, fontcolor=white
]{examplebox}{Exemple}[section]



\newcommand\R{\mathbb{R}}
\newcommand\Z{\mathbb{Z}}
\newcommand\N{\mathbb{N}}
\newcommand\E{\mathbb{E}}
\newcommand\F{\mathcal{F}}
\newcommand\cH{\mathcal{H}}
\newcommand\V{\mathbb{V}}
\newcommand\dmo{ ^{-1} }
\newcommand\kapa{\kappa}
\newcommand\im{Im}
\newcommand\hs{\mathcal{H}}





\usepackage{soul}

\makeatletter
\newcommand*{\whiten}[1]{\llap{\textcolor{white}{{\the\SOUL@token}}\hspace{#1pt}}}
\DeclareRobustCommand*\myul{%
    \def\SOUL@everyspace{\underline{\space}\kern\z@}%
    \def\SOUL@everytoken{%
     \setbox0=\hbox{\the\SOUL@token}%
     \ifdim\dp0>\z@
        \raisebox{\dp0}{\underline{\phantom{\the\SOUL@token}}}%
        \whiten{1}\whiten{0}%
        \whiten{-1}\whiten{-2}%
        \llap{\the\SOUL@token}%
     \else
        \underline{\the\SOUL@token}%
     \fi}%
\SOUL@}
\makeatother

\newcommand*{\demp}{\fontfamily{lmtt}\selectfont}

\DeclareTextFontCommand{\textdemp}{\demp}

\begin{document}

\ifcomment
Multiline
comment
\fi
\ifcomment
\myul{Typesetting test}
% \color[rgb]{1,1,1}
$∑_i^n≠ 60º±∞π∆¬≈√j∫h≤≥µ$

$\CR \R\pro\ind\pro\gS\pro
\mqty[a&b\\c&d]$
$\pro\mathbb{P}$
$\dd{x}$

  \[
    \alpha(x)=\left\{
                \begin{array}{ll}
                  x\\
                  \frac{1}{1+e^{-kx}}\\
                  \frac{e^x-e^{-x}}{e^x+e^{-x}}
                \end{array}
              \right.
  \]

  $\expval{x}$
  
  $\chi_\rho(ghg\dmo)=\Tr(\rho_{ghg\dmo})=\Tr(\rho_g\circ\rho_h\circ\rho\dmo_g)=\Tr(\rho_h)\overset{\mbox{\scalebox{0.5}{$\Tr(AB)=\Tr(BA)$}}}{=}\chi_\rho(h)$
  	$\mathop{\oplus}_{\substack{x\in X}}$

$\mat(\rho_g)=(a_{ij}(g))_{\scriptsize \substack{1\leq i\leq d \\ 1\leq j\leq d}}$ et $\mat(\rho'_g)=(a'_{ij}(g))_{\scriptsize \substack{1\leq i'\leq d' \\ 1\leq j'\leq d'}}$



\[\int_a^b{\mathbb{R}^2}g(u, v)\dd{P_{XY}}(u, v)=\iint g(u,v) f_{XY}(u, v)\dd \lambda(u) \dd \lambda(v)\]
$$\lim_{x\to\infty} f(x)$$	
$$\iiiint_V \mu(t,u,v,w) \,dt\,du\,dv\,dw$$
$$\sum_{n=1}^{\infty} 2^{-n} = 1$$	
\begin{definition}
	Si $X$ et $Y$ sont 2 v.a. ou definit la \textsc{Covariance} entre $X$ et $Y$ comme
	$\cov(X,Y)\overset{\text{def}}{=}\E\left[(X-\E(X))(Y-\E(Y))\right]=\E(XY)-\E(X)\E(Y)$.
\end{definition}
\fi
\pagebreak

% \tableofcontents

% insert your code here
%\input{./algebra/main.tex}
%\input{./geometrie-differentielle/main.tex}
%\input{./probabilite/main.tex}
%\input{./analyse-fonctionnelle/main.tex}
% \input{./Analyse-convexe-et-dualite-en-optimisation/main.tex}
%\input{./tikz/main.tex}
%\input{./Theorie-du-distributions/main.tex}
%\input{./optimisation/mine.tex}
 \input{./modelisation/main.tex}

% yves.aubry@univ-tln.fr : algebra

\end{document}


% yves.aubry@univ-tln.fr : algebra

\end{document}


% yves.aubry@univ-tln.fr : algebra

\end{document}

%% !TEX encoding = UTF-8 Unicode
% !TEX TS-program = xelatex

\documentclass[french]{report}

%\usepackage[utf8]{inputenc}
%\usepackage[T1]{fontenc}
\usepackage{babel}


\newif\ifcomment
%\commenttrue # Show comments

\usepackage{physics}
\usepackage{amssymb}


\usepackage{amsthm}
% \usepackage{thmtools}
\usepackage{mathtools}
\usepackage{amsfonts}

\usepackage{color}

\usepackage{tikz}

\usepackage{geometry}
\geometry{a5paper, margin=0.1in, right=1cm}

\usepackage{dsfont}

\usepackage{graphicx}
\graphicspath{ {images/} }

\usepackage{faktor}

\usepackage{IEEEtrantools}
\usepackage{enumerate}   
\usepackage[PostScript=dvips]{"/Users/aware/Documents/Courses/diagrams"}


\newtheorem{theorem}{Théorème}[section]
\renewcommand{\thetheorem}{\arabic{theorem}}
\newtheorem{lemme}{Lemme}[section]
\renewcommand{\thelemme}{\arabic{lemme}}
\newtheorem{proposition}{Proposition}[section]
\renewcommand{\theproposition}{\arabic{proposition}}
\newtheorem{notations}{Notations}[section]
\newtheorem{problem}{Problème}[section]
\newtheorem{corollary}{Corollaire}[theorem]
\renewcommand{\thecorollary}{\arabic{corollary}}
\newtheorem{property}{Propriété}[section]
\newtheorem{objective}{Objectif}[section]

\theoremstyle{definition}
\newtheorem{definition}{Définition}[section]
\renewcommand{\thedefinition}{\arabic{definition}}
\newtheorem{exercise}{Exercice}[chapter]
\renewcommand{\theexercise}{\arabic{exercise}}
\newtheorem{example}{Exemple}[chapter]
\renewcommand{\theexample}{\arabic{example}}
\newtheorem*{solution}{Solution}
\newtheorem*{application}{Application}
\newtheorem*{notation}{Notation}
\newtheorem*{vocabulary}{Vocabulaire}
\newtheorem*{properties}{Propriétés}



\theoremstyle{remark}
\newtheorem*{remark}{Remarque}
\newtheorem*{rappel}{Rappel}


\usepackage{etoolbox}
\AtBeginEnvironment{exercise}{\small}
\AtBeginEnvironment{example}{\small}

\usepackage{cases}
\usepackage[red]{mypack}

\usepackage[framemethod=TikZ]{mdframed}

\definecolor{bg}{rgb}{0.4,0.25,0.95}
\definecolor{pagebg}{rgb}{0,0,0.5}
\surroundwithmdframed[
   topline=false,
   rightline=false,
   bottomline=false,
   leftmargin=\parindent,
   skipabove=8pt,
   skipbelow=8pt,
   linecolor=blue,
   innerbottommargin=10pt,
   % backgroundcolor=bg,font=\color{orange}\sffamily, fontcolor=white
]{definition}

\usepackage{empheq}
\usepackage[most]{tcolorbox}

\newtcbox{\mymath}[1][]{%
    nobeforeafter, math upper, tcbox raise base,
    enhanced, colframe=blue!30!black,
    colback=red!10, boxrule=1pt,
    #1}

\usepackage{unixode}


\DeclareMathOperator{\ord}{ord}
\DeclareMathOperator{\orb}{orb}
\DeclareMathOperator{\stab}{stab}
\DeclareMathOperator{\Stab}{stab}
\DeclareMathOperator{\ppcm}{ppcm}
\DeclareMathOperator{\conj}{Conj}
\DeclareMathOperator{\End}{End}
\DeclareMathOperator{\rot}{rot}
\DeclareMathOperator{\trs}{trace}
\DeclareMathOperator{\Ind}{Ind}
\DeclareMathOperator{\mat}{Mat}
\DeclareMathOperator{\id}{Id}
\DeclareMathOperator{\vect}{vect}
\DeclareMathOperator{\img}{img}
\DeclareMathOperator{\cov}{Cov}
\DeclareMathOperator{\dist}{dist}
\DeclareMathOperator{\irr}{Irr}
\DeclareMathOperator{\image}{Im}
\DeclareMathOperator{\pd}{\partial}
\DeclareMathOperator{\epi}{epi}
\DeclareMathOperator{\Argmin}{Argmin}
\DeclareMathOperator{\dom}{dom}
\DeclareMathOperator{\proj}{proj}
\DeclareMathOperator{\ctg}{ctg}
\DeclareMathOperator{\supp}{supp}
\DeclareMathOperator{\argmin}{argmin}
\DeclareMathOperator{\mult}{mult}
\DeclareMathOperator{\ch}{ch}
\DeclareMathOperator{\sh}{sh}
\DeclareMathOperator{\rang}{rang}
\DeclareMathOperator{\diam}{diam}
\DeclareMathOperator{\Epigraphe}{Epigraphe}




\usepackage{xcolor}
\everymath{\color{blue}}
%\everymath{\color[rgb]{0,1,1}}
%\pagecolor[rgb]{0,0,0.5}


\newcommand*{\pdtest}[3][]{\ensuremath{\frac{\partial^{#1} #2}{\partial #3}}}

\newcommand*{\deffunc}[6][]{\ensuremath{
\begin{array}{rcl}
#2 : #3 &\rightarrow& #4\\
#5 &\mapsto& #6
\end{array}
}}

\newcommand{\eqcolon}{\mathrel{\resizebox{\widthof{$\mathord{=}$}}{\height}{ $\!\!=\!\!\resizebox{1.2\width}{0.8\height}{\raisebox{0.23ex}{$\mathop{:}$}}\!\!$ }}}
\newcommand{\coloneq}{\mathrel{\resizebox{\widthof{$\mathord{=}$}}{\height}{ $\!\!\resizebox{1.2\width}{0.8\height}{\raisebox{0.23ex}{$\mathop{:}$}}\!\!=\!\!$ }}}
\newcommand{\eqcolonl}{\ensuremath{\mathrel{=\!\!\mathop{:}}}}
\newcommand{\coloneql}{\ensuremath{\mathrel{\mathop{:} \!\! =}}}
\newcommand{\vc}[1]{% inline column vector
  \left(\begin{smallmatrix}#1\end{smallmatrix}\right)%
}
\newcommand{\vr}[1]{% inline row vector
  \begin{smallmatrix}(\,#1\,)\end{smallmatrix}%
}
\makeatletter
\newcommand*{\defeq}{\ =\mathrel{\rlap{%
                     \raisebox{0.3ex}{$\m@th\cdot$}}%
                     \raisebox{-0.3ex}{$\m@th\cdot$}}%
                     }
\makeatother

\newcommand{\mathcircle}[1]{% inline row vector
 \overset{\circ}{#1}
}
\newcommand{\ulim}{% low limit
 \underline{\lim}
}
\newcommand{\ssi}{% iff
\iff
}
\newcommand{\ps}[2]{
\expval{#1 | #2}
}
\newcommand{\df}[1]{
\mqty{#1}
}
\newcommand{\n}[1]{
\norm{#1}
}
\newcommand{\sys}[1]{
\left\{\smqty{#1}\right.
}


\newcommand{\eqdef}{\ensuremath{\overset{\text{def}}=}}


\def\Circlearrowright{\ensuremath{%
  \rotatebox[origin=c]{230}{$\circlearrowright$}}}

\newcommand\ct[1]{\text{\rmfamily\upshape #1}}
\newcommand\question[1]{ {\color{red} ...!? \small #1}}
\newcommand\caz[1]{\left\{\begin{array} #1 \end{array}\right.}
\newcommand\const{\text{\rmfamily\upshape const}}
\newcommand\toP{ \overset{\pro}{\to}}
\newcommand\toPP{ \overset{\text{PP}}{\to}}
\newcommand{\oeq}{\mathrel{\text{\textcircled{$=$}}}}





\usepackage{xcolor}
% \usepackage[normalem]{ulem}
\usepackage{lipsum}
\makeatletter
% \newcommand\colorwave[1][blue]{\bgroup \markoverwith{\lower3.5\p@\hbox{\sixly \textcolor{#1}{\char58}}}\ULon}
%\font\sixly=lasy6 % does not re-load if already loaded, so no memory problem.

\newmdtheoremenv[
linewidth= 1pt,linecolor= blue,%
leftmargin=20,rightmargin=20,innertopmargin=0pt, innerrightmargin=40,%
tikzsetting = { draw=lightgray, line width = 0.3pt,dashed,%
dash pattern = on 15pt off 3pt},%
splittopskip=\topskip,skipbelow=\baselineskip,%
skipabove=\baselineskip,ntheorem,roundcorner=0pt,
% backgroundcolor=pagebg,font=\color{orange}\sffamily, fontcolor=white
]{examplebox}{Exemple}[section]



\newcommand\R{\mathbb{R}}
\newcommand\Z{\mathbb{Z}}
\newcommand\N{\mathbb{N}}
\newcommand\E{\mathbb{E}}
\newcommand\F{\mathcal{F}}
\newcommand\cH{\mathcal{H}}
\newcommand\V{\mathbb{V}}
\newcommand\dmo{ ^{-1} }
\newcommand\kapa{\kappa}
\newcommand\im{Im}
\newcommand\hs{\mathcal{H}}





\usepackage{soul}

\makeatletter
\newcommand*{\whiten}[1]{\llap{\textcolor{white}{{\the\SOUL@token}}\hspace{#1pt}}}
\DeclareRobustCommand*\myul{%
    \def\SOUL@everyspace{\underline{\space}\kern\z@}%
    \def\SOUL@everytoken{%
     \setbox0=\hbox{\the\SOUL@token}%
     \ifdim\dp0>\z@
        \raisebox{\dp0}{\underline{\phantom{\the\SOUL@token}}}%
        \whiten{1}\whiten{0}%
        \whiten{-1}\whiten{-2}%
        \llap{\the\SOUL@token}%
     \else
        \underline{\the\SOUL@token}%
     \fi}%
\SOUL@}
\makeatother

\newcommand*{\demp}{\fontfamily{lmtt}\selectfont}

\DeclareTextFontCommand{\textdemp}{\demp}

\begin{document}

\ifcomment
Multiline
comment
\fi
\ifcomment
\myul{Typesetting test}
% \color[rgb]{1,1,1}
$∑_i^n≠ 60º±∞π∆¬≈√j∫h≤≥µ$

$\CR \R\pro\ind\pro\gS\pro
\mqty[a&b\\c&d]$
$\pro\mathbb{P}$
$\dd{x}$

  \[
    \alpha(x)=\left\{
                \begin{array}{ll}
                  x\\
                  \frac{1}{1+e^{-kx}}\\
                  \frac{e^x-e^{-x}}{e^x+e^{-x}}
                \end{array}
              \right.
  \]

  $\expval{x}$
  
  $\chi_\rho(ghg\dmo)=\Tr(\rho_{ghg\dmo})=\Tr(\rho_g\circ\rho_h\circ\rho\dmo_g)=\Tr(\rho_h)\overset{\mbox{\scalebox{0.5}{$\Tr(AB)=\Tr(BA)$}}}{=}\chi_\rho(h)$
  	$\mathop{\oplus}_{\substack{x\in X}}$

$\mat(\rho_g)=(a_{ij}(g))_{\scriptsize \substack{1\leq i\leq d \\ 1\leq j\leq d}}$ et $\mat(\rho'_g)=(a'_{ij}(g))_{\scriptsize \substack{1\leq i'\leq d' \\ 1\leq j'\leq d'}}$



\[\int_a^b{\mathbb{R}^2}g(u, v)\dd{P_{XY}}(u, v)=\iint g(u,v) f_{XY}(u, v)\dd \lambda(u) \dd \lambda(v)\]
$$\lim_{x\to\infty} f(x)$$	
$$\iiiint_V \mu(t,u,v,w) \,dt\,du\,dv\,dw$$
$$\sum_{n=1}^{\infty} 2^{-n} = 1$$	
\begin{definition}
	Si $X$ et $Y$ sont 2 v.a. ou definit la \textsc{Covariance} entre $X$ et $Y$ comme
	$\cov(X,Y)\overset{\text{def}}{=}\E\left[(X-\E(X))(Y-\E(Y))\right]=\E(XY)-\E(X)\E(Y)$.
\end{definition}
\fi
\pagebreak

% \tableofcontents

% insert your code here
%% !TEX encoding = UTF-8 Unicode
% !TEX TS-program = xelatex

\documentclass[french]{report}

%\usepackage[utf8]{inputenc}
%\usepackage[T1]{fontenc}
\usepackage{babel}


\newif\ifcomment
%\commenttrue # Show comments

\usepackage{physics}
\usepackage{amssymb}


\usepackage{amsthm}
% \usepackage{thmtools}
\usepackage{mathtools}
\usepackage{amsfonts}

\usepackage{color}

\usepackage{tikz}

\usepackage{geometry}
\geometry{a5paper, margin=0.1in, right=1cm}

\usepackage{dsfont}

\usepackage{graphicx}
\graphicspath{ {images/} }

\usepackage{faktor}

\usepackage{IEEEtrantools}
\usepackage{enumerate}   
\usepackage[PostScript=dvips]{"/Users/aware/Documents/Courses/diagrams"}


\newtheorem{theorem}{Théorème}[section]
\renewcommand{\thetheorem}{\arabic{theorem}}
\newtheorem{lemme}{Lemme}[section]
\renewcommand{\thelemme}{\arabic{lemme}}
\newtheorem{proposition}{Proposition}[section]
\renewcommand{\theproposition}{\arabic{proposition}}
\newtheorem{notations}{Notations}[section]
\newtheorem{problem}{Problème}[section]
\newtheorem{corollary}{Corollaire}[theorem]
\renewcommand{\thecorollary}{\arabic{corollary}}
\newtheorem{property}{Propriété}[section]
\newtheorem{objective}{Objectif}[section]

\theoremstyle{definition}
\newtheorem{definition}{Définition}[section]
\renewcommand{\thedefinition}{\arabic{definition}}
\newtheorem{exercise}{Exercice}[chapter]
\renewcommand{\theexercise}{\arabic{exercise}}
\newtheorem{example}{Exemple}[chapter]
\renewcommand{\theexample}{\arabic{example}}
\newtheorem*{solution}{Solution}
\newtheorem*{application}{Application}
\newtheorem*{notation}{Notation}
\newtheorem*{vocabulary}{Vocabulaire}
\newtheorem*{properties}{Propriétés}



\theoremstyle{remark}
\newtheorem*{remark}{Remarque}
\newtheorem*{rappel}{Rappel}


\usepackage{etoolbox}
\AtBeginEnvironment{exercise}{\small}
\AtBeginEnvironment{example}{\small}

\usepackage{cases}
\usepackage[red]{mypack}

\usepackage[framemethod=TikZ]{mdframed}

\definecolor{bg}{rgb}{0.4,0.25,0.95}
\definecolor{pagebg}{rgb}{0,0,0.5}
\surroundwithmdframed[
   topline=false,
   rightline=false,
   bottomline=false,
   leftmargin=\parindent,
   skipabove=8pt,
   skipbelow=8pt,
   linecolor=blue,
   innerbottommargin=10pt,
   % backgroundcolor=bg,font=\color{orange}\sffamily, fontcolor=white
]{definition}

\usepackage{empheq}
\usepackage[most]{tcolorbox}

\newtcbox{\mymath}[1][]{%
    nobeforeafter, math upper, tcbox raise base,
    enhanced, colframe=blue!30!black,
    colback=red!10, boxrule=1pt,
    #1}

\usepackage{unixode}


\DeclareMathOperator{\ord}{ord}
\DeclareMathOperator{\orb}{orb}
\DeclareMathOperator{\stab}{stab}
\DeclareMathOperator{\Stab}{stab}
\DeclareMathOperator{\ppcm}{ppcm}
\DeclareMathOperator{\conj}{Conj}
\DeclareMathOperator{\End}{End}
\DeclareMathOperator{\rot}{rot}
\DeclareMathOperator{\trs}{trace}
\DeclareMathOperator{\Ind}{Ind}
\DeclareMathOperator{\mat}{Mat}
\DeclareMathOperator{\id}{Id}
\DeclareMathOperator{\vect}{vect}
\DeclareMathOperator{\img}{img}
\DeclareMathOperator{\cov}{Cov}
\DeclareMathOperator{\dist}{dist}
\DeclareMathOperator{\irr}{Irr}
\DeclareMathOperator{\image}{Im}
\DeclareMathOperator{\pd}{\partial}
\DeclareMathOperator{\epi}{epi}
\DeclareMathOperator{\Argmin}{Argmin}
\DeclareMathOperator{\dom}{dom}
\DeclareMathOperator{\proj}{proj}
\DeclareMathOperator{\ctg}{ctg}
\DeclareMathOperator{\supp}{supp}
\DeclareMathOperator{\argmin}{argmin}
\DeclareMathOperator{\mult}{mult}
\DeclareMathOperator{\ch}{ch}
\DeclareMathOperator{\sh}{sh}
\DeclareMathOperator{\rang}{rang}
\DeclareMathOperator{\diam}{diam}
\DeclareMathOperator{\Epigraphe}{Epigraphe}




\usepackage{xcolor}
\everymath{\color{blue}}
%\everymath{\color[rgb]{0,1,1}}
%\pagecolor[rgb]{0,0,0.5}


\newcommand*{\pdtest}[3][]{\ensuremath{\frac{\partial^{#1} #2}{\partial #3}}}

\newcommand*{\deffunc}[6][]{\ensuremath{
\begin{array}{rcl}
#2 : #3 &\rightarrow& #4\\
#5 &\mapsto& #6
\end{array}
}}

\newcommand{\eqcolon}{\mathrel{\resizebox{\widthof{$\mathord{=}$}}{\height}{ $\!\!=\!\!\resizebox{1.2\width}{0.8\height}{\raisebox{0.23ex}{$\mathop{:}$}}\!\!$ }}}
\newcommand{\coloneq}{\mathrel{\resizebox{\widthof{$\mathord{=}$}}{\height}{ $\!\!\resizebox{1.2\width}{0.8\height}{\raisebox{0.23ex}{$\mathop{:}$}}\!\!=\!\!$ }}}
\newcommand{\eqcolonl}{\ensuremath{\mathrel{=\!\!\mathop{:}}}}
\newcommand{\coloneql}{\ensuremath{\mathrel{\mathop{:} \!\! =}}}
\newcommand{\vc}[1]{% inline column vector
  \left(\begin{smallmatrix}#1\end{smallmatrix}\right)%
}
\newcommand{\vr}[1]{% inline row vector
  \begin{smallmatrix}(\,#1\,)\end{smallmatrix}%
}
\makeatletter
\newcommand*{\defeq}{\ =\mathrel{\rlap{%
                     \raisebox{0.3ex}{$\m@th\cdot$}}%
                     \raisebox{-0.3ex}{$\m@th\cdot$}}%
                     }
\makeatother

\newcommand{\mathcircle}[1]{% inline row vector
 \overset{\circ}{#1}
}
\newcommand{\ulim}{% low limit
 \underline{\lim}
}
\newcommand{\ssi}{% iff
\iff
}
\newcommand{\ps}[2]{
\expval{#1 | #2}
}
\newcommand{\df}[1]{
\mqty{#1}
}
\newcommand{\n}[1]{
\norm{#1}
}
\newcommand{\sys}[1]{
\left\{\smqty{#1}\right.
}


\newcommand{\eqdef}{\ensuremath{\overset{\text{def}}=}}


\def\Circlearrowright{\ensuremath{%
  \rotatebox[origin=c]{230}{$\circlearrowright$}}}

\newcommand\ct[1]{\text{\rmfamily\upshape #1}}
\newcommand\question[1]{ {\color{red} ...!? \small #1}}
\newcommand\caz[1]{\left\{\begin{array} #1 \end{array}\right.}
\newcommand\const{\text{\rmfamily\upshape const}}
\newcommand\toP{ \overset{\pro}{\to}}
\newcommand\toPP{ \overset{\text{PP}}{\to}}
\newcommand{\oeq}{\mathrel{\text{\textcircled{$=$}}}}





\usepackage{xcolor}
% \usepackage[normalem]{ulem}
\usepackage{lipsum}
\makeatletter
% \newcommand\colorwave[1][blue]{\bgroup \markoverwith{\lower3.5\p@\hbox{\sixly \textcolor{#1}{\char58}}}\ULon}
%\font\sixly=lasy6 % does not re-load if already loaded, so no memory problem.

\newmdtheoremenv[
linewidth= 1pt,linecolor= blue,%
leftmargin=20,rightmargin=20,innertopmargin=0pt, innerrightmargin=40,%
tikzsetting = { draw=lightgray, line width = 0.3pt,dashed,%
dash pattern = on 15pt off 3pt},%
splittopskip=\topskip,skipbelow=\baselineskip,%
skipabove=\baselineskip,ntheorem,roundcorner=0pt,
% backgroundcolor=pagebg,font=\color{orange}\sffamily, fontcolor=white
]{examplebox}{Exemple}[section]



\newcommand\R{\mathbb{R}}
\newcommand\Z{\mathbb{Z}}
\newcommand\N{\mathbb{N}}
\newcommand\E{\mathbb{E}}
\newcommand\F{\mathcal{F}}
\newcommand\cH{\mathcal{H}}
\newcommand\V{\mathbb{V}}
\newcommand\dmo{ ^{-1} }
\newcommand\kapa{\kappa}
\newcommand\im{Im}
\newcommand\hs{\mathcal{H}}





\usepackage{soul}

\makeatletter
\newcommand*{\whiten}[1]{\llap{\textcolor{white}{{\the\SOUL@token}}\hspace{#1pt}}}
\DeclareRobustCommand*\myul{%
    \def\SOUL@everyspace{\underline{\space}\kern\z@}%
    \def\SOUL@everytoken{%
     \setbox0=\hbox{\the\SOUL@token}%
     \ifdim\dp0>\z@
        \raisebox{\dp0}{\underline{\phantom{\the\SOUL@token}}}%
        \whiten{1}\whiten{0}%
        \whiten{-1}\whiten{-2}%
        \llap{\the\SOUL@token}%
     \else
        \underline{\the\SOUL@token}%
     \fi}%
\SOUL@}
\makeatother

\newcommand*{\demp}{\fontfamily{lmtt}\selectfont}

\DeclareTextFontCommand{\textdemp}{\demp}

\begin{document}

\ifcomment
Multiline
comment
\fi
\ifcomment
\myul{Typesetting test}
% \color[rgb]{1,1,1}
$∑_i^n≠ 60º±∞π∆¬≈√j∫h≤≥µ$

$\CR \R\pro\ind\pro\gS\pro
\mqty[a&b\\c&d]$
$\pro\mathbb{P}$
$\dd{x}$

  \[
    \alpha(x)=\left\{
                \begin{array}{ll}
                  x\\
                  \frac{1}{1+e^{-kx}}\\
                  \frac{e^x-e^{-x}}{e^x+e^{-x}}
                \end{array}
              \right.
  \]

  $\expval{x}$
  
  $\chi_\rho(ghg\dmo)=\Tr(\rho_{ghg\dmo})=\Tr(\rho_g\circ\rho_h\circ\rho\dmo_g)=\Tr(\rho_h)\overset{\mbox{\scalebox{0.5}{$\Tr(AB)=\Tr(BA)$}}}{=}\chi_\rho(h)$
  	$\mathop{\oplus}_{\substack{x\in X}}$

$\mat(\rho_g)=(a_{ij}(g))_{\scriptsize \substack{1\leq i\leq d \\ 1\leq j\leq d}}$ et $\mat(\rho'_g)=(a'_{ij}(g))_{\scriptsize \substack{1\leq i'\leq d' \\ 1\leq j'\leq d'}}$



\[\int_a^b{\mathbb{R}^2}g(u, v)\dd{P_{XY}}(u, v)=\iint g(u,v) f_{XY}(u, v)\dd \lambda(u) \dd \lambda(v)\]
$$\lim_{x\to\infty} f(x)$$	
$$\iiiint_V \mu(t,u,v,w) \,dt\,du\,dv\,dw$$
$$\sum_{n=1}^{\infty} 2^{-n} = 1$$	
\begin{definition}
	Si $X$ et $Y$ sont 2 v.a. ou definit la \textsc{Covariance} entre $X$ et $Y$ comme
	$\cov(X,Y)\overset{\text{def}}{=}\E\left[(X-\E(X))(Y-\E(Y))\right]=\E(XY)-\E(X)\E(Y)$.
\end{definition}
\fi
\pagebreak

% \tableofcontents

% insert your code here
%% !TEX encoding = UTF-8 Unicode
% !TEX TS-program = xelatex

\documentclass[french]{report}

%\usepackage[utf8]{inputenc}
%\usepackage[T1]{fontenc}
\usepackage{babel}


\newif\ifcomment
%\commenttrue # Show comments

\usepackage{physics}
\usepackage{amssymb}


\usepackage{amsthm}
% \usepackage{thmtools}
\usepackage{mathtools}
\usepackage{amsfonts}

\usepackage{color}

\usepackage{tikz}

\usepackage{geometry}
\geometry{a5paper, margin=0.1in, right=1cm}

\usepackage{dsfont}

\usepackage{graphicx}
\graphicspath{ {images/} }

\usepackage{faktor}

\usepackage{IEEEtrantools}
\usepackage{enumerate}   
\usepackage[PostScript=dvips]{"/Users/aware/Documents/Courses/diagrams"}


\newtheorem{theorem}{Théorème}[section]
\renewcommand{\thetheorem}{\arabic{theorem}}
\newtheorem{lemme}{Lemme}[section]
\renewcommand{\thelemme}{\arabic{lemme}}
\newtheorem{proposition}{Proposition}[section]
\renewcommand{\theproposition}{\arabic{proposition}}
\newtheorem{notations}{Notations}[section]
\newtheorem{problem}{Problème}[section]
\newtheorem{corollary}{Corollaire}[theorem]
\renewcommand{\thecorollary}{\arabic{corollary}}
\newtheorem{property}{Propriété}[section]
\newtheorem{objective}{Objectif}[section]

\theoremstyle{definition}
\newtheorem{definition}{Définition}[section]
\renewcommand{\thedefinition}{\arabic{definition}}
\newtheorem{exercise}{Exercice}[chapter]
\renewcommand{\theexercise}{\arabic{exercise}}
\newtheorem{example}{Exemple}[chapter]
\renewcommand{\theexample}{\arabic{example}}
\newtheorem*{solution}{Solution}
\newtheorem*{application}{Application}
\newtheorem*{notation}{Notation}
\newtheorem*{vocabulary}{Vocabulaire}
\newtheorem*{properties}{Propriétés}



\theoremstyle{remark}
\newtheorem*{remark}{Remarque}
\newtheorem*{rappel}{Rappel}


\usepackage{etoolbox}
\AtBeginEnvironment{exercise}{\small}
\AtBeginEnvironment{example}{\small}

\usepackage{cases}
\usepackage[red]{mypack}

\usepackage[framemethod=TikZ]{mdframed}

\definecolor{bg}{rgb}{0.4,0.25,0.95}
\definecolor{pagebg}{rgb}{0,0,0.5}
\surroundwithmdframed[
   topline=false,
   rightline=false,
   bottomline=false,
   leftmargin=\parindent,
   skipabove=8pt,
   skipbelow=8pt,
   linecolor=blue,
   innerbottommargin=10pt,
   % backgroundcolor=bg,font=\color{orange}\sffamily, fontcolor=white
]{definition}

\usepackage{empheq}
\usepackage[most]{tcolorbox}

\newtcbox{\mymath}[1][]{%
    nobeforeafter, math upper, tcbox raise base,
    enhanced, colframe=blue!30!black,
    colback=red!10, boxrule=1pt,
    #1}

\usepackage{unixode}


\DeclareMathOperator{\ord}{ord}
\DeclareMathOperator{\orb}{orb}
\DeclareMathOperator{\stab}{stab}
\DeclareMathOperator{\Stab}{stab}
\DeclareMathOperator{\ppcm}{ppcm}
\DeclareMathOperator{\conj}{Conj}
\DeclareMathOperator{\End}{End}
\DeclareMathOperator{\rot}{rot}
\DeclareMathOperator{\trs}{trace}
\DeclareMathOperator{\Ind}{Ind}
\DeclareMathOperator{\mat}{Mat}
\DeclareMathOperator{\id}{Id}
\DeclareMathOperator{\vect}{vect}
\DeclareMathOperator{\img}{img}
\DeclareMathOperator{\cov}{Cov}
\DeclareMathOperator{\dist}{dist}
\DeclareMathOperator{\irr}{Irr}
\DeclareMathOperator{\image}{Im}
\DeclareMathOperator{\pd}{\partial}
\DeclareMathOperator{\epi}{epi}
\DeclareMathOperator{\Argmin}{Argmin}
\DeclareMathOperator{\dom}{dom}
\DeclareMathOperator{\proj}{proj}
\DeclareMathOperator{\ctg}{ctg}
\DeclareMathOperator{\supp}{supp}
\DeclareMathOperator{\argmin}{argmin}
\DeclareMathOperator{\mult}{mult}
\DeclareMathOperator{\ch}{ch}
\DeclareMathOperator{\sh}{sh}
\DeclareMathOperator{\rang}{rang}
\DeclareMathOperator{\diam}{diam}
\DeclareMathOperator{\Epigraphe}{Epigraphe}




\usepackage{xcolor}
\everymath{\color{blue}}
%\everymath{\color[rgb]{0,1,1}}
%\pagecolor[rgb]{0,0,0.5}


\newcommand*{\pdtest}[3][]{\ensuremath{\frac{\partial^{#1} #2}{\partial #3}}}

\newcommand*{\deffunc}[6][]{\ensuremath{
\begin{array}{rcl}
#2 : #3 &\rightarrow& #4\\
#5 &\mapsto& #6
\end{array}
}}

\newcommand{\eqcolon}{\mathrel{\resizebox{\widthof{$\mathord{=}$}}{\height}{ $\!\!=\!\!\resizebox{1.2\width}{0.8\height}{\raisebox{0.23ex}{$\mathop{:}$}}\!\!$ }}}
\newcommand{\coloneq}{\mathrel{\resizebox{\widthof{$\mathord{=}$}}{\height}{ $\!\!\resizebox{1.2\width}{0.8\height}{\raisebox{0.23ex}{$\mathop{:}$}}\!\!=\!\!$ }}}
\newcommand{\eqcolonl}{\ensuremath{\mathrel{=\!\!\mathop{:}}}}
\newcommand{\coloneql}{\ensuremath{\mathrel{\mathop{:} \!\! =}}}
\newcommand{\vc}[1]{% inline column vector
  \left(\begin{smallmatrix}#1\end{smallmatrix}\right)%
}
\newcommand{\vr}[1]{% inline row vector
  \begin{smallmatrix}(\,#1\,)\end{smallmatrix}%
}
\makeatletter
\newcommand*{\defeq}{\ =\mathrel{\rlap{%
                     \raisebox{0.3ex}{$\m@th\cdot$}}%
                     \raisebox{-0.3ex}{$\m@th\cdot$}}%
                     }
\makeatother

\newcommand{\mathcircle}[1]{% inline row vector
 \overset{\circ}{#1}
}
\newcommand{\ulim}{% low limit
 \underline{\lim}
}
\newcommand{\ssi}{% iff
\iff
}
\newcommand{\ps}[2]{
\expval{#1 | #2}
}
\newcommand{\df}[1]{
\mqty{#1}
}
\newcommand{\n}[1]{
\norm{#1}
}
\newcommand{\sys}[1]{
\left\{\smqty{#1}\right.
}


\newcommand{\eqdef}{\ensuremath{\overset{\text{def}}=}}


\def\Circlearrowright{\ensuremath{%
  \rotatebox[origin=c]{230}{$\circlearrowright$}}}

\newcommand\ct[1]{\text{\rmfamily\upshape #1}}
\newcommand\question[1]{ {\color{red} ...!? \small #1}}
\newcommand\caz[1]{\left\{\begin{array} #1 \end{array}\right.}
\newcommand\const{\text{\rmfamily\upshape const}}
\newcommand\toP{ \overset{\pro}{\to}}
\newcommand\toPP{ \overset{\text{PP}}{\to}}
\newcommand{\oeq}{\mathrel{\text{\textcircled{$=$}}}}





\usepackage{xcolor}
% \usepackage[normalem]{ulem}
\usepackage{lipsum}
\makeatletter
% \newcommand\colorwave[1][blue]{\bgroup \markoverwith{\lower3.5\p@\hbox{\sixly \textcolor{#1}{\char58}}}\ULon}
%\font\sixly=lasy6 % does not re-load if already loaded, so no memory problem.

\newmdtheoremenv[
linewidth= 1pt,linecolor= blue,%
leftmargin=20,rightmargin=20,innertopmargin=0pt, innerrightmargin=40,%
tikzsetting = { draw=lightgray, line width = 0.3pt,dashed,%
dash pattern = on 15pt off 3pt},%
splittopskip=\topskip,skipbelow=\baselineskip,%
skipabove=\baselineskip,ntheorem,roundcorner=0pt,
% backgroundcolor=pagebg,font=\color{orange}\sffamily, fontcolor=white
]{examplebox}{Exemple}[section]



\newcommand\R{\mathbb{R}}
\newcommand\Z{\mathbb{Z}}
\newcommand\N{\mathbb{N}}
\newcommand\E{\mathbb{E}}
\newcommand\F{\mathcal{F}}
\newcommand\cH{\mathcal{H}}
\newcommand\V{\mathbb{V}}
\newcommand\dmo{ ^{-1} }
\newcommand\kapa{\kappa}
\newcommand\im{Im}
\newcommand\hs{\mathcal{H}}





\usepackage{soul}

\makeatletter
\newcommand*{\whiten}[1]{\llap{\textcolor{white}{{\the\SOUL@token}}\hspace{#1pt}}}
\DeclareRobustCommand*\myul{%
    \def\SOUL@everyspace{\underline{\space}\kern\z@}%
    \def\SOUL@everytoken{%
     \setbox0=\hbox{\the\SOUL@token}%
     \ifdim\dp0>\z@
        \raisebox{\dp0}{\underline{\phantom{\the\SOUL@token}}}%
        \whiten{1}\whiten{0}%
        \whiten{-1}\whiten{-2}%
        \llap{\the\SOUL@token}%
     \else
        \underline{\the\SOUL@token}%
     \fi}%
\SOUL@}
\makeatother

\newcommand*{\demp}{\fontfamily{lmtt}\selectfont}

\DeclareTextFontCommand{\textdemp}{\demp}

\begin{document}

\ifcomment
Multiline
comment
\fi
\ifcomment
\myul{Typesetting test}
% \color[rgb]{1,1,1}
$∑_i^n≠ 60º±∞π∆¬≈√j∫h≤≥µ$

$\CR \R\pro\ind\pro\gS\pro
\mqty[a&b\\c&d]$
$\pro\mathbb{P}$
$\dd{x}$

  \[
    \alpha(x)=\left\{
                \begin{array}{ll}
                  x\\
                  \frac{1}{1+e^{-kx}}\\
                  \frac{e^x-e^{-x}}{e^x+e^{-x}}
                \end{array}
              \right.
  \]

  $\expval{x}$
  
  $\chi_\rho(ghg\dmo)=\Tr(\rho_{ghg\dmo})=\Tr(\rho_g\circ\rho_h\circ\rho\dmo_g)=\Tr(\rho_h)\overset{\mbox{\scalebox{0.5}{$\Tr(AB)=\Tr(BA)$}}}{=}\chi_\rho(h)$
  	$\mathop{\oplus}_{\substack{x\in X}}$

$\mat(\rho_g)=(a_{ij}(g))_{\scriptsize \substack{1\leq i\leq d \\ 1\leq j\leq d}}$ et $\mat(\rho'_g)=(a'_{ij}(g))_{\scriptsize \substack{1\leq i'\leq d' \\ 1\leq j'\leq d'}}$



\[\int_a^b{\mathbb{R}^2}g(u, v)\dd{P_{XY}}(u, v)=\iint g(u,v) f_{XY}(u, v)\dd \lambda(u) \dd \lambda(v)\]
$$\lim_{x\to\infty} f(x)$$	
$$\iiiint_V \mu(t,u,v,w) \,dt\,du\,dv\,dw$$
$$\sum_{n=1}^{\infty} 2^{-n} = 1$$	
\begin{definition}
	Si $X$ et $Y$ sont 2 v.a. ou definit la \textsc{Covariance} entre $X$ et $Y$ comme
	$\cov(X,Y)\overset{\text{def}}{=}\E\left[(X-\E(X))(Y-\E(Y))\right]=\E(XY)-\E(X)\E(Y)$.
\end{definition}
\fi
\pagebreak

% \tableofcontents

% insert your code here
%\input{./algebra/main.tex}
%\input{./geometrie-differentielle/main.tex}
%\input{./probabilite/main.tex}
%\input{./analyse-fonctionnelle/main.tex}
% \input{./Analyse-convexe-et-dualite-en-optimisation/main.tex}
%\input{./tikz/main.tex}
%\input{./Theorie-du-distributions/main.tex}
%\input{./optimisation/mine.tex}
 \input{./modelisation/main.tex}

% yves.aubry@univ-tln.fr : algebra

\end{document}

%% !TEX encoding = UTF-8 Unicode
% !TEX TS-program = xelatex

\documentclass[french]{report}

%\usepackage[utf8]{inputenc}
%\usepackage[T1]{fontenc}
\usepackage{babel}


\newif\ifcomment
%\commenttrue # Show comments

\usepackage{physics}
\usepackage{amssymb}


\usepackage{amsthm}
% \usepackage{thmtools}
\usepackage{mathtools}
\usepackage{amsfonts}

\usepackage{color}

\usepackage{tikz}

\usepackage{geometry}
\geometry{a5paper, margin=0.1in, right=1cm}

\usepackage{dsfont}

\usepackage{graphicx}
\graphicspath{ {images/} }

\usepackage{faktor}

\usepackage{IEEEtrantools}
\usepackage{enumerate}   
\usepackage[PostScript=dvips]{"/Users/aware/Documents/Courses/diagrams"}


\newtheorem{theorem}{Théorème}[section]
\renewcommand{\thetheorem}{\arabic{theorem}}
\newtheorem{lemme}{Lemme}[section]
\renewcommand{\thelemme}{\arabic{lemme}}
\newtheorem{proposition}{Proposition}[section]
\renewcommand{\theproposition}{\arabic{proposition}}
\newtheorem{notations}{Notations}[section]
\newtheorem{problem}{Problème}[section]
\newtheorem{corollary}{Corollaire}[theorem]
\renewcommand{\thecorollary}{\arabic{corollary}}
\newtheorem{property}{Propriété}[section]
\newtheorem{objective}{Objectif}[section]

\theoremstyle{definition}
\newtheorem{definition}{Définition}[section]
\renewcommand{\thedefinition}{\arabic{definition}}
\newtheorem{exercise}{Exercice}[chapter]
\renewcommand{\theexercise}{\arabic{exercise}}
\newtheorem{example}{Exemple}[chapter]
\renewcommand{\theexample}{\arabic{example}}
\newtheorem*{solution}{Solution}
\newtheorem*{application}{Application}
\newtheorem*{notation}{Notation}
\newtheorem*{vocabulary}{Vocabulaire}
\newtheorem*{properties}{Propriétés}



\theoremstyle{remark}
\newtheorem*{remark}{Remarque}
\newtheorem*{rappel}{Rappel}


\usepackage{etoolbox}
\AtBeginEnvironment{exercise}{\small}
\AtBeginEnvironment{example}{\small}

\usepackage{cases}
\usepackage[red]{mypack}

\usepackage[framemethod=TikZ]{mdframed}

\definecolor{bg}{rgb}{0.4,0.25,0.95}
\definecolor{pagebg}{rgb}{0,0,0.5}
\surroundwithmdframed[
   topline=false,
   rightline=false,
   bottomline=false,
   leftmargin=\parindent,
   skipabove=8pt,
   skipbelow=8pt,
   linecolor=blue,
   innerbottommargin=10pt,
   % backgroundcolor=bg,font=\color{orange}\sffamily, fontcolor=white
]{definition}

\usepackage{empheq}
\usepackage[most]{tcolorbox}

\newtcbox{\mymath}[1][]{%
    nobeforeafter, math upper, tcbox raise base,
    enhanced, colframe=blue!30!black,
    colback=red!10, boxrule=1pt,
    #1}

\usepackage{unixode}


\DeclareMathOperator{\ord}{ord}
\DeclareMathOperator{\orb}{orb}
\DeclareMathOperator{\stab}{stab}
\DeclareMathOperator{\Stab}{stab}
\DeclareMathOperator{\ppcm}{ppcm}
\DeclareMathOperator{\conj}{Conj}
\DeclareMathOperator{\End}{End}
\DeclareMathOperator{\rot}{rot}
\DeclareMathOperator{\trs}{trace}
\DeclareMathOperator{\Ind}{Ind}
\DeclareMathOperator{\mat}{Mat}
\DeclareMathOperator{\id}{Id}
\DeclareMathOperator{\vect}{vect}
\DeclareMathOperator{\img}{img}
\DeclareMathOperator{\cov}{Cov}
\DeclareMathOperator{\dist}{dist}
\DeclareMathOperator{\irr}{Irr}
\DeclareMathOperator{\image}{Im}
\DeclareMathOperator{\pd}{\partial}
\DeclareMathOperator{\epi}{epi}
\DeclareMathOperator{\Argmin}{Argmin}
\DeclareMathOperator{\dom}{dom}
\DeclareMathOperator{\proj}{proj}
\DeclareMathOperator{\ctg}{ctg}
\DeclareMathOperator{\supp}{supp}
\DeclareMathOperator{\argmin}{argmin}
\DeclareMathOperator{\mult}{mult}
\DeclareMathOperator{\ch}{ch}
\DeclareMathOperator{\sh}{sh}
\DeclareMathOperator{\rang}{rang}
\DeclareMathOperator{\diam}{diam}
\DeclareMathOperator{\Epigraphe}{Epigraphe}




\usepackage{xcolor}
\everymath{\color{blue}}
%\everymath{\color[rgb]{0,1,1}}
%\pagecolor[rgb]{0,0,0.5}


\newcommand*{\pdtest}[3][]{\ensuremath{\frac{\partial^{#1} #2}{\partial #3}}}

\newcommand*{\deffunc}[6][]{\ensuremath{
\begin{array}{rcl}
#2 : #3 &\rightarrow& #4\\
#5 &\mapsto& #6
\end{array}
}}

\newcommand{\eqcolon}{\mathrel{\resizebox{\widthof{$\mathord{=}$}}{\height}{ $\!\!=\!\!\resizebox{1.2\width}{0.8\height}{\raisebox{0.23ex}{$\mathop{:}$}}\!\!$ }}}
\newcommand{\coloneq}{\mathrel{\resizebox{\widthof{$\mathord{=}$}}{\height}{ $\!\!\resizebox{1.2\width}{0.8\height}{\raisebox{0.23ex}{$\mathop{:}$}}\!\!=\!\!$ }}}
\newcommand{\eqcolonl}{\ensuremath{\mathrel{=\!\!\mathop{:}}}}
\newcommand{\coloneql}{\ensuremath{\mathrel{\mathop{:} \!\! =}}}
\newcommand{\vc}[1]{% inline column vector
  \left(\begin{smallmatrix}#1\end{smallmatrix}\right)%
}
\newcommand{\vr}[1]{% inline row vector
  \begin{smallmatrix}(\,#1\,)\end{smallmatrix}%
}
\makeatletter
\newcommand*{\defeq}{\ =\mathrel{\rlap{%
                     \raisebox{0.3ex}{$\m@th\cdot$}}%
                     \raisebox{-0.3ex}{$\m@th\cdot$}}%
                     }
\makeatother

\newcommand{\mathcircle}[1]{% inline row vector
 \overset{\circ}{#1}
}
\newcommand{\ulim}{% low limit
 \underline{\lim}
}
\newcommand{\ssi}{% iff
\iff
}
\newcommand{\ps}[2]{
\expval{#1 | #2}
}
\newcommand{\df}[1]{
\mqty{#1}
}
\newcommand{\n}[1]{
\norm{#1}
}
\newcommand{\sys}[1]{
\left\{\smqty{#1}\right.
}


\newcommand{\eqdef}{\ensuremath{\overset{\text{def}}=}}


\def\Circlearrowright{\ensuremath{%
  \rotatebox[origin=c]{230}{$\circlearrowright$}}}

\newcommand\ct[1]{\text{\rmfamily\upshape #1}}
\newcommand\question[1]{ {\color{red} ...!? \small #1}}
\newcommand\caz[1]{\left\{\begin{array} #1 \end{array}\right.}
\newcommand\const{\text{\rmfamily\upshape const}}
\newcommand\toP{ \overset{\pro}{\to}}
\newcommand\toPP{ \overset{\text{PP}}{\to}}
\newcommand{\oeq}{\mathrel{\text{\textcircled{$=$}}}}





\usepackage{xcolor}
% \usepackage[normalem]{ulem}
\usepackage{lipsum}
\makeatletter
% \newcommand\colorwave[1][blue]{\bgroup \markoverwith{\lower3.5\p@\hbox{\sixly \textcolor{#1}{\char58}}}\ULon}
%\font\sixly=lasy6 % does not re-load if already loaded, so no memory problem.

\newmdtheoremenv[
linewidth= 1pt,linecolor= blue,%
leftmargin=20,rightmargin=20,innertopmargin=0pt, innerrightmargin=40,%
tikzsetting = { draw=lightgray, line width = 0.3pt,dashed,%
dash pattern = on 15pt off 3pt},%
splittopskip=\topskip,skipbelow=\baselineskip,%
skipabove=\baselineskip,ntheorem,roundcorner=0pt,
% backgroundcolor=pagebg,font=\color{orange}\sffamily, fontcolor=white
]{examplebox}{Exemple}[section]



\newcommand\R{\mathbb{R}}
\newcommand\Z{\mathbb{Z}}
\newcommand\N{\mathbb{N}}
\newcommand\E{\mathbb{E}}
\newcommand\F{\mathcal{F}}
\newcommand\cH{\mathcal{H}}
\newcommand\V{\mathbb{V}}
\newcommand\dmo{ ^{-1} }
\newcommand\kapa{\kappa}
\newcommand\im{Im}
\newcommand\hs{\mathcal{H}}





\usepackage{soul}

\makeatletter
\newcommand*{\whiten}[1]{\llap{\textcolor{white}{{\the\SOUL@token}}\hspace{#1pt}}}
\DeclareRobustCommand*\myul{%
    \def\SOUL@everyspace{\underline{\space}\kern\z@}%
    \def\SOUL@everytoken{%
     \setbox0=\hbox{\the\SOUL@token}%
     \ifdim\dp0>\z@
        \raisebox{\dp0}{\underline{\phantom{\the\SOUL@token}}}%
        \whiten{1}\whiten{0}%
        \whiten{-1}\whiten{-2}%
        \llap{\the\SOUL@token}%
     \else
        \underline{\the\SOUL@token}%
     \fi}%
\SOUL@}
\makeatother

\newcommand*{\demp}{\fontfamily{lmtt}\selectfont}

\DeclareTextFontCommand{\textdemp}{\demp}

\begin{document}

\ifcomment
Multiline
comment
\fi
\ifcomment
\myul{Typesetting test}
% \color[rgb]{1,1,1}
$∑_i^n≠ 60º±∞π∆¬≈√j∫h≤≥µ$

$\CR \R\pro\ind\pro\gS\pro
\mqty[a&b\\c&d]$
$\pro\mathbb{P}$
$\dd{x}$

  \[
    \alpha(x)=\left\{
                \begin{array}{ll}
                  x\\
                  \frac{1}{1+e^{-kx}}\\
                  \frac{e^x-e^{-x}}{e^x+e^{-x}}
                \end{array}
              \right.
  \]

  $\expval{x}$
  
  $\chi_\rho(ghg\dmo)=\Tr(\rho_{ghg\dmo})=\Tr(\rho_g\circ\rho_h\circ\rho\dmo_g)=\Tr(\rho_h)\overset{\mbox{\scalebox{0.5}{$\Tr(AB)=\Tr(BA)$}}}{=}\chi_\rho(h)$
  	$\mathop{\oplus}_{\substack{x\in X}}$

$\mat(\rho_g)=(a_{ij}(g))_{\scriptsize \substack{1\leq i\leq d \\ 1\leq j\leq d}}$ et $\mat(\rho'_g)=(a'_{ij}(g))_{\scriptsize \substack{1\leq i'\leq d' \\ 1\leq j'\leq d'}}$



\[\int_a^b{\mathbb{R}^2}g(u, v)\dd{P_{XY}}(u, v)=\iint g(u,v) f_{XY}(u, v)\dd \lambda(u) \dd \lambda(v)\]
$$\lim_{x\to\infty} f(x)$$	
$$\iiiint_V \mu(t,u,v,w) \,dt\,du\,dv\,dw$$
$$\sum_{n=1}^{\infty} 2^{-n} = 1$$	
\begin{definition}
	Si $X$ et $Y$ sont 2 v.a. ou definit la \textsc{Covariance} entre $X$ et $Y$ comme
	$\cov(X,Y)\overset{\text{def}}{=}\E\left[(X-\E(X))(Y-\E(Y))\right]=\E(XY)-\E(X)\E(Y)$.
\end{definition}
\fi
\pagebreak

% \tableofcontents

% insert your code here
%\input{./algebra/main.tex}
%\input{./geometrie-differentielle/main.tex}
%\input{./probabilite/main.tex}
%\input{./analyse-fonctionnelle/main.tex}
% \input{./Analyse-convexe-et-dualite-en-optimisation/main.tex}
%\input{./tikz/main.tex}
%\input{./Theorie-du-distributions/main.tex}
%\input{./optimisation/mine.tex}
 \input{./modelisation/main.tex}

% yves.aubry@univ-tln.fr : algebra

\end{document}

%% !TEX encoding = UTF-8 Unicode
% !TEX TS-program = xelatex

\documentclass[french]{report}

%\usepackage[utf8]{inputenc}
%\usepackage[T1]{fontenc}
\usepackage{babel}


\newif\ifcomment
%\commenttrue # Show comments

\usepackage{physics}
\usepackage{amssymb}


\usepackage{amsthm}
% \usepackage{thmtools}
\usepackage{mathtools}
\usepackage{amsfonts}

\usepackage{color}

\usepackage{tikz}

\usepackage{geometry}
\geometry{a5paper, margin=0.1in, right=1cm}

\usepackage{dsfont}

\usepackage{graphicx}
\graphicspath{ {images/} }

\usepackage{faktor}

\usepackage{IEEEtrantools}
\usepackage{enumerate}   
\usepackage[PostScript=dvips]{"/Users/aware/Documents/Courses/diagrams"}


\newtheorem{theorem}{Théorème}[section]
\renewcommand{\thetheorem}{\arabic{theorem}}
\newtheorem{lemme}{Lemme}[section]
\renewcommand{\thelemme}{\arabic{lemme}}
\newtheorem{proposition}{Proposition}[section]
\renewcommand{\theproposition}{\arabic{proposition}}
\newtheorem{notations}{Notations}[section]
\newtheorem{problem}{Problème}[section]
\newtheorem{corollary}{Corollaire}[theorem]
\renewcommand{\thecorollary}{\arabic{corollary}}
\newtheorem{property}{Propriété}[section]
\newtheorem{objective}{Objectif}[section]

\theoremstyle{definition}
\newtheorem{definition}{Définition}[section]
\renewcommand{\thedefinition}{\arabic{definition}}
\newtheorem{exercise}{Exercice}[chapter]
\renewcommand{\theexercise}{\arabic{exercise}}
\newtheorem{example}{Exemple}[chapter]
\renewcommand{\theexample}{\arabic{example}}
\newtheorem*{solution}{Solution}
\newtheorem*{application}{Application}
\newtheorem*{notation}{Notation}
\newtheorem*{vocabulary}{Vocabulaire}
\newtheorem*{properties}{Propriétés}



\theoremstyle{remark}
\newtheorem*{remark}{Remarque}
\newtheorem*{rappel}{Rappel}


\usepackage{etoolbox}
\AtBeginEnvironment{exercise}{\small}
\AtBeginEnvironment{example}{\small}

\usepackage{cases}
\usepackage[red]{mypack}

\usepackage[framemethod=TikZ]{mdframed}

\definecolor{bg}{rgb}{0.4,0.25,0.95}
\definecolor{pagebg}{rgb}{0,0,0.5}
\surroundwithmdframed[
   topline=false,
   rightline=false,
   bottomline=false,
   leftmargin=\parindent,
   skipabove=8pt,
   skipbelow=8pt,
   linecolor=blue,
   innerbottommargin=10pt,
   % backgroundcolor=bg,font=\color{orange}\sffamily, fontcolor=white
]{definition}

\usepackage{empheq}
\usepackage[most]{tcolorbox}

\newtcbox{\mymath}[1][]{%
    nobeforeafter, math upper, tcbox raise base,
    enhanced, colframe=blue!30!black,
    colback=red!10, boxrule=1pt,
    #1}

\usepackage{unixode}


\DeclareMathOperator{\ord}{ord}
\DeclareMathOperator{\orb}{orb}
\DeclareMathOperator{\stab}{stab}
\DeclareMathOperator{\Stab}{stab}
\DeclareMathOperator{\ppcm}{ppcm}
\DeclareMathOperator{\conj}{Conj}
\DeclareMathOperator{\End}{End}
\DeclareMathOperator{\rot}{rot}
\DeclareMathOperator{\trs}{trace}
\DeclareMathOperator{\Ind}{Ind}
\DeclareMathOperator{\mat}{Mat}
\DeclareMathOperator{\id}{Id}
\DeclareMathOperator{\vect}{vect}
\DeclareMathOperator{\img}{img}
\DeclareMathOperator{\cov}{Cov}
\DeclareMathOperator{\dist}{dist}
\DeclareMathOperator{\irr}{Irr}
\DeclareMathOperator{\image}{Im}
\DeclareMathOperator{\pd}{\partial}
\DeclareMathOperator{\epi}{epi}
\DeclareMathOperator{\Argmin}{Argmin}
\DeclareMathOperator{\dom}{dom}
\DeclareMathOperator{\proj}{proj}
\DeclareMathOperator{\ctg}{ctg}
\DeclareMathOperator{\supp}{supp}
\DeclareMathOperator{\argmin}{argmin}
\DeclareMathOperator{\mult}{mult}
\DeclareMathOperator{\ch}{ch}
\DeclareMathOperator{\sh}{sh}
\DeclareMathOperator{\rang}{rang}
\DeclareMathOperator{\diam}{diam}
\DeclareMathOperator{\Epigraphe}{Epigraphe}




\usepackage{xcolor}
\everymath{\color{blue}}
%\everymath{\color[rgb]{0,1,1}}
%\pagecolor[rgb]{0,0,0.5}


\newcommand*{\pdtest}[3][]{\ensuremath{\frac{\partial^{#1} #2}{\partial #3}}}

\newcommand*{\deffunc}[6][]{\ensuremath{
\begin{array}{rcl}
#2 : #3 &\rightarrow& #4\\
#5 &\mapsto& #6
\end{array}
}}

\newcommand{\eqcolon}{\mathrel{\resizebox{\widthof{$\mathord{=}$}}{\height}{ $\!\!=\!\!\resizebox{1.2\width}{0.8\height}{\raisebox{0.23ex}{$\mathop{:}$}}\!\!$ }}}
\newcommand{\coloneq}{\mathrel{\resizebox{\widthof{$\mathord{=}$}}{\height}{ $\!\!\resizebox{1.2\width}{0.8\height}{\raisebox{0.23ex}{$\mathop{:}$}}\!\!=\!\!$ }}}
\newcommand{\eqcolonl}{\ensuremath{\mathrel{=\!\!\mathop{:}}}}
\newcommand{\coloneql}{\ensuremath{\mathrel{\mathop{:} \!\! =}}}
\newcommand{\vc}[1]{% inline column vector
  \left(\begin{smallmatrix}#1\end{smallmatrix}\right)%
}
\newcommand{\vr}[1]{% inline row vector
  \begin{smallmatrix}(\,#1\,)\end{smallmatrix}%
}
\makeatletter
\newcommand*{\defeq}{\ =\mathrel{\rlap{%
                     \raisebox{0.3ex}{$\m@th\cdot$}}%
                     \raisebox{-0.3ex}{$\m@th\cdot$}}%
                     }
\makeatother

\newcommand{\mathcircle}[1]{% inline row vector
 \overset{\circ}{#1}
}
\newcommand{\ulim}{% low limit
 \underline{\lim}
}
\newcommand{\ssi}{% iff
\iff
}
\newcommand{\ps}[2]{
\expval{#1 | #2}
}
\newcommand{\df}[1]{
\mqty{#1}
}
\newcommand{\n}[1]{
\norm{#1}
}
\newcommand{\sys}[1]{
\left\{\smqty{#1}\right.
}


\newcommand{\eqdef}{\ensuremath{\overset{\text{def}}=}}


\def\Circlearrowright{\ensuremath{%
  \rotatebox[origin=c]{230}{$\circlearrowright$}}}

\newcommand\ct[1]{\text{\rmfamily\upshape #1}}
\newcommand\question[1]{ {\color{red} ...!? \small #1}}
\newcommand\caz[1]{\left\{\begin{array} #1 \end{array}\right.}
\newcommand\const{\text{\rmfamily\upshape const}}
\newcommand\toP{ \overset{\pro}{\to}}
\newcommand\toPP{ \overset{\text{PP}}{\to}}
\newcommand{\oeq}{\mathrel{\text{\textcircled{$=$}}}}





\usepackage{xcolor}
% \usepackage[normalem]{ulem}
\usepackage{lipsum}
\makeatletter
% \newcommand\colorwave[1][blue]{\bgroup \markoverwith{\lower3.5\p@\hbox{\sixly \textcolor{#1}{\char58}}}\ULon}
%\font\sixly=lasy6 % does not re-load if already loaded, so no memory problem.

\newmdtheoremenv[
linewidth= 1pt,linecolor= blue,%
leftmargin=20,rightmargin=20,innertopmargin=0pt, innerrightmargin=40,%
tikzsetting = { draw=lightgray, line width = 0.3pt,dashed,%
dash pattern = on 15pt off 3pt},%
splittopskip=\topskip,skipbelow=\baselineskip,%
skipabove=\baselineskip,ntheorem,roundcorner=0pt,
% backgroundcolor=pagebg,font=\color{orange}\sffamily, fontcolor=white
]{examplebox}{Exemple}[section]



\newcommand\R{\mathbb{R}}
\newcommand\Z{\mathbb{Z}}
\newcommand\N{\mathbb{N}}
\newcommand\E{\mathbb{E}}
\newcommand\F{\mathcal{F}}
\newcommand\cH{\mathcal{H}}
\newcommand\V{\mathbb{V}}
\newcommand\dmo{ ^{-1} }
\newcommand\kapa{\kappa}
\newcommand\im{Im}
\newcommand\hs{\mathcal{H}}





\usepackage{soul}

\makeatletter
\newcommand*{\whiten}[1]{\llap{\textcolor{white}{{\the\SOUL@token}}\hspace{#1pt}}}
\DeclareRobustCommand*\myul{%
    \def\SOUL@everyspace{\underline{\space}\kern\z@}%
    \def\SOUL@everytoken{%
     \setbox0=\hbox{\the\SOUL@token}%
     \ifdim\dp0>\z@
        \raisebox{\dp0}{\underline{\phantom{\the\SOUL@token}}}%
        \whiten{1}\whiten{0}%
        \whiten{-1}\whiten{-2}%
        \llap{\the\SOUL@token}%
     \else
        \underline{\the\SOUL@token}%
     \fi}%
\SOUL@}
\makeatother

\newcommand*{\demp}{\fontfamily{lmtt}\selectfont}

\DeclareTextFontCommand{\textdemp}{\demp}

\begin{document}

\ifcomment
Multiline
comment
\fi
\ifcomment
\myul{Typesetting test}
% \color[rgb]{1,1,1}
$∑_i^n≠ 60º±∞π∆¬≈√j∫h≤≥µ$

$\CR \R\pro\ind\pro\gS\pro
\mqty[a&b\\c&d]$
$\pro\mathbb{P}$
$\dd{x}$

  \[
    \alpha(x)=\left\{
                \begin{array}{ll}
                  x\\
                  \frac{1}{1+e^{-kx}}\\
                  \frac{e^x-e^{-x}}{e^x+e^{-x}}
                \end{array}
              \right.
  \]

  $\expval{x}$
  
  $\chi_\rho(ghg\dmo)=\Tr(\rho_{ghg\dmo})=\Tr(\rho_g\circ\rho_h\circ\rho\dmo_g)=\Tr(\rho_h)\overset{\mbox{\scalebox{0.5}{$\Tr(AB)=\Tr(BA)$}}}{=}\chi_\rho(h)$
  	$\mathop{\oplus}_{\substack{x\in X}}$

$\mat(\rho_g)=(a_{ij}(g))_{\scriptsize \substack{1\leq i\leq d \\ 1\leq j\leq d}}$ et $\mat(\rho'_g)=(a'_{ij}(g))_{\scriptsize \substack{1\leq i'\leq d' \\ 1\leq j'\leq d'}}$



\[\int_a^b{\mathbb{R}^2}g(u, v)\dd{P_{XY}}(u, v)=\iint g(u,v) f_{XY}(u, v)\dd \lambda(u) \dd \lambda(v)\]
$$\lim_{x\to\infty} f(x)$$	
$$\iiiint_V \mu(t,u,v,w) \,dt\,du\,dv\,dw$$
$$\sum_{n=1}^{\infty} 2^{-n} = 1$$	
\begin{definition}
	Si $X$ et $Y$ sont 2 v.a. ou definit la \textsc{Covariance} entre $X$ et $Y$ comme
	$\cov(X,Y)\overset{\text{def}}{=}\E\left[(X-\E(X))(Y-\E(Y))\right]=\E(XY)-\E(X)\E(Y)$.
\end{definition}
\fi
\pagebreak

% \tableofcontents

% insert your code here
%\input{./algebra/main.tex}
%\input{./geometrie-differentielle/main.tex}
%\input{./probabilite/main.tex}
%\input{./analyse-fonctionnelle/main.tex}
% \input{./Analyse-convexe-et-dualite-en-optimisation/main.tex}
%\input{./tikz/main.tex}
%\input{./Theorie-du-distributions/main.tex}
%\input{./optimisation/mine.tex}
 \input{./modelisation/main.tex}

% yves.aubry@univ-tln.fr : algebra

\end{document}

%% !TEX encoding = UTF-8 Unicode
% !TEX TS-program = xelatex

\documentclass[french]{report}

%\usepackage[utf8]{inputenc}
%\usepackage[T1]{fontenc}
\usepackage{babel}


\newif\ifcomment
%\commenttrue # Show comments

\usepackage{physics}
\usepackage{amssymb}


\usepackage{amsthm}
% \usepackage{thmtools}
\usepackage{mathtools}
\usepackage{amsfonts}

\usepackage{color}

\usepackage{tikz}

\usepackage{geometry}
\geometry{a5paper, margin=0.1in, right=1cm}

\usepackage{dsfont}

\usepackage{graphicx}
\graphicspath{ {images/} }

\usepackage{faktor}

\usepackage{IEEEtrantools}
\usepackage{enumerate}   
\usepackage[PostScript=dvips]{"/Users/aware/Documents/Courses/diagrams"}


\newtheorem{theorem}{Théorème}[section]
\renewcommand{\thetheorem}{\arabic{theorem}}
\newtheorem{lemme}{Lemme}[section]
\renewcommand{\thelemme}{\arabic{lemme}}
\newtheorem{proposition}{Proposition}[section]
\renewcommand{\theproposition}{\arabic{proposition}}
\newtheorem{notations}{Notations}[section]
\newtheorem{problem}{Problème}[section]
\newtheorem{corollary}{Corollaire}[theorem]
\renewcommand{\thecorollary}{\arabic{corollary}}
\newtheorem{property}{Propriété}[section]
\newtheorem{objective}{Objectif}[section]

\theoremstyle{definition}
\newtheorem{definition}{Définition}[section]
\renewcommand{\thedefinition}{\arabic{definition}}
\newtheorem{exercise}{Exercice}[chapter]
\renewcommand{\theexercise}{\arabic{exercise}}
\newtheorem{example}{Exemple}[chapter]
\renewcommand{\theexample}{\arabic{example}}
\newtheorem*{solution}{Solution}
\newtheorem*{application}{Application}
\newtheorem*{notation}{Notation}
\newtheorem*{vocabulary}{Vocabulaire}
\newtheorem*{properties}{Propriétés}



\theoremstyle{remark}
\newtheorem*{remark}{Remarque}
\newtheorem*{rappel}{Rappel}


\usepackage{etoolbox}
\AtBeginEnvironment{exercise}{\small}
\AtBeginEnvironment{example}{\small}

\usepackage{cases}
\usepackage[red]{mypack}

\usepackage[framemethod=TikZ]{mdframed}

\definecolor{bg}{rgb}{0.4,0.25,0.95}
\definecolor{pagebg}{rgb}{0,0,0.5}
\surroundwithmdframed[
   topline=false,
   rightline=false,
   bottomline=false,
   leftmargin=\parindent,
   skipabove=8pt,
   skipbelow=8pt,
   linecolor=blue,
   innerbottommargin=10pt,
   % backgroundcolor=bg,font=\color{orange}\sffamily, fontcolor=white
]{definition}

\usepackage{empheq}
\usepackage[most]{tcolorbox}

\newtcbox{\mymath}[1][]{%
    nobeforeafter, math upper, tcbox raise base,
    enhanced, colframe=blue!30!black,
    colback=red!10, boxrule=1pt,
    #1}

\usepackage{unixode}


\DeclareMathOperator{\ord}{ord}
\DeclareMathOperator{\orb}{orb}
\DeclareMathOperator{\stab}{stab}
\DeclareMathOperator{\Stab}{stab}
\DeclareMathOperator{\ppcm}{ppcm}
\DeclareMathOperator{\conj}{Conj}
\DeclareMathOperator{\End}{End}
\DeclareMathOperator{\rot}{rot}
\DeclareMathOperator{\trs}{trace}
\DeclareMathOperator{\Ind}{Ind}
\DeclareMathOperator{\mat}{Mat}
\DeclareMathOperator{\id}{Id}
\DeclareMathOperator{\vect}{vect}
\DeclareMathOperator{\img}{img}
\DeclareMathOperator{\cov}{Cov}
\DeclareMathOperator{\dist}{dist}
\DeclareMathOperator{\irr}{Irr}
\DeclareMathOperator{\image}{Im}
\DeclareMathOperator{\pd}{\partial}
\DeclareMathOperator{\epi}{epi}
\DeclareMathOperator{\Argmin}{Argmin}
\DeclareMathOperator{\dom}{dom}
\DeclareMathOperator{\proj}{proj}
\DeclareMathOperator{\ctg}{ctg}
\DeclareMathOperator{\supp}{supp}
\DeclareMathOperator{\argmin}{argmin}
\DeclareMathOperator{\mult}{mult}
\DeclareMathOperator{\ch}{ch}
\DeclareMathOperator{\sh}{sh}
\DeclareMathOperator{\rang}{rang}
\DeclareMathOperator{\diam}{diam}
\DeclareMathOperator{\Epigraphe}{Epigraphe}




\usepackage{xcolor}
\everymath{\color{blue}}
%\everymath{\color[rgb]{0,1,1}}
%\pagecolor[rgb]{0,0,0.5}


\newcommand*{\pdtest}[3][]{\ensuremath{\frac{\partial^{#1} #2}{\partial #3}}}

\newcommand*{\deffunc}[6][]{\ensuremath{
\begin{array}{rcl}
#2 : #3 &\rightarrow& #4\\
#5 &\mapsto& #6
\end{array}
}}

\newcommand{\eqcolon}{\mathrel{\resizebox{\widthof{$\mathord{=}$}}{\height}{ $\!\!=\!\!\resizebox{1.2\width}{0.8\height}{\raisebox{0.23ex}{$\mathop{:}$}}\!\!$ }}}
\newcommand{\coloneq}{\mathrel{\resizebox{\widthof{$\mathord{=}$}}{\height}{ $\!\!\resizebox{1.2\width}{0.8\height}{\raisebox{0.23ex}{$\mathop{:}$}}\!\!=\!\!$ }}}
\newcommand{\eqcolonl}{\ensuremath{\mathrel{=\!\!\mathop{:}}}}
\newcommand{\coloneql}{\ensuremath{\mathrel{\mathop{:} \!\! =}}}
\newcommand{\vc}[1]{% inline column vector
  \left(\begin{smallmatrix}#1\end{smallmatrix}\right)%
}
\newcommand{\vr}[1]{% inline row vector
  \begin{smallmatrix}(\,#1\,)\end{smallmatrix}%
}
\makeatletter
\newcommand*{\defeq}{\ =\mathrel{\rlap{%
                     \raisebox{0.3ex}{$\m@th\cdot$}}%
                     \raisebox{-0.3ex}{$\m@th\cdot$}}%
                     }
\makeatother

\newcommand{\mathcircle}[1]{% inline row vector
 \overset{\circ}{#1}
}
\newcommand{\ulim}{% low limit
 \underline{\lim}
}
\newcommand{\ssi}{% iff
\iff
}
\newcommand{\ps}[2]{
\expval{#1 | #2}
}
\newcommand{\df}[1]{
\mqty{#1}
}
\newcommand{\n}[1]{
\norm{#1}
}
\newcommand{\sys}[1]{
\left\{\smqty{#1}\right.
}


\newcommand{\eqdef}{\ensuremath{\overset{\text{def}}=}}


\def\Circlearrowright{\ensuremath{%
  \rotatebox[origin=c]{230}{$\circlearrowright$}}}

\newcommand\ct[1]{\text{\rmfamily\upshape #1}}
\newcommand\question[1]{ {\color{red} ...!? \small #1}}
\newcommand\caz[1]{\left\{\begin{array} #1 \end{array}\right.}
\newcommand\const{\text{\rmfamily\upshape const}}
\newcommand\toP{ \overset{\pro}{\to}}
\newcommand\toPP{ \overset{\text{PP}}{\to}}
\newcommand{\oeq}{\mathrel{\text{\textcircled{$=$}}}}





\usepackage{xcolor}
% \usepackage[normalem]{ulem}
\usepackage{lipsum}
\makeatletter
% \newcommand\colorwave[1][blue]{\bgroup \markoverwith{\lower3.5\p@\hbox{\sixly \textcolor{#1}{\char58}}}\ULon}
%\font\sixly=lasy6 % does not re-load if already loaded, so no memory problem.

\newmdtheoremenv[
linewidth= 1pt,linecolor= blue,%
leftmargin=20,rightmargin=20,innertopmargin=0pt, innerrightmargin=40,%
tikzsetting = { draw=lightgray, line width = 0.3pt,dashed,%
dash pattern = on 15pt off 3pt},%
splittopskip=\topskip,skipbelow=\baselineskip,%
skipabove=\baselineskip,ntheorem,roundcorner=0pt,
% backgroundcolor=pagebg,font=\color{orange}\sffamily, fontcolor=white
]{examplebox}{Exemple}[section]



\newcommand\R{\mathbb{R}}
\newcommand\Z{\mathbb{Z}}
\newcommand\N{\mathbb{N}}
\newcommand\E{\mathbb{E}}
\newcommand\F{\mathcal{F}}
\newcommand\cH{\mathcal{H}}
\newcommand\V{\mathbb{V}}
\newcommand\dmo{ ^{-1} }
\newcommand\kapa{\kappa}
\newcommand\im{Im}
\newcommand\hs{\mathcal{H}}





\usepackage{soul}

\makeatletter
\newcommand*{\whiten}[1]{\llap{\textcolor{white}{{\the\SOUL@token}}\hspace{#1pt}}}
\DeclareRobustCommand*\myul{%
    \def\SOUL@everyspace{\underline{\space}\kern\z@}%
    \def\SOUL@everytoken{%
     \setbox0=\hbox{\the\SOUL@token}%
     \ifdim\dp0>\z@
        \raisebox{\dp0}{\underline{\phantom{\the\SOUL@token}}}%
        \whiten{1}\whiten{0}%
        \whiten{-1}\whiten{-2}%
        \llap{\the\SOUL@token}%
     \else
        \underline{\the\SOUL@token}%
     \fi}%
\SOUL@}
\makeatother

\newcommand*{\demp}{\fontfamily{lmtt}\selectfont}

\DeclareTextFontCommand{\textdemp}{\demp}

\begin{document}

\ifcomment
Multiline
comment
\fi
\ifcomment
\myul{Typesetting test}
% \color[rgb]{1,1,1}
$∑_i^n≠ 60º±∞π∆¬≈√j∫h≤≥µ$

$\CR \R\pro\ind\pro\gS\pro
\mqty[a&b\\c&d]$
$\pro\mathbb{P}$
$\dd{x}$

  \[
    \alpha(x)=\left\{
                \begin{array}{ll}
                  x\\
                  \frac{1}{1+e^{-kx}}\\
                  \frac{e^x-e^{-x}}{e^x+e^{-x}}
                \end{array}
              \right.
  \]

  $\expval{x}$
  
  $\chi_\rho(ghg\dmo)=\Tr(\rho_{ghg\dmo})=\Tr(\rho_g\circ\rho_h\circ\rho\dmo_g)=\Tr(\rho_h)\overset{\mbox{\scalebox{0.5}{$\Tr(AB)=\Tr(BA)$}}}{=}\chi_\rho(h)$
  	$\mathop{\oplus}_{\substack{x\in X}}$

$\mat(\rho_g)=(a_{ij}(g))_{\scriptsize \substack{1\leq i\leq d \\ 1\leq j\leq d}}$ et $\mat(\rho'_g)=(a'_{ij}(g))_{\scriptsize \substack{1\leq i'\leq d' \\ 1\leq j'\leq d'}}$



\[\int_a^b{\mathbb{R}^2}g(u, v)\dd{P_{XY}}(u, v)=\iint g(u,v) f_{XY}(u, v)\dd \lambda(u) \dd \lambda(v)\]
$$\lim_{x\to\infty} f(x)$$	
$$\iiiint_V \mu(t,u,v,w) \,dt\,du\,dv\,dw$$
$$\sum_{n=1}^{\infty} 2^{-n} = 1$$	
\begin{definition}
	Si $X$ et $Y$ sont 2 v.a. ou definit la \textsc{Covariance} entre $X$ et $Y$ comme
	$\cov(X,Y)\overset{\text{def}}{=}\E\left[(X-\E(X))(Y-\E(Y))\right]=\E(XY)-\E(X)\E(Y)$.
\end{definition}
\fi
\pagebreak

% \tableofcontents

% insert your code here
%\input{./algebra/main.tex}
%\input{./geometrie-differentielle/main.tex}
%\input{./probabilite/main.tex}
%\input{./analyse-fonctionnelle/main.tex}
% \input{./Analyse-convexe-et-dualite-en-optimisation/main.tex}
%\input{./tikz/main.tex}
%\input{./Theorie-du-distributions/main.tex}
%\input{./optimisation/mine.tex}
 \input{./modelisation/main.tex}

% yves.aubry@univ-tln.fr : algebra

\end{document}

% % !TEX encoding = UTF-8 Unicode
% !TEX TS-program = xelatex

\documentclass[french]{report}

%\usepackage[utf8]{inputenc}
%\usepackage[T1]{fontenc}
\usepackage{babel}


\newif\ifcomment
%\commenttrue # Show comments

\usepackage{physics}
\usepackage{amssymb}


\usepackage{amsthm}
% \usepackage{thmtools}
\usepackage{mathtools}
\usepackage{amsfonts}

\usepackage{color}

\usepackage{tikz}

\usepackage{geometry}
\geometry{a5paper, margin=0.1in, right=1cm}

\usepackage{dsfont}

\usepackage{graphicx}
\graphicspath{ {images/} }

\usepackage{faktor}

\usepackage{IEEEtrantools}
\usepackage{enumerate}   
\usepackage[PostScript=dvips]{"/Users/aware/Documents/Courses/diagrams"}


\newtheorem{theorem}{Théorème}[section]
\renewcommand{\thetheorem}{\arabic{theorem}}
\newtheorem{lemme}{Lemme}[section]
\renewcommand{\thelemme}{\arabic{lemme}}
\newtheorem{proposition}{Proposition}[section]
\renewcommand{\theproposition}{\arabic{proposition}}
\newtheorem{notations}{Notations}[section]
\newtheorem{problem}{Problème}[section]
\newtheorem{corollary}{Corollaire}[theorem]
\renewcommand{\thecorollary}{\arabic{corollary}}
\newtheorem{property}{Propriété}[section]
\newtheorem{objective}{Objectif}[section]

\theoremstyle{definition}
\newtheorem{definition}{Définition}[section]
\renewcommand{\thedefinition}{\arabic{definition}}
\newtheorem{exercise}{Exercice}[chapter]
\renewcommand{\theexercise}{\arabic{exercise}}
\newtheorem{example}{Exemple}[chapter]
\renewcommand{\theexample}{\arabic{example}}
\newtheorem*{solution}{Solution}
\newtheorem*{application}{Application}
\newtheorem*{notation}{Notation}
\newtheorem*{vocabulary}{Vocabulaire}
\newtheorem*{properties}{Propriétés}



\theoremstyle{remark}
\newtheorem*{remark}{Remarque}
\newtheorem*{rappel}{Rappel}


\usepackage{etoolbox}
\AtBeginEnvironment{exercise}{\small}
\AtBeginEnvironment{example}{\small}

\usepackage{cases}
\usepackage[red]{mypack}

\usepackage[framemethod=TikZ]{mdframed}

\definecolor{bg}{rgb}{0.4,0.25,0.95}
\definecolor{pagebg}{rgb}{0,0,0.5}
\surroundwithmdframed[
   topline=false,
   rightline=false,
   bottomline=false,
   leftmargin=\parindent,
   skipabove=8pt,
   skipbelow=8pt,
   linecolor=blue,
   innerbottommargin=10pt,
   % backgroundcolor=bg,font=\color{orange}\sffamily, fontcolor=white
]{definition}

\usepackage{empheq}
\usepackage[most]{tcolorbox}

\newtcbox{\mymath}[1][]{%
    nobeforeafter, math upper, tcbox raise base,
    enhanced, colframe=blue!30!black,
    colback=red!10, boxrule=1pt,
    #1}

\usepackage{unixode}


\DeclareMathOperator{\ord}{ord}
\DeclareMathOperator{\orb}{orb}
\DeclareMathOperator{\stab}{stab}
\DeclareMathOperator{\Stab}{stab}
\DeclareMathOperator{\ppcm}{ppcm}
\DeclareMathOperator{\conj}{Conj}
\DeclareMathOperator{\End}{End}
\DeclareMathOperator{\rot}{rot}
\DeclareMathOperator{\trs}{trace}
\DeclareMathOperator{\Ind}{Ind}
\DeclareMathOperator{\mat}{Mat}
\DeclareMathOperator{\id}{Id}
\DeclareMathOperator{\vect}{vect}
\DeclareMathOperator{\img}{img}
\DeclareMathOperator{\cov}{Cov}
\DeclareMathOperator{\dist}{dist}
\DeclareMathOperator{\irr}{Irr}
\DeclareMathOperator{\image}{Im}
\DeclareMathOperator{\pd}{\partial}
\DeclareMathOperator{\epi}{epi}
\DeclareMathOperator{\Argmin}{Argmin}
\DeclareMathOperator{\dom}{dom}
\DeclareMathOperator{\proj}{proj}
\DeclareMathOperator{\ctg}{ctg}
\DeclareMathOperator{\supp}{supp}
\DeclareMathOperator{\argmin}{argmin}
\DeclareMathOperator{\mult}{mult}
\DeclareMathOperator{\ch}{ch}
\DeclareMathOperator{\sh}{sh}
\DeclareMathOperator{\rang}{rang}
\DeclareMathOperator{\diam}{diam}
\DeclareMathOperator{\Epigraphe}{Epigraphe}




\usepackage{xcolor}
\everymath{\color{blue}}
%\everymath{\color[rgb]{0,1,1}}
%\pagecolor[rgb]{0,0,0.5}


\newcommand*{\pdtest}[3][]{\ensuremath{\frac{\partial^{#1} #2}{\partial #3}}}

\newcommand*{\deffunc}[6][]{\ensuremath{
\begin{array}{rcl}
#2 : #3 &\rightarrow& #4\\
#5 &\mapsto& #6
\end{array}
}}

\newcommand{\eqcolon}{\mathrel{\resizebox{\widthof{$\mathord{=}$}}{\height}{ $\!\!=\!\!\resizebox{1.2\width}{0.8\height}{\raisebox{0.23ex}{$\mathop{:}$}}\!\!$ }}}
\newcommand{\coloneq}{\mathrel{\resizebox{\widthof{$\mathord{=}$}}{\height}{ $\!\!\resizebox{1.2\width}{0.8\height}{\raisebox{0.23ex}{$\mathop{:}$}}\!\!=\!\!$ }}}
\newcommand{\eqcolonl}{\ensuremath{\mathrel{=\!\!\mathop{:}}}}
\newcommand{\coloneql}{\ensuremath{\mathrel{\mathop{:} \!\! =}}}
\newcommand{\vc}[1]{% inline column vector
  \left(\begin{smallmatrix}#1\end{smallmatrix}\right)%
}
\newcommand{\vr}[1]{% inline row vector
  \begin{smallmatrix}(\,#1\,)\end{smallmatrix}%
}
\makeatletter
\newcommand*{\defeq}{\ =\mathrel{\rlap{%
                     \raisebox{0.3ex}{$\m@th\cdot$}}%
                     \raisebox{-0.3ex}{$\m@th\cdot$}}%
                     }
\makeatother

\newcommand{\mathcircle}[1]{% inline row vector
 \overset{\circ}{#1}
}
\newcommand{\ulim}{% low limit
 \underline{\lim}
}
\newcommand{\ssi}{% iff
\iff
}
\newcommand{\ps}[2]{
\expval{#1 | #2}
}
\newcommand{\df}[1]{
\mqty{#1}
}
\newcommand{\n}[1]{
\norm{#1}
}
\newcommand{\sys}[1]{
\left\{\smqty{#1}\right.
}


\newcommand{\eqdef}{\ensuremath{\overset{\text{def}}=}}


\def\Circlearrowright{\ensuremath{%
  \rotatebox[origin=c]{230}{$\circlearrowright$}}}

\newcommand\ct[1]{\text{\rmfamily\upshape #1}}
\newcommand\question[1]{ {\color{red} ...!? \small #1}}
\newcommand\caz[1]{\left\{\begin{array} #1 \end{array}\right.}
\newcommand\const{\text{\rmfamily\upshape const}}
\newcommand\toP{ \overset{\pro}{\to}}
\newcommand\toPP{ \overset{\text{PP}}{\to}}
\newcommand{\oeq}{\mathrel{\text{\textcircled{$=$}}}}





\usepackage{xcolor}
% \usepackage[normalem]{ulem}
\usepackage{lipsum}
\makeatletter
% \newcommand\colorwave[1][blue]{\bgroup \markoverwith{\lower3.5\p@\hbox{\sixly \textcolor{#1}{\char58}}}\ULon}
%\font\sixly=lasy6 % does not re-load if already loaded, so no memory problem.

\newmdtheoremenv[
linewidth= 1pt,linecolor= blue,%
leftmargin=20,rightmargin=20,innertopmargin=0pt, innerrightmargin=40,%
tikzsetting = { draw=lightgray, line width = 0.3pt,dashed,%
dash pattern = on 15pt off 3pt},%
splittopskip=\topskip,skipbelow=\baselineskip,%
skipabove=\baselineskip,ntheorem,roundcorner=0pt,
% backgroundcolor=pagebg,font=\color{orange}\sffamily, fontcolor=white
]{examplebox}{Exemple}[section]



\newcommand\R{\mathbb{R}}
\newcommand\Z{\mathbb{Z}}
\newcommand\N{\mathbb{N}}
\newcommand\E{\mathbb{E}}
\newcommand\F{\mathcal{F}}
\newcommand\cH{\mathcal{H}}
\newcommand\V{\mathbb{V}}
\newcommand\dmo{ ^{-1} }
\newcommand\kapa{\kappa}
\newcommand\im{Im}
\newcommand\hs{\mathcal{H}}





\usepackage{soul}

\makeatletter
\newcommand*{\whiten}[1]{\llap{\textcolor{white}{{\the\SOUL@token}}\hspace{#1pt}}}
\DeclareRobustCommand*\myul{%
    \def\SOUL@everyspace{\underline{\space}\kern\z@}%
    \def\SOUL@everytoken{%
     \setbox0=\hbox{\the\SOUL@token}%
     \ifdim\dp0>\z@
        \raisebox{\dp0}{\underline{\phantom{\the\SOUL@token}}}%
        \whiten{1}\whiten{0}%
        \whiten{-1}\whiten{-2}%
        \llap{\the\SOUL@token}%
     \else
        \underline{\the\SOUL@token}%
     \fi}%
\SOUL@}
\makeatother

\newcommand*{\demp}{\fontfamily{lmtt}\selectfont}

\DeclareTextFontCommand{\textdemp}{\demp}

\begin{document}

\ifcomment
Multiline
comment
\fi
\ifcomment
\myul{Typesetting test}
% \color[rgb]{1,1,1}
$∑_i^n≠ 60º±∞π∆¬≈√j∫h≤≥µ$

$\CR \R\pro\ind\pro\gS\pro
\mqty[a&b\\c&d]$
$\pro\mathbb{P}$
$\dd{x}$

  \[
    \alpha(x)=\left\{
                \begin{array}{ll}
                  x\\
                  \frac{1}{1+e^{-kx}}\\
                  \frac{e^x-e^{-x}}{e^x+e^{-x}}
                \end{array}
              \right.
  \]

  $\expval{x}$
  
  $\chi_\rho(ghg\dmo)=\Tr(\rho_{ghg\dmo})=\Tr(\rho_g\circ\rho_h\circ\rho\dmo_g)=\Tr(\rho_h)\overset{\mbox{\scalebox{0.5}{$\Tr(AB)=\Tr(BA)$}}}{=}\chi_\rho(h)$
  	$\mathop{\oplus}_{\substack{x\in X}}$

$\mat(\rho_g)=(a_{ij}(g))_{\scriptsize \substack{1\leq i\leq d \\ 1\leq j\leq d}}$ et $\mat(\rho'_g)=(a'_{ij}(g))_{\scriptsize \substack{1\leq i'\leq d' \\ 1\leq j'\leq d'}}$



\[\int_a^b{\mathbb{R}^2}g(u, v)\dd{P_{XY}}(u, v)=\iint g(u,v) f_{XY}(u, v)\dd \lambda(u) \dd \lambda(v)\]
$$\lim_{x\to\infty} f(x)$$	
$$\iiiint_V \mu(t,u,v,w) \,dt\,du\,dv\,dw$$
$$\sum_{n=1}^{\infty} 2^{-n} = 1$$	
\begin{definition}
	Si $X$ et $Y$ sont 2 v.a. ou definit la \textsc{Covariance} entre $X$ et $Y$ comme
	$\cov(X,Y)\overset{\text{def}}{=}\E\left[(X-\E(X))(Y-\E(Y))\right]=\E(XY)-\E(X)\E(Y)$.
\end{definition}
\fi
\pagebreak

% \tableofcontents

% insert your code here
%\input{./algebra/main.tex}
%\input{./geometrie-differentielle/main.tex}
%\input{./probabilite/main.tex}
%\input{./analyse-fonctionnelle/main.tex}
% \input{./Analyse-convexe-et-dualite-en-optimisation/main.tex}
%\input{./tikz/main.tex}
%\input{./Theorie-du-distributions/main.tex}
%\input{./optimisation/mine.tex}
 \input{./modelisation/main.tex}

% yves.aubry@univ-tln.fr : algebra

\end{document}

%% !TEX encoding = UTF-8 Unicode
% !TEX TS-program = xelatex

\documentclass[french]{report}

%\usepackage[utf8]{inputenc}
%\usepackage[T1]{fontenc}
\usepackage{babel}


\newif\ifcomment
%\commenttrue # Show comments

\usepackage{physics}
\usepackage{amssymb}


\usepackage{amsthm}
% \usepackage{thmtools}
\usepackage{mathtools}
\usepackage{amsfonts}

\usepackage{color}

\usepackage{tikz}

\usepackage{geometry}
\geometry{a5paper, margin=0.1in, right=1cm}

\usepackage{dsfont}

\usepackage{graphicx}
\graphicspath{ {images/} }

\usepackage{faktor}

\usepackage{IEEEtrantools}
\usepackage{enumerate}   
\usepackage[PostScript=dvips]{"/Users/aware/Documents/Courses/diagrams"}


\newtheorem{theorem}{Théorème}[section]
\renewcommand{\thetheorem}{\arabic{theorem}}
\newtheorem{lemme}{Lemme}[section]
\renewcommand{\thelemme}{\arabic{lemme}}
\newtheorem{proposition}{Proposition}[section]
\renewcommand{\theproposition}{\arabic{proposition}}
\newtheorem{notations}{Notations}[section]
\newtheorem{problem}{Problème}[section]
\newtheorem{corollary}{Corollaire}[theorem]
\renewcommand{\thecorollary}{\arabic{corollary}}
\newtheorem{property}{Propriété}[section]
\newtheorem{objective}{Objectif}[section]

\theoremstyle{definition}
\newtheorem{definition}{Définition}[section]
\renewcommand{\thedefinition}{\arabic{definition}}
\newtheorem{exercise}{Exercice}[chapter]
\renewcommand{\theexercise}{\arabic{exercise}}
\newtheorem{example}{Exemple}[chapter]
\renewcommand{\theexample}{\arabic{example}}
\newtheorem*{solution}{Solution}
\newtheorem*{application}{Application}
\newtheorem*{notation}{Notation}
\newtheorem*{vocabulary}{Vocabulaire}
\newtheorem*{properties}{Propriétés}



\theoremstyle{remark}
\newtheorem*{remark}{Remarque}
\newtheorem*{rappel}{Rappel}


\usepackage{etoolbox}
\AtBeginEnvironment{exercise}{\small}
\AtBeginEnvironment{example}{\small}

\usepackage{cases}
\usepackage[red]{mypack}

\usepackage[framemethod=TikZ]{mdframed}

\definecolor{bg}{rgb}{0.4,0.25,0.95}
\definecolor{pagebg}{rgb}{0,0,0.5}
\surroundwithmdframed[
   topline=false,
   rightline=false,
   bottomline=false,
   leftmargin=\parindent,
   skipabove=8pt,
   skipbelow=8pt,
   linecolor=blue,
   innerbottommargin=10pt,
   % backgroundcolor=bg,font=\color{orange}\sffamily, fontcolor=white
]{definition}

\usepackage{empheq}
\usepackage[most]{tcolorbox}

\newtcbox{\mymath}[1][]{%
    nobeforeafter, math upper, tcbox raise base,
    enhanced, colframe=blue!30!black,
    colback=red!10, boxrule=1pt,
    #1}

\usepackage{unixode}


\DeclareMathOperator{\ord}{ord}
\DeclareMathOperator{\orb}{orb}
\DeclareMathOperator{\stab}{stab}
\DeclareMathOperator{\Stab}{stab}
\DeclareMathOperator{\ppcm}{ppcm}
\DeclareMathOperator{\conj}{Conj}
\DeclareMathOperator{\End}{End}
\DeclareMathOperator{\rot}{rot}
\DeclareMathOperator{\trs}{trace}
\DeclareMathOperator{\Ind}{Ind}
\DeclareMathOperator{\mat}{Mat}
\DeclareMathOperator{\id}{Id}
\DeclareMathOperator{\vect}{vect}
\DeclareMathOperator{\img}{img}
\DeclareMathOperator{\cov}{Cov}
\DeclareMathOperator{\dist}{dist}
\DeclareMathOperator{\irr}{Irr}
\DeclareMathOperator{\image}{Im}
\DeclareMathOperator{\pd}{\partial}
\DeclareMathOperator{\epi}{epi}
\DeclareMathOperator{\Argmin}{Argmin}
\DeclareMathOperator{\dom}{dom}
\DeclareMathOperator{\proj}{proj}
\DeclareMathOperator{\ctg}{ctg}
\DeclareMathOperator{\supp}{supp}
\DeclareMathOperator{\argmin}{argmin}
\DeclareMathOperator{\mult}{mult}
\DeclareMathOperator{\ch}{ch}
\DeclareMathOperator{\sh}{sh}
\DeclareMathOperator{\rang}{rang}
\DeclareMathOperator{\diam}{diam}
\DeclareMathOperator{\Epigraphe}{Epigraphe}




\usepackage{xcolor}
\everymath{\color{blue}}
%\everymath{\color[rgb]{0,1,1}}
%\pagecolor[rgb]{0,0,0.5}


\newcommand*{\pdtest}[3][]{\ensuremath{\frac{\partial^{#1} #2}{\partial #3}}}

\newcommand*{\deffunc}[6][]{\ensuremath{
\begin{array}{rcl}
#2 : #3 &\rightarrow& #4\\
#5 &\mapsto& #6
\end{array}
}}

\newcommand{\eqcolon}{\mathrel{\resizebox{\widthof{$\mathord{=}$}}{\height}{ $\!\!=\!\!\resizebox{1.2\width}{0.8\height}{\raisebox{0.23ex}{$\mathop{:}$}}\!\!$ }}}
\newcommand{\coloneq}{\mathrel{\resizebox{\widthof{$\mathord{=}$}}{\height}{ $\!\!\resizebox{1.2\width}{0.8\height}{\raisebox{0.23ex}{$\mathop{:}$}}\!\!=\!\!$ }}}
\newcommand{\eqcolonl}{\ensuremath{\mathrel{=\!\!\mathop{:}}}}
\newcommand{\coloneql}{\ensuremath{\mathrel{\mathop{:} \!\! =}}}
\newcommand{\vc}[1]{% inline column vector
  \left(\begin{smallmatrix}#1\end{smallmatrix}\right)%
}
\newcommand{\vr}[1]{% inline row vector
  \begin{smallmatrix}(\,#1\,)\end{smallmatrix}%
}
\makeatletter
\newcommand*{\defeq}{\ =\mathrel{\rlap{%
                     \raisebox{0.3ex}{$\m@th\cdot$}}%
                     \raisebox{-0.3ex}{$\m@th\cdot$}}%
                     }
\makeatother

\newcommand{\mathcircle}[1]{% inline row vector
 \overset{\circ}{#1}
}
\newcommand{\ulim}{% low limit
 \underline{\lim}
}
\newcommand{\ssi}{% iff
\iff
}
\newcommand{\ps}[2]{
\expval{#1 | #2}
}
\newcommand{\df}[1]{
\mqty{#1}
}
\newcommand{\n}[1]{
\norm{#1}
}
\newcommand{\sys}[1]{
\left\{\smqty{#1}\right.
}


\newcommand{\eqdef}{\ensuremath{\overset{\text{def}}=}}


\def\Circlearrowright{\ensuremath{%
  \rotatebox[origin=c]{230}{$\circlearrowright$}}}

\newcommand\ct[1]{\text{\rmfamily\upshape #1}}
\newcommand\question[1]{ {\color{red} ...!? \small #1}}
\newcommand\caz[1]{\left\{\begin{array} #1 \end{array}\right.}
\newcommand\const{\text{\rmfamily\upshape const}}
\newcommand\toP{ \overset{\pro}{\to}}
\newcommand\toPP{ \overset{\text{PP}}{\to}}
\newcommand{\oeq}{\mathrel{\text{\textcircled{$=$}}}}





\usepackage{xcolor}
% \usepackage[normalem]{ulem}
\usepackage{lipsum}
\makeatletter
% \newcommand\colorwave[1][blue]{\bgroup \markoverwith{\lower3.5\p@\hbox{\sixly \textcolor{#1}{\char58}}}\ULon}
%\font\sixly=lasy6 % does not re-load if already loaded, so no memory problem.

\newmdtheoremenv[
linewidth= 1pt,linecolor= blue,%
leftmargin=20,rightmargin=20,innertopmargin=0pt, innerrightmargin=40,%
tikzsetting = { draw=lightgray, line width = 0.3pt,dashed,%
dash pattern = on 15pt off 3pt},%
splittopskip=\topskip,skipbelow=\baselineskip,%
skipabove=\baselineskip,ntheorem,roundcorner=0pt,
% backgroundcolor=pagebg,font=\color{orange}\sffamily, fontcolor=white
]{examplebox}{Exemple}[section]



\newcommand\R{\mathbb{R}}
\newcommand\Z{\mathbb{Z}}
\newcommand\N{\mathbb{N}}
\newcommand\E{\mathbb{E}}
\newcommand\F{\mathcal{F}}
\newcommand\cH{\mathcal{H}}
\newcommand\V{\mathbb{V}}
\newcommand\dmo{ ^{-1} }
\newcommand\kapa{\kappa}
\newcommand\im{Im}
\newcommand\hs{\mathcal{H}}





\usepackage{soul}

\makeatletter
\newcommand*{\whiten}[1]{\llap{\textcolor{white}{{\the\SOUL@token}}\hspace{#1pt}}}
\DeclareRobustCommand*\myul{%
    \def\SOUL@everyspace{\underline{\space}\kern\z@}%
    \def\SOUL@everytoken{%
     \setbox0=\hbox{\the\SOUL@token}%
     \ifdim\dp0>\z@
        \raisebox{\dp0}{\underline{\phantom{\the\SOUL@token}}}%
        \whiten{1}\whiten{0}%
        \whiten{-1}\whiten{-2}%
        \llap{\the\SOUL@token}%
     \else
        \underline{\the\SOUL@token}%
     \fi}%
\SOUL@}
\makeatother

\newcommand*{\demp}{\fontfamily{lmtt}\selectfont}

\DeclareTextFontCommand{\textdemp}{\demp}

\begin{document}

\ifcomment
Multiline
comment
\fi
\ifcomment
\myul{Typesetting test}
% \color[rgb]{1,1,1}
$∑_i^n≠ 60º±∞π∆¬≈√j∫h≤≥µ$

$\CR \R\pro\ind\pro\gS\pro
\mqty[a&b\\c&d]$
$\pro\mathbb{P}$
$\dd{x}$

  \[
    \alpha(x)=\left\{
                \begin{array}{ll}
                  x\\
                  \frac{1}{1+e^{-kx}}\\
                  \frac{e^x-e^{-x}}{e^x+e^{-x}}
                \end{array}
              \right.
  \]

  $\expval{x}$
  
  $\chi_\rho(ghg\dmo)=\Tr(\rho_{ghg\dmo})=\Tr(\rho_g\circ\rho_h\circ\rho\dmo_g)=\Tr(\rho_h)\overset{\mbox{\scalebox{0.5}{$\Tr(AB)=\Tr(BA)$}}}{=}\chi_\rho(h)$
  	$\mathop{\oplus}_{\substack{x\in X}}$

$\mat(\rho_g)=(a_{ij}(g))_{\scriptsize \substack{1\leq i\leq d \\ 1\leq j\leq d}}$ et $\mat(\rho'_g)=(a'_{ij}(g))_{\scriptsize \substack{1\leq i'\leq d' \\ 1\leq j'\leq d'}}$



\[\int_a^b{\mathbb{R}^2}g(u, v)\dd{P_{XY}}(u, v)=\iint g(u,v) f_{XY}(u, v)\dd \lambda(u) \dd \lambda(v)\]
$$\lim_{x\to\infty} f(x)$$	
$$\iiiint_V \mu(t,u,v,w) \,dt\,du\,dv\,dw$$
$$\sum_{n=1}^{\infty} 2^{-n} = 1$$	
\begin{definition}
	Si $X$ et $Y$ sont 2 v.a. ou definit la \textsc{Covariance} entre $X$ et $Y$ comme
	$\cov(X,Y)\overset{\text{def}}{=}\E\left[(X-\E(X))(Y-\E(Y))\right]=\E(XY)-\E(X)\E(Y)$.
\end{definition}
\fi
\pagebreak

% \tableofcontents

% insert your code here
%\input{./algebra/main.tex}
%\input{./geometrie-differentielle/main.tex}
%\input{./probabilite/main.tex}
%\input{./analyse-fonctionnelle/main.tex}
% \input{./Analyse-convexe-et-dualite-en-optimisation/main.tex}
%\input{./tikz/main.tex}
%\input{./Theorie-du-distributions/main.tex}
%\input{./optimisation/mine.tex}
 \input{./modelisation/main.tex}

% yves.aubry@univ-tln.fr : algebra

\end{document}

%% !TEX encoding = UTF-8 Unicode
% !TEX TS-program = xelatex

\documentclass[french]{report}

%\usepackage[utf8]{inputenc}
%\usepackage[T1]{fontenc}
\usepackage{babel}


\newif\ifcomment
%\commenttrue # Show comments

\usepackage{physics}
\usepackage{amssymb}


\usepackage{amsthm}
% \usepackage{thmtools}
\usepackage{mathtools}
\usepackage{amsfonts}

\usepackage{color}

\usepackage{tikz}

\usepackage{geometry}
\geometry{a5paper, margin=0.1in, right=1cm}

\usepackage{dsfont}

\usepackage{graphicx}
\graphicspath{ {images/} }

\usepackage{faktor}

\usepackage{IEEEtrantools}
\usepackage{enumerate}   
\usepackage[PostScript=dvips]{"/Users/aware/Documents/Courses/diagrams"}


\newtheorem{theorem}{Théorème}[section]
\renewcommand{\thetheorem}{\arabic{theorem}}
\newtheorem{lemme}{Lemme}[section]
\renewcommand{\thelemme}{\arabic{lemme}}
\newtheorem{proposition}{Proposition}[section]
\renewcommand{\theproposition}{\arabic{proposition}}
\newtheorem{notations}{Notations}[section]
\newtheorem{problem}{Problème}[section]
\newtheorem{corollary}{Corollaire}[theorem]
\renewcommand{\thecorollary}{\arabic{corollary}}
\newtheorem{property}{Propriété}[section]
\newtheorem{objective}{Objectif}[section]

\theoremstyle{definition}
\newtheorem{definition}{Définition}[section]
\renewcommand{\thedefinition}{\arabic{definition}}
\newtheorem{exercise}{Exercice}[chapter]
\renewcommand{\theexercise}{\arabic{exercise}}
\newtheorem{example}{Exemple}[chapter]
\renewcommand{\theexample}{\arabic{example}}
\newtheorem*{solution}{Solution}
\newtheorem*{application}{Application}
\newtheorem*{notation}{Notation}
\newtheorem*{vocabulary}{Vocabulaire}
\newtheorem*{properties}{Propriétés}



\theoremstyle{remark}
\newtheorem*{remark}{Remarque}
\newtheorem*{rappel}{Rappel}


\usepackage{etoolbox}
\AtBeginEnvironment{exercise}{\small}
\AtBeginEnvironment{example}{\small}

\usepackage{cases}
\usepackage[red]{mypack}

\usepackage[framemethod=TikZ]{mdframed}

\definecolor{bg}{rgb}{0.4,0.25,0.95}
\definecolor{pagebg}{rgb}{0,0,0.5}
\surroundwithmdframed[
   topline=false,
   rightline=false,
   bottomline=false,
   leftmargin=\parindent,
   skipabove=8pt,
   skipbelow=8pt,
   linecolor=blue,
   innerbottommargin=10pt,
   % backgroundcolor=bg,font=\color{orange}\sffamily, fontcolor=white
]{definition}

\usepackage{empheq}
\usepackage[most]{tcolorbox}

\newtcbox{\mymath}[1][]{%
    nobeforeafter, math upper, tcbox raise base,
    enhanced, colframe=blue!30!black,
    colback=red!10, boxrule=1pt,
    #1}

\usepackage{unixode}


\DeclareMathOperator{\ord}{ord}
\DeclareMathOperator{\orb}{orb}
\DeclareMathOperator{\stab}{stab}
\DeclareMathOperator{\Stab}{stab}
\DeclareMathOperator{\ppcm}{ppcm}
\DeclareMathOperator{\conj}{Conj}
\DeclareMathOperator{\End}{End}
\DeclareMathOperator{\rot}{rot}
\DeclareMathOperator{\trs}{trace}
\DeclareMathOperator{\Ind}{Ind}
\DeclareMathOperator{\mat}{Mat}
\DeclareMathOperator{\id}{Id}
\DeclareMathOperator{\vect}{vect}
\DeclareMathOperator{\img}{img}
\DeclareMathOperator{\cov}{Cov}
\DeclareMathOperator{\dist}{dist}
\DeclareMathOperator{\irr}{Irr}
\DeclareMathOperator{\image}{Im}
\DeclareMathOperator{\pd}{\partial}
\DeclareMathOperator{\epi}{epi}
\DeclareMathOperator{\Argmin}{Argmin}
\DeclareMathOperator{\dom}{dom}
\DeclareMathOperator{\proj}{proj}
\DeclareMathOperator{\ctg}{ctg}
\DeclareMathOperator{\supp}{supp}
\DeclareMathOperator{\argmin}{argmin}
\DeclareMathOperator{\mult}{mult}
\DeclareMathOperator{\ch}{ch}
\DeclareMathOperator{\sh}{sh}
\DeclareMathOperator{\rang}{rang}
\DeclareMathOperator{\diam}{diam}
\DeclareMathOperator{\Epigraphe}{Epigraphe}




\usepackage{xcolor}
\everymath{\color{blue}}
%\everymath{\color[rgb]{0,1,1}}
%\pagecolor[rgb]{0,0,0.5}


\newcommand*{\pdtest}[3][]{\ensuremath{\frac{\partial^{#1} #2}{\partial #3}}}

\newcommand*{\deffunc}[6][]{\ensuremath{
\begin{array}{rcl}
#2 : #3 &\rightarrow& #4\\
#5 &\mapsto& #6
\end{array}
}}

\newcommand{\eqcolon}{\mathrel{\resizebox{\widthof{$\mathord{=}$}}{\height}{ $\!\!=\!\!\resizebox{1.2\width}{0.8\height}{\raisebox{0.23ex}{$\mathop{:}$}}\!\!$ }}}
\newcommand{\coloneq}{\mathrel{\resizebox{\widthof{$\mathord{=}$}}{\height}{ $\!\!\resizebox{1.2\width}{0.8\height}{\raisebox{0.23ex}{$\mathop{:}$}}\!\!=\!\!$ }}}
\newcommand{\eqcolonl}{\ensuremath{\mathrel{=\!\!\mathop{:}}}}
\newcommand{\coloneql}{\ensuremath{\mathrel{\mathop{:} \!\! =}}}
\newcommand{\vc}[1]{% inline column vector
  \left(\begin{smallmatrix}#1\end{smallmatrix}\right)%
}
\newcommand{\vr}[1]{% inline row vector
  \begin{smallmatrix}(\,#1\,)\end{smallmatrix}%
}
\makeatletter
\newcommand*{\defeq}{\ =\mathrel{\rlap{%
                     \raisebox{0.3ex}{$\m@th\cdot$}}%
                     \raisebox{-0.3ex}{$\m@th\cdot$}}%
                     }
\makeatother

\newcommand{\mathcircle}[1]{% inline row vector
 \overset{\circ}{#1}
}
\newcommand{\ulim}{% low limit
 \underline{\lim}
}
\newcommand{\ssi}{% iff
\iff
}
\newcommand{\ps}[2]{
\expval{#1 | #2}
}
\newcommand{\df}[1]{
\mqty{#1}
}
\newcommand{\n}[1]{
\norm{#1}
}
\newcommand{\sys}[1]{
\left\{\smqty{#1}\right.
}


\newcommand{\eqdef}{\ensuremath{\overset{\text{def}}=}}


\def\Circlearrowright{\ensuremath{%
  \rotatebox[origin=c]{230}{$\circlearrowright$}}}

\newcommand\ct[1]{\text{\rmfamily\upshape #1}}
\newcommand\question[1]{ {\color{red} ...!? \small #1}}
\newcommand\caz[1]{\left\{\begin{array} #1 \end{array}\right.}
\newcommand\const{\text{\rmfamily\upshape const}}
\newcommand\toP{ \overset{\pro}{\to}}
\newcommand\toPP{ \overset{\text{PP}}{\to}}
\newcommand{\oeq}{\mathrel{\text{\textcircled{$=$}}}}





\usepackage{xcolor}
% \usepackage[normalem]{ulem}
\usepackage{lipsum}
\makeatletter
% \newcommand\colorwave[1][blue]{\bgroup \markoverwith{\lower3.5\p@\hbox{\sixly \textcolor{#1}{\char58}}}\ULon}
%\font\sixly=lasy6 % does not re-load if already loaded, so no memory problem.

\newmdtheoremenv[
linewidth= 1pt,linecolor= blue,%
leftmargin=20,rightmargin=20,innertopmargin=0pt, innerrightmargin=40,%
tikzsetting = { draw=lightgray, line width = 0.3pt,dashed,%
dash pattern = on 15pt off 3pt},%
splittopskip=\topskip,skipbelow=\baselineskip,%
skipabove=\baselineskip,ntheorem,roundcorner=0pt,
% backgroundcolor=pagebg,font=\color{orange}\sffamily, fontcolor=white
]{examplebox}{Exemple}[section]



\newcommand\R{\mathbb{R}}
\newcommand\Z{\mathbb{Z}}
\newcommand\N{\mathbb{N}}
\newcommand\E{\mathbb{E}}
\newcommand\F{\mathcal{F}}
\newcommand\cH{\mathcal{H}}
\newcommand\V{\mathbb{V}}
\newcommand\dmo{ ^{-1} }
\newcommand\kapa{\kappa}
\newcommand\im{Im}
\newcommand\hs{\mathcal{H}}





\usepackage{soul}

\makeatletter
\newcommand*{\whiten}[1]{\llap{\textcolor{white}{{\the\SOUL@token}}\hspace{#1pt}}}
\DeclareRobustCommand*\myul{%
    \def\SOUL@everyspace{\underline{\space}\kern\z@}%
    \def\SOUL@everytoken{%
     \setbox0=\hbox{\the\SOUL@token}%
     \ifdim\dp0>\z@
        \raisebox{\dp0}{\underline{\phantom{\the\SOUL@token}}}%
        \whiten{1}\whiten{0}%
        \whiten{-1}\whiten{-2}%
        \llap{\the\SOUL@token}%
     \else
        \underline{\the\SOUL@token}%
     \fi}%
\SOUL@}
\makeatother

\newcommand*{\demp}{\fontfamily{lmtt}\selectfont}

\DeclareTextFontCommand{\textdemp}{\demp}

\begin{document}

\ifcomment
Multiline
comment
\fi
\ifcomment
\myul{Typesetting test}
% \color[rgb]{1,1,1}
$∑_i^n≠ 60º±∞π∆¬≈√j∫h≤≥µ$

$\CR \R\pro\ind\pro\gS\pro
\mqty[a&b\\c&d]$
$\pro\mathbb{P}$
$\dd{x}$

  \[
    \alpha(x)=\left\{
                \begin{array}{ll}
                  x\\
                  \frac{1}{1+e^{-kx}}\\
                  \frac{e^x-e^{-x}}{e^x+e^{-x}}
                \end{array}
              \right.
  \]

  $\expval{x}$
  
  $\chi_\rho(ghg\dmo)=\Tr(\rho_{ghg\dmo})=\Tr(\rho_g\circ\rho_h\circ\rho\dmo_g)=\Tr(\rho_h)\overset{\mbox{\scalebox{0.5}{$\Tr(AB)=\Tr(BA)$}}}{=}\chi_\rho(h)$
  	$\mathop{\oplus}_{\substack{x\in X}}$

$\mat(\rho_g)=(a_{ij}(g))_{\scriptsize \substack{1\leq i\leq d \\ 1\leq j\leq d}}$ et $\mat(\rho'_g)=(a'_{ij}(g))_{\scriptsize \substack{1\leq i'\leq d' \\ 1\leq j'\leq d'}}$



\[\int_a^b{\mathbb{R}^2}g(u, v)\dd{P_{XY}}(u, v)=\iint g(u,v) f_{XY}(u, v)\dd \lambda(u) \dd \lambda(v)\]
$$\lim_{x\to\infty} f(x)$$	
$$\iiiint_V \mu(t,u,v,w) \,dt\,du\,dv\,dw$$
$$\sum_{n=1}^{\infty} 2^{-n} = 1$$	
\begin{definition}
	Si $X$ et $Y$ sont 2 v.a. ou definit la \textsc{Covariance} entre $X$ et $Y$ comme
	$\cov(X,Y)\overset{\text{def}}{=}\E\left[(X-\E(X))(Y-\E(Y))\right]=\E(XY)-\E(X)\E(Y)$.
\end{definition}
\fi
\pagebreak

% \tableofcontents

% insert your code here
%\input{./algebra/main.tex}
%\input{./geometrie-differentielle/main.tex}
%\input{./probabilite/main.tex}
%\input{./analyse-fonctionnelle/main.tex}
% \input{./Analyse-convexe-et-dualite-en-optimisation/main.tex}
%\input{./tikz/main.tex}
%\input{./Theorie-du-distributions/main.tex}
%\input{./optimisation/mine.tex}
 \input{./modelisation/main.tex}

% yves.aubry@univ-tln.fr : algebra

\end{document}

%\input{./optimisation/mine.tex}
 % !TEX encoding = UTF-8 Unicode
% !TEX TS-program = xelatex

\documentclass[french]{report}

%\usepackage[utf8]{inputenc}
%\usepackage[T1]{fontenc}
\usepackage{babel}


\newif\ifcomment
%\commenttrue # Show comments

\usepackage{physics}
\usepackage{amssymb}


\usepackage{amsthm}
% \usepackage{thmtools}
\usepackage{mathtools}
\usepackage{amsfonts}

\usepackage{color}

\usepackage{tikz}

\usepackage{geometry}
\geometry{a5paper, margin=0.1in, right=1cm}

\usepackage{dsfont}

\usepackage{graphicx}
\graphicspath{ {images/} }

\usepackage{faktor}

\usepackage{IEEEtrantools}
\usepackage{enumerate}   
\usepackage[PostScript=dvips]{"/Users/aware/Documents/Courses/diagrams"}


\newtheorem{theorem}{Théorème}[section]
\renewcommand{\thetheorem}{\arabic{theorem}}
\newtheorem{lemme}{Lemme}[section]
\renewcommand{\thelemme}{\arabic{lemme}}
\newtheorem{proposition}{Proposition}[section]
\renewcommand{\theproposition}{\arabic{proposition}}
\newtheorem{notations}{Notations}[section]
\newtheorem{problem}{Problème}[section]
\newtheorem{corollary}{Corollaire}[theorem]
\renewcommand{\thecorollary}{\arabic{corollary}}
\newtheorem{property}{Propriété}[section]
\newtheorem{objective}{Objectif}[section]

\theoremstyle{definition}
\newtheorem{definition}{Définition}[section]
\renewcommand{\thedefinition}{\arabic{definition}}
\newtheorem{exercise}{Exercice}[chapter]
\renewcommand{\theexercise}{\arabic{exercise}}
\newtheorem{example}{Exemple}[chapter]
\renewcommand{\theexample}{\arabic{example}}
\newtheorem*{solution}{Solution}
\newtheorem*{application}{Application}
\newtheorem*{notation}{Notation}
\newtheorem*{vocabulary}{Vocabulaire}
\newtheorem*{properties}{Propriétés}



\theoremstyle{remark}
\newtheorem*{remark}{Remarque}
\newtheorem*{rappel}{Rappel}


\usepackage{etoolbox}
\AtBeginEnvironment{exercise}{\small}
\AtBeginEnvironment{example}{\small}

\usepackage{cases}
\usepackage[red]{mypack}

\usepackage[framemethod=TikZ]{mdframed}

\definecolor{bg}{rgb}{0.4,0.25,0.95}
\definecolor{pagebg}{rgb}{0,0,0.5}
\surroundwithmdframed[
   topline=false,
   rightline=false,
   bottomline=false,
   leftmargin=\parindent,
   skipabove=8pt,
   skipbelow=8pt,
   linecolor=blue,
   innerbottommargin=10pt,
   % backgroundcolor=bg,font=\color{orange}\sffamily, fontcolor=white
]{definition}

\usepackage{empheq}
\usepackage[most]{tcolorbox}

\newtcbox{\mymath}[1][]{%
    nobeforeafter, math upper, tcbox raise base,
    enhanced, colframe=blue!30!black,
    colback=red!10, boxrule=1pt,
    #1}

\usepackage{unixode}


\DeclareMathOperator{\ord}{ord}
\DeclareMathOperator{\orb}{orb}
\DeclareMathOperator{\stab}{stab}
\DeclareMathOperator{\Stab}{stab}
\DeclareMathOperator{\ppcm}{ppcm}
\DeclareMathOperator{\conj}{Conj}
\DeclareMathOperator{\End}{End}
\DeclareMathOperator{\rot}{rot}
\DeclareMathOperator{\trs}{trace}
\DeclareMathOperator{\Ind}{Ind}
\DeclareMathOperator{\mat}{Mat}
\DeclareMathOperator{\id}{Id}
\DeclareMathOperator{\vect}{vect}
\DeclareMathOperator{\img}{img}
\DeclareMathOperator{\cov}{Cov}
\DeclareMathOperator{\dist}{dist}
\DeclareMathOperator{\irr}{Irr}
\DeclareMathOperator{\image}{Im}
\DeclareMathOperator{\pd}{\partial}
\DeclareMathOperator{\epi}{epi}
\DeclareMathOperator{\Argmin}{Argmin}
\DeclareMathOperator{\dom}{dom}
\DeclareMathOperator{\proj}{proj}
\DeclareMathOperator{\ctg}{ctg}
\DeclareMathOperator{\supp}{supp}
\DeclareMathOperator{\argmin}{argmin}
\DeclareMathOperator{\mult}{mult}
\DeclareMathOperator{\ch}{ch}
\DeclareMathOperator{\sh}{sh}
\DeclareMathOperator{\rang}{rang}
\DeclareMathOperator{\diam}{diam}
\DeclareMathOperator{\Epigraphe}{Epigraphe}




\usepackage{xcolor}
\everymath{\color{blue}}
%\everymath{\color[rgb]{0,1,1}}
%\pagecolor[rgb]{0,0,0.5}


\newcommand*{\pdtest}[3][]{\ensuremath{\frac{\partial^{#1} #2}{\partial #3}}}

\newcommand*{\deffunc}[6][]{\ensuremath{
\begin{array}{rcl}
#2 : #3 &\rightarrow& #4\\
#5 &\mapsto& #6
\end{array}
}}

\newcommand{\eqcolon}{\mathrel{\resizebox{\widthof{$\mathord{=}$}}{\height}{ $\!\!=\!\!\resizebox{1.2\width}{0.8\height}{\raisebox{0.23ex}{$\mathop{:}$}}\!\!$ }}}
\newcommand{\coloneq}{\mathrel{\resizebox{\widthof{$\mathord{=}$}}{\height}{ $\!\!\resizebox{1.2\width}{0.8\height}{\raisebox{0.23ex}{$\mathop{:}$}}\!\!=\!\!$ }}}
\newcommand{\eqcolonl}{\ensuremath{\mathrel{=\!\!\mathop{:}}}}
\newcommand{\coloneql}{\ensuremath{\mathrel{\mathop{:} \!\! =}}}
\newcommand{\vc}[1]{% inline column vector
  \left(\begin{smallmatrix}#1\end{smallmatrix}\right)%
}
\newcommand{\vr}[1]{% inline row vector
  \begin{smallmatrix}(\,#1\,)\end{smallmatrix}%
}
\makeatletter
\newcommand*{\defeq}{\ =\mathrel{\rlap{%
                     \raisebox{0.3ex}{$\m@th\cdot$}}%
                     \raisebox{-0.3ex}{$\m@th\cdot$}}%
                     }
\makeatother

\newcommand{\mathcircle}[1]{% inline row vector
 \overset{\circ}{#1}
}
\newcommand{\ulim}{% low limit
 \underline{\lim}
}
\newcommand{\ssi}{% iff
\iff
}
\newcommand{\ps}[2]{
\expval{#1 | #2}
}
\newcommand{\df}[1]{
\mqty{#1}
}
\newcommand{\n}[1]{
\norm{#1}
}
\newcommand{\sys}[1]{
\left\{\smqty{#1}\right.
}


\newcommand{\eqdef}{\ensuremath{\overset{\text{def}}=}}


\def\Circlearrowright{\ensuremath{%
  \rotatebox[origin=c]{230}{$\circlearrowright$}}}

\newcommand\ct[1]{\text{\rmfamily\upshape #1}}
\newcommand\question[1]{ {\color{red} ...!? \small #1}}
\newcommand\caz[1]{\left\{\begin{array} #1 \end{array}\right.}
\newcommand\const{\text{\rmfamily\upshape const}}
\newcommand\toP{ \overset{\pro}{\to}}
\newcommand\toPP{ \overset{\text{PP}}{\to}}
\newcommand{\oeq}{\mathrel{\text{\textcircled{$=$}}}}





\usepackage{xcolor}
% \usepackage[normalem]{ulem}
\usepackage{lipsum}
\makeatletter
% \newcommand\colorwave[1][blue]{\bgroup \markoverwith{\lower3.5\p@\hbox{\sixly \textcolor{#1}{\char58}}}\ULon}
%\font\sixly=lasy6 % does not re-load if already loaded, so no memory problem.

\newmdtheoremenv[
linewidth= 1pt,linecolor= blue,%
leftmargin=20,rightmargin=20,innertopmargin=0pt, innerrightmargin=40,%
tikzsetting = { draw=lightgray, line width = 0.3pt,dashed,%
dash pattern = on 15pt off 3pt},%
splittopskip=\topskip,skipbelow=\baselineskip,%
skipabove=\baselineskip,ntheorem,roundcorner=0pt,
% backgroundcolor=pagebg,font=\color{orange}\sffamily, fontcolor=white
]{examplebox}{Exemple}[section]



\newcommand\R{\mathbb{R}}
\newcommand\Z{\mathbb{Z}}
\newcommand\N{\mathbb{N}}
\newcommand\E{\mathbb{E}}
\newcommand\F{\mathcal{F}}
\newcommand\cH{\mathcal{H}}
\newcommand\V{\mathbb{V}}
\newcommand\dmo{ ^{-1} }
\newcommand\kapa{\kappa}
\newcommand\im{Im}
\newcommand\hs{\mathcal{H}}





\usepackage{soul}

\makeatletter
\newcommand*{\whiten}[1]{\llap{\textcolor{white}{{\the\SOUL@token}}\hspace{#1pt}}}
\DeclareRobustCommand*\myul{%
    \def\SOUL@everyspace{\underline{\space}\kern\z@}%
    \def\SOUL@everytoken{%
     \setbox0=\hbox{\the\SOUL@token}%
     \ifdim\dp0>\z@
        \raisebox{\dp0}{\underline{\phantom{\the\SOUL@token}}}%
        \whiten{1}\whiten{0}%
        \whiten{-1}\whiten{-2}%
        \llap{\the\SOUL@token}%
     \else
        \underline{\the\SOUL@token}%
     \fi}%
\SOUL@}
\makeatother

\newcommand*{\demp}{\fontfamily{lmtt}\selectfont}

\DeclareTextFontCommand{\textdemp}{\demp}

\begin{document}

\ifcomment
Multiline
comment
\fi
\ifcomment
\myul{Typesetting test}
% \color[rgb]{1,1,1}
$∑_i^n≠ 60º±∞π∆¬≈√j∫h≤≥µ$

$\CR \R\pro\ind\pro\gS\pro
\mqty[a&b\\c&d]$
$\pro\mathbb{P}$
$\dd{x}$

  \[
    \alpha(x)=\left\{
                \begin{array}{ll}
                  x\\
                  \frac{1}{1+e^{-kx}}\\
                  \frac{e^x-e^{-x}}{e^x+e^{-x}}
                \end{array}
              \right.
  \]

  $\expval{x}$
  
  $\chi_\rho(ghg\dmo)=\Tr(\rho_{ghg\dmo})=\Tr(\rho_g\circ\rho_h\circ\rho\dmo_g)=\Tr(\rho_h)\overset{\mbox{\scalebox{0.5}{$\Tr(AB)=\Tr(BA)$}}}{=}\chi_\rho(h)$
  	$\mathop{\oplus}_{\substack{x\in X}}$

$\mat(\rho_g)=(a_{ij}(g))_{\scriptsize \substack{1\leq i\leq d \\ 1\leq j\leq d}}$ et $\mat(\rho'_g)=(a'_{ij}(g))_{\scriptsize \substack{1\leq i'\leq d' \\ 1\leq j'\leq d'}}$



\[\int_a^b{\mathbb{R}^2}g(u, v)\dd{P_{XY}}(u, v)=\iint g(u,v) f_{XY}(u, v)\dd \lambda(u) \dd \lambda(v)\]
$$\lim_{x\to\infty} f(x)$$	
$$\iiiint_V \mu(t,u,v,w) \,dt\,du\,dv\,dw$$
$$\sum_{n=1}^{\infty} 2^{-n} = 1$$	
\begin{definition}
	Si $X$ et $Y$ sont 2 v.a. ou definit la \textsc{Covariance} entre $X$ et $Y$ comme
	$\cov(X,Y)\overset{\text{def}}{=}\E\left[(X-\E(X))(Y-\E(Y))\right]=\E(XY)-\E(X)\E(Y)$.
\end{definition}
\fi
\pagebreak

% \tableofcontents

% insert your code here
%\input{./algebra/main.tex}
%\input{./geometrie-differentielle/main.tex}
%\input{./probabilite/main.tex}
%\input{./analyse-fonctionnelle/main.tex}
% \input{./Analyse-convexe-et-dualite-en-optimisation/main.tex}
%\input{./tikz/main.tex}
%\input{./Theorie-du-distributions/main.tex}
%\input{./optimisation/mine.tex}
 \input{./modelisation/main.tex}

% yves.aubry@univ-tln.fr : algebra

\end{document}


% yves.aubry@univ-tln.fr : algebra

\end{document}

%% !TEX encoding = UTF-8 Unicode
% !TEX TS-program = xelatex

\documentclass[french]{report}

%\usepackage[utf8]{inputenc}
%\usepackage[T1]{fontenc}
\usepackage{babel}


\newif\ifcomment
%\commenttrue # Show comments

\usepackage{physics}
\usepackage{amssymb}


\usepackage{amsthm}
% \usepackage{thmtools}
\usepackage{mathtools}
\usepackage{amsfonts}

\usepackage{color}

\usepackage{tikz}

\usepackage{geometry}
\geometry{a5paper, margin=0.1in, right=1cm}

\usepackage{dsfont}

\usepackage{graphicx}
\graphicspath{ {images/} }

\usepackage{faktor}

\usepackage{IEEEtrantools}
\usepackage{enumerate}   
\usepackage[PostScript=dvips]{"/Users/aware/Documents/Courses/diagrams"}


\newtheorem{theorem}{Théorème}[section]
\renewcommand{\thetheorem}{\arabic{theorem}}
\newtheorem{lemme}{Lemme}[section]
\renewcommand{\thelemme}{\arabic{lemme}}
\newtheorem{proposition}{Proposition}[section]
\renewcommand{\theproposition}{\arabic{proposition}}
\newtheorem{notations}{Notations}[section]
\newtheorem{problem}{Problème}[section]
\newtheorem{corollary}{Corollaire}[theorem]
\renewcommand{\thecorollary}{\arabic{corollary}}
\newtheorem{property}{Propriété}[section]
\newtheorem{objective}{Objectif}[section]

\theoremstyle{definition}
\newtheorem{definition}{Définition}[section]
\renewcommand{\thedefinition}{\arabic{definition}}
\newtheorem{exercise}{Exercice}[chapter]
\renewcommand{\theexercise}{\arabic{exercise}}
\newtheorem{example}{Exemple}[chapter]
\renewcommand{\theexample}{\arabic{example}}
\newtheorem*{solution}{Solution}
\newtheorem*{application}{Application}
\newtheorem*{notation}{Notation}
\newtheorem*{vocabulary}{Vocabulaire}
\newtheorem*{properties}{Propriétés}



\theoremstyle{remark}
\newtheorem*{remark}{Remarque}
\newtheorem*{rappel}{Rappel}


\usepackage{etoolbox}
\AtBeginEnvironment{exercise}{\small}
\AtBeginEnvironment{example}{\small}

\usepackage{cases}
\usepackage[red]{mypack}

\usepackage[framemethod=TikZ]{mdframed}

\definecolor{bg}{rgb}{0.4,0.25,0.95}
\definecolor{pagebg}{rgb}{0,0,0.5}
\surroundwithmdframed[
   topline=false,
   rightline=false,
   bottomline=false,
   leftmargin=\parindent,
   skipabove=8pt,
   skipbelow=8pt,
   linecolor=blue,
   innerbottommargin=10pt,
   % backgroundcolor=bg,font=\color{orange}\sffamily, fontcolor=white
]{definition}

\usepackage{empheq}
\usepackage[most]{tcolorbox}

\newtcbox{\mymath}[1][]{%
    nobeforeafter, math upper, tcbox raise base,
    enhanced, colframe=blue!30!black,
    colback=red!10, boxrule=1pt,
    #1}

\usepackage{unixode}


\DeclareMathOperator{\ord}{ord}
\DeclareMathOperator{\orb}{orb}
\DeclareMathOperator{\stab}{stab}
\DeclareMathOperator{\Stab}{stab}
\DeclareMathOperator{\ppcm}{ppcm}
\DeclareMathOperator{\conj}{Conj}
\DeclareMathOperator{\End}{End}
\DeclareMathOperator{\rot}{rot}
\DeclareMathOperator{\trs}{trace}
\DeclareMathOperator{\Ind}{Ind}
\DeclareMathOperator{\mat}{Mat}
\DeclareMathOperator{\id}{Id}
\DeclareMathOperator{\vect}{vect}
\DeclareMathOperator{\img}{img}
\DeclareMathOperator{\cov}{Cov}
\DeclareMathOperator{\dist}{dist}
\DeclareMathOperator{\irr}{Irr}
\DeclareMathOperator{\image}{Im}
\DeclareMathOperator{\pd}{\partial}
\DeclareMathOperator{\epi}{epi}
\DeclareMathOperator{\Argmin}{Argmin}
\DeclareMathOperator{\dom}{dom}
\DeclareMathOperator{\proj}{proj}
\DeclareMathOperator{\ctg}{ctg}
\DeclareMathOperator{\supp}{supp}
\DeclareMathOperator{\argmin}{argmin}
\DeclareMathOperator{\mult}{mult}
\DeclareMathOperator{\ch}{ch}
\DeclareMathOperator{\sh}{sh}
\DeclareMathOperator{\rang}{rang}
\DeclareMathOperator{\diam}{diam}
\DeclareMathOperator{\Epigraphe}{Epigraphe}




\usepackage{xcolor}
\everymath{\color{blue}}
%\everymath{\color[rgb]{0,1,1}}
%\pagecolor[rgb]{0,0,0.5}


\newcommand*{\pdtest}[3][]{\ensuremath{\frac{\partial^{#1} #2}{\partial #3}}}

\newcommand*{\deffunc}[6][]{\ensuremath{
\begin{array}{rcl}
#2 : #3 &\rightarrow& #4\\
#5 &\mapsto& #6
\end{array}
}}

\newcommand{\eqcolon}{\mathrel{\resizebox{\widthof{$\mathord{=}$}}{\height}{ $\!\!=\!\!\resizebox{1.2\width}{0.8\height}{\raisebox{0.23ex}{$\mathop{:}$}}\!\!$ }}}
\newcommand{\coloneq}{\mathrel{\resizebox{\widthof{$\mathord{=}$}}{\height}{ $\!\!\resizebox{1.2\width}{0.8\height}{\raisebox{0.23ex}{$\mathop{:}$}}\!\!=\!\!$ }}}
\newcommand{\eqcolonl}{\ensuremath{\mathrel{=\!\!\mathop{:}}}}
\newcommand{\coloneql}{\ensuremath{\mathrel{\mathop{:} \!\! =}}}
\newcommand{\vc}[1]{% inline column vector
  \left(\begin{smallmatrix}#1\end{smallmatrix}\right)%
}
\newcommand{\vr}[1]{% inline row vector
  \begin{smallmatrix}(\,#1\,)\end{smallmatrix}%
}
\makeatletter
\newcommand*{\defeq}{\ =\mathrel{\rlap{%
                     \raisebox{0.3ex}{$\m@th\cdot$}}%
                     \raisebox{-0.3ex}{$\m@th\cdot$}}%
                     }
\makeatother

\newcommand{\mathcircle}[1]{% inline row vector
 \overset{\circ}{#1}
}
\newcommand{\ulim}{% low limit
 \underline{\lim}
}
\newcommand{\ssi}{% iff
\iff
}
\newcommand{\ps}[2]{
\expval{#1 | #2}
}
\newcommand{\df}[1]{
\mqty{#1}
}
\newcommand{\n}[1]{
\norm{#1}
}
\newcommand{\sys}[1]{
\left\{\smqty{#1}\right.
}


\newcommand{\eqdef}{\ensuremath{\overset{\text{def}}=}}


\def\Circlearrowright{\ensuremath{%
  \rotatebox[origin=c]{230}{$\circlearrowright$}}}

\newcommand\ct[1]{\text{\rmfamily\upshape #1}}
\newcommand\question[1]{ {\color{red} ...!? \small #1}}
\newcommand\caz[1]{\left\{\begin{array} #1 \end{array}\right.}
\newcommand\const{\text{\rmfamily\upshape const}}
\newcommand\toP{ \overset{\pro}{\to}}
\newcommand\toPP{ \overset{\text{PP}}{\to}}
\newcommand{\oeq}{\mathrel{\text{\textcircled{$=$}}}}





\usepackage{xcolor}
% \usepackage[normalem]{ulem}
\usepackage{lipsum}
\makeatletter
% \newcommand\colorwave[1][blue]{\bgroup \markoverwith{\lower3.5\p@\hbox{\sixly \textcolor{#1}{\char58}}}\ULon}
%\font\sixly=lasy6 % does not re-load if already loaded, so no memory problem.

\newmdtheoremenv[
linewidth= 1pt,linecolor= blue,%
leftmargin=20,rightmargin=20,innertopmargin=0pt, innerrightmargin=40,%
tikzsetting = { draw=lightgray, line width = 0.3pt,dashed,%
dash pattern = on 15pt off 3pt},%
splittopskip=\topskip,skipbelow=\baselineskip,%
skipabove=\baselineskip,ntheorem,roundcorner=0pt,
% backgroundcolor=pagebg,font=\color{orange}\sffamily, fontcolor=white
]{examplebox}{Exemple}[section]



\newcommand\R{\mathbb{R}}
\newcommand\Z{\mathbb{Z}}
\newcommand\N{\mathbb{N}}
\newcommand\E{\mathbb{E}}
\newcommand\F{\mathcal{F}}
\newcommand\cH{\mathcal{H}}
\newcommand\V{\mathbb{V}}
\newcommand\dmo{ ^{-1} }
\newcommand\kapa{\kappa}
\newcommand\im{Im}
\newcommand\hs{\mathcal{H}}





\usepackage{soul}

\makeatletter
\newcommand*{\whiten}[1]{\llap{\textcolor{white}{{\the\SOUL@token}}\hspace{#1pt}}}
\DeclareRobustCommand*\myul{%
    \def\SOUL@everyspace{\underline{\space}\kern\z@}%
    \def\SOUL@everytoken{%
     \setbox0=\hbox{\the\SOUL@token}%
     \ifdim\dp0>\z@
        \raisebox{\dp0}{\underline{\phantom{\the\SOUL@token}}}%
        \whiten{1}\whiten{0}%
        \whiten{-1}\whiten{-2}%
        \llap{\the\SOUL@token}%
     \else
        \underline{\the\SOUL@token}%
     \fi}%
\SOUL@}
\makeatother

\newcommand*{\demp}{\fontfamily{lmtt}\selectfont}

\DeclareTextFontCommand{\textdemp}{\demp}

\begin{document}

\ifcomment
Multiline
comment
\fi
\ifcomment
\myul{Typesetting test}
% \color[rgb]{1,1,1}
$∑_i^n≠ 60º±∞π∆¬≈√j∫h≤≥µ$

$\CR \R\pro\ind\pro\gS\pro
\mqty[a&b\\c&d]$
$\pro\mathbb{P}$
$\dd{x}$

  \[
    \alpha(x)=\left\{
                \begin{array}{ll}
                  x\\
                  \frac{1}{1+e^{-kx}}\\
                  \frac{e^x-e^{-x}}{e^x+e^{-x}}
                \end{array}
              \right.
  \]

  $\expval{x}$
  
  $\chi_\rho(ghg\dmo)=\Tr(\rho_{ghg\dmo})=\Tr(\rho_g\circ\rho_h\circ\rho\dmo_g)=\Tr(\rho_h)\overset{\mbox{\scalebox{0.5}{$\Tr(AB)=\Tr(BA)$}}}{=}\chi_\rho(h)$
  	$\mathop{\oplus}_{\substack{x\in X}}$

$\mat(\rho_g)=(a_{ij}(g))_{\scriptsize \substack{1\leq i\leq d \\ 1\leq j\leq d}}$ et $\mat(\rho'_g)=(a'_{ij}(g))_{\scriptsize \substack{1\leq i'\leq d' \\ 1\leq j'\leq d'}}$



\[\int_a^b{\mathbb{R}^2}g(u, v)\dd{P_{XY}}(u, v)=\iint g(u,v) f_{XY}(u, v)\dd \lambda(u) \dd \lambda(v)\]
$$\lim_{x\to\infty} f(x)$$	
$$\iiiint_V \mu(t,u,v,w) \,dt\,du\,dv\,dw$$
$$\sum_{n=1}^{\infty} 2^{-n} = 1$$	
\begin{definition}
	Si $X$ et $Y$ sont 2 v.a. ou definit la \textsc{Covariance} entre $X$ et $Y$ comme
	$\cov(X,Y)\overset{\text{def}}{=}\E\left[(X-\E(X))(Y-\E(Y))\right]=\E(XY)-\E(X)\E(Y)$.
\end{definition}
\fi
\pagebreak

% \tableofcontents

% insert your code here
%% !TEX encoding = UTF-8 Unicode
% !TEX TS-program = xelatex

\documentclass[french]{report}

%\usepackage[utf8]{inputenc}
%\usepackage[T1]{fontenc}
\usepackage{babel}


\newif\ifcomment
%\commenttrue # Show comments

\usepackage{physics}
\usepackage{amssymb}


\usepackage{amsthm}
% \usepackage{thmtools}
\usepackage{mathtools}
\usepackage{amsfonts}

\usepackage{color}

\usepackage{tikz}

\usepackage{geometry}
\geometry{a5paper, margin=0.1in, right=1cm}

\usepackage{dsfont}

\usepackage{graphicx}
\graphicspath{ {images/} }

\usepackage{faktor}

\usepackage{IEEEtrantools}
\usepackage{enumerate}   
\usepackage[PostScript=dvips]{"/Users/aware/Documents/Courses/diagrams"}


\newtheorem{theorem}{Théorème}[section]
\renewcommand{\thetheorem}{\arabic{theorem}}
\newtheorem{lemme}{Lemme}[section]
\renewcommand{\thelemme}{\arabic{lemme}}
\newtheorem{proposition}{Proposition}[section]
\renewcommand{\theproposition}{\arabic{proposition}}
\newtheorem{notations}{Notations}[section]
\newtheorem{problem}{Problème}[section]
\newtheorem{corollary}{Corollaire}[theorem]
\renewcommand{\thecorollary}{\arabic{corollary}}
\newtheorem{property}{Propriété}[section]
\newtheorem{objective}{Objectif}[section]

\theoremstyle{definition}
\newtheorem{definition}{Définition}[section]
\renewcommand{\thedefinition}{\arabic{definition}}
\newtheorem{exercise}{Exercice}[chapter]
\renewcommand{\theexercise}{\arabic{exercise}}
\newtheorem{example}{Exemple}[chapter]
\renewcommand{\theexample}{\arabic{example}}
\newtheorem*{solution}{Solution}
\newtheorem*{application}{Application}
\newtheorem*{notation}{Notation}
\newtheorem*{vocabulary}{Vocabulaire}
\newtheorem*{properties}{Propriétés}



\theoremstyle{remark}
\newtheorem*{remark}{Remarque}
\newtheorem*{rappel}{Rappel}


\usepackage{etoolbox}
\AtBeginEnvironment{exercise}{\small}
\AtBeginEnvironment{example}{\small}

\usepackage{cases}
\usepackage[red]{mypack}

\usepackage[framemethod=TikZ]{mdframed}

\definecolor{bg}{rgb}{0.4,0.25,0.95}
\definecolor{pagebg}{rgb}{0,0,0.5}
\surroundwithmdframed[
   topline=false,
   rightline=false,
   bottomline=false,
   leftmargin=\parindent,
   skipabove=8pt,
   skipbelow=8pt,
   linecolor=blue,
   innerbottommargin=10pt,
   % backgroundcolor=bg,font=\color{orange}\sffamily, fontcolor=white
]{definition}

\usepackage{empheq}
\usepackage[most]{tcolorbox}

\newtcbox{\mymath}[1][]{%
    nobeforeafter, math upper, tcbox raise base,
    enhanced, colframe=blue!30!black,
    colback=red!10, boxrule=1pt,
    #1}

\usepackage{unixode}


\DeclareMathOperator{\ord}{ord}
\DeclareMathOperator{\orb}{orb}
\DeclareMathOperator{\stab}{stab}
\DeclareMathOperator{\Stab}{stab}
\DeclareMathOperator{\ppcm}{ppcm}
\DeclareMathOperator{\conj}{Conj}
\DeclareMathOperator{\End}{End}
\DeclareMathOperator{\rot}{rot}
\DeclareMathOperator{\trs}{trace}
\DeclareMathOperator{\Ind}{Ind}
\DeclareMathOperator{\mat}{Mat}
\DeclareMathOperator{\id}{Id}
\DeclareMathOperator{\vect}{vect}
\DeclareMathOperator{\img}{img}
\DeclareMathOperator{\cov}{Cov}
\DeclareMathOperator{\dist}{dist}
\DeclareMathOperator{\irr}{Irr}
\DeclareMathOperator{\image}{Im}
\DeclareMathOperator{\pd}{\partial}
\DeclareMathOperator{\epi}{epi}
\DeclareMathOperator{\Argmin}{Argmin}
\DeclareMathOperator{\dom}{dom}
\DeclareMathOperator{\proj}{proj}
\DeclareMathOperator{\ctg}{ctg}
\DeclareMathOperator{\supp}{supp}
\DeclareMathOperator{\argmin}{argmin}
\DeclareMathOperator{\mult}{mult}
\DeclareMathOperator{\ch}{ch}
\DeclareMathOperator{\sh}{sh}
\DeclareMathOperator{\rang}{rang}
\DeclareMathOperator{\diam}{diam}
\DeclareMathOperator{\Epigraphe}{Epigraphe}




\usepackage{xcolor}
\everymath{\color{blue}}
%\everymath{\color[rgb]{0,1,1}}
%\pagecolor[rgb]{0,0,0.5}


\newcommand*{\pdtest}[3][]{\ensuremath{\frac{\partial^{#1} #2}{\partial #3}}}

\newcommand*{\deffunc}[6][]{\ensuremath{
\begin{array}{rcl}
#2 : #3 &\rightarrow& #4\\
#5 &\mapsto& #6
\end{array}
}}

\newcommand{\eqcolon}{\mathrel{\resizebox{\widthof{$\mathord{=}$}}{\height}{ $\!\!=\!\!\resizebox{1.2\width}{0.8\height}{\raisebox{0.23ex}{$\mathop{:}$}}\!\!$ }}}
\newcommand{\coloneq}{\mathrel{\resizebox{\widthof{$\mathord{=}$}}{\height}{ $\!\!\resizebox{1.2\width}{0.8\height}{\raisebox{0.23ex}{$\mathop{:}$}}\!\!=\!\!$ }}}
\newcommand{\eqcolonl}{\ensuremath{\mathrel{=\!\!\mathop{:}}}}
\newcommand{\coloneql}{\ensuremath{\mathrel{\mathop{:} \!\! =}}}
\newcommand{\vc}[1]{% inline column vector
  \left(\begin{smallmatrix}#1\end{smallmatrix}\right)%
}
\newcommand{\vr}[1]{% inline row vector
  \begin{smallmatrix}(\,#1\,)\end{smallmatrix}%
}
\makeatletter
\newcommand*{\defeq}{\ =\mathrel{\rlap{%
                     \raisebox{0.3ex}{$\m@th\cdot$}}%
                     \raisebox{-0.3ex}{$\m@th\cdot$}}%
                     }
\makeatother

\newcommand{\mathcircle}[1]{% inline row vector
 \overset{\circ}{#1}
}
\newcommand{\ulim}{% low limit
 \underline{\lim}
}
\newcommand{\ssi}{% iff
\iff
}
\newcommand{\ps}[2]{
\expval{#1 | #2}
}
\newcommand{\df}[1]{
\mqty{#1}
}
\newcommand{\n}[1]{
\norm{#1}
}
\newcommand{\sys}[1]{
\left\{\smqty{#1}\right.
}


\newcommand{\eqdef}{\ensuremath{\overset{\text{def}}=}}


\def\Circlearrowright{\ensuremath{%
  \rotatebox[origin=c]{230}{$\circlearrowright$}}}

\newcommand\ct[1]{\text{\rmfamily\upshape #1}}
\newcommand\question[1]{ {\color{red} ...!? \small #1}}
\newcommand\caz[1]{\left\{\begin{array} #1 \end{array}\right.}
\newcommand\const{\text{\rmfamily\upshape const}}
\newcommand\toP{ \overset{\pro}{\to}}
\newcommand\toPP{ \overset{\text{PP}}{\to}}
\newcommand{\oeq}{\mathrel{\text{\textcircled{$=$}}}}





\usepackage{xcolor}
% \usepackage[normalem]{ulem}
\usepackage{lipsum}
\makeatletter
% \newcommand\colorwave[1][blue]{\bgroup \markoverwith{\lower3.5\p@\hbox{\sixly \textcolor{#1}{\char58}}}\ULon}
%\font\sixly=lasy6 % does not re-load if already loaded, so no memory problem.

\newmdtheoremenv[
linewidth= 1pt,linecolor= blue,%
leftmargin=20,rightmargin=20,innertopmargin=0pt, innerrightmargin=40,%
tikzsetting = { draw=lightgray, line width = 0.3pt,dashed,%
dash pattern = on 15pt off 3pt},%
splittopskip=\topskip,skipbelow=\baselineskip,%
skipabove=\baselineskip,ntheorem,roundcorner=0pt,
% backgroundcolor=pagebg,font=\color{orange}\sffamily, fontcolor=white
]{examplebox}{Exemple}[section]



\newcommand\R{\mathbb{R}}
\newcommand\Z{\mathbb{Z}}
\newcommand\N{\mathbb{N}}
\newcommand\E{\mathbb{E}}
\newcommand\F{\mathcal{F}}
\newcommand\cH{\mathcal{H}}
\newcommand\V{\mathbb{V}}
\newcommand\dmo{ ^{-1} }
\newcommand\kapa{\kappa}
\newcommand\im{Im}
\newcommand\hs{\mathcal{H}}





\usepackage{soul}

\makeatletter
\newcommand*{\whiten}[1]{\llap{\textcolor{white}{{\the\SOUL@token}}\hspace{#1pt}}}
\DeclareRobustCommand*\myul{%
    \def\SOUL@everyspace{\underline{\space}\kern\z@}%
    \def\SOUL@everytoken{%
     \setbox0=\hbox{\the\SOUL@token}%
     \ifdim\dp0>\z@
        \raisebox{\dp0}{\underline{\phantom{\the\SOUL@token}}}%
        \whiten{1}\whiten{0}%
        \whiten{-1}\whiten{-2}%
        \llap{\the\SOUL@token}%
     \else
        \underline{\the\SOUL@token}%
     \fi}%
\SOUL@}
\makeatother

\newcommand*{\demp}{\fontfamily{lmtt}\selectfont}

\DeclareTextFontCommand{\textdemp}{\demp}

\begin{document}

\ifcomment
Multiline
comment
\fi
\ifcomment
\myul{Typesetting test}
% \color[rgb]{1,1,1}
$∑_i^n≠ 60º±∞π∆¬≈√j∫h≤≥µ$

$\CR \R\pro\ind\pro\gS\pro
\mqty[a&b\\c&d]$
$\pro\mathbb{P}$
$\dd{x}$

  \[
    \alpha(x)=\left\{
                \begin{array}{ll}
                  x\\
                  \frac{1}{1+e^{-kx}}\\
                  \frac{e^x-e^{-x}}{e^x+e^{-x}}
                \end{array}
              \right.
  \]

  $\expval{x}$
  
  $\chi_\rho(ghg\dmo)=\Tr(\rho_{ghg\dmo})=\Tr(\rho_g\circ\rho_h\circ\rho\dmo_g)=\Tr(\rho_h)\overset{\mbox{\scalebox{0.5}{$\Tr(AB)=\Tr(BA)$}}}{=}\chi_\rho(h)$
  	$\mathop{\oplus}_{\substack{x\in X}}$

$\mat(\rho_g)=(a_{ij}(g))_{\scriptsize \substack{1\leq i\leq d \\ 1\leq j\leq d}}$ et $\mat(\rho'_g)=(a'_{ij}(g))_{\scriptsize \substack{1\leq i'\leq d' \\ 1\leq j'\leq d'}}$



\[\int_a^b{\mathbb{R}^2}g(u, v)\dd{P_{XY}}(u, v)=\iint g(u,v) f_{XY}(u, v)\dd \lambda(u) \dd \lambda(v)\]
$$\lim_{x\to\infty} f(x)$$	
$$\iiiint_V \mu(t,u,v,w) \,dt\,du\,dv\,dw$$
$$\sum_{n=1}^{\infty} 2^{-n} = 1$$	
\begin{definition}
	Si $X$ et $Y$ sont 2 v.a. ou definit la \textsc{Covariance} entre $X$ et $Y$ comme
	$\cov(X,Y)\overset{\text{def}}{=}\E\left[(X-\E(X))(Y-\E(Y))\right]=\E(XY)-\E(X)\E(Y)$.
\end{definition}
\fi
\pagebreak

% \tableofcontents

% insert your code here
%\input{./algebra/main.tex}
%\input{./geometrie-differentielle/main.tex}
%\input{./probabilite/main.tex}
%\input{./analyse-fonctionnelle/main.tex}
% \input{./Analyse-convexe-et-dualite-en-optimisation/main.tex}
%\input{./tikz/main.tex}
%\input{./Theorie-du-distributions/main.tex}
%\input{./optimisation/mine.tex}
 \input{./modelisation/main.tex}

% yves.aubry@univ-tln.fr : algebra

\end{document}

%% !TEX encoding = UTF-8 Unicode
% !TEX TS-program = xelatex

\documentclass[french]{report}

%\usepackage[utf8]{inputenc}
%\usepackage[T1]{fontenc}
\usepackage{babel}


\newif\ifcomment
%\commenttrue # Show comments

\usepackage{physics}
\usepackage{amssymb}


\usepackage{amsthm}
% \usepackage{thmtools}
\usepackage{mathtools}
\usepackage{amsfonts}

\usepackage{color}

\usepackage{tikz}

\usepackage{geometry}
\geometry{a5paper, margin=0.1in, right=1cm}

\usepackage{dsfont}

\usepackage{graphicx}
\graphicspath{ {images/} }

\usepackage{faktor}

\usepackage{IEEEtrantools}
\usepackage{enumerate}   
\usepackage[PostScript=dvips]{"/Users/aware/Documents/Courses/diagrams"}


\newtheorem{theorem}{Théorème}[section]
\renewcommand{\thetheorem}{\arabic{theorem}}
\newtheorem{lemme}{Lemme}[section]
\renewcommand{\thelemme}{\arabic{lemme}}
\newtheorem{proposition}{Proposition}[section]
\renewcommand{\theproposition}{\arabic{proposition}}
\newtheorem{notations}{Notations}[section]
\newtheorem{problem}{Problème}[section]
\newtheorem{corollary}{Corollaire}[theorem]
\renewcommand{\thecorollary}{\arabic{corollary}}
\newtheorem{property}{Propriété}[section]
\newtheorem{objective}{Objectif}[section]

\theoremstyle{definition}
\newtheorem{definition}{Définition}[section]
\renewcommand{\thedefinition}{\arabic{definition}}
\newtheorem{exercise}{Exercice}[chapter]
\renewcommand{\theexercise}{\arabic{exercise}}
\newtheorem{example}{Exemple}[chapter]
\renewcommand{\theexample}{\arabic{example}}
\newtheorem*{solution}{Solution}
\newtheorem*{application}{Application}
\newtheorem*{notation}{Notation}
\newtheorem*{vocabulary}{Vocabulaire}
\newtheorem*{properties}{Propriétés}



\theoremstyle{remark}
\newtheorem*{remark}{Remarque}
\newtheorem*{rappel}{Rappel}


\usepackage{etoolbox}
\AtBeginEnvironment{exercise}{\small}
\AtBeginEnvironment{example}{\small}

\usepackage{cases}
\usepackage[red]{mypack}

\usepackage[framemethod=TikZ]{mdframed}

\definecolor{bg}{rgb}{0.4,0.25,0.95}
\definecolor{pagebg}{rgb}{0,0,0.5}
\surroundwithmdframed[
   topline=false,
   rightline=false,
   bottomline=false,
   leftmargin=\parindent,
   skipabove=8pt,
   skipbelow=8pt,
   linecolor=blue,
   innerbottommargin=10pt,
   % backgroundcolor=bg,font=\color{orange}\sffamily, fontcolor=white
]{definition}

\usepackage{empheq}
\usepackage[most]{tcolorbox}

\newtcbox{\mymath}[1][]{%
    nobeforeafter, math upper, tcbox raise base,
    enhanced, colframe=blue!30!black,
    colback=red!10, boxrule=1pt,
    #1}

\usepackage{unixode}


\DeclareMathOperator{\ord}{ord}
\DeclareMathOperator{\orb}{orb}
\DeclareMathOperator{\stab}{stab}
\DeclareMathOperator{\Stab}{stab}
\DeclareMathOperator{\ppcm}{ppcm}
\DeclareMathOperator{\conj}{Conj}
\DeclareMathOperator{\End}{End}
\DeclareMathOperator{\rot}{rot}
\DeclareMathOperator{\trs}{trace}
\DeclareMathOperator{\Ind}{Ind}
\DeclareMathOperator{\mat}{Mat}
\DeclareMathOperator{\id}{Id}
\DeclareMathOperator{\vect}{vect}
\DeclareMathOperator{\img}{img}
\DeclareMathOperator{\cov}{Cov}
\DeclareMathOperator{\dist}{dist}
\DeclareMathOperator{\irr}{Irr}
\DeclareMathOperator{\image}{Im}
\DeclareMathOperator{\pd}{\partial}
\DeclareMathOperator{\epi}{epi}
\DeclareMathOperator{\Argmin}{Argmin}
\DeclareMathOperator{\dom}{dom}
\DeclareMathOperator{\proj}{proj}
\DeclareMathOperator{\ctg}{ctg}
\DeclareMathOperator{\supp}{supp}
\DeclareMathOperator{\argmin}{argmin}
\DeclareMathOperator{\mult}{mult}
\DeclareMathOperator{\ch}{ch}
\DeclareMathOperator{\sh}{sh}
\DeclareMathOperator{\rang}{rang}
\DeclareMathOperator{\diam}{diam}
\DeclareMathOperator{\Epigraphe}{Epigraphe}




\usepackage{xcolor}
\everymath{\color{blue}}
%\everymath{\color[rgb]{0,1,1}}
%\pagecolor[rgb]{0,0,0.5}


\newcommand*{\pdtest}[3][]{\ensuremath{\frac{\partial^{#1} #2}{\partial #3}}}

\newcommand*{\deffunc}[6][]{\ensuremath{
\begin{array}{rcl}
#2 : #3 &\rightarrow& #4\\
#5 &\mapsto& #6
\end{array}
}}

\newcommand{\eqcolon}{\mathrel{\resizebox{\widthof{$\mathord{=}$}}{\height}{ $\!\!=\!\!\resizebox{1.2\width}{0.8\height}{\raisebox{0.23ex}{$\mathop{:}$}}\!\!$ }}}
\newcommand{\coloneq}{\mathrel{\resizebox{\widthof{$\mathord{=}$}}{\height}{ $\!\!\resizebox{1.2\width}{0.8\height}{\raisebox{0.23ex}{$\mathop{:}$}}\!\!=\!\!$ }}}
\newcommand{\eqcolonl}{\ensuremath{\mathrel{=\!\!\mathop{:}}}}
\newcommand{\coloneql}{\ensuremath{\mathrel{\mathop{:} \!\! =}}}
\newcommand{\vc}[1]{% inline column vector
  \left(\begin{smallmatrix}#1\end{smallmatrix}\right)%
}
\newcommand{\vr}[1]{% inline row vector
  \begin{smallmatrix}(\,#1\,)\end{smallmatrix}%
}
\makeatletter
\newcommand*{\defeq}{\ =\mathrel{\rlap{%
                     \raisebox{0.3ex}{$\m@th\cdot$}}%
                     \raisebox{-0.3ex}{$\m@th\cdot$}}%
                     }
\makeatother

\newcommand{\mathcircle}[1]{% inline row vector
 \overset{\circ}{#1}
}
\newcommand{\ulim}{% low limit
 \underline{\lim}
}
\newcommand{\ssi}{% iff
\iff
}
\newcommand{\ps}[2]{
\expval{#1 | #2}
}
\newcommand{\df}[1]{
\mqty{#1}
}
\newcommand{\n}[1]{
\norm{#1}
}
\newcommand{\sys}[1]{
\left\{\smqty{#1}\right.
}


\newcommand{\eqdef}{\ensuremath{\overset{\text{def}}=}}


\def\Circlearrowright{\ensuremath{%
  \rotatebox[origin=c]{230}{$\circlearrowright$}}}

\newcommand\ct[1]{\text{\rmfamily\upshape #1}}
\newcommand\question[1]{ {\color{red} ...!? \small #1}}
\newcommand\caz[1]{\left\{\begin{array} #1 \end{array}\right.}
\newcommand\const{\text{\rmfamily\upshape const}}
\newcommand\toP{ \overset{\pro}{\to}}
\newcommand\toPP{ \overset{\text{PP}}{\to}}
\newcommand{\oeq}{\mathrel{\text{\textcircled{$=$}}}}





\usepackage{xcolor}
% \usepackage[normalem]{ulem}
\usepackage{lipsum}
\makeatletter
% \newcommand\colorwave[1][blue]{\bgroup \markoverwith{\lower3.5\p@\hbox{\sixly \textcolor{#1}{\char58}}}\ULon}
%\font\sixly=lasy6 % does not re-load if already loaded, so no memory problem.

\newmdtheoremenv[
linewidth= 1pt,linecolor= blue,%
leftmargin=20,rightmargin=20,innertopmargin=0pt, innerrightmargin=40,%
tikzsetting = { draw=lightgray, line width = 0.3pt,dashed,%
dash pattern = on 15pt off 3pt},%
splittopskip=\topskip,skipbelow=\baselineskip,%
skipabove=\baselineskip,ntheorem,roundcorner=0pt,
% backgroundcolor=pagebg,font=\color{orange}\sffamily, fontcolor=white
]{examplebox}{Exemple}[section]



\newcommand\R{\mathbb{R}}
\newcommand\Z{\mathbb{Z}}
\newcommand\N{\mathbb{N}}
\newcommand\E{\mathbb{E}}
\newcommand\F{\mathcal{F}}
\newcommand\cH{\mathcal{H}}
\newcommand\V{\mathbb{V}}
\newcommand\dmo{ ^{-1} }
\newcommand\kapa{\kappa}
\newcommand\im{Im}
\newcommand\hs{\mathcal{H}}





\usepackage{soul}

\makeatletter
\newcommand*{\whiten}[1]{\llap{\textcolor{white}{{\the\SOUL@token}}\hspace{#1pt}}}
\DeclareRobustCommand*\myul{%
    \def\SOUL@everyspace{\underline{\space}\kern\z@}%
    \def\SOUL@everytoken{%
     \setbox0=\hbox{\the\SOUL@token}%
     \ifdim\dp0>\z@
        \raisebox{\dp0}{\underline{\phantom{\the\SOUL@token}}}%
        \whiten{1}\whiten{0}%
        \whiten{-1}\whiten{-2}%
        \llap{\the\SOUL@token}%
     \else
        \underline{\the\SOUL@token}%
     \fi}%
\SOUL@}
\makeatother

\newcommand*{\demp}{\fontfamily{lmtt}\selectfont}

\DeclareTextFontCommand{\textdemp}{\demp}

\begin{document}

\ifcomment
Multiline
comment
\fi
\ifcomment
\myul{Typesetting test}
% \color[rgb]{1,1,1}
$∑_i^n≠ 60º±∞π∆¬≈√j∫h≤≥µ$

$\CR \R\pro\ind\pro\gS\pro
\mqty[a&b\\c&d]$
$\pro\mathbb{P}$
$\dd{x}$

  \[
    \alpha(x)=\left\{
                \begin{array}{ll}
                  x\\
                  \frac{1}{1+e^{-kx}}\\
                  \frac{e^x-e^{-x}}{e^x+e^{-x}}
                \end{array}
              \right.
  \]

  $\expval{x}$
  
  $\chi_\rho(ghg\dmo)=\Tr(\rho_{ghg\dmo})=\Tr(\rho_g\circ\rho_h\circ\rho\dmo_g)=\Tr(\rho_h)\overset{\mbox{\scalebox{0.5}{$\Tr(AB)=\Tr(BA)$}}}{=}\chi_\rho(h)$
  	$\mathop{\oplus}_{\substack{x\in X}}$

$\mat(\rho_g)=(a_{ij}(g))_{\scriptsize \substack{1\leq i\leq d \\ 1\leq j\leq d}}$ et $\mat(\rho'_g)=(a'_{ij}(g))_{\scriptsize \substack{1\leq i'\leq d' \\ 1\leq j'\leq d'}}$



\[\int_a^b{\mathbb{R}^2}g(u, v)\dd{P_{XY}}(u, v)=\iint g(u,v) f_{XY}(u, v)\dd \lambda(u) \dd \lambda(v)\]
$$\lim_{x\to\infty} f(x)$$	
$$\iiiint_V \mu(t,u,v,w) \,dt\,du\,dv\,dw$$
$$\sum_{n=1}^{\infty} 2^{-n} = 1$$	
\begin{definition}
	Si $X$ et $Y$ sont 2 v.a. ou definit la \textsc{Covariance} entre $X$ et $Y$ comme
	$\cov(X,Y)\overset{\text{def}}{=}\E\left[(X-\E(X))(Y-\E(Y))\right]=\E(XY)-\E(X)\E(Y)$.
\end{definition}
\fi
\pagebreak

% \tableofcontents

% insert your code here
%\input{./algebra/main.tex}
%\input{./geometrie-differentielle/main.tex}
%\input{./probabilite/main.tex}
%\input{./analyse-fonctionnelle/main.tex}
% \input{./Analyse-convexe-et-dualite-en-optimisation/main.tex}
%\input{./tikz/main.tex}
%\input{./Theorie-du-distributions/main.tex}
%\input{./optimisation/mine.tex}
 \input{./modelisation/main.tex}

% yves.aubry@univ-tln.fr : algebra

\end{document}

%% !TEX encoding = UTF-8 Unicode
% !TEX TS-program = xelatex

\documentclass[french]{report}

%\usepackage[utf8]{inputenc}
%\usepackage[T1]{fontenc}
\usepackage{babel}


\newif\ifcomment
%\commenttrue # Show comments

\usepackage{physics}
\usepackage{amssymb}


\usepackage{amsthm}
% \usepackage{thmtools}
\usepackage{mathtools}
\usepackage{amsfonts}

\usepackage{color}

\usepackage{tikz}

\usepackage{geometry}
\geometry{a5paper, margin=0.1in, right=1cm}

\usepackage{dsfont}

\usepackage{graphicx}
\graphicspath{ {images/} }

\usepackage{faktor}

\usepackage{IEEEtrantools}
\usepackage{enumerate}   
\usepackage[PostScript=dvips]{"/Users/aware/Documents/Courses/diagrams"}


\newtheorem{theorem}{Théorème}[section]
\renewcommand{\thetheorem}{\arabic{theorem}}
\newtheorem{lemme}{Lemme}[section]
\renewcommand{\thelemme}{\arabic{lemme}}
\newtheorem{proposition}{Proposition}[section]
\renewcommand{\theproposition}{\arabic{proposition}}
\newtheorem{notations}{Notations}[section]
\newtheorem{problem}{Problème}[section]
\newtheorem{corollary}{Corollaire}[theorem]
\renewcommand{\thecorollary}{\arabic{corollary}}
\newtheorem{property}{Propriété}[section]
\newtheorem{objective}{Objectif}[section]

\theoremstyle{definition}
\newtheorem{definition}{Définition}[section]
\renewcommand{\thedefinition}{\arabic{definition}}
\newtheorem{exercise}{Exercice}[chapter]
\renewcommand{\theexercise}{\arabic{exercise}}
\newtheorem{example}{Exemple}[chapter]
\renewcommand{\theexample}{\arabic{example}}
\newtheorem*{solution}{Solution}
\newtheorem*{application}{Application}
\newtheorem*{notation}{Notation}
\newtheorem*{vocabulary}{Vocabulaire}
\newtheorem*{properties}{Propriétés}



\theoremstyle{remark}
\newtheorem*{remark}{Remarque}
\newtheorem*{rappel}{Rappel}


\usepackage{etoolbox}
\AtBeginEnvironment{exercise}{\small}
\AtBeginEnvironment{example}{\small}

\usepackage{cases}
\usepackage[red]{mypack}

\usepackage[framemethod=TikZ]{mdframed}

\definecolor{bg}{rgb}{0.4,0.25,0.95}
\definecolor{pagebg}{rgb}{0,0,0.5}
\surroundwithmdframed[
   topline=false,
   rightline=false,
   bottomline=false,
   leftmargin=\parindent,
   skipabove=8pt,
   skipbelow=8pt,
   linecolor=blue,
   innerbottommargin=10pt,
   % backgroundcolor=bg,font=\color{orange}\sffamily, fontcolor=white
]{definition}

\usepackage{empheq}
\usepackage[most]{tcolorbox}

\newtcbox{\mymath}[1][]{%
    nobeforeafter, math upper, tcbox raise base,
    enhanced, colframe=blue!30!black,
    colback=red!10, boxrule=1pt,
    #1}

\usepackage{unixode}


\DeclareMathOperator{\ord}{ord}
\DeclareMathOperator{\orb}{orb}
\DeclareMathOperator{\stab}{stab}
\DeclareMathOperator{\Stab}{stab}
\DeclareMathOperator{\ppcm}{ppcm}
\DeclareMathOperator{\conj}{Conj}
\DeclareMathOperator{\End}{End}
\DeclareMathOperator{\rot}{rot}
\DeclareMathOperator{\trs}{trace}
\DeclareMathOperator{\Ind}{Ind}
\DeclareMathOperator{\mat}{Mat}
\DeclareMathOperator{\id}{Id}
\DeclareMathOperator{\vect}{vect}
\DeclareMathOperator{\img}{img}
\DeclareMathOperator{\cov}{Cov}
\DeclareMathOperator{\dist}{dist}
\DeclareMathOperator{\irr}{Irr}
\DeclareMathOperator{\image}{Im}
\DeclareMathOperator{\pd}{\partial}
\DeclareMathOperator{\epi}{epi}
\DeclareMathOperator{\Argmin}{Argmin}
\DeclareMathOperator{\dom}{dom}
\DeclareMathOperator{\proj}{proj}
\DeclareMathOperator{\ctg}{ctg}
\DeclareMathOperator{\supp}{supp}
\DeclareMathOperator{\argmin}{argmin}
\DeclareMathOperator{\mult}{mult}
\DeclareMathOperator{\ch}{ch}
\DeclareMathOperator{\sh}{sh}
\DeclareMathOperator{\rang}{rang}
\DeclareMathOperator{\diam}{diam}
\DeclareMathOperator{\Epigraphe}{Epigraphe}




\usepackage{xcolor}
\everymath{\color{blue}}
%\everymath{\color[rgb]{0,1,1}}
%\pagecolor[rgb]{0,0,0.5}


\newcommand*{\pdtest}[3][]{\ensuremath{\frac{\partial^{#1} #2}{\partial #3}}}

\newcommand*{\deffunc}[6][]{\ensuremath{
\begin{array}{rcl}
#2 : #3 &\rightarrow& #4\\
#5 &\mapsto& #6
\end{array}
}}

\newcommand{\eqcolon}{\mathrel{\resizebox{\widthof{$\mathord{=}$}}{\height}{ $\!\!=\!\!\resizebox{1.2\width}{0.8\height}{\raisebox{0.23ex}{$\mathop{:}$}}\!\!$ }}}
\newcommand{\coloneq}{\mathrel{\resizebox{\widthof{$\mathord{=}$}}{\height}{ $\!\!\resizebox{1.2\width}{0.8\height}{\raisebox{0.23ex}{$\mathop{:}$}}\!\!=\!\!$ }}}
\newcommand{\eqcolonl}{\ensuremath{\mathrel{=\!\!\mathop{:}}}}
\newcommand{\coloneql}{\ensuremath{\mathrel{\mathop{:} \!\! =}}}
\newcommand{\vc}[1]{% inline column vector
  \left(\begin{smallmatrix}#1\end{smallmatrix}\right)%
}
\newcommand{\vr}[1]{% inline row vector
  \begin{smallmatrix}(\,#1\,)\end{smallmatrix}%
}
\makeatletter
\newcommand*{\defeq}{\ =\mathrel{\rlap{%
                     \raisebox{0.3ex}{$\m@th\cdot$}}%
                     \raisebox{-0.3ex}{$\m@th\cdot$}}%
                     }
\makeatother

\newcommand{\mathcircle}[1]{% inline row vector
 \overset{\circ}{#1}
}
\newcommand{\ulim}{% low limit
 \underline{\lim}
}
\newcommand{\ssi}{% iff
\iff
}
\newcommand{\ps}[2]{
\expval{#1 | #2}
}
\newcommand{\df}[1]{
\mqty{#1}
}
\newcommand{\n}[1]{
\norm{#1}
}
\newcommand{\sys}[1]{
\left\{\smqty{#1}\right.
}


\newcommand{\eqdef}{\ensuremath{\overset{\text{def}}=}}


\def\Circlearrowright{\ensuremath{%
  \rotatebox[origin=c]{230}{$\circlearrowright$}}}

\newcommand\ct[1]{\text{\rmfamily\upshape #1}}
\newcommand\question[1]{ {\color{red} ...!? \small #1}}
\newcommand\caz[1]{\left\{\begin{array} #1 \end{array}\right.}
\newcommand\const{\text{\rmfamily\upshape const}}
\newcommand\toP{ \overset{\pro}{\to}}
\newcommand\toPP{ \overset{\text{PP}}{\to}}
\newcommand{\oeq}{\mathrel{\text{\textcircled{$=$}}}}





\usepackage{xcolor}
% \usepackage[normalem]{ulem}
\usepackage{lipsum}
\makeatletter
% \newcommand\colorwave[1][blue]{\bgroup \markoverwith{\lower3.5\p@\hbox{\sixly \textcolor{#1}{\char58}}}\ULon}
%\font\sixly=lasy6 % does not re-load if already loaded, so no memory problem.

\newmdtheoremenv[
linewidth= 1pt,linecolor= blue,%
leftmargin=20,rightmargin=20,innertopmargin=0pt, innerrightmargin=40,%
tikzsetting = { draw=lightgray, line width = 0.3pt,dashed,%
dash pattern = on 15pt off 3pt},%
splittopskip=\topskip,skipbelow=\baselineskip,%
skipabove=\baselineskip,ntheorem,roundcorner=0pt,
% backgroundcolor=pagebg,font=\color{orange}\sffamily, fontcolor=white
]{examplebox}{Exemple}[section]



\newcommand\R{\mathbb{R}}
\newcommand\Z{\mathbb{Z}}
\newcommand\N{\mathbb{N}}
\newcommand\E{\mathbb{E}}
\newcommand\F{\mathcal{F}}
\newcommand\cH{\mathcal{H}}
\newcommand\V{\mathbb{V}}
\newcommand\dmo{ ^{-1} }
\newcommand\kapa{\kappa}
\newcommand\im{Im}
\newcommand\hs{\mathcal{H}}





\usepackage{soul}

\makeatletter
\newcommand*{\whiten}[1]{\llap{\textcolor{white}{{\the\SOUL@token}}\hspace{#1pt}}}
\DeclareRobustCommand*\myul{%
    \def\SOUL@everyspace{\underline{\space}\kern\z@}%
    \def\SOUL@everytoken{%
     \setbox0=\hbox{\the\SOUL@token}%
     \ifdim\dp0>\z@
        \raisebox{\dp0}{\underline{\phantom{\the\SOUL@token}}}%
        \whiten{1}\whiten{0}%
        \whiten{-1}\whiten{-2}%
        \llap{\the\SOUL@token}%
     \else
        \underline{\the\SOUL@token}%
     \fi}%
\SOUL@}
\makeatother

\newcommand*{\demp}{\fontfamily{lmtt}\selectfont}

\DeclareTextFontCommand{\textdemp}{\demp}

\begin{document}

\ifcomment
Multiline
comment
\fi
\ifcomment
\myul{Typesetting test}
% \color[rgb]{1,1,1}
$∑_i^n≠ 60º±∞π∆¬≈√j∫h≤≥µ$

$\CR \R\pro\ind\pro\gS\pro
\mqty[a&b\\c&d]$
$\pro\mathbb{P}$
$\dd{x}$

  \[
    \alpha(x)=\left\{
                \begin{array}{ll}
                  x\\
                  \frac{1}{1+e^{-kx}}\\
                  \frac{e^x-e^{-x}}{e^x+e^{-x}}
                \end{array}
              \right.
  \]

  $\expval{x}$
  
  $\chi_\rho(ghg\dmo)=\Tr(\rho_{ghg\dmo})=\Tr(\rho_g\circ\rho_h\circ\rho\dmo_g)=\Tr(\rho_h)\overset{\mbox{\scalebox{0.5}{$\Tr(AB)=\Tr(BA)$}}}{=}\chi_\rho(h)$
  	$\mathop{\oplus}_{\substack{x\in X}}$

$\mat(\rho_g)=(a_{ij}(g))_{\scriptsize \substack{1\leq i\leq d \\ 1\leq j\leq d}}$ et $\mat(\rho'_g)=(a'_{ij}(g))_{\scriptsize \substack{1\leq i'\leq d' \\ 1\leq j'\leq d'}}$



\[\int_a^b{\mathbb{R}^2}g(u, v)\dd{P_{XY}}(u, v)=\iint g(u,v) f_{XY}(u, v)\dd \lambda(u) \dd \lambda(v)\]
$$\lim_{x\to\infty} f(x)$$	
$$\iiiint_V \mu(t,u,v,w) \,dt\,du\,dv\,dw$$
$$\sum_{n=1}^{\infty} 2^{-n} = 1$$	
\begin{definition}
	Si $X$ et $Y$ sont 2 v.a. ou definit la \textsc{Covariance} entre $X$ et $Y$ comme
	$\cov(X,Y)\overset{\text{def}}{=}\E\left[(X-\E(X))(Y-\E(Y))\right]=\E(XY)-\E(X)\E(Y)$.
\end{definition}
\fi
\pagebreak

% \tableofcontents

% insert your code here
%\input{./algebra/main.tex}
%\input{./geometrie-differentielle/main.tex}
%\input{./probabilite/main.tex}
%\input{./analyse-fonctionnelle/main.tex}
% \input{./Analyse-convexe-et-dualite-en-optimisation/main.tex}
%\input{./tikz/main.tex}
%\input{./Theorie-du-distributions/main.tex}
%\input{./optimisation/mine.tex}
 \input{./modelisation/main.tex}

% yves.aubry@univ-tln.fr : algebra

\end{document}

%% !TEX encoding = UTF-8 Unicode
% !TEX TS-program = xelatex

\documentclass[french]{report}

%\usepackage[utf8]{inputenc}
%\usepackage[T1]{fontenc}
\usepackage{babel}


\newif\ifcomment
%\commenttrue # Show comments

\usepackage{physics}
\usepackage{amssymb}


\usepackage{amsthm}
% \usepackage{thmtools}
\usepackage{mathtools}
\usepackage{amsfonts}

\usepackage{color}

\usepackage{tikz}

\usepackage{geometry}
\geometry{a5paper, margin=0.1in, right=1cm}

\usepackage{dsfont}

\usepackage{graphicx}
\graphicspath{ {images/} }

\usepackage{faktor}

\usepackage{IEEEtrantools}
\usepackage{enumerate}   
\usepackage[PostScript=dvips]{"/Users/aware/Documents/Courses/diagrams"}


\newtheorem{theorem}{Théorème}[section]
\renewcommand{\thetheorem}{\arabic{theorem}}
\newtheorem{lemme}{Lemme}[section]
\renewcommand{\thelemme}{\arabic{lemme}}
\newtheorem{proposition}{Proposition}[section]
\renewcommand{\theproposition}{\arabic{proposition}}
\newtheorem{notations}{Notations}[section]
\newtheorem{problem}{Problème}[section]
\newtheorem{corollary}{Corollaire}[theorem]
\renewcommand{\thecorollary}{\arabic{corollary}}
\newtheorem{property}{Propriété}[section]
\newtheorem{objective}{Objectif}[section]

\theoremstyle{definition}
\newtheorem{definition}{Définition}[section]
\renewcommand{\thedefinition}{\arabic{definition}}
\newtheorem{exercise}{Exercice}[chapter]
\renewcommand{\theexercise}{\arabic{exercise}}
\newtheorem{example}{Exemple}[chapter]
\renewcommand{\theexample}{\arabic{example}}
\newtheorem*{solution}{Solution}
\newtheorem*{application}{Application}
\newtheorem*{notation}{Notation}
\newtheorem*{vocabulary}{Vocabulaire}
\newtheorem*{properties}{Propriétés}



\theoremstyle{remark}
\newtheorem*{remark}{Remarque}
\newtheorem*{rappel}{Rappel}


\usepackage{etoolbox}
\AtBeginEnvironment{exercise}{\small}
\AtBeginEnvironment{example}{\small}

\usepackage{cases}
\usepackage[red]{mypack}

\usepackage[framemethod=TikZ]{mdframed}

\definecolor{bg}{rgb}{0.4,0.25,0.95}
\definecolor{pagebg}{rgb}{0,0,0.5}
\surroundwithmdframed[
   topline=false,
   rightline=false,
   bottomline=false,
   leftmargin=\parindent,
   skipabove=8pt,
   skipbelow=8pt,
   linecolor=blue,
   innerbottommargin=10pt,
   % backgroundcolor=bg,font=\color{orange}\sffamily, fontcolor=white
]{definition}

\usepackage{empheq}
\usepackage[most]{tcolorbox}

\newtcbox{\mymath}[1][]{%
    nobeforeafter, math upper, tcbox raise base,
    enhanced, colframe=blue!30!black,
    colback=red!10, boxrule=1pt,
    #1}

\usepackage{unixode}


\DeclareMathOperator{\ord}{ord}
\DeclareMathOperator{\orb}{orb}
\DeclareMathOperator{\stab}{stab}
\DeclareMathOperator{\Stab}{stab}
\DeclareMathOperator{\ppcm}{ppcm}
\DeclareMathOperator{\conj}{Conj}
\DeclareMathOperator{\End}{End}
\DeclareMathOperator{\rot}{rot}
\DeclareMathOperator{\trs}{trace}
\DeclareMathOperator{\Ind}{Ind}
\DeclareMathOperator{\mat}{Mat}
\DeclareMathOperator{\id}{Id}
\DeclareMathOperator{\vect}{vect}
\DeclareMathOperator{\img}{img}
\DeclareMathOperator{\cov}{Cov}
\DeclareMathOperator{\dist}{dist}
\DeclareMathOperator{\irr}{Irr}
\DeclareMathOperator{\image}{Im}
\DeclareMathOperator{\pd}{\partial}
\DeclareMathOperator{\epi}{epi}
\DeclareMathOperator{\Argmin}{Argmin}
\DeclareMathOperator{\dom}{dom}
\DeclareMathOperator{\proj}{proj}
\DeclareMathOperator{\ctg}{ctg}
\DeclareMathOperator{\supp}{supp}
\DeclareMathOperator{\argmin}{argmin}
\DeclareMathOperator{\mult}{mult}
\DeclareMathOperator{\ch}{ch}
\DeclareMathOperator{\sh}{sh}
\DeclareMathOperator{\rang}{rang}
\DeclareMathOperator{\diam}{diam}
\DeclareMathOperator{\Epigraphe}{Epigraphe}




\usepackage{xcolor}
\everymath{\color{blue}}
%\everymath{\color[rgb]{0,1,1}}
%\pagecolor[rgb]{0,0,0.5}


\newcommand*{\pdtest}[3][]{\ensuremath{\frac{\partial^{#1} #2}{\partial #3}}}

\newcommand*{\deffunc}[6][]{\ensuremath{
\begin{array}{rcl}
#2 : #3 &\rightarrow& #4\\
#5 &\mapsto& #6
\end{array}
}}

\newcommand{\eqcolon}{\mathrel{\resizebox{\widthof{$\mathord{=}$}}{\height}{ $\!\!=\!\!\resizebox{1.2\width}{0.8\height}{\raisebox{0.23ex}{$\mathop{:}$}}\!\!$ }}}
\newcommand{\coloneq}{\mathrel{\resizebox{\widthof{$\mathord{=}$}}{\height}{ $\!\!\resizebox{1.2\width}{0.8\height}{\raisebox{0.23ex}{$\mathop{:}$}}\!\!=\!\!$ }}}
\newcommand{\eqcolonl}{\ensuremath{\mathrel{=\!\!\mathop{:}}}}
\newcommand{\coloneql}{\ensuremath{\mathrel{\mathop{:} \!\! =}}}
\newcommand{\vc}[1]{% inline column vector
  \left(\begin{smallmatrix}#1\end{smallmatrix}\right)%
}
\newcommand{\vr}[1]{% inline row vector
  \begin{smallmatrix}(\,#1\,)\end{smallmatrix}%
}
\makeatletter
\newcommand*{\defeq}{\ =\mathrel{\rlap{%
                     \raisebox{0.3ex}{$\m@th\cdot$}}%
                     \raisebox{-0.3ex}{$\m@th\cdot$}}%
                     }
\makeatother

\newcommand{\mathcircle}[1]{% inline row vector
 \overset{\circ}{#1}
}
\newcommand{\ulim}{% low limit
 \underline{\lim}
}
\newcommand{\ssi}{% iff
\iff
}
\newcommand{\ps}[2]{
\expval{#1 | #2}
}
\newcommand{\df}[1]{
\mqty{#1}
}
\newcommand{\n}[1]{
\norm{#1}
}
\newcommand{\sys}[1]{
\left\{\smqty{#1}\right.
}


\newcommand{\eqdef}{\ensuremath{\overset{\text{def}}=}}


\def\Circlearrowright{\ensuremath{%
  \rotatebox[origin=c]{230}{$\circlearrowright$}}}

\newcommand\ct[1]{\text{\rmfamily\upshape #1}}
\newcommand\question[1]{ {\color{red} ...!? \small #1}}
\newcommand\caz[1]{\left\{\begin{array} #1 \end{array}\right.}
\newcommand\const{\text{\rmfamily\upshape const}}
\newcommand\toP{ \overset{\pro}{\to}}
\newcommand\toPP{ \overset{\text{PP}}{\to}}
\newcommand{\oeq}{\mathrel{\text{\textcircled{$=$}}}}





\usepackage{xcolor}
% \usepackage[normalem]{ulem}
\usepackage{lipsum}
\makeatletter
% \newcommand\colorwave[1][blue]{\bgroup \markoverwith{\lower3.5\p@\hbox{\sixly \textcolor{#1}{\char58}}}\ULon}
%\font\sixly=lasy6 % does not re-load if already loaded, so no memory problem.

\newmdtheoremenv[
linewidth= 1pt,linecolor= blue,%
leftmargin=20,rightmargin=20,innertopmargin=0pt, innerrightmargin=40,%
tikzsetting = { draw=lightgray, line width = 0.3pt,dashed,%
dash pattern = on 15pt off 3pt},%
splittopskip=\topskip,skipbelow=\baselineskip,%
skipabove=\baselineskip,ntheorem,roundcorner=0pt,
% backgroundcolor=pagebg,font=\color{orange}\sffamily, fontcolor=white
]{examplebox}{Exemple}[section]



\newcommand\R{\mathbb{R}}
\newcommand\Z{\mathbb{Z}}
\newcommand\N{\mathbb{N}}
\newcommand\E{\mathbb{E}}
\newcommand\F{\mathcal{F}}
\newcommand\cH{\mathcal{H}}
\newcommand\V{\mathbb{V}}
\newcommand\dmo{ ^{-1} }
\newcommand\kapa{\kappa}
\newcommand\im{Im}
\newcommand\hs{\mathcal{H}}





\usepackage{soul}

\makeatletter
\newcommand*{\whiten}[1]{\llap{\textcolor{white}{{\the\SOUL@token}}\hspace{#1pt}}}
\DeclareRobustCommand*\myul{%
    \def\SOUL@everyspace{\underline{\space}\kern\z@}%
    \def\SOUL@everytoken{%
     \setbox0=\hbox{\the\SOUL@token}%
     \ifdim\dp0>\z@
        \raisebox{\dp0}{\underline{\phantom{\the\SOUL@token}}}%
        \whiten{1}\whiten{0}%
        \whiten{-1}\whiten{-2}%
        \llap{\the\SOUL@token}%
     \else
        \underline{\the\SOUL@token}%
     \fi}%
\SOUL@}
\makeatother

\newcommand*{\demp}{\fontfamily{lmtt}\selectfont}

\DeclareTextFontCommand{\textdemp}{\demp}

\begin{document}

\ifcomment
Multiline
comment
\fi
\ifcomment
\myul{Typesetting test}
% \color[rgb]{1,1,1}
$∑_i^n≠ 60º±∞π∆¬≈√j∫h≤≥µ$

$\CR \R\pro\ind\pro\gS\pro
\mqty[a&b\\c&d]$
$\pro\mathbb{P}$
$\dd{x}$

  \[
    \alpha(x)=\left\{
                \begin{array}{ll}
                  x\\
                  \frac{1}{1+e^{-kx}}\\
                  \frac{e^x-e^{-x}}{e^x+e^{-x}}
                \end{array}
              \right.
  \]

  $\expval{x}$
  
  $\chi_\rho(ghg\dmo)=\Tr(\rho_{ghg\dmo})=\Tr(\rho_g\circ\rho_h\circ\rho\dmo_g)=\Tr(\rho_h)\overset{\mbox{\scalebox{0.5}{$\Tr(AB)=\Tr(BA)$}}}{=}\chi_\rho(h)$
  	$\mathop{\oplus}_{\substack{x\in X}}$

$\mat(\rho_g)=(a_{ij}(g))_{\scriptsize \substack{1\leq i\leq d \\ 1\leq j\leq d}}$ et $\mat(\rho'_g)=(a'_{ij}(g))_{\scriptsize \substack{1\leq i'\leq d' \\ 1\leq j'\leq d'}}$



\[\int_a^b{\mathbb{R}^2}g(u, v)\dd{P_{XY}}(u, v)=\iint g(u,v) f_{XY}(u, v)\dd \lambda(u) \dd \lambda(v)\]
$$\lim_{x\to\infty} f(x)$$	
$$\iiiint_V \mu(t,u,v,w) \,dt\,du\,dv\,dw$$
$$\sum_{n=1}^{\infty} 2^{-n} = 1$$	
\begin{definition}
	Si $X$ et $Y$ sont 2 v.a. ou definit la \textsc{Covariance} entre $X$ et $Y$ comme
	$\cov(X,Y)\overset{\text{def}}{=}\E\left[(X-\E(X))(Y-\E(Y))\right]=\E(XY)-\E(X)\E(Y)$.
\end{definition}
\fi
\pagebreak

% \tableofcontents

% insert your code here
%\input{./algebra/main.tex}
%\input{./geometrie-differentielle/main.tex}
%\input{./probabilite/main.tex}
%\input{./analyse-fonctionnelle/main.tex}
% \input{./Analyse-convexe-et-dualite-en-optimisation/main.tex}
%\input{./tikz/main.tex}
%\input{./Theorie-du-distributions/main.tex}
%\input{./optimisation/mine.tex}
 \input{./modelisation/main.tex}

% yves.aubry@univ-tln.fr : algebra

\end{document}

% % !TEX encoding = UTF-8 Unicode
% !TEX TS-program = xelatex

\documentclass[french]{report}

%\usepackage[utf8]{inputenc}
%\usepackage[T1]{fontenc}
\usepackage{babel}


\newif\ifcomment
%\commenttrue # Show comments

\usepackage{physics}
\usepackage{amssymb}


\usepackage{amsthm}
% \usepackage{thmtools}
\usepackage{mathtools}
\usepackage{amsfonts}

\usepackage{color}

\usepackage{tikz}

\usepackage{geometry}
\geometry{a5paper, margin=0.1in, right=1cm}

\usepackage{dsfont}

\usepackage{graphicx}
\graphicspath{ {images/} }

\usepackage{faktor}

\usepackage{IEEEtrantools}
\usepackage{enumerate}   
\usepackage[PostScript=dvips]{"/Users/aware/Documents/Courses/diagrams"}


\newtheorem{theorem}{Théorème}[section]
\renewcommand{\thetheorem}{\arabic{theorem}}
\newtheorem{lemme}{Lemme}[section]
\renewcommand{\thelemme}{\arabic{lemme}}
\newtheorem{proposition}{Proposition}[section]
\renewcommand{\theproposition}{\arabic{proposition}}
\newtheorem{notations}{Notations}[section]
\newtheorem{problem}{Problème}[section]
\newtheorem{corollary}{Corollaire}[theorem]
\renewcommand{\thecorollary}{\arabic{corollary}}
\newtheorem{property}{Propriété}[section]
\newtheorem{objective}{Objectif}[section]

\theoremstyle{definition}
\newtheorem{definition}{Définition}[section]
\renewcommand{\thedefinition}{\arabic{definition}}
\newtheorem{exercise}{Exercice}[chapter]
\renewcommand{\theexercise}{\arabic{exercise}}
\newtheorem{example}{Exemple}[chapter]
\renewcommand{\theexample}{\arabic{example}}
\newtheorem*{solution}{Solution}
\newtheorem*{application}{Application}
\newtheorem*{notation}{Notation}
\newtheorem*{vocabulary}{Vocabulaire}
\newtheorem*{properties}{Propriétés}



\theoremstyle{remark}
\newtheorem*{remark}{Remarque}
\newtheorem*{rappel}{Rappel}


\usepackage{etoolbox}
\AtBeginEnvironment{exercise}{\small}
\AtBeginEnvironment{example}{\small}

\usepackage{cases}
\usepackage[red]{mypack}

\usepackage[framemethod=TikZ]{mdframed}

\definecolor{bg}{rgb}{0.4,0.25,0.95}
\definecolor{pagebg}{rgb}{0,0,0.5}
\surroundwithmdframed[
   topline=false,
   rightline=false,
   bottomline=false,
   leftmargin=\parindent,
   skipabove=8pt,
   skipbelow=8pt,
   linecolor=blue,
   innerbottommargin=10pt,
   % backgroundcolor=bg,font=\color{orange}\sffamily, fontcolor=white
]{definition}

\usepackage{empheq}
\usepackage[most]{tcolorbox}

\newtcbox{\mymath}[1][]{%
    nobeforeafter, math upper, tcbox raise base,
    enhanced, colframe=blue!30!black,
    colback=red!10, boxrule=1pt,
    #1}

\usepackage{unixode}


\DeclareMathOperator{\ord}{ord}
\DeclareMathOperator{\orb}{orb}
\DeclareMathOperator{\stab}{stab}
\DeclareMathOperator{\Stab}{stab}
\DeclareMathOperator{\ppcm}{ppcm}
\DeclareMathOperator{\conj}{Conj}
\DeclareMathOperator{\End}{End}
\DeclareMathOperator{\rot}{rot}
\DeclareMathOperator{\trs}{trace}
\DeclareMathOperator{\Ind}{Ind}
\DeclareMathOperator{\mat}{Mat}
\DeclareMathOperator{\id}{Id}
\DeclareMathOperator{\vect}{vect}
\DeclareMathOperator{\img}{img}
\DeclareMathOperator{\cov}{Cov}
\DeclareMathOperator{\dist}{dist}
\DeclareMathOperator{\irr}{Irr}
\DeclareMathOperator{\image}{Im}
\DeclareMathOperator{\pd}{\partial}
\DeclareMathOperator{\epi}{epi}
\DeclareMathOperator{\Argmin}{Argmin}
\DeclareMathOperator{\dom}{dom}
\DeclareMathOperator{\proj}{proj}
\DeclareMathOperator{\ctg}{ctg}
\DeclareMathOperator{\supp}{supp}
\DeclareMathOperator{\argmin}{argmin}
\DeclareMathOperator{\mult}{mult}
\DeclareMathOperator{\ch}{ch}
\DeclareMathOperator{\sh}{sh}
\DeclareMathOperator{\rang}{rang}
\DeclareMathOperator{\diam}{diam}
\DeclareMathOperator{\Epigraphe}{Epigraphe}




\usepackage{xcolor}
\everymath{\color{blue}}
%\everymath{\color[rgb]{0,1,1}}
%\pagecolor[rgb]{0,0,0.5}


\newcommand*{\pdtest}[3][]{\ensuremath{\frac{\partial^{#1} #2}{\partial #3}}}

\newcommand*{\deffunc}[6][]{\ensuremath{
\begin{array}{rcl}
#2 : #3 &\rightarrow& #4\\
#5 &\mapsto& #6
\end{array}
}}

\newcommand{\eqcolon}{\mathrel{\resizebox{\widthof{$\mathord{=}$}}{\height}{ $\!\!=\!\!\resizebox{1.2\width}{0.8\height}{\raisebox{0.23ex}{$\mathop{:}$}}\!\!$ }}}
\newcommand{\coloneq}{\mathrel{\resizebox{\widthof{$\mathord{=}$}}{\height}{ $\!\!\resizebox{1.2\width}{0.8\height}{\raisebox{0.23ex}{$\mathop{:}$}}\!\!=\!\!$ }}}
\newcommand{\eqcolonl}{\ensuremath{\mathrel{=\!\!\mathop{:}}}}
\newcommand{\coloneql}{\ensuremath{\mathrel{\mathop{:} \!\! =}}}
\newcommand{\vc}[1]{% inline column vector
  \left(\begin{smallmatrix}#1\end{smallmatrix}\right)%
}
\newcommand{\vr}[1]{% inline row vector
  \begin{smallmatrix}(\,#1\,)\end{smallmatrix}%
}
\makeatletter
\newcommand*{\defeq}{\ =\mathrel{\rlap{%
                     \raisebox{0.3ex}{$\m@th\cdot$}}%
                     \raisebox{-0.3ex}{$\m@th\cdot$}}%
                     }
\makeatother

\newcommand{\mathcircle}[1]{% inline row vector
 \overset{\circ}{#1}
}
\newcommand{\ulim}{% low limit
 \underline{\lim}
}
\newcommand{\ssi}{% iff
\iff
}
\newcommand{\ps}[2]{
\expval{#1 | #2}
}
\newcommand{\df}[1]{
\mqty{#1}
}
\newcommand{\n}[1]{
\norm{#1}
}
\newcommand{\sys}[1]{
\left\{\smqty{#1}\right.
}


\newcommand{\eqdef}{\ensuremath{\overset{\text{def}}=}}


\def\Circlearrowright{\ensuremath{%
  \rotatebox[origin=c]{230}{$\circlearrowright$}}}

\newcommand\ct[1]{\text{\rmfamily\upshape #1}}
\newcommand\question[1]{ {\color{red} ...!? \small #1}}
\newcommand\caz[1]{\left\{\begin{array} #1 \end{array}\right.}
\newcommand\const{\text{\rmfamily\upshape const}}
\newcommand\toP{ \overset{\pro}{\to}}
\newcommand\toPP{ \overset{\text{PP}}{\to}}
\newcommand{\oeq}{\mathrel{\text{\textcircled{$=$}}}}





\usepackage{xcolor}
% \usepackage[normalem]{ulem}
\usepackage{lipsum}
\makeatletter
% \newcommand\colorwave[1][blue]{\bgroup \markoverwith{\lower3.5\p@\hbox{\sixly \textcolor{#1}{\char58}}}\ULon}
%\font\sixly=lasy6 % does not re-load if already loaded, so no memory problem.

\newmdtheoremenv[
linewidth= 1pt,linecolor= blue,%
leftmargin=20,rightmargin=20,innertopmargin=0pt, innerrightmargin=40,%
tikzsetting = { draw=lightgray, line width = 0.3pt,dashed,%
dash pattern = on 15pt off 3pt},%
splittopskip=\topskip,skipbelow=\baselineskip,%
skipabove=\baselineskip,ntheorem,roundcorner=0pt,
% backgroundcolor=pagebg,font=\color{orange}\sffamily, fontcolor=white
]{examplebox}{Exemple}[section]



\newcommand\R{\mathbb{R}}
\newcommand\Z{\mathbb{Z}}
\newcommand\N{\mathbb{N}}
\newcommand\E{\mathbb{E}}
\newcommand\F{\mathcal{F}}
\newcommand\cH{\mathcal{H}}
\newcommand\V{\mathbb{V}}
\newcommand\dmo{ ^{-1} }
\newcommand\kapa{\kappa}
\newcommand\im{Im}
\newcommand\hs{\mathcal{H}}





\usepackage{soul}

\makeatletter
\newcommand*{\whiten}[1]{\llap{\textcolor{white}{{\the\SOUL@token}}\hspace{#1pt}}}
\DeclareRobustCommand*\myul{%
    \def\SOUL@everyspace{\underline{\space}\kern\z@}%
    \def\SOUL@everytoken{%
     \setbox0=\hbox{\the\SOUL@token}%
     \ifdim\dp0>\z@
        \raisebox{\dp0}{\underline{\phantom{\the\SOUL@token}}}%
        \whiten{1}\whiten{0}%
        \whiten{-1}\whiten{-2}%
        \llap{\the\SOUL@token}%
     \else
        \underline{\the\SOUL@token}%
     \fi}%
\SOUL@}
\makeatother

\newcommand*{\demp}{\fontfamily{lmtt}\selectfont}

\DeclareTextFontCommand{\textdemp}{\demp}

\begin{document}

\ifcomment
Multiline
comment
\fi
\ifcomment
\myul{Typesetting test}
% \color[rgb]{1,1,1}
$∑_i^n≠ 60º±∞π∆¬≈√j∫h≤≥µ$

$\CR \R\pro\ind\pro\gS\pro
\mqty[a&b\\c&d]$
$\pro\mathbb{P}$
$\dd{x}$

  \[
    \alpha(x)=\left\{
                \begin{array}{ll}
                  x\\
                  \frac{1}{1+e^{-kx}}\\
                  \frac{e^x-e^{-x}}{e^x+e^{-x}}
                \end{array}
              \right.
  \]

  $\expval{x}$
  
  $\chi_\rho(ghg\dmo)=\Tr(\rho_{ghg\dmo})=\Tr(\rho_g\circ\rho_h\circ\rho\dmo_g)=\Tr(\rho_h)\overset{\mbox{\scalebox{0.5}{$\Tr(AB)=\Tr(BA)$}}}{=}\chi_\rho(h)$
  	$\mathop{\oplus}_{\substack{x\in X}}$

$\mat(\rho_g)=(a_{ij}(g))_{\scriptsize \substack{1\leq i\leq d \\ 1\leq j\leq d}}$ et $\mat(\rho'_g)=(a'_{ij}(g))_{\scriptsize \substack{1\leq i'\leq d' \\ 1\leq j'\leq d'}}$



\[\int_a^b{\mathbb{R}^2}g(u, v)\dd{P_{XY}}(u, v)=\iint g(u,v) f_{XY}(u, v)\dd \lambda(u) \dd \lambda(v)\]
$$\lim_{x\to\infty} f(x)$$	
$$\iiiint_V \mu(t,u,v,w) \,dt\,du\,dv\,dw$$
$$\sum_{n=1}^{\infty} 2^{-n} = 1$$	
\begin{definition}
	Si $X$ et $Y$ sont 2 v.a. ou definit la \textsc{Covariance} entre $X$ et $Y$ comme
	$\cov(X,Y)\overset{\text{def}}{=}\E\left[(X-\E(X))(Y-\E(Y))\right]=\E(XY)-\E(X)\E(Y)$.
\end{definition}
\fi
\pagebreak

% \tableofcontents

% insert your code here
%\input{./algebra/main.tex}
%\input{./geometrie-differentielle/main.tex}
%\input{./probabilite/main.tex}
%\input{./analyse-fonctionnelle/main.tex}
% \input{./Analyse-convexe-et-dualite-en-optimisation/main.tex}
%\input{./tikz/main.tex}
%\input{./Theorie-du-distributions/main.tex}
%\input{./optimisation/mine.tex}
 \input{./modelisation/main.tex}

% yves.aubry@univ-tln.fr : algebra

\end{document}

%% !TEX encoding = UTF-8 Unicode
% !TEX TS-program = xelatex

\documentclass[french]{report}

%\usepackage[utf8]{inputenc}
%\usepackage[T1]{fontenc}
\usepackage{babel}


\newif\ifcomment
%\commenttrue # Show comments

\usepackage{physics}
\usepackage{amssymb}


\usepackage{amsthm}
% \usepackage{thmtools}
\usepackage{mathtools}
\usepackage{amsfonts}

\usepackage{color}

\usepackage{tikz}

\usepackage{geometry}
\geometry{a5paper, margin=0.1in, right=1cm}

\usepackage{dsfont}

\usepackage{graphicx}
\graphicspath{ {images/} }

\usepackage{faktor}

\usepackage{IEEEtrantools}
\usepackage{enumerate}   
\usepackage[PostScript=dvips]{"/Users/aware/Documents/Courses/diagrams"}


\newtheorem{theorem}{Théorème}[section]
\renewcommand{\thetheorem}{\arabic{theorem}}
\newtheorem{lemme}{Lemme}[section]
\renewcommand{\thelemme}{\arabic{lemme}}
\newtheorem{proposition}{Proposition}[section]
\renewcommand{\theproposition}{\arabic{proposition}}
\newtheorem{notations}{Notations}[section]
\newtheorem{problem}{Problème}[section]
\newtheorem{corollary}{Corollaire}[theorem]
\renewcommand{\thecorollary}{\arabic{corollary}}
\newtheorem{property}{Propriété}[section]
\newtheorem{objective}{Objectif}[section]

\theoremstyle{definition}
\newtheorem{definition}{Définition}[section]
\renewcommand{\thedefinition}{\arabic{definition}}
\newtheorem{exercise}{Exercice}[chapter]
\renewcommand{\theexercise}{\arabic{exercise}}
\newtheorem{example}{Exemple}[chapter]
\renewcommand{\theexample}{\arabic{example}}
\newtheorem*{solution}{Solution}
\newtheorem*{application}{Application}
\newtheorem*{notation}{Notation}
\newtheorem*{vocabulary}{Vocabulaire}
\newtheorem*{properties}{Propriétés}



\theoremstyle{remark}
\newtheorem*{remark}{Remarque}
\newtheorem*{rappel}{Rappel}


\usepackage{etoolbox}
\AtBeginEnvironment{exercise}{\small}
\AtBeginEnvironment{example}{\small}

\usepackage{cases}
\usepackage[red]{mypack}

\usepackage[framemethod=TikZ]{mdframed}

\definecolor{bg}{rgb}{0.4,0.25,0.95}
\definecolor{pagebg}{rgb}{0,0,0.5}
\surroundwithmdframed[
   topline=false,
   rightline=false,
   bottomline=false,
   leftmargin=\parindent,
   skipabove=8pt,
   skipbelow=8pt,
   linecolor=blue,
   innerbottommargin=10pt,
   % backgroundcolor=bg,font=\color{orange}\sffamily, fontcolor=white
]{definition}

\usepackage{empheq}
\usepackage[most]{tcolorbox}

\newtcbox{\mymath}[1][]{%
    nobeforeafter, math upper, tcbox raise base,
    enhanced, colframe=blue!30!black,
    colback=red!10, boxrule=1pt,
    #1}

\usepackage{unixode}


\DeclareMathOperator{\ord}{ord}
\DeclareMathOperator{\orb}{orb}
\DeclareMathOperator{\stab}{stab}
\DeclareMathOperator{\Stab}{stab}
\DeclareMathOperator{\ppcm}{ppcm}
\DeclareMathOperator{\conj}{Conj}
\DeclareMathOperator{\End}{End}
\DeclareMathOperator{\rot}{rot}
\DeclareMathOperator{\trs}{trace}
\DeclareMathOperator{\Ind}{Ind}
\DeclareMathOperator{\mat}{Mat}
\DeclareMathOperator{\id}{Id}
\DeclareMathOperator{\vect}{vect}
\DeclareMathOperator{\img}{img}
\DeclareMathOperator{\cov}{Cov}
\DeclareMathOperator{\dist}{dist}
\DeclareMathOperator{\irr}{Irr}
\DeclareMathOperator{\image}{Im}
\DeclareMathOperator{\pd}{\partial}
\DeclareMathOperator{\epi}{epi}
\DeclareMathOperator{\Argmin}{Argmin}
\DeclareMathOperator{\dom}{dom}
\DeclareMathOperator{\proj}{proj}
\DeclareMathOperator{\ctg}{ctg}
\DeclareMathOperator{\supp}{supp}
\DeclareMathOperator{\argmin}{argmin}
\DeclareMathOperator{\mult}{mult}
\DeclareMathOperator{\ch}{ch}
\DeclareMathOperator{\sh}{sh}
\DeclareMathOperator{\rang}{rang}
\DeclareMathOperator{\diam}{diam}
\DeclareMathOperator{\Epigraphe}{Epigraphe}




\usepackage{xcolor}
\everymath{\color{blue}}
%\everymath{\color[rgb]{0,1,1}}
%\pagecolor[rgb]{0,0,0.5}


\newcommand*{\pdtest}[3][]{\ensuremath{\frac{\partial^{#1} #2}{\partial #3}}}

\newcommand*{\deffunc}[6][]{\ensuremath{
\begin{array}{rcl}
#2 : #3 &\rightarrow& #4\\
#5 &\mapsto& #6
\end{array}
}}

\newcommand{\eqcolon}{\mathrel{\resizebox{\widthof{$\mathord{=}$}}{\height}{ $\!\!=\!\!\resizebox{1.2\width}{0.8\height}{\raisebox{0.23ex}{$\mathop{:}$}}\!\!$ }}}
\newcommand{\coloneq}{\mathrel{\resizebox{\widthof{$\mathord{=}$}}{\height}{ $\!\!\resizebox{1.2\width}{0.8\height}{\raisebox{0.23ex}{$\mathop{:}$}}\!\!=\!\!$ }}}
\newcommand{\eqcolonl}{\ensuremath{\mathrel{=\!\!\mathop{:}}}}
\newcommand{\coloneql}{\ensuremath{\mathrel{\mathop{:} \!\! =}}}
\newcommand{\vc}[1]{% inline column vector
  \left(\begin{smallmatrix}#1\end{smallmatrix}\right)%
}
\newcommand{\vr}[1]{% inline row vector
  \begin{smallmatrix}(\,#1\,)\end{smallmatrix}%
}
\makeatletter
\newcommand*{\defeq}{\ =\mathrel{\rlap{%
                     \raisebox{0.3ex}{$\m@th\cdot$}}%
                     \raisebox{-0.3ex}{$\m@th\cdot$}}%
                     }
\makeatother

\newcommand{\mathcircle}[1]{% inline row vector
 \overset{\circ}{#1}
}
\newcommand{\ulim}{% low limit
 \underline{\lim}
}
\newcommand{\ssi}{% iff
\iff
}
\newcommand{\ps}[2]{
\expval{#1 | #2}
}
\newcommand{\df}[1]{
\mqty{#1}
}
\newcommand{\n}[1]{
\norm{#1}
}
\newcommand{\sys}[1]{
\left\{\smqty{#1}\right.
}


\newcommand{\eqdef}{\ensuremath{\overset{\text{def}}=}}


\def\Circlearrowright{\ensuremath{%
  \rotatebox[origin=c]{230}{$\circlearrowright$}}}

\newcommand\ct[1]{\text{\rmfamily\upshape #1}}
\newcommand\question[1]{ {\color{red} ...!? \small #1}}
\newcommand\caz[1]{\left\{\begin{array} #1 \end{array}\right.}
\newcommand\const{\text{\rmfamily\upshape const}}
\newcommand\toP{ \overset{\pro}{\to}}
\newcommand\toPP{ \overset{\text{PP}}{\to}}
\newcommand{\oeq}{\mathrel{\text{\textcircled{$=$}}}}





\usepackage{xcolor}
% \usepackage[normalem]{ulem}
\usepackage{lipsum}
\makeatletter
% \newcommand\colorwave[1][blue]{\bgroup \markoverwith{\lower3.5\p@\hbox{\sixly \textcolor{#1}{\char58}}}\ULon}
%\font\sixly=lasy6 % does not re-load if already loaded, so no memory problem.

\newmdtheoremenv[
linewidth= 1pt,linecolor= blue,%
leftmargin=20,rightmargin=20,innertopmargin=0pt, innerrightmargin=40,%
tikzsetting = { draw=lightgray, line width = 0.3pt,dashed,%
dash pattern = on 15pt off 3pt},%
splittopskip=\topskip,skipbelow=\baselineskip,%
skipabove=\baselineskip,ntheorem,roundcorner=0pt,
% backgroundcolor=pagebg,font=\color{orange}\sffamily, fontcolor=white
]{examplebox}{Exemple}[section]



\newcommand\R{\mathbb{R}}
\newcommand\Z{\mathbb{Z}}
\newcommand\N{\mathbb{N}}
\newcommand\E{\mathbb{E}}
\newcommand\F{\mathcal{F}}
\newcommand\cH{\mathcal{H}}
\newcommand\V{\mathbb{V}}
\newcommand\dmo{ ^{-1} }
\newcommand\kapa{\kappa}
\newcommand\im{Im}
\newcommand\hs{\mathcal{H}}





\usepackage{soul}

\makeatletter
\newcommand*{\whiten}[1]{\llap{\textcolor{white}{{\the\SOUL@token}}\hspace{#1pt}}}
\DeclareRobustCommand*\myul{%
    \def\SOUL@everyspace{\underline{\space}\kern\z@}%
    \def\SOUL@everytoken{%
     \setbox0=\hbox{\the\SOUL@token}%
     \ifdim\dp0>\z@
        \raisebox{\dp0}{\underline{\phantom{\the\SOUL@token}}}%
        \whiten{1}\whiten{0}%
        \whiten{-1}\whiten{-2}%
        \llap{\the\SOUL@token}%
     \else
        \underline{\the\SOUL@token}%
     \fi}%
\SOUL@}
\makeatother

\newcommand*{\demp}{\fontfamily{lmtt}\selectfont}

\DeclareTextFontCommand{\textdemp}{\demp}

\begin{document}

\ifcomment
Multiline
comment
\fi
\ifcomment
\myul{Typesetting test}
% \color[rgb]{1,1,1}
$∑_i^n≠ 60º±∞π∆¬≈√j∫h≤≥µ$

$\CR \R\pro\ind\pro\gS\pro
\mqty[a&b\\c&d]$
$\pro\mathbb{P}$
$\dd{x}$

  \[
    \alpha(x)=\left\{
                \begin{array}{ll}
                  x\\
                  \frac{1}{1+e^{-kx}}\\
                  \frac{e^x-e^{-x}}{e^x+e^{-x}}
                \end{array}
              \right.
  \]

  $\expval{x}$
  
  $\chi_\rho(ghg\dmo)=\Tr(\rho_{ghg\dmo})=\Tr(\rho_g\circ\rho_h\circ\rho\dmo_g)=\Tr(\rho_h)\overset{\mbox{\scalebox{0.5}{$\Tr(AB)=\Tr(BA)$}}}{=}\chi_\rho(h)$
  	$\mathop{\oplus}_{\substack{x\in X}}$

$\mat(\rho_g)=(a_{ij}(g))_{\scriptsize \substack{1\leq i\leq d \\ 1\leq j\leq d}}$ et $\mat(\rho'_g)=(a'_{ij}(g))_{\scriptsize \substack{1\leq i'\leq d' \\ 1\leq j'\leq d'}}$



\[\int_a^b{\mathbb{R}^2}g(u, v)\dd{P_{XY}}(u, v)=\iint g(u,v) f_{XY}(u, v)\dd \lambda(u) \dd \lambda(v)\]
$$\lim_{x\to\infty} f(x)$$	
$$\iiiint_V \mu(t,u,v,w) \,dt\,du\,dv\,dw$$
$$\sum_{n=1}^{\infty} 2^{-n} = 1$$	
\begin{definition}
	Si $X$ et $Y$ sont 2 v.a. ou definit la \textsc{Covariance} entre $X$ et $Y$ comme
	$\cov(X,Y)\overset{\text{def}}{=}\E\left[(X-\E(X))(Y-\E(Y))\right]=\E(XY)-\E(X)\E(Y)$.
\end{definition}
\fi
\pagebreak

% \tableofcontents

% insert your code here
%\input{./algebra/main.tex}
%\input{./geometrie-differentielle/main.tex}
%\input{./probabilite/main.tex}
%\input{./analyse-fonctionnelle/main.tex}
% \input{./Analyse-convexe-et-dualite-en-optimisation/main.tex}
%\input{./tikz/main.tex}
%\input{./Theorie-du-distributions/main.tex}
%\input{./optimisation/mine.tex}
 \input{./modelisation/main.tex}

% yves.aubry@univ-tln.fr : algebra

\end{document}

%% !TEX encoding = UTF-8 Unicode
% !TEX TS-program = xelatex

\documentclass[french]{report}

%\usepackage[utf8]{inputenc}
%\usepackage[T1]{fontenc}
\usepackage{babel}


\newif\ifcomment
%\commenttrue # Show comments

\usepackage{physics}
\usepackage{amssymb}


\usepackage{amsthm}
% \usepackage{thmtools}
\usepackage{mathtools}
\usepackage{amsfonts}

\usepackage{color}

\usepackage{tikz}

\usepackage{geometry}
\geometry{a5paper, margin=0.1in, right=1cm}

\usepackage{dsfont}

\usepackage{graphicx}
\graphicspath{ {images/} }

\usepackage{faktor}

\usepackage{IEEEtrantools}
\usepackage{enumerate}   
\usepackage[PostScript=dvips]{"/Users/aware/Documents/Courses/diagrams"}


\newtheorem{theorem}{Théorème}[section]
\renewcommand{\thetheorem}{\arabic{theorem}}
\newtheorem{lemme}{Lemme}[section]
\renewcommand{\thelemme}{\arabic{lemme}}
\newtheorem{proposition}{Proposition}[section]
\renewcommand{\theproposition}{\arabic{proposition}}
\newtheorem{notations}{Notations}[section]
\newtheorem{problem}{Problème}[section]
\newtheorem{corollary}{Corollaire}[theorem]
\renewcommand{\thecorollary}{\arabic{corollary}}
\newtheorem{property}{Propriété}[section]
\newtheorem{objective}{Objectif}[section]

\theoremstyle{definition}
\newtheorem{definition}{Définition}[section]
\renewcommand{\thedefinition}{\arabic{definition}}
\newtheorem{exercise}{Exercice}[chapter]
\renewcommand{\theexercise}{\arabic{exercise}}
\newtheorem{example}{Exemple}[chapter]
\renewcommand{\theexample}{\arabic{example}}
\newtheorem*{solution}{Solution}
\newtheorem*{application}{Application}
\newtheorem*{notation}{Notation}
\newtheorem*{vocabulary}{Vocabulaire}
\newtheorem*{properties}{Propriétés}



\theoremstyle{remark}
\newtheorem*{remark}{Remarque}
\newtheorem*{rappel}{Rappel}


\usepackage{etoolbox}
\AtBeginEnvironment{exercise}{\small}
\AtBeginEnvironment{example}{\small}

\usepackage{cases}
\usepackage[red]{mypack}

\usepackage[framemethod=TikZ]{mdframed}

\definecolor{bg}{rgb}{0.4,0.25,0.95}
\definecolor{pagebg}{rgb}{0,0,0.5}
\surroundwithmdframed[
   topline=false,
   rightline=false,
   bottomline=false,
   leftmargin=\parindent,
   skipabove=8pt,
   skipbelow=8pt,
   linecolor=blue,
   innerbottommargin=10pt,
   % backgroundcolor=bg,font=\color{orange}\sffamily, fontcolor=white
]{definition}

\usepackage{empheq}
\usepackage[most]{tcolorbox}

\newtcbox{\mymath}[1][]{%
    nobeforeafter, math upper, tcbox raise base,
    enhanced, colframe=blue!30!black,
    colback=red!10, boxrule=1pt,
    #1}

\usepackage{unixode}


\DeclareMathOperator{\ord}{ord}
\DeclareMathOperator{\orb}{orb}
\DeclareMathOperator{\stab}{stab}
\DeclareMathOperator{\Stab}{stab}
\DeclareMathOperator{\ppcm}{ppcm}
\DeclareMathOperator{\conj}{Conj}
\DeclareMathOperator{\End}{End}
\DeclareMathOperator{\rot}{rot}
\DeclareMathOperator{\trs}{trace}
\DeclareMathOperator{\Ind}{Ind}
\DeclareMathOperator{\mat}{Mat}
\DeclareMathOperator{\id}{Id}
\DeclareMathOperator{\vect}{vect}
\DeclareMathOperator{\img}{img}
\DeclareMathOperator{\cov}{Cov}
\DeclareMathOperator{\dist}{dist}
\DeclareMathOperator{\irr}{Irr}
\DeclareMathOperator{\image}{Im}
\DeclareMathOperator{\pd}{\partial}
\DeclareMathOperator{\epi}{epi}
\DeclareMathOperator{\Argmin}{Argmin}
\DeclareMathOperator{\dom}{dom}
\DeclareMathOperator{\proj}{proj}
\DeclareMathOperator{\ctg}{ctg}
\DeclareMathOperator{\supp}{supp}
\DeclareMathOperator{\argmin}{argmin}
\DeclareMathOperator{\mult}{mult}
\DeclareMathOperator{\ch}{ch}
\DeclareMathOperator{\sh}{sh}
\DeclareMathOperator{\rang}{rang}
\DeclareMathOperator{\diam}{diam}
\DeclareMathOperator{\Epigraphe}{Epigraphe}




\usepackage{xcolor}
\everymath{\color{blue}}
%\everymath{\color[rgb]{0,1,1}}
%\pagecolor[rgb]{0,0,0.5}


\newcommand*{\pdtest}[3][]{\ensuremath{\frac{\partial^{#1} #2}{\partial #3}}}

\newcommand*{\deffunc}[6][]{\ensuremath{
\begin{array}{rcl}
#2 : #3 &\rightarrow& #4\\
#5 &\mapsto& #6
\end{array}
}}

\newcommand{\eqcolon}{\mathrel{\resizebox{\widthof{$\mathord{=}$}}{\height}{ $\!\!=\!\!\resizebox{1.2\width}{0.8\height}{\raisebox{0.23ex}{$\mathop{:}$}}\!\!$ }}}
\newcommand{\coloneq}{\mathrel{\resizebox{\widthof{$\mathord{=}$}}{\height}{ $\!\!\resizebox{1.2\width}{0.8\height}{\raisebox{0.23ex}{$\mathop{:}$}}\!\!=\!\!$ }}}
\newcommand{\eqcolonl}{\ensuremath{\mathrel{=\!\!\mathop{:}}}}
\newcommand{\coloneql}{\ensuremath{\mathrel{\mathop{:} \!\! =}}}
\newcommand{\vc}[1]{% inline column vector
  \left(\begin{smallmatrix}#1\end{smallmatrix}\right)%
}
\newcommand{\vr}[1]{% inline row vector
  \begin{smallmatrix}(\,#1\,)\end{smallmatrix}%
}
\makeatletter
\newcommand*{\defeq}{\ =\mathrel{\rlap{%
                     \raisebox{0.3ex}{$\m@th\cdot$}}%
                     \raisebox{-0.3ex}{$\m@th\cdot$}}%
                     }
\makeatother

\newcommand{\mathcircle}[1]{% inline row vector
 \overset{\circ}{#1}
}
\newcommand{\ulim}{% low limit
 \underline{\lim}
}
\newcommand{\ssi}{% iff
\iff
}
\newcommand{\ps}[2]{
\expval{#1 | #2}
}
\newcommand{\df}[1]{
\mqty{#1}
}
\newcommand{\n}[1]{
\norm{#1}
}
\newcommand{\sys}[1]{
\left\{\smqty{#1}\right.
}


\newcommand{\eqdef}{\ensuremath{\overset{\text{def}}=}}


\def\Circlearrowright{\ensuremath{%
  \rotatebox[origin=c]{230}{$\circlearrowright$}}}

\newcommand\ct[1]{\text{\rmfamily\upshape #1}}
\newcommand\question[1]{ {\color{red} ...!? \small #1}}
\newcommand\caz[1]{\left\{\begin{array} #1 \end{array}\right.}
\newcommand\const{\text{\rmfamily\upshape const}}
\newcommand\toP{ \overset{\pro}{\to}}
\newcommand\toPP{ \overset{\text{PP}}{\to}}
\newcommand{\oeq}{\mathrel{\text{\textcircled{$=$}}}}





\usepackage{xcolor}
% \usepackage[normalem]{ulem}
\usepackage{lipsum}
\makeatletter
% \newcommand\colorwave[1][blue]{\bgroup \markoverwith{\lower3.5\p@\hbox{\sixly \textcolor{#1}{\char58}}}\ULon}
%\font\sixly=lasy6 % does not re-load if already loaded, so no memory problem.

\newmdtheoremenv[
linewidth= 1pt,linecolor= blue,%
leftmargin=20,rightmargin=20,innertopmargin=0pt, innerrightmargin=40,%
tikzsetting = { draw=lightgray, line width = 0.3pt,dashed,%
dash pattern = on 15pt off 3pt},%
splittopskip=\topskip,skipbelow=\baselineskip,%
skipabove=\baselineskip,ntheorem,roundcorner=0pt,
% backgroundcolor=pagebg,font=\color{orange}\sffamily, fontcolor=white
]{examplebox}{Exemple}[section]



\newcommand\R{\mathbb{R}}
\newcommand\Z{\mathbb{Z}}
\newcommand\N{\mathbb{N}}
\newcommand\E{\mathbb{E}}
\newcommand\F{\mathcal{F}}
\newcommand\cH{\mathcal{H}}
\newcommand\V{\mathbb{V}}
\newcommand\dmo{ ^{-1} }
\newcommand\kapa{\kappa}
\newcommand\im{Im}
\newcommand\hs{\mathcal{H}}





\usepackage{soul}

\makeatletter
\newcommand*{\whiten}[1]{\llap{\textcolor{white}{{\the\SOUL@token}}\hspace{#1pt}}}
\DeclareRobustCommand*\myul{%
    \def\SOUL@everyspace{\underline{\space}\kern\z@}%
    \def\SOUL@everytoken{%
     \setbox0=\hbox{\the\SOUL@token}%
     \ifdim\dp0>\z@
        \raisebox{\dp0}{\underline{\phantom{\the\SOUL@token}}}%
        \whiten{1}\whiten{0}%
        \whiten{-1}\whiten{-2}%
        \llap{\the\SOUL@token}%
     \else
        \underline{\the\SOUL@token}%
     \fi}%
\SOUL@}
\makeatother

\newcommand*{\demp}{\fontfamily{lmtt}\selectfont}

\DeclareTextFontCommand{\textdemp}{\demp}

\begin{document}

\ifcomment
Multiline
comment
\fi
\ifcomment
\myul{Typesetting test}
% \color[rgb]{1,1,1}
$∑_i^n≠ 60º±∞π∆¬≈√j∫h≤≥µ$

$\CR \R\pro\ind\pro\gS\pro
\mqty[a&b\\c&d]$
$\pro\mathbb{P}$
$\dd{x}$

  \[
    \alpha(x)=\left\{
                \begin{array}{ll}
                  x\\
                  \frac{1}{1+e^{-kx}}\\
                  \frac{e^x-e^{-x}}{e^x+e^{-x}}
                \end{array}
              \right.
  \]

  $\expval{x}$
  
  $\chi_\rho(ghg\dmo)=\Tr(\rho_{ghg\dmo})=\Tr(\rho_g\circ\rho_h\circ\rho\dmo_g)=\Tr(\rho_h)\overset{\mbox{\scalebox{0.5}{$\Tr(AB)=\Tr(BA)$}}}{=}\chi_\rho(h)$
  	$\mathop{\oplus}_{\substack{x\in X}}$

$\mat(\rho_g)=(a_{ij}(g))_{\scriptsize \substack{1\leq i\leq d \\ 1\leq j\leq d}}$ et $\mat(\rho'_g)=(a'_{ij}(g))_{\scriptsize \substack{1\leq i'\leq d' \\ 1\leq j'\leq d'}}$



\[\int_a^b{\mathbb{R}^2}g(u, v)\dd{P_{XY}}(u, v)=\iint g(u,v) f_{XY}(u, v)\dd \lambda(u) \dd \lambda(v)\]
$$\lim_{x\to\infty} f(x)$$	
$$\iiiint_V \mu(t,u,v,w) \,dt\,du\,dv\,dw$$
$$\sum_{n=1}^{\infty} 2^{-n} = 1$$	
\begin{definition}
	Si $X$ et $Y$ sont 2 v.a. ou definit la \textsc{Covariance} entre $X$ et $Y$ comme
	$\cov(X,Y)\overset{\text{def}}{=}\E\left[(X-\E(X))(Y-\E(Y))\right]=\E(XY)-\E(X)\E(Y)$.
\end{definition}
\fi
\pagebreak

% \tableofcontents

% insert your code here
%\input{./algebra/main.tex}
%\input{./geometrie-differentielle/main.tex}
%\input{./probabilite/main.tex}
%\input{./analyse-fonctionnelle/main.tex}
% \input{./Analyse-convexe-et-dualite-en-optimisation/main.tex}
%\input{./tikz/main.tex}
%\input{./Theorie-du-distributions/main.tex}
%\input{./optimisation/mine.tex}
 \input{./modelisation/main.tex}

% yves.aubry@univ-tln.fr : algebra

\end{document}

%\input{./optimisation/mine.tex}
 % !TEX encoding = UTF-8 Unicode
% !TEX TS-program = xelatex

\documentclass[french]{report}

%\usepackage[utf8]{inputenc}
%\usepackage[T1]{fontenc}
\usepackage{babel}


\newif\ifcomment
%\commenttrue # Show comments

\usepackage{physics}
\usepackage{amssymb}


\usepackage{amsthm}
% \usepackage{thmtools}
\usepackage{mathtools}
\usepackage{amsfonts}

\usepackage{color}

\usepackage{tikz}

\usepackage{geometry}
\geometry{a5paper, margin=0.1in, right=1cm}

\usepackage{dsfont}

\usepackage{graphicx}
\graphicspath{ {images/} }

\usepackage{faktor}

\usepackage{IEEEtrantools}
\usepackage{enumerate}   
\usepackage[PostScript=dvips]{"/Users/aware/Documents/Courses/diagrams"}


\newtheorem{theorem}{Théorème}[section]
\renewcommand{\thetheorem}{\arabic{theorem}}
\newtheorem{lemme}{Lemme}[section]
\renewcommand{\thelemme}{\arabic{lemme}}
\newtheorem{proposition}{Proposition}[section]
\renewcommand{\theproposition}{\arabic{proposition}}
\newtheorem{notations}{Notations}[section]
\newtheorem{problem}{Problème}[section]
\newtheorem{corollary}{Corollaire}[theorem]
\renewcommand{\thecorollary}{\arabic{corollary}}
\newtheorem{property}{Propriété}[section]
\newtheorem{objective}{Objectif}[section]

\theoremstyle{definition}
\newtheorem{definition}{Définition}[section]
\renewcommand{\thedefinition}{\arabic{definition}}
\newtheorem{exercise}{Exercice}[chapter]
\renewcommand{\theexercise}{\arabic{exercise}}
\newtheorem{example}{Exemple}[chapter]
\renewcommand{\theexample}{\arabic{example}}
\newtheorem*{solution}{Solution}
\newtheorem*{application}{Application}
\newtheorem*{notation}{Notation}
\newtheorem*{vocabulary}{Vocabulaire}
\newtheorem*{properties}{Propriétés}



\theoremstyle{remark}
\newtheorem*{remark}{Remarque}
\newtheorem*{rappel}{Rappel}


\usepackage{etoolbox}
\AtBeginEnvironment{exercise}{\small}
\AtBeginEnvironment{example}{\small}

\usepackage{cases}
\usepackage[red]{mypack}

\usepackage[framemethod=TikZ]{mdframed}

\definecolor{bg}{rgb}{0.4,0.25,0.95}
\definecolor{pagebg}{rgb}{0,0,0.5}
\surroundwithmdframed[
   topline=false,
   rightline=false,
   bottomline=false,
   leftmargin=\parindent,
   skipabove=8pt,
   skipbelow=8pt,
   linecolor=blue,
   innerbottommargin=10pt,
   % backgroundcolor=bg,font=\color{orange}\sffamily, fontcolor=white
]{definition}

\usepackage{empheq}
\usepackage[most]{tcolorbox}

\newtcbox{\mymath}[1][]{%
    nobeforeafter, math upper, tcbox raise base,
    enhanced, colframe=blue!30!black,
    colback=red!10, boxrule=1pt,
    #1}

\usepackage{unixode}


\DeclareMathOperator{\ord}{ord}
\DeclareMathOperator{\orb}{orb}
\DeclareMathOperator{\stab}{stab}
\DeclareMathOperator{\Stab}{stab}
\DeclareMathOperator{\ppcm}{ppcm}
\DeclareMathOperator{\conj}{Conj}
\DeclareMathOperator{\End}{End}
\DeclareMathOperator{\rot}{rot}
\DeclareMathOperator{\trs}{trace}
\DeclareMathOperator{\Ind}{Ind}
\DeclareMathOperator{\mat}{Mat}
\DeclareMathOperator{\id}{Id}
\DeclareMathOperator{\vect}{vect}
\DeclareMathOperator{\img}{img}
\DeclareMathOperator{\cov}{Cov}
\DeclareMathOperator{\dist}{dist}
\DeclareMathOperator{\irr}{Irr}
\DeclareMathOperator{\image}{Im}
\DeclareMathOperator{\pd}{\partial}
\DeclareMathOperator{\epi}{epi}
\DeclareMathOperator{\Argmin}{Argmin}
\DeclareMathOperator{\dom}{dom}
\DeclareMathOperator{\proj}{proj}
\DeclareMathOperator{\ctg}{ctg}
\DeclareMathOperator{\supp}{supp}
\DeclareMathOperator{\argmin}{argmin}
\DeclareMathOperator{\mult}{mult}
\DeclareMathOperator{\ch}{ch}
\DeclareMathOperator{\sh}{sh}
\DeclareMathOperator{\rang}{rang}
\DeclareMathOperator{\diam}{diam}
\DeclareMathOperator{\Epigraphe}{Epigraphe}




\usepackage{xcolor}
\everymath{\color{blue}}
%\everymath{\color[rgb]{0,1,1}}
%\pagecolor[rgb]{0,0,0.5}


\newcommand*{\pdtest}[3][]{\ensuremath{\frac{\partial^{#1} #2}{\partial #3}}}

\newcommand*{\deffunc}[6][]{\ensuremath{
\begin{array}{rcl}
#2 : #3 &\rightarrow& #4\\
#5 &\mapsto& #6
\end{array}
}}

\newcommand{\eqcolon}{\mathrel{\resizebox{\widthof{$\mathord{=}$}}{\height}{ $\!\!=\!\!\resizebox{1.2\width}{0.8\height}{\raisebox{0.23ex}{$\mathop{:}$}}\!\!$ }}}
\newcommand{\coloneq}{\mathrel{\resizebox{\widthof{$\mathord{=}$}}{\height}{ $\!\!\resizebox{1.2\width}{0.8\height}{\raisebox{0.23ex}{$\mathop{:}$}}\!\!=\!\!$ }}}
\newcommand{\eqcolonl}{\ensuremath{\mathrel{=\!\!\mathop{:}}}}
\newcommand{\coloneql}{\ensuremath{\mathrel{\mathop{:} \!\! =}}}
\newcommand{\vc}[1]{% inline column vector
  \left(\begin{smallmatrix}#1\end{smallmatrix}\right)%
}
\newcommand{\vr}[1]{% inline row vector
  \begin{smallmatrix}(\,#1\,)\end{smallmatrix}%
}
\makeatletter
\newcommand*{\defeq}{\ =\mathrel{\rlap{%
                     \raisebox{0.3ex}{$\m@th\cdot$}}%
                     \raisebox{-0.3ex}{$\m@th\cdot$}}%
                     }
\makeatother

\newcommand{\mathcircle}[1]{% inline row vector
 \overset{\circ}{#1}
}
\newcommand{\ulim}{% low limit
 \underline{\lim}
}
\newcommand{\ssi}{% iff
\iff
}
\newcommand{\ps}[2]{
\expval{#1 | #2}
}
\newcommand{\df}[1]{
\mqty{#1}
}
\newcommand{\n}[1]{
\norm{#1}
}
\newcommand{\sys}[1]{
\left\{\smqty{#1}\right.
}


\newcommand{\eqdef}{\ensuremath{\overset{\text{def}}=}}


\def\Circlearrowright{\ensuremath{%
  \rotatebox[origin=c]{230}{$\circlearrowright$}}}

\newcommand\ct[1]{\text{\rmfamily\upshape #1}}
\newcommand\question[1]{ {\color{red} ...!? \small #1}}
\newcommand\caz[1]{\left\{\begin{array} #1 \end{array}\right.}
\newcommand\const{\text{\rmfamily\upshape const}}
\newcommand\toP{ \overset{\pro}{\to}}
\newcommand\toPP{ \overset{\text{PP}}{\to}}
\newcommand{\oeq}{\mathrel{\text{\textcircled{$=$}}}}





\usepackage{xcolor}
% \usepackage[normalem]{ulem}
\usepackage{lipsum}
\makeatletter
% \newcommand\colorwave[1][blue]{\bgroup \markoverwith{\lower3.5\p@\hbox{\sixly \textcolor{#1}{\char58}}}\ULon}
%\font\sixly=lasy6 % does not re-load if already loaded, so no memory problem.

\newmdtheoremenv[
linewidth= 1pt,linecolor= blue,%
leftmargin=20,rightmargin=20,innertopmargin=0pt, innerrightmargin=40,%
tikzsetting = { draw=lightgray, line width = 0.3pt,dashed,%
dash pattern = on 15pt off 3pt},%
splittopskip=\topskip,skipbelow=\baselineskip,%
skipabove=\baselineskip,ntheorem,roundcorner=0pt,
% backgroundcolor=pagebg,font=\color{orange}\sffamily, fontcolor=white
]{examplebox}{Exemple}[section]



\newcommand\R{\mathbb{R}}
\newcommand\Z{\mathbb{Z}}
\newcommand\N{\mathbb{N}}
\newcommand\E{\mathbb{E}}
\newcommand\F{\mathcal{F}}
\newcommand\cH{\mathcal{H}}
\newcommand\V{\mathbb{V}}
\newcommand\dmo{ ^{-1} }
\newcommand\kapa{\kappa}
\newcommand\im{Im}
\newcommand\hs{\mathcal{H}}





\usepackage{soul}

\makeatletter
\newcommand*{\whiten}[1]{\llap{\textcolor{white}{{\the\SOUL@token}}\hspace{#1pt}}}
\DeclareRobustCommand*\myul{%
    \def\SOUL@everyspace{\underline{\space}\kern\z@}%
    \def\SOUL@everytoken{%
     \setbox0=\hbox{\the\SOUL@token}%
     \ifdim\dp0>\z@
        \raisebox{\dp0}{\underline{\phantom{\the\SOUL@token}}}%
        \whiten{1}\whiten{0}%
        \whiten{-1}\whiten{-2}%
        \llap{\the\SOUL@token}%
     \else
        \underline{\the\SOUL@token}%
     \fi}%
\SOUL@}
\makeatother

\newcommand*{\demp}{\fontfamily{lmtt}\selectfont}

\DeclareTextFontCommand{\textdemp}{\demp}

\begin{document}

\ifcomment
Multiline
comment
\fi
\ifcomment
\myul{Typesetting test}
% \color[rgb]{1,1,1}
$∑_i^n≠ 60º±∞π∆¬≈√j∫h≤≥µ$

$\CR \R\pro\ind\pro\gS\pro
\mqty[a&b\\c&d]$
$\pro\mathbb{P}$
$\dd{x}$

  \[
    \alpha(x)=\left\{
                \begin{array}{ll}
                  x\\
                  \frac{1}{1+e^{-kx}}\\
                  \frac{e^x-e^{-x}}{e^x+e^{-x}}
                \end{array}
              \right.
  \]

  $\expval{x}$
  
  $\chi_\rho(ghg\dmo)=\Tr(\rho_{ghg\dmo})=\Tr(\rho_g\circ\rho_h\circ\rho\dmo_g)=\Tr(\rho_h)\overset{\mbox{\scalebox{0.5}{$\Tr(AB)=\Tr(BA)$}}}{=}\chi_\rho(h)$
  	$\mathop{\oplus}_{\substack{x\in X}}$

$\mat(\rho_g)=(a_{ij}(g))_{\scriptsize \substack{1\leq i\leq d \\ 1\leq j\leq d}}$ et $\mat(\rho'_g)=(a'_{ij}(g))_{\scriptsize \substack{1\leq i'\leq d' \\ 1\leq j'\leq d'}}$



\[\int_a^b{\mathbb{R}^2}g(u, v)\dd{P_{XY}}(u, v)=\iint g(u,v) f_{XY}(u, v)\dd \lambda(u) \dd \lambda(v)\]
$$\lim_{x\to\infty} f(x)$$	
$$\iiiint_V \mu(t,u,v,w) \,dt\,du\,dv\,dw$$
$$\sum_{n=1}^{\infty} 2^{-n} = 1$$	
\begin{definition}
	Si $X$ et $Y$ sont 2 v.a. ou definit la \textsc{Covariance} entre $X$ et $Y$ comme
	$\cov(X,Y)\overset{\text{def}}{=}\E\left[(X-\E(X))(Y-\E(Y))\right]=\E(XY)-\E(X)\E(Y)$.
\end{definition}
\fi
\pagebreak

% \tableofcontents

% insert your code here
%\input{./algebra/main.tex}
%\input{./geometrie-differentielle/main.tex}
%\input{./probabilite/main.tex}
%\input{./analyse-fonctionnelle/main.tex}
% \input{./Analyse-convexe-et-dualite-en-optimisation/main.tex}
%\input{./tikz/main.tex}
%\input{./Theorie-du-distributions/main.tex}
%\input{./optimisation/mine.tex}
 \input{./modelisation/main.tex}

% yves.aubry@univ-tln.fr : algebra

\end{document}


% yves.aubry@univ-tln.fr : algebra

\end{document}

%% !TEX encoding = UTF-8 Unicode
% !TEX TS-program = xelatex

\documentclass[french]{report}

%\usepackage[utf8]{inputenc}
%\usepackage[T1]{fontenc}
\usepackage{babel}


\newif\ifcomment
%\commenttrue # Show comments

\usepackage{physics}
\usepackage{amssymb}


\usepackage{amsthm}
% \usepackage{thmtools}
\usepackage{mathtools}
\usepackage{amsfonts}

\usepackage{color}

\usepackage{tikz}

\usepackage{geometry}
\geometry{a5paper, margin=0.1in, right=1cm}

\usepackage{dsfont}

\usepackage{graphicx}
\graphicspath{ {images/} }

\usepackage{faktor}

\usepackage{IEEEtrantools}
\usepackage{enumerate}   
\usepackage[PostScript=dvips]{"/Users/aware/Documents/Courses/diagrams"}


\newtheorem{theorem}{Théorème}[section]
\renewcommand{\thetheorem}{\arabic{theorem}}
\newtheorem{lemme}{Lemme}[section]
\renewcommand{\thelemme}{\arabic{lemme}}
\newtheorem{proposition}{Proposition}[section]
\renewcommand{\theproposition}{\arabic{proposition}}
\newtheorem{notations}{Notations}[section]
\newtheorem{problem}{Problème}[section]
\newtheorem{corollary}{Corollaire}[theorem]
\renewcommand{\thecorollary}{\arabic{corollary}}
\newtheorem{property}{Propriété}[section]
\newtheorem{objective}{Objectif}[section]

\theoremstyle{definition}
\newtheorem{definition}{Définition}[section]
\renewcommand{\thedefinition}{\arabic{definition}}
\newtheorem{exercise}{Exercice}[chapter]
\renewcommand{\theexercise}{\arabic{exercise}}
\newtheorem{example}{Exemple}[chapter]
\renewcommand{\theexample}{\arabic{example}}
\newtheorem*{solution}{Solution}
\newtheorem*{application}{Application}
\newtheorem*{notation}{Notation}
\newtheorem*{vocabulary}{Vocabulaire}
\newtheorem*{properties}{Propriétés}



\theoremstyle{remark}
\newtheorem*{remark}{Remarque}
\newtheorem*{rappel}{Rappel}


\usepackage{etoolbox}
\AtBeginEnvironment{exercise}{\small}
\AtBeginEnvironment{example}{\small}

\usepackage{cases}
\usepackage[red]{mypack}

\usepackage[framemethod=TikZ]{mdframed}

\definecolor{bg}{rgb}{0.4,0.25,0.95}
\definecolor{pagebg}{rgb}{0,0,0.5}
\surroundwithmdframed[
   topline=false,
   rightline=false,
   bottomline=false,
   leftmargin=\parindent,
   skipabove=8pt,
   skipbelow=8pt,
   linecolor=blue,
   innerbottommargin=10pt,
   % backgroundcolor=bg,font=\color{orange}\sffamily, fontcolor=white
]{definition}

\usepackage{empheq}
\usepackage[most]{tcolorbox}

\newtcbox{\mymath}[1][]{%
    nobeforeafter, math upper, tcbox raise base,
    enhanced, colframe=blue!30!black,
    colback=red!10, boxrule=1pt,
    #1}

\usepackage{unixode}


\DeclareMathOperator{\ord}{ord}
\DeclareMathOperator{\orb}{orb}
\DeclareMathOperator{\stab}{stab}
\DeclareMathOperator{\Stab}{stab}
\DeclareMathOperator{\ppcm}{ppcm}
\DeclareMathOperator{\conj}{Conj}
\DeclareMathOperator{\End}{End}
\DeclareMathOperator{\rot}{rot}
\DeclareMathOperator{\trs}{trace}
\DeclareMathOperator{\Ind}{Ind}
\DeclareMathOperator{\mat}{Mat}
\DeclareMathOperator{\id}{Id}
\DeclareMathOperator{\vect}{vect}
\DeclareMathOperator{\img}{img}
\DeclareMathOperator{\cov}{Cov}
\DeclareMathOperator{\dist}{dist}
\DeclareMathOperator{\irr}{Irr}
\DeclareMathOperator{\image}{Im}
\DeclareMathOperator{\pd}{\partial}
\DeclareMathOperator{\epi}{epi}
\DeclareMathOperator{\Argmin}{Argmin}
\DeclareMathOperator{\dom}{dom}
\DeclareMathOperator{\proj}{proj}
\DeclareMathOperator{\ctg}{ctg}
\DeclareMathOperator{\supp}{supp}
\DeclareMathOperator{\argmin}{argmin}
\DeclareMathOperator{\mult}{mult}
\DeclareMathOperator{\ch}{ch}
\DeclareMathOperator{\sh}{sh}
\DeclareMathOperator{\rang}{rang}
\DeclareMathOperator{\diam}{diam}
\DeclareMathOperator{\Epigraphe}{Epigraphe}




\usepackage{xcolor}
\everymath{\color{blue}}
%\everymath{\color[rgb]{0,1,1}}
%\pagecolor[rgb]{0,0,0.5}


\newcommand*{\pdtest}[3][]{\ensuremath{\frac{\partial^{#1} #2}{\partial #3}}}

\newcommand*{\deffunc}[6][]{\ensuremath{
\begin{array}{rcl}
#2 : #3 &\rightarrow& #4\\
#5 &\mapsto& #6
\end{array}
}}

\newcommand{\eqcolon}{\mathrel{\resizebox{\widthof{$\mathord{=}$}}{\height}{ $\!\!=\!\!\resizebox{1.2\width}{0.8\height}{\raisebox{0.23ex}{$\mathop{:}$}}\!\!$ }}}
\newcommand{\coloneq}{\mathrel{\resizebox{\widthof{$\mathord{=}$}}{\height}{ $\!\!\resizebox{1.2\width}{0.8\height}{\raisebox{0.23ex}{$\mathop{:}$}}\!\!=\!\!$ }}}
\newcommand{\eqcolonl}{\ensuremath{\mathrel{=\!\!\mathop{:}}}}
\newcommand{\coloneql}{\ensuremath{\mathrel{\mathop{:} \!\! =}}}
\newcommand{\vc}[1]{% inline column vector
  \left(\begin{smallmatrix}#1\end{smallmatrix}\right)%
}
\newcommand{\vr}[1]{% inline row vector
  \begin{smallmatrix}(\,#1\,)\end{smallmatrix}%
}
\makeatletter
\newcommand*{\defeq}{\ =\mathrel{\rlap{%
                     \raisebox{0.3ex}{$\m@th\cdot$}}%
                     \raisebox{-0.3ex}{$\m@th\cdot$}}%
                     }
\makeatother

\newcommand{\mathcircle}[1]{% inline row vector
 \overset{\circ}{#1}
}
\newcommand{\ulim}{% low limit
 \underline{\lim}
}
\newcommand{\ssi}{% iff
\iff
}
\newcommand{\ps}[2]{
\expval{#1 | #2}
}
\newcommand{\df}[1]{
\mqty{#1}
}
\newcommand{\n}[1]{
\norm{#1}
}
\newcommand{\sys}[1]{
\left\{\smqty{#1}\right.
}


\newcommand{\eqdef}{\ensuremath{\overset{\text{def}}=}}


\def\Circlearrowright{\ensuremath{%
  \rotatebox[origin=c]{230}{$\circlearrowright$}}}

\newcommand\ct[1]{\text{\rmfamily\upshape #1}}
\newcommand\question[1]{ {\color{red} ...!? \small #1}}
\newcommand\caz[1]{\left\{\begin{array} #1 \end{array}\right.}
\newcommand\const{\text{\rmfamily\upshape const}}
\newcommand\toP{ \overset{\pro}{\to}}
\newcommand\toPP{ \overset{\text{PP}}{\to}}
\newcommand{\oeq}{\mathrel{\text{\textcircled{$=$}}}}





\usepackage{xcolor}
% \usepackage[normalem]{ulem}
\usepackage{lipsum}
\makeatletter
% \newcommand\colorwave[1][blue]{\bgroup \markoverwith{\lower3.5\p@\hbox{\sixly \textcolor{#1}{\char58}}}\ULon}
%\font\sixly=lasy6 % does not re-load if already loaded, so no memory problem.

\newmdtheoremenv[
linewidth= 1pt,linecolor= blue,%
leftmargin=20,rightmargin=20,innertopmargin=0pt, innerrightmargin=40,%
tikzsetting = { draw=lightgray, line width = 0.3pt,dashed,%
dash pattern = on 15pt off 3pt},%
splittopskip=\topskip,skipbelow=\baselineskip,%
skipabove=\baselineskip,ntheorem,roundcorner=0pt,
% backgroundcolor=pagebg,font=\color{orange}\sffamily, fontcolor=white
]{examplebox}{Exemple}[section]



\newcommand\R{\mathbb{R}}
\newcommand\Z{\mathbb{Z}}
\newcommand\N{\mathbb{N}}
\newcommand\E{\mathbb{E}}
\newcommand\F{\mathcal{F}}
\newcommand\cH{\mathcal{H}}
\newcommand\V{\mathbb{V}}
\newcommand\dmo{ ^{-1} }
\newcommand\kapa{\kappa}
\newcommand\im{Im}
\newcommand\hs{\mathcal{H}}





\usepackage{soul}

\makeatletter
\newcommand*{\whiten}[1]{\llap{\textcolor{white}{{\the\SOUL@token}}\hspace{#1pt}}}
\DeclareRobustCommand*\myul{%
    \def\SOUL@everyspace{\underline{\space}\kern\z@}%
    \def\SOUL@everytoken{%
     \setbox0=\hbox{\the\SOUL@token}%
     \ifdim\dp0>\z@
        \raisebox{\dp0}{\underline{\phantom{\the\SOUL@token}}}%
        \whiten{1}\whiten{0}%
        \whiten{-1}\whiten{-2}%
        \llap{\the\SOUL@token}%
     \else
        \underline{\the\SOUL@token}%
     \fi}%
\SOUL@}
\makeatother

\newcommand*{\demp}{\fontfamily{lmtt}\selectfont}

\DeclareTextFontCommand{\textdemp}{\demp}

\begin{document}

\ifcomment
Multiline
comment
\fi
\ifcomment
\myul{Typesetting test}
% \color[rgb]{1,1,1}
$∑_i^n≠ 60º±∞π∆¬≈√j∫h≤≥µ$

$\CR \R\pro\ind\pro\gS\pro
\mqty[a&b\\c&d]$
$\pro\mathbb{P}$
$\dd{x}$

  \[
    \alpha(x)=\left\{
                \begin{array}{ll}
                  x\\
                  \frac{1}{1+e^{-kx}}\\
                  \frac{e^x-e^{-x}}{e^x+e^{-x}}
                \end{array}
              \right.
  \]

  $\expval{x}$
  
  $\chi_\rho(ghg\dmo)=\Tr(\rho_{ghg\dmo})=\Tr(\rho_g\circ\rho_h\circ\rho\dmo_g)=\Tr(\rho_h)\overset{\mbox{\scalebox{0.5}{$\Tr(AB)=\Tr(BA)$}}}{=}\chi_\rho(h)$
  	$\mathop{\oplus}_{\substack{x\in X}}$

$\mat(\rho_g)=(a_{ij}(g))_{\scriptsize \substack{1\leq i\leq d \\ 1\leq j\leq d}}$ et $\mat(\rho'_g)=(a'_{ij}(g))_{\scriptsize \substack{1\leq i'\leq d' \\ 1\leq j'\leq d'}}$



\[\int_a^b{\mathbb{R}^2}g(u, v)\dd{P_{XY}}(u, v)=\iint g(u,v) f_{XY}(u, v)\dd \lambda(u) \dd \lambda(v)\]
$$\lim_{x\to\infty} f(x)$$	
$$\iiiint_V \mu(t,u,v,w) \,dt\,du\,dv\,dw$$
$$\sum_{n=1}^{\infty} 2^{-n} = 1$$	
\begin{definition}
	Si $X$ et $Y$ sont 2 v.a. ou definit la \textsc{Covariance} entre $X$ et $Y$ comme
	$\cov(X,Y)\overset{\text{def}}{=}\E\left[(X-\E(X))(Y-\E(Y))\right]=\E(XY)-\E(X)\E(Y)$.
\end{definition}
\fi
\pagebreak

% \tableofcontents

% insert your code here
%% !TEX encoding = UTF-8 Unicode
% !TEX TS-program = xelatex

\documentclass[french]{report}

%\usepackage[utf8]{inputenc}
%\usepackage[T1]{fontenc}
\usepackage{babel}


\newif\ifcomment
%\commenttrue # Show comments

\usepackage{physics}
\usepackage{amssymb}


\usepackage{amsthm}
% \usepackage{thmtools}
\usepackage{mathtools}
\usepackage{amsfonts}

\usepackage{color}

\usepackage{tikz}

\usepackage{geometry}
\geometry{a5paper, margin=0.1in, right=1cm}

\usepackage{dsfont}

\usepackage{graphicx}
\graphicspath{ {images/} }

\usepackage{faktor}

\usepackage{IEEEtrantools}
\usepackage{enumerate}   
\usepackage[PostScript=dvips]{"/Users/aware/Documents/Courses/diagrams"}


\newtheorem{theorem}{Théorème}[section]
\renewcommand{\thetheorem}{\arabic{theorem}}
\newtheorem{lemme}{Lemme}[section]
\renewcommand{\thelemme}{\arabic{lemme}}
\newtheorem{proposition}{Proposition}[section]
\renewcommand{\theproposition}{\arabic{proposition}}
\newtheorem{notations}{Notations}[section]
\newtheorem{problem}{Problème}[section]
\newtheorem{corollary}{Corollaire}[theorem]
\renewcommand{\thecorollary}{\arabic{corollary}}
\newtheorem{property}{Propriété}[section]
\newtheorem{objective}{Objectif}[section]

\theoremstyle{definition}
\newtheorem{definition}{Définition}[section]
\renewcommand{\thedefinition}{\arabic{definition}}
\newtheorem{exercise}{Exercice}[chapter]
\renewcommand{\theexercise}{\arabic{exercise}}
\newtheorem{example}{Exemple}[chapter]
\renewcommand{\theexample}{\arabic{example}}
\newtheorem*{solution}{Solution}
\newtheorem*{application}{Application}
\newtheorem*{notation}{Notation}
\newtheorem*{vocabulary}{Vocabulaire}
\newtheorem*{properties}{Propriétés}



\theoremstyle{remark}
\newtheorem*{remark}{Remarque}
\newtheorem*{rappel}{Rappel}


\usepackage{etoolbox}
\AtBeginEnvironment{exercise}{\small}
\AtBeginEnvironment{example}{\small}

\usepackage{cases}
\usepackage[red]{mypack}

\usepackage[framemethod=TikZ]{mdframed}

\definecolor{bg}{rgb}{0.4,0.25,0.95}
\definecolor{pagebg}{rgb}{0,0,0.5}
\surroundwithmdframed[
   topline=false,
   rightline=false,
   bottomline=false,
   leftmargin=\parindent,
   skipabove=8pt,
   skipbelow=8pt,
   linecolor=blue,
   innerbottommargin=10pt,
   % backgroundcolor=bg,font=\color{orange}\sffamily, fontcolor=white
]{definition}

\usepackage{empheq}
\usepackage[most]{tcolorbox}

\newtcbox{\mymath}[1][]{%
    nobeforeafter, math upper, tcbox raise base,
    enhanced, colframe=blue!30!black,
    colback=red!10, boxrule=1pt,
    #1}

\usepackage{unixode}


\DeclareMathOperator{\ord}{ord}
\DeclareMathOperator{\orb}{orb}
\DeclareMathOperator{\stab}{stab}
\DeclareMathOperator{\Stab}{stab}
\DeclareMathOperator{\ppcm}{ppcm}
\DeclareMathOperator{\conj}{Conj}
\DeclareMathOperator{\End}{End}
\DeclareMathOperator{\rot}{rot}
\DeclareMathOperator{\trs}{trace}
\DeclareMathOperator{\Ind}{Ind}
\DeclareMathOperator{\mat}{Mat}
\DeclareMathOperator{\id}{Id}
\DeclareMathOperator{\vect}{vect}
\DeclareMathOperator{\img}{img}
\DeclareMathOperator{\cov}{Cov}
\DeclareMathOperator{\dist}{dist}
\DeclareMathOperator{\irr}{Irr}
\DeclareMathOperator{\image}{Im}
\DeclareMathOperator{\pd}{\partial}
\DeclareMathOperator{\epi}{epi}
\DeclareMathOperator{\Argmin}{Argmin}
\DeclareMathOperator{\dom}{dom}
\DeclareMathOperator{\proj}{proj}
\DeclareMathOperator{\ctg}{ctg}
\DeclareMathOperator{\supp}{supp}
\DeclareMathOperator{\argmin}{argmin}
\DeclareMathOperator{\mult}{mult}
\DeclareMathOperator{\ch}{ch}
\DeclareMathOperator{\sh}{sh}
\DeclareMathOperator{\rang}{rang}
\DeclareMathOperator{\diam}{diam}
\DeclareMathOperator{\Epigraphe}{Epigraphe}




\usepackage{xcolor}
\everymath{\color{blue}}
%\everymath{\color[rgb]{0,1,1}}
%\pagecolor[rgb]{0,0,0.5}


\newcommand*{\pdtest}[3][]{\ensuremath{\frac{\partial^{#1} #2}{\partial #3}}}

\newcommand*{\deffunc}[6][]{\ensuremath{
\begin{array}{rcl}
#2 : #3 &\rightarrow& #4\\
#5 &\mapsto& #6
\end{array}
}}

\newcommand{\eqcolon}{\mathrel{\resizebox{\widthof{$\mathord{=}$}}{\height}{ $\!\!=\!\!\resizebox{1.2\width}{0.8\height}{\raisebox{0.23ex}{$\mathop{:}$}}\!\!$ }}}
\newcommand{\coloneq}{\mathrel{\resizebox{\widthof{$\mathord{=}$}}{\height}{ $\!\!\resizebox{1.2\width}{0.8\height}{\raisebox{0.23ex}{$\mathop{:}$}}\!\!=\!\!$ }}}
\newcommand{\eqcolonl}{\ensuremath{\mathrel{=\!\!\mathop{:}}}}
\newcommand{\coloneql}{\ensuremath{\mathrel{\mathop{:} \!\! =}}}
\newcommand{\vc}[1]{% inline column vector
  \left(\begin{smallmatrix}#1\end{smallmatrix}\right)%
}
\newcommand{\vr}[1]{% inline row vector
  \begin{smallmatrix}(\,#1\,)\end{smallmatrix}%
}
\makeatletter
\newcommand*{\defeq}{\ =\mathrel{\rlap{%
                     \raisebox{0.3ex}{$\m@th\cdot$}}%
                     \raisebox{-0.3ex}{$\m@th\cdot$}}%
                     }
\makeatother

\newcommand{\mathcircle}[1]{% inline row vector
 \overset{\circ}{#1}
}
\newcommand{\ulim}{% low limit
 \underline{\lim}
}
\newcommand{\ssi}{% iff
\iff
}
\newcommand{\ps}[2]{
\expval{#1 | #2}
}
\newcommand{\df}[1]{
\mqty{#1}
}
\newcommand{\n}[1]{
\norm{#1}
}
\newcommand{\sys}[1]{
\left\{\smqty{#1}\right.
}


\newcommand{\eqdef}{\ensuremath{\overset{\text{def}}=}}


\def\Circlearrowright{\ensuremath{%
  \rotatebox[origin=c]{230}{$\circlearrowright$}}}

\newcommand\ct[1]{\text{\rmfamily\upshape #1}}
\newcommand\question[1]{ {\color{red} ...!? \small #1}}
\newcommand\caz[1]{\left\{\begin{array} #1 \end{array}\right.}
\newcommand\const{\text{\rmfamily\upshape const}}
\newcommand\toP{ \overset{\pro}{\to}}
\newcommand\toPP{ \overset{\text{PP}}{\to}}
\newcommand{\oeq}{\mathrel{\text{\textcircled{$=$}}}}





\usepackage{xcolor}
% \usepackage[normalem]{ulem}
\usepackage{lipsum}
\makeatletter
% \newcommand\colorwave[1][blue]{\bgroup \markoverwith{\lower3.5\p@\hbox{\sixly \textcolor{#1}{\char58}}}\ULon}
%\font\sixly=lasy6 % does not re-load if already loaded, so no memory problem.

\newmdtheoremenv[
linewidth= 1pt,linecolor= blue,%
leftmargin=20,rightmargin=20,innertopmargin=0pt, innerrightmargin=40,%
tikzsetting = { draw=lightgray, line width = 0.3pt,dashed,%
dash pattern = on 15pt off 3pt},%
splittopskip=\topskip,skipbelow=\baselineskip,%
skipabove=\baselineskip,ntheorem,roundcorner=0pt,
% backgroundcolor=pagebg,font=\color{orange}\sffamily, fontcolor=white
]{examplebox}{Exemple}[section]



\newcommand\R{\mathbb{R}}
\newcommand\Z{\mathbb{Z}}
\newcommand\N{\mathbb{N}}
\newcommand\E{\mathbb{E}}
\newcommand\F{\mathcal{F}}
\newcommand\cH{\mathcal{H}}
\newcommand\V{\mathbb{V}}
\newcommand\dmo{ ^{-1} }
\newcommand\kapa{\kappa}
\newcommand\im{Im}
\newcommand\hs{\mathcal{H}}





\usepackage{soul}

\makeatletter
\newcommand*{\whiten}[1]{\llap{\textcolor{white}{{\the\SOUL@token}}\hspace{#1pt}}}
\DeclareRobustCommand*\myul{%
    \def\SOUL@everyspace{\underline{\space}\kern\z@}%
    \def\SOUL@everytoken{%
     \setbox0=\hbox{\the\SOUL@token}%
     \ifdim\dp0>\z@
        \raisebox{\dp0}{\underline{\phantom{\the\SOUL@token}}}%
        \whiten{1}\whiten{0}%
        \whiten{-1}\whiten{-2}%
        \llap{\the\SOUL@token}%
     \else
        \underline{\the\SOUL@token}%
     \fi}%
\SOUL@}
\makeatother

\newcommand*{\demp}{\fontfamily{lmtt}\selectfont}

\DeclareTextFontCommand{\textdemp}{\demp}

\begin{document}

\ifcomment
Multiline
comment
\fi
\ifcomment
\myul{Typesetting test}
% \color[rgb]{1,1,1}
$∑_i^n≠ 60º±∞π∆¬≈√j∫h≤≥µ$

$\CR \R\pro\ind\pro\gS\pro
\mqty[a&b\\c&d]$
$\pro\mathbb{P}$
$\dd{x}$

  \[
    \alpha(x)=\left\{
                \begin{array}{ll}
                  x\\
                  \frac{1}{1+e^{-kx}}\\
                  \frac{e^x-e^{-x}}{e^x+e^{-x}}
                \end{array}
              \right.
  \]

  $\expval{x}$
  
  $\chi_\rho(ghg\dmo)=\Tr(\rho_{ghg\dmo})=\Tr(\rho_g\circ\rho_h\circ\rho\dmo_g)=\Tr(\rho_h)\overset{\mbox{\scalebox{0.5}{$\Tr(AB)=\Tr(BA)$}}}{=}\chi_\rho(h)$
  	$\mathop{\oplus}_{\substack{x\in X}}$

$\mat(\rho_g)=(a_{ij}(g))_{\scriptsize \substack{1\leq i\leq d \\ 1\leq j\leq d}}$ et $\mat(\rho'_g)=(a'_{ij}(g))_{\scriptsize \substack{1\leq i'\leq d' \\ 1\leq j'\leq d'}}$



\[\int_a^b{\mathbb{R}^2}g(u, v)\dd{P_{XY}}(u, v)=\iint g(u,v) f_{XY}(u, v)\dd \lambda(u) \dd \lambda(v)\]
$$\lim_{x\to\infty} f(x)$$	
$$\iiiint_V \mu(t,u,v,w) \,dt\,du\,dv\,dw$$
$$\sum_{n=1}^{\infty} 2^{-n} = 1$$	
\begin{definition}
	Si $X$ et $Y$ sont 2 v.a. ou definit la \textsc{Covariance} entre $X$ et $Y$ comme
	$\cov(X,Y)\overset{\text{def}}{=}\E\left[(X-\E(X))(Y-\E(Y))\right]=\E(XY)-\E(X)\E(Y)$.
\end{definition}
\fi
\pagebreak

% \tableofcontents

% insert your code here
%\input{./algebra/main.tex}
%\input{./geometrie-differentielle/main.tex}
%\input{./probabilite/main.tex}
%\input{./analyse-fonctionnelle/main.tex}
% \input{./Analyse-convexe-et-dualite-en-optimisation/main.tex}
%\input{./tikz/main.tex}
%\input{./Theorie-du-distributions/main.tex}
%\input{./optimisation/mine.tex}
 \input{./modelisation/main.tex}

% yves.aubry@univ-tln.fr : algebra

\end{document}

%% !TEX encoding = UTF-8 Unicode
% !TEX TS-program = xelatex

\documentclass[french]{report}

%\usepackage[utf8]{inputenc}
%\usepackage[T1]{fontenc}
\usepackage{babel}


\newif\ifcomment
%\commenttrue # Show comments

\usepackage{physics}
\usepackage{amssymb}


\usepackage{amsthm}
% \usepackage{thmtools}
\usepackage{mathtools}
\usepackage{amsfonts}

\usepackage{color}

\usepackage{tikz}

\usepackage{geometry}
\geometry{a5paper, margin=0.1in, right=1cm}

\usepackage{dsfont}

\usepackage{graphicx}
\graphicspath{ {images/} }

\usepackage{faktor}

\usepackage{IEEEtrantools}
\usepackage{enumerate}   
\usepackage[PostScript=dvips]{"/Users/aware/Documents/Courses/diagrams"}


\newtheorem{theorem}{Théorème}[section]
\renewcommand{\thetheorem}{\arabic{theorem}}
\newtheorem{lemme}{Lemme}[section]
\renewcommand{\thelemme}{\arabic{lemme}}
\newtheorem{proposition}{Proposition}[section]
\renewcommand{\theproposition}{\arabic{proposition}}
\newtheorem{notations}{Notations}[section]
\newtheorem{problem}{Problème}[section]
\newtheorem{corollary}{Corollaire}[theorem]
\renewcommand{\thecorollary}{\arabic{corollary}}
\newtheorem{property}{Propriété}[section]
\newtheorem{objective}{Objectif}[section]

\theoremstyle{definition}
\newtheorem{definition}{Définition}[section]
\renewcommand{\thedefinition}{\arabic{definition}}
\newtheorem{exercise}{Exercice}[chapter]
\renewcommand{\theexercise}{\arabic{exercise}}
\newtheorem{example}{Exemple}[chapter]
\renewcommand{\theexample}{\arabic{example}}
\newtheorem*{solution}{Solution}
\newtheorem*{application}{Application}
\newtheorem*{notation}{Notation}
\newtheorem*{vocabulary}{Vocabulaire}
\newtheorem*{properties}{Propriétés}



\theoremstyle{remark}
\newtheorem*{remark}{Remarque}
\newtheorem*{rappel}{Rappel}


\usepackage{etoolbox}
\AtBeginEnvironment{exercise}{\small}
\AtBeginEnvironment{example}{\small}

\usepackage{cases}
\usepackage[red]{mypack}

\usepackage[framemethod=TikZ]{mdframed}

\definecolor{bg}{rgb}{0.4,0.25,0.95}
\definecolor{pagebg}{rgb}{0,0,0.5}
\surroundwithmdframed[
   topline=false,
   rightline=false,
   bottomline=false,
   leftmargin=\parindent,
   skipabove=8pt,
   skipbelow=8pt,
   linecolor=blue,
   innerbottommargin=10pt,
   % backgroundcolor=bg,font=\color{orange}\sffamily, fontcolor=white
]{definition}

\usepackage{empheq}
\usepackage[most]{tcolorbox}

\newtcbox{\mymath}[1][]{%
    nobeforeafter, math upper, tcbox raise base,
    enhanced, colframe=blue!30!black,
    colback=red!10, boxrule=1pt,
    #1}

\usepackage{unixode}


\DeclareMathOperator{\ord}{ord}
\DeclareMathOperator{\orb}{orb}
\DeclareMathOperator{\stab}{stab}
\DeclareMathOperator{\Stab}{stab}
\DeclareMathOperator{\ppcm}{ppcm}
\DeclareMathOperator{\conj}{Conj}
\DeclareMathOperator{\End}{End}
\DeclareMathOperator{\rot}{rot}
\DeclareMathOperator{\trs}{trace}
\DeclareMathOperator{\Ind}{Ind}
\DeclareMathOperator{\mat}{Mat}
\DeclareMathOperator{\id}{Id}
\DeclareMathOperator{\vect}{vect}
\DeclareMathOperator{\img}{img}
\DeclareMathOperator{\cov}{Cov}
\DeclareMathOperator{\dist}{dist}
\DeclareMathOperator{\irr}{Irr}
\DeclareMathOperator{\image}{Im}
\DeclareMathOperator{\pd}{\partial}
\DeclareMathOperator{\epi}{epi}
\DeclareMathOperator{\Argmin}{Argmin}
\DeclareMathOperator{\dom}{dom}
\DeclareMathOperator{\proj}{proj}
\DeclareMathOperator{\ctg}{ctg}
\DeclareMathOperator{\supp}{supp}
\DeclareMathOperator{\argmin}{argmin}
\DeclareMathOperator{\mult}{mult}
\DeclareMathOperator{\ch}{ch}
\DeclareMathOperator{\sh}{sh}
\DeclareMathOperator{\rang}{rang}
\DeclareMathOperator{\diam}{diam}
\DeclareMathOperator{\Epigraphe}{Epigraphe}




\usepackage{xcolor}
\everymath{\color{blue}}
%\everymath{\color[rgb]{0,1,1}}
%\pagecolor[rgb]{0,0,0.5}


\newcommand*{\pdtest}[3][]{\ensuremath{\frac{\partial^{#1} #2}{\partial #3}}}

\newcommand*{\deffunc}[6][]{\ensuremath{
\begin{array}{rcl}
#2 : #3 &\rightarrow& #4\\
#5 &\mapsto& #6
\end{array}
}}

\newcommand{\eqcolon}{\mathrel{\resizebox{\widthof{$\mathord{=}$}}{\height}{ $\!\!=\!\!\resizebox{1.2\width}{0.8\height}{\raisebox{0.23ex}{$\mathop{:}$}}\!\!$ }}}
\newcommand{\coloneq}{\mathrel{\resizebox{\widthof{$\mathord{=}$}}{\height}{ $\!\!\resizebox{1.2\width}{0.8\height}{\raisebox{0.23ex}{$\mathop{:}$}}\!\!=\!\!$ }}}
\newcommand{\eqcolonl}{\ensuremath{\mathrel{=\!\!\mathop{:}}}}
\newcommand{\coloneql}{\ensuremath{\mathrel{\mathop{:} \!\! =}}}
\newcommand{\vc}[1]{% inline column vector
  \left(\begin{smallmatrix}#1\end{smallmatrix}\right)%
}
\newcommand{\vr}[1]{% inline row vector
  \begin{smallmatrix}(\,#1\,)\end{smallmatrix}%
}
\makeatletter
\newcommand*{\defeq}{\ =\mathrel{\rlap{%
                     \raisebox{0.3ex}{$\m@th\cdot$}}%
                     \raisebox{-0.3ex}{$\m@th\cdot$}}%
                     }
\makeatother

\newcommand{\mathcircle}[1]{% inline row vector
 \overset{\circ}{#1}
}
\newcommand{\ulim}{% low limit
 \underline{\lim}
}
\newcommand{\ssi}{% iff
\iff
}
\newcommand{\ps}[2]{
\expval{#1 | #2}
}
\newcommand{\df}[1]{
\mqty{#1}
}
\newcommand{\n}[1]{
\norm{#1}
}
\newcommand{\sys}[1]{
\left\{\smqty{#1}\right.
}


\newcommand{\eqdef}{\ensuremath{\overset{\text{def}}=}}


\def\Circlearrowright{\ensuremath{%
  \rotatebox[origin=c]{230}{$\circlearrowright$}}}

\newcommand\ct[1]{\text{\rmfamily\upshape #1}}
\newcommand\question[1]{ {\color{red} ...!? \small #1}}
\newcommand\caz[1]{\left\{\begin{array} #1 \end{array}\right.}
\newcommand\const{\text{\rmfamily\upshape const}}
\newcommand\toP{ \overset{\pro}{\to}}
\newcommand\toPP{ \overset{\text{PP}}{\to}}
\newcommand{\oeq}{\mathrel{\text{\textcircled{$=$}}}}





\usepackage{xcolor}
% \usepackage[normalem]{ulem}
\usepackage{lipsum}
\makeatletter
% \newcommand\colorwave[1][blue]{\bgroup \markoverwith{\lower3.5\p@\hbox{\sixly \textcolor{#1}{\char58}}}\ULon}
%\font\sixly=lasy6 % does not re-load if already loaded, so no memory problem.

\newmdtheoremenv[
linewidth= 1pt,linecolor= blue,%
leftmargin=20,rightmargin=20,innertopmargin=0pt, innerrightmargin=40,%
tikzsetting = { draw=lightgray, line width = 0.3pt,dashed,%
dash pattern = on 15pt off 3pt},%
splittopskip=\topskip,skipbelow=\baselineskip,%
skipabove=\baselineskip,ntheorem,roundcorner=0pt,
% backgroundcolor=pagebg,font=\color{orange}\sffamily, fontcolor=white
]{examplebox}{Exemple}[section]



\newcommand\R{\mathbb{R}}
\newcommand\Z{\mathbb{Z}}
\newcommand\N{\mathbb{N}}
\newcommand\E{\mathbb{E}}
\newcommand\F{\mathcal{F}}
\newcommand\cH{\mathcal{H}}
\newcommand\V{\mathbb{V}}
\newcommand\dmo{ ^{-1} }
\newcommand\kapa{\kappa}
\newcommand\im{Im}
\newcommand\hs{\mathcal{H}}





\usepackage{soul}

\makeatletter
\newcommand*{\whiten}[1]{\llap{\textcolor{white}{{\the\SOUL@token}}\hspace{#1pt}}}
\DeclareRobustCommand*\myul{%
    \def\SOUL@everyspace{\underline{\space}\kern\z@}%
    \def\SOUL@everytoken{%
     \setbox0=\hbox{\the\SOUL@token}%
     \ifdim\dp0>\z@
        \raisebox{\dp0}{\underline{\phantom{\the\SOUL@token}}}%
        \whiten{1}\whiten{0}%
        \whiten{-1}\whiten{-2}%
        \llap{\the\SOUL@token}%
     \else
        \underline{\the\SOUL@token}%
     \fi}%
\SOUL@}
\makeatother

\newcommand*{\demp}{\fontfamily{lmtt}\selectfont}

\DeclareTextFontCommand{\textdemp}{\demp}

\begin{document}

\ifcomment
Multiline
comment
\fi
\ifcomment
\myul{Typesetting test}
% \color[rgb]{1,1,1}
$∑_i^n≠ 60º±∞π∆¬≈√j∫h≤≥µ$

$\CR \R\pro\ind\pro\gS\pro
\mqty[a&b\\c&d]$
$\pro\mathbb{P}$
$\dd{x}$

  \[
    \alpha(x)=\left\{
                \begin{array}{ll}
                  x\\
                  \frac{1}{1+e^{-kx}}\\
                  \frac{e^x-e^{-x}}{e^x+e^{-x}}
                \end{array}
              \right.
  \]

  $\expval{x}$
  
  $\chi_\rho(ghg\dmo)=\Tr(\rho_{ghg\dmo})=\Tr(\rho_g\circ\rho_h\circ\rho\dmo_g)=\Tr(\rho_h)\overset{\mbox{\scalebox{0.5}{$\Tr(AB)=\Tr(BA)$}}}{=}\chi_\rho(h)$
  	$\mathop{\oplus}_{\substack{x\in X}}$

$\mat(\rho_g)=(a_{ij}(g))_{\scriptsize \substack{1\leq i\leq d \\ 1\leq j\leq d}}$ et $\mat(\rho'_g)=(a'_{ij}(g))_{\scriptsize \substack{1\leq i'\leq d' \\ 1\leq j'\leq d'}}$



\[\int_a^b{\mathbb{R}^2}g(u, v)\dd{P_{XY}}(u, v)=\iint g(u,v) f_{XY}(u, v)\dd \lambda(u) \dd \lambda(v)\]
$$\lim_{x\to\infty} f(x)$$	
$$\iiiint_V \mu(t,u,v,w) \,dt\,du\,dv\,dw$$
$$\sum_{n=1}^{\infty} 2^{-n} = 1$$	
\begin{definition}
	Si $X$ et $Y$ sont 2 v.a. ou definit la \textsc{Covariance} entre $X$ et $Y$ comme
	$\cov(X,Y)\overset{\text{def}}{=}\E\left[(X-\E(X))(Y-\E(Y))\right]=\E(XY)-\E(X)\E(Y)$.
\end{definition}
\fi
\pagebreak

% \tableofcontents

% insert your code here
%\input{./algebra/main.tex}
%\input{./geometrie-differentielle/main.tex}
%\input{./probabilite/main.tex}
%\input{./analyse-fonctionnelle/main.tex}
% \input{./Analyse-convexe-et-dualite-en-optimisation/main.tex}
%\input{./tikz/main.tex}
%\input{./Theorie-du-distributions/main.tex}
%\input{./optimisation/mine.tex}
 \input{./modelisation/main.tex}

% yves.aubry@univ-tln.fr : algebra

\end{document}

%% !TEX encoding = UTF-8 Unicode
% !TEX TS-program = xelatex

\documentclass[french]{report}

%\usepackage[utf8]{inputenc}
%\usepackage[T1]{fontenc}
\usepackage{babel}


\newif\ifcomment
%\commenttrue # Show comments

\usepackage{physics}
\usepackage{amssymb}


\usepackage{amsthm}
% \usepackage{thmtools}
\usepackage{mathtools}
\usepackage{amsfonts}

\usepackage{color}

\usepackage{tikz}

\usepackage{geometry}
\geometry{a5paper, margin=0.1in, right=1cm}

\usepackage{dsfont}

\usepackage{graphicx}
\graphicspath{ {images/} }

\usepackage{faktor}

\usepackage{IEEEtrantools}
\usepackage{enumerate}   
\usepackage[PostScript=dvips]{"/Users/aware/Documents/Courses/diagrams"}


\newtheorem{theorem}{Théorème}[section]
\renewcommand{\thetheorem}{\arabic{theorem}}
\newtheorem{lemme}{Lemme}[section]
\renewcommand{\thelemme}{\arabic{lemme}}
\newtheorem{proposition}{Proposition}[section]
\renewcommand{\theproposition}{\arabic{proposition}}
\newtheorem{notations}{Notations}[section]
\newtheorem{problem}{Problème}[section]
\newtheorem{corollary}{Corollaire}[theorem]
\renewcommand{\thecorollary}{\arabic{corollary}}
\newtheorem{property}{Propriété}[section]
\newtheorem{objective}{Objectif}[section]

\theoremstyle{definition}
\newtheorem{definition}{Définition}[section]
\renewcommand{\thedefinition}{\arabic{definition}}
\newtheorem{exercise}{Exercice}[chapter]
\renewcommand{\theexercise}{\arabic{exercise}}
\newtheorem{example}{Exemple}[chapter]
\renewcommand{\theexample}{\arabic{example}}
\newtheorem*{solution}{Solution}
\newtheorem*{application}{Application}
\newtheorem*{notation}{Notation}
\newtheorem*{vocabulary}{Vocabulaire}
\newtheorem*{properties}{Propriétés}



\theoremstyle{remark}
\newtheorem*{remark}{Remarque}
\newtheorem*{rappel}{Rappel}


\usepackage{etoolbox}
\AtBeginEnvironment{exercise}{\small}
\AtBeginEnvironment{example}{\small}

\usepackage{cases}
\usepackage[red]{mypack}

\usepackage[framemethod=TikZ]{mdframed}

\definecolor{bg}{rgb}{0.4,0.25,0.95}
\definecolor{pagebg}{rgb}{0,0,0.5}
\surroundwithmdframed[
   topline=false,
   rightline=false,
   bottomline=false,
   leftmargin=\parindent,
   skipabove=8pt,
   skipbelow=8pt,
   linecolor=blue,
   innerbottommargin=10pt,
   % backgroundcolor=bg,font=\color{orange}\sffamily, fontcolor=white
]{definition}

\usepackage{empheq}
\usepackage[most]{tcolorbox}

\newtcbox{\mymath}[1][]{%
    nobeforeafter, math upper, tcbox raise base,
    enhanced, colframe=blue!30!black,
    colback=red!10, boxrule=1pt,
    #1}

\usepackage{unixode}


\DeclareMathOperator{\ord}{ord}
\DeclareMathOperator{\orb}{orb}
\DeclareMathOperator{\stab}{stab}
\DeclareMathOperator{\Stab}{stab}
\DeclareMathOperator{\ppcm}{ppcm}
\DeclareMathOperator{\conj}{Conj}
\DeclareMathOperator{\End}{End}
\DeclareMathOperator{\rot}{rot}
\DeclareMathOperator{\trs}{trace}
\DeclareMathOperator{\Ind}{Ind}
\DeclareMathOperator{\mat}{Mat}
\DeclareMathOperator{\id}{Id}
\DeclareMathOperator{\vect}{vect}
\DeclareMathOperator{\img}{img}
\DeclareMathOperator{\cov}{Cov}
\DeclareMathOperator{\dist}{dist}
\DeclareMathOperator{\irr}{Irr}
\DeclareMathOperator{\image}{Im}
\DeclareMathOperator{\pd}{\partial}
\DeclareMathOperator{\epi}{epi}
\DeclareMathOperator{\Argmin}{Argmin}
\DeclareMathOperator{\dom}{dom}
\DeclareMathOperator{\proj}{proj}
\DeclareMathOperator{\ctg}{ctg}
\DeclareMathOperator{\supp}{supp}
\DeclareMathOperator{\argmin}{argmin}
\DeclareMathOperator{\mult}{mult}
\DeclareMathOperator{\ch}{ch}
\DeclareMathOperator{\sh}{sh}
\DeclareMathOperator{\rang}{rang}
\DeclareMathOperator{\diam}{diam}
\DeclareMathOperator{\Epigraphe}{Epigraphe}




\usepackage{xcolor}
\everymath{\color{blue}}
%\everymath{\color[rgb]{0,1,1}}
%\pagecolor[rgb]{0,0,0.5}


\newcommand*{\pdtest}[3][]{\ensuremath{\frac{\partial^{#1} #2}{\partial #3}}}

\newcommand*{\deffunc}[6][]{\ensuremath{
\begin{array}{rcl}
#2 : #3 &\rightarrow& #4\\
#5 &\mapsto& #6
\end{array}
}}

\newcommand{\eqcolon}{\mathrel{\resizebox{\widthof{$\mathord{=}$}}{\height}{ $\!\!=\!\!\resizebox{1.2\width}{0.8\height}{\raisebox{0.23ex}{$\mathop{:}$}}\!\!$ }}}
\newcommand{\coloneq}{\mathrel{\resizebox{\widthof{$\mathord{=}$}}{\height}{ $\!\!\resizebox{1.2\width}{0.8\height}{\raisebox{0.23ex}{$\mathop{:}$}}\!\!=\!\!$ }}}
\newcommand{\eqcolonl}{\ensuremath{\mathrel{=\!\!\mathop{:}}}}
\newcommand{\coloneql}{\ensuremath{\mathrel{\mathop{:} \!\! =}}}
\newcommand{\vc}[1]{% inline column vector
  \left(\begin{smallmatrix}#1\end{smallmatrix}\right)%
}
\newcommand{\vr}[1]{% inline row vector
  \begin{smallmatrix}(\,#1\,)\end{smallmatrix}%
}
\makeatletter
\newcommand*{\defeq}{\ =\mathrel{\rlap{%
                     \raisebox{0.3ex}{$\m@th\cdot$}}%
                     \raisebox{-0.3ex}{$\m@th\cdot$}}%
                     }
\makeatother

\newcommand{\mathcircle}[1]{% inline row vector
 \overset{\circ}{#1}
}
\newcommand{\ulim}{% low limit
 \underline{\lim}
}
\newcommand{\ssi}{% iff
\iff
}
\newcommand{\ps}[2]{
\expval{#1 | #2}
}
\newcommand{\df}[1]{
\mqty{#1}
}
\newcommand{\n}[1]{
\norm{#1}
}
\newcommand{\sys}[1]{
\left\{\smqty{#1}\right.
}


\newcommand{\eqdef}{\ensuremath{\overset{\text{def}}=}}


\def\Circlearrowright{\ensuremath{%
  \rotatebox[origin=c]{230}{$\circlearrowright$}}}

\newcommand\ct[1]{\text{\rmfamily\upshape #1}}
\newcommand\question[1]{ {\color{red} ...!? \small #1}}
\newcommand\caz[1]{\left\{\begin{array} #1 \end{array}\right.}
\newcommand\const{\text{\rmfamily\upshape const}}
\newcommand\toP{ \overset{\pro}{\to}}
\newcommand\toPP{ \overset{\text{PP}}{\to}}
\newcommand{\oeq}{\mathrel{\text{\textcircled{$=$}}}}





\usepackage{xcolor}
% \usepackage[normalem]{ulem}
\usepackage{lipsum}
\makeatletter
% \newcommand\colorwave[1][blue]{\bgroup \markoverwith{\lower3.5\p@\hbox{\sixly \textcolor{#1}{\char58}}}\ULon}
%\font\sixly=lasy6 % does not re-load if already loaded, so no memory problem.

\newmdtheoremenv[
linewidth= 1pt,linecolor= blue,%
leftmargin=20,rightmargin=20,innertopmargin=0pt, innerrightmargin=40,%
tikzsetting = { draw=lightgray, line width = 0.3pt,dashed,%
dash pattern = on 15pt off 3pt},%
splittopskip=\topskip,skipbelow=\baselineskip,%
skipabove=\baselineskip,ntheorem,roundcorner=0pt,
% backgroundcolor=pagebg,font=\color{orange}\sffamily, fontcolor=white
]{examplebox}{Exemple}[section]



\newcommand\R{\mathbb{R}}
\newcommand\Z{\mathbb{Z}}
\newcommand\N{\mathbb{N}}
\newcommand\E{\mathbb{E}}
\newcommand\F{\mathcal{F}}
\newcommand\cH{\mathcal{H}}
\newcommand\V{\mathbb{V}}
\newcommand\dmo{ ^{-1} }
\newcommand\kapa{\kappa}
\newcommand\im{Im}
\newcommand\hs{\mathcal{H}}





\usepackage{soul}

\makeatletter
\newcommand*{\whiten}[1]{\llap{\textcolor{white}{{\the\SOUL@token}}\hspace{#1pt}}}
\DeclareRobustCommand*\myul{%
    \def\SOUL@everyspace{\underline{\space}\kern\z@}%
    \def\SOUL@everytoken{%
     \setbox0=\hbox{\the\SOUL@token}%
     \ifdim\dp0>\z@
        \raisebox{\dp0}{\underline{\phantom{\the\SOUL@token}}}%
        \whiten{1}\whiten{0}%
        \whiten{-1}\whiten{-2}%
        \llap{\the\SOUL@token}%
     \else
        \underline{\the\SOUL@token}%
     \fi}%
\SOUL@}
\makeatother

\newcommand*{\demp}{\fontfamily{lmtt}\selectfont}

\DeclareTextFontCommand{\textdemp}{\demp}

\begin{document}

\ifcomment
Multiline
comment
\fi
\ifcomment
\myul{Typesetting test}
% \color[rgb]{1,1,1}
$∑_i^n≠ 60º±∞π∆¬≈√j∫h≤≥µ$

$\CR \R\pro\ind\pro\gS\pro
\mqty[a&b\\c&d]$
$\pro\mathbb{P}$
$\dd{x}$

  \[
    \alpha(x)=\left\{
                \begin{array}{ll}
                  x\\
                  \frac{1}{1+e^{-kx}}\\
                  \frac{e^x-e^{-x}}{e^x+e^{-x}}
                \end{array}
              \right.
  \]

  $\expval{x}$
  
  $\chi_\rho(ghg\dmo)=\Tr(\rho_{ghg\dmo})=\Tr(\rho_g\circ\rho_h\circ\rho\dmo_g)=\Tr(\rho_h)\overset{\mbox{\scalebox{0.5}{$\Tr(AB)=\Tr(BA)$}}}{=}\chi_\rho(h)$
  	$\mathop{\oplus}_{\substack{x\in X}}$

$\mat(\rho_g)=(a_{ij}(g))_{\scriptsize \substack{1\leq i\leq d \\ 1\leq j\leq d}}$ et $\mat(\rho'_g)=(a'_{ij}(g))_{\scriptsize \substack{1\leq i'\leq d' \\ 1\leq j'\leq d'}}$



\[\int_a^b{\mathbb{R}^2}g(u, v)\dd{P_{XY}}(u, v)=\iint g(u,v) f_{XY}(u, v)\dd \lambda(u) \dd \lambda(v)\]
$$\lim_{x\to\infty} f(x)$$	
$$\iiiint_V \mu(t,u,v,w) \,dt\,du\,dv\,dw$$
$$\sum_{n=1}^{\infty} 2^{-n} = 1$$	
\begin{definition}
	Si $X$ et $Y$ sont 2 v.a. ou definit la \textsc{Covariance} entre $X$ et $Y$ comme
	$\cov(X,Y)\overset{\text{def}}{=}\E\left[(X-\E(X))(Y-\E(Y))\right]=\E(XY)-\E(X)\E(Y)$.
\end{definition}
\fi
\pagebreak

% \tableofcontents

% insert your code here
%\input{./algebra/main.tex}
%\input{./geometrie-differentielle/main.tex}
%\input{./probabilite/main.tex}
%\input{./analyse-fonctionnelle/main.tex}
% \input{./Analyse-convexe-et-dualite-en-optimisation/main.tex}
%\input{./tikz/main.tex}
%\input{./Theorie-du-distributions/main.tex}
%\input{./optimisation/mine.tex}
 \input{./modelisation/main.tex}

% yves.aubry@univ-tln.fr : algebra

\end{document}

%% !TEX encoding = UTF-8 Unicode
% !TEX TS-program = xelatex

\documentclass[french]{report}

%\usepackage[utf8]{inputenc}
%\usepackage[T1]{fontenc}
\usepackage{babel}


\newif\ifcomment
%\commenttrue # Show comments

\usepackage{physics}
\usepackage{amssymb}


\usepackage{amsthm}
% \usepackage{thmtools}
\usepackage{mathtools}
\usepackage{amsfonts}

\usepackage{color}

\usepackage{tikz}

\usepackage{geometry}
\geometry{a5paper, margin=0.1in, right=1cm}

\usepackage{dsfont}

\usepackage{graphicx}
\graphicspath{ {images/} }

\usepackage{faktor}

\usepackage{IEEEtrantools}
\usepackage{enumerate}   
\usepackage[PostScript=dvips]{"/Users/aware/Documents/Courses/diagrams"}


\newtheorem{theorem}{Théorème}[section]
\renewcommand{\thetheorem}{\arabic{theorem}}
\newtheorem{lemme}{Lemme}[section]
\renewcommand{\thelemme}{\arabic{lemme}}
\newtheorem{proposition}{Proposition}[section]
\renewcommand{\theproposition}{\arabic{proposition}}
\newtheorem{notations}{Notations}[section]
\newtheorem{problem}{Problème}[section]
\newtheorem{corollary}{Corollaire}[theorem]
\renewcommand{\thecorollary}{\arabic{corollary}}
\newtheorem{property}{Propriété}[section]
\newtheorem{objective}{Objectif}[section]

\theoremstyle{definition}
\newtheorem{definition}{Définition}[section]
\renewcommand{\thedefinition}{\arabic{definition}}
\newtheorem{exercise}{Exercice}[chapter]
\renewcommand{\theexercise}{\arabic{exercise}}
\newtheorem{example}{Exemple}[chapter]
\renewcommand{\theexample}{\arabic{example}}
\newtheorem*{solution}{Solution}
\newtheorem*{application}{Application}
\newtheorem*{notation}{Notation}
\newtheorem*{vocabulary}{Vocabulaire}
\newtheorem*{properties}{Propriétés}



\theoremstyle{remark}
\newtheorem*{remark}{Remarque}
\newtheorem*{rappel}{Rappel}


\usepackage{etoolbox}
\AtBeginEnvironment{exercise}{\small}
\AtBeginEnvironment{example}{\small}

\usepackage{cases}
\usepackage[red]{mypack}

\usepackage[framemethod=TikZ]{mdframed}

\definecolor{bg}{rgb}{0.4,0.25,0.95}
\definecolor{pagebg}{rgb}{0,0,0.5}
\surroundwithmdframed[
   topline=false,
   rightline=false,
   bottomline=false,
   leftmargin=\parindent,
   skipabove=8pt,
   skipbelow=8pt,
   linecolor=blue,
   innerbottommargin=10pt,
   % backgroundcolor=bg,font=\color{orange}\sffamily, fontcolor=white
]{definition}

\usepackage{empheq}
\usepackage[most]{tcolorbox}

\newtcbox{\mymath}[1][]{%
    nobeforeafter, math upper, tcbox raise base,
    enhanced, colframe=blue!30!black,
    colback=red!10, boxrule=1pt,
    #1}

\usepackage{unixode}


\DeclareMathOperator{\ord}{ord}
\DeclareMathOperator{\orb}{orb}
\DeclareMathOperator{\stab}{stab}
\DeclareMathOperator{\Stab}{stab}
\DeclareMathOperator{\ppcm}{ppcm}
\DeclareMathOperator{\conj}{Conj}
\DeclareMathOperator{\End}{End}
\DeclareMathOperator{\rot}{rot}
\DeclareMathOperator{\trs}{trace}
\DeclareMathOperator{\Ind}{Ind}
\DeclareMathOperator{\mat}{Mat}
\DeclareMathOperator{\id}{Id}
\DeclareMathOperator{\vect}{vect}
\DeclareMathOperator{\img}{img}
\DeclareMathOperator{\cov}{Cov}
\DeclareMathOperator{\dist}{dist}
\DeclareMathOperator{\irr}{Irr}
\DeclareMathOperator{\image}{Im}
\DeclareMathOperator{\pd}{\partial}
\DeclareMathOperator{\epi}{epi}
\DeclareMathOperator{\Argmin}{Argmin}
\DeclareMathOperator{\dom}{dom}
\DeclareMathOperator{\proj}{proj}
\DeclareMathOperator{\ctg}{ctg}
\DeclareMathOperator{\supp}{supp}
\DeclareMathOperator{\argmin}{argmin}
\DeclareMathOperator{\mult}{mult}
\DeclareMathOperator{\ch}{ch}
\DeclareMathOperator{\sh}{sh}
\DeclareMathOperator{\rang}{rang}
\DeclareMathOperator{\diam}{diam}
\DeclareMathOperator{\Epigraphe}{Epigraphe}




\usepackage{xcolor}
\everymath{\color{blue}}
%\everymath{\color[rgb]{0,1,1}}
%\pagecolor[rgb]{0,0,0.5}


\newcommand*{\pdtest}[3][]{\ensuremath{\frac{\partial^{#1} #2}{\partial #3}}}

\newcommand*{\deffunc}[6][]{\ensuremath{
\begin{array}{rcl}
#2 : #3 &\rightarrow& #4\\
#5 &\mapsto& #6
\end{array}
}}

\newcommand{\eqcolon}{\mathrel{\resizebox{\widthof{$\mathord{=}$}}{\height}{ $\!\!=\!\!\resizebox{1.2\width}{0.8\height}{\raisebox{0.23ex}{$\mathop{:}$}}\!\!$ }}}
\newcommand{\coloneq}{\mathrel{\resizebox{\widthof{$\mathord{=}$}}{\height}{ $\!\!\resizebox{1.2\width}{0.8\height}{\raisebox{0.23ex}{$\mathop{:}$}}\!\!=\!\!$ }}}
\newcommand{\eqcolonl}{\ensuremath{\mathrel{=\!\!\mathop{:}}}}
\newcommand{\coloneql}{\ensuremath{\mathrel{\mathop{:} \!\! =}}}
\newcommand{\vc}[1]{% inline column vector
  \left(\begin{smallmatrix}#1\end{smallmatrix}\right)%
}
\newcommand{\vr}[1]{% inline row vector
  \begin{smallmatrix}(\,#1\,)\end{smallmatrix}%
}
\makeatletter
\newcommand*{\defeq}{\ =\mathrel{\rlap{%
                     \raisebox{0.3ex}{$\m@th\cdot$}}%
                     \raisebox{-0.3ex}{$\m@th\cdot$}}%
                     }
\makeatother

\newcommand{\mathcircle}[1]{% inline row vector
 \overset{\circ}{#1}
}
\newcommand{\ulim}{% low limit
 \underline{\lim}
}
\newcommand{\ssi}{% iff
\iff
}
\newcommand{\ps}[2]{
\expval{#1 | #2}
}
\newcommand{\df}[1]{
\mqty{#1}
}
\newcommand{\n}[1]{
\norm{#1}
}
\newcommand{\sys}[1]{
\left\{\smqty{#1}\right.
}


\newcommand{\eqdef}{\ensuremath{\overset{\text{def}}=}}


\def\Circlearrowright{\ensuremath{%
  \rotatebox[origin=c]{230}{$\circlearrowright$}}}

\newcommand\ct[1]{\text{\rmfamily\upshape #1}}
\newcommand\question[1]{ {\color{red} ...!? \small #1}}
\newcommand\caz[1]{\left\{\begin{array} #1 \end{array}\right.}
\newcommand\const{\text{\rmfamily\upshape const}}
\newcommand\toP{ \overset{\pro}{\to}}
\newcommand\toPP{ \overset{\text{PP}}{\to}}
\newcommand{\oeq}{\mathrel{\text{\textcircled{$=$}}}}





\usepackage{xcolor}
% \usepackage[normalem]{ulem}
\usepackage{lipsum}
\makeatletter
% \newcommand\colorwave[1][blue]{\bgroup \markoverwith{\lower3.5\p@\hbox{\sixly \textcolor{#1}{\char58}}}\ULon}
%\font\sixly=lasy6 % does not re-load if already loaded, so no memory problem.

\newmdtheoremenv[
linewidth= 1pt,linecolor= blue,%
leftmargin=20,rightmargin=20,innertopmargin=0pt, innerrightmargin=40,%
tikzsetting = { draw=lightgray, line width = 0.3pt,dashed,%
dash pattern = on 15pt off 3pt},%
splittopskip=\topskip,skipbelow=\baselineskip,%
skipabove=\baselineskip,ntheorem,roundcorner=0pt,
% backgroundcolor=pagebg,font=\color{orange}\sffamily, fontcolor=white
]{examplebox}{Exemple}[section]



\newcommand\R{\mathbb{R}}
\newcommand\Z{\mathbb{Z}}
\newcommand\N{\mathbb{N}}
\newcommand\E{\mathbb{E}}
\newcommand\F{\mathcal{F}}
\newcommand\cH{\mathcal{H}}
\newcommand\V{\mathbb{V}}
\newcommand\dmo{ ^{-1} }
\newcommand\kapa{\kappa}
\newcommand\im{Im}
\newcommand\hs{\mathcal{H}}





\usepackage{soul}

\makeatletter
\newcommand*{\whiten}[1]{\llap{\textcolor{white}{{\the\SOUL@token}}\hspace{#1pt}}}
\DeclareRobustCommand*\myul{%
    \def\SOUL@everyspace{\underline{\space}\kern\z@}%
    \def\SOUL@everytoken{%
     \setbox0=\hbox{\the\SOUL@token}%
     \ifdim\dp0>\z@
        \raisebox{\dp0}{\underline{\phantom{\the\SOUL@token}}}%
        \whiten{1}\whiten{0}%
        \whiten{-1}\whiten{-2}%
        \llap{\the\SOUL@token}%
     \else
        \underline{\the\SOUL@token}%
     \fi}%
\SOUL@}
\makeatother

\newcommand*{\demp}{\fontfamily{lmtt}\selectfont}

\DeclareTextFontCommand{\textdemp}{\demp}

\begin{document}

\ifcomment
Multiline
comment
\fi
\ifcomment
\myul{Typesetting test}
% \color[rgb]{1,1,1}
$∑_i^n≠ 60º±∞π∆¬≈√j∫h≤≥µ$

$\CR \R\pro\ind\pro\gS\pro
\mqty[a&b\\c&d]$
$\pro\mathbb{P}$
$\dd{x}$

  \[
    \alpha(x)=\left\{
                \begin{array}{ll}
                  x\\
                  \frac{1}{1+e^{-kx}}\\
                  \frac{e^x-e^{-x}}{e^x+e^{-x}}
                \end{array}
              \right.
  \]

  $\expval{x}$
  
  $\chi_\rho(ghg\dmo)=\Tr(\rho_{ghg\dmo})=\Tr(\rho_g\circ\rho_h\circ\rho\dmo_g)=\Tr(\rho_h)\overset{\mbox{\scalebox{0.5}{$\Tr(AB)=\Tr(BA)$}}}{=}\chi_\rho(h)$
  	$\mathop{\oplus}_{\substack{x\in X}}$

$\mat(\rho_g)=(a_{ij}(g))_{\scriptsize \substack{1\leq i\leq d \\ 1\leq j\leq d}}$ et $\mat(\rho'_g)=(a'_{ij}(g))_{\scriptsize \substack{1\leq i'\leq d' \\ 1\leq j'\leq d'}}$



\[\int_a^b{\mathbb{R}^2}g(u, v)\dd{P_{XY}}(u, v)=\iint g(u,v) f_{XY}(u, v)\dd \lambda(u) \dd \lambda(v)\]
$$\lim_{x\to\infty} f(x)$$	
$$\iiiint_V \mu(t,u,v,w) \,dt\,du\,dv\,dw$$
$$\sum_{n=1}^{\infty} 2^{-n} = 1$$	
\begin{definition}
	Si $X$ et $Y$ sont 2 v.a. ou definit la \textsc{Covariance} entre $X$ et $Y$ comme
	$\cov(X,Y)\overset{\text{def}}{=}\E\left[(X-\E(X))(Y-\E(Y))\right]=\E(XY)-\E(X)\E(Y)$.
\end{definition}
\fi
\pagebreak

% \tableofcontents

% insert your code here
%\input{./algebra/main.tex}
%\input{./geometrie-differentielle/main.tex}
%\input{./probabilite/main.tex}
%\input{./analyse-fonctionnelle/main.tex}
% \input{./Analyse-convexe-et-dualite-en-optimisation/main.tex}
%\input{./tikz/main.tex}
%\input{./Theorie-du-distributions/main.tex}
%\input{./optimisation/mine.tex}
 \input{./modelisation/main.tex}

% yves.aubry@univ-tln.fr : algebra

\end{document}

% % !TEX encoding = UTF-8 Unicode
% !TEX TS-program = xelatex

\documentclass[french]{report}

%\usepackage[utf8]{inputenc}
%\usepackage[T1]{fontenc}
\usepackage{babel}


\newif\ifcomment
%\commenttrue # Show comments

\usepackage{physics}
\usepackage{amssymb}


\usepackage{amsthm}
% \usepackage{thmtools}
\usepackage{mathtools}
\usepackage{amsfonts}

\usepackage{color}

\usepackage{tikz}

\usepackage{geometry}
\geometry{a5paper, margin=0.1in, right=1cm}

\usepackage{dsfont}

\usepackage{graphicx}
\graphicspath{ {images/} }

\usepackage{faktor}

\usepackage{IEEEtrantools}
\usepackage{enumerate}   
\usepackage[PostScript=dvips]{"/Users/aware/Documents/Courses/diagrams"}


\newtheorem{theorem}{Théorème}[section]
\renewcommand{\thetheorem}{\arabic{theorem}}
\newtheorem{lemme}{Lemme}[section]
\renewcommand{\thelemme}{\arabic{lemme}}
\newtheorem{proposition}{Proposition}[section]
\renewcommand{\theproposition}{\arabic{proposition}}
\newtheorem{notations}{Notations}[section]
\newtheorem{problem}{Problème}[section]
\newtheorem{corollary}{Corollaire}[theorem]
\renewcommand{\thecorollary}{\arabic{corollary}}
\newtheorem{property}{Propriété}[section]
\newtheorem{objective}{Objectif}[section]

\theoremstyle{definition}
\newtheorem{definition}{Définition}[section]
\renewcommand{\thedefinition}{\arabic{definition}}
\newtheorem{exercise}{Exercice}[chapter]
\renewcommand{\theexercise}{\arabic{exercise}}
\newtheorem{example}{Exemple}[chapter]
\renewcommand{\theexample}{\arabic{example}}
\newtheorem*{solution}{Solution}
\newtheorem*{application}{Application}
\newtheorem*{notation}{Notation}
\newtheorem*{vocabulary}{Vocabulaire}
\newtheorem*{properties}{Propriétés}



\theoremstyle{remark}
\newtheorem*{remark}{Remarque}
\newtheorem*{rappel}{Rappel}


\usepackage{etoolbox}
\AtBeginEnvironment{exercise}{\small}
\AtBeginEnvironment{example}{\small}

\usepackage{cases}
\usepackage[red]{mypack}

\usepackage[framemethod=TikZ]{mdframed}

\definecolor{bg}{rgb}{0.4,0.25,0.95}
\definecolor{pagebg}{rgb}{0,0,0.5}
\surroundwithmdframed[
   topline=false,
   rightline=false,
   bottomline=false,
   leftmargin=\parindent,
   skipabove=8pt,
   skipbelow=8pt,
   linecolor=blue,
   innerbottommargin=10pt,
   % backgroundcolor=bg,font=\color{orange}\sffamily, fontcolor=white
]{definition}

\usepackage{empheq}
\usepackage[most]{tcolorbox}

\newtcbox{\mymath}[1][]{%
    nobeforeafter, math upper, tcbox raise base,
    enhanced, colframe=blue!30!black,
    colback=red!10, boxrule=1pt,
    #1}

\usepackage{unixode}


\DeclareMathOperator{\ord}{ord}
\DeclareMathOperator{\orb}{orb}
\DeclareMathOperator{\stab}{stab}
\DeclareMathOperator{\Stab}{stab}
\DeclareMathOperator{\ppcm}{ppcm}
\DeclareMathOperator{\conj}{Conj}
\DeclareMathOperator{\End}{End}
\DeclareMathOperator{\rot}{rot}
\DeclareMathOperator{\trs}{trace}
\DeclareMathOperator{\Ind}{Ind}
\DeclareMathOperator{\mat}{Mat}
\DeclareMathOperator{\id}{Id}
\DeclareMathOperator{\vect}{vect}
\DeclareMathOperator{\img}{img}
\DeclareMathOperator{\cov}{Cov}
\DeclareMathOperator{\dist}{dist}
\DeclareMathOperator{\irr}{Irr}
\DeclareMathOperator{\image}{Im}
\DeclareMathOperator{\pd}{\partial}
\DeclareMathOperator{\epi}{epi}
\DeclareMathOperator{\Argmin}{Argmin}
\DeclareMathOperator{\dom}{dom}
\DeclareMathOperator{\proj}{proj}
\DeclareMathOperator{\ctg}{ctg}
\DeclareMathOperator{\supp}{supp}
\DeclareMathOperator{\argmin}{argmin}
\DeclareMathOperator{\mult}{mult}
\DeclareMathOperator{\ch}{ch}
\DeclareMathOperator{\sh}{sh}
\DeclareMathOperator{\rang}{rang}
\DeclareMathOperator{\diam}{diam}
\DeclareMathOperator{\Epigraphe}{Epigraphe}




\usepackage{xcolor}
\everymath{\color{blue}}
%\everymath{\color[rgb]{0,1,1}}
%\pagecolor[rgb]{0,0,0.5}


\newcommand*{\pdtest}[3][]{\ensuremath{\frac{\partial^{#1} #2}{\partial #3}}}

\newcommand*{\deffunc}[6][]{\ensuremath{
\begin{array}{rcl}
#2 : #3 &\rightarrow& #4\\
#5 &\mapsto& #6
\end{array}
}}

\newcommand{\eqcolon}{\mathrel{\resizebox{\widthof{$\mathord{=}$}}{\height}{ $\!\!=\!\!\resizebox{1.2\width}{0.8\height}{\raisebox{0.23ex}{$\mathop{:}$}}\!\!$ }}}
\newcommand{\coloneq}{\mathrel{\resizebox{\widthof{$\mathord{=}$}}{\height}{ $\!\!\resizebox{1.2\width}{0.8\height}{\raisebox{0.23ex}{$\mathop{:}$}}\!\!=\!\!$ }}}
\newcommand{\eqcolonl}{\ensuremath{\mathrel{=\!\!\mathop{:}}}}
\newcommand{\coloneql}{\ensuremath{\mathrel{\mathop{:} \!\! =}}}
\newcommand{\vc}[1]{% inline column vector
  \left(\begin{smallmatrix}#1\end{smallmatrix}\right)%
}
\newcommand{\vr}[1]{% inline row vector
  \begin{smallmatrix}(\,#1\,)\end{smallmatrix}%
}
\makeatletter
\newcommand*{\defeq}{\ =\mathrel{\rlap{%
                     \raisebox{0.3ex}{$\m@th\cdot$}}%
                     \raisebox{-0.3ex}{$\m@th\cdot$}}%
                     }
\makeatother

\newcommand{\mathcircle}[1]{% inline row vector
 \overset{\circ}{#1}
}
\newcommand{\ulim}{% low limit
 \underline{\lim}
}
\newcommand{\ssi}{% iff
\iff
}
\newcommand{\ps}[2]{
\expval{#1 | #2}
}
\newcommand{\df}[1]{
\mqty{#1}
}
\newcommand{\n}[1]{
\norm{#1}
}
\newcommand{\sys}[1]{
\left\{\smqty{#1}\right.
}


\newcommand{\eqdef}{\ensuremath{\overset{\text{def}}=}}


\def\Circlearrowright{\ensuremath{%
  \rotatebox[origin=c]{230}{$\circlearrowright$}}}

\newcommand\ct[1]{\text{\rmfamily\upshape #1}}
\newcommand\question[1]{ {\color{red} ...!? \small #1}}
\newcommand\caz[1]{\left\{\begin{array} #1 \end{array}\right.}
\newcommand\const{\text{\rmfamily\upshape const}}
\newcommand\toP{ \overset{\pro}{\to}}
\newcommand\toPP{ \overset{\text{PP}}{\to}}
\newcommand{\oeq}{\mathrel{\text{\textcircled{$=$}}}}





\usepackage{xcolor}
% \usepackage[normalem]{ulem}
\usepackage{lipsum}
\makeatletter
% \newcommand\colorwave[1][blue]{\bgroup \markoverwith{\lower3.5\p@\hbox{\sixly \textcolor{#1}{\char58}}}\ULon}
%\font\sixly=lasy6 % does not re-load if already loaded, so no memory problem.

\newmdtheoremenv[
linewidth= 1pt,linecolor= blue,%
leftmargin=20,rightmargin=20,innertopmargin=0pt, innerrightmargin=40,%
tikzsetting = { draw=lightgray, line width = 0.3pt,dashed,%
dash pattern = on 15pt off 3pt},%
splittopskip=\topskip,skipbelow=\baselineskip,%
skipabove=\baselineskip,ntheorem,roundcorner=0pt,
% backgroundcolor=pagebg,font=\color{orange}\sffamily, fontcolor=white
]{examplebox}{Exemple}[section]



\newcommand\R{\mathbb{R}}
\newcommand\Z{\mathbb{Z}}
\newcommand\N{\mathbb{N}}
\newcommand\E{\mathbb{E}}
\newcommand\F{\mathcal{F}}
\newcommand\cH{\mathcal{H}}
\newcommand\V{\mathbb{V}}
\newcommand\dmo{ ^{-1} }
\newcommand\kapa{\kappa}
\newcommand\im{Im}
\newcommand\hs{\mathcal{H}}





\usepackage{soul}

\makeatletter
\newcommand*{\whiten}[1]{\llap{\textcolor{white}{{\the\SOUL@token}}\hspace{#1pt}}}
\DeclareRobustCommand*\myul{%
    \def\SOUL@everyspace{\underline{\space}\kern\z@}%
    \def\SOUL@everytoken{%
     \setbox0=\hbox{\the\SOUL@token}%
     \ifdim\dp0>\z@
        \raisebox{\dp0}{\underline{\phantom{\the\SOUL@token}}}%
        \whiten{1}\whiten{0}%
        \whiten{-1}\whiten{-2}%
        \llap{\the\SOUL@token}%
     \else
        \underline{\the\SOUL@token}%
     \fi}%
\SOUL@}
\makeatother

\newcommand*{\demp}{\fontfamily{lmtt}\selectfont}

\DeclareTextFontCommand{\textdemp}{\demp}

\begin{document}

\ifcomment
Multiline
comment
\fi
\ifcomment
\myul{Typesetting test}
% \color[rgb]{1,1,1}
$∑_i^n≠ 60º±∞π∆¬≈√j∫h≤≥µ$

$\CR \R\pro\ind\pro\gS\pro
\mqty[a&b\\c&d]$
$\pro\mathbb{P}$
$\dd{x}$

  \[
    \alpha(x)=\left\{
                \begin{array}{ll}
                  x\\
                  \frac{1}{1+e^{-kx}}\\
                  \frac{e^x-e^{-x}}{e^x+e^{-x}}
                \end{array}
              \right.
  \]

  $\expval{x}$
  
  $\chi_\rho(ghg\dmo)=\Tr(\rho_{ghg\dmo})=\Tr(\rho_g\circ\rho_h\circ\rho\dmo_g)=\Tr(\rho_h)\overset{\mbox{\scalebox{0.5}{$\Tr(AB)=\Tr(BA)$}}}{=}\chi_\rho(h)$
  	$\mathop{\oplus}_{\substack{x\in X}}$

$\mat(\rho_g)=(a_{ij}(g))_{\scriptsize \substack{1\leq i\leq d \\ 1\leq j\leq d}}$ et $\mat(\rho'_g)=(a'_{ij}(g))_{\scriptsize \substack{1\leq i'\leq d' \\ 1\leq j'\leq d'}}$



\[\int_a^b{\mathbb{R}^2}g(u, v)\dd{P_{XY}}(u, v)=\iint g(u,v) f_{XY}(u, v)\dd \lambda(u) \dd \lambda(v)\]
$$\lim_{x\to\infty} f(x)$$	
$$\iiiint_V \mu(t,u,v,w) \,dt\,du\,dv\,dw$$
$$\sum_{n=1}^{\infty} 2^{-n} = 1$$	
\begin{definition}
	Si $X$ et $Y$ sont 2 v.a. ou definit la \textsc{Covariance} entre $X$ et $Y$ comme
	$\cov(X,Y)\overset{\text{def}}{=}\E\left[(X-\E(X))(Y-\E(Y))\right]=\E(XY)-\E(X)\E(Y)$.
\end{definition}
\fi
\pagebreak

% \tableofcontents

% insert your code here
%\input{./algebra/main.tex}
%\input{./geometrie-differentielle/main.tex}
%\input{./probabilite/main.tex}
%\input{./analyse-fonctionnelle/main.tex}
% \input{./Analyse-convexe-et-dualite-en-optimisation/main.tex}
%\input{./tikz/main.tex}
%\input{./Theorie-du-distributions/main.tex}
%\input{./optimisation/mine.tex}
 \input{./modelisation/main.tex}

% yves.aubry@univ-tln.fr : algebra

\end{document}

%% !TEX encoding = UTF-8 Unicode
% !TEX TS-program = xelatex

\documentclass[french]{report}

%\usepackage[utf8]{inputenc}
%\usepackage[T1]{fontenc}
\usepackage{babel}


\newif\ifcomment
%\commenttrue # Show comments

\usepackage{physics}
\usepackage{amssymb}


\usepackage{amsthm}
% \usepackage{thmtools}
\usepackage{mathtools}
\usepackage{amsfonts}

\usepackage{color}

\usepackage{tikz}

\usepackage{geometry}
\geometry{a5paper, margin=0.1in, right=1cm}

\usepackage{dsfont}

\usepackage{graphicx}
\graphicspath{ {images/} }

\usepackage{faktor}

\usepackage{IEEEtrantools}
\usepackage{enumerate}   
\usepackage[PostScript=dvips]{"/Users/aware/Documents/Courses/diagrams"}


\newtheorem{theorem}{Théorème}[section]
\renewcommand{\thetheorem}{\arabic{theorem}}
\newtheorem{lemme}{Lemme}[section]
\renewcommand{\thelemme}{\arabic{lemme}}
\newtheorem{proposition}{Proposition}[section]
\renewcommand{\theproposition}{\arabic{proposition}}
\newtheorem{notations}{Notations}[section]
\newtheorem{problem}{Problème}[section]
\newtheorem{corollary}{Corollaire}[theorem]
\renewcommand{\thecorollary}{\arabic{corollary}}
\newtheorem{property}{Propriété}[section]
\newtheorem{objective}{Objectif}[section]

\theoremstyle{definition}
\newtheorem{definition}{Définition}[section]
\renewcommand{\thedefinition}{\arabic{definition}}
\newtheorem{exercise}{Exercice}[chapter]
\renewcommand{\theexercise}{\arabic{exercise}}
\newtheorem{example}{Exemple}[chapter]
\renewcommand{\theexample}{\arabic{example}}
\newtheorem*{solution}{Solution}
\newtheorem*{application}{Application}
\newtheorem*{notation}{Notation}
\newtheorem*{vocabulary}{Vocabulaire}
\newtheorem*{properties}{Propriétés}



\theoremstyle{remark}
\newtheorem*{remark}{Remarque}
\newtheorem*{rappel}{Rappel}


\usepackage{etoolbox}
\AtBeginEnvironment{exercise}{\small}
\AtBeginEnvironment{example}{\small}

\usepackage{cases}
\usepackage[red]{mypack}

\usepackage[framemethod=TikZ]{mdframed}

\definecolor{bg}{rgb}{0.4,0.25,0.95}
\definecolor{pagebg}{rgb}{0,0,0.5}
\surroundwithmdframed[
   topline=false,
   rightline=false,
   bottomline=false,
   leftmargin=\parindent,
   skipabove=8pt,
   skipbelow=8pt,
   linecolor=blue,
   innerbottommargin=10pt,
   % backgroundcolor=bg,font=\color{orange}\sffamily, fontcolor=white
]{definition}

\usepackage{empheq}
\usepackage[most]{tcolorbox}

\newtcbox{\mymath}[1][]{%
    nobeforeafter, math upper, tcbox raise base,
    enhanced, colframe=blue!30!black,
    colback=red!10, boxrule=1pt,
    #1}

\usepackage{unixode}


\DeclareMathOperator{\ord}{ord}
\DeclareMathOperator{\orb}{orb}
\DeclareMathOperator{\stab}{stab}
\DeclareMathOperator{\Stab}{stab}
\DeclareMathOperator{\ppcm}{ppcm}
\DeclareMathOperator{\conj}{Conj}
\DeclareMathOperator{\End}{End}
\DeclareMathOperator{\rot}{rot}
\DeclareMathOperator{\trs}{trace}
\DeclareMathOperator{\Ind}{Ind}
\DeclareMathOperator{\mat}{Mat}
\DeclareMathOperator{\id}{Id}
\DeclareMathOperator{\vect}{vect}
\DeclareMathOperator{\img}{img}
\DeclareMathOperator{\cov}{Cov}
\DeclareMathOperator{\dist}{dist}
\DeclareMathOperator{\irr}{Irr}
\DeclareMathOperator{\image}{Im}
\DeclareMathOperator{\pd}{\partial}
\DeclareMathOperator{\epi}{epi}
\DeclareMathOperator{\Argmin}{Argmin}
\DeclareMathOperator{\dom}{dom}
\DeclareMathOperator{\proj}{proj}
\DeclareMathOperator{\ctg}{ctg}
\DeclareMathOperator{\supp}{supp}
\DeclareMathOperator{\argmin}{argmin}
\DeclareMathOperator{\mult}{mult}
\DeclareMathOperator{\ch}{ch}
\DeclareMathOperator{\sh}{sh}
\DeclareMathOperator{\rang}{rang}
\DeclareMathOperator{\diam}{diam}
\DeclareMathOperator{\Epigraphe}{Epigraphe}




\usepackage{xcolor}
\everymath{\color{blue}}
%\everymath{\color[rgb]{0,1,1}}
%\pagecolor[rgb]{0,0,0.5}


\newcommand*{\pdtest}[3][]{\ensuremath{\frac{\partial^{#1} #2}{\partial #3}}}

\newcommand*{\deffunc}[6][]{\ensuremath{
\begin{array}{rcl}
#2 : #3 &\rightarrow& #4\\
#5 &\mapsto& #6
\end{array}
}}

\newcommand{\eqcolon}{\mathrel{\resizebox{\widthof{$\mathord{=}$}}{\height}{ $\!\!=\!\!\resizebox{1.2\width}{0.8\height}{\raisebox{0.23ex}{$\mathop{:}$}}\!\!$ }}}
\newcommand{\coloneq}{\mathrel{\resizebox{\widthof{$\mathord{=}$}}{\height}{ $\!\!\resizebox{1.2\width}{0.8\height}{\raisebox{0.23ex}{$\mathop{:}$}}\!\!=\!\!$ }}}
\newcommand{\eqcolonl}{\ensuremath{\mathrel{=\!\!\mathop{:}}}}
\newcommand{\coloneql}{\ensuremath{\mathrel{\mathop{:} \!\! =}}}
\newcommand{\vc}[1]{% inline column vector
  \left(\begin{smallmatrix}#1\end{smallmatrix}\right)%
}
\newcommand{\vr}[1]{% inline row vector
  \begin{smallmatrix}(\,#1\,)\end{smallmatrix}%
}
\makeatletter
\newcommand*{\defeq}{\ =\mathrel{\rlap{%
                     \raisebox{0.3ex}{$\m@th\cdot$}}%
                     \raisebox{-0.3ex}{$\m@th\cdot$}}%
                     }
\makeatother

\newcommand{\mathcircle}[1]{% inline row vector
 \overset{\circ}{#1}
}
\newcommand{\ulim}{% low limit
 \underline{\lim}
}
\newcommand{\ssi}{% iff
\iff
}
\newcommand{\ps}[2]{
\expval{#1 | #2}
}
\newcommand{\df}[1]{
\mqty{#1}
}
\newcommand{\n}[1]{
\norm{#1}
}
\newcommand{\sys}[1]{
\left\{\smqty{#1}\right.
}


\newcommand{\eqdef}{\ensuremath{\overset{\text{def}}=}}


\def\Circlearrowright{\ensuremath{%
  \rotatebox[origin=c]{230}{$\circlearrowright$}}}

\newcommand\ct[1]{\text{\rmfamily\upshape #1}}
\newcommand\question[1]{ {\color{red} ...!? \small #1}}
\newcommand\caz[1]{\left\{\begin{array} #1 \end{array}\right.}
\newcommand\const{\text{\rmfamily\upshape const}}
\newcommand\toP{ \overset{\pro}{\to}}
\newcommand\toPP{ \overset{\text{PP}}{\to}}
\newcommand{\oeq}{\mathrel{\text{\textcircled{$=$}}}}





\usepackage{xcolor}
% \usepackage[normalem]{ulem}
\usepackage{lipsum}
\makeatletter
% \newcommand\colorwave[1][blue]{\bgroup \markoverwith{\lower3.5\p@\hbox{\sixly \textcolor{#1}{\char58}}}\ULon}
%\font\sixly=lasy6 % does not re-load if already loaded, so no memory problem.

\newmdtheoremenv[
linewidth= 1pt,linecolor= blue,%
leftmargin=20,rightmargin=20,innertopmargin=0pt, innerrightmargin=40,%
tikzsetting = { draw=lightgray, line width = 0.3pt,dashed,%
dash pattern = on 15pt off 3pt},%
splittopskip=\topskip,skipbelow=\baselineskip,%
skipabove=\baselineskip,ntheorem,roundcorner=0pt,
% backgroundcolor=pagebg,font=\color{orange}\sffamily, fontcolor=white
]{examplebox}{Exemple}[section]



\newcommand\R{\mathbb{R}}
\newcommand\Z{\mathbb{Z}}
\newcommand\N{\mathbb{N}}
\newcommand\E{\mathbb{E}}
\newcommand\F{\mathcal{F}}
\newcommand\cH{\mathcal{H}}
\newcommand\V{\mathbb{V}}
\newcommand\dmo{ ^{-1} }
\newcommand\kapa{\kappa}
\newcommand\im{Im}
\newcommand\hs{\mathcal{H}}





\usepackage{soul}

\makeatletter
\newcommand*{\whiten}[1]{\llap{\textcolor{white}{{\the\SOUL@token}}\hspace{#1pt}}}
\DeclareRobustCommand*\myul{%
    \def\SOUL@everyspace{\underline{\space}\kern\z@}%
    \def\SOUL@everytoken{%
     \setbox0=\hbox{\the\SOUL@token}%
     \ifdim\dp0>\z@
        \raisebox{\dp0}{\underline{\phantom{\the\SOUL@token}}}%
        \whiten{1}\whiten{0}%
        \whiten{-1}\whiten{-2}%
        \llap{\the\SOUL@token}%
     \else
        \underline{\the\SOUL@token}%
     \fi}%
\SOUL@}
\makeatother

\newcommand*{\demp}{\fontfamily{lmtt}\selectfont}

\DeclareTextFontCommand{\textdemp}{\demp}

\begin{document}

\ifcomment
Multiline
comment
\fi
\ifcomment
\myul{Typesetting test}
% \color[rgb]{1,1,1}
$∑_i^n≠ 60º±∞π∆¬≈√j∫h≤≥µ$

$\CR \R\pro\ind\pro\gS\pro
\mqty[a&b\\c&d]$
$\pro\mathbb{P}$
$\dd{x}$

  \[
    \alpha(x)=\left\{
                \begin{array}{ll}
                  x\\
                  \frac{1}{1+e^{-kx}}\\
                  \frac{e^x-e^{-x}}{e^x+e^{-x}}
                \end{array}
              \right.
  \]

  $\expval{x}$
  
  $\chi_\rho(ghg\dmo)=\Tr(\rho_{ghg\dmo})=\Tr(\rho_g\circ\rho_h\circ\rho\dmo_g)=\Tr(\rho_h)\overset{\mbox{\scalebox{0.5}{$\Tr(AB)=\Tr(BA)$}}}{=}\chi_\rho(h)$
  	$\mathop{\oplus}_{\substack{x\in X}}$

$\mat(\rho_g)=(a_{ij}(g))_{\scriptsize \substack{1\leq i\leq d \\ 1\leq j\leq d}}$ et $\mat(\rho'_g)=(a'_{ij}(g))_{\scriptsize \substack{1\leq i'\leq d' \\ 1\leq j'\leq d'}}$



\[\int_a^b{\mathbb{R}^2}g(u, v)\dd{P_{XY}}(u, v)=\iint g(u,v) f_{XY}(u, v)\dd \lambda(u) \dd \lambda(v)\]
$$\lim_{x\to\infty} f(x)$$	
$$\iiiint_V \mu(t,u,v,w) \,dt\,du\,dv\,dw$$
$$\sum_{n=1}^{\infty} 2^{-n} = 1$$	
\begin{definition}
	Si $X$ et $Y$ sont 2 v.a. ou definit la \textsc{Covariance} entre $X$ et $Y$ comme
	$\cov(X,Y)\overset{\text{def}}{=}\E\left[(X-\E(X))(Y-\E(Y))\right]=\E(XY)-\E(X)\E(Y)$.
\end{definition}
\fi
\pagebreak

% \tableofcontents

% insert your code here
%\input{./algebra/main.tex}
%\input{./geometrie-differentielle/main.tex}
%\input{./probabilite/main.tex}
%\input{./analyse-fonctionnelle/main.tex}
% \input{./Analyse-convexe-et-dualite-en-optimisation/main.tex}
%\input{./tikz/main.tex}
%\input{./Theorie-du-distributions/main.tex}
%\input{./optimisation/mine.tex}
 \input{./modelisation/main.tex}

% yves.aubry@univ-tln.fr : algebra

\end{document}

%% !TEX encoding = UTF-8 Unicode
% !TEX TS-program = xelatex

\documentclass[french]{report}

%\usepackage[utf8]{inputenc}
%\usepackage[T1]{fontenc}
\usepackage{babel}


\newif\ifcomment
%\commenttrue # Show comments

\usepackage{physics}
\usepackage{amssymb}


\usepackage{amsthm}
% \usepackage{thmtools}
\usepackage{mathtools}
\usepackage{amsfonts}

\usepackage{color}

\usepackage{tikz}

\usepackage{geometry}
\geometry{a5paper, margin=0.1in, right=1cm}

\usepackage{dsfont}

\usepackage{graphicx}
\graphicspath{ {images/} }

\usepackage{faktor}

\usepackage{IEEEtrantools}
\usepackage{enumerate}   
\usepackage[PostScript=dvips]{"/Users/aware/Documents/Courses/diagrams"}


\newtheorem{theorem}{Théorème}[section]
\renewcommand{\thetheorem}{\arabic{theorem}}
\newtheorem{lemme}{Lemme}[section]
\renewcommand{\thelemme}{\arabic{lemme}}
\newtheorem{proposition}{Proposition}[section]
\renewcommand{\theproposition}{\arabic{proposition}}
\newtheorem{notations}{Notations}[section]
\newtheorem{problem}{Problème}[section]
\newtheorem{corollary}{Corollaire}[theorem]
\renewcommand{\thecorollary}{\arabic{corollary}}
\newtheorem{property}{Propriété}[section]
\newtheorem{objective}{Objectif}[section]

\theoremstyle{definition}
\newtheorem{definition}{Définition}[section]
\renewcommand{\thedefinition}{\arabic{definition}}
\newtheorem{exercise}{Exercice}[chapter]
\renewcommand{\theexercise}{\arabic{exercise}}
\newtheorem{example}{Exemple}[chapter]
\renewcommand{\theexample}{\arabic{example}}
\newtheorem*{solution}{Solution}
\newtheorem*{application}{Application}
\newtheorem*{notation}{Notation}
\newtheorem*{vocabulary}{Vocabulaire}
\newtheorem*{properties}{Propriétés}



\theoremstyle{remark}
\newtheorem*{remark}{Remarque}
\newtheorem*{rappel}{Rappel}


\usepackage{etoolbox}
\AtBeginEnvironment{exercise}{\small}
\AtBeginEnvironment{example}{\small}

\usepackage{cases}
\usepackage[red]{mypack}

\usepackage[framemethod=TikZ]{mdframed}

\definecolor{bg}{rgb}{0.4,0.25,0.95}
\definecolor{pagebg}{rgb}{0,0,0.5}
\surroundwithmdframed[
   topline=false,
   rightline=false,
   bottomline=false,
   leftmargin=\parindent,
   skipabove=8pt,
   skipbelow=8pt,
   linecolor=blue,
   innerbottommargin=10pt,
   % backgroundcolor=bg,font=\color{orange}\sffamily, fontcolor=white
]{definition}

\usepackage{empheq}
\usepackage[most]{tcolorbox}

\newtcbox{\mymath}[1][]{%
    nobeforeafter, math upper, tcbox raise base,
    enhanced, colframe=blue!30!black,
    colback=red!10, boxrule=1pt,
    #1}

\usepackage{unixode}


\DeclareMathOperator{\ord}{ord}
\DeclareMathOperator{\orb}{orb}
\DeclareMathOperator{\stab}{stab}
\DeclareMathOperator{\Stab}{stab}
\DeclareMathOperator{\ppcm}{ppcm}
\DeclareMathOperator{\conj}{Conj}
\DeclareMathOperator{\End}{End}
\DeclareMathOperator{\rot}{rot}
\DeclareMathOperator{\trs}{trace}
\DeclareMathOperator{\Ind}{Ind}
\DeclareMathOperator{\mat}{Mat}
\DeclareMathOperator{\id}{Id}
\DeclareMathOperator{\vect}{vect}
\DeclareMathOperator{\img}{img}
\DeclareMathOperator{\cov}{Cov}
\DeclareMathOperator{\dist}{dist}
\DeclareMathOperator{\irr}{Irr}
\DeclareMathOperator{\image}{Im}
\DeclareMathOperator{\pd}{\partial}
\DeclareMathOperator{\epi}{epi}
\DeclareMathOperator{\Argmin}{Argmin}
\DeclareMathOperator{\dom}{dom}
\DeclareMathOperator{\proj}{proj}
\DeclareMathOperator{\ctg}{ctg}
\DeclareMathOperator{\supp}{supp}
\DeclareMathOperator{\argmin}{argmin}
\DeclareMathOperator{\mult}{mult}
\DeclareMathOperator{\ch}{ch}
\DeclareMathOperator{\sh}{sh}
\DeclareMathOperator{\rang}{rang}
\DeclareMathOperator{\diam}{diam}
\DeclareMathOperator{\Epigraphe}{Epigraphe}




\usepackage{xcolor}
\everymath{\color{blue}}
%\everymath{\color[rgb]{0,1,1}}
%\pagecolor[rgb]{0,0,0.5}


\newcommand*{\pdtest}[3][]{\ensuremath{\frac{\partial^{#1} #2}{\partial #3}}}

\newcommand*{\deffunc}[6][]{\ensuremath{
\begin{array}{rcl}
#2 : #3 &\rightarrow& #4\\
#5 &\mapsto& #6
\end{array}
}}

\newcommand{\eqcolon}{\mathrel{\resizebox{\widthof{$\mathord{=}$}}{\height}{ $\!\!=\!\!\resizebox{1.2\width}{0.8\height}{\raisebox{0.23ex}{$\mathop{:}$}}\!\!$ }}}
\newcommand{\coloneq}{\mathrel{\resizebox{\widthof{$\mathord{=}$}}{\height}{ $\!\!\resizebox{1.2\width}{0.8\height}{\raisebox{0.23ex}{$\mathop{:}$}}\!\!=\!\!$ }}}
\newcommand{\eqcolonl}{\ensuremath{\mathrel{=\!\!\mathop{:}}}}
\newcommand{\coloneql}{\ensuremath{\mathrel{\mathop{:} \!\! =}}}
\newcommand{\vc}[1]{% inline column vector
  \left(\begin{smallmatrix}#1\end{smallmatrix}\right)%
}
\newcommand{\vr}[1]{% inline row vector
  \begin{smallmatrix}(\,#1\,)\end{smallmatrix}%
}
\makeatletter
\newcommand*{\defeq}{\ =\mathrel{\rlap{%
                     \raisebox{0.3ex}{$\m@th\cdot$}}%
                     \raisebox{-0.3ex}{$\m@th\cdot$}}%
                     }
\makeatother

\newcommand{\mathcircle}[1]{% inline row vector
 \overset{\circ}{#1}
}
\newcommand{\ulim}{% low limit
 \underline{\lim}
}
\newcommand{\ssi}{% iff
\iff
}
\newcommand{\ps}[2]{
\expval{#1 | #2}
}
\newcommand{\df}[1]{
\mqty{#1}
}
\newcommand{\n}[1]{
\norm{#1}
}
\newcommand{\sys}[1]{
\left\{\smqty{#1}\right.
}


\newcommand{\eqdef}{\ensuremath{\overset{\text{def}}=}}


\def\Circlearrowright{\ensuremath{%
  \rotatebox[origin=c]{230}{$\circlearrowright$}}}

\newcommand\ct[1]{\text{\rmfamily\upshape #1}}
\newcommand\question[1]{ {\color{red} ...!? \small #1}}
\newcommand\caz[1]{\left\{\begin{array} #1 \end{array}\right.}
\newcommand\const{\text{\rmfamily\upshape const}}
\newcommand\toP{ \overset{\pro}{\to}}
\newcommand\toPP{ \overset{\text{PP}}{\to}}
\newcommand{\oeq}{\mathrel{\text{\textcircled{$=$}}}}





\usepackage{xcolor}
% \usepackage[normalem]{ulem}
\usepackage{lipsum}
\makeatletter
% \newcommand\colorwave[1][blue]{\bgroup \markoverwith{\lower3.5\p@\hbox{\sixly \textcolor{#1}{\char58}}}\ULon}
%\font\sixly=lasy6 % does not re-load if already loaded, so no memory problem.

\newmdtheoremenv[
linewidth= 1pt,linecolor= blue,%
leftmargin=20,rightmargin=20,innertopmargin=0pt, innerrightmargin=40,%
tikzsetting = { draw=lightgray, line width = 0.3pt,dashed,%
dash pattern = on 15pt off 3pt},%
splittopskip=\topskip,skipbelow=\baselineskip,%
skipabove=\baselineskip,ntheorem,roundcorner=0pt,
% backgroundcolor=pagebg,font=\color{orange}\sffamily, fontcolor=white
]{examplebox}{Exemple}[section]



\newcommand\R{\mathbb{R}}
\newcommand\Z{\mathbb{Z}}
\newcommand\N{\mathbb{N}}
\newcommand\E{\mathbb{E}}
\newcommand\F{\mathcal{F}}
\newcommand\cH{\mathcal{H}}
\newcommand\V{\mathbb{V}}
\newcommand\dmo{ ^{-1} }
\newcommand\kapa{\kappa}
\newcommand\im{Im}
\newcommand\hs{\mathcal{H}}





\usepackage{soul}

\makeatletter
\newcommand*{\whiten}[1]{\llap{\textcolor{white}{{\the\SOUL@token}}\hspace{#1pt}}}
\DeclareRobustCommand*\myul{%
    \def\SOUL@everyspace{\underline{\space}\kern\z@}%
    \def\SOUL@everytoken{%
     \setbox0=\hbox{\the\SOUL@token}%
     \ifdim\dp0>\z@
        \raisebox{\dp0}{\underline{\phantom{\the\SOUL@token}}}%
        \whiten{1}\whiten{0}%
        \whiten{-1}\whiten{-2}%
        \llap{\the\SOUL@token}%
     \else
        \underline{\the\SOUL@token}%
     \fi}%
\SOUL@}
\makeatother

\newcommand*{\demp}{\fontfamily{lmtt}\selectfont}

\DeclareTextFontCommand{\textdemp}{\demp}

\begin{document}

\ifcomment
Multiline
comment
\fi
\ifcomment
\myul{Typesetting test}
% \color[rgb]{1,1,1}
$∑_i^n≠ 60º±∞π∆¬≈√j∫h≤≥µ$

$\CR \R\pro\ind\pro\gS\pro
\mqty[a&b\\c&d]$
$\pro\mathbb{P}$
$\dd{x}$

  \[
    \alpha(x)=\left\{
                \begin{array}{ll}
                  x\\
                  \frac{1}{1+e^{-kx}}\\
                  \frac{e^x-e^{-x}}{e^x+e^{-x}}
                \end{array}
              \right.
  \]

  $\expval{x}$
  
  $\chi_\rho(ghg\dmo)=\Tr(\rho_{ghg\dmo})=\Tr(\rho_g\circ\rho_h\circ\rho\dmo_g)=\Tr(\rho_h)\overset{\mbox{\scalebox{0.5}{$\Tr(AB)=\Tr(BA)$}}}{=}\chi_\rho(h)$
  	$\mathop{\oplus}_{\substack{x\in X}}$

$\mat(\rho_g)=(a_{ij}(g))_{\scriptsize \substack{1\leq i\leq d \\ 1\leq j\leq d}}$ et $\mat(\rho'_g)=(a'_{ij}(g))_{\scriptsize \substack{1\leq i'\leq d' \\ 1\leq j'\leq d'}}$



\[\int_a^b{\mathbb{R}^2}g(u, v)\dd{P_{XY}}(u, v)=\iint g(u,v) f_{XY}(u, v)\dd \lambda(u) \dd \lambda(v)\]
$$\lim_{x\to\infty} f(x)$$	
$$\iiiint_V \mu(t,u,v,w) \,dt\,du\,dv\,dw$$
$$\sum_{n=1}^{\infty} 2^{-n} = 1$$	
\begin{definition}
	Si $X$ et $Y$ sont 2 v.a. ou definit la \textsc{Covariance} entre $X$ et $Y$ comme
	$\cov(X,Y)\overset{\text{def}}{=}\E\left[(X-\E(X))(Y-\E(Y))\right]=\E(XY)-\E(X)\E(Y)$.
\end{definition}
\fi
\pagebreak

% \tableofcontents

% insert your code here
%\input{./algebra/main.tex}
%\input{./geometrie-differentielle/main.tex}
%\input{./probabilite/main.tex}
%\input{./analyse-fonctionnelle/main.tex}
% \input{./Analyse-convexe-et-dualite-en-optimisation/main.tex}
%\input{./tikz/main.tex}
%\input{./Theorie-du-distributions/main.tex}
%\input{./optimisation/mine.tex}
 \input{./modelisation/main.tex}

% yves.aubry@univ-tln.fr : algebra

\end{document}

%\input{./optimisation/mine.tex}
 % !TEX encoding = UTF-8 Unicode
% !TEX TS-program = xelatex

\documentclass[french]{report}

%\usepackage[utf8]{inputenc}
%\usepackage[T1]{fontenc}
\usepackage{babel}


\newif\ifcomment
%\commenttrue # Show comments

\usepackage{physics}
\usepackage{amssymb}


\usepackage{amsthm}
% \usepackage{thmtools}
\usepackage{mathtools}
\usepackage{amsfonts}

\usepackage{color}

\usepackage{tikz}

\usepackage{geometry}
\geometry{a5paper, margin=0.1in, right=1cm}

\usepackage{dsfont}

\usepackage{graphicx}
\graphicspath{ {images/} }

\usepackage{faktor}

\usepackage{IEEEtrantools}
\usepackage{enumerate}   
\usepackage[PostScript=dvips]{"/Users/aware/Documents/Courses/diagrams"}


\newtheorem{theorem}{Théorème}[section]
\renewcommand{\thetheorem}{\arabic{theorem}}
\newtheorem{lemme}{Lemme}[section]
\renewcommand{\thelemme}{\arabic{lemme}}
\newtheorem{proposition}{Proposition}[section]
\renewcommand{\theproposition}{\arabic{proposition}}
\newtheorem{notations}{Notations}[section]
\newtheorem{problem}{Problème}[section]
\newtheorem{corollary}{Corollaire}[theorem]
\renewcommand{\thecorollary}{\arabic{corollary}}
\newtheorem{property}{Propriété}[section]
\newtheorem{objective}{Objectif}[section]

\theoremstyle{definition}
\newtheorem{definition}{Définition}[section]
\renewcommand{\thedefinition}{\arabic{definition}}
\newtheorem{exercise}{Exercice}[chapter]
\renewcommand{\theexercise}{\arabic{exercise}}
\newtheorem{example}{Exemple}[chapter]
\renewcommand{\theexample}{\arabic{example}}
\newtheorem*{solution}{Solution}
\newtheorem*{application}{Application}
\newtheorem*{notation}{Notation}
\newtheorem*{vocabulary}{Vocabulaire}
\newtheorem*{properties}{Propriétés}



\theoremstyle{remark}
\newtheorem*{remark}{Remarque}
\newtheorem*{rappel}{Rappel}


\usepackage{etoolbox}
\AtBeginEnvironment{exercise}{\small}
\AtBeginEnvironment{example}{\small}

\usepackage{cases}
\usepackage[red]{mypack}

\usepackage[framemethod=TikZ]{mdframed}

\definecolor{bg}{rgb}{0.4,0.25,0.95}
\definecolor{pagebg}{rgb}{0,0,0.5}
\surroundwithmdframed[
   topline=false,
   rightline=false,
   bottomline=false,
   leftmargin=\parindent,
   skipabove=8pt,
   skipbelow=8pt,
   linecolor=blue,
   innerbottommargin=10pt,
   % backgroundcolor=bg,font=\color{orange}\sffamily, fontcolor=white
]{definition}

\usepackage{empheq}
\usepackage[most]{tcolorbox}

\newtcbox{\mymath}[1][]{%
    nobeforeafter, math upper, tcbox raise base,
    enhanced, colframe=blue!30!black,
    colback=red!10, boxrule=1pt,
    #1}

\usepackage{unixode}


\DeclareMathOperator{\ord}{ord}
\DeclareMathOperator{\orb}{orb}
\DeclareMathOperator{\stab}{stab}
\DeclareMathOperator{\Stab}{stab}
\DeclareMathOperator{\ppcm}{ppcm}
\DeclareMathOperator{\conj}{Conj}
\DeclareMathOperator{\End}{End}
\DeclareMathOperator{\rot}{rot}
\DeclareMathOperator{\trs}{trace}
\DeclareMathOperator{\Ind}{Ind}
\DeclareMathOperator{\mat}{Mat}
\DeclareMathOperator{\id}{Id}
\DeclareMathOperator{\vect}{vect}
\DeclareMathOperator{\img}{img}
\DeclareMathOperator{\cov}{Cov}
\DeclareMathOperator{\dist}{dist}
\DeclareMathOperator{\irr}{Irr}
\DeclareMathOperator{\image}{Im}
\DeclareMathOperator{\pd}{\partial}
\DeclareMathOperator{\epi}{epi}
\DeclareMathOperator{\Argmin}{Argmin}
\DeclareMathOperator{\dom}{dom}
\DeclareMathOperator{\proj}{proj}
\DeclareMathOperator{\ctg}{ctg}
\DeclareMathOperator{\supp}{supp}
\DeclareMathOperator{\argmin}{argmin}
\DeclareMathOperator{\mult}{mult}
\DeclareMathOperator{\ch}{ch}
\DeclareMathOperator{\sh}{sh}
\DeclareMathOperator{\rang}{rang}
\DeclareMathOperator{\diam}{diam}
\DeclareMathOperator{\Epigraphe}{Epigraphe}




\usepackage{xcolor}
\everymath{\color{blue}}
%\everymath{\color[rgb]{0,1,1}}
%\pagecolor[rgb]{0,0,0.5}


\newcommand*{\pdtest}[3][]{\ensuremath{\frac{\partial^{#1} #2}{\partial #3}}}

\newcommand*{\deffunc}[6][]{\ensuremath{
\begin{array}{rcl}
#2 : #3 &\rightarrow& #4\\
#5 &\mapsto& #6
\end{array}
}}

\newcommand{\eqcolon}{\mathrel{\resizebox{\widthof{$\mathord{=}$}}{\height}{ $\!\!=\!\!\resizebox{1.2\width}{0.8\height}{\raisebox{0.23ex}{$\mathop{:}$}}\!\!$ }}}
\newcommand{\coloneq}{\mathrel{\resizebox{\widthof{$\mathord{=}$}}{\height}{ $\!\!\resizebox{1.2\width}{0.8\height}{\raisebox{0.23ex}{$\mathop{:}$}}\!\!=\!\!$ }}}
\newcommand{\eqcolonl}{\ensuremath{\mathrel{=\!\!\mathop{:}}}}
\newcommand{\coloneql}{\ensuremath{\mathrel{\mathop{:} \!\! =}}}
\newcommand{\vc}[1]{% inline column vector
  \left(\begin{smallmatrix}#1\end{smallmatrix}\right)%
}
\newcommand{\vr}[1]{% inline row vector
  \begin{smallmatrix}(\,#1\,)\end{smallmatrix}%
}
\makeatletter
\newcommand*{\defeq}{\ =\mathrel{\rlap{%
                     \raisebox{0.3ex}{$\m@th\cdot$}}%
                     \raisebox{-0.3ex}{$\m@th\cdot$}}%
                     }
\makeatother

\newcommand{\mathcircle}[1]{% inline row vector
 \overset{\circ}{#1}
}
\newcommand{\ulim}{% low limit
 \underline{\lim}
}
\newcommand{\ssi}{% iff
\iff
}
\newcommand{\ps}[2]{
\expval{#1 | #2}
}
\newcommand{\df}[1]{
\mqty{#1}
}
\newcommand{\n}[1]{
\norm{#1}
}
\newcommand{\sys}[1]{
\left\{\smqty{#1}\right.
}


\newcommand{\eqdef}{\ensuremath{\overset{\text{def}}=}}


\def\Circlearrowright{\ensuremath{%
  \rotatebox[origin=c]{230}{$\circlearrowright$}}}

\newcommand\ct[1]{\text{\rmfamily\upshape #1}}
\newcommand\question[1]{ {\color{red} ...!? \small #1}}
\newcommand\caz[1]{\left\{\begin{array} #1 \end{array}\right.}
\newcommand\const{\text{\rmfamily\upshape const}}
\newcommand\toP{ \overset{\pro}{\to}}
\newcommand\toPP{ \overset{\text{PP}}{\to}}
\newcommand{\oeq}{\mathrel{\text{\textcircled{$=$}}}}





\usepackage{xcolor}
% \usepackage[normalem]{ulem}
\usepackage{lipsum}
\makeatletter
% \newcommand\colorwave[1][blue]{\bgroup \markoverwith{\lower3.5\p@\hbox{\sixly \textcolor{#1}{\char58}}}\ULon}
%\font\sixly=lasy6 % does not re-load if already loaded, so no memory problem.

\newmdtheoremenv[
linewidth= 1pt,linecolor= blue,%
leftmargin=20,rightmargin=20,innertopmargin=0pt, innerrightmargin=40,%
tikzsetting = { draw=lightgray, line width = 0.3pt,dashed,%
dash pattern = on 15pt off 3pt},%
splittopskip=\topskip,skipbelow=\baselineskip,%
skipabove=\baselineskip,ntheorem,roundcorner=0pt,
% backgroundcolor=pagebg,font=\color{orange}\sffamily, fontcolor=white
]{examplebox}{Exemple}[section]



\newcommand\R{\mathbb{R}}
\newcommand\Z{\mathbb{Z}}
\newcommand\N{\mathbb{N}}
\newcommand\E{\mathbb{E}}
\newcommand\F{\mathcal{F}}
\newcommand\cH{\mathcal{H}}
\newcommand\V{\mathbb{V}}
\newcommand\dmo{ ^{-1} }
\newcommand\kapa{\kappa}
\newcommand\im{Im}
\newcommand\hs{\mathcal{H}}





\usepackage{soul}

\makeatletter
\newcommand*{\whiten}[1]{\llap{\textcolor{white}{{\the\SOUL@token}}\hspace{#1pt}}}
\DeclareRobustCommand*\myul{%
    \def\SOUL@everyspace{\underline{\space}\kern\z@}%
    \def\SOUL@everytoken{%
     \setbox0=\hbox{\the\SOUL@token}%
     \ifdim\dp0>\z@
        \raisebox{\dp0}{\underline{\phantom{\the\SOUL@token}}}%
        \whiten{1}\whiten{0}%
        \whiten{-1}\whiten{-2}%
        \llap{\the\SOUL@token}%
     \else
        \underline{\the\SOUL@token}%
     \fi}%
\SOUL@}
\makeatother

\newcommand*{\demp}{\fontfamily{lmtt}\selectfont}

\DeclareTextFontCommand{\textdemp}{\demp}

\begin{document}

\ifcomment
Multiline
comment
\fi
\ifcomment
\myul{Typesetting test}
% \color[rgb]{1,1,1}
$∑_i^n≠ 60º±∞π∆¬≈√j∫h≤≥µ$

$\CR \R\pro\ind\pro\gS\pro
\mqty[a&b\\c&d]$
$\pro\mathbb{P}$
$\dd{x}$

  \[
    \alpha(x)=\left\{
                \begin{array}{ll}
                  x\\
                  \frac{1}{1+e^{-kx}}\\
                  \frac{e^x-e^{-x}}{e^x+e^{-x}}
                \end{array}
              \right.
  \]

  $\expval{x}$
  
  $\chi_\rho(ghg\dmo)=\Tr(\rho_{ghg\dmo})=\Tr(\rho_g\circ\rho_h\circ\rho\dmo_g)=\Tr(\rho_h)\overset{\mbox{\scalebox{0.5}{$\Tr(AB)=\Tr(BA)$}}}{=}\chi_\rho(h)$
  	$\mathop{\oplus}_{\substack{x\in X}}$

$\mat(\rho_g)=(a_{ij}(g))_{\scriptsize \substack{1\leq i\leq d \\ 1\leq j\leq d}}$ et $\mat(\rho'_g)=(a'_{ij}(g))_{\scriptsize \substack{1\leq i'\leq d' \\ 1\leq j'\leq d'}}$



\[\int_a^b{\mathbb{R}^2}g(u, v)\dd{P_{XY}}(u, v)=\iint g(u,v) f_{XY}(u, v)\dd \lambda(u) \dd \lambda(v)\]
$$\lim_{x\to\infty} f(x)$$	
$$\iiiint_V \mu(t,u,v,w) \,dt\,du\,dv\,dw$$
$$\sum_{n=1}^{\infty} 2^{-n} = 1$$	
\begin{definition}
	Si $X$ et $Y$ sont 2 v.a. ou definit la \textsc{Covariance} entre $X$ et $Y$ comme
	$\cov(X,Y)\overset{\text{def}}{=}\E\left[(X-\E(X))(Y-\E(Y))\right]=\E(XY)-\E(X)\E(Y)$.
\end{definition}
\fi
\pagebreak

% \tableofcontents

% insert your code here
%\input{./algebra/main.tex}
%\input{./geometrie-differentielle/main.tex}
%\input{./probabilite/main.tex}
%\input{./analyse-fonctionnelle/main.tex}
% \input{./Analyse-convexe-et-dualite-en-optimisation/main.tex}
%\input{./tikz/main.tex}
%\input{./Theorie-du-distributions/main.tex}
%\input{./optimisation/mine.tex}
 \input{./modelisation/main.tex}

% yves.aubry@univ-tln.fr : algebra

\end{document}


% yves.aubry@univ-tln.fr : algebra

\end{document}

%% !TEX encoding = UTF-8 Unicode
% !TEX TS-program = xelatex

\documentclass[french]{report}

%\usepackage[utf8]{inputenc}
%\usepackage[T1]{fontenc}
\usepackage{babel}


\newif\ifcomment
%\commenttrue # Show comments

\usepackage{physics}
\usepackage{amssymb}


\usepackage{amsthm}
% \usepackage{thmtools}
\usepackage{mathtools}
\usepackage{amsfonts}

\usepackage{color}

\usepackage{tikz}

\usepackage{geometry}
\geometry{a5paper, margin=0.1in, right=1cm}

\usepackage{dsfont}

\usepackage{graphicx}
\graphicspath{ {images/} }

\usepackage{faktor}

\usepackage{IEEEtrantools}
\usepackage{enumerate}   
\usepackage[PostScript=dvips]{"/Users/aware/Documents/Courses/diagrams"}


\newtheorem{theorem}{Théorème}[section]
\renewcommand{\thetheorem}{\arabic{theorem}}
\newtheorem{lemme}{Lemme}[section]
\renewcommand{\thelemme}{\arabic{lemme}}
\newtheorem{proposition}{Proposition}[section]
\renewcommand{\theproposition}{\arabic{proposition}}
\newtheorem{notations}{Notations}[section]
\newtheorem{problem}{Problème}[section]
\newtheorem{corollary}{Corollaire}[theorem]
\renewcommand{\thecorollary}{\arabic{corollary}}
\newtheorem{property}{Propriété}[section]
\newtheorem{objective}{Objectif}[section]

\theoremstyle{definition}
\newtheorem{definition}{Définition}[section]
\renewcommand{\thedefinition}{\arabic{definition}}
\newtheorem{exercise}{Exercice}[chapter]
\renewcommand{\theexercise}{\arabic{exercise}}
\newtheorem{example}{Exemple}[chapter]
\renewcommand{\theexample}{\arabic{example}}
\newtheorem*{solution}{Solution}
\newtheorem*{application}{Application}
\newtheorem*{notation}{Notation}
\newtheorem*{vocabulary}{Vocabulaire}
\newtheorem*{properties}{Propriétés}



\theoremstyle{remark}
\newtheorem*{remark}{Remarque}
\newtheorem*{rappel}{Rappel}


\usepackage{etoolbox}
\AtBeginEnvironment{exercise}{\small}
\AtBeginEnvironment{example}{\small}

\usepackage{cases}
\usepackage[red]{mypack}

\usepackage[framemethod=TikZ]{mdframed}

\definecolor{bg}{rgb}{0.4,0.25,0.95}
\definecolor{pagebg}{rgb}{0,0,0.5}
\surroundwithmdframed[
   topline=false,
   rightline=false,
   bottomline=false,
   leftmargin=\parindent,
   skipabove=8pt,
   skipbelow=8pt,
   linecolor=blue,
   innerbottommargin=10pt,
   % backgroundcolor=bg,font=\color{orange}\sffamily, fontcolor=white
]{definition}

\usepackage{empheq}
\usepackage[most]{tcolorbox}

\newtcbox{\mymath}[1][]{%
    nobeforeafter, math upper, tcbox raise base,
    enhanced, colframe=blue!30!black,
    colback=red!10, boxrule=1pt,
    #1}

\usepackage{unixode}


\DeclareMathOperator{\ord}{ord}
\DeclareMathOperator{\orb}{orb}
\DeclareMathOperator{\stab}{stab}
\DeclareMathOperator{\Stab}{stab}
\DeclareMathOperator{\ppcm}{ppcm}
\DeclareMathOperator{\conj}{Conj}
\DeclareMathOperator{\End}{End}
\DeclareMathOperator{\rot}{rot}
\DeclareMathOperator{\trs}{trace}
\DeclareMathOperator{\Ind}{Ind}
\DeclareMathOperator{\mat}{Mat}
\DeclareMathOperator{\id}{Id}
\DeclareMathOperator{\vect}{vect}
\DeclareMathOperator{\img}{img}
\DeclareMathOperator{\cov}{Cov}
\DeclareMathOperator{\dist}{dist}
\DeclareMathOperator{\irr}{Irr}
\DeclareMathOperator{\image}{Im}
\DeclareMathOperator{\pd}{\partial}
\DeclareMathOperator{\epi}{epi}
\DeclareMathOperator{\Argmin}{Argmin}
\DeclareMathOperator{\dom}{dom}
\DeclareMathOperator{\proj}{proj}
\DeclareMathOperator{\ctg}{ctg}
\DeclareMathOperator{\supp}{supp}
\DeclareMathOperator{\argmin}{argmin}
\DeclareMathOperator{\mult}{mult}
\DeclareMathOperator{\ch}{ch}
\DeclareMathOperator{\sh}{sh}
\DeclareMathOperator{\rang}{rang}
\DeclareMathOperator{\diam}{diam}
\DeclareMathOperator{\Epigraphe}{Epigraphe}




\usepackage{xcolor}
\everymath{\color{blue}}
%\everymath{\color[rgb]{0,1,1}}
%\pagecolor[rgb]{0,0,0.5}


\newcommand*{\pdtest}[3][]{\ensuremath{\frac{\partial^{#1} #2}{\partial #3}}}

\newcommand*{\deffunc}[6][]{\ensuremath{
\begin{array}{rcl}
#2 : #3 &\rightarrow& #4\\
#5 &\mapsto& #6
\end{array}
}}

\newcommand{\eqcolon}{\mathrel{\resizebox{\widthof{$\mathord{=}$}}{\height}{ $\!\!=\!\!\resizebox{1.2\width}{0.8\height}{\raisebox{0.23ex}{$\mathop{:}$}}\!\!$ }}}
\newcommand{\coloneq}{\mathrel{\resizebox{\widthof{$\mathord{=}$}}{\height}{ $\!\!\resizebox{1.2\width}{0.8\height}{\raisebox{0.23ex}{$\mathop{:}$}}\!\!=\!\!$ }}}
\newcommand{\eqcolonl}{\ensuremath{\mathrel{=\!\!\mathop{:}}}}
\newcommand{\coloneql}{\ensuremath{\mathrel{\mathop{:} \!\! =}}}
\newcommand{\vc}[1]{% inline column vector
  \left(\begin{smallmatrix}#1\end{smallmatrix}\right)%
}
\newcommand{\vr}[1]{% inline row vector
  \begin{smallmatrix}(\,#1\,)\end{smallmatrix}%
}
\makeatletter
\newcommand*{\defeq}{\ =\mathrel{\rlap{%
                     \raisebox{0.3ex}{$\m@th\cdot$}}%
                     \raisebox{-0.3ex}{$\m@th\cdot$}}%
                     }
\makeatother

\newcommand{\mathcircle}[1]{% inline row vector
 \overset{\circ}{#1}
}
\newcommand{\ulim}{% low limit
 \underline{\lim}
}
\newcommand{\ssi}{% iff
\iff
}
\newcommand{\ps}[2]{
\expval{#1 | #2}
}
\newcommand{\df}[1]{
\mqty{#1}
}
\newcommand{\n}[1]{
\norm{#1}
}
\newcommand{\sys}[1]{
\left\{\smqty{#1}\right.
}


\newcommand{\eqdef}{\ensuremath{\overset{\text{def}}=}}


\def\Circlearrowright{\ensuremath{%
  \rotatebox[origin=c]{230}{$\circlearrowright$}}}

\newcommand\ct[1]{\text{\rmfamily\upshape #1}}
\newcommand\question[1]{ {\color{red} ...!? \small #1}}
\newcommand\caz[1]{\left\{\begin{array} #1 \end{array}\right.}
\newcommand\const{\text{\rmfamily\upshape const}}
\newcommand\toP{ \overset{\pro}{\to}}
\newcommand\toPP{ \overset{\text{PP}}{\to}}
\newcommand{\oeq}{\mathrel{\text{\textcircled{$=$}}}}





\usepackage{xcolor}
% \usepackage[normalem]{ulem}
\usepackage{lipsum}
\makeatletter
% \newcommand\colorwave[1][blue]{\bgroup \markoverwith{\lower3.5\p@\hbox{\sixly \textcolor{#1}{\char58}}}\ULon}
%\font\sixly=lasy6 % does not re-load if already loaded, so no memory problem.

\newmdtheoremenv[
linewidth= 1pt,linecolor= blue,%
leftmargin=20,rightmargin=20,innertopmargin=0pt, innerrightmargin=40,%
tikzsetting = { draw=lightgray, line width = 0.3pt,dashed,%
dash pattern = on 15pt off 3pt},%
splittopskip=\topskip,skipbelow=\baselineskip,%
skipabove=\baselineskip,ntheorem,roundcorner=0pt,
% backgroundcolor=pagebg,font=\color{orange}\sffamily, fontcolor=white
]{examplebox}{Exemple}[section]



\newcommand\R{\mathbb{R}}
\newcommand\Z{\mathbb{Z}}
\newcommand\N{\mathbb{N}}
\newcommand\E{\mathbb{E}}
\newcommand\F{\mathcal{F}}
\newcommand\cH{\mathcal{H}}
\newcommand\V{\mathbb{V}}
\newcommand\dmo{ ^{-1} }
\newcommand\kapa{\kappa}
\newcommand\im{Im}
\newcommand\hs{\mathcal{H}}





\usepackage{soul}

\makeatletter
\newcommand*{\whiten}[1]{\llap{\textcolor{white}{{\the\SOUL@token}}\hspace{#1pt}}}
\DeclareRobustCommand*\myul{%
    \def\SOUL@everyspace{\underline{\space}\kern\z@}%
    \def\SOUL@everytoken{%
     \setbox0=\hbox{\the\SOUL@token}%
     \ifdim\dp0>\z@
        \raisebox{\dp0}{\underline{\phantom{\the\SOUL@token}}}%
        \whiten{1}\whiten{0}%
        \whiten{-1}\whiten{-2}%
        \llap{\the\SOUL@token}%
     \else
        \underline{\the\SOUL@token}%
     \fi}%
\SOUL@}
\makeatother

\newcommand*{\demp}{\fontfamily{lmtt}\selectfont}

\DeclareTextFontCommand{\textdemp}{\demp}

\begin{document}

\ifcomment
Multiline
comment
\fi
\ifcomment
\myul{Typesetting test}
% \color[rgb]{1,1,1}
$∑_i^n≠ 60º±∞π∆¬≈√j∫h≤≥µ$

$\CR \R\pro\ind\pro\gS\pro
\mqty[a&b\\c&d]$
$\pro\mathbb{P}$
$\dd{x}$

  \[
    \alpha(x)=\left\{
                \begin{array}{ll}
                  x\\
                  \frac{1}{1+e^{-kx}}\\
                  \frac{e^x-e^{-x}}{e^x+e^{-x}}
                \end{array}
              \right.
  \]

  $\expval{x}$
  
  $\chi_\rho(ghg\dmo)=\Tr(\rho_{ghg\dmo})=\Tr(\rho_g\circ\rho_h\circ\rho\dmo_g)=\Tr(\rho_h)\overset{\mbox{\scalebox{0.5}{$\Tr(AB)=\Tr(BA)$}}}{=}\chi_\rho(h)$
  	$\mathop{\oplus}_{\substack{x\in X}}$

$\mat(\rho_g)=(a_{ij}(g))_{\scriptsize \substack{1\leq i\leq d \\ 1\leq j\leq d}}$ et $\mat(\rho'_g)=(a'_{ij}(g))_{\scriptsize \substack{1\leq i'\leq d' \\ 1\leq j'\leq d'}}$



\[\int_a^b{\mathbb{R}^2}g(u, v)\dd{P_{XY}}(u, v)=\iint g(u,v) f_{XY}(u, v)\dd \lambda(u) \dd \lambda(v)\]
$$\lim_{x\to\infty} f(x)$$	
$$\iiiint_V \mu(t,u,v,w) \,dt\,du\,dv\,dw$$
$$\sum_{n=1}^{\infty} 2^{-n} = 1$$	
\begin{definition}
	Si $X$ et $Y$ sont 2 v.a. ou definit la \textsc{Covariance} entre $X$ et $Y$ comme
	$\cov(X,Y)\overset{\text{def}}{=}\E\left[(X-\E(X))(Y-\E(Y))\right]=\E(XY)-\E(X)\E(Y)$.
\end{definition}
\fi
\pagebreak

% \tableofcontents

% insert your code here
%% !TEX encoding = UTF-8 Unicode
% !TEX TS-program = xelatex

\documentclass[french]{report}

%\usepackage[utf8]{inputenc}
%\usepackage[T1]{fontenc}
\usepackage{babel}


\newif\ifcomment
%\commenttrue # Show comments

\usepackage{physics}
\usepackage{amssymb}


\usepackage{amsthm}
% \usepackage{thmtools}
\usepackage{mathtools}
\usepackage{amsfonts}

\usepackage{color}

\usepackage{tikz}

\usepackage{geometry}
\geometry{a5paper, margin=0.1in, right=1cm}

\usepackage{dsfont}

\usepackage{graphicx}
\graphicspath{ {images/} }

\usepackage{faktor}

\usepackage{IEEEtrantools}
\usepackage{enumerate}   
\usepackage[PostScript=dvips]{"/Users/aware/Documents/Courses/diagrams"}


\newtheorem{theorem}{Théorème}[section]
\renewcommand{\thetheorem}{\arabic{theorem}}
\newtheorem{lemme}{Lemme}[section]
\renewcommand{\thelemme}{\arabic{lemme}}
\newtheorem{proposition}{Proposition}[section]
\renewcommand{\theproposition}{\arabic{proposition}}
\newtheorem{notations}{Notations}[section]
\newtheorem{problem}{Problème}[section]
\newtheorem{corollary}{Corollaire}[theorem]
\renewcommand{\thecorollary}{\arabic{corollary}}
\newtheorem{property}{Propriété}[section]
\newtheorem{objective}{Objectif}[section]

\theoremstyle{definition}
\newtheorem{definition}{Définition}[section]
\renewcommand{\thedefinition}{\arabic{definition}}
\newtheorem{exercise}{Exercice}[chapter]
\renewcommand{\theexercise}{\arabic{exercise}}
\newtheorem{example}{Exemple}[chapter]
\renewcommand{\theexample}{\arabic{example}}
\newtheorem*{solution}{Solution}
\newtheorem*{application}{Application}
\newtheorem*{notation}{Notation}
\newtheorem*{vocabulary}{Vocabulaire}
\newtheorem*{properties}{Propriétés}



\theoremstyle{remark}
\newtheorem*{remark}{Remarque}
\newtheorem*{rappel}{Rappel}


\usepackage{etoolbox}
\AtBeginEnvironment{exercise}{\small}
\AtBeginEnvironment{example}{\small}

\usepackage{cases}
\usepackage[red]{mypack}

\usepackage[framemethod=TikZ]{mdframed}

\definecolor{bg}{rgb}{0.4,0.25,0.95}
\definecolor{pagebg}{rgb}{0,0,0.5}
\surroundwithmdframed[
   topline=false,
   rightline=false,
   bottomline=false,
   leftmargin=\parindent,
   skipabove=8pt,
   skipbelow=8pt,
   linecolor=blue,
   innerbottommargin=10pt,
   % backgroundcolor=bg,font=\color{orange}\sffamily, fontcolor=white
]{definition}

\usepackage{empheq}
\usepackage[most]{tcolorbox}

\newtcbox{\mymath}[1][]{%
    nobeforeafter, math upper, tcbox raise base,
    enhanced, colframe=blue!30!black,
    colback=red!10, boxrule=1pt,
    #1}

\usepackage{unixode}


\DeclareMathOperator{\ord}{ord}
\DeclareMathOperator{\orb}{orb}
\DeclareMathOperator{\stab}{stab}
\DeclareMathOperator{\Stab}{stab}
\DeclareMathOperator{\ppcm}{ppcm}
\DeclareMathOperator{\conj}{Conj}
\DeclareMathOperator{\End}{End}
\DeclareMathOperator{\rot}{rot}
\DeclareMathOperator{\trs}{trace}
\DeclareMathOperator{\Ind}{Ind}
\DeclareMathOperator{\mat}{Mat}
\DeclareMathOperator{\id}{Id}
\DeclareMathOperator{\vect}{vect}
\DeclareMathOperator{\img}{img}
\DeclareMathOperator{\cov}{Cov}
\DeclareMathOperator{\dist}{dist}
\DeclareMathOperator{\irr}{Irr}
\DeclareMathOperator{\image}{Im}
\DeclareMathOperator{\pd}{\partial}
\DeclareMathOperator{\epi}{epi}
\DeclareMathOperator{\Argmin}{Argmin}
\DeclareMathOperator{\dom}{dom}
\DeclareMathOperator{\proj}{proj}
\DeclareMathOperator{\ctg}{ctg}
\DeclareMathOperator{\supp}{supp}
\DeclareMathOperator{\argmin}{argmin}
\DeclareMathOperator{\mult}{mult}
\DeclareMathOperator{\ch}{ch}
\DeclareMathOperator{\sh}{sh}
\DeclareMathOperator{\rang}{rang}
\DeclareMathOperator{\diam}{diam}
\DeclareMathOperator{\Epigraphe}{Epigraphe}




\usepackage{xcolor}
\everymath{\color{blue}}
%\everymath{\color[rgb]{0,1,1}}
%\pagecolor[rgb]{0,0,0.5}


\newcommand*{\pdtest}[3][]{\ensuremath{\frac{\partial^{#1} #2}{\partial #3}}}

\newcommand*{\deffunc}[6][]{\ensuremath{
\begin{array}{rcl}
#2 : #3 &\rightarrow& #4\\
#5 &\mapsto& #6
\end{array}
}}

\newcommand{\eqcolon}{\mathrel{\resizebox{\widthof{$\mathord{=}$}}{\height}{ $\!\!=\!\!\resizebox{1.2\width}{0.8\height}{\raisebox{0.23ex}{$\mathop{:}$}}\!\!$ }}}
\newcommand{\coloneq}{\mathrel{\resizebox{\widthof{$\mathord{=}$}}{\height}{ $\!\!\resizebox{1.2\width}{0.8\height}{\raisebox{0.23ex}{$\mathop{:}$}}\!\!=\!\!$ }}}
\newcommand{\eqcolonl}{\ensuremath{\mathrel{=\!\!\mathop{:}}}}
\newcommand{\coloneql}{\ensuremath{\mathrel{\mathop{:} \!\! =}}}
\newcommand{\vc}[1]{% inline column vector
  \left(\begin{smallmatrix}#1\end{smallmatrix}\right)%
}
\newcommand{\vr}[1]{% inline row vector
  \begin{smallmatrix}(\,#1\,)\end{smallmatrix}%
}
\makeatletter
\newcommand*{\defeq}{\ =\mathrel{\rlap{%
                     \raisebox{0.3ex}{$\m@th\cdot$}}%
                     \raisebox{-0.3ex}{$\m@th\cdot$}}%
                     }
\makeatother

\newcommand{\mathcircle}[1]{% inline row vector
 \overset{\circ}{#1}
}
\newcommand{\ulim}{% low limit
 \underline{\lim}
}
\newcommand{\ssi}{% iff
\iff
}
\newcommand{\ps}[2]{
\expval{#1 | #2}
}
\newcommand{\df}[1]{
\mqty{#1}
}
\newcommand{\n}[1]{
\norm{#1}
}
\newcommand{\sys}[1]{
\left\{\smqty{#1}\right.
}


\newcommand{\eqdef}{\ensuremath{\overset{\text{def}}=}}


\def\Circlearrowright{\ensuremath{%
  \rotatebox[origin=c]{230}{$\circlearrowright$}}}

\newcommand\ct[1]{\text{\rmfamily\upshape #1}}
\newcommand\question[1]{ {\color{red} ...!? \small #1}}
\newcommand\caz[1]{\left\{\begin{array} #1 \end{array}\right.}
\newcommand\const{\text{\rmfamily\upshape const}}
\newcommand\toP{ \overset{\pro}{\to}}
\newcommand\toPP{ \overset{\text{PP}}{\to}}
\newcommand{\oeq}{\mathrel{\text{\textcircled{$=$}}}}





\usepackage{xcolor}
% \usepackage[normalem]{ulem}
\usepackage{lipsum}
\makeatletter
% \newcommand\colorwave[1][blue]{\bgroup \markoverwith{\lower3.5\p@\hbox{\sixly \textcolor{#1}{\char58}}}\ULon}
%\font\sixly=lasy6 % does not re-load if already loaded, so no memory problem.

\newmdtheoremenv[
linewidth= 1pt,linecolor= blue,%
leftmargin=20,rightmargin=20,innertopmargin=0pt, innerrightmargin=40,%
tikzsetting = { draw=lightgray, line width = 0.3pt,dashed,%
dash pattern = on 15pt off 3pt},%
splittopskip=\topskip,skipbelow=\baselineskip,%
skipabove=\baselineskip,ntheorem,roundcorner=0pt,
% backgroundcolor=pagebg,font=\color{orange}\sffamily, fontcolor=white
]{examplebox}{Exemple}[section]



\newcommand\R{\mathbb{R}}
\newcommand\Z{\mathbb{Z}}
\newcommand\N{\mathbb{N}}
\newcommand\E{\mathbb{E}}
\newcommand\F{\mathcal{F}}
\newcommand\cH{\mathcal{H}}
\newcommand\V{\mathbb{V}}
\newcommand\dmo{ ^{-1} }
\newcommand\kapa{\kappa}
\newcommand\im{Im}
\newcommand\hs{\mathcal{H}}





\usepackage{soul}

\makeatletter
\newcommand*{\whiten}[1]{\llap{\textcolor{white}{{\the\SOUL@token}}\hspace{#1pt}}}
\DeclareRobustCommand*\myul{%
    \def\SOUL@everyspace{\underline{\space}\kern\z@}%
    \def\SOUL@everytoken{%
     \setbox0=\hbox{\the\SOUL@token}%
     \ifdim\dp0>\z@
        \raisebox{\dp0}{\underline{\phantom{\the\SOUL@token}}}%
        \whiten{1}\whiten{0}%
        \whiten{-1}\whiten{-2}%
        \llap{\the\SOUL@token}%
     \else
        \underline{\the\SOUL@token}%
     \fi}%
\SOUL@}
\makeatother

\newcommand*{\demp}{\fontfamily{lmtt}\selectfont}

\DeclareTextFontCommand{\textdemp}{\demp}

\begin{document}

\ifcomment
Multiline
comment
\fi
\ifcomment
\myul{Typesetting test}
% \color[rgb]{1,1,1}
$∑_i^n≠ 60º±∞π∆¬≈√j∫h≤≥µ$

$\CR \R\pro\ind\pro\gS\pro
\mqty[a&b\\c&d]$
$\pro\mathbb{P}$
$\dd{x}$

  \[
    \alpha(x)=\left\{
                \begin{array}{ll}
                  x\\
                  \frac{1}{1+e^{-kx}}\\
                  \frac{e^x-e^{-x}}{e^x+e^{-x}}
                \end{array}
              \right.
  \]

  $\expval{x}$
  
  $\chi_\rho(ghg\dmo)=\Tr(\rho_{ghg\dmo})=\Tr(\rho_g\circ\rho_h\circ\rho\dmo_g)=\Tr(\rho_h)\overset{\mbox{\scalebox{0.5}{$\Tr(AB)=\Tr(BA)$}}}{=}\chi_\rho(h)$
  	$\mathop{\oplus}_{\substack{x\in X}}$

$\mat(\rho_g)=(a_{ij}(g))_{\scriptsize \substack{1\leq i\leq d \\ 1\leq j\leq d}}$ et $\mat(\rho'_g)=(a'_{ij}(g))_{\scriptsize \substack{1\leq i'\leq d' \\ 1\leq j'\leq d'}}$



\[\int_a^b{\mathbb{R}^2}g(u, v)\dd{P_{XY}}(u, v)=\iint g(u,v) f_{XY}(u, v)\dd \lambda(u) \dd \lambda(v)\]
$$\lim_{x\to\infty} f(x)$$	
$$\iiiint_V \mu(t,u,v,w) \,dt\,du\,dv\,dw$$
$$\sum_{n=1}^{\infty} 2^{-n} = 1$$	
\begin{definition}
	Si $X$ et $Y$ sont 2 v.a. ou definit la \textsc{Covariance} entre $X$ et $Y$ comme
	$\cov(X,Y)\overset{\text{def}}{=}\E\left[(X-\E(X))(Y-\E(Y))\right]=\E(XY)-\E(X)\E(Y)$.
\end{definition}
\fi
\pagebreak

% \tableofcontents

% insert your code here
%\input{./algebra/main.tex}
%\input{./geometrie-differentielle/main.tex}
%\input{./probabilite/main.tex}
%\input{./analyse-fonctionnelle/main.tex}
% \input{./Analyse-convexe-et-dualite-en-optimisation/main.tex}
%\input{./tikz/main.tex}
%\input{./Theorie-du-distributions/main.tex}
%\input{./optimisation/mine.tex}
 \input{./modelisation/main.tex}

% yves.aubry@univ-tln.fr : algebra

\end{document}

%% !TEX encoding = UTF-8 Unicode
% !TEX TS-program = xelatex

\documentclass[french]{report}

%\usepackage[utf8]{inputenc}
%\usepackage[T1]{fontenc}
\usepackage{babel}


\newif\ifcomment
%\commenttrue # Show comments

\usepackage{physics}
\usepackage{amssymb}


\usepackage{amsthm}
% \usepackage{thmtools}
\usepackage{mathtools}
\usepackage{amsfonts}

\usepackage{color}

\usepackage{tikz}

\usepackage{geometry}
\geometry{a5paper, margin=0.1in, right=1cm}

\usepackage{dsfont}

\usepackage{graphicx}
\graphicspath{ {images/} }

\usepackage{faktor}

\usepackage{IEEEtrantools}
\usepackage{enumerate}   
\usepackage[PostScript=dvips]{"/Users/aware/Documents/Courses/diagrams"}


\newtheorem{theorem}{Théorème}[section]
\renewcommand{\thetheorem}{\arabic{theorem}}
\newtheorem{lemme}{Lemme}[section]
\renewcommand{\thelemme}{\arabic{lemme}}
\newtheorem{proposition}{Proposition}[section]
\renewcommand{\theproposition}{\arabic{proposition}}
\newtheorem{notations}{Notations}[section]
\newtheorem{problem}{Problème}[section]
\newtheorem{corollary}{Corollaire}[theorem]
\renewcommand{\thecorollary}{\arabic{corollary}}
\newtheorem{property}{Propriété}[section]
\newtheorem{objective}{Objectif}[section]

\theoremstyle{definition}
\newtheorem{definition}{Définition}[section]
\renewcommand{\thedefinition}{\arabic{definition}}
\newtheorem{exercise}{Exercice}[chapter]
\renewcommand{\theexercise}{\arabic{exercise}}
\newtheorem{example}{Exemple}[chapter]
\renewcommand{\theexample}{\arabic{example}}
\newtheorem*{solution}{Solution}
\newtheorem*{application}{Application}
\newtheorem*{notation}{Notation}
\newtheorem*{vocabulary}{Vocabulaire}
\newtheorem*{properties}{Propriétés}



\theoremstyle{remark}
\newtheorem*{remark}{Remarque}
\newtheorem*{rappel}{Rappel}


\usepackage{etoolbox}
\AtBeginEnvironment{exercise}{\small}
\AtBeginEnvironment{example}{\small}

\usepackage{cases}
\usepackage[red]{mypack}

\usepackage[framemethod=TikZ]{mdframed}

\definecolor{bg}{rgb}{0.4,0.25,0.95}
\definecolor{pagebg}{rgb}{0,0,0.5}
\surroundwithmdframed[
   topline=false,
   rightline=false,
   bottomline=false,
   leftmargin=\parindent,
   skipabove=8pt,
   skipbelow=8pt,
   linecolor=blue,
   innerbottommargin=10pt,
   % backgroundcolor=bg,font=\color{orange}\sffamily, fontcolor=white
]{definition}

\usepackage{empheq}
\usepackage[most]{tcolorbox}

\newtcbox{\mymath}[1][]{%
    nobeforeafter, math upper, tcbox raise base,
    enhanced, colframe=blue!30!black,
    colback=red!10, boxrule=1pt,
    #1}

\usepackage{unixode}


\DeclareMathOperator{\ord}{ord}
\DeclareMathOperator{\orb}{orb}
\DeclareMathOperator{\stab}{stab}
\DeclareMathOperator{\Stab}{stab}
\DeclareMathOperator{\ppcm}{ppcm}
\DeclareMathOperator{\conj}{Conj}
\DeclareMathOperator{\End}{End}
\DeclareMathOperator{\rot}{rot}
\DeclareMathOperator{\trs}{trace}
\DeclareMathOperator{\Ind}{Ind}
\DeclareMathOperator{\mat}{Mat}
\DeclareMathOperator{\id}{Id}
\DeclareMathOperator{\vect}{vect}
\DeclareMathOperator{\img}{img}
\DeclareMathOperator{\cov}{Cov}
\DeclareMathOperator{\dist}{dist}
\DeclareMathOperator{\irr}{Irr}
\DeclareMathOperator{\image}{Im}
\DeclareMathOperator{\pd}{\partial}
\DeclareMathOperator{\epi}{epi}
\DeclareMathOperator{\Argmin}{Argmin}
\DeclareMathOperator{\dom}{dom}
\DeclareMathOperator{\proj}{proj}
\DeclareMathOperator{\ctg}{ctg}
\DeclareMathOperator{\supp}{supp}
\DeclareMathOperator{\argmin}{argmin}
\DeclareMathOperator{\mult}{mult}
\DeclareMathOperator{\ch}{ch}
\DeclareMathOperator{\sh}{sh}
\DeclareMathOperator{\rang}{rang}
\DeclareMathOperator{\diam}{diam}
\DeclareMathOperator{\Epigraphe}{Epigraphe}




\usepackage{xcolor}
\everymath{\color{blue}}
%\everymath{\color[rgb]{0,1,1}}
%\pagecolor[rgb]{0,0,0.5}


\newcommand*{\pdtest}[3][]{\ensuremath{\frac{\partial^{#1} #2}{\partial #3}}}

\newcommand*{\deffunc}[6][]{\ensuremath{
\begin{array}{rcl}
#2 : #3 &\rightarrow& #4\\
#5 &\mapsto& #6
\end{array}
}}

\newcommand{\eqcolon}{\mathrel{\resizebox{\widthof{$\mathord{=}$}}{\height}{ $\!\!=\!\!\resizebox{1.2\width}{0.8\height}{\raisebox{0.23ex}{$\mathop{:}$}}\!\!$ }}}
\newcommand{\coloneq}{\mathrel{\resizebox{\widthof{$\mathord{=}$}}{\height}{ $\!\!\resizebox{1.2\width}{0.8\height}{\raisebox{0.23ex}{$\mathop{:}$}}\!\!=\!\!$ }}}
\newcommand{\eqcolonl}{\ensuremath{\mathrel{=\!\!\mathop{:}}}}
\newcommand{\coloneql}{\ensuremath{\mathrel{\mathop{:} \!\! =}}}
\newcommand{\vc}[1]{% inline column vector
  \left(\begin{smallmatrix}#1\end{smallmatrix}\right)%
}
\newcommand{\vr}[1]{% inline row vector
  \begin{smallmatrix}(\,#1\,)\end{smallmatrix}%
}
\makeatletter
\newcommand*{\defeq}{\ =\mathrel{\rlap{%
                     \raisebox{0.3ex}{$\m@th\cdot$}}%
                     \raisebox{-0.3ex}{$\m@th\cdot$}}%
                     }
\makeatother

\newcommand{\mathcircle}[1]{% inline row vector
 \overset{\circ}{#1}
}
\newcommand{\ulim}{% low limit
 \underline{\lim}
}
\newcommand{\ssi}{% iff
\iff
}
\newcommand{\ps}[2]{
\expval{#1 | #2}
}
\newcommand{\df}[1]{
\mqty{#1}
}
\newcommand{\n}[1]{
\norm{#1}
}
\newcommand{\sys}[1]{
\left\{\smqty{#1}\right.
}


\newcommand{\eqdef}{\ensuremath{\overset{\text{def}}=}}


\def\Circlearrowright{\ensuremath{%
  \rotatebox[origin=c]{230}{$\circlearrowright$}}}

\newcommand\ct[1]{\text{\rmfamily\upshape #1}}
\newcommand\question[1]{ {\color{red} ...!? \small #1}}
\newcommand\caz[1]{\left\{\begin{array} #1 \end{array}\right.}
\newcommand\const{\text{\rmfamily\upshape const}}
\newcommand\toP{ \overset{\pro}{\to}}
\newcommand\toPP{ \overset{\text{PP}}{\to}}
\newcommand{\oeq}{\mathrel{\text{\textcircled{$=$}}}}





\usepackage{xcolor}
% \usepackage[normalem]{ulem}
\usepackage{lipsum}
\makeatletter
% \newcommand\colorwave[1][blue]{\bgroup \markoverwith{\lower3.5\p@\hbox{\sixly \textcolor{#1}{\char58}}}\ULon}
%\font\sixly=lasy6 % does not re-load if already loaded, so no memory problem.

\newmdtheoremenv[
linewidth= 1pt,linecolor= blue,%
leftmargin=20,rightmargin=20,innertopmargin=0pt, innerrightmargin=40,%
tikzsetting = { draw=lightgray, line width = 0.3pt,dashed,%
dash pattern = on 15pt off 3pt},%
splittopskip=\topskip,skipbelow=\baselineskip,%
skipabove=\baselineskip,ntheorem,roundcorner=0pt,
% backgroundcolor=pagebg,font=\color{orange}\sffamily, fontcolor=white
]{examplebox}{Exemple}[section]



\newcommand\R{\mathbb{R}}
\newcommand\Z{\mathbb{Z}}
\newcommand\N{\mathbb{N}}
\newcommand\E{\mathbb{E}}
\newcommand\F{\mathcal{F}}
\newcommand\cH{\mathcal{H}}
\newcommand\V{\mathbb{V}}
\newcommand\dmo{ ^{-1} }
\newcommand\kapa{\kappa}
\newcommand\im{Im}
\newcommand\hs{\mathcal{H}}





\usepackage{soul}

\makeatletter
\newcommand*{\whiten}[1]{\llap{\textcolor{white}{{\the\SOUL@token}}\hspace{#1pt}}}
\DeclareRobustCommand*\myul{%
    \def\SOUL@everyspace{\underline{\space}\kern\z@}%
    \def\SOUL@everytoken{%
     \setbox0=\hbox{\the\SOUL@token}%
     \ifdim\dp0>\z@
        \raisebox{\dp0}{\underline{\phantom{\the\SOUL@token}}}%
        \whiten{1}\whiten{0}%
        \whiten{-1}\whiten{-2}%
        \llap{\the\SOUL@token}%
     \else
        \underline{\the\SOUL@token}%
     \fi}%
\SOUL@}
\makeatother

\newcommand*{\demp}{\fontfamily{lmtt}\selectfont}

\DeclareTextFontCommand{\textdemp}{\demp}

\begin{document}

\ifcomment
Multiline
comment
\fi
\ifcomment
\myul{Typesetting test}
% \color[rgb]{1,1,1}
$∑_i^n≠ 60º±∞π∆¬≈√j∫h≤≥µ$

$\CR \R\pro\ind\pro\gS\pro
\mqty[a&b\\c&d]$
$\pro\mathbb{P}$
$\dd{x}$

  \[
    \alpha(x)=\left\{
                \begin{array}{ll}
                  x\\
                  \frac{1}{1+e^{-kx}}\\
                  \frac{e^x-e^{-x}}{e^x+e^{-x}}
                \end{array}
              \right.
  \]

  $\expval{x}$
  
  $\chi_\rho(ghg\dmo)=\Tr(\rho_{ghg\dmo})=\Tr(\rho_g\circ\rho_h\circ\rho\dmo_g)=\Tr(\rho_h)\overset{\mbox{\scalebox{0.5}{$\Tr(AB)=\Tr(BA)$}}}{=}\chi_\rho(h)$
  	$\mathop{\oplus}_{\substack{x\in X}}$

$\mat(\rho_g)=(a_{ij}(g))_{\scriptsize \substack{1\leq i\leq d \\ 1\leq j\leq d}}$ et $\mat(\rho'_g)=(a'_{ij}(g))_{\scriptsize \substack{1\leq i'\leq d' \\ 1\leq j'\leq d'}}$



\[\int_a^b{\mathbb{R}^2}g(u, v)\dd{P_{XY}}(u, v)=\iint g(u,v) f_{XY}(u, v)\dd \lambda(u) \dd \lambda(v)\]
$$\lim_{x\to\infty} f(x)$$	
$$\iiiint_V \mu(t,u,v,w) \,dt\,du\,dv\,dw$$
$$\sum_{n=1}^{\infty} 2^{-n} = 1$$	
\begin{definition}
	Si $X$ et $Y$ sont 2 v.a. ou definit la \textsc{Covariance} entre $X$ et $Y$ comme
	$\cov(X,Y)\overset{\text{def}}{=}\E\left[(X-\E(X))(Y-\E(Y))\right]=\E(XY)-\E(X)\E(Y)$.
\end{definition}
\fi
\pagebreak

% \tableofcontents

% insert your code here
%\input{./algebra/main.tex}
%\input{./geometrie-differentielle/main.tex}
%\input{./probabilite/main.tex}
%\input{./analyse-fonctionnelle/main.tex}
% \input{./Analyse-convexe-et-dualite-en-optimisation/main.tex}
%\input{./tikz/main.tex}
%\input{./Theorie-du-distributions/main.tex}
%\input{./optimisation/mine.tex}
 \input{./modelisation/main.tex}

% yves.aubry@univ-tln.fr : algebra

\end{document}

%% !TEX encoding = UTF-8 Unicode
% !TEX TS-program = xelatex

\documentclass[french]{report}

%\usepackage[utf8]{inputenc}
%\usepackage[T1]{fontenc}
\usepackage{babel}


\newif\ifcomment
%\commenttrue # Show comments

\usepackage{physics}
\usepackage{amssymb}


\usepackage{amsthm}
% \usepackage{thmtools}
\usepackage{mathtools}
\usepackage{amsfonts}

\usepackage{color}

\usepackage{tikz}

\usepackage{geometry}
\geometry{a5paper, margin=0.1in, right=1cm}

\usepackage{dsfont}

\usepackage{graphicx}
\graphicspath{ {images/} }

\usepackage{faktor}

\usepackage{IEEEtrantools}
\usepackage{enumerate}   
\usepackage[PostScript=dvips]{"/Users/aware/Documents/Courses/diagrams"}


\newtheorem{theorem}{Théorème}[section]
\renewcommand{\thetheorem}{\arabic{theorem}}
\newtheorem{lemme}{Lemme}[section]
\renewcommand{\thelemme}{\arabic{lemme}}
\newtheorem{proposition}{Proposition}[section]
\renewcommand{\theproposition}{\arabic{proposition}}
\newtheorem{notations}{Notations}[section]
\newtheorem{problem}{Problème}[section]
\newtheorem{corollary}{Corollaire}[theorem]
\renewcommand{\thecorollary}{\arabic{corollary}}
\newtheorem{property}{Propriété}[section]
\newtheorem{objective}{Objectif}[section]

\theoremstyle{definition}
\newtheorem{definition}{Définition}[section]
\renewcommand{\thedefinition}{\arabic{definition}}
\newtheorem{exercise}{Exercice}[chapter]
\renewcommand{\theexercise}{\arabic{exercise}}
\newtheorem{example}{Exemple}[chapter]
\renewcommand{\theexample}{\arabic{example}}
\newtheorem*{solution}{Solution}
\newtheorem*{application}{Application}
\newtheorem*{notation}{Notation}
\newtheorem*{vocabulary}{Vocabulaire}
\newtheorem*{properties}{Propriétés}



\theoremstyle{remark}
\newtheorem*{remark}{Remarque}
\newtheorem*{rappel}{Rappel}


\usepackage{etoolbox}
\AtBeginEnvironment{exercise}{\small}
\AtBeginEnvironment{example}{\small}

\usepackage{cases}
\usepackage[red]{mypack}

\usepackage[framemethod=TikZ]{mdframed}

\definecolor{bg}{rgb}{0.4,0.25,0.95}
\definecolor{pagebg}{rgb}{0,0,0.5}
\surroundwithmdframed[
   topline=false,
   rightline=false,
   bottomline=false,
   leftmargin=\parindent,
   skipabove=8pt,
   skipbelow=8pt,
   linecolor=blue,
   innerbottommargin=10pt,
   % backgroundcolor=bg,font=\color{orange}\sffamily, fontcolor=white
]{definition}

\usepackage{empheq}
\usepackage[most]{tcolorbox}

\newtcbox{\mymath}[1][]{%
    nobeforeafter, math upper, tcbox raise base,
    enhanced, colframe=blue!30!black,
    colback=red!10, boxrule=1pt,
    #1}

\usepackage{unixode}


\DeclareMathOperator{\ord}{ord}
\DeclareMathOperator{\orb}{orb}
\DeclareMathOperator{\stab}{stab}
\DeclareMathOperator{\Stab}{stab}
\DeclareMathOperator{\ppcm}{ppcm}
\DeclareMathOperator{\conj}{Conj}
\DeclareMathOperator{\End}{End}
\DeclareMathOperator{\rot}{rot}
\DeclareMathOperator{\trs}{trace}
\DeclareMathOperator{\Ind}{Ind}
\DeclareMathOperator{\mat}{Mat}
\DeclareMathOperator{\id}{Id}
\DeclareMathOperator{\vect}{vect}
\DeclareMathOperator{\img}{img}
\DeclareMathOperator{\cov}{Cov}
\DeclareMathOperator{\dist}{dist}
\DeclareMathOperator{\irr}{Irr}
\DeclareMathOperator{\image}{Im}
\DeclareMathOperator{\pd}{\partial}
\DeclareMathOperator{\epi}{epi}
\DeclareMathOperator{\Argmin}{Argmin}
\DeclareMathOperator{\dom}{dom}
\DeclareMathOperator{\proj}{proj}
\DeclareMathOperator{\ctg}{ctg}
\DeclareMathOperator{\supp}{supp}
\DeclareMathOperator{\argmin}{argmin}
\DeclareMathOperator{\mult}{mult}
\DeclareMathOperator{\ch}{ch}
\DeclareMathOperator{\sh}{sh}
\DeclareMathOperator{\rang}{rang}
\DeclareMathOperator{\diam}{diam}
\DeclareMathOperator{\Epigraphe}{Epigraphe}




\usepackage{xcolor}
\everymath{\color{blue}}
%\everymath{\color[rgb]{0,1,1}}
%\pagecolor[rgb]{0,0,0.5}


\newcommand*{\pdtest}[3][]{\ensuremath{\frac{\partial^{#1} #2}{\partial #3}}}

\newcommand*{\deffunc}[6][]{\ensuremath{
\begin{array}{rcl}
#2 : #3 &\rightarrow& #4\\
#5 &\mapsto& #6
\end{array}
}}

\newcommand{\eqcolon}{\mathrel{\resizebox{\widthof{$\mathord{=}$}}{\height}{ $\!\!=\!\!\resizebox{1.2\width}{0.8\height}{\raisebox{0.23ex}{$\mathop{:}$}}\!\!$ }}}
\newcommand{\coloneq}{\mathrel{\resizebox{\widthof{$\mathord{=}$}}{\height}{ $\!\!\resizebox{1.2\width}{0.8\height}{\raisebox{0.23ex}{$\mathop{:}$}}\!\!=\!\!$ }}}
\newcommand{\eqcolonl}{\ensuremath{\mathrel{=\!\!\mathop{:}}}}
\newcommand{\coloneql}{\ensuremath{\mathrel{\mathop{:} \!\! =}}}
\newcommand{\vc}[1]{% inline column vector
  \left(\begin{smallmatrix}#1\end{smallmatrix}\right)%
}
\newcommand{\vr}[1]{% inline row vector
  \begin{smallmatrix}(\,#1\,)\end{smallmatrix}%
}
\makeatletter
\newcommand*{\defeq}{\ =\mathrel{\rlap{%
                     \raisebox{0.3ex}{$\m@th\cdot$}}%
                     \raisebox{-0.3ex}{$\m@th\cdot$}}%
                     }
\makeatother

\newcommand{\mathcircle}[1]{% inline row vector
 \overset{\circ}{#1}
}
\newcommand{\ulim}{% low limit
 \underline{\lim}
}
\newcommand{\ssi}{% iff
\iff
}
\newcommand{\ps}[2]{
\expval{#1 | #2}
}
\newcommand{\df}[1]{
\mqty{#1}
}
\newcommand{\n}[1]{
\norm{#1}
}
\newcommand{\sys}[1]{
\left\{\smqty{#1}\right.
}


\newcommand{\eqdef}{\ensuremath{\overset{\text{def}}=}}


\def\Circlearrowright{\ensuremath{%
  \rotatebox[origin=c]{230}{$\circlearrowright$}}}

\newcommand\ct[1]{\text{\rmfamily\upshape #1}}
\newcommand\question[1]{ {\color{red} ...!? \small #1}}
\newcommand\caz[1]{\left\{\begin{array} #1 \end{array}\right.}
\newcommand\const{\text{\rmfamily\upshape const}}
\newcommand\toP{ \overset{\pro}{\to}}
\newcommand\toPP{ \overset{\text{PP}}{\to}}
\newcommand{\oeq}{\mathrel{\text{\textcircled{$=$}}}}





\usepackage{xcolor}
% \usepackage[normalem]{ulem}
\usepackage{lipsum}
\makeatletter
% \newcommand\colorwave[1][blue]{\bgroup \markoverwith{\lower3.5\p@\hbox{\sixly \textcolor{#1}{\char58}}}\ULon}
%\font\sixly=lasy6 % does not re-load if already loaded, so no memory problem.

\newmdtheoremenv[
linewidth= 1pt,linecolor= blue,%
leftmargin=20,rightmargin=20,innertopmargin=0pt, innerrightmargin=40,%
tikzsetting = { draw=lightgray, line width = 0.3pt,dashed,%
dash pattern = on 15pt off 3pt},%
splittopskip=\topskip,skipbelow=\baselineskip,%
skipabove=\baselineskip,ntheorem,roundcorner=0pt,
% backgroundcolor=pagebg,font=\color{orange}\sffamily, fontcolor=white
]{examplebox}{Exemple}[section]



\newcommand\R{\mathbb{R}}
\newcommand\Z{\mathbb{Z}}
\newcommand\N{\mathbb{N}}
\newcommand\E{\mathbb{E}}
\newcommand\F{\mathcal{F}}
\newcommand\cH{\mathcal{H}}
\newcommand\V{\mathbb{V}}
\newcommand\dmo{ ^{-1} }
\newcommand\kapa{\kappa}
\newcommand\im{Im}
\newcommand\hs{\mathcal{H}}





\usepackage{soul}

\makeatletter
\newcommand*{\whiten}[1]{\llap{\textcolor{white}{{\the\SOUL@token}}\hspace{#1pt}}}
\DeclareRobustCommand*\myul{%
    \def\SOUL@everyspace{\underline{\space}\kern\z@}%
    \def\SOUL@everytoken{%
     \setbox0=\hbox{\the\SOUL@token}%
     \ifdim\dp0>\z@
        \raisebox{\dp0}{\underline{\phantom{\the\SOUL@token}}}%
        \whiten{1}\whiten{0}%
        \whiten{-1}\whiten{-2}%
        \llap{\the\SOUL@token}%
     \else
        \underline{\the\SOUL@token}%
     \fi}%
\SOUL@}
\makeatother

\newcommand*{\demp}{\fontfamily{lmtt}\selectfont}

\DeclareTextFontCommand{\textdemp}{\demp}

\begin{document}

\ifcomment
Multiline
comment
\fi
\ifcomment
\myul{Typesetting test}
% \color[rgb]{1,1,1}
$∑_i^n≠ 60º±∞π∆¬≈√j∫h≤≥µ$

$\CR \R\pro\ind\pro\gS\pro
\mqty[a&b\\c&d]$
$\pro\mathbb{P}$
$\dd{x}$

  \[
    \alpha(x)=\left\{
                \begin{array}{ll}
                  x\\
                  \frac{1}{1+e^{-kx}}\\
                  \frac{e^x-e^{-x}}{e^x+e^{-x}}
                \end{array}
              \right.
  \]

  $\expval{x}$
  
  $\chi_\rho(ghg\dmo)=\Tr(\rho_{ghg\dmo})=\Tr(\rho_g\circ\rho_h\circ\rho\dmo_g)=\Tr(\rho_h)\overset{\mbox{\scalebox{0.5}{$\Tr(AB)=\Tr(BA)$}}}{=}\chi_\rho(h)$
  	$\mathop{\oplus}_{\substack{x\in X}}$

$\mat(\rho_g)=(a_{ij}(g))_{\scriptsize \substack{1\leq i\leq d \\ 1\leq j\leq d}}$ et $\mat(\rho'_g)=(a'_{ij}(g))_{\scriptsize \substack{1\leq i'\leq d' \\ 1\leq j'\leq d'}}$



\[\int_a^b{\mathbb{R}^2}g(u, v)\dd{P_{XY}}(u, v)=\iint g(u,v) f_{XY}(u, v)\dd \lambda(u) \dd \lambda(v)\]
$$\lim_{x\to\infty} f(x)$$	
$$\iiiint_V \mu(t,u,v,w) \,dt\,du\,dv\,dw$$
$$\sum_{n=1}^{\infty} 2^{-n} = 1$$	
\begin{definition}
	Si $X$ et $Y$ sont 2 v.a. ou definit la \textsc{Covariance} entre $X$ et $Y$ comme
	$\cov(X,Y)\overset{\text{def}}{=}\E\left[(X-\E(X))(Y-\E(Y))\right]=\E(XY)-\E(X)\E(Y)$.
\end{definition}
\fi
\pagebreak

% \tableofcontents

% insert your code here
%\input{./algebra/main.tex}
%\input{./geometrie-differentielle/main.tex}
%\input{./probabilite/main.tex}
%\input{./analyse-fonctionnelle/main.tex}
% \input{./Analyse-convexe-et-dualite-en-optimisation/main.tex}
%\input{./tikz/main.tex}
%\input{./Theorie-du-distributions/main.tex}
%\input{./optimisation/mine.tex}
 \input{./modelisation/main.tex}

% yves.aubry@univ-tln.fr : algebra

\end{document}

%% !TEX encoding = UTF-8 Unicode
% !TEX TS-program = xelatex

\documentclass[french]{report}

%\usepackage[utf8]{inputenc}
%\usepackage[T1]{fontenc}
\usepackage{babel}


\newif\ifcomment
%\commenttrue # Show comments

\usepackage{physics}
\usepackage{amssymb}


\usepackage{amsthm}
% \usepackage{thmtools}
\usepackage{mathtools}
\usepackage{amsfonts}

\usepackage{color}

\usepackage{tikz}

\usepackage{geometry}
\geometry{a5paper, margin=0.1in, right=1cm}

\usepackage{dsfont}

\usepackage{graphicx}
\graphicspath{ {images/} }

\usepackage{faktor}

\usepackage{IEEEtrantools}
\usepackage{enumerate}   
\usepackage[PostScript=dvips]{"/Users/aware/Documents/Courses/diagrams"}


\newtheorem{theorem}{Théorème}[section]
\renewcommand{\thetheorem}{\arabic{theorem}}
\newtheorem{lemme}{Lemme}[section]
\renewcommand{\thelemme}{\arabic{lemme}}
\newtheorem{proposition}{Proposition}[section]
\renewcommand{\theproposition}{\arabic{proposition}}
\newtheorem{notations}{Notations}[section]
\newtheorem{problem}{Problème}[section]
\newtheorem{corollary}{Corollaire}[theorem]
\renewcommand{\thecorollary}{\arabic{corollary}}
\newtheorem{property}{Propriété}[section]
\newtheorem{objective}{Objectif}[section]

\theoremstyle{definition}
\newtheorem{definition}{Définition}[section]
\renewcommand{\thedefinition}{\arabic{definition}}
\newtheorem{exercise}{Exercice}[chapter]
\renewcommand{\theexercise}{\arabic{exercise}}
\newtheorem{example}{Exemple}[chapter]
\renewcommand{\theexample}{\arabic{example}}
\newtheorem*{solution}{Solution}
\newtheorem*{application}{Application}
\newtheorem*{notation}{Notation}
\newtheorem*{vocabulary}{Vocabulaire}
\newtheorem*{properties}{Propriétés}



\theoremstyle{remark}
\newtheorem*{remark}{Remarque}
\newtheorem*{rappel}{Rappel}


\usepackage{etoolbox}
\AtBeginEnvironment{exercise}{\small}
\AtBeginEnvironment{example}{\small}

\usepackage{cases}
\usepackage[red]{mypack}

\usepackage[framemethod=TikZ]{mdframed}

\definecolor{bg}{rgb}{0.4,0.25,0.95}
\definecolor{pagebg}{rgb}{0,0,0.5}
\surroundwithmdframed[
   topline=false,
   rightline=false,
   bottomline=false,
   leftmargin=\parindent,
   skipabove=8pt,
   skipbelow=8pt,
   linecolor=blue,
   innerbottommargin=10pt,
   % backgroundcolor=bg,font=\color{orange}\sffamily, fontcolor=white
]{definition}

\usepackage{empheq}
\usepackage[most]{tcolorbox}

\newtcbox{\mymath}[1][]{%
    nobeforeafter, math upper, tcbox raise base,
    enhanced, colframe=blue!30!black,
    colback=red!10, boxrule=1pt,
    #1}

\usepackage{unixode}


\DeclareMathOperator{\ord}{ord}
\DeclareMathOperator{\orb}{orb}
\DeclareMathOperator{\stab}{stab}
\DeclareMathOperator{\Stab}{stab}
\DeclareMathOperator{\ppcm}{ppcm}
\DeclareMathOperator{\conj}{Conj}
\DeclareMathOperator{\End}{End}
\DeclareMathOperator{\rot}{rot}
\DeclareMathOperator{\trs}{trace}
\DeclareMathOperator{\Ind}{Ind}
\DeclareMathOperator{\mat}{Mat}
\DeclareMathOperator{\id}{Id}
\DeclareMathOperator{\vect}{vect}
\DeclareMathOperator{\img}{img}
\DeclareMathOperator{\cov}{Cov}
\DeclareMathOperator{\dist}{dist}
\DeclareMathOperator{\irr}{Irr}
\DeclareMathOperator{\image}{Im}
\DeclareMathOperator{\pd}{\partial}
\DeclareMathOperator{\epi}{epi}
\DeclareMathOperator{\Argmin}{Argmin}
\DeclareMathOperator{\dom}{dom}
\DeclareMathOperator{\proj}{proj}
\DeclareMathOperator{\ctg}{ctg}
\DeclareMathOperator{\supp}{supp}
\DeclareMathOperator{\argmin}{argmin}
\DeclareMathOperator{\mult}{mult}
\DeclareMathOperator{\ch}{ch}
\DeclareMathOperator{\sh}{sh}
\DeclareMathOperator{\rang}{rang}
\DeclareMathOperator{\diam}{diam}
\DeclareMathOperator{\Epigraphe}{Epigraphe}




\usepackage{xcolor}
\everymath{\color{blue}}
%\everymath{\color[rgb]{0,1,1}}
%\pagecolor[rgb]{0,0,0.5}


\newcommand*{\pdtest}[3][]{\ensuremath{\frac{\partial^{#1} #2}{\partial #3}}}

\newcommand*{\deffunc}[6][]{\ensuremath{
\begin{array}{rcl}
#2 : #3 &\rightarrow& #4\\
#5 &\mapsto& #6
\end{array}
}}

\newcommand{\eqcolon}{\mathrel{\resizebox{\widthof{$\mathord{=}$}}{\height}{ $\!\!=\!\!\resizebox{1.2\width}{0.8\height}{\raisebox{0.23ex}{$\mathop{:}$}}\!\!$ }}}
\newcommand{\coloneq}{\mathrel{\resizebox{\widthof{$\mathord{=}$}}{\height}{ $\!\!\resizebox{1.2\width}{0.8\height}{\raisebox{0.23ex}{$\mathop{:}$}}\!\!=\!\!$ }}}
\newcommand{\eqcolonl}{\ensuremath{\mathrel{=\!\!\mathop{:}}}}
\newcommand{\coloneql}{\ensuremath{\mathrel{\mathop{:} \!\! =}}}
\newcommand{\vc}[1]{% inline column vector
  \left(\begin{smallmatrix}#1\end{smallmatrix}\right)%
}
\newcommand{\vr}[1]{% inline row vector
  \begin{smallmatrix}(\,#1\,)\end{smallmatrix}%
}
\makeatletter
\newcommand*{\defeq}{\ =\mathrel{\rlap{%
                     \raisebox{0.3ex}{$\m@th\cdot$}}%
                     \raisebox{-0.3ex}{$\m@th\cdot$}}%
                     }
\makeatother

\newcommand{\mathcircle}[1]{% inline row vector
 \overset{\circ}{#1}
}
\newcommand{\ulim}{% low limit
 \underline{\lim}
}
\newcommand{\ssi}{% iff
\iff
}
\newcommand{\ps}[2]{
\expval{#1 | #2}
}
\newcommand{\df}[1]{
\mqty{#1}
}
\newcommand{\n}[1]{
\norm{#1}
}
\newcommand{\sys}[1]{
\left\{\smqty{#1}\right.
}


\newcommand{\eqdef}{\ensuremath{\overset{\text{def}}=}}


\def\Circlearrowright{\ensuremath{%
  \rotatebox[origin=c]{230}{$\circlearrowright$}}}

\newcommand\ct[1]{\text{\rmfamily\upshape #1}}
\newcommand\question[1]{ {\color{red} ...!? \small #1}}
\newcommand\caz[1]{\left\{\begin{array} #1 \end{array}\right.}
\newcommand\const{\text{\rmfamily\upshape const}}
\newcommand\toP{ \overset{\pro}{\to}}
\newcommand\toPP{ \overset{\text{PP}}{\to}}
\newcommand{\oeq}{\mathrel{\text{\textcircled{$=$}}}}





\usepackage{xcolor}
% \usepackage[normalem]{ulem}
\usepackage{lipsum}
\makeatletter
% \newcommand\colorwave[1][blue]{\bgroup \markoverwith{\lower3.5\p@\hbox{\sixly \textcolor{#1}{\char58}}}\ULon}
%\font\sixly=lasy6 % does not re-load if already loaded, so no memory problem.

\newmdtheoremenv[
linewidth= 1pt,linecolor= blue,%
leftmargin=20,rightmargin=20,innertopmargin=0pt, innerrightmargin=40,%
tikzsetting = { draw=lightgray, line width = 0.3pt,dashed,%
dash pattern = on 15pt off 3pt},%
splittopskip=\topskip,skipbelow=\baselineskip,%
skipabove=\baselineskip,ntheorem,roundcorner=0pt,
% backgroundcolor=pagebg,font=\color{orange}\sffamily, fontcolor=white
]{examplebox}{Exemple}[section]



\newcommand\R{\mathbb{R}}
\newcommand\Z{\mathbb{Z}}
\newcommand\N{\mathbb{N}}
\newcommand\E{\mathbb{E}}
\newcommand\F{\mathcal{F}}
\newcommand\cH{\mathcal{H}}
\newcommand\V{\mathbb{V}}
\newcommand\dmo{ ^{-1} }
\newcommand\kapa{\kappa}
\newcommand\im{Im}
\newcommand\hs{\mathcal{H}}





\usepackage{soul}

\makeatletter
\newcommand*{\whiten}[1]{\llap{\textcolor{white}{{\the\SOUL@token}}\hspace{#1pt}}}
\DeclareRobustCommand*\myul{%
    \def\SOUL@everyspace{\underline{\space}\kern\z@}%
    \def\SOUL@everytoken{%
     \setbox0=\hbox{\the\SOUL@token}%
     \ifdim\dp0>\z@
        \raisebox{\dp0}{\underline{\phantom{\the\SOUL@token}}}%
        \whiten{1}\whiten{0}%
        \whiten{-1}\whiten{-2}%
        \llap{\the\SOUL@token}%
     \else
        \underline{\the\SOUL@token}%
     \fi}%
\SOUL@}
\makeatother

\newcommand*{\demp}{\fontfamily{lmtt}\selectfont}

\DeclareTextFontCommand{\textdemp}{\demp}

\begin{document}

\ifcomment
Multiline
comment
\fi
\ifcomment
\myul{Typesetting test}
% \color[rgb]{1,1,1}
$∑_i^n≠ 60º±∞π∆¬≈√j∫h≤≥µ$

$\CR \R\pro\ind\pro\gS\pro
\mqty[a&b\\c&d]$
$\pro\mathbb{P}$
$\dd{x}$

  \[
    \alpha(x)=\left\{
                \begin{array}{ll}
                  x\\
                  \frac{1}{1+e^{-kx}}\\
                  \frac{e^x-e^{-x}}{e^x+e^{-x}}
                \end{array}
              \right.
  \]

  $\expval{x}$
  
  $\chi_\rho(ghg\dmo)=\Tr(\rho_{ghg\dmo})=\Tr(\rho_g\circ\rho_h\circ\rho\dmo_g)=\Tr(\rho_h)\overset{\mbox{\scalebox{0.5}{$\Tr(AB)=\Tr(BA)$}}}{=}\chi_\rho(h)$
  	$\mathop{\oplus}_{\substack{x\in X}}$

$\mat(\rho_g)=(a_{ij}(g))_{\scriptsize \substack{1\leq i\leq d \\ 1\leq j\leq d}}$ et $\mat(\rho'_g)=(a'_{ij}(g))_{\scriptsize \substack{1\leq i'\leq d' \\ 1\leq j'\leq d'}}$



\[\int_a^b{\mathbb{R}^2}g(u, v)\dd{P_{XY}}(u, v)=\iint g(u,v) f_{XY}(u, v)\dd \lambda(u) \dd \lambda(v)\]
$$\lim_{x\to\infty} f(x)$$	
$$\iiiint_V \mu(t,u,v,w) \,dt\,du\,dv\,dw$$
$$\sum_{n=1}^{\infty} 2^{-n} = 1$$	
\begin{definition}
	Si $X$ et $Y$ sont 2 v.a. ou definit la \textsc{Covariance} entre $X$ et $Y$ comme
	$\cov(X,Y)\overset{\text{def}}{=}\E\left[(X-\E(X))(Y-\E(Y))\right]=\E(XY)-\E(X)\E(Y)$.
\end{definition}
\fi
\pagebreak

% \tableofcontents

% insert your code here
%\input{./algebra/main.tex}
%\input{./geometrie-differentielle/main.tex}
%\input{./probabilite/main.tex}
%\input{./analyse-fonctionnelle/main.tex}
% \input{./Analyse-convexe-et-dualite-en-optimisation/main.tex}
%\input{./tikz/main.tex}
%\input{./Theorie-du-distributions/main.tex}
%\input{./optimisation/mine.tex}
 \input{./modelisation/main.tex}

% yves.aubry@univ-tln.fr : algebra

\end{document}

% % !TEX encoding = UTF-8 Unicode
% !TEX TS-program = xelatex

\documentclass[french]{report}

%\usepackage[utf8]{inputenc}
%\usepackage[T1]{fontenc}
\usepackage{babel}


\newif\ifcomment
%\commenttrue # Show comments

\usepackage{physics}
\usepackage{amssymb}


\usepackage{amsthm}
% \usepackage{thmtools}
\usepackage{mathtools}
\usepackage{amsfonts}

\usepackage{color}

\usepackage{tikz}

\usepackage{geometry}
\geometry{a5paper, margin=0.1in, right=1cm}

\usepackage{dsfont}

\usepackage{graphicx}
\graphicspath{ {images/} }

\usepackage{faktor}

\usepackage{IEEEtrantools}
\usepackage{enumerate}   
\usepackage[PostScript=dvips]{"/Users/aware/Documents/Courses/diagrams"}


\newtheorem{theorem}{Théorème}[section]
\renewcommand{\thetheorem}{\arabic{theorem}}
\newtheorem{lemme}{Lemme}[section]
\renewcommand{\thelemme}{\arabic{lemme}}
\newtheorem{proposition}{Proposition}[section]
\renewcommand{\theproposition}{\arabic{proposition}}
\newtheorem{notations}{Notations}[section]
\newtheorem{problem}{Problème}[section]
\newtheorem{corollary}{Corollaire}[theorem]
\renewcommand{\thecorollary}{\arabic{corollary}}
\newtheorem{property}{Propriété}[section]
\newtheorem{objective}{Objectif}[section]

\theoremstyle{definition}
\newtheorem{definition}{Définition}[section]
\renewcommand{\thedefinition}{\arabic{definition}}
\newtheorem{exercise}{Exercice}[chapter]
\renewcommand{\theexercise}{\arabic{exercise}}
\newtheorem{example}{Exemple}[chapter]
\renewcommand{\theexample}{\arabic{example}}
\newtheorem*{solution}{Solution}
\newtheorem*{application}{Application}
\newtheorem*{notation}{Notation}
\newtheorem*{vocabulary}{Vocabulaire}
\newtheorem*{properties}{Propriétés}



\theoremstyle{remark}
\newtheorem*{remark}{Remarque}
\newtheorem*{rappel}{Rappel}


\usepackage{etoolbox}
\AtBeginEnvironment{exercise}{\small}
\AtBeginEnvironment{example}{\small}

\usepackage{cases}
\usepackage[red]{mypack}

\usepackage[framemethod=TikZ]{mdframed}

\definecolor{bg}{rgb}{0.4,0.25,0.95}
\definecolor{pagebg}{rgb}{0,0,0.5}
\surroundwithmdframed[
   topline=false,
   rightline=false,
   bottomline=false,
   leftmargin=\parindent,
   skipabove=8pt,
   skipbelow=8pt,
   linecolor=blue,
   innerbottommargin=10pt,
   % backgroundcolor=bg,font=\color{orange}\sffamily, fontcolor=white
]{definition}

\usepackage{empheq}
\usepackage[most]{tcolorbox}

\newtcbox{\mymath}[1][]{%
    nobeforeafter, math upper, tcbox raise base,
    enhanced, colframe=blue!30!black,
    colback=red!10, boxrule=1pt,
    #1}

\usepackage{unixode}


\DeclareMathOperator{\ord}{ord}
\DeclareMathOperator{\orb}{orb}
\DeclareMathOperator{\stab}{stab}
\DeclareMathOperator{\Stab}{stab}
\DeclareMathOperator{\ppcm}{ppcm}
\DeclareMathOperator{\conj}{Conj}
\DeclareMathOperator{\End}{End}
\DeclareMathOperator{\rot}{rot}
\DeclareMathOperator{\trs}{trace}
\DeclareMathOperator{\Ind}{Ind}
\DeclareMathOperator{\mat}{Mat}
\DeclareMathOperator{\id}{Id}
\DeclareMathOperator{\vect}{vect}
\DeclareMathOperator{\img}{img}
\DeclareMathOperator{\cov}{Cov}
\DeclareMathOperator{\dist}{dist}
\DeclareMathOperator{\irr}{Irr}
\DeclareMathOperator{\image}{Im}
\DeclareMathOperator{\pd}{\partial}
\DeclareMathOperator{\epi}{epi}
\DeclareMathOperator{\Argmin}{Argmin}
\DeclareMathOperator{\dom}{dom}
\DeclareMathOperator{\proj}{proj}
\DeclareMathOperator{\ctg}{ctg}
\DeclareMathOperator{\supp}{supp}
\DeclareMathOperator{\argmin}{argmin}
\DeclareMathOperator{\mult}{mult}
\DeclareMathOperator{\ch}{ch}
\DeclareMathOperator{\sh}{sh}
\DeclareMathOperator{\rang}{rang}
\DeclareMathOperator{\diam}{diam}
\DeclareMathOperator{\Epigraphe}{Epigraphe}




\usepackage{xcolor}
\everymath{\color{blue}}
%\everymath{\color[rgb]{0,1,1}}
%\pagecolor[rgb]{0,0,0.5}


\newcommand*{\pdtest}[3][]{\ensuremath{\frac{\partial^{#1} #2}{\partial #3}}}

\newcommand*{\deffunc}[6][]{\ensuremath{
\begin{array}{rcl}
#2 : #3 &\rightarrow& #4\\
#5 &\mapsto& #6
\end{array}
}}

\newcommand{\eqcolon}{\mathrel{\resizebox{\widthof{$\mathord{=}$}}{\height}{ $\!\!=\!\!\resizebox{1.2\width}{0.8\height}{\raisebox{0.23ex}{$\mathop{:}$}}\!\!$ }}}
\newcommand{\coloneq}{\mathrel{\resizebox{\widthof{$\mathord{=}$}}{\height}{ $\!\!\resizebox{1.2\width}{0.8\height}{\raisebox{0.23ex}{$\mathop{:}$}}\!\!=\!\!$ }}}
\newcommand{\eqcolonl}{\ensuremath{\mathrel{=\!\!\mathop{:}}}}
\newcommand{\coloneql}{\ensuremath{\mathrel{\mathop{:} \!\! =}}}
\newcommand{\vc}[1]{% inline column vector
  \left(\begin{smallmatrix}#1\end{smallmatrix}\right)%
}
\newcommand{\vr}[1]{% inline row vector
  \begin{smallmatrix}(\,#1\,)\end{smallmatrix}%
}
\makeatletter
\newcommand*{\defeq}{\ =\mathrel{\rlap{%
                     \raisebox{0.3ex}{$\m@th\cdot$}}%
                     \raisebox{-0.3ex}{$\m@th\cdot$}}%
                     }
\makeatother

\newcommand{\mathcircle}[1]{% inline row vector
 \overset{\circ}{#1}
}
\newcommand{\ulim}{% low limit
 \underline{\lim}
}
\newcommand{\ssi}{% iff
\iff
}
\newcommand{\ps}[2]{
\expval{#1 | #2}
}
\newcommand{\df}[1]{
\mqty{#1}
}
\newcommand{\n}[1]{
\norm{#1}
}
\newcommand{\sys}[1]{
\left\{\smqty{#1}\right.
}


\newcommand{\eqdef}{\ensuremath{\overset{\text{def}}=}}


\def\Circlearrowright{\ensuremath{%
  \rotatebox[origin=c]{230}{$\circlearrowright$}}}

\newcommand\ct[1]{\text{\rmfamily\upshape #1}}
\newcommand\question[1]{ {\color{red} ...!? \small #1}}
\newcommand\caz[1]{\left\{\begin{array} #1 \end{array}\right.}
\newcommand\const{\text{\rmfamily\upshape const}}
\newcommand\toP{ \overset{\pro}{\to}}
\newcommand\toPP{ \overset{\text{PP}}{\to}}
\newcommand{\oeq}{\mathrel{\text{\textcircled{$=$}}}}





\usepackage{xcolor}
% \usepackage[normalem]{ulem}
\usepackage{lipsum}
\makeatletter
% \newcommand\colorwave[1][blue]{\bgroup \markoverwith{\lower3.5\p@\hbox{\sixly \textcolor{#1}{\char58}}}\ULon}
%\font\sixly=lasy6 % does not re-load if already loaded, so no memory problem.

\newmdtheoremenv[
linewidth= 1pt,linecolor= blue,%
leftmargin=20,rightmargin=20,innertopmargin=0pt, innerrightmargin=40,%
tikzsetting = { draw=lightgray, line width = 0.3pt,dashed,%
dash pattern = on 15pt off 3pt},%
splittopskip=\topskip,skipbelow=\baselineskip,%
skipabove=\baselineskip,ntheorem,roundcorner=0pt,
% backgroundcolor=pagebg,font=\color{orange}\sffamily, fontcolor=white
]{examplebox}{Exemple}[section]



\newcommand\R{\mathbb{R}}
\newcommand\Z{\mathbb{Z}}
\newcommand\N{\mathbb{N}}
\newcommand\E{\mathbb{E}}
\newcommand\F{\mathcal{F}}
\newcommand\cH{\mathcal{H}}
\newcommand\V{\mathbb{V}}
\newcommand\dmo{ ^{-1} }
\newcommand\kapa{\kappa}
\newcommand\im{Im}
\newcommand\hs{\mathcal{H}}





\usepackage{soul}

\makeatletter
\newcommand*{\whiten}[1]{\llap{\textcolor{white}{{\the\SOUL@token}}\hspace{#1pt}}}
\DeclareRobustCommand*\myul{%
    \def\SOUL@everyspace{\underline{\space}\kern\z@}%
    \def\SOUL@everytoken{%
     \setbox0=\hbox{\the\SOUL@token}%
     \ifdim\dp0>\z@
        \raisebox{\dp0}{\underline{\phantom{\the\SOUL@token}}}%
        \whiten{1}\whiten{0}%
        \whiten{-1}\whiten{-2}%
        \llap{\the\SOUL@token}%
     \else
        \underline{\the\SOUL@token}%
     \fi}%
\SOUL@}
\makeatother

\newcommand*{\demp}{\fontfamily{lmtt}\selectfont}

\DeclareTextFontCommand{\textdemp}{\demp}

\begin{document}

\ifcomment
Multiline
comment
\fi
\ifcomment
\myul{Typesetting test}
% \color[rgb]{1,1,1}
$∑_i^n≠ 60º±∞π∆¬≈√j∫h≤≥µ$

$\CR \R\pro\ind\pro\gS\pro
\mqty[a&b\\c&d]$
$\pro\mathbb{P}$
$\dd{x}$

  \[
    \alpha(x)=\left\{
                \begin{array}{ll}
                  x\\
                  \frac{1}{1+e^{-kx}}\\
                  \frac{e^x-e^{-x}}{e^x+e^{-x}}
                \end{array}
              \right.
  \]

  $\expval{x}$
  
  $\chi_\rho(ghg\dmo)=\Tr(\rho_{ghg\dmo})=\Tr(\rho_g\circ\rho_h\circ\rho\dmo_g)=\Tr(\rho_h)\overset{\mbox{\scalebox{0.5}{$\Tr(AB)=\Tr(BA)$}}}{=}\chi_\rho(h)$
  	$\mathop{\oplus}_{\substack{x\in X}}$

$\mat(\rho_g)=(a_{ij}(g))_{\scriptsize \substack{1\leq i\leq d \\ 1\leq j\leq d}}$ et $\mat(\rho'_g)=(a'_{ij}(g))_{\scriptsize \substack{1\leq i'\leq d' \\ 1\leq j'\leq d'}}$



\[\int_a^b{\mathbb{R}^2}g(u, v)\dd{P_{XY}}(u, v)=\iint g(u,v) f_{XY}(u, v)\dd \lambda(u) \dd \lambda(v)\]
$$\lim_{x\to\infty} f(x)$$	
$$\iiiint_V \mu(t,u,v,w) \,dt\,du\,dv\,dw$$
$$\sum_{n=1}^{\infty} 2^{-n} = 1$$	
\begin{definition}
	Si $X$ et $Y$ sont 2 v.a. ou definit la \textsc{Covariance} entre $X$ et $Y$ comme
	$\cov(X,Y)\overset{\text{def}}{=}\E\left[(X-\E(X))(Y-\E(Y))\right]=\E(XY)-\E(X)\E(Y)$.
\end{definition}
\fi
\pagebreak

% \tableofcontents

% insert your code here
%\input{./algebra/main.tex}
%\input{./geometrie-differentielle/main.tex}
%\input{./probabilite/main.tex}
%\input{./analyse-fonctionnelle/main.tex}
% \input{./Analyse-convexe-et-dualite-en-optimisation/main.tex}
%\input{./tikz/main.tex}
%\input{./Theorie-du-distributions/main.tex}
%\input{./optimisation/mine.tex}
 \input{./modelisation/main.tex}

% yves.aubry@univ-tln.fr : algebra

\end{document}

%% !TEX encoding = UTF-8 Unicode
% !TEX TS-program = xelatex

\documentclass[french]{report}

%\usepackage[utf8]{inputenc}
%\usepackage[T1]{fontenc}
\usepackage{babel}


\newif\ifcomment
%\commenttrue # Show comments

\usepackage{physics}
\usepackage{amssymb}


\usepackage{amsthm}
% \usepackage{thmtools}
\usepackage{mathtools}
\usepackage{amsfonts}

\usepackage{color}

\usepackage{tikz}

\usepackage{geometry}
\geometry{a5paper, margin=0.1in, right=1cm}

\usepackage{dsfont}

\usepackage{graphicx}
\graphicspath{ {images/} }

\usepackage{faktor}

\usepackage{IEEEtrantools}
\usepackage{enumerate}   
\usepackage[PostScript=dvips]{"/Users/aware/Documents/Courses/diagrams"}


\newtheorem{theorem}{Théorème}[section]
\renewcommand{\thetheorem}{\arabic{theorem}}
\newtheorem{lemme}{Lemme}[section]
\renewcommand{\thelemme}{\arabic{lemme}}
\newtheorem{proposition}{Proposition}[section]
\renewcommand{\theproposition}{\arabic{proposition}}
\newtheorem{notations}{Notations}[section]
\newtheorem{problem}{Problème}[section]
\newtheorem{corollary}{Corollaire}[theorem]
\renewcommand{\thecorollary}{\arabic{corollary}}
\newtheorem{property}{Propriété}[section]
\newtheorem{objective}{Objectif}[section]

\theoremstyle{definition}
\newtheorem{definition}{Définition}[section]
\renewcommand{\thedefinition}{\arabic{definition}}
\newtheorem{exercise}{Exercice}[chapter]
\renewcommand{\theexercise}{\arabic{exercise}}
\newtheorem{example}{Exemple}[chapter]
\renewcommand{\theexample}{\arabic{example}}
\newtheorem*{solution}{Solution}
\newtheorem*{application}{Application}
\newtheorem*{notation}{Notation}
\newtheorem*{vocabulary}{Vocabulaire}
\newtheorem*{properties}{Propriétés}



\theoremstyle{remark}
\newtheorem*{remark}{Remarque}
\newtheorem*{rappel}{Rappel}


\usepackage{etoolbox}
\AtBeginEnvironment{exercise}{\small}
\AtBeginEnvironment{example}{\small}

\usepackage{cases}
\usepackage[red]{mypack}

\usepackage[framemethod=TikZ]{mdframed}

\definecolor{bg}{rgb}{0.4,0.25,0.95}
\definecolor{pagebg}{rgb}{0,0,0.5}
\surroundwithmdframed[
   topline=false,
   rightline=false,
   bottomline=false,
   leftmargin=\parindent,
   skipabove=8pt,
   skipbelow=8pt,
   linecolor=blue,
   innerbottommargin=10pt,
   % backgroundcolor=bg,font=\color{orange}\sffamily, fontcolor=white
]{definition}

\usepackage{empheq}
\usepackage[most]{tcolorbox}

\newtcbox{\mymath}[1][]{%
    nobeforeafter, math upper, tcbox raise base,
    enhanced, colframe=blue!30!black,
    colback=red!10, boxrule=1pt,
    #1}

\usepackage{unixode}


\DeclareMathOperator{\ord}{ord}
\DeclareMathOperator{\orb}{orb}
\DeclareMathOperator{\stab}{stab}
\DeclareMathOperator{\Stab}{stab}
\DeclareMathOperator{\ppcm}{ppcm}
\DeclareMathOperator{\conj}{Conj}
\DeclareMathOperator{\End}{End}
\DeclareMathOperator{\rot}{rot}
\DeclareMathOperator{\trs}{trace}
\DeclareMathOperator{\Ind}{Ind}
\DeclareMathOperator{\mat}{Mat}
\DeclareMathOperator{\id}{Id}
\DeclareMathOperator{\vect}{vect}
\DeclareMathOperator{\img}{img}
\DeclareMathOperator{\cov}{Cov}
\DeclareMathOperator{\dist}{dist}
\DeclareMathOperator{\irr}{Irr}
\DeclareMathOperator{\image}{Im}
\DeclareMathOperator{\pd}{\partial}
\DeclareMathOperator{\epi}{epi}
\DeclareMathOperator{\Argmin}{Argmin}
\DeclareMathOperator{\dom}{dom}
\DeclareMathOperator{\proj}{proj}
\DeclareMathOperator{\ctg}{ctg}
\DeclareMathOperator{\supp}{supp}
\DeclareMathOperator{\argmin}{argmin}
\DeclareMathOperator{\mult}{mult}
\DeclareMathOperator{\ch}{ch}
\DeclareMathOperator{\sh}{sh}
\DeclareMathOperator{\rang}{rang}
\DeclareMathOperator{\diam}{diam}
\DeclareMathOperator{\Epigraphe}{Epigraphe}




\usepackage{xcolor}
\everymath{\color{blue}}
%\everymath{\color[rgb]{0,1,1}}
%\pagecolor[rgb]{0,0,0.5}


\newcommand*{\pdtest}[3][]{\ensuremath{\frac{\partial^{#1} #2}{\partial #3}}}

\newcommand*{\deffunc}[6][]{\ensuremath{
\begin{array}{rcl}
#2 : #3 &\rightarrow& #4\\
#5 &\mapsto& #6
\end{array}
}}

\newcommand{\eqcolon}{\mathrel{\resizebox{\widthof{$\mathord{=}$}}{\height}{ $\!\!=\!\!\resizebox{1.2\width}{0.8\height}{\raisebox{0.23ex}{$\mathop{:}$}}\!\!$ }}}
\newcommand{\coloneq}{\mathrel{\resizebox{\widthof{$\mathord{=}$}}{\height}{ $\!\!\resizebox{1.2\width}{0.8\height}{\raisebox{0.23ex}{$\mathop{:}$}}\!\!=\!\!$ }}}
\newcommand{\eqcolonl}{\ensuremath{\mathrel{=\!\!\mathop{:}}}}
\newcommand{\coloneql}{\ensuremath{\mathrel{\mathop{:} \!\! =}}}
\newcommand{\vc}[1]{% inline column vector
  \left(\begin{smallmatrix}#1\end{smallmatrix}\right)%
}
\newcommand{\vr}[1]{% inline row vector
  \begin{smallmatrix}(\,#1\,)\end{smallmatrix}%
}
\makeatletter
\newcommand*{\defeq}{\ =\mathrel{\rlap{%
                     \raisebox{0.3ex}{$\m@th\cdot$}}%
                     \raisebox{-0.3ex}{$\m@th\cdot$}}%
                     }
\makeatother

\newcommand{\mathcircle}[1]{% inline row vector
 \overset{\circ}{#1}
}
\newcommand{\ulim}{% low limit
 \underline{\lim}
}
\newcommand{\ssi}{% iff
\iff
}
\newcommand{\ps}[2]{
\expval{#1 | #2}
}
\newcommand{\df}[1]{
\mqty{#1}
}
\newcommand{\n}[1]{
\norm{#1}
}
\newcommand{\sys}[1]{
\left\{\smqty{#1}\right.
}


\newcommand{\eqdef}{\ensuremath{\overset{\text{def}}=}}


\def\Circlearrowright{\ensuremath{%
  \rotatebox[origin=c]{230}{$\circlearrowright$}}}

\newcommand\ct[1]{\text{\rmfamily\upshape #1}}
\newcommand\question[1]{ {\color{red} ...!? \small #1}}
\newcommand\caz[1]{\left\{\begin{array} #1 \end{array}\right.}
\newcommand\const{\text{\rmfamily\upshape const}}
\newcommand\toP{ \overset{\pro}{\to}}
\newcommand\toPP{ \overset{\text{PP}}{\to}}
\newcommand{\oeq}{\mathrel{\text{\textcircled{$=$}}}}





\usepackage{xcolor}
% \usepackage[normalem]{ulem}
\usepackage{lipsum}
\makeatletter
% \newcommand\colorwave[1][blue]{\bgroup \markoverwith{\lower3.5\p@\hbox{\sixly \textcolor{#1}{\char58}}}\ULon}
%\font\sixly=lasy6 % does not re-load if already loaded, so no memory problem.

\newmdtheoremenv[
linewidth= 1pt,linecolor= blue,%
leftmargin=20,rightmargin=20,innertopmargin=0pt, innerrightmargin=40,%
tikzsetting = { draw=lightgray, line width = 0.3pt,dashed,%
dash pattern = on 15pt off 3pt},%
splittopskip=\topskip,skipbelow=\baselineskip,%
skipabove=\baselineskip,ntheorem,roundcorner=0pt,
% backgroundcolor=pagebg,font=\color{orange}\sffamily, fontcolor=white
]{examplebox}{Exemple}[section]



\newcommand\R{\mathbb{R}}
\newcommand\Z{\mathbb{Z}}
\newcommand\N{\mathbb{N}}
\newcommand\E{\mathbb{E}}
\newcommand\F{\mathcal{F}}
\newcommand\cH{\mathcal{H}}
\newcommand\V{\mathbb{V}}
\newcommand\dmo{ ^{-1} }
\newcommand\kapa{\kappa}
\newcommand\im{Im}
\newcommand\hs{\mathcal{H}}





\usepackage{soul}

\makeatletter
\newcommand*{\whiten}[1]{\llap{\textcolor{white}{{\the\SOUL@token}}\hspace{#1pt}}}
\DeclareRobustCommand*\myul{%
    \def\SOUL@everyspace{\underline{\space}\kern\z@}%
    \def\SOUL@everytoken{%
     \setbox0=\hbox{\the\SOUL@token}%
     \ifdim\dp0>\z@
        \raisebox{\dp0}{\underline{\phantom{\the\SOUL@token}}}%
        \whiten{1}\whiten{0}%
        \whiten{-1}\whiten{-2}%
        \llap{\the\SOUL@token}%
     \else
        \underline{\the\SOUL@token}%
     \fi}%
\SOUL@}
\makeatother

\newcommand*{\demp}{\fontfamily{lmtt}\selectfont}

\DeclareTextFontCommand{\textdemp}{\demp}

\begin{document}

\ifcomment
Multiline
comment
\fi
\ifcomment
\myul{Typesetting test}
% \color[rgb]{1,1,1}
$∑_i^n≠ 60º±∞π∆¬≈√j∫h≤≥µ$

$\CR \R\pro\ind\pro\gS\pro
\mqty[a&b\\c&d]$
$\pro\mathbb{P}$
$\dd{x}$

  \[
    \alpha(x)=\left\{
                \begin{array}{ll}
                  x\\
                  \frac{1}{1+e^{-kx}}\\
                  \frac{e^x-e^{-x}}{e^x+e^{-x}}
                \end{array}
              \right.
  \]

  $\expval{x}$
  
  $\chi_\rho(ghg\dmo)=\Tr(\rho_{ghg\dmo})=\Tr(\rho_g\circ\rho_h\circ\rho\dmo_g)=\Tr(\rho_h)\overset{\mbox{\scalebox{0.5}{$\Tr(AB)=\Tr(BA)$}}}{=}\chi_\rho(h)$
  	$\mathop{\oplus}_{\substack{x\in X}}$

$\mat(\rho_g)=(a_{ij}(g))_{\scriptsize \substack{1\leq i\leq d \\ 1\leq j\leq d}}$ et $\mat(\rho'_g)=(a'_{ij}(g))_{\scriptsize \substack{1\leq i'\leq d' \\ 1\leq j'\leq d'}}$



\[\int_a^b{\mathbb{R}^2}g(u, v)\dd{P_{XY}}(u, v)=\iint g(u,v) f_{XY}(u, v)\dd \lambda(u) \dd \lambda(v)\]
$$\lim_{x\to\infty} f(x)$$	
$$\iiiint_V \mu(t,u,v,w) \,dt\,du\,dv\,dw$$
$$\sum_{n=1}^{\infty} 2^{-n} = 1$$	
\begin{definition}
	Si $X$ et $Y$ sont 2 v.a. ou definit la \textsc{Covariance} entre $X$ et $Y$ comme
	$\cov(X,Y)\overset{\text{def}}{=}\E\left[(X-\E(X))(Y-\E(Y))\right]=\E(XY)-\E(X)\E(Y)$.
\end{definition}
\fi
\pagebreak

% \tableofcontents

% insert your code here
%\input{./algebra/main.tex}
%\input{./geometrie-differentielle/main.tex}
%\input{./probabilite/main.tex}
%\input{./analyse-fonctionnelle/main.tex}
% \input{./Analyse-convexe-et-dualite-en-optimisation/main.tex}
%\input{./tikz/main.tex}
%\input{./Theorie-du-distributions/main.tex}
%\input{./optimisation/mine.tex}
 \input{./modelisation/main.tex}

% yves.aubry@univ-tln.fr : algebra

\end{document}

%% !TEX encoding = UTF-8 Unicode
% !TEX TS-program = xelatex

\documentclass[french]{report}

%\usepackage[utf8]{inputenc}
%\usepackage[T1]{fontenc}
\usepackage{babel}


\newif\ifcomment
%\commenttrue # Show comments

\usepackage{physics}
\usepackage{amssymb}


\usepackage{amsthm}
% \usepackage{thmtools}
\usepackage{mathtools}
\usepackage{amsfonts}

\usepackage{color}

\usepackage{tikz}

\usepackage{geometry}
\geometry{a5paper, margin=0.1in, right=1cm}

\usepackage{dsfont}

\usepackage{graphicx}
\graphicspath{ {images/} }

\usepackage{faktor}

\usepackage{IEEEtrantools}
\usepackage{enumerate}   
\usepackage[PostScript=dvips]{"/Users/aware/Documents/Courses/diagrams"}


\newtheorem{theorem}{Théorème}[section]
\renewcommand{\thetheorem}{\arabic{theorem}}
\newtheorem{lemme}{Lemme}[section]
\renewcommand{\thelemme}{\arabic{lemme}}
\newtheorem{proposition}{Proposition}[section]
\renewcommand{\theproposition}{\arabic{proposition}}
\newtheorem{notations}{Notations}[section]
\newtheorem{problem}{Problème}[section]
\newtheorem{corollary}{Corollaire}[theorem]
\renewcommand{\thecorollary}{\arabic{corollary}}
\newtheorem{property}{Propriété}[section]
\newtheorem{objective}{Objectif}[section]

\theoremstyle{definition}
\newtheorem{definition}{Définition}[section]
\renewcommand{\thedefinition}{\arabic{definition}}
\newtheorem{exercise}{Exercice}[chapter]
\renewcommand{\theexercise}{\arabic{exercise}}
\newtheorem{example}{Exemple}[chapter]
\renewcommand{\theexample}{\arabic{example}}
\newtheorem*{solution}{Solution}
\newtheorem*{application}{Application}
\newtheorem*{notation}{Notation}
\newtheorem*{vocabulary}{Vocabulaire}
\newtheorem*{properties}{Propriétés}



\theoremstyle{remark}
\newtheorem*{remark}{Remarque}
\newtheorem*{rappel}{Rappel}


\usepackage{etoolbox}
\AtBeginEnvironment{exercise}{\small}
\AtBeginEnvironment{example}{\small}

\usepackage{cases}
\usepackage[red]{mypack}

\usepackage[framemethod=TikZ]{mdframed}

\definecolor{bg}{rgb}{0.4,0.25,0.95}
\definecolor{pagebg}{rgb}{0,0,0.5}
\surroundwithmdframed[
   topline=false,
   rightline=false,
   bottomline=false,
   leftmargin=\parindent,
   skipabove=8pt,
   skipbelow=8pt,
   linecolor=blue,
   innerbottommargin=10pt,
   % backgroundcolor=bg,font=\color{orange}\sffamily, fontcolor=white
]{definition}

\usepackage{empheq}
\usepackage[most]{tcolorbox}

\newtcbox{\mymath}[1][]{%
    nobeforeafter, math upper, tcbox raise base,
    enhanced, colframe=blue!30!black,
    colback=red!10, boxrule=1pt,
    #1}

\usepackage{unixode}


\DeclareMathOperator{\ord}{ord}
\DeclareMathOperator{\orb}{orb}
\DeclareMathOperator{\stab}{stab}
\DeclareMathOperator{\Stab}{stab}
\DeclareMathOperator{\ppcm}{ppcm}
\DeclareMathOperator{\conj}{Conj}
\DeclareMathOperator{\End}{End}
\DeclareMathOperator{\rot}{rot}
\DeclareMathOperator{\trs}{trace}
\DeclareMathOperator{\Ind}{Ind}
\DeclareMathOperator{\mat}{Mat}
\DeclareMathOperator{\id}{Id}
\DeclareMathOperator{\vect}{vect}
\DeclareMathOperator{\img}{img}
\DeclareMathOperator{\cov}{Cov}
\DeclareMathOperator{\dist}{dist}
\DeclareMathOperator{\irr}{Irr}
\DeclareMathOperator{\image}{Im}
\DeclareMathOperator{\pd}{\partial}
\DeclareMathOperator{\epi}{epi}
\DeclareMathOperator{\Argmin}{Argmin}
\DeclareMathOperator{\dom}{dom}
\DeclareMathOperator{\proj}{proj}
\DeclareMathOperator{\ctg}{ctg}
\DeclareMathOperator{\supp}{supp}
\DeclareMathOperator{\argmin}{argmin}
\DeclareMathOperator{\mult}{mult}
\DeclareMathOperator{\ch}{ch}
\DeclareMathOperator{\sh}{sh}
\DeclareMathOperator{\rang}{rang}
\DeclareMathOperator{\diam}{diam}
\DeclareMathOperator{\Epigraphe}{Epigraphe}




\usepackage{xcolor}
\everymath{\color{blue}}
%\everymath{\color[rgb]{0,1,1}}
%\pagecolor[rgb]{0,0,0.5}


\newcommand*{\pdtest}[3][]{\ensuremath{\frac{\partial^{#1} #2}{\partial #3}}}

\newcommand*{\deffunc}[6][]{\ensuremath{
\begin{array}{rcl}
#2 : #3 &\rightarrow& #4\\
#5 &\mapsto& #6
\end{array}
}}

\newcommand{\eqcolon}{\mathrel{\resizebox{\widthof{$\mathord{=}$}}{\height}{ $\!\!=\!\!\resizebox{1.2\width}{0.8\height}{\raisebox{0.23ex}{$\mathop{:}$}}\!\!$ }}}
\newcommand{\coloneq}{\mathrel{\resizebox{\widthof{$\mathord{=}$}}{\height}{ $\!\!\resizebox{1.2\width}{0.8\height}{\raisebox{0.23ex}{$\mathop{:}$}}\!\!=\!\!$ }}}
\newcommand{\eqcolonl}{\ensuremath{\mathrel{=\!\!\mathop{:}}}}
\newcommand{\coloneql}{\ensuremath{\mathrel{\mathop{:} \!\! =}}}
\newcommand{\vc}[1]{% inline column vector
  \left(\begin{smallmatrix}#1\end{smallmatrix}\right)%
}
\newcommand{\vr}[1]{% inline row vector
  \begin{smallmatrix}(\,#1\,)\end{smallmatrix}%
}
\makeatletter
\newcommand*{\defeq}{\ =\mathrel{\rlap{%
                     \raisebox{0.3ex}{$\m@th\cdot$}}%
                     \raisebox{-0.3ex}{$\m@th\cdot$}}%
                     }
\makeatother

\newcommand{\mathcircle}[1]{% inline row vector
 \overset{\circ}{#1}
}
\newcommand{\ulim}{% low limit
 \underline{\lim}
}
\newcommand{\ssi}{% iff
\iff
}
\newcommand{\ps}[2]{
\expval{#1 | #2}
}
\newcommand{\df}[1]{
\mqty{#1}
}
\newcommand{\n}[1]{
\norm{#1}
}
\newcommand{\sys}[1]{
\left\{\smqty{#1}\right.
}


\newcommand{\eqdef}{\ensuremath{\overset{\text{def}}=}}


\def\Circlearrowright{\ensuremath{%
  \rotatebox[origin=c]{230}{$\circlearrowright$}}}

\newcommand\ct[1]{\text{\rmfamily\upshape #1}}
\newcommand\question[1]{ {\color{red} ...!? \small #1}}
\newcommand\caz[1]{\left\{\begin{array} #1 \end{array}\right.}
\newcommand\const{\text{\rmfamily\upshape const}}
\newcommand\toP{ \overset{\pro}{\to}}
\newcommand\toPP{ \overset{\text{PP}}{\to}}
\newcommand{\oeq}{\mathrel{\text{\textcircled{$=$}}}}





\usepackage{xcolor}
% \usepackage[normalem]{ulem}
\usepackage{lipsum}
\makeatletter
% \newcommand\colorwave[1][blue]{\bgroup \markoverwith{\lower3.5\p@\hbox{\sixly \textcolor{#1}{\char58}}}\ULon}
%\font\sixly=lasy6 % does not re-load if already loaded, so no memory problem.

\newmdtheoremenv[
linewidth= 1pt,linecolor= blue,%
leftmargin=20,rightmargin=20,innertopmargin=0pt, innerrightmargin=40,%
tikzsetting = { draw=lightgray, line width = 0.3pt,dashed,%
dash pattern = on 15pt off 3pt},%
splittopskip=\topskip,skipbelow=\baselineskip,%
skipabove=\baselineskip,ntheorem,roundcorner=0pt,
% backgroundcolor=pagebg,font=\color{orange}\sffamily, fontcolor=white
]{examplebox}{Exemple}[section]



\newcommand\R{\mathbb{R}}
\newcommand\Z{\mathbb{Z}}
\newcommand\N{\mathbb{N}}
\newcommand\E{\mathbb{E}}
\newcommand\F{\mathcal{F}}
\newcommand\cH{\mathcal{H}}
\newcommand\V{\mathbb{V}}
\newcommand\dmo{ ^{-1} }
\newcommand\kapa{\kappa}
\newcommand\im{Im}
\newcommand\hs{\mathcal{H}}





\usepackage{soul}

\makeatletter
\newcommand*{\whiten}[1]{\llap{\textcolor{white}{{\the\SOUL@token}}\hspace{#1pt}}}
\DeclareRobustCommand*\myul{%
    \def\SOUL@everyspace{\underline{\space}\kern\z@}%
    \def\SOUL@everytoken{%
     \setbox0=\hbox{\the\SOUL@token}%
     \ifdim\dp0>\z@
        \raisebox{\dp0}{\underline{\phantom{\the\SOUL@token}}}%
        \whiten{1}\whiten{0}%
        \whiten{-1}\whiten{-2}%
        \llap{\the\SOUL@token}%
     \else
        \underline{\the\SOUL@token}%
     \fi}%
\SOUL@}
\makeatother

\newcommand*{\demp}{\fontfamily{lmtt}\selectfont}

\DeclareTextFontCommand{\textdemp}{\demp}

\begin{document}

\ifcomment
Multiline
comment
\fi
\ifcomment
\myul{Typesetting test}
% \color[rgb]{1,1,1}
$∑_i^n≠ 60º±∞π∆¬≈√j∫h≤≥µ$

$\CR \R\pro\ind\pro\gS\pro
\mqty[a&b\\c&d]$
$\pro\mathbb{P}$
$\dd{x}$

  \[
    \alpha(x)=\left\{
                \begin{array}{ll}
                  x\\
                  \frac{1}{1+e^{-kx}}\\
                  \frac{e^x-e^{-x}}{e^x+e^{-x}}
                \end{array}
              \right.
  \]

  $\expval{x}$
  
  $\chi_\rho(ghg\dmo)=\Tr(\rho_{ghg\dmo})=\Tr(\rho_g\circ\rho_h\circ\rho\dmo_g)=\Tr(\rho_h)\overset{\mbox{\scalebox{0.5}{$\Tr(AB)=\Tr(BA)$}}}{=}\chi_\rho(h)$
  	$\mathop{\oplus}_{\substack{x\in X}}$

$\mat(\rho_g)=(a_{ij}(g))_{\scriptsize \substack{1\leq i\leq d \\ 1\leq j\leq d}}$ et $\mat(\rho'_g)=(a'_{ij}(g))_{\scriptsize \substack{1\leq i'\leq d' \\ 1\leq j'\leq d'}}$



\[\int_a^b{\mathbb{R}^2}g(u, v)\dd{P_{XY}}(u, v)=\iint g(u,v) f_{XY}(u, v)\dd \lambda(u) \dd \lambda(v)\]
$$\lim_{x\to\infty} f(x)$$	
$$\iiiint_V \mu(t,u,v,w) \,dt\,du\,dv\,dw$$
$$\sum_{n=1}^{\infty} 2^{-n} = 1$$	
\begin{definition}
	Si $X$ et $Y$ sont 2 v.a. ou definit la \textsc{Covariance} entre $X$ et $Y$ comme
	$\cov(X,Y)\overset{\text{def}}{=}\E\left[(X-\E(X))(Y-\E(Y))\right]=\E(XY)-\E(X)\E(Y)$.
\end{definition}
\fi
\pagebreak

% \tableofcontents

% insert your code here
%\input{./algebra/main.tex}
%\input{./geometrie-differentielle/main.tex}
%\input{./probabilite/main.tex}
%\input{./analyse-fonctionnelle/main.tex}
% \input{./Analyse-convexe-et-dualite-en-optimisation/main.tex}
%\input{./tikz/main.tex}
%\input{./Theorie-du-distributions/main.tex}
%\input{./optimisation/mine.tex}
 \input{./modelisation/main.tex}

% yves.aubry@univ-tln.fr : algebra

\end{document}

%\input{./optimisation/mine.tex}
 % !TEX encoding = UTF-8 Unicode
% !TEX TS-program = xelatex

\documentclass[french]{report}

%\usepackage[utf8]{inputenc}
%\usepackage[T1]{fontenc}
\usepackage{babel}


\newif\ifcomment
%\commenttrue # Show comments

\usepackage{physics}
\usepackage{amssymb}


\usepackage{amsthm}
% \usepackage{thmtools}
\usepackage{mathtools}
\usepackage{amsfonts}

\usepackage{color}

\usepackage{tikz}

\usepackage{geometry}
\geometry{a5paper, margin=0.1in, right=1cm}

\usepackage{dsfont}

\usepackage{graphicx}
\graphicspath{ {images/} }

\usepackage{faktor}

\usepackage{IEEEtrantools}
\usepackage{enumerate}   
\usepackage[PostScript=dvips]{"/Users/aware/Documents/Courses/diagrams"}


\newtheorem{theorem}{Théorème}[section]
\renewcommand{\thetheorem}{\arabic{theorem}}
\newtheorem{lemme}{Lemme}[section]
\renewcommand{\thelemme}{\arabic{lemme}}
\newtheorem{proposition}{Proposition}[section]
\renewcommand{\theproposition}{\arabic{proposition}}
\newtheorem{notations}{Notations}[section]
\newtheorem{problem}{Problème}[section]
\newtheorem{corollary}{Corollaire}[theorem]
\renewcommand{\thecorollary}{\arabic{corollary}}
\newtheorem{property}{Propriété}[section]
\newtheorem{objective}{Objectif}[section]

\theoremstyle{definition}
\newtheorem{definition}{Définition}[section]
\renewcommand{\thedefinition}{\arabic{definition}}
\newtheorem{exercise}{Exercice}[chapter]
\renewcommand{\theexercise}{\arabic{exercise}}
\newtheorem{example}{Exemple}[chapter]
\renewcommand{\theexample}{\arabic{example}}
\newtheorem*{solution}{Solution}
\newtheorem*{application}{Application}
\newtheorem*{notation}{Notation}
\newtheorem*{vocabulary}{Vocabulaire}
\newtheorem*{properties}{Propriétés}



\theoremstyle{remark}
\newtheorem*{remark}{Remarque}
\newtheorem*{rappel}{Rappel}


\usepackage{etoolbox}
\AtBeginEnvironment{exercise}{\small}
\AtBeginEnvironment{example}{\small}

\usepackage{cases}
\usepackage[red]{mypack}

\usepackage[framemethod=TikZ]{mdframed}

\definecolor{bg}{rgb}{0.4,0.25,0.95}
\definecolor{pagebg}{rgb}{0,0,0.5}
\surroundwithmdframed[
   topline=false,
   rightline=false,
   bottomline=false,
   leftmargin=\parindent,
   skipabove=8pt,
   skipbelow=8pt,
   linecolor=blue,
   innerbottommargin=10pt,
   % backgroundcolor=bg,font=\color{orange}\sffamily, fontcolor=white
]{definition}

\usepackage{empheq}
\usepackage[most]{tcolorbox}

\newtcbox{\mymath}[1][]{%
    nobeforeafter, math upper, tcbox raise base,
    enhanced, colframe=blue!30!black,
    colback=red!10, boxrule=1pt,
    #1}

\usepackage{unixode}


\DeclareMathOperator{\ord}{ord}
\DeclareMathOperator{\orb}{orb}
\DeclareMathOperator{\stab}{stab}
\DeclareMathOperator{\Stab}{stab}
\DeclareMathOperator{\ppcm}{ppcm}
\DeclareMathOperator{\conj}{Conj}
\DeclareMathOperator{\End}{End}
\DeclareMathOperator{\rot}{rot}
\DeclareMathOperator{\trs}{trace}
\DeclareMathOperator{\Ind}{Ind}
\DeclareMathOperator{\mat}{Mat}
\DeclareMathOperator{\id}{Id}
\DeclareMathOperator{\vect}{vect}
\DeclareMathOperator{\img}{img}
\DeclareMathOperator{\cov}{Cov}
\DeclareMathOperator{\dist}{dist}
\DeclareMathOperator{\irr}{Irr}
\DeclareMathOperator{\image}{Im}
\DeclareMathOperator{\pd}{\partial}
\DeclareMathOperator{\epi}{epi}
\DeclareMathOperator{\Argmin}{Argmin}
\DeclareMathOperator{\dom}{dom}
\DeclareMathOperator{\proj}{proj}
\DeclareMathOperator{\ctg}{ctg}
\DeclareMathOperator{\supp}{supp}
\DeclareMathOperator{\argmin}{argmin}
\DeclareMathOperator{\mult}{mult}
\DeclareMathOperator{\ch}{ch}
\DeclareMathOperator{\sh}{sh}
\DeclareMathOperator{\rang}{rang}
\DeclareMathOperator{\diam}{diam}
\DeclareMathOperator{\Epigraphe}{Epigraphe}




\usepackage{xcolor}
\everymath{\color{blue}}
%\everymath{\color[rgb]{0,1,1}}
%\pagecolor[rgb]{0,0,0.5}


\newcommand*{\pdtest}[3][]{\ensuremath{\frac{\partial^{#1} #2}{\partial #3}}}

\newcommand*{\deffunc}[6][]{\ensuremath{
\begin{array}{rcl}
#2 : #3 &\rightarrow& #4\\
#5 &\mapsto& #6
\end{array}
}}

\newcommand{\eqcolon}{\mathrel{\resizebox{\widthof{$\mathord{=}$}}{\height}{ $\!\!=\!\!\resizebox{1.2\width}{0.8\height}{\raisebox{0.23ex}{$\mathop{:}$}}\!\!$ }}}
\newcommand{\coloneq}{\mathrel{\resizebox{\widthof{$\mathord{=}$}}{\height}{ $\!\!\resizebox{1.2\width}{0.8\height}{\raisebox{0.23ex}{$\mathop{:}$}}\!\!=\!\!$ }}}
\newcommand{\eqcolonl}{\ensuremath{\mathrel{=\!\!\mathop{:}}}}
\newcommand{\coloneql}{\ensuremath{\mathrel{\mathop{:} \!\! =}}}
\newcommand{\vc}[1]{% inline column vector
  \left(\begin{smallmatrix}#1\end{smallmatrix}\right)%
}
\newcommand{\vr}[1]{% inline row vector
  \begin{smallmatrix}(\,#1\,)\end{smallmatrix}%
}
\makeatletter
\newcommand*{\defeq}{\ =\mathrel{\rlap{%
                     \raisebox{0.3ex}{$\m@th\cdot$}}%
                     \raisebox{-0.3ex}{$\m@th\cdot$}}%
                     }
\makeatother

\newcommand{\mathcircle}[1]{% inline row vector
 \overset{\circ}{#1}
}
\newcommand{\ulim}{% low limit
 \underline{\lim}
}
\newcommand{\ssi}{% iff
\iff
}
\newcommand{\ps}[2]{
\expval{#1 | #2}
}
\newcommand{\df}[1]{
\mqty{#1}
}
\newcommand{\n}[1]{
\norm{#1}
}
\newcommand{\sys}[1]{
\left\{\smqty{#1}\right.
}


\newcommand{\eqdef}{\ensuremath{\overset{\text{def}}=}}


\def\Circlearrowright{\ensuremath{%
  \rotatebox[origin=c]{230}{$\circlearrowright$}}}

\newcommand\ct[1]{\text{\rmfamily\upshape #1}}
\newcommand\question[1]{ {\color{red} ...!? \small #1}}
\newcommand\caz[1]{\left\{\begin{array} #1 \end{array}\right.}
\newcommand\const{\text{\rmfamily\upshape const}}
\newcommand\toP{ \overset{\pro}{\to}}
\newcommand\toPP{ \overset{\text{PP}}{\to}}
\newcommand{\oeq}{\mathrel{\text{\textcircled{$=$}}}}





\usepackage{xcolor}
% \usepackage[normalem]{ulem}
\usepackage{lipsum}
\makeatletter
% \newcommand\colorwave[1][blue]{\bgroup \markoverwith{\lower3.5\p@\hbox{\sixly \textcolor{#1}{\char58}}}\ULon}
%\font\sixly=lasy6 % does not re-load if already loaded, so no memory problem.

\newmdtheoremenv[
linewidth= 1pt,linecolor= blue,%
leftmargin=20,rightmargin=20,innertopmargin=0pt, innerrightmargin=40,%
tikzsetting = { draw=lightgray, line width = 0.3pt,dashed,%
dash pattern = on 15pt off 3pt},%
splittopskip=\topskip,skipbelow=\baselineskip,%
skipabove=\baselineskip,ntheorem,roundcorner=0pt,
% backgroundcolor=pagebg,font=\color{orange}\sffamily, fontcolor=white
]{examplebox}{Exemple}[section]



\newcommand\R{\mathbb{R}}
\newcommand\Z{\mathbb{Z}}
\newcommand\N{\mathbb{N}}
\newcommand\E{\mathbb{E}}
\newcommand\F{\mathcal{F}}
\newcommand\cH{\mathcal{H}}
\newcommand\V{\mathbb{V}}
\newcommand\dmo{ ^{-1} }
\newcommand\kapa{\kappa}
\newcommand\im{Im}
\newcommand\hs{\mathcal{H}}





\usepackage{soul}

\makeatletter
\newcommand*{\whiten}[1]{\llap{\textcolor{white}{{\the\SOUL@token}}\hspace{#1pt}}}
\DeclareRobustCommand*\myul{%
    \def\SOUL@everyspace{\underline{\space}\kern\z@}%
    \def\SOUL@everytoken{%
     \setbox0=\hbox{\the\SOUL@token}%
     \ifdim\dp0>\z@
        \raisebox{\dp0}{\underline{\phantom{\the\SOUL@token}}}%
        \whiten{1}\whiten{0}%
        \whiten{-1}\whiten{-2}%
        \llap{\the\SOUL@token}%
     \else
        \underline{\the\SOUL@token}%
     \fi}%
\SOUL@}
\makeatother

\newcommand*{\demp}{\fontfamily{lmtt}\selectfont}

\DeclareTextFontCommand{\textdemp}{\demp}

\begin{document}

\ifcomment
Multiline
comment
\fi
\ifcomment
\myul{Typesetting test}
% \color[rgb]{1,1,1}
$∑_i^n≠ 60º±∞π∆¬≈√j∫h≤≥µ$

$\CR \R\pro\ind\pro\gS\pro
\mqty[a&b\\c&d]$
$\pro\mathbb{P}$
$\dd{x}$

  \[
    \alpha(x)=\left\{
                \begin{array}{ll}
                  x\\
                  \frac{1}{1+e^{-kx}}\\
                  \frac{e^x-e^{-x}}{e^x+e^{-x}}
                \end{array}
              \right.
  \]

  $\expval{x}$
  
  $\chi_\rho(ghg\dmo)=\Tr(\rho_{ghg\dmo})=\Tr(\rho_g\circ\rho_h\circ\rho\dmo_g)=\Tr(\rho_h)\overset{\mbox{\scalebox{0.5}{$\Tr(AB)=\Tr(BA)$}}}{=}\chi_\rho(h)$
  	$\mathop{\oplus}_{\substack{x\in X}}$

$\mat(\rho_g)=(a_{ij}(g))_{\scriptsize \substack{1\leq i\leq d \\ 1\leq j\leq d}}$ et $\mat(\rho'_g)=(a'_{ij}(g))_{\scriptsize \substack{1\leq i'\leq d' \\ 1\leq j'\leq d'}}$



\[\int_a^b{\mathbb{R}^2}g(u, v)\dd{P_{XY}}(u, v)=\iint g(u,v) f_{XY}(u, v)\dd \lambda(u) \dd \lambda(v)\]
$$\lim_{x\to\infty} f(x)$$	
$$\iiiint_V \mu(t,u,v,w) \,dt\,du\,dv\,dw$$
$$\sum_{n=1}^{\infty} 2^{-n} = 1$$	
\begin{definition}
	Si $X$ et $Y$ sont 2 v.a. ou definit la \textsc{Covariance} entre $X$ et $Y$ comme
	$\cov(X,Y)\overset{\text{def}}{=}\E\left[(X-\E(X))(Y-\E(Y))\right]=\E(XY)-\E(X)\E(Y)$.
\end{definition}
\fi
\pagebreak

% \tableofcontents

% insert your code here
%\input{./algebra/main.tex}
%\input{./geometrie-differentielle/main.tex}
%\input{./probabilite/main.tex}
%\input{./analyse-fonctionnelle/main.tex}
% \input{./Analyse-convexe-et-dualite-en-optimisation/main.tex}
%\input{./tikz/main.tex}
%\input{./Theorie-du-distributions/main.tex}
%\input{./optimisation/mine.tex}
 \input{./modelisation/main.tex}

% yves.aubry@univ-tln.fr : algebra

\end{document}


% yves.aubry@univ-tln.fr : algebra

\end{document}

% % !TEX encoding = UTF-8 Unicode
% !TEX TS-program = xelatex

\documentclass[french]{report}

%\usepackage[utf8]{inputenc}
%\usepackage[T1]{fontenc}
\usepackage{babel}


\newif\ifcomment
%\commenttrue # Show comments

\usepackage{physics}
\usepackage{amssymb}


\usepackage{amsthm}
% \usepackage{thmtools}
\usepackage{mathtools}
\usepackage{amsfonts}

\usepackage{color}

\usepackage{tikz}

\usepackage{geometry}
\geometry{a5paper, margin=0.1in, right=1cm}

\usepackage{dsfont}

\usepackage{graphicx}
\graphicspath{ {images/} }

\usepackage{faktor}

\usepackage{IEEEtrantools}
\usepackage{enumerate}   
\usepackage[PostScript=dvips]{"/Users/aware/Documents/Courses/diagrams"}


\newtheorem{theorem}{Théorème}[section]
\renewcommand{\thetheorem}{\arabic{theorem}}
\newtheorem{lemme}{Lemme}[section]
\renewcommand{\thelemme}{\arabic{lemme}}
\newtheorem{proposition}{Proposition}[section]
\renewcommand{\theproposition}{\arabic{proposition}}
\newtheorem{notations}{Notations}[section]
\newtheorem{problem}{Problème}[section]
\newtheorem{corollary}{Corollaire}[theorem]
\renewcommand{\thecorollary}{\arabic{corollary}}
\newtheorem{property}{Propriété}[section]
\newtheorem{objective}{Objectif}[section]

\theoremstyle{definition}
\newtheorem{definition}{Définition}[section]
\renewcommand{\thedefinition}{\arabic{definition}}
\newtheorem{exercise}{Exercice}[chapter]
\renewcommand{\theexercise}{\arabic{exercise}}
\newtheorem{example}{Exemple}[chapter]
\renewcommand{\theexample}{\arabic{example}}
\newtheorem*{solution}{Solution}
\newtheorem*{application}{Application}
\newtheorem*{notation}{Notation}
\newtheorem*{vocabulary}{Vocabulaire}
\newtheorem*{properties}{Propriétés}



\theoremstyle{remark}
\newtheorem*{remark}{Remarque}
\newtheorem*{rappel}{Rappel}


\usepackage{etoolbox}
\AtBeginEnvironment{exercise}{\small}
\AtBeginEnvironment{example}{\small}

\usepackage{cases}
\usepackage[red]{mypack}

\usepackage[framemethod=TikZ]{mdframed}

\definecolor{bg}{rgb}{0.4,0.25,0.95}
\definecolor{pagebg}{rgb}{0,0,0.5}
\surroundwithmdframed[
   topline=false,
   rightline=false,
   bottomline=false,
   leftmargin=\parindent,
   skipabove=8pt,
   skipbelow=8pt,
   linecolor=blue,
   innerbottommargin=10pt,
   % backgroundcolor=bg,font=\color{orange}\sffamily, fontcolor=white
]{definition}

\usepackage{empheq}
\usepackage[most]{tcolorbox}

\newtcbox{\mymath}[1][]{%
    nobeforeafter, math upper, tcbox raise base,
    enhanced, colframe=blue!30!black,
    colback=red!10, boxrule=1pt,
    #1}

\usepackage{unixode}


\DeclareMathOperator{\ord}{ord}
\DeclareMathOperator{\orb}{orb}
\DeclareMathOperator{\stab}{stab}
\DeclareMathOperator{\Stab}{stab}
\DeclareMathOperator{\ppcm}{ppcm}
\DeclareMathOperator{\conj}{Conj}
\DeclareMathOperator{\End}{End}
\DeclareMathOperator{\rot}{rot}
\DeclareMathOperator{\trs}{trace}
\DeclareMathOperator{\Ind}{Ind}
\DeclareMathOperator{\mat}{Mat}
\DeclareMathOperator{\id}{Id}
\DeclareMathOperator{\vect}{vect}
\DeclareMathOperator{\img}{img}
\DeclareMathOperator{\cov}{Cov}
\DeclareMathOperator{\dist}{dist}
\DeclareMathOperator{\irr}{Irr}
\DeclareMathOperator{\image}{Im}
\DeclareMathOperator{\pd}{\partial}
\DeclareMathOperator{\epi}{epi}
\DeclareMathOperator{\Argmin}{Argmin}
\DeclareMathOperator{\dom}{dom}
\DeclareMathOperator{\proj}{proj}
\DeclareMathOperator{\ctg}{ctg}
\DeclareMathOperator{\supp}{supp}
\DeclareMathOperator{\argmin}{argmin}
\DeclareMathOperator{\mult}{mult}
\DeclareMathOperator{\ch}{ch}
\DeclareMathOperator{\sh}{sh}
\DeclareMathOperator{\rang}{rang}
\DeclareMathOperator{\diam}{diam}
\DeclareMathOperator{\Epigraphe}{Epigraphe}




\usepackage{xcolor}
\everymath{\color{blue}}
%\everymath{\color[rgb]{0,1,1}}
%\pagecolor[rgb]{0,0,0.5}


\newcommand*{\pdtest}[3][]{\ensuremath{\frac{\partial^{#1} #2}{\partial #3}}}

\newcommand*{\deffunc}[6][]{\ensuremath{
\begin{array}{rcl}
#2 : #3 &\rightarrow& #4\\
#5 &\mapsto& #6
\end{array}
}}

\newcommand{\eqcolon}{\mathrel{\resizebox{\widthof{$\mathord{=}$}}{\height}{ $\!\!=\!\!\resizebox{1.2\width}{0.8\height}{\raisebox{0.23ex}{$\mathop{:}$}}\!\!$ }}}
\newcommand{\coloneq}{\mathrel{\resizebox{\widthof{$\mathord{=}$}}{\height}{ $\!\!\resizebox{1.2\width}{0.8\height}{\raisebox{0.23ex}{$\mathop{:}$}}\!\!=\!\!$ }}}
\newcommand{\eqcolonl}{\ensuremath{\mathrel{=\!\!\mathop{:}}}}
\newcommand{\coloneql}{\ensuremath{\mathrel{\mathop{:} \!\! =}}}
\newcommand{\vc}[1]{% inline column vector
  \left(\begin{smallmatrix}#1\end{smallmatrix}\right)%
}
\newcommand{\vr}[1]{% inline row vector
  \begin{smallmatrix}(\,#1\,)\end{smallmatrix}%
}
\makeatletter
\newcommand*{\defeq}{\ =\mathrel{\rlap{%
                     \raisebox{0.3ex}{$\m@th\cdot$}}%
                     \raisebox{-0.3ex}{$\m@th\cdot$}}%
                     }
\makeatother

\newcommand{\mathcircle}[1]{% inline row vector
 \overset{\circ}{#1}
}
\newcommand{\ulim}{% low limit
 \underline{\lim}
}
\newcommand{\ssi}{% iff
\iff
}
\newcommand{\ps}[2]{
\expval{#1 | #2}
}
\newcommand{\df}[1]{
\mqty{#1}
}
\newcommand{\n}[1]{
\norm{#1}
}
\newcommand{\sys}[1]{
\left\{\smqty{#1}\right.
}


\newcommand{\eqdef}{\ensuremath{\overset{\text{def}}=}}


\def\Circlearrowright{\ensuremath{%
  \rotatebox[origin=c]{230}{$\circlearrowright$}}}

\newcommand\ct[1]{\text{\rmfamily\upshape #1}}
\newcommand\question[1]{ {\color{red} ...!? \small #1}}
\newcommand\caz[1]{\left\{\begin{array} #1 \end{array}\right.}
\newcommand\const{\text{\rmfamily\upshape const}}
\newcommand\toP{ \overset{\pro}{\to}}
\newcommand\toPP{ \overset{\text{PP}}{\to}}
\newcommand{\oeq}{\mathrel{\text{\textcircled{$=$}}}}





\usepackage{xcolor}
% \usepackage[normalem]{ulem}
\usepackage{lipsum}
\makeatletter
% \newcommand\colorwave[1][blue]{\bgroup \markoverwith{\lower3.5\p@\hbox{\sixly \textcolor{#1}{\char58}}}\ULon}
%\font\sixly=lasy6 % does not re-load if already loaded, so no memory problem.

\newmdtheoremenv[
linewidth= 1pt,linecolor= blue,%
leftmargin=20,rightmargin=20,innertopmargin=0pt, innerrightmargin=40,%
tikzsetting = { draw=lightgray, line width = 0.3pt,dashed,%
dash pattern = on 15pt off 3pt},%
splittopskip=\topskip,skipbelow=\baselineskip,%
skipabove=\baselineskip,ntheorem,roundcorner=0pt,
% backgroundcolor=pagebg,font=\color{orange}\sffamily, fontcolor=white
]{examplebox}{Exemple}[section]



\newcommand\R{\mathbb{R}}
\newcommand\Z{\mathbb{Z}}
\newcommand\N{\mathbb{N}}
\newcommand\E{\mathbb{E}}
\newcommand\F{\mathcal{F}}
\newcommand\cH{\mathcal{H}}
\newcommand\V{\mathbb{V}}
\newcommand\dmo{ ^{-1} }
\newcommand\kapa{\kappa}
\newcommand\im{Im}
\newcommand\hs{\mathcal{H}}





\usepackage{soul}

\makeatletter
\newcommand*{\whiten}[1]{\llap{\textcolor{white}{{\the\SOUL@token}}\hspace{#1pt}}}
\DeclareRobustCommand*\myul{%
    \def\SOUL@everyspace{\underline{\space}\kern\z@}%
    \def\SOUL@everytoken{%
     \setbox0=\hbox{\the\SOUL@token}%
     \ifdim\dp0>\z@
        \raisebox{\dp0}{\underline{\phantom{\the\SOUL@token}}}%
        \whiten{1}\whiten{0}%
        \whiten{-1}\whiten{-2}%
        \llap{\the\SOUL@token}%
     \else
        \underline{\the\SOUL@token}%
     \fi}%
\SOUL@}
\makeatother

\newcommand*{\demp}{\fontfamily{lmtt}\selectfont}

\DeclareTextFontCommand{\textdemp}{\demp}

\begin{document}

\ifcomment
Multiline
comment
\fi
\ifcomment
\myul{Typesetting test}
% \color[rgb]{1,1,1}
$∑_i^n≠ 60º±∞π∆¬≈√j∫h≤≥µ$

$\CR \R\pro\ind\pro\gS\pro
\mqty[a&b\\c&d]$
$\pro\mathbb{P}$
$\dd{x}$

  \[
    \alpha(x)=\left\{
                \begin{array}{ll}
                  x\\
                  \frac{1}{1+e^{-kx}}\\
                  \frac{e^x-e^{-x}}{e^x+e^{-x}}
                \end{array}
              \right.
  \]

  $\expval{x}$
  
  $\chi_\rho(ghg\dmo)=\Tr(\rho_{ghg\dmo})=\Tr(\rho_g\circ\rho_h\circ\rho\dmo_g)=\Tr(\rho_h)\overset{\mbox{\scalebox{0.5}{$\Tr(AB)=\Tr(BA)$}}}{=}\chi_\rho(h)$
  	$\mathop{\oplus}_{\substack{x\in X}}$

$\mat(\rho_g)=(a_{ij}(g))_{\scriptsize \substack{1\leq i\leq d \\ 1\leq j\leq d}}$ et $\mat(\rho'_g)=(a'_{ij}(g))_{\scriptsize \substack{1\leq i'\leq d' \\ 1\leq j'\leq d'}}$



\[\int_a^b{\mathbb{R}^2}g(u, v)\dd{P_{XY}}(u, v)=\iint g(u,v) f_{XY}(u, v)\dd \lambda(u) \dd \lambda(v)\]
$$\lim_{x\to\infty} f(x)$$	
$$\iiiint_V \mu(t,u,v,w) \,dt\,du\,dv\,dw$$
$$\sum_{n=1}^{\infty} 2^{-n} = 1$$	
\begin{definition}
	Si $X$ et $Y$ sont 2 v.a. ou definit la \textsc{Covariance} entre $X$ et $Y$ comme
	$\cov(X,Y)\overset{\text{def}}{=}\E\left[(X-\E(X))(Y-\E(Y))\right]=\E(XY)-\E(X)\E(Y)$.
\end{definition}
\fi
\pagebreak

% \tableofcontents

% insert your code here
%% !TEX encoding = UTF-8 Unicode
% !TEX TS-program = xelatex

\documentclass[french]{report}

%\usepackage[utf8]{inputenc}
%\usepackage[T1]{fontenc}
\usepackage{babel}


\newif\ifcomment
%\commenttrue # Show comments

\usepackage{physics}
\usepackage{amssymb}


\usepackage{amsthm}
% \usepackage{thmtools}
\usepackage{mathtools}
\usepackage{amsfonts}

\usepackage{color}

\usepackage{tikz}

\usepackage{geometry}
\geometry{a5paper, margin=0.1in, right=1cm}

\usepackage{dsfont}

\usepackage{graphicx}
\graphicspath{ {images/} }

\usepackage{faktor}

\usepackage{IEEEtrantools}
\usepackage{enumerate}   
\usepackage[PostScript=dvips]{"/Users/aware/Documents/Courses/diagrams"}


\newtheorem{theorem}{Théorème}[section]
\renewcommand{\thetheorem}{\arabic{theorem}}
\newtheorem{lemme}{Lemme}[section]
\renewcommand{\thelemme}{\arabic{lemme}}
\newtheorem{proposition}{Proposition}[section]
\renewcommand{\theproposition}{\arabic{proposition}}
\newtheorem{notations}{Notations}[section]
\newtheorem{problem}{Problème}[section]
\newtheorem{corollary}{Corollaire}[theorem]
\renewcommand{\thecorollary}{\arabic{corollary}}
\newtheorem{property}{Propriété}[section]
\newtheorem{objective}{Objectif}[section]

\theoremstyle{definition}
\newtheorem{definition}{Définition}[section]
\renewcommand{\thedefinition}{\arabic{definition}}
\newtheorem{exercise}{Exercice}[chapter]
\renewcommand{\theexercise}{\arabic{exercise}}
\newtheorem{example}{Exemple}[chapter]
\renewcommand{\theexample}{\arabic{example}}
\newtheorem*{solution}{Solution}
\newtheorem*{application}{Application}
\newtheorem*{notation}{Notation}
\newtheorem*{vocabulary}{Vocabulaire}
\newtheorem*{properties}{Propriétés}



\theoremstyle{remark}
\newtheorem*{remark}{Remarque}
\newtheorem*{rappel}{Rappel}


\usepackage{etoolbox}
\AtBeginEnvironment{exercise}{\small}
\AtBeginEnvironment{example}{\small}

\usepackage{cases}
\usepackage[red]{mypack}

\usepackage[framemethod=TikZ]{mdframed}

\definecolor{bg}{rgb}{0.4,0.25,0.95}
\definecolor{pagebg}{rgb}{0,0,0.5}
\surroundwithmdframed[
   topline=false,
   rightline=false,
   bottomline=false,
   leftmargin=\parindent,
   skipabove=8pt,
   skipbelow=8pt,
   linecolor=blue,
   innerbottommargin=10pt,
   % backgroundcolor=bg,font=\color{orange}\sffamily, fontcolor=white
]{definition}

\usepackage{empheq}
\usepackage[most]{tcolorbox}

\newtcbox{\mymath}[1][]{%
    nobeforeafter, math upper, tcbox raise base,
    enhanced, colframe=blue!30!black,
    colback=red!10, boxrule=1pt,
    #1}

\usepackage{unixode}


\DeclareMathOperator{\ord}{ord}
\DeclareMathOperator{\orb}{orb}
\DeclareMathOperator{\stab}{stab}
\DeclareMathOperator{\Stab}{stab}
\DeclareMathOperator{\ppcm}{ppcm}
\DeclareMathOperator{\conj}{Conj}
\DeclareMathOperator{\End}{End}
\DeclareMathOperator{\rot}{rot}
\DeclareMathOperator{\trs}{trace}
\DeclareMathOperator{\Ind}{Ind}
\DeclareMathOperator{\mat}{Mat}
\DeclareMathOperator{\id}{Id}
\DeclareMathOperator{\vect}{vect}
\DeclareMathOperator{\img}{img}
\DeclareMathOperator{\cov}{Cov}
\DeclareMathOperator{\dist}{dist}
\DeclareMathOperator{\irr}{Irr}
\DeclareMathOperator{\image}{Im}
\DeclareMathOperator{\pd}{\partial}
\DeclareMathOperator{\epi}{epi}
\DeclareMathOperator{\Argmin}{Argmin}
\DeclareMathOperator{\dom}{dom}
\DeclareMathOperator{\proj}{proj}
\DeclareMathOperator{\ctg}{ctg}
\DeclareMathOperator{\supp}{supp}
\DeclareMathOperator{\argmin}{argmin}
\DeclareMathOperator{\mult}{mult}
\DeclareMathOperator{\ch}{ch}
\DeclareMathOperator{\sh}{sh}
\DeclareMathOperator{\rang}{rang}
\DeclareMathOperator{\diam}{diam}
\DeclareMathOperator{\Epigraphe}{Epigraphe}




\usepackage{xcolor}
\everymath{\color{blue}}
%\everymath{\color[rgb]{0,1,1}}
%\pagecolor[rgb]{0,0,0.5}


\newcommand*{\pdtest}[3][]{\ensuremath{\frac{\partial^{#1} #2}{\partial #3}}}

\newcommand*{\deffunc}[6][]{\ensuremath{
\begin{array}{rcl}
#2 : #3 &\rightarrow& #4\\
#5 &\mapsto& #6
\end{array}
}}

\newcommand{\eqcolon}{\mathrel{\resizebox{\widthof{$\mathord{=}$}}{\height}{ $\!\!=\!\!\resizebox{1.2\width}{0.8\height}{\raisebox{0.23ex}{$\mathop{:}$}}\!\!$ }}}
\newcommand{\coloneq}{\mathrel{\resizebox{\widthof{$\mathord{=}$}}{\height}{ $\!\!\resizebox{1.2\width}{0.8\height}{\raisebox{0.23ex}{$\mathop{:}$}}\!\!=\!\!$ }}}
\newcommand{\eqcolonl}{\ensuremath{\mathrel{=\!\!\mathop{:}}}}
\newcommand{\coloneql}{\ensuremath{\mathrel{\mathop{:} \!\! =}}}
\newcommand{\vc}[1]{% inline column vector
  \left(\begin{smallmatrix}#1\end{smallmatrix}\right)%
}
\newcommand{\vr}[1]{% inline row vector
  \begin{smallmatrix}(\,#1\,)\end{smallmatrix}%
}
\makeatletter
\newcommand*{\defeq}{\ =\mathrel{\rlap{%
                     \raisebox{0.3ex}{$\m@th\cdot$}}%
                     \raisebox{-0.3ex}{$\m@th\cdot$}}%
                     }
\makeatother

\newcommand{\mathcircle}[1]{% inline row vector
 \overset{\circ}{#1}
}
\newcommand{\ulim}{% low limit
 \underline{\lim}
}
\newcommand{\ssi}{% iff
\iff
}
\newcommand{\ps}[2]{
\expval{#1 | #2}
}
\newcommand{\df}[1]{
\mqty{#1}
}
\newcommand{\n}[1]{
\norm{#1}
}
\newcommand{\sys}[1]{
\left\{\smqty{#1}\right.
}


\newcommand{\eqdef}{\ensuremath{\overset{\text{def}}=}}


\def\Circlearrowright{\ensuremath{%
  \rotatebox[origin=c]{230}{$\circlearrowright$}}}

\newcommand\ct[1]{\text{\rmfamily\upshape #1}}
\newcommand\question[1]{ {\color{red} ...!? \small #1}}
\newcommand\caz[1]{\left\{\begin{array} #1 \end{array}\right.}
\newcommand\const{\text{\rmfamily\upshape const}}
\newcommand\toP{ \overset{\pro}{\to}}
\newcommand\toPP{ \overset{\text{PP}}{\to}}
\newcommand{\oeq}{\mathrel{\text{\textcircled{$=$}}}}





\usepackage{xcolor}
% \usepackage[normalem]{ulem}
\usepackage{lipsum}
\makeatletter
% \newcommand\colorwave[1][blue]{\bgroup \markoverwith{\lower3.5\p@\hbox{\sixly \textcolor{#1}{\char58}}}\ULon}
%\font\sixly=lasy6 % does not re-load if already loaded, so no memory problem.

\newmdtheoremenv[
linewidth= 1pt,linecolor= blue,%
leftmargin=20,rightmargin=20,innertopmargin=0pt, innerrightmargin=40,%
tikzsetting = { draw=lightgray, line width = 0.3pt,dashed,%
dash pattern = on 15pt off 3pt},%
splittopskip=\topskip,skipbelow=\baselineskip,%
skipabove=\baselineskip,ntheorem,roundcorner=0pt,
% backgroundcolor=pagebg,font=\color{orange}\sffamily, fontcolor=white
]{examplebox}{Exemple}[section]



\newcommand\R{\mathbb{R}}
\newcommand\Z{\mathbb{Z}}
\newcommand\N{\mathbb{N}}
\newcommand\E{\mathbb{E}}
\newcommand\F{\mathcal{F}}
\newcommand\cH{\mathcal{H}}
\newcommand\V{\mathbb{V}}
\newcommand\dmo{ ^{-1} }
\newcommand\kapa{\kappa}
\newcommand\im{Im}
\newcommand\hs{\mathcal{H}}





\usepackage{soul}

\makeatletter
\newcommand*{\whiten}[1]{\llap{\textcolor{white}{{\the\SOUL@token}}\hspace{#1pt}}}
\DeclareRobustCommand*\myul{%
    \def\SOUL@everyspace{\underline{\space}\kern\z@}%
    \def\SOUL@everytoken{%
     \setbox0=\hbox{\the\SOUL@token}%
     \ifdim\dp0>\z@
        \raisebox{\dp0}{\underline{\phantom{\the\SOUL@token}}}%
        \whiten{1}\whiten{0}%
        \whiten{-1}\whiten{-2}%
        \llap{\the\SOUL@token}%
     \else
        \underline{\the\SOUL@token}%
     \fi}%
\SOUL@}
\makeatother

\newcommand*{\demp}{\fontfamily{lmtt}\selectfont}

\DeclareTextFontCommand{\textdemp}{\demp}

\begin{document}

\ifcomment
Multiline
comment
\fi
\ifcomment
\myul{Typesetting test}
% \color[rgb]{1,1,1}
$∑_i^n≠ 60º±∞π∆¬≈√j∫h≤≥µ$

$\CR \R\pro\ind\pro\gS\pro
\mqty[a&b\\c&d]$
$\pro\mathbb{P}$
$\dd{x}$

  \[
    \alpha(x)=\left\{
                \begin{array}{ll}
                  x\\
                  \frac{1}{1+e^{-kx}}\\
                  \frac{e^x-e^{-x}}{e^x+e^{-x}}
                \end{array}
              \right.
  \]

  $\expval{x}$
  
  $\chi_\rho(ghg\dmo)=\Tr(\rho_{ghg\dmo})=\Tr(\rho_g\circ\rho_h\circ\rho\dmo_g)=\Tr(\rho_h)\overset{\mbox{\scalebox{0.5}{$\Tr(AB)=\Tr(BA)$}}}{=}\chi_\rho(h)$
  	$\mathop{\oplus}_{\substack{x\in X}}$

$\mat(\rho_g)=(a_{ij}(g))_{\scriptsize \substack{1\leq i\leq d \\ 1\leq j\leq d}}$ et $\mat(\rho'_g)=(a'_{ij}(g))_{\scriptsize \substack{1\leq i'\leq d' \\ 1\leq j'\leq d'}}$



\[\int_a^b{\mathbb{R}^2}g(u, v)\dd{P_{XY}}(u, v)=\iint g(u,v) f_{XY}(u, v)\dd \lambda(u) \dd \lambda(v)\]
$$\lim_{x\to\infty} f(x)$$	
$$\iiiint_V \mu(t,u,v,w) \,dt\,du\,dv\,dw$$
$$\sum_{n=1}^{\infty} 2^{-n} = 1$$	
\begin{definition}
	Si $X$ et $Y$ sont 2 v.a. ou definit la \textsc{Covariance} entre $X$ et $Y$ comme
	$\cov(X,Y)\overset{\text{def}}{=}\E\left[(X-\E(X))(Y-\E(Y))\right]=\E(XY)-\E(X)\E(Y)$.
\end{definition}
\fi
\pagebreak

% \tableofcontents

% insert your code here
%\input{./algebra/main.tex}
%\input{./geometrie-differentielle/main.tex}
%\input{./probabilite/main.tex}
%\input{./analyse-fonctionnelle/main.tex}
% \input{./Analyse-convexe-et-dualite-en-optimisation/main.tex}
%\input{./tikz/main.tex}
%\input{./Theorie-du-distributions/main.tex}
%\input{./optimisation/mine.tex}
 \input{./modelisation/main.tex}

% yves.aubry@univ-tln.fr : algebra

\end{document}

%% !TEX encoding = UTF-8 Unicode
% !TEX TS-program = xelatex

\documentclass[french]{report}

%\usepackage[utf8]{inputenc}
%\usepackage[T1]{fontenc}
\usepackage{babel}


\newif\ifcomment
%\commenttrue # Show comments

\usepackage{physics}
\usepackage{amssymb}


\usepackage{amsthm}
% \usepackage{thmtools}
\usepackage{mathtools}
\usepackage{amsfonts}

\usepackage{color}

\usepackage{tikz}

\usepackage{geometry}
\geometry{a5paper, margin=0.1in, right=1cm}

\usepackage{dsfont}

\usepackage{graphicx}
\graphicspath{ {images/} }

\usepackage{faktor}

\usepackage{IEEEtrantools}
\usepackage{enumerate}   
\usepackage[PostScript=dvips]{"/Users/aware/Documents/Courses/diagrams"}


\newtheorem{theorem}{Théorème}[section]
\renewcommand{\thetheorem}{\arabic{theorem}}
\newtheorem{lemme}{Lemme}[section]
\renewcommand{\thelemme}{\arabic{lemme}}
\newtheorem{proposition}{Proposition}[section]
\renewcommand{\theproposition}{\arabic{proposition}}
\newtheorem{notations}{Notations}[section]
\newtheorem{problem}{Problème}[section]
\newtheorem{corollary}{Corollaire}[theorem]
\renewcommand{\thecorollary}{\arabic{corollary}}
\newtheorem{property}{Propriété}[section]
\newtheorem{objective}{Objectif}[section]

\theoremstyle{definition}
\newtheorem{definition}{Définition}[section]
\renewcommand{\thedefinition}{\arabic{definition}}
\newtheorem{exercise}{Exercice}[chapter]
\renewcommand{\theexercise}{\arabic{exercise}}
\newtheorem{example}{Exemple}[chapter]
\renewcommand{\theexample}{\arabic{example}}
\newtheorem*{solution}{Solution}
\newtheorem*{application}{Application}
\newtheorem*{notation}{Notation}
\newtheorem*{vocabulary}{Vocabulaire}
\newtheorem*{properties}{Propriétés}



\theoremstyle{remark}
\newtheorem*{remark}{Remarque}
\newtheorem*{rappel}{Rappel}


\usepackage{etoolbox}
\AtBeginEnvironment{exercise}{\small}
\AtBeginEnvironment{example}{\small}

\usepackage{cases}
\usepackage[red]{mypack}

\usepackage[framemethod=TikZ]{mdframed}

\definecolor{bg}{rgb}{0.4,0.25,0.95}
\definecolor{pagebg}{rgb}{0,0,0.5}
\surroundwithmdframed[
   topline=false,
   rightline=false,
   bottomline=false,
   leftmargin=\parindent,
   skipabove=8pt,
   skipbelow=8pt,
   linecolor=blue,
   innerbottommargin=10pt,
   % backgroundcolor=bg,font=\color{orange}\sffamily, fontcolor=white
]{definition}

\usepackage{empheq}
\usepackage[most]{tcolorbox}

\newtcbox{\mymath}[1][]{%
    nobeforeafter, math upper, tcbox raise base,
    enhanced, colframe=blue!30!black,
    colback=red!10, boxrule=1pt,
    #1}

\usepackage{unixode}


\DeclareMathOperator{\ord}{ord}
\DeclareMathOperator{\orb}{orb}
\DeclareMathOperator{\stab}{stab}
\DeclareMathOperator{\Stab}{stab}
\DeclareMathOperator{\ppcm}{ppcm}
\DeclareMathOperator{\conj}{Conj}
\DeclareMathOperator{\End}{End}
\DeclareMathOperator{\rot}{rot}
\DeclareMathOperator{\trs}{trace}
\DeclareMathOperator{\Ind}{Ind}
\DeclareMathOperator{\mat}{Mat}
\DeclareMathOperator{\id}{Id}
\DeclareMathOperator{\vect}{vect}
\DeclareMathOperator{\img}{img}
\DeclareMathOperator{\cov}{Cov}
\DeclareMathOperator{\dist}{dist}
\DeclareMathOperator{\irr}{Irr}
\DeclareMathOperator{\image}{Im}
\DeclareMathOperator{\pd}{\partial}
\DeclareMathOperator{\epi}{epi}
\DeclareMathOperator{\Argmin}{Argmin}
\DeclareMathOperator{\dom}{dom}
\DeclareMathOperator{\proj}{proj}
\DeclareMathOperator{\ctg}{ctg}
\DeclareMathOperator{\supp}{supp}
\DeclareMathOperator{\argmin}{argmin}
\DeclareMathOperator{\mult}{mult}
\DeclareMathOperator{\ch}{ch}
\DeclareMathOperator{\sh}{sh}
\DeclareMathOperator{\rang}{rang}
\DeclareMathOperator{\diam}{diam}
\DeclareMathOperator{\Epigraphe}{Epigraphe}




\usepackage{xcolor}
\everymath{\color{blue}}
%\everymath{\color[rgb]{0,1,1}}
%\pagecolor[rgb]{0,0,0.5}


\newcommand*{\pdtest}[3][]{\ensuremath{\frac{\partial^{#1} #2}{\partial #3}}}

\newcommand*{\deffunc}[6][]{\ensuremath{
\begin{array}{rcl}
#2 : #3 &\rightarrow& #4\\
#5 &\mapsto& #6
\end{array}
}}

\newcommand{\eqcolon}{\mathrel{\resizebox{\widthof{$\mathord{=}$}}{\height}{ $\!\!=\!\!\resizebox{1.2\width}{0.8\height}{\raisebox{0.23ex}{$\mathop{:}$}}\!\!$ }}}
\newcommand{\coloneq}{\mathrel{\resizebox{\widthof{$\mathord{=}$}}{\height}{ $\!\!\resizebox{1.2\width}{0.8\height}{\raisebox{0.23ex}{$\mathop{:}$}}\!\!=\!\!$ }}}
\newcommand{\eqcolonl}{\ensuremath{\mathrel{=\!\!\mathop{:}}}}
\newcommand{\coloneql}{\ensuremath{\mathrel{\mathop{:} \!\! =}}}
\newcommand{\vc}[1]{% inline column vector
  \left(\begin{smallmatrix}#1\end{smallmatrix}\right)%
}
\newcommand{\vr}[1]{% inline row vector
  \begin{smallmatrix}(\,#1\,)\end{smallmatrix}%
}
\makeatletter
\newcommand*{\defeq}{\ =\mathrel{\rlap{%
                     \raisebox{0.3ex}{$\m@th\cdot$}}%
                     \raisebox{-0.3ex}{$\m@th\cdot$}}%
                     }
\makeatother

\newcommand{\mathcircle}[1]{% inline row vector
 \overset{\circ}{#1}
}
\newcommand{\ulim}{% low limit
 \underline{\lim}
}
\newcommand{\ssi}{% iff
\iff
}
\newcommand{\ps}[2]{
\expval{#1 | #2}
}
\newcommand{\df}[1]{
\mqty{#1}
}
\newcommand{\n}[1]{
\norm{#1}
}
\newcommand{\sys}[1]{
\left\{\smqty{#1}\right.
}


\newcommand{\eqdef}{\ensuremath{\overset{\text{def}}=}}


\def\Circlearrowright{\ensuremath{%
  \rotatebox[origin=c]{230}{$\circlearrowright$}}}

\newcommand\ct[1]{\text{\rmfamily\upshape #1}}
\newcommand\question[1]{ {\color{red} ...!? \small #1}}
\newcommand\caz[1]{\left\{\begin{array} #1 \end{array}\right.}
\newcommand\const{\text{\rmfamily\upshape const}}
\newcommand\toP{ \overset{\pro}{\to}}
\newcommand\toPP{ \overset{\text{PP}}{\to}}
\newcommand{\oeq}{\mathrel{\text{\textcircled{$=$}}}}





\usepackage{xcolor}
% \usepackage[normalem]{ulem}
\usepackage{lipsum}
\makeatletter
% \newcommand\colorwave[1][blue]{\bgroup \markoverwith{\lower3.5\p@\hbox{\sixly \textcolor{#1}{\char58}}}\ULon}
%\font\sixly=lasy6 % does not re-load if already loaded, so no memory problem.

\newmdtheoremenv[
linewidth= 1pt,linecolor= blue,%
leftmargin=20,rightmargin=20,innertopmargin=0pt, innerrightmargin=40,%
tikzsetting = { draw=lightgray, line width = 0.3pt,dashed,%
dash pattern = on 15pt off 3pt},%
splittopskip=\topskip,skipbelow=\baselineskip,%
skipabove=\baselineskip,ntheorem,roundcorner=0pt,
% backgroundcolor=pagebg,font=\color{orange}\sffamily, fontcolor=white
]{examplebox}{Exemple}[section]



\newcommand\R{\mathbb{R}}
\newcommand\Z{\mathbb{Z}}
\newcommand\N{\mathbb{N}}
\newcommand\E{\mathbb{E}}
\newcommand\F{\mathcal{F}}
\newcommand\cH{\mathcal{H}}
\newcommand\V{\mathbb{V}}
\newcommand\dmo{ ^{-1} }
\newcommand\kapa{\kappa}
\newcommand\im{Im}
\newcommand\hs{\mathcal{H}}





\usepackage{soul}

\makeatletter
\newcommand*{\whiten}[1]{\llap{\textcolor{white}{{\the\SOUL@token}}\hspace{#1pt}}}
\DeclareRobustCommand*\myul{%
    \def\SOUL@everyspace{\underline{\space}\kern\z@}%
    \def\SOUL@everytoken{%
     \setbox0=\hbox{\the\SOUL@token}%
     \ifdim\dp0>\z@
        \raisebox{\dp0}{\underline{\phantom{\the\SOUL@token}}}%
        \whiten{1}\whiten{0}%
        \whiten{-1}\whiten{-2}%
        \llap{\the\SOUL@token}%
     \else
        \underline{\the\SOUL@token}%
     \fi}%
\SOUL@}
\makeatother

\newcommand*{\demp}{\fontfamily{lmtt}\selectfont}

\DeclareTextFontCommand{\textdemp}{\demp}

\begin{document}

\ifcomment
Multiline
comment
\fi
\ifcomment
\myul{Typesetting test}
% \color[rgb]{1,1,1}
$∑_i^n≠ 60º±∞π∆¬≈√j∫h≤≥µ$

$\CR \R\pro\ind\pro\gS\pro
\mqty[a&b\\c&d]$
$\pro\mathbb{P}$
$\dd{x}$

  \[
    \alpha(x)=\left\{
                \begin{array}{ll}
                  x\\
                  \frac{1}{1+e^{-kx}}\\
                  \frac{e^x-e^{-x}}{e^x+e^{-x}}
                \end{array}
              \right.
  \]

  $\expval{x}$
  
  $\chi_\rho(ghg\dmo)=\Tr(\rho_{ghg\dmo})=\Tr(\rho_g\circ\rho_h\circ\rho\dmo_g)=\Tr(\rho_h)\overset{\mbox{\scalebox{0.5}{$\Tr(AB)=\Tr(BA)$}}}{=}\chi_\rho(h)$
  	$\mathop{\oplus}_{\substack{x\in X}}$

$\mat(\rho_g)=(a_{ij}(g))_{\scriptsize \substack{1\leq i\leq d \\ 1\leq j\leq d}}$ et $\mat(\rho'_g)=(a'_{ij}(g))_{\scriptsize \substack{1\leq i'\leq d' \\ 1\leq j'\leq d'}}$



\[\int_a^b{\mathbb{R}^2}g(u, v)\dd{P_{XY}}(u, v)=\iint g(u,v) f_{XY}(u, v)\dd \lambda(u) \dd \lambda(v)\]
$$\lim_{x\to\infty} f(x)$$	
$$\iiiint_V \mu(t,u,v,w) \,dt\,du\,dv\,dw$$
$$\sum_{n=1}^{\infty} 2^{-n} = 1$$	
\begin{definition}
	Si $X$ et $Y$ sont 2 v.a. ou definit la \textsc{Covariance} entre $X$ et $Y$ comme
	$\cov(X,Y)\overset{\text{def}}{=}\E\left[(X-\E(X))(Y-\E(Y))\right]=\E(XY)-\E(X)\E(Y)$.
\end{definition}
\fi
\pagebreak

% \tableofcontents

% insert your code here
%\input{./algebra/main.tex}
%\input{./geometrie-differentielle/main.tex}
%\input{./probabilite/main.tex}
%\input{./analyse-fonctionnelle/main.tex}
% \input{./Analyse-convexe-et-dualite-en-optimisation/main.tex}
%\input{./tikz/main.tex}
%\input{./Theorie-du-distributions/main.tex}
%\input{./optimisation/mine.tex}
 \input{./modelisation/main.tex}

% yves.aubry@univ-tln.fr : algebra

\end{document}

%% !TEX encoding = UTF-8 Unicode
% !TEX TS-program = xelatex

\documentclass[french]{report}

%\usepackage[utf8]{inputenc}
%\usepackage[T1]{fontenc}
\usepackage{babel}


\newif\ifcomment
%\commenttrue # Show comments

\usepackage{physics}
\usepackage{amssymb}


\usepackage{amsthm}
% \usepackage{thmtools}
\usepackage{mathtools}
\usepackage{amsfonts}

\usepackage{color}

\usepackage{tikz}

\usepackage{geometry}
\geometry{a5paper, margin=0.1in, right=1cm}

\usepackage{dsfont}

\usepackage{graphicx}
\graphicspath{ {images/} }

\usepackage{faktor}

\usepackage{IEEEtrantools}
\usepackage{enumerate}   
\usepackage[PostScript=dvips]{"/Users/aware/Documents/Courses/diagrams"}


\newtheorem{theorem}{Théorème}[section]
\renewcommand{\thetheorem}{\arabic{theorem}}
\newtheorem{lemme}{Lemme}[section]
\renewcommand{\thelemme}{\arabic{lemme}}
\newtheorem{proposition}{Proposition}[section]
\renewcommand{\theproposition}{\arabic{proposition}}
\newtheorem{notations}{Notations}[section]
\newtheorem{problem}{Problème}[section]
\newtheorem{corollary}{Corollaire}[theorem]
\renewcommand{\thecorollary}{\arabic{corollary}}
\newtheorem{property}{Propriété}[section]
\newtheorem{objective}{Objectif}[section]

\theoremstyle{definition}
\newtheorem{definition}{Définition}[section]
\renewcommand{\thedefinition}{\arabic{definition}}
\newtheorem{exercise}{Exercice}[chapter]
\renewcommand{\theexercise}{\arabic{exercise}}
\newtheorem{example}{Exemple}[chapter]
\renewcommand{\theexample}{\arabic{example}}
\newtheorem*{solution}{Solution}
\newtheorem*{application}{Application}
\newtheorem*{notation}{Notation}
\newtheorem*{vocabulary}{Vocabulaire}
\newtheorem*{properties}{Propriétés}



\theoremstyle{remark}
\newtheorem*{remark}{Remarque}
\newtheorem*{rappel}{Rappel}


\usepackage{etoolbox}
\AtBeginEnvironment{exercise}{\small}
\AtBeginEnvironment{example}{\small}

\usepackage{cases}
\usepackage[red]{mypack}

\usepackage[framemethod=TikZ]{mdframed}

\definecolor{bg}{rgb}{0.4,0.25,0.95}
\definecolor{pagebg}{rgb}{0,0,0.5}
\surroundwithmdframed[
   topline=false,
   rightline=false,
   bottomline=false,
   leftmargin=\parindent,
   skipabove=8pt,
   skipbelow=8pt,
   linecolor=blue,
   innerbottommargin=10pt,
   % backgroundcolor=bg,font=\color{orange}\sffamily, fontcolor=white
]{definition}

\usepackage{empheq}
\usepackage[most]{tcolorbox}

\newtcbox{\mymath}[1][]{%
    nobeforeafter, math upper, tcbox raise base,
    enhanced, colframe=blue!30!black,
    colback=red!10, boxrule=1pt,
    #1}

\usepackage{unixode}


\DeclareMathOperator{\ord}{ord}
\DeclareMathOperator{\orb}{orb}
\DeclareMathOperator{\stab}{stab}
\DeclareMathOperator{\Stab}{stab}
\DeclareMathOperator{\ppcm}{ppcm}
\DeclareMathOperator{\conj}{Conj}
\DeclareMathOperator{\End}{End}
\DeclareMathOperator{\rot}{rot}
\DeclareMathOperator{\trs}{trace}
\DeclareMathOperator{\Ind}{Ind}
\DeclareMathOperator{\mat}{Mat}
\DeclareMathOperator{\id}{Id}
\DeclareMathOperator{\vect}{vect}
\DeclareMathOperator{\img}{img}
\DeclareMathOperator{\cov}{Cov}
\DeclareMathOperator{\dist}{dist}
\DeclareMathOperator{\irr}{Irr}
\DeclareMathOperator{\image}{Im}
\DeclareMathOperator{\pd}{\partial}
\DeclareMathOperator{\epi}{epi}
\DeclareMathOperator{\Argmin}{Argmin}
\DeclareMathOperator{\dom}{dom}
\DeclareMathOperator{\proj}{proj}
\DeclareMathOperator{\ctg}{ctg}
\DeclareMathOperator{\supp}{supp}
\DeclareMathOperator{\argmin}{argmin}
\DeclareMathOperator{\mult}{mult}
\DeclareMathOperator{\ch}{ch}
\DeclareMathOperator{\sh}{sh}
\DeclareMathOperator{\rang}{rang}
\DeclareMathOperator{\diam}{diam}
\DeclareMathOperator{\Epigraphe}{Epigraphe}




\usepackage{xcolor}
\everymath{\color{blue}}
%\everymath{\color[rgb]{0,1,1}}
%\pagecolor[rgb]{0,0,0.5}


\newcommand*{\pdtest}[3][]{\ensuremath{\frac{\partial^{#1} #2}{\partial #3}}}

\newcommand*{\deffunc}[6][]{\ensuremath{
\begin{array}{rcl}
#2 : #3 &\rightarrow& #4\\
#5 &\mapsto& #6
\end{array}
}}

\newcommand{\eqcolon}{\mathrel{\resizebox{\widthof{$\mathord{=}$}}{\height}{ $\!\!=\!\!\resizebox{1.2\width}{0.8\height}{\raisebox{0.23ex}{$\mathop{:}$}}\!\!$ }}}
\newcommand{\coloneq}{\mathrel{\resizebox{\widthof{$\mathord{=}$}}{\height}{ $\!\!\resizebox{1.2\width}{0.8\height}{\raisebox{0.23ex}{$\mathop{:}$}}\!\!=\!\!$ }}}
\newcommand{\eqcolonl}{\ensuremath{\mathrel{=\!\!\mathop{:}}}}
\newcommand{\coloneql}{\ensuremath{\mathrel{\mathop{:} \!\! =}}}
\newcommand{\vc}[1]{% inline column vector
  \left(\begin{smallmatrix}#1\end{smallmatrix}\right)%
}
\newcommand{\vr}[1]{% inline row vector
  \begin{smallmatrix}(\,#1\,)\end{smallmatrix}%
}
\makeatletter
\newcommand*{\defeq}{\ =\mathrel{\rlap{%
                     \raisebox{0.3ex}{$\m@th\cdot$}}%
                     \raisebox{-0.3ex}{$\m@th\cdot$}}%
                     }
\makeatother

\newcommand{\mathcircle}[1]{% inline row vector
 \overset{\circ}{#1}
}
\newcommand{\ulim}{% low limit
 \underline{\lim}
}
\newcommand{\ssi}{% iff
\iff
}
\newcommand{\ps}[2]{
\expval{#1 | #2}
}
\newcommand{\df}[1]{
\mqty{#1}
}
\newcommand{\n}[1]{
\norm{#1}
}
\newcommand{\sys}[1]{
\left\{\smqty{#1}\right.
}


\newcommand{\eqdef}{\ensuremath{\overset{\text{def}}=}}


\def\Circlearrowright{\ensuremath{%
  \rotatebox[origin=c]{230}{$\circlearrowright$}}}

\newcommand\ct[1]{\text{\rmfamily\upshape #1}}
\newcommand\question[1]{ {\color{red} ...!? \small #1}}
\newcommand\caz[1]{\left\{\begin{array} #1 \end{array}\right.}
\newcommand\const{\text{\rmfamily\upshape const}}
\newcommand\toP{ \overset{\pro}{\to}}
\newcommand\toPP{ \overset{\text{PP}}{\to}}
\newcommand{\oeq}{\mathrel{\text{\textcircled{$=$}}}}





\usepackage{xcolor}
% \usepackage[normalem]{ulem}
\usepackage{lipsum}
\makeatletter
% \newcommand\colorwave[1][blue]{\bgroup \markoverwith{\lower3.5\p@\hbox{\sixly \textcolor{#1}{\char58}}}\ULon}
%\font\sixly=lasy6 % does not re-load if already loaded, so no memory problem.

\newmdtheoremenv[
linewidth= 1pt,linecolor= blue,%
leftmargin=20,rightmargin=20,innertopmargin=0pt, innerrightmargin=40,%
tikzsetting = { draw=lightgray, line width = 0.3pt,dashed,%
dash pattern = on 15pt off 3pt},%
splittopskip=\topskip,skipbelow=\baselineskip,%
skipabove=\baselineskip,ntheorem,roundcorner=0pt,
% backgroundcolor=pagebg,font=\color{orange}\sffamily, fontcolor=white
]{examplebox}{Exemple}[section]



\newcommand\R{\mathbb{R}}
\newcommand\Z{\mathbb{Z}}
\newcommand\N{\mathbb{N}}
\newcommand\E{\mathbb{E}}
\newcommand\F{\mathcal{F}}
\newcommand\cH{\mathcal{H}}
\newcommand\V{\mathbb{V}}
\newcommand\dmo{ ^{-1} }
\newcommand\kapa{\kappa}
\newcommand\im{Im}
\newcommand\hs{\mathcal{H}}





\usepackage{soul}

\makeatletter
\newcommand*{\whiten}[1]{\llap{\textcolor{white}{{\the\SOUL@token}}\hspace{#1pt}}}
\DeclareRobustCommand*\myul{%
    \def\SOUL@everyspace{\underline{\space}\kern\z@}%
    \def\SOUL@everytoken{%
     \setbox0=\hbox{\the\SOUL@token}%
     \ifdim\dp0>\z@
        \raisebox{\dp0}{\underline{\phantom{\the\SOUL@token}}}%
        \whiten{1}\whiten{0}%
        \whiten{-1}\whiten{-2}%
        \llap{\the\SOUL@token}%
     \else
        \underline{\the\SOUL@token}%
     \fi}%
\SOUL@}
\makeatother

\newcommand*{\demp}{\fontfamily{lmtt}\selectfont}

\DeclareTextFontCommand{\textdemp}{\demp}

\begin{document}

\ifcomment
Multiline
comment
\fi
\ifcomment
\myul{Typesetting test}
% \color[rgb]{1,1,1}
$∑_i^n≠ 60º±∞π∆¬≈√j∫h≤≥µ$

$\CR \R\pro\ind\pro\gS\pro
\mqty[a&b\\c&d]$
$\pro\mathbb{P}$
$\dd{x}$

  \[
    \alpha(x)=\left\{
                \begin{array}{ll}
                  x\\
                  \frac{1}{1+e^{-kx}}\\
                  \frac{e^x-e^{-x}}{e^x+e^{-x}}
                \end{array}
              \right.
  \]

  $\expval{x}$
  
  $\chi_\rho(ghg\dmo)=\Tr(\rho_{ghg\dmo})=\Tr(\rho_g\circ\rho_h\circ\rho\dmo_g)=\Tr(\rho_h)\overset{\mbox{\scalebox{0.5}{$\Tr(AB)=\Tr(BA)$}}}{=}\chi_\rho(h)$
  	$\mathop{\oplus}_{\substack{x\in X}}$

$\mat(\rho_g)=(a_{ij}(g))_{\scriptsize \substack{1\leq i\leq d \\ 1\leq j\leq d}}$ et $\mat(\rho'_g)=(a'_{ij}(g))_{\scriptsize \substack{1\leq i'\leq d' \\ 1\leq j'\leq d'}}$



\[\int_a^b{\mathbb{R}^2}g(u, v)\dd{P_{XY}}(u, v)=\iint g(u,v) f_{XY}(u, v)\dd \lambda(u) \dd \lambda(v)\]
$$\lim_{x\to\infty} f(x)$$	
$$\iiiint_V \mu(t,u,v,w) \,dt\,du\,dv\,dw$$
$$\sum_{n=1}^{\infty} 2^{-n} = 1$$	
\begin{definition}
	Si $X$ et $Y$ sont 2 v.a. ou definit la \textsc{Covariance} entre $X$ et $Y$ comme
	$\cov(X,Y)\overset{\text{def}}{=}\E\left[(X-\E(X))(Y-\E(Y))\right]=\E(XY)-\E(X)\E(Y)$.
\end{definition}
\fi
\pagebreak

% \tableofcontents

% insert your code here
%\input{./algebra/main.tex}
%\input{./geometrie-differentielle/main.tex}
%\input{./probabilite/main.tex}
%\input{./analyse-fonctionnelle/main.tex}
% \input{./Analyse-convexe-et-dualite-en-optimisation/main.tex}
%\input{./tikz/main.tex}
%\input{./Theorie-du-distributions/main.tex}
%\input{./optimisation/mine.tex}
 \input{./modelisation/main.tex}

% yves.aubry@univ-tln.fr : algebra

\end{document}

%% !TEX encoding = UTF-8 Unicode
% !TEX TS-program = xelatex

\documentclass[french]{report}

%\usepackage[utf8]{inputenc}
%\usepackage[T1]{fontenc}
\usepackage{babel}


\newif\ifcomment
%\commenttrue # Show comments

\usepackage{physics}
\usepackage{amssymb}


\usepackage{amsthm}
% \usepackage{thmtools}
\usepackage{mathtools}
\usepackage{amsfonts}

\usepackage{color}

\usepackage{tikz}

\usepackage{geometry}
\geometry{a5paper, margin=0.1in, right=1cm}

\usepackage{dsfont}

\usepackage{graphicx}
\graphicspath{ {images/} }

\usepackage{faktor}

\usepackage{IEEEtrantools}
\usepackage{enumerate}   
\usepackage[PostScript=dvips]{"/Users/aware/Documents/Courses/diagrams"}


\newtheorem{theorem}{Théorème}[section]
\renewcommand{\thetheorem}{\arabic{theorem}}
\newtheorem{lemme}{Lemme}[section]
\renewcommand{\thelemme}{\arabic{lemme}}
\newtheorem{proposition}{Proposition}[section]
\renewcommand{\theproposition}{\arabic{proposition}}
\newtheorem{notations}{Notations}[section]
\newtheorem{problem}{Problème}[section]
\newtheorem{corollary}{Corollaire}[theorem]
\renewcommand{\thecorollary}{\arabic{corollary}}
\newtheorem{property}{Propriété}[section]
\newtheorem{objective}{Objectif}[section]

\theoremstyle{definition}
\newtheorem{definition}{Définition}[section]
\renewcommand{\thedefinition}{\arabic{definition}}
\newtheorem{exercise}{Exercice}[chapter]
\renewcommand{\theexercise}{\arabic{exercise}}
\newtheorem{example}{Exemple}[chapter]
\renewcommand{\theexample}{\arabic{example}}
\newtheorem*{solution}{Solution}
\newtheorem*{application}{Application}
\newtheorem*{notation}{Notation}
\newtheorem*{vocabulary}{Vocabulaire}
\newtheorem*{properties}{Propriétés}



\theoremstyle{remark}
\newtheorem*{remark}{Remarque}
\newtheorem*{rappel}{Rappel}


\usepackage{etoolbox}
\AtBeginEnvironment{exercise}{\small}
\AtBeginEnvironment{example}{\small}

\usepackage{cases}
\usepackage[red]{mypack}

\usepackage[framemethod=TikZ]{mdframed}

\definecolor{bg}{rgb}{0.4,0.25,0.95}
\definecolor{pagebg}{rgb}{0,0,0.5}
\surroundwithmdframed[
   topline=false,
   rightline=false,
   bottomline=false,
   leftmargin=\parindent,
   skipabove=8pt,
   skipbelow=8pt,
   linecolor=blue,
   innerbottommargin=10pt,
   % backgroundcolor=bg,font=\color{orange}\sffamily, fontcolor=white
]{definition}

\usepackage{empheq}
\usepackage[most]{tcolorbox}

\newtcbox{\mymath}[1][]{%
    nobeforeafter, math upper, tcbox raise base,
    enhanced, colframe=blue!30!black,
    colback=red!10, boxrule=1pt,
    #1}

\usepackage{unixode}


\DeclareMathOperator{\ord}{ord}
\DeclareMathOperator{\orb}{orb}
\DeclareMathOperator{\stab}{stab}
\DeclareMathOperator{\Stab}{stab}
\DeclareMathOperator{\ppcm}{ppcm}
\DeclareMathOperator{\conj}{Conj}
\DeclareMathOperator{\End}{End}
\DeclareMathOperator{\rot}{rot}
\DeclareMathOperator{\trs}{trace}
\DeclareMathOperator{\Ind}{Ind}
\DeclareMathOperator{\mat}{Mat}
\DeclareMathOperator{\id}{Id}
\DeclareMathOperator{\vect}{vect}
\DeclareMathOperator{\img}{img}
\DeclareMathOperator{\cov}{Cov}
\DeclareMathOperator{\dist}{dist}
\DeclareMathOperator{\irr}{Irr}
\DeclareMathOperator{\image}{Im}
\DeclareMathOperator{\pd}{\partial}
\DeclareMathOperator{\epi}{epi}
\DeclareMathOperator{\Argmin}{Argmin}
\DeclareMathOperator{\dom}{dom}
\DeclareMathOperator{\proj}{proj}
\DeclareMathOperator{\ctg}{ctg}
\DeclareMathOperator{\supp}{supp}
\DeclareMathOperator{\argmin}{argmin}
\DeclareMathOperator{\mult}{mult}
\DeclareMathOperator{\ch}{ch}
\DeclareMathOperator{\sh}{sh}
\DeclareMathOperator{\rang}{rang}
\DeclareMathOperator{\diam}{diam}
\DeclareMathOperator{\Epigraphe}{Epigraphe}




\usepackage{xcolor}
\everymath{\color{blue}}
%\everymath{\color[rgb]{0,1,1}}
%\pagecolor[rgb]{0,0,0.5}


\newcommand*{\pdtest}[3][]{\ensuremath{\frac{\partial^{#1} #2}{\partial #3}}}

\newcommand*{\deffunc}[6][]{\ensuremath{
\begin{array}{rcl}
#2 : #3 &\rightarrow& #4\\
#5 &\mapsto& #6
\end{array}
}}

\newcommand{\eqcolon}{\mathrel{\resizebox{\widthof{$\mathord{=}$}}{\height}{ $\!\!=\!\!\resizebox{1.2\width}{0.8\height}{\raisebox{0.23ex}{$\mathop{:}$}}\!\!$ }}}
\newcommand{\coloneq}{\mathrel{\resizebox{\widthof{$\mathord{=}$}}{\height}{ $\!\!\resizebox{1.2\width}{0.8\height}{\raisebox{0.23ex}{$\mathop{:}$}}\!\!=\!\!$ }}}
\newcommand{\eqcolonl}{\ensuremath{\mathrel{=\!\!\mathop{:}}}}
\newcommand{\coloneql}{\ensuremath{\mathrel{\mathop{:} \!\! =}}}
\newcommand{\vc}[1]{% inline column vector
  \left(\begin{smallmatrix}#1\end{smallmatrix}\right)%
}
\newcommand{\vr}[1]{% inline row vector
  \begin{smallmatrix}(\,#1\,)\end{smallmatrix}%
}
\makeatletter
\newcommand*{\defeq}{\ =\mathrel{\rlap{%
                     \raisebox{0.3ex}{$\m@th\cdot$}}%
                     \raisebox{-0.3ex}{$\m@th\cdot$}}%
                     }
\makeatother

\newcommand{\mathcircle}[1]{% inline row vector
 \overset{\circ}{#1}
}
\newcommand{\ulim}{% low limit
 \underline{\lim}
}
\newcommand{\ssi}{% iff
\iff
}
\newcommand{\ps}[2]{
\expval{#1 | #2}
}
\newcommand{\df}[1]{
\mqty{#1}
}
\newcommand{\n}[1]{
\norm{#1}
}
\newcommand{\sys}[1]{
\left\{\smqty{#1}\right.
}


\newcommand{\eqdef}{\ensuremath{\overset{\text{def}}=}}


\def\Circlearrowright{\ensuremath{%
  \rotatebox[origin=c]{230}{$\circlearrowright$}}}

\newcommand\ct[1]{\text{\rmfamily\upshape #1}}
\newcommand\question[1]{ {\color{red} ...!? \small #1}}
\newcommand\caz[1]{\left\{\begin{array} #1 \end{array}\right.}
\newcommand\const{\text{\rmfamily\upshape const}}
\newcommand\toP{ \overset{\pro}{\to}}
\newcommand\toPP{ \overset{\text{PP}}{\to}}
\newcommand{\oeq}{\mathrel{\text{\textcircled{$=$}}}}





\usepackage{xcolor}
% \usepackage[normalem]{ulem}
\usepackage{lipsum}
\makeatletter
% \newcommand\colorwave[1][blue]{\bgroup \markoverwith{\lower3.5\p@\hbox{\sixly \textcolor{#1}{\char58}}}\ULon}
%\font\sixly=lasy6 % does not re-load if already loaded, so no memory problem.

\newmdtheoremenv[
linewidth= 1pt,linecolor= blue,%
leftmargin=20,rightmargin=20,innertopmargin=0pt, innerrightmargin=40,%
tikzsetting = { draw=lightgray, line width = 0.3pt,dashed,%
dash pattern = on 15pt off 3pt},%
splittopskip=\topskip,skipbelow=\baselineskip,%
skipabove=\baselineskip,ntheorem,roundcorner=0pt,
% backgroundcolor=pagebg,font=\color{orange}\sffamily, fontcolor=white
]{examplebox}{Exemple}[section]



\newcommand\R{\mathbb{R}}
\newcommand\Z{\mathbb{Z}}
\newcommand\N{\mathbb{N}}
\newcommand\E{\mathbb{E}}
\newcommand\F{\mathcal{F}}
\newcommand\cH{\mathcal{H}}
\newcommand\V{\mathbb{V}}
\newcommand\dmo{ ^{-1} }
\newcommand\kapa{\kappa}
\newcommand\im{Im}
\newcommand\hs{\mathcal{H}}





\usepackage{soul}

\makeatletter
\newcommand*{\whiten}[1]{\llap{\textcolor{white}{{\the\SOUL@token}}\hspace{#1pt}}}
\DeclareRobustCommand*\myul{%
    \def\SOUL@everyspace{\underline{\space}\kern\z@}%
    \def\SOUL@everytoken{%
     \setbox0=\hbox{\the\SOUL@token}%
     \ifdim\dp0>\z@
        \raisebox{\dp0}{\underline{\phantom{\the\SOUL@token}}}%
        \whiten{1}\whiten{0}%
        \whiten{-1}\whiten{-2}%
        \llap{\the\SOUL@token}%
     \else
        \underline{\the\SOUL@token}%
     \fi}%
\SOUL@}
\makeatother

\newcommand*{\demp}{\fontfamily{lmtt}\selectfont}

\DeclareTextFontCommand{\textdemp}{\demp}

\begin{document}

\ifcomment
Multiline
comment
\fi
\ifcomment
\myul{Typesetting test}
% \color[rgb]{1,1,1}
$∑_i^n≠ 60º±∞π∆¬≈√j∫h≤≥µ$

$\CR \R\pro\ind\pro\gS\pro
\mqty[a&b\\c&d]$
$\pro\mathbb{P}$
$\dd{x}$

  \[
    \alpha(x)=\left\{
                \begin{array}{ll}
                  x\\
                  \frac{1}{1+e^{-kx}}\\
                  \frac{e^x-e^{-x}}{e^x+e^{-x}}
                \end{array}
              \right.
  \]

  $\expval{x}$
  
  $\chi_\rho(ghg\dmo)=\Tr(\rho_{ghg\dmo})=\Tr(\rho_g\circ\rho_h\circ\rho\dmo_g)=\Tr(\rho_h)\overset{\mbox{\scalebox{0.5}{$\Tr(AB)=\Tr(BA)$}}}{=}\chi_\rho(h)$
  	$\mathop{\oplus}_{\substack{x\in X}}$

$\mat(\rho_g)=(a_{ij}(g))_{\scriptsize \substack{1\leq i\leq d \\ 1\leq j\leq d}}$ et $\mat(\rho'_g)=(a'_{ij}(g))_{\scriptsize \substack{1\leq i'\leq d' \\ 1\leq j'\leq d'}}$



\[\int_a^b{\mathbb{R}^2}g(u, v)\dd{P_{XY}}(u, v)=\iint g(u,v) f_{XY}(u, v)\dd \lambda(u) \dd \lambda(v)\]
$$\lim_{x\to\infty} f(x)$$	
$$\iiiint_V \mu(t,u,v,w) \,dt\,du\,dv\,dw$$
$$\sum_{n=1}^{\infty} 2^{-n} = 1$$	
\begin{definition}
	Si $X$ et $Y$ sont 2 v.a. ou definit la \textsc{Covariance} entre $X$ et $Y$ comme
	$\cov(X,Y)\overset{\text{def}}{=}\E\left[(X-\E(X))(Y-\E(Y))\right]=\E(XY)-\E(X)\E(Y)$.
\end{definition}
\fi
\pagebreak

% \tableofcontents

% insert your code here
%\input{./algebra/main.tex}
%\input{./geometrie-differentielle/main.tex}
%\input{./probabilite/main.tex}
%\input{./analyse-fonctionnelle/main.tex}
% \input{./Analyse-convexe-et-dualite-en-optimisation/main.tex}
%\input{./tikz/main.tex}
%\input{./Theorie-du-distributions/main.tex}
%\input{./optimisation/mine.tex}
 \input{./modelisation/main.tex}

% yves.aubry@univ-tln.fr : algebra

\end{document}

% % !TEX encoding = UTF-8 Unicode
% !TEX TS-program = xelatex

\documentclass[french]{report}

%\usepackage[utf8]{inputenc}
%\usepackage[T1]{fontenc}
\usepackage{babel}


\newif\ifcomment
%\commenttrue # Show comments

\usepackage{physics}
\usepackage{amssymb}


\usepackage{amsthm}
% \usepackage{thmtools}
\usepackage{mathtools}
\usepackage{amsfonts}

\usepackage{color}

\usepackage{tikz}

\usepackage{geometry}
\geometry{a5paper, margin=0.1in, right=1cm}

\usepackage{dsfont}

\usepackage{graphicx}
\graphicspath{ {images/} }

\usepackage{faktor}

\usepackage{IEEEtrantools}
\usepackage{enumerate}   
\usepackage[PostScript=dvips]{"/Users/aware/Documents/Courses/diagrams"}


\newtheorem{theorem}{Théorème}[section]
\renewcommand{\thetheorem}{\arabic{theorem}}
\newtheorem{lemme}{Lemme}[section]
\renewcommand{\thelemme}{\arabic{lemme}}
\newtheorem{proposition}{Proposition}[section]
\renewcommand{\theproposition}{\arabic{proposition}}
\newtheorem{notations}{Notations}[section]
\newtheorem{problem}{Problème}[section]
\newtheorem{corollary}{Corollaire}[theorem]
\renewcommand{\thecorollary}{\arabic{corollary}}
\newtheorem{property}{Propriété}[section]
\newtheorem{objective}{Objectif}[section]

\theoremstyle{definition}
\newtheorem{definition}{Définition}[section]
\renewcommand{\thedefinition}{\arabic{definition}}
\newtheorem{exercise}{Exercice}[chapter]
\renewcommand{\theexercise}{\arabic{exercise}}
\newtheorem{example}{Exemple}[chapter]
\renewcommand{\theexample}{\arabic{example}}
\newtheorem*{solution}{Solution}
\newtheorem*{application}{Application}
\newtheorem*{notation}{Notation}
\newtheorem*{vocabulary}{Vocabulaire}
\newtheorem*{properties}{Propriétés}



\theoremstyle{remark}
\newtheorem*{remark}{Remarque}
\newtheorem*{rappel}{Rappel}


\usepackage{etoolbox}
\AtBeginEnvironment{exercise}{\small}
\AtBeginEnvironment{example}{\small}

\usepackage{cases}
\usepackage[red]{mypack}

\usepackage[framemethod=TikZ]{mdframed}

\definecolor{bg}{rgb}{0.4,0.25,0.95}
\definecolor{pagebg}{rgb}{0,0,0.5}
\surroundwithmdframed[
   topline=false,
   rightline=false,
   bottomline=false,
   leftmargin=\parindent,
   skipabove=8pt,
   skipbelow=8pt,
   linecolor=blue,
   innerbottommargin=10pt,
   % backgroundcolor=bg,font=\color{orange}\sffamily, fontcolor=white
]{definition}

\usepackage{empheq}
\usepackage[most]{tcolorbox}

\newtcbox{\mymath}[1][]{%
    nobeforeafter, math upper, tcbox raise base,
    enhanced, colframe=blue!30!black,
    colback=red!10, boxrule=1pt,
    #1}

\usepackage{unixode}


\DeclareMathOperator{\ord}{ord}
\DeclareMathOperator{\orb}{orb}
\DeclareMathOperator{\stab}{stab}
\DeclareMathOperator{\Stab}{stab}
\DeclareMathOperator{\ppcm}{ppcm}
\DeclareMathOperator{\conj}{Conj}
\DeclareMathOperator{\End}{End}
\DeclareMathOperator{\rot}{rot}
\DeclareMathOperator{\trs}{trace}
\DeclareMathOperator{\Ind}{Ind}
\DeclareMathOperator{\mat}{Mat}
\DeclareMathOperator{\id}{Id}
\DeclareMathOperator{\vect}{vect}
\DeclareMathOperator{\img}{img}
\DeclareMathOperator{\cov}{Cov}
\DeclareMathOperator{\dist}{dist}
\DeclareMathOperator{\irr}{Irr}
\DeclareMathOperator{\image}{Im}
\DeclareMathOperator{\pd}{\partial}
\DeclareMathOperator{\epi}{epi}
\DeclareMathOperator{\Argmin}{Argmin}
\DeclareMathOperator{\dom}{dom}
\DeclareMathOperator{\proj}{proj}
\DeclareMathOperator{\ctg}{ctg}
\DeclareMathOperator{\supp}{supp}
\DeclareMathOperator{\argmin}{argmin}
\DeclareMathOperator{\mult}{mult}
\DeclareMathOperator{\ch}{ch}
\DeclareMathOperator{\sh}{sh}
\DeclareMathOperator{\rang}{rang}
\DeclareMathOperator{\diam}{diam}
\DeclareMathOperator{\Epigraphe}{Epigraphe}




\usepackage{xcolor}
\everymath{\color{blue}}
%\everymath{\color[rgb]{0,1,1}}
%\pagecolor[rgb]{0,0,0.5}


\newcommand*{\pdtest}[3][]{\ensuremath{\frac{\partial^{#1} #2}{\partial #3}}}

\newcommand*{\deffunc}[6][]{\ensuremath{
\begin{array}{rcl}
#2 : #3 &\rightarrow& #4\\
#5 &\mapsto& #6
\end{array}
}}

\newcommand{\eqcolon}{\mathrel{\resizebox{\widthof{$\mathord{=}$}}{\height}{ $\!\!=\!\!\resizebox{1.2\width}{0.8\height}{\raisebox{0.23ex}{$\mathop{:}$}}\!\!$ }}}
\newcommand{\coloneq}{\mathrel{\resizebox{\widthof{$\mathord{=}$}}{\height}{ $\!\!\resizebox{1.2\width}{0.8\height}{\raisebox{0.23ex}{$\mathop{:}$}}\!\!=\!\!$ }}}
\newcommand{\eqcolonl}{\ensuremath{\mathrel{=\!\!\mathop{:}}}}
\newcommand{\coloneql}{\ensuremath{\mathrel{\mathop{:} \!\! =}}}
\newcommand{\vc}[1]{% inline column vector
  \left(\begin{smallmatrix}#1\end{smallmatrix}\right)%
}
\newcommand{\vr}[1]{% inline row vector
  \begin{smallmatrix}(\,#1\,)\end{smallmatrix}%
}
\makeatletter
\newcommand*{\defeq}{\ =\mathrel{\rlap{%
                     \raisebox{0.3ex}{$\m@th\cdot$}}%
                     \raisebox{-0.3ex}{$\m@th\cdot$}}%
                     }
\makeatother

\newcommand{\mathcircle}[1]{% inline row vector
 \overset{\circ}{#1}
}
\newcommand{\ulim}{% low limit
 \underline{\lim}
}
\newcommand{\ssi}{% iff
\iff
}
\newcommand{\ps}[2]{
\expval{#1 | #2}
}
\newcommand{\df}[1]{
\mqty{#1}
}
\newcommand{\n}[1]{
\norm{#1}
}
\newcommand{\sys}[1]{
\left\{\smqty{#1}\right.
}


\newcommand{\eqdef}{\ensuremath{\overset{\text{def}}=}}


\def\Circlearrowright{\ensuremath{%
  \rotatebox[origin=c]{230}{$\circlearrowright$}}}

\newcommand\ct[1]{\text{\rmfamily\upshape #1}}
\newcommand\question[1]{ {\color{red} ...!? \small #1}}
\newcommand\caz[1]{\left\{\begin{array} #1 \end{array}\right.}
\newcommand\const{\text{\rmfamily\upshape const}}
\newcommand\toP{ \overset{\pro}{\to}}
\newcommand\toPP{ \overset{\text{PP}}{\to}}
\newcommand{\oeq}{\mathrel{\text{\textcircled{$=$}}}}





\usepackage{xcolor}
% \usepackage[normalem]{ulem}
\usepackage{lipsum}
\makeatletter
% \newcommand\colorwave[1][blue]{\bgroup \markoverwith{\lower3.5\p@\hbox{\sixly \textcolor{#1}{\char58}}}\ULon}
%\font\sixly=lasy6 % does not re-load if already loaded, so no memory problem.

\newmdtheoremenv[
linewidth= 1pt,linecolor= blue,%
leftmargin=20,rightmargin=20,innertopmargin=0pt, innerrightmargin=40,%
tikzsetting = { draw=lightgray, line width = 0.3pt,dashed,%
dash pattern = on 15pt off 3pt},%
splittopskip=\topskip,skipbelow=\baselineskip,%
skipabove=\baselineskip,ntheorem,roundcorner=0pt,
% backgroundcolor=pagebg,font=\color{orange}\sffamily, fontcolor=white
]{examplebox}{Exemple}[section]



\newcommand\R{\mathbb{R}}
\newcommand\Z{\mathbb{Z}}
\newcommand\N{\mathbb{N}}
\newcommand\E{\mathbb{E}}
\newcommand\F{\mathcal{F}}
\newcommand\cH{\mathcal{H}}
\newcommand\V{\mathbb{V}}
\newcommand\dmo{ ^{-1} }
\newcommand\kapa{\kappa}
\newcommand\im{Im}
\newcommand\hs{\mathcal{H}}





\usepackage{soul}

\makeatletter
\newcommand*{\whiten}[1]{\llap{\textcolor{white}{{\the\SOUL@token}}\hspace{#1pt}}}
\DeclareRobustCommand*\myul{%
    \def\SOUL@everyspace{\underline{\space}\kern\z@}%
    \def\SOUL@everytoken{%
     \setbox0=\hbox{\the\SOUL@token}%
     \ifdim\dp0>\z@
        \raisebox{\dp0}{\underline{\phantom{\the\SOUL@token}}}%
        \whiten{1}\whiten{0}%
        \whiten{-1}\whiten{-2}%
        \llap{\the\SOUL@token}%
     \else
        \underline{\the\SOUL@token}%
     \fi}%
\SOUL@}
\makeatother

\newcommand*{\demp}{\fontfamily{lmtt}\selectfont}

\DeclareTextFontCommand{\textdemp}{\demp}

\begin{document}

\ifcomment
Multiline
comment
\fi
\ifcomment
\myul{Typesetting test}
% \color[rgb]{1,1,1}
$∑_i^n≠ 60º±∞π∆¬≈√j∫h≤≥µ$

$\CR \R\pro\ind\pro\gS\pro
\mqty[a&b\\c&d]$
$\pro\mathbb{P}$
$\dd{x}$

  \[
    \alpha(x)=\left\{
                \begin{array}{ll}
                  x\\
                  \frac{1}{1+e^{-kx}}\\
                  \frac{e^x-e^{-x}}{e^x+e^{-x}}
                \end{array}
              \right.
  \]

  $\expval{x}$
  
  $\chi_\rho(ghg\dmo)=\Tr(\rho_{ghg\dmo})=\Tr(\rho_g\circ\rho_h\circ\rho\dmo_g)=\Tr(\rho_h)\overset{\mbox{\scalebox{0.5}{$\Tr(AB)=\Tr(BA)$}}}{=}\chi_\rho(h)$
  	$\mathop{\oplus}_{\substack{x\in X}}$

$\mat(\rho_g)=(a_{ij}(g))_{\scriptsize \substack{1\leq i\leq d \\ 1\leq j\leq d}}$ et $\mat(\rho'_g)=(a'_{ij}(g))_{\scriptsize \substack{1\leq i'\leq d' \\ 1\leq j'\leq d'}}$



\[\int_a^b{\mathbb{R}^2}g(u, v)\dd{P_{XY}}(u, v)=\iint g(u,v) f_{XY}(u, v)\dd \lambda(u) \dd \lambda(v)\]
$$\lim_{x\to\infty} f(x)$$	
$$\iiiint_V \mu(t,u,v,w) \,dt\,du\,dv\,dw$$
$$\sum_{n=1}^{\infty} 2^{-n} = 1$$	
\begin{definition}
	Si $X$ et $Y$ sont 2 v.a. ou definit la \textsc{Covariance} entre $X$ et $Y$ comme
	$\cov(X,Y)\overset{\text{def}}{=}\E\left[(X-\E(X))(Y-\E(Y))\right]=\E(XY)-\E(X)\E(Y)$.
\end{definition}
\fi
\pagebreak

% \tableofcontents

% insert your code here
%\input{./algebra/main.tex}
%\input{./geometrie-differentielle/main.tex}
%\input{./probabilite/main.tex}
%\input{./analyse-fonctionnelle/main.tex}
% \input{./Analyse-convexe-et-dualite-en-optimisation/main.tex}
%\input{./tikz/main.tex}
%\input{./Theorie-du-distributions/main.tex}
%\input{./optimisation/mine.tex}
 \input{./modelisation/main.tex}

% yves.aubry@univ-tln.fr : algebra

\end{document}

%% !TEX encoding = UTF-8 Unicode
% !TEX TS-program = xelatex

\documentclass[french]{report}

%\usepackage[utf8]{inputenc}
%\usepackage[T1]{fontenc}
\usepackage{babel}


\newif\ifcomment
%\commenttrue # Show comments

\usepackage{physics}
\usepackage{amssymb}


\usepackage{amsthm}
% \usepackage{thmtools}
\usepackage{mathtools}
\usepackage{amsfonts}

\usepackage{color}

\usepackage{tikz}

\usepackage{geometry}
\geometry{a5paper, margin=0.1in, right=1cm}

\usepackage{dsfont}

\usepackage{graphicx}
\graphicspath{ {images/} }

\usepackage{faktor}

\usepackage{IEEEtrantools}
\usepackage{enumerate}   
\usepackage[PostScript=dvips]{"/Users/aware/Documents/Courses/diagrams"}


\newtheorem{theorem}{Théorème}[section]
\renewcommand{\thetheorem}{\arabic{theorem}}
\newtheorem{lemme}{Lemme}[section]
\renewcommand{\thelemme}{\arabic{lemme}}
\newtheorem{proposition}{Proposition}[section]
\renewcommand{\theproposition}{\arabic{proposition}}
\newtheorem{notations}{Notations}[section]
\newtheorem{problem}{Problème}[section]
\newtheorem{corollary}{Corollaire}[theorem]
\renewcommand{\thecorollary}{\arabic{corollary}}
\newtheorem{property}{Propriété}[section]
\newtheorem{objective}{Objectif}[section]

\theoremstyle{definition}
\newtheorem{definition}{Définition}[section]
\renewcommand{\thedefinition}{\arabic{definition}}
\newtheorem{exercise}{Exercice}[chapter]
\renewcommand{\theexercise}{\arabic{exercise}}
\newtheorem{example}{Exemple}[chapter]
\renewcommand{\theexample}{\arabic{example}}
\newtheorem*{solution}{Solution}
\newtheorem*{application}{Application}
\newtheorem*{notation}{Notation}
\newtheorem*{vocabulary}{Vocabulaire}
\newtheorem*{properties}{Propriétés}



\theoremstyle{remark}
\newtheorem*{remark}{Remarque}
\newtheorem*{rappel}{Rappel}


\usepackage{etoolbox}
\AtBeginEnvironment{exercise}{\small}
\AtBeginEnvironment{example}{\small}

\usepackage{cases}
\usepackage[red]{mypack}

\usepackage[framemethod=TikZ]{mdframed}

\definecolor{bg}{rgb}{0.4,0.25,0.95}
\definecolor{pagebg}{rgb}{0,0,0.5}
\surroundwithmdframed[
   topline=false,
   rightline=false,
   bottomline=false,
   leftmargin=\parindent,
   skipabove=8pt,
   skipbelow=8pt,
   linecolor=blue,
   innerbottommargin=10pt,
   % backgroundcolor=bg,font=\color{orange}\sffamily, fontcolor=white
]{definition}

\usepackage{empheq}
\usepackage[most]{tcolorbox}

\newtcbox{\mymath}[1][]{%
    nobeforeafter, math upper, tcbox raise base,
    enhanced, colframe=blue!30!black,
    colback=red!10, boxrule=1pt,
    #1}

\usepackage{unixode}


\DeclareMathOperator{\ord}{ord}
\DeclareMathOperator{\orb}{orb}
\DeclareMathOperator{\stab}{stab}
\DeclareMathOperator{\Stab}{stab}
\DeclareMathOperator{\ppcm}{ppcm}
\DeclareMathOperator{\conj}{Conj}
\DeclareMathOperator{\End}{End}
\DeclareMathOperator{\rot}{rot}
\DeclareMathOperator{\trs}{trace}
\DeclareMathOperator{\Ind}{Ind}
\DeclareMathOperator{\mat}{Mat}
\DeclareMathOperator{\id}{Id}
\DeclareMathOperator{\vect}{vect}
\DeclareMathOperator{\img}{img}
\DeclareMathOperator{\cov}{Cov}
\DeclareMathOperator{\dist}{dist}
\DeclareMathOperator{\irr}{Irr}
\DeclareMathOperator{\image}{Im}
\DeclareMathOperator{\pd}{\partial}
\DeclareMathOperator{\epi}{epi}
\DeclareMathOperator{\Argmin}{Argmin}
\DeclareMathOperator{\dom}{dom}
\DeclareMathOperator{\proj}{proj}
\DeclareMathOperator{\ctg}{ctg}
\DeclareMathOperator{\supp}{supp}
\DeclareMathOperator{\argmin}{argmin}
\DeclareMathOperator{\mult}{mult}
\DeclareMathOperator{\ch}{ch}
\DeclareMathOperator{\sh}{sh}
\DeclareMathOperator{\rang}{rang}
\DeclareMathOperator{\diam}{diam}
\DeclareMathOperator{\Epigraphe}{Epigraphe}




\usepackage{xcolor}
\everymath{\color{blue}}
%\everymath{\color[rgb]{0,1,1}}
%\pagecolor[rgb]{0,0,0.5}


\newcommand*{\pdtest}[3][]{\ensuremath{\frac{\partial^{#1} #2}{\partial #3}}}

\newcommand*{\deffunc}[6][]{\ensuremath{
\begin{array}{rcl}
#2 : #3 &\rightarrow& #4\\
#5 &\mapsto& #6
\end{array}
}}

\newcommand{\eqcolon}{\mathrel{\resizebox{\widthof{$\mathord{=}$}}{\height}{ $\!\!=\!\!\resizebox{1.2\width}{0.8\height}{\raisebox{0.23ex}{$\mathop{:}$}}\!\!$ }}}
\newcommand{\coloneq}{\mathrel{\resizebox{\widthof{$\mathord{=}$}}{\height}{ $\!\!\resizebox{1.2\width}{0.8\height}{\raisebox{0.23ex}{$\mathop{:}$}}\!\!=\!\!$ }}}
\newcommand{\eqcolonl}{\ensuremath{\mathrel{=\!\!\mathop{:}}}}
\newcommand{\coloneql}{\ensuremath{\mathrel{\mathop{:} \!\! =}}}
\newcommand{\vc}[1]{% inline column vector
  \left(\begin{smallmatrix}#1\end{smallmatrix}\right)%
}
\newcommand{\vr}[1]{% inline row vector
  \begin{smallmatrix}(\,#1\,)\end{smallmatrix}%
}
\makeatletter
\newcommand*{\defeq}{\ =\mathrel{\rlap{%
                     \raisebox{0.3ex}{$\m@th\cdot$}}%
                     \raisebox{-0.3ex}{$\m@th\cdot$}}%
                     }
\makeatother

\newcommand{\mathcircle}[1]{% inline row vector
 \overset{\circ}{#1}
}
\newcommand{\ulim}{% low limit
 \underline{\lim}
}
\newcommand{\ssi}{% iff
\iff
}
\newcommand{\ps}[2]{
\expval{#1 | #2}
}
\newcommand{\df}[1]{
\mqty{#1}
}
\newcommand{\n}[1]{
\norm{#1}
}
\newcommand{\sys}[1]{
\left\{\smqty{#1}\right.
}


\newcommand{\eqdef}{\ensuremath{\overset{\text{def}}=}}


\def\Circlearrowright{\ensuremath{%
  \rotatebox[origin=c]{230}{$\circlearrowright$}}}

\newcommand\ct[1]{\text{\rmfamily\upshape #1}}
\newcommand\question[1]{ {\color{red} ...!? \small #1}}
\newcommand\caz[1]{\left\{\begin{array} #1 \end{array}\right.}
\newcommand\const{\text{\rmfamily\upshape const}}
\newcommand\toP{ \overset{\pro}{\to}}
\newcommand\toPP{ \overset{\text{PP}}{\to}}
\newcommand{\oeq}{\mathrel{\text{\textcircled{$=$}}}}





\usepackage{xcolor}
% \usepackage[normalem]{ulem}
\usepackage{lipsum}
\makeatletter
% \newcommand\colorwave[1][blue]{\bgroup \markoverwith{\lower3.5\p@\hbox{\sixly \textcolor{#1}{\char58}}}\ULon}
%\font\sixly=lasy6 % does not re-load if already loaded, so no memory problem.

\newmdtheoremenv[
linewidth= 1pt,linecolor= blue,%
leftmargin=20,rightmargin=20,innertopmargin=0pt, innerrightmargin=40,%
tikzsetting = { draw=lightgray, line width = 0.3pt,dashed,%
dash pattern = on 15pt off 3pt},%
splittopskip=\topskip,skipbelow=\baselineskip,%
skipabove=\baselineskip,ntheorem,roundcorner=0pt,
% backgroundcolor=pagebg,font=\color{orange}\sffamily, fontcolor=white
]{examplebox}{Exemple}[section]



\newcommand\R{\mathbb{R}}
\newcommand\Z{\mathbb{Z}}
\newcommand\N{\mathbb{N}}
\newcommand\E{\mathbb{E}}
\newcommand\F{\mathcal{F}}
\newcommand\cH{\mathcal{H}}
\newcommand\V{\mathbb{V}}
\newcommand\dmo{ ^{-1} }
\newcommand\kapa{\kappa}
\newcommand\im{Im}
\newcommand\hs{\mathcal{H}}





\usepackage{soul}

\makeatletter
\newcommand*{\whiten}[1]{\llap{\textcolor{white}{{\the\SOUL@token}}\hspace{#1pt}}}
\DeclareRobustCommand*\myul{%
    \def\SOUL@everyspace{\underline{\space}\kern\z@}%
    \def\SOUL@everytoken{%
     \setbox0=\hbox{\the\SOUL@token}%
     \ifdim\dp0>\z@
        \raisebox{\dp0}{\underline{\phantom{\the\SOUL@token}}}%
        \whiten{1}\whiten{0}%
        \whiten{-1}\whiten{-2}%
        \llap{\the\SOUL@token}%
     \else
        \underline{\the\SOUL@token}%
     \fi}%
\SOUL@}
\makeatother

\newcommand*{\demp}{\fontfamily{lmtt}\selectfont}

\DeclareTextFontCommand{\textdemp}{\demp}

\begin{document}

\ifcomment
Multiline
comment
\fi
\ifcomment
\myul{Typesetting test}
% \color[rgb]{1,1,1}
$∑_i^n≠ 60º±∞π∆¬≈√j∫h≤≥µ$

$\CR \R\pro\ind\pro\gS\pro
\mqty[a&b\\c&d]$
$\pro\mathbb{P}$
$\dd{x}$

  \[
    \alpha(x)=\left\{
                \begin{array}{ll}
                  x\\
                  \frac{1}{1+e^{-kx}}\\
                  \frac{e^x-e^{-x}}{e^x+e^{-x}}
                \end{array}
              \right.
  \]

  $\expval{x}$
  
  $\chi_\rho(ghg\dmo)=\Tr(\rho_{ghg\dmo})=\Tr(\rho_g\circ\rho_h\circ\rho\dmo_g)=\Tr(\rho_h)\overset{\mbox{\scalebox{0.5}{$\Tr(AB)=\Tr(BA)$}}}{=}\chi_\rho(h)$
  	$\mathop{\oplus}_{\substack{x\in X}}$

$\mat(\rho_g)=(a_{ij}(g))_{\scriptsize \substack{1\leq i\leq d \\ 1\leq j\leq d}}$ et $\mat(\rho'_g)=(a'_{ij}(g))_{\scriptsize \substack{1\leq i'\leq d' \\ 1\leq j'\leq d'}}$



\[\int_a^b{\mathbb{R}^2}g(u, v)\dd{P_{XY}}(u, v)=\iint g(u,v) f_{XY}(u, v)\dd \lambda(u) \dd \lambda(v)\]
$$\lim_{x\to\infty} f(x)$$	
$$\iiiint_V \mu(t,u,v,w) \,dt\,du\,dv\,dw$$
$$\sum_{n=1}^{\infty} 2^{-n} = 1$$	
\begin{definition}
	Si $X$ et $Y$ sont 2 v.a. ou definit la \textsc{Covariance} entre $X$ et $Y$ comme
	$\cov(X,Y)\overset{\text{def}}{=}\E\left[(X-\E(X))(Y-\E(Y))\right]=\E(XY)-\E(X)\E(Y)$.
\end{definition}
\fi
\pagebreak

% \tableofcontents

% insert your code here
%\input{./algebra/main.tex}
%\input{./geometrie-differentielle/main.tex}
%\input{./probabilite/main.tex}
%\input{./analyse-fonctionnelle/main.tex}
% \input{./Analyse-convexe-et-dualite-en-optimisation/main.tex}
%\input{./tikz/main.tex}
%\input{./Theorie-du-distributions/main.tex}
%\input{./optimisation/mine.tex}
 \input{./modelisation/main.tex}

% yves.aubry@univ-tln.fr : algebra

\end{document}

%% !TEX encoding = UTF-8 Unicode
% !TEX TS-program = xelatex

\documentclass[french]{report}

%\usepackage[utf8]{inputenc}
%\usepackage[T1]{fontenc}
\usepackage{babel}


\newif\ifcomment
%\commenttrue # Show comments

\usepackage{physics}
\usepackage{amssymb}


\usepackage{amsthm}
% \usepackage{thmtools}
\usepackage{mathtools}
\usepackage{amsfonts}

\usepackage{color}

\usepackage{tikz}

\usepackage{geometry}
\geometry{a5paper, margin=0.1in, right=1cm}

\usepackage{dsfont}

\usepackage{graphicx}
\graphicspath{ {images/} }

\usepackage{faktor}

\usepackage{IEEEtrantools}
\usepackage{enumerate}   
\usepackage[PostScript=dvips]{"/Users/aware/Documents/Courses/diagrams"}


\newtheorem{theorem}{Théorème}[section]
\renewcommand{\thetheorem}{\arabic{theorem}}
\newtheorem{lemme}{Lemme}[section]
\renewcommand{\thelemme}{\arabic{lemme}}
\newtheorem{proposition}{Proposition}[section]
\renewcommand{\theproposition}{\arabic{proposition}}
\newtheorem{notations}{Notations}[section]
\newtheorem{problem}{Problème}[section]
\newtheorem{corollary}{Corollaire}[theorem]
\renewcommand{\thecorollary}{\arabic{corollary}}
\newtheorem{property}{Propriété}[section]
\newtheorem{objective}{Objectif}[section]

\theoremstyle{definition}
\newtheorem{definition}{Définition}[section]
\renewcommand{\thedefinition}{\arabic{definition}}
\newtheorem{exercise}{Exercice}[chapter]
\renewcommand{\theexercise}{\arabic{exercise}}
\newtheorem{example}{Exemple}[chapter]
\renewcommand{\theexample}{\arabic{example}}
\newtheorem*{solution}{Solution}
\newtheorem*{application}{Application}
\newtheorem*{notation}{Notation}
\newtheorem*{vocabulary}{Vocabulaire}
\newtheorem*{properties}{Propriétés}



\theoremstyle{remark}
\newtheorem*{remark}{Remarque}
\newtheorem*{rappel}{Rappel}


\usepackage{etoolbox}
\AtBeginEnvironment{exercise}{\small}
\AtBeginEnvironment{example}{\small}

\usepackage{cases}
\usepackage[red]{mypack}

\usepackage[framemethod=TikZ]{mdframed}

\definecolor{bg}{rgb}{0.4,0.25,0.95}
\definecolor{pagebg}{rgb}{0,0,0.5}
\surroundwithmdframed[
   topline=false,
   rightline=false,
   bottomline=false,
   leftmargin=\parindent,
   skipabove=8pt,
   skipbelow=8pt,
   linecolor=blue,
   innerbottommargin=10pt,
   % backgroundcolor=bg,font=\color{orange}\sffamily, fontcolor=white
]{definition}

\usepackage{empheq}
\usepackage[most]{tcolorbox}

\newtcbox{\mymath}[1][]{%
    nobeforeafter, math upper, tcbox raise base,
    enhanced, colframe=blue!30!black,
    colback=red!10, boxrule=1pt,
    #1}

\usepackage{unixode}


\DeclareMathOperator{\ord}{ord}
\DeclareMathOperator{\orb}{orb}
\DeclareMathOperator{\stab}{stab}
\DeclareMathOperator{\Stab}{stab}
\DeclareMathOperator{\ppcm}{ppcm}
\DeclareMathOperator{\conj}{Conj}
\DeclareMathOperator{\End}{End}
\DeclareMathOperator{\rot}{rot}
\DeclareMathOperator{\trs}{trace}
\DeclareMathOperator{\Ind}{Ind}
\DeclareMathOperator{\mat}{Mat}
\DeclareMathOperator{\id}{Id}
\DeclareMathOperator{\vect}{vect}
\DeclareMathOperator{\img}{img}
\DeclareMathOperator{\cov}{Cov}
\DeclareMathOperator{\dist}{dist}
\DeclareMathOperator{\irr}{Irr}
\DeclareMathOperator{\image}{Im}
\DeclareMathOperator{\pd}{\partial}
\DeclareMathOperator{\epi}{epi}
\DeclareMathOperator{\Argmin}{Argmin}
\DeclareMathOperator{\dom}{dom}
\DeclareMathOperator{\proj}{proj}
\DeclareMathOperator{\ctg}{ctg}
\DeclareMathOperator{\supp}{supp}
\DeclareMathOperator{\argmin}{argmin}
\DeclareMathOperator{\mult}{mult}
\DeclareMathOperator{\ch}{ch}
\DeclareMathOperator{\sh}{sh}
\DeclareMathOperator{\rang}{rang}
\DeclareMathOperator{\diam}{diam}
\DeclareMathOperator{\Epigraphe}{Epigraphe}




\usepackage{xcolor}
\everymath{\color{blue}}
%\everymath{\color[rgb]{0,1,1}}
%\pagecolor[rgb]{0,0,0.5}


\newcommand*{\pdtest}[3][]{\ensuremath{\frac{\partial^{#1} #2}{\partial #3}}}

\newcommand*{\deffunc}[6][]{\ensuremath{
\begin{array}{rcl}
#2 : #3 &\rightarrow& #4\\
#5 &\mapsto& #6
\end{array}
}}

\newcommand{\eqcolon}{\mathrel{\resizebox{\widthof{$\mathord{=}$}}{\height}{ $\!\!=\!\!\resizebox{1.2\width}{0.8\height}{\raisebox{0.23ex}{$\mathop{:}$}}\!\!$ }}}
\newcommand{\coloneq}{\mathrel{\resizebox{\widthof{$\mathord{=}$}}{\height}{ $\!\!\resizebox{1.2\width}{0.8\height}{\raisebox{0.23ex}{$\mathop{:}$}}\!\!=\!\!$ }}}
\newcommand{\eqcolonl}{\ensuremath{\mathrel{=\!\!\mathop{:}}}}
\newcommand{\coloneql}{\ensuremath{\mathrel{\mathop{:} \!\! =}}}
\newcommand{\vc}[1]{% inline column vector
  \left(\begin{smallmatrix}#1\end{smallmatrix}\right)%
}
\newcommand{\vr}[1]{% inline row vector
  \begin{smallmatrix}(\,#1\,)\end{smallmatrix}%
}
\makeatletter
\newcommand*{\defeq}{\ =\mathrel{\rlap{%
                     \raisebox{0.3ex}{$\m@th\cdot$}}%
                     \raisebox{-0.3ex}{$\m@th\cdot$}}%
                     }
\makeatother

\newcommand{\mathcircle}[1]{% inline row vector
 \overset{\circ}{#1}
}
\newcommand{\ulim}{% low limit
 \underline{\lim}
}
\newcommand{\ssi}{% iff
\iff
}
\newcommand{\ps}[2]{
\expval{#1 | #2}
}
\newcommand{\df}[1]{
\mqty{#1}
}
\newcommand{\n}[1]{
\norm{#1}
}
\newcommand{\sys}[1]{
\left\{\smqty{#1}\right.
}


\newcommand{\eqdef}{\ensuremath{\overset{\text{def}}=}}


\def\Circlearrowright{\ensuremath{%
  \rotatebox[origin=c]{230}{$\circlearrowright$}}}

\newcommand\ct[1]{\text{\rmfamily\upshape #1}}
\newcommand\question[1]{ {\color{red} ...!? \small #1}}
\newcommand\caz[1]{\left\{\begin{array} #1 \end{array}\right.}
\newcommand\const{\text{\rmfamily\upshape const}}
\newcommand\toP{ \overset{\pro}{\to}}
\newcommand\toPP{ \overset{\text{PP}}{\to}}
\newcommand{\oeq}{\mathrel{\text{\textcircled{$=$}}}}





\usepackage{xcolor}
% \usepackage[normalem]{ulem}
\usepackage{lipsum}
\makeatletter
% \newcommand\colorwave[1][blue]{\bgroup \markoverwith{\lower3.5\p@\hbox{\sixly \textcolor{#1}{\char58}}}\ULon}
%\font\sixly=lasy6 % does not re-load if already loaded, so no memory problem.

\newmdtheoremenv[
linewidth= 1pt,linecolor= blue,%
leftmargin=20,rightmargin=20,innertopmargin=0pt, innerrightmargin=40,%
tikzsetting = { draw=lightgray, line width = 0.3pt,dashed,%
dash pattern = on 15pt off 3pt},%
splittopskip=\topskip,skipbelow=\baselineskip,%
skipabove=\baselineskip,ntheorem,roundcorner=0pt,
% backgroundcolor=pagebg,font=\color{orange}\sffamily, fontcolor=white
]{examplebox}{Exemple}[section]



\newcommand\R{\mathbb{R}}
\newcommand\Z{\mathbb{Z}}
\newcommand\N{\mathbb{N}}
\newcommand\E{\mathbb{E}}
\newcommand\F{\mathcal{F}}
\newcommand\cH{\mathcal{H}}
\newcommand\V{\mathbb{V}}
\newcommand\dmo{ ^{-1} }
\newcommand\kapa{\kappa}
\newcommand\im{Im}
\newcommand\hs{\mathcal{H}}





\usepackage{soul}

\makeatletter
\newcommand*{\whiten}[1]{\llap{\textcolor{white}{{\the\SOUL@token}}\hspace{#1pt}}}
\DeclareRobustCommand*\myul{%
    \def\SOUL@everyspace{\underline{\space}\kern\z@}%
    \def\SOUL@everytoken{%
     \setbox0=\hbox{\the\SOUL@token}%
     \ifdim\dp0>\z@
        \raisebox{\dp0}{\underline{\phantom{\the\SOUL@token}}}%
        \whiten{1}\whiten{0}%
        \whiten{-1}\whiten{-2}%
        \llap{\the\SOUL@token}%
     \else
        \underline{\the\SOUL@token}%
     \fi}%
\SOUL@}
\makeatother

\newcommand*{\demp}{\fontfamily{lmtt}\selectfont}

\DeclareTextFontCommand{\textdemp}{\demp}

\begin{document}

\ifcomment
Multiline
comment
\fi
\ifcomment
\myul{Typesetting test}
% \color[rgb]{1,1,1}
$∑_i^n≠ 60º±∞π∆¬≈√j∫h≤≥µ$

$\CR \R\pro\ind\pro\gS\pro
\mqty[a&b\\c&d]$
$\pro\mathbb{P}$
$\dd{x}$

  \[
    \alpha(x)=\left\{
                \begin{array}{ll}
                  x\\
                  \frac{1}{1+e^{-kx}}\\
                  \frac{e^x-e^{-x}}{e^x+e^{-x}}
                \end{array}
              \right.
  \]

  $\expval{x}$
  
  $\chi_\rho(ghg\dmo)=\Tr(\rho_{ghg\dmo})=\Tr(\rho_g\circ\rho_h\circ\rho\dmo_g)=\Tr(\rho_h)\overset{\mbox{\scalebox{0.5}{$\Tr(AB)=\Tr(BA)$}}}{=}\chi_\rho(h)$
  	$\mathop{\oplus}_{\substack{x\in X}}$

$\mat(\rho_g)=(a_{ij}(g))_{\scriptsize \substack{1\leq i\leq d \\ 1\leq j\leq d}}$ et $\mat(\rho'_g)=(a'_{ij}(g))_{\scriptsize \substack{1\leq i'\leq d' \\ 1\leq j'\leq d'}}$



\[\int_a^b{\mathbb{R}^2}g(u, v)\dd{P_{XY}}(u, v)=\iint g(u,v) f_{XY}(u, v)\dd \lambda(u) \dd \lambda(v)\]
$$\lim_{x\to\infty} f(x)$$	
$$\iiiint_V \mu(t,u,v,w) \,dt\,du\,dv\,dw$$
$$\sum_{n=1}^{\infty} 2^{-n} = 1$$	
\begin{definition}
	Si $X$ et $Y$ sont 2 v.a. ou definit la \textsc{Covariance} entre $X$ et $Y$ comme
	$\cov(X,Y)\overset{\text{def}}{=}\E\left[(X-\E(X))(Y-\E(Y))\right]=\E(XY)-\E(X)\E(Y)$.
\end{definition}
\fi
\pagebreak

% \tableofcontents

% insert your code here
%\input{./algebra/main.tex}
%\input{./geometrie-differentielle/main.tex}
%\input{./probabilite/main.tex}
%\input{./analyse-fonctionnelle/main.tex}
% \input{./Analyse-convexe-et-dualite-en-optimisation/main.tex}
%\input{./tikz/main.tex}
%\input{./Theorie-du-distributions/main.tex}
%\input{./optimisation/mine.tex}
 \input{./modelisation/main.tex}

% yves.aubry@univ-tln.fr : algebra

\end{document}

%\input{./optimisation/mine.tex}
 % !TEX encoding = UTF-8 Unicode
% !TEX TS-program = xelatex

\documentclass[french]{report}

%\usepackage[utf8]{inputenc}
%\usepackage[T1]{fontenc}
\usepackage{babel}


\newif\ifcomment
%\commenttrue # Show comments

\usepackage{physics}
\usepackage{amssymb}


\usepackage{amsthm}
% \usepackage{thmtools}
\usepackage{mathtools}
\usepackage{amsfonts}

\usepackage{color}

\usepackage{tikz}

\usepackage{geometry}
\geometry{a5paper, margin=0.1in, right=1cm}

\usepackage{dsfont}

\usepackage{graphicx}
\graphicspath{ {images/} }

\usepackage{faktor}

\usepackage{IEEEtrantools}
\usepackage{enumerate}   
\usepackage[PostScript=dvips]{"/Users/aware/Documents/Courses/diagrams"}


\newtheorem{theorem}{Théorème}[section]
\renewcommand{\thetheorem}{\arabic{theorem}}
\newtheorem{lemme}{Lemme}[section]
\renewcommand{\thelemme}{\arabic{lemme}}
\newtheorem{proposition}{Proposition}[section]
\renewcommand{\theproposition}{\arabic{proposition}}
\newtheorem{notations}{Notations}[section]
\newtheorem{problem}{Problème}[section]
\newtheorem{corollary}{Corollaire}[theorem]
\renewcommand{\thecorollary}{\arabic{corollary}}
\newtheorem{property}{Propriété}[section]
\newtheorem{objective}{Objectif}[section]

\theoremstyle{definition}
\newtheorem{definition}{Définition}[section]
\renewcommand{\thedefinition}{\arabic{definition}}
\newtheorem{exercise}{Exercice}[chapter]
\renewcommand{\theexercise}{\arabic{exercise}}
\newtheorem{example}{Exemple}[chapter]
\renewcommand{\theexample}{\arabic{example}}
\newtheorem*{solution}{Solution}
\newtheorem*{application}{Application}
\newtheorem*{notation}{Notation}
\newtheorem*{vocabulary}{Vocabulaire}
\newtheorem*{properties}{Propriétés}



\theoremstyle{remark}
\newtheorem*{remark}{Remarque}
\newtheorem*{rappel}{Rappel}


\usepackage{etoolbox}
\AtBeginEnvironment{exercise}{\small}
\AtBeginEnvironment{example}{\small}

\usepackage{cases}
\usepackage[red]{mypack}

\usepackage[framemethod=TikZ]{mdframed}

\definecolor{bg}{rgb}{0.4,0.25,0.95}
\definecolor{pagebg}{rgb}{0,0,0.5}
\surroundwithmdframed[
   topline=false,
   rightline=false,
   bottomline=false,
   leftmargin=\parindent,
   skipabove=8pt,
   skipbelow=8pt,
   linecolor=blue,
   innerbottommargin=10pt,
   % backgroundcolor=bg,font=\color{orange}\sffamily, fontcolor=white
]{definition}

\usepackage{empheq}
\usepackage[most]{tcolorbox}

\newtcbox{\mymath}[1][]{%
    nobeforeafter, math upper, tcbox raise base,
    enhanced, colframe=blue!30!black,
    colback=red!10, boxrule=1pt,
    #1}

\usepackage{unixode}


\DeclareMathOperator{\ord}{ord}
\DeclareMathOperator{\orb}{orb}
\DeclareMathOperator{\stab}{stab}
\DeclareMathOperator{\Stab}{stab}
\DeclareMathOperator{\ppcm}{ppcm}
\DeclareMathOperator{\conj}{Conj}
\DeclareMathOperator{\End}{End}
\DeclareMathOperator{\rot}{rot}
\DeclareMathOperator{\trs}{trace}
\DeclareMathOperator{\Ind}{Ind}
\DeclareMathOperator{\mat}{Mat}
\DeclareMathOperator{\id}{Id}
\DeclareMathOperator{\vect}{vect}
\DeclareMathOperator{\img}{img}
\DeclareMathOperator{\cov}{Cov}
\DeclareMathOperator{\dist}{dist}
\DeclareMathOperator{\irr}{Irr}
\DeclareMathOperator{\image}{Im}
\DeclareMathOperator{\pd}{\partial}
\DeclareMathOperator{\epi}{epi}
\DeclareMathOperator{\Argmin}{Argmin}
\DeclareMathOperator{\dom}{dom}
\DeclareMathOperator{\proj}{proj}
\DeclareMathOperator{\ctg}{ctg}
\DeclareMathOperator{\supp}{supp}
\DeclareMathOperator{\argmin}{argmin}
\DeclareMathOperator{\mult}{mult}
\DeclareMathOperator{\ch}{ch}
\DeclareMathOperator{\sh}{sh}
\DeclareMathOperator{\rang}{rang}
\DeclareMathOperator{\diam}{diam}
\DeclareMathOperator{\Epigraphe}{Epigraphe}




\usepackage{xcolor}
\everymath{\color{blue}}
%\everymath{\color[rgb]{0,1,1}}
%\pagecolor[rgb]{0,0,0.5}


\newcommand*{\pdtest}[3][]{\ensuremath{\frac{\partial^{#1} #2}{\partial #3}}}

\newcommand*{\deffunc}[6][]{\ensuremath{
\begin{array}{rcl}
#2 : #3 &\rightarrow& #4\\
#5 &\mapsto& #6
\end{array}
}}

\newcommand{\eqcolon}{\mathrel{\resizebox{\widthof{$\mathord{=}$}}{\height}{ $\!\!=\!\!\resizebox{1.2\width}{0.8\height}{\raisebox{0.23ex}{$\mathop{:}$}}\!\!$ }}}
\newcommand{\coloneq}{\mathrel{\resizebox{\widthof{$\mathord{=}$}}{\height}{ $\!\!\resizebox{1.2\width}{0.8\height}{\raisebox{0.23ex}{$\mathop{:}$}}\!\!=\!\!$ }}}
\newcommand{\eqcolonl}{\ensuremath{\mathrel{=\!\!\mathop{:}}}}
\newcommand{\coloneql}{\ensuremath{\mathrel{\mathop{:} \!\! =}}}
\newcommand{\vc}[1]{% inline column vector
  \left(\begin{smallmatrix}#1\end{smallmatrix}\right)%
}
\newcommand{\vr}[1]{% inline row vector
  \begin{smallmatrix}(\,#1\,)\end{smallmatrix}%
}
\makeatletter
\newcommand*{\defeq}{\ =\mathrel{\rlap{%
                     \raisebox{0.3ex}{$\m@th\cdot$}}%
                     \raisebox{-0.3ex}{$\m@th\cdot$}}%
                     }
\makeatother

\newcommand{\mathcircle}[1]{% inline row vector
 \overset{\circ}{#1}
}
\newcommand{\ulim}{% low limit
 \underline{\lim}
}
\newcommand{\ssi}{% iff
\iff
}
\newcommand{\ps}[2]{
\expval{#1 | #2}
}
\newcommand{\df}[1]{
\mqty{#1}
}
\newcommand{\n}[1]{
\norm{#1}
}
\newcommand{\sys}[1]{
\left\{\smqty{#1}\right.
}


\newcommand{\eqdef}{\ensuremath{\overset{\text{def}}=}}


\def\Circlearrowright{\ensuremath{%
  \rotatebox[origin=c]{230}{$\circlearrowright$}}}

\newcommand\ct[1]{\text{\rmfamily\upshape #1}}
\newcommand\question[1]{ {\color{red} ...!? \small #1}}
\newcommand\caz[1]{\left\{\begin{array} #1 \end{array}\right.}
\newcommand\const{\text{\rmfamily\upshape const}}
\newcommand\toP{ \overset{\pro}{\to}}
\newcommand\toPP{ \overset{\text{PP}}{\to}}
\newcommand{\oeq}{\mathrel{\text{\textcircled{$=$}}}}





\usepackage{xcolor}
% \usepackage[normalem]{ulem}
\usepackage{lipsum}
\makeatletter
% \newcommand\colorwave[1][blue]{\bgroup \markoverwith{\lower3.5\p@\hbox{\sixly \textcolor{#1}{\char58}}}\ULon}
%\font\sixly=lasy6 % does not re-load if already loaded, so no memory problem.

\newmdtheoremenv[
linewidth= 1pt,linecolor= blue,%
leftmargin=20,rightmargin=20,innertopmargin=0pt, innerrightmargin=40,%
tikzsetting = { draw=lightgray, line width = 0.3pt,dashed,%
dash pattern = on 15pt off 3pt},%
splittopskip=\topskip,skipbelow=\baselineskip,%
skipabove=\baselineskip,ntheorem,roundcorner=0pt,
% backgroundcolor=pagebg,font=\color{orange}\sffamily, fontcolor=white
]{examplebox}{Exemple}[section]



\newcommand\R{\mathbb{R}}
\newcommand\Z{\mathbb{Z}}
\newcommand\N{\mathbb{N}}
\newcommand\E{\mathbb{E}}
\newcommand\F{\mathcal{F}}
\newcommand\cH{\mathcal{H}}
\newcommand\V{\mathbb{V}}
\newcommand\dmo{ ^{-1} }
\newcommand\kapa{\kappa}
\newcommand\im{Im}
\newcommand\hs{\mathcal{H}}





\usepackage{soul}

\makeatletter
\newcommand*{\whiten}[1]{\llap{\textcolor{white}{{\the\SOUL@token}}\hspace{#1pt}}}
\DeclareRobustCommand*\myul{%
    \def\SOUL@everyspace{\underline{\space}\kern\z@}%
    \def\SOUL@everytoken{%
     \setbox0=\hbox{\the\SOUL@token}%
     \ifdim\dp0>\z@
        \raisebox{\dp0}{\underline{\phantom{\the\SOUL@token}}}%
        \whiten{1}\whiten{0}%
        \whiten{-1}\whiten{-2}%
        \llap{\the\SOUL@token}%
     \else
        \underline{\the\SOUL@token}%
     \fi}%
\SOUL@}
\makeatother

\newcommand*{\demp}{\fontfamily{lmtt}\selectfont}

\DeclareTextFontCommand{\textdemp}{\demp}

\begin{document}

\ifcomment
Multiline
comment
\fi
\ifcomment
\myul{Typesetting test}
% \color[rgb]{1,1,1}
$∑_i^n≠ 60º±∞π∆¬≈√j∫h≤≥µ$

$\CR \R\pro\ind\pro\gS\pro
\mqty[a&b\\c&d]$
$\pro\mathbb{P}$
$\dd{x}$

  \[
    \alpha(x)=\left\{
                \begin{array}{ll}
                  x\\
                  \frac{1}{1+e^{-kx}}\\
                  \frac{e^x-e^{-x}}{e^x+e^{-x}}
                \end{array}
              \right.
  \]

  $\expval{x}$
  
  $\chi_\rho(ghg\dmo)=\Tr(\rho_{ghg\dmo})=\Tr(\rho_g\circ\rho_h\circ\rho\dmo_g)=\Tr(\rho_h)\overset{\mbox{\scalebox{0.5}{$\Tr(AB)=\Tr(BA)$}}}{=}\chi_\rho(h)$
  	$\mathop{\oplus}_{\substack{x\in X}}$

$\mat(\rho_g)=(a_{ij}(g))_{\scriptsize \substack{1\leq i\leq d \\ 1\leq j\leq d}}$ et $\mat(\rho'_g)=(a'_{ij}(g))_{\scriptsize \substack{1\leq i'\leq d' \\ 1\leq j'\leq d'}}$



\[\int_a^b{\mathbb{R}^2}g(u, v)\dd{P_{XY}}(u, v)=\iint g(u,v) f_{XY}(u, v)\dd \lambda(u) \dd \lambda(v)\]
$$\lim_{x\to\infty} f(x)$$	
$$\iiiint_V \mu(t,u,v,w) \,dt\,du\,dv\,dw$$
$$\sum_{n=1}^{\infty} 2^{-n} = 1$$	
\begin{definition}
	Si $X$ et $Y$ sont 2 v.a. ou definit la \textsc{Covariance} entre $X$ et $Y$ comme
	$\cov(X,Y)\overset{\text{def}}{=}\E\left[(X-\E(X))(Y-\E(Y))\right]=\E(XY)-\E(X)\E(Y)$.
\end{definition}
\fi
\pagebreak

% \tableofcontents

% insert your code here
%\input{./algebra/main.tex}
%\input{./geometrie-differentielle/main.tex}
%\input{./probabilite/main.tex}
%\input{./analyse-fonctionnelle/main.tex}
% \input{./Analyse-convexe-et-dualite-en-optimisation/main.tex}
%\input{./tikz/main.tex}
%\input{./Theorie-du-distributions/main.tex}
%\input{./optimisation/mine.tex}
 \input{./modelisation/main.tex}

% yves.aubry@univ-tln.fr : algebra

\end{document}


% yves.aubry@univ-tln.fr : algebra

\end{document}

%% !TEX encoding = UTF-8 Unicode
% !TEX TS-program = xelatex

\documentclass[french]{report}

%\usepackage[utf8]{inputenc}
%\usepackage[T1]{fontenc}
\usepackage{babel}


\newif\ifcomment
%\commenttrue # Show comments

\usepackage{physics}
\usepackage{amssymb}


\usepackage{amsthm}
% \usepackage{thmtools}
\usepackage{mathtools}
\usepackage{amsfonts}

\usepackage{color}

\usepackage{tikz}

\usepackage{geometry}
\geometry{a5paper, margin=0.1in, right=1cm}

\usepackage{dsfont}

\usepackage{graphicx}
\graphicspath{ {images/} }

\usepackage{faktor}

\usepackage{IEEEtrantools}
\usepackage{enumerate}   
\usepackage[PostScript=dvips]{"/Users/aware/Documents/Courses/diagrams"}


\newtheorem{theorem}{Théorème}[section]
\renewcommand{\thetheorem}{\arabic{theorem}}
\newtheorem{lemme}{Lemme}[section]
\renewcommand{\thelemme}{\arabic{lemme}}
\newtheorem{proposition}{Proposition}[section]
\renewcommand{\theproposition}{\arabic{proposition}}
\newtheorem{notations}{Notations}[section]
\newtheorem{problem}{Problème}[section]
\newtheorem{corollary}{Corollaire}[theorem]
\renewcommand{\thecorollary}{\arabic{corollary}}
\newtheorem{property}{Propriété}[section]
\newtheorem{objective}{Objectif}[section]

\theoremstyle{definition}
\newtheorem{definition}{Définition}[section]
\renewcommand{\thedefinition}{\arabic{definition}}
\newtheorem{exercise}{Exercice}[chapter]
\renewcommand{\theexercise}{\arabic{exercise}}
\newtheorem{example}{Exemple}[chapter]
\renewcommand{\theexample}{\arabic{example}}
\newtheorem*{solution}{Solution}
\newtheorem*{application}{Application}
\newtheorem*{notation}{Notation}
\newtheorem*{vocabulary}{Vocabulaire}
\newtheorem*{properties}{Propriétés}



\theoremstyle{remark}
\newtheorem*{remark}{Remarque}
\newtheorem*{rappel}{Rappel}


\usepackage{etoolbox}
\AtBeginEnvironment{exercise}{\small}
\AtBeginEnvironment{example}{\small}

\usepackage{cases}
\usepackage[red]{mypack}

\usepackage[framemethod=TikZ]{mdframed}

\definecolor{bg}{rgb}{0.4,0.25,0.95}
\definecolor{pagebg}{rgb}{0,0,0.5}
\surroundwithmdframed[
   topline=false,
   rightline=false,
   bottomline=false,
   leftmargin=\parindent,
   skipabove=8pt,
   skipbelow=8pt,
   linecolor=blue,
   innerbottommargin=10pt,
   % backgroundcolor=bg,font=\color{orange}\sffamily, fontcolor=white
]{definition}

\usepackage{empheq}
\usepackage[most]{tcolorbox}

\newtcbox{\mymath}[1][]{%
    nobeforeafter, math upper, tcbox raise base,
    enhanced, colframe=blue!30!black,
    colback=red!10, boxrule=1pt,
    #1}

\usepackage{unixode}


\DeclareMathOperator{\ord}{ord}
\DeclareMathOperator{\orb}{orb}
\DeclareMathOperator{\stab}{stab}
\DeclareMathOperator{\Stab}{stab}
\DeclareMathOperator{\ppcm}{ppcm}
\DeclareMathOperator{\conj}{Conj}
\DeclareMathOperator{\End}{End}
\DeclareMathOperator{\rot}{rot}
\DeclareMathOperator{\trs}{trace}
\DeclareMathOperator{\Ind}{Ind}
\DeclareMathOperator{\mat}{Mat}
\DeclareMathOperator{\id}{Id}
\DeclareMathOperator{\vect}{vect}
\DeclareMathOperator{\img}{img}
\DeclareMathOperator{\cov}{Cov}
\DeclareMathOperator{\dist}{dist}
\DeclareMathOperator{\irr}{Irr}
\DeclareMathOperator{\image}{Im}
\DeclareMathOperator{\pd}{\partial}
\DeclareMathOperator{\epi}{epi}
\DeclareMathOperator{\Argmin}{Argmin}
\DeclareMathOperator{\dom}{dom}
\DeclareMathOperator{\proj}{proj}
\DeclareMathOperator{\ctg}{ctg}
\DeclareMathOperator{\supp}{supp}
\DeclareMathOperator{\argmin}{argmin}
\DeclareMathOperator{\mult}{mult}
\DeclareMathOperator{\ch}{ch}
\DeclareMathOperator{\sh}{sh}
\DeclareMathOperator{\rang}{rang}
\DeclareMathOperator{\diam}{diam}
\DeclareMathOperator{\Epigraphe}{Epigraphe}




\usepackage{xcolor}
\everymath{\color{blue}}
%\everymath{\color[rgb]{0,1,1}}
%\pagecolor[rgb]{0,0,0.5}


\newcommand*{\pdtest}[3][]{\ensuremath{\frac{\partial^{#1} #2}{\partial #3}}}

\newcommand*{\deffunc}[6][]{\ensuremath{
\begin{array}{rcl}
#2 : #3 &\rightarrow& #4\\
#5 &\mapsto& #6
\end{array}
}}

\newcommand{\eqcolon}{\mathrel{\resizebox{\widthof{$\mathord{=}$}}{\height}{ $\!\!=\!\!\resizebox{1.2\width}{0.8\height}{\raisebox{0.23ex}{$\mathop{:}$}}\!\!$ }}}
\newcommand{\coloneq}{\mathrel{\resizebox{\widthof{$\mathord{=}$}}{\height}{ $\!\!\resizebox{1.2\width}{0.8\height}{\raisebox{0.23ex}{$\mathop{:}$}}\!\!=\!\!$ }}}
\newcommand{\eqcolonl}{\ensuremath{\mathrel{=\!\!\mathop{:}}}}
\newcommand{\coloneql}{\ensuremath{\mathrel{\mathop{:} \!\! =}}}
\newcommand{\vc}[1]{% inline column vector
  \left(\begin{smallmatrix}#1\end{smallmatrix}\right)%
}
\newcommand{\vr}[1]{% inline row vector
  \begin{smallmatrix}(\,#1\,)\end{smallmatrix}%
}
\makeatletter
\newcommand*{\defeq}{\ =\mathrel{\rlap{%
                     \raisebox{0.3ex}{$\m@th\cdot$}}%
                     \raisebox{-0.3ex}{$\m@th\cdot$}}%
                     }
\makeatother

\newcommand{\mathcircle}[1]{% inline row vector
 \overset{\circ}{#1}
}
\newcommand{\ulim}{% low limit
 \underline{\lim}
}
\newcommand{\ssi}{% iff
\iff
}
\newcommand{\ps}[2]{
\expval{#1 | #2}
}
\newcommand{\df}[1]{
\mqty{#1}
}
\newcommand{\n}[1]{
\norm{#1}
}
\newcommand{\sys}[1]{
\left\{\smqty{#1}\right.
}


\newcommand{\eqdef}{\ensuremath{\overset{\text{def}}=}}


\def\Circlearrowright{\ensuremath{%
  \rotatebox[origin=c]{230}{$\circlearrowright$}}}

\newcommand\ct[1]{\text{\rmfamily\upshape #1}}
\newcommand\question[1]{ {\color{red} ...!? \small #1}}
\newcommand\caz[1]{\left\{\begin{array} #1 \end{array}\right.}
\newcommand\const{\text{\rmfamily\upshape const}}
\newcommand\toP{ \overset{\pro}{\to}}
\newcommand\toPP{ \overset{\text{PP}}{\to}}
\newcommand{\oeq}{\mathrel{\text{\textcircled{$=$}}}}





\usepackage{xcolor}
% \usepackage[normalem]{ulem}
\usepackage{lipsum}
\makeatletter
% \newcommand\colorwave[1][blue]{\bgroup \markoverwith{\lower3.5\p@\hbox{\sixly \textcolor{#1}{\char58}}}\ULon}
%\font\sixly=lasy6 % does not re-load if already loaded, so no memory problem.

\newmdtheoremenv[
linewidth= 1pt,linecolor= blue,%
leftmargin=20,rightmargin=20,innertopmargin=0pt, innerrightmargin=40,%
tikzsetting = { draw=lightgray, line width = 0.3pt,dashed,%
dash pattern = on 15pt off 3pt},%
splittopskip=\topskip,skipbelow=\baselineskip,%
skipabove=\baselineskip,ntheorem,roundcorner=0pt,
% backgroundcolor=pagebg,font=\color{orange}\sffamily, fontcolor=white
]{examplebox}{Exemple}[section]



\newcommand\R{\mathbb{R}}
\newcommand\Z{\mathbb{Z}}
\newcommand\N{\mathbb{N}}
\newcommand\E{\mathbb{E}}
\newcommand\F{\mathcal{F}}
\newcommand\cH{\mathcal{H}}
\newcommand\V{\mathbb{V}}
\newcommand\dmo{ ^{-1} }
\newcommand\kapa{\kappa}
\newcommand\im{Im}
\newcommand\hs{\mathcal{H}}





\usepackage{soul}

\makeatletter
\newcommand*{\whiten}[1]{\llap{\textcolor{white}{{\the\SOUL@token}}\hspace{#1pt}}}
\DeclareRobustCommand*\myul{%
    \def\SOUL@everyspace{\underline{\space}\kern\z@}%
    \def\SOUL@everytoken{%
     \setbox0=\hbox{\the\SOUL@token}%
     \ifdim\dp0>\z@
        \raisebox{\dp0}{\underline{\phantom{\the\SOUL@token}}}%
        \whiten{1}\whiten{0}%
        \whiten{-1}\whiten{-2}%
        \llap{\the\SOUL@token}%
     \else
        \underline{\the\SOUL@token}%
     \fi}%
\SOUL@}
\makeatother

\newcommand*{\demp}{\fontfamily{lmtt}\selectfont}

\DeclareTextFontCommand{\textdemp}{\demp}

\begin{document}

\ifcomment
Multiline
comment
\fi
\ifcomment
\myul{Typesetting test}
% \color[rgb]{1,1,1}
$∑_i^n≠ 60º±∞π∆¬≈√j∫h≤≥µ$

$\CR \R\pro\ind\pro\gS\pro
\mqty[a&b\\c&d]$
$\pro\mathbb{P}$
$\dd{x}$

  \[
    \alpha(x)=\left\{
                \begin{array}{ll}
                  x\\
                  \frac{1}{1+e^{-kx}}\\
                  \frac{e^x-e^{-x}}{e^x+e^{-x}}
                \end{array}
              \right.
  \]

  $\expval{x}$
  
  $\chi_\rho(ghg\dmo)=\Tr(\rho_{ghg\dmo})=\Tr(\rho_g\circ\rho_h\circ\rho\dmo_g)=\Tr(\rho_h)\overset{\mbox{\scalebox{0.5}{$\Tr(AB)=\Tr(BA)$}}}{=}\chi_\rho(h)$
  	$\mathop{\oplus}_{\substack{x\in X}}$

$\mat(\rho_g)=(a_{ij}(g))_{\scriptsize \substack{1\leq i\leq d \\ 1\leq j\leq d}}$ et $\mat(\rho'_g)=(a'_{ij}(g))_{\scriptsize \substack{1\leq i'\leq d' \\ 1\leq j'\leq d'}}$



\[\int_a^b{\mathbb{R}^2}g(u, v)\dd{P_{XY}}(u, v)=\iint g(u,v) f_{XY}(u, v)\dd \lambda(u) \dd \lambda(v)\]
$$\lim_{x\to\infty} f(x)$$	
$$\iiiint_V \mu(t,u,v,w) \,dt\,du\,dv\,dw$$
$$\sum_{n=1}^{\infty} 2^{-n} = 1$$	
\begin{definition}
	Si $X$ et $Y$ sont 2 v.a. ou definit la \textsc{Covariance} entre $X$ et $Y$ comme
	$\cov(X,Y)\overset{\text{def}}{=}\E\left[(X-\E(X))(Y-\E(Y))\right]=\E(XY)-\E(X)\E(Y)$.
\end{definition}
\fi
\pagebreak

% \tableofcontents

% insert your code here
%% !TEX encoding = UTF-8 Unicode
% !TEX TS-program = xelatex

\documentclass[french]{report}

%\usepackage[utf8]{inputenc}
%\usepackage[T1]{fontenc}
\usepackage{babel}


\newif\ifcomment
%\commenttrue # Show comments

\usepackage{physics}
\usepackage{amssymb}


\usepackage{amsthm}
% \usepackage{thmtools}
\usepackage{mathtools}
\usepackage{amsfonts}

\usepackage{color}

\usepackage{tikz}

\usepackage{geometry}
\geometry{a5paper, margin=0.1in, right=1cm}

\usepackage{dsfont}

\usepackage{graphicx}
\graphicspath{ {images/} }

\usepackage{faktor}

\usepackage{IEEEtrantools}
\usepackage{enumerate}   
\usepackage[PostScript=dvips]{"/Users/aware/Documents/Courses/diagrams"}


\newtheorem{theorem}{Théorème}[section]
\renewcommand{\thetheorem}{\arabic{theorem}}
\newtheorem{lemme}{Lemme}[section]
\renewcommand{\thelemme}{\arabic{lemme}}
\newtheorem{proposition}{Proposition}[section]
\renewcommand{\theproposition}{\arabic{proposition}}
\newtheorem{notations}{Notations}[section]
\newtheorem{problem}{Problème}[section]
\newtheorem{corollary}{Corollaire}[theorem]
\renewcommand{\thecorollary}{\arabic{corollary}}
\newtheorem{property}{Propriété}[section]
\newtheorem{objective}{Objectif}[section]

\theoremstyle{definition}
\newtheorem{definition}{Définition}[section]
\renewcommand{\thedefinition}{\arabic{definition}}
\newtheorem{exercise}{Exercice}[chapter]
\renewcommand{\theexercise}{\arabic{exercise}}
\newtheorem{example}{Exemple}[chapter]
\renewcommand{\theexample}{\arabic{example}}
\newtheorem*{solution}{Solution}
\newtheorem*{application}{Application}
\newtheorem*{notation}{Notation}
\newtheorem*{vocabulary}{Vocabulaire}
\newtheorem*{properties}{Propriétés}



\theoremstyle{remark}
\newtheorem*{remark}{Remarque}
\newtheorem*{rappel}{Rappel}


\usepackage{etoolbox}
\AtBeginEnvironment{exercise}{\small}
\AtBeginEnvironment{example}{\small}

\usepackage{cases}
\usepackage[red]{mypack}

\usepackage[framemethod=TikZ]{mdframed}

\definecolor{bg}{rgb}{0.4,0.25,0.95}
\definecolor{pagebg}{rgb}{0,0,0.5}
\surroundwithmdframed[
   topline=false,
   rightline=false,
   bottomline=false,
   leftmargin=\parindent,
   skipabove=8pt,
   skipbelow=8pt,
   linecolor=blue,
   innerbottommargin=10pt,
   % backgroundcolor=bg,font=\color{orange}\sffamily, fontcolor=white
]{definition}

\usepackage{empheq}
\usepackage[most]{tcolorbox}

\newtcbox{\mymath}[1][]{%
    nobeforeafter, math upper, tcbox raise base,
    enhanced, colframe=blue!30!black,
    colback=red!10, boxrule=1pt,
    #1}

\usepackage{unixode}


\DeclareMathOperator{\ord}{ord}
\DeclareMathOperator{\orb}{orb}
\DeclareMathOperator{\stab}{stab}
\DeclareMathOperator{\Stab}{stab}
\DeclareMathOperator{\ppcm}{ppcm}
\DeclareMathOperator{\conj}{Conj}
\DeclareMathOperator{\End}{End}
\DeclareMathOperator{\rot}{rot}
\DeclareMathOperator{\trs}{trace}
\DeclareMathOperator{\Ind}{Ind}
\DeclareMathOperator{\mat}{Mat}
\DeclareMathOperator{\id}{Id}
\DeclareMathOperator{\vect}{vect}
\DeclareMathOperator{\img}{img}
\DeclareMathOperator{\cov}{Cov}
\DeclareMathOperator{\dist}{dist}
\DeclareMathOperator{\irr}{Irr}
\DeclareMathOperator{\image}{Im}
\DeclareMathOperator{\pd}{\partial}
\DeclareMathOperator{\epi}{epi}
\DeclareMathOperator{\Argmin}{Argmin}
\DeclareMathOperator{\dom}{dom}
\DeclareMathOperator{\proj}{proj}
\DeclareMathOperator{\ctg}{ctg}
\DeclareMathOperator{\supp}{supp}
\DeclareMathOperator{\argmin}{argmin}
\DeclareMathOperator{\mult}{mult}
\DeclareMathOperator{\ch}{ch}
\DeclareMathOperator{\sh}{sh}
\DeclareMathOperator{\rang}{rang}
\DeclareMathOperator{\diam}{diam}
\DeclareMathOperator{\Epigraphe}{Epigraphe}




\usepackage{xcolor}
\everymath{\color{blue}}
%\everymath{\color[rgb]{0,1,1}}
%\pagecolor[rgb]{0,0,0.5}


\newcommand*{\pdtest}[3][]{\ensuremath{\frac{\partial^{#1} #2}{\partial #3}}}

\newcommand*{\deffunc}[6][]{\ensuremath{
\begin{array}{rcl}
#2 : #3 &\rightarrow& #4\\
#5 &\mapsto& #6
\end{array}
}}

\newcommand{\eqcolon}{\mathrel{\resizebox{\widthof{$\mathord{=}$}}{\height}{ $\!\!=\!\!\resizebox{1.2\width}{0.8\height}{\raisebox{0.23ex}{$\mathop{:}$}}\!\!$ }}}
\newcommand{\coloneq}{\mathrel{\resizebox{\widthof{$\mathord{=}$}}{\height}{ $\!\!\resizebox{1.2\width}{0.8\height}{\raisebox{0.23ex}{$\mathop{:}$}}\!\!=\!\!$ }}}
\newcommand{\eqcolonl}{\ensuremath{\mathrel{=\!\!\mathop{:}}}}
\newcommand{\coloneql}{\ensuremath{\mathrel{\mathop{:} \!\! =}}}
\newcommand{\vc}[1]{% inline column vector
  \left(\begin{smallmatrix}#1\end{smallmatrix}\right)%
}
\newcommand{\vr}[1]{% inline row vector
  \begin{smallmatrix}(\,#1\,)\end{smallmatrix}%
}
\makeatletter
\newcommand*{\defeq}{\ =\mathrel{\rlap{%
                     \raisebox{0.3ex}{$\m@th\cdot$}}%
                     \raisebox{-0.3ex}{$\m@th\cdot$}}%
                     }
\makeatother

\newcommand{\mathcircle}[1]{% inline row vector
 \overset{\circ}{#1}
}
\newcommand{\ulim}{% low limit
 \underline{\lim}
}
\newcommand{\ssi}{% iff
\iff
}
\newcommand{\ps}[2]{
\expval{#1 | #2}
}
\newcommand{\df}[1]{
\mqty{#1}
}
\newcommand{\n}[1]{
\norm{#1}
}
\newcommand{\sys}[1]{
\left\{\smqty{#1}\right.
}


\newcommand{\eqdef}{\ensuremath{\overset{\text{def}}=}}


\def\Circlearrowright{\ensuremath{%
  \rotatebox[origin=c]{230}{$\circlearrowright$}}}

\newcommand\ct[1]{\text{\rmfamily\upshape #1}}
\newcommand\question[1]{ {\color{red} ...!? \small #1}}
\newcommand\caz[1]{\left\{\begin{array} #1 \end{array}\right.}
\newcommand\const{\text{\rmfamily\upshape const}}
\newcommand\toP{ \overset{\pro}{\to}}
\newcommand\toPP{ \overset{\text{PP}}{\to}}
\newcommand{\oeq}{\mathrel{\text{\textcircled{$=$}}}}





\usepackage{xcolor}
% \usepackage[normalem]{ulem}
\usepackage{lipsum}
\makeatletter
% \newcommand\colorwave[1][blue]{\bgroup \markoverwith{\lower3.5\p@\hbox{\sixly \textcolor{#1}{\char58}}}\ULon}
%\font\sixly=lasy6 % does not re-load if already loaded, so no memory problem.

\newmdtheoremenv[
linewidth= 1pt,linecolor= blue,%
leftmargin=20,rightmargin=20,innertopmargin=0pt, innerrightmargin=40,%
tikzsetting = { draw=lightgray, line width = 0.3pt,dashed,%
dash pattern = on 15pt off 3pt},%
splittopskip=\topskip,skipbelow=\baselineskip,%
skipabove=\baselineskip,ntheorem,roundcorner=0pt,
% backgroundcolor=pagebg,font=\color{orange}\sffamily, fontcolor=white
]{examplebox}{Exemple}[section]



\newcommand\R{\mathbb{R}}
\newcommand\Z{\mathbb{Z}}
\newcommand\N{\mathbb{N}}
\newcommand\E{\mathbb{E}}
\newcommand\F{\mathcal{F}}
\newcommand\cH{\mathcal{H}}
\newcommand\V{\mathbb{V}}
\newcommand\dmo{ ^{-1} }
\newcommand\kapa{\kappa}
\newcommand\im{Im}
\newcommand\hs{\mathcal{H}}





\usepackage{soul}

\makeatletter
\newcommand*{\whiten}[1]{\llap{\textcolor{white}{{\the\SOUL@token}}\hspace{#1pt}}}
\DeclareRobustCommand*\myul{%
    \def\SOUL@everyspace{\underline{\space}\kern\z@}%
    \def\SOUL@everytoken{%
     \setbox0=\hbox{\the\SOUL@token}%
     \ifdim\dp0>\z@
        \raisebox{\dp0}{\underline{\phantom{\the\SOUL@token}}}%
        \whiten{1}\whiten{0}%
        \whiten{-1}\whiten{-2}%
        \llap{\the\SOUL@token}%
     \else
        \underline{\the\SOUL@token}%
     \fi}%
\SOUL@}
\makeatother

\newcommand*{\demp}{\fontfamily{lmtt}\selectfont}

\DeclareTextFontCommand{\textdemp}{\demp}

\begin{document}

\ifcomment
Multiline
comment
\fi
\ifcomment
\myul{Typesetting test}
% \color[rgb]{1,1,1}
$∑_i^n≠ 60º±∞π∆¬≈√j∫h≤≥µ$

$\CR \R\pro\ind\pro\gS\pro
\mqty[a&b\\c&d]$
$\pro\mathbb{P}$
$\dd{x}$

  \[
    \alpha(x)=\left\{
                \begin{array}{ll}
                  x\\
                  \frac{1}{1+e^{-kx}}\\
                  \frac{e^x-e^{-x}}{e^x+e^{-x}}
                \end{array}
              \right.
  \]

  $\expval{x}$
  
  $\chi_\rho(ghg\dmo)=\Tr(\rho_{ghg\dmo})=\Tr(\rho_g\circ\rho_h\circ\rho\dmo_g)=\Tr(\rho_h)\overset{\mbox{\scalebox{0.5}{$\Tr(AB)=\Tr(BA)$}}}{=}\chi_\rho(h)$
  	$\mathop{\oplus}_{\substack{x\in X}}$

$\mat(\rho_g)=(a_{ij}(g))_{\scriptsize \substack{1\leq i\leq d \\ 1\leq j\leq d}}$ et $\mat(\rho'_g)=(a'_{ij}(g))_{\scriptsize \substack{1\leq i'\leq d' \\ 1\leq j'\leq d'}}$



\[\int_a^b{\mathbb{R}^2}g(u, v)\dd{P_{XY}}(u, v)=\iint g(u,v) f_{XY}(u, v)\dd \lambda(u) \dd \lambda(v)\]
$$\lim_{x\to\infty} f(x)$$	
$$\iiiint_V \mu(t,u,v,w) \,dt\,du\,dv\,dw$$
$$\sum_{n=1}^{\infty} 2^{-n} = 1$$	
\begin{definition}
	Si $X$ et $Y$ sont 2 v.a. ou definit la \textsc{Covariance} entre $X$ et $Y$ comme
	$\cov(X,Y)\overset{\text{def}}{=}\E\left[(X-\E(X))(Y-\E(Y))\right]=\E(XY)-\E(X)\E(Y)$.
\end{definition}
\fi
\pagebreak

% \tableofcontents

% insert your code here
%\input{./algebra/main.tex}
%\input{./geometrie-differentielle/main.tex}
%\input{./probabilite/main.tex}
%\input{./analyse-fonctionnelle/main.tex}
% \input{./Analyse-convexe-et-dualite-en-optimisation/main.tex}
%\input{./tikz/main.tex}
%\input{./Theorie-du-distributions/main.tex}
%\input{./optimisation/mine.tex}
 \input{./modelisation/main.tex}

% yves.aubry@univ-tln.fr : algebra

\end{document}

%% !TEX encoding = UTF-8 Unicode
% !TEX TS-program = xelatex

\documentclass[french]{report}

%\usepackage[utf8]{inputenc}
%\usepackage[T1]{fontenc}
\usepackage{babel}


\newif\ifcomment
%\commenttrue # Show comments

\usepackage{physics}
\usepackage{amssymb}


\usepackage{amsthm}
% \usepackage{thmtools}
\usepackage{mathtools}
\usepackage{amsfonts}

\usepackage{color}

\usepackage{tikz}

\usepackage{geometry}
\geometry{a5paper, margin=0.1in, right=1cm}

\usepackage{dsfont}

\usepackage{graphicx}
\graphicspath{ {images/} }

\usepackage{faktor}

\usepackage{IEEEtrantools}
\usepackage{enumerate}   
\usepackage[PostScript=dvips]{"/Users/aware/Documents/Courses/diagrams"}


\newtheorem{theorem}{Théorème}[section]
\renewcommand{\thetheorem}{\arabic{theorem}}
\newtheorem{lemme}{Lemme}[section]
\renewcommand{\thelemme}{\arabic{lemme}}
\newtheorem{proposition}{Proposition}[section]
\renewcommand{\theproposition}{\arabic{proposition}}
\newtheorem{notations}{Notations}[section]
\newtheorem{problem}{Problème}[section]
\newtheorem{corollary}{Corollaire}[theorem]
\renewcommand{\thecorollary}{\arabic{corollary}}
\newtheorem{property}{Propriété}[section]
\newtheorem{objective}{Objectif}[section]

\theoremstyle{definition}
\newtheorem{definition}{Définition}[section]
\renewcommand{\thedefinition}{\arabic{definition}}
\newtheorem{exercise}{Exercice}[chapter]
\renewcommand{\theexercise}{\arabic{exercise}}
\newtheorem{example}{Exemple}[chapter]
\renewcommand{\theexample}{\arabic{example}}
\newtheorem*{solution}{Solution}
\newtheorem*{application}{Application}
\newtheorem*{notation}{Notation}
\newtheorem*{vocabulary}{Vocabulaire}
\newtheorem*{properties}{Propriétés}



\theoremstyle{remark}
\newtheorem*{remark}{Remarque}
\newtheorem*{rappel}{Rappel}


\usepackage{etoolbox}
\AtBeginEnvironment{exercise}{\small}
\AtBeginEnvironment{example}{\small}

\usepackage{cases}
\usepackage[red]{mypack}

\usepackage[framemethod=TikZ]{mdframed}

\definecolor{bg}{rgb}{0.4,0.25,0.95}
\definecolor{pagebg}{rgb}{0,0,0.5}
\surroundwithmdframed[
   topline=false,
   rightline=false,
   bottomline=false,
   leftmargin=\parindent,
   skipabove=8pt,
   skipbelow=8pt,
   linecolor=blue,
   innerbottommargin=10pt,
   % backgroundcolor=bg,font=\color{orange}\sffamily, fontcolor=white
]{definition}

\usepackage{empheq}
\usepackage[most]{tcolorbox}

\newtcbox{\mymath}[1][]{%
    nobeforeafter, math upper, tcbox raise base,
    enhanced, colframe=blue!30!black,
    colback=red!10, boxrule=1pt,
    #1}

\usepackage{unixode}


\DeclareMathOperator{\ord}{ord}
\DeclareMathOperator{\orb}{orb}
\DeclareMathOperator{\stab}{stab}
\DeclareMathOperator{\Stab}{stab}
\DeclareMathOperator{\ppcm}{ppcm}
\DeclareMathOperator{\conj}{Conj}
\DeclareMathOperator{\End}{End}
\DeclareMathOperator{\rot}{rot}
\DeclareMathOperator{\trs}{trace}
\DeclareMathOperator{\Ind}{Ind}
\DeclareMathOperator{\mat}{Mat}
\DeclareMathOperator{\id}{Id}
\DeclareMathOperator{\vect}{vect}
\DeclareMathOperator{\img}{img}
\DeclareMathOperator{\cov}{Cov}
\DeclareMathOperator{\dist}{dist}
\DeclareMathOperator{\irr}{Irr}
\DeclareMathOperator{\image}{Im}
\DeclareMathOperator{\pd}{\partial}
\DeclareMathOperator{\epi}{epi}
\DeclareMathOperator{\Argmin}{Argmin}
\DeclareMathOperator{\dom}{dom}
\DeclareMathOperator{\proj}{proj}
\DeclareMathOperator{\ctg}{ctg}
\DeclareMathOperator{\supp}{supp}
\DeclareMathOperator{\argmin}{argmin}
\DeclareMathOperator{\mult}{mult}
\DeclareMathOperator{\ch}{ch}
\DeclareMathOperator{\sh}{sh}
\DeclareMathOperator{\rang}{rang}
\DeclareMathOperator{\diam}{diam}
\DeclareMathOperator{\Epigraphe}{Epigraphe}




\usepackage{xcolor}
\everymath{\color{blue}}
%\everymath{\color[rgb]{0,1,1}}
%\pagecolor[rgb]{0,0,0.5}


\newcommand*{\pdtest}[3][]{\ensuremath{\frac{\partial^{#1} #2}{\partial #3}}}

\newcommand*{\deffunc}[6][]{\ensuremath{
\begin{array}{rcl}
#2 : #3 &\rightarrow& #4\\
#5 &\mapsto& #6
\end{array}
}}

\newcommand{\eqcolon}{\mathrel{\resizebox{\widthof{$\mathord{=}$}}{\height}{ $\!\!=\!\!\resizebox{1.2\width}{0.8\height}{\raisebox{0.23ex}{$\mathop{:}$}}\!\!$ }}}
\newcommand{\coloneq}{\mathrel{\resizebox{\widthof{$\mathord{=}$}}{\height}{ $\!\!\resizebox{1.2\width}{0.8\height}{\raisebox{0.23ex}{$\mathop{:}$}}\!\!=\!\!$ }}}
\newcommand{\eqcolonl}{\ensuremath{\mathrel{=\!\!\mathop{:}}}}
\newcommand{\coloneql}{\ensuremath{\mathrel{\mathop{:} \!\! =}}}
\newcommand{\vc}[1]{% inline column vector
  \left(\begin{smallmatrix}#1\end{smallmatrix}\right)%
}
\newcommand{\vr}[1]{% inline row vector
  \begin{smallmatrix}(\,#1\,)\end{smallmatrix}%
}
\makeatletter
\newcommand*{\defeq}{\ =\mathrel{\rlap{%
                     \raisebox{0.3ex}{$\m@th\cdot$}}%
                     \raisebox{-0.3ex}{$\m@th\cdot$}}%
                     }
\makeatother

\newcommand{\mathcircle}[1]{% inline row vector
 \overset{\circ}{#1}
}
\newcommand{\ulim}{% low limit
 \underline{\lim}
}
\newcommand{\ssi}{% iff
\iff
}
\newcommand{\ps}[2]{
\expval{#1 | #2}
}
\newcommand{\df}[1]{
\mqty{#1}
}
\newcommand{\n}[1]{
\norm{#1}
}
\newcommand{\sys}[1]{
\left\{\smqty{#1}\right.
}


\newcommand{\eqdef}{\ensuremath{\overset{\text{def}}=}}


\def\Circlearrowright{\ensuremath{%
  \rotatebox[origin=c]{230}{$\circlearrowright$}}}

\newcommand\ct[1]{\text{\rmfamily\upshape #1}}
\newcommand\question[1]{ {\color{red} ...!? \small #1}}
\newcommand\caz[1]{\left\{\begin{array} #1 \end{array}\right.}
\newcommand\const{\text{\rmfamily\upshape const}}
\newcommand\toP{ \overset{\pro}{\to}}
\newcommand\toPP{ \overset{\text{PP}}{\to}}
\newcommand{\oeq}{\mathrel{\text{\textcircled{$=$}}}}





\usepackage{xcolor}
% \usepackage[normalem]{ulem}
\usepackage{lipsum}
\makeatletter
% \newcommand\colorwave[1][blue]{\bgroup \markoverwith{\lower3.5\p@\hbox{\sixly \textcolor{#1}{\char58}}}\ULon}
%\font\sixly=lasy6 % does not re-load if already loaded, so no memory problem.

\newmdtheoremenv[
linewidth= 1pt,linecolor= blue,%
leftmargin=20,rightmargin=20,innertopmargin=0pt, innerrightmargin=40,%
tikzsetting = { draw=lightgray, line width = 0.3pt,dashed,%
dash pattern = on 15pt off 3pt},%
splittopskip=\topskip,skipbelow=\baselineskip,%
skipabove=\baselineskip,ntheorem,roundcorner=0pt,
% backgroundcolor=pagebg,font=\color{orange}\sffamily, fontcolor=white
]{examplebox}{Exemple}[section]



\newcommand\R{\mathbb{R}}
\newcommand\Z{\mathbb{Z}}
\newcommand\N{\mathbb{N}}
\newcommand\E{\mathbb{E}}
\newcommand\F{\mathcal{F}}
\newcommand\cH{\mathcal{H}}
\newcommand\V{\mathbb{V}}
\newcommand\dmo{ ^{-1} }
\newcommand\kapa{\kappa}
\newcommand\im{Im}
\newcommand\hs{\mathcal{H}}





\usepackage{soul}

\makeatletter
\newcommand*{\whiten}[1]{\llap{\textcolor{white}{{\the\SOUL@token}}\hspace{#1pt}}}
\DeclareRobustCommand*\myul{%
    \def\SOUL@everyspace{\underline{\space}\kern\z@}%
    \def\SOUL@everytoken{%
     \setbox0=\hbox{\the\SOUL@token}%
     \ifdim\dp0>\z@
        \raisebox{\dp0}{\underline{\phantom{\the\SOUL@token}}}%
        \whiten{1}\whiten{0}%
        \whiten{-1}\whiten{-2}%
        \llap{\the\SOUL@token}%
     \else
        \underline{\the\SOUL@token}%
     \fi}%
\SOUL@}
\makeatother

\newcommand*{\demp}{\fontfamily{lmtt}\selectfont}

\DeclareTextFontCommand{\textdemp}{\demp}

\begin{document}

\ifcomment
Multiline
comment
\fi
\ifcomment
\myul{Typesetting test}
% \color[rgb]{1,1,1}
$∑_i^n≠ 60º±∞π∆¬≈√j∫h≤≥µ$

$\CR \R\pro\ind\pro\gS\pro
\mqty[a&b\\c&d]$
$\pro\mathbb{P}$
$\dd{x}$

  \[
    \alpha(x)=\left\{
                \begin{array}{ll}
                  x\\
                  \frac{1}{1+e^{-kx}}\\
                  \frac{e^x-e^{-x}}{e^x+e^{-x}}
                \end{array}
              \right.
  \]

  $\expval{x}$
  
  $\chi_\rho(ghg\dmo)=\Tr(\rho_{ghg\dmo})=\Tr(\rho_g\circ\rho_h\circ\rho\dmo_g)=\Tr(\rho_h)\overset{\mbox{\scalebox{0.5}{$\Tr(AB)=\Tr(BA)$}}}{=}\chi_\rho(h)$
  	$\mathop{\oplus}_{\substack{x\in X}}$

$\mat(\rho_g)=(a_{ij}(g))_{\scriptsize \substack{1\leq i\leq d \\ 1\leq j\leq d}}$ et $\mat(\rho'_g)=(a'_{ij}(g))_{\scriptsize \substack{1\leq i'\leq d' \\ 1\leq j'\leq d'}}$



\[\int_a^b{\mathbb{R}^2}g(u, v)\dd{P_{XY}}(u, v)=\iint g(u,v) f_{XY}(u, v)\dd \lambda(u) \dd \lambda(v)\]
$$\lim_{x\to\infty} f(x)$$	
$$\iiiint_V \mu(t,u,v,w) \,dt\,du\,dv\,dw$$
$$\sum_{n=1}^{\infty} 2^{-n} = 1$$	
\begin{definition}
	Si $X$ et $Y$ sont 2 v.a. ou definit la \textsc{Covariance} entre $X$ et $Y$ comme
	$\cov(X,Y)\overset{\text{def}}{=}\E\left[(X-\E(X))(Y-\E(Y))\right]=\E(XY)-\E(X)\E(Y)$.
\end{definition}
\fi
\pagebreak

% \tableofcontents

% insert your code here
%\input{./algebra/main.tex}
%\input{./geometrie-differentielle/main.tex}
%\input{./probabilite/main.tex}
%\input{./analyse-fonctionnelle/main.tex}
% \input{./Analyse-convexe-et-dualite-en-optimisation/main.tex}
%\input{./tikz/main.tex}
%\input{./Theorie-du-distributions/main.tex}
%\input{./optimisation/mine.tex}
 \input{./modelisation/main.tex}

% yves.aubry@univ-tln.fr : algebra

\end{document}

%% !TEX encoding = UTF-8 Unicode
% !TEX TS-program = xelatex

\documentclass[french]{report}

%\usepackage[utf8]{inputenc}
%\usepackage[T1]{fontenc}
\usepackage{babel}


\newif\ifcomment
%\commenttrue # Show comments

\usepackage{physics}
\usepackage{amssymb}


\usepackage{amsthm}
% \usepackage{thmtools}
\usepackage{mathtools}
\usepackage{amsfonts}

\usepackage{color}

\usepackage{tikz}

\usepackage{geometry}
\geometry{a5paper, margin=0.1in, right=1cm}

\usepackage{dsfont}

\usepackage{graphicx}
\graphicspath{ {images/} }

\usepackage{faktor}

\usepackage{IEEEtrantools}
\usepackage{enumerate}   
\usepackage[PostScript=dvips]{"/Users/aware/Documents/Courses/diagrams"}


\newtheorem{theorem}{Théorème}[section]
\renewcommand{\thetheorem}{\arabic{theorem}}
\newtheorem{lemme}{Lemme}[section]
\renewcommand{\thelemme}{\arabic{lemme}}
\newtheorem{proposition}{Proposition}[section]
\renewcommand{\theproposition}{\arabic{proposition}}
\newtheorem{notations}{Notations}[section]
\newtheorem{problem}{Problème}[section]
\newtheorem{corollary}{Corollaire}[theorem]
\renewcommand{\thecorollary}{\arabic{corollary}}
\newtheorem{property}{Propriété}[section]
\newtheorem{objective}{Objectif}[section]

\theoremstyle{definition}
\newtheorem{definition}{Définition}[section]
\renewcommand{\thedefinition}{\arabic{definition}}
\newtheorem{exercise}{Exercice}[chapter]
\renewcommand{\theexercise}{\arabic{exercise}}
\newtheorem{example}{Exemple}[chapter]
\renewcommand{\theexample}{\arabic{example}}
\newtheorem*{solution}{Solution}
\newtheorem*{application}{Application}
\newtheorem*{notation}{Notation}
\newtheorem*{vocabulary}{Vocabulaire}
\newtheorem*{properties}{Propriétés}



\theoremstyle{remark}
\newtheorem*{remark}{Remarque}
\newtheorem*{rappel}{Rappel}


\usepackage{etoolbox}
\AtBeginEnvironment{exercise}{\small}
\AtBeginEnvironment{example}{\small}

\usepackage{cases}
\usepackage[red]{mypack}

\usepackage[framemethod=TikZ]{mdframed}

\definecolor{bg}{rgb}{0.4,0.25,0.95}
\definecolor{pagebg}{rgb}{0,0,0.5}
\surroundwithmdframed[
   topline=false,
   rightline=false,
   bottomline=false,
   leftmargin=\parindent,
   skipabove=8pt,
   skipbelow=8pt,
   linecolor=blue,
   innerbottommargin=10pt,
   % backgroundcolor=bg,font=\color{orange}\sffamily, fontcolor=white
]{definition}

\usepackage{empheq}
\usepackage[most]{tcolorbox}

\newtcbox{\mymath}[1][]{%
    nobeforeafter, math upper, tcbox raise base,
    enhanced, colframe=blue!30!black,
    colback=red!10, boxrule=1pt,
    #1}

\usepackage{unixode}


\DeclareMathOperator{\ord}{ord}
\DeclareMathOperator{\orb}{orb}
\DeclareMathOperator{\stab}{stab}
\DeclareMathOperator{\Stab}{stab}
\DeclareMathOperator{\ppcm}{ppcm}
\DeclareMathOperator{\conj}{Conj}
\DeclareMathOperator{\End}{End}
\DeclareMathOperator{\rot}{rot}
\DeclareMathOperator{\trs}{trace}
\DeclareMathOperator{\Ind}{Ind}
\DeclareMathOperator{\mat}{Mat}
\DeclareMathOperator{\id}{Id}
\DeclareMathOperator{\vect}{vect}
\DeclareMathOperator{\img}{img}
\DeclareMathOperator{\cov}{Cov}
\DeclareMathOperator{\dist}{dist}
\DeclareMathOperator{\irr}{Irr}
\DeclareMathOperator{\image}{Im}
\DeclareMathOperator{\pd}{\partial}
\DeclareMathOperator{\epi}{epi}
\DeclareMathOperator{\Argmin}{Argmin}
\DeclareMathOperator{\dom}{dom}
\DeclareMathOperator{\proj}{proj}
\DeclareMathOperator{\ctg}{ctg}
\DeclareMathOperator{\supp}{supp}
\DeclareMathOperator{\argmin}{argmin}
\DeclareMathOperator{\mult}{mult}
\DeclareMathOperator{\ch}{ch}
\DeclareMathOperator{\sh}{sh}
\DeclareMathOperator{\rang}{rang}
\DeclareMathOperator{\diam}{diam}
\DeclareMathOperator{\Epigraphe}{Epigraphe}




\usepackage{xcolor}
\everymath{\color{blue}}
%\everymath{\color[rgb]{0,1,1}}
%\pagecolor[rgb]{0,0,0.5}


\newcommand*{\pdtest}[3][]{\ensuremath{\frac{\partial^{#1} #2}{\partial #3}}}

\newcommand*{\deffunc}[6][]{\ensuremath{
\begin{array}{rcl}
#2 : #3 &\rightarrow& #4\\
#5 &\mapsto& #6
\end{array}
}}

\newcommand{\eqcolon}{\mathrel{\resizebox{\widthof{$\mathord{=}$}}{\height}{ $\!\!=\!\!\resizebox{1.2\width}{0.8\height}{\raisebox{0.23ex}{$\mathop{:}$}}\!\!$ }}}
\newcommand{\coloneq}{\mathrel{\resizebox{\widthof{$\mathord{=}$}}{\height}{ $\!\!\resizebox{1.2\width}{0.8\height}{\raisebox{0.23ex}{$\mathop{:}$}}\!\!=\!\!$ }}}
\newcommand{\eqcolonl}{\ensuremath{\mathrel{=\!\!\mathop{:}}}}
\newcommand{\coloneql}{\ensuremath{\mathrel{\mathop{:} \!\! =}}}
\newcommand{\vc}[1]{% inline column vector
  \left(\begin{smallmatrix}#1\end{smallmatrix}\right)%
}
\newcommand{\vr}[1]{% inline row vector
  \begin{smallmatrix}(\,#1\,)\end{smallmatrix}%
}
\makeatletter
\newcommand*{\defeq}{\ =\mathrel{\rlap{%
                     \raisebox{0.3ex}{$\m@th\cdot$}}%
                     \raisebox{-0.3ex}{$\m@th\cdot$}}%
                     }
\makeatother

\newcommand{\mathcircle}[1]{% inline row vector
 \overset{\circ}{#1}
}
\newcommand{\ulim}{% low limit
 \underline{\lim}
}
\newcommand{\ssi}{% iff
\iff
}
\newcommand{\ps}[2]{
\expval{#1 | #2}
}
\newcommand{\df}[1]{
\mqty{#1}
}
\newcommand{\n}[1]{
\norm{#1}
}
\newcommand{\sys}[1]{
\left\{\smqty{#1}\right.
}


\newcommand{\eqdef}{\ensuremath{\overset{\text{def}}=}}


\def\Circlearrowright{\ensuremath{%
  \rotatebox[origin=c]{230}{$\circlearrowright$}}}

\newcommand\ct[1]{\text{\rmfamily\upshape #1}}
\newcommand\question[1]{ {\color{red} ...!? \small #1}}
\newcommand\caz[1]{\left\{\begin{array} #1 \end{array}\right.}
\newcommand\const{\text{\rmfamily\upshape const}}
\newcommand\toP{ \overset{\pro}{\to}}
\newcommand\toPP{ \overset{\text{PP}}{\to}}
\newcommand{\oeq}{\mathrel{\text{\textcircled{$=$}}}}





\usepackage{xcolor}
% \usepackage[normalem]{ulem}
\usepackage{lipsum}
\makeatletter
% \newcommand\colorwave[1][blue]{\bgroup \markoverwith{\lower3.5\p@\hbox{\sixly \textcolor{#1}{\char58}}}\ULon}
%\font\sixly=lasy6 % does not re-load if already loaded, so no memory problem.

\newmdtheoremenv[
linewidth= 1pt,linecolor= blue,%
leftmargin=20,rightmargin=20,innertopmargin=0pt, innerrightmargin=40,%
tikzsetting = { draw=lightgray, line width = 0.3pt,dashed,%
dash pattern = on 15pt off 3pt},%
splittopskip=\topskip,skipbelow=\baselineskip,%
skipabove=\baselineskip,ntheorem,roundcorner=0pt,
% backgroundcolor=pagebg,font=\color{orange}\sffamily, fontcolor=white
]{examplebox}{Exemple}[section]



\newcommand\R{\mathbb{R}}
\newcommand\Z{\mathbb{Z}}
\newcommand\N{\mathbb{N}}
\newcommand\E{\mathbb{E}}
\newcommand\F{\mathcal{F}}
\newcommand\cH{\mathcal{H}}
\newcommand\V{\mathbb{V}}
\newcommand\dmo{ ^{-1} }
\newcommand\kapa{\kappa}
\newcommand\im{Im}
\newcommand\hs{\mathcal{H}}





\usepackage{soul}

\makeatletter
\newcommand*{\whiten}[1]{\llap{\textcolor{white}{{\the\SOUL@token}}\hspace{#1pt}}}
\DeclareRobustCommand*\myul{%
    \def\SOUL@everyspace{\underline{\space}\kern\z@}%
    \def\SOUL@everytoken{%
     \setbox0=\hbox{\the\SOUL@token}%
     \ifdim\dp0>\z@
        \raisebox{\dp0}{\underline{\phantom{\the\SOUL@token}}}%
        \whiten{1}\whiten{0}%
        \whiten{-1}\whiten{-2}%
        \llap{\the\SOUL@token}%
     \else
        \underline{\the\SOUL@token}%
     \fi}%
\SOUL@}
\makeatother

\newcommand*{\demp}{\fontfamily{lmtt}\selectfont}

\DeclareTextFontCommand{\textdemp}{\demp}

\begin{document}

\ifcomment
Multiline
comment
\fi
\ifcomment
\myul{Typesetting test}
% \color[rgb]{1,1,1}
$∑_i^n≠ 60º±∞π∆¬≈√j∫h≤≥µ$

$\CR \R\pro\ind\pro\gS\pro
\mqty[a&b\\c&d]$
$\pro\mathbb{P}$
$\dd{x}$

  \[
    \alpha(x)=\left\{
                \begin{array}{ll}
                  x\\
                  \frac{1}{1+e^{-kx}}\\
                  \frac{e^x-e^{-x}}{e^x+e^{-x}}
                \end{array}
              \right.
  \]

  $\expval{x}$
  
  $\chi_\rho(ghg\dmo)=\Tr(\rho_{ghg\dmo})=\Tr(\rho_g\circ\rho_h\circ\rho\dmo_g)=\Tr(\rho_h)\overset{\mbox{\scalebox{0.5}{$\Tr(AB)=\Tr(BA)$}}}{=}\chi_\rho(h)$
  	$\mathop{\oplus}_{\substack{x\in X}}$

$\mat(\rho_g)=(a_{ij}(g))_{\scriptsize \substack{1\leq i\leq d \\ 1\leq j\leq d}}$ et $\mat(\rho'_g)=(a'_{ij}(g))_{\scriptsize \substack{1\leq i'\leq d' \\ 1\leq j'\leq d'}}$



\[\int_a^b{\mathbb{R}^2}g(u, v)\dd{P_{XY}}(u, v)=\iint g(u,v) f_{XY}(u, v)\dd \lambda(u) \dd \lambda(v)\]
$$\lim_{x\to\infty} f(x)$$	
$$\iiiint_V \mu(t,u,v,w) \,dt\,du\,dv\,dw$$
$$\sum_{n=1}^{\infty} 2^{-n} = 1$$	
\begin{definition}
	Si $X$ et $Y$ sont 2 v.a. ou definit la \textsc{Covariance} entre $X$ et $Y$ comme
	$\cov(X,Y)\overset{\text{def}}{=}\E\left[(X-\E(X))(Y-\E(Y))\right]=\E(XY)-\E(X)\E(Y)$.
\end{definition}
\fi
\pagebreak

% \tableofcontents

% insert your code here
%\input{./algebra/main.tex}
%\input{./geometrie-differentielle/main.tex}
%\input{./probabilite/main.tex}
%\input{./analyse-fonctionnelle/main.tex}
% \input{./Analyse-convexe-et-dualite-en-optimisation/main.tex}
%\input{./tikz/main.tex}
%\input{./Theorie-du-distributions/main.tex}
%\input{./optimisation/mine.tex}
 \input{./modelisation/main.tex}

% yves.aubry@univ-tln.fr : algebra

\end{document}

%% !TEX encoding = UTF-8 Unicode
% !TEX TS-program = xelatex

\documentclass[french]{report}

%\usepackage[utf8]{inputenc}
%\usepackage[T1]{fontenc}
\usepackage{babel}


\newif\ifcomment
%\commenttrue # Show comments

\usepackage{physics}
\usepackage{amssymb}


\usepackage{amsthm}
% \usepackage{thmtools}
\usepackage{mathtools}
\usepackage{amsfonts}

\usepackage{color}

\usepackage{tikz}

\usepackage{geometry}
\geometry{a5paper, margin=0.1in, right=1cm}

\usepackage{dsfont}

\usepackage{graphicx}
\graphicspath{ {images/} }

\usepackage{faktor}

\usepackage{IEEEtrantools}
\usepackage{enumerate}   
\usepackage[PostScript=dvips]{"/Users/aware/Documents/Courses/diagrams"}


\newtheorem{theorem}{Théorème}[section]
\renewcommand{\thetheorem}{\arabic{theorem}}
\newtheorem{lemme}{Lemme}[section]
\renewcommand{\thelemme}{\arabic{lemme}}
\newtheorem{proposition}{Proposition}[section]
\renewcommand{\theproposition}{\arabic{proposition}}
\newtheorem{notations}{Notations}[section]
\newtheorem{problem}{Problème}[section]
\newtheorem{corollary}{Corollaire}[theorem]
\renewcommand{\thecorollary}{\arabic{corollary}}
\newtheorem{property}{Propriété}[section]
\newtheorem{objective}{Objectif}[section]

\theoremstyle{definition}
\newtheorem{definition}{Définition}[section]
\renewcommand{\thedefinition}{\arabic{definition}}
\newtheorem{exercise}{Exercice}[chapter]
\renewcommand{\theexercise}{\arabic{exercise}}
\newtheorem{example}{Exemple}[chapter]
\renewcommand{\theexample}{\arabic{example}}
\newtheorem*{solution}{Solution}
\newtheorem*{application}{Application}
\newtheorem*{notation}{Notation}
\newtheorem*{vocabulary}{Vocabulaire}
\newtheorem*{properties}{Propriétés}



\theoremstyle{remark}
\newtheorem*{remark}{Remarque}
\newtheorem*{rappel}{Rappel}


\usepackage{etoolbox}
\AtBeginEnvironment{exercise}{\small}
\AtBeginEnvironment{example}{\small}

\usepackage{cases}
\usepackage[red]{mypack}

\usepackage[framemethod=TikZ]{mdframed}

\definecolor{bg}{rgb}{0.4,0.25,0.95}
\definecolor{pagebg}{rgb}{0,0,0.5}
\surroundwithmdframed[
   topline=false,
   rightline=false,
   bottomline=false,
   leftmargin=\parindent,
   skipabove=8pt,
   skipbelow=8pt,
   linecolor=blue,
   innerbottommargin=10pt,
   % backgroundcolor=bg,font=\color{orange}\sffamily, fontcolor=white
]{definition}

\usepackage{empheq}
\usepackage[most]{tcolorbox}

\newtcbox{\mymath}[1][]{%
    nobeforeafter, math upper, tcbox raise base,
    enhanced, colframe=blue!30!black,
    colback=red!10, boxrule=1pt,
    #1}

\usepackage{unixode}


\DeclareMathOperator{\ord}{ord}
\DeclareMathOperator{\orb}{orb}
\DeclareMathOperator{\stab}{stab}
\DeclareMathOperator{\Stab}{stab}
\DeclareMathOperator{\ppcm}{ppcm}
\DeclareMathOperator{\conj}{Conj}
\DeclareMathOperator{\End}{End}
\DeclareMathOperator{\rot}{rot}
\DeclareMathOperator{\trs}{trace}
\DeclareMathOperator{\Ind}{Ind}
\DeclareMathOperator{\mat}{Mat}
\DeclareMathOperator{\id}{Id}
\DeclareMathOperator{\vect}{vect}
\DeclareMathOperator{\img}{img}
\DeclareMathOperator{\cov}{Cov}
\DeclareMathOperator{\dist}{dist}
\DeclareMathOperator{\irr}{Irr}
\DeclareMathOperator{\image}{Im}
\DeclareMathOperator{\pd}{\partial}
\DeclareMathOperator{\epi}{epi}
\DeclareMathOperator{\Argmin}{Argmin}
\DeclareMathOperator{\dom}{dom}
\DeclareMathOperator{\proj}{proj}
\DeclareMathOperator{\ctg}{ctg}
\DeclareMathOperator{\supp}{supp}
\DeclareMathOperator{\argmin}{argmin}
\DeclareMathOperator{\mult}{mult}
\DeclareMathOperator{\ch}{ch}
\DeclareMathOperator{\sh}{sh}
\DeclareMathOperator{\rang}{rang}
\DeclareMathOperator{\diam}{diam}
\DeclareMathOperator{\Epigraphe}{Epigraphe}




\usepackage{xcolor}
\everymath{\color{blue}}
%\everymath{\color[rgb]{0,1,1}}
%\pagecolor[rgb]{0,0,0.5}


\newcommand*{\pdtest}[3][]{\ensuremath{\frac{\partial^{#1} #2}{\partial #3}}}

\newcommand*{\deffunc}[6][]{\ensuremath{
\begin{array}{rcl}
#2 : #3 &\rightarrow& #4\\
#5 &\mapsto& #6
\end{array}
}}

\newcommand{\eqcolon}{\mathrel{\resizebox{\widthof{$\mathord{=}$}}{\height}{ $\!\!=\!\!\resizebox{1.2\width}{0.8\height}{\raisebox{0.23ex}{$\mathop{:}$}}\!\!$ }}}
\newcommand{\coloneq}{\mathrel{\resizebox{\widthof{$\mathord{=}$}}{\height}{ $\!\!\resizebox{1.2\width}{0.8\height}{\raisebox{0.23ex}{$\mathop{:}$}}\!\!=\!\!$ }}}
\newcommand{\eqcolonl}{\ensuremath{\mathrel{=\!\!\mathop{:}}}}
\newcommand{\coloneql}{\ensuremath{\mathrel{\mathop{:} \!\! =}}}
\newcommand{\vc}[1]{% inline column vector
  \left(\begin{smallmatrix}#1\end{smallmatrix}\right)%
}
\newcommand{\vr}[1]{% inline row vector
  \begin{smallmatrix}(\,#1\,)\end{smallmatrix}%
}
\makeatletter
\newcommand*{\defeq}{\ =\mathrel{\rlap{%
                     \raisebox{0.3ex}{$\m@th\cdot$}}%
                     \raisebox{-0.3ex}{$\m@th\cdot$}}%
                     }
\makeatother

\newcommand{\mathcircle}[1]{% inline row vector
 \overset{\circ}{#1}
}
\newcommand{\ulim}{% low limit
 \underline{\lim}
}
\newcommand{\ssi}{% iff
\iff
}
\newcommand{\ps}[2]{
\expval{#1 | #2}
}
\newcommand{\df}[1]{
\mqty{#1}
}
\newcommand{\n}[1]{
\norm{#1}
}
\newcommand{\sys}[1]{
\left\{\smqty{#1}\right.
}


\newcommand{\eqdef}{\ensuremath{\overset{\text{def}}=}}


\def\Circlearrowright{\ensuremath{%
  \rotatebox[origin=c]{230}{$\circlearrowright$}}}

\newcommand\ct[1]{\text{\rmfamily\upshape #1}}
\newcommand\question[1]{ {\color{red} ...!? \small #1}}
\newcommand\caz[1]{\left\{\begin{array} #1 \end{array}\right.}
\newcommand\const{\text{\rmfamily\upshape const}}
\newcommand\toP{ \overset{\pro}{\to}}
\newcommand\toPP{ \overset{\text{PP}}{\to}}
\newcommand{\oeq}{\mathrel{\text{\textcircled{$=$}}}}





\usepackage{xcolor}
% \usepackage[normalem]{ulem}
\usepackage{lipsum}
\makeatletter
% \newcommand\colorwave[1][blue]{\bgroup \markoverwith{\lower3.5\p@\hbox{\sixly \textcolor{#1}{\char58}}}\ULon}
%\font\sixly=lasy6 % does not re-load if already loaded, so no memory problem.

\newmdtheoremenv[
linewidth= 1pt,linecolor= blue,%
leftmargin=20,rightmargin=20,innertopmargin=0pt, innerrightmargin=40,%
tikzsetting = { draw=lightgray, line width = 0.3pt,dashed,%
dash pattern = on 15pt off 3pt},%
splittopskip=\topskip,skipbelow=\baselineskip,%
skipabove=\baselineskip,ntheorem,roundcorner=0pt,
% backgroundcolor=pagebg,font=\color{orange}\sffamily, fontcolor=white
]{examplebox}{Exemple}[section]



\newcommand\R{\mathbb{R}}
\newcommand\Z{\mathbb{Z}}
\newcommand\N{\mathbb{N}}
\newcommand\E{\mathbb{E}}
\newcommand\F{\mathcal{F}}
\newcommand\cH{\mathcal{H}}
\newcommand\V{\mathbb{V}}
\newcommand\dmo{ ^{-1} }
\newcommand\kapa{\kappa}
\newcommand\im{Im}
\newcommand\hs{\mathcal{H}}





\usepackage{soul}

\makeatletter
\newcommand*{\whiten}[1]{\llap{\textcolor{white}{{\the\SOUL@token}}\hspace{#1pt}}}
\DeclareRobustCommand*\myul{%
    \def\SOUL@everyspace{\underline{\space}\kern\z@}%
    \def\SOUL@everytoken{%
     \setbox0=\hbox{\the\SOUL@token}%
     \ifdim\dp0>\z@
        \raisebox{\dp0}{\underline{\phantom{\the\SOUL@token}}}%
        \whiten{1}\whiten{0}%
        \whiten{-1}\whiten{-2}%
        \llap{\the\SOUL@token}%
     \else
        \underline{\the\SOUL@token}%
     \fi}%
\SOUL@}
\makeatother

\newcommand*{\demp}{\fontfamily{lmtt}\selectfont}

\DeclareTextFontCommand{\textdemp}{\demp}

\begin{document}

\ifcomment
Multiline
comment
\fi
\ifcomment
\myul{Typesetting test}
% \color[rgb]{1,1,1}
$∑_i^n≠ 60º±∞π∆¬≈√j∫h≤≥µ$

$\CR \R\pro\ind\pro\gS\pro
\mqty[a&b\\c&d]$
$\pro\mathbb{P}$
$\dd{x}$

  \[
    \alpha(x)=\left\{
                \begin{array}{ll}
                  x\\
                  \frac{1}{1+e^{-kx}}\\
                  \frac{e^x-e^{-x}}{e^x+e^{-x}}
                \end{array}
              \right.
  \]

  $\expval{x}$
  
  $\chi_\rho(ghg\dmo)=\Tr(\rho_{ghg\dmo})=\Tr(\rho_g\circ\rho_h\circ\rho\dmo_g)=\Tr(\rho_h)\overset{\mbox{\scalebox{0.5}{$\Tr(AB)=\Tr(BA)$}}}{=}\chi_\rho(h)$
  	$\mathop{\oplus}_{\substack{x\in X}}$

$\mat(\rho_g)=(a_{ij}(g))_{\scriptsize \substack{1\leq i\leq d \\ 1\leq j\leq d}}$ et $\mat(\rho'_g)=(a'_{ij}(g))_{\scriptsize \substack{1\leq i'\leq d' \\ 1\leq j'\leq d'}}$



\[\int_a^b{\mathbb{R}^2}g(u, v)\dd{P_{XY}}(u, v)=\iint g(u,v) f_{XY}(u, v)\dd \lambda(u) \dd \lambda(v)\]
$$\lim_{x\to\infty} f(x)$$	
$$\iiiint_V \mu(t,u,v,w) \,dt\,du\,dv\,dw$$
$$\sum_{n=1}^{\infty} 2^{-n} = 1$$	
\begin{definition}
	Si $X$ et $Y$ sont 2 v.a. ou definit la \textsc{Covariance} entre $X$ et $Y$ comme
	$\cov(X,Y)\overset{\text{def}}{=}\E\left[(X-\E(X))(Y-\E(Y))\right]=\E(XY)-\E(X)\E(Y)$.
\end{definition}
\fi
\pagebreak

% \tableofcontents

% insert your code here
%\input{./algebra/main.tex}
%\input{./geometrie-differentielle/main.tex}
%\input{./probabilite/main.tex}
%\input{./analyse-fonctionnelle/main.tex}
% \input{./Analyse-convexe-et-dualite-en-optimisation/main.tex}
%\input{./tikz/main.tex}
%\input{./Theorie-du-distributions/main.tex}
%\input{./optimisation/mine.tex}
 \input{./modelisation/main.tex}

% yves.aubry@univ-tln.fr : algebra

\end{document}

% % !TEX encoding = UTF-8 Unicode
% !TEX TS-program = xelatex

\documentclass[french]{report}

%\usepackage[utf8]{inputenc}
%\usepackage[T1]{fontenc}
\usepackage{babel}


\newif\ifcomment
%\commenttrue # Show comments

\usepackage{physics}
\usepackage{amssymb}


\usepackage{amsthm}
% \usepackage{thmtools}
\usepackage{mathtools}
\usepackage{amsfonts}

\usepackage{color}

\usepackage{tikz}

\usepackage{geometry}
\geometry{a5paper, margin=0.1in, right=1cm}

\usepackage{dsfont}

\usepackage{graphicx}
\graphicspath{ {images/} }

\usepackage{faktor}

\usepackage{IEEEtrantools}
\usepackage{enumerate}   
\usepackage[PostScript=dvips]{"/Users/aware/Documents/Courses/diagrams"}


\newtheorem{theorem}{Théorème}[section]
\renewcommand{\thetheorem}{\arabic{theorem}}
\newtheorem{lemme}{Lemme}[section]
\renewcommand{\thelemme}{\arabic{lemme}}
\newtheorem{proposition}{Proposition}[section]
\renewcommand{\theproposition}{\arabic{proposition}}
\newtheorem{notations}{Notations}[section]
\newtheorem{problem}{Problème}[section]
\newtheorem{corollary}{Corollaire}[theorem]
\renewcommand{\thecorollary}{\arabic{corollary}}
\newtheorem{property}{Propriété}[section]
\newtheorem{objective}{Objectif}[section]

\theoremstyle{definition}
\newtheorem{definition}{Définition}[section]
\renewcommand{\thedefinition}{\arabic{definition}}
\newtheorem{exercise}{Exercice}[chapter]
\renewcommand{\theexercise}{\arabic{exercise}}
\newtheorem{example}{Exemple}[chapter]
\renewcommand{\theexample}{\arabic{example}}
\newtheorem*{solution}{Solution}
\newtheorem*{application}{Application}
\newtheorem*{notation}{Notation}
\newtheorem*{vocabulary}{Vocabulaire}
\newtheorem*{properties}{Propriétés}



\theoremstyle{remark}
\newtheorem*{remark}{Remarque}
\newtheorem*{rappel}{Rappel}


\usepackage{etoolbox}
\AtBeginEnvironment{exercise}{\small}
\AtBeginEnvironment{example}{\small}

\usepackage{cases}
\usepackage[red]{mypack}

\usepackage[framemethod=TikZ]{mdframed}

\definecolor{bg}{rgb}{0.4,0.25,0.95}
\definecolor{pagebg}{rgb}{0,0,0.5}
\surroundwithmdframed[
   topline=false,
   rightline=false,
   bottomline=false,
   leftmargin=\parindent,
   skipabove=8pt,
   skipbelow=8pt,
   linecolor=blue,
   innerbottommargin=10pt,
   % backgroundcolor=bg,font=\color{orange}\sffamily, fontcolor=white
]{definition}

\usepackage{empheq}
\usepackage[most]{tcolorbox}

\newtcbox{\mymath}[1][]{%
    nobeforeafter, math upper, tcbox raise base,
    enhanced, colframe=blue!30!black,
    colback=red!10, boxrule=1pt,
    #1}

\usepackage{unixode}


\DeclareMathOperator{\ord}{ord}
\DeclareMathOperator{\orb}{orb}
\DeclareMathOperator{\stab}{stab}
\DeclareMathOperator{\Stab}{stab}
\DeclareMathOperator{\ppcm}{ppcm}
\DeclareMathOperator{\conj}{Conj}
\DeclareMathOperator{\End}{End}
\DeclareMathOperator{\rot}{rot}
\DeclareMathOperator{\trs}{trace}
\DeclareMathOperator{\Ind}{Ind}
\DeclareMathOperator{\mat}{Mat}
\DeclareMathOperator{\id}{Id}
\DeclareMathOperator{\vect}{vect}
\DeclareMathOperator{\img}{img}
\DeclareMathOperator{\cov}{Cov}
\DeclareMathOperator{\dist}{dist}
\DeclareMathOperator{\irr}{Irr}
\DeclareMathOperator{\image}{Im}
\DeclareMathOperator{\pd}{\partial}
\DeclareMathOperator{\epi}{epi}
\DeclareMathOperator{\Argmin}{Argmin}
\DeclareMathOperator{\dom}{dom}
\DeclareMathOperator{\proj}{proj}
\DeclareMathOperator{\ctg}{ctg}
\DeclareMathOperator{\supp}{supp}
\DeclareMathOperator{\argmin}{argmin}
\DeclareMathOperator{\mult}{mult}
\DeclareMathOperator{\ch}{ch}
\DeclareMathOperator{\sh}{sh}
\DeclareMathOperator{\rang}{rang}
\DeclareMathOperator{\diam}{diam}
\DeclareMathOperator{\Epigraphe}{Epigraphe}




\usepackage{xcolor}
\everymath{\color{blue}}
%\everymath{\color[rgb]{0,1,1}}
%\pagecolor[rgb]{0,0,0.5}


\newcommand*{\pdtest}[3][]{\ensuremath{\frac{\partial^{#1} #2}{\partial #3}}}

\newcommand*{\deffunc}[6][]{\ensuremath{
\begin{array}{rcl}
#2 : #3 &\rightarrow& #4\\
#5 &\mapsto& #6
\end{array}
}}

\newcommand{\eqcolon}{\mathrel{\resizebox{\widthof{$\mathord{=}$}}{\height}{ $\!\!=\!\!\resizebox{1.2\width}{0.8\height}{\raisebox{0.23ex}{$\mathop{:}$}}\!\!$ }}}
\newcommand{\coloneq}{\mathrel{\resizebox{\widthof{$\mathord{=}$}}{\height}{ $\!\!\resizebox{1.2\width}{0.8\height}{\raisebox{0.23ex}{$\mathop{:}$}}\!\!=\!\!$ }}}
\newcommand{\eqcolonl}{\ensuremath{\mathrel{=\!\!\mathop{:}}}}
\newcommand{\coloneql}{\ensuremath{\mathrel{\mathop{:} \!\! =}}}
\newcommand{\vc}[1]{% inline column vector
  \left(\begin{smallmatrix}#1\end{smallmatrix}\right)%
}
\newcommand{\vr}[1]{% inline row vector
  \begin{smallmatrix}(\,#1\,)\end{smallmatrix}%
}
\makeatletter
\newcommand*{\defeq}{\ =\mathrel{\rlap{%
                     \raisebox{0.3ex}{$\m@th\cdot$}}%
                     \raisebox{-0.3ex}{$\m@th\cdot$}}%
                     }
\makeatother

\newcommand{\mathcircle}[1]{% inline row vector
 \overset{\circ}{#1}
}
\newcommand{\ulim}{% low limit
 \underline{\lim}
}
\newcommand{\ssi}{% iff
\iff
}
\newcommand{\ps}[2]{
\expval{#1 | #2}
}
\newcommand{\df}[1]{
\mqty{#1}
}
\newcommand{\n}[1]{
\norm{#1}
}
\newcommand{\sys}[1]{
\left\{\smqty{#1}\right.
}


\newcommand{\eqdef}{\ensuremath{\overset{\text{def}}=}}


\def\Circlearrowright{\ensuremath{%
  \rotatebox[origin=c]{230}{$\circlearrowright$}}}

\newcommand\ct[1]{\text{\rmfamily\upshape #1}}
\newcommand\question[1]{ {\color{red} ...!? \small #1}}
\newcommand\caz[1]{\left\{\begin{array} #1 \end{array}\right.}
\newcommand\const{\text{\rmfamily\upshape const}}
\newcommand\toP{ \overset{\pro}{\to}}
\newcommand\toPP{ \overset{\text{PP}}{\to}}
\newcommand{\oeq}{\mathrel{\text{\textcircled{$=$}}}}





\usepackage{xcolor}
% \usepackage[normalem]{ulem}
\usepackage{lipsum}
\makeatletter
% \newcommand\colorwave[1][blue]{\bgroup \markoverwith{\lower3.5\p@\hbox{\sixly \textcolor{#1}{\char58}}}\ULon}
%\font\sixly=lasy6 % does not re-load if already loaded, so no memory problem.

\newmdtheoremenv[
linewidth= 1pt,linecolor= blue,%
leftmargin=20,rightmargin=20,innertopmargin=0pt, innerrightmargin=40,%
tikzsetting = { draw=lightgray, line width = 0.3pt,dashed,%
dash pattern = on 15pt off 3pt},%
splittopskip=\topskip,skipbelow=\baselineskip,%
skipabove=\baselineskip,ntheorem,roundcorner=0pt,
% backgroundcolor=pagebg,font=\color{orange}\sffamily, fontcolor=white
]{examplebox}{Exemple}[section]



\newcommand\R{\mathbb{R}}
\newcommand\Z{\mathbb{Z}}
\newcommand\N{\mathbb{N}}
\newcommand\E{\mathbb{E}}
\newcommand\F{\mathcal{F}}
\newcommand\cH{\mathcal{H}}
\newcommand\V{\mathbb{V}}
\newcommand\dmo{ ^{-1} }
\newcommand\kapa{\kappa}
\newcommand\im{Im}
\newcommand\hs{\mathcal{H}}





\usepackage{soul}

\makeatletter
\newcommand*{\whiten}[1]{\llap{\textcolor{white}{{\the\SOUL@token}}\hspace{#1pt}}}
\DeclareRobustCommand*\myul{%
    \def\SOUL@everyspace{\underline{\space}\kern\z@}%
    \def\SOUL@everytoken{%
     \setbox0=\hbox{\the\SOUL@token}%
     \ifdim\dp0>\z@
        \raisebox{\dp0}{\underline{\phantom{\the\SOUL@token}}}%
        \whiten{1}\whiten{0}%
        \whiten{-1}\whiten{-2}%
        \llap{\the\SOUL@token}%
     \else
        \underline{\the\SOUL@token}%
     \fi}%
\SOUL@}
\makeatother

\newcommand*{\demp}{\fontfamily{lmtt}\selectfont}

\DeclareTextFontCommand{\textdemp}{\demp}

\begin{document}

\ifcomment
Multiline
comment
\fi
\ifcomment
\myul{Typesetting test}
% \color[rgb]{1,1,1}
$∑_i^n≠ 60º±∞π∆¬≈√j∫h≤≥µ$

$\CR \R\pro\ind\pro\gS\pro
\mqty[a&b\\c&d]$
$\pro\mathbb{P}$
$\dd{x}$

  \[
    \alpha(x)=\left\{
                \begin{array}{ll}
                  x\\
                  \frac{1}{1+e^{-kx}}\\
                  \frac{e^x-e^{-x}}{e^x+e^{-x}}
                \end{array}
              \right.
  \]

  $\expval{x}$
  
  $\chi_\rho(ghg\dmo)=\Tr(\rho_{ghg\dmo})=\Tr(\rho_g\circ\rho_h\circ\rho\dmo_g)=\Tr(\rho_h)\overset{\mbox{\scalebox{0.5}{$\Tr(AB)=\Tr(BA)$}}}{=}\chi_\rho(h)$
  	$\mathop{\oplus}_{\substack{x\in X}}$

$\mat(\rho_g)=(a_{ij}(g))_{\scriptsize \substack{1\leq i\leq d \\ 1\leq j\leq d}}$ et $\mat(\rho'_g)=(a'_{ij}(g))_{\scriptsize \substack{1\leq i'\leq d' \\ 1\leq j'\leq d'}}$



\[\int_a^b{\mathbb{R}^2}g(u, v)\dd{P_{XY}}(u, v)=\iint g(u,v) f_{XY}(u, v)\dd \lambda(u) \dd \lambda(v)\]
$$\lim_{x\to\infty} f(x)$$	
$$\iiiint_V \mu(t,u,v,w) \,dt\,du\,dv\,dw$$
$$\sum_{n=1}^{\infty} 2^{-n} = 1$$	
\begin{definition}
	Si $X$ et $Y$ sont 2 v.a. ou definit la \textsc{Covariance} entre $X$ et $Y$ comme
	$\cov(X,Y)\overset{\text{def}}{=}\E\left[(X-\E(X))(Y-\E(Y))\right]=\E(XY)-\E(X)\E(Y)$.
\end{definition}
\fi
\pagebreak

% \tableofcontents

% insert your code here
%\input{./algebra/main.tex}
%\input{./geometrie-differentielle/main.tex}
%\input{./probabilite/main.tex}
%\input{./analyse-fonctionnelle/main.tex}
% \input{./Analyse-convexe-et-dualite-en-optimisation/main.tex}
%\input{./tikz/main.tex}
%\input{./Theorie-du-distributions/main.tex}
%\input{./optimisation/mine.tex}
 \input{./modelisation/main.tex}

% yves.aubry@univ-tln.fr : algebra

\end{document}

%% !TEX encoding = UTF-8 Unicode
% !TEX TS-program = xelatex

\documentclass[french]{report}

%\usepackage[utf8]{inputenc}
%\usepackage[T1]{fontenc}
\usepackage{babel}


\newif\ifcomment
%\commenttrue # Show comments

\usepackage{physics}
\usepackage{amssymb}


\usepackage{amsthm}
% \usepackage{thmtools}
\usepackage{mathtools}
\usepackage{amsfonts}

\usepackage{color}

\usepackage{tikz}

\usepackage{geometry}
\geometry{a5paper, margin=0.1in, right=1cm}

\usepackage{dsfont}

\usepackage{graphicx}
\graphicspath{ {images/} }

\usepackage{faktor}

\usepackage{IEEEtrantools}
\usepackage{enumerate}   
\usepackage[PostScript=dvips]{"/Users/aware/Documents/Courses/diagrams"}


\newtheorem{theorem}{Théorème}[section]
\renewcommand{\thetheorem}{\arabic{theorem}}
\newtheorem{lemme}{Lemme}[section]
\renewcommand{\thelemme}{\arabic{lemme}}
\newtheorem{proposition}{Proposition}[section]
\renewcommand{\theproposition}{\arabic{proposition}}
\newtheorem{notations}{Notations}[section]
\newtheorem{problem}{Problème}[section]
\newtheorem{corollary}{Corollaire}[theorem]
\renewcommand{\thecorollary}{\arabic{corollary}}
\newtheorem{property}{Propriété}[section]
\newtheorem{objective}{Objectif}[section]

\theoremstyle{definition}
\newtheorem{definition}{Définition}[section]
\renewcommand{\thedefinition}{\arabic{definition}}
\newtheorem{exercise}{Exercice}[chapter]
\renewcommand{\theexercise}{\arabic{exercise}}
\newtheorem{example}{Exemple}[chapter]
\renewcommand{\theexample}{\arabic{example}}
\newtheorem*{solution}{Solution}
\newtheorem*{application}{Application}
\newtheorem*{notation}{Notation}
\newtheorem*{vocabulary}{Vocabulaire}
\newtheorem*{properties}{Propriétés}



\theoremstyle{remark}
\newtheorem*{remark}{Remarque}
\newtheorem*{rappel}{Rappel}


\usepackage{etoolbox}
\AtBeginEnvironment{exercise}{\small}
\AtBeginEnvironment{example}{\small}

\usepackage{cases}
\usepackage[red]{mypack}

\usepackage[framemethod=TikZ]{mdframed}

\definecolor{bg}{rgb}{0.4,0.25,0.95}
\definecolor{pagebg}{rgb}{0,0,0.5}
\surroundwithmdframed[
   topline=false,
   rightline=false,
   bottomline=false,
   leftmargin=\parindent,
   skipabove=8pt,
   skipbelow=8pt,
   linecolor=blue,
   innerbottommargin=10pt,
   % backgroundcolor=bg,font=\color{orange}\sffamily, fontcolor=white
]{definition}

\usepackage{empheq}
\usepackage[most]{tcolorbox}

\newtcbox{\mymath}[1][]{%
    nobeforeafter, math upper, tcbox raise base,
    enhanced, colframe=blue!30!black,
    colback=red!10, boxrule=1pt,
    #1}

\usepackage{unixode}


\DeclareMathOperator{\ord}{ord}
\DeclareMathOperator{\orb}{orb}
\DeclareMathOperator{\stab}{stab}
\DeclareMathOperator{\Stab}{stab}
\DeclareMathOperator{\ppcm}{ppcm}
\DeclareMathOperator{\conj}{Conj}
\DeclareMathOperator{\End}{End}
\DeclareMathOperator{\rot}{rot}
\DeclareMathOperator{\trs}{trace}
\DeclareMathOperator{\Ind}{Ind}
\DeclareMathOperator{\mat}{Mat}
\DeclareMathOperator{\id}{Id}
\DeclareMathOperator{\vect}{vect}
\DeclareMathOperator{\img}{img}
\DeclareMathOperator{\cov}{Cov}
\DeclareMathOperator{\dist}{dist}
\DeclareMathOperator{\irr}{Irr}
\DeclareMathOperator{\image}{Im}
\DeclareMathOperator{\pd}{\partial}
\DeclareMathOperator{\epi}{epi}
\DeclareMathOperator{\Argmin}{Argmin}
\DeclareMathOperator{\dom}{dom}
\DeclareMathOperator{\proj}{proj}
\DeclareMathOperator{\ctg}{ctg}
\DeclareMathOperator{\supp}{supp}
\DeclareMathOperator{\argmin}{argmin}
\DeclareMathOperator{\mult}{mult}
\DeclareMathOperator{\ch}{ch}
\DeclareMathOperator{\sh}{sh}
\DeclareMathOperator{\rang}{rang}
\DeclareMathOperator{\diam}{diam}
\DeclareMathOperator{\Epigraphe}{Epigraphe}




\usepackage{xcolor}
\everymath{\color{blue}}
%\everymath{\color[rgb]{0,1,1}}
%\pagecolor[rgb]{0,0,0.5}


\newcommand*{\pdtest}[3][]{\ensuremath{\frac{\partial^{#1} #2}{\partial #3}}}

\newcommand*{\deffunc}[6][]{\ensuremath{
\begin{array}{rcl}
#2 : #3 &\rightarrow& #4\\
#5 &\mapsto& #6
\end{array}
}}

\newcommand{\eqcolon}{\mathrel{\resizebox{\widthof{$\mathord{=}$}}{\height}{ $\!\!=\!\!\resizebox{1.2\width}{0.8\height}{\raisebox{0.23ex}{$\mathop{:}$}}\!\!$ }}}
\newcommand{\coloneq}{\mathrel{\resizebox{\widthof{$\mathord{=}$}}{\height}{ $\!\!\resizebox{1.2\width}{0.8\height}{\raisebox{0.23ex}{$\mathop{:}$}}\!\!=\!\!$ }}}
\newcommand{\eqcolonl}{\ensuremath{\mathrel{=\!\!\mathop{:}}}}
\newcommand{\coloneql}{\ensuremath{\mathrel{\mathop{:} \!\! =}}}
\newcommand{\vc}[1]{% inline column vector
  \left(\begin{smallmatrix}#1\end{smallmatrix}\right)%
}
\newcommand{\vr}[1]{% inline row vector
  \begin{smallmatrix}(\,#1\,)\end{smallmatrix}%
}
\makeatletter
\newcommand*{\defeq}{\ =\mathrel{\rlap{%
                     \raisebox{0.3ex}{$\m@th\cdot$}}%
                     \raisebox{-0.3ex}{$\m@th\cdot$}}%
                     }
\makeatother

\newcommand{\mathcircle}[1]{% inline row vector
 \overset{\circ}{#1}
}
\newcommand{\ulim}{% low limit
 \underline{\lim}
}
\newcommand{\ssi}{% iff
\iff
}
\newcommand{\ps}[2]{
\expval{#1 | #2}
}
\newcommand{\df}[1]{
\mqty{#1}
}
\newcommand{\n}[1]{
\norm{#1}
}
\newcommand{\sys}[1]{
\left\{\smqty{#1}\right.
}


\newcommand{\eqdef}{\ensuremath{\overset{\text{def}}=}}


\def\Circlearrowright{\ensuremath{%
  \rotatebox[origin=c]{230}{$\circlearrowright$}}}

\newcommand\ct[1]{\text{\rmfamily\upshape #1}}
\newcommand\question[1]{ {\color{red} ...!? \small #1}}
\newcommand\caz[1]{\left\{\begin{array} #1 \end{array}\right.}
\newcommand\const{\text{\rmfamily\upshape const}}
\newcommand\toP{ \overset{\pro}{\to}}
\newcommand\toPP{ \overset{\text{PP}}{\to}}
\newcommand{\oeq}{\mathrel{\text{\textcircled{$=$}}}}





\usepackage{xcolor}
% \usepackage[normalem]{ulem}
\usepackage{lipsum}
\makeatletter
% \newcommand\colorwave[1][blue]{\bgroup \markoverwith{\lower3.5\p@\hbox{\sixly \textcolor{#1}{\char58}}}\ULon}
%\font\sixly=lasy6 % does not re-load if already loaded, so no memory problem.

\newmdtheoremenv[
linewidth= 1pt,linecolor= blue,%
leftmargin=20,rightmargin=20,innertopmargin=0pt, innerrightmargin=40,%
tikzsetting = { draw=lightgray, line width = 0.3pt,dashed,%
dash pattern = on 15pt off 3pt},%
splittopskip=\topskip,skipbelow=\baselineskip,%
skipabove=\baselineskip,ntheorem,roundcorner=0pt,
% backgroundcolor=pagebg,font=\color{orange}\sffamily, fontcolor=white
]{examplebox}{Exemple}[section]



\newcommand\R{\mathbb{R}}
\newcommand\Z{\mathbb{Z}}
\newcommand\N{\mathbb{N}}
\newcommand\E{\mathbb{E}}
\newcommand\F{\mathcal{F}}
\newcommand\cH{\mathcal{H}}
\newcommand\V{\mathbb{V}}
\newcommand\dmo{ ^{-1} }
\newcommand\kapa{\kappa}
\newcommand\im{Im}
\newcommand\hs{\mathcal{H}}





\usepackage{soul}

\makeatletter
\newcommand*{\whiten}[1]{\llap{\textcolor{white}{{\the\SOUL@token}}\hspace{#1pt}}}
\DeclareRobustCommand*\myul{%
    \def\SOUL@everyspace{\underline{\space}\kern\z@}%
    \def\SOUL@everytoken{%
     \setbox0=\hbox{\the\SOUL@token}%
     \ifdim\dp0>\z@
        \raisebox{\dp0}{\underline{\phantom{\the\SOUL@token}}}%
        \whiten{1}\whiten{0}%
        \whiten{-1}\whiten{-2}%
        \llap{\the\SOUL@token}%
     \else
        \underline{\the\SOUL@token}%
     \fi}%
\SOUL@}
\makeatother

\newcommand*{\demp}{\fontfamily{lmtt}\selectfont}

\DeclareTextFontCommand{\textdemp}{\demp}

\begin{document}

\ifcomment
Multiline
comment
\fi
\ifcomment
\myul{Typesetting test}
% \color[rgb]{1,1,1}
$∑_i^n≠ 60º±∞π∆¬≈√j∫h≤≥µ$

$\CR \R\pro\ind\pro\gS\pro
\mqty[a&b\\c&d]$
$\pro\mathbb{P}$
$\dd{x}$

  \[
    \alpha(x)=\left\{
                \begin{array}{ll}
                  x\\
                  \frac{1}{1+e^{-kx}}\\
                  \frac{e^x-e^{-x}}{e^x+e^{-x}}
                \end{array}
              \right.
  \]

  $\expval{x}$
  
  $\chi_\rho(ghg\dmo)=\Tr(\rho_{ghg\dmo})=\Tr(\rho_g\circ\rho_h\circ\rho\dmo_g)=\Tr(\rho_h)\overset{\mbox{\scalebox{0.5}{$\Tr(AB)=\Tr(BA)$}}}{=}\chi_\rho(h)$
  	$\mathop{\oplus}_{\substack{x\in X}}$

$\mat(\rho_g)=(a_{ij}(g))_{\scriptsize \substack{1\leq i\leq d \\ 1\leq j\leq d}}$ et $\mat(\rho'_g)=(a'_{ij}(g))_{\scriptsize \substack{1\leq i'\leq d' \\ 1\leq j'\leq d'}}$



\[\int_a^b{\mathbb{R}^2}g(u, v)\dd{P_{XY}}(u, v)=\iint g(u,v) f_{XY}(u, v)\dd \lambda(u) \dd \lambda(v)\]
$$\lim_{x\to\infty} f(x)$$	
$$\iiiint_V \mu(t,u,v,w) \,dt\,du\,dv\,dw$$
$$\sum_{n=1}^{\infty} 2^{-n} = 1$$	
\begin{definition}
	Si $X$ et $Y$ sont 2 v.a. ou definit la \textsc{Covariance} entre $X$ et $Y$ comme
	$\cov(X,Y)\overset{\text{def}}{=}\E\left[(X-\E(X))(Y-\E(Y))\right]=\E(XY)-\E(X)\E(Y)$.
\end{definition}
\fi
\pagebreak

% \tableofcontents

% insert your code here
%\input{./algebra/main.tex}
%\input{./geometrie-differentielle/main.tex}
%\input{./probabilite/main.tex}
%\input{./analyse-fonctionnelle/main.tex}
% \input{./Analyse-convexe-et-dualite-en-optimisation/main.tex}
%\input{./tikz/main.tex}
%\input{./Theorie-du-distributions/main.tex}
%\input{./optimisation/mine.tex}
 \input{./modelisation/main.tex}

% yves.aubry@univ-tln.fr : algebra

\end{document}

%% !TEX encoding = UTF-8 Unicode
% !TEX TS-program = xelatex

\documentclass[french]{report}

%\usepackage[utf8]{inputenc}
%\usepackage[T1]{fontenc}
\usepackage{babel}


\newif\ifcomment
%\commenttrue # Show comments

\usepackage{physics}
\usepackage{amssymb}


\usepackage{amsthm}
% \usepackage{thmtools}
\usepackage{mathtools}
\usepackage{amsfonts}

\usepackage{color}

\usepackage{tikz}

\usepackage{geometry}
\geometry{a5paper, margin=0.1in, right=1cm}

\usepackage{dsfont}

\usepackage{graphicx}
\graphicspath{ {images/} }

\usepackage{faktor}

\usepackage{IEEEtrantools}
\usepackage{enumerate}   
\usepackage[PostScript=dvips]{"/Users/aware/Documents/Courses/diagrams"}


\newtheorem{theorem}{Théorème}[section]
\renewcommand{\thetheorem}{\arabic{theorem}}
\newtheorem{lemme}{Lemme}[section]
\renewcommand{\thelemme}{\arabic{lemme}}
\newtheorem{proposition}{Proposition}[section]
\renewcommand{\theproposition}{\arabic{proposition}}
\newtheorem{notations}{Notations}[section]
\newtheorem{problem}{Problème}[section]
\newtheorem{corollary}{Corollaire}[theorem]
\renewcommand{\thecorollary}{\arabic{corollary}}
\newtheorem{property}{Propriété}[section]
\newtheorem{objective}{Objectif}[section]

\theoremstyle{definition}
\newtheorem{definition}{Définition}[section]
\renewcommand{\thedefinition}{\arabic{definition}}
\newtheorem{exercise}{Exercice}[chapter]
\renewcommand{\theexercise}{\arabic{exercise}}
\newtheorem{example}{Exemple}[chapter]
\renewcommand{\theexample}{\arabic{example}}
\newtheorem*{solution}{Solution}
\newtheorem*{application}{Application}
\newtheorem*{notation}{Notation}
\newtheorem*{vocabulary}{Vocabulaire}
\newtheorem*{properties}{Propriétés}



\theoremstyle{remark}
\newtheorem*{remark}{Remarque}
\newtheorem*{rappel}{Rappel}


\usepackage{etoolbox}
\AtBeginEnvironment{exercise}{\small}
\AtBeginEnvironment{example}{\small}

\usepackage{cases}
\usepackage[red]{mypack}

\usepackage[framemethod=TikZ]{mdframed}

\definecolor{bg}{rgb}{0.4,0.25,0.95}
\definecolor{pagebg}{rgb}{0,0,0.5}
\surroundwithmdframed[
   topline=false,
   rightline=false,
   bottomline=false,
   leftmargin=\parindent,
   skipabove=8pt,
   skipbelow=8pt,
   linecolor=blue,
   innerbottommargin=10pt,
   % backgroundcolor=bg,font=\color{orange}\sffamily, fontcolor=white
]{definition}

\usepackage{empheq}
\usepackage[most]{tcolorbox}

\newtcbox{\mymath}[1][]{%
    nobeforeafter, math upper, tcbox raise base,
    enhanced, colframe=blue!30!black,
    colback=red!10, boxrule=1pt,
    #1}

\usepackage{unixode}


\DeclareMathOperator{\ord}{ord}
\DeclareMathOperator{\orb}{orb}
\DeclareMathOperator{\stab}{stab}
\DeclareMathOperator{\Stab}{stab}
\DeclareMathOperator{\ppcm}{ppcm}
\DeclareMathOperator{\conj}{Conj}
\DeclareMathOperator{\End}{End}
\DeclareMathOperator{\rot}{rot}
\DeclareMathOperator{\trs}{trace}
\DeclareMathOperator{\Ind}{Ind}
\DeclareMathOperator{\mat}{Mat}
\DeclareMathOperator{\id}{Id}
\DeclareMathOperator{\vect}{vect}
\DeclareMathOperator{\img}{img}
\DeclareMathOperator{\cov}{Cov}
\DeclareMathOperator{\dist}{dist}
\DeclareMathOperator{\irr}{Irr}
\DeclareMathOperator{\image}{Im}
\DeclareMathOperator{\pd}{\partial}
\DeclareMathOperator{\epi}{epi}
\DeclareMathOperator{\Argmin}{Argmin}
\DeclareMathOperator{\dom}{dom}
\DeclareMathOperator{\proj}{proj}
\DeclareMathOperator{\ctg}{ctg}
\DeclareMathOperator{\supp}{supp}
\DeclareMathOperator{\argmin}{argmin}
\DeclareMathOperator{\mult}{mult}
\DeclareMathOperator{\ch}{ch}
\DeclareMathOperator{\sh}{sh}
\DeclareMathOperator{\rang}{rang}
\DeclareMathOperator{\diam}{diam}
\DeclareMathOperator{\Epigraphe}{Epigraphe}




\usepackage{xcolor}
\everymath{\color{blue}}
%\everymath{\color[rgb]{0,1,1}}
%\pagecolor[rgb]{0,0,0.5}


\newcommand*{\pdtest}[3][]{\ensuremath{\frac{\partial^{#1} #2}{\partial #3}}}

\newcommand*{\deffunc}[6][]{\ensuremath{
\begin{array}{rcl}
#2 : #3 &\rightarrow& #4\\
#5 &\mapsto& #6
\end{array}
}}

\newcommand{\eqcolon}{\mathrel{\resizebox{\widthof{$\mathord{=}$}}{\height}{ $\!\!=\!\!\resizebox{1.2\width}{0.8\height}{\raisebox{0.23ex}{$\mathop{:}$}}\!\!$ }}}
\newcommand{\coloneq}{\mathrel{\resizebox{\widthof{$\mathord{=}$}}{\height}{ $\!\!\resizebox{1.2\width}{0.8\height}{\raisebox{0.23ex}{$\mathop{:}$}}\!\!=\!\!$ }}}
\newcommand{\eqcolonl}{\ensuremath{\mathrel{=\!\!\mathop{:}}}}
\newcommand{\coloneql}{\ensuremath{\mathrel{\mathop{:} \!\! =}}}
\newcommand{\vc}[1]{% inline column vector
  \left(\begin{smallmatrix}#1\end{smallmatrix}\right)%
}
\newcommand{\vr}[1]{% inline row vector
  \begin{smallmatrix}(\,#1\,)\end{smallmatrix}%
}
\makeatletter
\newcommand*{\defeq}{\ =\mathrel{\rlap{%
                     \raisebox{0.3ex}{$\m@th\cdot$}}%
                     \raisebox{-0.3ex}{$\m@th\cdot$}}%
                     }
\makeatother

\newcommand{\mathcircle}[1]{% inline row vector
 \overset{\circ}{#1}
}
\newcommand{\ulim}{% low limit
 \underline{\lim}
}
\newcommand{\ssi}{% iff
\iff
}
\newcommand{\ps}[2]{
\expval{#1 | #2}
}
\newcommand{\df}[1]{
\mqty{#1}
}
\newcommand{\n}[1]{
\norm{#1}
}
\newcommand{\sys}[1]{
\left\{\smqty{#1}\right.
}


\newcommand{\eqdef}{\ensuremath{\overset{\text{def}}=}}


\def\Circlearrowright{\ensuremath{%
  \rotatebox[origin=c]{230}{$\circlearrowright$}}}

\newcommand\ct[1]{\text{\rmfamily\upshape #1}}
\newcommand\question[1]{ {\color{red} ...!? \small #1}}
\newcommand\caz[1]{\left\{\begin{array} #1 \end{array}\right.}
\newcommand\const{\text{\rmfamily\upshape const}}
\newcommand\toP{ \overset{\pro}{\to}}
\newcommand\toPP{ \overset{\text{PP}}{\to}}
\newcommand{\oeq}{\mathrel{\text{\textcircled{$=$}}}}





\usepackage{xcolor}
% \usepackage[normalem]{ulem}
\usepackage{lipsum}
\makeatletter
% \newcommand\colorwave[1][blue]{\bgroup \markoverwith{\lower3.5\p@\hbox{\sixly \textcolor{#1}{\char58}}}\ULon}
%\font\sixly=lasy6 % does not re-load if already loaded, so no memory problem.

\newmdtheoremenv[
linewidth= 1pt,linecolor= blue,%
leftmargin=20,rightmargin=20,innertopmargin=0pt, innerrightmargin=40,%
tikzsetting = { draw=lightgray, line width = 0.3pt,dashed,%
dash pattern = on 15pt off 3pt},%
splittopskip=\topskip,skipbelow=\baselineskip,%
skipabove=\baselineskip,ntheorem,roundcorner=0pt,
% backgroundcolor=pagebg,font=\color{orange}\sffamily, fontcolor=white
]{examplebox}{Exemple}[section]



\newcommand\R{\mathbb{R}}
\newcommand\Z{\mathbb{Z}}
\newcommand\N{\mathbb{N}}
\newcommand\E{\mathbb{E}}
\newcommand\F{\mathcal{F}}
\newcommand\cH{\mathcal{H}}
\newcommand\V{\mathbb{V}}
\newcommand\dmo{ ^{-1} }
\newcommand\kapa{\kappa}
\newcommand\im{Im}
\newcommand\hs{\mathcal{H}}





\usepackage{soul}

\makeatletter
\newcommand*{\whiten}[1]{\llap{\textcolor{white}{{\the\SOUL@token}}\hspace{#1pt}}}
\DeclareRobustCommand*\myul{%
    \def\SOUL@everyspace{\underline{\space}\kern\z@}%
    \def\SOUL@everytoken{%
     \setbox0=\hbox{\the\SOUL@token}%
     \ifdim\dp0>\z@
        \raisebox{\dp0}{\underline{\phantom{\the\SOUL@token}}}%
        \whiten{1}\whiten{0}%
        \whiten{-1}\whiten{-2}%
        \llap{\the\SOUL@token}%
     \else
        \underline{\the\SOUL@token}%
     \fi}%
\SOUL@}
\makeatother

\newcommand*{\demp}{\fontfamily{lmtt}\selectfont}

\DeclareTextFontCommand{\textdemp}{\demp}

\begin{document}

\ifcomment
Multiline
comment
\fi
\ifcomment
\myul{Typesetting test}
% \color[rgb]{1,1,1}
$∑_i^n≠ 60º±∞π∆¬≈√j∫h≤≥µ$

$\CR \R\pro\ind\pro\gS\pro
\mqty[a&b\\c&d]$
$\pro\mathbb{P}$
$\dd{x}$

  \[
    \alpha(x)=\left\{
                \begin{array}{ll}
                  x\\
                  \frac{1}{1+e^{-kx}}\\
                  \frac{e^x-e^{-x}}{e^x+e^{-x}}
                \end{array}
              \right.
  \]

  $\expval{x}$
  
  $\chi_\rho(ghg\dmo)=\Tr(\rho_{ghg\dmo})=\Tr(\rho_g\circ\rho_h\circ\rho\dmo_g)=\Tr(\rho_h)\overset{\mbox{\scalebox{0.5}{$\Tr(AB)=\Tr(BA)$}}}{=}\chi_\rho(h)$
  	$\mathop{\oplus}_{\substack{x\in X}}$

$\mat(\rho_g)=(a_{ij}(g))_{\scriptsize \substack{1\leq i\leq d \\ 1\leq j\leq d}}$ et $\mat(\rho'_g)=(a'_{ij}(g))_{\scriptsize \substack{1\leq i'\leq d' \\ 1\leq j'\leq d'}}$



\[\int_a^b{\mathbb{R}^2}g(u, v)\dd{P_{XY}}(u, v)=\iint g(u,v) f_{XY}(u, v)\dd \lambda(u) \dd \lambda(v)\]
$$\lim_{x\to\infty} f(x)$$	
$$\iiiint_V \mu(t,u,v,w) \,dt\,du\,dv\,dw$$
$$\sum_{n=1}^{\infty} 2^{-n} = 1$$	
\begin{definition}
	Si $X$ et $Y$ sont 2 v.a. ou definit la \textsc{Covariance} entre $X$ et $Y$ comme
	$\cov(X,Y)\overset{\text{def}}{=}\E\left[(X-\E(X))(Y-\E(Y))\right]=\E(XY)-\E(X)\E(Y)$.
\end{definition}
\fi
\pagebreak

% \tableofcontents

% insert your code here
%\input{./algebra/main.tex}
%\input{./geometrie-differentielle/main.tex}
%\input{./probabilite/main.tex}
%\input{./analyse-fonctionnelle/main.tex}
% \input{./Analyse-convexe-et-dualite-en-optimisation/main.tex}
%\input{./tikz/main.tex}
%\input{./Theorie-du-distributions/main.tex}
%\input{./optimisation/mine.tex}
 \input{./modelisation/main.tex}

% yves.aubry@univ-tln.fr : algebra

\end{document}

%\input{./optimisation/mine.tex}
 % !TEX encoding = UTF-8 Unicode
% !TEX TS-program = xelatex

\documentclass[french]{report}

%\usepackage[utf8]{inputenc}
%\usepackage[T1]{fontenc}
\usepackage{babel}


\newif\ifcomment
%\commenttrue # Show comments

\usepackage{physics}
\usepackage{amssymb}


\usepackage{amsthm}
% \usepackage{thmtools}
\usepackage{mathtools}
\usepackage{amsfonts}

\usepackage{color}

\usepackage{tikz}

\usepackage{geometry}
\geometry{a5paper, margin=0.1in, right=1cm}

\usepackage{dsfont}

\usepackage{graphicx}
\graphicspath{ {images/} }

\usepackage{faktor}

\usepackage{IEEEtrantools}
\usepackage{enumerate}   
\usepackage[PostScript=dvips]{"/Users/aware/Documents/Courses/diagrams"}


\newtheorem{theorem}{Théorème}[section]
\renewcommand{\thetheorem}{\arabic{theorem}}
\newtheorem{lemme}{Lemme}[section]
\renewcommand{\thelemme}{\arabic{lemme}}
\newtheorem{proposition}{Proposition}[section]
\renewcommand{\theproposition}{\arabic{proposition}}
\newtheorem{notations}{Notations}[section]
\newtheorem{problem}{Problème}[section]
\newtheorem{corollary}{Corollaire}[theorem]
\renewcommand{\thecorollary}{\arabic{corollary}}
\newtheorem{property}{Propriété}[section]
\newtheorem{objective}{Objectif}[section]

\theoremstyle{definition}
\newtheorem{definition}{Définition}[section]
\renewcommand{\thedefinition}{\arabic{definition}}
\newtheorem{exercise}{Exercice}[chapter]
\renewcommand{\theexercise}{\arabic{exercise}}
\newtheorem{example}{Exemple}[chapter]
\renewcommand{\theexample}{\arabic{example}}
\newtheorem*{solution}{Solution}
\newtheorem*{application}{Application}
\newtheorem*{notation}{Notation}
\newtheorem*{vocabulary}{Vocabulaire}
\newtheorem*{properties}{Propriétés}



\theoremstyle{remark}
\newtheorem*{remark}{Remarque}
\newtheorem*{rappel}{Rappel}


\usepackage{etoolbox}
\AtBeginEnvironment{exercise}{\small}
\AtBeginEnvironment{example}{\small}

\usepackage{cases}
\usepackage[red]{mypack}

\usepackage[framemethod=TikZ]{mdframed}

\definecolor{bg}{rgb}{0.4,0.25,0.95}
\definecolor{pagebg}{rgb}{0,0,0.5}
\surroundwithmdframed[
   topline=false,
   rightline=false,
   bottomline=false,
   leftmargin=\parindent,
   skipabove=8pt,
   skipbelow=8pt,
   linecolor=blue,
   innerbottommargin=10pt,
   % backgroundcolor=bg,font=\color{orange}\sffamily, fontcolor=white
]{definition}

\usepackage{empheq}
\usepackage[most]{tcolorbox}

\newtcbox{\mymath}[1][]{%
    nobeforeafter, math upper, tcbox raise base,
    enhanced, colframe=blue!30!black,
    colback=red!10, boxrule=1pt,
    #1}

\usepackage{unixode}


\DeclareMathOperator{\ord}{ord}
\DeclareMathOperator{\orb}{orb}
\DeclareMathOperator{\stab}{stab}
\DeclareMathOperator{\Stab}{stab}
\DeclareMathOperator{\ppcm}{ppcm}
\DeclareMathOperator{\conj}{Conj}
\DeclareMathOperator{\End}{End}
\DeclareMathOperator{\rot}{rot}
\DeclareMathOperator{\trs}{trace}
\DeclareMathOperator{\Ind}{Ind}
\DeclareMathOperator{\mat}{Mat}
\DeclareMathOperator{\id}{Id}
\DeclareMathOperator{\vect}{vect}
\DeclareMathOperator{\img}{img}
\DeclareMathOperator{\cov}{Cov}
\DeclareMathOperator{\dist}{dist}
\DeclareMathOperator{\irr}{Irr}
\DeclareMathOperator{\image}{Im}
\DeclareMathOperator{\pd}{\partial}
\DeclareMathOperator{\epi}{epi}
\DeclareMathOperator{\Argmin}{Argmin}
\DeclareMathOperator{\dom}{dom}
\DeclareMathOperator{\proj}{proj}
\DeclareMathOperator{\ctg}{ctg}
\DeclareMathOperator{\supp}{supp}
\DeclareMathOperator{\argmin}{argmin}
\DeclareMathOperator{\mult}{mult}
\DeclareMathOperator{\ch}{ch}
\DeclareMathOperator{\sh}{sh}
\DeclareMathOperator{\rang}{rang}
\DeclareMathOperator{\diam}{diam}
\DeclareMathOperator{\Epigraphe}{Epigraphe}




\usepackage{xcolor}
\everymath{\color{blue}}
%\everymath{\color[rgb]{0,1,1}}
%\pagecolor[rgb]{0,0,0.5}


\newcommand*{\pdtest}[3][]{\ensuremath{\frac{\partial^{#1} #2}{\partial #3}}}

\newcommand*{\deffunc}[6][]{\ensuremath{
\begin{array}{rcl}
#2 : #3 &\rightarrow& #4\\
#5 &\mapsto& #6
\end{array}
}}

\newcommand{\eqcolon}{\mathrel{\resizebox{\widthof{$\mathord{=}$}}{\height}{ $\!\!=\!\!\resizebox{1.2\width}{0.8\height}{\raisebox{0.23ex}{$\mathop{:}$}}\!\!$ }}}
\newcommand{\coloneq}{\mathrel{\resizebox{\widthof{$\mathord{=}$}}{\height}{ $\!\!\resizebox{1.2\width}{0.8\height}{\raisebox{0.23ex}{$\mathop{:}$}}\!\!=\!\!$ }}}
\newcommand{\eqcolonl}{\ensuremath{\mathrel{=\!\!\mathop{:}}}}
\newcommand{\coloneql}{\ensuremath{\mathrel{\mathop{:} \!\! =}}}
\newcommand{\vc}[1]{% inline column vector
  \left(\begin{smallmatrix}#1\end{smallmatrix}\right)%
}
\newcommand{\vr}[1]{% inline row vector
  \begin{smallmatrix}(\,#1\,)\end{smallmatrix}%
}
\makeatletter
\newcommand*{\defeq}{\ =\mathrel{\rlap{%
                     \raisebox{0.3ex}{$\m@th\cdot$}}%
                     \raisebox{-0.3ex}{$\m@th\cdot$}}%
                     }
\makeatother

\newcommand{\mathcircle}[1]{% inline row vector
 \overset{\circ}{#1}
}
\newcommand{\ulim}{% low limit
 \underline{\lim}
}
\newcommand{\ssi}{% iff
\iff
}
\newcommand{\ps}[2]{
\expval{#1 | #2}
}
\newcommand{\df}[1]{
\mqty{#1}
}
\newcommand{\n}[1]{
\norm{#1}
}
\newcommand{\sys}[1]{
\left\{\smqty{#1}\right.
}


\newcommand{\eqdef}{\ensuremath{\overset{\text{def}}=}}


\def\Circlearrowright{\ensuremath{%
  \rotatebox[origin=c]{230}{$\circlearrowright$}}}

\newcommand\ct[1]{\text{\rmfamily\upshape #1}}
\newcommand\question[1]{ {\color{red} ...!? \small #1}}
\newcommand\caz[1]{\left\{\begin{array} #1 \end{array}\right.}
\newcommand\const{\text{\rmfamily\upshape const}}
\newcommand\toP{ \overset{\pro}{\to}}
\newcommand\toPP{ \overset{\text{PP}}{\to}}
\newcommand{\oeq}{\mathrel{\text{\textcircled{$=$}}}}





\usepackage{xcolor}
% \usepackage[normalem]{ulem}
\usepackage{lipsum}
\makeatletter
% \newcommand\colorwave[1][blue]{\bgroup \markoverwith{\lower3.5\p@\hbox{\sixly \textcolor{#1}{\char58}}}\ULon}
%\font\sixly=lasy6 % does not re-load if already loaded, so no memory problem.

\newmdtheoremenv[
linewidth= 1pt,linecolor= blue,%
leftmargin=20,rightmargin=20,innertopmargin=0pt, innerrightmargin=40,%
tikzsetting = { draw=lightgray, line width = 0.3pt,dashed,%
dash pattern = on 15pt off 3pt},%
splittopskip=\topskip,skipbelow=\baselineskip,%
skipabove=\baselineskip,ntheorem,roundcorner=0pt,
% backgroundcolor=pagebg,font=\color{orange}\sffamily, fontcolor=white
]{examplebox}{Exemple}[section]



\newcommand\R{\mathbb{R}}
\newcommand\Z{\mathbb{Z}}
\newcommand\N{\mathbb{N}}
\newcommand\E{\mathbb{E}}
\newcommand\F{\mathcal{F}}
\newcommand\cH{\mathcal{H}}
\newcommand\V{\mathbb{V}}
\newcommand\dmo{ ^{-1} }
\newcommand\kapa{\kappa}
\newcommand\im{Im}
\newcommand\hs{\mathcal{H}}





\usepackage{soul}

\makeatletter
\newcommand*{\whiten}[1]{\llap{\textcolor{white}{{\the\SOUL@token}}\hspace{#1pt}}}
\DeclareRobustCommand*\myul{%
    \def\SOUL@everyspace{\underline{\space}\kern\z@}%
    \def\SOUL@everytoken{%
     \setbox0=\hbox{\the\SOUL@token}%
     \ifdim\dp0>\z@
        \raisebox{\dp0}{\underline{\phantom{\the\SOUL@token}}}%
        \whiten{1}\whiten{0}%
        \whiten{-1}\whiten{-2}%
        \llap{\the\SOUL@token}%
     \else
        \underline{\the\SOUL@token}%
     \fi}%
\SOUL@}
\makeatother

\newcommand*{\demp}{\fontfamily{lmtt}\selectfont}

\DeclareTextFontCommand{\textdemp}{\demp}

\begin{document}

\ifcomment
Multiline
comment
\fi
\ifcomment
\myul{Typesetting test}
% \color[rgb]{1,1,1}
$∑_i^n≠ 60º±∞π∆¬≈√j∫h≤≥µ$

$\CR \R\pro\ind\pro\gS\pro
\mqty[a&b\\c&d]$
$\pro\mathbb{P}$
$\dd{x}$

  \[
    \alpha(x)=\left\{
                \begin{array}{ll}
                  x\\
                  \frac{1}{1+e^{-kx}}\\
                  \frac{e^x-e^{-x}}{e^x+e^{-x}}
                \end{array}
              \right.
  \]

  $\expval{x}$
  
  $\chi_\rho(ghg\dmo)=\Tr(\rho_{ghg\dmo})=\Tr(\rho_g\circ\rho_h\circ\rho\dmo_g)=\Tr(\rho_h)\overset{\mbox{\scalebox{0.5}{$\Tr(AB)=\Tr(BA)$}}}{=}\chi_\rho(h)$
  	$\mathop{\oplus}_{\substack{x\in X}}$

$\mat(\rho_g)=(a_{ij}(g))_{\scriptsize \substack{1\leq i\leq d \\ 1\leq j\leq d}}$ et $\mat(\rho'_g)=(a'_{ij}(g))_{\scriptsize \substack{1\leq i'\leq d' \\ 1\leq j'\leq d'}}$



\[\int_a^b{\mathbb{R}^2}g(u, v)\dd{P_{XY}}(u, v)=\iint g(u,v) f_{XY}(u, v)\dd \lambda(u) \dd \lambda(v)\]
$$\lim_{x\to\infty} f(x)$$	
$$\iiiint_V \mu(t,u,v,w) \,dt\,du\,dv\,dw$$
$$\sum_{n=1}^{\infty} 2^{-n} = 1$$	
\begin{definition}
	Si $X$ et $Y$ sont 2 v.a. ou definit la \textsc{Covariance} entre $X$ et $Y$ comme
	$\cov(X,Y)\overset{\text{def}}{=}\E\left[(X-\E(X))(Y-\E(Y))\right]=\E(XY)-\E(X)\E(Y)$.
\end{definition}
\fi
\pagebreak

% \tableofcontents

% insert your code here
%\input{./algebra/main.tex}
%\input{./geometrie-differentielle/main.tex}
%\input{./probabilite/main.tex}
%\input{./analyse-fonctionnelle/main.tex}
% \input{./Analyse-convexe-et-dualite-en-optimisation/main.tex}
%\input{./tikz/main.tex}
%\input{./Theorie-du-distributions/main.tex}
%\input{./optimisation/mine.tex}
 \input{./modelisation/main.tex}

% yves.aubry@univ-tln.fr : algebra

\end{document}


% yves.aubry@univ-tln.fr : algebra

\end{document}

%% !TEX encoding = UTF-8 Unicode
% !TEX TS-program = xelatex

\documentclass[french]{report}

%\usepackage[utf8]{inputenc}
%\usepackage[T1]{fontenc}
\usepackage{babel}


\newif\ifcomment
%\commenttrue # Show comments

\usepackage{physics}
\usepackage{amssymb}


\usepackage{amsthm}
% \usepackage{thmtools}
\usepackage{mathtools}
\usepackage{amsfonts}

\usepackage{color}

\usepackage{tikz}

\usepackage{geometry}
\geometry{a5paper, margin=0.1in, right=1cm}

\usepackage{dsfont}

\usepackage{graphicx}
\graphicspath{ {images/} }

\usepackage{faktor}

\usepackage{IEEEtrantools}
\usepackage{enumerate}   
\usepackage[PostScript=dvips]{"/Users/aware/Documents/Courses/diagrams"}


\newtheorem{theorem}{Théorème}[section]
\renewcommand{\thetheorem}{\arabic{theorem}}
\newtheorem{lemme}{Lemme}[section]
\renewcommand{\thelemme}{\arabic{lemme}}
\newtheorem{proposition}{Proposition}[section]
\renewcommand{\theproposition}{\arabic{proposition}}
\newtheorem{notations}{Notations}[section]
\newtheorem{problem}{Problème}[section]
\newtheorem{corollary}{Corollaire}[theorem]
\renewcommand{\thecorollary}{\arabic{corollary}}
\newtheorem{property}{Propriété}[section]
\newtheorem{objective}{Objectif}[section]

\theoremstyle{definition}
\newtheorem{definition}{Définition}[section]
\renewcommand{\thedefinition}{\arabic{definition}}
\newtheorem{exercise}{Exercice}[chapter]
\renewcommand{\theexercise}{\arabic{exercise}}
\newtheorem{example}{Exemple}[chapter]
\renewcommand{\theexample}{\arabic{example}}
\newtheorem*{solution}{Solution}
\newtheorem*{application}{Application}
\newtheorem*{notation}{Notation}
\newtheorem*{vocabulary}{Vocabulaire}
\newtheorem*{properties}{Propriétés}



\theoremstyle{remark}
\newtheorem*{remark}{Remarque}
\newtheorem*{rappel}{Rappel}


\usepackage{etoolbox}
\AtBeginEnvironment{exercise}{\small}
\AtBeginEnvironment{example}{\small}

\usepackage{cases}
\usepackage[red]{mypack}

\usepackage[framemethod=TikZ]{mdframed}

\definecolor{bg}{rgb}{0.4,0.25,0.95}
\definecolor{pagebg}{rgb}{0,0,0.5}
\surroundwithmdframed[
   topline=false,
   rightline=false,
   bottomline=false,
   leftmargin=\parindent,
   skipabove=8pt,
   skipbelow=8pt,
   linecolor=blue,
   innerbottommargin=10pt,
   % backgroundcolor=bg,font=\color{orange}\sffamily, fontcolor=white
]{definition}

\usepackage{empheq}
\usepackage[most]{tcolorbox}

\newtcbox{\mymath}[1][]{%
    nobeforeafter, math upper, tcbox raise base,
    enhanced, colframe=blue!30!black,
    colback=red!10, boxrule=1pt,
    #1}

\usepackage{unixode}


\DeclareMathOperator{\ord}{ord}
\DeclareMathOperator{\orb}{orb}
\DeclareMathOperator{\stab}{stab}
\DeclareMathOperator{\Stab}{stab}
\DeclareMathOperator{\ppcm}{ppcm}
\DeclareMathOperator{\conj}{Conj}
\DeclareMathOperator{\End}{End}
\DeclareMathOperator{\rot}{rot}
\DeclareMathOperator{\trs}{trace}
\DeclareMathOperator{\Ind}{Ind}
\DeclareMathOperator{\mat}{Mat}
\DeclareMathOperator{\id}{Id}
\DeclareMathOperator{\vect}{vect}
\DeclareMathOperator{\img}{img}
\DeclareMathOperator{\cov}{Cov}
\DeclareMathOperator{\dist}{dist}
\DeclareMathOperator{\irr}{Irr}
\DeclareMathOperator{\image}{Im}
\DeclareMathOperator{\pd}{\partial}
\DeclareMathOperator{\epi}{epi}
\DeclareMathOperator{\Argmin}{Argmin}
\DeclareMathOperator{\dom}{dom}
\DeclareMathOperator{\proj}{proj}
\DeclareMathOperator{\ctg}{ctg}
\DeclareMathOperator{\supp}{supp}
\DeclareMathOperator{\argmin}{argmin}
\DeclareMathOperator{\mult}{mult}
\DeclareMathOperator{\ch}{ch}
\DeclareMathOperator{\sh}{sh}
\DeclareMathOperator{\rang}{rang}
\DeclareMathOperator{\diam}{diam}
\DeclareMathOperator{\Epigraphe}{Epigraphe}




\usepackage{xcolor}
\everymath{\color{blue}}
%\everymath{\color[rgb]{0,1,1}}
%\pagecolor[rgb]{0,0,0.5}


\newcommand*{\pdtest}[3][]{\ensuremath{\frac{\partial^{#1} #2}{\partial #3}}}

\newcommand*{\deffunc}[6][]{\ensuremath{
\begin{array}{rcl}
#2 : #3 &\rightarrow& #4\\
#5 &\mapsto& #6
\end{array}
}}

\newcommand{\eqcolon}{\mathrel{\resizebox{\widthof{$\mathord{=}$}}{\height}{ $\!\!=\!\!\resizebox{1.2\width}{0.8\height}{\raisebox{0.23ex}{$\mathop{:}$}}\!\!$ }}}
\newcommand{\coloneq}{\mathrel{\resizebox{\widthof{$\mathord{=}$}}{\height}{ $\!\!\resizebox{1.2\width}{0.8\height}{\raisebox{0.23ex}{$\mathop{:}$}}\!\!=\!\!$ }}}
\newcommand{\eqcolonl}{\ensuremath{\mathrel{=\!\!\mathop{:}}}}
\newcommand{\coloneql}{\ensuremath{\mathrel{\mathop{:} \!\! =}}}
\newcommand{\vc}[1]{% inline column vector
  \left(\begin{smallmatrix}#1\end{smallmatrix}\right)%
}
\newcommand{\vr}[1]{% inline row vector
  \begin{smallmatrix}(\,#1\,)\end{smallmatrix}%
}
\makeatletter
\newcommand*{\defeq}{\ =\mathrel{\rlap{%
                     \raisebox{0.3ex}{$\m@th\cdot$}}%
                     \raisebox{-0.3ex}{$\m@th\cdot$}}%
                     }
\makeatother

\newcommand{\mathcircle}[1]{% inline row vector
 \overset{\circ}{#1}
}
\newcommand{\ulim}{% low limit
 \underline{\lim}
}
\newcommand{\ssi}{% iff
\iff
}
\newcommand{\ps}[2]{
\expval{#1 | #2}
}
\newcommand{\df}[1]{
\mqty{#1}
}
\newcommand{\n}[1]{
\norm{#1}
}
\newcommand{\sys}[1]{
\left\{\smqty{#1}\right.
}


\newcommand{\eqdef}{\ensuremath{\overset{\text{def}}=}}


\def\Circlearrowright{\ensuremath{%
  \rotatebox[origin=c]{230}{$\circlearrowright$}}}

\newcommand\ct[1]{\text{\rmfamily\upshape #1}}
\newcommand\question[1]{ {\color{red} ...!? \small #1}}
\newcommand\caz[1]{\left\{\begin{array} #1 \end{array}\right.}
\newcommand\const{\text{\rmfamily\upshape const}}
\newcommand\toP{ \overset{\pro}{\to}}
\newcommand\toPP{ \overset{\text{PP}}{\to}}
\newcommand{\oeq}{\mathrel{\text{\textcircled{$=$}}}}





\usepackage{xcolor}
% \usepackage[normalem]{ulem}
\usepackage{lipsum}
\makeatletter
% \newcommand\colorwave[1][blue]{\bgroup \markoverwith{\lower3.5\p@\hbox{\sixly \textcolor{#1}{\char58}}}\ULon}
%\font\sixly=lasy6 % does not re-load if already loaded, so no memory problem.

\newmdtheoremenv[
linewidth= 1pt,linecolor= blue,%
leftmargin=20,rightmargin=20,innertopmargin=0pt, innerrightmargin=40,%
tikzsetting = { draw=lightgray, line width = 0.3pt,dashed,%
dash pattern = on 15pt off 3pt},%
splittopskip=\topskip,skipbelow=\baselineskip,%
skipabove=\baselineskip,ntheorem,roundcorner=0pt,
% backgroundcolor=pagebg,font=\color{orange}\sffamily, fontcolor=white
]{examplebox}{Exemple}[section]



\newcommand\R{\mathbb{R}}
\newcommand\Z{\mathbb{Z}}
\newcommand\N{\mathbb{N}}
\newcommand\E{\mathbb{E}}
\newcommand\F{\mathcal{F}}
\newcommand\cH{\mathcal{H}}
\newcommand\V{\mathbb{V}}
\newcommand\dmo{ ^{-1} }
\newcommand\kapa{\kappa}
\newcommand\im{Im}
\newcommand\hs{\mathcal{H}}





\usepackage{soul}

\makeatletter
\newcommand*{\whiten}[1]{\llap{\textcolor{white}{{\the\SOUL@token}}\hspace{#1pt}}}
\DeclareRobustCommand*\myul{%
    \def\SOUL@everyspace{\underline{\space}\kern\z@}%
    \def\SOUL@everytoken{%
     \setbox0=\hbox{\the\SOUL@token}%
     \ifdim\dp0>\z@
        \raisebox{\dp0}{\underline{\phantom{\the\SOUL@token}}}%
        \whiten{1}\whiten{0}%
        \whiten{-1}\whiten{-2}%
        \llap{\the\SOUL@token}%
     \else
        \underline{\the\SOUL@token}%
     \fi}%
\SOUL@}
\makeatother

\newcommand*{\demp}{\fontfamily{lmtt}\selectfont}

\DeclareTextFontCommand{\textdemp}{\demp}

\begin{document}

\ifcomment
Multiline
comment
\fi
\ifcomment
\myul{Typesetting test}
% \color[rgb]{1,1,1}
$∑_i^n≠ 60º±∞π∆¬≈√j∫h≤≥µ$

$\CR \R\pro\ind\pro\gS\pro
\mqty[a&b\\c&d]$
$\pro\mathbb{P}$
$\dd{x}$

  \[
    \alpha(x)=\left\{
                \begin{array}{ll}
                  x\\
                  \frac{1}{1+e^{-kx}}\\
                  \frac{e^x-e^{-x}}{e^x+e^{-x}}
                \end{array}
              \right.
  \]

  $\expval{x}$
  
  $\chi_\rho(ghg\dmo)=\Tr(\rho_{ghg\dmo})=\Tr(\rho_g\circ\rho_h\circ\rho\dmo_g)=\Tr(\rho_h)\overset{\mbox{\scalebox{0.5}{$\Tr(AB)=\Tr(BA)$}}}{=}\chi_\rho(h)$
  	$\mathop{\oplus}_{\substack{x\in X}}$

$\mat(\rho_g)=(a_{ij}(g))_{\scriptsize \substack{1\leq i\leq d \\ 1\leq j\leq d}}$ et $\mat(\rho'_g)=(a'_{ij}(g))_{\scriptsize \substack{1\leq i'\leq d' \\ 1\leq j'\leq d'}}$



\[\int_a^b{\mathbb{R}^2}g(u, v)\dd{P_{XY}}(u, v)=\iint g(u,v) f_{XY}(u, v)\dd \lambda(u) \dd \lambda(v)\]
$$\lim_{x\to\infty} f(x)$$	
$$\iiiint_V \mu(t,u,v,w) \,dt\,du\,dv\,dw$$
$$\sum_{n=1}^{\infty} 2^{-n} = 1$$	
\begin{definition}
	Si $X$ et $Y$ sont 2 v.a. ou definit la \textsc{Covariance} entre $X$ et $Y$ comme
	$\cov(X,Y)\overset{\text{def}}{=}\E\left[(X-\E(X))(Y-\E(Y))\right]=\E(XY)-\E(X)\E(Y)$.
\end{definition}
\fi
\pagebreak

% \tableofcontents

% insert your code here
%% !TEX encoding = UTF-8 Unicode
% !TEX TS-program = xelatex

\documentclass[french]{report}

%\usepackage[utf8]{inputenc}
%\usepackage[T1]{fontenc}
\usepackage{babel}


\newif\ifcomment
%\commenttrue # Show comments

\usepackage{physics}
\usepackage{amssymb}


\usepackage{amsthm}
% \usepackage{thmtools}
\usepackage{mathtools}
\usepackage{amsfonts}

\usepackage{color}

\usepackage{tikz}

\usepackage{geometry}
\geometry{a5paper, margin=0.1in, right=1cm}

\usepackage{dsfont}

\usepackage{graphicx}
\graphicspath{ {images/} }

\usepackage{faktor}

\usepackage{IEEEtrantools}
\usepackage{enumerate}   
\usepackage[PostScript=dvips]{"/Users/aware/Documents/Courses/diagrams"}


\newtheorem{theorem}{Théorème}[section]
\renewcommand{\thetheorem}{\arabic{theorem}}
\newtheorem{lemme}{Lemme}[section]
\renewcommand{\thelemme}{\arabic{lemme}}
\newtheorem{proposition}{Proposition}[section]
\renewcommand{\theproposition}{\arabic{proposition}}
\newtheorem{notations}{Notations}[section]
\newtheorem{problem}{Problème}[section]
\newtheorem{corollary}{Corollaire}[theorem]
\renewcommand{\thecorollary}{\arabic{corollary}}
\newtheorem{property}{Propriété}[section]
\newtheorem{objective}{Objectif}[section]

\theoremstyle{definition}
\newtheorem{definition}{Définition}[section]
\renewcommand{\thedefinition}{\arabic{definition}}
\newtheorem{exercise}{Exercice}[chapter]
\renewcommand{\theexercise}{\arabic{exercise}}
\newtheorem{example}{Exemple}[chapter]
\renewcommand{\theexample}{\arabic{example}}
\newtheorem*{solution}{Solution}
\newtheorem*{application}{Application}
\newtheorem*{notation}{Notation}
\newtheorem*{vocabulary}{Vocabulaire}
\newtheorem*{properties}{Propriétés}



\theoremstyle{remark}
\newtheorem*{remark}{Remarque}
\newtheorem*{rappel}{Rappel}


\usepackage{etoolbox}
\AtBeginEnvironment{exercise}{\small}
\AtBeginEnvironment{example}{\small}

\usepackage{cases}
\usepackage[red]{mypack}

\usepackage[framemethod=TikZ]{mdframed}

\definecolor{bg}{rgb}{0.4,0.25,0.95}
\definecolor{pagebg}{rgb}{0,0,0.5}
\surroundwithmdframed[
   topline=false,
   rightline=false,
   bottomline=false,
   leftmargin=\parindent,
   skipabove=8pt,
   skipbelow=8pt,
   linecolor=blue,
   innerbottommargin=10pt,
   % backgroundcolor=bg,font=\color{orange}\sffamily, fontcolor=white
]{definition}

\usepackage{empheq}
\usepackage[most]{tcolorbox}

\newtcbox{\mymath}[1][]{%
    nobeforeafter, math upper, tcbox raise base,
    enhanced, colframe=blue!30!black,
    colback=red!10, boxrule=1pt,
    #1}

\usepackage{unixode}


\DeclareMathOperator{\ord}{ord}
\DeclareMathOperator{\orb}{orb}
\DeclareMathOperator{\stab}{stab}
\DeclareMathOperator{\Stab}{stab}
\DeclareMathOperator{\ppcm}{ppcm}
\DeclareMathOperator{\conj}{Conj}
\DeclareMathOperator{\End}{End}
\DeclareMathOperator{\rot}{rot}
\DeclareMathOperator{\trs}{trace}
\DeclareMathOperator{\Ind}{Ind}
\DeclareMathOperator{\mat}{Mat}
\DeclareMathOperator{\id}{Id}
\DeclareMathOperator{\vect}{vect}
\DeclareMathOperator{\img}{img}
\DeclareMathOperator{\cov}{Cov}
\DeclareMathOperator{\dist}{dist}
\DeclareMathOperator{\irr}{Irr}
\DeclareMathOperator{\image}{Im}
\DeclareMathOperator{\pd}{\partial}
\DeclareMathOperator{\epi}{epi}
\DeclareMathOperator{\Argmin}{Argmin}
\DeclareMathOperator{\dom}{dom}
\DeclareMathOperator{\proj}{proj}
\DeclareMathOperator{\ctg}{ctg}
\DeclareMathOperator{\supp}{supp}
\DeclareMathOperator{\argmin}{argmin}
\DeclareMathOperator{\mult}{mult}
\DeclareMathOperator{\ch}{ch}
\DeclareMathOperator{\sh}{sh}
\DeclareMathOperator{\rang}{rang}
\DeclareMathOperator{\diam}{diam}
\DeclareMathOperator{\Epigraphe}{Epigraphe}




\usepackage{xcolor}
\everymath{\color{blue}}
%\everymath{\color[rgb]{0,1,1}}
%\pagecolor[rgb]{0,0,0.5}


\newcommand*{\pdtest}[3][]{\ensuremath{\frac{\partial^{#1} #2}{\partial #3}}}

\newcommand*{\deffunc}[6][]{\ensuremath{
\begin{array}{rcl}
#2 : #3 &\rightarrow& #4\\
#5 &\mapsto& #6
\end{array}
}}

\newcommand{\eqcolon}{\mathrel{\resizebox{\widthof{$\mathord{=}$}}{\height}{ $\!\!=\!\!\resizebox{1.2\width}{0.8\height}{\raisebox{0.23ex}{$\mathop{:}$}}\!\!$ }}}
\newcommand{\coloneq}{\mathrel{\resizebox{\widthof{$\mathord{=}$}}{\height}{ $\!\!\resizebox{1.2\width}{0.8\height}{\raisebox{0.23ex}{$\mathop{:}$}}\!\!=\!\!$ }}}
\newcommand{\eqcolonl}{\ensuremath{\mathrel{=\!\!\mathop{:}}}}
\newcommand{\coloneql}{\ensuremath{\mathrel{\mathop{:} \!\! =}}}
\newcommand{\vc}[1]{% inline column vector
  \left(\begin{smallmatrix}#1\end{smallmatrix}\right)%
}
\newcommand{\vr}[1]{% inline row vector
  \begin{smallmatrix}(\,#1\,)\end{smallmatrix}%
}
\makeatletter
\newcommand*{\defeq}{\ =\mathrel{\rlap{%
                     \raisebox{0.3ex}{$\m@th\cdot$}}%
                     \raisebox{-0.3ex}{$\m@th\cdot$}}%
                     }
\makeatother

\newcommand{\mathcircle}[1]{% inline row vector
 \overset{\circ}{#1}
}
\newcommand{\ulim}{% low limit
 \underline{\lim}
}
\newcommand{\ssi}{% iff
\iff
}
\newcommand{\ps}[2]{
\expval{#1 | #2}
}
\newcommand{\df}[1]{
\mqty{#1}
}
\newcommand{\n}[1]{
\norm{#1}
}
\newcommand{\sys}[1]{
\left\{\smqty{#1}\right.
}


\newcommand{\eqdef}{\ensuremath{\overset{\text{def}}=}}


\def\Circlearrowright{\ensuremath{%
  \rotatebox[origin=c]{230}{$\circlearrowright$}}}

\newcommand\ct[1]{\text{\rmfamily\upshape #1}}
\newcommand\question[1]{ {\color{red} ...!? \small #1}}
\newcommand\caz[1]{\left\{\begin{array} #1 \end{array}\right.}
\newcommand\const{\text{\rmfamily\upshape const}}
\newcommand\toP{ \overset{\pro}{\to}}
\newcommand\toPP{ \overset{\text{PP}}{\to}}
\newcommand{\oeq}{\mathrel{\text{\textcircled{$=$}}}}





\usepackage{xcolor}
% \usepackage[normalem]{ulem}
\usepackage{lipsum}
\makeatletter
% \newcommand\colorwave[1][blue]{\bgroup \markoverwith{\lower3.5\p@\hbox{\sixly \textcolor{#1}{\char58}}}\ULon}
%\font\sixly=lasy6 % does not re-load if already loaded, so no memory problem.

\newmdtheoremenv[
linewidth= 1pt,linecolor= blue,%
leftmargin=20,rightmargin=20,innertopmargin=0pt, innerrightmargin=40,%
tikzsetting = { draw=lightgray, line width = 0.3pt,dashed,%
dash pattern = on 15pt off 3pt},%
splittopskip=\topskip,skipbelow=\baselineskip,%
skipabove=\baselineskip,ntheorem,roundcorner=0pt,
% backgroundcolor=pagebg,font=\color{orange}\sffamily, fontcolor=white
]{examplebox}{Exemple}[section]



\newcommand\R{\mathbb{R}}
\newcommand\Z{\mathbb{Z}}
\newcommand\N{\mathbb{N}}
\newcommand\E{\mathbb{E}}
\newcommand\F{\mathcal{F}}
\newcommand\cH{\mathcal{H}}
\newcommand\V{\mathbb{V}}
\newcommand\dmo{ ^{-1} }
\newcommand\kapa{\kappa}
\newcommand\im{Im}
\newcommand\hs{\mathcal{H}}





\usepackage{soul}

\makeatletter
\newcommand*{\whiten}[1]{\llap{\textcolor{white}{{\the\SOUL@token}}\hspace{#1pt}}}
\DeclareRobustCommand*\myul{%
    \def\SOUL@everyspace{\underline{\space}\kern\z@}%
    \def\SOUL@everytoken{%
     \setbox0=\hbox{\the\SOUL@token}%
     \ifdim\dp0>\z@
        \raisebox{\dp0}{\underline{\phantom{\the\SOUL@token}}}%
        \whiten{1}\whiten{0}%
        \whiten{-1}\whiten{-2}%
        \llap{\the\SOUL@token}%
     \else
        \underline{\the\SOUL@token}%
     \fi}%
\SOUL@}
\makeatother

\newcommand*{\demp}{\fontfamily{lmtt}\selectfont}

\DeclareTextFontCommand{\textdemp}{\demp}

\begin{document}

\ifcomment
Multiline
comment
\fi
\ifcomment
\myul{Typesetting test}
% \color[rgb]{1,1,1}
$∑_i^n≠ 60º±∞π∆¬≈√j∫h≤≥µ$

$\CR \R\pro\ind\pro\gS\pro
\mqty[a&b\\c&d]$
$\pro\mathbb{P}$
$\dd{x}$

  \[
    \alpha(x)=\left\{
                \begin{array}{ll}
                  x\\
                  \frac{1}{1+e^{-kx}}\\
                  \frac{e^x-e^{-x}}{e^x+e^{-x}}
                \end{array}
              \right.
  \]

  $\expval{x}$
  
  $\chi_\rho(ghg\dmo)=\Tr(\rho_{ghg\dmo})=\Tr(\rho_g\circ\rho_h\circ\rho\dmo_g)=\Tr(\rho_h)\overset{\mbox{\scalebox{0.5}{$\Tr(AB)=\Tr(BA)$}}}{=}\chi_\rho(h)$
  	$\mathop{\oplus}_{\substack{x\in X}}$

$\mat(\rho_g)=(a_{ij}(g))_{\scriptsize \substack{1\leq i\leq d \\ 1\leq j\leq d}}$ et $\mat(\rho'_g)=(a'_{ij}(g))_{\scriptsize \substack{1\leq i'\leq d' \\ 1\leq j'\leq d'}}$



\[\int_a^b{\mathbb{R}^2}g(u, v)\dd{P_{XY}}(u, v)=\iint g(u,v) f_{XY}(u, v)\dd \lambda(u) \dd \lambda(v)\]
$$\lim_{x\to\infty} f(x)$$	
$$\iiiint_V \mu(t,u,v,w) \,dt\,du\,dv\,dw$$
$$\sum_{n=1}^{\infty} 2^{-n} = 1$$	
\begin{definition}
	Si $X$ et $Y$ sont 2 v.a. ou definit la \textsc{Covariance} entre $X$ et $Y$ comme
	$\cov(X,Y)\overset{\text{def}}{=}\E\left[(X-\E(X))(Y-\E(Y))\right]=\E(XY)-\E(X)\E(Y)$.
\end{definition}
\fi
\pagebreak

% \tableofcontents

% insert your code here
%\input{./algebra/main.tex}
%\input{./geometrie-differentielle/main.tex}
%\input{./probabilite/main.tex}
%\input{./analyse-fonctionnelle/main.tex}
% \input{./Analyse-convexe-et-dualite-en-optimisation/main.tex}
%\input{./tikz/main.tex}
%\input{./Theorie-du-distributions/main.tex}
%\input{./optimisation/mine.tex}
 \input{./modelisation/main.tex}

% yves.aubry@univ-tln.fr : algebra

\end{document}

%% !TEX encoding = UTF-8 Unicode
% !TEX TS-program = xelatex

\documentclass[french]{report}

%\usepackage[utf8]{inputenc}
%\usepackage[T1]{fontenc}
\usepackage{babel}


\newif\ifcomment
%\commenttrue # Show comments

\usepackage{physics}
\usepackage{amssymb}


\usepackage{amsthm}
% \usepackage{thmtools}
\usepackage{mathtools}
\usepackage{amsfonts}

\usepackage{color}

\usepackage{tikz}

\usepackage{geometry}
\geometry{a5paper, margin=0.1in, right=1cm}

\usepackage{dsfont}

\usepackage{graphicx}
\graphicspath{ {images/} }

\usepackage{faktor}

\usepackage{IEEEtrantools}
\usepackage{enumerate}   
\usepackage[PostScript=dvips]{"/Users/aware/Documents/Courses/diagrams"}


\newtheorem{theorem}{Théorème}[section]
\renewcommand{\thetheorem}{\arabic{theorem}}
\newtheorem{lemme}{Lemme}[section]
\renewcommand{\thelemme}{\arabic{lemme}}
\newtheorem{proposition}{Proposition}[section]
\renewcommand{\theproposition}{\arabic{proposition}}
\newtheorem{notations}{Notations}[section]
\newtheorem{problem}{Problème}[section]
\newtheorem{corollary}{Corollaire}[theorem]
\renewcommand{\thecorollary}{\arabic{corollary}}
\newtheorem{property}{Propriété}[section]
\newtheorem{objective}{Objectif}[section]

\theoremstyle{definition}
\newtheorem{definition}{Définition}[section]
\renewcommand{\thedefinition}{\arabic{definition}}
\newtheorem{exercise}{Exercice}[chapter]
\renewcommand{\theexercise}{\arabic{exercise}}
\newtheorem{example}{Exemple}[chapter]
\renewcommand{\theexample}{\arabic{example}}
\newtheorem*{solution}{Solution}
\newtheorem*{application}{Application}
\newtheorem*{notation}{Notation}
\newtheorem*{vocabulary}{Vocabulaire}
\newtheorem*{properties}{Propriétés}



\theoremstyle{remark}
\newtheorem*{remark}{Remarque}
\newtheorem*{rappel}{Rappel}


\usepackage{etoolbox}
\AtBeginEnvironment{exercise}{\small}
\AtBeginEnvironment{example}{\small}

\usepackage{cases}
\usepackage[red]{mypack}

\usepackage[framemethod=TikZ]{mdframed}

\definecolor{bg}{rgb}{0.4,0.25,0.95}
\definecolor{pagebg}{rgb}{0,0,0.5}
\surroundwithmdframed[
   topline=false,
   rightline=false,
   bottomline=false,
   leftmargin=\parindent,
   skipabove=8pt,
   skipbelow=8pt,
   linecolor=blue,
   innerbottommargin=10pt,
   % backgroundcolor=bg,font=\color{orange}\sffamily, fontcolor=white
]{definition}

\usepackage{empheq}
\usepackage[most]{tcolorbox}

\newtcbox{\mymath}[1][]{%
    nobeforeafter, math upper, tcbox raise base,
    enhanced, colframe=blue!30!black,
    colback=red!10, boxrule=1pt,
    #1}

\usepackage{unixode}


\DeclareMathOperator{\ord}{ord}
\DeclareMathOperator{\orb}{orb}
\DeclareMathOperator{\stab}{stab}
\DeclareMathOperator{\Stab}{stab}
\DeclareMathOperator{\ppcm}{ppcm}
\DeclareMathOperator{\conj}{Conj}
\DeclareMathOperator{\End}{End}
\DeclareMathOperator{\rot}{rot}
\DeclareMathOperator{\trs}{trace}
\DeclareMathOperator{\Ind}{Ind}
\DeclareMathOperator{\mat}{Mat}
\DeclareMathOperator{\id}{Id}
\DeclareMathOperator{\vect}{vect}
\DeclareMathOperator{\img}{img}
\DeclareMathOperator{\cov}{Cov}
\DeclareMathOperator{\dist}{dist}
\DeclareMathOperator{\irr}{Irr}
\DeclareMathOperator{\image}{Im}
\DeclareMathOperator{\pd}{\partial}
\DeclareMathOperator{\epi}{epi}
\DeclareMathOperator{\Argmin}{Argmin}
\DeclareMathOperator{\dom}{dom}
\DeclareMathOperator{\proj}{proj}
\DeclareMathOperator{\ctg}{ctg}
\DeclareMathOperator{\supp}{supp}
\DeclareMathOperator{\argmin}{argmin}
\DeclareMathOperator{\mult}{mult}
\DeclareMathOperator{\ch}{ch}
\DeclareMathOperator{\sh}{sh}
\DeclareMathOperator{\rang}{rang}
\DeclareMathOperator{\diam}{diam}
\DeclareMathOperator{\Epigraphe}{Epigraphe}




\usepackage{xcolor}
\everymath{\color{blue}}
%\everymath{\color[rgb]{0,1,1}}
%\pagecolor[rgb]{0,0,0.5}


\newcommand*{\pdtest}[3][]{\ensuremath{\frac{\partial^{#1} #2}{\partial #3}}}

\newcommand*{\deffunc}[6][]{\ensuremath{
\begin{array}{rcl}
#2 : #3 &\rightarrow& #4\\
#5 &\mapsto& #6
\end{array}
}}

\newcommand{\eqcolon}{\mathrel{\resizebox{\widthof{$\mathord{=}$}}{\height}{ $\!\!=\!\!\resizebox{1.2\width}{0.8\height}{\raisebox{0.23ex}{$\mathop{:}$}}\!\!$ }}}
\newcommand{\coloneq}{\mathrel{\resizebox{\widthof{$\mathord{=}$}}{\height}{ $\!\!\resizebox{1.2\width}{0.8\height}{\raisebox{0.23ex}{$\mathop{:}$}}\!\!=\!\!$ }}}
\newcommand{\eqcolonl}{\ensuremath{\mathrel{=\!\!\mathop{:}}}}
\newcommand{\coloneql}{\ensuremath{\mathrel{\mathop{:} \!\! =}}}
\newcommand{\vc}[1]{% inline column vector
  \left(\begin{smallmatrix}#1\end{smallmatrix}\right)%
}
\newcommand{\vr}[1]{% inline row vector
  \begin{smallmatrix}(\,#1\,)\end{smallmatrix}%
}
\makeatletter
\newcommand*{\defeq}{\ =\mathrel{\rlap{%
                     \raisebox{0.3ex}{$\m@th\cdot$}}%
                     \raisebox{-0.3ex}{$\m@th\cdot$}}%
                     }
\makeatother

\newcommand{\mathcircle}[1]{% inline row vector
 \overset{\circ}{#1}
}
\newcommand{\ulim}{% low limit
 \underline{\lim}
}
\newcommand{\ssi}{% iff
\iff
}
\newcommand{\ps}[2]{
\expval{#1 | #2}
}
\newcommand{\df}[1]{
\mqty{#1}
}
\newcommand{\n}[1]{
\norm{#1}
}
\newcommand{\sys}[1]{
\left\{\smqty{#1}\right.
}


\newcommand{\eqdef}{\ensuremath{\overset{\text{def}}=}}


\def\Circlearrowright{\ensuremath{%
  \rotatebox[origin=c]{230}{$\circlearrowright$}}}

\newcommand\ct[1]{\text{\rmfamily\upshape #1}}
\newcommand\question[1]{ {\color{red} ...!? \small #1}}
\newcommand\caz[1]{\left\{\begin{array} #1 \end{array}\right.}
\newcommand\const{\text{\rmfamily\upshape const}}
\newcommand\toP{ \overset{\pro}{\to}}
\newcommand\toPP{ \overset{\text{PP}}{\to}}
\newcommand{\oeq}{\mathrel{\text{\textcircled{$=$}}}}





\usepackage{xcolor}
% \usepackage[normalem]{ulem}
\usepackage{lipsum}
\makeatletter
% \newcommand\colorwave[1][blue]{\bgroup \markoverwith{\lower3.5\p@\hbox{\sixly \textcolor{#1}{\char58}}}\ULon}
%\font\sixly=lasy6 % does not re-load if already loaded, so no memory problem.

\newmdtheoremenv[
linewidth= 1pt,linecolor= blue,%
leftmargin=20,rightmargin=20,innertopmargin=0pt, innerrightmargin=40,%
tikzsetting = { draw=lightgray, line width = 0.3pt,dashed,%
dash pattern = on 15pt off 3pt},%
splittopskip=\topskip,skipbelow=\baselineskip,%
skipabove=\baselineskip,ntheorem,roundcorner=0pt,
% backgroundcolor=pagebg,font=\color{orange}\sffamily, fontcolor=white
]{examplebox}{Exemple}[section]



\newcommand\R{\mathbb{R}}
\newcommand\Z{\mathbb{Z}}
\newcommand\N{\mathbb{N}}
\newcommand\E{\mathbb{E}}
\newcommand\F{\mathcal{F}}
\newcommand\cH{\mathcal{H}}
\newcommand\V{\mathbb{V}}
\newcommand\dmo{ ^{-1} }
\newcommand\kapa{\kappa}
\newcommand\im{Im}
\newcommand\hs{\mathcal{H}}





\usepackage{soul}

\makeatletter
\newcommand*{\whiten}[1]{\llap{\textcolor{white}{{\the\SOUL@token}}\hspace{#1pt}}}
\DeclareRobustCommand*\myul{%
    \def\SOUL@everyspace{\underline{\space}\kern\z@}%
    \def\SOUL@everytoken{%
     \setbox0=\hbox{\the\SOUL@token}%
     \ifdim\dp0>\z@
        \raisebox{\dp0}{\underline{\phantom{\the\SOUL@token}}}%
        \whiten{1}\whiten{0}%
        \whiten{-1}\whiten{-2}%
        \llap{\the\SOUL@token}%
     \else
        \underline{\the\SOUL@token}%
     \fi}%
\SOUL@}
\makeatother

\newcommand*{\demp}{\fontfamily{lmtt}\selectfont}

\DeclareTextFontCommand{\textdemp}{\demp}

\begin{document}

\ifcomment
Multiline
comment
\fi
\ifcomment
\myul{Typesetting test}
% \color[rgb]{1,1,1}
$∑_i^n≠ 60º±∞π∆¬≈√j∫h≤≥µ$

$\CR \R\pro\ind\pro\gS\pro
\mqty[a&b\\c&d]$
$\pro\mathbb{P}$
$\dd{x}$

  \[
    \alpha(x)=\left\{
                \begin{array}{ll}
                  x\\
                  \frac{1}{1+e^{-kx}}\\
                  \frac{e^x-e^{-x}}{e^x+e^{-x}}
                \end{array}
              \right.
  \]

  $\expval{x}$
  
  $\chi_\rho(ghg\dmo)=\Tr(\rho_{ghg\dmo})=\Tr(\rho_g\circ\rho_h\circ\rho\dmo_g)=\Tr(\rho_h)\overset{\mbox{\scalebox{0.5}{$\Tr(AB)=\Tr(BA)$}}}{=}\chi_\rho(h)$
  	$\mathop{\oplus}_{\substack{x\in X}}$

$\mat(\rho_g)=(a_{ij}(g))_{\scriptsize \substack{1\leq i\leq d \\ 1\leq j\leq d}}$ et $\mat(\rho'_g)=(a'_{ij}(g))_{\scriptsize \substack{1\leq i'\leq d' \\ 1\leq j'\leq d'}}$



\[\int_a^b{\mathbb{R}^2}g(u, v)\dd{P_{XY}}(u, v)=\iint g(u,v) f_{XY}(u, v)\dd \lambda(u) \dd \lambda(v)\]
$$\lim_{x\to\infty} f(x)$$	
$$\iiiint_V \mu(t,u,v,w) \,dt\,du\,dv\,dw$$
$$\sum_{n=1}^{\infty} 2^{-n} = 1$$	
\begin{definition}
	Si $X$ et $Y$ sont 2 v.a. ou definit la \textsc{Covariance} entre $X$ et $Y$ comme
	$\cov(X,Y)\overset{\text{def}}{=}\E\left[(X-\E(X))(Y-\E(Y))\right]=\E(XY)-\E(X)\E(Y)$.
\end{definition}
\fi
\pagebreak

% \tableofcontents

% insert your code here
%\input{./algebra/main.tex}
%\input{./geometrie-differentielle/main.tex}
%\input{./probabilite/main.tex}
%\input{./analyse-fonctionnelle/main.tex}
% \input{./Analyse-convexe-et-dualite-en-optimisation/main.tex}
%\input{./tikz/main.tex}
%\input{./Theorie-du-distributions/main.tex}
%\input{./optimisation/mine.tex}
 \input{./modelisation/main.tex}

% yves.aubry@univ-tln.fr : algebra

\end{document}

%% !TEX encoding = UTF-8 Unicode
% !TEX TS-program = xelatex

\documentclass[french]{report}

%\usepackage[utf8]{inputenc}
%\usepackage[T1]{fontenc}
\usepackage{babel}


\newif\ifcomment
%\commenttrue # Show comments

\usepackage{physics}
\usepackage{amssymb}


\usepackage{amsthm}
% \usepackage{thmtools}
\usepackage{mathtools}
\usepackage{amsfonts}

\usepackage{color}

\usepackage{tikz}

\usepackage{geometry}
\geometry{a5paper, margin=0.1in, right=1cm}

\usepackage{dsfont}

\usepackage{graphicx}
\graphicspath{ {images/} }

\usepackage{faktor}

\usepackage{IEEEtrantools}
\usepackage{enumerate}   
\usepackage[PostScript=dvips]{"/Users/aware/Documents/Courses/diagrams"}


\newtheorem{theorem}{Théorème}[section]
\renewcommand{\thetheorem}{\arabic{theorem}}
\newtheorem{lemme}{Lemme}[section]
\renewcommand{\thelemme}{\arabic{lemme}}
\newtheorem{proposition}{Proposition}[section]
\renewcommand{\theproposition}{\arabic{proposition}}
\newtheorem{notations}{Notations}[section]
\newtheorem{problem}{Problème}[section]
\newtheorem{corollary}{Corollaire}[theorem]
\renewcommand{\thecorollary}{\arabic{corollary}}
\newtheorem{property}{Propriété}[section]
\newtheorem{objective}{Objectif}[section]

\theoremstyle{definition}
\newtheorem{definition}{Définition}[section]
\renewcommand{\thedefinition}{\arabic{definition}}
\newtheorem{exercise}{Exercice}[chapter]
\renewcommand{\theexercise}{\arabic{exercise}}
\newtheorem{example}{Exemple}[chapter]
\renewcommand{\theexample}{\arabic{example}}
\newtheorem*{solution}{Solution}
\newtheorem*{application}{Application}
\newtheorem*{notation}{Notation}
\newtheorem*{vocabulary}{Vocabulaire}
\newtheorem*{properties}{Propriétés}



\theoremstyle{remark}
\newtheorem*{remark}{Remarque}
\newtheorem*{rappel}{Rappel}


\usepackage{etoolbox}
\AtBeginEnvironment{exercise}{\small}
\AtBeginEnvironment{example}{\small}

\usepackage{cases}
\usepackage[red]{mypack}

\usepackage[framemethod=TikZ]{mdframed}

\definecolor{bg}{rgb}{0.4,0.25,0.95}
\definecolor{pagebg}{rgb}{0,0,0.5}
\surroundwithmdframed[
   topline=false,
   rightline=false,
   bottomline=false,
   leftmargin=\parindent,
   skipabove=8pt,
   skipbelow=8pt,
   linecolor=blue,
   innerbottommargin=10pt,
   % backgroundcolor=bg,font=\color{orange}\sffamily, fontcolor=white
]{definition}

\usepackage{empheq}
\usepackage[most]{tcolorbox}

\newtcbox{\mymath}[1][]{%
    nobeforeafter, math upper, tcbox raise base,
    enhanced, colframe=blue!30!black,
    colback=red!10, boxrule=1pt,
    #1}

\usepackage{unixode}


\DeclareMathOperator{\ord}{ord}
\DeclareMathOperator{\orb}{orb}
\DeclareMathOperator{\stab}{stab}
\DeclareMathOperator{\Stab}{stab}
\DeclareMathOperator{\ppcm}{ppcm}
\DeclareMathOperator{\conj}{Conj}
\DeclareMathOperator{\End}{End}
\DeclareMathOperator{\rot}{rot}
\DeclareMathOperator{\trs}{trace}
\DeclareMathOperator{\Ind}{Ind}
\DeclareMathOperator{\mat}{Mat}
\DeclareMathOperator{\id}{Id}
\DeclareMathOperator{\vect}{vect}
\DeclareMathOperator{\img}{img}
\DeclareMathOperator{\cov}{Cov}
\DeclareMathOperator{\dist}{dist}
\DeclareMathOperator{\irr}{Irr}
\DeclareMathOperator{\image}{Im}
\DeclareMathOperator{\pd}{\partial}
\DeclareMathOperator{\epi}{epi}
\DeclareMathOperator{\Argmin}{Argmin}
\DeclareMathOperator{\dom}{dom}
\DeclareMathOperator{\proj}{proj}
\DeclareMathOperator{\ctg}{ctg}
\DeclareMathOperator{\supp}{supp}
\DeclareMathOperator{\argmin}{argmin}
\DeclareMathOperator{\mult}{mult}
\DeclareMathOperator{\ch}{ch}
\DeclareMathOperator{\sh}{sh}
\DeclareMathOperator{\rang}{rang}
\DeclareMathOperator{\diam}{diam}
\DeclareMathOperator{\Epigraphe}{Epigraphe}




\usepackage{xcolor}
\everymath{\color{blue}}
%\everymath{\color[rgb]{0,1,1}}
%\pagecolor[rgb]{0,0,0.5}


\newcommand*{\pdtest}[3][]{\ensuremath{\frac{\partial^{#1} #2}{\partial #3}}}

\newcommand*{\deffunc}[6][]{\ensuremath{
\begin{array}{rcl}
#2 : #3 &\rightarrow& #4\\
#5 &\mapsto& #6
\end{array}
}}

\newcommand{\eqcolon}{\mathrel{\resizebox{\widthof{$\mathord{=}$}}{\height}{ $\!\!=\!\!\resizebox{1.2\width}{0.8\height}{\raisebox{0.23ex}{$\mathop{:}$}}\!\!$ }}}
\newcommand{\coloneq}{\mathrel{\resizebox{\widthof{$\mathord{=}$}}{\height}{ $\!\!\resizebox{1.2\width}{0.8\height}{\raisebox{0.23ex}{$\mathop{:}$}}\!\!=\!\!$ }}}
\newcommand{\eqcolonl}{\ensuremath{\mathrel{=\!\!\mathop{:}}}}
\newcommand{\coloneql}{\ensuremath{\mathrel{\mathop{:} \!\! =}}}
\newcommand{\vc}[1]{% inline column vector
  \left(\begin{smallmatrix}#1\end{smallmatrix}\right)%
}
\newcommand{\vr}[1]{% inline row vector
  \begin{smallmatrix}(\,#1\,)\end{smallmatrix}%
}
\makeatletter
\newcommand*{\defeq}{\ =\mathrel{\rlap{%
                     \raisebox{0.3ex}{$\m@th\cdot$}}%
                     \raisebox{-0.3ex}{$\m@th\cdot$}}%
                     }
\makeatother

\newcommand{\mathcircle}[1]{% inline row vector
 \overset{\circ}{#1}
}
\newcommand{\ulim}{% low limit
 \underline{\lim}
}
\newcommand{\ssi}{% iff
\iff
}
\newcommand{\ps}[2]{
\expval{#1 | #2}
}
\newcommand{\df}[1]{
\mqty{#1}
}
\newcommand{\n}[1]{
\norm{#1}
}
\newcommand{\sys}[1]{
\left\{\smqty{#1}\right.
}


\newcommand{\eqdef}{\ensuremath{\overset{\text{def}}=}}


\def\Circlearrowright{\ensuremath{%
  \rotatebox[origin=c]{230}{$\circlearrowright$}}}

\newcommand\ct[1]{\text{\rmfamily\upshape #1}}
\newcommand\question[1]{ {\color{red} ...!? \small #1}}
\newcommand\caz[1]{\left\{\begin{array} #1 \end{array}\right.}
\newcommand\const{\text{\rmfamily\upshape const}}
\newcommand\toP{ \overset{\pro}{\to}}
\newcommand\toPP{ \overset{\text{PP}}{\to}}
\newcommand{\oeq}{\mathrel{\text{\textcircled{$=$}}}}





\usepackage{xcolor}
% \usepackage[normalem]{ulem}
\usepackage{lipsum}
\makeatletter
% \newcommand\colorwave[1][blue]{\bgroup \markoverwith{\lower3.5\p@\hbox{\sixly \textcolor{#1}{\char58}}}\ULon}
%\font\sixly=lasy6 % does not re-load if already loaded, so no memory problem.

\newmdtheoremenv[
linewidth= 1pt,linecolor= blue,%
leftmargin=20,rightmargin=20,innertopmargin=0pt, innerrightmargin=40,%
tikzsetting = { draw=lightgray, line width = 0.3pt,dashed,%
dash pattern = on 15pt off 3pt},%
splittopskip=\topskip,skipbelow=\baselineskip,%
skipabove=\baselineskip,ntheorem,roundcorner=0pt,
% backgroundcolor=pagebg,font=\color{orange}\sffamily, fontcolor=white
]{examplebox}{Exemple}[section]



\newcommand\R{\mathbb{R}}
\newcommand\Z{\mathbb{Z}}
\newcommand\N{\mathbb{N}}
\newcommand\E{\mathbb{E}}
\newcommand\F{\mathcal{F}}
\newcommand\cH{\mathcal{H}}
\newcommand\V{\mathbb{V}}
\newcommand\dmo{ ^{-1} }
\newcommand\kapa{\kappa}
\newcommand\im{Im}
\newcommand\hs{\mathcal{H}}





\usepackage{soul}

\makeatletter
\newcommand*{\whiten}[1]{\llap{\textcolor{white}{{\the\SOUL@token}}\hspace{#1pt}}}
\DeclareRobustCommand*\myul{%
    \def\SOUL@everyspace{\underline{\space}\kern\z@}%
    \def\SOUL@everytoken{%
     \setbox0=\hbox{\the\SOUL@token}%
     \ifdim\dp0>\z@
        \raisebox{\dp0}{\underline{\phantom{\the\SOUL@token}}}%
        \whiten{1}\whiten{0}%
        \whiten{-1}\whiten{-2}%
        \llap{\the\SOUL@token}%
     \else
        \underline{\the\SOUL@token}%
     \fi}%
\SOUL@}
\makeatother

\newcommand*{\demp}{\fontfamily{lmtt}\selectfont}

\DeclareTextFontCommand{\textdemp}{\demp}

\begin{document}

\ifcomment
Multiline
comment
\fi
\ifcomment
\myul{Typesetting test}
% \color[rgb]{1,1,1}
$∑_i^n≠ 60º±∞π∆¬≈√j∫h≤≥µ$

$\CR \R\pro\ind\pro\gS\pro
\mqty[a&b\\c&d]$
$\pro\mathbb{P}$
$\dd{x}$

  \[
    \alpha(x)=\left\{
                \begin{array}{ll}
                  x\\
                  \frac{1}{1+e^{-kx}}\\
                  \frac{e^x-e^{-x}}{e^x+e^{-x}}
                \end{array}
              \right.
  \]

  $\expval{x}$
  
  $\chi_\rho(ghg\dmo)=\Tr(\rho_{ghg\dmo})=\Tr(\rho_g\circ\rho_h\circ\rho\dmo_g)=\Tr(\rho_h)\overset{\mbox{\scalebox{0.5}{$\Tr(AB)=\Tr(BA)$}}}{=}\chi_\rho(h)$
  	$\mathop{\oplus}_{\substack{x\in X}}$

$\mat(\rho_g)=(a_{ij}(g))_{\scriptsize \substack{1\leq i\leq d \\ 1\leq j\leq d}}$ et $\mat(\rho'_g)=(a'_{ij}(g))_{\scriptsize \substack{1\leq i'\leq d' \\ 1\leq j'\leq d'}}$



\[\int_a^b{\mathbb{R}^2}g(u, v)\dd{P_{XY}}(u, v)=\iint g(u,v) f_{XY}(u, v)\dd \lambda(u) \dd \lambda(v)\]
$$\lim_{x\to\infty} f(x)$$	
$$\iiiint_V \mu(t,u,v,w) \,dt\,du\,dv\,dw$$
$$\sum_{n=1}^{\infty} 2^{-n} = 1$$	
\begin{definition}
	Si $X$ et $Y$ sont 2 v.a. ou definit la \textsc{Covariance} entre $X$ et $Y$ comme
	$\cov(X,Y)\overset{\text{def}}{=}\E\left[(X-\E(X))(Y-\E(Y))\right]=\E(XY)-\E(X)\E(Y)$.
\end{definition}
\fi
\pagebreak

% \tableofcontents

% insert your code here
%\input{./algebra/main.tex}
%\input{./geometrie-differentielle/main.tex}
%\input{./probabilite/main.tex}
%\input{./analyse-fonctionnelle/main.tex}
% \input{./Analyse-convexe-et-dualite-en-optimisation/main.tex}
%\input{./tikz/main.tex}
%\input{./Theorie-du-distributions/main.tex}
%\input{./optimisation/mine.tex}
 \input{./modelisation/main.tex}

% yves.aubry@univ-tln.fr : algebra

\end{document}

%% !TEX encoding = UTF-8 Unicode
% !TEX TS-program = xelatex

\documentclass[french]{report}

%\usepackage[utf8]{inputenc}
%\usepackage[T1]{fontenc}
\usepackage{babel}


\newif\ifcomment
%\commenttrue # Show comments

\usepackage{physics}
\usepackage{amssymb}


\usepackage{amsthm}
% \usepackage{thmtools}
\usepackage{mathtools}
\usepackage{amsfonts}

\usepackage{color}

\usepackage{tikz}

\usepackage{geometry}
\geometry{a5paper, margin=0.1in, right=1cm}

\usepackage{dsfont}

\usepackage{graphicx}
\graphicspath{ {images/} }

\usepackage{faktor}

\usepackage{IEEEtrantools}
\usepackage{enumerate}   
\usepackage[PostScript=dvips]{"/Users/aware/Documents/Courses/diagrams"}


\newtheorem{theorem}{Théorème}[section]
\renewcommand{\thetheorem}{\arabic{theorem}}
\newtheorem{lemme}{Lemme}[section]
\renewcommand{\thelemme}{\arabic{lemme}}
\newtheorem{proposition}{Proposition}[section]
\renewcommand{\theproposition}{\arabic{proposition}}
\newtheorem{notations}{Notations}[section]
\newtheorem{problem}{Problème}[section]
\newtheorem{corollary}{Corollaire}[theorem]
\renewcommand{\thecorollary}{\arabic{corollary}}
\newtheorem{property}{Propriété}[section]
\newtheorem{objective}{Objectif}[section]

\theoremstyle{definition}
\newtheorem{definition}{Définition}[section]
\renewcommand{\thedefinition}{\arabic{definition}}
\newtheorem{exercise}{Exercice}[chapter]
\renewcommand{\theexercise}{\arabic{exercise}}
\newtheorem{example}{Exemple}[chapter]
\renewcommand{\theexample}{\arabic{example}}
\newtheorem*{solution}{Solution}
\newtheorem*{application}{Application}
\newtheorem*{notation}{Notation}
\newtheorem*{vocabulary}{Vocabulaire}
\newtheorem*{properties}{Propriétés}



\theoremstyle{remark}
\newtheorem*{remark}{Remarque}
\newtheorem*{rappel}{Rappel}


\usepackage{etoolbox}
\AtBeginEnvironment{exercise}{\small}
\AtBeginEnvironment{example}{\small}

\usepackage{cases}
\usepackage[red]{mypack}

\usepackage[framemethod=TikZ]{mdframed}

\definecolor{bg}{rgb}{0.4,0.25,0.95}
\definecolor{pagebg}{rgb}{0,0,0.5}
\surroundwithmdframed[
   topline=false,
   rightline=false,
   bottomline=false,
   leftmargin=\parindent,
   skipabove=8pt,
   skipbelow=8pt,
   linecolor=blue,
   innerbottommargin=10pt,
   % backgroundcolor=bg,font=\color{orange}\sffamily, fontcolor=white
]{definition}

\usepackage{empheq}
\usepackage[most]{tcolorbox}

\newtcbox{\mymath}[1][]{%
    nobeforeafter, math upper, tcbox raise base,
    enhanced, colframe=blue!30!black,
    colback=red!10, boxrule=1pt,
    #1}

\usepackage{unixode}


\DeclareMathOperator{\ord}{ord}
\DeclareMathOperator{\orb}{orb}
\DeclareMathOperator{\stab}{stab}
\DeclareMathOperator{\Stab}{stab}
\DeclareMathOperator{\ppcm}{ppcm}
\DeclareMathOperator{\conj}{Conj}
\DeclareMathOperator{\End}{End}
\DeclareMathOperator{\rot}{rot}
\DeclareMathOperator{\trs}{trace}
\DeclareMathOperator{\Ind}{Ind}
\DeclareMathOperator{\mat}{Mat}
\DeclareMathOperator{\id}{Id}
\DeclareMathOperator{\vect}{vect}
\DeclareMathOperator{\img}{img}
\DeclareMathOperator{\cov}{Cov}
\DeclareMathOperator{\dist}{dist}
\DeclareMathOperator{\irr}{Irr}
\DeclareMathOperator{\image}{Im}
\DeclareMathOperator{\pd}{\partial}
\DeclareMathOperator{\epi}{epi}
\DeclareMathOperator{\Argmin}{Argmin}
\DeclareMathOperator{\dom}{dom}
\DeclareMathOperator{\proj}{proj}
\DeclareMathOperator{\ctg}{ctg}
\DeclareMathOperator{\supp}{supp}
\DeclareMathOperator{\argmin}{argmin}
\DeclareMathOperator{\mult}{mult}
\DeclareMathOperator{\ch}{ch}
\DeclareMathOperator{\sh}{sh}
\DeclareMathOperator{\rang}{rang}
\DeclareMathOperator{\diam}{diam}
\DeclareMathOperator{\Epigraphe}{Epigraphe}




\usepackage{xcolor}
\everymath{\color{blue}}
%\everymath{\color[rgb]{0,1,1}}
%\pagecolor[rgb]{0,0,0.5}


\newcommand*{\pdtest}[3][]{\ensuremath{\frac{\partial^{#1} #2}{\partial #3}}}

\newcommand*{\deffunc}[6][]{\ensuremath{
\begin{array}{rcl}
#2 : #3 &\rightarrow& #4\\
#5 &\mapsto& #6
\end{array}
}}

\newcommand{\eqcolon}{\mathrel{\resizebox{\widthof{$\mathord{=}$}}{\height}{ $\!\!=\!\!\resizebox{1.2\width}{0.8\height}{\raisebox{0.23ex}{$\mathop{:}$}}\!\!$ }}}
\newcommand{\coloneq}{\mathrel{\resizebox{\widthof{$\mathord{=}$}}{\height}{ $\!\!\resizebox{1.2\width}{0.8\height}{\raisebox{0.23ex}{$\mathop{:}$}}\!\!=\!\!$ }}}
\newcommand{\eqcolonl}{\ensuremath{\mathrel{=\!\!\mathop{:}}}}
\newcommand{\coloneql}{\ensuremath{\mathrel{\mathop{:} \!\! =}}}
\newcommand{\vc}[1]{% inline column vector
  \left(\begin{smallmatrix}#1\end{smallmatrix}\right)%
}
\newcommand{\vr}[1]{% inline row vector
  \begin{smallmatrix}(\,#1\,)\end{smallmatrix}%
}
\makeatletter
\newcommand*{\defeq}{\ =\mathrel{\rlap{%
                     \raisebox{0.3ex}{$\m@th\cdot$}}%
                     \raisebox{-0.3ex}{$\m@th\cdot$}}%
                     }
\makeatother

\newcommand{\mathcircle}[1]{% inline row vector
 \overset{\circ}{#1}
}
\newcommand{\ulim}{% low limit
 \underline{\lim}
}
\newcommand{\ssi}{% iff
\iff
}
\newcommand{\ps}[2]{
\expval{#1 | #2}
}
\newcommand{\df}[1]{
\mqty{#1}
}
\newcommand{\n}[1]{
\norm{#1}
}
\newcommand{\sys}[1]{
\left\{\smqty{#1}\right.
}


\newcommand{\eqdef}{\ensuremath{\overset{\text{def}}=}}


\def\Circlearrowright{\ensuremath{%
  \rotatebox[origin=c]{230}{$\circlearrowright$}}}

\newcommand\ct[1]{\text{\rmfamily\upshape #1}}
\newcommand\question[1]{ {\color{red} ...!? \small #1}}
\newcommand\caz[1]{\left\{\begin{array} #1 \end{array}\right.}
\newcommand\const{\text{\rmfamily\upshape const}}
\newcommand\toP{ \overset{\pro}{\to}}
\newcommand\toPP{ \overset{\text{PP}}{\to}}
\newcommand{\oeq}{\mathrel{\text{\textcircled{$=$}}}}





\usepackage{xcolor}
% \usepackage[normalem]{ulem}
\usepackage{lipsum}
\makeatletter
% \newcommand\colorwave[1][blue]{\bgroup \markoverwith{\lower3.5\p@\hbox{\sixly \textcolor{#1}{\char58}}}\ULon}
%\font\sixly=lasy6 % does not re-load if already loaded, so no memory problem.

\newmdtheoremenv[
linewidth= 1pt,linecolor= blue,%
leftmargin=20,rightmargin=20,innertopmargin=0pt, innerrightmargin=40,%
tikzsetting = { draw=lightgray, line width = 0.3pt,dashed,%
dash pattern = on 15pt off 3pt},%
splittopskip=\topskip,skipbelow=\baselineskip,%
skipabove=\baselineskip,ntheorem,roundcorner=0pt,
% backgroundcolor=pagebg,font=\color{orange}\sffamily, fontcolor=white
]{examplebox}{Exemple}[section]



\newcommand\R{\mathbb{R}}
\newcommand\Z{\mathbb{Z}}
\newcommand\N{\mathbb{N}}
\newcommand\E{\mathbb{E}}
\newcommand\F{\mathcal{F}}
\newcommand\cH{\mathcal{H}}
\newcommand\V{\mathbb{V}}
\newcommand\dmo{ ^{-1} }
\newcommand\kapa{\kappa}
\newcommand\im{Im}
\newcommand\hs{\mathcal{H}}





\usepackage{soul}

\makeatletter
\newcommand*{\whiten}[1]{\llap{\textcolor{white}{{\the\SOUL@token}}\hspace{#1pt}}}
\DeclareRobustCommand*\myul{%
    \def\SOUL@everyspace{\underline{\space}\kern\z@}%
    \def\SOUL@everytoken{%
     \setbox0=\hbox{\the\SOUL@token}%
     \ifdim\dp0>\z@
        \raisebox{\dp0}{\underline{\phantom{\the\SOUL@token}}}%
        \whiten{1}\whiten{0}%
        \whiten{-1}\whiten{-2}%
        \llap{\the\SOUL@token}%
     \else
        \underline{\the\SOUL@token}%
     \fi}%
\SOUL@}
\makeatother

\newcommand*{\demp}{\fontfamily{lmtt}\selectfont}

\DeclareTextFontCommand{\textdemp}{\demp}

\begin{document}

\ifcomment
Multiline
comment
\fi
\ifcomment
\myul{Typesetting test}
% \color[rgb]{1,1,1}
$∑_i^n≠ 60º±∞π∆¬≈√j∫h≤≥µ$

$\CR \R\pro\ind\pro\gS\pro
\mqty[a&b\\c&d]$
$\pro\mathbb{P}$
$\dd{x}$

  \[
    \alpha(x)=\left\{
                \begin{array}{ll}
                  x\\
                  \frac{1}{1+e^{-kx}}\\
                  \frac{e^x-e^{-x}}{e^x+e^{-x}}
                \end{array}
              \right.
  \]

  $\expval{x}$
  
  $\chi_\rho(ghg\dmo)=\Tr(\rho_{ghg\dmo})=\Tr(\rho_g\circ\rho_h\circ\rho\dmo_g)=\Tr(\rho_h)\overset{\mbox{\scalebox{0.5}{$\Tr(AB)=\Tr(BA)$}}}{=}\chi_\rho(h)$
  	$\mathop{\oplus}_{\substack{x\in X}}$

$\mat(\rho_g)=(a_{ij}(g))_{\scriptsize \substack{1\leq i\leq d \\ 1\leq j\leq d}}$ et $\mat(\rho'_g)=(a'_{ij}(g))_{\scriptsize \substack{1\leq i'\leq d' \\ 1\leq j'\leq d'}}$



\[\int_a^b{\mathbb{R}^2}g(u, v)\dd{P_{XY}}(u, v)=\iint g(u,v) f_{XY}(u, v)\dd \lambda(u) \dd \lambda(v)\]
$$\lim_{x\to\infty} f(x)$$	
$$\iiiint_V \mu(t,u,v,w) \,dt\,du\,dv\,dw$$
$$\sum_{n=1}^{\infty} 2^{-n} = 1$$	
\begin{definition}
	Si $X$ et $Y$ sont 2 v.a. ou definit la \textsc{Covariance} entre $X$ et $Y$ comme
	$\cov(X,Y)\overset{\text{def}}{=}\E\left[(X-\E(X))(Y-\E(Y))\right]=\E(XY)-\E(X)\E(Y)$.
\end{definition}
\fi
\pagebreak

% \tableofcontents

% insert your code here
%\input{./algebra/main.tex}
%\input{./geometrie-differentielle/main.tex}
%\input{./probabilite/main.tex}
%\input{./analyse-fonctionnelle/main.tex}
% \input{./Analyse-convexe-et-dualite-en-optimisation/main.tex}
%\input{./tikz/main.tex}
%\input{./Theorie-du-distributions/main.tex}
%\input{./optimisation/mine.tex}
 \input{./modelisation/main.tex}

% yves.aubry@univ-tln.fr : algebra

\end{document}

% % !TEX encoding = UTF-8 Unicode
% !TEX TS-program = xelatex

\documentclass[french]{report}

%\usepackage[utf8]{inputenc}
%\usepackage[T1]{fontenc}
\usepackage{babel}


\newif\ifcomment
%\commenttrue # Show comments

\usepackage{physics}
\usepackage{amssymb}


\usepackage{amsthm}
% \usepackage{thmtools}
\usepackage{mathtools}
\usepackage{amsfonts}

\usepackage{color}

\usepackage{tikz}

\usepackage{geometry}
\geometry{a5paper, margin=0.1in, right=1cm}

\usepackage{dsfont}

\usepackage{graphicx}
\graphicspath{ {images/} }

\usepackage{faktor}

\usepackage{IEEEtrantools}
\usepackage{enumerate}   
\usepackage[PostScript=dvips]{"/Users/aware/Documents/Courses/diagrams"}


\newtheorem{theorem}{Théorème}[section]
\renewcommand{\thetheorem}{\arabic{theorem}}
\newtheorem{lemme}{Lemme}[section]
\renewcommand{\thelemme}{\arabic{lemme}}
\newtheorem{proposition}{Proposition}[section]
\renewcommand{\theproposition}{\arabic{proposition}}
\newtheorem{notations}{Notations}[section]
\newtheorem{problem}{Problème}[section]
\newtheorem{corollary}{Corollaire}[theorem]
\renewcommand{\thecorollary}{\arabic{corollary}}
\newtheorem{property}{Propriété}[section]
\newtheorem{objective}{Objectif}[section]

\theoremstyle{definition}
\newtheorem{definition}{Définition}[section]
\renewcommand{\thedefinition}{\arabic{definition}}
\newtheorem{exercise}{Exercice}[chapter]
\renewcommand{\theexercise}{\arabic{exercise}}
\newtheorem{example}{Exemple}[chapter]
\renewcommand{\theexample}{\arabic{example}}
\newtheorem*{solution}{Solution}
\newtheorem*{application}{Application}
\newtheorem*{notation}{Notation}
\newtheorem*{vocabulary}{Vocabulaire}
\newtheorem*{properties}{Propriétés}



\theoremstyle{remark}
\newtheorem*{remark}{Remarque}
\newtheorem*{rappel}{Rappel}


\usepackage{etoolbox}
\AtBeginEnvironment{exercise}{\small}
\AtBeginEnvironment{example}{\small}

\usepackage{cases}
\usepackage[red]{mypack}

\usepackage[framemethod=TikZ]{mdframed}

\definecolor{bg}{rgb}{0.4,0.25,0.95}
\definecolor{pagebg}{rgb}{0,0,0.5}
\surroundwithmdframed[
   topline=false,
   rightline=false,
   bottomline=false,
   leftmargin=\parindent,
   skipabove=8pt,
   skipbelow=8pt,
   linecolor=blue,
   innerbottommargin=10pt,
   % backgroundcolor=bg,font=\color{orange}\sffamily, fontcolor=white
]{definition}

\usepackage{empheq}
\usepackage[most]{tcolorbox}

\newtcbox{\mymath}[1][]{%
    nobeforeafter, math upper, tcbox raise base,
    enhanced, colframe=blue!30!black,
    colback=red!10, boxrule=1pt,
    #1}

\usepackage{unixode}


\DeclareMathOperator{\ord}{ord}
\DeclareMathOperator{\orb}{orb}
\DeclareMathOperator{\stab}{stab}
\DeclareMathOperator{\Stab}{stab}
\DeclareMathOperator{\ppcm}{ppcm}
\DeclareMathOperator{\conj}{Conj}
\DeclareMathOperator{\End}{End}
\DeclareMathOperator{\rot}{rot}
\DeclareMathOperator{\trs}{trace}
\DeclareMathOperator{\Ind}{Ind}
\DeclareMathOperator{\mat}{Mat}
\DeclareMathOperator{\id}{Id}
\DeclareMathOperator{\vect}{vect}
\DeclareMathOperator{\img}{img}
\DeclareMathOperator{\cov}{Cov}
\DeclareMathOperator{\dist}{dist}
\DeclareMathOperator{\irr}{Irr}
\DeclareMathOperator{\image}{Im}
\DeclareMathOperator{\pd}{\partial}
\DeclareMathOperator{\epi}{epi}
\DeclareMathOperator{\Argmin}{Argmin}
\DeclareMathOperator{\dom}{dom}
\DeclareMathOperator{\proj}{proj}
\DeclareMathOperator{\ctg}{ctg}
\DeclareMathOperator{\supp}{supp}
\DeclareMathOperator{\argmin}{argmin}
\DeclareMathOperator{\mult}{mult}
\DeclareMathOperator{\ch}{ch}
\DeclareMathOperator{\sh}{sh}
\DeclareMathOperator{\rang}{rang}
\DeclareMathOperator{\diam}{diam}
\DeclareMathOperator{\Epigraphe}{Epigraphe}




\usepackage{xcolor}
\everymath{\color{blue}}
%\everymath{\color[rgb]{0,1,1}}
%\pagecolor[rgb]{0,0,0.5}


\newcommand*{\pdtest}[3][]{\ensuremath{\frac{\partial^{#1} #2}{\partial #3}}}

\newcommand*{\deffunc}[6][]{\ensuremath{
\begin{array}{rcl}
#2 : #3 &\rightarrow& #4\\
#5 &\mapsto& #6
\end{array}
}}

\newcommand{\eqcolon}{\mathrel{\resizebox{\widthof{$\mathord{=}$}}{\height}{ $\!\!=\!\!\resizebox{1.2\width}{0.8\height}{\raisebox{0.23ex}{$\mathop{:}$}}\!\!$ }}}
\newcommand{\coloneq}{\mathrel{\resizebox{\widthof{$\mathord{=}$}}{\height}{ $\!\!\resizebox{1.2\width}{0.8\height}{\raisebox{0.23ex}{$\mathop{:}$}}\!\!=\!\!$ }}}
\newcommand{\eqcolonl}{\ensuremath{\mathrel{=\!\!\mathop{:}}}}
\newcommand{\coloneql}{\ensuremath{\mathrel{\mathop{:} \!\! =}}}
\newcommand{\vc}[1]{% inline column vector
  \left(\begin{smallmatrix}#1\end{smallmatrix}\right)%
}
\newcommand{\vr}[1]{% inline row vector
  \begin{smallmatrix}(\,#1\,)\end{smallmatrix}%
}
\makeatletter
\newcommand*{\defeq}{\ =\mathrel{\rlap{%
                     \raisebox{0.3ex}{$\m@th\cdot$}}%
                     \raisebox{-0.3ex}{$\m@th\cdot$}}%
                     }
\makeatother

\newcommand{\mathcircle}[1]{% inline row vector
 \overset{\circ}{#1}
}
\newcommand{\ulim}{% low limit
 \underline{\lim}
}
\newcommand{\ssi}{% iff
\iff
}
\newcommand{\ps}[2]{
\expval{#1 | #2}
}
\newcommand{\df}[1]{
\mqty{#1}
}
\newcommand{\n}[1]{
\norm{#1}
}
\newcommand{\sys}[1]{
\left\{\smqty{#1}\right.
}


\newcommand{\eqdef}{\ensuremath{\overset{\text{def}}=}}


\def\Circlearrowright{\ensuremath{%
  \rotatebox[origin=c]{230}{$\circlearrowright$}}}

\newcommand\ct[1]{\text{\rmfamily\upshape #1}}
\newcommand\question[1]{ {\color{red} ...!? \small #1}}
\newcommand\caz[1]{\left\{\begin{array} #1 \end{array}\right.}
\newcommand\const{\text{\rmfamily\upshape const}}
\newcommand\toP{ \overset{\pro}{\to}}
\newcommand\toPP{ \overset{\text{PP}}{\to}}
\newcommand{\oeq}{\mathrel{\text{\textcircled{$=$}}}}





\usepackage{xcolor}
% \usepackage[normalem]{ulem}
\usepackage{lipsum}
\makeatletter
% \newcommand\colorwave[1][blue]{\bgroup \markoverwith{\lower3.5\p@\hbox{\sixly \textcolor{#1}{\char58}}}\ULon}
%\font\sixly=lasy6 % does not re-load if already loaded, so no memory problem.

\newmdtheoremenv[
linewidth= 1pt,linecolor= blue,%
leftmargin=20,rightmargin=20,innertopmargin=0pt, innerrightmargin=40,%
tikzsetting = { draw=lightgray, line width = 0.3pt,dashed,%
dash pattern = on 15pt off 3pt},%
splittopskip=\topskip,skipbelow=\baselineskip,%
skipabove=\baselineskip,ntheorem,roundcorner=0pt,
% backgroundcolor=pagebg,font=\color{orange}\sffamily, fontcolor=white
]{examplebox}{Exemple}[section]



\newcommand\R{\mathbb{R}}
\newcommand\Z{\mathbb{Z}}
\newcommand\N{\mathbb{N}}
\newcommand\E{\mathbb{E}}
\newcommand\F{\mathcal{F}}
\newcommand\cH{\mathcal{H}}
\newcommand\V{\mathbb{V}}
\newcommand\dmo{ ^{-1} }
\newcommand\kapa{\kappa}
\newcommand\im{Im}
\newcommand\hs{\mathcal{H}}





\usepackage{soul}

\makeatletter
\newcommand*{\whiten}[1]{\llap{\textcolor{white}{{\the\SOUL@token}}\hspace{#1pt}}}
\DeclareRobustCommand*\myul{%
    \def\SOUL@everyspace{\underline{\space}\kern\z@}%
    \def\SOUL@everytoken{%
     \setbox0=\hbox{\the\SOUL@token}%
     \ifdim\dp0>\z@
        \raisebox{\dp0}{\underline{\phantom{\the\SOUL@token}}}%
        \whiten{1}\whiten{0}%
        \whiten{-1}\whiten{-2}%
        \llap{\the\SOUL@token}%
     \else
        \underline{\the\SOUL@token}%
     \fi}%
\SOUL@}
\makeatother

\newcommand*{\demp}{\fontfamily{lmtt}\selectfont}

\DeclareTextFontCommand{\textdemp}{\demp}

\begin{document}

\ifcomment
Multiline
comment
\fi
\ifcomment
\myul{Typesetting test}
% \color[rgb]{1,1,1}
$∑_i^n≠ 60º±∞π∆¬≈√j∫h≤≥µ$

$\CR \R\pro\ind\pro\gS\pro
\mqty[a&b\\c&d]$
$\pro\mathbb{P}$
$\dd{x}$

  \[
    \alpha(x)=\left\{
                \begin{array}{ll}
                  x\\
                  \frac{1}{1+e^{-kx}}\\
                  \frac{e^x-e^{-x}}{e^x+e^{-x}}
                \end{array}
              \right.
  \]

  $\expval{x}$
  
  $\chi_\rho(ghg\dmo)=\Tr(\rho_{ghg\dmo})=\Tr(\rho_g\circ\rho_h\circ\rho\dmo_g)=\Tr(\rho_h)\overset{\mbox{\scalebox{0.5}{$\Tr(AB)=\Tr(BA)$}}}{=}\chi_\rho(h)$
  	$\mathop{\oplus}_{\substack{x\in X}}$

$\mat(\rho_g)=(a_{ij}(g))_{\scriptsize \substack{1\leq i\leq d \\ 1\leq j\leq d}}$ et $\mat(\rho'_g)=(a'_{ij}(g))_{\scriptsize \substack{1\leq i'\leq d' \\ 1\leq j'\leq d'}}$



\[\int_a^b{\mathbb{R}^2}g(u, v)\dd{P_{XY}}(u, v)=\iint g(u,v) f_{XY}(u, v)\dd \lambda(u) \dd \lambda(v)\]
$$\lim_{x\to\infty} f(x)$$	
$$\iiiint_V \mu(t,u,v,w) \,dt\,du\,dv\,dw$$
$$\sum_{n=1}^{\infty} 2^{-n} = 1$$	
\begin{definition}
	Si $X$ et $Y$ sont 2 v.a. ou definit la \textsc{Covariance} entre $X$ et $Y$ comme
	$\cov(X,Y)\overset{\text{def}}{=}\E\left[(X-\E(X))(Y-\E(Y))\right]=\E(XY)-\E(X)\E(Y)$.
\end{definition}
\fi
\pagebreak

% \tableofcontents

% insert your code here
%\input{./algebra/main.tex}
%\input{./geometrie-differentielle/main.tex}
%\input{./probabilite/main.tex}
%\input{./analyse-fonctionnelle/main.tex}
% \input{./Analyse-convexe-et-dualite-en-optimisation/main.tex}
%\input{./tikz/main.tex}
%\input{./Theorie-du-distributions/main.tex}
%\input{./optimisation/mine.tex}
 \input{./modelisation/main.tex}

% yves.aubry@univ-tln.fr : algebra

\end{document}

%% !TEX encoding = UTF-8 Unicode
% !TEX TS-program = xelatex

\documentclass[french]{report}

%\usepackage[utf8]{inputenc}
%\usepackage[T1]{fontenc}
\usepackage{babel}


\newif\ifcomment
%\commenttrue # Show comments

\usepackage{physics}
\usepackage{amssymb}


\usepackage{amsthm}
% \usepackage{thmtools}
\usepackage{mathtools}
\usepackage{amsfonts}

\usepackage{color}

\usepackage{tikz}

\usepackage{geometry}
\geometry{a5paper, margin=0.1in, right=1cm}

\usepackage{dsfont}

\usepackage{graphicx}
\graphicspath{ {images/} }

\usepackage{faktor}

\usepackage{IEEEtrantools}
\usepackage{enumerate}   
\usepackage[PostScript=dvips]{"/Users/aware/Documents/Courses/diagrams"}


\newtheorem{theorem}{Théorème}[section]
\renewcommand{\thetheorem}{\arabic{theorem}}
\newtheorem{lemme}{Lemme}[section]
\renewcommand{\thelemme}{\arabic{lemme}}
\newtheorem{proposition}{Proposition}[section]
\renewcommand{\theproposition}{\arabic{proposition}}
\newtheorem{notations}{Notations}[section]
\newtheorem{problem}{Problème}[section]
\newtheorem{corollary}{Corollaire}[theorem]
\renewcommand{\thecorollary}{\arabic{corollary}}
\newtheorem{property}{Propriété}[section]
\newtheorem{objective}{Objectif}[section]

\theoremstyle{definition}
\newtheorem{definition}{Définition}[section]
\renewcommand{\thedefinition}{\arabic{definition}}
\newtheorem{exercise}{Exercice}[chapter]
\renewcommand{\theexercise}{\arabic{exercise}}
\newtheorem{example}{Exemple}[chapter]
\renewcommand{\theexample}{\arabic{example}}
\newtheorem*{solution}{Solution}
\newtheorem*{application}{Application}
\newtheorem*{notation}{Notation}
\newtheorem*{vocabulary}{Vocabulaire}
\newtheorem*{properties}{Propriétés}



\theoremstyle{remark}
\newtheorem*{remark}{Remarque}
\newtheorem*{rappel}{Rappel}


\usepackage{etoolbox}
\AtBeginEnvironment{exercise}{\small}
\AtBeginEnvironment{example}{\small}

\usepackage{cases}
\usepackage[red]{mypack}

\usepackage[framemethod=TikZ]{mdframed}

\definecolor{bg}{rgb}{0.4,0.25,0.95}
\definecolor{pagebg}{rgb}{0,0,0.5}
\surroundwithmdframed[
   topline=false,
   rightline=false,
   bottomline=false,
   leftmargin=\parindent,
   skipabove=8pt,
   skipbelow=8pt,
   linecolor=blue,
   innerbottommargin=10pt,
   % backgroundcolor=bg,font=\color{orange}\sffamily, fontcolor=white
]{definition}

\usepackage{empheq}
\usepackage[most]{tcolorbox}

\newtcbox{\mymath}[1][]{%
    nobeforeafter, math upper, tcbox raise base,
    enhanced, colframe=blue!30!black,
    colback=red!10, boxrule=1pt,
    #1}

\usepackage{unixode}


\DeclareMathOperator{\ord}{ord}
\DeclareMathOperator{\orb}{orb}
\DeclareMathOperator{\stab}{stab}
\DeclareMathOperator{\Stab}{stab}
\DeclareMathOperator{\ppcm}{ppcm}
\DeclareMathOperator{\conj}{Conj}
\DeclareMathOperator{\End}{End}
\DeclareMathOperator{\rot}{rot}
\DeclareMathOperator{\trs}{trace}
\DeclareMathOperator{\Ind}{Ind}
\DeclareMathOperator{\mat}{Mat}
\DeclareMathOperator{\id}{Id}
\DeclareMathOperator{\vect}{vect}
\DeclareMathOperator{\img}{img}
\DeclareMathOperator{\cov}{Cov}
\DeclareMathOperator{\dist}{dist}
\DeclareMathOperator{\irr}{Irr}
\DeclareMathOperator{\image}{Im}
\DeclareMathOperator{\pd}{\partial}
\DeclareMathOperator{\epi}{epi}
\DeclareMathOperator{\Argmin}{Argmin}
\DeclareMathOperator{\dom}{dom}
\DeclareMathOperator{\proj}{proj}
\DeclareMathOperator{\ctg}{ctg}
\DeclareMathOperator{\supp}{supp}
\DeclareMathOperator{\argmin}{argmin}
\DeclareMathOperator{\mult}{mult}
\DeclareMathOperator{\ch}{ch}
\DeclareMathOperator{\sh}{sh}
\DeclareMathOperator{\rang}{rang}
\DeclareMathOperator{\diam}{diam}
\DeclareMathOperator{\Epigraphe}{Epigraphe}




\usepackage{xcolor}
\everymath{\color{blue}}
%\everymath{\color[rgb]{0,1,1}}
%\pagecolor[rgb]{0,0,0.5}


\newcommand*{\pdtest}[3][]{\ensuremath{\frac{\partial^{#1} #2}{\partial #3}}}

\newcommand*{\deffunc}[6][]{\ensuremath{
\begin{array}{rcl}
#2 : #3 &\rightarrow& #4\\
#5 &\mapsto& #6
\end{array}
}}

\newcommand{\eqcolon}{\mathrel{\resizebox{\widthof{$\mathord{=}$}}{\height}{ $\!\!=\!\!\resizebox{1.2\width}{0.8\height}{\raisebox{0.23ex}{$\mathop{:}$}}\!\!$ }}}
\newcommand{\coloneq}{\mathrel{\resizebox{\widthof{$\mathord{=}$}}{\height}{ $\!\!\resizebox{1.2\width}{0.8\height}{\raisebox{0.23ex}{$\mathop{:}$}}\!\!=\!\!$ }}}
\newcommand{\eqcolonl}{\ensuremath{\mathrel{=\!\!\mathop{:}}}}
\newcommand{\coloneql}{\ensuremath{\mathrel{\mathop{:} \!\! =}}}
\newcommand{\vc}[1]{% inline column vector
  \left(\begin{smallmatrix}#1\end{smallmatrix}\right)%
}
\newcommand{\vr}[1]{% inline row vector
  \begin{smallmatrix}(\,#1\,)\end{smallmatrix}%
}
\makeatletter
\newcommand*{\defeq}{\ =\mathrel{\rlap{%
                     \raisebox{0.3ex}{$\m@th\cdot$}}%
                     \raisebox{-0.3ex}{$\m@th\cdot$}}%
                     }
\makeatother

\newcommand{\mathcircle}[1]{% inline row vector
 \overset{\circ}{#1}
}
\newcommand{\ulim}{% low limit
 \underline{\lim}
}
\newcommand{\ssi}{% iff
\iff
}
\newcommand{\ps}[2]{
\expval{#1 | #2}
}
\newcommand{\df}[1]{
\mqty{#1}
}
\newcommand{\n}[1]{
\norm{#1}
}
\newcommand{\sys}[1]{
\left\{\smqty{#1}\right.
}


\newcommand{\eqdef}{\ensuremath{\overset{\text{def}}=}}


\def\Circlearrowright{\ensuremath{%
  \rotatebox[origin=c]{230}{$\circlearrowright$}}}

\newcommand\ct[1]{\text{\rmfamily\upshape #1}}
\newcommand\question[1]{ {\color{red} ...!? \small #1}}
\newcommand\caz[1]{\left\{\begin{array} #1 \end{array}\right.}
\newcommand\const{\text{\rmfamily\upshape const}}
\newcommand\toP{ \overset{\pro}{\to}}
\newcommand\toPP{ \overset{\text{PP}}{\to}}
\newcommand{\oeq}{\mathrel{\text{\textcircled{$=$}}}}





\usepackage{xcolor}
% \usepackage[normalem]{ulem}
\usepackage{lipsum}
\makeatletter
% \newcommand\colorwave[1][blue]{\bgroup \markoverwith{\lower3.5\p@\hbox{\sixly \textcolor{#1}{\char58}}}\ULon}
%\font\sixly=lasy6 % does not re-load if already loaded, so no memory problem.

\newmdtheoremenv[
linewidth= 1pt,linecolor= blue,%
leftmargin=20,rightmargin=20,innertopmargin=0pt, innerrightmargin=40,%
tikzsetting = { draw=lightgray, line width = 0.3pt,dashed,%
dash pattern = on 15pt off 3pt},%
splittopskip=\topskip,skipbelow=\baselineskip,%
skipabove=\baselineskip,ntheorem,roundcorner=0pt,
% backgroundcolor=pagebg,font=\color{orange}\sffamily, fontcolor=white
]{examplebox}{Exemple}[section]



\newcommand\R{\mathbb{R}}
\newcommand\Z{\mathbb{Z}}
\newcommand\N{\mathbb{N}}
\newcommand\E{\mathbb{E}}
\newcommand\F{\mathcal{F}}
\newcommand\cH{\mathcal{H}}
\newcommand\V{\mathbb{V}}
\newcommand\dmo{ ^{-1} }
\newcommand\kapa{\kappa}
\newcommand\im{Im}
\newcommand\hs{\mathcal{H}}





\usepackage{soul}

\makeatletter
\newcommand*{\whiten}[1]{\llap{\textcolor{white}{{\the\SOUL@token}}\hspace{#1pt}}}
\DeclareRobustCommand*\myul{%
    \def\SOUL@everyspace{\underline{\space}\kern\z@}%
    \def\SOUL@everytoken{%
     \setbox0=\hbox{\the\SOUL@token}%
     \ifdim\dp0>\z@
        \raisebox{\dp0}{\underline{\phantom{\the\SOUL@token}}}%
        \whiten{1}\whiten{0}%
        \whiten{-1}\whiten{-2}%
        \llap{\the\SOUL@token}%
     \else
        \underline{\the\SOUL@token}%
     \fi}%
\SOUL@}
\makeatother

\newcommand*{\demp}{\fontfamily{lmtt}\selectfont}

\DeclareTextFontCommand{\textdemp}{\demp}

\begin{document}

\ifcomment
Multiline
comment
\fi
\ifcomment
\myul{Typesetting test}
% \color[rgb]{1,1,1}
$∑_i^n≠ 60º±∞π∆¬≈√j∫h≤≥µ$

$\CR \R\pro\ind\pro\gS\pro
\mqty[a&b\\c&d]$
$\pro\mathbb{P}$
$\dd{x}$

  \[
    \alpha(x)=\left\{
                \begin{array}{ll}
                  x\\
                  \frac{1}{1+e^{-kx}}\\
                  \frac{e^x-e^{-x}}{e^x+e^{-x}}
                \end{array}
              \right.
  \]

  $\expval{x}$
  
  $\chi_\rho(ghg\dmo)=\Tr(\rho_{ghg\dmo})=\Tr(\rho_g\circ\rho_h\circ\rho\dmo_g)=\Tr(\rho_h)\overset{\mbox{\scalebox{0.5}{$\Tr(AB)=\Tr(BA)$}}}{=}\chi_\rho(h)$
  	$\mathop{\oplus}_{\substack{x\in X}}$

$\mat(\rho_g)=(a_{ij}(g))_{\scriptsize \substack{1\leq i\leq d \\ 1\leq j\leq d}}$ et $\mat(\rho'_g)=(a'_{ij}(g))_{\scriptsize \substack{1\leq i'\leq d' \\ 1\leq j'\leq d'}}$



\[\int_a^b{\mathbb{R}^2}g(u, v)\dd{P_{XY}}(u, v)=\iint g(u,v) f_{XY}(u, v)\dd \lambda(u) \dd \lambda(v)\]
$$\lim_{x\to\infty} f(x)$$	
$$\iiiint_V \mu(t,u,v,w) \,dt\,du\,dv\,dw$$
$$\sum_{n=1}^{\infty} 2^{-n} = 1$$	
\begin{definition}
	Si $X$ et $Y$ sont 2 v.a. ou definit la \textsc{Covariance} entre $X$ et $Y$ comme
	$\cov(X,Y)\overset{\text{def}}{=}\E\left[(X-\E(X))(Y-\E(Y))\right]=\E(XY)-\E(X)\E(Y)$.
\end{definition}
\fi
\pagebreak

% \tableofcontents

% insert your code here
%\input{./algebra/main.tex}
%\input{./geometrie-differentielle/main.tex}
%\input{./probabilite/main.tex}
%\input{./analyse-fonctionnelle/main.tex}
% \input{./Analyse-convexe-et-dualite-en-optimisation/main.tex}
%\input{./tikz/main.tex}
%\input{./Theorie-du-distributions/main.tex}
%\input{./optimisation/mine.tex}
 \input{./modelisation/main.tex}

% yves.aubry@univ-tln.fr : algebra

\end{document}

%% !TEX encoding = UTF-8 Unicode
% !TEX TS-program = xelatex

\documentclass[french]{report}

%\usepackage[utf8]{inputenc}
%\usepackage[T1]{fontenc}
\usepackage{babel}


\newif\ifcomment
%\commenttrue # Show comments

\usepackage{physics}
\usepackage{amssymb}


\usepackage{amsthm}
% \usepackage{thmtools}
\usepackage{mathtools}
\usepackage{amsfonts}

\usepackage{color}

\usepackage{tikz}

\usepackage{geometry}
\geometry{a5paper, margin=0.1in, right=1cm}

\usepackage{dsfont}

\usepackage{graphicx}
\graphicspath{ {images/} }

\usepackage{faktor}

\usepackage{IEEEtrantools}
\usepackage{enumerate}   
\usepackage[PostScript=dvips]{"/Users/aware/Documents/Courses/diagrams"}


\newtheorem{theorem}{Théorème}[section]
\renewcommand{\thetheorem}{\arabic{theorem}}
\newtheorem{lemme}{Lemme}[section]
\renewcommand{\thelemme}{\arabic{lemme}}
\newtheorem{proposition}{Proposition}[section]
\renewcommand{\theproposition}{\arabic{proposition}}
\newtheorem{notations}{Notations}[section]
\newtheorem{problem}{Problème}[section]
\newtheorem{corollary}{Corollaire}[theorem]
\renewcommand{\thecorollary}{\arabic{corollary}}
\newtheorem{property}{Propriété}[section]
\newtheorem{objective}{Objectif}[section]

\theoremstyle{definition}
\newtheorem{definition}{Définition}[section]
\renewcommand{\thedefinition}{\arabic{definition}}
\newtheorem{exercise}{Exercice}[chapter]
\renewcommand{\theexercise}{\arabic{exercise}}
\newtheorem{example}{Exemple}[chapter]
\renewcommand{\theexample}{\arabic{example}}
\newtheorem*{solution}{Solution}
\newtheorem*{application}{Application}
\newtheorem*{notation}{Notation}
\newtheorem*{vocabulary}{Vocabulaire}
\newtheorem*{properties}{Propriétés}



\theoremstyle{remark}
\newtheorem*{remark}{Remarque}
\newtheorem*{rappel}{Rappel}


\usepackage{etoolbox}
\AtBeginEnvironment{exercise}{\small}
\AtBeginEnvironment{example}{\small}

\usepackage{cases}
\usepackage[red]{mypack}

\usepackage[framemethod=TikZ]{mdframed}

\definecolor{bg}{rgb}{0.4,0.25,0.95}
\definecolor{pagebg}{rgb}{0,0,0.5}
\surroundwithmdframed[
   topline=false,
   rightline=false,
   bottomline=false,
   leftmargin=\parindent,
   skipabove=8pt,
   skipbelow=8pt,
   linecolor=blue,
   innerbottommargin=10pt,
   % backgroundcolor=bg,font=\color{orange}\sffamily, fontcolor=white
]{definition}

\usepackage{empheq}
\usepackage[most]{tcolorbox}

\newtcbox{\mymath}[1][]{%
    nobeforeafter, math upper, tcbox raise base,
    enhanced, colframe=blue!30!black,
    colback=red!10, boxrule=1pt,
    #1}

\usepackage{unixode}


\DeclareMathOperator{\ord}{ord}
\DeclareMathOperator{\orb}{orb}
\DeclareMathOperator{\stab}{stab}
\DeclareMathOperator{\Stab}{stab}
\DeclareMathOperator{\ppcm}{ppcm}
\DeclareMathOperator{\conj}{Conj}
\DeclareMathOperator{\End}{End}
\DeclareMathOperator{\rot}{rot}
\DeclareMathOperator{\trs}{trace}
\DeclareMathOperator{\Ind}{Ind}
\DeclareMathOperator{\mat}{Mat}
\DeclareMathOperator{\id}{Id}
\DeclareMathOperator{\vect}{vect}
\DeclareMathOperator{\img}{img}
\DeclareMathOperator{\cov}{Cov}
\DeclareMathOperator{\dist}{dist}
\DeclareMathOperator{\irr}{Irr}
\DeclareMathOperator{\image}{Im}
\DeclareMathOperator{\pd}{\partial}
\DeclareMathOperator{\epi}{epi}
\DeclareMathOperator{\Argmin}{Argmin}
\DeclareMathOperator{\dom}{dom}
\DeclareMathOperator{\proj}{proj}
\DeclareMathOperator{\ctg}{ctg}
\DeclareMathOperator{\supp}{supp}
\DeclareMathOperator{\argmin}{argmin}
\DeclareMathOperator{\mult}{mult}
\DeclareMathOperator{\ch}{ch}
\DeclareMathOperator{\sh}{sh}
\DeclareMathOperator{\rang}{rang}
\DeclareMathOperator{\diam}{diam}
\DeclareMathOperator{\Epigraphe}{Epigraphe}




\usepackage{xcolor}
\everymath{\color{blue}}
%\everymath{\color[rgb]{0,1,1}}
%\pagecolor[rgb]{0,0,0.5}


\newcommand*{\pdtest}[3][]{\ensuremath{\frac{\partial^{#1} #2}{\partial #3}}}

\newcommand*{\deffunc}[6][]{\ensuremath{
\begin{array}{rcl}
#2 : #3 &\rightarrow& #4\\
#5 &\mapsto& #6
\end{array}
}}

\newcommand{\eqcolon}{\mathrel{\resizebox{\widthof{$\mathord{=}$}}{\height}{ $\!\!=\!\!\resizebox{1.2\width}{0.8\height}{\raisebox{0.23ex}{$\mathop{:}$}}\!\!$ }}}
\newcommand{\coloneq}{\mathrel{\resizebox{\widthof{$\mathord{=}$}}{\height}{ $\!\!\resizebox{1.2\width}{0.8\height}{\raisebox{0.23ex}{$\mathop{:}$}}\!\!=\!\!$ }}}
\newcommand{\eqcolonl}{\ensuremath{\mathrel{=\!\!\mathop{:}}}}
\newcommand{\coloneql}{\ensuremath{\mathrel{\mathop{:} \!\! =}}}
\newcommand{\vc}[1]{% inline column vector
  \left(\begin{smallmatrix}#1\end{smallmatrix}\right)%
}
\newcommand{\vr}[1]{% inline row vector
  \begin{smallmatrix}(\,#1\,)\end{smallmatrix}%
}
\makeatletter
\newcommand*{\defeq}{\ =\mathrel{\rlap{%
                     \raisebox{0.3ex}{$\m@th\cdot$}}%
                     \raisebox{-0.3ex}{$\m@th\cdot$}}%
                     }
\makeatother

\newcommand{\mathcircle}[1]{% inline row vector
 \overset{\circ}{#1}
}
\newcommand{\ulim}{% low limit
 \underline{\lim}
}
\newcommand{\ssi}{% iff
\iff
}
\newcommand{\ps}[2]{
\expval{#1 | #2}
}
\newcommand{\df}[1]{
\mqty{#1}
}
\newcommand{\n}[1]{
\norm{#1}
}
\newcommand{\sys}[1]{
\left\{\smqty{#1}\right.
}


\newcommand{\eqdef}{\ensuremath{\overset{\text{def}}=}}


\def\Circlearrowright{\ensuremath{%
  \rotatebox[origin=c]{230}{$\circlearrowright$}}}

\newcommand\ct[1]{\text{\rmfamily\upshape #1}}
\newcommand\question[1]{ {\color{red} ...!? \small #1}}
\newcommand\caz[1]{\left\{\begin{array} #1 \end{array}\right.}
\newcommand\const{\text{\rmfamily\upshape const}}
\newcommand\toP{ \overset{\pro}{\to}}
\newcommand\toPP{ \overset{\text{PP}}{\to}}
\newcommand{\oeq}{\mathrel{\text{\textcircled{$=$}}}}





\usepackage{xcolor}
% \usepackage[normalem]{ulem}
\usepackage{lipsum}
\makeatletter
% \newcommand\colorwave[1][blue]{\bgroup \markoverwith{\lower3.5\p@\hbox{\sixly \textcolor{#1}{\char58}}}\ULon}
%\font\sixly=lasy6 % does not re-load if already loaded, so no memory problem.

\newmdtheoremenv[
linewidth= 1pt,linecolor= blue,%
leftmargin=20,rightmargin=20,innertopmargin=0pt, innerrightmargin=40,%
tikzsetting = { draw=lightgray, line width = 0.3pt,dashed,%
dash pattern = on 15pt off 3pt},%
splittopskip=\topskip,skipbelow=\baselineskip,%
skipabove=\baselineskip,ntheorem,roundcorner=0pt,
% backgroundcolor=pagebg,font=\color{orange}\sffamily, fontcolor=white
]{examplebox}{Exemple}[section]



\newcommand\R{\mathbb{R}}
\newcommand\Z{\mathbb{Z}}
\newcommand\N{\mathbb{N}}
\newcommand\E{\mathbb{E}}
\newcommand\F{\mathcal{F}}
\newcommand\cH{\mathcal{H}}
\newcommand\V{\mathbb{V}}
\newcommand\dmo{ ^{-1} }
\newcommand\kapa{\kappa}
\newcommand\im{Im}
\newcommand\hs{\mathcal{H}}





\usepackage{soul}

\makeatletter
\newcommand*{\whiten}[1]{\llap{\textcolor{white}{{\the\SOUL@token}}\hspace{#1pt}}}
\DeclareRobustCommand*\myul{%
    \def\SOUL@everyspace{\underline{\space}\kern\z@}%
    \def\SOUL@everytoken{%
     \setbox0=\hbox{\the\SOUL@token}%
     \ifdim\dp0>\z@
        \raisebox{\dp0}{\underline{\phantom{\the\SOUL@token}}}%
        \whiten{1}\whiten{0}%
        \whiten{-1}\whiten{-2}%
        \llap{\the\SOUL@token}%
     \else
        \underline{\the\SOUL@token}%
     \fi}%
\SOUL@}
\makeatother

\newcommand*{\demp}{\fontfamily{lmtt}\selectfont}

\DeclareTextFontCommand{\textdemp}{\demp}

\begin{document}

\ifcomment
Multiline
comment
\fi
\ifcomment
\myul{Typesetting test}
% \color[rgb]{1,1,1}
$∑_i^n≠ 60º±∞π∆¬≈√j∫h≤≥µ$

$\CR \R\pro\ind\pro\gS\pro
\mqty[a&b\\c&d]$
$\pro\mathbb{P}$
$\dd{x}$

  \[
    \alpha(x)=\left\{
                \begin{array}{ll}
                  x\\
                  \frac{1}{1+e^{-kx}}\\
                  \frac{e^x-e^{-x}}{e^x+e^{-x}}
                \end{array}
              \right.
  \]

  $\expval{x}$
  
  $\chi_\rho(ghg\dmo)=\Tr(\rho_{ghg\dmo})=\Tr(\rho_g\circ\rho_h\circ\rho\dmo_g)=\Tr(\rho_h)\overset{\mbox{\scalebox{0.5}{$\Tr(AB)=\Tr(BA)$}}}{=}\chi_\rho(h)$
  	$\mathop{\oplus}_{\substack{x\in X}}$

$\mat(\rho_g)=(a_{ij}(g))_{\scriptsize \substack{1\leq i\leq d \\ 1\leq j\leq d}}$ et $\mat(\rho'_g)=(a'_{ij}(g))_{\scriptsize \substack{1\leq i'\leq d' \\ 1\leq j'\leq d'}}$



\[\int_a^b{\mathbb{R}^2}g(u, v)\dd{P_{XY}}(u, v)=\iint g(u,v) f_{XY}(u, v)\dd \lambda(u) \dd \lambda(v)\]
$$\lim_{x\to\infty} f(x)$$	
$$\iiiint_V \mu(t,u,v,w) \,dt\,du\,dv\,dw$$
$$\sum_{n=1}^{\infty} 2^{-n} = 1$$	
\begin{definition}
	Si $X$ et $Y$ sont 2 v.a. ou definit la \textsc{Covariance} entre $X$ et $Y$ comme
	$\cov(X,Y)\overset{\text{def}}{=}\E\left[(X-\E(X))(Y-\E(Y))\right]=\E(XY)-\E(X)\E(Y)$.
\end{definition}
\fi
\pagebreak

% \tableofcontents

% insert your code here
%\input{./algebra/main.tex}
%\input{./geometrie-differentielle/main.tex}
%\input{./probabilite/main.tex}
%\input{./analyse-fonctionnelle/main.tex}
% \input{./Analyse-convexe-et-dualite-en-optimisation/main.tex}
%\input{./tikz/main.tex}
%\input{./Theorie-du-distributions/main.tex}
%\input{./optimisation/mine.tex}
 \input{./modelisation/main.tex}

% yves.aubry@univ-tln.fr : algebra

\end{document}

%\input{./optimisation/mine.tex}
 % !TEX encoding = UTF-8 Unicode
% !TEX TS-program = xelatex

\documentclass[french]{report}

%\usepackage[utf8]{inputenc}
%\usepackage[T1]{fontenc}
\usepackage{babel}


\newif\ifcomment
%\commenttrue # Show comments

\usepackage{physics}
\usepackage{amssymb}


\usepackage{amsthm}
% \usepackage{thmtools}
\usepackage{mathtools}
\usepackage{amsfonts}

\usepackage{color}

\usepackage{tikz}

\usepackage{geometry}
\geometry{a5paper, margin=0.1in, right=1cm}

\usepackage{dsfont}

\usepackage{graphicx}
\graphicspath{ {images/} }

\usepackage{faktor}

\usepackage{IEEEtrantools}
\usepackage{enumerate}   
\usepackage[PostScript=dvips]{"/Users/aware/Documents/Courses/diagrams"}


\newtheorem{theorem}{Théorème}[section]
\renewcommand{\thetheorem}{\arabic{theorem}}
\newtheorem{lemme}{Lemme}[section]
\renewcommand{\thelemme}{\arabic{lemme}}
\newtheorem{proposition}{Proposition}[section]
\renewcommand{\theproposition}{\arabic{proposition}}
\newtheorem{notations}{Notations}[section]
\newtheorem{problem}{Problème}[section]
\newtheorem{corollary}{Corollaire}[theorem]
\renewcommand{\thecorollary}{\arabic{corollary}}
\newtheorem{property}{Propriété}[section]
\newtheorem{objective}{Objectif}[section]

\theoremstyle{definition}
\newtheorem{definition}{Définition}[section]
\renewcommand{\thedefinition}{\arabic{definition}}
\newtheorem{exercise}{Exercice}[chapter]
\renewcommand{\theexercise}{\arabic{exercise}}
\newtheorem{example}{Exemple}[chapter]
\renewcommand{\theexample}{\arabic{example}}
\newtheorem*{solution}{Solution}
\newtheorem*{application}{Application}
\newtheorem*{notation}{Notation}
\newtheorem*{vocabulary}{Vocabulaire}
\newtheorem*{properties}{Propriétés}



\theoremstyle{remark}
\newtheorem*{remark}{Remarque}
\newtheorem*{rappel}{Rappel}


\usepackage{etoolbox}
\AtBeginEnvironment{exercise}{\small}
\AtBeginEnvironment{example}{\small}

\usepackage{cases}
\usepackage[red]{mypack}

\usepackage[framemethod=TikZ]{mdframed}

\definecolor{bg}{rgb}{0.4,0.25,0.95}
\definecolor{pagebg}{rgb}{0,0,0.5}
\surroundwithmdframed[
   topline=false,
   rightline=false,
   bottomline=false,
   leftmargin=\parindent,
   skipabove=8pt,
   skipbelow=8pt,
   linecolor=blue,
   innerbottommargin=10pt,
   % backgroundcolor=bg,font=\color{orange}\sffamily, fontcolor=white
]{definition}

\usepackage{empheq}
\usepackage[most]{tcolorbox}

\newtcbox{\mymath}[1][]{%
    nobeforeafter, math upper, tcbox raise base,
    enhanced, colframe=blue!30!black,
    colback=red!10, boxrule=1pt,
    #1}

\usepackage{unixode}


\DeclareMathOperator{\ord}{ord}
\DeclareMathOperator{\orb}{orb}
\DeclareMathOperator{\stab}{stab}
\DeclareMathOperator{\Stab}{stab}
\DeclareMathOperator{\ppcm}{ppcm}
\DeclareMathOperator{\conj}{Conj}
\DeclareMathOperator{\End}{End}
\DeclareMathOperator{\rot}{rot}
\DeclareMathOperator{\trs}{trace}
\DeclareMathOperator{\Ind}{Ind}
\DeclareMathOperator{\mat}{Mat}
\DeclareMathOperator{\id}{Id}
\DeclareMathOperator{\vect}{vect}
\DeclareMathOperator{\img}{img}
\DeclareMathOperator{\cov}{Cov}
\DeclareMathOperator{\dist}{dist}
\DeclareMathOperator{\irr}{Irr}
\DeclareMathOperator{\image}{Im}
\DeclareMathOperator{\pd}{\partial}
\DeclareMathOperator{\epi}{epi}
\DeclareMathOperator{\Argmin}{Argmin}
\DeclareMathOperator{\dom}{dom}
\DeclareMathOperator{\proj}{proj}
\DeclareMathOperator{\ctg}{ctg}
\DeclareMathOperator{\supp}{supp}
\DeclareMathOperator{\argmin}{argmin}
\DeclareMathOperator{\mult}{mult}
\DeclareMathOperator{\ch}{ch}
\DeclareMathOperator{\sh}{sh}
\DeclareMathOperator{\rang}{rang}
\DeclareMathOperator{\diam}{diam}
\DeclareMathOperator{\Epigraphe}{Epigraphe}




\usepackage{xcolor}
\everymath{\color{blue}}
%\everymath{\color[rgb]{0,1,1}}
%\pagecolor[rgb]{0,0,0.5}


\newcommand*{\pdtest}[3][]{\ensuremath{\frac{\partial^{#1} #2}{\partial #3}}}

\newcommand*{\deffunc}[6][]{\ensuremath{
\begin{array}{rcl}
#2 : #3 &\rightarrow& #4\\
#5 &\mapsto& #6
\end{array}
}}

\newcommand{\eqcolon}{\mathrel{\resizebox{\widthof{$\mathord{=}$}}{\height}{ $\!\!=\!\!\resizebox{1.2\width}{0.8\height}{\raisebox{0.23ex}{$\mathop{:}$}}\!\!$ }}}
\newcommand{\coloneq}{\mathrel{\resizebox{\widthof{$\mathord{=}$}}{\height}{ $\!\!\resizebox{1.2\width}{0.8\height}{\raisebox{0.23ex}{$\mathop{:}$}}\!\!=\!\!$ }}}
\newcommand{\eqcolonl}{\ensuremath{\mathrel{=\!\!\mathop{:}}}}
\newcommand{\coloneql}{\ensuremath{\mathrel{\mathop{:} \!\! =}}}
\newcommand{\vc}[1]{% inline column vector
  \left(\begin{smallmatrix}#1\end{smallmatrix}\right)%
}
\newcommand{\vr}[1]{% inline row vector
  \begin{smallmatrix}(\,#1\,)\end{smallmatrix}%
}
\makeatletter
\newcommand*{\defeq}{\ =\mathrel{\rlap{%
                     \raisebox{0.3ex}{$\m@th\cdot$}}%
                     \raisebox{-0.3ex}{$\m@th\cdot$}}%
                     }
\makeatother

\newcommand{\mathcircle}[1]{% inline row vector
 \overset{\circ}{#1}
}
\newcommand{\ulim}{% low limit
 \underline{\lim}
}
\newcommand{\ssi}{% iff
\iff
}
\newcommand{\ps}[2]{
\expval{#1 | #2}
}
\newcommand{\df}[1]{
\mqty{#1}
}
\newcommand{\n}[1]{
\norm{#1}
}
\newcommand{\sys}[1]{
\left\{\smqty{#1}\right.
}


\newcommand{\eqdef}{\ensuremath{\overset{\text{def}}=}}


\def\Circlearrowright{\ensuremath{%
  \rotatebox[origin=c]{230}{$\circlearrowright$}}}

\newcommand\ct[1]{\text{\rmfamily\upshape #1}}
\newcommand\question[1]{ {\color{red} ...!? \small #1}}
\newcommand\caz[1]{\left\{\begin{array} #1 \end{array}\right.}
\newcommand\const{\text{\rmfamily\upshape const}}
\newcommand\toP{ \overset{\pro}{\to}}
\newcommand\toPP{ \overset{\text{PP}}{\to}}
\newcommand{\oeq}{\mathrel{\text{\textcircled{$=$}}}}





\usepackage{xcolor}
% \usepackage[normalem]{ulem}
\usepackage{lipsum}
\makeatletter
% \newcommand\colorwave[1][blue]{\bgroup \markoverwith{\lower3.5\p@\hbox{\sixly \textcolor{#1}{\char58}}}\ULon}
%\font\sixly=lasy6 % does not re-load if already loaded, so no memory problem.

\newmdtheoremenv[
linewidth= 1pt,linecolor= blue,%
leftmargin=20,rightmargin=20,innertopmargin=0pt, innerrightmargin=40,%
tikzsetting = { draw=lightgray, line width = 0.3pt,dashed,%
dash pattern = on 15pt off 3pt},%
splittopskip=\topskip,skipbelow=\baselineskip,%
skipabove=\baselineskip,ntheorem,roundcorner=0pt,
% backgroundcolor=pagebg,font=\color{orange}\sffamily, fontcolor=white
]{examplebox}{Exemple}[section]



\newcommand\R{\mathbb{R}}
\newcommand\Z{\mathbb{Z}}
\newcommand\N{\mathbb{N}}
\newcommand\E{\mathbb{E}}
\newcommand\F{\mathcal{F}}
\newcommand\cH{\mathcal{H}}
\newcommand\V{\mathbb{V}}
\newcommand\dmo{ ^{-1} }
\newcommand\kapa{\kappa}
\newcommand\im{Im}
\newcommand\hs{\mathcal{H}}





\usepackage{soul}

\makeatletter
\newcommand*{\whiten}[1]{\llap{\textcolor{white}{{\the\SOUL@token}}\hspace{#1pt}}}
\DeclareRobustCommand*\myul{%
    \def\SOUL@everyspace{\underline{\space}\kern\z@}%
    \def\SOUL@everytoken{%
     \setbox0=\hbox{\the\SOUL@token}%
     \ifdim\dp0>\z@
        \raisebox{\dp0}{\underline{\phantom{\the\SOUL@token}}}%
        \whiten{1}\whiten{0}%
        \whiten{-1}\whiten{-2}%
        \llap{\the\SOUL@token}%
     \else
        \underline{\the\SOUL@token}%
     \fi}%
\SOUL@}
\makeatother

\newcommand*{\demp}{\fontfamily{lmtt}\selectfont}

\DeclareTextFontCommand{\textdemp}{\demp}

\begin{document}

\ifcomment
Multiline
comment
\fi
\ifcomment
\myul{Typesetting test}
% \color[rgb]{1,1,1}
$∑_i^n≠ 60º±∞π∆¬≈√j∫h≤≥µ$

$\CR \R\pro\ind\pro\gS\pro
\mqty[a&b\\c&d]$
$\pro\mathbb{P}$
$\dd{x}$

  \[
    \alpha(x)=\left\{
                \begin{array}{ll}
                  x\\
                  \frac{1}{1+e^{-kx}}\\
                  \frac{e^x-e^{-x}}{e^x+e^{-x}}
                \end{array}
              \right.
  \]

  $\expval{x}$
  
  $\chi_\rho(ghg\dmo)=\Tr(\rho_{ghg\dmo})=\Tr(\rho_g\circ\rho_h\circ\rho\dmo_g)=\Tr(\rho_h)\overset{\mbox{\scalebox{0.5}{$\Tr(AB)=\Tr(BA)$}}}{=}\chi_\rho(h)$
  	$\mathop{\oplus}_{\substack{x\in X}}$

$\mat(\rho_g)=(a_{ij}(g))_{\scriptsize \substack{1\leq i\leq d \\ 1\leq j\leq d}}$ et $\mat(\rho'_g)=(a'_{ij}(g))_{\scriptsize \substack{1\leq i'\leq d' \\ 1\leq j'\leq d'}}$



\[\int_a^b{\mathbb{R}^2}g(u, v)\dd{P_{XY}}(u, v)=\iint g(u,v) f_{XY}(u, v)\dd \lambda(u) \dd \lambda(v)\]
$$\lim_{x\to\infty} f(x)$$	
$$\iiiint_V \mu(t,u,v,w) \,dt\,du\,dv\,dw$$
$$\sum_{n=1}^{\infty} 2^{-n} = 1$$	
\begin{definition}
	Si $X$ et $Y$ sont 2 v.a. ou definit la \textsc{Covariance} entre $X$ et $Y$ comme
	$\cov(X,Y)\overset{\text{def}}{=}\E\left[(X-\E(X))(Y-\E(Y))\right]=\E(XY)-\E(X)\E(Y)$.
\end{definition}
\fi
\pagebreak

% \tableofcontents

% insert your code here
%\input{./algebra/main.tex}
%\input{./geometrie-differentielle/main.tex}
%\input{./probabilite/main.tex}
%\input{./analyse-fonctionnelle/main.tex}
% \input{./Analyse-convexe-et-dualite-en-optimisation/main.tex}
%\input{./tikz/main.tex}
%\input{./Theorie-du-distributions/main.tex}
%\input{./optimisation/mine.tex}
 \input{./modelisation/main.tex}

% yves.aubry@univ-tln.fr : algebra

\end{document}


% yves.aubry@univ-tln.fr : algebra

\end{document}

%\input{./optimisation/mine.tex}
 % !TEX encoding = UTF-8 Unicode
% !TEX TS-program = xelatex

\documentclass[french]{report}

%\usepackage[utf8]{inputenc}
%\usepackage[T1]{fontenc}
\usepackage{babel}


\newif\ifcomment
%\commenttrue # Show comments

\usepackage{physics}
\usepackage{amssymb}


\usepackage{amsthm}
% \usepackage{thmtools}
\usepackage{mathtools}
\usepackage{amsfonts}

\usepackage{color}

\usepackage{tikz}

\usepackage{geometry}
\geometry{a5paper, margin=0.1in, right=1cm}

\usepackage{dsfont}

\usepackage{graphicx}
\graphicspath{ {images/} }

\usepackage{faktor}

\usepackage{IEEEtrantools}
\usepackage{enumerate}   
\usepackage[PostScript=dvips]{"/Users/aware/Documents/Courses/diagrams"}


\newtheorem{theorem}{Théorème}[section]
\renewcommand{\thetheorem}{\arabic{theorem}}
\newtheorem{lemme}{Lemme}[section]
\renewcommand{\thelemme}{\arabic{lemme}}
\newtheorem{proposition}{Proposition}[section]
\renewcommand{\theproposition}{\arabic{proposition}}
\newtheorem{notations}{Notations}[section]
\newtheorem{problem}{Problème}[section]
\newtheorem{corollary}{Corollaire}[theorem]
\renewcommand{\thecorollary}{\arabic{corollary}}
\newtheorem{property}{Propriété}[section]
\newtheorem{objective}{Objectif}[section]

\theoremstyle{definition}
\newtheorem{definition}{Définition}[section]
\renewcommand{\thedefinition}{\arabic{definition}}
\newtheorem{exercise}{Exercice}[chapter]
\renewcommand{\theexercise}{\arabic{exercise}}
\newtheorem{example}{Exemple}[chapter]
\renewcommand{\theexample}{\arabic{example}}
\newtheorem*{solution}{Solution}
\newtheorem*{application}{Application}
\newtheorem*{notation}{Notation}
\newtheorem*{vocabulary}{Vocabulaire}
\newtheorem*{properties}{Propriétés}



\theoremstyle{remark}
\newtheorem*{remark}{Remarque}
\newtheorem*{rappel}{Rappel}


\usepackage{etoolbox}
\AtBeginEnvironment{exercise}{\small}
\AtBeginEnvironment{example}{\small}

\usepackage{cases}
\usepackage[red]{mypack}

\usepackage[framemethod=TikZ]{mdframed}

\definecolor{bg}{rgb}{0.4,0.25,0.95}
\definecolor{pagebg}{rgb}{0,0,0.5}
\surroundwithmdframed[
   topline=false,
   rightline=false,
   bottomline=false,
   leftmargin=\parindent,
   skipabove=8pt,
   skipbelow=8pt,
   linecolor=blue,
   innerbottommargin=10pt,
   % backgroundcolor=bg,font=\color{orange}\sffamily, fontcolor=white
]{definition}

\usepackage{empheq}
\usepackage[most]{tcolorbox}

\newtcbox{\mymath}[1][]{%
    nobeforeafter, math upper, tcbox raise base,
    enhanced, colframe=blue!30!black,
    colback=red!10, boxrule=1pt,
    #1}

\usepackage{unixode}


\DeclareMathOperator{\ord}{ord}
\DeclareMathOperator{\orb}{orb}
\DeclareMathOperator{\stab}{stab}
\DeclareMathOperator{\Stab}{stab}
\DeclareMathOperator{\ppcm}{ppcm}
\DeclareMathOperator{\conj}{Conj}
\DeclareMathOperator{\End}{End}
\DeclareMathOperator{\rot}{rot}
\DeclareMathOperator{\trs}{trace}
\DeclareMathOperator{\Ind}{Ind}
\DeclareMathOperator{\mat}{Mat}
\DeclareMathOperator{\id}{Id}
\DeclareMathOperator{\vect}{vect}
\DeclareMathOperator{\img}{img}
\DeclareMathOperator{\cov}{Cov}
\DeclareMathOperator{\dist}{dist}
\DeclareMathOperator{\irr}{Irr}
\DeclareMathOperator{\image}{Im}
\DeclareMathOperator{\pd}{\partial}
\DeclareMathOperator{\epi}{epi}
\DeclareMathOperator{\Argmin}{Argmin}
\DeclareMathOperator{\dom}{dom}
\DeclareMathOperator{\proj}{proj}
\DeclareMathOperator{\ctg}{ctg}
\DeclareMathOperator{\supp}{supp}
\DeclareMathOperator{\argmin}{argmin}
\DeclareMathOperator{\mult}{mult}
\DeclareMathOperator{\ch}{ch}
\DeclareMathOperator{\sh}{sh}
\DeclareMathOperator{\rang}{rang}
\DeclareMathOperator{\diam}{diam}
\DeclareMathOperator{\Epigraphe}{Epigraphe}




\usepackage{xcolor}
\everymath{\color{blue}}
%\everymath{\color[rgb]{0,1,1}}
%\pagecolor[rgb]{0,0,0.5}


\newcommand*{\pdtest}[3][]{\ensuremath{\frac{\partial^{#1} #2}{\partial #3}}}

\newcommand*{\deffunc}[6][]{\ensuremath{
\begin{array}{rcl}
#2 : #3 &\rightarrow& #4\\
#5 &\mapsto& #6
\end{array}
}}

\newcommand{\eqcolon}{\mathrel{\resizebox{\widthof{$\mathord{=}$}}{\height}{ $\!\!=\!\!\resizebox{1.2\width}{0.8\height}{\raisebox{0.23ex}{$\mathop{:}$}}\!\!$ }}}
\newcommand{\coloneq}{\mathrel{\resizebox{\widthof{$\mathord{=}$}}{\height}{ $\!\!\resizebox{1.2\width}{0.8\height}{\raisebox{0.23ex}{$\mathop{:}$}}\!\!=\!\!$ }}}
\newcommand{\eqcolonl}{\ensuremath{\mathrel{=\!\!\mathop{:}}}}
\newcommand{\coloneql}{\ensuremath{\mathrel{\mathop{:} \!\! =}}}
\newcommand{\vc}[1]{% inline column vector
  \left(\begin{smallmatrix}#1\end{smallmatrix}\right)%
}
\newcommand{\vr}[1]{% inline row vector
  \begin{smallmatrix}(\,#1\,)\end{smallmatrix}%
}
\makeatletter
\newcommand*{\defeq}{\ =\mathrel{\rlap{%
                     \raisebox{0.3ex}{$\m@th\cdot$}}%
                     \raisebox{-0.3ex}{$\m@th\cdot$}}%
                     }
\makeatother

\newcommand{\mathcircle}[1]{% inline row vector
 \overset{\circ}{#1}
}
\newcommand{\ulim}{% low limit
 \underline{\lim}
}
\newcommand{\ssi}{% iff
\iff
}
\newcommand{\ps}[2]{
\expval{#1 | #2}
}
\newcommand{\df}[1]{
\mqty{#1}
}
\newcommand{\n}[1]{
\norm{#1}
}
\newcommand{\sys}[1]{
\left\{\smqty{#1}\right.
}


\newcommand{\eqdef}{\ensuremath{\overset{\text{def}}=}}


\def\Circlearrowright{\ensuremath{%
  \rotatebox[origin=c]{230}{$\circlearrowright$}}}

\newcommand\ct[1]{\text{\rmfamily\upshape #1}}
\newcommand\question[1]{ {\color{red} ...!? \small #1}}
\newcommand\caz[1]{\left\{\begin{array} #1 \end{array}\right.}
\newcommand\const{\text{\rmfamily\upshape const}}
\newcommand\toP{ \overset{\pro}{\to}}
\newcommand\toPP{ \overset{\text{PP}}{\to}}
\newcommand{\oeq}{\mathrel{\text{\textcircled{$=$}}}}





\usepackage{xcolor}
% \usepackage[normalem]{ulem}
\usepackage{lipsum}
\makeatletter
% \newcommand\colorwave[1][blue]{\bgroup \markoverwith{\lower3.5\p@\hbox{\sixly \textcolor{#1}{\char58}}}\ULon}
%\font\sixly=lasy6 % does not re-load if already loaded, so no memory problem.

\newmdtheoremenv[
linewidth= 1pt,linecolor= blue,%
leftmargin=20,rightmargin=20,innertopmargin=0pt, innerrightmargin=40,%
tikzsetting = { draw=lightgray, line width = 0.3pt,dashed,%
dash pattern = on 15pt off 3pt},%
splittopskip=\topskip,skipbelow=\baselineskip,%
skipabove=\baselineskip,ntheorem,roundcorner=0pt,
% backgroundcolor=pagebg,font=\color{orange}\sffamily, fontcolor=white
]{examplebox}{Exemple}[section]



\newcommand\R{\mathbb{R}}
\newcommand\Z{\mathbb{Z}}
\newcommand\N{\mathbb{N}}
\newcommand\E{\mathbb{E}}
\newcommand\F{\mathcal{F}}
\newcommand\cH{\mathcal{H}}
\newcommand\V{\mathbb{V}}
\newcommand\dmo{ ^{-1} }
\newcommand\kapa{\kappa}
\newcommand\im{Im}
\newcommand\hs{\mathcal{H}}





\usepackage{soul}

\makeatletter
\newcommand*{\whiten}[1]{\llap{\textcolor{white}{{\the\SOUL@token}}\hspace{#1pt}}}
\DeclareRobustCommand*\myul{%
    \def\SOUL@everyspace{\underline{\space}\kern\z@}%
    \def\SOUL@everytoken{%
     \setbox0=\hbox{\the\SOUL@token}%
     \ifdim\dp0>\z@
        \raisebox{\dp0}{\underline{\phantom{\the\SOUL@token}}}%
        \whiten{1}\whiten{0}%
        \whiten{-1}\whiten{-2}%
        \llap{\the\SOUL@token}%
     \else
        \underline{\the\SOUL@token}%
     \fi}%
\SOUL@}
\makeatother

\newcommand*{\demp}{\fontfamily{lmtt}\selectfont}

\DeclareTextFontCommand{\textdemp}{\demp}

\begin{document}

\ifcomment
Multiline
comment
\fi
\ifcomment
\myul{Typesetting test}
% \color[rgb]{1,1,1}
$∑_i^n≠ 60º±∞π∆¬≈√j∫h≤≥µ$

$\CR \R\pro\ind\pro\gS\pro
\mqty[a&b\\c&d]$
$\pro\mathbb{P}$
$\dd{x}$

  \[
    \alpha(x)=\left\{
                \begin{array}{ll}
                  x\\
                  \frac{1}{1+e^{-kx}}\\
                  \frac{e^x-e^{-x}}{e^x+e^{-x}}
                \end{array}
              \right.
  \]

  $\expval{x}$
  
  $\chi_\rho(ghg\dmo)=\Tr(\rho_{ghg\dmo})=\Tr(\rho_g\circ\rho_h\circ\rho\dmo_g)=\Tr(\rho_h)\overset{\mbox{\scalebox{0.5}{$\Tr(AB)=\Tr(BA)$}}}{=}\chi_\rho(h)$
  	$\mathop{\oplus}_{\substack{x\in X}}$

$\mat(\rho_g)=(a_{ij}(g))_{\scriptsize \substack{1\leq i\leq d \\ 1\leq j\leq d}}$ et $\mat(\rho'_g)=(a'_{ij}(g))_{\scriptsize \substack{1\leq i'\leq d' \\ 1\leq j'\leq d'}}$



\[\int_a^b{\mathbb{R}^2}g(u, v)\dd{P_{XY}}(u, v)=\iint g(u,v) f_{XY}(u, v)\dd \lambda(u) \dd \lambda(v)\]
$$\lim_{x\to\infty} f(x)$$	
$$\iiiint_V \mu(t,u,v,w) \,dt\,du\,dv\,dw$$
$$\sum_{n=1}^{\infty} 2^{-n} = 1$$	
\begin{definition}
	Si $X$ et $Y$ sont 2 v.a. ou definit la \textsc{Covariance} entre $X$ et $Y$ comme
	$\cov(X,Y)\overset{\text{def}}{=}\E\left[(X-\E(X))(Y-\E(Y))\right]=\E(XY)-\E(X)\E(Y)$.
\end{definition}
\fi
\pagebreak

% \tableofcontents

% insert your code here
%% !TEX encoding = UTF-8 Unicode
% !TEX TS-program = xelatex

\documentclass[french]{report}

%\usepackage[utf8]{inputenc}
%\usepackage[T1]{fontenc}
\usepackage{babel}


\newif\ifcomment
%\commenttrue # Show comments

\usepackage{physics}
\usepackage{amssymb}


\usepackage{amsthm}
% \usepackage{thmtools}
\usepackage{mathtools}
\usepackage{amsfonts}

\usepackage{color}

\usepackage{tikz}

\usepackage{geometry}
\geometry{a5paper, margin=0.1in, right=1cm}

\usepackage{dsfont}

\usepackage{graphicx}
\graphicspath{ {images/} }

\usepackage{faktor}

\usepackage{IEEEtrantools}
\usepackage{enumerate}   
\usepackage[PostScript=dvips]{"/Users/aware/Documents/Courses/diagrams"}


\newtheorem{theorem}{Théorème}[section]
\renewcommand{\thetheorem}{\arabic{theorem}}
\newtheorem{lemme}{Lemme}[section]
\renewcommand{\thelemme}{\arabic{lemme}}
\newtheorem{proposition}{Proposition}[section]
\renewcommand{\theproposition}{\arabic{proposition}}
\newtheorem{notations}{Notations}[section]
\newtheorem{problem}{Problème}[section]
\newtheorem{corollary}{Corollaire}[theorem]
\renewcommand{\thecorollary}{\arabic{corollary}}
\newtheorem{property}{Propriété}[section]
\newtheorem{objective}{Objectif}[section]

\theoremstyle{definition}
\newtheorem{definition}{Définition}[section]
\renewcommand{\thedefinition}{\arabic{definition}}
\newtheorem{exercise}{Exercice}[chapter]
\renewcommand{\theexercise}{\arabic{exercise}}
\newtheorem{example}{Exemple}[chapter]
\renewcommand{\theexample}{\arabic{example}}
\newtheorem*{solution}{Solution}
\newtheorem*{application}{Application}
\newtheorem*{notation}{Notation}
\newtheorem*{vocabulary}{Vocabulaire}
\newtheorem*{properties}{Propriétés}



\theoremstyle{remark}
\newtheorem*{remark}{Remarque}
\newtheorem*{rappel}{Rappel}


\usepackage{etoolbox}
\AtBeginEnvironment{exercise}{\small}
\AtBeginEnvironment{example}{\small}

\usepackage{cases}
\usepackage[red]{mypack}

\usepackage[framemethod=TikZ]{mdframed}

\definecolor{bg}{rgb}{0.4,0.25,0.95}
\definecolor{pagebg}{rgb}{0,0,0.5}
\surroundwithmdframed[
   topline=false,
   rightline=false,
   bottomline=false,
   leftmargin=\parindent,
   skipabove=8pt,
   skipbelow=8pt,
   linecolor=blue,
   innerbottommargin=10pt,
   % backgroundcolor=bg,font=\color{orange}\sffamily, fontcolor=white
]{definition}

\usepackage{empheq}
\usepackage[most]{tcolorbox}

\newtcbox{\mymath}[1][]{%
    nobeforeafter, math upper, tcbox raise base,
    enhanced, colframe=blue!30!black,
    colback=red!10, boxrule=1pt,
    #1}

\usepackage{unixode}


\DeclareMathOperator{\ord}{ord}
\DeclareMathOperator{\orb}{orb}
\DeclareMathOperator{\stab}{stab}
\DeclareMathOperator{\Stab}{stab}
\DeclareMathOperator{\ppcm}{ppcm}
\DeclareMathOperator{\conj}{Conj}
\DeclareMathOperator{\End}{End}
\DeclareMathOperator{\rot}{rot}
\DeclareMathOperator{\trs}{trace}
\DeclareMathOperator{\Ind}{Ind}
\DeclareMathOperator{\mat}{Mat}
\DeclareMathOperator{\id}{Id}
\DeclareMathOperator{\vect}{vect}
\DeclareMathOperator{\img}{img}
\DeclareMathOperator{\cov}{Cov}
\DeclareMathOperator{\dist}{dist}
\DeclareMathOperator{\irr}{Irr}
\DeclareMathOperator{\image}{Im}
\DeclareMathOperator{\pd}{\partial}
\DeclareMathOperator{\epi}{epi}
\DeclareMathOperator{\Argmin}{Argmin}
\DeclareMathOperator{\dom}{dom}
\DeclareMathOperator{\proj}{proj}
\DeclareMathOperator{\ctg}{ctg}
\DeclareMathOperator{\supp}{supp}
\DeclareMathOperator{\argmin}{argmin}
\DeclareMathOperator{\mult}{mult}
\DeclareMathOperator{\ch}{ch}
\DeclareMathOperator{\sh}{sh}
\DeclareMathOperator{\rang}{rang}
\DeclareMathOperator{\diam}{diam}
\DeclareMathOperator{\Epigraphe}{Epigraphe}




\usepackage{xcolor}
\everymath{\color{blue}}
%\everymath{\color[rgb]{0,1,1}}
%\pagecolor[rgb]{0,0,0.5}


\newcommand*{\pdtest}[3][]{\ensuremath{\frac{\partial^{#1} #2}{\partial #3}}}

\newcommand*{\deffunc}[6][]{\ensuremath{
\begin{array}{rcl}
#2 : #3 &\rightarrow& #4\\
#5 &\mapsto& #6
\end{array}
}}

\newcommand{\eqcolon}{\mathrel{\resizebox{\widthof{$\mathord{=}$}}{\height}{ $\!\!=\!\!\resizebox{1.2\width}{0.8\height}{\raisebox{0.23ex}{$\mathop{:}$}}\!\!$ }}}
\newcommand{\coloneq}{\mathrel{\resizebox{\widthof{$\mathord{=}$}}{\height}{ $\!\!\resizebox{1.2\width}{0.8\height}{\raisebox{0.23ex}{$\mathop{:}$}}\!\!=\!\!$ }}}
\newcommand{\eqcolonl}{\ensuremath{\mathrel{=\!\!\mathop{:}}}}
\newcommand{\coloneql}{\ensuremath{\mathrel{\mathop{:} \!\! =}}}
\newcommand{\vc}[1]{% inline column vector
  \left(\begin{smallmatrix}#1\end{smallmatrix}\right)%
}
\newcommand{\vr}[1]{% inline row vector
  \begin{smallmatrix}(\,#1\,)\end{smallmatrix}%
}
\makeatletter
\newcommand*{\defeq}{\ =\mathrel{\rlap{%
                     \raisebox{0.3ex}{$\m@th\cdot$}}%
                     \raisebox{-0.3ex}{$\m@th\cdot$}}%
                     }
\makeatother

\newcommand{\mathcircle}[1]{% inline row vector
 \overset{\circ}{#1}
}
\newcommand{\ulim}{% low limit
 \underline{\lim}
}
\newcommand{\ssi}{% iff
\iff
}
\newcommand{\ps}[2]{
\expval{#1 | #2}
}
\newcommand{\df}[1]{
\mqty{#1}
}
\newcommand{\n}[1]{
\norm{#1}
}
\newcommand{\sys}[1]{
\left\{\smqty{#1}\right.
}


\newcommand{\eqdef}{\ensuremath{\overset{\text{def}}=}}


\def\Circlearrowright{\ensuremath{%
  \rotatebox[origin=c]{230}{$\circlearrowright$}}}

\newcommand\ct[1]{\text{\rmfamily\upshape #1}}
\newcommand\question[1]{ {\color{red} ...!? \small #1}}
\newcommand\caz[1]{\left\{\begin{array} #1 \end{array}\right.}
\newcommand\const{\text{\rmfamily\upshape const}}
\newcommand\toP{ \overset{\pro}{\to}}
\newcommand\toPP{ \overset{\text{PP}}{\to}}
\newcommand{\oeq}{\mathrel{\text{\textcircled{$=$}}}}





\usepackage{xcolor}
% \usepackage[normalem]{ulem}
\usepackage{lipsum}
\makeatletter
% \newcommand\colorwave[1][blue]{\bgroup \markoverwith{\lower3.5\p@\hbox{\sixly \textcolor{#1}{\char58}}}\ULon}
%\font\sixly=lasy6 % does not re-load if already loaded, so no memory problem.

\newmdtheoremenv[
linewidth= 1pt,linecolor= blue,%
leftmargin=20,rightmargin=20,innertopmargin=0pt, innerrightmargin=40,%
tikzsetting = { draw=lightgray, line width = 0.3pt,dashed,%
dash pattern = on 15pt off 3pt},%
splittopskip=\topskip,skipbelow=\baselineskip,%
skipabove=\baselineskip,ntheorem,roundcorner=0pt,
% backgroundcolor=pagebg,font=\color{orange}\sffamily, fontcolor=white
]{examplebox}{Exemple}[section]



\newcommand\R{\mathbb{R}}
\newcommand\Z{\mathbb{Z}}
\newcommand\N{\mathbb{N}}
\newcommand\E{\mathbb{E}}
\newcommand\F{\mathcal{F}}
\newcommand\cH{\mathcal{H}}
\newcommand\V{\mathbb{V}}
\newcommand\dmo{ ^{-1} }
\newcommand\kapa{\kappa}
\newcommand\im{Im}
\newcommand\hs{\mathcal{H}}





\usepackage{soul}

\makeatletter
\newcommand*{\whiten}[1]{\llap{\textcolor{white}{{\the\SOUL@token}}\hspace{#1pt}}}
\DeclareRobustCommand*\myul{%
    \def\SOUL@everyspace{\underline{\space}\kern\z@}%
    \def\SOUL@everytoken{%
     \setbox0=\hbox{\the\SOUL@token}%
     \ifdim\dp0>\z@
        \raisebox{\dp0}{\underline{\phantom{\the\SOUL@token}}}%
        \whiten{1}\whiten{0}%
        \whiten{-1}\whiten{-2}%
        \llap{\the\SOUL@token}%
     \else
        \underline{\the\SOUL@token}%
     \fi}%
\SOUL@}
\makeatother

\newcommand*{\demp}{\fontfamily{lmtt}\selectfont}

\DeclareTextFontCommand{\textdemp}{\demp}

\begin{document}

\ifcomment
Multiline
comment
\fi
\ifcomment
\myul{Typesetting test}
% \color[rgb]{1,1,1}
$∑_i^n≠ 60º±∞π∆¬≈√j∫h≤≥µ$

$\CR \R\pro\ind\pro\gS\pro
\mqty[a&b\\c&d]$
$\pro\mathbb{P}$
$\dd{x}$

  \[
    \alpha(x)=\left\{
                \begin{array}{ll}
                  x\\
                  \frac{1}{1+e^{-kx}}\\
                  \frac{e^x-e^{-x}}{e^x+e^{-x}}
                \end{array}
              \right.
  \]

  $\expval{x}$
  
  $\chi_\rho(ghg\dmo)=\Tr(\rho_{ghg\dmo})=\Tr(\rho_g\circ\rho_h\circ\rho\dmo_g)=\Tr(\rho_h)\overset{\mbox{\scalebox{0.5}{$\Tr(AB)=\Tr(BA)$}}}{=}\chi_\rho(h)$
  	$\mathop{\oplus}_{\substack{x\in X}}$

$\mat(\rho_g)=(a_{ij}(g))_{\scriptsize \substack{1\leq i\leq d \\ 1\leq j\leq d}}$ et $\mat(\rho'_g)=(a'_{ij}(g))_{\scriptsize \substack{1\leq i'\leq d' \\ 1\leq j'\leq d'}}$



\[\int_a^b{\mathbb{R}^2}g(u, v)\dd{P_{XY}}(u, v)=\iint g(u,v) f_{XY}(u, v)\dd \lambda(u) \dd \lambda(v)\]
$$\lim_{x\to\infty} f(x)$$	
$$\iiiint_V \mu(t,u,v,w) \,dt\,du\,dv\,dw$$
$$\sum_{n=1}^{\infty} 2^{-n} = 1$$	
\begin{definition}
	Si $X$ et $Y$ sont 2 v.a. ou definit la \textsc{Covariance} entre $X$ et $Y$ comme
	$\cov(X,Y)\overset{\text{def}}{=}\E\left[(X-\E(X))(Y-\E(Y))\right]=\E(XY)-\E(X)\E(Y)$.
\end{definition}
\fi
\pagebreak

% \tableofcontents

% insert your code here
%\input{./algebra/main.tex}
%\input{./geometrie-differentielle/main.tex}
%\input{./probabilite/main.tex}
%\input{./analyse-fonctionnelle/main.tex}
% \input{./Analyse-convexe-et-dualite-en-optimisation/main.tex}
%\input{./tikz/main.tex}
%\input{./Theorie-du-distributions/main.tex}
%\input{./optimisation/mine.tex}
 \input{./modelisation/main.tex}

% yves.aubry@univ-tln.fr : algebra

\end{document}

%% !TEX encoding = UTF-8 Unicode
% !TEX TS-program = xelatex

\documentclass[french]{report}

%\usepackage[utf8]{inputenc}
%\usepackage[T1]{fontenc}
\usepackage{babel}


\newif\ifcomment
%\commenttrue # Show comments

\usepackage{physics}
\usepackage{amssymb}


\usepackage{amsthm}
% \usepackage{thmtools}
\usepackage{mathtools}
\usepackage{amsfonts}

\usepackage{color}

\usepackage{tikz}

\usepackage{geometry}
\geometry{a5paper, margin=0.1in, right=1cm}

\usepackage{dsfont}

\usepackage{graphicx}
\graphicspath{ {images/} }

\usepackage{faktor}

\usepackage{IEEEtrantools}
\usepackage{enumerate}   
\usepackage[PostScript=dvips]{"/Users/aware/Documents/Courses/diagrams"}


\newtheorem{theorem}{Théorème}[section]
\renewcommand{\thetheorem}{\arabic{theorem}}
\newtheorem{lemme}{Lemme}[section]
\renewcommand{\thelemme}{\arabic{lemme}}
\newtheorem{proposition}{Proposition}[section]
\renewcommand{\theproposition}{\arabic{proposition}}
\newtheorem{notations}{Notations}[section]
\newtheorem{problem}{Problème}[section]
\newtheorem{corollary}{Corollaire}[theorem]
\renewcommand{\thecorollary}{\arabic{corollary}}
\newtheorem{property}{Propriété}[section]
\newtheorem{objective}{Objectif}[section]

\theoremstyle{definition}
\newtheorem{definition}{Définition}[section]
\renewcommand{\thedefinition}{\arabic{definition}}
\newtheorem{exercise}{Exercice}[chapter]
\renewcommand{\theexercise}{\arabic{exercise}}
\newtheorem{example}{Exemple}[chapter]
\renewcommand{\theexample}{\arabic{example}}
\newtheorem*{solution}{Solution}
\newtheorem*{application}{Application}
\newtheorem*{notation}{Notation}
\newtheorem*{vocabulary}{Vocabulaire}
\newtheorem*{properties}{Propriétés}



\theoremstyle{remark}
\newtheorem*{remark}{Remarque}
\newtheorem*{rappel}{Rappel}


\usepackage{etoolbox}
\AtBeginEnvironment{exercise}{\small}
\AtBeginEnvironment{example}{\small}

\usepackage{cases}
\usepackage[red]{mypack}

\usepackage[framemethod=TikZ]{mdframed}

\definecolor{bg}{rgb}{0.4,0.25,0.95}
\definecolor{pagebg}{rgb}{0,0,0.5}
\surroundwithmdframed[
   topline=false,
   rightline=false,
   bottomline=false,
   leftmargin=\parindent,
   skipabove=8pt,
   skipbelow=8pt,
   linecolor=blue,
   innerbottommargin=10pt,
   % backgroundcolor=bg,font=\color{orange}\sffamily, fontcolor=white
]{definition}

\usepackage{empheq}
\usepackage[most]{tcolorbox}

\newtcbox{\mymath}[1][]{%
    nobeforeafter, math upper, tcbox raise base,
    enhanced, colframe=blue!30!black,
    colback=red!10, boxrule=1pt,
    #1}

\usepackage{unixode}


\DeclareMathOperator{\ord}{ord}
\DeclareMathOperator{\orb}{orb}
\DeclareMathOperator{\stab}{stab}
\DeclareMathOperator{\Stab}{stab}
\DeclareMathOperator{\ppcm}{ppcm}
\DeclareMathOperator{\conj}{Conj}
\DeclareMathOperator{\End}{End}
\DeclareMathOperator{\rot}{rot}
\DeclareMathOperator{\trs}{trace}
\DeclareMathOperator{\Ind}{Ind}
\DeclareMathOperator{\mat}{Mat}
\DeclareMathOperator{\id}{Id}
\DeclareMathOperator{\vect}{vect}
\DeclareMathOperator{\img}{img}
\DeclareMathOperator{\cov}{Cov}
\DeclareMathOperator{\dist}{dist}
\DeclareMathOperator{\irr}{Irr}
\DeclareMathOperator{\image}{Im}
\DeclareMathOperator{\pd}{\partial}
\DeclareMathOperator{\epi}{epi}
\DeclareMathOperator{\Argmin}{Argmin}
\DeclareMathOperator{\dom}{dom}
\DeclareMathOperator{\proj}{proj}
\DeclareMathOperator{\ctg}{ctg}
\DeclareMathOperator{\supp}{supp}
\DeclareMathOperator{\argmin}{argmin}
\DeclareMathOperator{\mult}{mult}
\DeclareMathOperator{\ch}{ch}
\DeclareMathOperator{\sh}{sh}
\DeclareMathOperator{\rang}{rang}
\DeclareMathOperator{\diam}{diam}
\DeclareMathOperator{\Epigraphe}{Epigraphe}




\usepackage{xcolor}
\everymath{\color{blue}}
%\everymath{\color[rgb]{0,1,1}}
%\pagecolor[rgb]{0,0,0.5}


\newcommand*{\pdtest}[3][]{\ensuremath{\frac{\partial^{#1} #2}{\partial #3}}}

\newcommand*{\deffunc}[6][]{\ensuremath{
\begin{array}{rcl}
#2 : #3 &\rightarrow& #4\\
#5 &\mapsto& #6
\end{array}
}}

\newcommand{\eqcolon}{\mathrel{\resizebox{\widthof{$\mathord{=}$}}{\height}{ $\!\!=\!\!\resizebox{1.2\width}{0.8\height}{\raisebox{0.23ex}{$\mathop{:}$}}\!\!$ }}}
\newcommand{\coloneq}{\mathrel{\resizebox{\widthof{$\mathord{=}$}}{\height}{ $\!\!\resizebox{1.2\width}{0.8\height}{\raisebox{0.23ex}{$\mathop{:}$}}\!\!=\!\!$ }}}
\newcommand{\eqcolonl}{\ensuremath{\mathrel{=\!\!\mathop{:}}}}
\newcommand{\coloneql}{\ensuremath{\mathrel{\mathop{:} \!\! =}}}
\newcommand{\vc}[1]{% inline column vector
  \left(\begin{smallmatrix}#1\end{smallmatrix}\right)%
}
\newcommand{\vr}[1]{% inline row vector
  \begin{smallmatrix}(\,#1\,)\end{smallmatrix}%
}
\makeatletter
\newcommand*{\defeq}{\ =\mathrel{\rlap{%
                     \raisebox{0.3ex}{$\m@th\cdot$}}%
                     \raisebox{-0.3ex}{$\m@th\cdot$}}%
                     }
\makeatother

\newcommand{\mathcircle}[1]{% inline row vector
 \overset{\circ}{#1}
}
\newcommand{\ulim}{% low limit
 \underline{\lim}
}
\newcommand{\ssi}{% iff
\iff
}
\newcommand{\ps}[2]{
\expval{#1 | #2}
}
\newcommand{\df}[1]{
\mqty{#1}
}
\newcommand{\n}[1]{
\norm{#1}
}
\newcommand{\sys}[1]{
\left\{\smqty{#1}\right.
}


\newcommand{\eqdef}{\ensuremath{\overset{\text{def}}=}}


\def\Circlearrowright{\ensuremath{%
  \rotatebox[origin=c]{230}{$\circlearrowright$}}}

\newcommand\ct[1]{\text{\rmfamily\upshape #1}}
\newcommand\question[1]{ {\color{red} ...!? \small #1}}
\newcommand\caz[1]{\left\{\begin{array} #1 \end{array}\right.}
\newcommand\const{\text{\rmfamily\upshape const}}
\newcommand\toP{ \overset{\pro}{\to}}
\newcommand\toPP{ \overset{\text{PP}}{\to}}
\newcommand{\oeq}{\mathrel{\text{\textcircled{$=$}}}}





\usepackage{xcolor}
% \usepackage[normalem]{ulem}
\usepackage{lipsum}
\makeatletter
% \newcommand\colorwave[1][blue]{\bgroup \markoverwith{\lower3.5\p@\hbox{\sixly \textcolor{#1}{\char58}}}\ULon}
%\font\sixly=lasy6 % does not re-load if already loaded, so no memory problem.

\newmdtheoremenv[
linewidth= 1pt,linecolor= blue,%
leftmargin=20,rightmargin=20,innertopmargin=0pt, innerrightmargin=40,%
tikzsetting = { draw=lightgray, line width = 0.3pt,dashed,%
dash pattern = on 15pt off 3pt},%
splittopskip=\topskip,skipbelow=\baselineskip,%
skipabove=\baselineskip,ntheorem,roundcorner=0pt,
% backgroundcolor=pagebg,font=\color{orange}\sffamily, fontcolor=white
]{examplebox}{Exemple}[section]



\newcommand\R{\mathbb{R}}
\newcommand\Z{\mathbb{Z}}
\newcommand\N{\mathbb{N}}
\newcommand\E{\mathbb{E}}
\newcommand\F{\mathcal{F}}
\newcommand\cH{\mathcal{H}}
\newcommand\V{\mathbb{V}}
\newcommand\dmo{ ^{-1} }
\newcommand\kapa{\kappa}
\newcommand\im{Im}
\newcommand\hs{\mathcal{H}}





\usepackage{soul}

\makeatletter
\newcommand*{\whiten}[1]{\llap{\textcolor{white}{{\the\SOUL@token}}\hspace{#1pt}}}
\DeclareRobustCommand*\myul{%
    \def\SOUL@everyspace{\underline{\space}\kern\z@}%
    \def\SOUL@everytoken{%
     \setbox0=\hbox{\the\SOUL@token}%
     \ifdim\dp0>\z@
        \raisebox{\dp0}{\underline{\phantom{\the\SOUL@token}}}%
        \whiten{1}\whiten{0}%
        \whiten{-1}\whiten{-2}%
        \llap{\the\SOUL@token}%
     \else
        \underline{\the\SOUL@token}%
     \fi}%
\SOUL@}
\makeatother

\newcommand*{\demp}{\fontfamily{lmtt}\selectfont}

\DeclareTextFontCommand{\textdemp}{\demp}

\begin{document}

\ifcomment
Multiline
comment
\fi
\ifcomment
\myul{Typesetting test}
% \color[rgb]{1,1,1}
$∑_i^n≠ 60º±∞π∆¬≈√j∫h≤≥µ$

$\CR \R\pro\ind\pro\gS\pro
\mqty[a&b\\c&d]$
$\pro\mathbb{P}$
$\dd{x}$

  \[
    \alpha(x)=\left\{
                \begin{array}{ll}
                  x\\
                  \frac{1}{1+e^{-kx}}\\
                  \frac{e^x-e^{-x}}{e^x+e^{-x}}
                \end{array}
              \right.
  \]

  $\expval{x}$
  
  $\chi_\rho(ghg\dmo)=\Tr(\rho_{ghg\dmo})=\Tr(\rho_g\circ\rho_h\circ\rho\dmo_g)=\Tr(\rho_h)\overset{\mbox{\scalebox{0.5}{$\Tr(AB)=\Tr(BA)$}}}{=}\chi_\rho(h)$
  	$\mathop{\oplus}_{\substack{x\in X}}$

$\mat(\rho_g)=(a_{ij}(g))_{\scriptsize \substack{1\leq i\leq d \\ 1\leq j\leq d}}$ et $\mat(\rho'_g)=(a'_{ij}(g))_{\scriptsize \substack{1\leq i'\leq d' \\ 1\leq j'\leq d'}}$



\[\int_a^b{\mathbb{R}^2}g(u, v)\dd{P_{XY}}(u, v)=\iint g(u,v) f_{XY}(u, v)\dd \lambda(u) \dd \lambda(v)\]
$$\lim_{x\to\infty} f(x)$$	
$$\iiiint_V \mu(t,u,v,w) \,dt\,du\,dv\,dw$$
$$\sum_{n=1}^{\infty} 2^{-n} = 1$$	
\begin{definition}
	Si $X$ et $Y$ sont 2 v.a. ou definit la \textsc{Covariance} entre $X$ et $Y$ comme
	$\cov(X,Y)\overset{\text{def}}{=}\E\left[(X-\E(X))(Y-\E(Y))\right]=\E(XY)-\E(X)\E(Y)$.
\end{definition}
\fi
\pagebreak

% \tableofcontents

% insert your code here
%\input{./algebra/main.tex}
%\input{./geometrie-differentielle/main.tex}
%\input{./probabilite/main.tex}
%\input{./analyse-fonctionnelle/main.tex}
% \input{./Analyse-convexe-et-dualite-en-optimisation/main.tex}
%\input{./tikz/main.tex}
%\input{./Theorie-du-distributions/main.tex}
%\input{./optimisation/mine.tex}
 \input{./modelisation/main.tex}

% yves.aubry@univ-tln.fr : algebra

\end{document}

%% !TEX encoding = UTF-8 Unicode
% !TEX TS-program = xelatex

\documentclass[french]{report}

%\usepackage[utf8]{inputenc}
%\usepackage[T1]{fontenc}
\usepackage{babel}


\newif\ifcomment
%\commenttrue # Show comments

\usepackage{physics}
\usepackage{amssymb}


\usepackage{amsthm}
% \usepackage{thmtools}
\usepackage{mathtools}
\usepackage{amsfonts}

\usepackage{color}

\usepackage{tikz}

\usepackage{geometry}
\geometry{a5paper, margin=0.1in, right=1cm}

\usepackage{dsfont}

\usepackage{graphicx}
\graphicspath{ {images/} }

\usepackage{faktor}

\usepackage{IEEEtrantools}
\usepackage{enumerate}   
\usepackage[PostScript=dvips]{"/Users/aware/Documents/Courses/diagrams"}


\newtheorem{theorem}{Théorème}[section]
\renewcommand{\thetheorem}{\arabic{theorem}}
\newtheorem{lemme}{Lemme}[section]
\renewcommand{\thelemme}{\arabic{lemme}}
\newtheorem{proposition}{Proposition}[section]
\renewcommand{\theproposition}{\arabic{proposition}}
\newtheorem{notations}{Notations}[section]
\newtheorem{problem}{Problème}[section]
\newtheorem{corollary}{Corollaire}[theorem]
\renewcommand{\thecorollary}{\arabic{corollary}}
\newtheorem{property}{Propriété}[section]
\newtheorem{objective}{Objectif}[section]

\theoremstyle{definition}
\newtheorem{definition}{Définition}[section]
\renewcommand{\thedefinition}{\arabic{definition}}
\newtheorem{exercise}{Exercice}[chapter]
\renewcommand{\theexercise}{\arabic{exercise}}
\newtheorem{example}{Exemple}[chapter]
\renewcommand{\theexample}{\arabic{example}}
\newtheorem*{solution}{Solution}
\newtheorem*{application}{Application}
\newtheorem*{notation}{Notation}
\newtheorem*{vocabulary}{Vocabulaire}
\newtheorem*{properties}{Propriétés}



\theoremstyle{remark}
\newtheorem*{remark}{Remarque}
\newtheorem*{rappel}{Rappel}


\usepackage{etoolbox}
\AtBeginEnvironment{exercise}{\small}
\AtBeginEnvironment{example}{\small}

\usepackage{cases}
\usepackage[red]{mypack}

\usepackage[framemethod=TikZ]{mdframed}

\definecolor{bg}{rgb}{0.4,0.25,0.95}
\definecolor{pagebg}{rgb}{0,0,0.5}
\surroundwithmdframed[
   topline=false,
   rightline=false,
   bottomline=false,
   leftmargin=\parindent,
   skipabove=8pt,
   skipbelow=8pt,
   linecolor=blue,
   innerbottommargin=10pt,
   % backgroundcolor=bg,font=\color{orange}\sffamily, fontcolor=white
]{definition}

\usepackage{empheq}
\usepackage[most]{tcolorbox}

\newtcbox{\mymath}[1][]{%
    nobeforeafter, math upper, tcbox raise base,
    enhanced, colframe=blue!30!black,
    colback=red!10, boxrule=1pt,
    #1}

\usepackage{unixode}


\DeclareMathOperator{\ord}{ord}
\DeclareMathOperator{\orb}{orb}
\DeclareMathOperator{\stab}{stab}
\DeclareMathOperator{\Stab}{stab}
\DeclareMathOperator{\ppcm}{ppcm}
\DeclareMathOperator{\conj}{Conj}
\DeclareMathOperator{\End}{End}
\DeclareMathOperator{\rot}{rot}
\DeclareMathOperator{\trs}{trace}
\DeclareMathOperator{\Ind}{Ind}
\DeclareMathOperator{\mat}{Mat}
\DeclareMathOperator{\id}{Id}
\DeclareMathOperator{\vect}{vect}
\DeclareMathOperator{\img}{img}
\DeclareMathOperator{\cov}{Cov}
\DeclareMathOperator{\dist}{dist}
\DeclareMathOperator{\irr}{Irr}
\DeclareMathOperator{\image}{Im}
\DeclareMathOperator{\pd}{\partial}
\DeclareMathOperator{\epi}{epi}
\DeclareMathOperator{\Argmin}{Argmin}
\DeclareMathOperator{\dom}{dom}
\DeclareMathOperator{\proj}{proj}
\DeclareMathOperator{\ctg}{ctg}
\DeclareMathOperator{\supp}{supp}
\DeclareMathOperator{\argmin}{argmin}
\DeclareMathOperator{\mult}{mult}
\DeclareMathOperator{\ch}{ch}
\DeclareMathOperator{\sh}{sh}
\DeclareMathOperator{\rang}{rang}
\DeclareMathOperator{\diam}{diam}
\DeclareMathOperator{\Epigraphe}{Epigraphe}




\usepackage{xcolor}
\everymath{\color{blue}}
%\everymath{\color[rgb]{0,1,1}}
%\pagecolor[rgb]{0,0,0.5}


\newcommand*{\pdtest}[3][]{\ensuremath{\frac{\partial^{#1} #2}{\partial #3}}}

\newcommand*{\deffunc}[6][]{\ensuremath{
\begin{array}{rcl}
#2 : #3 &\rightarrow& #4\\
#5 &\mapsto& #6
\end{array}
}}

\newcommand{\eqcolon}{\mathrel{\resizebox{\widthof{$\mathord{=}$}}{\height}{ $\!\!=\!\!\resizebox{1.2\width}{0.8\height}{\raisebox{0.23ex}{$\mathop{:}$}}\!\!$ }}}
\newcommand{\coloneq}{\mathrel{\resizebox{\widthof{$\mathord{=}$}}{\height}{ $\!\!\resizebox{1.2\width}{0.8\height}{\raisebox{0.23ex}{$\mathop{:}$}}\!\!=\!\!$ }}}
\newcommand{\eqcolonl}{\ensuremath{\mathrel{=\!\!\mathop{:}}}}
\newcommand{\coloneql}{\ensuremath{\mathrel{\mathop{:} \!\! =}}}
\newcommand{\vc}[1]{% inline column vector
  \left(\begin{smallmatrix}#1\end{smallmatrix}\right)%
}
\newcommand{\vr}[1]{% inline row vector
  \begin{smallmatrix}(\,#1\,)\end{smallmatrix}%
}
\makeatletter
\newcommand*{\defeq}{\ =\mathrel{\rlap{%
                     \raisebox{0.3ex}{$\m@th\cdot$}}%
                     \raisebox{-0.3ex}{$\m@th\cdot$}}%
                     }
\makeatother

\newcommand{\mathcircle}[1]{% inline row vector
 \overset{\circ}{#1}
}
\newcommand{\ulim}{% low limit
 \underline{\lim}
}
\newcommand{\ssi}{% iff
\iff
}
\newcommand{\ps}[2]{
\expval{#1 | #2}
}
\newcommand{\df}[1]{
\mqty{#1}
}
\newcommand{\n}[1]{
\norm{#1}
}
\newcommand{\sys}[1]{
\left\{\smqty{#1}\right.
}


\newcommand{\eqdef}{\ensuremath{\overset{\text{def}}=}}


\def\Circlearrowright{\ensuremath{%
  \rotatebox[origin=c]{230}{$\circlearrowright$}}}

\newcommand\ct[1]{\text{\rmfamily\upshape #1}}
\newcommand\question[1]{ {\color{red} ...!? \small #1}}
\newcommand\caz[1]{\left\{\begin{array} #1 \end{array}\right.}
\newcommand\const{\text{\rmfamily\upshape const}}
\newcommand\toP{ \overset{\pro}{\to}}
\newcommand\toPP{ \overset{\text{PP}}{\to}}
\newcommand{\oeq}{\mathrel{\text{\textcircled{$=$}}}}





\usepackage{xcolor}
% \usepackage[normalem]{ulem}
\usepackage{lipsum}
\makeatletter
% \newcommand\colorwave[1][blue]{\bgroup \markoverwith{\lower3.5\p@\hbox{\sixly \textcolor{#1}{\char58}}}\ULon}
%\font\sixly=lasy6 % does not re-load if already loaded, so no memory problem.

\newmdtheoremenv[
linewidth= 1pt,linecolor= blue,%
leftmargin=20,rightmargin=20,innertopmargin=0pt, innerrightmargin=40,%
tikzsetting = { draw=lightgray, line width = 0.3pt,dashed,%
dash pattern = on 15pt off 3pt},%
splittopskip=\topskip,skipbelow=\baselineskip,%
skipabove=\baselineskip,ntheorem,roundcorner=0pt,
% backgroundcolor=pagebg,font=\color{orange}\sffamily, fontcolor=white
]{examplebox}{Exemple}[section]



\newcommand\R{\mathbb{R}}
\newcommand\Z{\mathbb{Z}}
\newcommand\N{\mathbb{N}}
\newcommand\E{\mathbb{E}}
\newcommand\F{\mathcal{F}}
\newcommand\cH{\mathcal{H}}
\newcommand\V{\mathbb{V}}
\newcommand\dmo{ ^{-1} }
\newcommand\kapa{\kappa}
\newcommand\im{Im}
\newcommand\hs{\mathcal{H}}





\usepackage{soul}

\makeatletter
\newcommand*{\whiten}[1]{\llap{\textcolor{white}{{\the\SOUL@token}}\hspace{#1pt}}}
\DeclareRobustCommand*\myul{%
    \def\SOUL@everyspace{\underline{\space}\kern\z@}%
    \def\SOUL@everytoken{%
     \setbox0=\hbox{\the\SOUL@token}%
     \ifdim\dp0>\z@
        \raisebox{\dp0}{\underline{\phantom{\the\SOUL@token}}}%
        \whiten{1}\whiten{0}%
        \whiten{-1}\whiten{-2}%
        \llap{\the\SOUL@token}%
     \else
        \underline{\the\SOUL@token}%
     \fi}%
\SOUL@}
\makeatother

\newcommand*{\demp}{\fontfamily{lmtt}\selectfont}

\DeclareTextFontCommand{\textdemp}{\demp}

\begin{document}

\ifcomment
Multiline
comment
\fi
\ifcomment
\myul{Typesetting test}
% \color[rgb]{1,1,1}
$∑_i^n≠ 60º±∞π∆¬≈√j∫h≤≥µ$

$\CR \R\pro\ind\pro\gS\pro
\mqty[a&b\\c&d]$
$\pro\mathbb{P}$
$\dd{x}$

  \[
    \alpha(x)=\left\{
                \begin{array}{ll}
                  x\\
                  \frac{1}{1+e^{-kx}}\\
                  \frac{e^x-e^{-x}}{e^x+e^{-x}}
                \end{array}
              \right.
  \]

  $\expval{x}$
  
  $\chi_\rho(ghg\dmo)=\Tr(\rho_{ghg\dmo})=\Tr(\rho_g\circ\rho_h\circ\rho\dmo_g)=\Tr(\rho_h)\overset{\mbox{\scalebox{0.5}{$\Tr(AB)=\Tr(BA)$}}}{=}\chi_\rho(h)$
  	$\mathop{\oplus}_{\substack{x\in X}}$

$\mat(\rho_g)=(a_{ij}(g))_{\scriptsize \substack{1\leq i\leq d \\ 1\leq j\leq d}}$ et $\mat(\rho'_g)=(a'_{ij}(g))_{\scriptsize \substack{1\leq i'\leq d' \\ 1\leq j'\leq d'}}$



\[\int_a^b{\mathbb{R}^2}g(u, v)\dd{P_{XY}}(u, v)=\iint g(u,v) f_{XY}(u, v)\dd \lambda(u) \dd \lambda(v)\]
$$\lim_{x\to\infty} f(x)$$	
$$\iiiint_V \mu(t,u,v,w) \,dt\,du\,dv\,dw$$
$$\sum_{n=1}^{\infty} 2^{-n} = 1$$	
\begin{definition}
	Si $X$ et $Y$ sont 2 v.a. ou definit la \textsc{Covariance} entre $X$ et $Y$ comme
	$\cov(X,Y)\overset{\text{def}}{=}\E\left[(X-\E(X))(Y-\E(Y))\right]=\E(XY)-\E(X)\E(Y)$.
\end{definition}
\fi
\pagebreak

% \tableofcontents

% insert your code here
%\input{./algebra/main.tex}
%\input{./geometrie-differentielle/main.tex}
%\input{./probabilite/main.tex}
%\input{./analyse-fonctionnelle/main.tex}
% \input{./Analyse-convexe-et-dualite-en-optimisation/main.tex}
%\input{./tikz/main.tex}
%\input{./Theorie-du-distributions/main.tex}
%\input{./optimisation/mine.tex}
 \input{./modelisation/main.tex}

% yves.aubry@univ-tln.fr : algebra

\end{document}

%% !TEX encoding = UTF-8 Unicode
% !TEX TS-program = xelatex

\documentclass[french]{report}

%\usepackage[utf8]{inputenc}
%\usepackage[T1]{fontenc}
\usepackage{babel}


\newif\ifcomment
%\commenttrue # Show comments

\usepackage{physics}
\usepackage{amssymb}


\usepackage{amsthm}
% \usepackage{thmtools}
\usepackage{mathtools}
\usepackage{amsfonts}

\usepackage{color}

\usepackage{tikz}

\usepackage{geometry}
\geometry{a5paper, margin=0.1in, right=1cm}

\usepackage{dsfont}

\usepackage{graphicx}
\graphicspath{ {images/} }

\usepackage{faktor}

\usepackage{IEEEtrantools}
\usepackage{enumerate}   
\usepackage[PostScript=dvips]{"/Users/aware/Documents/Courses/diagrams"}


\newtheorem{theorem}{Théorème}[section]
\renewcommand{\thetheorem}{\arabic{theorem}}
\newtheorem{lemme}{Lemme}[section]
\renewcommand{\thelemme}{\arabic{lemme}}
\newtheorem{proposition}{Proposition}[section]
\renewcommand{\theproposition}{\arabic{proposition}}
\newtheorem{notations}{Notations}[section]
\newtheorem{problem}{Problème}[section]
\newtheorem{corollary}{Corollaire}[theorem]
\renewcommand{\thecorollary}{\arabic{corollary}}
\newtheorem{property}{Propriété}[section]
\newtheorem{objective}{Objectif}[section]

\theoremstyle{definition}
\newtheorem{definition}{Définition}[section]
\renewcommand{\thedefinition}{\arabic{definition}}
\newtheorem{exercise}{Exercice}[chapter]
\renewcommand{\theexercise}{\arabic{exercise}}
\newtheorem{example}{Exemple}[chapter]
\renewcommand{\theexample}{\arabic{example}}
\newtheorem*{solution}{Solution}
\newtheorem*{application}{Application}
\newtheorem*{notation}{Notation}
\newtheorem*{vocabulary}{Vocabulaire}
\newtheorem*{properties}{Propriétés}



\theoremstyle{remark}
\newtheorem*{remark}{Remarque}
\newtheorem*{rappel}{Rappel}


\usepackage{etoolbox}
\AtBeginEnvironment{exercise}{\small}
\AtBeginEnvironment{example}{\small}

\usepackage{cases}
\usepackage[red]{mypack}

\usepackage[framemethod=TikZ]{mdframed}

\definecolor{bg}{rgb}{0.4,0.25,0.95}
\definecolor{pagebg}{rgb}{0,0,0.5}
\surroundwithmdframed[
   topline=false,
   rightline=false,
   bottomline=false,
   leftmargin=\parindent,
   skipabove=8pt,
   skipbelow=8pt,
   linecolor=blue,
   innerbottommargin=10pt,
   % backgroundcolor=bg,font=\color{orange}\sffamily, fontcolor=white
]{definition}

\usepackage{empheq}
\usepackage[most]{tcolorbox}

\newtcbox{\mymath}[1][]{%
    nobeforeafter, math upper, tcbox raise base,
    enhanced, colframe=blue!30!black,
    colback=red!10, boxrule=1pt,
    #1}

\usepackage{unixode}


\DeclareMathOperator{\ord}{ord}
\DeclareMathOperator{\orb}{orb}
\DeclareMathOperator{\stab}{stab}
\DeclareMathOperator{\Stab}{stab}
\DeclareMathOperator{\ppcm}{ppcm}
\DeclareMathOperator{\conj}{Conj}
\DeclareMathOperator{\End}{End}
\DeclareMathOperator{\rot}{rot}
\DeclareMathOperator{\trs}{trace}
\DeclareMathOperator{\Ind}{Ind}
\DeclareMathOperator{\mat}{Mat}
\DeclareMathOperator{\id}{Id}
\DeclareMathOperator{\vect}{vect}
\DeclareMathOperator{\img}{img}
\DeclareMathOperator{\cov}{Cov}
\DeclareMathOperator{\dist}{dist}
\DeclareMathOperator{\irr}{Irr}
\DeclareMathOperator{\image}{Im}
\DeclareMathOperator{\pd}{\partial}
\DeclareMathOperator{\epi}{epi}
\DeclareMathOperator{\Argmin}{Argmin}
\DeclareMathOperator{\dom}{dom}
\DeclareMathOperator{\proj}{proj}
\DeclareMathOperator{\ctg}{ctg}
\DeclareMathOperator{\supp}{supp}
\DeclareMathOperator{\argmin}{argmin}
\DeclareMathOperator{\mult}{mult}
\DeclareMathOperator{\ch}{ch}
\DeclareMathOperator{\sh}{sh}
\DeclareMathOperator{\rang}{rang}
\DeclareMathOperator{\diam}{diam}
\DeclareMathOperator{\Epigraphe}{Epigraphe}




\usepackage{xcolor}
\everymath{\color{blue}}
%\everymath{\color[rgb]{0,1,1}}
%\pagecolor[rgb]{0,0,0.5}


\newcommand*{\pdtest}[3][]{\ensuremath{\frac{\partial^{#1} #2}{\partial #3}}}

\newcommand*{\deffunc}[6][]{\ensuremath{
\begin{array}{rcl}
#2 : #3 &\rightarrow& #4\\
#5 &\mapsto& #6
\end{array}
}}

\newcommand{\eqcolon}{\mathrel{\resizebox{\widthof{$\mathord{=}$}}{\height}{ $\!\!=\!\!\resizebox{1.2\width}{0.8\height}{\raisebox{0.23ex}{$\mathop{:}$}}\!\!$ }}}
\newcommand{\coloneq}{\mathrel{\resizebox{\widthof{$\mathord{=}$}}{\height}{ $\!\!\resizebox{1.2\width}{0.8\height}{\raisebox{0.23ex}{$\mathop{:}$}}\!\!=\!\!$ }}}
\newcommand{\eqcolonl}{\ensuremath{\mathrel{=\!\!\mathop{:}}}}
\newcommand{\coloneql}{\ensuremath{\mathrel{\mathop{:} \!\! =}}}
\newcommand{\vc}[1]{% inline column vector
  \left(\begin{smallmatrix}#1\end{smallmatrix}\right)%
}
\newcommand{\vr}[1]{% inline row vector
  \begin{smallmatrix}(\,#1\,)\end{smallmatrix}%
}
\makeatletter
\newcommand*{\defeq}{\ =\mathrel{\rlap{%
                     \raisebox{0.3ex}{$\m@th\cdot$}}%
                     \raisebox{-0.3ex}{$\m@th\cdot$}}%
                     }
\makeatother

\newcommand{\mathcircle}[1]{% inline row vector
 \overset{\circ}{#1}
}
\newcommand{\ulim}{% low limit
 \underline{\lim}
}
\newcommand{\ssi}{% iff
\iff
}
\newcommand{\ps}[2]{
\expval{#1 | #2}
}
\newcommand{\df}[1]{
\mqty{#1}
}
\newcommand{\n}[1]{
\norm{#1}
}
\newcommand{\sys}[1]{
\left\{\smqty{#1}\right.
}


\newcommand{\eqdef}{\ensuremath{\overset{\text{def}}=}}


\def\Circlearrowright{\ensuremath{%
  \rotatebox[origin=c]{230}{$\circlearrowright$}}}

\newcommand\ct[1]{\text{\rmfamily\upshape #1}}
\newcommand\question[1]{ {\color{red} ...!? \small #1}}
\newcommand\caz[1]{\left\{\begin{array} #1 \end{array}\right.}
\newcommand\const{\text{\rmfamily\upshape const}}
\newcommand\toP{ \overset{\pro}{\to}}
\newcommand\toPP{ \overset{\text{PP}}{\to}}
\newcommand{\oeq}{\mathrel{\text{\textcircled{$=$}}}}





\usepackage{xcolor}
% \usepackage[normalem]{ulem}
\usepackage{lipsum}
\makeatletter
% \newcommand\colorwave[1][blue]{\bgroup \markoverwith{\lower3.5\p@\hbox{\sixly \textcolor{#1}{\char58}}}\ULon}
%\font\sixly=lasy6 % does not re-load if already loaded, so no memory problem.

\newmdtheoremenv[
linewidth= 1pt,linecolor= blue,%
leftmargin=20,rightmargin=20,innertopmargin=0pt, innerrightmargin=40,%
tikzsetting = { draw=lightgray, line width = 0.3pt,dashed,%
dash pattern = on 15pt off 3pt},%
splittopskip=\topskip,skipbelow=\baselineskip,%
skipabove=\baselineskip,ntheorem,roundcorner=0pt,
% backgroundcolor=pagebg,font=\color{orange}\sffamily, fontcolor=white
]{examplebox}{Exemple}[section]



\newcommand\R{\mathbb{R}}
\newcommand\Z{\mathbb{Z}}
\newcommand\N{\mathbb{N}}
\newcommand\E{\mathbb{E}}
\newcommand\F{\mathcal{F}}
\newcommand\cH{\mathcal{H}}
\newcommand\V{\mathbb{V}}
\newcommand\dmo{ ^{-1} }
\newcommand\kapa{\kappa}
\newcommand\im{Im}
\newcommand\hs{\mathcal{H}}





\usepackage{soul}

\makeatletter
\newcommand*{\whiten}[1]{\llap{\textcolor{white}{{\the\SOUL@token}}\hspace{#1pt}}}
\DeclareRobustCommand*\myul{%
    \def\SOUL@everyspace{\underline{\space}\kern\z@}%
    \def\SOUL@everytoken{%
     \setbox0=\hbox{\the\SOUL@token}%
     \ifdim\dp0>\z@
        \raisebox{\dp0}{\underline{\phantom{\the\SOUL@token}}}%
        \whiten{1}\whiten{0}%
        \whiten{-1}\whiten{-2}%
        \llap{\the\SOUL@token}%
     \else
        \underline{\the\SOUL@token}%
     \fi}%
\SOUL@}
\makeatother

\newcommand*{\demp}{\fontfamily{lmtt}\selectfont}

\DeclareTextFontCommand{\textdemp}{\demp}

\begin{document}

\ifcomment
Multiline
comment
\fi
\ifcomment
\myul{Typesetting test}
% \color[rgb]{1,1,1}
$∑_i^n≠ 60º±∞π∆¬≈√j∫h≤≥µ$

$\CR \R\pro\ind\pro\gS\pro
\mqty[a&b\\c&d]$
$\pro\mathbb{P}$
$\dd{x}$

  \[
    \alpha(x)=\left\{
                \begin{array}{ll}
                  x\\
                  \frac{1}{1+e^{-kx}}\\
                  \frac{e^x-e^{-x}}{e^x+e^{-x}}
                \end{array}
              \right.
  \]

  $\expval{x}$
  
  $\chi_\rho(ghg\dmo)=\Tr(\rho_{ghg\dmo})=\Tr(\rho_g\circ\rho_h\circ\rho\dmo_g)=\Tr(\rho_h)\overset{\mbox{\scalebox{0.5}{$\Tr(AB)=\Tr(BA)$}}}{=}\chi_\rho(h)$
  	$\mathop{\oplus}_{\substack{x\in X}}$

$\mat(\rho_g)=(a_{ij}(g))_{\scriptsize \substack{1\leq i\leq d \\ 1\leq j\leq d}}$ et $\mat(\rho'_g)=(a'_{ij}(g))_{\scriptsize \substack{1\leq i'\leq d' \\ 1\leq j'\leq d'}}$



\[\int_a^b{\mathbb{R}^2}g(u, v)\dd{P_{XY}}(u, v)=\iint g(u,v) f_{XY}(u, v)\dd \lambda(u) \dd \lambda(v)\]
$$\lim_{x\to\infty} f(x)$$	
$$\iiiint_V \mu(t,u,v,w) \,dt\,du\,dv\,dw$$
$$\sum_{n=1}^{\infty} 2^{-n} = 1$$	
\begin{definition}
	Si $X$ et $Y$ sont 2 v.a. ou definit la \textsc{Covariance} entre $X$ et $Y$ comme
	$\cov(X,Y)\overset{\text{def}}{=}\E\left[(X-\E(X))(Y-\E(Y))\right]=\E(XY)-\E(X)\E(Y)$.
\end{definition}
\fi
\pagebreak

% \tableofcontents

% insert your code here
%\input{./algebra/main.tex}
%\input{./geometrie-differentielle/main.tex}
%\input{./probabilite/main.tex}
%\input{./analyse-fonctionnelle/main.tex}
% \input{./Analyse-convexe-et-dualite-en-optimisation/main.tex}
%\input{./tikz/main.tex}
%\input{./Theorie-du-distributions/main.tex}
%\input{./optimisation/mine.tex}
 \input{./modelisation/main.tex}

% yves.aubry@univ-tln.fr : algebra

\end{document}

% % !TEX encoding = UTF-8 Unicode
% !TEX TS-program = xelatex

\documentclass[french]{report}

%\usepackage[utf8]{inputenc}
%\usepackage[T1]{fontenc}
\usepackage{babel}


\newif\ifcomment
%\commenttrue # Show comments

\usepackage{physics}
\usepackage{amssymb}


\usepackage{amsthm}
% \usepackage{thmtools}
\usepackage{mathtools}
\usepackage{amsfonts}

\usepackage{color}

\usepackage{tikz}

\usepackage{geometry}
\geometry{a5paper, margin=0.1in, right=1cm}

\usepackage{dsfont}

\usepackage{graphicx}
\graphicspath{ {images/} }

\usepackage{faktor}

\usepackage{IEEEtrantools}
\usepackage{enumerate}   
\usepackage[PostScript=dvips]{"/Users/aware/Documents/Courses/diagrams"}


\newtheorem{theorem}{Théorème}[section]
\renewcommand{\thetheorem}{\arabic{theorem}}
\newtheorem{lemme}{Lemme}[section]
\renewcommand{\thelemme}{\arabic{lemme}}
\newtheorem{proposition}{Proposition}[section]
\renewcommand{\theproposition}{\arabic{proposition}}
\newtheorem{notations}{Notations}[section]
\newtheorem{problem}{Problème}[section]
\newtheorem{corollary}{Corollaire}[theorem]
\renewcommand{\thecorollary}{\arabic{corollary}}
\newtheorem{property}{Propriété}[section]
\newtheorem{objective}{Objectif}[section]

\theoremstyle{definition}
\newtheorem{definition}{Définition}[section]
\renewcommand{\thedefinition}{\arabic{definition}}
\newtheorem{exercise}{Exercice}[chapter]
\renewcommand{\theexercise}{\arabic{exercise}}
\newtheorem{example}{Exemple}[chapter]
\renewcommand{\theexample}{\arabic{example}}
\newtheorem*{solution}{Solution}
\newtheorem*{application}{Application}
\newtheorem*{notation}{Notation}
\newtheorem*{vocabulary}{Vocabulaire}
\newtheorem*{properties}{Propriétés}



\theoremstyle{remark}
\newtheorem*{remark}{Remarque}
\newtheorem*{rappel}{Rappel}


\usepackage{etoolbox}
\AtBeginEnvironment{exercise}{\small}
\AtBeginEnvironment{example}{\small}

\usepackage{cases}
\usepackage[red]{mypack}

\usepackage[framemethod=TikZ]{mdframed}

\definecolor{bg}{rgb}{0.4,0.25,0.95}
\definecolor{pagebg}{rgb}{0,0,0.5}
\surroundwithmdframed[
   topline=false,
   rightline=false,
   bottomline=false,
   leftmargin=\parindent,
   skipabove=8pt,
   skipbelow=8pt,
   linecolor=blue,
   innerbottommargin=10pt,
   % backgroundcolor=bg,font=\color{orange}\sffamily, fontcolor=white
]{definition}

\usepackage{empheq}
\usepackage[most]{tcolorbox}

\newtcbox{\mymath}[1][]{%
    nobeforeafter, math upper, tcbox raise base,
    enhanced, colframe=blue!30!black,
    colback=red!10, boxrule=1pt,
    #1}

\usepackage{unixode}


\DeclareMathOperator{\ord}{ord}
\DeclareMathOperator{\orb}{orb}
\DeclareMathOperator{\stab}{stab}
\DeclareMathOperator{\Stab}{stab}
\DeclareMathOperator{\ppcm}{ppcm}
\DeclareMathOperator{\conj}{Conj}
\DeclareMathOperator{\End}{End}
\DeclareMathOperator{\rot}{rot}
\DeclareMathOperator{\trs}{trace}
\DeclareMathOperator{\Ind}{Ind}
\DeclareMathOperator{\mat}{Mat}
\DeclareMathOperator{\id}{Id}
\DeclareMathOperator{\vect}{vect}
\DeclareMathOperator{\img}{img}
\DeclareMathOperator{\cov}{Cov}
\DeclareMathOperator{\dist}{dist}
\DeclareMathOperator{\irr}{Irr}
\DeclareMathOperator{\image}{Im}
\DeclareMathOperator{\pd}{\partial}
\DeclareMathOperator{\epi}{epi}
\DeclareMathOperator{\Argmin}{Argmin}
\DeclareMathOperator{\dom}{dom}
\DeclareMathOperator{\proj}{proj}
\DeclareMathOperator{\ctg}{ctg}
\DeclareMathOperator{\supp}{supp}
\DeclareMathOperator{\argmin}{argmin}
\DeclareMathOperator{\mult}{mult}
\DeclareMathOperator{\ch}{ch}
\DeclareMathOperator{\sh}{sh}
\DeclareMathOperator{\rang}{rang}
\DeclareMathOperator{\diam}{diam}
\DeclareMathOperator{\Epigraphe}{Epigraphe}




\usepackage{xcolor}
\everymath{\color{blue}}
%\everymath{\color[rgb]{0,1,1}}
%\pagecolor[rgb]{0,0,0.5}


\newcommand*{\pdtest}[3][]{\ensuremath{\frac{\partial^{#1} #2}{\partial #3}}}

\newcommand*{\deffunc}[6][]{\ensuremath{
\begin{array}{rcl}
#2 : #3 &\rightarrow& #4\\
#5 &\mapsto& #6
\end{array}
}}

\newcommand{\eqcolon}{\mathrel{\resizebox{\widthof{$\mathord{=}$}}{\height}{ $\!\!=\!\!\resizebox{1.2\width}{0.8\height}{\raisebox{0.23ex}{$\mathop{:}$}}\!\!$ }}}
\newcommand{\coloneq}{\mathrel{\resizebox{\widthof{$\mathord{=}$}}{\height}{ $\!\!\resizebox{1.2\width}{0.8\height}{\raisebox{0.23ex}{$\mathop{:}$}}\!\!=\!\!$ }}}
\newcommand{\eqcolonl}{\ensuremath{\mathrel{=\!\!\mathop{:}}}}
\newcommand{\coloneql}{\ensuremath{\mathrel{\mathop{:} \!\! =}}}
\newcommand{\vc}[1]{% inline column vector
  \left(\begin{smallmatrix}#1\end{smallmatrix}\right)%
}
\newcommand{\vr}[1]{% inline row vector
  \begin{smallmatrix}(\,#1\,)\end{smallmatrix}%
}
\makeatletter
\newcommand*{\defeq}{\ =\mathrel{\rlap{%
                     \raisebox{0.3ex}{$\m@th\cdot$}}%
                     \raisebox{-0.3ex}{$\m@th\cdot$}}%
                     }
\makeatother

\newcommand{\mathcircle}[1]{% inline row vector
 \overset{\circ}{#1}
}
\newcommand{\ulim}{% low limit
 \underline{\lim}
}
\newcommand{\ssi}{% iff
\iff
}
\newcommand{\ps}[2]{
\expval{#1 | #2}
}
\newcommand{\df}[1]{
\mqty{#1}
}
\newcommand{\n}[1]{
\norm{#1}
}
\newcommand{\sys}[1]{
\left\{\smqty{#1}\right.
}


\newcommand{\eqdef}{\ensuremath{\overset{\text{def}}=}}


\def\Circlearrowright{\ensuremath{%
  \rotatebox[origin=c]{230}{$\circlearrowright$}}}

\newcommand\ct[1]{\text{\rmfamily\upshape #1}}
\newcommand\question[1]{ {\color{red} ...!? \small #1}}
\newcommand\caz[1]{\left\{\begin{array} #1 \end{array}\right.}
\newcommand\const{\text{\rmfamily\upshape const}}
\newcommand\toP{ \overset{\pro}{\to}}
\newcommand\toPP{ \overset{\text{PP}}{\to}}
\newcommand{\oeq}{\mathrel{\text{\textcircled{$=$}}}}





\usepackage{xcolor}
% \usepackage[normalem]{ulem}
\usepackage{lipsum}
\makeatletter
% \newcommand\colorwave[1][blue]{\bgroup \markoverwith{\lower3.5\p@\hbox{\sixly \textcolor{#1}{\char58}}}\ULon}
%\font\sixly=lasy6 % does not re-load if already loaded, so no memory problem.

\newmdtheoremenv[
linewidth= 1pt,linecolor= blue,%
leftmargin=20,rightmargin=20,innertopmargin=0pt, innerrightmargin=40,%
tikzsetting = { draw=lightgray, line width = 0.3pt,dashed,%
dash pattern = on 15pt off 3pt},%
splittopskip=\topskip,skipbelow=\baselineskip,%
skipabove=\baselineskip,ntheorem,roundcorner=0pt,
% backgroundcolor=pagebg,font=\color{orange}\sffamily, fontcolor=white
]{examplebox}{Exemple}[section]



\newcommand\R{\mathbb{R}}
\newcommand\Z{\mathbb{Z}}
\newcommand\N{\mathbb{N}}
\newcommand\E{\mathbb{E}}
\newcommand\F{\mathcal{F}}
\newcommand\cH{\mathcal{H}}
\newcommand\V{\mathbb{V}}
\newcommand\dmo{ ^{-1} }
\newcommand\kapa{\kappa}
\newcommand\im{Im}
\newcommand\hs{\mathcal{H}}





\usepackage{soul}

\makeatletter
\newcommand*{\whiten}[1]{\llap{\textcolor{white}{{\the\SOUL@token}}\hspace{#1pt}}}
\DeclareRobustCommand*\myul{%
    \def\SOUL@everyspace{\underline{\space}\kern\z@}%
    \def\SOUL@everytoken{%
     \setbox0=\hbox{\the\SOUL@token}%
     \ifdim\dp0>\z@
        \raisebox{\dp0}{\underline{\phantom{\the\SOUL@token}}}%
        \whiten{1}\whiten{0}%
        \whiten{-1}\whiten{-2}%
        \llap{\the\SOUL@token}%
     \else
        \underline{\the\SOUL@token}%
     \fi}%
\SOUL@}
\makeatother

\newcommand*{\demp}{\fontfamily{lmtt}\selectfont}

\DeclareTextFontCommand{\textdemp}{\demp}

\begin{document}

\ifcomment
Multiline
comment
\fi
\ifcomment
\myul{Typesetting test}
% \color[rgb]{1,1,1}
$∑_i^n≠ 60º±∞π∆¬≈√j∫h≤≥µ$

$\CR \R\pro\ind\pro\gS\pro
\mqty[a&b\\c&d]$
$\pro\mathbb{P}$
$\dd{x}$

  \[
    \alpha(x)=\left\{
                \begin{array}{ll}
                  x\\
                  \frac{1}{1+e^{-kx}}\\
                  \frac{e^x-e^{-x}}{e^x+e^{-x}}
                \end{array}
              \right.
  \]

  $\expval{x}$
  
  $\chi_\rho(ghg\dmo)=\Tr(\rho_{ghg\dmo})=\Tr(\rho_g\circ\rho_h\circ\rho\dmo_g)=\Tr(\rho_h)\overset{\mbox{\scalebox{0.5}{$\Tr(AB)=\Tr(BA)$}}}{=}\chi_\rho(h)$
  	$\mathop{\oplus}_{\substack{x\in X}}$

$\mat(\rho_g)=(a_{ij}(g))_{\scriptsize \substack{1\leq i\leq d \\ 1\leq j\leq d}}$ et $\mat(\rho'_g)=(a'_{ij}(g))_{\scriptsize \substack{1\leq i'\leq d' \\ 1\leq j'\leq d'}}$



\[\int_a^b{\mathbb{R}^2}g(u, v)\dd{P_{XY}}(u, v)=\iint g(u,v) f_{XY}(u, v)\dd \lambda(u) \dd \lambda(v)\]
$$\lim_{x\to\infty} f(x)$$	
$$\iiiint_V \mu(t,u,v,w) \,dt\,du\,dv\,dw$$
$$\sum_{n=1}^{\infty} 2^{-n} = 1$$	
\begin{definition}
	Si $X$ et $Y$ sont 2 v.a. ou definit la \textsc{Covariance} entre $X$ et $Y$ comme
	$\cov(X,Y)\overset{\text{def}}{=}\E\left[(X-\E(X))(Y-\E(Y))\right]=\E(XY)-\E(X)\E(Y)$.
\end{definition}
\fi
\pagebreak

% \tableofcontents

% insert your code here
%\input{./algebra/main.tex}
%\input{./geometrie-differentielle/main.tex}
%\input{./probabilite/main.tex}
%\input{./analyse-fonctionnelle/main.tex}
% \input{./Analyse-convexe-et-dualite-en-optimisation/main.tex}
%\input{./tikz/main.tex}
%\input{./Theorie-du-distributions/main.tex}
%\input{./optimisation/mine.tex}
 \input{./modelisation/main.tex}

% yves.aubry@univ-tln.fr : algebra

\end{document}

%% !TEX encoding = UTF-8 Unicode
% !TEX TS-program = xelatex

\documentclass[french]{report}

%\usepackage[utf8]{inputenc}
%\usepackage[T1]{fontenc}
\usepackage{babel}


\newif\ifcomment
%\commenttrue # Show comments

\usepackage{physics}
\usepackage{amssymb}


\usepackage{amsthm}
% \usepackage{thmtools}
\usepackage{mathtools}
\usepackage{amsfonts}

\usepackage{color}

\usepackage{tikz}

\usepackage{geometry}
\geometry{a5paper, margin=0.1in, right=1cm}

\usepackage{dsfont}

\usepackage{graphicx}
\graphicspath{ {images/} }

\usepackage{faktor}

\usepackage{IEEEtrantools}
\usepackage{enumerate}   
\usepackage[PostScript=dvips]{"/Users/aware/Documents/Courses/diagrams"}


\newtheorem{theorem}{Théorème}[section]
\renewcommand{\thetheorem}{\arabic{theorem}}
\newtheorem{lemme}{Lemme}[section]
\renewcommand{\thelemme}{\arabic{lemme}}
\newtheorem{proposition}{Proposition}[section]
\renewcommand{\theproposition}{\arabic{proposition}}
\newtheorem{notations}{Notations}[section]
\newtheorem{problem}{Problème}[section]
\newtheorem{corollary}{Corollaire}[theorem]
\renewcommand{\thecorollary}{\arabic{corollary}}
\newtheorem{property}{Propriété}[section]
\newtheorem{objective}{Objectif}[section]

\theoremstyle{definition}
\newtheorem{definition}{Définition}[section]
\renewcommand{\thedefinition}{\arabic{definition}}
\newtheorem{exercise}{Exercice}[chapter]
\renewcommand{\theexercise}{\arabic{exercise}}
\newtheorem{example}{Exemple}[chapter]
\renewcommand{\theexample}{\arabic{example}}
\newtheorem*{solution}{Solution}
\newtheorem*{application}{Application}
\newtheorem*{notation}{Notation}
\newtheorem*{vocabulary}{Vocabulaire}
\newtheorem*{properties}{Propriétés}



\theoremstyle{remark}
\newtheorem*{remark}{Remarque}
\newtheorem*{rappel}{Rappel}


\usepackage{etoolbox}
\AtBeginEnvironment{exercise}{\small}
\AtBeginEnvironment{example}{\small}

\usepackage{cases}
\usepackage[red]{mypack}

\usepackage[framemethod=TikZ]{mdframed}

\definecolor{bg}{rgb}{0.4,0.25,0.95}
\definecolor{pagebg}{rgb}{0,0,0.5}
\surroundwithmdframed[
   topline=false,
   rightline=false,
   bottomline=false,
   leftmargin=\parindent,
   skipabove=8pt,
   skipbelow=8pt,
   linecolor=blue,
   innerbottommargin=10pt,
   % backgroundcolor=bg,font=\color{orange}\sffamily, fontcolor=white
]{definition}

\usepackage{empheq}
\usepackage[most]{tcolorbox}

\newtcbox{\mymath}[1][]{%
    nobeforeafter, math upper, tcbox raise base,
    enhanced, colframe=blue!30!black,
    colback=red!10, boxrule=1pt,
    #1}

\usepackage{unixode}


\DeclareMathOperator{\ord}{ord}
\DeclareMathOperator{\orb}{orb}
\DeclareMathOperator{\stab}{stab}
\DeclareMathOperator{\Stab}{stab}
\DeclareMathOperator{\ppcm}{ppcm}
\DeclareMathOperator{\conj}{Conj}
\DeclareMathOperator{\End}{End}
\DeclareMathOperator{\rot}{rot}
\DeclareMathOperator{\trs}{trace}
\DeclareMathOperator{\Ind}{Ind}
\DeclareMathOperator{\mat}{Mat}
\DeclareMathOperator{\id}{Id}
\DeclareMathOperator{\vect}{vect}
\DeclareMathOperator{\img}{img}
\DeclareMathOperator{\cov}{Cov}
\DeclareMathOperator{\dist}{dist}
\DeclareMathOperator{\irr}{Irr}
\DeclareMathOperator{\image}{Im}
\DeclareMathOperator{\pd}{\partial}
\DeclareMathOperator{\epi}{epi}
\DeclareMathOperator{\Argmin}{Argmin}
\DeclareMathOperator{\dom}{dom}
\DeclareMathOperator{\proj}{proj}
\DeclareMathOperator{\ctg}{ctg}
\DeclareMathOperator{\supp}{supp}
\DeclareMathOperator{\argmin}{argmin}
\DeclareMathOperator{\mult}{mult}
\DeclareMathOperator{\ch}{ch}
\DeclareMathOperator{\sh}{sh}
\DeclareMathOperator{\rang}{rang}
\DeclareMathOperator{\diam}{diam}
\DeclareMathOperator{\Epigraphe}{Epigraphe}




\usepackage{xcolor}
\everymath{\color{blue}}
%\everymath{\color[rgb]{0,1,1}}
%\pagecolor[rgb]{0,0,0.5}


\newcommand*{\pdtest}[3][]{\ensuremath{\frac{\partial^{#1} #2}{\partial #3}}}

\newcommand*{\deffunc}[6][]{\ensuremath{
\begin{array}{rcl}
#2 : #3 &\rightarrow& #4\\
#5 &\mapsto& #6
\end{array}
}}

\newcommand{\eqcolon}{\mathrel{\resizebox{\widthof{$\mathord{=}$}}{\height}{ $\!\!=\!\!\resizebox{1.2\width}{0.8\height}{\raisebox{0.23ex}{$\mathop{:}$}}\!\!$ }}}
\newcommand{\coloneq}{\mathrel{\resizebox{\widthof{$\mathord{=}$}}{\height}{ $\!\!\resizebox{1.2\width}{0.8\height}{\raisebox{0.23ex}{$\mathop{:}$}}\!\!=\!\!$ }}}
\newcommand{\eqcolonl}{\ensuremath{\mathrel{=\!\!\mathop{:}}}}
\newcommand{\coloneql}{\ensuremath{\mathrel{\mathop{:} \!\! =}}}
\newcommand{\vc}[1]{% inline column vector
  \left(\begin{smallmatrix}#1\end{smallmatrix}\right)%
}
\newcommand{\vr}[1]{% inline row vector
  \begin{smallmatrix}(\,#1\,)\end{smallmatrix}%
}
\makeatletter
\newcommand*{\defeq}{\ =\mathrel{\rlap{%
                     \raisebox{0.3ex}{$\m@th\cdot$}}%
                     \raisebox{-0.3ex}{$\m@th\cdot$}}%
                     }
\makeatother

\newcommand{\mathcircle}[1]{% inline row vector
 \overset{\circ}{#1}
}
\newcommand{\ulim}{% low limit
 \underline{\lim}
}
\newcommand{\ssi}{% iff
\iff
}
\newcommand{\ps}[2]{
\expval{#1 | #2}
}
\newcommand{\df}[1]{
\mqty{#1}
}
\newcommand{\n}[1]{
\norm{#1}
}
\newcommand{\sys}[1]{
\left\{\smqty{#1}\right.
}


\newcommand{\eqdef}{\ensuremath{\overset{\text{def}}=}}


\def\Circlearrowright{\ensuremath{%
  \rotatebox[origin=c]{230}{$\circlearrowright$}}}

\newcommand\ct[1]{\text{\rmfamily\upshape #1}}
\newcommand\question[1]{ {\color{red} ...!? \small #1}}
\newcommand\caz[1]{\left\{\begin{array} #1 \end{array}\right.}
\newcommand\const{\text{\rmfamily\upshape const}}
\newcommand\toP{ \overset{\pro}{\to}}
\newcommand\toPP{ \overset{\text{PP}}{\to}}
\newcommand{\oeq}{\mathrel{\text{\textcircled{$=$}}}}





\usepackage{xcolor}
% \usepackage[normalem]{ulem}
\usepackage{lipsum}
\makeatletter
% \newcommand\colorwave[1][blue]{\bgroup \markoverwith{\lower3.5\p@\hbox{\sixly \textcolor{#1}{\char58}}}\ULon}
%\font\sixly=lasy6 % does not re-load if already loaded, so no memory problem.

\newmdtheoremenv[
linewidth= 1pt,linecolor= blue,%
leftmargin=20,rightmargin=20,innertopmargin=0pt, innerrightmargin=40,%
tikzsetting = { draw=lightgray, line width = 0.3pt,dashed,%
dash pattern = on 15pt off 3pt},%
splittopskip=\topskip,skipbelow=\baselineskip,%
skipabove=\baselineskip,ntheorem,roundcorner=0pt,
% backgroundcolor=pagebg,font=\color{orange}\sffamily, fontcolor=white
]{examplebox}{Exemple}[section]



\newcommand\R{\mathbb{R}}
\newcommand\Z{\mathbb{Z}}
\newcommand\N{\mathbb{N}}
\newcommand\E{\mathbb{E}}
\newcommand\F{\mathcal{F}}
\newcommand\cH{\mathcal{H}}
\newcommand\V{\mathbb{V}}
\newcommand\dmo{ ^{-1} }
\newcommand\kapa{\kappa}
\newcommand\im{Im}
\newcommand\hs{\mathcal{H}}





\usepackage{soul}

\makeatletter
\newcommand*{\whiten}[1]{\llap{\textcolor{white}{{\the\SOUL@token}}\hspace{#1pt}}}
\DeclareRobustCommand*\myul{%
    \def\SOUL@everyspace{\underline{\space}\kern\z@}%
    \def\SOUL@everytoken{%
     \setbox0=\hbox{\the\SOUL@token}%
     \ifdim\dp0>\z@
        \raisebox{\dp0}{\underline{\phantom{\the\SOUL@token}}}%
        \whiten{1}\whiten{0}%
        \whiten{-1}\whiten{-2}%
        \llap{\the\SOUL@token}%
     \else
        \underline{\the\SOUL@token}%
     \fi}%
\SOUL@}
\makeatother

\newcommand*{\demp}{\fontfamily{lmtt}\selectfont}

\DeclareTextFontCommand{\textdemp}{\demp}

\begin{document}

\ifcomment
Multiline
comment
\fi
\ifcomment
\myul{Typesetting test}
% \color[rgb]{1,1,1}
$∑_i^n≠ 60º±∞π∆¬≈√j∫h≤≥µ$

$\CR \R\pro\ind\pro\gS\pro
\mqty[a&b\\c&d]$
$\pro\mathbb{P}$
$\dd{x}$

  \[
    \alpha(x)=\left\{
                \begin{array}{ll}
                  x\\
                  \frac{1}{1+e^{-kx}}\\
                  \frac{e^x-e^{-x}}{e^x+e^{-x}}
                \end{array}
              \right.
  \]

  $\expval{x}$
  
  $\chi_\rho(ghg\dmo)=\Tr(\rho_{ghg\dmo})=\Tr(\rho_g\circ\rho_h\circ\rho\dmo_g)=\Tr(\rho_h)\overset{\mbox{\scalebox{0.5}{$\Tr(AB)=\Tr(BA)$}}}{=}\chi_\rho(h)$
  	$\mathop{\oplus}_{\substack{x\in X}}$

$\mat(\rho_g)=(a_{ij}(g))_{\scriptsize \substack{1\leq i\leq d \\ 1\leq j\leq d}}$ et $\mat(\rho'_g)=(a'_{ij}(g))_{\scriptsize \substack{1\leq i'\leq d' \\ 1\leq j'\leq d'}}$



\[\int_a^b{\mathbb{R}^2}g(u, v)\dd{P_{XY}}(u, v)=\iint g(u,v) f_{XY}(u, v)\dd \lambda(u) \dd \lambda(v)\]
$$\lim_{x\to\infty} f(x)$$	
$$\iiiint_V \mu(t,u,v,w) \,dt\,du\,dv\,dw$$
$$\sum_{n=1}^{\infty} 2^{-n} = 1$$	
\begin{definition}
	Si $X$ et $Y$ sont 2 v.a. ou definit la \textsc{Covariance} entre $X$ et $Y$ comme
	$\cov(X,Y)\overset{\text{def}}{=}\E\left[(X-\E(X))(Y-\E(Y))\right]=\E(XY)-\E(X)\E(Y)$.
\end{definition}
\fi
\pagebreak

% \tableofcontents

% insert your code here
%\input{./algebra/main.tex}
%\input{./geometrie-differentielle/main.tex}
%\input{./probabilite/main.tex}
%\input{./analyse-fonctionnelle/main.tex}
% \input{./Analyse-convexe-et-dualite-en-optimisation/main.tex}
%\input{./tikz/main.tex}
%\input{./Theorie-du-distributions/main.tex}
%\input{./optimisation/mine.tex}
 \input{./modelisation/main.tex}

% yves.aubry@univ-tln.fr : algebra

\end{document}

%% !TEX encoding = UTF-8 Unicode
% !TEX TS-program = xelatex

\documentclass[french]{report}

%\usepackage[utf8]{inputenc}
%\usepackage[T1]{fontenc}
\usepackage{babel}


\newif\ifcomment
%\commenttrue # Show comments

\usepackage{physics}
\usepackage{amssymb}


\usepackage{amsthm}
% \usepackage{thmtools}
\usepackage{mathtools}
\usepackage{amsfonts}

\usepackage{color}

\usepackage{tikz}

\usepackage{geometry}
\geometry{a5paper, margin=0.1in, right=1cm}

\usepackage{dsfont}

\usepackage{graphicx}
\graphicspath{ {images/} }

\usepackage{faktor}

\usepackage{IEEEtrantools}
\usepackage{enumerate}   
\usepackage[PostScript=dvips]{"/Users/aware/Documents/Courses/diagrams"}


\newtheorem{theorem}{Théorème}[section]
\renewcommand{\thetheorem}{\arabic{theorem}}
\newtheorem{lemme}{Lemme}[section]
\renewcommand{\thelemme}{\arabic{lemme}}
\newtheorem{proposition}{Proposition}[section]
\renewcommand{\theproposition}{\arabic{proposition}}
\newtheorem{notations}{Notations}[section]
\newtheorem{problem}{Problème}[section]
\newtheorem{corollary}{Corollaire}[theorem]
\renewcommand{\thecorollary}{\arabic{corollary}}
\newtheorem{property}{Propriété}[section]
\newtheorem{objective}{Objectif}[section]

\theoremstyle{definition}
\newtheorem{definition}{Définition}[section]
\renewcommand{\thedefinition}{\arabic{definition}}
\newtheorem{exercise}{Exercice}[chapter]
\renewcommand{\theexercise}{\arabic{exercise}}
\newtheorem{example}{Exemple}[chapter]
\renewcommand{\theexample}{\arabic{example}}
\newtheorem*{solution}{Solution}
\newtheorem*{application}{Application}
\newtheorem*{notation}{Notation}
\newtheorem*{vocabulary}{Vocabulaire}
\newtheorem*{properties}{Propriétés}



\theoremstyle{remark}
\newtheorem*{remark}{Remarque}
\newtheorem*{rappel}{Rappel}


\usepackage{etoolbox}
\AtBeginEnvironment{exercise}{\small}
\AtBeginEnvironment{example}{\small}

\usepackage{cases}
\usepackage[red]{mypack}

\usepackage[framemethod=TikZ]{mdframed}

\definecolor{bg}{rgb}{0.4,0.25,0.95}
\definecolor{pagebg}{rgb}{0,0,0.5}
\surroundwithmdframed[
   topline=false,
   rightline=false,
   bottomline=false,
   leftmargin=\parindent,
   skipabove=8pt,
   skipbelow=8pt,
   linecolor=blue,
   innerbottommargin=10pt,
   % backgroundcolor=bg,font=\color{orange}\sffamily, fontcolor=white
]{definition}

\usepackage{empheq}
\usepackage[most]{tcolorbox}

\newtcbox{\mymath}[1][]{%
    nobeforeafter, math upper, tcbox raise base,
    enhanced, colframe=blue!30!black,
    colback=red!10, boxrule=1pt,
    #1}

\usepackage{unixode}


\DeclareMathOperator{\ord}{ord}
\DeclareMathOperator{\orb}{orb}
\DeclareMathOperator{\stab}{stab}
\DeclareMathOperator{\Stab}{stab}
\DeclareMathOperator{\ppcm}{ppcm}
\DeclareMathOperator{\conj}{Conj}
\DeclareMathOperator{\End}{End}
\DeclareMathOperator{\rot}{rot}
\DeclareMathOperator{\trs}{trace}
\DeclareMathOperator{\Ind}{Ind}
\DeclareMathOperator{\mat}{Mat}
\DeclareMathOperator{\id}{Id}
\DeclareMathOperator{\vect}{vect}
\DeclareMathOperator{\img}{img}
\DeclareMathOperator{\cov}{Cov}
\DeclareMathOperator{\dist}{dist}
\DeclareMathOperator{\irr}{Irr}
\DeclareMathOperator{\image}{Im}
\DeclareMathOperator{\pd}{\partial}
\DeclareMathOperator{\epi}{epi}
\DeclareMathOperator{\Argmin}{Argmin}
\DeclareMathOperator{\dom}{dom}
\DeclareMathOperator{\proj}{proj}
\DeclareMathOperator{\ctg}{ctg}
\DeclareMathOperator{\supp}{supp}
\DeclareMathOperator{\argmin}{argmin}
\DeclareMathOperator{\mult}{mult}
\DeclareMathOperator{\ch}{ch}
\DeclareMathOperator{\sh}{sh}
\DeclareMathOperator{\rang}{rang}
\DeclareMathOperator{\diam}{diam}
\DeclareMathOperator{\Epigraphe}{Epigraphe}




\usepackage{xcolor}
\everymath{\color{blue}}
%\everymath{\color[rgb]{0,1,1}}
%\pagecolor[rgb]{0,0,0.5}


\newcommand*{\pdtest}[3][]{\ensuremath{\frac{\partial^{#1} #2}{\partial #3}}}

\newcommand*{\deffunc}[6][]{\ensuremath{
\begin{array}{rcl}
#2 : #3 &\rightarrow& #4\\
#5 &\mapsto& #6
\end{array}
}}

\newcommand{\eqcolon}{\mathrel{\resizebox{\widthof{$\mathord{=}$}}{\height}{ $\!\!=\!\!\resizebox{1.2\width}{0.8\height}{\raisebox{0.23ex}{$\mathop{:}$}}\!\!$ }}}
\newcommand{\coloneq}{\mathrel{\resizebox{\widthof{$\mathord{=}$}}{\height}{ $\!\!\resizebox{1.2\width}{0.8\height}{\raisebox{0.23ex}{$\mathop{:}$}}\!\!=\!\!$ }}}
\newcommand{\eqcolonl}{\ensuremath{\mathrel{=\!\!\mathop{:}}}}
\newcommand{\coloneql}{\ensuremath{\mathrel{\mathop{:} \!\! =}}}
\newcommand{\vc}[1]{% inline column vector
  \left(\begin{smallmatrix}#1\end{smallmatrix}\right)%
}
\newcommand{\vr}[1]{% inline row vector
  \begin{smallmatrix}(\,#1\,)\end{smallmatrix}%
}
\makeatletter
\newcommand*{\defeq}{\ =\mathrel{\rlap{%
                     \raisebox{0.3ex}{$\m@th\cdot$}}%
                     \raisebox{-0.3ex}{$\m@th\cdot$}}%
                     }
\makeatother

\newcommand{\mathcircle}[1]{% inline row vector
 \overset{\circ}{#1}
}
\newcommand{\ulim}{% low limit
 \underline{\lim}
}
\newcommand{\ssi}{% iff
\iff
}
\newcommand{\ps}[2]{
\expval{#1 | #2}
}
\newcommand{\df}[1]{
\mqty{#1}
}
\newcommand{\n}[1]{
\norm{#1}
}
\newcommand{\sys}[1]{
\left\{\smqty{#1}\right.
}


\newcommand{\eqdef}{\ensuremath{\overset{\text{def}}=}}


\def\Circlearrowright{\ensuremath{%
  \rotatebox[origin=c]{230}{$\circlearrowright$}}}

\newcommand\ct[1]{\text{\rmfamily\upshape #1}}
\newcommand\question[1]{ {\color{red} ...!? \small #1}}
\newcommand\caz[1]{\left\{\begin{array} #1 \end{array}\right.}
\newcommand\const{\text{\rmfamily\upshape const}}
\newcommand\toP{ \overset{\pro}{\to}}
\newcommand\toPP{ \overset{\text{PP}}{\to}}
\newcommand{\oeq}{\mathrel{\text{\textcircled{$=$}}}}





\usepackage{xcolor}
% \usepackage[normalem]{ulem}
\usepackage{lipsum}
\makeatletter
% \newcommand\colorwave[1][blue]{\bgroup \markoverwith{\lower3.5\p@\hbox{\sixly \textcolor{#1}{\char58}}}\ULon}
%\font\sixly=lasy6 % does not re-load if already loaded, so no memory problem.

\newmdtheoremenv[
linewidth= 1pt,linecolor= blue,%
leftmargin=20,rightmargin=20,innertopmargin=0pt, innerrightmargin=40,%
tikzsetting = { draw=lightgray, line width = 0.3pt,dashed,%
dash pattern = on 15pt off 3pt},%
splittopskip=\topskip,skipbelow=\baselineskip,%
skipabove=\baselineskip,ntheorem,roundcorner=0pt,
% backgroundcolor=pagebg,font=\color{orange}\sffamily, fontcolor=white
]{examplebox}{Exemple}[section]



\newcommand\R{\mathbb{R}}
\newcommand\Z{\mathbb{Z}}
\newcommand\N{\mathbb{N}}
\newcommand\E{\mathbb{E}}
\newcommand\F{\mathcal{F}}
\newcommand\cH{\mathcal{H}}
\newcommand\V{\mathbb{V}}
\newcommand\dmo{ ^{-1} }
\newcommand\kapa{\kappa}
\newcommand\im{Im}
\newcommand\hs{\mathcal{H}}





\usepackage{soul}

\makeatletter
\newcommand*{\whiten}[1]{\llap{\textcolor{white}{{\the\SOUL@token}}\hspace{#1pt}}}
\DeclareRobustCommand*\myul{%
    \def\SOUL@everyspace{\underline{\space}\kern\z@}%
    \def\SOUL@everytoken{%
     \setbox0=\hbox{\the\SOUL@token}%
     \ifdim\dp0>\z@
        \raisebox{\dp0}{\underline{\phantom{\the\SOUL@token}}}%
        \whiten{1}\whiten{0}%
        \whiten{-1}\whiten{-2}%
        \llap{\the\SOUL@token}%
     \else
        \underline{\the\SOUL@token}%
     \fi}%
\SOUL@}
\makeatother

\newcommand*{\demp}{\fontfamily{lmtt}\selectfont}

\DeclareTextFontCommand{\textdemp}{\demp}

\begin{document}

\ifcomment
Multiline
comment
\fi
\ifcomment
\myul{Typesetting test}
% \color[rgb]{1,1,1}
$∑_i^n≠ 60º±∞π∆¬≈√j∫h≤≥µ$

$\CR \R\pro\ind\pro\gS\pro
\mqty[a&b\\c&d]$
$\pro\mathbb{P}$
$\dd{x}$

  \[
    \alpha(x)=\left\{
                \begin{array}{ll}
                  x\\
                  \frac{1}{1+e^{-kx}}\\
                  \frac{e^x-e^{-x}}{e^x+e^{-x}}
                \end{array}
              \right.
  \]

  $\expval{x}$
  
  $\chi_\rho(ghg\dmo)=\Tr(\rho_{ghg\dmo})=\Tr(\rho_g\circ\rho_h\circ\rho\dmo_g)=\Tr(\rho_h)\overset{\mbox{\scalebox{0.5}{$\Tr(AB)=\Tr(BA)$}}}{=}\chi_\rho(h)$
  	$\mathop{\oplus}_{\substack{x\in X}}$

$\mat(\rho_g)=(a_{ij}(g))_{\scriptsize \substack{1\leq i\leq d \\ 1\leq j\leq d}}$ et $\mat(\rho'_g)=(a'_{ij}(g))_{\scriptsize \substack{1\leq i'\leq d' \\ 1\leq j'\leq d'}}$



\[\int_a^b{\mathbb{R}^2}g(u, v)\dd{P_{XY}}(u, v)=\iint g(u,v) f_{XY}(u, v)\dd \lambda(u) \dd \lambda(v)\]
$$\lim_{x\to\infty} f(x)$$	
$$\iiiint_V \mu(t,u,v,w) \,dt\,du\,dv\,dw$$
$$\sum_{n=1}^{\infty} 2^{-n} = 1$$	
\begin{definition}
	Si $X$ et $Y$ sont 2 v.a. ou definit la \textsc{Covariance} entre $X$ et $Y$ comme
	$\cov(X,Y)\overset{\text{def}}{=}\E\left[(X-\E(X))(Y-\E(Y))\right]=\E(XY)-\E(X)\E(Y)$.
\end{definition}
\fi
\pagebreak

% \tableofcontents

% insert your code here
%\input{./algebra/main.tex}
%\input{./geometrie-differentielle/main.tex}
%\input{./probabilite/main.tex}
%\input{./analyse-fonctionnelle/main.tex}
% \input{./Analyse-convexe-et-dualite-en-optimisation/main.tex}
%\input{./tikz/main.tex}
%\input{./Theorie-du-distributions/main.tex}
%\input{./optimisation/mine.tex}
 \input{./modelisation/main.tex}

% yves.aubry@univ-tln.fr : algebra

\end{document}

%\input{./optimisation/mine.tex}
 % !TEX encoding = UTF-8 Unicode
% !TEX TS-program = xelatex

\documentclass[french]{report}

%\usepackage[utf8]{inputenc}
%\usepackage[T1]{fontenc}
\usepackage{babel}


\newif\ifcomment
%\commenttrue # Show comments

\usepackage{physics}
\usepackage{amssymb}


\usepackage{amsthm}
% \usepackage{thmtools}
\usepackage{mathtools}
\usepackage{amsfonts}

\usepackage{color}

\usepackage{tikz}

\usepackage{geometry}
\geometry{a5paper, margin=0.1in, right=1cm}

\usepackage{dsfont}

\usepackage{graphicx}
\graphicspath{ {images/} }

\usepackage{faktor}

\usepackage{IEEEtrantools}
\usepackage{enumerate}   
\usepackage[PostScript=dvips]{"/Users/aware/Documents/Courses/diagrams"}


\newtheorem{theorem}{Théorème}[section]
\renewcommand{\thetheorem}{\arabic{theorem}}
\newtheorem{lemme}{Lemme}[section]
\renewcommand{\thelemme}{\arabic{lemme}}
\newtheorem{proposition}{Proposition}[section]
\renewcommand{\theproposition}{\arabic{proposition}}
\newtheorem{notations}{Notations}[section]
\newtheorem{problem}{Problème}[section]
\newtheorem{corollary}{Corollaire}[theorem]
\renewcommand{\thecorollary}{\arabic{corollary}}
\newtheorem{property}{Propriété}[section]
\newtheorem{objective}{Objectif}[section]

\theoremstyle{definition}
\newtheorem{definition}{Définition}[section]
\renewcommand{\thedefinition}{\arabic{definition}}
\newtheorem{exercise}{Exercice}[chapter]
\renewcommand{\theexercise}{\arabic{exercise}}
\newtheorem{example}{Exemple}[chapter]
\renewcommand{\theexample}{\arabic{example}}
\newtheorem*{solution}{Solution}
\newtheorem*{application}{Application}
\newtheorem*{notation}{Notation}
\newtheorem*{vocabulary}{Vocabulaire}
\newtheorem*{properties}{Propriétés}



\theoremstyle{remark}
\newtheorem*{remark}{Remarque}
\newtheorem*{rappel}{Rappel}


\usepackage{etoolbox}
\AtBeginEnvironment{exercise}{\small}
\AtBeginEnvironment{example}{\small}

\usepackage{cases}
\usepackage[red]{mypack}

\usepackage[framemethod=TikZ]{mdframed}

\definecolor{bg}{rgb}{0.4,0.25,0.95}
\definecolor{pagebg}{rgb}{0,0,0.5}
\surroundwithmdframed[
   topline=false,
   rightline=false,
   bottomline=false,
   leftmargin=\parindent,
   skipabove=8pt,
   skipbelow=8pt,
   linecolor=blue,
   innerbottommargin=10pt,
   % backgroundcolor=bg,font=\color{orange}\sffamily, fontcolor=white
]{definition}

\usepackage{empheq}
\usepackage[most]{tcolorbox}

\newtcbox{\mymath}[1][]{%
    nobeforeafter, math upper, tcbox raise base,
    enhanced, colframe=blue!30!black,
    colback=red!10, boxrule=1pt,
    #1}

\usepackage{unixode}


\DeclareMathOperator{\ord}{ord}
\DeclareMathOperator{\orb}{orb}
\DeclareMathOperator{\stab}{stab}
\DeclareMathOperator{\Stab}{stab}
\DeclareMathOperator{\ppcm}{ppcm}
\DeclareMathOperator{\conj}{Conj}
\DeclareMathOperator{\End}{End}
\DeclareMathOperator{\rot}{rot}
\DeclareMathOperator{\trs}{trace}
\DeclareMathOperator{\Ind}{Ind}
\DeclareMathOperator{\mat}{Mat}
\DeclareMathOperator{\id}{Id}
\DeclareMathOperator{\vect}{vect}
\DeclareMathOperator{\img}{img}
\DeclareMathOperator{\cov}{Cov}
\DeclareMathOperator{\dist}{dist}
\DeclareMathOperator{\irr}{Irr}
\DeclareMathOperator{\image}{Im}
\DeclareMathOperator{\pd}{\partial}
\DeclareMathOperator{\epi}{epi}
\DeclareMathOperator{\Argmin}{Argmin}
\DeclareMathOperator{\dom}{dom}
\DeclareMathOperator{\proj}{proj}
\DeclareMathOperator{\ctg}{ctg}
\DeclareMathOperator{\supp}{supp}
\DeclareMathOperator{\argmin}{argmin}
\DeclareMathOperator{\mult}{mult}
\DeclareMathOperator{\ch}{ch}
\DeclareMathOperator{\sh}{sh}
\DeclareMathOperator{\rang}{rang}
\DeclareMathOperator{\diam}{diam}
\DeclareMathOperator{\Epigraphe}{Epigraphe}




\usepackage{xcolor}
\everymath{\color{blue}}
%\everymath{\color[rgb]{0,1,1}}
%\pagecolor[rgb]{0,0,0.5}


\newcommand*{\pdtest}[3][]{\ensuremath{\frac{\partial^{#1} #2}{\partial #3}}}

\newcommand*{\deffunc}[6][]{\ensuremath{
\begin{array}{rcl}
#2 : #3 &\rightarrow& #4\\
#5 &\mapsto& #6
\end{array}
}}

\newcommand{\eqcolon}{\mathrel{\resizebox{\widthof{$\mathord{=}$}}{\height}{ $\!\!=\!\!\resizebox{1.2\width}{0.8\height}{\raisebox{0.23ex}{$\mathop{:}$}}\!\!$ }}}
\newcommand{\coloneq}{\mathrel{\resizebox{\widthof{$\mathord{=}$}}{\height}{ $\!\!\resizebox{1.2\width}{0.8\height}{\raisebox{0.23ex}{$\mathop{:}$}}\!\!=\!\!$ }}}
\newcommand{\eqcolonl}{\ensuremath{\mathrel{=\!\!\mathop{:}}}}
\newcommand{\coloneql}{\ensuremath{\mathrel{\mathop{:} \!\! =}}}
\newcommand{\vc}[1]{% inline column vector
  \left(\begin{smallmatrix}#1\end{smallmatrix}\right)%
}
\newcommand{\vr}[1]{% inline row vector
  \begin{smallmatrix}(\,#1\,)\end{smallmatrix}%
}
\makeatletter
\newcommand*{\defeq}{\ =\mathrel{\rlap{%
                     \raisebox{0.3ex}{$\m@th\cdot$}}%
                     \raisebox{-0.3ex}{$\m@th\cdot$}}%
                     }
\makeatother

\newcommand{\mathcircle}[1]{% inline row vector
 \overset{\circ}{#1}
}
\newcommand{\ulim}{% low limit
 \underline{\lim}
}
\newcommand{\ssi}{% iff
\iff
}
\newcommand{\ps}[2]{
\expval{#1 | #2}
}
\newcommand{\df}[1]{
\mqty{#1}
}
\newcommand{\n}[1]{
\norm{#1}
}
\newcommand{\sys}[1]{
\left\{\smqty{#1}\right.
}


\newcommand{\eqdef}{\ensuremath{\overset{\text{def}}=}}


\def\Circlearrowright{\ensuremath{%
  \rotatebox[origin=c]{230}{$\circlearrowright$}}}

\newcommand\ct[1]{\text{\rmfamily\upshape #1}}
\newcommand\question[1]{ {\color{red} ...!? \small #1}}
\newcommand\caz[1]{\left\{\begin{array} #1 \end{array}\right.}
\newcommand\const{\text{\rmfamily\upshape const}}
\newcommand\toP{ \overset{\pro}{\to}}
\newcommand\toPP{ \overset{\text{PP}}{\to}}
\newcommand{\oeq}{\mathrel{\text{\textcircled{$=$}}}}





\usepackage{xcolor}
% \usepackage[normalem]{ulem}
\usepackage{lipsum}
\makeatletter
% \newcommand\colorwave[1][blue]{\bgroup \markoverwith{\lower3.5\p@\hbox{\sixly \textcolor{#1}{\char58}}}\ULon}
%\font\sixly=lasy6 % does not re-load if already loaded, so no memory problem.

\newmdtheoremenv[
linewidth= 1pt,linecolor= blue,%
leftmargin=20,rightmargin=20,innertopmargin=0pt, innerrightmargin=40,%
tikzsetting = { draw=lightgray, line width = 0.3pt,dashed,%
dash pattern = on 15pt off 3pt},%
splittopskip=\topskip,skipbelow=\baselineskip,%
skipabove=\baselineskip,ntheorem,roundcorner=0pt,
% backgroundcolor=pagebg,font=\color{orange}\sffamily, fontcolor=white
]{examplebox}{Exemple}[section]



\newcommand\R{\mathbb{R}}
\newcommand\Z{\mathbb{Z}}
\newcommand\N{\mathbb{N}}
\newcommand\E{\mathbb{E}}
\newcommand\F{\mathcal{F}}
\newcommand\cH{\mathcal{H}}
\newcommand\V{\mathbb{V}}
\newcommand\dmo{ ^{-1} }
\newcommand\kapa{\kappa}
\newcommand\im{Im}
\newcommand\hs{\mathcal{H}}





\usepackage{soul}

\makeatletter
\newcommand*{\whiten}[1]{\llap{\textcolor{white}{{\the\SOUL@token}}\hspace{#1pt}}}
\DeclareRobustCommand*\myul{%
    \def\SOUL@everyspace{\underline{\space}\kern\z@}%
    \def\SOUL@everytoken{%
     \setbox0=\hbox{\the\SOUL@token}%
     \ifdim\dp0>\z@
        \raisebox{\dp0}{\underline{\phantom{\the\SOUL@token}}}%
        \whiten{1}\whiten{0}%
        \whiten{-1}\whiten{-2}%
        \llap{\the\SOUL@token}%
     \else
        \underline{\the\SOUL@token}%
     \fi}%
\SOUL@}
\makeatother

\newcommand*{\demp}{\fontfamily{lmtt}\selectfont}

\DeclareTextFontCommand{\textdemp}{\demp}

\begin{document}

\ifcomment
Multiline
comment
\fi
\ifcomment
\myul{Typesetting test}
% \color[rgb]{1,1,1}
$∑_i^n≠ 60º±∞π∆¬≈√j∫h≤≥µ$

$\CR \R\pro\ind\pro\gS\pro
\mqty[a&b\\c&d]$
$\pro\mathbb{P}$
$\dd{x}$

  \[
    \alpha(x)=\left\{
                \begin{array}{ll}
                  x\\
                  \frac{1}{1+e^{-kx}}\\
                  \frac{e^x-e^{-x}}{e^x+e^{-x}}
                \end{array}
              \right.
  \]

  $\expval{x}$
  
  $\chi_\rho(ghg\dmo)=\Tr(\rho_{ghg\dmo})=\Tr(\rho_g\circ\rho_h\circ\rho\dmo_g)=\Tr(\rho_h)\overset{\mbox{\scalebox{0.5}{$\Tr(AB)=\Tr(BA)$}}}{=}\chi_\rho(h)$
  	$\mathop{\oplus}_{\substack{x\in X}}$

$\mat(\rho_g)=(a_{ij}(g))_{\scriptsize \substack{1\leq i\leq d \\ 1\leq j\leq d}}$ et $\mat(\rho'_g)=(a'_{ij}(g))_{\scriptsize \substack{1\leq i'\leq d' \\ 1\leq j'\leq d'}}$



\[\int_a^b{\mathbb{R}^2}g(u, v)\dd{P_{XY}}(u, v)=\iint g(u,v) f_{XY}(u, v)\dd \lambda(u) \dd \lambda(v)\]
$$\lim_{x\to\infty} f(x)$$	
$$\iiiint_V \mu(t,u,v,w) \,dt\,du\,dv\,dw$$
$$\sum_{n=1}^{\infty} 2^{-n} = 1$$	
\begin{definition}
	Si $X$ et $Y$ sont 2 v.a. ou definit la \textsc{Covariance} entre $X$ et $Y$ comme
	$\cov(X,Y)\overset{\text{def}}{=}\E\left[(X-\E(X))(Y-\E(Y))\right]=\E(XY)-\E(X)\E(Y)$.
\end{definition}
\fi
\pagebreak

% \tableofcontents

% insert your code here
%\input{./algebra/main.tex}
%\input{./geometrie-differentielle/main.tex}
%\input{./probabilite/main.tex}
%\input{./analyse-fonctionnelle/main.tex}
% \input{./Analyse-convexe-et-dualite-en-optimisation/main.tex}
%\input{./tikz/main.tex}
%\input{./Theorie-du-distributions/main.tex}
%\input{./optimisation/mine.tex}
 \input{./modelisation/main.tex}

% yves.aubry@univ-tln.fr : algebra

\end{document}


% yves.aubry@univ-tln.fr : algebra

\end{document}


% yves.aubry@univ-tln.fr : algebra

\end{document}

%% !TEX encoding = UTF-8 Unicode
% !TEX TS-program = xelatex

\documentclass[french]{report}

%\usepackage[utf8]{inputenc}
%\usepackage[T1]{fontenc}
\usepackage{babel}


\newif\ifcomment
%\commenttrue # Show comments

\usepackage{physics}
\usepackage{amssymb}


\usepackage{amsthm}
% \usepackage{thmtools}
\usepackage{mathtools}
\usepackage{amsfonts}

\usepackage{color}

\usepackage{tikz}

\usepackage{geometry}
\geometry{a5paper, margin=0.1in, right=1cm}

\usepackage{dsfont}

\usepackage{graphicx}
\graphicspath{ {images/} }

\usepackage{faktor}

\usepackage{IEEEtrantools}
\usepackage{enumerate}   
\usepackage[PostScript=dvips]{"/Users/aware/Documents/Courses/diagrams"}


\newtheorem{theorem}{Théorème}[section]
\renewcommand{\thetheorem}{\arabic{theorem}}
\newtheorem{lemme}{Lemme}[section]
\renewcommand{\thelemme}{\arabic{lemme}}
\newtheorem{proposition}{Proposition}[section]
\renewcommand{\theproposition}{\arabic{proposition}}
\newtheorem{notations}{Notations}[section]
\newtheorem{problem}{Problème}[section]
\newtheorem{corollary}{Corollaire}[theorem]
\renewcommand{\thecorollary}{\arabic{corollary}}
\newtheorem{property}{Propriété}[section]
\newtheorem{objective}{Objectif}[section]

\theoremstyle{definition}
\newtheorem{definition}{Définition}[section]
\renewcommand{\thedefinition}{\arabic{definition}}
\newtheorem{exercise}{Exercice}[chapter]
\renewcommand{\theexercise}{\arabic{exercise}}
\newtheorem{example}{Exemple}[chapter]
\renewcommand{\theexample}{\arabic{example}}
\newtheorem*{solution}{Solution}
\newtheorem*{application}{Application}
\newtheorem*{notation}{Notation}
\newtheorem*{vocabulary}{Vocabulaire}
\newtheorem*{properties}{Propriétés}



\theoremstyle{remark}
\newtheorem*{remark}{Remarque}
\newtheorem*{rappel}{Rappel}


\usepackage{etoolbox}
\AtBeginEnvironment{exercise}{\small}
\AtBeginEnvironment{example}{\small}

\usepackage{cases}
\usepackage[red]{mypack}

\usepackage[framemethod=TikZ]{mdframed}

\definecolor{bg}{rgb}{0.4,0.25,0.95}
\definecolor{pagebg}{rgb}{0,0,0.5}
\surroundwithmdframed[
   topline=false,
   rightline=false,
   bottomline=false,
   leftmargin=\parindent,
   skipabove=8pt,
   skipbelow=8pt,
   linecolor=blue,
   innerbottommargin=10pt,
   % backgroundcolor=bg,font=\color{orange}\sffamily, fontcolor=white
]{definition}

\usepackage{empheq}
\usepackage[most]{tcolorbox}

\newtcbox{\mymath}[1][]{%
    nobeforeafter, math upper, tcbox raise base,
    enhanced, colframe=blue!30!black,
    colback=red!10, boxrule=1pt,
    #1}

\usepackage{unixode}


\DeclareMathOperator{\ord}{ord}
\DeclareMathOperator{\orb}{orb}
\DeclareMathOperator{\stab}{stab}
\DeclareMathOperator{\Stab}{stab}
\DeclareMathOperator{\ppcm}{ppcm}
\DeclareMathOperator{\conj}{Conj}
\DeclareMathOperator{\End}{End}
\DeclareMathOperator{\rot}{rot}
\DeclareMathOperator{\trs}{trace}
\DeclareMathOperator{\Ind}{Ind}
\DeclareMathOperator{\mat}{Mat}
\DeclareMathOperator{\id}{Id}
\DeclareMathOperator{\vect}{vect}
\DeclareMathOperator{\img}{img}
\DeclareMathOperator{\cov}{Cov}
\DeclareMathOperator{\dist}{dist}
\DeclareMathOperator{\irr}{Irr}
\DeclareMathOperator{\image}{Im}
\DeclareMathOperator{\pd}{\partial}
\DeclareMathOperator{\epi}{epi}
\DeclareMathOperator{\Argmin}{Argmin}
\DeclareMathOperator{\dom}{dom}
\DeclareMathOperator{\proj}{proj}
\DeclareMathOperator{\ctg}{ctg}
\DeclareMathOperator{\supp}{supp}
\DeclareMathOperator{\argmin}{argmin}
\DeclareMathOperator{\mult}{mult}
\DeclareMathOperator{\ch}{ch}
\DeclareMathOperator{\sh}{sh}
\DeclareMathOperator{\rang}{rang}
\DeclareMathOperator{\diam}{diam}
\DeclareMathOperator{\Epigraphe}{Epigraphe}




\usepackage{xcolor}
\everymath{\color{blue}}
%\everymath{\color[rgb]{0,1,1}}
%\pagecolor[rgb]{0,0,0.5}


\newcommand*{\pdtest}[3][]{\ensuremath{\frac{\partial^{#1} #2}{\partial #3}}}

\newcommand*{\deffunc}[6][]{\ensuremath{
\begin{array}{rcl}
#2 : #3 &\rightarrow& #4\\
#5 &\mapsto& #6
\end{array}
}}

\newcommand{\eqcolon}{\mathrel{\resizebox{\widthof{$\mathord{=}$}}{\height}{ $\!\!=\!\!\resizebox{1.2\width}{0.8\height}{\raisebox{0.23ex}{$\mathop{:}$}}\!\!$ }}}
\newcommand{\coloneq}{\mathrel{\resizebox{\widthof{$\mathord{=}$}}{\height}{ $\!\!\resizebox{1.2\width}{0.8\height}{\raisebox{0.23ex}{$\mathop{:}$}}\!\!=\!\!$ }}}
\newcommand{\eqcolonl}{\ensuremath{\mathrel{=\!\!\mathop{:}}}}
\newcommand{\coloneql}{\ensuremath{\mathrel{\mathop{:} \!\! =}}}
\newcommand{\vc}[1]{% inline column vector
  \left(\begin{smallmatrix}#1\end{smallmatrix}\right)%
}
\newcommand{\vr}[1]{% inline row vector
  \begin{smallmatrix}(\,#1\,)\end{smallmatrix}%
}
\makeatletter
\newcommand*{\defeq}{\ =\mathrel{\rlap{%
                     \raisebox{0.3ex}{$\m@th\cdot$}}%
                     \raisebox{-0.3ex}{$\m@th\cdot$}}%
                     }
\makeatother

\newcommand{\mathcircle}[1]{% inline row vector
 \overset{\circ}{#1}
}
\newcommand{\ulim}{% low limit
 \underline{\lim}
}
\newcommand{\ssi}{% iff
\iff
}
\newcommand{\ps}[2]{
\expval{#1 | #2}
}
\newcommand{\df}[1]{
\mqty{#1}
}
\newcommand{\n}[1]{
\norm{#1}
}
\newcommand{\sys}[1]{
\left\{\smqty{#1}\right.
}


\newcommand{\eqdef}{\ensuremath{\overset{\text{def}}=}}


\def\Circlearrowright{\ensuremath{%
  \rotatebox[origin=c]{230}{$\circlearrowright$}}}

\newcommand\ct[1]{\text{\rmfamily\upshape #1}}
\newcommand\question[1]{ {\color{red} ...!? \small #1}}
\newcommand\caz[1]{\left\{\begin{array} #1 \end{array}\right.}
\newcommand\const{\text{\rmfamily\upshape const}}
\newcommand\toP{ \overset{\pro}{\to}}
\newcommand\toPP{ \overset{\text{PP}}{\to}}
\newcommand{\oeq}{\mathrel{\text{\textcircled{$=$}}}}





\usepackage{xcolor}
% \usepackage[normalem]{ulem}
\usepackage{lipsum}
\makeatletter
% \newcommand\colorwave[1][blue]{\bgroup \markoverwith{\lower3.5\p@\hbox{\sixly \textcolor{#1}{\char58}}}\ULon}
%\font\sixly=lasy6 % does not re-load if already loaded, so no memory problem.

\newmdtheoremenv[
linewidth= 1pt,linecolor= blue,%
leftmargin=20,rightmargin=20,innertopmargin=0pt, innerrightmargin=40,%
tikzsetting = { draw=lightgray, line width = 0.3pt,dashed,%
dash pattern = on 15pt off 3pt},%
splittopskip=\topskip,skipbelow=\baselineskip,%
skipabove=\baselineskip,ntheorem,roundcorner=0pt,
% backgroundcolor=pagebg,font=\color{orange}\sffamily, fontcolor=white
]{examplebox}{Exemple}[section]



\newcommand\R{\mathbb{R}}
\newcommand\Z{\mathbb{Z}}
\newcommand\N{\mathbb{N}}
\newcommand\E{\mathbb{E}}
\newcommand\F{\mathcal{F}}
\newcommand\cH{\mathcal{H}}
\newcommand\V{\mathbb{V}}
\newcommand\dmo{ ^{-1} }
\newcommand\kapa{\kappa}
\newcommand\im{Im}
\newcommand\hs{\mathcal{H}}





\usepackage{soul}

\makeatletter
\newcommand*{\whiten}[1]{\llap{\textcolor{white}{{\the\SOUL@token}}\hspace{#1pt}}}
\DeclareRobustCommand*\myul{%
    \def\SOUL@everyspace{\underline{\space}\kern\z@}%
    \def\SOUL@everytoken{%
     \setbox0=\hbox{\the\SOUL@token}%
     \ifdim\dp0>\z@
        \raisebox{\dp0}{\underline{\phantom{\the\SOUL@token}}}%
        \whiten{1}\whiten{0}%
        \whiten{-1}\whiten{-2}%
        \llap{\the\SOUL@token}%
     \else
        \underline{\the\SOUL@token}%
     \fi}%
\SOUL@}
\makeatother

\newcommand*{\demp}{\fontfamily{lmtt}\selectfont}

\DeclareTextFontCommand{\textdemp}{\demp}

\begin{document}

\ifcomment
Multiline
comment
\fi
\ifcomment
\myul{Typesetting test}
% \color[rgb]{1,1,1}
$∑_i^n≠ 60º±∞π∆¬≈√j∫h≤≥µ$

$\CR \R\pro\ind\pro\gS\pro
\mqty[a&b\\c&d]$
$\pro\mathbb{P}$
$\dd{x}$

  \[
    \alpha(x)=\left\{
                \begin{array}{ll}
                  x\\
                  \frac{1}{1+e^{-kx}}\\
                  \frac{e^x-e^{-x}}{e^x+e^{-x}}
                \end{array}
              \right.
  \]

  $\expval{x}$
  
  $\chi_\rho(ghg\dmo)=\Tr(\rho_{ghg\dmo})=\Tr(\rho_g\circ\rho_h\circ\rho\dmo_g)=\Tr(\rho_h)\overset{\mbox{\scalebox{0.5}{$\Tr(AB)=\Tr(BA)$}}}{=}\chi_\rho(h)$
  	$\mathop{\oplus}_{\substack{x\in X}}$

$\mat(\rho_g)=(a_{ij}(g))_{\scriptsize \substack{1\leq i\leq d \\ 1\leq j\leq d}}$ et $\mat(\rho'_g)=(a'_{ij}(g))_{\scriptsize \substack{1\leq i'\leq d' \\ 1\leq j'\leq d'}}$



\[\int_a^b{\mathbb{R}^2}g(u, v)\dd{P_{XY}}(u, v)=\iint g(u,v) f_{XY}(u, v)\dd \lambda(u) \dd \lambda(v)\]
$$\lim_{x\to\infty} f(x)$$	
$$\iiiint_V \mu(t,u,v,w) \,dt\,du\,dv\,dw$$
$$\sum_{n=1}^{\infty} 2^{-n} = 1$$	
\begin{definition}
	Si $X$ et $Y$ sont 2 v.a. ou definit la \textsc{Covariance} entre $X$ et $Y$ comme
	$\cov(X,Y)\overset{\text{def}}{=}\E\left[(X-\E(X))(Y-\E(Y))\right]=\E(XY)-\E(X)\E(Y)$.
\end{definition}
\fi
\pagebreak

% \tableofcontents

% insert your code here
%% !TEX encoding = UTF-8 Unicode
% !TEX TS-program = xelatex

\documentclass[french]{report}

%\usepackage[utf8]{inputenc}
%\usepackage[T1]{fontenc}
\usepackage{babel}


\newif\ifcomment
%\commenttrue # Show comments

\usepackage{physics}
\usepackage{amssymb}


\usepackage{amsthm}
% \usepackage{thmtools}
\usepackage{mathtools}
\usepackage{amsfonts}

\usepackage{color}

\usepackage{tikz}

\usepackage{geometry}
\geometry{a5paper, margin=0.1in, right=1cm}

\usepackage{dsfont}

\usepackage{graphicx}
\graphicspath{ {images/} }

\usepackage{faktor}

\usepackage{IEEEtrantools}
\usepackage{enumerate}   
\usepackage[PostScript=dvips]{"/Users/aware/Documents/Courses/diagrams"}


\newtheorem{theorem}{Théorème}[section]
\renewcommand{\thetheorem}{\arabic{theorem}}
\newtheorem{lemme}{Lemme}[section]
\renewcommand{\thelemme}{\arabic{lemme}}
\newtheorem{proposition}{Proposition}[section]
\renewcommand{\theproposition}{\arabic{proposition}}
\newtheorem{notations}{Notations}[section]
\newtheorem{problem}{Problème}[section]
\newtheorem{corollary}{Corollaire}[theorem]
\renewcommand{\thecorollary}{\arabic{corollary}}
\newtheorem{property}{Propriété}[section]
\newtheorem{objective}{Objectif}[section]

\theoremstyle{definition}
\newtheorem{definition}{Définition}[section]
\renewcommand{\thedefinition}{\arabic{definition}}
\newtheorem{exercise}{Exercice}[chapter]
\renewcommand{\theexercise}{\arabic{exercise}}
\newtheorem{example}{Exemple}[chapter]
\renewcommand{\theexample}{\arabic{example}}
\newtheorem*{solution}{Solution}
\newtheorem*{application}{Application}
\newtheorem*{notation}{Notation}
\newtheorem*{vocabulary}{Vocabulaire}
\newtheorem*{properties}{Propriétés}



\theoremstyle{remark}
\newtheorem*{remark}{Remarque}
\newtheorem*{rappel}{Rappel}


\usepackage{etoolbox}
\AtBeginEnvironment{exercise}{\small}
\AtBeginEnvironment{example}{\small}

\usepackage{cases}
\usepackage[red]{mypack}

\usepackage[framemethod=TikZ]{mdframed}

\definecolor{bg}{rgb}{0.4,0.25,0.95}
\definecolor{pagebg}{rgb}{0,0,0.5}
\surroundwithmdframed[
   topline=false,
   rightline=false,
   bottomline=false,
   leftmargin=\parindent,
   skipabove=8pt,
   skipbelow=8pt,
   linecolor=blue,
   innerbottommargin=10pt,
   % backgroundcolor=bg,font=\color{orange}\sffamily, fontcolor=white
]{definition}

\usepackage{empheq}
\usepackage[most]{tcolorbox}

\newtcbox{\mymath}[1][]{%
    nobeforeafter, math upper, tcbox raise base,
    enhanced, colframe=blue!30!black,
    colback=red!10, boxrule=1pt,
    #1}

\usepackage{unixode}


\DeclareMathOperator{\ord}{ord}
\DeclareMathOperator{\orb}{orb}
\DeclareMathOperator{\stab}{stab}
\DeclareMathOperator{\Stab}{stab}
\DeclareMathOperator{\ppcm}{ppcm}
\DeclareMathOperator{\conj}{Conj}
\DeclareMathOperator{\End}{End}
\DeclareMathOperator{\rot}{rot}
\DeclareMathOperator{\trs}{trace}
\DeclareMathOperator{\Ind}{Ind}
\DeclareMathOperator{\mat}{Mat}
\DeclareMathOperator{\id}{Id}
\DeclareMathOperator{\vect}{vect}
\DeclareMathOperator{\img}{img}
\DeclareMathOperator{\cov}{Cov}
\DeclareMathOperator{\dist}{dist}
\DeclareMathOperator{\irr}{Irr}
\DeclareMathOperator{\image}{Im}
\DeclareMathOperator{\pd}{\partial}
\DeclareMathOperator{\epi}{epi}
\DeclareMathOperator{\Argmin}{Argmin}
\DeclareMathOperator{\dom}{dom}
\DeclareMathOperator{\proj}{proj}
\DeclareMathOperator{\ctg}{ctg}
\DeclareMathOperator{\supp}{supp}
\DeclareMathOperator{\argmin}{argmin}
\DeclareMathOperator{\mult}{mult}
\DeclareMathOperator{\ch}{ch}
\DeclareMathOperator{\sh}{sh}
\DeclareMathOperator{\rang}{rang}
\DeclareMathOperator{\diam}{diam}
\DeclareMathOperator{\Epigraphe}{Epigraphe}




\usepackage{xcolor}
\everymath{\color{blue}}
%\everymath{\color[rgb]{0,1,1}}
%\pagecolor[rgb]{0,0,0.5}


\newcommand*{\pdtest}[3][]{\ensuremath{\frac{\partial^{#1} #2}{\partial #3}}}

\newcommand*{\deffunc}[6][]{\ensuremath{
\begin{array}{rcl}
#2 : #3 &\rightarrow& #4\\
#5 &\mapsto& #6
\end{array}
}}

\newcommand{\eqcolon}{\mathrel{\resizebox{\widthof{$\mathord{=}$}}{\height}{ $\!\!=\!\!\resizebox{1.2\width}{0.8\height}{\raisebox{0.23ex}{$\mathop{:}$}}\!\!$ }}}
\newcommand{\coloneq}{\mathrel{\resizebox{\widthof{$\mathord{=}$}}{\height}{ $\!\!\resizebox{1.2\width}{0.8\height}{\raisebox{0.23ex}{$\mathop{:}$}}\!\!=\!\!$ }}}
\newcommand{\eqcolonl}{\ensuremath{\mathrel{=\!\!\mathop{:}}}}
\newcommand{\coloneql}{\ensuremath{\mathrel{\mathop{:} \!\! =}}}
\newcommand{\vc}[1]{% inline column vector
  \left(\begin{smallmatrix}#1\end{smallmatrix}\right)%
}
\newcommand{\vr}[1]{% inline row vector
  \begin{smallmatrix}(\,#1\,)\end{smallmatrix}%
}
\makeatletter
\newcommand*{\defeq}{\ =\mathrel{\rlap{%
                     \raisebox{0.3ex}{$\m@th\cdot$}}%
                     \raisebox{-0.3ex}{$\m@th\cdot$}}%
                     }
\makeatother

\newcommand{\mathcircle}[1]{% inline row vector
 \overset{\circ}{#1}
}
\newcommand{\ulim}{% low limit
 \underline{\lim}
}
\newcommand{\ssi}{% iff
\iff
}
\newcommand{\ps}[2]{
\expval{#1 | #2}
}
\newcommand{\df}[1]{
\mqty{#1}
}
\newcommand{\n}[1]{
\norm{#1}
}
\newcommand{\sys}[1]{
\left\{\smqty{#1}\right.
}


\newcommand{\eqdef}{\ensuremath{\overset{\text{def}}=}}


\def\Circlearrowright{\ensuremath{%
  \rotatebox[origin=c]{230}{$\circlearrowright$}}}

\newcommand\ct[1]{\text{\rmfamily\upshape #1}}
\newcommand\question[1]{ {\color{red} ...!? \small #1}}
\newcommand\caz[1]{\left\{\begin{array} #1 \end{array}\right.}
\newcommand\const{\text{\rmfamily\upshape const}}
\newcommand\toP{ \overset{\pro}{\to}}
\newcommand\toPP{ \overset{\text{PP}}{\to}}
\newcommand{\oeq}{\mathrel{\text{\textcircled{$=$}}}}





\usepackage{xcolor}
% \usepackage[normalem]{ulem}
\usepackage{lipsum}
\makeatletter
% \newcommand\colorwave[1][blue]{\bgroup \markoverwith{\lower3.5\p@\hbox{\sixly \textcolor{#1}{\char58}}}\ULon}
%\font\sixly=lasy6 % does not re-load if already loaded, so no memory problem.

\newmdtheoremenv[
linewidth= 1pt,linecolor= blue,%
leftmargin=20,rightmargin=20,innertopmargin=0pt, innerrightmargin=40,%
tikzsetting = { draw=lightgray, line width = 0.3pt,dashed,%
dash pattern = on 15pt off 3pt},%
splittopskip=\topskip,skipbelow=\baselineskip,%
skipabove=\baselineskip,ntheorem,roundcorner=0pt,
% backgroundcolor=pagebg,font=\color{orange}\sffamily, fontcolor=white
]{examplebox}{Exemple}[section]



\newcommand\R{\mathbb{R}}
\newcommand\Z{\mathbb{Z}}
\newcommand\N{\mathbb{N}}
\newcommand\E{\mathbb{E}}
\newcommand\F{\mathcal{F}}
\newcommand\cH{\mathcal{H}}
\newcommand\V{\mathbb{V}}
\newcommand\dmo{ ^{-1} }
\newcommand\kapa{\kappa}
\newcommand\im{Im}
\newcommand\hs{\mathcal{H}}





\usepackage{soul}

\makeatletter
\newcommand*{\whiten}[1]{\llap{\textcolor{white}{{\the\SOUL@token}}\hspace{#1pt}}}
\DeclareRobustCommand*\myul{%
    \def\SOUL@everyspace{\underline{\space}\kern\z@}%
    \def\SOUL@everytoken{%
     \setbox0=\hbox{\the\SOUL@token}%
     \ifdim\dp0>\z@
        \raisebox{\dp0}{\underline{\phantom{\the\SOUL@token}}}%
        \whiten{1}\whiten{0}%
        \whiten{-1}\whiten{-2}%
        \llap{\the\SOUL@token}%
     \else
        \underline{\the\SOUL@token}%
     \fi}%
\SOUL@}
\makeatother

\newcommand*{\demp}{\fontfamily{lmtt}\selectfont}

\DeclareTextFontCommand{\textdemp}{\demp}

\begin{document}

\ifcomment
Multiline
comment
\fi
\ifcomment
\myul{Typesetting test}
% \color[rgb]{1,1,1}
$∑_i^n≠ 60º±∞π∆¬≈√j∫h≤≥µ$

$\CR \R\pro\ind\pro\gS\pro
\mqty[a&b\\c&d]$
$\pro\mathbb{P}$
$\dd{x}$

  \[
    \alpha(x)=\left\{
                \begin{array}{ll}
                  x\\
                  \frac{1}{1+e^{-kx}}\\
                  \frac{e^x-e^{-x}}{e^x+e^{-x}}
                \end{array}
              \right.
  \]

  $\expval{x}$
  
  $\chi_\rho(ghg\dmo)=\Tr(\rho_{ghg\dmo})=\Tr(\rho_g\circ\rho_h\circ\rho\dmo_g)=\Tr(\rho_h)\overset{\mbox{\scalebox{0.5}{$\Tr(AB)=\Tr(BA)$}}}{=}\chi_\rho(h)$
  	$\mathop{\oplus}_{\substack{x\in X}}$

$\mat(\rho_g)=(a_{ij}(g))_{\scriptsize \substack{1\leq i\leq d \\ 1\leq j\leq d}}$ et $\mat(\rho'_g)=(a'_{ij}(g))_{\scriptsize \substack{1\leq i'\leq d' \\ 1\leq j'\leq d'}}$



\[\int_a^b{\mathbb{R}^2}g(u, v)\dd{P_{XY}}(u, v)=\iint g(u,v) f_{XY}(u, v)\dd \lambda(u) \dd \lambda(v)\]
$$\lim_{x\to\infty} f(x)$$	
$$\iiiint_V \mu(t,u,v,w) \,dt\,du\,dv\,dw$$
$$\sum_{n=1}^{\infty} 2^{-n} = 1$$	
\begin{definition}
	Si $X$ et $Y$ sont 2 v.a. ou definit la \textsc{Covariance} entre $X$ et $Y$ comme
	$\cov(X,Y)\overset{\text{def}}{=}\E\left[(X-\E(X))(Y-\E(Y))\right]=\E(XY)-\E(X)\E(Y)$.
\end{definition}
\fi
\pagebreak

% \tableofcontents

% insert your code here
%% !TEX encoding = UTF-8 Unicode
% !TEX TS-program = xelatex

\documentclass[french]{report}

%\usepackage[utf8]{inputenc}
%\usepackage[T1]{fontenc}
\usepackage{babel}


\newif\ifcomment
%\commenttrue # Show comments

\usepackage{physics}
\usepackage{amssymb}


\usepackage{amsthm}
% \usepackage{thmtools}
\usepackage{mathtools}
\usepackage{amsfonts}

\usepackage{color}

\usepackage{tikz}

\usepackage{geometry}
\geometry{a5paper, margin=0.1in, right=1cm}

\usepackage{dsfont}

\usepackage{graphicx}
\graphicspath{ {images/} }

\usepackage{faktor}

\usepackage{IEEEtrantools}
\usepackage{enumerate}   
\usepackage[PostScript=dvips]{"/Users/aware/Documents/Courses/diagrams"}


\newtheorem{theorem}{Théorème}[section]
\renewcommand{\thetheorem}{\arabic{theorem}}
\newtheorem{lemme}{Lemme}[section]
\renewcommand{\thelemme}{\arabic{lemme}}
\newtheorem{proposition}{Proposition}[section]
\renewcommand{\theproposition}{\arabic{proposition}}
\newtheorem{notations}{Notations}[section]
\newtheorem{problem}{Problème}[section]
\newtheorem{corollary}{Corollaire}[theorem]
\renewcommand{\thecorollary}{\arabic{corollary}}
\newtheorem{property}{Propriété}[section]
\newtheorem{objective}{Objectif}[section]

\theoremstyle{definition}
\newtheorem{definition}{Définition}[section]
\renewcommand{\thedefinition}{\arabic{definition}}
\newtheorem{exercise}{Exercice}[chapter]
\renewcommand{\theexercise}{\arabic{exercise}}
\newtheorem{example}{Exemple}[chapter]
\renewcommand{\theexample}{\arabic{example}}
\newtheorem*{solution}{Solution}
\newtheorem*{application}{Application}
\newtheorem*{notation}{Notation}
\newtheorem*{vocabulary}{Vocabulaire}
\newtheorem*{properties}{Propriétés}



\theoremstyle{remark}
\newtheorem*{remark}{Remarque}
\newtheorem*{rappel}{Rappel}


\usepackage{etoolbox}
\AtBeginEnvironment{exercise}{\small}
\AtBeginEnvironment{example}{\small}

\usepackage{cases}
\usepackage[red]{mypack}

\usepackage[framemethod=TikZ]{mdframed}

\definecolor{bg}{rgb}{0.4,0.25,0.95}
\definecolor{pagebg}{rgb}{0,0,0.5}
\surroundwithmdframed[
   topline=false,
   rightline=false,
   bottomline=false,
   leftmargin=\parindent,
   skipabove=8pt,
   skipbelow=8pt,
   linecolor=blue,
   innerbottommargin=10pt,
   % backgroundcolor=bg,font=\color{orange}\sffamily, fontcolor=white
]{definition}

\usepackage{empheq}
\usepackage[most]{tcolorbox}

\newtcbox{\mymath}[1][]{%
    nobeforeafter, math upper, tcbox raise base,
    enhanced, colframe=blue!30!black,
    colback=red!10, boxrule=1pt,
    #1}

\usepackage{unixode}


\DeclareMathOperator{\ord}{ord}
\DeclareMathOperator{\orb}{orb}
\DeclareMathOperator{\stab}{stab}
\DeclareMathOperator{\Stab}{stab}
\DeclareMathOperator{\ppcm}{ppcm}
\DeclareMathOperator{\conj}{Conj}
\DeclareMathOperator{\End}{End}
\DeclareMathOperator{\rot}{rot}
\DeclareMathOperator{\trs}{trace}
\DeclareMathOperator{\Ind}{Ind}
\DeclareMathOperator{\mat}{Mat}
\DeclareMathOperator{\id}{Id}
\DeclareMathOperator{\vect}{vect}
\DeclareMathOperator{\img}{img}
\DeclareMathOperator{\cov}{Cov}
\DeclareMathOperator{\dist}{dist}
\DeclareMathOperator{\irr}{Irr}
\DeclareMathOperator{\image}{Im}
\DeclareMathOperator{\pd}{\partial}
\DeclareMathOperator{\epi}{epi}
\DeclareMathOperator{\Argmin}{Argmin}
\DeclareMathOperator{\dom}{dom}
\DeclareMathOperator{\proj}{proj}
\DeclareMathOperator{\ctg}{ctg}
\DeclareMathOperator{\supp}{supp}
\DeclareMathOperator{\argmin}{argmin}
\DeclareMathOperator{\mult}{mult}
\DeclareMathOperator{\ch}{ch}
\DeclareMathOperator{\sh}{sh}
\DeclareMathOperator{\rang}{rang}
\DeclareMathOperator{\diam}{diam}
\DeclareMathOperator{\Epigraphe}{Epigraphe}




\usepackage{xcolor}
\everymath{\color{blue}}
%\everymath{\color[rgb]{0,1,1}}
%\pagecolor[rgb]{0,0,0.5}


\newcommand*{\pdtest}[3][]{\ensuremath{\frac{\partial^{#1} #2}{\partial #3}}}

\newcommand*{\deffunc}[6][]{\ensuremath{
\begin{array}{rcl}
#2 : #3 &\rightarrow& #4\\
#5 &\mapsto& #6
\end{array}
}}

\newcommand{\eqcolon}{\mathrel{\resizebox{\widthof{$\mathord{=}$}}{\height}{ $\!\!=\!\!\resizebox{1.2\width}{0.8\height}{\raisebox{0.23ex}{$\mathop{:}$}}\!\!$ }}}
\newcommand{\coloneq}{\mathrel{\resizebox{\widthof{$\mathord{=}$}}{\height}{ $\!\!\resizebox{1.2\width}{0.8\height}{\raisebox{0.23ex}{$\mathop{:}$}}\!\!=\!\!$ }}}
\newcommand{\eqcolonl}{\ensuremath{\mathrel{=\!\!\mathop{:}}}}
\newcommand{\coloneql}{\ensuremath{\mathrel{\mathop{:} \!\! =}}}
\newcommand{\vc}[1]{% inline column vector
  \left(\begin{smallmatrix}#1\end{smallmatrix}\right)%
}
\newcommand{\vr}[1]{% inline row vector
  \begin{smallmatrix}(\,#1\,)\end{smallmatrix}%
}
\makeatletter
\newcommand*{\defeq}{\ =\mathrel{\rlap{%
                     \raisebox{0.3ex}{$\m@th\cdot$}}%
                     \raisebox{-0.3ex}{$\m@th\cdot$}}%
                     }
\makeatother

\newcommand{\mathcircle}[1]{% inline row vector
 \overset{\circ}{#1}
}
\newcommand{\ulim}{% low limit
 \underline{\lim}
}
\newcommand{\ssi}{% iff
\iff
}
\newcommand{\ps}[2]{
\expval{#1 | #2}
}
\newcommand{\df}[1]{
\mqty{#1}
}
\newcommand{\n}[1]{
\norm{#1}
}
\newcommand{\sys}[1]{
\left\{\smqty{#1}\right.
}


\newcommand{\eqdef}{\ensuremath{\overset{\text{def}}=}}


\def\Circlearrowright{\ensuremath{%
  \rotatebox[origin=c]{230}{$\circlearrowright$}}}

\newcommand\ct[1]{\text{\rmfamily\upshape #1}}
\newcommand\question[1]{ {\color{red} ...!? \small #1}}
\newcommand\caz[1]{\left\{\begin{array} #1 \end{array}\right.}
\newcommand\const{\text{\rmfamily\upshape const}}
\newcommand\toP{ \overset{\pro}{\to}}
\newcommand\toPP{ \overset{\text{PP}}{\to}}
\newcommand{\oeq}{\mathrel{\text{\textcircled{$=$}}}}





\usepackage{xcolor}
% \usepackage[normalem]{ulem}
\usepackage{lipsum}
\makeatletter
% \newcommand\colorwave[1][blue]{\bgroup \markoverwith{\lower3.5\p@\hbox{\sixly \textcolor{#1}{\char58}}}\ULon}
%\font\sixly=lasy6 % does not re-load if already loaded, so no memory problem.

\newmdtheoremenv[
linewidth= 1pt,linecolor= blue,%
leftmargin=20,rightmargin=20,innertopmargin=0pt, innerrightmargin=40,%
tikzsetting = { draw=lightgray, line width = 0.3pt,dashed,%
dash pattern = on 15pt off 3pt},%
splittopskip=\topskip,skipbelow=\baselineskip,%
skipabove=\baselineskip,ntheorem,roundcorner=0pt,
% backgroundcolor=pagebg,font=\color{orange}\sffamily, fontcolor=white
]{examplebox}{Exemple}[section]



\newcommand\R{\mathbb{R}}
\newcommand\Z{\mathbb{Z}}
\newcommand\N{\mathbb{N}}
\newcommand\E{\mathbb{E}}
\newcommand\F{\mathcal{F}}
\newcommand\cH{\mathcal{H}}
\newcommand\V{\mathbb{V}}
\newcommand\dmo{ ^{-1} }
\newcommand\kapa{\kappa}
\newcommand\im{Im}
\newcommand\hs{\mathcal{H}}





\usepackage{soul}

\makeatletter
\newcommand*{\whiten}[1]{\llap{\textcolor{white}{{\the\SOUL@token}}\hspace{#1pt}}}
\DeclareRobustCommand*\myul{%
    \def\SOUL@everyspace{\underline{\space}\kern\z@}%
    \def\SOUL@everytoken{%
     \setbox0=\hbox{\the\SOUL@token}%
     \ifdim\dp0>\z@
        \raisebox{\dp0}{\underline{\phantom{\the\SOUL@token}}}%
        \whiten{1}\whiten{0}%
        \whiten{-1}\whiten{-2}%
        \llap{\the\SOUL@token}%
     \else
        \underline{\the\SOUL@token}%
     \fi}%
\SOUL@}
\makeatother

\newcommand*{\demp}{\fontfamily{lmtt}\selectfont}

\DeclareTextFontCommand{\textdemp}{\demp}

\begin{document}

\ifcomment
Multiline
comment
\fi
\ifcomment
\myul{Typesetting test}
% \color[rgb]{1,1,1}
$∑_i^n≠ 60º±∞π∆¬≈√j∫h≤≥µ$

$\CR \R\pro\ind\pro\gS\pro
\mqty[a&b\\c&d]$
$\pro\mathbb{P}$
$\dd{x}$

  \[
    \alpha(x)=\left\{
                \begin{array}{ll}
                  x\\
                  \frac{1}{1+e^{-kx}}\\
                  \frac{e^x-e^{-x}}{e^x+e^{-x}}
                \end{array}
              \right.
  \]

  $\expval{x}$
  
  $\chi_\rho(ghg\dmo)=\Tr(\rho_{ghg\dmo})=\Tr(\rho_g\circ\rho_h\circ\rho\dmo_g)=\Tr(\rho_h)\overset{\mbox{\scalebox{0.5}{$\Tr(AB)=\Tr(BA)$}}}{=}\chi_\rho(h)$
  	$\mathop{\oplus}_{\substack{x\in X}}$

$\mat(\rho_g)=(a_{ij}(g))_{\scriptsize \substack{1\leq i\leq d \\ 1\leq j\leq d}}$ et $\mat(\rho'_g)=(a'_{ij}(g))_{\scriptsize \substack{1\leq i'\leq d' \\ 1\leq j'\leq d'}}$



\[\int_a^b{\mathbb{R}^2}g(u, v)\dd{P_{XY}}(u, v)=\iint g(u,v) f_{XY}(u, v)\dd \lambda(u) \dd \lambda(v)\]
$$\lim_{x\to\infty} f(x)$$	
$$\iiiint_V \mu(t,u,v,w) \,dt\,du\,dv\,dw$$
$$\sum_{n=1}^{\infty} 2^{-n} = 1$$	
\begin{definition}
	Si $X$ et $Y$ sont 2 v.a. ou definit la \textsc{Covariance} entre $X$ et $Y$ comme
	$\cov(X,Y)\overset{\text{def}}{=}\E\left[(X-\E(X))(Y-\E(Y))\right]=\E(XY)-\E(X)\E(Y)$.
\end{definition}
\fi
\pagebreak

% \tableofcontents

% insert your code here
%\input{./algebra/main.tex}
%\input{./geometrie-differentielle/main.tex}
%\input{./probabilite/main.tex}
%\input{./analyse-fonctionnelle/main.tex}
% \input{./Analyse-convexe-et-dualite-en-optimisation/main.tex}
%\input{./tikz/main.tex}
%\input{./Theorie-du-distributions/main.tex}
%\input{./optimisation/mine.tex}
 \input{./modelisation/main.tex}

% yves.aubry@univ-tln.fr : algebra

\end{document}

%% !TEX encoding = UTF-8 Unicode
% !TEX TS-program = xelatex

\documentclass[french]{report}

%\usepackage[utf8]{inputenc}
%\usepackage[T1]{fontenc}
\usepackage{babel}


\newif\ifcomment
%\commenttrue # Show comments

\usepackage{physics}
\usepackage{amssymb}


\usepackage{amsthm}
% \usepackage{thmtools}
\usepackage{mathtools}
\usepackage{amsfonts}

\usepackage{color}

\usepackage{tikz}

\usepackage{geometry}
\geometry{a5paper, margin=0.1in, right=1cm}

\usepackage{dsfont}

\usepackage{graphicx}
\graphicspath{ {images/} }

\usepackage{faktor}

\usepackage{IEEEtrantools}
\usepackage{enumerate}   
\usepackage[PostScript=dvips]{"/Users/aware/Documents/Courses/diagrams"}


\newtheorem{theorem}{Théorème}[section]
\renewcommand{\thetheorem}{\arabic{theorem}}
\newtheorem{lemme}{Lemme}[section]
\renewcommand{\thelemme}{\arabic{lemme}}
\newtheorem{proposition}{Proposition}[section]
\renewcommand{\theproposition}{\arabic{proposition}}
\newtheorem{notations}{Notations}[section]
\newtheorem{problem}{Problème}[section]
\newtheorem{corollary}{Corollaire}[theorem]
\renewcommand{\thecorollary}{\arabic{corollary}}
\newtheorem{property}{Propriété}[section]
\newtheorem{objective}{Objectif}[section]

\theoremstyle{definition}
\newtheorem{definition}{Définition}[section]
\renewcommand{\thedefinition}{\arabic{definition}}
\newtheorem{exercise}{Exercice}[chapter]
\renewcommand{\theexercise}{\arabic{exercise}}
\newtheorem{example}{Exemple}[chapter]
\renewcommand{\theexample}{\arabic{example}}
\newtheorem*{solution}{Solution}
\newtheorem*{application}{Application}
\newtheorem*{notation}{Notation}
\newtheorem*{vocabulary}{Vocabulaire}
\newtheorem*{properties}{Propriétés}



\theoremstyle{remark}
\newtheorem*{remark}{Remarque}
\newtheorem*{rappel}{Rappel}


\usepackage{etoolbox}
\AtBeginEnvironment{exercise}{\small}
\AtBeginEnvironment{example}{\small}

\usepackage{cases}
\usepackage[red]{mypack}

\usepackage[framemethod=TikZ]{mdframed}

\definecolor{bg}{rgb}{0.4,0.25,0.95}
\definecolor{pagebg}{rgb}{0,0,0.5}
\surroundwithmdframed[
   topline=false,
   rightline=false,
   bottomline=false,
   leftmargin=\parindent,
   skipabove=8pt,
   skipbelow=8pt,
   linecolor=blue,
   innerbottommargin=10pt,
   % backgroundcolor=bg,font=\color{orange}\sffamily, fontcolor=white
]{definition}

\usepackage{empheq}
\usepackage[most]{tcolorbox}

\newtcbox{\mymath}[1][]{%
    nobeforeafter, math upper, tcbox raise base,
    enhanced, colframe=blue!30!black,
    colback=red!10, boxrule=1pt,
    #1}

\usepackage{unixode}


\DeclareMathOperator{\ord}{ord}
\DeclareMathOperator{\orb}{orb}
\DeclareMathOperator{\stab}{stab}
\DeclareMathOperator{\Stab}{stab}
\DeclareMathOperator{\ppcm}{ppcm}
\DeclareMathOperator{\conj}{Conj}
\DeclareMathOperator{\End}{End}
\DeclareMathOperator{\rot}{rot}
\DeclareMathOperator{\trs}{trace}
\DeclareMathOperator{\Ind}{Ind}
\DeclareMathOperator{\mat}{Mat}
\DeclareMathOperator{\id}{Id}
\DeclareMathOperator{\vect}{vect}
\DeclareMathOperator{\img}{img}
\DeclareMathOperator{\cov}{Cov}
\DeclareMathOperator{\dist}{dist}
\DeclareMathOperator{\irr}{Irr}
\DeclareMathOperator{\image}{Im}
\DeclareMathOperator{\pd}{\partial}
\DeclareMathOperator{\epi}{epi}
\DeclareMathOperator{\Argmin}{Argmin}
\DeclareMathOperator{\dom}{dom}
\DeclareMathOperator{\proj}{proj}
\DeclareMathOperator{\ctg}{ctg}
\DeclareMathOperator{\supp}{supp}
\DeclareMathOperator{\argmin}{argmin}
\DeclareMathOperator{\mult}{mult}
\DeclareMathOperator{\ch}{ch}
\DeclareMathOperator{\sh}{sh}
\DeclareMathOperator{\rang}{rang}
\DeclareMathOperator{\diam}{diam}
\DeclareMathOperator{\Epigraphe}{Epigraphe}




\usepackage{xcolor}
\everymath{\color{blue}}
%\everymath{\color[rgb]{0,1,1}}
%\pagecolor[rgb]{0,0,0.5}


\newcommand*{\pdtest}[3][]{\ensuremath{\frac{\partial^{#1} #2}{\partial #3}}}

\newcommand*{\deffunc}[6][]{\ensuremath{
\begin{array}{rcl}
#2 : #3 &\rightarrow& #4\\
#5 &\mapsto& #6
\end{array}
}}

\newcommand{\eqcolon}{\mathrel{\resizebox{\widthof{$\mathord{=}$}}{\height}{ $\!\!=\!\!\resizebox{1.2\width}{0.8\height}{\raisebox{0.23ex}{$\mathop{:}$}}\!\!$ }}}
\newcommand{\coloneq}{\mathrel{\resizebox{\widthof{$\mathord{=}$}}{\height}{ $\!\!\resizebox{1.2\width}{0.8\height}{\raisebox{0.23ex}{$\mathop{:}$}}\!\!=\!\!$ }}}
\newcommand{\eqcolonl}{\ensuremath{\mathrel{=\!\!\mathop{:}}}}
\newcommand{\coloneql}{\ensuremath{\mathrel{\mathop{:} \!\! =}}}
\newcommand{\vc}[1]{% inline column vector
  \left(\begin{smallmatrix}#1\end{smallmatrix}\right)%
}
\newcommand{\vr}[1]{% inline row vector
  \begin{smallmatrix}(\,#1\,)\end{smallmatrix}%
}
\makeatletter
\newcommand*{\defeq}{\ =\mathrel{\rlap{%
                     \raisebox{0.3ex}{$\m@th\cdot$}}%
                     \raisebox{-0.3ex}{$\m@th\cdot$}}%
                     }
\makeatother

\newcommand{\mathcircle}[1]{% inline row vector
 \overset{\circ}{#1}
}
\newcommand{\ulim}{% low limit
 \underline{\lim}
}
\newcommand{\ssi}{% iff
\iff
}
\newcommand{\ps}[2]{
\expval{#1 | #2}
}
\newcommand{\df}[1]{
\mqty{#1}
}
\newcommand{\n}[1]{
\norm{#1}
}
\newcommand{\sys}[1]{
\left\{\smqty{#1}\right.
}


\newcommand{\eqdef}{\ensuremath{\overset{\text{def}}=}}


\def\Circlearrowright{\ensuremath{%
  \rotatebox[origin=c]{230}{$\circlearrowright$}}}

\newcommand\ct[1]{\text{\rmfamily\upshape #1}}
\newcommand\question[1]{ {\color{red} ...!? \small #1}}
\newcommand\caz[1]{\left\{\begin{array} #1 \end{array}\right.}
\newcommand\const{\text{\rmfamily\upshape const}}
\newcommand\toP{ \overset{\pro}{\to}}
\newcommand\toPP{ \overset{\text{PP}}{\to}}
\newcommand{\oeq}{\mathrel{\text{\textcircled{$=$}}}}





\usepackage{xcolor}
% \usepackage[normalem]{ulem}
\usepackage{lipsum}
\makeatletter
% \newcommand\colorwave[1][blue]{\bgroup \markoverwith{\lower3.5\p@\hbox{\sixly \textcolor{#1}{\char58}}}\ULon}
%\font\sixly=lasy6 % does not re-load if already loaded, so no memory problem.

\newmdtheoremenv[
linewidth= 1pt,linecolor= blue,%
leftmargin=20,rightmargin=20,innertopmargin=0pt, innerrightmargin=40,%
tikzsetting = { draw=lightgray, line width = 0.3pt,dashed,%
dash pattern = on 15pt off 3pt},%
splittopskip=\topskip,skipbelow=\baselineskip,%
skipabove=\baselineskip,ntheorem,roundcorner=0pt,
% backgroundcolor=pagebg,font=\color{orange}\sffamily, fontcolor=white
]{examplebox}{Exemple}[section]



\newcommand\R{\mathbb{R}}
\newcommand\Z{\mathbb{Z}}
\newcommand\N{\mathbb{N}}
\newcommand\E{\mathbb{E}}
\newcommand\F{\mathcal{F}}
\newcommand\cH{\mathcal{H}}
\newcommand\V{\mathbb{V}}
\newcommand\dmo{ ^{-1} }
\newcommand\kapa{\kappa}
\newcommand\im{Im}
\newcommand\hs{\mathcal{H}}





\usepackage{soul}

\makeatletter
\newcommand*{\whiten}[1]{\llap{\textcolor{white}{{\the\SOUL@token}}\hspace{#1pt}}}
\DeclareRobustCommand*\myul{%
    \def\SOUL@everyspace{\underline{\space}\kern\z@}%
    \def\SOUL@everytoken{%
     \setbox0=\hbox{\the\SOUL@token}%
     \ifdim\dp0>\z@
        \raisebox{\dp0}{\underline{\phantom{\the\SOUL@token}}}%
        \whiten{1}\whiten{0}%
        \whiten{-1}\whiten{-2}%
        \llap{\the\SOUL@token}%
     \else
        \underline{\the\SOUL@token}%
     \fi}%
\SOUL@}
\makeatother

\newcommand*{\demp}{\fontfamily{lmtt}\selectfont}

\DeclareTextFontCommand{\textdemp}{\demp}

\begin{document}

\ifcomment
Multiline
comment
\fi
\ifcomment
\myul{Typesetting test}
% \color[rgb]{1,1,1}
$∑_i^n≠ 60º±∞π∆¬≈√j∫h≤≥µ$

$\CR \R\pro\ind\pro\gS\pro
\mqty[a&b\\c&d]$
$\pro\mathbb{P}$
$\dd{x}$

  \[
    \alpha(x)=\left\{
                \begin{array}{ll}
                  x\\
                  \frac{1}{1+e^{-kx}}\\
                  \frac{e^x-e^{-x}}{e^x+e^{-x}}
                \end{array}
              \right.
  \]

  $\expval{x}$
  
  $\chi_\rho(ghg\dmo)=\Tr(\rho_{ghg\dmo})=\Tr(\rho_g\circ\rho_h\circ\rho\dmo_g)=\Tr(\rho_h)\overset{\mbox{\scalebox{0.5}{$\Tr(AB)=\Tr(BA)$}}}{=}\chi_\rho(h)$
  	$\mathop{\oplus}_{\substack{x\in X}}$

$\mat(\rho_g)=(a_{ij}(g))_{\scriptsize \substack{1\leq i\leq d \\ 1\leq j\leq d}}$ et $\mat(\rho'_g)=(a'_{ij}(g))_{\scriptsize \substack{1\leq i'\leq d' \\ 1\leq j'\leq d'}}$



\[\int_a^b{\mathbb{R}^2}g(u, v)\dd{P_{XY}}(u, v)=\iint g(u,v) f_{XY}(u, v)\dd \lambda(u) \dd \lambda(v)\]
$$\lim_{x\to\infty} f(x)$$	
$$\iiiint_V \mu(t,u,v,w) \,dt\,du\,dv\,dw$$
$$\sum_{n=1}^{\infty} 2^{-n} = 1$$	
\begin{definition}
	Si $X$ et $Y$ sont 2 v.a. ou definit la \textsc{Covariance} entre $X$ et $Y$ comme
	$\cov(X,Y)\overset{\text{def}}{=}\E\left[(X-\E(X))(Y-\E(Y))\right]=\E(XY)-\E(X)\E(Y)$.
\end{definition}
\fi
\pagebreak

% \tableofcontents

% insert your code here
%\input{./algebra/main.tex}
%\input{./geometrie-differentielle/main.tex}
%\input{./probabilite/main.tex}
%\input{./analyse-fonctionnelle/main.tex}
% \input{./Analyse-convexe-et-dualite-en-optimisation/main.tex}
%\input{./tikz/main.tex}
%\input{./Theorie-du-distributions/main.tex}
%\input{./optimisation/mine.tex}
 \input{./modelisation/main.tex}

% yves.aubry@univ-tln.fr : algebra

\end{document}

%% !TEX encoding = UTF-8 Unicode
% !TEX TS-program = xelatex

\documentclass[french]{report}

%\usepackage[utf8]{inputenc}
%\usepackage[T1]{fontenc}
\usepackage{babel}


\newif\ifcomment
%\commenttrue # Show comments

\usepackage{physics}
\usepackage{amssymb}


\usepackage{amsthm}
% \usepackage{thmtools}
\usepackage{mathtools}
\usepackage{amsfonts}

\usepackage{color}

\usepackage{tikz}

\usepackage{geometry}
\geometry{a5paper, margin=0.1in, right=1cm}

\usepackage{dsfont}

\usepackage{graphicx}
\graphicspath{ {images/} }

\usepackage{faktor}

\usepackage{IEEEtrantools}
\usepackage{enumerate}   
\usepackage[PostScript=dvips]{"/Users/aware/Documents/Courses/diagrams"}


\newtheorem{theorem}{Théorème}[section]
\renewcommand{\thetheorem}{\arabic{theorem}}
\newtheorem{lemme}{Lemme}[section]
\renewcommand{\thelemme}{\arabic{lemme}}
\newtheorem{proposition}{Proposition}[section]
\renewcommand{\theproposition}{\arabic{proposition}}
\newtheorem{notations}{Notations}[section]
\newtheorem{problem}{Problème}[section]
\newtheorem{corollary}{Corollaire}[theorem]
\renewcommand{\thecorollary}{\arabic{corollary}}
\newtheorem{property}{Propriété}[section]
\newtheorem{objective}{Objectif}[section]

\theoremstyle{definition}
\newtheorem{definition}{Définition}[section]
\renewcommand{\thedefinition}{\arabic{definition}}
\newtheorem{exercise}{Exercice}[chapter]
\renewcommand{\theexercise}{\arabic{exercise}}
\newtheorem{example}{Exemple}[chapter]
\renewcommand{\theexample}{\arabic{example}}
\newtheorem*{solution}{Solution}
\newtheorem*{application}{Application}
\newtheorem*{notation}{Notation}
\newtheorem*{vocabulary}{Vocabulaire}
\newtheorem*{properties}{Propriétés}



\theoremstyle{remark}
\newtheorem*{remark}{Remarque}
\newtheorem*{rappel}{Rappel}


\usepackage{etoolbox}
\AtBeginEnvironment{exercise}{\small}
\AtBeginEnvironment{example}{\small}

\usepackage{cases}
\usepackage[red]{mypack}

\usepackage[framemethod=TikZ]{mdframed}

\definecolor{bg}{rgb}{0.4,0.25,0.95}
\definecolor{pagebg}{rgb}{0,0,0.5}
\surroundwithmdframed[
   topline=false,
   rightline=false,
   bottomline=false,
   leftmargin=\parindent,
   skipabove=8pt,
   skipbelow=8pt,
   linecolor=blue,
   innerbottommargin=10pt,
   % backgroundcolor=bg,font=\color{orange}\sffamily, fontcolor=white
]{definition}

\usepackage{empheq}
\usepackage[most]{tcolorbox}

\newtcbox{\mymath}[1][]{%
    nobeforeafter, math upper, tcbox raise base,
    enhanced, colframe=blue!30!black,
    colback=red!10, boxrule=1pt,
    #1}

\usepackage{unixode}


\DeclareMathOperator{\ord}{ord}
\DeclareMathOperator{\orb}{orb}
\DeclareMathOperator{\stab}{stab}
\DeclareMathOperator{\Stab}{stab}
\DeclareMathOperator{\ppcm}{ppcm}
\DeclareMathOperator{\conj}{Conj}
\DeclareMathOperator{\End}{End}
\DeclareMathOperator{\rot}{rot}
\DeclareMathOperator{\trs}{trace}
\DeclareMathOperator{\Ind}{Ind}
\DeclareMathOperator{\mat}{Mat}
\DeclareMathOperator{\id}{Id}
\DeclareMathOperator{\vect}{vect}
\DeclareMathOperator{\img}{img}
\DeclareMathOperator{\cov}{Cov}
\DeclareMathOperator{\dist}{dist}
\DeclareMathOperator{\irr}{Irr}
\DeclareMathOperator{\image}{Im}
\DeclareMathOperator{\pd}{\partial}
\DeclareMathOperator{\epi}{epi}
\DeclareMathOperator{\Argmin}{Argmin}
\DeclareMathOperator{\dom}{dom}
\DeclareMathOperator{\proj}{proj}
\DeclareMathOperator{\ctg}{ctg}
\DeclareMathOperator{\supp}{supp}
\DeclareMathOperator{\argmin}{argmin}
\DeclareMathOperator{\mult}{mult}
\DeclareMathOperator{\ch}{ch}
\DeclareMathOperator{\sh}{sh}
\DeclareMathOperator{\rang}{rang}
\DeclareMathOperator{\diam}{diam}
\DeclareMathOperator{\Epigraphe}{Epigraphe}




\usepackage{xcolor}
\everymath{\color{blue}}
%\everymath{\color[rgb]{0,1,1}}
%\pagecolor[rgb]{0,0,0.5}


\newcommand*{\pdtest}[3][]{\ensuremath{\frac{\partial^{#1} #2}{\partial #3}}}

\newcommand*{\deffunc}[6][]{\ensuremath{
\begin{array}{rcl}
#2 : #3 &\rightarrow& #4\\
#5 &\mapsto& #6
\end{array}
}}

\newcommand{\eqcolon}{\mathrel{\resizebox{\widthof{$\mathord{=}$}}{\height}{ $\!\!=\!\!\resizebox{1.2\width}{0.8\height}{\raisebox{0.23ex}{$\mathop{:}$}}\!\!$ }}}
\newcommand{\coloneq}{\mathrel{\resizebox{\widthof{$\mathord{=}$}}{\height}{ $\!\!\resizebox{1.2\width}{0.8\height}{\raisebox{0.23ex}{$\mathop{:}$}}\!\!=\!\!$ }}}
\newcommand{\eqcolonl}{\ensuremath{\mathrel{=\!\!\mathop{:}}}}
\newcommand{\coloneql}{\ensuremath{\mathrel{\mathop{:} \!\! =}}}
\newcommand{\vc}[1]{% inline column vector
  \left(\begin{smallmatrix}#1\end{smallmatrix}\right)%
}
\newcommand{\vr}[1]{% inline row vector
  \begin{smallmatrix}(\,#1\,)\end{smallmatrix}%
}
\makeatletter
\newcommand*{\defeq}{\ =\mathrel{\rlap{%
                     \raisebox{0.3ex}{$\m@th\cdot$}}%
                     \raisebox{-0.3ex}{$\m@th\cdot$}}%
                     }
\makeatother

\newcommand{\mathcircle}[1]{% inline row vector
 \overset{\circ}{#1}
}
\newcommand{\ulim}{% low limit
 \underline{\lim}
}
\newcommand{\ssi}{% iff
\iff
}
\newcommand{\ps}[2]{
\expval{#1 | #2}
}
\newcommand{\df}[1]{
\mqty{#1}
}
\newcommand{\n}[1]{
\norm{#1}
}
\newcommand{\sys}[1]{
\left\{\smqty{#1}\right.
}


\newcommand{\eqdef}{\ensuremath{\overset{\text{def}}=}}


\def\Circlearrowright{\ensuremath{%
  \rotatebox[origin=c]{230}{$\circlearrowright$}}}

\newcommand\ct[1]{\text{\rmfamily\upshape #1}}
\newcommand\question[1]{ {\color{red} ...!? \small #1}}
\newcommand\caz[1]{\left\{\begin{array} #1 \end{array}\right.}
\newcommand\const{\text{\rmfamily\upshape const}}
\newcommand\toP{ \overset{\pro}{\to}}
\newcommand\toPP{ \overset{\text{PP}}{\to}}
\newcommand{\oeq}{\mathrel{\text{\textcircled{$=$}}}}





\usepackage{xcolor}
% \usepackage[normalem]{ulem}
\usepackage{lipsum}
\makeatletter
% \newcommand\colorwave[1][blue]{\bgroup \markoverwith{\lower3.5\p@\hbox{\sixly \textcolor{#1}{\char58}}}\ULon}
%\font\sixly=lasy6 % does not re-load if already loaded, so no memory problem.

\newmdtheoremenv[
linewidth= 1pt,linecolor= blue,%
leftmargin=20,rightmargin=20,innertopmargin=0pt, innerrightmargin=40,%
tikzsetting = { draw=lightgray, line width = 0.3pt,dashed,%
dash pattern = on 15pt off 3pt},%
splittopskip=\topskip,skipbelow=\baselineskip,%
skipabove=\baselineskip,ntheorem,roundcorner=0pt,
% backgroundcolor=pagebg,font=\color{orange}\sffamily, fontcolor=white
]{examplebox}{Exemple}[section]



\newcommand\R{\mathbb{R}}
\newcommand\Z{\mathbb{Z}}
\newcommand\N{\mathbb{N}}
\newcommand\E{\mathbb{E}}
\newcommand\F{\mathcal{F}}
\newcommand\cH{\mathcal{H}}
\newcommand\V{\mathbb{V}}
\newcommand\dmo{ ^{-1} }
\newcommand\kapa{\kappa}
\newcommand\im{Im}
\newcommand\hs{\mathcal{H}}





\usepackage{soul}

\makeatletter
\newcommand*{\whiten}[1]{\llap{\textcolor{white}{{\the\SOUL@token}}\hspace{#1pt}}}
\DeclareRobustCommand*\myul{%
    \def\SOUL@everyspace{\underline{\space}\kern\z@}%
    \def\SOUL@everytoken{%
     \setbox0=\hbox{\the\SOUL@token}%
     \ifdim\dp0>\z@
        \raisebox{\dp0}{\underline{\phantom{\the\SOUL@token}}}%
        \whiten{1}\whiten{0}%
        \whiten{-1}\whiten{-2}%
        \llap{\the\SOUL@token}%
     \else
        \underline{\the\SOUL@token}%
     \fi}%
\SOUL@}
\makeatother

\newcommand*{\demp}{\fontfamily{lmtt}\selectfont}

\DeclareTextFontCommand{\textdemp}{\demp}

\begin{document}

\ifcomment
Multiline
comment
\fi
\ifcomment
\myul{Typesetting test}
% \color[rgb]{1,1,1}
$∑_i^n≠ 60º±∞π∆¬≈√j∫h≤≥µ$

$\CR \R\pro\ind\pro\gS\pro
\mqty[a&b\\c&d]$
$\pro\mathbb{P}$
$\dd{x}$

  \[
    \alpha(x)=\left\{
                \begin{array}{ll}
                  x\\
                  \frac{1}{1+e^{-kx}}\\
                  \frac{e^x-e^{-x}}{e^x+e^{-x}}
                \end{array}
              \right.
  \]

  $\expval{x}$
  
  $\chi_\rho(ghg\dmo)=\Tr(\rho_{ghg\dmo})=\Tr(\rho_g\circ\rho_h\circ\rho\dmo_g)=\Tr(\rho_h)\overset{\mbox{\scalebox{0.5}{$\Tr(AB)=\Tr(BA)$}}}{=}\chi_\rho(h)$
  	$\mathop{\oplus}_{\substack{x\in X}}$

$\mat(\rho_g)=(a_{ij}(g))_{\scriptsize \substack{1\leq i\leq d \\ 1\leq j\leq d}}$ et $\mat(\rho'_g)=(a'_{ij}(g))_{\scriptsize \substack{1\leq i'\leq d' \\ 1\leq j'\leq d'}}$



\[\int_a^b{\mathbb{R}^2}g(u, v)\dd{P_{XY}}(u, v)=\iint g(u,v) f_{XY}(u, v)\dd \lambda(u) \dd \lambda(v)\]
$$\lim_{x\to\infty} f(x)$$	
$$\iiiint_V \mu(t,u,v,w) \,dt\,du\,dv\,dw$$
$$\sum_{n=1}^{\infty} 2^{-n} = 1$$	
\begin{definition}
	Si $X$ et $Y$ sont 2 v.a. ou definit la \textsc{Covariance} entre $X$ et $Y$ comme
	$\cov(X,Y)\overset{\text{def}}{=}\E\left[(X-\E(X))(Y-\E(Y))\right]=\E(XY)-\E(X)\E(Y)$.
\end{definition}
\fi
\pagebreak

% \tableofcontents

% insert your code here
%\input{./algebra/main.tex}
%\input{./geometrie-differentielle/main.tex}
%\input{./probabilite/main.tex}
%\input{./analyse-fonctionnelle/main.tex}
% \input{./Analyse-convexe-et-dualite-en-optimisation/main.tex}
%\input{./tikz/main.tex}
%\input{./Theorie-du-distributions/main.tex}
%\input{./optimisation/mine.tex}
 \input{./modelisation/main.tex}

% yves.aubry@univ-tln.fr : algebra

\end{document}

%% !TEX encoding = UTF-8 Unicode
% !TEX TS-program = xelatex

\documentclass[french]{report}

%\usepackage[utf8]{inputenc}
%\usepackage[T1]{fontenc}
\usepackage{babel}


\newif\ifcomment
%\commenttrue # Show comments

\usepackage{physics}
\usepackage{amssymb}


\usepackage{amsthm}
% \usepackage{thmtools}
\usepackage{mathtools}
\usepackage{amsfonts}

\usepackage{color}

\usepackage{tikz}

\usepackage{geometry}
\geometry{a5paper, margin=0.1in, right=1cm}

\usepackage{dsfont}

\usepackage{graphicx}
\graphicspath{ {images/} }

\usepackage{faktor}

\usepackage{IEEEtrantools}
\usepackage{enumerate}   
\usepackage[PostScript=dvips]{"/Users/aware/Documents/Courses/diagrams"}


\newtheorem{theorem}{Théorème}[section]
\renewcommand{\thetheorem}{\arabic{theorem}}
\newtheorem{lemme}{Lemme}[section]
\renewcommand{\thelemme}{\arabic{lemme}}
\newtheorem{proposition}{Proposition}[section]
\renewcommand{\theproposition}{\arabic{proposition}}
\newtheorem{notations}{Notations}[section]
\newtheorem{problem}{Problème}[section]
\newtheorem{corollary}{Corollaire}[theorem]
\renewcommand{\thecorollary}{\arabic{corollary}}
\newtheorem{property}{Propriété}[section]
\newtheorem{objective}{Objectif}[section]

\theoremstyle{definition}
\newtheorem{definition}{Définition}[section]
\renewcommand{\thedefinition}{\arabic{definition}}
\newtheorem{exercise}{Exercice}[chapter]
\renewcommand{\theexercise}{\arabic{exercise}}
\newtheorem{example}{Exemple}[chapter]
\renewcommand{\theexample}{\arabic{example}}
\newtheorem*{solution}{Solution}
\newtheorem*{application}{Application}
\newtheorem*{notation}{Notation}
\newtheorem*{vocabulary}{Vocabulaire}
\newtheorem*{properties}{Propriétés}



\theoremstyle{remark}
\newtheorem*{remark}{Remarque}
\newtheorem*{rappel}{Rappel}


\usepackage{etoolbox}
\AtBeginEnvironment{exercise}{\small}
\AtBeginEnvironment{example}{\small}

\usepackage{cases}
\usepackage[red]{mypack}

\usepackage[framemethod=TikZ]{mdframed}

\definecolor{bg}{rgb}{0.4,0.25,0.95}
\definecolor{pagebg}{rgb}{0,0,0.5}
\surroundwithmdframed[
   topline=false,
   rightline=false,
   bottomline=false,
   leftmargin=\parindent,
   skipabove=8pt,
   skipbelow=8pt,
   linecolor=blue,
   innerbottommargin=10pt,
   % backgroundcolor=bg,font=\color{orange}\sffamily, fontcolor=white
]{definition}

\usepackage{empheq}
\usepackage[most]{tcolorbox}

\newtcbox{\mymath}[1][]{%
    nobeforeafter, math upper, tcbox raise base,
    enhanced, colframe=blue!30!black,
    colback=red!10, boxrule=1pt,
    #1}

\usepackage{unixode}


\DeclareMathOperator{\ord}{ord}
\DeclareMathOperator{\orb}{orb}
\DeclareMathOperator{\stab}{stab}
\DeclareMathOperator{\Stab}{stab}
\DeclareMathOperator{\ppcm}{ppcm}
\DeclareMathOperator{\conj}{Conj}
\DeclareMathOperator{\End}{End}
\DeclareMathOperator{\rot}{rot}
\DeclareMathOperator{\trs}{trace}
\DeclareMathOperator{\Ind}{Ind}
\DeclareMathOperator{\mat}{Mat}
\DeclareMathOperator{\id}{Id}
\DeclareMathOperator{\vect}{vect}
\DeclareMathOperator{\img}{img}
\DeclareMathOperator{\cov}{Cov}
\DeclareMathOperator{\dist}{dist}
\DeclareMathOperator{\irr}{Irr}
\DeclareMathOperator{\image}{Im}
\DeclareMathOperator{\pd}{\partial}
\DeclareMathOperator{\epi}{epi}
\DeclareMathOperator{\Argmin}{Argmin}
\DeclareMathOperator{\dom}{dom}
\DeclareMathOperator{\proj}{proj}
\DeclareMathOperator{\ctg}{ctg}
\DeclareMathOperator{\supp}{supp}
\DeclareMathOperator{\argmin}{argmin}
\DeclareMathOperator{\mult}{mult}
\DeclareMathOperator{\ch}{ch}
\DeclareMathOperator{\sh}{sh}
\DeclareMathOperator{\rang}{rang}
\DeclareMathOperator{\diam}{diam}
\DeclareMathOperator{\Epigraphe}{Epigraphe}




\usepackage{xcolor}
\everymath{\color{blue}}
%\everymath{\color[rgb]{0,1,1}}
%\pagecolor[rgb]{0,0,0.5}


\newcommand*{\pdtest}[3][]{\ensuremath{\frac{\partial^{#1} #2}{\partial #3}}}

\newcommand*{\deffunc}[6][]{\ensuremath{
\begin{array}{rcl}
#2 : #3 &\rightarrow& #4\\
#5 &\mapsto& #6
\end{array}
}}

\newcommand{\eqcolon}{\mathrel{\resizebox{\widthof{$\mathord{=}$}}{\height}{ $\!\!=\!\!\resizebox{1.2\width}{0.8\height}{\raisebox{0.23ex}{$\mathop{:}$}}\!\!$ }}}
\newcommand{\coloneq}{\mathrel{\resizebox{\widthof{$\mathord{=}$}}{\height}{ $\!\!\resizebox{1.2\width}{0.8\height}{\raisebox{0.23ex}{$\mathop{:}$}}\!\!=\!\!$ }}}
\newcommand{\eqcolonl}{\ensuremath{\mathrel{=\!\!\mathop{:}}}}
\newcommand{\coloneql}{\ensuremath{\mathrel{\mathop{:} \!\! =}}}
\newcommand{\vc}[1]{% inline column vector
  \left(\begin{smallmatrix}#1\end{smallmatrix}\right)%
}
\newcommand{\vr}[1]{% inline row vector
  \begin{smallmatrix}(\,#1\,)\end{smallmatrix}%
}
\makeatletter
\newcommand*{\defeq}{\ =\mathrel{\rlap{%
                     \raisebox{0.3ex}{$\m@th\cdot$}}%
                     \raisebox{-0.3ex}{$\m@th\cdot$}}%
                     }
\makeatother

\newcommand{\mathcircle}[1]{% inline row vector
 \overset{\circ}{#1}
}
\newcommand{\ulim}{% low limit
 \underline{\lim}
}
\newcommand{\ssi}{% iff
\iff
}
\newcommand{\ps}[2]{
\expval{#1 | #2}
}
\newcommand{\df}[1]{
\mqty{#1}
}
\newcommand{\n}[1]{
\norm{#1}
}
\newcommand{\sys}[1]{
\left\{\smqty{#1}\right.
}


\newcommand{\eqdef}{\ensuremath{\overset{\text{def}}=}}


\def\Circlearrowright{\ensuremath{%
  \rotatebox[origin=c]{230}{$\circlearrowright$}}}

\newcommand\ct[1]{\text{\rmfamily\upshape #1}}
\newcommand\question[1]{ {\color{red} ...!? \small #1}}
\newcommand\caz[1]{\left\{\begin{array} #1 \end{array}\right.}
\newcommand\const{\text{\rmfamily\upshape const}}
\newcommand\toP{ \overset{\pro}{\to}}
\newcommand\toPP{ \overset{\text{PP}}{\to}}
\newcommand{\oeq}{\mathrel{\text{\textcircled{$=$}}}}





\usepackage{xcolor}
% \usepackage[normalem]{ulem}
\usepackage{lipsum}
\makeatletter
% \newcommand\colorwave[1][blue]{\bgroup \markoverwith{\lower3.5\p@\hbox{\sixly \textcolor{#1}{\char58}}}\ULon}
%\font\sixly=lasy6 % does not re-load if already loaded, so no memory problem.

\newmdtheoremenv[
linewidth= 1pt,linecolor= blue,%
leftmargin=20,rightmargin=20,innertopmargin=0pt, innerrightmargin=40,%
tikzsetting = { draw=lightgray, line width = 0.3pt,dashed,%
dash pattern = on 15pt off 3pt},%
splittopskip=\topskip,skipbelow=\baselineskip,%
skipabove=\baselineskip,ntheorem,roundcorner=0pt,
% backgroundcolor=pagebg,font=\color{orange}\sffamily, fontcolor=white
]{examplebox}{Exemple}[section]



\newcommand\R{\mathbb{R}}
\newcommand\Z{\mathbb{Z}}
\newcommand\N{\mathbb{N}}
\newcommand\E{\mathbb{E}}
\newcommand\F{\mathcal{F}}
\newcommand\cH{\mathcal{H}}
\newcommand\V{\mathbb{V}}
\newcommand\dmo{ ^{-1} }
\newcommand\kapa{\kappa}
\newcommand\im{Im}
\newcommand\hs{\mathcal{H}}





\usepackage{soul}

\makeatletter
\newcommand*{\whiten}[1]{\llap{\textcolor{white}{{\the\SOUL@token}}\hspace{#1pt}}}
\DeclareRobustCommand*\myul{%
    \def\SOUL@everyspace{\underline{\space}\kern\z@}%
    \def\SOUL@everytoken{%
     \setbox0=\hbox{\the\SOUL@token}%
     \ifdim\dp0>\z@
        \raisebox{\dp0}{\underline{\phantom{\the\SOUL@token}}}%
        \whiten{1}\whiten{0}%
        \whiten{-1}\whiten{-2}%
        \llap{\the\SOUL@token}%
     \else
        \underline{\the\SOUL@token}%
     \fi}%
\SOUL@}
\makeatother

\newcommand*{\demp}{\fontfamily{lmtt}\selectfont}

\DeclareTextFontCommand{\textdemp}{\demp}

\begin{document}

\ifcomment
Multiline
comment
\fi
\ifcomment
\myul{Typesetting test}
% \color[rgb]{1,1,1}
$∑_i^n≠ 60º±∞π∆¬≈√j∫h≤≥µ$

$\CR \R\pro\ind\pro\gS\pro
\mqty[a&b\\c&d]$
$\pro\mathbb{P}$
$\dd{x}$

  \[
    \alpha(x)=\left\{
                \begin{array}{ll}
                  x\\
                  \frac{1}{1+e^{-kx}}\\
                  \frac{e^x-e^{-x}}{e^x+e^{-x}}
                \end{array}
              \right.
  \]

  $\expval{x}$
  
  $\chi_\rho(ghg\dmo)=\Tr(\rho_{ghg\dmo})=\Tr(\rho_g\circ\rho_h\circ\rho\dmo_g)=\Tr(\rho_h)\overset{\mbox{\scalebox{0.5}{$\Tr(AB)=\Tr(BA)$}}}{=}\chi_\rho(h)$
  	$\mathop{\oplus}_{\substack{x\in X}}$

$\mat(\rho_g)=(a_{ij}(g))_{\scriptsize \substack{1\leq i\leq d \\ 1\leq j\leq d}}$ et $\mat(\rho'_g)=(a'_{ij}(g))_{\scriptsize \substack{1\leq i'\leq d' \\ 1\leq j'\leq d'}}$



\[\int_a^b{\mathbb{R}^2}g(u, v)\dd{P_{XY}}(u, v)=\iint g(u,v) f_{XY}(u, v)\dd \lambda(u) \dd \lambda(v)\]
$$\lim_{x\to\infty} f(x)$$	
$$\iiiint_V \mu(t,u,v,w) \,dt\,du\,dv\,dw$$
$$\sum_{n=1}^{\infty} 2^{-n} = 1$$	
\begin{definition}
	Si $X$ et $Y$ sont 2 v.a. ou definit la \textsc{Covariance} entre $X$ et $Y$ comme
	$\cov(X,Y)\overset{\text{def}}{=}\E\left[(X-\E(X))(Y-\E(Y))\right]=\E(XY)-\E(X)\E(Y)$.
\end{definition}
\fi
\pagebreak

% \tableofcontents

% insert your code here
%\input{./algebra/main.tex}
%\input{./geometrie-differentielle/main.tex}
%\input{./probabilite/main.tex}
%\input{./analyse-fonctionnelle/main.tex}
% \input{./Analyse-convexe-et-dualite-en-optimisation/main.tex}
%\input{./tikz/main.tex}
%\input{./Theorie-du-distributions/main.tex}
%\input{./optimisation/mine.tex}
 \input{./modelisation/main.tex}

% yves.aubry@univ-tln.fr : algebra

\end{document}

% % !TEX encoding = UTF-8 Unicode
% !TEX TS-program = xelatex

\documentclass[french]{report}

%\usepackage[utf8]{inputenc}
%\usepackage[T1]{fontenc}
\usepackage{babel}


\newif\ifcomment
%\commenttrue # Show comments

\usepackage{physics}
\usepackage{amssymb}


\usepackage{amsthm}
% \usepackage{thmtools}
\usepackage{mathtools}
\usepackage{amsfonts}

\usepackage{color}

\usepackage{tikz}

\usepackage{geometry}
\geometry{a5paper, margin=0.1in, right=1cm}

\usepackage{dsfont}

\usepackage{graphicx}
\graphicspath{ {images/} }

\usepackage{faktor}

\usepackage{IEEEtrantools}
\usepackage{enumerate}   
\usepackage[PostScript=dvips]{"/Users/aware/Documents/Courses/diagrams"}


\newtheorem{theorem}{Théorème}[section]
\renewcommand{\thetheorem}{\arabic{theorem}}
\newtheorem{lemme}{Lemme}[section]
\renewcommand{\thelemme}{\arabic{lemme}}
\newtheorem{proposition}{Proposition}[section]
\renewcommand{\theproposition}{\arabic{proposition}}
\newtheorem{notations}{Notations}[section]
\newtheorem{problem}{Problème}[section]
\newtheorem{corollary}{Corollaire}[theorem]
\renewcommand{\thecorollary}{\arabic{corollary}}
\newtheorem{property}{Propriété}[section]
\newtheorem{objective}{Objectif}[section]

\theoremstyle{definition}
\newtheorem{definition}{Définition}[section]
\renewcommand{\thedefinition}{\arabic{definition}}
\newtheorem{exercise}{Exercice}[chapter]
\renewcommand{\theexercise}{\arabic{exercise}}
\newtheorem{example}{Exemple}[chapter]
\renewcommand{\theexample}{\arabic{example}}
\newtheorem*{solution}{Solution}
\newtheorem*{application}{Application}
\newtheorem*{notation}{Notation}
\newtheorem*{vocabulary}{Vocabulaire}
\newtheorem*{properties}{Propriétés}



\theoremstyle{remark}
\newtheorem*{remark}{Remarque}
\newtheorem*{rappel}{Rappel}


\usepackage{etoolbox}
\AtBeginEnvironment{exercise}{\small}
\AtBeginEnvironment{example}{\small}

\usepackage{cases}
\usepackage[red]{mypack}

\usepackage[framemethod=TikZ]{mdframed}

\definecolor{bg}{rgb}{0.4,0.25,0.95}
\definecolor{pagebg}{rgb}{0,0,0.5}
\surroundwithmdframed[
   topline=false,
   rightline=false,
   bottomline=false,
   leftmargin=\parindent,
   skipabove=8pt,
   skipbelow=8pt,
   linecolor=blue,
   innerbottommargin=10pt,
   % backgroundcolor=bg,font=\color{orange}\sffamily, fontcolor=white
]{definition}

\usepackage{empheq}
\usepackage[most]{tcolorbox}

\newtcbox{\mymath}[1][]{%
    nobeforeafter, math upper, tcbox raise base,
    enhanced, colframe=blue!30!black,
    colback=red!10, boxrule=1pt,
    #1}

\usepackage{unixode}


\DeclareMathOperator{\ord}{ord}
\DeclareMathOperator{\orb}{orb}
\DeclareMathOperator{\stab}{stab}
\DeclareMathOperator{\Stab}{stab}
\DeclareMathOperator{\ppcm}{ppcm}
\DeclareMathOperator{\conj}{Conj}
\DeclareMathOperator{\End}{End}
\DeclareMathOperator{\rot}{rot}
\DeclareMathOperator{\trs}{trace}
\DeclareMathOperator{\Ind}{Ind}
\DeclareMathOperator{\mat}{Mat}
\DeclareMathOperator{\id}{Id}
\DeclareMathOperator{\vect}{vect}
\DeclareMathOperator{\img}{img}
\DeclareMathOperator{\cov}{Cov}
\DeclareMathOperator{\dist}{dist}
\DeclareMathOperator{\irr}{Irr}
\DeclareMathOperator{\image}{Im}
\DeclareMathOperator{\pd}{\partial}
\DeclareMathOperator{\epi}{epi}
\DeclareMathOperator{\Argmin}{Argmin}
\DeclareMathOperator{\dom}{dom}
\DeclareMathOperator{\proj}{proj}
\DeclareMathOperator{\ctg}{ctg}
\DeclareMathOperator{\supp}{supp}
\DeclareMathOperator{\argmin}{argmin}
\DeclareMathOperator{\mult}{mult}
\DeclareMathOperator{\ch}{ch}
\DeclareMathOperator{\sh}{sh}
\DeclareMathOperator{\rang}{rang}
\DeclareMathOperator{\diam}{diam}
\DeclareMathOperator{\Epigraphe}{Epigraphe}




\usepackage{xcolor}
\everymath{\color{blue}}
%\everymath{\color[rgb]{0,1,1}}
%\pagecolor[rgb]{0,0,0.5}


\newcommand*{\pdtest}[3][]{\ensuremath{\frac{\partial^{#1} #2}{\partial #3}}}

\newcommand*{\deffunc}[6][]{\ensuremath{
\begin{array}{rcl}
#2 : #3 &\rightarrow& #4\\
#5 &\mapsto& #6
\end{array}
}}

\newcommand{\eqcolon}{\mathrel{\resizebox{\widthof{$\mathord{=}$}}{\height}{ $\!\!=\!\!\resizebox{1.2\width}{0.8\height}{\raisebox{0.23ex}{$\mathop{:}$}}\!\!$ }}}
\newcommand{\coloneq}{\mathrel{\resizebox{\widthof{$\mathord{=}$}}{\height}{ $\!\!\resizebox{1.2\width}{0.8\height}{\raisebox{0.23ex}{$\mathop{:}$}}\!\!=\!\!$ }}}
\newcommand{\eqcolonl}{\ensuremath{\mathrel{=\!\!\mathop{:}}}}
\newcommand{\coloneql}{\ensuremath{\mathrel{\mathop{:} \!\! =}}}
\newcommand{\vc}[1]{% inline column vector
  \left(\begin{smallmatrix}#1\end{smallmatrix}\right)%
}
\newcommand{\vr}[1]{% inline row vector
  \begin{smallmatrix}(\,#1\,)\end{smallmatrix}%
}
\makeatletter
\newcommand*{\defeq}{\ =\mathrel{\rlap{%
                     \raisebox{0.3ex}{$\m@th\cdot$}}%
                     \raisebox{-0.3ex}{$\m@th\cdot$}}%
                     }
\makeatother

\newcommand{\mathcircle}[1]{% inline row vector
 \overset{\circ}{#1}
}
\newcommand{\ulim}{% low limit
 \underline{\lim}
}
\newcommand{\ssi}{% iff
\iff
}
\newcommand{\ps}[2]{
\expval{#1 | #2}
}
\newcommand{\df}[1]{
\mqty{#1}
}
\newcommand{\n}[1]{
\norm{#1}
}
\newcommand{\sys}[1]{
\left\{\smqty{#1}\right.
}


\newcommand{\eqdef}{\ensuremath{\overset{\text{def}}=}}


\def\Circlearrowright{\ensuremath{%
  \rotatebox[origin=c]{230}{$\circlearrowright$}}}

\newcommand\ct[1]{\text{\rmfamily\upshape #1}}
\newcommand\question[1]{ {\color{red} ...!? \small #1}}
\newcommand\caz[1]{\left\{\begin{array} #1 \end{array}\right.}
\newcommand\const{\text{\rmfamily\upshape const}}
\newcommand\toP{ \overset{\pro}{\to}}
\newcommand\toPP{ \overset{\text{PP}}{\to}}
\newcommand{\oeq}{\mathrel{\text{\textcircled{$=$}}}}





\usepackage{xcolor}
% \usepackage[normalem]{ulem}
\usepackage{lipsum}
\makeatletter
% \newcommand\colorwave[1][blue]{\bgroup \markoverwith{\lower3.5\p@\hbox{\sixly \textcolor{#1}{\char58}}}\ULon}
%\font\sixly=lasy6 % does not re-load if already loaded, so no memory problem.

\newmdtheoremenv[
linewidth= 1pt,linecolor= blue,%
leftmargin=20,rightmargin=20,innertopmargin=0pt, innerrightmargin=40,%
tikzsetting = { draw=lightgray, line width = 0.3pt,dashed,%
dash pattern = on 15pt off 3pt},%
splittopskip=\topskip,skipbelow=\baselineskip,%
skipabove=\baselineskip,ntheorem,roundcorner=0pt,
% backgroundcolor=pagebg,font=\color{orange}\sffamily, fontcolor=white
]{examplebox}{Exemple}[section]



\newcommand\R{\mathbb{R}}
\newcommand\Z{\mathbb{Z}}
\newcommand\N{\mathbb{N}}
\newcommand\E{\mathbb{E}}
\newcommand\F{\mathcal{F}}
\newcommand\cH{\mathcal{H}}
\newcommand\V{\mathbb{V}}
\newcommand\dmo{ ^{-1} }
\newcommand\kapa{\kappa}
\newcommand\im{Im}
\newcommand\hs{\mathcal{H}}





\usepackage{soul}

\makeatletter
\newcommand*{\whiten}[1]{\llap{\textcolor{white}{{\the\SOUL@token}}\hspace{#1pt}}}
\DeclareRobustCommand*\myul{%
    \def\SOUL@everyspace{\underline{\space}\kern\z@}%
    \def\SOUL@everytoken{%
     \setbox0=\hbox{\the\SOUL@token}%
     \ifdim\dp0>\z@
        \raisebox{\dp0}{\underline{\phantom{\the\SOUL@token}}}%
        \whiten{1}\whiten{0}%
        \whiten{-1}\whiten{-2}%
        \llap{\the\SOUL@token}%
     \else
        \underline{\the\SOUL@token}%
     \fi}%
\SOUL@}
\makeatother

\newcommand*{\demp}{\fontfamily{lmtt}\selectfont}

\DeclareTextFontCommand{\textdemp}{\demp}

\begin{document}

\ifcomment
Multiline
comment
\fi
\ifcomment
\myul{Typesetting test}
% \color[rgb]{1,1,1}
$∑_i^n≠ 60º±∞π∆¬≈√j∫h≤≥µ$

$\CR \R\pro\ind\pro\gS\pro
\mqty[a&b\\c&d]$
$\pro\mathbb{P}$
$\dd{x}$

  \[
    \alpha(x)=\left\{
                \begin{array}{ll}
                  x\\
                  \frac{1}{1+e^{-kx}}\\
                  \frac{e^x-e^{-x}}{e^x+e^{-x}}
                \end{array}
              \right.
  \]

  $\expval{x}$
  
  $\chi_\rho(ghg\dmo)=\Tr(\rho_{ghg\dmo})=\Tr(\rho_g\circ\rho_h\circ\rho\dmo_g)=\Tr(\rho_h)\overset{\mbox{\scalebox{0.5}{$\Tr(AB)=\Tr(BA)$}}}{=}\chi_\rho(h)$
  	$\mathop{\oplus}_{\substack{x\in X}}$

$\mat(\rho_g)=(a_{ij}(g))_{\scriptsize \substack{1\leq i\leq d \\ 1\leq j\leq d}}$ et $\mat(\rho'_g)=(a'_{ij}(g))_{\scriptsize \substack{1\leq i'\leq d' \\ 1\leq j'\leq d'}}$



\[\int_a^b{\mathbb{R}^2}g(u, v)\dd{P_{XY}}(u, v)=\iint g(u,v) f_{XY}(u, v)\dd \lambda(u) \dd \lambda(v)\]
$$\lim_{x\to\infty} f(x)$$	
$$\iiiint_V \mu(t,u,v,w) \,dt\,du\,dv\,dw$$
$$\sum_{n=1}^{\infty} 2^{-n} = 1$$	
\begin{definition}
	Si $X$ et $Y$ sont 2 v.a. ou definit la \textsc{Covariance} entre $X$ et $Y$ comme
	$\cov(X,Y)\overset{\text{def}}{=}\E\left[(X-\E(X))(Y-\E(Y))\right]=\E(XY)-\E(X)\E(Y)$.
\end{definition}
\fi
\pagebreak

% \tableofcontents

% insert your code here
%\input{./algebra/main.tex}
%\input{./geometrie-differentielle/main.tex}
%\input{./probabilite/main.tex}
%\input{./analyse-fonctionnelle/main.tex}
% \input{./Analyse-convexe-et-dualite-en-optimisation/main.tex}
%\input{./tikz/main.tex}
%\input{./Theorie-du-distributions/main.tex}
%\input{./optimisation/mine.tex}
 \input{./modelisation/main.tex}

% yves.aubry@univ-tln.fr : algebra

\end{document}

%% !TEX encoding = UTF-8 Unicode
% !TEX TS-program = xelatex

\documentclass[french]{report}

%\usepackage[utf8]{inputenc}
%\usepackage[T1]{fontenc}
\usepackage{babel}


\newif\ifcomment
%\commenttrue # Show comments

\usepackage{physics}
\usepackage{amssymb}


\usepackage{amsthm}
% \usepackage{thmtools}
\usepackage{mathtools}
\usepackage{amsfonts}

\usepackage{color}

\usepackage{tikz}

\usepackage{geometry}
\geometry{a5paper, margin=0.1in, right=1cm}

\usepackage{dsfont}

\usepackage{graphicx}
\graphicspath{ {images/} }

\usepackage{faktor}

\usepackage{IEEEtrantools}
\usepackage{enumerate}   
\usepackage[PostScript=dvips]{"/Users/aware/Documents/Courses/diagrams"}


\newtheorem{theorem}{Théorème}[section]
\renewcommand{\thetheorem}{\arabic{theorem}}
\newtheorem{lemme}{Lemme}[section]
\renewcommand{\thelemme}{\arabic{lemme}}
\newtheorem{proposition}{Proposition}[section]
\renewcommand{\theproposition}{\arabic{proposition}}
\newtheorem{notations}{Notations}[section]
\newtheorem{problem}{Problème}[section]
\newtheorem{corollary}{Corollaire}[theorem]
\renewcommand{\thecorollary}{\arabic{corollary}}
\newtheorem{property}{Propriété}[section]
\newtheorem{objective}{Objectif}[section]

\theoremstyle{definition}
\newtheorem{definition}{Définition}[section]
\renewcommand{\thedefinition}{\arabic{definition}}
\newtheorem{exercise}{Exercice}[chapter]
\renewcommand{\theexercise}{\arabic{exercise}}
\newtheorem{example}{Exemple}[chapter]
\renewcommand{\theexample}{\arabic{example}}
\newtheorem*{solution}{Solution}
\newtheorem*{application}{Application}
\newtheorem*{notation}{Notation}
\newtheorem*{vocabulary}{Vocabulaire}
\newtheorem*{properties}{Propriétés}



\theoremstyle{remark}
\newtheorem*{remark}{Remarque}
\newtheorem*{rappel}{Rappel}


\usepackage{etoolbox}
\AtBeginEnvironment{exercise}{\small}
\AtBeginEnvironment{example}{\small}

\usepackage{cases}
\usepackage[red]{mypack}

\usepackage[framemethod=TikZ]{mdframed}

\definecolor{bg}{rgb}{0.4,0.25,0.95}
\definecolor{pagebg}{rgb}{0,0,0.5}
\surroundwithmdframed[
   topline=false,
   rightline=false,
   bottomline=false,
   leftmargin=\parindent,
   skipabove=8pt,
   skipbelow=8pt,
   linecolor=blue,
   innerbottommargin=10pt,
   % backgroundcolor=bg,font=\color{orange}\sffamily, fontcolor=white
]{definition}

\usepackage{empheq}
\usepackage[most]{tcolorbox}

\newtcbox{\mymath}[1][]{%
    nobeforeafter, math upper, tcbox raise base,
    enhanced, colframe=blue!30!black,
    colback=red!10, boxrule=1pt,
    #1}

\usepackage{unixode}


\DeclareMathOperator{\ord}{ord}
\DeclareMathOperator{\orb}{orb}
\DeclareMathOperator{\stab}{stab}
\DeclareMathOperator{\Stab}{stab}
\DeclareMathOperator{\ppcm}{ppcm}
\DeclareMathOperator{\conj}{Conj}
\DeclareMathOperator{\End}{End}
\DeclareMathOperator{\rot}{rot}
\DeclareMathOperator{\trs}{trace}
\DeclareMathOperator{\Ind}{Ind}
\DeclareMathOperator{\mat}{Mat}
\DeclareMathOperator{\id}{Id}
\DeclareMathOperator{\vect}{vect}
\DeclareMathOperator{\img}{img}
\DeclareMathOperator{\cov}{Cov}
\DeclareMathOperator{\dist}{dist}
\DeclareMathOperator{\irr}{Irr}
\DeclareMathOperator{\image}{Im}
\DeclareMathOperator{\pd}{\partial}
\DeclareMathOperator{\epi}{epi}
\DeclareMathOperator{\Argmin}{Argmin}
\DeclareMathOperator{\dom}{dom}
\DeclareMathOperator{\proj}{proj}
\DeclareMathOperator{\ctg}{ctg}
\DeclareMathOperator{\supp}{supp}
\DeclareMathOperator{\argmin}{argmin}
\DeclareMathOperator{\mult}{mult}
\DeclareMathOperator{\ch}{ch}
\DeclareMathOperator{\sh}{sh}
\DeclareMathOperator{\rang}{rang}
\DeclareMathOperator{\diam}{diam}
\DeclareMathOperator{\Epigraphe}{Epigraphe}




\usepackage{xcolor}
\everymath{\color{blue}}
%\everymath{\color[rgb]{0,1,1}}
%\pagecolor[rgb]{0,0,0.5}


\newcommand*{\pdtest}[3][]{\ensuremath{\frac{\partial^{#1} #2}{\partial #3}}}

\newcommand*{\deffunc}[6][]{\ensuremath{
\begin{array}{rcl}
#2 : #3 &\rightarrow& #4\\
#5 &\mapsto& #6
\end{array}
}}

\newcommand{\eqcolon}{\mathrel{\resizebox{\widthof{$\mathord{=}$}}{\height}{ $\!\!=\!\!\resizebox{1.2\width}{0.8\height}{\raisebox{0.23ex}{$\mathop{:}$}}\!\!$ }}}
\newcommand{\coloneq}{\mathrel{\resizebox{\widthof{$\mathord{=}$}}{\height}{ $\!\!\resizebox{1.2\width}{0.8\height}{\raisebox{0.23ex}{$\mathop{:}$}}\!\!=\!\!$ }}}
\newcommand{\eqcolonl}{\ensuremath{\mathrel{=\!\!\mathop{:}}}}
\newcommand{\coloneql}{\ensuremath{\mathrel{\mathop{:} \!\! =}}}
\newcommand{\vc}[1]{% inline column vector
  \left(\begin{smallmatrix}#1\end{smallmatrix}\right)%
}
\newcommand{\vr}[1]{% inline row vector
  \begin{smallmatrix}(\,#1\,)\end{smallmatrix}%
}
\makeatletter
\newcommand*{\defeq}{\ =\mathrel{\rlap{%
                     \raisebox{0.3ex}{$\m@th\cdot$}}%
                     \raisebox{-0.3ex}{$\m@th\cdot$}}%
                     }
\makeatother

\newcommand{\mathcircle}[1]{% inline row vector
 \overset{\circ}{#1}
}
\newcommand{\ulim}{% low limit
 \underline{\lim}
}
\newcommand{\ssi}{% iff
\iff
}
\newcommand{\ps}[2]{
\expval{#1 | #2}
}
\newcommand{\df}[1]{
\mqty{#1}
}
\newcommand{\n}[1]{
\norm{#1}
}
\newcommand{\sys}[1]{
\left\{\smqty{#1}\right.
}


\newcommand{\eqdef}{\ensuremath{\overset{\text{def}}=}}


\def\Circlearrowright{\ensuremath{%
  \rotatebox[origin=c]{230}{$\circlearrowright$}}}

\newcommand\ct[1]{\text{\rmfamily\upshape #1}}
\newcommand\question[1]{ {\color{red} ...!? \small #1}}
\newcommand\caz[1]{\left\{\begin{array} #1 \end{array}\right.}
\newcommand\const{\text{\rmfamily\upshape const}}
\newcommand\toP{ \overset{\pro}{\to}}
\newcommand\toPP{ \overset{\text{PP}}{\to}}
\newcommand{\oeq}{\mathrel{\text{\textcircled{$=$}}}}





\usepackage{xcolor}
% \usepackage[normalem]{ulem}
\usepackage{lipsum}
\makeatletter
% \newcommand\colorwave[1][blue]{\bgroup \markoverwith{\lower3.5\p@\hbox{\sixly \textcolor{#1}{\char58}}}\ULon}
%\font\sixly=lasy6 % does not re-load if already loaded, so no memory problem.

\newmdtheoremenv[
linewidth= 1pt,linecolor= blue,%
leftmargin=20,rightmargin=20,innertopmargin=0pt, innerrightmargin=40,%
tikzsetting = { draw=lightgray, line width = 0.3pt,dashed,%
dash pattern = on 15pt off 3pt},%
splittopskip=\topskip,skipbelow=\baselineskip,%
skipabove=\baselineskip,ntheorem,roundcorner=0pt,
% backgroundcolor=pagebg,font=\color{orange}\sffamily, fontcolor=white
]{examplebox}{Exemple}[section]



\newcommand\R{\mathbb{R}}
\newcommand\Z{\mathbb{Z}}
\newcommand\N{\mathbb{N}}
\newcommand\E{\mathbb{E}}
\newcommand\F{\mathcal{F}}
\newcommand\cH{\mathcal{H}}
\newcommand\V{\mathbb{V}}
\newcommand\dmo{ ^{-1} }
\newcommand\kapa{\kappa}
\newcommand\im{Im}
\newcommand\hs{\mathcal{H}}





\usepackage{soul}

\makeatletter
\newcommand*{\whiten}[1]{\llap{\textcolor{white}{{\the\SOUL@token}}\hspace{#1pt}}}
\DeclareRobustCommand*\myul{%
    \def\SOUL@everyspace{\underline{\space}\kern\z@}%
    \def\SOUL@everytoken{%
     \setbox0=\hbox{\the\SOUL@token}%
     \ifdim\dp0>\z@
        \raisebox{\dp0}{\underline{\phantom{\the\SOUL@token}}}%
        \whiten{1}\whiten{0}%
        \whiten{-1}\whiten{-2}%
        \llap{\the\SOUL@token}%
     \else
        \underline{\the\SOUL@token}%
     \fi}%
\SOUL@}
\makeatother

\newcommand*{\demp}{\fontfamily{lmtt}\selectfont}

\DeclareTextFontCommand{\textdemp}{\demp}

\begin{document}

\ifcomment
Multiline
comment
\fi
\ifcomment
\myul{Typesetting test}
% \color[rgb]{1,1,1}
$∑_i^n≠ 60º±∞π∆¬≈√j∫h≤≥µ$

$\CR \R\pro\ind\pro\gS\pro
\mqty[a&b\\c&d]$
$\pro\mathbb{P}$
$\dd{x}$

  \[
    \alpha(x)=\left\{
                \begin{array}{ll}
                  x\\
                  \frac{1}{1+e^{-kx}}\\
                  \frac{e^x-e^{-x}}{e^x+e^{-x}}
                \end{array}
              \right.
  \]

  $\expval{x}$
  
  $\chi_\rho(ghg\dmo)=\Tr(\rho_{ghg\dmo})=\Tr(\rho_g\circ\rho_h\circ\rho\dmo_g)=\Tr(\rho_h)\overset{\mbox{\scalebox{0.5}{$\Tr(AB)=\Tr(BA)$}}}{=}\chi_\rho(h)$
  	$\mathop{\oplus}_{\substack{x\in X}}$

$\mat(\rho_g)=(a_{ij}(g))_{\scriptsize \substack{1\leq i\leq d \\ 1\leq j\leq d}}$ et $\mat(\rho'_g)=(a'_{ij}(g))_{\scriptsize \substack{1\leq i'\leq d' \\ 1\leq j'\leq d'}}$



\[\int_a^b{\mathbb{R}^2}g(u, v)\dd{P_{XY}}(u, v)=\iint g(u,v) f_{XY}(u, v)\dd \lambda(u) \dd \lambda(v)\]
$$\lim_{x\to\infty} f(x)$$	
$$\iiiint_V \mu(t,u,v,w) \,dt\,du\,dv\,dw$$
$$\sum_{n=1}^{\infty} 2^{-n} = 1$$	
\begin{definition}
	Si $X$ et $Y$ sont 2 v.a. ou definit la \textsc{Covariance} entre $X$ et $Y$ comme
	$\cov(X,Y)\overset{\text{def}}{=}\E\left[(X-\E(X))(Y-\E(Y))\right]=\E(XY)-\E(X)\E(Y)$.
\end{definition}
\fi
\pagebreak

% \tableofcontents

% insert your code here
%\input{./algebra/main.tex}
%\input{./geometrie-differentielle/main.tex}
%\input{./probabilite/main.tex}
%\input{./analyse-fonctionnelle/main.tex}
% \input{./Analyse-convexe-et-dualite-en-optimisation/main.tex}
%\input{./tikz/main.tex}
%\input{./Theorie-du-distributions/main.tex}
%\input{./optimisation/mine.tex}
 \input{./modelisation/main.tex}

% yves.aubry@univ-tln.fr : algebra

\end{document}

%% !TEX encoding = UTF-8 Unicode
% !TEX TS-program = xelatex

\documentclass[french]{report}

%\usepackage[utf8]{inputenc}
%\usepackage[T1]{fontenc}
\usepackage{babel}


\newif\ifcomment
%\commenttrue # Show comments

\usepackage{physics}
\usepackage{amssymb}


\usepackage{amsthm}
% \usepackage{thmtools}
\usepackage{mathtools}
\usepackage{amsfonts}

\usepackage{color}

\usepackage{tikz}

\usepackage{geometry}
\geometry{a5paper, margin=0.1in, right=1cm}

\usepackage{dsfont}

\usepackage{graphicx}
\graphicspath{ {images/} }

\usepackage{faktor}

\usepackage{IEEEtrantools}
\usepackage{enumerate}   
\usepackage[PostScript=dvips]{"/Users/aware/Documents/Courses/diagrams"}


\newtheorem{theorem}{Théorème}[section]
\renewcommand{\thetheorem}{\arabic{theorem}}
\newtheorem{lemme}{Lemme}[section]
\renewcommand{\thelemme}{\arabic{lemme}}
\newtheorem{proposition}{Proposition}[section]
\renewcommand{\theproposition}{\arabic{proposition}}
\newtheorem{notations}{Notations}[section]
\newtheorem{problem}{Problème}[section]
\newtheorem{corollary}{Corollaire}[theorem]
\renewcommand{\thecorollary}{\arabic{corollary}}
\newtheorem{property}{Propriété}[section]
\newtheorem{objective}{Objectif}[section]

\theoremstyle{definition}
\newtheorem{definition}{Définition}[section]
\renewcommand{\thedefinition}{\arabic{definition}}
\newtheorem{exercise}{Exercice}[chapter]
\renewcommand{\theexercise}{\arabic{exercise}}
\newtheorem{example}{Exemple}[chapter]
\renewcommand{\theexample}{\arabic{example}}
\newtheorem*{solution}{Solution}
\newtheorem*{application}{Application}
\newtheorem*{notation}{Notation}
\newtheorem*{vocabulary}{Vocabulaire}
\newtheorem*{properties}{Propriétés}



\theoremstyle{remark}
\newtheorem*{remark}{Remarque}
\newtheorem*{rappel}{Rappel}


\usepackage{etoolbox}
\AtBeginEnvironment{exercise}{\small}
\AtBeginEnvironment{example}{\small}

\usepackage{cases}
\usepackage[red]{mypack}

\usepackage[framemethod=TikZ]{mdframed}

\definecolor{bg}{rgb}{0.4,0.25,0.95}
\definecolor{pagebg}{rgb}{0,0,0.5}
\surroundwithmdframed[
   topline=false,
   rightline=false,
   bottomline=false,
   leftmargin=\parindent,
   skipabove=8pt,
   skipbelow=8pt,
   linecolor=blue,
   innerbottommargin=10pt,
   % backgroundcolor=bg,font=\color{orange}\sffamily, fontcolor=white
]{definition}

\usepackage{empheq}
\usepackage[most]{tcolorbox}

\newtcbox{\mymath}[1][]{%
    nobeforeafter, math upper, tcbox raise base,
    enhanced, colframe=blue!30!black,
    colback=red!10, boxrule=1pt,
    #1}

\usepackage{unixode}


\DeclareMathOperator{\ord}{ord}
\DeclareMathOperator{\orb}{orb}
\DeclareMathOperator{\stab}{stab}
\DeclareMathOperator{\Stab}{stab}
\DeclareMathOperator{\ppcm}{ppcm}
\DeclareMathOperator{\conj}{Conj}
\DeclareMathOperator{\End}{End}
\DeclareMathOperator{\rot}{rot}
\DeclareMathOperator{\trs}{trace}
\DeclareMathOperator{\Ind}{Ind}
\DeclareMathOperator{\mat}{Mat}
\DeclareMathOperator{\id}{Id}
\DeclareMathOperator{\vect}{vect}
\DeclareMathOperator{\img}{img}
\DeclareMathOperator{\cov}{Cov}
\DeclareMathOperator{\dist}{dist}
\DeclareMathOperator{\irr}{Irr}
\DeclareMathOperator{\image}{Im}
\DeclareMathOperator{\pd}{\partial}
\DeclareMathOperator{\epi}{epi}
\DeclareMathOperator{\Argmin}{Argmin}
\DeclareMathOperator{\dom}{dom}
\DeclareMathOperator{\proj}{proj}
\DeclareMathOperator{\ctg}{ctg}
\DeclareMathOperator{\supp}{supp}
\DeclareMathOperator{\argmin}{argmin}
\DeclareMathOperator{\mult}{mult}
\DeclareMathOperator{\ch}{ch}
\DeclareMathOperator{\sh}{sh}
\DeclareMathOperator{\rang}{rang}
\DeclareMathOperator{\diam}{diam}
\DeclareMathOperator{\Epigraphe}{Epigraphe}




\usepackage{xcolor}
\everymath{\color{blue}}
%\everymath{\color[rgb]{0,1,1}}
%\pagecolor[rgb]{0,0,0.5}


\newcommand*{\pdtest}[3][]{\ensuremath{\frac{\partial^{#1} #2}{\partial #3}}}

\newcommand*{\deffunc}[6][]{\ensuremath{
\begin{array}{rcl}
#2 : #3 &\rightarrow& #4\\
#5 &\mapsto& #6
\end{array}
}}

\newcommand{\eqcolon}{\mathrel{\resizebox{\widthof{$\mathord{=}$}}{\height}{ $\!\!=\!\!\resizebox{1.2\width}{0.8\height}{\raisebox{0.23ex}{$\mathop{:}$}}\!\!$ }}}
\newcommand{\coloneq}{\mathrel{\resizebox{\widthof{$\mathord{=}$}}{\height}{ $\!\!\resizebox{1.2\width}{0.8\height}{\raisebox{0.23ex}{$\mathop{:}$}}\!\!=\!\!$ }}}
\newcommand{\eqcolonl}{\ensuremath{\mathrel{=\!\!\mathop{:}}}}
\newcommand{\coloneql}{\ensuremath{\mathrel{\mathop{:} \!\! =}}}
\newcommand{\vc}[1]{% inline column vector
  \left(\begin{smallmatrix}#1\end{smallmatrix}\right)%
}
\newcommand{\vr}[1]{% inline row vector
  \begin{smallmatrix}(\,#1\,)\end{smallmatrix}%
}
\makeatletter
\newcommand*{\defeq}{\ =\mathrel{\rlap{%
                     \raisebox{0.3ex}{$\m@th\cdot$}}%
                     \raisebox{-0.3ex}{$\m@th\cdot$}}%
                     }
\makeatother

\newcommand{\mathcircle}[1]{% inline row vector
 \overset{\circ}{#1}
}
\newcommand{\ulim}{% low limit
 \underline{\lim}
}
\newcommand{\ssi}{% iff
\iff
}
\newcommand{\ps}[2]{
\expval{#1 | #2}
}
\newcommand{\df}[1]{
\mqty{#1}
}
\newcommand{\n}[1]{
\norm{#1}
}
\newcommand{\sys}[1]{
\left\{\smqty{#1}\right.
}


\newcommand{\eqdef}{\ensuremath{\overset{\text{def}}=}}


\def\Circlearrowright{\ensuremath{%
  \rotatebox[origin=c]{230}{$\circlearrowright$}}}

\newcommand\ct[1]{\text{\rmfamily\upshape #1}}
\newcommand\question[1]{ {\color{red} ...!? \small #1}}
\newcommand\caz[1]{\left\{\begin{array} #1 \end{array}\right.}
\newcommand\const{\text{\rmfamily\upshape const}}
\newcommand\toP{ \overset{\pro}{\to}}
\newcommand\toPP{ \overset{\text{PP}}{\to}}
\newcommand{\oeq}{\mathrel{\text{\textcircled{$=$}}}}





\usepackage{xcolor}
% \usepackage[normalem]{ulem}
\usepackage{lipsum}
\makeatletter
% \newcommand\colorwave[1][blue]{\bgroup \markoverwith{\lower3.5\p@\hbox{\sixly \textcolor{#1}{\char58}}}\ULon}
%\font\sixly=lasy6 % does not re-load if already loaded, so no memory problem.

\newmdtheoremenv[
linewidth= 1pt,linecolor= blue,%
leftmargin=20,rightmargin=20,innertopmargin=0pt, innerrightmargin=40,%
tikzsetting = { draw=lightgray, line width = 0.3pt,dashed,%
dash pattern = on 15pt off 3pt},%
splittopskip=\topskip,skipbelow=\baselineskip,%
skipabove=\baselineskip,ntheorem,roundcorner=0pt,
% backgroundcolor=pagebg,font=\color{orange}\sffamily, fontcolor=white
]{examplebox}{Exemple}[section]



\newcommand\R{\mathbb{R}}
\newcommand\Z{\mathbb{Z}}
\newcommand\N{\mathbb{N}}
\newcommand\E{\mathbb{E}}
\newcommand\F{\mathcal{F}}
\newcommand\cH{\mathcal{H}}
\newcommand\V{\mathbb{V}}
\newcommand\dmo{ ^{-1} }
\newcommand\kapa{\kappa}
\newcommand\im{Im}
\newcommand\hs{\mathcal{H}}





\usepackage{soul}

\makeatletter
\newcommand*{\whiten}[1]{\llap{\textcolor{white}{{\the\SOUL@token}}\hspace{#1pt}}}
\DeclareRobustCommand*\myul{%
    \def\SOUL@everyspace{\underline{\space}\kern\z@}%
    \def\SOUL@everytoken{%
     \setbox0=\hbox{\the\SOUL@token}%
     \ifdim\dp0>\z@
        \raisebox{\dp0}{\underline{\phantom{\the\SOUL@token}}}%
        \whiten{1}\whiten{0}%
        \whiten{-1}\whiten{-2}%
        \llap{\the\SOUL@token}%
     \else
        \underline{\the\SOUL@token}%
     \fi}%
\SOUL@}
\makeatother

\newcommand*{\demp}{\fontfamily{lmtt}\selectfont}

\DeclareTextFontCommand{\textdemp}{\demp}

\begin{document}

\ifcomment
Multiline
comment
\fi
\ifcomment
\myul{Typesetting test}
% \color[rgb]{1,1,1}
$∑_i^n≠ 60º±∞π∆¬≈√j∫h≤≥µ$

$\CR \R\pro\ind\pro\gS\pro
\mqty[a&b\\c&d]$
$\pro\mathbb{P}$
$\dd{x}$

  \[
    \alpha(x)=\left\{
                \begin{array}{ll}
                  x\\
                  \frac{1}{1+e^{-kx}}\\
                  \frac{e^x-e^{-x}}{e^x+e^{-x}}
                \end{array}
              \right.
  \]

  $\expval{x}$
  
  $\chi_\rho(ghg\dmo)=\Tr(\rho_{ghg\dmo})=\Tr(\rho_g\circ\rho_h\circ\rho\dmo_g)=\Tr(\rho_h)\overset{\mbox{\scalebox{0.5}{$\Tr(AB)=\Tr(BA)$}}}{=}\chi_\rho(h)$
  	$\mathop{\oplus}_{\substack{x\in X}}$

$\mat(\rho_g)=(a_{ij}(g))_{\scriptsize \substack{1\leq i\leq d \\ 1\leq j\leq d}}$ et $\mat(\rho'_g)=(a'_{ij}(g))_{\scriptsize \substack{1\leq i'\leq d' \\ 1\leq j'\leq d'}}$



\[\int_a^b{\mathbb{R}^2}g(u, v)\dd{P_{XY}}(u, v)=\iint g(u,v) f_{XY}(u, v)\dd \lambda(u) \dd \lambda(v)\]
$$\lim_{x\to\infty} f(x)$$	
$$\iiiint_V \mu(t,u,v,w) \,dt\,du\,dv\,dw$$
$$\sum_{n=1}^{\infty} 2^{-n} = 1$$	
\begin{definition}
	Si $X$ et $Y$ sont 2 v.a. ou definit la \textsc{Covariance} entre $X$ et $Y$ comme
	$\cov(X,Y)\overset{\text{def}}{=}\E\left[(X-\E(X))(Y-\E(Y))\right]=\E(XY)-\E(X)\E(Y)$.
\end{definition}
\fi
\pagebreak

% \tableofcontents

% insert your code here
%\input{./algebra/main.tex}
%\input{./geometrie-differentielle/main.tex}
%\input{./probabilite/main.tex}
%\input{./analyse-fonctionnelle/main.tex}
% \input{./Analyse-convexe-et-dualite-en-optimisation/main.tex}
%\input{./tikz/main.tex}
%\input{./Theorie-du-distributions/main.tex}
%\input{./optimisation/mine.tex}
 \input{./modelisation/main.tex}

% yves.aubry@univ-tln.fr : algebra

\end{document}

%\input{./optimisation/mine.tex}
 % !TEX encoding = UTF-8 Unicode
% !TEX TS-program = xelatex

\documentclass[french]{report}

%\usepackage[utf8]{inputenc}
%\usepackage[T1]{fontenc}
\usepackage{babel}


\newif\ifcomment
%\commenttrue # Show comments

\usepackage{physics}
\usepackage{amssymb}


\usepackage{amsthm}
% \usepackage{thmtools}
\usepackage{mathtools}
\usepackage{amsfonts}

\usepackage{color}

\usepackage{tikz}

\usepackage{geometry}
\geometry{a5paper, margin=0.1in, right=1cm}

\usepackage{dsfont}

\usepackage{graphicx}
\graphicspath{ {images/} }

\usepackage{faktor}

\usepackage{IEEEtrantools}
\usepackage{enumerate}   
\usepackage[PostScript=dvips]{"/Users/aware/Documents/Courses/diagrams"}


\newtheorem{theorem}{Théorème}[section]
\renewcommand{\thetheorem}{\arabic{theorem}}
\newtheorem{lemme}{Lemme}[section]
\renewcommand{\thelemme}{\arabic{lemme}}
\newtheorem{proposition}{Proposition}[section]
\renewcommand{\theproposition}{\arabic{proposition}}
\newtheorem{notations}{Notations}[section]
\newtheorem{problem}{Problème}[section]
\newtheorem{corollary}{Corollaire}[theorem]
\renewcommand{\thecorollary}{\arabic{corollary}}
\newtheorem{property}{Propriété}[section]
\newtheorem{objective}{Objectif}[section]

\theoremstyle{definition}
\newtheorem{definition}{Définition}[section]
\renewcommand{\thedefinition}{\arabic{definition}}
\newtheorem{exercise}{Exercice}[chapter]
\renewcommand{\theexercise}{\arabic{exercise}}
\newtheorem{example}{Exemple}[chapter]
\renewcommand{\theexample}{\arabic{example}}
\newtheorem*{solution}{Solution}
\newtheorem*{application}{Application}
\newtheorem*{notation}{Notation}
\newtheorem*{vocabulary}{Vocabulaire}
\newtheorem*{properties}{Propriétés}



\theoremstyle{remark}
\newtheorem*{remark}{Remarque}
\newtheorem*{rappel}{Rappel}


\usepackage{etoolbox}
\AtBeginEnvironment{exercise}{\small}
\AtBeginEnvironment{example}{\small}

\usepackage{cases}
\usepackage[red]{mypack}

\usepackage[framemethod=TikZ]{mdframed}

\definecolor{bg}{rgb}{0.4,0.25,0.95}
\definecolor{pagebg}{rgb}{0,0,0.5}
\surroundwithmdframed[
   topline=false,
   rightline=false,
   bottomline=false,
   leftmargin=\parindent,
   skipabove=8pt,
   skipbelow=8pt,
   linecolor=blue,
   innerbottommargin=10pt,
   % backgroundcolor=bg,font=\color{orange}\sffamily, fontcolor=white
]{definition}

\usepackage{empheq}
\usepackage[most]{tcolorbox}

\newtcbox{\mymath}[1][]{%
    nobeforeafter, math upper, tcbox raise base,
    enhanced, colframe=blue!30!black,
    colback=red!10, boxrule=1pt,
    #1}

\usepackage{unixode}


\DeclareMathOperator{\ord}{ord}
\DeclareMathOperator{\orb}{orb}
\DeclareMathOperator{\stab}{stab}
\DeclareMathOperator{\Stab}{stab}
\DeclareMathOperator{\ppcm}{ppcm}
\DeclareMathOperator{\conj}{Conj}
\DeclareMathOperator{\End}{End}
\DeclareMathOperator{\rot}{rot}
\DeclareMathOperator{\trs}{trace}
\DeclareMathOperator{\Ind}{Ind}
\DeclareMathOperator{\mat}{Mat}
\DeclareMathOperator{\id}{Id}
\DeclareMathOperator{\vect}{vect}
\DeclareMathOperator{\img}{img}
\DeclareMathOperator{\cov}{Cov}
\DeclareMathOperator{\dist}{dist}
\DeclareMathOperator{\irr}{Irr}
\DeclareMathOperator{\image}{Im}
\DeclareMathOperator{\pd}{\partial}
\DeclareMathOperator{\epi}{epi}
\DeclareMathOperator{\Argmin}{Argmin}
\DeclareMathOperator{\dom}{dom}
\DeclareMathOperator{\proj}{proj}
\DeclareMathOperator{\ctg}{ctg}
\DeclareMathOperator{\supp}{supp}
\DeclareMathOperator{\argmin}{argmin}
\DeclareMathOperator{\mult}{mult}
\DeclareMathOperator{\ch}{ch}
\DeclareMathOperator{\sh}{sh}
\DeclareMathOperator{\rang}{rang}
\DeclareMathOperator{\diam}{diam}
\DeclareMathOperator{\Epigraphe}{Epigraphe}




\usepackage{xcolor}
\everymath{\color{blue}}
%\everymath{\color[rgb]{0,1,1}}
%\pagecolor[rgb]{0,0,0.5}


\newcommand*{\pdtest}[3][]{\ensuremath{\frac{\partial^{#1} #2}{\partial #3}}}

\newcommand*{\deffunc}[6][]{\ensuremath{
\begin{array}{rcl}
#2 : #3 &\rightarrow& #4\\
#5 &\mapsto& #6
\end{array}
}}

\newcommand{\eqcolon}{\mathrel{\resizebox{\widthof{$\mathord{=}$}}{\height}{ $\!\!=\!\!\resizebox{1.2\width}{0.8\height}{\raisebox{0.23ex}{$\mathop{:}$}}\!\!$ }}}
\newcommand{\coloneq}{\mathrel{\resizebox{\widthof{$\mathord{=}$}}{\height}{ $\!\!\resizebox{1.2\width}{0.8\height}{\raisebox{0.23ex}{$\mathop{:}$}}\!\!=\!\!$ }}}
\newcommand{\eqcolonl}{\ensuremath{\mathrel{=\!\!\mathop{:}}}}
\newcommand{\coloneql}{\ensuremath{\mathrel{\mathop{:} \!\! =}}}
\newcommand{\vc}[1]{% inline column vector
  \left(\begin{smallmatrix}#1\end{smallmatrix}\right)%
}
\newcommand{\vr}[1]{% inline row vector
  \begin{smallmatrix}(\,#1\,)\end{smallmatrix}%
}
\makeatletter
\newcommand*{\defeq}{\ =\mathrel{\rlap{%
                     \raisebox{0.3ex}{$\m@th\cdot$}}%
                     \raisebox{-0.3ex}{$\m@th\cdot$}}%
                     }
\makeatother

\newcommand{\mathcircle}[1]{% inline row vector
 \overset{\circ}{#1}
}
\newcommand{\ulim}{% low limit
 \underline{\lim}
}
\newcommand{\ssi}{% iff
\iff
}
\newcommand{\ps}[2]{
\expval{#1 | #2}
}
\newcommand{\df}[1]{
\mqty{#1}
}
\newcommand{\n}[1]{
\norm{#1}
}
\newcommand{\sys}[1]{
\left\{\smqty{#1}\right.
}


\newcommand{\eqdef}{\ensuremath{\overset{\text{def}}=}}


\def\Circlearrowright{\ensuremath{%
  \rotatebox[origin=c]{230}{$\circlearrowright$}}}

\newcommand\ct[1]{\text{\rmfamily\upshape #1}}
\newcommand\question[1]{ {\color{red} ...!? \small #1}}
\newcommand\caz[1]{\left\{\begin{array} #1 \end{array}\right.}
\newcommand\const{\text{\rmfamily\upshape const}}
\newcommand\toP{ \overset{\pro}{\to}}
\newcommand\toPP{ \overset{\text{PP}}{\to}}
\newcommand{\oeq}{\mathrel{\text{\textcircled{$=$}}}}





\usepackage{xcolor}
% \usepackage[normalem]{ulem}
\usepackage{lipsum}
\makeatletter
% \newcommand\colorwave[1][blue]{\bgroup \markoverwith{\lower3.5\p@\hbox{\sixly \textcolor{#1}{\char58}}}\ULon}
%\font\sixly=lasy6 % does not re-load if already loaded, so no memory problem.

\newmdtheoremenv[
linewidth= 1pt,linecolor= blue,%
leftmargin=20,rightmargin=20,innertopmargin=0pt, innerrightmargin=40,%
tikzsetting = { draw=lightgray, line width = 0.3pt,dashed,%
dash pattern = on 15pt off 3pt},%
splittopskip=\topskip,skipbelow=\baselineskip,%
skipabove=\baselineskip,ntheorem,roundcorner=0pt,
% backgroundcolor=pagebg,font=\color{orange}\sffamily, fontcolor=white
]{examplebox}{Exemple}[section]



\newcommand\R{\mathbb{R}}
\newcommand\Z{\mathbb{Z}}
\newcommand\N{\mathbb{N}}
\newcommand\E{\mathbb{E}}
\newcommand\F{\mathcal{F}}
\newcommand\cH{\mathcal{H}}
\newcommand\V{\mathbb{V}}
\newcommand\dmo{ ^{-1} }
\newcommand\kapa{\kappa}
\newcommand\im{Im}
\newcommand\hs{\mathcal{H}}





\usepackage{soul}

\makeatletter
\newcommand*{\whiten}[1]{\llap{\textcolor{white}{{\the\SOUL@token}}\hspace{#1pt}}}
\DeclareRobustCommand*\myul{%
    \def\SOUL@everyspace{\underline{\space}\kern\z@}%
    \def\SOUL@everytoken{%
     \setbox0=\hbox{\the\SOUL@token}%
     \ifdim\dp0>\z@
        \raisebox{\dp0}{\underline{\phantom{\the\SOUL@token}}}%
        \whiten{1}\whiten{0}%
        \whiten{-1}\whiten{-2}%
        \llap{\the\SOUL@token}%
     \else
        \underline{\the\SOUL@token}%
     \fi}%
\SOUL@}
\makeatother

\newcommand*{\demp}{\fontfamily{lmtt}\selectfont}

\DeclareTextFontCommand{\textdemp}{\demp}

\begin{document}

\ifcomment
Multiline
comment
\fi
\ifcomment
\myul{Typesetting test}
% \color[rgb]{1,1,1}
$∑_i^n≠ 60º±∞π∆¬≈√j∫h≤≥µ$

$\CR \R\pro\ind\pro\gS\pro
\mqty[a&b\\c&d]$
$\pro\mathbb{P}$
$\dd{x}$

  \[
    \alpha(x)=\left\{
                \begin{array}{ll}
                  x\\
                  \frac{1}{1+e^{-kx}}\\
                  \frac{e^x-e^{-x}}{e^x+e^{-x}}
                \end{array}
              \right.
  \]

  $\expval{x}$
  
  $\chi_\rho(ghg\dmo)=\Tr(\rho_{ghg\dmo})=\Tr(\rho_g\circ\rho_h\circ\rho\dmo_g)=\Tr(\rho_h)\overset{\mbox{\scalebox{0.5}{$\Tr(AB)=\Tr(BA)$}}}{=}\chi_\rho(h)$
  	$\mathop{\oplus}_{\substack{x\in X}}$

$\mat(\rho_g)=(a_{ij}(g))_{\scriptsize \substack{1\leq i\leq d \\ 1\leq j\leq d}}$ et $\mat(\rho'_g)=(a'_{ij}(g))_{\scriptsize \substack{1\leq i'\leq d' \\ 1\leq j'\leq d'}}$



\[\int_a^b{\mathbb{R}^2}g(u, v)\dd{P_{XY}}(u, v)=\iint g(u,v) f_{XY}(u, v)\dd \lambda(u) \dd \lambda(v)\]
$$\lim_{x\to\infty} f(x)$$	
$$\iiiint_V \mu(t,u,v,w) \,dt\,du\,dv\,dw$$
$$\sum_{n=1}^{\infty} 2^{-n} = 1$$	
\begin{definition}
	Si $X$ et $Y$ sont 2 v.a. ou definit la \textsc{Covariance} entre $X$ et $Y$ comme
	$\cov(X,Y)\overset{\text{def}}{=}\E\left[(X-\E(X))(Y-\E(Y))\right]=\E(XY)-\E(X)\E(Y)$.
\end{definition}
\fi
\pagebreak

% \tableofcontents

% insert your code here
%\input{./algebra/main.tex}
%\input{./geometrie-differentielle/main.tex}
%\input{./probabilite/main.tex}
%\input{./analyse-fonctionnelle/main.tex}
% \input{./Analyse-convexe-et-dualite-en-optimisation/main.tex}
%\input{./tikz/main.tex}
%\input{./Theorie-du-distributions/main.tex}
%\input{./optimisation/mine.tex}
 \input{./modelisation/main.tex}

% yves.aubry@univ-tln.fr : algebra

\end{document}


% yves.aubry@univ-tln.fr : algebra

\end{document}

%% !TEX encoding = UTF-8 Unicode
% !TEX TS-program = xelatex

\documentclass[french]{report}

%\usepackage[utf8]{inputenc}
%\usepackage[T1]{fontenc}
\usepackage{babel}


\newif\ifcomment
%\commenttrue # Show comments

\usepackage{physics}
\usepackage{amssymb}


\usepackage{amsthm}
% \usepackage{thmtools}
\usepackage{mathtools}
\usepackage{amsfonts}

\usepackage{color}

\usepackage{tikz}

\usepackage{geometry}
\geometry{a5paper, margin=0.1in, right=1cm}

\usepackage{dsfont}

\usepackage{graphicx}
\graphicspath{ {images/} }

\usepackage{faktor}

\usepackage{IEEEtrantools}
\usepackage{enumerate}   
\usepackage[PostScript=dvips]{"/Users/aware/Documents/Courses/diagrams"}


\newtheorem{theorem}{Théorème}[section]
\renewcommand{\thetheorem}{\arabic{theorem}}
\newtheorem{lemme}{Lemme}[section]
\renewcommand{\thelemme}{\arabic{lemme}}
\newtheorem{proposition}{Proposition}[section]
\renewcommand{\theproposition}{\arabic{proposition}}
\newtheorem{notations}{Notations}[section]
\newtheorem{problem}{Problème}[section]
\newtheorem{corollary}{Corollaire}[theorem]
\renewcommand{\thecorollary}{\arabic{corollary}}
\newtheorem{property}{Propriété}[section]
\newtheorem{objective}{Objectif}[section]

\theoremstyle{definition}
\newtheorem{definition}{Définition}[section]
\renewcommand{\thedefinition}{\arabic{definition}}
\newtheorem{exercise}{Exercice}[chapter]
\renewcommand{\theexercise}{\arabic{exercise}}
\newtheorem{example}{Exemple}[chapter]
\renewcommand{\theexample}{\arabic{example}}
\newtheorem*{solution}{Solution}
\newtheorem*{application}{Application}
\newtheorem*{notation}{Notation}
\newtheorem*{vocabulary}{Vocabulaire}
\newtheorem*{properties}{Propriétés}



\theoremstyle{remark}
\newtheorem*{remark}{Remarque}
\newtheorem*{rappel}{Rappel}


\usepackage{etoolbox}
\AtBeginEnvironment{exercise}{\small}
\AtBeginEnvironment{example}{\small}

\usepackage{cases}
\usepackage[red]{mypack}

\usepackage[framemethod=TikZ]{mdframed}

\definecolor{bg}{rgb}{0.4,0.25,0.95}
\definecolor{pagebg}{rgb}{0,0,0.5}
\surroundwithmdframed[
   topline=false,
   rightline=false,
   bottomline=false,
   leftmargin=\parindent,
   skipabove=8pt,
   skipbelow=8pt,
   linecolor=blue,
   innerbottommargin=10pt,
   % backgroundcolor=bg,font=\color{orange}\sffamily, fontcolor=white
]{definition}

\usepackage{empheq}
\usepackage[most]{tcolorbox}

\newtcbox{\mymath}[1][]{%
    nobeforeafter, math upper, tcbox raise base,
    enhanced, colframe=blue!30!black,
    colback=red!10, boxrule=1pt,
    #1}

\usepackage{unixode}


\DeclareMathOperator{\ord}{ord}
\DeclareMathOperator{\orb}{orb}
\DeclareMathOperator{\stab}{stab}
\DeclareMathOperator{\Stab}{stab}
\DeclareMathOperator{\ppcm}{ppcm}
\DeclareMathOperator{\conj}{Conj}
\DeclareMathOperator{\End}{End}
\DeclareMathOperator{\rot}{rot}
\DeclareMathOperator{\trs}{trace}
\DeclareMathOperator{\Ind}{Ind}
\DeclareMathOperator{\mat}{Mat}
\DeclareMathOperator{\id}{Id}
\DeclareMathOperator{\vect}{vect}
\DeclareMathOperator{\img}{img}
\DeclareMathOperator{\cov}{Cov}
\DeclareMathOperator{\dist}{dist}
\DeclareMathOperator{\irr}{Irr}
\DeclareMathOperator{\image}{Im}
\DeclareMathOperator{\pd}{\partial}
\DeclareMathOperator{\epi}{epi}
\DeclareMathOperator{\Argmin}{Argmin}
\DeclareMathOperator{\dom}{dom}
\DeclareMathOperator{\proj}{proj}
\DeclareMathOperator{\ctg}{ctg}
\DeclareMathOperator{\supp}{supp}
\DeclareMathOperator{\argmin}{argmin}
\DeclareMathOperator{\mult}{mult}
\DeclareMathOperator{\ch}{ch}
\DeclareMathOperator{\sh}{sh}
\DeclareMathOperator{\rang}{rang}
\DeclareMathOperator{\diam}{diam}
\DeclareMathOperator{\Epigraphe}{Epigraphe}




\usepackage{xcolor}
\everymath{\color{blue}}
%\everymath{\color[rgb]{0,1,1}}
%\pagecolor[rgb]{0,0,0.5}


\newcommand*{\pdtest}[3][]{\ensuremath{\frac{\partial^{#1} #2}{\partial #3}}}

\newcommand*{\deffunc}[6][]{\ensuremath{
\begin{array}{rcl}
#2 : #3 &\rightarrow& #4\\
#5 &\mapsto& #6
\end{array}
}}

\newcommand{\eqcolon}{\mathrel{\resizebox{\widthof{$\mathord{=}$}}{\height}{ $\!\!=\!\!\resizebox{1.2\width}{0.8\height}{\raisebox{0.23ex}{$\mathop{:}$}}\!\!$ }}}
\newcommand{\coloneq}{\mathrel{\resizebox{\widthof{$\mathord{=}$}}{\height}{ $\!\!\resizebox{1.2\width}{0.8\height}{\raisebox{0.23ex}{$\mathop{:}$}}\!\!=\!\!$ }}}
\newcommand{\eqcolonl}{\ensuremath{\mathrel{=\!\!\mathop{:}}}}
\newcommand{\coloneql}{\ensuremath{\mathrel{\mathop{:} \!\! =}}}
\newcommand{\vc}[1]{% inline column vector
  \left(\begin{smallmatrix}#1\end{smallmatrix}\right)%
}
\newcommand{\vr}[1]{% inline row vector
  \begin{smallmatrix}(\,#1\,)\end{smallmatrix}%
}
\makeatletter
\newcommand*{\defeq}{\ =\mathrel{\rlap{%
                     \raisebox{0.3ex}{$\m@th\cdot$}}%
                     \raisebox{-0.3ex}{$\m@th\cdot$}}%
                     }
\makeatother

\newcommand{\mathcircle}[1]{% inline row vector
 \overset{\circ}{#1}
}
\newcommand{\ulim}{% low limit
 \underline{\lim}
}
\newcommand{\ssi}{% iff
\iff
}
\newcommand{\ps}[2]{
\expval{#1 | #2}
}
\newcommand{\df}[1]{
\mqty{#1}
}
\newcommand{\n}[1]{
\norm{#1}
}
\newcommand{\sys}[1]{
\left\{\smqty{#1}\right.
}


\newcommand{\eqdef}{\ensuremath{\overset{\text{def}}=}}


\def\Circlearrowright{\ensuremath{%
  \rotatebox[origin=c]{230}{$\circlearrowright$}}}

\newcommand\ct[1]{\text{\rmfamily\upshape #1}}
\newcommand\question[1]{ {\color{red} ...!? \small #1}}
\newcommand\caz[1]{\left\{\begin{array} #1 \end{array}\right.}
\newcommand\const{\text{\rmfamily\upshape const}}
\newcommand\toP{ \overset{\pro}{\to}}
\newcommand\toPP{ \overset{\text{PP}}{\to}}
\newcommand{\oeq}{\mathrel{\text{\textcircled{$=$}}}}





\usepackage{xcolor}
% \usepackage[normalem]{ulem}
\usepackage{lipsum}
\makeatletter
% \newcommand\colorwave[1][blue]{\bgroup \markoverwith{\lower3.5\p@\hbox{\sixly \textcolor{#1}{\char58}}}\ULon}
%\font\sixly=lasy6 % does not re-load if already loaded, so no memory problem.

\newmdtheoremenv[
linewidth= 1pt,linecolor= blue,%
leftmargin=20,rightmargin=20,innertopmargin=0pt, innerrightmargin=40,%
tikzsetting = { draw=lightgray, line width = 0.3pt,dashed,%
dash pattern = on 15pt off 3pt},%
splittopskip=\topskip,skipbelow=\baselineskip,%
skipabove=\baselineskip,ntheorem,roundcorner=0pt,
% backgroundcolor=pagebg,font=\color{orange}\sffamily, fontcolor=white
]{examplebox}{Exemple}[section]



\newcommand\R{\mathbb{R}}
\newcommand\Z{\mathbb{Z}}
\newcommand\N{\mathbb{N}}
\newcommand\E{\mathbb{E}}
\newcommand\F{\mathcal{F}}
\newcommand\cH{\mathcal{H}}
\newcommand\V{\mathbb{V}}
\newcommand\dmo{ ^{-1} }
\newcommand\kapa{\kappa}
\newcommand\im{Im}
\newcommand\hs{\mathcal{H}}





\usepackage{soul}

\makeatletter
\newcommand*{\whiten}[1]{\llap{\textcolor{white}{{\the\SOUL@token}}\hspace{#1pt}}}
\DeclareRobustCommand*\myul{%
    \def\SOUL@everyspace{\underline{\space}\kern\z@}%
    \def\SOUL@everytoken{%
     \setbox0=\hbox{\the\SOUL@token}%
     \ifdim\dp0>\z@
        \raisebox{\dp0}{\underline{\phantom{\the\SOUL@token}}}%
        \whiten{1}\whiten{0}%
        \whiten{-1}\whiten{-2}%
        \llap{\the\SOUL@token}%
     \else
        \underline{\the\SOUL@token}%
     \fi}%
\SOUL@}
\makeatother

\newcommand*{\demp}{\fontfamily{lmtt}\selectfont}

\DeclareTextFontCommand{\textdemp}{\demp}

\begin{document}

\ifcomment
Multiline
comment
\fi
\ifcomment
\myul{Typesetting test}
% \color[rgb]{1,1,1}
$∑_i^n≠ 60º±∞π∆¬≈√j∫h≤≥µ$

$\CR \R\pro\ind\pro\gS\pro
\mqty[a&b\\c&d]$
$\pro\mathbb{P}$
$\dd{x}$

  \[
    \alpha(x)=\left\{
                \begin{array}{ll}
                  x\\
                  \frac{1}{1+e^{-kx}}\\
                  \frac{e^x-e^{-x}}{e^x+e^{-x}}
                \end{array}
              \right.
  \]

  $\expval{x}$
  
  $\chi_\rho(ghg\dmo)=\Tr(\rho_{ghg\dmo})=\Tr(\rho_g\circ\rho_h\circ\rho\dmo_g)=\Tr(\rho_h)\overset{\mbox{\scalebox{0.5}{$\Tr(AB)=\Tr(BA)$}}}{=}\chi_\rho(h)$
  	$\mathop{\oplus}_{\substack{x\in X}}$

$\mat(\rho_g)=(a_{ij}(g))_{\scriptsize \substack{1\leq i\leq d \\ 1\leq j\leq d}}$ et $\mat(\rho'_g)=(a'_{ij}(g))_{\scriptsize \substack{1\leq i'\leq d' \\ 1\leq j'\leq d'}}$



\[\int_a^b{\mathbb{R}^2}g(u, v)\dd{P_{XY}}(u, v)=\iint g(u,v) f_{XY}(u, v)\dd \lambda(u) \dd \lambda(v)\]
$$\lim_{x\to\infty} f(x)$$	
$$\iiiint_V \mu(t,u,v,w) \,dt\,du\,dv\,dw$$
$$\sum_{n=1}^{\infty} 2^{-n} = 1$$	
\begin{definition}
	Si $X$ et $Y$ sont 2 v.a. ou definit la \textsc{Covariance} entre $X$ et $Y$ comme
	$\cov(X,Y)\overset{\text{def}}{=}\E\left[(X-\E(X))(Y-\E(Y))\right]=\E(XY)-\E(X)\E(Y)$.
\end{definition}
\fi
\pagebreak

% \tableofcontents

% insert your code here
%% !TEX encoding = UTF-8 Unicode
% !TEX TS-program = xelatex

\documentclass[french]{report}

%\usepackage[utf8]{inputenc}
%\usepackage[T1]{fontenc}
\usepackage{babel}


\newif\ifcomment
%\commenttrue # Show comments

\usepackage{physics}
\usepackage{amssymb}


\usepackage{amsthm}
% \usepackage{thmtools}
\usepackage{mathtools}
\usepackage{amsfonts}

\usepackage{color}

\usepackage{tikz}

\usepackage{geometry}
\geometry{a5paper, margin=0.1in, right=1cm}

\usepackage{dsfont}

\usepackage{graphicx}
\graphicspath{ {images/} }

\usepackage{faktor}

\usepackage{IEEEtrantools}
\usepackage{enumerate}   
\usepackage[PostScript=dvips]{"/Users/aware/Documents/Courses/diagrams"}


\newtheorem{theorem}{Théorème}[section]
\renewcommand{\thetheorem}{\arabic{theorem}}
\newtheorem{lemme}{Lemme}[section]
\renewcommand{\thelemme}{\arabic{lemme}}
\newtheorem{proposition}{Proposition}[section]
\renewcommand{\theproposition}{\arabic{proposition}}
\newtheorem{notations}{Notations}[section]
\newtheorem{problem}{Problème}[section]
\newtheorem{corollary}{Corollaire}[theorem]
\renewcommand{\thecorollary}{\arabic{corollary}}
\newtheorem{property}{Propriété}[section]
\newtheorem{objective}{Objectif}[section]

\theoremstyle{definition}
\newtheorem{definition}{Définition}[section]
\renewcommand{\thedefinition}{\arabic{definition}}
\newtheorem{exercise}{Exercice}[chapter]
\renewcommand{\theexercise}{\arabic{exercise}}
\newtheorem{example}{Exemple}[chapter]
\renewcommand{\theexample}{\arabic{example}}
\newtheorem*{solution}{Solution}
\newtheorem*{application}{Application}
\newtheorem*{notation}{Notation}
\newtheorem*{vocabulary}{Vocabulaire}
\newtheorem*{properties}{Propriétés}



\theoremstyle{remark}
\newtheorem*{remark}{Remarque}
\newtheorem*{rappel}{Rappel}


\usepackage{etoolbox}
\AtBeginEnvironment{exercise}{\small}
\AtBeginEnvironment{example}{\small}

\usepackage{cases}
\usepackage[red]{mypack}

\usepackage[framemethod=TikZ]{mdframed}

\definecolor{bg}{rgb}{0.4,0.25,0.95}
\definecolor{pagebg}{rgb}{0,0,0.5}
\surroundwithmdframed[
   topline=false,
   rightline=false,
   bottomline=false,
   leftmargin=\parindent,
   skipabove=8pt,
   skipbelow=8pt,
   linecolor=blue,
   innerbottommargin=10pt,
   % backgroundcolor=bg,font=\color{orange}\sffamily, fontcolor=white
]{definition}

\usepackage{empheq}
\usepackage[most]{tcolorbox}

\newtcbox{\mymath}[1][]{%
    nobeforeafter, math upper, tcbox raise base,
    enhanced, colframe=blue!30!black,
    colback=red!10, boxrule=1pt,
    #1}

\usepackage{unixode}


\DeclareMathOperator{\ord}{ord}
\DeclareMathOperator{\orb}{orb}
\DeclareMathOperator{\stab}{stab}
\DeclareMathOperator{\Stab}{stab}
\DeclareMathOperator{\ppcm}{ppcm}
\DeclareMathOperator{\conj}{Conj}
\DeclareMathOperator{\End}{End}
\DeclareMathOperator{\rot}{rot}
\DeclareMathOperator{\trs}{trace}
\DeclareMathOperator{\Ind}{Ind}
\DeclareMathOperator{\mat}{Mat}
\DeclareMathOperator{\id}{Id}
\DeclareMathOperator{\vect}{vect}
\DeclareMathOperator{\img}{img}
\DeclareMathOperator{\cov}{Cov}
\DeclareMathOperator{\dist}{dist}
\DeclareMathOperator{\irr}{Irr}
\DeclareMathOperator{\image}{Im}
\DeclareMathOperator{\pd}{\partial}
\DeclareMathOperator{\epi}{epi}
\DeclareMathOperator{\Argmin}{Argmin}
\DeclareMathOperator{\dom}{dom}
\DeclareMathOperator{\proj}{proj}
\DeclareMathOperator{\ctg}{ctg}
\DeclareMathOperator{\supp}{supp}
\DeclareMathOperator{\argmin}{argmin}
\DeclareMathOperator{\mult}{mult}
\DeclareMathOperator{\ch}{ch}
\DeclareMathOperator{\sh}{sh}
\DeclareMathOperator{\rang}{rang}
\DeclareMathOperator{\diam}{diam}
\DeclareMathOperator{\Epigraphe}{Epigraphe}




\usepackage{xcolor}
\everymath{\color{blue}}
%\everymath{\color[rgb]{0,1,1}}
%\pagecolor[rgb]{0,0,0.5}


\newcommand*{\pdtest}[3][]{\ensuremath{\frac{\partial^{#1} #2}{\partial #3}}}

\newcommand*{\deffunc}[6][]{\ensuremath{
\begin{array}{rcl}
#2 : #3 &\rightarrow& #4\\
#5 &\mapsto& #6
\end{array}
}}

\newcommand{\eqcolon}{\mathrel{\resizebox{\widthof{$\mathord{=}$}}{\height}{ $\!\!=\!\!\resizebox{1.2\width}{0.8\height}{\raisebox{0.23ex}{$\mathop{:}$}}\!\!$ }}}
\newcommand{\coloneq}{\mathrel{\resizebox{\widthof{$\mathord{=}$}}{\height}{ $\!\!\resizebox{1.2\width}{0.8\height}{\raisebox{0.23ex}{$\mathop{:}$}}\!\!=\!\!$ }}}
\newcommand{\eqcolonl}{\ensuremath{\mathrel{=\!\!\mathop{:}}}}
\newcommand{\coloneql}{\ensuremath{\mathrel{\mathop{:} \!\! =}}}
\newcommand{\vc}[1]{% inline column vector
  \left(\begin{smallmatrix}#1\end{smallmatrix}\right)%
}
\newcommand{\vr}[1]{% inline row vector
  \begin{smallmatrix}(\,#1\,)\end{smallmatrix}%
}
\makeatletter
\newcommand*{\defeq}{\ =\mathrel{\rlap{%
                     \raisebox{0.3ex}{$\m@th\cdot$}}%
                     \raisebox{-0.3ex}{$\m@th\cdot$}}%
                     }
\makeatother

\newcommand{\mathcircle}[1]{% inline row vector
 \overset{\circ}{#1}
}
\newcommand{\ulim}{% low limit
 \underline{\lim}
}
\newcommand{\ssi}{% iff
\iff
}
\newcommand{\ps}[2]{
\expval{#1 | #2}
}
\newcommand{\df}[1]{
\mqty{#1}
}
\newcommand{\n}[1]{
\norm{#1}
}
\newcommand{\sys}[1]{
\left\{\smqty{#1}\right.
}


\newcommand{\eqdef}{\ensuremath{\overset{\text{def}}=}}


\def\Circlearrowright{\ensuremath{%
  \rotatebox[origin=c]{230}{$\circlearrowright$}}}

\newcommand\ct[1]{\text{\rmfamily\upshape #1}}
\newcommand\question[1]{ {\color{red} ...!? \small #1}}
\newcommand\caz[1]{\left\{\begin{array} #1 \end{array}\right.}
\newcommand\const{\text{\rmfamily\upshape const}}
\newcommand\toP{ \overset{\pro}{\to}}
\newcommand\toPP{ \overset{\text{PP}}{\to}}
\newcommand{\oeq}{\mathrel{\text{\textcircled{$=$}}}}





\usepackage{xcolor}
% \usepackage[normalem]{ulem}
\usepackage{lipsum}
\makeatletter
% \newcommand\colorwave[1][blue]{\bgroup \markoverwith{\lower3.5\p@\hbox{\sixly \textcolor{#1}{\char58}}}\ULon}
%\font\sixly=lasy6 % does not re-load if already loaded, so no memory problem.

\newmdtheoremenv[
linewidth= 1pt,linecolor= blue,%
leftmargin=20,rightmargin=20,innertopmargin=0pt, innerrightmargin=40,%
tikzsetting = { draw=lightgray, line width = 0.3pt,dashed,%
dash pattern = on 15pt off 3pt},%
splittopskip=\topskip,skipbelow=\baselineskip,%
skipabove=\baselineskip,ntheorem,roundcorner=0pt,
% backgroundcolor=pagebg,font=\color{orange}\sffamily, fontcolor=white
]{examplebox}{Exemple}[section]



\newcommand\R{\mathbb{R}}
\newcommand\Z{\mathbb{Z}}
\newcommand\N{\mathbb{N}}
\newcommand\E{\mathbb{E}}
\newcommand\F{\mathcal{F}}
\newcommand\cH{\mathcal{H}}
\newcommand\V{\mathbb{V}}
\newcommand\dmo{ ^{-1} }
\newcommand\kapa{\kappa}
\newcommand\im{Im}
\newcommand\hs{\mathcal{H}}





\usepackage{soul}

\makeatletter
\newcommand*{\whiten}[1]{\llap{\textcolor{white}{{\the\SOUL@token}}\hspace{#1pt}}}
\DeclareRobustCommand*\myul{%
    \def\SOUL@everyspace{\underline{\space}\kern\z@}%
    \def\SOUL@everytoken{%
     \setbox0=\hbox{\the\SOUL@token}%
     \ifdim\dp0>\z@
        \raisebox{\dp0}{\underline{\phantom{\the\SOUL@token}}}%
        \whiten{1}\whiten{0}%
        \whiten{-1}\whiten{-2}%
        \llap{\the\SOUL@token}%
     \else
        \underline{\the\SOUL@token}%
     \fi}%
\SOUL@}
\makeatother

\newcommand*{\demp}{\fontfamily{lmtt}\selectfont}

\DeclareTextFontCommand{\textdemp}{\demp}

\begin{document}

\ifcomment
Multiline
comment
\fi
\ifcomment
\myul{Typesetting test}
% \color[rgb]{1,1,1}
$∑_i^n≠ 60º±∞π∆¬≈√j∫h≤≥µ$

$\CR \R\pro\ind\pro\gS\pro
\mqty[a&b\\c&d]$
$\pro\mathbb{P}$
$\dd{x}$

  \[
    \alpha(x)=\left\{
                \begin{array}{ll}
                  x\\
                  \frac{1}{1+e^{-kx}}\\
                  \frac{e^x-e^{-x}}{e^x+e^{-x}}
                \end{array}
              \right.
  \]

  $\expval{x}$
  
  $\chi_\rho(ghg\dmo)=\Tr(\rho_{ghg\dmo})=\Tr(\rho_g\circ\rho_h\circ\rho\dmo_g)=\Tr(\rho_h)\overset{\mbox{\scalebox{0.5}{$\Tr(AB)=\Tr(BA)$}}}{=}\chi_\rho(h)$
  	$\mathop{\oplus}_{\substack{x\in X}}$

$\mat(\rho_g)=(a_{ij}(g))_{\scriptsize \substack{1\leq i\leq d \\ 1\leq j\leq d}}$ et $\mat(\rho'_g)=(a'_{ij}(g))_{\scriptsize \substack{1\leq i'\leq d' \\ 1\leq j'\leq d'}}$



\[\int_a^b{\mathbb{R}^2}g(u, v)\dd{P_{XY}}(u, v)=\iint g(u,v) f_{XY}(u, v)\dd \lambda(u) \dd \lambda(v)\]
$$\lim_{x\to\infty} f(x)$$	
$$\iiiint_V \mu(t,u,v,w) \,dt\,du\,dv\,dw$$
$$\sum_{n=1}^{\infty} 2^{-n} = 1$$	
\begin{definition}
	Si $X$ et $Y$ sont 2 v.a. ou definit la \textsc{Covariance} entre $X$ et $Y$ comme
	$\cov(X,Y)\overset{\text{def}}{=}\E\left[(X-\E(X))(Y-\E(Y))\right]=\E(XY)-\E(X)\E(Y)$.
\end{definition}
\fi
\pagebreak

% \tableofcontents

% insert your code here
%\input{./algebra/main.tex}
%\input{./geometrie-differentielle/main.tex}
%\input{./probabilite/main.tex}
%\input{./analyse-fonctionnelle/main.tex}
% \input{./Analyse-convexe-et-dualite-en-optimisation/main.tex}
%\input{./tikz/main.tex}
%\input{./Theorie-du-distributions/main.tex}
%\input{./optimisation/mine.tex}
 \input{./modelisation/main.tex}

% yves.aubry@univ-tln.fr : algebra

\end{document}

%% !TEX encoding = UTF-8 Unicode
% !TEX TS-program = xelatex

\documentclass[french]{report}

%\usepackage[utf8]{inputenc}
%\usepackage[T1]{fontenc}
\usepackage{babel}


\newif\ifcomment
%\commenttrue # Show comments

\usepackage{physics}
\usepackage{amssymb}


\usepackage{amsthm}
% \usepackage{thmtools}
\usepackage{mathtools}
\usepackage{amsfonts}

\usepackage{color}

\usepackage{tikz}

\usepackage{geometry}
\geometry{a5paper, margin=0.1in, right=1cm}

\usepackage{dsfont}

\usepackage{graphicx}
\graphicspath{ {images/} }

\usepackage{faktor}

\usepackage{IEEEtrantools}
\usepackage{enumerate}   
\usepackage[PostScript=dvips]{"/Users/aware/Documents/Courses/diagrams"}


\newtheorem{theorem}{Théorème}[section]
\renewcommand{\thetheorem}{\arabic{theorem}}
\newtheorem{lemme}{Lemme}[section]
\renewcommand{\thelemme}{\arabic{lemme}}
\newtheorem{proposition}{Proposition}[section]
\renewcommand{\theproposition}{\arabic{proposition}}
\newtheorem{notations}{Notations}[section]
\newtheorem{problem}{Problème}[section]
\newtheorem{corollary}{Corollaire}[theorem]
\renewcommand{\thecorollary}{\arabic{corollary}}
\newtheorem{property}{Propriété}[section]
\newtheorem{objective}{Objectif}[section]

\theoremstyle{definition}
\newtheorem{definition}{Définition}[section]
\renewcommand{\thedefinition}{\arabic{definition}}
\newtheorem{exercise}{Exercice}[chapter]
\renewcommand{\theexercise}{\arabic{exercise}}
\newtheorem{example}{Exemple}[chapter]
\renewcommand{\theexample}{\arabic{example}}
\newtheorem*{solution}{Solution}
\newtheorem*{application}{Application}
\newtheorem*{notation}{Notation}
\newtheorem*{vocabulary}{Vocabulaire}
\newtheorem*{properties}{Propriétés}



\theoremstyle{remark}
\newtheorem*{remark}{Remarque}
\newtheorem*{rappel}{Rappel}


\usepackage{etoolbox}
\AtBeginEnvironment{exercise}{\small}
\AtBeginEnvironment{example}{\small}

\usepackage{cases}
\usepackage[red]{mypack}

\usepackage[framemethod=TikZ]{mdframed}

\definecolor{bg}{rgb}{0.4,0.25,0.95}
\definecolor{pagebg}{rgb}{0,0,0.5}
\surroundwithmdframed[
   topline=false,
   rightline=false,
   bottomline=false,
   leftmargin=\parindent,
   skipabove=8pt,
   skipbelow=8pt,
   linecolor=blue,
   innerbottommargin=10pt,
   % backgroundcolor=bg,font=\color{orange}\sffamily, fontcolor=white
]{definition}

\usepackage{empheq}
\usepackage[most]{tcolorbox}

\newtcbox{\mymath}[1][]{%
    nobeforeafter, math upper, tcbox raise base,
    enhanced, colframe=blue!30!black,
    colback=red!10, boxrule=1pt,
    #1}

\usepackage{unixode}


\DeclareMathOperator{\ord}{ord}
\DeclareMathOperator{\orb}{orb}
\DeclareMathOperator{\stab}{stab}
\DeclareMathOperator{\Stab}{stab}
\DeclareMathOperator{\ppcm}{ppcm}
\DeclareMathOperator{\conj}{Conj}
\DeclareMathOperator{\End}{End}
\DeclareMathOperator{\rot}{rot}
\DeclareMathOperator{\trs}{trace}
\DeclareMathOperator{\Ind}{Ind}
\DeclareMathOperator{\mat}{Mat}
\DeclareMathOperator{\id}{Id}
\DeclareMathOperator{\vect}{vect}
\DeclareMathOperator{\img}{img}
\DeclareMathOperator{\cov}{Cov}
\DeclareMathOperator{\dist}{dist}
\DeclareMathOperator{\irr}{Irr}
\DeclareMathOperator{\image}{Im}
\DeclareMathOperator{\pd}{\partial}
\DeclareMathOperator{\epi}{epi}
\DeclareMathOperator{\Argmin}{Argmin}
\DeclareMathOperator{\dom}{dom}
\DeclareMathOperator{\proj}{proj}
\DeclareMathOperator{\ctg}{ctg}
\DeclareMathOperator{\supp}{supp}
\DeclareMathOperator{\argmin}{argmin}
\DeclareMathOperator{\mult}{mult}
\DeclareMathOperator{\ch}{ch}
\DeclareMathOperator{\sh}{sh}
\DeclareMathOperator{\rang}{rang}
\DeclareMathOperator{\diam}{diam}
\DeclareMathOperator{\Epigraphe}{Epigraphe}




\usepackage{xcolor}
\everymath{\color{blue}}
%\everymath{\color[rgb]{0,1,1}}
%\pagecolor[rgb]{0,0,0.5}


\newcommand*{\pdtest}[3][]{\ensuremath{\frac{\partial^{#1} #2}{\partial #3}}}

\newcommand*{\deffunc}[6][]{\ensuremath{
\begin{array}{rcl}
#2 : #3 &\rightarrow& #4\\
#5 &\mapsto& #6
\end{array}
}}

\newcommand{\eqcolon}{\mathrel{\resizebox{\widthof{$\mathord{=}$}}{\height}{ $\!\!=\!\!\resizebox{1.2\width}{0.8\height}{\raisebox{0.23ex}{$\mathop{:}$}}\!\!$ }}}
\newcommand{\coloneq}{\mathrel{\resizebox{\widthof{$\mathord{=}$}}{\height}{ $\!\!\resizebox{1.2\width}{0.8\height}{\raisebox{0.23ex}{$\mathop{:}$}}\!\!=\!\!$ }}}
\newcommand{\eqcolonl}{\ensuremath{\mathrel{=\!\!\mathop{:}}}}
\newcommand{\coloneql}{\ensuremath{\mathrel{\mathop{:} \!\! =}}}
\newcommand{\vc}[1]{% inline column vector
  \left(\begin{smallmatrix}#1\end{smallmatrix}\right)%
}
\newcommand{\vr}[1]{% inline row vector
  \begin{smallmatrix}(\,#1\,)\end{smallmatrix}%
}
\makeatletter
\newcommand*{\defeq}{\ =\mathrel{\rlap{%
                     \raisebox{0.3ex}{$\m@th\cdot$}}%
                     \raisebox{-0.3ex}{$\m@th\cdot$}}%
                     }
\makeatother

\newcommand{\mathcircle}[1]{% inline row vector
 \overset{\circ}{#1}
}
\newcommand{\ulim}{% low limit
 \underline{\lim}
}
\newcommand{\ssi}{% iff
\iff
}
\newcommand{\ps}[2]{
\expval{#1 | #2}
}
\newcommand{\df}[1]{
\mqty{#1}
}
\newcommand{\n}[1]{
\norm{#1}
}
\newcommand{\sys}[1]{
\left\{\smqty{#1}\right.
}


\newcommand{\eqdef}{\ensuremath{\overset{\text{def}}=}}


\def\Circlearrowright{\ensuremath{%
  \rotatebox[origin=c]{230}{$\circlearrowright$}}}

\newcommand\ct[1]{\text{\rmfamily\upshape #1}}
\newcommand\question[1]{ {\color{red} ...!? \small #1}}
\newcommand\caz[1]{\left\{\begin{array} #1 \end{array}\right.}
\newcommand\const{\text{\rmfamily\upshape const}}
\newcommand\toP{ \overset{\pro}{\to}}
\newcommand\toPP{ \overset{\text{PP}}{\to}}
\newcommand{\oeq}{\mathrel{\text{\textcircled{$=$}}}}





\usepackage{xcolor}
% \usepackage[normalem]{ulem}
\usepackage{lipsum}
\makeatletter
% \newcommand\colorwave[1][blue]{\bgroup \markoverwith{\lower3.5\p@\hbox{\sixly \textcolor{#1}{\char58}}}\ULon}
%\font\sixly=lasy6 % does not re-load if already loaded, so no memory problem.

\newmdtheoremenv[
linewidth= 1pt,linecolor= blue,%
leftmargin=20,rightmargin=20,innertopmargin=0pt, innerrightmargin=40,%
tikzsetting = { draw=lightgray, line width = 0.3pt,dashed,%
dash pattern = on 15pt off 3pt},%
splittopskip=\topskip,skipbelow=\baselineskip,%
skipabove=\baselineskip,ntheorem,roundcorner=0pt,
% backgroundcolor=pagebg,font=\color{orange}\sffamily, fontcolor=white
]{examplebox}{Exemple}[section]



\newcommand\R{\mathbb{R}}
\newcommand\Z{\mathbb{Z}}
\newcommand\N{\mathbb{N}}
\newcommand\E{\mathbb{E}}
\newcommand\F{\mathcal{F}}
\newcommand\cH{\mathcal{H}}
\newcommand\V{\mathbb{V}}
\newcommand\dmo{ ^{-1} }
\newcommand\kapa{\kappa}
\newcommand\im{Im}
\newcommand\hs{\mathcal{H}}





\usepackage{soul}

\makeatletter
\newcommand*{\whiten}[1]{\llap{\textcolor{white}{{\the\SOUL@token}}\hspace{#1pt}}}
\DeclareRobustCommand*\myul{%
    \def\SOUL@everyspace{\underline{\space}\kern\z@}%
    \def\SOUL@everytoken{%
     \setbox0=\hbox{\the\SOUL@token}%
     \ifdim\dp0>\z@
        \raisebox{\dp0}{\underline{\phantom{\the\SOUL@token}}}%
        \whiten{1}\whiten{0}%
        \whiten{-1}\whiten{-2}%
        \llap{\the\SOUL@token}%
     \else
        \underline{\the\SOUL@token}%
     \fi}%
\SOUL@}
\makeatother

\newcommand*{\demp}{\fontfamily{lmtt}\selectfont}

\DeclareTextFontCommand{\textdemp}{\demp}

\begin{document}

\ifcomment
Multiline
comment
\fi
\ifcomment
\myul{Typesetting test}
% \color[rgb]{1,1,1}
$∑_i^n≠ 60º±∞π∆¬≈√j∫h≤≥µ$

$\CR \R\pro\ind\pro\gS\pro
\mqty[a&b\\c&d]$
$\pro\mathbb{P}$
$\dd{x}$

  \[
    \alpha(x)=\left\{
                \begin{array}{ll}
                  x\\
                  \frac{1}{1+e^{-kx}}\\
                  \frac{e^x-e^{-x}}{e^x+e^{-x}}
                \end{array}
              \right.
  \]

  $\expval{x}$
  
  $\chi_\rho(ghg\dmo)=\Tr(\rho_{ghg\dmo})=\Tr(\rho_g\circ\rho_h\circ\rho\dmo_g)=\Tr(\rho_h)\overset{\mbox{\scalebox{0.5}{$\Tr(AB)=\Tr(BA)$}}}{=}\chi_\rho(h)$
  	$\mathop{\oplus}_{\substack{x\in X}}$

$\mat(\rho_g)=(a_{ij}(g))_{\scriptsize \substack{1\leq i\leq d \\ 1\leq j\leq d}}$ et $\mat(\rho'_g)=(a'_{ij}(g))_{\scriptsize \substack{1\leq i'\leq d' \\ 1\leq j'\leq d'}}$



\[\int_a^b{\mathbb{R}^2}g(u, v)\dd{P_{XY}}(u, v)=\iint g(u,v) f_{XY}(u, v)\dd \lambda(u) \dd \lambda(v)\]
$$\lim_{x\to\infty} f(x)$$	
$$\iiiint_V \mu(t,u,v,w) \,dt\,du\,dv\,dw$$
$$\sum_{n=1}^{\infty} 2^{-n} = 1$$	
\begin{definition}
	Si $X$ et $Y$ sont 2 v.a. ou definit la \textsc{Covariance} entre $X$ et $Y$ comme
	$\cov(X,Y)\overset{\text{def}}{=}\E\left[(X-\E(X))(Y-\E(Y))\right]=\E(XY)-\E(X)\E(Y)$.
\end{definition}
\fi
\pagebreak

% \tableofcontents

% insert your code here
%\input{./algebra/main.tex}
%\input{./geometrie-differentielle/main.tex}
%\input{./probabilite/main.tex}
%\input{./analyse-fonctionnelle/main.tex}
% \input{./Analyse-convexe-et-dualite-en-optimisation/main.tex}
%\input{./tikz/main.tex}
%\input{./Theorie-du-distributions/main.tex}
%\input{./optimisation/mine.tex}
 \input{./modelisation/main.tex}

% yves.aubry@univ-tln.fr : algebra

\end{document}

%% !TEX encoding = UTF-8 Unicode
% !TEX TS-program = xelatex

\documentclass[french]{report}

%\usepackage[utf8]{inputenc}
%\usepackage[T1]{fontenc}
\usepackage{babel}


\newif\ifcomment
%\commenttrue # Show comments

\usepackage{physics}
\usepackage{amssymb}


\usepackage{amsthm}
% \usepackage{thmtools}
\usepackage{mathtools}
\usepackage{amsfonts}

\usepackage{color}

\usepackage{tikz}

\usepackage{geometry}
\geometry{a5paper, margin=0.1in, right=1cm}

\usepackage{dsfont}

\usepackage{graphicx}
\graphicspath{ {images/} }

\usepackage{faktor}

\usepackage{IEEEtrantools}
\usepackage{enumerate}   
\usepackage[PostScript=dvips]{"/Users/aware/Documents/Courses/diagrams"}


\newtheorem{theorem}{Théorème}[section]
\renewcommand{\thetheorem}{\arabic{theorem}}
\newtheorem{lemme}{Lemme}[section]
\renewcommand{\thelemme}{\arabic{lemme}}
\newtheorem{proposition}{Proposition}[section]
\renewcommand{\theproposition}{\arabic{proposition}}
\newtheorem{notations}{Notations}[section]
\newtheorem{problem}{Problème}[section]
\newtheorem{corollary}{Corollaire}[theorem]
\renewcommand{\thecorollary}{\arabic{corollary}}
\newtheorem{property}{Propriété}[section]
\newtheorem{objective}{Objectif}[section]

\theoremstyle{definition}
\newtheorem{definition}{Définition}[section]
\renewcommand{\thedefinition}{\arabic{definition}}
\newtheorem{exercise}{Exercice}[chapter]
\renewcommand{\theexercise}{\arabic{exercise}}
\newtheorem{example}{Exemple}[chapter]
\renewcommand{\theexample}{\arabic{example}}
\newtheorem*{solution}{Solution}
\newtheorem*{application}{Application}
\newtheorem*{notation}{Notation}
\newtheorem*{vocabulary}{Vocabulaire}
\newtheorem*{properties}{Propriétés}



\theoremstyle{remark}
\newtheorem*{remark}{Remarque}
\newtheorem*{rappel}{Rappel}


\usepackage{etoolbox}
\AtBeginEnvironment{exercise}{\small}
\AtBeginEnvironment{example}{\small}

\usepackage{cases}
\usepackage[red]{mypack}

\usepackage[framemethod=TikZ]{mdframed}

\definecolor{bg}{rgb}{0.4,0.25,0.95}
\definecolor{pagebg}{rgb}{0,0,0.5}
\surroundwithmdframed[
   topline=false,
   rightline=false,
   bottomline=false,
   leftmargin=\parindent,
   skipabove=8pt,
   skipbelow=8pt,
   linecolor=blue,
   innerbottommargin=10pt,
   % backgroundcolor=bg,font=\color{orange}\sffamily, fontcolor=white
]{definition}

\usepackage{empheq}
\usepackage[most]{tcolorbox}

\newtcbox{\mymath}[1][]{%
    nobeforeafter, math upper, tcbox raise base,
    enhanced, colframe=blue!30!black,
    colback=red!10, boxrule=1pt,
    #1}

\usepackage{unixode}


\DeclareMathOperator{\ord}{ord}
\DeclareMathOperator{\orb}{orb}
\DeclareMathOperator{\stab}{stab}
\DeclareMathOperator{\Stab}{stab}
\DeclareMathOperator{\ppcm}{ppcm}
\DeclareMathOperator{\conj}{Conj}
\DeclareMathOperator{\End}{End}
\DeclareMathOperator{\rot}{rot}
\DeclareMathOperator{\trs}{trace}
\DeclareMathOperator{\Ind}{Ind}
\DeclareMathOperator{\mat}{Mat}
\DeclareMathOperator{\id}{Id}
\DeclareMathOperator{\vect}{vect}
\DeclareMathOperator{\img}{img}
\DeclareMathOperator{\cov}{Cov}
\DeclareMathOperator{\dist}{dist}
\DeclareMathOperator{\irr}{Irr}
\DeclareMathOperator{\image}{Im}
\DeclareMathOperator{\pd}{\partial}
\DeclareMathOperator{\epi}{epi}
\DeclareMathOperator{\Argmin}{Argmin}
\DeclareMathOperator{\dom}{dom}
\DeclareMathOperator{\proj}{proj}
\DeclareMathOperator{\ctg}{ctg}
\DeclareMathOperator{\supp}{supp}
\DeclareMathOperator{\argmin}{argmin}
\DeclareMathOperator{\mult}{mult}
\DeclareMathOperator{\ch}{ch}
\DeclareMathOperator{\sh}{sh}
\DeclareMathOperator{\rang}{rang}
\DeclareMathOperator{\diam}{diam}
\DeclareMathOperator{\Epigraphe}{Epigraphe}




\usepackage{xcolor}
\everymath{\color{blue}}
%\everymath{\color[rgb]{0,1,1}}
%\pagecolor[rgb]{0,0,0.5}


\newcommand*{\pdtest}[3][]{\ensuremath{\frac{\partial^{#1} #2}{\partial #3}}}

\newcommand*{\deffunc}[6][]{\ensuremath{
\begin{array}{rcl}
#2 : #3 &\rightarrow& #4\\
#5 &\mapsto& #6
\end{array}
}}

\newcommand{\eqcolon}{\mathrel{\resizebox{\widthof{$\mathord{=}$}}{\height}{ $\!\!=\!\!\resizebox{1.2\width}{0.8\height}{\raisebox{0.23ex}{$\mathop{:}$}}\!\!$ }}}
\newcommand{\coloneq}{\mathrel{\resizebox{\widthof{$\mathord{=}$}}{\height}{ $\!\!\resizebox{1.2\width}{0.8\height}{\raisebox{0.23ex}{$\mathop{:}$}}\!\!=\!\!$ }}}
\newcommand{\eqcolonl}{\ensuremath{\mathrel{=\!\!\mathop{:}}}}
\newcommand{\coloneql}{\ensuremath{\mathrel{\mathop{:} \!\! =}}}
\newcommand{\vc}[1]{% inline column vector
  \left(\begin{smallmatrix}#1\end{smallmatrix}\right)%
}
\newcommand{\vr}[1]{% inline row vector
  \begin{smallmatrix}(\,#1\,)\end{smallmatrix}%
}
\makeatletter
\newcommand*{\defeq}{\ =\mathrel{\rlap{%
                     \raisebox{0.3ex}{$\m@th\cdot$}}%
                     \raisebox{-0.3ex}{$\m@th\cdot$}}%
                     }
\makeatother

\newcommand{\mathcircle}[1]{% inline row vector
 \overset{\circ}{#1}
}
\newcommand{\ulim}{% low limit
 \underline{\lim}
}
\newcommand{\ssi}{% iff
\iff
}
\newcommand{\ps}[2]{
\expval{#1 | #2}
}
\newcommand{\df}[1]{
\mqty{#1}
}
\newcommand{\n}[1]{
\norm{#1}
}
\newcommand{\sys}[1]{
\left\{\smqty{#1}\right.
}


\newcommand{\eqdef}{\ensuremath{\overset{\text{def}}=}}


\def\Circlearrowright{\ensuremath{%
  \rotatebox[origin=c]{230}{$\circlearrowright$}}}

\newcommand\ct[1]{\text{\rmfamily\upshape #1}}
\newcommand\question[1]{ {\color{red} ...!? \small #1}}
\newcommand\caz[1]{\left\{\begin{array} #1 \end{array}\right.}
\newcommand\const{\text{\rmfamily\upshape const}}
\newcommand\toP{ \overset{\pro}{\to}}
\newcommand\toPP{ \overset{\text{PP}}{\to}}
\newcommand{\oeq}{\mathrel{\text{\textcircled{$=$}}}}





\usepackage{xcolor}
% \usepackage[normalem]{ulem}
\usepackage{lipsum}
\makeatletter
% \newcommand\colorwave[1][blue]{\bgroup \markoverwith{\lower3.5\p@\hbox{\sixly \textcolor{#1}{\char58}}}\ULon}
%\font\sixly=lasy6 % does not re-load if already loaded, so no memory problem.

\newmdtheoremenv[
linewidth= 1pt,linecolor= blue,%
leftmargin=20,rightmargin=20,innertopmargin=0pt, innerrightmargin=40,%
tikzsetting = { draw=lightgray, line width = 0.3pt,dashed,%
dash pattern = on 15pt off 3pt},%
splittopskip=\topskip,skipbelow=\baselineskip,%
skipabove=\baselineskip,ntheorem,roundcorner=0pt,
% backgroundcolor=pagebg,font=\color{orange}\sffamily, fontcolor=white
]{examplebox}{Exemple}[section]



\newcommand\R{\mathbb{R}}
\newcommand\Z{\mathbb{Z}}
\newcommand\N{\mathbb{N}}
\newcommand\E{\mathbb{E}}
\newcommand\F{\mathcal{F}}
\newcommand\cH{\mathcal{H}}
\newcommand\V{\mathbb{V}}
\newcommand\dmo{ ^{-1} }
\newcommand\kapa{\kappa}
\newcommand\im{Im}
\newcommand\hs{\mathcal{H}}





\usepackage{soul}

\makeatletter
\newcommand*{\whiten}[1]{\llap{\textcolor{white}{{\the\SOUL@token}}\hspace{#1pt}}}
\DeclareRobustCommand*\myul{%
    \def\SOUL@everyspace{\underline{\space}\kern\z@}%
    \def\SOUL@everytoken{%
     \setbox0=\hbox{\the\SOUL@token}%
     \ifdim\dp0>\z@
        \raisebox{\dp0}{\underline{\phantom{\the\SOUL@token}}}%
        \whiten{1}\whiten{0}%
        \whiten{-1}\whiten{-2}%
        \llap{\the\SOUL@token}%
     \else
        \underline{\the\SOUL@token}%
     \fi}%
\SOUL@}
\makeatother

\newcommand*{\demp}{\fontfamily{lmtt}\selectfont}

\DeclareTextFontCommand{\textdemp}{\demp}

\begin{document}

\ifcomment
Multiline
comment
\fi
\ifcomment
\myul{Typesetting test}
% \color[rgb]{1,1,1}
$∑_i^n≠ 60º±∞π∆¬≈√j∫h≤≥µ$

$\CR \R\pro\ind\pro\gS\pro
\mqty[a&b\\c&d]$
$\pro\mathbb{P}$
$\dd{x}$

  \[
    \alpha(x)=\left\{
                \begin{array}{ll}
                  x\\
                  \frac{1}{1+e^{-kx}}\\
                  \frac{e^x-e^{-x}}{e^x+e^{-x}}
                \end{array}
              \right.
  \]

  $\expval{x}$
  
  $\chi_\rho(ghg\dmo)=\Tr(\rho_{ghg\dmo})=\Tr(\rho_g\circ\rho_h\circ\rho\dmo_g)=\Tr(\rho_h)\overset{\mbox{\scalebox{0.5}{$\Tr(AB)=\Tr(BA)$}}}{=}\chi_\rho(h)$
  	$\mathop{\oplus}_{\substack{x\in X}}$

$\mat(\rho_g)=(a_{ij}(g))_{\scriptsize \substack{1\leq i\leq d \\ 1\leq j\leq d}}$ et $\mat(\rho'_g)=(a'_{ij}(g))_{\scriptsize \substack{1\leq i'\leq d' \\ 1\leq j'\leq d'}}$



\[\int_a^b{\mathbb{R}^2}g(u, v)\dd{P_{XY}}(u, v)=\iint g(u,v) f_{XY}(u, v)\dd \lambda(u) \dd \lambda(v)\]
$$\lim_{x\to\infty} f(x)$$	
$$\iiiint_V \mu(t,u,v,w) \,dt\,du\,dv\,dw$$
$$\sum_{n=1}^{\infty} 2^{-n} = 1$$	
\begin{definition}
	Si $X$ et $Y$ sont 2 v.a. ou definit la \textsc{Covariance} entre $X$ et $Y$ comme
	$\cov(X,Y)\overset{\text{def}}{=}\E\left[(X-\E(X))(Y-\E(Y))\right]=\E(XY)-\E(X)\E(Y)$.
\end{definition}
\fi
\pagebreak

% \tableofcontents

% insert your code here
%\input{./algebra/main.tex}
%\input{./geometrie-differentielle/main.tex}
%\input{./probabilite/main.tex}
%\input{./analyse-fonctionnelle/main.tex}
% \input{./Analyse-convexe-et-dualite-en-optimisation/main.tex}
%\input{./tikz/main.tex}
%\input{./Theorie-du-distributions/main.tex}
%\input{./optimisation/mine.tex}
 \input{./modelisation/main.tex}

% yves.aubry@univ-tln.fr : algebra

\end{document}

%% !TEX encoding = UTF-8 Unicode
% !TEX TS-program = xelatex

\documentclass[french]{report}

%\usepackage[utf8]{inputenc}
%\usepackage[T1]{fontenc}
\usepackage{babel}


\newif\ifcomment
%\commenttrue # Show comments

\usepackage{physics}
\usepackage{amssymb}


\usepackage{amsthm}
% \usepackage{thmtools}
\usepackage{mathtools}
\usepackage{amsfonts}

\usepackage{color}

\usepackage{tikz}

\usepackage{geometry}
\geometry{a5paper, margin=0.1in, right=1cm}

\usepackage{dsfont}

\usepackage{graphicx}
\graphicspath{ {images/} }

\usepackage{faktor}

\usepackage{IEEEtrantools}
\usepackage{enumerate}   
\usepackage[PostScript=dvips]{"/Users/aware/Documents/Courses/diagrams"}


\newtheorem{theorem}{Théorème}[section]
\renewcommand{\thetheorem}{\arabic{theorem}}
\newtheorem{lemme}{Lemme}[section]
\renewcommand{\thelemme}{\arabic{lemme}}
\newtheorem{proposition}{Proposition}[section]
\renewcommand{\theproposition}{\arabic{proposition}}
\newtheorem{notations}{Notations}[section]
\newtheorem{problem}{Problème}[section]
\newtheorem{corollary}{Corollaire}[theorem]
\renewcommand{\thecorollary}{\arabic{corollary}}
\newtheorem{property}{Propriété}[section]
\newtheorem{objective}{Objectif}[section]

\theoremstyle{definition}
\newtheorem{definition}{Définition}[section]
\renewcommand{\thedefinition}{\arabic{definition}}
\newtheorem{exercise}{Exercice}[chapter]
\renewcommand{\theexercise}{\arabic{exercise}}
\newtheorem{example}{Exemple}[chapter]
\renewcommand{\theexample}{\arabic{example}}
\newtheorem*{solution}{Solution}
\newtheorem*{application}{Application}
\newtheorem*{notation}{Notation}
\newtheorem*{vocabulary}{Vocabulaire}
\newtheorem*{properties}{Propriétés}



\theoremstyle{remark}
\newtheorem*{remark}{Remarque}
\newtheorem*{rappel}{Rappel}


\usepackage{etoolbox}
\AtBeginEnvironment{exercise}{\small}
\AtBeginEnvironment{example}{\small}

\usepackage{cases}
\usepackage[red]{mypack}

\usepackage[framemethod=TikZ]{mdframed}

\definecolor{bg}{rgb}{0.4,0.25,0.95}
\definecolor{pagebg}{rgb}{0,0,0.5}
\surroundwithmdframed[
   topline=false,
   rightline=false,
   bottomline=false,
   leftmargin=\parindent,
   skipabove=8pt,
   skipbelow=8pt,
   linecolor=blue,
   innerbottommargin=10pt,
   % backgroundcolor=bg,font=\color{orange}\sffamily, fontcolor=white
]{definition}

\usepackage{empheq}
\usepackage[most]{tcolorbox}

\newtcbox{\mymath}[1][]{%
    nobeforeafter, math upper, tcbox raise base,
    enhanced, colframe=blue!30!black,
    colback=red!10, boxrule=1pt,
    #1}

\usepackage{unixode}


\DeclareMathOperator{\ord}{ord}
\DeclareMathOperator{\orb}{orb}
\DeclareMathOperator{\stab}{stab}
\DeclareMathOperator{\Stab}{stab}
\DeclareMathOperator{\ppcm}{ppcm}
\DeclareMathOperator{\conj}{Conj}
\DeclareMathOperator{\End}{End}
\DeclareMathOperator{\rot}{rot}
\DeclareMathOperator{\trs}{trace}
\DeclareMathOperator{\Ind}{Ind}
\DeclareMathOperator{\mat}{Mat}
\DeclareMathOperator{\id}{Id}
\DeclareMathOperator{\vect}{vect}
\DeclareMathOperator{\img}{img}
\DeclareMathOperator{\cov}{Cov}
\DeclareMathOperator{\dist}{dist}
\DeclareMathOperator{\irr}{Irr}
\DeclareMathOperator{\image}{Im}
\DeclareMathOperator{\pd}{\partial}
\DeclareMathOperator{\epi}{epi}
\DeclareMathOperator{\Argmin}{Argmin}
\DeclareMathOperator{\dom}{dom}
\DeclareMathOperator{\proj}{proj}
\DeclareMathOperator{\ctg}{ctg}
\DeclareMathOperator{\supp}{supp}
\DeclareMathOperator{\argmin}{argmin}
\DeclareMathOperator{\mult}{mult}
\DeclareMathOperator{\ch}{ch}
\DeclareMathOperator{\sh}{sh}
\DeclareMathOperator{\rang}{rang}
\DeclareMathOperator{\diam}{diam}
\DeclareMathOperator{\Epigraphe}{Epigraphe}




\usepackage{xcolor}
\everymath{\color{blue}}
%\everymath{\color[rgb]{0,1,1}}
%\pagecolor[rgb]{0,0,0.5}


\newcommand*{\pdtest}[3][]{\ensuremath{\frac{\partial^{#1} #2}{\partial #3}}}

\newcommand*{\deffunc}[6][]{\ensuremath{
\begin{array}{rcl}
#2 : #3 &\rightarrow& #4\\
#5 &\mapsto& #6
\end{array}
}}

\newcommand{\eqcolon}{\mathrel{\resizebox{\widthof{$\mathord{=}$}}{\height}{ $\!\!=\!\!\resizebox{1.2\width}{0.8\height}{\raisebox{0.23ex}{$\mathop{:}$}}\!\!$ }}}
\newcommand{\coloneq}{\mathrel{\resizebox{\widthof{$\mathord{=}$}}{\height}{ $\!\!\resizebox{1.2\width}{0.8\height}{\raisebox{0.23ex}{$\mathop{:}$}}\!\!=\!\!$ }}}
\newcommand{\eqcolonl}{\ensuremath{\mathrel{=\!\!\mathop{:}}}}
\newcommand{\coloneql}{\ensuremath{\mathrel{\mathop{:} \!\! =}}}
\newcommand{\vc}[1]{% inline column vector
  \left(\begin{smallmatrix}#1\end{smallmatrix}\right)%
}
\newcommand{\vr}[1]{% inline row vector
  \begin{smallmatrix}(\,#1\,)\end{smallmatrix}%
}
\makeatletter
\newcommand*{\defeq}{\ =\mathrel{\rlap{%
                     \raisebox{0.3ex}{$\m@th\cdot$}}%
                     \raisebox{-0.3ex}{$\m@th\cdot$}}%
                     }
\makeatother

\newcommand{\mathcircle}[1]{% inline row vector
 \overset{\circ}{#1}
}
\newcommand{\ulim}{% low limit
 \underline{\lim}
}
\newcommand{\ssi}{% iff
\iff
}
\newcommand{\ps}[2]{
\expval{#1 | #2}
}
\newcommand{\df}[1]{
\mqty{#1}
}
\newcommand{\n}[1]{
\norm{#1}
}
\newcommand{\sys}[1]{
\left\{\smqty{#1}\right.
}


\newcommand{\eqdef}{\ensuremath{\overset{\text{def}}=}}


\def\Circlearrowright{\ensuremath{%
  \rotatebox[origin=c]{230}{$\circlearrowright$}}}

\newcommand\ct[1]{\text{\rmfamily\upshape #1}}
\newcommand\question[1]{ {\color{red} ...!? \small #1}}
\newcommand\caz[1]{\left\{\begin{array} #1 \end{array}\right.}
\newcommand\const{\text{\rmfamily\upshape const}}
\newcommand\toP{ \overset{\pro}{\to}}
\newcommand\toPP{ \overset{\text{PP}}{\to}}
\newcommand{\oeq}{\mathrel{\text{\textcircled{$=$}}}}





\usepackage{xcolor}
% \usepackage[normalem]{ulem}
\usepackage{lipsum}
\makeatletter
% \newcommand\colorwave[1][blue]{\bgroup \markoverwith{\lower3.5\p@\hbox{\sixly \textcolor{#1}{\char58}}}\ULon}
%\font\sixly=lasy6 % does not re-load if already loaded, so no memory problem.

\newmdtheoremenv[
linewidth= 1pt,linecolor= blue,%
leftmargin=20,rightmargin=20,innertopmargin=0pt, innerrightmargin=40,%
tikzsetting = { draw=lightgray, line width = 0.3pt,dashed,%
dash pattern = on 15pt off 3pt},%
splittopskip=\topskip,skipbelow=\baselineskip,%
skipabove=\baselineskip,ntheorem,roundcorner=0pt,
% backgroundcolor=pagebg,font=\color{orange}\sffamily, fontcolor=white
]{examplebox}{Exemple}[section]



\newcommand\R{\mathbb{R}}
\newcommand\Z{\mathbb{Z}}
\newcommand\N{\mathbb{N}}
\newcommand\E{\mathbb{E}}
\newcommand\F{\mathcal{F}}
\newcommand\cH{\mathcal{H}}
\newcommand\V{\mathbb{V}}
\newcommand\dmo{ ^{-1} }
\newcommand\kapa{\kappa}
\newcommand\im{Im}
\newcommand\hs{\mathcal{H}}





\usepackage{soul}

\makeatletter
\newcommand*{\whiten}[1]{\llap{\textcolor{white}{{\the\SOUL@token}}\hspace{#1pt}}}
\DeclareRobustCommand*\myul{%
    \def\SOUL@everyspace{\underline{\space}\kern\z@}%
    \def\SOUL@everytoken{%
     \setbox0=\hbox{\the\SOUL@token}%
     \ifdim\dp0>\z@
        \raisebox{\dp0}{\underline{\phantom{\the\SOUL@token}}}%
        \whiten{1}\whiten{0}%
        \whiten{-1}\whiten{-2}%
        \llap{\the\SOUL@token}%
     \else
        \underline{\the\SOUL@token}%
     \fi}%
\SOUL@}
\makeatother

\newcommand*{\demp}{\fontfamily{lmtt}\selectfont}

\DeclareTextFontCommand{\textdemp}{\demp}

\begin{document}

\ifcomment
Multiline
comment
\fi
\ifcomment
\myul{Typesetting test}
% \color[rgb]{1,1,1}
$∑_i^n≠ 60º±∞π∆¬≈√j∫h≤≥µ$

$\CR \R\pro\ind\pro\gS\pro
\mqty[a&b\\c&d]$
$\pro\mathbb{P}$
$\dd{x}$

  \[
    \alpha(x)=\left\{
                \begin{array}{ll}
                  x\\
                  \frac{1}{1+e^{-kx}}\\
                  \frac{e^x-e^{-x}}{e^x+e^{-x}}
                \end{array}
              \right.
  \]

  $\expval{x}$
  
  $\chi_\rho(ghg\dmo)=\Tr(\rho_{ghg\dmo})=\Tr(\rho_g\circ\rho_h\circ\rho\dmo_g)=\Tr(\rho_h)\overset{\mbox{\scalebox{0.5}{$\Tr(AB)=\Tr(BA)$}}}{=}\chi_\rho(h)$
  	$\mathop{\oplus}_{\substack{x\in X}}$

$\mat(\rho_g)=(a_{ij}(g))_{\scriptsize \substack{1\leq i\leq d \\ 1\leq j\leq d}}$ et $\mat(\rho'_g)=(a'_{ij}(g))_{\scriptsize \substack{1\leq i'\leq d' \\ 1\leq j'\leq d'}}$



\[\int_a^b{\mathbb{R}^2}g(u, v)\dd{P_{XY}}(u, v)=\iint g(u,v) f_{XY}(u, v)\dd \lambda(u) \dd \lambda(v)\]
$$\lim_{x\to\infty} f(x)$$	
$$\iiiint_V \mu(t,u,v,w) \,dt\,du\,dv\,dw$$
$$\sum_{n=1}^{\infty} 2^{-n} = 1$$	
\begin{definition}
	Si $X$ et $Y$ sont 2 v.a. ou definit la \textsc{Covariance} entre $X$ et $Y$ comme
	$\cov(X,Y)\overset{\text{def}}{=}\E\left[(X-\E(X))(Y-\E(Y))\right]=\E(XY)-\E(X)\E(Y)$.
\end{definition}
\fi
\pagebreak

% \tableofcontents

% insert your code here
%\input{./algebra/main.tex}
%\input{./geometrie-differentielle/main.tex}
%\input{./probabilite/main.tex}
%\input{./analyse-fonctionnelle/main.tex}
% \input{./Analyse-convexe-et-dualite-en-optimisation/main.tex}
%\input{./tikz/main.tex}
%\input{./Theorie-du-distributions/main.tex}
%\input{./optimisation/mine.tex}
 \input{./modelisation/main.tex}

% yves.aubry@univ-tln.fr : algebra

\end{document}

% % !TEX encoding = UTF-8 Unicode
% !TEX TS-program = xelatex

\documentclass[french]{report}

%\usepackage[utf8]{inputenc}
%\usepackage[T1]{fontenc}
\usepackage{babel}


\newif\ifcomment
%\commenttrue # Show comments

\usepackage{physics}
\usepackage{amssymb}


\usepackage{amsthm}
% \usepackage{thmtools}
\usepackage{mathtools}
\usepackage{amsfonts}

\usepackage{color}

\usepackage{tikz}

\usepackage{geometry}
\geometry{a5paper, margin=0.1in, right=1cm}

\usepackage{dsfont}

\usepackage{graphicx}
\graphicspath{ {images/} }

\usepackage{faktor}

\usepackage{IEEEtrantools}
\usepackage{enumerate}   
\usepackage[PostScript=dvips]{"/Users/aware/Documents/Courses/diagrams"}


\newtheorem{theorem}{Théorème}[section]
\renewcommand{\thetheorem}{\arabic{theorem}}
\newtheorem{lemme}{Lemme}[section]
\renewcommand{\thelemme}{\arabic{lemme}}
\newtheorem{proposition}{Proposition}[section]
\renewcommand{\theproposition}{\arabic{proposition}}
\newtheorem{notations}{Notations}[section]
\newtheorem{problem}{Problème}[section]
\newtheorem{corollary}{Corollaire}[theorem]
\renewcommand{\thecorollary}{\arabic{corollary}}
\newtheorem{property}{Propriété}[section]
\newtheorem{objective}{Objectif}[section]

\theoremstyle{definition}
\newtheorem{definition}{Définition}[section]
\renewcommand{\thedefinition}{\arabic{definition}}
\newtheorem{exercise}{Exercice}[chapter]
\renewcommand{\theexercise}{\arabic{exercise}}
\newtheorem{example}{Exemple}[chapter]
\renewcommand{\theexample}{\arabic{example}}
\newtheorem*{solution}{Solution}
\newtheorem*{application}{Application}
\newtheorem*{notation}{Notation}
\newtheorem*{vocabulary}{Vocabulaire}
\newtheorem*{properties}{Propriétés}



\theoremstyle{remark}
\newtheorem*{remark}{Remarque}
\newtheorem*{rappel}{Rappel}


\usepackage{etoolbox}
\AtBeginEnvironment{exercise}{\small}
\AtBeginEnvironment{example}{\small}

\usepackage{cases}
\usepackage[red]{mypack}

\usepackage[framemethod=TikZ]{mdframed}

\definecolor{bg}{rgb}{0.4,0.25,0.95}
\definecolor{pagebg}{rgb}{0,0,0.5}
\surroundwithmdframed[
   topline=false,
   rightline=false,
   bottomline=false,
   leftmargin=\parindent,
   skipabove=8pt,
   skipbelow=8pt,
   linecolor=blue,
   innerbottommargin=10pt,
   % backgroundcolor=bg,font=\color{orange}\sffamily, fontcolor=white
]{definition}

\usepackage{empheq}
\usepackage[most]{tcolorbox}

\newtcbox{\mymath}[1][]{%
    nobeforeafter, math upper, tcbox raise base,
    enhanced, colframe=blue!30!black,
    colback=red!10, boxrule=1pt,
    #1}

\usepackage{unixode}


\DeclareMathOperator{\ord}{ord}
\DeclareMathOperator{\orb}{orb}
\DeclareMathOperator{\stab}{stab}
\DeclareMathOperator{\Stab}{stab}
\DeclareMathOperator{\ppcm}{ppcm}
\DeclareMathOperator{\conj}{Conj}
\DeclareMathOperator{\End}{End}
\DeclareMathOperator{\rot}{rot}
\DeclareMathOperator{\trs}{trace}
\DeclareMathOperator{\Ind}{Ind}
\DeclareMathOperator{\mat}{Mat}
\DeclareMathOperator{\id}{Id}
\DeclareMathOperator{\vect}{vect}
\DeclareMathOperator{\img}{img}
\DeclareMathOperator{\cov}{Cov}
\DeclareMathOperator{\dist}{dist}
\DeclareMathOperator{\irr}{Irr}
\DeclareMathOperator{\image}{Im}
\DeclareMathOperator{\pd}{\partial}
\DeclareMathOperator{\epi}{epi}
\DeclareMathOperator{\Argmin}{Argmin}
\DeclareMathOperator{\dom}{dom}
\DeclareMathOperator{\proj}{proj}
\DeclareMathOperator{\ctg}{ctg}
\DeclareMathOperator{\supp}{supp}
\DeclareMathOperator{\argmin}{argmin}
\DeclareMathOperator{\mult}{mult}
\DeclareMathOperator{\ch}{ch}
\DeclareMathOperator{\sh}{sh}
\DeclareMathOperator{\rang}{rang}
\DeclareMathOperator{\diam}{diam}
\DeclareMathOperator{\Epigraphe}{Epigraphe}




\usepackage{xcolor}
\everymath{\color{blue}}
%\everymath{\color[rgb]{0,1,1}}
%\pagecolor[rgb]{0,0,0.5}


\newcommand*{\pdtest}[3][]{\ensuremath{\frac{\partial^{#1} #2}{\partial #3}}}

\newcommand*{\deffunc}[6][]{\ensuremath{
\begin{array}{rcl}
#2 : #3 &\rightarrow& #4\\
#5 &\mapsto& #6
\end{array}
}}

\newcommand{\eqcolon}{\mathrel{\resizebox{\widthof{$\mathord{=}$}}{\height}{ $\!\!=\!\!\resizebox{1.2\width}{0.8\height}{\raisebox{0.23ex}{$\mathop{:}$}}\!\!$ }}}
\newcommand{\coloneq}{\mathrel{\resizebox{\widthof{$\mathord{=}$}}{\height}{ $\!\!\resizebox{1.2\width}{0.8\height}{\raisebox{0.23ex}{$\mathop{:}$}}\!\!=\!\!$ }}}
\newcommand{\eqcolonl}{\ensuremath{\mathrel{=\!\!\mathop{:}}}}
\newcommand{\coloneql}{\ensuremath{\mathrel{\mathop{:} \!\! =}}}
\newcommand{\vc}[1]{% inline column vector
  \left(\begin{smallmatrix}#1\end{smallmatrix}\right)%
}
\newcommand{\vr}[1]{% inline row vector
  \begin{smallmatrix}(\,#1\,)\end{smallmatrix}%
}
\makeatletter
\newcommand*{\defeq}{\ =\mathrel{\rlap{%
                     \raisebox{0.3ex}{$\m@th\cdot$}}%
                     \raisebox{-0.3ex}{$\m@th\cdot$}}%
                     }
\makeatother

\newcommand{\mathcircle}[1]{% inline row vector
 \overset{\circ}{#1}
}
\newcommand{\ulim}{% low limit
 \underline{\lim}
}
\newcommand{\ssi}{% iff
\iff
}
\newcommand{\ps}[2]{
\expval{#1 | #2}
}
\newcommand{\df}[1]{
\mqty{#1}
}
\newcommand{\n}[1]{
\norm{#1}
}
\newcommand{\sys}[1]{
\left\{\smqty{#1}\right.
}


\newcommand{\eqdef}{\ensuremath{\overset{\text{def}}=}}


\def\Circlearrowright{\ensuremath{%
  \rotatebox[origin=c]{230}{$\circlearrowright$}}}

\newcommand\ct[1]{\text{\rmfamily\upshape #1}}
\newcommand\question[1]{ {\color{red} ...!? \small #1}}
\newcommand\caz[1]{\left\{\begin{array} #1 \end{array}\right.}
\newcommand\const{\text{\rmfamily\upshape const}}
\newcommand\toP{ \overset{\pro}{\to}}
\newcommand\toPP{ \overset{\text{PP}}{\to}}
\newcommand{\oeq}{\mathrel{\text{\textcircled{$=$}}}}





\usepackage{xcolor}
% \usepackage[normalem]{ulem}
\usepackage{lipsum}
\makeatletter
% \newcommand\colorwave[1][blue]{\bgroup \markoverwith{\lower3.5\p@\hbox{\sixly \textcolor{#1}{\char58}}}\ULon}
%\font\sixly=lasy6 % does not re-load if already loaded, so no memory problem.

\newmdtheoremenv[
linewidth= 1pt,linecolor= blue,%
leftmargin=20,rightmargin=20,innertopmargin=0pt, innerrightmargin=40,%
tikzsetting = { draw=lightgray, line width = 0.3pt,dashed,%
dash pattern = on 15pt off 3pt},%
splittopskip=\topskip,skipbelow=\baselineskip,%
skipabove=\baselineskip,ntheorem,roundcorner=0pt,
% backgroundcolor=pagebg,font=\color{orange}\sffamily, fontcolor=white
]{examplebox}{Exemple}[section]



\newcommand\R{\mathbb{R}}
\newcommand\Z{\mathbb{Z}}
\newcommand\N{\mathbb{N}}
\newcommand\E{\mathbb{E}}
\newcommand\F{\mathcal{F}}
\newcommand\cH{\mathcal{H}}
\newcommand\V{\mathbb{V}}
\newcommand\dmo{ ^{-1} }
\newcommand\kapa{\kappa}
\newcommand\im{Im}
\newcommand\hs{\mathcal{H}}





\usepackage{soul}

\makeatletter
\newcommand*{\whiten}[1]{\llap{\textcolor{white}{{\the\SOUL@token}}\hspace{#1pt}}}
\DeclareRobustCommand*\myul{%
    \def\SOUL@everyspace{\underline{\space}\kern\z@}%
    \def\SOUL@everytoken{%
     \setbox0=\hbox{\the\SOUL@token}%
     \ifdim\dp0>\z@
        \raisebox{\dp0}{\underline{\phantom{\the\SOUL@token}}}%
        \whiten{1}\whiten{0}%
        \whiten{-1}\whiten{-2}%
        \llap{\the\SOUL@token}%
     \else
        \underline{\the\SOUL@token}%
     \fi}%
\SOUL@}
\makeatother

\newcommand*{\demp}{\fontfamily{lmtt}\selectfont}

\DeclareTextFontCommand{\textdemp}{\demp}

\begin{document}

\ifcomment
Multiline
comment
\fi
\ifcomment
\myul{Typesetting test}
% \color[rgb]{1,1,1}
$∑_i^n≠ 60º±∞π∆¬≈√j∫h≤≥µ$

$\CR \R\pro\ind\pro\gS\pro
\mqty[a&b\\c&d]$
$\pro\mathbb{P}$
$\dd{x}$

  \[
    \alpha(x)=\left\{
                \begin{array}{ll}
                  x\\
                  \frac{1}{1+e^{-kx}}\\
                  \frac{e^x-e^{-x}}{e^x+e^{-x}}
                \end{array}
              \right.
  \]

  $\expval{x}$
  
  $\chi_\rho(ghg\dmo)=\Tr(\rho_{ghg\dmo})=\Tr(\rho_g\circ\rho_h\circ\rho\dmo_g)=\Tr(\rho_h)\overset{\mbox{\scalebox{0.5}{$\Tr(AB)=\Tr(BA)$}}}{=}\chi_\rho(h)$
  	$\mathop{\oplus}_{\substack{x\in X}}$

$\mat(\rho_g)=(a_{ij}(g))_{\scriptsize \substack{1\leq i\leq d \\ 1\leq j\leq d}}$ et $\mat(\rho'_g)=(a'_{ij}(g))_{\scriptsize \substack{1\leq i'\leq d' \\ 1\leq j'\leq d'}}$



\[\int_a^b{\mathbb{R}^2}g(u, v)\dd{P_{XY}}(u, v)=\iint g(u,v) f_{XY}(u, v)\dd \lambda(u) \dd \lambda(v)\]
$$\lim_{x\to\infty} f(x)$$	
$$\iiiint_V \mu(t,u,v,w) \,dt\,du\,dv\,dw$$
$$\sum_{n=1}^{\infty} 2^{-n} = 1$$	
\begin{definition}
	Si $X$ et $Y$ sont 2 v.a. ou definit la \textsc{Covariance} entre $X$ et $Y$ comme
	$\cov(X,Y)\overset{\text{def}}{=}\E\left[(X-\E(X))(Y-\E(Y))\right]=\E(XY)-\E(X)\E(Y)$.
\end{definition}
\fi
\pagebreak

% \tableofcontents

% insert your code here
%\input{./algebra/main.tex}
%\input{./geometrie-differentielle/main.tex}
%\input{./probabilite/main.tex}
%\input{./analyse-fonctionnelle/main.tex}
% \input{./Analyse-convexe-et-dualite-en-optimisation/main.tex}
%\input{./tikz/main.tex}
%\input{./Theorie-du-distributions/main.tex}
%\input{./optimisation/mine.tex}
 \input{./modelisation/main.tex}

% yves.aubry@univ-tln.fr : algebra

\end{document}

%% !TEX encoding = UTF-8 Unicode
% !TEX TS-program = xelatex

\documentclass[french]{report}

%\usepackage[utf8]{inputenc}
%\usepackage[T1]{fontenc}
\usepackage{babel}


\newif\ifcomment
%\commenttrue # Show comments

\usepackage{physics}
\usepackage{amssymb}


\usepackage{amsthm}
% \usepackage{thmtools}
\usepackage{mathtools}
\usepackage{amsfonts}

\usepackage{color}

\usepackage{tikz}

\usepackage{geometry}
\geometry{a5paper, margin=0.1in, right=1cm}

\usepackage{dsfont}

\usepackage{graphicx}
\graphicspath{ {images/} }

\usepackage{faktor}

\usepackage{IEEEtrantools}
\usepackage{enumerate}   
\usepackage[PostScript=dvips]{"/Users/aware/Documents/Courses/diagrams"}


\newtheorem{theorem}{Théorème}[section]
\renewcommand{\thetheorem}{\arabic{theorem}}
\newtheorem{lemme}{Lemme}[section]
\renewcommand{\thelemme}{\arabic{lemme}}
\newtheorem{proposition}{Proposition}[section]
\renewcommand{\theproposition}{\arabic{proposition}}
\newtheorem{notations}{Notations}[section]
\newtheorem{problem}{Problème}[section]
\newtheorem{corollary}{Corollaire}[theorem]
\renewcommand{\thecorollary}{\arabic{corollary}}
\newtheorem{property}{Propriété}[section]
\newtheorem{objective}{Objectif}[section]

\theoremstyle{definition}
\newtheorem{definition}{Définition}[section]
\renewcommand{\thedefinition}{\arabic{definition}}
\newtheorem{exercise}{Exercice}[chapter]
\renewcommand{\theexercise}{\arabic{exercise}}
\newtheorem{example}{Exemple}[chapter]
\renewcommand{\theexample}{\arabic{example}}
\newtheorem*{solution}{Solution}
\newtheorem*{application}{Application}
\newtheorem*{notation}{Notation}
\newtheorem*{vocabulary}{Vocabulaire}
\newtheorem*{properties}{Propriétés}



\theoremstyle{remark}
\newtheorem*{remark}{Remarque}
\newtheorem*{rappel}{Rappel}


\usepackage{etoolbox}
\AtBeginEnvironment{exercise}{\small}
\AtBeginEnvironment{example}{\small}

\usepackage{cases}
\usepackage[red]{mypack}

\usepackage[framemethod=TikZ]{mdframed}

\definecolor{bg}{rgb}{0.4,0.25,0.95}
\definecolor{pagebg}{rgb}{0,0,0.5}
\surroundwithmdframed[
   topline=false,
   rightline=false,
   bottomline=false,
   leftmargin=\parindent,
   skipabove=8pt,
   skipbelow=8pt,
   linecolor=blue,
   innerbottommargin=10pt,
   % backgroundcolor=bg,font=\color{orange}\sffamily, fontcolor=white
]{definition}

\usepackage{empheq}
\usepackage[most]{tcolorbox}

\newtcbox{\mymath}[1][]{%
    nobeforeafter, math upper, tcbox raise base,
    enhanced, colframe=blue!30!black,
    colback=red!10, boxrule=1pt,
    #1}

\usepackage{unixode}


\DeclareMathOperator{\ord}{ord}
\DeclareMathOperator{\orb}{orb}
\DeclareMathOperator{\stab}{stab}
\DeclareMathOperator{\Stab}{stab}
\DeclareMathOperator{\ppcm}{ppcm}
\DeclareMathOperator{\conj}{Conj}
\DeclareMathOperator{\End}{End}
\DeclareMathOperator{\rot}{rot}
\DeclareMathOperator{\trs}{trace}
\DeclareMathOperator{\Ind}{Ind}
\DeclareMathOperator{\mat}{Mat}
\DeclareMathOperator{\id}{Id}
\DeclareMathOperator{\vect}{vect}
\DeclareMathOperator{\img}{img}
\DeclareMathOperator{\cov}{Cov}
\DeclareMathOperator{\dist}{dist}
\DeclareMathOperator{\irr}{Irr}
\DeclareMathOperator{\image}{Im}
\DeclareMathOperator{\pd}{\partial}
\DeclareMathOperator{\epi}{epi}
\DeclareMathOperator{\Argmin}{Argmin}
\DeclareMathOperator{\dom}{dom}
\DeclareMathOperator{\proj}{proj}
\DeclareMathOperator{\ctg}{ctg}
\DeclareMathOperator{\supp}{supp}
\DeclareMathOperator{\argmin}{argmin}
\DeclareMathOperator{\mult}{mult}
\DeclareMathOperator{\ch}{ch}
\DeclareMathOperator{\sh}{sh}
\DeclareMathOperator{\rang}{rang}
\DeclareMathOperator{\diam}{diam}
\DeclareMathOperator{\Epigraphe}{Epigraphe}




\usepackage{xcolor}
\everymath{\color{blue}}
%\everymath{\color[rgb]{0,1,1}}
%\pagecolor[rgb]{0,0,0.5}


\newcommand*{\pdtest}[3][]{\ensuremath{\frac{\partial^{#1} #2}{\partial #3}}}

\newcommand*{\deffunc}[6][]{\ensuremath{
\begin{array}{rcl}
#2 : #3 &\rightarrow& #4\\
#5 &\mapsto& #6
\end{array}
}}

\newcommand{\eqcolon}{\mathrel{\resizebox{\widthof{$\mathord{=}$}}{\height}{ $\!\!=\!\!\resizebox{1.2\width}{0.8\height}{\raisebox{0.23ex}{$\mathop{:}$}}\!\!$ }}}
\newcommand{\coloneq}{\mathrel{\resizebox{\widthof{$\mathord{=}$}}{\height}{ $\!\!\resizebox{1.2\width}{0.8\height}{\raisebox{0.23ex}{$\mathop{:}$}}\!\!=\!\!$ }}}
\newcommand{\eqcolonl}{\ensuremath{\mathrel{=\!\!\mathop{:}}}}
\newcommand{\coloneql}{\ensuremath{\mathrel{\mathop{:} \!\! =}}}
\newcommand{\vc}[1]{% inline column vector
  \left(\begin{smallmatrix}#1\end{smallmatrix}\right)%
}
\newcommand{\vr}[1]{% inline row vector
  \begin{smallmatrix}(\,#1\,)\end{smallmatrix}%
}
\makeatletter
\newcommand*{\defeq}{\ =\mathrel{\rlap{%
                     \raisebox{0.3ex}{$\m@th\cdot$}}%
                     \raisebox{-0.3ex}{$\m@th\cdot$}}%
                     }
\makeatother

\newcommand{\mathcircle}[1]{% inline row vector
 \overset{\circ}{#1}
}
\newcommand{\ulim}{% low limit
 \underline{\lim}
}
\newcommand{\ssi}{% iff
\iff
}
\newcommand{\ps}[2]{
\expval{#1 | #2}
}
\newcommand{\df}[1]{
\mqty{#1}
}
\newcommand{\n}[1]{
\norm{#1}
}
\newcommand{\sys}[1]{
\left\{\smqty{#1}\right.
}


\newcommand{\eqdef}{\ensuremath{\overset{\text{def}}=}}


\def\Circlearrowright{\ensuremath{%
  \rotatebox[origin=c]{230}{$\circlearrowright$}}}

\newcommand\ct[1]{\text{\rmfamily\upshape #1}}
\newcommand\question[1]{ {\color{red} ...!? \small #1}}
\newcommand\caz[1]{\left\{\begin{array} #1 \end{array}\right.}
\newcommand\const{\text{\rmfamily\upshape const}}
\newcommand\toP{ \overset{\pro}{\to}}
\newcommand\toPP{ \overset{\text{PP}}{\to}}
\newcommand{\oeq}{\mathrel{\text{\textcircled{$=$}}}}





\usepackage{xcolor}
% \usepackage[normalem]{ulem}
\usepackage{lipsum}
\makeatletter
% \newcommand\colorwave[1][blue]{\bgroup \markoverwith{\lower3.5\p@\hbox{\sixly \textcolor{#1}{\char58}}}\ULon}
%\font\sixly=lasy6 % does not re-load if already loaded, so no memory problem.

\newmdtheoremenv[
linewidth= 1pt,linecolor= blue,%
leftmargin=20,rightmargin=20,innertopmargin=0pt, innerrightmargin=40,%
tikzsetting = { draw=lightgray, line width = 0.3pt,dashed,%
dash pattern = on 15pt off 3pt},%
splittopskip=\topskip,skipbelow=\baselineskip,%
skipabove=\baselineskip,ntheorem,roundcorner=0pt,
% backgroundcolor=pagebg,font=\color{orange}\sffamily, fontcolor=white
]{examplebox}{Exemple}[section]



\newcommand\R{\mathbb{R}}
\newcommand\Z{\mathbb{Z}}
\newcommand\N{\mathbb{N}}
\newcommand\E{\mathbb{E}}
\newcommand\F{\mathcal{F}}
\newcommand\cH{\mathcal{H}}
\newcommand\V{\mathbb{V}}
\newcommand\dmo{ ^{-1} }
\newcommand\kapa{\kappa}
\newcommand\im{Im}
\newcommand\hs{\mathcal{H}}





\usepackage{soul}

\makeatletter
\newcommand*{\whiten}[1]{\llap{\textcolor{white}{{\the\SOUL@token}}\hspace{#1pt}}}
\DeclareRobustCommand*\myul{%
    \def\SOUL@everyspace{\underline{\space}\kern\z@}%
    \def\SOUL@everytoken{%
     \setbox0=\hbox{\the\SOUL@token}%
     \ifdim\dp0>\z@
        \raisebox{\dp0}{\underline{\phantom{\the\SOUL@token}}}%
        \whiten{1}\whiten{0}%
        \whiten{-1}\whiten{-2}%
        \llap{\the\SOUL@token}%
     \else
        \underline{\the\SOUL@token}%
     \fi}%
\SOUL@}
\makeatother

\newcommand*{\demp}{\fontfamily{lmtt}\selectfont}

\DeclareTextFontCommand{\textdemp}{\demp}

\begin{document}

\ifcomment
Multiline
comment
\fi
\ifcomment
\myul{Typesetting test}
% \color[rgb]{1,1,1}
$∑_i^n≠ 60º±∞π∆¬≈√j∫h≤≥µ$

$\CR \R\pro\ind\pro\gS\pro
\mqty[a&b\\c&d]$
$\pro\mathbb{P}$
$\dd{x}$

  \[
    \alpha(x)=\left\{
                \begin{array}{ll}
                  x\\
                  \frac{1}{1+e^{-kx}}\\
                  \frac{e^x-e^{-x}}{e^x+e^{-x}}
                \end{array}
              \right.
  \]

  $\expval{x}$
  
  $\chi_\rho(ghg\dmo)=\Tr(\rho_{ghg\dmo})=\Tr(\rho_g\circ\rho_h\circ\rho\dmo_g)=\Tr(\rho_h)\overset{\mbox{\scalebox{0.5}{$\Tr(AB)=\Tr(BA)$}}}{=}\chi_\rho(h)$
  	$\mathop{\oplus}_{\substack{x\in X}}$

$\mat(\rho_g)=(a_{ij}(g))_{\scriptsize \substack{1\leq i\leq d \\ 1\leq j\leq d}}$ et $\mat(\rho'_g)=(a'_{ij}(g))_{\scriptsize \substack{1\leq i'\leq d' \\ 1\leq j'\leq d'}}$



\[\int_a^b{\mathbb{R}^2}g(u, v)\dd{P_{XY}}(u, v)=\iint g(u,v) f_{XY}(u, v)\dd \lambda(u) \dd \lambda(v)\]
$$\lim_{x\to\infty} f(x)$$	
$$\iiiint_V \mu(t,u,v,w) \,dt\,du\,dv\,dw$$
$$\sum_{n=1}^{\infty} 2^{-n} = 1$$	
\begin{definition}
	Si $X$ et $Y$ sont 2 v.a. ou definit la \textsc{Covariance} entre $X$ et $Y$ comme
	$\cov(X,Y)\overset{\text{def}}{=}\E\left[(X-\E(X))(Y-\E(Y))\right]=\E(XY)-\E(X)\E(Y)$.
\end{definition}
\fi
\pagebreak

% \tableofcontents

% insert your code here
%\input{./algebra/main.tex}
%\input{./geometrie-differentielle/main.tex}
%\input{./probabilite/main.tex}
%\input{./analyse-fonctionnelle/main.tex}
% \input{./Analyse-convexe-et-dualite-en-optimisation/main.tex}
%\input{./tikz/main.tex}
%\input{./Theorie-du-distributions/main.tex}
%\input{./optimisation/mine.tex}
 \input{./modelisation/main.tex}

% yves.aubry@univ-tln.fr : algebra

\end{document}

%% !TEX encoding = UTF-8 Unicode
% !TEX TS-program = xelatex

\documentclass[french]{report}

%\usepackage[utf8]{inputenc}
%\usepackage[T1]{fontenc}
\usepackage{babel}


\newif\ifcomment
%\commenttrue # Show comments

\usepackage{physics}
\usepackage{amssymb}


\usepackage{amsthm}
% \usepackage{thmtools}
\usepackage{mathtools}
\usepackage{amsfonts}

\usepackage{color}

\usepackage{tikz}

\usepackage{geometry}
\geometry{a5paper, margin=0.1in, right=1cm}

\usepackage{dsfont}

\usepackage{graphicx}
\graphicspath{ {images/} }

\usepackage{faktor}

\usepackage{IEEEtrantools}
\usepackage{enumerate}   
\usepackage[PostScript=dvips]{"/Users/aware/Documents/Courses/diagrams"}


\newtheorem{theorem}{Théorème}[section]
\renewcommand{\thetheorem}{\arabic{theorem}}
\newtheorem{lemme}{Lemme}[section]
\renewcommand{\thelemme}{\arabic{lemme}}
\newtheorem{proposition}{Proposition}[section]
\renewcommand{\theproposition}{\arabic{proposition}}
\newtheorem{notations}{Notations}[section]
\newtheorem{problem}{Problème}[section]
\newtheorem{corollary}{Corollaire}[theorem]
\renewcommand{\thecorollary}{\arabic{corollary}}
\newtheorem{property}{Propriété}[section]
\newtheorem{objective}{Objectif}[section]

\theoremstyle{definition}
\newtheorem{definition}{Définition}[section]
\renewcommand{\thedefinition}{\arabic{definition}}
\newtheorem{exercise}{Exercice}[chapter]
\renewcommand{\theexercise}{\arabic{exercise}}
\newtheorem{example}{Exemple}[chapter]
\renewcommand{\theexample}{\arabic{example}}
\newtheorem*{solution}{Solution}
\newtheorem*{application}{Application}
\newtheorem*{notation}{Notation}
\newtheorem*{vocabulary}{Vocabulaire}
\newtheorem*{properties}{Propriétés}



\theoremstyle{remark}
\newtheorem*{remark}{Remarque}
\newtheorem*{rappel}{Rappel}


\usepackage{etoolbox}
\AtBeginEnvironment{exercise}{\small}
\AtBeginEnvironment{example}{\small}

\usepackage{cases}
\usepackage[red]{mypack}

\usepackage[framemethod=TikZ]{mdframed}

\definecolor{bg}{rgb}{0.4,0.25,0.95}
\definecolor{pagebg}{rgb}{0,0,0.5}
\surroundwithmdframed[
   topline=false,
   rightline=false,
   bottomline=false,
   leftmargin=\parindent,
   skipabove=8pt,
   skipbelow=8pt,
   linecolor=blue,
   innerbottommargin=10pt,
   % backgroundcolor=bg,font=\color{orange}\sffamily, fontcolor=white
]{definition}

\usepackage{empheq}
\usepackage[most]{tcolorbox}

\newtcbox{\mymath}[1][]{%
    nobeforeafter, math upper, tcbox raise base,
    enhanced, colframe=blue!30!black,
    colback=red!10, boxrule=1pt,
    #1}

\usepackage{unixode}


\DeclareMathOperator{\ord}{ord}
\DeclareMathOperator{\orb}{orb}
\DeclareMathOperator{\stab}{stab}
\DeclareMathOperator{\Stab}{stab}
\DeclareMathOperator{\ppcm}{ppcm}
\DeclareMathOperator{\conj}{Conj}
\DeclareMathOperator{\End}{End}
\DeclareMathOperator{\rot}{rot}
\DeclareMathOperator{\trs}{trace}
\DeclareMathOperator{\Ind}{Ind}
\DeclareMathOperator{\mat}{Mat}
\DeclareMathOperator{\id}{Id}
\DeclareMathOperator{\vect}{vect}
\DeclareMathOperator{\img}{img}
\DeclareMathOperator{\cov}{Cov}
\DeclareMathOperator{\dist}{dist}
\DeclareMathOperator{\irr}{Irr}
\DeclareMathOperator{\image}{Im}
\DeclareMathOperator{\pd}{\partial}
\DeclareMathOperator{\epi}{epi}
\DeclareMathOperator{\Argmin}{Argmin}
\DeclareMathOperator{\dom}{dom}
\DeclareMathOperator{\proj}{proj}
\DeclareMathOperator{\ctg}{ctg}
\DeclareMathOperator{\supp}{supp}
\DeclareMathOperator{\argmin}{argmin}
\DeclareMathOperator{\mult}{mult}
\DeclareMathOperator{\ch}{ch}
\DeclareMathOperator{\sh}{sh}
\DeclareMathOperator{\rang}{rang}
\DeclareMathOperator{\diam}{diam}
\DeclareMathOperator{\Epigraphe}{Epigraphe}




\usepackage{xcolor}
\everymath{\color{blue}}
%\everymath{\color[rgb]{0,1,1}}
%\pagecolor[rgb]{0,0,0.5}


\newcommand*{\pdtest}[3][]{\ensuremath{\frac{\partial^{#1} #2}{\partial #3}}}

\newcommand*{\deffunc}[6][]{\ensuremath{
\begin{array}{rcl}
#2 : #3 &\rightarrow& #4\\
#5 &\mapsto& #6
\end{array}
}}

\newcommand{\eqcolon}{\mathrel{\resizebox{\widthof{$\mathord{=}$}}{\height}{ $\!\!=\!\!\resizebox{1.2\width}{0.8\height}{\raisebox{0.23ex}{$\mathop{:}$}}\!\!$ }}}
\newcommand{\coloneq}{\mathrel{\resizebox{\widthof{$\mathord{=}$}}{\height}{ $\!\!\resizebox{1.2\width}{0.8\height}{\raisebox{0.23ex}{$\mathop{:}$}}\!\!=\!\!$ }}}
\newcommand{\eqcolonl}{\ensuremath{\mathrel{=\!\!\mathop{:}}}}
\newcommand{\coloneql}{\ensuremath{\mathrel{\mathop{:} \!\! =}}}
\newcommand{\vc}[1]{% inline column vector
  \left(\begin{smallmatrix}#1\end{smallmatrix}\right)%
}
\newcommand{\vr}[1]{% inline row vector
  \begin{smallmatrix}(\,#1\,)\end{smallmatrix}%
}
\makeatletter
\newcommand*{\defeq}{\ =\mathrel{\rlap{%
                     \raisebox{0.3ex}{$\m@th\cdot$}}%
                     \raisebox{-0.3ex}{$\m@th\cdot$}}%
                     }
\makeatother

\newcommand{\mathcircle}[1]{% inline row vector
 \overset{\circ}{#1}
}
\newcommand{\ulim}{% low limit
 \underline{\lim}
}
\newcommand{\ssi}{% iff
\iff
}
\newcommand{\ps}[2]{
\expval{#1 | #2}
}
\newcommand{\df}[1]{
\mqty{#1}
}
\newcommand{\n}[1]{
\norm{#1}
}
\newcommand{\sys}[1]{
\left\{\smqty{#1}\right.
}


\newcommand{\eqdef}{\ensuremath{\overset{\text{def}}=}}


\def\Circlearrowright{\ensuremath{%
  \rotatebox[origin=c]{230}{$\circlearrowright$}}}

\newcommand\ct[1]{\text{\rmfamily\upshape #1}}
\newcommand\question[1]{ {\color{red} ...!? \small #1}}
\newcommand\caz[1]{\left\{\begin{array} #1 \end{array}\right.}
\newcommand\const{\text{\rmfamily\upshape const}}
\newcommand\toP{ \overset{\pro}{\to}}
\newcommand\toPP{ \overset{\text{PP}}{\to}}
\newcommand{\oeq}{\mathrel{\text{\textcircled{$=$}}}}





\usepackage{xcolor}
% \usepackage[normalem]{ulem}
\usepackage{lipsum}
\makeatletter
% \newcommand\colorwave[1][blue]{\bgroup \markoverwith{\lower3.5\p@\hbox{\sixly \textcolor{#1}{\char58}}}\ULon}
%\font\sixly=lasy6 % does not re-load if already loaded, so no memory problem.

\newmdtheoremenv[
linewidth= 1pt,linecolor= blue,%
leftmargin=20,rightmargin=20,innertopmargin=0pt, innerrightmargin=40,%
tikzsetting = { draw=lightgray, line width = 0.3pt,dashed,%
dash pattern = on 15pt off 3pt},%
splittopskip=\topskip,skipbelow=\baselineskip,%
skipabove=\baselineskip,ntheorem,roundcorner=0pt,
% backgroundcolor=pagebg,font=\color{orange}\sffamily, fontcolor=white
]{examplebox}{Exemple}[section]



\newcommand\R{\mathbb{R}}
\newcommand\Z{\mathbb{Z}}
\newcommand\N{\mathbb{N}}
\newcommand\E{\mathbb{E}}
\newcommand\F{\mathcal{F}}
\newcommand\cH{\mathcal{H}}
\newcommand\V{\mathbb{V}}
\newcommand\dmo{ ^{-1} }
\newcommand\kapa{\kappa}
\newcommand\im{Im}
\newcommand\hs{\mathcal{H}}





\usepackage{soul}

\makeatletter
\newcommand*{\whiten}[1]{\llap{\textcolor{white}{{\the\SOUL@token}}\hspace{#1pt}}}
\DeclareRobustCommand*\myul{%
    \def\SOUL@everyspace{\underline{\space}\kern\z@}%
    \def\SOUL@everytoken{%
     \setbox0=\hbox{\the\SOUL@token}%
     \ifdim\dp0>\z@
        \raisebox{\dp0}{\underline{\phantom{\the\SOUL@token}}}%
        \whiten{1}\whiten{0}%
        \whiten{-1}\whiten{-2}%
        \llap{\the\SOUL@token}%
     \else
        \underline{\the\SOUL@token}%
     \fi}%
\SOUL@}
\makeatother

\newcommand*{\demp}{\fontfamily{lmtt}\selectfont}

\DeclareTextFontCommand{\textdemp}{\demp}

\begin{document}

\ifcomment
Multiline
comment
\fi
\ifcomment
\myul{Typesetting test}
% \color[rgb]{1,1,1}
$∑_i^n≠ 60º±∞π∆¬≈√j∫h≤≥µ$

$\CR \R\pro\ind\pro\gS\pro
\mqty[a&b\\c&d]$
$\pro\mathbb{P}$
$\dd{x}$

  \[
    \alpha(x)=\left\{
                \begin{array}{ll}
                  x\\
                  \frac{1}{1+e^{-kx}}\\
                  \frac{e^x-e^{-x}}{e^x+e^{-x}}
                \end{array}
              \right.
  \]

  $\expval{x}$
  
  $\chi_\rho(ghg\dmo)=\Tr(\rho_{ghg\dmo})=\Tr(\rho_g\circ\rho_h\circ\rho\dmo_g)=\Tr(\rho_h)\overset{\mbox{\scalebox{0.5}{$\Tr(AB)=\Tr(BA)$}}}{=}\chi_\rho(h)$
  	$\mathop{\oplus}_{\substack{x\in X}}$

$\mat(\rho_g)=(a_{ij}(g))_{\scriptsize \substack{1\leq i\leq d \\ 1\leq j\leq d}}$ et $\mat(\rho'_g)=(a'_{ij}(g))_{\scriptsize \substack{1\leq i'\leq d' \\ 1\leq j'\leq d'}}$



\[\int_a^b{\mathbb{R}^2}g(u, v)\dd{P_{XY}}(u, v)=\iint g(u,v) f_{XY}(u, v)\dd \lambda(u) \dd \lambda(v)\]
$$\lim_{x\to\infty} f(x)$$	
$$\iiiint_V \mu(t,u,v,w) \,dt\,du\,dv\,dw$$
$$\sum_{n=1}^{\infty} 2^{-n} = 1$$	
\begin{definition}
	Si $X$ et $Y$ sont 2 v.a. ou definit la \textsc{Covariance} entre $X$ et $Y$ comme
	$\cov(X,Y)\overset{\text{def}}{=}\E\left[(X-\E(X))(Y-\E(Y))\right]=\E(XY)-\E(X)\E(Y)$.
\end{definition}
\fi
\pagebreak

% \tableofcontents

% insert your code here
%\input{./algebra/main.tex}
%\input{./geometrie-differentielle/main.tex}
%\input{./probabilite/main.tex}
%\input{./analyse-fonctionnelle/main.tex}
% \input{./Analyse-convexe-et-dualite-en-optimisation/main.tex}
%\input{./tikz/main.tex}
%\input{./Theorie-du-distributions/main.tex}
%\input{./optimisation/mine.tex}
 \input{./modelisation/main.tex}

% yves.aubry@univ-tln.fr : algebra

\end{document}

%\input{./optimisation/mine.tex}
 % !TEX encoding = UTF-8 Unicode
% !TEX TS-program = xelatex

\documentclass[french]{report}

%\usepackage[utf8]{inputenc}
%\usepackage[T1]{fontenc}
\usepackage{babel}


\newif\ifcomment
%\commenttrue # Show comments

\usepackage{physics}
\usepackage{amssymb}


\usepackage{amsthm}
% \usepackage{thmtools}
\usepackage{mathtools}
\usepackage{amsfonts}

\usepackage{color}

\usepackage{tikz}

\usepackage{geometry}
\geometry{a5paper, margin=0.1in, right=1cm}

\usepackage{dsfont}

\usepackage{graphicx}
\graphicspath{ {images/} }

\usepackage{faktor}

\usepackage{IEEEtrantools}
\usepackage{enumerate}   
\usepackage[PostScript=dvips]{"/Users/aware/Documents/Courses/diagrams"}


\newtheorem{theorem}{Théorème}[section]
\renewcommand{\thetheorem}{\arabic{theorem}}
\newtheorem{lemme}{Lemme}[section]
\renewcommand{\thelemme}{\arabic{lemme}}
\newtheorem{proposition}{Proposition}[section]
\renewcommand{\theproposition}{\arabic{proposition}}
\newtheorem{notations}{Notations}[section]
\newtheorem{problem}{Problème}[section]
\newtheorem{corollary}{Corollaire}[theorem]
\renewcommand{\thecorollary}{\arabic{corollary}}
\newtheorem{property}{Propriété}[section]
\newtheorem{objective}{Objectif}[section]

\theoremstyle{definition}
\newtheorem{definition}{Définition}[section]
\renewcommand{\thedefinition}{\arabic{definition}}
\newtheorem{exercise}{Exercice}[chapter]
\renewcommand{\theexercise}{\arabic{exercise}}
\newtheorem{example}{Exemple}[chapter]
\renewcommand{\theexample}{\arabic{example}}
\newtheorem*{solution}{Solution}
\newtheorem*{application}{Application}
\newtheorem*{notation}{Notation}
\newtheorem*{vocabulary}{Vocabulaire}
\newtheorem*{properties}{Propriétés}



\theoremstyle{remark}
\newtheorem*{remark}{Remarque}
\newtheorem*{rappel}{Rappel}


\usepackage{etoolbox}
\AtBeginEnvironment{exercise}{\small}
\AtBeginEnvironment{example}{\small}

\usepackage{cases}
\usepackage[red]{mypack}

\usepackage[framemethod=TikZ]{mdframed}

\definecolor{bg}{rgb}{0.4,0.25,0.95}
\definecolor{pagebg}{rgb}{0,0,0.5}
\surroundwithmdframed[
   topline=false,
   rightline=false,
   bottomline=false,
   leftmargin=\parindent,
   skipabove=8pt,
   skipbelow=8pt,
   linecolor=blue,
   innerbottommargin=10pt,
   % backgroundcolor=bg,font=\color{orange}\sffamily, fontcolor=white
]{definition}

\usepackage{empheq}
\usepackage[most]{tcolorbox}

\newtcbox{\mymath}[1][]{%
    nobeforeafter, math upper, tcbox raise base,
    enhanced, colframe=blue!30!black,
    colback=red!10, boxrule=1pt,
    #1}

\usepackage{unixode}


\DeclareMathOperator{\ord}{ord}
\DeclareMathOperator{\orb}{orb}
\DeclareMathOperator{\stab}{stab}
\DeclareMathOperator{\Stab}{stab}
\DeclareMathOperator{\ppcm}{ppcm}
\DeclareMathOperator{\conj}{Conj}
\DeclareMathOperator{\End}{End}
\DeclareMathOperator{\rot}{rot}
\DeclareMathOperator{\trs}{trace}
\DeclareMathOperator{\Ind}{Ind}
\DeclareMathOperator{\mat}{Mat}
\DeclareMathOperator{\id}{Id}
\DeclareMathOperator{\vect}{vect}
\DeclareMathOperator{\img}{img}
\DeclareMathOperator{\cov}{Cov}
\DeclareMathOperator{\dist}{dist}
\DeclareMathOperator{\irr}{Irr}
\DeclareMathOperator{\image}{Im}
\DeclareMathOperator{\pd}{\partial}
\DeclareMathOperator{\epi}{epi}
\DeclareMathOperator{\Argmin}{Argmin}
\DeclareMathOperator{\dom}{dom}
\DeclareMathOperator{\proj}{proj}
\DeclareMathOperator{\ctg}{ctg}
\DeclareMathOperator{\supp}{supp}
\DeclareMathOperator{\argmin}{argmin}
\DeclareMathOperator{\mult}{mult}
\DeclareMathOperator{\ch}{ch}
\DeclareMathOperator{\sh}{sh}
\DeclareMathOperator{\rang}{rang}
\DeclareMathOperator{\diam}{diam}
\DeclareMathOperator{\Epigraphe}{Epigraphe}




\usepackage{xcolor}
\everymath{\color{blue}}
%\everymath{\color[rgb]{0,1,1}}
%\pagecolor[rgb]{0,0,0.5}


\newcommand*{\pdtest}[3][]{\ensuremath{\frac{\partial^{#1} #2}{\partial #3}}}

\newcommand*{\deffunc}[6][]{\ensuremath{
\begin{array}{rcl}
#2 : #3 &\rightarrow& #4\\
#5 &\mapsto& #6
\end{array}
}}

\newcommand{\eqcolon}{\mathrel{\resizebox{\widthof{$\mathord{=}$}}{\height}{ $\!\!=\!\!\resizebox{1.2\width}{0.8\height}{\raisebox{0.23ex}{$\mathop{:}$}}\!\!$ }}}
\newcommand{\coloneq}{\mathrel{\resizebox{\widthof{$\mathord{=}$}}{\height}{ $\!\!\resizebox{1.2\width}{0.8\height}{\raisebox{0.23ex}{$\mathop{:}$}}\!\!=\!\!$ }}}
\newcommand{\eqcolonl}{\ensuremath{\mathrel{=\!\!\mathop{:}}}}
\newcommand{\coloneql}{\ensuremath{\mathrel{\mathop{:} \!\! =}}}
\newcommand{\vc}[1]{% inline column vector
  \left(\begin{smallmatrix}#1\end{smallmatrix}\right)%
}
\newcommand{\vr}[1]{% inline row vector
  \begin{smallmatrix}(\,#1\,)\end{smallmatrix}%
}
\makeatletter
\newcommand*{\defeq}{\ =\mathrel{\rlap{%
                     \raisebox{0.3ex}{$\m@th\cdot$}}%
                     \raisebox{-0.3ex}{$\m@th\cdot$}}%
                     }
\makeatother

\newcommand{\mathcircle}[1]{% inline row vector
 \overset{\circ}{#1}
}
\newcommand{\ulim}{% low limit
 \underline{\lim}
}
\newcommand{\ssi}{% iff
\iff
}
\newcommand{\ps}[2]{
\expval{#1 | #2}
}
\newcommand{\df}[1]{
\mqty{#1}
}
\newcommand{\n}[1]{
\norm{#1}
}
\newcommand{\sys}[1]{
\left\{\smqty{#1}\right.
}


\newcommand{\eqdef}{\ensuremath{\overset{\text{def}}=}}


\def\Circlearrowright{\ensuremath{%
  \rotatebox[origin=c]{230}{$\circlearrowright$}}}

\newcommand\ct[1]{\text{\rmfamily\upshape #1}}
\newcommand\question[1]{ {\color{red} ...!? \small #1}}
\newcommand\caz[1]{\left\{\begin{array} #1 \end{array}\right.}
\newcommand\const{\text{\rmfamily\upshape const}}
\newcommand\toP{ \overset{\pro}{\to}}
\newcommand\toPP{ \overset{\text{PP}}{\to}}
\newcommand{\oeq}{\mathrel{\text{\textcircled{$=$}}}}





\usepackage{xcolor}
% \usepackage[normalem]{ulem}
\usepackage{lipsum}
\makeatletter
% \newcommand\colorwave[1][blue]{\bgroup \markoverwith{\lower3.5\p@\hbox{\sixly \textcolor{#1}{\char58}}}\ULon}
%\font\sixly=lasy6 % does not re-load if already loaded, so no memory problem.

\newmdtheoremenv[
linewidth= 1pt,linecolor= blue,%
leftmargin=20,rightmargin=20,innertopmargin=0pt, innerrightmargin=40,%
tikzsetting = { draw=lightgray, line width = 0.3pt,dashed,%
dash pattern = on 15pt off 3pt},%
splittopskip=\topskip,skipbelow=\baselineskip,%
skipabove=\baselineskip,ntheorem,roundcorner=0pt,
% backgroundcolor=pagebg,font=\color{orange}\sffamily, fontcolor=white
]{examplebox}{Exemple}[section]



\newcommand\R{\mathbb{R}}
\newcommand\Z{\mathbb{Z}}
\newcommand\N{\mathbb{N}}
\newcommand\E{\mathbb{E}}
\newcommand\F{\mathcal{F}}
\newcommand\cH{\mathcal{H}}
\newcommand\V{\mathbb{V}}
\newcommand\dmo{ ^{-1} }
\newcommand\kapa{\kappa}
\newcommand\im{Im}
\newcommand\hs{\mathcal{H}}





\usepackage{soul}

\makeatletter
\newcommand*{\whiten}[1]{\llap{\textcolor{white}{{\the\SOUL@token}}\hspace{#1pt}}}
\DeclareRobustCommand*\myul{%
    \def\SOUL@everyspace{\underline{\space}\kern\z@}%
    \def\SOUL@everytoken{%
     \setbox0=\hbox{\the\SOUL@token}%
     \ifdim\dp0>\z@
        \raisebox{\dp0}{\underline{\phantom{\the\SOUL@token}}}%
        \whiten{1}\whiten{0}%
        \whiten{-1}\whiten{-2}%
        \llap{\the\SOUL@token}%
     \else
        \underline{\the\SOUL@token}%
     \fi}%
\SOUL@}
\makeatother

\newcommand*{\demp}{\fontfamily{lmtt}\selectfont}

\DeclareTextFontCommand{\textdemp}{\demp}

\begin{document}

\ifcomment
Multiline
comment
\fi
\ifcomment
\myul{Typesetting test}
% \color[rgb]{1,1,1}
$∑_i^n≠ 60º±∞π∆¬≈√j∫h≤≥µ$

$\CR \R\pro\ind\pro\gS\pro
\mqty[a&b\\c&d]$
$\pro\mathbb{P}$
$\dd{x}$

  \[
    \alpha(x)=\left\{
                \begin{array}{ll}
                  x\\
                  \frac{1}{1+e^{-kx}}\\
                  \frac{e^x-e^{-x}}{e^x+e^{-x}}
                \end{array}
              \right.
  \]

  $\expval{x}$
  
  $\chi_\rho(ghg\dmo)=\Tr(\rho_{ghg\dmo})=\Tr(\rho_g\circ\rho_h\circ\rho\dmo_g)=\Tr(\rho_h)\overset{\mbox{\scalebox{0.5}{$\Tr(AB)=\Tr(BA)$}}}{=}\chi_\rho(h)$
  	$\mathop{\oplus}_{\substack{x\in X}}$

$\mat(\rho_g)=(a_{ij}(g))_{\scriptsize \substack{1\leq i\leq d \\ 1\leq j\leq d}}$ et $\mat(\rho'_g)=(a'_{ij}(g))_{\scriptsize \substack{1\leq i'\leq d' \\ 1\leq j'\leq d'}}$



\[\int_a^b{\mathbb{R}^2}g(u, v)\dd{P_{XY}}(u, v)=\iint g(u,v) f_{XY}(u, v)\dd \lambda(u) \dd \lambda(v)\]
$$\lim_{x\to\infty} f(x)$$	
$$\iiiint_V \mu(t,u,v,w) \,dt\,du\,dv\,dw$$
$$\sum_{n=1}^{\infty} 2^{-n} = 1$$	
\begin{definition}
	Si $X$ et $Y$ sont 2 v.a. ou definit la \textsc{Covariance} entre $X$ et $Y$ comme
	$\cov(X,Y)\overset{\text{def}}{=}\E\left[(X-\E(X))(Y-\E(Y))\right]=\E(XY)-\E(X)\E(Y)$.
\end{definition}
\fi
\pagebreak

% \tableofcontents

% insert your code here
%\input{./algebra/main.tex}
%\input{./geometrie-differentielle/main.tex}
%\input{./probabilite/main.tex}
%\input{./analyse-fonctionnelle/main.tex}
% \input{./Analyse-convexe-et-dualite-en-optimisation/main.tex}
%\input{./tikz/main.tex}
%\input{./Theorie-du-distributions/main.tex}
%\input{./optimisation/mine.tex}
 \input{./modelisation/main.tex}

% yves.aubry@univ-tln.fr : algebra

\end{document}


% yves.aubry@univ-tln.fr : algebra

\end{document}

%% !TEX encoding = UTF-8 Unicode
% !TEX TS-program = xelatex

\documentclass[french]{report}

%\usepackage[utf8]{inputenc}
%\usepackage[T1]{fontenc}
\usepackage{babel}


\newif\ifcomment
%\commenttrue # Show comments

\usepackage{physics}
\usepackage{amssymb}


\usepackage{amsthm}
% \usepackage{thmtools}
\usepackage{mathtools}
\usepackage{amsfonts}

\usepackage{color}

\usepackage{tikz}

\usepackage{geometry}
\geometry{a5paper, margin=0.1in, right=1cm}

\usepackage{dsfont}

\usepackage{graphicx}
\graphicspath{ {images/} }

\usepackage{faktor}

\usepackage{IEEEtrantools}
\usepackage{enumerate}   
\usepackage[PostScript=dvips]{"/Users/aware/Documents/Courses/diagrams"}


\newtheorem{theorem}{Théorème}[section]
\renewcommand{\thetheorem}{\arabic{theorem}}
\newtheorem{lemme}{Lemme}[section]
\renewcommand{\thelemme}{\arabic{lemme}}
\newtheorem{proposition}{Proposition}[section]
\renewcommand{\theproposition}{\arabic{proposition}}
\newtheorem{notations}{Notations}[section]
\newtheorem{problem}{Problème}[section]
\newtheorem{corollary}{Corollaire}[theorem]
\renewcommand{\thecorollary}{\arabic{corollary}}
\newtheorem{property}{Propriété}[section]
\newtheorem{objective}{Objectif}[section]

\theoremstyle{definition}
\newtheorem{definition}{Définition}[section]
\renewcommand{\thedefinition}{\arabic{definition}}
\newtheorem{exercise}{Exercice}[chapter]
\renewcommand{\theexercise}{\arabic{exercise}}
\newtheorem{example}{Exemple}[chapter]
\renewcommand{\theexample}{\arabic{example}}
\newtheorem*{solution}{Solution}
\newtheorem*{application}{Application}
\newtheorem*{notation}{Notation}
\newtheorem*{vocabulary}{Vocabulaire}
\newtheorem*{properties}{Propriétés}



\theoremstyle{remark}
\newtheorem*{remark}{Remarque}
\newtheorem*{rappel}{Rappel}


\usepackage{etoolbox}
\AtBeginEnvironment{exercise}{\small}
\AtBeginEnvironment{example}{\small}

\usepackage{cases}
\usepackage[red]{mypack}

\usepackage[framemethod=TikZ]{mdframed}

\definecolor{bg}{rgb}{0.4,0.25,0.95}
\definecolor{pagebg}{rgb}{0,0,0.5}
\surroundwithmdframed[
   topline=false,
   rightline=false,
   bottomline=false,
   leftmargin=\parindent,
   skipabove=8pt,
   skipbelow=8pt,
   linecolor=blue,
   innerbottommargin=10pt,
   % backgroundcolor=bg,font=\color{orange}\sffamily, fontcolor=white
]{definition}

\usepackage{empheq}
\usepackage[most]{tcolorbox}

\newtcbox{\mymath}[1][]{%
    nobeforeafter, math upper, tcbox raise base,
    enhanced, colframe=blue!30!black,
    colback=red!10, boxrule=1pt,
    #1}

\usepackage{unixode}


\DeclareMathOperator{\ord}{ord}
\DeclareMathOperator{\orb}{orb}
\DeclareMathOperator{\stab}{stab}
\DeclareMathOperator{\Stab}{stab}
\DeclareMathOperator{\ppcm}{ppcm}
\DeclareMathOperator{\conj}{Conj}
\DeclareMathOperator{\End}{End}
\DeclareMathOperator{\rot}{rot}
\DeclareMathOperator{\trs}{trace}
\DeclareMathOperator{\Ind}{Ind}
\DeclareMathOperator{\mat}{Mat}
\DeclareMathOperator{\id}{Id}
\DeclareMathOperator{\vect}{vect}
\DeclareMathOperator{\img}{img}
\DeclareMathOperator{\cov}{Cov}
\DeclareMathOperator{\dist}{dist}
\DeclareMathOperator{\irr}{Irr}
\DeclareMathOperator{\image}{Im}
\DeclareMathOperator{\pd}{\partial}
\DeclareMathOperator{\epi}{epi}
\DeclareMathOperator{\Argmin}{Argmin}
\DeclareMathOperator{\dom}{dom}
\DeclareMathOperator{\proj}{proj}
\DeclareMathOperator{\ctg}{ctg}
\DeclareMathOperator{\supp}{supp}
\DeclareMathOperator{\argmin}{argmin}
\DeclareMathOperator{\mult}{mult}
\DeclareMathOperator{\ch}{ch}
\DeclareMathOperator{\sh}{sh}
\DeclareMathOperator{\rang}{rang}
\DeclareMathOperator{\diam}{diam}
\DeclareMathOperator{\Epigraphe}{Epigraphe}




\usepackage{xcolor}
\everymath{\color{blue}}
%\everymath{\color[rgb]{0,1,1}}
%\pagecolor[rgb]{0,0,0.5}


\newcommand*{\pdtest}[3][]{\ensuremath{\frac{\partial^{#1} #2}{\partial #3}}}

\newcommand*{\deffunc}[6][]{\ensuremath{
\begin{array}{rcl}
#2 : #3 &\rightarrow& #4\\
#5 &\mapsto& #6
\end{array}
}}

\newcommand{\eqcolon}{\mathrel{\resizebox{\widthof{$\mathord{=}$}}{\height}{ $\!\!=\!\!\resizebox{1.2\width}{0.8\height}{\raisebox{0.23ex}{$\mathop{:}$}}\!\!$ }}}
\newcommand{\coloneq}{\mathrel{\resizebox{\widthof{$\mathord{=}$}}{\height}{ $\!\!\resizebox{1.2\width}{0.8\height}{\raisebox{0.23ex}{$\mathop{:}$}}\!\!=\!\!$ }}}
\newcommand{\eqcolonl}{\ensuremath{\mathrel{=\!\!\mathop{:}}}}
\newcommand{\coloneql}{\ensuremath{\mathrel{\mathop{:} \!\! =}}}
\newcommand{\vc}[1]{% inline column vector
  \left(\begin{smallmatrix}#1\end{smallmatrix}\right)%
}
\newcommand{\vr}[1]{% inline row vector
  \begin{smallmatrix}(\,#1\,)\end{smallmatrix}%
}
\makeatletter
\newcommand*{\defeq}{\ =\mathrel{\rlap{%
                     \raisebox{0.3ex}{$\m@th\cdot$}}%
                     \raisebox{-0.3ex}{$\m@th\cdot$}}%
                     }
\makeatother

\newcommand{\mathcircle}[1]{% inline row vector
 \overset{\circ}{#1}
}
\newcommand{\ulim}{% low limit
 \underline{\lim}
}
\newcommand{\ssi}{% iff
\iff
}
\newcommand{\ps}[2]{
\expval{#1 | #2}
}
\newcommand{\df}[1]{
\mqty{#1}
}
\newcommand{\n}[1]{
\norm{#1}
}
\newcommand{\sys}[1]{
\left\{\smqty{#1}\right.
}


\newcommand{\eqdef}{\ensuremath{\overset{\text{def}}=}}


\def\Circlearrowright{\ensuremath{%
  \rotatebox[origin=c]{230}{$\circlearrowright$}}}

\newcommand\ct[1]{\text{\rmfamily\upshape #1}}
\newcommand\question[1]{ {\color{red} ...!? \small #1}}
\newcommand\caz[1]{\left\{\begin{array} #1 \end{array}\right.}
\newcommand\const{\text{\rmfamily\upshape const}}
\newcommand\toP{ \overset{\pro}{\to}}
\newcommand\toPP{ \overset{\text{PP}}{\to}}
\newcommand{\oeq}{\mathrel{\text{\textcircled{$=$}}}}





\usepackage{xcolor}
% \usepackage[normalem]{ulem}
\usepackage{lipsum}
\makeatletter
% \newcommand\colorwave[1][blue]{\bgroup \markoverwith{\lower3.5\p@\hbox{\sixly \textcolor{#1}{\char58}}}\ULon}
%\font\sixly=lasy6 % does not re-load if already loaded, so no memory problem.

\newmdtheoremenv[
linewidth= 1pt,linecolor= blue,%
leftmargin=20,rightmargin=20,innertopmargin=0pt, innerrightmargin=40,%
tikzsetting = { draw=lightgray, line width = 0.3pt,dashed,%
dash pattern = on 15pt off 3pt},%
splittopskip=\topskip,skipbelow=\baselineskip,%
skipabove=\baselineskip,ntheorem,roundcorner=0pt,
% backgroundcolor=pagebg,font=\color{orange}\sffamily, fontcolor=white
]{examplebox}{Exemple}[section]



\newcommand\R{\mathbb{R}}
\newcommand\Z{\mathbb{Z}}
\newcommand\N{\mathbb{N}}
\newcommand\E{\mathbb{E}}
\newcommand\F{\mathcal{F}}
\newcommand\cH{\mathcal{H}}
\newcommand\V{\mathbb{V}}
\newcommand\dmo{ ^{-1} }
\newcommand\kapa{\kappa}
\newcommand\im{Im}
\newcommand\hs{\mathcal{H}}





\usepackage{soul}

\makeatletter
\newcommand*{\whiten}[1]{\llap{\textcolor{white}{{\the\SOUL@token}}\hspace{#1pt}}}
\DeclareRobustCommand*\myul{%
    \def\SOUL@everyspace{\underline{\space}\kern\z@}%
    \def\SOUL@everytoken{%
     \setbox0=\hbox{\the\SOUL@token}%
     \ifdim\dp0>\z@
        \raisebox{\dp0}{\underline{\phantom{\the\SOUL@token}}}%
        \whiten{1}\whiten{0}%
        \whiten{-1}\whiten{-2}%
        \llap{\the\SOUL@token}%
     \else
        \underline{\the\SOUL@token}%
     \fi}%
\SOUL@}
\makeatother

\newcommand*{\demp}{\fontfamily{lmtt}\selectfont}

\DeclareTextFontCommand{\textdemp}{\demp}

\begin{document}

\ifcomment
Multiline
comment
\fi
\ifcomment
\myul{Typesetting test}
% \color[rgb]{1,1,1}
$∑_i^n≠ 60º±∞π∆¬≈√j∫h≤≥µ$

$\CR \R\pro\ind\pro\gS\pro
\mqty[a&b\\c&d]$
$\pro\mathbb{P}$
$\dd{x}$

  \[
    \alpha(x)=\left\{
                \begin{array}{ll}
                  x\\
                  \frac{1}{1+e^{-kx}}\\
                  \frac{e^x-e^{-x}}{e^x+e^{-x}}
                \end{array}
              \right.
  \]

  $\expval{x}$
  
  $\chi_\rho(ghg\dmo)=\Tr(\rho_{ghg\dmo})=\Tr(\rho_g\circ\rho_h\circ\rho\dmo_g)=\Tr(\rho_h)\overset{\mbox{\scalebox{0.5}{$\Tr(AB)=\Tr(BA)$}}}{=}\chi_\rho(h)$
  	$\mathop{\oplus}_{\substack{x\in X}}$

$\mat(\rho_g)=(a_{ij}(g))_{\scriptsize \substack{1\leq i\leq d \\ 1\leq j\leq d}}$ et $\mat(\rho'_g)=(a'_{ij}(g))_{\scriptsize \substack{1\leq i'\leq d' \\ 1\leq j'\leq d'}}$



\[\int_a^b{\mathbb{R}^2}g(u, v)\dd{P_{XY}}(u, v)=\iint g(u,v) f_{XY}(u, v)\dd \lambda(u) \dd \lambda(v)\]
$$\lim_{x\to\infty} f(x)$$	
$$\iiiint_V \mu(t,u,v,w) \,dt\,du\,dv\,dw$$
$$\sum_{n=1}^{\infty} 2^{-n} = 1$$	
\begin{definition}
	Si $X$ et $Y$ sont 2 v.a. ou definit la \textsc{Covariance} entre $X$ et $Y$ comme
	$\cov(X,Y)\overset{\text{def}}{=}\E\left[(X-\E(X))(Y-\E(Y))\right]=\E(XY)-\E(X)\E(Y)$.
\end{definition}
\fi
\pagebreak

% \tableofcontents

% insert your code here
%% !TEX encoding = UTF-8 Unicode
% !TEX TS-program = xelatex

\documentclass[french]{report}

%\usepackage[utf8]{inputenc}
%\usepackage[T1]{fontenc}
\usepackage{babel}


\newif\ifcomment
%\commenttrue # Show comments

\usepackage{physics}
\usepackage{amssymb}


\usepackage{amsthm}
% \usepackage{thmtools}
\usepackage{mathtools}
\usepackage{amsfonts}

\usepackage{color}

\usepackage{tikz}

\usepackage{geometry}
\geometry{a5paper, margin=0.1in, right=1cm}

\usepackage{dsfont}

\usepackage{graphicx}
\graphicspath{ {images/} }

\usepackage{faktor}

\usepackage{IEEEtrantools}
\usepackage{enumerate}   
\usepackage[PostScript=dvips]{"/Users/aware/Documents/Courses/diagrams"}


\newtheorem{theorem}{Théorème}[section]
\renewcommand{\thetheorem}{\arabic{theorem}}
\newtheorem{lemme}{Lemme}[section]
\renewcommand{\thelemme}{\arabic{lemme}}
\newtheorem{proposition}{Proposition}[section]
\renewcommand{\theproposition}{\arabic{proposition}}
\newtheorem{notations}{Notations}[section]
\newtheorem{problem}{Problème}[section]
\newtheorem{corollary}{Corollaire}[theorem]
\renewcommand{\thecorollary}{\arabic{corollary}}
\newtheorem{property}{Propriété}[section]
\newtheorem{objective}{Objectif}[section]

\theoremstyle{definition}
\newtheorem{definition}{Définition}[section]
\renewcommand{\thedefinition}{\arabic{definition}}
\newtheorem{exercise}{Exercice}[chapter]
\renewcommand{\theexercise}{\arabic{exercise}}
\newtheorem{example}{Exemple}[chapter]
\renewcommand{\theexample}{\arabic{example}}
\newtheorem*{solution}{Solution}
\newtheorem*{application}{Application}
\newtheorem*{notation}{Notation}
\newtheorem*{vocabulary}{Vocabulaire}
\newtheorem*{properties}{Propriétés}



\theoremstyle{remark}
\newtheorem*{remark}{Remarque}
\newtheorem*{rappel}{Rappel}


\usepackage{etoolbox}
\AtBeginEnvironment{exercise}{\small}
\AtBeginEnvironment{example}{\small}

\usepackage{cases}
\usepackage[red]{mypack}

\usepackage[framemethod=TikZ]{mdframed}

\definecolor{bg}{rgb}{0.4,0.25,0.95}
\definecolor{pagebg}{rgb}{0,0,0.5}
\surroundwithmdframed[
   topline=false,
   rightline=false,
   bottomline=false,
   leftmargin=\parindent,
   skipabove=8pt,
   skipbelow=8pt,
   linecolor=blue,
   innerbottommargin=10pt,
   % backgroundcolor=bg,font=\color{orange}\sffamily, fontcolor=white
]{definition}

\usepackage{empheq}
\usepackage[most]{tcolorbox}

\newtcbox{\mymath}[1][]{%
    nobeforeafter, math upper, tcbox raise base,
    enhanced, colframe=blue!30!black,
    colback=red!10, boxrule=1pt,
    #1}

\usepackage{unixode}


\DeclareMathOperator{\ord}{ord}
\DeclareMathOperator{\orb}{orb}
\DeclareMathOperator{\stab}{stab}
\DeclareMathOperator{\Stab}{stab}
\DeclareMathOperator{\ppcm}{ppcm}
\DeclareMathOperator{\conj}{Conj}
\DeclareMathOperator{\End}{End}
\DeclareMathOperator{\rot}{rot}
\DeclareMathOperator{\trs}{trace}
\DeclareMathOperator{\Ind}{Ind}
\DeclareMathOperator{\mat}{Mat}
\DeclareMathOperator{\id}{Id}
\DeclareMathOperator{\vect}{vect}
\DeclareMathOperator{\img}{img}
\DeclareMathOperator{\cov}{Cov}
\DeclareMathOperator{\dist}{dist}
\DeclareMathOperator{\irr}{Irr}
\DeclareMathOperator{\image}{Im}
\DeclareMathOperator{\pd}{\partial}
\DeclareMathOperator{\epi}{epi}
\DeclareMathOperator{\Argmin}{Argmin}
\DeclareMathOperator{\dom}{dom}
\DeclareMathOperator{\proj}{proj}
\DeclareMathOperator{\ctg}{ctg}
\DeclareMathOperator{\supp}{supp}
\DeclareMathOperator{\argmin}{argmin}
\DeclareMathOperator{\mult}{mult}
\DeclareMathOperator{\ch}{ch}
\DeclareMathOperator{\sh}{sh}
\DeclareMathOperator{\rang}{rang}
\DeclareMathOperator{\diam}{diam}
\DeclareMathOperator{\Epigraphe}{Epigraphe}




\usepackage{xcolor}
\everymath{\color{blue}}
%\everymath{\color[rgb]{0,1,1}}
%\pagecolor[rgb]{0,0,0.5}


\newcommand*{\pdtest}[3][]{\ensuremath{\frac{\partial^{#1} #2}{\partial #3}}}

\newcommand*{\deffunc}[6][]{\ensuremath{
\begin{array}{rcl}
#2 : #3 &\rightarrow& #4\\
#5 &\mapsto& #6
\end{array}
}}

\newcommand{\eqcolon}{\mathrel{\resizebox{\widthof{$\mathord{=}$}}{\height}{ $\!\!=\!\!\resizebox{1.2\width}{0.8\height}{\raisebox{0.23ex}{$\mathop{:}$}}\!\!$ }}}
\newcommand{\coloneq}{\mathrel{\resizebox{\widthof{$\mathord{=}$}}{\height}{ $\!\!\resizebox{1.2\width}{0.8\height}{\raisebox{0.23ex}{$\mathop{:}$}}\!\!=\!\!$ }}}
\newcommand{\eqcolonl}{\ensuremath{\mathrel{=\!\!\mathop{:}}}}
\newcommand{\coloneql}{\ensuremath{\mathrel{\mathop{:} \!\! =}}}
\newcommand{\vc}[1]{% inline column vector
  \left(\begin{smallmatrix}#1\end{smallmatrix}\right)%
}
\newcommand{\vr}[1]{% inline row vector
  \begin{smallmatrix}(\,#1\,)\end{smallmatrix}%
}
\makeatletter
\newcommand*{\defeq}{\ =\mathrel{\rlap{%
                     \raisebox{0.3ex}{$\m@th\cdot$}}%
                     \raisebox{-0.3ex}{$\m@th\cdot$}}%
                     }
\makeatother

\newcommand{\mathcircle}[1]{% inline row vector
 \overset{\circ}{#1}
}
\newcommand{\ulim}{% low limit
 \underline{\lim}
}
\newcommand{\ssi}{% iff
\iff
}
\newcommand{\ps}[2]{
\expval{#1 | #2}
}
\newcommand{\df}[1]{
\mqty{#1}
}
\newcommand{\n}[1]{
\norm{#1}
}
\newcommand{\sys}[1]{
\left\{\smqty{#1}\right.
}


\newcommand{\eqdef}{\ensuremath{\overset{\text{def}}=}}


\def\Circlearrowright{\ensuremath{%
  \rotatebox[origin=c]{230}{$\circlearrowright$}}}

\newcommand\ct[1]{\text{\rmfamily\upshape #1}}
\newcommand\question[1]{ {\color{red} ...!? \small #1}}
\newcommand\caz[1]{\left\{\begin{array} #1 \end{array}\right.}
\newcommand\const{\text{\rmfamily\upshape const}}
\newcommand\toP{ \overset{\pro}{\to}}
\newcommand\toPP{ \overset{\text{PP}}{\to}}
\newcommand{\oeq}{\mathrel{\text{\textcircled{$=$}}}}





\usepackage{xcolor}
% \usepackage[normalem]{ulem}
\usepackage{lipsum}
\makeatletter
% \newcommand\colorwave[1][blue]{\bgroup \markoverwith{\lower3.5\p@\hbox{\sixly \textcolor{#1}{\char58}}}\ULon}
%\font\sixly=lasy6 % does not re-load if already loaded, so no memory problem.

\newmdtheoremenv[
linewidth= 1pt,linecolor= blue,%
leftmargin=20,rightmargin=20,innertopmargin=0pt, innerrightmargin=40,%
tikzsetting = { draw=lightgray, line width = 0.3pt,dashed,%
dash pattern = on 15pt off 3pt},%
splittopskip=\topskip,skipbelow=\baselineskip,%
skipabove=\baselineskip,ntheorem,roundcorner=0pt,
% backgroundcolor=pagebg,font=\color{orange}\sffamily, fontcolor=white
]{examplebox}{Exemple}[section]



\newcommand\R{\mathbb{R}}
\newcommand\Z{\mathbb{Z}}
\newcommand\N{\mathbb{N}}
\newcommand\E{\mathbb{E}}
\newcommand\F{\mathcal{F}}
\newcommand\cH{\mathcal{H}}
\newcommand\V{\mathbb{V}}
\newcommand\dmo{ ^{-1} }
\newcommand\kapa{\kappa}
\newcommand\im{Im}
\newcommand\hs{\mathcal{H}}





\usepackage{soul}

\makeatletter
\newcommand*{\whiten}[1]{\llap{\textcolor{white}{{\the\SOUL@token}}\hspace{#1pt}}}
\DeclareRobustCommand*\myul{%
    \def\SOUL@everyspace{\underline{\space}\kern\z@}%
    \def\SOUL@everytoken{%
     \setbox0=\hbox{\the\SOUL@token}%
     \ifdim\dp0>\z@
        \raisebox{\dp0}{\underline{\phantom{\the\SOUL@token}}}%
        \whiten{1}\whiten{0}%
        \whiten{-1}\whiten{-2}%
        \llap{\the\SOUL@token}%
     \else
        \underline{\the\SOUL@token}%
     \fi}%
\SOUL@}
\makeatother

\newcommand*{\demp}{\fontfamily{lmtt}\selectfont}

\DeclareTextFontCommand{\textdemp}{\demp}

\begin{document}

\ifcomment
Multiline
comment
\fi
\ifcomment
\myul{Typesetting test}
% \color[rgb]{1,1,1}
$∑_i^n≠ 60º±∞π∆¬≈√j∫h≤≥µ$

$\CR \R\pro\ind\pro\gS\pro
\mqty[a&b\\c&d]$
$\pro\mathbb{P}$
$\dd{x}$

  \[
    \alpha(x)=\left\{
                \begin{array}{ll}
                  x\\
                  \frac{1}{1+e^{-kx}}\\
                  \frac{e^x-e^{-x}}{e^x+e^{-x}}
                \end{array}
              \right.
  \]

  $\expval{x}$
  
  $\chi_\rho(ghg\dmo)=\Tr(\rho_{ghg\dmo})=\Tr(\rho_g\circ\rho_h\circ\rho\dmo_g)=\Tr(\rho_h)\overset{\mbox{\scalebox{0.5}{$\Tr(AB)=\Tr(BA)$}}}{=}\chi_\rho(h)$
  	$\mathop{\oplus}_{\substack{x\in X}}$

$\mat(\rho_g)=(a_{ij}(g))_{\scriptsize \substack{1\leq i\leq d \\ 1\leq j\leq d}}$ et $\mat(\rho'_g)=(a'_{ij}(g))_{\scriptsize \substack{1\leq i'\leq d' \\ 1\leq j'\leq d'}}$



\[\int_a^b{\mathbb{R}^2}g(u, v)\dd{P_{XY}}(u, v)=\iint g(u,v) f_{XY}(u, v)\dd \lambda(u) \dd \lambda(v)\]
$$\lim_{x\to\infty} f(x)$$	
$$\iiiint_V \mu(t,u,v,w) \,dt\,du\,dv\,dw$$
$$\sum_{n=1}^{\infty} 2^{-n} = 1$$	
\begin{definition}
	Si $X$ et $Y$ sont 2 v.a. ou definit la \textsc{Covariance} entre $X$ et $Y$ comme
	$\cov(X,Y)\overset{\text{def}}{=}\E\left[(X-\E(X))(Y-\E(Y))\right]=\E(XY)-\E(X)\E(Y)$.
\end{definition}
\fi
\pagebreak

% \tableofcontents

% insert your code here
%\input{./algebra/main.tex}
%\input{./geometrie-differentielle/main.tex}
%\input{./probabilite/main.tex}
%\input{./analyse-fonctionnelle/main.tex}
% \input{./Analyse-convexe-et-dualite-en-optimisation/main.tex}
%\input{./tikz/main.tex}
%\input{./Theorie-du-distributions/main.tex}
%\input{./optimisation/mine.tex}
 \input{./modelisation/main.tex}

% yves.aubry@univ-tln.fr : algebra

\end{document}

%% !TEX encoding = UTF-8 Unicode
% !TEX TS-program = xelatex

\documentclass[french]{report}

%\usepackage[utf8]{inputenc}
%\usepackage[T1]{fontenc}
\usepackage{babel}


\newif\ifcomment
%\commenttrue # Show comments

\usepackage{physics}
\usepackage{amssymb}


\usepackage{amsthm}
% \usepackage{thmtools}
\usepackage{mathtools}
\usepackage{amsfonts}

\usepackage{color}

\usepackage{tikz}

\usepackage{geometry}
\geometry{a5paper, margin=0.1in, right=1cm}

\usepackage{dsfont}

\usepackage{graphicx}
\graphicspath{ {images/} }

\usepackage{faktor}

\usepackage{IEEEtrantools}
\usepackage{enumerate}   
\usepackage[PostScript=dvips]{"/Users/aware/Documents/Courses/diagrams"}


\newtheorem{theorem}{Théorème}[section]
\renewcommand{\thetheorem}{\arabic{theorem}}
\newtheorem{lemme}{Lemme}[section]
\renewcommand{\thelemme}{\arabic{lemme}}
\newtheorem{proposition}{Proposition}[section]
\renewcommand{\theproposition}{\arabic{proposition}}
\newtheorem{notations}{Notations}[section]
\newtheorem{problem}{Problème}[section]
\newtheorem{corollary}{Corollaire}[theorem]
\renewcommand{\thecorollary}{\arabic{corollary}}
\newtheorem{property}{Propriété}[section]
\newtheorem{objective}{Objectif}[section]

\theoremstyle{definition}
\newtheorem{definition}{Définition}[section]
\renewcommand{\thedefinition}{\arabic{definition}}
\newtheorem{exercise}{Exercice}[chapter]
\renewcommand{\theexercise}{\arabic{exercise}}
\newtheorem{example}{Exemple}[chapter]
\renewcommand{\theexample}{\arabic{example}}
\newtheorem*{solution}{Solution}
\newtheorem*{application}{Application}
\newtheorem*{notation}{Notation}
\newtheorem*{vocabulary}{Vocabulaire}
\newtheorem*{properties}{Propriétés}



\theoremstyle{remark}
\newtheorem*{remark}{Remarque}
\newtheorem*{rappel}{Rappel}


\usepackage{etoolbox}
\AtBeginEnvironment{exercise}{\small}
\AtBeginEnvironment{example}{\small}

\usepackage{cases}
\usepackage[red]{mypack}

\usepackage[framemethod=TikZ]{mdframed}

\definecolor{bg}{rgb}{0.4,0.25,0.95}
\definecolor{pagebg}{rgb}{0,0,0.5}
\surroundwithmdframed[
   topline=false,
   rightline=false,
   bottomline=false,
   leftmargin=\parindent,
   skipabove=8pt,
   skipbelow=8pt,
   linecolor=blue,
   innerbottommargin=10pt,
   % backgroundcolor=bg,font=\color{orange}\sffamily, fontcolor=white
]{definition}

\usepackage{empheq}
\usepackage[most]{tcolorbox}

\newtcbox{\mymath}[1][]{%
    nobeforeafter, math upper, tcbox raise base,
    enhanced, colframe=blue!30!black,
    colback=red!10, boxrule=1pt,
    #1}

\usepackage{unixode}


\DeclareMathOperator{\ord}{ord}
\DeclareMathOperator{\orb}{orb}
\DeclareMathOperator{\stab}{stab}
\DeclareMathOperator{\Stab}{stab}
\DeclareMathOperator{\ppcm}{ppcm}
\DeclareMathOperator{\conj}{Conj}
\DeclareMathOperator{\End}{End}
\DeclareMathOperator{\rot}{rot}
\DeclareMathOperator{\trs}{trace}
\DeclareMathOperator{\Ind}{Ind}
\DeclareMathOperator{\mat}{Mat}
\DeclareMathOperator{\id}{Id}
\DeclareMathOperator{\vect}{vect}
\DeclareMathOperator{\img}{img}
\DeclareMathOperator{\cov}{Cov}
\DeclareMathOperator{\dist}{dist}
\DeclareMathOperator{\irr}{Irr}
\DeclareMathOperator{\image}{Im}
\DeclareMathOperator{\pd}{\partial}
\DeclareMathOperator{\epi}{epi}
\DeclareMathOperator{\Argmin}{Argmin}
\DeclareMathOperator{\dom}{dom}
\DeclareMathOperator{\proj}{proj}
\DeclareMathOperator{\ctg}{ctg}
\DeclareMathOperator{\supp}{supp}
\DeclareMathOperator{\argmin}{argmin}
\DeclareMathOperator{\mult}{mult}
\DeclareMathOperator{\ch}{ch}
\DeclareMathOperator{\sh}{sh}
\DeclareMathOperator{\rang}{rang}
\DeclareMathOperator{\diam}{diam}
\DeclareMathOperator{\Epigraphe}{Epigraphe}




\usepackage{xcolor}
\everymath{\color{blue}}
%\everymath{\color[rgb]{0,1,1}}
%\pagecolor[rgb]{0,0,0.5}


\newcommand*{\pdtest}[3][]{\ensuremath{\frac{\partial^{#1} #2}{\partial #3}}}

\newcommand*{\deffunc}[6][]{\ensuremath{
\begin{array}{rcl}
#2 : #3 &\rightarrow& #4\\
#5 &\mapsto& #6
\end{array}
}}

\newcommand{\eqcolon}{\mathrel{\resizebox{\widthof{$\mathord{=}$}}{\height}{ $\!\!=\!\!\resizebox{1.2\width}{0.8\height}{\raisebox{0.23ex}{$\mathop{:}$}}\!\!$ }}}
\newcommand{\coloneq}{\mathrel{\resizebox{\widthof{$\mathord{=}$}}{\height}{ $\!\!\resizebox{1.2\width}{0.8\height}{\raisebox{0.23ex}{$\mathop{:}$}}\!\!=\!\!$ }}}
\newcommand{\eqcolonl}{\ensuremath{\mathrel{=\!\!\mathop{:}}}}
\newcommand{\coloneql}{\ensuremath{\mathrel{\mathop{:} \!\! =}}}
\newcommand{\vc}[1]{% inline column vector
  \left(\begin{smallmatrix}#1\end{smallmatrix}\right)%
}
\newcommand{\vr}[1]{% inline row vector
  \begin{smallmatrix}(\,#1\,)\end{smallmatrix}%
}
\makeatletter
\newcommand*{\defeq}{\ =\mathrel{\rlap{%
                     \raisebox{0.3ex}{$\m@th\cdot$}}%
                     \raisebox{-0.3ex}{$\m@th\cdot$}}%
                     }
\makeatother

\newcommand{\mathcircle}[1]{% inline row vector
 \overset{\circ}{#1}
}
\newcommand{\ulim}{% low limit
 \underline{\lim}
}
\newcommand{\ssi}{% iff
\iff
}
\newcommand{\ps}[2]{
\expval{#1 | #2}
}
\newcommand{\df}[1]{
\mqty{#1}
}
\newcommand{\n}[1]{
\norm{#1}
}
\newcommand{\sys}[1]{
\left\{\smqty{#1}\right.
}


\newcommand{\eqdef}{\ensuremath{\overset{\text{def}}=}}


\def\Circlearrowright{\ensuremath{%
  \rotatebox[origin=c]{230}{$\circlearrowright$}}}

\newcommand\ct[1]{\text{\rmfamily\upshape #1}}
\newcommand\question[1]{ {\color{red} ...!? \small #1}}
\newcommand\caz[1]{\left\{\begin{array} #1 \end{array}\right.}
\newcommand\const{\text{\rmfamily\upshape const}}
\newcommand\toP{ \overset{\pro}{\to}}
\newcommand\toPP{ \overset{\text{PP}}{\to}}
\newcommand{\oeq}{\mathrel{\text{\textcircled{$=$}}}}





\usepackage{xcolor}
% \usepackage[normalem]{ulem}
\usepackage{lipsum}
\makeatletter
% \newcommand\colorwave[1][blue]{\bgroup \markoverwith{\lower3.5\p@\hbox{\sixly \textcolor{#1}{\char58}}}\ULon}
%\font\sixly=lasy6 % does not re-load if already loaded, so no memory problem.

\newmdtheoremenv[
linewidth= 1pt,linecolor= blue,%
leftmargin=20,rightmargin=20,innertopmargin=0pt, innerrightmargin=40,%
tikzsetting = { draw=lightgray, line width = 0.3pt,dashed,%
dash pattern = on 15pt off 3pt},%
splittopskip=\topskip,skipbelow=\baselineskip,%
skipabove=\baselineskip,ntheorem,roundcorner=0pt,
% backgroundcolor=pagebg,font=\color{orange}\sffamily, fontcolor=white
]{examplebox}{Exemple}[section]



\newcommand\R{\mathbb{R}}
\newcommand\Z{\mathbb{Z}}
\newcommand\N{\mathbb{N}}
\newcommand\E{\mathbb{E}}
\newcommand\F{\mathcal{F}}
\newcommand\cH{\mathcal{H}}
\newcommand\V{\mathbb{V}}
\newcommand\dmo{ ^{-1} }
\newcommand\kapa{\kappa}
\newcommand\im{Im}
\newcommand\hs{\mathcal{H}}





\usepackage{soul}

\makeatletter
\newcommand*{\whiten}[1]{\llap{\textcolor{white}{{\the\SOUL@token}}\hspace{#1pt}}}
\DeclareRobustCommand*\myul{%
    \def\SOUL@everyspace{\underline{\space}\kern\z@}%
    \def\SOUL@everytoken{%
     \setbox0=\hbox{\the\SOUL@token}%
     \ifdim\dp0>\z@
        \raisebox{\dp0}{\underline{\phantom{\the\SOUL@token}}}%
        \whiten{1}\whiten{0}%
        \whiten{-1}\whiten{-2}%
        \llap{\the\SOUL@token}%
     \else
        \underline{\the\SOUL@token}%
     \fi}%
\SOUL@}
\makeatother

\newcommand*{\demp}{\fontfamily{lmtt}\selectfont}

\DeclareTextFontCommand{\textdemp}{\demp}

\begin{document}

\ifcomment
Multiline
comment
\fi
\ifcomment
\myul{Typesetting test}
% \color[rgb]{1,1,1}
$∑_i^n≠ 60º±∞π∆¬≈√j∫h≤≥µ$

$\CR \R\pro\ind\pro\gS\pro
\mqty[a&b\\c&d]$
$\pro\mathbb{P}$
$\dd{x}$

  \[
    \alpha(x)=\left\{
                \begin{array}{ll}
                  x\\
                  \frac{1}{1+e^{-kx}}\\
                  \frac{e^x-e^{-x}}{e^x+e^{-x}}
                \end{array}
              \right.
  \]

  $\expval{x}$
  
  $\chi_\rho(ghg\dmo)=\Tr(\rho_{ghg\dmo})=\Tr(\rho_g\circ\rho_h\circ\rho\dmo_g)=\Tr(\rho_h)\overset{\mbox{\scalebox{0.5}{$\Tr(AB)=\Tr(BA)$}}}{=}\chi_\rho(h)$
  	$\mathop{\oplus}_{\substack{x\in X}}$

$\mat(\rho_g)=(a_{ij}(g))_{\scriptsize \substack{1\leq i\leq d \\ 1\leq j\leq d}}$ et $\mat(\rho'_g)=(a'_{ij}(g))_{\scriptsize \substack{1\leq i'\leq d' \\ 1\leq j'\leq d'}}$



\[\int_a^b{\mathbb{R}^2}g(u, v)\dd{P_{XY}}(u, v)=\iint g(u,v) f_{XY}(u, v)\dd \lambda(u) \dd \lambda(v)\]
$$\lim_{x\to\infty} f(x)$$	
$$\iiiint_V \mu(t,u,v,w) \,dt\,du\,dv\,dw$$
$$\sum_{n=1}^{\infty} 2^{-n} = 1$$	
\begin{definition}
	Si $X$ et $Y$ sont 2 v.a. ou definit la \textsc{Covariance} entre $X$ et $Y$ comme
	$\cov(X,Y)\overset{\text{def}}{=}\E\left[(X-\E(X))(Y-\E(Y))\right]=\E(XY)-\E(X)\E(Y)$.
\end{definition}
\fi
\pagebreak

% \tableofcontents

% insert your code here
%\input{./algebra/main.tex}
%\input{./geometrie-differentielle/main.tex}
%\input{./probabilite/main.tex}
%\input{./analyse-fonctionnelle/main.tex}
% \input{./Analyse-convexe-et-dualite-en-optimisation/main.tex}
%\input{./tikz/main.tex}
%\input{./Theorie-du-distributions/main.tex}
%\input{./optimisation/mine.tex}
 \input{./modelisation/main.tex}

% yves.aubry@univ-tln.fr : algebra

\end{document}

%% !TEX encoding = UTF-8 Unicode
% !TEX TS-program = xelatex

\documentclass[french]{report}

%\usepackage[utf8]{inputenc}
%\usepackage[T1]{fontenc}
\usepackage{babel}


\newif\ifcomment
%\commenttrue # Show comments

\usepackage{physics}
\usepackage{amssymb}


\usepackage{amsthm}
% \usepackage{thmtools}
\usepackage{mathtools}
\usepackage{amsfonts}

\usepackage{color}

\usepackage{tikz}

\usepackage{geometry}
\geometry{a5paper, margin=0.1in, right=1cm}

\usepackage{dsfont}

\usepackage{graphicx}
\graphicspath{ {images/} }

\usepackage{faktor}

\usepackage{IEEEtrantools}
\usepackage{enumerate}   
\usepackage[PostScript=dvips]{"/Users/aware/Documents/Courses/diagrams"}


\newtheorem{theorem}{Théorème}[section]
\renewcommand{\thetheorem}{\arabic{theorem}}
\newtheorem{lemme}{Lemme}[section]
\renewcommand{\thelemme}{\arabic{lemme}}
\newtheorem{proposition}{Proposition}[section]
\renewcommand{\theproposition}{\arabic{proposition}}
\newtheorem{notations}{Notations}[section]
\newtheorem{problem}{Problème}[section]
\newtheorem{corollary}{Corollaire}[theorem]
\renewcommand{\thecorollary}{\arabic{corollary}}
\newtheorem{property}{Propriété}[section]
\newtheorem{objective}{Objectif}[section]

\theoremstyle{definition}
\newtheorem{definition}{Définition}[section]
\renewcommand{\thedefinition}{\arabic{definition}}
\newtheorem{exercise}{Exercice}[chapter]
\renewcommand{\theexercise}{\arabic{exercise}}
\newtheorem{example}{Exemple}[chapter]
\renewcommand{\theexample}{\arabic{example}}
\newtheorem*{solution}{Solution}
\newtheorem*{application}{Application}
\newtheorem*{notation}{Notation}
\newtheorem*{vocabulary}{Vocabulaire}
\newtheorem*{properties}{Propriétés}



\theoremstyle{remark}
\newtheorem*{remark}{Remarque}
\newtheorem*{rappel}{Rappel}


\usepackage{etoolbox}
\AtBeginEnvironment{exercise}{\small}
\AtBeginEnvironment{example}{\small}

\usepackage{cases}
\usepackage[red]{mypack}

\usepackage[framemethod=TikZ]{mdframed}

\definecolor{bg}{rgb}{0.4,0.25,0.95}
\definecolor{pagebg}{rgb}{0,0,0.5}
\surroundwithmdframed[
   topline=false,
   rightline=false,
   bottomline=false,
   leftmargin=\parindent,
   skipabove=8pt,
   skipbelow=8pt,
   linecolor=blue,
   innerbottommargin=10pt,
   % backgroundcolor=bg,font=\color{orange}\sffamily, fontcolor=white
]{definition}

\usepackage{empheq}
\usepackage[most]{tcolorbox}

\newtcbox{\mymath}[1][]{%
    nobeforeafter, math upper, tcbox raise base,
    enhanced, colframe=blue!30!black,
    colback=red!10, boxrule=1pt,
    #1}

\usepackage{unixode}


\DeclareMathOperator{\ord}{ord}
\DeclareMathOperator{\orb}{orb}
\DeclareMathOperator{\stab}{stab}
\DeclareMathOperator{\Stab}{stab}
\DeclareMathOperator{\ppcm}{ppcm}
\DeclareMathOperator{\conj}{Conj}
\DeclareMathOperator{\End}{End}
\DeclareMathOperator{\rot}{rot}
\DeclareMathOperator{\trs}{trace}
\DeclareMathOperator{\Ind}{Ind}
\DeclareMathOperator{\mat}{Mat}
\DeclareMathOperator{\id}{Id}
\DeclareMathOperator{\vect}{vect}
\DeclareMathOperator{\img}{img}
\DeclareMathOperator{\cov}{Cov}
\DeclareMathOperator{\dist}{dist}
\DeclareMathOperator{\irr}{Irr}
\DeclareMathOperator{\image}{Im}
\DeclareMathOperator{\pd}{\partial}
\DeclareMathOperator{\epi}{epi}
\DeclareMathOperator{\Argmin}{Argmin}
\DeclareMathOperator{\dom}{dom}
\DeclareMathOperator{\proj}{proj}
\DeclareMathOperator{\ctg}{ctg}
\DeclareMathOperator{\supp}{supp}
\DeclareMathOperator{\argmin}{argmin}
\DeclareMathOperator{\mult}{mult}
\DeclareMathOperator{\ch}{ch}
\DeclareMathOperator{\sh}{sh}
\DeclareMathOperator{\rang}{rang}
\DeclareMathOperator{\diam}{diam}
\DeclareMathOperator{\Epigraphe}{Epigraphe}




\usepackage{xcolor}
\everymath{\color{blue}}
%\everymath{\color[rgb]{0,1,1}}
%\pagecolor[rgb]{0,0,0.5}


\newcommand*{\pdtest}[3][]{\ensuremath{\frac{\partial^{#1} #2}{\partial #3}}}

\newcommand*{\deffunc}[6][]{\ensuremath{
\begin{array}{rcl}
#2 : #3 &\rightarrow& #4\\
#5 &\mapsto& #6
\end{array}
}}

\newcommand{\eqcolon}{\mathrel{\resizebox{\widthof{$\mathord{=}$}}{\height}{ $\!\!=\!\!\resizebox{1.2\width}{0.8\height}{\raisebox{0.23ex}{$\mathop{:}$}}\!\!$ }}}
\newcommand{\coloneq}{\mathrel{\resizebox{\widthof{$\mathord{=}$}}{\height}{ $\!\!\resizebox{1.2\width}{0.8\height}{\raisebox{0.23ex}{$\mathop{:}$}}\!\!=\!\!$ }}}
\newcommand{\eqcolonl}{\ensuremath{\mathrel{=\!\!\mathop{:}}}}
\newcommand{\coloneql}{\ensuremath{\mathrel{\mathop{:} \!\! =}}}
\newcommand{\vc}[1]{% inline column vector
  \left(\begin{smallmatrix}#1\end{smallmatrix}\right)%
}
\newcommand{\vr}[1]{% inline row vector
  \begin{smallmatrix}(\,#1\,)\end{smallmatrix}%
}
\makeatletter
\newcommand*{\defeq}{\ =\mathrel{\rlap{%
                     \raisebox{0.3ex}{$\m@th\cdot$}}%
                     \raisebox{-0.3ex}{$\m@th\cdot$}}%
                     }
\makeatother

\newcommand{\mathcircle}[1]{% inline row vector
 \overset{\circ}{#1}
}
\newcommand{\ulim}{% low limit
 \underline{\lim}
}
\newcommand{\ssi}{% iff
\iff
}
\newcommand{\ps}[2]{
\expval{#1 | #2}
}
\newcommand{\df}[1]{
\mqty{#1}
}
\newcommand{\n}[1]{
\norm{#1}
}
\newcommand{\sys}[1]{
\left\{\smqty{#1}\right.
}


\newcommand{\eqdef}{\ensuremath{\overset{\text{def}}=}}


\def\Circlearrowright{\ensuremath{%
  \rotatebox[origin=c]{230}{$\circlearrowright$}}}

\newcommand\ct[1]{\text{\rmfamily\upshape #1}}
\newcommand\question[1]{ {\color{red} ...!? \small #1}}
\newcommand\caz[1]{\left\{\begin{array} #1 \end{array}\right.}
\newcommand\const{\text{\rmfamily\upshape const}}
\newcommand\toP{ \overset{\pro}{\to}}
\newcommand\toPP{ \overset{\text{PP}}{\to}}
\newcommand{\oeq}{\mathrel{\text{\textcircled{$=$}}}}





\usepackage{xcolor}
% \usepackage[normalem]{ulem}
\usepackage{lipsum}
\makeatletter
% \newcommand\colorwave[1][blue]{\bgroup \markoverwith{\lower3.5\p@\hbox{\sixly \textcolor{#1}{\char58}}}\ULon}
%\font\sixly=lasy6 % does not re-load if already loaded, so no memory problem.

\newmdtheoremenv[
linewidth= 1pt,linecolor= blue,%
leftmargin=20,rightmargin=20,innertopmargin=0pt, innerrightmargin=40,%
tikzsetting = { draw=lightgray, line width = 0.3pt,dashed,%
dash pattern = on 15pt off 3pt},%
splittopskip=\topskip,skipbelow=\baselineskip,%
skipabove=\baselineskip,ntheorem,roundcorner=0pt,
% backgroundcolor=pagebg,font=\color{orange}\sffamily, fontcolor=white
]{examplebox}{Exemple}[section]



\newcommand\R{\mathbb{R}}
\newcommand\Z{\mathbb{Z}}
\newcommand\N{\mathbb{N}}
\newcommand\E{\mathbb{E}}
\newcommand\F{\mathcal{F}}
\newcommand\cH{\mathcal{H}}
\newcommand\V{\mathbb{V}}
\newcommand\dmo{ ^{-1} }
\newcommand\kapa{\kappa}
\newcommand\im{Im}
\newcommand\hs{\mathcal{H}}





\usepackage{soul}

\makeatletter
\newcommand*{\whiten}[1]{\llap{\textcolor{white}{{\the\SOUL@token}}\hspace{#1pt}}}
\DeclareRobustCommand*\myul{%
    \def\SOUL@everyspace{\underline{\space}\kern\z@}%
    \def\SOUL@everytoken{%
     \setbox0=\hbox{\the\SOUL@token}%
     \ifdim\dp0>\z@
        \raisebox{\dp0}{\underline{\phantom{\the\SOUL@token}}}%
        \whiten{1}\whiten{0}%
        \whiten{-1}\whiten{-2}%
        \llap{\the\SOUL@token}%
     \else
        \underline{\the\SOUL@token}%
     \fi}%
\SOUL@}
\makeatother

\newcommand*{\demp}{\fontfamily{lmtt}\selectfont}

\DeclareTextFontCommand{\textdemp}{\demp}

\begin{document}

\ifcomment
Multiline
comment
\fi
\ifcomment
\myul{Typesetting test}
% \color[rgb]{1,1,1}
$∑_i^n≠ 60º±∞π∆¬≈√j∫h≤≥µ$

$\CR \R\pro\ind\pro\gS\pro
\mqty[a&b\\c&d]$
$\pro\mathbb{P}$
$\dd{x}$

  \[
    \alpha(x)=\left\{
                \begin{array}{ll}
                  x\\
                  \frac{1}{1+e^{-kx}}\\
                  \frac{e^x-e^{-x}}{e^x+e^{-x}}
                \end{array}
              \right.
  \]

  $\expval{x}$
  
  $\chi_\rho(ghg\dmo)=\Tr(\rho_{ghg\dmo})=\Tr(\rho_g\circ\rho_h\circ\rho\dmo_g)=\Tr(\rho_h)\overset{\mbox{\scalebox{0.5}{$\Tr(AB)=\Tr(BA)$}}}{=}\chi_\rho(h)$
  	$\mathop{\oplus}_{\substack{x\in X}}$

$\mat(\rho_g)=(a_{ij}(g))_{\scriptsize \substack{1\leq i\leq d \\ 1\leq j\leq d}}$ et $\mat(\rho'_g)=(a'_{ij}(g))_{\scriptsize \substack{1\leq i'\leq d' \\ 1\leq j'\leq d'}}$



\[\int_a^b{\mathbb{R}^2}g(u, v)\dd{P_{XY}}(u, v)=\iint g(u,v) f_{XY}(u, v)\dd \lambda(u) \dd \lambda(v)\]
$$\lim_{x\to\infty} f(x)$$	
$$\iiiint_V \mu(t,u,v,w) \,dt\,du\,dv\,dw$$
$$\sum_{n=1}^{\infty} 2^{-n} = 1$$	
\begin{definition}
	Si $X$ et $Y$ sont 2 v.a. ou definit la \textsc{Covariance} entre $X$ et $Y$ comme
	$\cov(X,Y)\overset{\text{def}}{=}\E\left[(X-\E(X))(Y-\E(Y))\right]=\E(XY)-\E(X)\E(Y)$.
\end{definition}
\fi
\pagebreak

% \tableofcontents

% insert your code here
%\input{./algebra/main.tex}
%\input{./geometrie-differentielle/main.tex}
%\input{./probabilite/main.tex}
%\input{./analyse-fonctionnelle/main.tex}
% \input{./Analyse-convexe-et-dualite-en-optimisation/main.tex}
%\input{./tikz/main.tex}
%\input{./Theorie-du-distributions/main.tex}
%\input{./optimisation/mine.tex}
 \input{./modelisation/main.tex}

% yves.aubry@univ-tln.fr : algebra

\end{document}

%% !TEX encoding = UTF-8 Unicode
% !TEX TS-program = xelatex

\documentclass[french]{report}

%\usepackage[utf8]{inputenc}
%\usepackage[T1]{fontenc}
\usepackage{babel}


\newif\ifcomment
%\commenttrue # Show comments

\usepackage{physics}
\usepackage{amssymb}


\usepackage{amsthm}
% \usepackage{thmtools}
\usepackage{mathtools}
\usepackage{amsfonts}

\usepackage{color}

\usepackage{tikz}

\usepackage{geometry}
\geometry{a5paper, margin=0.1in, right=1cm}

\usepackage{dsfont}

\usepackage{graphicx}
\graphicspath{ {images/} }

\usepackage{faktor}

\usepackage{IEEEtrantools}
\usepackage{enumerate}   
\usepackage[PostScript=dvips]{"/Users/aware/Documents/Courses/diagrams"}


\newtheorem{theorem}{Théorème}[section]
\renewcommand{\thetheorem}{\arabic{theorem}}
\newtheorem{lemme}{Lemme}[section]
\renewcommand{\thelemme}{\arabic{lemme}}
\newtheorem{proposition}{Proposition}[section]
\renewcommand{\theproposition}{\arabic{proposition}}
\newtheorem{notations}{Notations}[section]
\newtheorem{problem}{Problème}[section]
\newtheorem{corollary}{Corollaire}[theorem]
\renewcommand{\thecorollary}{\arabic{corollary}}
\newtheorem{property}{Propriété}[section]
\newtheorem{objective}{Objectif}[section]

\theoremstyle{definition}
\newtheorem{definition}{Définition}[section]
\renewcommand{\thedefinition}{\arabic{definition}}
\newtheorem{exercise}{Exercice}[chapter]
\renewcommand{\theexercise}{\arabic{exercise}}
\newtheorem{example}{Exemple}[chapter]
\renewcommand{\theexample}{\arabic{example}}
\newtheorem*{solution}{Solution}
\newtheorem*{application}{Application}
\newtheorem*{notation}{Notation}
\newtheorem*{vocabulary}{Vocabulaire}
\newtheorem*{properties}{Propriétés}



\theoremstyle{remark}
\newtheorem*{remark}{Remarque}
\newtheorem*{rappel}{Rappel}


\usepackage{etoolbox}
\AtBeginEnvironment{exercise}{\small}
\AtBeginEnvironment{example}{\small}

\usepackage{cases}
\usepackage[red]{mypack}

\usepackage[framemethod=TikZ]{mdframed}

\definecolor{bg}{rgb}{0.4,0.25,0.95}
\definecolor{pagebg}{rgb}{0,0,0.5}
\surroundwithmdframed[
   topline=false,
   rightline=false,
   bottomline=false,
   leftmargin=\parindent,
   skipabove=8pt,
   skipbelow=8pt,
   linecolor=blue,
   innerbottommargin=10pt,
   % backgroundcolor=bg,font=\color{orange}\sffamily, fontcolor=white
]{definition}

\usepackage{empheq}
\usepackage[most]{tcolorbox}

\newtcbox{\mymath}[1][]{%
    nobeforeafter, math upper, tcbox raise base,
    enhanced, colframe=blue!30!black,
    colback=red!10, boxrule=1pt,
    #1}

\usepackage{unixode}


\DeclareMathOperator{\ord}{ord}
\DeclareMathOperator{\orb}{orb}
\DeclareMathOperator{\stab}{stab}
\DeclareMathOperator{\Stab}{stab}
\DeclareMathOperator{\ppcm}{ppcm}
\DeclareMathOperator{\conj}{Conj}
\DeclareMathOperator{\End}{End}
\DeclareMathOperator{\rot}{rot}
\DeclareMathOperator{\trs}{trace}
\DeclareMathOperator{\Ind}{Ind}
\DeclareMathOperator{\mat}{Mat}
\DeclareMathOperator{\id}{Id}
\DeclareMathOperator{\vect}{vect}
\DeclareMathOperator{\img}{img}
\DeclareMathOperator{\cov}{Cov}
\DeclareMathOperator{\dist}{dist}
\DeclareMathOperator{\irr}{Irr}
\DeclareMathOperator{\image}{Im}
\DeclareMathOperator{\pd}{\partial}
\DeclareMathOperator{\epi}{epi}
\DeclareMathOperator{\Argmin}{Argmin}
\DeclareMathOperator{\dom}{dom}
\DeclareMathOperator{\proj}{proj}
\DeclareMathOperator{\ctg}{ctg}
\DeclareMathOperator{\supp}{supp}
\DeclareMathOperator{\argmin}{argmin}
\DeclareMathOperator{\mult}{mult}
\DeclareMathOperator{\ch}{ch}
\DeclareMathOperator{\sh}{sh}
\DeclareMathOperator{\rang}{rang}
\DeclareMathOperator{\diam}{diam}
\DeclareMathOperator{\Epigraphe}{Epigraphe}




\usepackage{xcolor}
\everymath{\color{blue}}
%\everymath{\color[rgb]{0,1,1}}
%\pagecolor[rgb]{0,0,0.5}


\newcommand*{\pdtest}[3][]{\ensuremath{\frac{\partial^{#1} #2}{\partial #3}}}

\newcommand*{\deffunc}[6][]{\ensuremath{
\begin{array}{rcl}
#2 : #3 &\rightarrow& #4\\
#5 &\mapsto& #6
\end{array}
}}

\newcommand{\eqcolon}{\mathrel{\resizebox{\widthof{$\mathord{=}$}}{\height}{ $\!\!=\!\!\resizebox{1.2\width}{0.8\height}{\raisebox{0.23ex}{$\mathop{:}$}}\!\!$ }}}
\newcommand{\coloneq}{\mathrel{\resizebox{\widthof{$\mathord{=}$}}{\height}{ $\!\!\resizebox{1.2\width}{0.8\height}{\raisebox{0.23ex}{$\mathop{:}$}}\!\!=\!\!$ }}}
\newcommand{\eqcolonl}{\ensuremath{\mathrel{=\!\!\mathop{:}}}}
\newcommand{\coloneql}{\ensuremath{\mathrel{\mathop{:} \!\! =}}}
\newcommand{\vc}[1]{% inline column vector
  \left(\begin{smallmatrix}#1\end{smallmatrix}\right)%
}
\newcommand{\vr}[1]{% inline row vector
  \begin{smallmatrix}(\,#1\,)\end{smallmatrix}%
}
\makeatletter
\newcommand*{\defeq}{\ =\mathrel{\rlap{%
                     \raisebox{0.3ex}{$\m@th\cdot$}}%
                     \raisebox{-0.3ex}{$\m@th\cdot$}}%
                     }
\makeatother

\newcommand{\mathcircle}[1]{% inline row vector
 \overset{\circ}{#1}
}
\newcommand{\ulim}{% low limit
 \underline{\lim}
}
\newcommand{\ssi}{% iff
\iff
}
\newcommand{\ps}[2]{
\expval{#1 | #2}
}
\newcommand{\df}[1]{
\mqty{#1}
}
\newcommand{\n}[1]{
\norm{#1}
}
\newcommand{\sys}[1]{
\left\{\smqty{#1}\right.
}


\newcommand{\eqdef}{\ensuremath{\overset{\text{def}}=}}


\def\Circlearrowright{\ensuremath{%
  \rotatebox[origin=c]{230}{$\circlearrowright$}}}

\newcommand\ct[1]{\text{\rmfamily\upshape #1}}
\newcommand\question[1]{ {\color{red} ...!? \small #1}}
\newcommand\caz[1]{\left\{\begin{array} #1 \end{array}\right.}
\newcommand\const{\text{\rmfamily\upshape const}}
\newcommand\toP{ \overset{\pro}{\to}}
\newcommand\toPP{ \overset{\text{PP}}{\to}}
\newcommand{\oeq}{\mathrel{\text{\textcircled{$=$}}}}





\usepackage{xcolor}
% \usepackage[normalem]{ulem}
\usepackage{lipsum}
\makeatletter
% \newcommand\colorwave[1][blue]{\bgroup \markoverwith{\lower3.5\p@\hbox{\sixly \textcolor{#1}{\char58}}}\ULon}
%\font\sixly=lasy6 % does not re-load if already loaded, so no memory problem.

\newmdtheoremenv[
linewidth= 1pt,linecolor= blue,%
leftmargin=20,rightmargin=20,innertopmargin=0pt, innerrightmargin=40,%
tikzsetting = { draw=lightgray, line width = 0.3pt,dashed,%
dash pattern = on 15pt off 3pt},%
splittopskip=\topskip,skipbelow=\baselineskip,%
skipabove=\baselineskip,ntheorem,roundcorner=0pt,
% backgroundcolor=pagebg,font=\color{orange}\sffamily, fontcolor=white
]{examplebox}{Exemple}[section]



\newcommand\R{\mathbb{R}}
\newcommand\Z{\mathbb{Z}}
\newcommand\N{\mathbb{N}}
\newcommand\E{\mathbb{E}}
\newcommand\F{\mathcal{F}}
\newcommand\cH{\mathcal{H}}
\newcommand\V{\mathbb{V}}
\newcommand\dmo{ ^{-1} }
\newcommand\kapa{\kappa}
\newcommand\im{Im}
\newcommand\hs{\mathcal{H}}





\usepackage{soul}

\makeatletter
\newcommand*{\whiten}[1]{\llap{\textcolor{white}{{\the\SOUL@token}}\hspace{#1pt}}}
\DeclareRobustCommand*\myul{%
    \def\SOUL@everyspace{\underline{\space}\kern\z@}%
    \def\SOUL@everytoken{%
     \setbox0=\hbox{\the\SOUL@token}%
     \ifdim\dp0>\z@
        \raisebox{\dp0}{\underline{\phantom{\the\SOUL@token}}}%
        \whiten{1}\whiten{0}%
        \whiten{-1}\whiten{-2}%
        \llap{\the\SOUL@token}%
     \else
        \underline{\the\SOUL@token}%
     \fi}%
\SOUL@}
\makeatother

\newcommand*{\demp}{\fontfamily{lmtt}\selectfont}

\DeclareTextFontCommand{\textdemp}{\demp}

\begin{document}

\ifcomment
Multiline
comment
\fi
\ifcomment
\myul{Typesetting test}
% \color[rgb]{1,1,1}
$∑_i^n≠ 60º±∞π∆¬≈√j∫h≤≥µ$

$\CR \R\pro\ind\pro\gS\pro
\mqty[a&b\\c&d]$
$\pro\mathbb{P}$
$\dd{x}$

  \[
    \alpha(x)=\left\{
                \begin{array}{ll}
                  x\\
                  \frac{1}{1+e^{-kx}}\\
                  \frac{e^x-e^{-x}}{e^x+e^{-x}}
                \end{array}
              \right.
  \]

  $\expval{x}$
  
  $\chi_\rho(ghg\dmo)=\Tr(\rho_{ghg\dmo})=\Tr(\rho_g\circ\rho_h\circ\rho\dmo_g)=\Tr(\rho_h)\overset{\mbox{\scalebox{0.5}{$\Tr(AB)=\Tr(BA)$}}}{=}\chi_\rho(h)$
  	$\mathop{\oplus}_{\substack{x\in X}}$

$\mat(\rho_g)=(a_{ij}(g))_{\scriptsize \substack{1\leq i\leq d \\ 1\leq j\leq d}}$ et $\mat(\rho'_g)=(a'_{ij}(g))_{\scriptsize \substack{1\leq i'\leq d' \\ 1\leq j'\leq d'}}$



\[\int_a^b{\mathbb{R}^2}g(u, v)\dd{P_{XY}}(u, v)=\iint g(u,v) f_{XY}(u, v)\dd \lambda(u) \dd \lambda(v)\]
$$\lim_{x\to\infty} f(x)$$	
$$\iiiint_V \mu(t,u,v,w) \,dt\,du\,dv\,dw$$
$$\sum_{n=1}^{\infty} 2^{-n} = 1$$	
\begin{definition}
	Si $X$ et $Y$ sont 2 v.a. ou definit la \textsc{Covariance} entre $X$ et $Y$ comme
	$\cov(X,Y)\overset{\text{def}}{=}\E\left[(X-\E(X))(Y-\E(Y))\right]=\E(XY)-\E(X)\E(Y)$.
\end{definition}
\fi
\pagebreak

% \tableofcontents

% insert your code here
%\input{./algebra/main.tex}
%\input{./geometrie-differentielle/main.tex}
%\input{./probabilite/main.tex}
%\input{./analyse-fonctionnelle/main.tex}
% \input{./Analyse-convexe-et-dualite-en-optimisation/main.tex}
%\input{./tikz/main.tex}
%\input{./Theorie-du-distributions/main.tex}
%\input{./optimisation/mine.tex}
 \input{./modelisation/main.tex}

% yves.aubry@univ-tln.fr : algebra

\end{document}

% % !TEX encoding = UTF-8 Unicode
% !TEX TS-program = xelatex

\documentclass[french]{report}

%\usepackage[utf8]{inputenc}
%\usepackage[T1]{fontenc}
\usepackage{babel}


\newif\ifcomment
%\commenttrue # Show comments

\usepackage{physics}
\usepackage{amssymb}


\usepackage{amsthm}
% \usepackage{thmtools}
\usepackage{mathtools}
\usepackage{amsfonts}

\usepackage{color}

\usepackage{tikz}

\usepackage{geometry}
\geometry{a5paper, margin=0.1in, right=1cm}

\usepackage{dsfont}

\usepackage{graphicx}
\graphicspath{ {images/} }

\usepackage{faktor}

\usepackage{IEEEtrantools}
\usepackage{enumerate}   
\usepackage[PostScript=dvips]{"/Users/aware/Documents/Courses/diagrams"}


\newtheorem{theorem}{Théorème}[section]
\renewcommand{\thetheorem}{\arabic{theorem}}
\newtheorem{lemme}{Lemme}[section]
\renewcommand{\thelemme}{\arabic{lemme}}
\newtheorem{proposition}{Proposition}[section]
\renewcommand{\theproposition}{\arabic{proposition}}
\newtheorem{notations}{Notations}[section]
\newtheorem{problem}{Problème}[section]
\newtheorem{corollary}{Corollaire}[theorem]
\renewcommand{\thecorollary}{\arabic{corollary}}
\newtheorem{property}{Propriété}[section]
\newtheorem{objective}{Objectif}[section]

\theoremstyle{definition}
\newtheorem{definition}{Définition}[section]
\renewcommand{\thedefinition}{\arabic{definition}}
\newtheorem{exercise}{Exercice}[chapter]
\renewcommand{\theexercise}{\arabic{exercise}}
\newtheorem{example}{Exemple}[chapter]
\renewcommand{\theexample}{\arabic{example}}
\newtheorem*{solution}{Solution}
\newtheorem*{application}{Application}
\newtheorem*{notation}{Notation}
\newtheorem*{vocabulary}{Vocabulaire}
\newtheorem*{properties}{Propriétés}



\theoremstyle{remark}
\newtheorem*{remark}{Remarque}
\newtheorem*{rappel}{Rappel}


\usepackage{etoolbox}
\AtBeginEnvironment{exercise}{\small}
\AtBeginEnvironment{example}{\small}

\usepackage{cases}
\usepackage[red]{mypack}

\usepackage[framemethod=TikZ]{mdframed}

\definecolor{bg}{rgb}{0.4,0.25,0.95}
\definecolor{pagebg}{rgb}{0,0,0.5}
\surroundwithmdframed[
   topline=false,
   rightline=false,
   bottomline=false,
   leftmargin=\parindent,
   skipabove=8pt,
   skipbelow=8pt,
   linecolor=blue,
   innerbottommargin=10pt,
   % backgroundcolor=bg,font=\color{orange}\sffamily, fontcolor=white
]{definition}

\usepackage{empheq}
\usepackage[most]{tcolorbox}

\newtcbox{\mymath}[1][]{%
    nobeforeafter, math upper, tcbox raise base,
    enhanced, colframe=blue!30!black,
    colback=red!10, boxrule=1pt,
    #1}

\usepackage{unixode}


\DeclareMathOperator{\ord}{ord}
\DeclareMathOperator{\orb}{orb}
\DeclareMathOperator{\stab}{stab}
\DeclareMathOperator{\Stab}{stab}
\DeclareMathOperator{\ppcm}{ppcm}
\DeclareMathOperator{\conj}{Conj}
\DeclareMathOperator{\End}{End}
\DeclareMathOperator{\rot}{rot}
\DeclareMathOperator{\trs}{trace}
\DeclareMathOperator{\Ind}{Ind}
\DeclareMathOperator{\mat}{Mat}
\DeclareMathOperator{\id}{Id}
\DeclareMathOperator{\vect}{vect}
\DeclareMathOperator{\img}{img}
\DeclareMathOperator{\cov}{Cov}
\DeclareMathOperator{\dist}{dist}
\DeclareMathOperator{\irr}{Irr}
\DeclareMathOperator{\image}{Im}
\DeclareMathOperator{\pd}{\partial}
\DeclareMathOperator{\epi}{epi}
\DeclareMathOperator{\Argmin}{Argmin}
\DeclareMathOperator{\dom}{dom}
\DeclareMathOperator{\proj}{proj}
\DeclareMathOperator{\ctg}{ctg}
\DeclareMathOperator{\supp}{supp}
\DeclareMathOperator{\argmin}{argmin}
\DeclareMathOperator{\mult}{mult}
\DeclareMathOperator{\ch}{ch}
\DeclareMathOperator{\sh}{sh}
\DeclareMathOperator{\rang}{rang}
\DeclareMathOperator{\diam}{diam}
\DeclareMathOperator{\Epigraphe}{Epigraphe}




\usepackage{xcolor}
\everymath{\color{blue}}
%\everymath{\color[rgb]{0,1,1}}
%\pagecolor[rgb]{0,0,0.5}


\newcommand*{\pdtest}[3][]{\ensuremath{\frac{\partial^{#1} #2}{\partial #3}}}

\newcommand*{\deffunc}[6][]{\ensuremath{
\begin{array}{rcl}
#2 : #3 &\rightarrow& #4\\
#5 &\mapsto& #6
\end{array}
}}

\newcommand{\eqcolon}{\mathrel{\resizebox{\widthof{$\mathord{=}$}}{\height}{ $\!\!=\!\!\resizebox{1.2\width}{0.8\height}{\raisebox{0.23ex}{$\mathop{:}$}}\!\!$ }}}
\newcommand{\coloneq}{\mathrel{\resizebox{\widthof{$\mathord{=}$}}{\height}{ $\!\!\resizebox{1.2\width}{0.8\height}{\raisebox{0.23ex}{$\mathop{:}$}}\!\!=\!\!$ }}}
\newcommand{\eqcolonl}{\ensuremath{\mathrel{=\!\!\mathop{:}}}}
\newcommand{\coloneql}{\ensuremath{\mathrel{\mathop{:} \!\! =}}}
\newcommand{\vc}[1]{% inline column vector
  \left(\begin{smallmatrix}#1\end{smallmatrix}\right)%
}
\newcommand{\vr}[1]{% inline row vector
  \begin{smallmatrix}(\,#1\,)\end{smallmatrix}%
}
\makeatletter
\newcommand*{\defeq}{\ =\mathrel{\rlap{%
                     \raisebox{0.3ex}{$\m@th\cdot$}}%
                     \raisebox{-0.3ex}{$\m@th\cdot$}}%
                     }
\makeatother

\newcommand{\mathcircle}[1]{% inline row vector
 \overset{\circ}{#1}
}
\newcommand{\ulim}{% low limit
 \underline{\lim}
}
\newcommand{\ssi}{% iff
\iff
}
\newcommand{\ps}[2]{
\expval{#1 | #2}
}
\newcommand{\df}[1]{
\mqty{#1}
}
\newcommand{\n}[1]{
\norm{#1}
}
\newcommand{\sys}[1]{
\left\{\smqty{#1}\right.
}


\newcommand{\eqdef}{\ensuremath{\overset{\text{def}}=}}


\def\Circlearrowright{\ensuremath{%
  \rotatebox[origin=c]{230}{$\circlearrowright$}}}

\newcommand\ct[1]{\text{\rmfamily\upshape #1}}
\newcommand\question[1]{ {\color{red} ...!? \small #1}}
\newcommand\caz[1]{\left\{\begin{array} #1 \end{array}\right.}
\newcommand\const{\text{\rmfamily\upshape const}}
\newcommand\toP{ \overset{\pro}{\to}}
\newcommand\toPP{ \overset{\text{PP}}{\to}}
\newcommand{\oeq}{\mathrel{\text{\textcircled{$=$}}}}





\usepackage{xcolor}
% \usepackage[normalem]{ulem}
\usepackage{lipsum}
\makeatletter
% \newcommand\colorwave[1][blue]{\bgroup \markoverwith{\lower3.5\p@\hbox{\sixly \textcolor{#1}{\char58}}}\ULon}
%\font\sixly=lasy6 % does not re-load if already loaded, so no memory problem.

\newmdtheoremenv[
linewidth= 1pt,linecolor= blue,%
leftmargin=20,rightmargin=20,innertopmargin=0pt, innerrightmargin=40,%
tikzsetting = { draw=lightgray, line width = 0.3pt,dashed,%
dash pattern = on 15pt off 3pt},%
splittopskip=\topskip,skipbelow=\baselineskip,%
skipabove=\baselineskip,ntheorem,roundcorner=0pt,
% backgroundcolor=pagebg,font=\color{orange}\sffamily, fontcolor=white
]{examplebox}{Exemple}[section]



\newcommand\R{\mathbb{R}}
\newcommand\Z{\mathbb{Z}}
\newcommand\N{\mathbb{N}}
\newcommand\E{\mathbb{E}}
\newcommand\F{\mathcal{F}}
\newcommand\cH{\mathcal{H}}
\newcommand\V{\mathbb{V}}
\newcommand\dmo{ ^{-1} }
\newcommand\kapa{\kappa}
\newcommand\im{Im}
\newcommand\hs{\mathcal{H}}





\usepackage{soul}

\makeatletter
\newcommand*{\whiten}[1]{\llap{\textcolor{white}{{\the\SOUL@token}}\hspace{#1pt}}}
\DeclareRobustCommand*\myul{%
    \def\SOUL@everyspace{\underline{\space}\kern\z@}%
    \def\SOUL@everytoken{%
     \setbox0=\hbox{\the\SOUL@token}%
     \ifdim\dp0>\z@
        \raisebox{\dp0}{\underline{\phantom{\the\SOUL@token}}}%
        \whiten{1}\whiten{0}%
        \whiten{-1}\whiten{-2}%
        \llap{\the\SOUL@token}%
     \else
        \underline{\the\SOUL@token}%
     \fi}%
\SOUL@}
\makeatother

\newcommand*{\demp}{\fontfamily{lmtt}\selectfont}

\DeclareTextFontCommand{\textdemp}{\demp}

\begin{document}

\ifcomment
Multiline
comment
\fi
\ifcomment
\myul{Typesetting test}
% \color[rgb]{1,1,1}
$∑_i^n≠ 60º±∞π∆¬≈√j∫h≤≥µ$

$\CR \R\pro\ind\pro\gS\pro
\mqty[a&b\\c&d]$
$\pro\mathbb{P}$
$\dd{x}$

  \[
    \alpha(x)=\left\{
                \begin{array}{ll}
                  x\\
                  \frac{1}{1+e^{-kx}}\\
                  \frac{e^x-e^{-x}}{e^x+e^{-x}}
                \end{array}
              \right.
  \]

  $\expval{x}$
  
  $\chi_\rho(ghg\dmo)=\Tr(\rho_{ghg\dmo})=\Tr(\rho_g\circ\rho_h\circ\rho\dmo_g)=\Tr(\rho_h)\overset{\mbox{\scalebox{0.5}{$\Tr(AB)=\Tr(BA)$}}}{=}\chi_\rho(h)$
  	$\mathop{\oplus}_{\substack{x\in X}}$

$\mat(\rho_g)=(a_{ij}(g))_{\scriptsize \substack{1\leq i\leq d \\ 1\leq j\leq d}}$ et $\mat(\rho'_g)=(a'_{ij}(g))_{\scriptsize \substack{1\leq i'\leq d' \\ 1\leq j'\leq d'}}$



\[\int_a^b{\mathbb{R}^2}g(u, v)\dd{P_{XY}}(u, v)=\iint g(u,v) f_{XY}(u, v)\dd \lambda(u) \dd \lambda(v)\]
$$\lim_{x\to\infty} f(x)$$	
$$\iiiint_V \mu(t,u,v,w) \,dt\,du\,dv\,dw$$
$$\sum_{n=1}^{\infty} 2^{-n} = 1$$	
\begin{definition}
	Si $X$ et $Y$ sont 2 v.a. ou definit la \textsc{Covariance} entre $X$ et $Y$ comme
	$\cov(X,Y)\overset{\text{def}}{=}\E\left[(X-\E(X))(Y-\E(Y))\right]=\E(XY)-\E(X)\E(Y)$.
\end{definition}
\fi
\pagebreak

% \tableofcontents

% insert your code here
%\input{./algebra/main.tex}
%\input{./geometrie-differentielle/main.tex}
%\input{./probabilite/main.tex}
%\input{./analyse-fonctionnelle/main.tex}
% \input{./Analyse-convexe-et-dualite-en-optimisation/main.tex}
%\input{./tikz/main.tex}
%\input{./Theorie-du-distributions/main.tex}
%\input{./optimisation/mine.tex}
 \input{./modelisation/main.tex}

% yves.aubry@univ-tln.fr : algebra

\end{document}

%% !TEX encoding = UTF-8 Unicode
% !TEX TS-program = xelatex

\documentclass[french]{report}

%\usepackage[utf8]{inputenc}
%\usepackage[T1]{fontenc}
\usepackage{babel}


\newif\ifcomment
%\commenttrue # Show comments

\usepackage{physics}
\usepackage{amssymb}


\usepackage{amsthm}
% \usepackage{thmtools}
\usepackage{mathtools}
\usepackage{amsfonts}

\usepackage{color}

\usepackage{tikz}

\usepackage{geometry}
\geometry{a5paper, margin=0.1in, right=1cm}

\usepackage{dsfont}

\usepackage{graphicx}
\graphicspath{ {images/} }

\usepackage{faktor}

\usepackage{IEEEtrantools}
\usepackage{enumerate}   
\usepackage[PostScript=dvips]{"/Users/aware/Documents/Courses/diagrams"}


\newtheorem{theorem}{Théorème}[section]
\renewcommand{\thetheorem}{\arabic{theorem}}
\newtheorem{lemme}{Lemme}[section]
\renewcommand{\thelemme}{\arabic{lemme}}
\newtheorem{proposition}{Proposition}[section]
\renewcommand{\theproposition}{\arabic{proposition}}
\newtheorem{notations}{Notations}[section]
\newtheorem{problem}{Problème}[section]
\newtheorem{corollary}{Corollaire}[theorem]
\renewcommand{\thecorollary}{\arabic{corollary}}
\newtheorem{property}{Propriété}[section]
\newtheorem{objective}{Objectif}[section]

\theoremstyle{definition}
\newtheorem{definition}{Définition}[section]
\renewcommand{\thedefinition}{\arabic{definition}}
\newtheorem{exercise}{Exercice}[chapter]
\renewcommand{\theexercise}{\arabic{exercise}}
\newtheorem{example}{Exemple}[chapter]
\renewcommand{\theexample}{\arabic{example}}
\newtheorem*{solution}{Solution}
\newtheorem*{application}{Application}
\newtheorem*{notation}{Notation}
\newtheorem*{vocabulary}{Vocabulaire}
\newtheorem*{properties}{Propriétés}



\theoremstyle{remark}
\newtheorem*{remark}{Remarque}
\newtheorem*{rappel}{Rappel}


\usepackage{etoolbox}
\AtBeginEnvironment{exercise}{\small}
\AtBeginEnvironment{example}{\small}

\usepackage{cases}
\usepackage[red]{mypack}

\usepackage[framemethod=TikZ]{mdframed}

\definecolor{bg}{rgb}{0.4,0.25,0.95}
\definecolor{pagebg}{rgb}{0,0,0.5}
\surroundwithmdframed[
   topline=false,
   rightline=false,
   bottomline=false,
   leftmargin=\parindent,
   skipabove=8pt,
   skipbelow=8pt,
   linecolor=blue,
   innerbottommargin=10pt,
   % backgroundcolor=bg,font=\color{orange}\sffamily, fontcolor=white
]{definition}

\usepackage{empheq}
\usepackage[most]{tcolorbox}

\newtcbox{\mymath}[1][]{%
    nobeforeafter, math upper, tcbox raise base,
    enhanced, colframe=blue!30!black,
    colback=red!10, boxrule=1pt,
    #1}

\usepackage{unixode}


\DeclareMathOperator{\ord}{ord}
\DeclareMathOperator{\orb}{orb}
\DeclareMathOperator{\stab}{stab}
\DeclareMathOperator{\Stab}{stab}
\DeclareMathOperator{\ppcm}{ppcm}
\DeclareMathOperator{\conj}{Conj}
\DeclareMathOperator{\End}{End}
\DeclareMathOperator{\rot}{rot}
\DeclareMathOperator{\trs}{trace}
\DeclareMathOperator{\Ind}{Ind}
\DeclareMathOperator{\mat}{Mat}
\DeclareMathOperator{\id}{Id}
\DeclareMathOperator{\vect}{vect}
\DeclareMathOperator{\img}{img}
\DeclareMathOperator{\cov}{Cov}
\DeclareMathOperator{\dist}{dist}
\DeclareMathOperator{\irr}{Irr}
\DeclareMathOperator{\image}{Im}
\DeclareMathOperator{\pd}{\partial}
\DeclareMathOperator{\epi}{epi}
\DeclareMathOperator{\Argmin}{Argmin}
\DeclareMathOperator{\dom}{dom}
\DeclareMathOperator{\proj}{proj}
\DeclareMathOperator{\ctg}{ctg}
\DeclareMathOperator{\supp}{supp}
\DeclareMathOperator{\argmin}{argmin}
\DeclareMathOperator{\mult}{mult}
\DeclareMathOperator{\ch}{ch}
\DeclareMathOperator{\sh}{sh}
\DeclareMathOperator{\rang}{rang}
\DeclareMathOperator{\diam}{diam}
\DeclareMathOperator{\Epigraphe}{Epigraphe}




\usepackage{xcolor}
\everymath{\color{blue}}
%\everymath{\color[rgb]{0,1,1}}
%\pagecolor[rgb]{0,0,0.5}


\newcommand*{\pdtest}[3][]{\ensuremath{\frac{\partial^{#1} #2}{\partial #3}}}

\newcommand*{\deffunc}[6][]{\ensuremath{
\begin{array}{rcl}
#2 : #3 &\rightarrow& #4\\
#5 &\mapsto& #6
\end{array}
}}

\newcommand{\eqcolon}{\mathrel{\resizebox{\widthof{$\mathord{=}$}}{\height}{ $\!\!=\!\!\resizebox{1.2\width}{0.8\height}{\raisebox{0.23ex}{$\mathop{:}$}}\!\!$ }}}
\newcommand{\coloneq}{\mathrel{\resizebox{\widthof{$\mathord{=}$}}{\height}{ $\!\!\resizebox{1.2\width}{0.8\height}{\raisebox{0.23ex}{$\mathop{:}$}}\!\!=\!\!$ }}}
\newcommand{\eqcolonl}{\ensuremath{\mathrel{=\!\!\mathop{:}}}}
\newcommand{\coloneql}{\ensuremath{\mathrel{\mathop{:} \!\! =}}}
\newcommand{\vc}[1]{% inline column vector
  \left(\begin{smallmatrix}#1\end{smallmatrix}\right)%
}
\newcommand{\vr}[1]{% inline row vector
  \begin{smallmatrix}(\,#1\,)\end{smallmatrix}%
}
\makeatletter
\newcommand*{\defeq}{\ =\mathrel{\rlap{%
                     \raisebox{0.3ex}{$\m@th\cdot$}}%
                     \raisebox{-0.3ex}{$\m@th\cdot$}}%
                     }
\makeatother

\newcommand{\mathcircle}[1]{% inline row vector
 \overset{\circ}{#1}
}
\newcommand{\ulim}{% low limit
 \underline{\lim}
}
\newcommand{\ssi}{% iff
\iff
}
\newcommand{\ps}[2]{
\expval{#1 | #2}
}
\newcommand{\df}[1]{
\mqty{#1}
}
\newcommand{\n}[1]{
\norm{#1}
}
\newcommand{\sys}[1]{
\left\{\smqty{#1}\right.
}


\newcommand{\eqdef}{\ensuremath{\overset{\text{def}}=}}


\def\Circlearrowright{\ensuremath{%
  \rotatebox[origin=c]{230}{$\circlearrowright$}}}

\newcommand\ct[1]{\text{\rmfamily\upshape #1}}
\newcommand\question[1]{ {\color{red} ...!? \small #1}}
\newcommand\caz[1]{\left\{\begin{array} #1 \end{array}\right.}
\newcommand\const{\text{\rmfamily\upshape const}}
\newcommand\toP{ \overset{\pro}{\to}}
\newcommand\toPP{ \overset{\text{PP}}{\to}}
\newcommand{\oeq}{\mathrel{\text{\textcircled{$=$}}}}





\usepackage{xcolor}
% \usepackage[normalem]{ulem}
\usepackage{lipsum}
\makeatletter
% \newcommand\colorwave[1][blue]{\bgroup \markoverwith{\lower3.5\p@\hbox{\sixly \textcolor{#1}{\char58}}}\ULon}
%\font\sixly=lasy6 % does not re-load if already loaded, so no memory problem.

\newmdtheoremenv[
linewidth= 1pt,linecolor= blue,%
leftmargin=20,rightmargin=20,innertopmargin=0pt, innerrightmargin=40,%
tikzsetting = { draw=lightgray, line width = 0.3pt,dashed,%
dash pattern = on 15pt off 3pt},%
splittopskip=\topskip,skipbelow=\baselineskip,%
skipabove=\baselineskip,ntheorem,roundcorner=0pt,
% backgroundcolor=pagebg,font=\color{orange}\sffamily, fontcolor=white
]{examplebox}{Exemple}[section]



\newcommand\R{\mathbb{R}}
\newcommand\Z{\mathbb{Z}}
\newcommand\N{\mathbb{N}}
\newcommand\E{\mathbb{E}}
\newcommand\F{\mathcal{F}}
\newcommand\cH{\mathcal{H}}
\newcommand\V{\mathbb{V}}
\newcommand\dmo{ ^{-1} }
\newcommand\kapa{\kappa}
\newcommand\im{Im}
\newcommand\hs{\mathcal{H}}





\usepackage{soul}

\makeatletter
\newcommand*{\whiten}[1]{\llap{\textcolor{white}{{\the\SOUL@token}}\hspace{#1pt}}}
\DeclareRobustCommand*\myul{%
    \def\SOUL@everyspace{\underline{\space}\kern\z@}%
    \def\SOUL@everytoken{%
     \setbox0=\hbox{\the\SOUL@token}%
     \ifdim\dp0>\z@
        \raisebox{\dp0}{\underline{\phantom{\the\SOUL@token}}}%
        \whiten{1}\whiten{0}%
        \whiten{-1}\whiten{-2}%
        \llap{\the\SOUL@token}%
     \else
        \underline{\the\SOUL@token}%
     \fi}%
\SOUL@}
\makeatother

\newcommand*{\demp}{\fontfamily{lmtt}\selectfont}

\DeclareTextFontCommand{\textdemp}{\demp}

\begin{document}

\ifcomment
Multiline
comment
\fi
\ifcomment
\myul{Typesetting test}
% \color[rgb]{1,1,1}
$∑_i^n≠ 60º±∞π∆¬≈√j∫h≤≥µ$

$\CR \R\pro\ind\pro\gS\pro
\mqty[a&b\\c&d]$
$\pro\mathbb{P}$
$\dd{x}$

  \[
    \alpha(x)=\left\{
                \begin{array}{ll}
                  x\\
                  \frac{1}{1+e^{-kx}}\\
                  \frac{e^x-e^{-x}}{e^x+e^{-x}}
                \end{array}
              \right.
  \]

  $\expval{x}$
  
  $\chi_\rho(ghg\dmo)=\Tr(\rho_{ghg\dmo})=\Tr(\rho_g\circ\rho_h\circ\rho\dmo_g)=\Tr(\rho_h)\overset{\mbox{\scalebox{0.5}{$\Tr(AB)=\Tr(BA)$}}}{=}\chi_\rho(h)$
  	$\mathop{\oplus}_{\substack{x\in X}}$

$\mat(\rho_g)=(a_{ij}(g))_{\scriptsize \substack{1\leq i\leq d \\ 1\leq j\leq d}}$ et $\mat(\rho'_g)=(a'_{ij}(g))_{\scriptsize \substack{1\leq i'\leq d' \\ 1\leq j'\leq d'}}$



\[\int_a^b{\mathbb{R}^2}g(u, v)\dd{P_{XY}}(u, v)=\iint g(u,v) f_{XY}(u, v)\dd \lambda(u) \dd \lambda(v)\]
$$\lim_{x\to\infty} f(x)$$	
$$\iiiint_V \mu(t,u,v,w) \,dt\,du\,dv\,dw$$
$$\sum_{n=1}^{\infty} 2^{-n} = 1$$	
\begin{definition}
	Si $X$ et $Y$ sont 2 v.a. ou definit la \textsc{Covariance} entre $X$ et $Y$ comme
	$\cov(X,Y)\overset{\text{def}}{=}\E\left[(X-\E(X))(Y-\E(Y))\right]=\E(XY)-\E(X)\E(Y)$.
\end{definition}
\fi
\pagebreak

% \tableofcontents

% insert your code here
%\input{./algebra/main.tex}
%\input{./geometrie-differentielle/main.tex}
%\input{./probabilite/main.tex}
%\input{./analyse-fonctionnelle/main.tex}
% \input{./Analyse-convexe-et-dualite-en-optimisation/main.tex}
%\input{./tikz/main.tex}
%\input{./Theorie-du-distributions/main.tex}
%\input{./optimisation/mine.tex}
 \input{./modelisation/main.tex}

% yves.aubry@univ-tln.fr : algebra

\end{document}

%% !TEX encoding = UTF-8 Unicode
% !TEX TS-program = xelatex

\documentclass[french]{report}

%\usepackage[utf8]{inputenc}
%\usepackage[T1]{fontenc}
\usepackage{babel}


\newif\ifcomment
%\commenttrue # Show comments

\usepackage{physics}
\usepackage{amssymb}


\usepackage{amsthm}
% \usepackage{thmtools}
\usepackage{mathtools}
\usepackage{amsfonts}

\usepackage{color}

\usepackage{tikz}

\usepackage{geometry}
\geometry{a5paper, margin=0.1in, right=1cm}

\usepackage{dsfont}

\usepackage{graphicx}
\graphicspath{ {images/} }

\usepackage{faktor}

\usepackage{IEEEtrantools}
\usepackage{enumerate}   
\usepackage[PostScript=dvips]{"/Users/aware/Documents/Courses/diagrams"}


\newtheorem{theorem}{Théorème}[section]
\renewcommand{\thetheorem}{\arabic{theorem}}
\newtheorem{lemme}{Lemme}[section]
\renewcommand{\thelemme}{\arabic{lemme}}
\newtheorem{proposition}{Proposition}[section]
\renewcommand{\theproposition}{\arabic{proposition}}
\newtheorem{notations}{Notations}[section]
\newtheorem{problem}{Problème}[section]
\newtheorem{corollary}{Corollaire}[theorem]
\renewcommand{\thecorollary}{\arabic{corollary}}
\newtheorem{property}{Propriété}[section]
\newtheorem{objective}{Objectif}[section]

\theoremstyle{definition}
\newtheorem{definition}{Définition}[section]
\renewcommand{\thedefinition}{\arabic{definition}}
\newtheorem{exercise}{Exercice}[chapter]
\renewcommand{\theexercise}{\arabic{exercise}}
\newtheorem{example}{Exemple}[chapter]
\renewcommand{\theexample}{\arabic{example}}
\newtheorem*{solution}{Solution}
\newtheorem*{application}{Application}
\newtheorem*{notation}{Notation}
\newtheorem*{vocabulary}{Vocabulaire}
\newtheorem*{properties}{Propriétés}



\theoremstyle{remark}
\newtheorem*{remark}{Remarque}
\newtheorem*{rappel}{Rappel}


\usepackage{etoolbox}
\AtBeginEnvironment{exercise}{\small}
\AtBeginEnvironment{example}{\small}

\usepackage{cases}
\usepackage[red]{mypack}

\usepackage[framemethod=TikZ]{mdframed}

\definecolor{bg}{rgb}{0.4,0.25,0.95}
\definecolor{pagebg}{rgb}{0,0,0.5}
\surroundwithmdframed[
   topline=false,
   rightline=false,
   bottomline=false,
   leftmargin=\parindent,
   skipabove=8pt,
   skipbelow=8pt,
   linecolor=blue,
   innerbottommargin=10pt,
   % backgroundcolor=bg,font=\color{orange}\sffamily, fontcolor=white
]{definition}

\usepackage{empheq}
\usepackage[most]{tcolorbox}

\newtcbox{\mymath}[1][]{%
    nobeforeafter, math upper, tcbox raise base,
    enhanced, colframe=blue!30!black,
    colback=red!10, boxrule=1pt,
    #1}

\usepackage{unixode}


\DeclareMathOperator{\ord}{ord}
\DeclareMathOperator{\orb}{orb}
\DeclareMathOperator{\stab}{stab}
\DeclareMathOperator{\Stab}{stab}
\DeclareMathOperator{\ppcm}{ppcm}
\DeclareMathOperator{\conj}{Conj}
\DeclareMathOperator{\End}{End}
\DeclareMathOperator{\rot}{rot}
\DeclareMathOperator{\trs}{trace}
\DeclareMathOperator{\Ind}{Ind}
\DeclareMathOperator{\mat}{Mat}
\DeclareMathOperator{\id}{Id}
\DeclareMathOperator{\vect}{vect}
\DeclareMathOperator{\img}{img}
\DeclareMathOperator{\cov}{Cov}
\DeclareMathOperator{\dist}{dist}
\DeclareMathOperator{\irr}{Irr}
\DeclareMathOperator{\image}{Im}
\DeclareMathOperator{\pd}{\partial}
\DeclareMathOperator{\epi}{epi}
\DeclareMathOperator{\Argmin}{Argmin}
\DeclareMathOperator{\dom}{dom}
\DeclareMathOperator{\proj}{proj}
\DeclareMathOperator{\ctg}{ctg}
\DeclareMathOperator{\supp}{supp}
\DeclareMathOperator{\argmin}{argmin}
\DeclareMathOperator{\mult}{mult}
\DeclareMathOperator{\ch}{ch}
\DeclareMathOperator{\sh}{sh}
\DeclareMathOperator{\rang}{rang}
\DeclareMathOperator{\diam}{diam}
\DeclareMathOperator{\Epigraphe}{Epigraphe}




\usepackage{xcolor}
\everymath{\color{blue}}
%\everymath{\color[rgb]{0,1,1}}
%\pagecolor[rgb]{0,0,0.5}


\newcommand*{\pdtest}[3][]{\ensuremath{\frac{\partial^{#1} #2}{\partial #3}}}

\newcommand*{\deffunc}[6][]{\ensuremath{
\begin{array}{rcl}
#2 : #3 &\rightarrow& #4\\
#5 &\mapsto& #6
\end{array}
}}

\newcommand{\eqcolon}{\mathrel{\resizebox{\widthof{$\mathord{=}$}}{\height}{ $\!\!=\!\!\resizebox{1.2\width}{0.8\height}{\raisebox{0.23ex}{$\mathop{:}$}}\!\!$ }}}
\newcommand{\coloneq}{\mathrel{\resizebox{\widthof{$\mathord{=}$}}{\height}{ $\!\!\resizebox{1.2\width}{0.8\height}{\raisebox{0.23ex}{$\mathop{:}$}}\!\!=\!\!$ }}}
\newcommand{\eqcolonl}{\ensuremath{\mathrel{=\!\!\mathop{:}}}}
\newcommand{\coloneql}{\ensuremath{\mathrel{\mathop{:} \!\! =}}}
\newcommand{\vc}[1]{% inline column vector
  \left(\begin{smallmatrix}#1\end{smallmatrix}\right)%
}
\newcommand{\vr}[1]{% inline row vector
  \begin{smallmatrix}(\,#1\,)\end{smallmatrix}%
}
\makeatletter
\newcommand*{\defeq}{\ =\mathrel{\rlap{%
                     \raisebox{0.3ex}{$\m@th\cdot$}}%
                     \raisebox{-0.3ex}{$\m@th\cdot$}}%
                     }
\makeatother

\newcommand{\mathcircle}[1]{% inline row vector
 \overset{\circ}{#1}
}
\newcommand{\ulim}{% low limit
 \underline{\lim}
}
\newcommand{\ssi}{% iff
\iff
}
\newcommand{\ps}[2]{
\expval{#1 | #2}
}
\newcommand{\df}[1]{
\mqty{#1}
}
\newcommand{\n}[1]{
\norm{#1}
}
\newcommand{\sys}[1]{
\left\{\smqty{#1}\right.
}


\newcommand{\eqdef}{\ensuremath{\overset{\text{def}}=}}


\def\Circlearrowright{\ensuremath{%
  \rotatebox[origin=c]{230}{$\circlearrowright$}}}

\newcommand\ct[1]{\text{\rmfamily\upshape #1}}
\newcommand\question[1]{ {\color{red} ...!? \small #1}}
\newcommand\caz[1]{\left\{\begin{array} #1 \end{array}\right.}
\newcommand\const{\text{\rmfamily\upshape const}}
\newcommand\toP{ \overset{\pro}{\to}}
\newcommand\toPP{ \overset{\text{PP}}{\to}}
\newcommand{\oeq}{\mathrel{\text{\textcircled{$=$}}}}





\usepackage{xcolor}
% \usepackage[normalem]{ulem}
\usepackage{lipsum}
\makeatletter
% \newcommand\colorwave[1][blue]{\bgroup \markoverwith{\lower3.5\p@\hbox{\sixly \textcolor{#1}{\char58}}}\ULon}
%\font\sixly=lasy6 % does not re-load if already loaded, so no memory problem.

\newmdtheoremenv[
linewidth= 1pt,linecolor= blue,%
leftmargin=20,rightmargin=20,innertopmargin=0pt, innerrightmargin=40,%
tikzsetting = { draw=lightgray, line width = 0.3pt,dashed,%
dash pattern = on 15pt off 3pt},%
splittopskip=\topskip,skipbelow=\baselineskip,%
skipabove=\baselineskip,ntheorem,roundcorner=0pt,
% backgroundcolor=pagebg,font=\color{orange}\sffamily, fontcolor=white
]{examplebox}{Exemple}[section]



\newcommand\R{\mathbb{R}}
\newcommand\Z{\mathbb{Z}}
\newcommand\N{\mathbb{N}}
\newcommand\E{\mathbb{E}}
\newcommand\F{\mathcal{F}}
\newcommand\cH{\mathcal{H}}
\newcommand\V{\mathbb{V}}
\newcommand\dmo{ ^{-1} }
\newcommand\kapa{\kappa}
\newcommand\im{Im}
\newcommand\hs{\mathcal{H}}





\usepackage{soul}

\makeatletter
\newcommand*{\whiten}[1]{\llap{\textcolor{white}{{\the\SOUL@token}}\hspace{#1pt}}}
\DeclareRobustCommand*\myul{%
    \def\SOUL@everyspace{\underline{\space}\kern\z@}%
    \def\SOUL@everytoken{%
     \setbox0=\hbox{\the\SOUL@token}%
     \ifdim\dp0>\z@
        \raisebox{\dp0}{\underline{\phantom{\the\SOUL@token}}}%
        \whiten{1}\whiten{0}%
        \whiten{-1}\whiten{-2}%
        \llap{\the\SOUL@token}%
     \else
        \underline{\the\SOUL@token}%
     \fi}%
\SOUL@}
\makeatother

\newcommand*{\demp}{\fontfamily{lmtt}\selectfont}

\DeclareTextFontCommand{\textdemp}{\demp}

\begin{document}

\ifcomment
Multiline
comment
\fi
\ifcomment
\myul{Typesetting test}
% \color[rgb]{1,1,1}
$∑_i^n≠ 60º±∞π∆¬≈√j∫h≤≥µ$

$\CR \R\pro\ind\pro\gS\pro
\mqty[a&b\\c&d]$
$\pro\mathbb{P}$
$\dd{x}$

  \[
    \alpha(x)=\left\{
                \begin{array}{ll}
                  x\\
                  \frac{1}{1+e^{-kx}}\\
                  \frac{e^x-e^{-x}}{e^x+e^{-x}}
                \end{array}
              \right.
  \]

  $\expval{x}$
  
  $\chi_\rho(ghg\dmo)=\Tr(\rho_{ghg\dmo})=\Tr(\rho_g\circ\rho_h\circ\rho\dmo_g)=\Tr(\rho_h)\overset{\mbox{\scalebox{0.5}{$\Tr(AB)=\Tr(BA)$}}}{=}\chi_\rho(h)$
  	$\mathop{\oplus}_{\substack{x\in X}}$

$\mat(\rho_g)=(a_{ij}(g))_{\scriptsize \substack{1\leq i\leq d \\ 1\leq j\leq d}}$ et $\mat(\rho'_g)=(a'_{ij}(g))_{\scriptsize \substack{1\leq i'\leq d' \\ 1\leq j'\leq d'}}$



\[\int_a^b{\mathbb{R}^2}g(u, v)\dd{P_{XY}}(u, v)=\iint g(u,v) f_{XY}(u, v)\dd \lambda(u) \dd \lambda(v)\]
$$\lim_{x\to\infty} f(x)$$	
$$\iiiint_V \mu(t,u,v,w) \,dt\,du\,dv\,dw$$
$$\sum_{n=1}^{\infty} 2^{-n} = 1$$	
\begin{definition}
	Si $X$ et $Y$ sont 2 v.a. ou definit la \textsc{Covariance} entre $X$ et $Y$ comme
	$\cov(X,Y)\overset{\text{def}}{=}\E\left[(X-\E(X))(Y-\E(Y))\right]=\E(XY)-\E(X)\E(Y)$.
\end{definition}
\fi
\pagebreak

% \tableofcontents

% insert your code here
%\input{./algebra/main.tex}
%\input{./geometrie-differentielle/main.tex}
%\input{./probabilite/main.tex}
%\input{./analyse-fonctionnelle/main.tex}
% \input{./Analyse-convexe-et-dualite-en-optimisation/main.tex}
%\input{./tikz/main.tex}
%\input{./Theorie-du-distributions/main.tex}
%\input{./optimisation/mine.tex}
 \input{./modelisation/main.tex}

% yves.aubry@univ-tln.fr : algebra

\end{document}

%\input{./optimisation/mine.tex}
 % !TEX encoding = UTF-8 Unicode
% !TEX TS-program = xelatex

\documentclass[french]{report}

%\usepackage[utf8]{inputenc}
%\usepackage[T1]{fontenc}
\usepackage{babel}


\newif\ifcomment
%\commenttrue # Show comments

\usepackage{physics}
\usepackage{amssymb}


\usepackage{amsthm}
% \usepackage{thmtools}
\usepackage{mathtools}
\usepackage{amsfonts}

\usepackage{color}

\usepackage{tikz}

\usepackage{geometry}
\geometry{a5paper, margin=0.1in, right=1cm}

\usepackage{dsfont}

\usepackage{graphicx}
\graphicspath{ {images/} }

\usepackage{faktor}

\usepackage{IEEEtrantools}
\usepackage{enumerate}   
\usepackage[PostScript=dvips]{"/Users/aware/Documents/Courses/diagrams"}


\newtheorem{theorem}{Théorème}[section]
\renewcommand{\thetheorem}{\arabic{theorem}}
\newtheorem{lemme}{Lemme}[section]
\renewcommand{\thelemme}{\arabic{lemme}}
\newtheorem{proposition}{Proposition}[section]
\renewcommand{\theproposition}{\arabic{proposition}}
\newtheorem{notations}{Notations}[section]
\newtheorem{problem}{Problème}[section]
\newtheorem{corollary}{Corollaire}[theorem]
\renewcommand{\thecorollary}{\arabic{corollary}}
\newtheorem{property}{Propriété}[section]
\newtheorem{objective}{Objectif}[section]

\theoremstyle{definition}
\newtheorem{definition}{Définition}[section]
\renewcommand{\thedefinition}{\arabic{definition}}
\newtheorem{exercise}{Exercice}[chapter]
\renewcommand{\theexercise}{\arabic{exercise}}
\newtheorem{example}{Exemple}[chapter]
\renewcommand{\theexample}{\arabic{example}}
\newtheorem*{solution}{Solution}
\newtheorem*{application}{Application}
\newtheorem*{notation}{Notation}
\newtheorem*{vocabulary}{Vocabulaire}
\newtheorem*{properties}{Propriétés}



\theoremstyle{remark}
\newtheorem*{remark}{Remarque}
\newtheorem*{rappel}{Rappel}


\usepackage{etoolbox}
\AtBeginEnvironment{exercise}{\small}
\AtBeginEnvironment{example}{\small}

\usepackage{cases}
\usepackage[red]{mypack}

\usepackage[framemethod=TikZ]{mdframed}

\definecolor{bg}{rgb}{0.4,0.25,0.95}
\definecolor{pagebg}{rgb}{0,0,0.5}
\surroundwithmdframed[
   topline=false,
   rightline=false,
   bottomline=false,
   leftmargin=\parindent,
   skipabove=8pt,
   skipbelow=8pt,
   linecolor=blue,
   innerbottommargin=10pt,
   % backgroundcolor=bg,font=\color{orange}\sffamily, fontcolor=white
]{definition}

\usepackage{empheq}
\usepackage[most]{tcolorbox}

\newtcbox{\mymath}[1][]{%
    nobeforeafter, math upper, tcbox raise base,
    enhanced, colframe=blue!30!black,
    colback=red!10, boxrule=1pt,
    #1}

\usepackage{unixode}


\DeclareMathOperator{\ord}{ord}
\DeclareMathOperator{\orb}{orb}
\DeclareMathOperator{\stab}{stab}
\DeclareMathOperator{\Stab}{stab}
\DeclareMathOperator{\ppcm}{ppcm}
\DeclareMathOperator{\conj}{Conj}
\DeclareMathOperator{\End}{End}
\DeclareMathOperator{\rot}{rot}
\DeclareMathOperator{\trs}{trace}
\DeclareMathOperator{\Ind}{Ind}
\DeclareMathOperator{\mat}{Mat}
\DeclareMathOperator{\id}{Id}
\DeclareMathOperator{\vect}{vect}
\DeclareMathOperator{\img}{img}
\DeclareMathOperator{\cov}{Cov}
\DeclareMathOperator{\dist}{dist}
\DeclareMathOperator{\irr}{Irr}
\DeclareMathOperator{\image}{Im}
\DeclareMathOperator{\pd}{\partial}
\DeclareMathOperator{\epi}{epi}
\DeclareMathOperator{\Argmin}{Argmin}
\DeclareMathOperator{\dom}{dom}
\DeclareMathOperator{\proj}{proj}
\DeclareMathOperator{\ctg}{ctg}
\DeclareMathOperator{\supp}{supp}
\DeclareMathOperator{\argmin}{argmin}
\DeclareMathOperator{\mult}{mult}
\DeclareMathOperator{\ch}{ch}
\DeclareMathOperator{\sh}{sh}
\DeclareMathOperator{\rang}{rang}
\DeclareMathOperator{\diam}{diam}
\DeclareMathOperator{\Epigraphe}{Epigraphe}




\usepackage{xcolor}
\everymath{\color{blue}}
%\everymath{\color[rgb]{0,1,1}}
%\pagecolor[rgb]{0,0,0.5}


\newcommand*{\pdtest}[3][]{\ensuremath{\frac{\partial^{#1} #2}{\partial #3}}}

\newcommand*{\deffunc}[6][]{\ensuremath{
\begin{array}{rcl}
#2 : #3 &\rightarrow& #4\\
#5 &\mapsto& #6
\end{array}
}}

\newcommand{\eqcolon}{\mathrel{\resizebox{\widthof{$\mathord{=}$}}{\height}{ $\!\!=\!\!\resizebox{1.2\width}{0.8\height}{\raisebox{0.23ex}{$\mathop{:}$}}\!\!$ }}}
\newcommand{\coloneq}{\mathrel{\resizebox{\widthof{$\mathord{=}$}}{\height}{ $\!\!\resizebox{1.2\width}{0.8\height}{\raisebox{0.23ex}{$\mathop{:}$}}\!\!=\!\!$ }}}
\newcommand{\eqcolonl}{\ensuremath{\mathrel{=\!\!\mathop{:}}}}
\newcommand{\coloneql}{\ensuremath{\mathrel{\mathop{:} \!\! =}}}
\newcommand{\vc}[1]{% inline column vector
  \left(\begin{smallmatrix}#1\end{smallmatrix}\right)%
}
\newcommand{\vr}[1]{% inline row vector
  \begin{smallmatrix}(\,#1\,)\end{smallmatrix}%
}
\makeatletter
\newcommand*{\defeq}{\ =\mathrel{\rlap{%
                     \raisebox{0.3ex}{$\m@th\cdot$}}%
                     \raisebox{-0.3ex}{$\m@th\cdot$}}%
                     }
\makeatother

\newcommand{\mathcircle}[1]{% inline row vector
 \overset{\circ}{#1}
}
\newcommand{\ulim}{% low limit
 \underline{\lim}
}
\newcommand{\ssi}{% iff
\iff
}
\newcommand{\ps}[2]{
\expval{#1 | #2}
}
\newcommand{\df}[1]{
\mqty{#1}
}
\newcommand{\n}[1]{
\norm{#1}
}
\newcommand{\sys}[1]{
\left\{\smqty{#1}\right.
}


\newcommand{\eqdef}{\ensuremath{\overset{\text{def}}=}}


\def\Circlearrowright{\ensuremath{%
  \rotatebox[origin=c]{230}{$\circlearrowright$}}}

\newcommand\ct[1]{\text{\rmfamily\upshape #1}}
\newcommand\question[1]{ {\color{red} ...!? \small #1}}
\newcommand\caz[1]{\left\{\begin{array} #1 \end{array}\right.}
\newcommand\const{\text{\rmfamily\upshape const}}
\newcommand\toP{ \overset{\pro}{\to}}
\newcommand\toPP{ \overset{\text{PP}}{\to}}
\newcommand{\oeq}{\mathrel{\text{\textcircled{$=$}}}}





\usepackage{xcolor}
% \usepackage[normalem]{ulem}
\usepackage{lipsum}
\makeatletter
% \newcommand\colorwave[1][blue]{\bgroup \markoverwith{\lower3.5\p@\hbox{\sixly \textcolor{#1}{\char58}}}\ULon}
%\font\sixly=lasy6 % does not re-load if already loaded, so no memory problem.

\newmdtheoremenv[
linewidth= 1pt,linecolor= blue,%
leftmargin=20,rightmargin=20,innertopmargin=0pt, innerrightmargin=40,%
tikzsetting = { draw=lightgray, line width = 0.3pt,dashed,%
dash pattern = on 15pt off 3pt},%
splittopskip=\topskip,skipbelow=\baselineskip,%
skipabove=\baselineskip,ntheorem,roundcorner=0pt,
% backgroundcolor=pagebg,font=\color{orange}\sffamily, fontcolor=white
]{examplebox}{Exemple}[section]



\newcommand\R{\mathbb{R}}
\newcommand\Z{\mathbb{Z}}
\newcommand\N{\mathbb{N}}
\newcommand\E{\mathbb{E}}
\newcommand\F{\mathcal{F}}
\newcommand\cH{\mathcal{H}}
\newcommand\V{\mathbb{V}}
\newcommand\dmo{ ^{-1} }
\newcommand\kapa{\kappa}
\newcommand\im{Im}
\newcommand\hs{\mathcal{H}}





\usepackage{soul}

\makeatletter
\newcommand*{\whiten}[1]{\llap{\textcolor{white}{{\the\SOUL@token}}\hspace{#1pt}}}
\DeclareRobustCommand*\myul{%
    \def\SOUL@everyspace{\underline{\space}\kern\z@}%
    \def\SOUL@everytoken{%
     \setbox0=\hbox{\the\SOUL@token}%
     \ifdim\dp0>\z@
        \raisebox{\dp0}{\underline{\phantom{\the\SOUL@token}}}%
        \whiten{1}\whiten{0}%
        \whiten{-1}\whiten{-2}%
        \llap{\the\SOUL@token}%
     \else
        \underline{\the\SOUL@token}%
     \fi}%
\SOUL@}
\makeatother

\newcommand*{\demp}{\fontfamily{lmtt}\selectfont}

\DeclareTextFontCommand{\textdemp}{\demp}

\begin{document}

\ifcomment
Multiline
comment
\fi
\ifcomment
\myul{Typesetting test}
% \color[rgb]{1,1,1}
$∑_i^n≠ 60º±∞π∆¬≈√j∫h≤≥µ$

$\CR \R\pro\ind\pro\gS\pro
\mqty[a&b\\c&d]$
$\pro\mathbb{P}$
$\dd{x}$

  \[
    \alpha(x)=\left\{
                \begin{array}{ll}
                  x\\
                  \frac{1}{1+e^{-kx}}\\
                  \frac{e^x-e^{-x}}{e^x+e^{-x}}
                \end{array}
              \right.
  \]

  $\expval{x}$
  
  $\chi_\rho(ghg\dmo)=\Tr(\rho_{ghg\dmo})=\Tr(\rho_g\circ\rho_h\circ\rho\dmo_g)=\Tr(\rho_h)\overset{\mbox{\scalebox{0.5}{$\Tr(AB)=\Tr(BA)$}}}{=}\chi_\rho(h)$
  	$\mathop{\oplus}_{\substack{x\in X}}$

$\mat(\rho_g)=(a_{ij}(g))_{\scriptsize \substack{1\leq i\leq d \\ 1\leq j\leq d}}$ et $\mat(\rho'_g)=(a'_{ij}(g))_{\scriptsize \substack{1\leq i'\leq d' \\ 1\leq j'\leq d'}}$



\[\int_a^b{\mathbb{R}^2}g(u, v)\dd{P_{XY}}(u, v)=\iint g(u,v) f_{XY}(u, v)\dd \lambda(u) \dd \lambda(v)\]
$$\lim_{x\to\infty} f(x)$$	
$$\iiiint_V \mu(t,u,v,w) \,dt\,du\,dv\,dw$$
$$\sum_{n=1}^{\infty} 2^{-n} = 1$$	
\begin{definition}
	Si $X$ et $Y$ sont 2 v.a. ou definit la \textsc{Covariance} entre $X$ et $Y$ comme
	$\cov(X,Y)\overset{\text{def}}{=}\E\left[(X-\E(X))(Y-\E(Y))\right]=\E(XY)-\E(X)\E(Y)$.
\end{definition}
\fi
\pagebreak

% \tableofcontents

% insert your code here
%\input{./algebra/main.tex}
%\input{./geometrie-differentielle/main.tex}
%\input{./probabilite/main.tex}
%\input{./analyse-fonctionnelle/main.tex}
% \input{./Analyse-convexe-et-dualite-en-optimisation/main.tex}
%\input{./tikz/main.tex}
%\input{./Theorie-du-distributions/main.tex}
%\input{./optimisation/mine.tex}
 \input{./modelisation/main.tex}

% yves.aubry@univ-tln.fr : algebra

\end{document}


% yves.aubry@univ-tln.fr : algebra

\end{document}

%% !TEX encoding = UTF-8 Unicode
% !TEX TS-program = xelatex

\documentclass[french]{report}

%\usepackage[utf8]{inputenc}
%\usepackage[T1]{fontenc}
\usepackage{babel}


\newif\ifcomment
%\commenttrue # Show comments

\usepackage{physics}
\usepackage{amssymb}


\usepackage{amsthm}
% \usepackage{thmtools}
\usepackage{mathtools}
\usepackage{amsfonts}

\usepackage{color}

\usepackage{tikz}

\usepackage{geometry}
\geometry{a5paper, margin=0.1in, right=1cm}

\usepackage{dsfont}

\usepackage{graphicx}
\graphicspath{ {images/} }

\usepackage{faktor}

\usepackage{IEEEtrantools}
\usepackage{enumerate}   
\usepackage[PostScript=dvips]{"/Users/aware/Documents/Courses/diagrams"}


\newtheorem{theorem}{Théorème}[section]
\renewcommand{\thetheorem}{\arabic{theorem}}
\newtheorem{lemme}{Lemme}[section]
\renewcommand{\thelemme}{\arabic{lemme}}
\newtheorem{proposition}{Proposition}[section]
\renewcommand{\theproposition}{\arabic{proposition}}
\newtheorem{notations}{Notations}[section]
\newtheorem{problem}{Problème}[section]
\newtheorem{corollary}{Corollaire}[theorem]
\renewcommand{\thecorollary}{\arabic{corollary}}
\newtheorem{property}{Propriété}[section]
\newtheorem{objective}{Objectif}[section]

\theoremstyle{definition}
\newtheorem{definition}{Définition}[section]
\renewcommand{\thedefinition}{\arabic{definition}}
\newtheorem{exercise}{Exercice}[chapter]
\renewcommand{\theexercise}{\arabic{exercise}}
\newtheorem{example}{Exemple}[chapter]
\renewcommand{\theexample}{\arabic{example}}
\newtheorem*{solution}{Solution}
\newtheorem*{application}{Application}
\newtheorem*{notation}{Notation}
\newtheorem*{vocabulary}{Vocabulaire}
\newtheorem*{properties}{Propriétés}



\theoremstyle{remark}
\newtheorem*{remark}{Remarque}
\newtheorem*{rappel}{Rappel}


\usepackage{etoolbox}
\AtBeginEnvironment{exercise}{\small}
\AtBeginEnvironment{example}{\small}

\usepackage{cases}
\usepackage[red]{mypack}

\usepackage[framemethod=TikZ]{mdframed}

\definecolor{bg}{rgb}{0.4,0.25,0.95}
\definecolor{pagebg}{rgb}{0,0,0.5}
\surroundwithmdframed[
   topline=false,
   rightline=false,
   bottomline=false,
   leftmargin=\parindent,
   skipabove=8pt,
   skipbelow=8pt,
   linecolor=blue,
   innerbottommargin=10pt,
   % backgroundcolor=bg,font=\color{orange}\sffamily, fontcolor=white
]{definition}

\usepackage{empheq}
\usepackage[most]{tcolorbox}

\newtcbox{\mymath}[1][]{%
    nobeforeafter, math upper, tcbox raise base,
    enhanced, colframe=blue!30!black,
    colback=red!10, boxrule=1pt,
    #1}

\usepackage{unixode}


\DeclareMathOperator{\ord}{ord}
\DeclareMathOperator{\orb}{orb}
\DeclareMathOperator{\stab}{stab}
\DeclareMathOperator{\Stab}{stab}
\DeclareMathOperator{\ppcm}{ppcm}
\DeclareMathOperator{\conj}{Conj}
\DeclareMathOperator{\End}{End}
\DeclareMathOperator{\rot}{rot}
\DeclareMathOperator{\trs}{trace}
\DeclareMathOperator{\Ind}{Ind}
\DeclareMathOperator{\mat}{Mat}
\DeclareMathOperator{\id}{Id}
\DeclareMathOperator{\vect}{vect}
\DeclareMathOperator{\img}{img}
\DeclareMathOperator{\cov}{Cov}
\DeclareMathOperator{\dist}{dist}
\DeclareMathOperator{\irr}{Irr}
\DeclareMathOperator{\image}{Im}
\DeclareMathOperator{\pd}{\partial}
\DeclareMathOperator{\epi}{epi}
\DeclareMathOperator{\Argmin}{Argmin}
\DeclareMathOperator{\dom}{dom}
\DeclareMathOperator{\proj}{proj}
\DeclareMathOperator{\ctg}{ctg}
\DeclareMathOperator{\supp}{supp}
\DeclareMathOperator{\argmin}{argmin}
\DeclareMathOperator{\mult}{mult}
\DeclareMathOperator{\ch}{ch}
\DeclareMathOperator{\sh}{sh}
\DeclareMathOperator{\rang}{rang}
\DeclareMathOperator{\diam}{diam}
\DeclareMathOperator{\Epigraphe}{Epigraphe}




\usepackage{xcolor}
\everymath{\color{blue}}
%\everymath{\color[rgb]{0,1,1}}
%\pagecolor[rgb]{0,0,0.5}


\newcommand*{\pdtest}[3][]{\ensuremath{\frac{\partial^{#1} #2}{\partial #3}}}

\newcommand*{\deffunc}[6][]{\ensuremath{
\begin{array}{rcl}
#2 : #3 &\rightarrow& #4\\
#5 &\mapsto& #6
\end{array}
}}

\newcommand{\eqcolon}{\mathrel{\resizebox{\widthof{$\mathord{=}$}}{\height}{ $\!\!=\!\!\resizebox{1.2\width}{0.8\height}{\raisebox{0.23ex}{$\mathop{:}$}}\!\!$ }}}
\newcommand{\coloneq}{\mathrel{\resizebox{\widthof{$\mathord{=}$}}{\height}{ $\!\!\resizebox{1.2\width}{0.8\height}{\raisebox{0.23ex}{$\mathop{:}$}}\!\!=\!\!$ }}}
\newcommand{\eqcolonl}{\ensuremath{\mathrel{=\!\!\mathop{:}}}}
\newcommand{\coloneql}{\ensuremath{\mathrel{\mathop{:} \!\! =}}}
\newcommand{\vc}[1]{% inline column vector
  \left(\begin{smallmatrix}#1\end{smallmatrix}\right)%
}
\newcommand{\vr}[1]{% inline row vector
  \begin{smallmatrix}(\,#1\,)\end{smallmatrix}%
}
\makeatletter
\newcommand*{\defeq}{\ =\mathrel{\rlap{%
                     \raisebox{0.3ex}{$\m@th\cdot$}}%
                     \raisebox{-0.3ex}{$\m@th\cdot$}}%
                     }
\makeatother

\newcommand{\mathcircle}[1]{% inline row vector
 \overset{\circ}{#1}
}
\newcommand{\ulim}{% low limit
 \underline{\lim}
}
\newcommand{\ssi}{% iff
\iff
}
\newcommand{\ps}[2]{
\expval{#1 | #2}
}
\newcommand{\df}[1]{
\mqty{#1}
}
\newcommand{\n}[1]{
\norm{#1}
}
\newcommand{\sys}[1]{
\left\{\smqty{#1}\right.
}


\newcommand{\eqdef}{\ensuremath{\overset{\text{def}}=}}


\def\Circlearrowright{\ensuremath{%
  \rotatebox[origin=c]{230}{$\circlearrowright$}}}

\newcommand\ct[1]{\text{\rmfamily\upshape #1}}
\newcommand\question[1]{ {\color{red} ...!? \small #1}}
\newcommand\caz[1]{\left\{\begin{array} #1 \end{array}\right.}
\newcommand\const{\text{\rmfamily\upshape const}}
\newcommand\toP{ \overset{\pro}{\to}}
\newcommand\toPP{ \overset{\text{PP}}{\to}}
\newcommand{\oeq}{\mathrel{\text{\textcircled{$=$}}}}





\usepackage{xcolor}
% \usepackage[normalem]{ulem}
\usepackage{lipsum}
\makeatletter
% \newcommand\colorwave[1][blue]{\bgroup \markoverwith{\lower3.5\p@\hbox{\sixly \textcolor{#1}{\char58}}}\ULon}
%\font\sixly=lasy6 % does not re-load if already loaded, so no memory problem.

\newmdtheoremenv[
linewidth= 1pt,linecolor= blue,%
leftmargin=20,rightmargin=20,innertopmargin=0pt, innerrightmargin=40,%
tikzsetting = { draw=lightgray, line width = 0.3pt,dashed,%
dash pattern = on 15pt off 3pt},%
splittopskip=\topskip,skipbelow=\baselineskip,%
skipabove=\baselineskip,ntheorem,roundcorner=0pt,
% backgroundcolor=pagebg,font=\color{orange}\sffamily, fontcolor=white
]{examplebox}{Exemple}[section]



\newcommand\R{\mathbb{R}}
\newcommand\Z{\mathbb{Z}}
\newcommand\N{\mathbb{N}}
\newcommand\E{\mathbb{E}}
\newcommand\F{\mathcal{F}}
\newcommand\cH{\mathcal{H}}
\newcommand\V{\mathbb{V}}
\newcommand\dmo{ ^{-1} }
\newcommand\kapa{\kappa}
\newcommand\im{Im}
\newcommand\hs{\mathcal{H}}





\usepackage{soul}

\makeatletter
\newcommand*{\whiten}[1]{\llap{\textcolor{white}{{\the\SOUL@token}}\hspace{#1pt}}}
\DeclareRobustCommand*\myul{%
    \def\SOUL@everyspace{\underline{\space}\kern\z@}%
    \def\SOUL@everytoken{%
     \setbox0=\hbox{\the\SOUL@token}%
     \ifdim\dp0>\z@
        \raisebox{\dp0}{\underline{\phantom{\the\SOUL@token}}}%
        \whiten{1}\whiten{0}%
        \whiten{-1}\whiten{-2}%
        \llap{\the\SOUL@token}%
     \else
        \underline{\the\SOUL@token}%
     \fi}%
\SOUL@}
\makeatother

\newcommand*{\demp}{\fontfamily{lmtt}\selectfont}

\DeclareTextFontCommand{\textdemp}{\demp}

\begin{document}

\ifcomment
Multiline
comment
\fi
\ifcomment
\myul{Typesetting test}
% \color[rgb]{1,1,1}
$∑_i^n≠ 60º±∞π∆¬≈√j∫h≤≥µ$

$\CR \R\pro\ind\pro\gS\pro
\mqty[a&b\\c&d]$
$\pro\mathbb{P}$
$\dd{x}$

  \[
    \alpha(x)=\left\{
                \begin{array}{ll}
                  x\\
                  \frac{1}{1+e^{-kx}}\\
                  \frac{e^x-e^{-x}}{e^x+e^{-x}}
                \end{array}
              \right.
  \]

  $\expval{x}$
  
  $\chi_\rho(ghg\dmo)=\Tr(\rho_{ghg\dmo})=\Tr(\rho_g\circ\rho_h\circ\rho\dmo_g)=\Tr(\rho_h)\overset{\mbox{\scalebox{0.5}{$\Tr(AB)=\Tr(BA)$}}}{=}\chi_\rho(h)$
  	$\mathop{\oplus}_{\substack{x\in X}}$

$\mat(\rho_g)=(a_{ij}(g))_{\scriptsize \substack{1\leq i\leq d \\ 1\leq j\leq d}}$ et $\mat(\rho'_g)=(a'_{ij}(g))_{\scriptsize \substack{1\leq i'\leq d' \\ 1\leq j'\leq d'}}$



\[\int_a^b{\mathbb{R}^2}g(u, v)\dd{P_{XY}}(u, v)=\iint g(u,v) f_{XY}(u, v)\dd \lambda(u) \dd \lambda(v)\]
$$\lim_{x\to\infty} f(x)$$	
$$\iiiint_V \mu(t,u,v,w) \,dt\,du\,dv\,dw$$
$$\sum_{n=1}^{\infty} 2^{-n} = 1$$	
\begin{definition}
	Si $X$ et $Y$ sont 2 v.a. ou definit la \textsc{Covariance} entre $X$ et $Y$ comme
	$\cov(X,Y)\overset{\text{def}}{=}\E\left[(X-\E(X))(Y-\E(Y))\right]=\E(XY)-\E(X)\E(Y)$.
\end{definition}
\fi
\pagebreak

% \tableofcontents

% insert your code here
%% !TEX encoding = UTF-8 Unicode
% !TEX TS-program = xelatex

\documentclass[french]{report}

%\usepackage[utf8]{inputenc}
%\usepackage[T1]{fontenc}
\usepackage{babel}


\newif\ifcomment
%\commenttrue # Show comments

\usepackage{physics}
\usepackage{amssymb}


\usepackage{amsthm}
% \usepackage{thmtools}
\usepackage{mathtools}
\usepackage{amsfonts}

\usepackage{color}

\usepackage{tikz}

\usepackage{geometry}
\geometry{a5paper, margin=0.1in, right=1cm}

\usepackage{dsfont}

\usepackage{graphicx}
\graphicspath{ {images/} }

\usepackage{faktor}

\usepackage{IEEEtrantools}
\usepackage{enumerate}   
\usepackage[PostScript=dvips]{"/Users/aware/Documents/Courses/diagrams"}


\newtheorem{theorem}{Théorème}[section]
\renewcommand{\thetheorem}{\arabic{theorem}}
\newtheorem{lemme}{Lemme}[section]
\renewcommand{\thelemme}{\arabic{lemme}}
\newtheorem{proposition}{Proposition}[section]
\renewcommand{\theproposition}{\arabic{proposition}}
\newtheorem{notations}{Notations}[section]
\newtheorem{problem}{Problème}[section]
\newtheorem{corollary}{Corollaire}[theorem]
\renewcommand{\thecorollary}{\arabic{corollary}}
\newtheorem{property}{Propriété}[section]
\newtheorem{objective}{Objectif}[section]

\theoremstyle{definition}
\newtheorem{definition}{Définition}[section]
\renewcommand{\thedefinition}{\arabic{definition}}
\newtheorem{exercise}{Exercice}[chapter]
\renewcommand{\theexercise}{\arabic{exercise}}
\newtheorem{example}{Exemple}[chapter]
\renewcommand{\theexample}{\arabic{example}}
\newtheorem*{solution}{Solution}
\newtheorem*{application}{Application}
\newtheorem*{notation}{Notation}
\newtheorem*{vocabulary}{Vocabulaire}
\newtheorem*{properties}{Propriétés}



\theoremstyle{remark}
\newtheorem*{remark}{Remarque}
\newtheorem*{rappel}{Rappel}


\usepackage{etoolbox}
\AtBeginEnvironment{exercise}{\small}
\AtBeginEnvironment{example}{\small}

\usepackage{cases}
\usepackage[red]{mypack}

\usepackage[framemethod=TikZ]{mdframed}

\definecolor{bg}{rgb}{0.4,0.25,0.95}
\definecolor{pagebg}{rgb}{0,0,0.5}
\surroundwithmdframed[
   topline=false,
   rightline=false,
   bottomline=false,
   leftmargin=\parindent,
   skipabove=8pt,
   skipbelow=8pt,
   linecolor=blue,
   innerbottommargin=10pt,
   % backgroundcolor=bg,font=\color{orange}\sffamily, fontcolor=white
]{definition}

\usepackage{empheq}
\usepackage[most]{tcolorbox}

\newtcbox{\mymath}[1][]{%
    nobeforeafter, math upper, tcbox raise base,
    enhanced, colframe=blue!30!black,
    colback=red!10, boxrule=1pt,
    #1}

\usepackage{unixode}


\DeclareMathOperator{\ord}{ord}
\DeclareMathOperator{\orb}{orb}
\DeclareMathOperator{\stab}{stab}
\DeclareMathOperator{\Stab}{stab}
\DeclareMathOperator{\ppcm}{ppcm}
\DeclareMathOperator{\conj}{Conj}
\DeclareMathOperator{\End}{End}
\DeclareMathOperator{\rot}{rot}
\DeclareMathOperator{\trs}{trace}
\DeclareMathOperator{\Ind}{Ind}
\DeclareMathOperator{\mat}{Mat}
\DeclareMathOperator{\id}{Id}
\DeclareMathOperator{\vect}{vect}
\DeclareMathOperator{\img}{img}
\DeclareMathOperator{\cov}{Cov}
\DeclareMathOperator{\dist}{dist}
\DeclareMathOperator{\irr}{Irr}
\DeclareMathOperator{\image}{Im}
\DeclareMathOperator{\pd}{\partial}
\DeclareMathOperator{\epi}{epi}
\DeclareMathOperator{\Argmin}{Argmin}
\DeclareMathOperator{\dom}{dom}
\DeclareMathOperator{\proj}{proj}
\DeclareMathOperator{\ctg}{ctg}
\DeclareMathOperator{\supp}{supp}
\DeclareMathOperator{\argmin}{argmin}
\DeclareMathOperator{\mult}{mult}
\DeclareMathOperator{\ch}{ch}
\DeclareMathOperator{\sh}{sh}
\DeclareMathOperator{\rang}{rang}
\DeclareMathOperator{\diam}{diam}
\DeclareMathOperator{\Epigraphe}{Epigraphe}




\usepackage{xcolor}
\everymath{\color{blue}}
%\everymath{\color[rgb]{0,1,1}}
%\pagecolor[rgb]{0,0,0.5}


\newcommand*{\pdtest}[3][]{\ensuremath{\frac{\partial^{#1} #2}{\partial #3}}}

\newcommand*{\deffunc}[6][]{\ensuremath{
\begin{array}{rcl}
#2 : #3 &\rightarrow& #4\\
#5 &\mapsto& #6
\end{array}
}}

\newcommand{\eqcolon}{\mathrel{\resizebox{\widthof{$\mathord{=}$}}{\height}{ $\!\!=\!\!\resizebox{1.2\width}{0.8\height}{\raisebox{0.23ex}{$\mathop{:}$}}\!\!$ }}}
\newcommand{\coloneq}{\mathrel{\resizebox{\widthof{$\mathord{=}$}}{\height}{ $\!\!\resizebox{1.2\width}{0.8\height}{\raisebox{0.23ex}{$\mathop{:}$}}\!\!=\!\!$ }}}
\newcommand{\eqcolonl}{\ensuremath{\mathrel{=\!\!\mathop{:}}}}
\newcommand{\coloneql}{\ensuremath{\mathrel{\mathop{:} \!\! =}}}
\newcommand{\vc}[1]{% inline column vector
  \left(\begin{smallmatrix}#1\end{smallmatrix}\right)%
}
\newcommand{\vr}[1]{% inline row vector
  \begin{smallmatrix}(\,#1\,)\end{smallmatrix}%
}
\makeatletter
\newcommand*{\defeq}{\ =\mathrel{\rlap{%
                     \raisebox{0.3ex}{$\m@th\cdot$}}%
                     \raisebox{-0.3ex}{$\m@th\cdot$}}%
                     }
\makeatother

\newcommand{\mathcircle}[1]{% inline row vector
 \overset{\circ}{#1}
}
\newcommand{\ulim}{% low limit
 \underline{\lim}
}
\newcommand{\ssi}{% iff
\iff
}
\newcommand{\ps}[2]{
\expval{#1 | #2}
}
\newcommand{\df}[1]{
\mqty{#1}
}
\newcommand{\n}[1]{
\norm{#1}
}
\newcommand{\sys}[1]{
\left\{\smqty{#1}\right.
}


\newcommand{\eqdef}{\ensuremath{\overset{\text{def}}=}}


\def\Circlearrowright{\ensuremath{%
  \rotatebox[origin=c]{230}{$\circlearrowright$}}}

\newcommand\ct[1]{\text{\rmfamily\upshape #1}}
\newcommand\question[1]{ {\color{red} ...!? \small #1}}
\newcommand\caz[1]{\left\{\begin{array} #1 \end{array}\right.}
\newcommand\const{\text{\rmfamily\upshape const}}
\newcommand\toP{ \overset{\pro}{\to}}
\newcommand\toPP{ \overset{\text{PP}}{\to}}
\newcommand{\oeq}{\mathrel{\text{\textcircled{$=$}}}}





\usepackage{xcolor}
% \usepackage[normalem]{ulem}
\usepackage{lipsum}
\makeatletter
% \newcommand\colorwave[1][blue]{\bgroup \markoverwith{\lower3.5\p@\hbox{\sixly \textcolor{#1}{\char58}}}\ULon}
%\font\sixly=lasy6 % does not re-load if already loaded, so no memory problem.

\newmdtheoremenv[
linewidth= 1pt,linecolor= blue,%
leftmargin=20,rightmargin=20,innertopmargin=0pt, innerrightmargin=40,%
tikzsetting = { draw=lightgray, line width = 0.3pt,dashed,%
dash pattern = on 15pt off 3pt},%
splittopskip=\topskip,skipbelow=\baselineskip,%
skipabove=\baselineskip,ntheorem,roundcorner=0pt,
% backgroundcolor=pagebg,font=\color{orange}\sffamily, fontcolor=white
]{examplebox}{Exemple}[section]



\newcommand\R{\mathbb{R}}
\newcommand\Z{\mathbb{Z}}
\newcommand\N{\mathbb{N}}
\newcommand\E{\mathbb{E}}
\newcommand\F{\mathcal{F}}
\newcommand\cH{\mathcal{H}}
\newcommand\V{\mathbb{V}}
\newcommand\dmo{ ^{-1} }
\newcommand\kapa{\kappa}
\newcommand\im{Im}
\newcommand\hs{\mathcal{H}}





\usepackage{soul}

\makeatletter
\newcommand*{\whiten}[1]{\llap{\textcolor{white}{{\the\SOUL@token}}\hspace{#1pt}}}
\DeclareRobustCommand*\myul{%
    \def\SOUL@everyspace{\underline{\space}\kern\z@}%
    \def\SOUL@everytoken{%
     \setbox0=\hbox{\the\SOUL@token}%
     \ifdim\dp0>\z@
        \raisebox{\dp0}{\underline{\phantom{\the\SOUL@token}}}%
        \whiten{1}\whiten{0}%
        \whiten{-1}\whiten{-2}%
        \llap{\the\SOUL@token}%
     \else
        \underline{\the\SOUL@token}%
     \fi}%
\SOUL@}
\makeatother

\newcommand*{\demp}{\fontfamily{lmtt}\selectfont}

\DeclareTextFontCommand{\textdemp}{\demp}

\begin{document}

\ifcomment
Multiline
comment
\fi
\ifcomment
\myul{Typesetting test}
% \color[rgb]{1,1,1}
$∑_i^n≠ 60º±∞π∆¬≈√j∫h≤≥µ$

$\CR \R\pro\ind\pro\gS\pro
\mqty[a&b\\c&d]$
$\pro\mathbb{P}$
$\dd{x}$

  \[
    \alpha(x)=\left\{
                \begin{array}{ll}
                  x\\
                  \frac{1}{1+e^{-kx}}\\
                  \frac{e^x-e^{-x}}{e^x+e^{-x}}
                \end{array}
              \right.
  \]

  $\expval{x}$
  
  $\chi_\rho(ghg\dmo)=\Tr(\rho_{ghg\dmo})=\Tr(\rho_g\circ\rho_h\circ\rho\dmo_g)=\Tr(\rho_h)\overset{\mbox{\scalebox{0.5}{$\Tr(AB)=\Tr(BA)$}}}{=}\chi_\rho(h)$
  	$\mathop{\oplus}_{\substack{x\in X}}$

$\mat(\rho_g)=(a_{ij}(g))_{\scriptsize \substack{1\leq i\leq d \\ 1\leq j\leq d}}$ et $\mat(\rho'_g)=(a'_{ij}(g))_{\scriptsize \substack{1\leq i'\leq d' \\ 1\leq j'\leq d'}}$



\[\int_a^b{\mathbb{R}^2}g(u, v)\dd{P_{XY}}(u, v)=\iint g(u,v) f_{XY}(u, v)\dd \lambda(u) \dd \lambda(v)\]
$$\lim_{x\to\infty} f(x)$$	
$$\iiiint_V \mu(t,u,v,w) \,dt\,du\,dv\,dw$$
$$\sum_{n=1}^{\infty} 2^{-n} = 1$$	
\begin{definition}
	Si $X$ et $Y$ sont 2 v.a. ou definit la \textsc{Covariance} entre $X$ et $Y$ comme
	$\cov(X,Y)\overset{\text{def}}{=}\E\left[(X-\E(X))(Y-\E(Y))\right]=\E(XY)-\E(X)\E(Y)$.
\end{definition}
\fi
\pagebreak

% \tableofcontents

% insert your code here
%\input{./algebra/main.tex}
%\input{./geometrie-differentielle/main.tex}
%\input{./probabilite/main.tex}
%\input{./analyse-fonctionnelle/main.tex}
% \input{./Analyse-convexe-et-dualite-en-optimisation/main.tex}
%\input{./tikz/main.tex}
%\input{./Theorie-du-distributions/main.tex}
%\input{./optimisation/mine.tex}
 \input{./modelisation/main.tex}

% yves.aubry@univ-tln.fr : algebra

\end{document}

%% !TEX encoding = UTF-8 Unicode
% !TEX TS-program = xelatex

\documentclass[french]{report}

%\usepackage[utf8]{inputenc}
%\usepackage[T1]{fontenc}
\usepackage{babel}


\newif\ifcomment
%\commenttrue # Show comments

\usepackage{physics}
\usepackage{amssymb}


\usepackage{amsthm}
% \usepackage{thmtools}
\usepackage{mathtools}
\usepackage{amsfonts}

\usepackage{color}

\usepackage{tikz}

\usepackage{geometry}
\geometry{a5paper, margin=0.1in, right=1cm}

\usepackage{dsfont}

\usepackage{graphicx}
\graphicspath{ {images/} }

\usepackage{faktor}

\usepackage{IEEEtrantools}
\usepackage{enumerate}   
\usepackage[PostScript=dvips]{"/Users/aware/Documents/Courses/diagrams"}


\newtheorem{theorem}{Théorème}[section]
\renewcommand{\thetheorem}{\arabic{theorem}}
\newtheorem{lemme}{Lemme}[section]
\renewcommand{\thelemme}{\arabic{lemme}}
\newtheorem{proposition}{Proposition}[section]
\renewcommand{\theproposition}{\arabic{proposition}}
\newtheorem{notations}{Notations}[section]
\newtheorem{problem}{Problème}[section]
\newtheorem{corollary}{Corollaire}[theorem]
\renewcommand{\thecorollary}{\arabic{corollary}}
\newtheorem{property}{Propriété}[section]
\newtheorem{objective}{Objectif}[section]

\theoremstyle{definition}
\newtheorem{definition}{Définition}[section]
\renewcommand{\thedefinition}{\arabic{definition}}
\newtheorem{exercise}{Exercice}[chapter]
\renewcommand{\theexercise}{\arabic{exercise}}
\newtheorem{example}{Exemple}[chapter]
\renewcommand{\theexample}{\arabic{example}}
\newtheorem*{solution}{Solution}
\newtheorem*{application}{Application}
\newtheorem*{notation}{Notation}
\newtheorem*{vocabulary}{Vocabulaire}
\newtheorem*{properties}{Propriétés}



\theoremstyle{remark}
\newtheorem*{remark}{Remarque}
\newtheorem*{rappel}{Rappel}


\usepackage{etoolbox}
\AtBeginEnvironment{exercise}{\small}
\AtBeginEnvironment{example}{\small}

\usepackage{cases}
\usepackage[red]{mypack}

\usepackage[framemethod=TikZ]{mdframed}

\definecolor{bg}{rgb}{0.4,0.25,0.95}
\definecolor{pagebg}{rgb}{0,0,0.5}
\surroundwithmdframed[
   topline=false,
   rightline=false,
   bottomline=false,
   leftmargin=\parindent,
   skipabove=8pt,
   skipbelow=8pt,
   linecolor=blue,
   innerbottommargin=10pt,
   % backgroundcolor=bg,font=\color{orange}\sffamily, fontcolor=white
]{definition}

\usepackage{empheq}
\usepackage[most]{tcolorbox}

\newtcbox{\mymath}[1][]{%
    nobeforeafter, math upper, tcbox raise base,
    enhanced, colframe=blue!30!black,
    colback=red!10, boxrule=1pt,
    #1}

\usepackage{unixode}


\DeclareMathOperator{\ord}{ord}
\DeclareMathOperator{\orb}{orb}
\DeclareMathOperator{\stab}{stab}
\DeclareMathOperator{\Stab}{stab}
\DeclareMathOperator{\ppcm}{ppcm}
\DeclareMathOperator{\conj}{Conj}
\DeclareMathOperator{\End}{End}
\DeclareMathOperator{\rot}{rot}
\DeclareMathOperator{\trs}{trace}
\DeclareMathOperator{\Ind}{Ind}
\DeclareMathOperator{\mat}{Mat}
\DeclareMathOperator{\id}{Id}
\DeclareMathOperator{\vect}{vect}
\DeclareMathOperator{\img}{img}
\DeclareMathOperator{\cov}{Cov}
\DeclareMathOperator{\dist}{dist}
\DeclareMathOperator{\irr}{Irr}
\DeclareMathOperator{\image}{Im}
\DeclareMathOperator{\pd}{\partial}
\DeclareMathOperator{\epi}{epi}
\DeclareMathOperator{\Argmin}{Argmin}
\DeclareMathOperator{\dom}{dom}
\DeclareMathOperator{\proj}{proj}
\DeclareMathOperator{\ctg}{ctg}
\DeclareMathOperator{\supp}{supp}
\DeclareMathOperator{\argmin}{argmin}
\DeclareMathOperator{\mult}{mult}
\DeclareMathOperator{\ch}{ch}
\DeclareMathOperator{\sh}{sh}
\DeclareMathOperator{\rang}{rang}
\DeclareMathOperator{\diam}{diam}
\DeclareMathOperator{\Epigraphe}{Epigraphe}




\usepackage{xcolor}
\everymath{\color{blue}}
%\everymath{\color[rgb]{0,1,1}}
%\pagecolor[rgb]{0,0,0.5}


\newcommand*{\pdtest}[3][]{\ensuremath{\frac{\partial^{#1} #2}{\partial #3}}}

\newcommand*{\deffunc}[6][]{\ensuremath{
\begin{array}{rcl}
#2 : #3 &\rightarrow& #4\\
#5 &\mapsto& #6
\end{array}
}}

\newcommand{\eqcolon}{\mathrel{\resizebox{\widthof{$\mathord{=}$}}{\height}{ $\!\!=\!\!\resizebox{1.2\width}{0.8\height}{\raisebox{0.23ex}{$\mathop{:}$}}\!\!$ }}}
\newcommand{\coloneq}{\mathrel{\resizebox{\widthof{$\mathord{=}$}}{\height}{ $\!\!\resizebox{1.2\width}{0.8\height}{\raisebox{0.23ex}{$\mathop{:}$}}\!\!=\!\!$ }}}
\newcommand{\eqcolonl}{\ensuremath{\mathrel{=\!\!\mathop{:}}}}
\newcommand{\coloneql}{\ensuremath{\mathrel{\mathop{:} \!\! =}}}
\newcommand{\vc}[1]{% inline column vector
  \left(\begin{smallmatrix}#1\end{smallmatrix}\right)%
}
\newcommand{\vr}[1]{% inline row vector
  \begin{smallmatrix}(\,#1\,)\end{smallmatrix}%
}
\makeatletter
\newcommand*{\defeq}{\ =\mathrel{\rlap{%
                     \raisebox{0.3ex}{$\m@th\cdot$}}%
                     \raisebox{-0.3ex}{$\m@th\cdot$}}%
                     }
\makeatother

\newcommand{\mathcircle}[1]{% inline row vector
 \overset{\circ}{#1}
}
\newcommand{\ulim}{% low limit
 \underline{\lim}
}
\newcommand{\ssi}{% iff
\iff
}
\newcommand{\ps}[2]{
\expval{#1 | #2}
}
\newcommand{\df}[1]{
\mqty{#1}
}
\newcommand{\n}[1]{
\norm{#1}
}
\newcommand{\sys}[1]{
\left\{\smqty{#1}\right.
}


\newcommand{\eqdef}{\ensuremath{\overset{\text{def}}=}}


\def\Circlearrowright{\ensuremath{%
  \rotatebox[origin=c]{230}{$\circlearrowright$}}}

\newcommand\ct[1]{\text{\rmfamily\upshape #1}}
\newcommand\question[1]{ {\color{red} ...!? \small #1}}
\newcommand\caz[1]{\left\{\begin{array} #1 \end{array}\right.}
\newcommand\const{\text{\rmfamily\upshape const}}
\newcommand\toP{ \overset{\pro}{\to}}
\newcommand\toPP{ \overset{\text{PP}}{\to}}
\newcommand{\oeq}{\mathrel{\text{\textcircled{$=$}}}}





\usepackage{xcolor}
% \usepackage[normalem]{ulem}
\usepackage{lipsum}
\makeatletter
% \newcommand\colorwave[1][blue]{\bgroup \markoverwith{\lower3.5\p@\hbox{\sixly \textcolor{#1}{\char58}}}\ULon}
%\font\sixly=lasy6 % does not re-load if already loaded, so no memory problem.

\newmdtheoremenv[
linewidth= 1pt,linecolor= blue,%
leftmargin=20,rightmargin=20,innertopmargin=0pt, innerrightmargin=40,%
tikzsetting = { draw=lightgray, line width = 0.3pt,dashed,%
dash pattern = on 15pt off 3pt},%
splittopskip=\topskip,skipbelow=\baselineskip,%
skipabove=\baselineskip,ntheorem,roundcorner=0pt,
% backgroundcolor=pagebg,font=\color{orange}\sffamily, fontcolor=white
]{examplebox}{Exemple}[section]



\newcommand\R{\mathbb{R}}
\newcommand\Z{\mathbb{Z}}
\newcommand\N{\mathbb{N}}
\newcommand\E{\mathbb{E}}
\newcommand\F{\mathcal{F}}
\newcommand\cH{\mathcal{H}}
\newcommand\V{\mathbb{V}}
\newcommand\dmo{ ^{-1} }
\newcommand\kapa{\kappa}
\newcommand\im{Im}
\newcommand\hs{\mathcal{H}}





\usepackage{soul}

\makeatletter
\newcommand*{\whiten}[1]{\llap{\textcolor{white}{{\the\SOUL@token}}\hspace{#1pt}}}
\DeclareRobustCommand*\myul{%
    \def\SOUL@everyspace{\underline{\space}\kern\z@}%
    \def\SOUL@everytoken{%
     \setbox0=\hbox{\the\SOUL@token}%
     \ifdim\dp0>\z@
        \raisebox{\dp0}{\underline{\phantom{\the\SOUL@token}}}%
        \whiten{1}\whiten{0}%
        \whiten{-1}\whiten{-2}%
        \llap{\the\SOUL@token}%
     \else
        \underline{\the\SOUL@token}%
     \fi}%
\SOUL@}
\makeatother

\newcommand*{\demp}{\fontfamily{lmtt}\selectfont}

\DeclareTextFontCommand{\textdemp}{\demp}

\begin{document}

\ifcomment
Multiline
comment
\fi
\ifcomment
\myul{Typesetting test}
% \color[rgb]{1,1,1}
$∑_i^n≠ 60º±∞π∆¬≈√j∫h≤≥µ$

$\CR \R\pro\ind\pro\gS\pro
\mqty[a&b\\c&d]$
$\pro\mathbb{P}$
$\dd{x}$

  \[
    \alpha(x)=\left\{
                \begin{array}{ll}
                  x\\
                  \frac{1}{1+e^{-kx}}\\
                  \frac{e^x-e^{-x}}{e^x+e^{-x}}
                \end{array}
              \right.
  \]

  $\expval{x}$
  
  $\chi_\rho(ghg\dmo)=\Tr(\rho_{ghg\dmo})=\Tr(\rho_g\circ\rho_h\circ\rho\dmo_g)=\Tr(\rho_h)\overset{\mbox{\scalebox{0.5}{$\Tr(AB)=\Tr(BA)$}}}{=}\chi_\rho(h)$
  	$\mathop{\oplus}_{\substack{x\in X}}$

$\mat(\rho_g)=(a_{ij}(g))_{\scriptsize \substack{1\leq i\leq d \\ 1\leq j\leq d}}$ et $\mat(\rho'_g)=(a'_{ij}(g))_{\scriptsize \substack{1\leq i'\leq d' \\ 1\leq j'\leq d'}}$



\[\int_a^b{\mathbb{R}^2}g(u, v)\dd{P_{XY}}(u, v)=\iint g(u,v) f_{XY}(u, v)\dd \lambda(u) \dd \lambda(v)\]
$$\lim_{x\to\infty} f(x)$$	
$$\iiiint_V \mu(t,u,v,w) \,dt\,du\,dv\,dw$$
$$\sum_{n=1}^{\infty} 2^{-n} = 1$$	
\begin{definition}
	Si $X$ et $Y$ sont 2 v.a. ou definit la \textsc{Covariance} entre $X$ et $Y$ comme
	$\cov(X,Y)\overset{\text{def}}{=}\E\left[(X-\E(X))(Y-\E(Y))\right]=\E(XY)-\E(X)\E(Y)$.
\end{definition}
\fi
\pagebreak

% \tableofcontents

% insert your code here
%\input{./algebra/main.tex}
%\input{./geometrie-differentielle/main.tex}
%\input{./probabilite/main.tex}
%\input{./analyse-fonctionnelle/main.tex}
% \input{./Analyse-convexe-et-dualite-en-optimisation/main.tex}
%\input{./tikz/main.tex}
%\input{./Theorie-du-distributions/main.tex}
%\input{./optimisation/mine.tex}
 \input{./modelisation/main.tex}

% yves.aubry@univ-tln.fr : algebra

\end{document}

%% !TEX encoding = UTF-8 Unicode
% !TEX TS-program = xelatex

\documentclass[french]{report}

%\usepackage[utf8]{inputenc}
%\usepackage[T1]{fontenc}
\usepackage{babel}


\newif\ifcomment
%\commenttrue # Show comments

\usepackage{physics}
\usepackage{amssymb}


\usepackage{amsthm}
% \usepackage{thmtools}
\usepackage{mathtools}
\usepackage{amsfonts}

\usepackage{color}

\usepackage{tikz}

\usepackage{geometry}
\geometry{a5paper, margin=0.1in, right=1cm}

\usepackage{dsfont}

\usepackage{graphicx}
\graphicspath{ {images/} }

\usepackage{faktor}

\usepackage{IEEEtrantools}
\usepackage{enumerate}   
\usepackage[PostScript=dvips]{"/Users/aware/Documents/Courses/diagrams"}


\newtheorem{theorem}{Théorème}[section]
\renewcommand{\thetheorem}{\arabic{theorem}}
\newtheorem{lemme}{Lemme}[section]
\renewcommand{\thelemme}{\arabic{lemme}}
\newtheorem{proposition}{Proposition}[section]
\renewcommand{\theproposition}{\arabic{proposition}}
\newtheorem{notations}{Notations}[section]
\newtheorem{problem}{Problème}[section]
\newtheorem{corollary}{Corollaire}[theorem]
\renewcommand{\thecorollary}{\arabic{corollary}}
\newtheorem{property}{Propriété}[section]
\newtheorem{objective}{Objectif}[section]

\theoremstyle{definition}
\newtheorem{definition}{Définition}[section]
\renewcommand{\thedefinition}{\arabic{definition}}
\newtheorem{exercise}{Exercice}[chapter]
\renewcommand{\theexercise}{\arabic{exercise}}
\newtheorem{example}{Exemple}[chapter]
\renewcommand{\theexample}{\arabic{example}}
\newtheorem*{solution}{Solution}
\newtheorem*{application}{Application}
\newtheorem*{notation}{Notation}
\newtheorem*{vocabulary}{Vocabulaire}
\newtheorem*{properties}{Propriétés}



\theoremstyle{remark}
\newtheorem*{remark}{Remarque}
\newtheorem*{rappel}{Rappel}


\usepackage{etoolbox}
\AtBeginEnvironment{exercise}{\small}
\AtBeginEnvironment{example}{\small}

\usepackage{cases}
\usepackage[red]{mypack}

\usepackage[framemethod=TikZ]{mdframed}

\definecolor{bg}{rgb}{0.4,0.25,0.95}
\definecolor{pagebg}{rgb}{0,0,0.5}
\surroundwithmdframed[
   topline=false,
   rightline=false,
   bottomline=false,
   leftmargin=\parindent,
   skipabove=8pt,
   skipbelow=8pt,
   linecolor=blue,
   innerbottommargin=10pt,
   % backgroundcolor=bg,font=\color{orange}\sffamily, fontcolor=white
]{definition}

\usepackage{empheq}
\usepackage[most]{tcolorbox}

\newtcbox{\mymath}[1][]{%
    nobeforeafter, math upper, tcbox raise base,
    enhanced, colframe=blue!30!black,
    colback=red!10, boxrule=1pt,
    #1}

\usepackage{unixode}


\DeclareMathOperator{\ord}{ord}
\DeclareMathOperator{\orb}{orb}
\DeclareMathOperator{\stab}{stab}
\DeclareMathOperator{\Stab}{stab}
\DeclareMathOperator{\ppcm}{ppcm}
\DeclareMathOperator{\conj}{Conj}
\DeclareMathOperator{\End}{End}
\DeclareMathOperator{\rot}{rot}
\DeclareMathOperator{\trs}{trace}
\DeclareMathOperator{\Ind}{Ind}
\DeclareMathOperator{\mat}{Mat}
\DeclareMathOperator{\id}{Id}
\DeclareMathOperator{\vect}{vect}
\DeclareMathOperator{\img}{img}
\DeclareMathOperator{\cov}{Cov}
\DeclareMathOperator{\dist}{dist}
\DeclareMathOperator{\irr}{Irr}
\DeclareMathOperator{\image}{Im}
\DeclareMathOperator{\pd}{\partial}
\DeclareMathOperator{\epi}{epi}
\DeclareMathOperator{\Argmin}{Argmin}
\DeclareMathOperator{\dom}{dom}
\DeclareMathOperator{\proj}{proj}
\DeclareMathOperator{\ctg}{ctg}
\DeclareMathOperator{\supp}{supp}
\DeclareMathOperator{\argmin}{argmin}
\DeclareMathOperator{\mult}{mult}
\DeclareMathOperator{\ch}{ch}
\DeclareMathOperator{\sh}{sh}
\DeclareMathOperator{\rang}{rang}
\DeclareMathOperator{\diam}{diam}
\DeclareMathOperator{\Epigraphe}{Epigraphe}




\usepackage{xcolor}
\everymath{\color{blue}}
%\everymath{\color[rgb]{0,1,1}}
%\pagecolor[rgb]{0,0,0.5}


\newcommand*{\pdtest}[3][]{\ensuremath{\frac{\partial^{#1} #2}{\partial #3}}}

\newcommand*{\deffunc}[6][]{\ensuremath{
\begin{array}{rcl}
#2 : #3 &\rightarrow& #4\\
#5 &\mapsto& #6
\end{array}
}}

\newcommand{\eqcolon}{\mathrel{\resizebox{\widthof{$\mathord{=}$}}{\height}{ $\!\!=\!\!\resizebox{1.2\width}{0.8\height}{\raisebox{0.23ex}{$\mathop{:}$}}\!\!$ }}}
\newcommand{\coloneq}{\mathrel{\resizebox{\widthof{$\mathord{=}$}}{\height}{ $\!\!\resizebox{1.2\width}{0.8\height}{\raisebox{0.23ex}{$\mathop{:}$}}\!\!=\!\!$ }}}
\newcommand{\eqcolonl}{\ensuremath{\mathrel{=\!\!\mathop{:}}}}
\newcommand{\coloneql}{\ensuremath{\mathrel{\mathop{:} \!\! =}}}
\newcommand{\vc}[1]{% inline column vector
  \left(\begin{smallmatrix}#1\end{smallmatrix}\right)%
}
\newcommand{\vr}[1]{% inline row vector
  \begin{smallmatrix}(\,#1\,)\end{smallmatrix}%
}
\makeatletter
\newcommand*{\defeq}{\ =\mathrel{\rlap{%
                     \raisebox{0.3ex}{$\m@th\cdot$}}%
                     \raisebox{-0.3ex}{$\m@th\cdot$}}%
                     }
\makeatother

\newcommand{\mathcircle}[1]{% inline row vector
 \overset{\circ}{#1}
}
\newcommand{\ulim}{% low limit
 \underline{\lim}
}
\newcommand{\ssi}{% iff
\iff
}
\newcommand{\ps}[2]{
\expval{#1 | #2}
}
\newcommand{\df}[1]{
\mqty{#1}
}
\newcommand{\n}[1]{
\norm{#1}
}
\newcommand{\sys}[1]{
\left\{\smqty{#1}\right.
}


\newcommand{\eqdef}{\ensuremath{\overset{\text{def}}=}}


\def\Circlearrowright{\ensuremath{%
  \rotatebox[origin=c]{230}{$\circlearrowright$}}}

\newcommand\ct[1]{\text{\rmfamily\upshape #1}}
\newcommand\question[1]{ {\color{red} ...!? \small #1}}
\newcommand\caz[1]{\left\{\begin{array} #1 \end{array}\right.}
\newcommand\const{\text{\rmfamily\upshape const}}
\newcommand\toP{ \overset{\pro}{\to}}
\newcommand\toPP{ \overset{\text{PP}}{\to}}
\newcommand{\oeq}{\mathrel{\text{\textcircled{$=$}}}}





\usepackage{xcolor}
% \usepackage[normalem]{ulem}
\usepackage{lipsum}
\makeatletter
% \newcommand\colorwave[1][blue]{\bgroup \markoverwith{\lower3.5\p@\hbox{\sixly \textcolor{#1}{\char58}}}\ULon}
%\font\sixly=lasy6 % does not re-load if already loaded, so no memory problem.

\newmdtheoremenv[
linewidth= 1pt,linecolor= blue,%
leftmargin=20,rightmargin=20,innertopmargin=0pt, innerrightmargin=40,%
tikzsetting = { draw=lightgray, line width = 0.3pt,dashed,%
dash pattern = on 15pt off 3pt},%
splittopskip=\topskip,skipbelow=\baselineskip,%
skipabove=\baselineskip,ntheorem,roundcorner=0pt,
% backgroundcolor=pagebg,font=\color{orange}\sffamily, fontcolor=white
]{examplebox}{Exemple}[section]



\newcommand\R{\mathbb{R}}
\newcommand\Z{\mathbb{Z}}
\newcommand\N{\mathbb{N}}
\newcommand\E{\mathbb{E}}
\newcommand\F{\mathcal{F}}
\newcommand\cH{\mathcal{H}}
\newcommand\V{\mathbb{V}}
\newcommand\dmo{ ^{-1} }
\newcommand\kapa{\kappa}
\newcommand\im{Im}
\newcommand\hs{\mathcal{H}}





\usepackage{soul}

\makeatletter
\newcommand*{\whiten}[1]{\llap{\textcolor{white}{{\the\SOUL@token}}\hspace{#1pt}}}
\DeclareRobustCommand*\myul{%
    \def\SOUL@everyspace{\underline{\space}\kern\z@}%
    \def\SOUL@everytoken{%
     \setbox0=\hbox{\the\SOUL@token}%
     \ifdim\dp0>\z@
        \raisebox{\dp0}{\underline{\phantom{\the\SOUL@token}}}%
        \whiten{1}\whiten{0}%
        \whiten{-1}\whiten{-2}%
        \llap{\the\SOUL@token}%
     \else
        \underline{\the\SOUL@token}%
     \fi}%
\SOUL@}
\makeatother

\newcommand*{\demp}{\fontfamily{lmtt}\selectfont}

\DeclareTextFontCommand{\textdemp}{\demp}

\begin{document}

\ifcomment
Multiline
comment
\fi
\ifcomment
\myul{Typesetting test}
% \color[rgb]{1,1,1}
$∑_i^n≠ 60º±∞π∆¬≈√j∫h≤≥µ$

$\CR \R\pro\ind\pro\gS\pro
\mqty[a&b\\c&d]$
$\pro\mathbb{P}$
$\dd{x}$

  \[
    \alpha(x)=\left\{
                \begin{array}{ll}
                  x\\
                  \frac{1}{1+e^{-kx}}\\
                  \frac{e^x-e^{-x}}{e^x+e^{-x}}
                \end{array}
              \right.
  \]

  $\expval{x}$
  
  $\chi_\rho(ghg\dmo)=\Tr(\rho_{ghg\dmo})=\Tr(\rho_g\circ\rho_h\circ\rho\dmo_g)=\Tr(\rho_h)\overset{\mbox{\scalebox{0.5}{$\Tr(AB)=\Tr(BA)$}}}{=}\chi_\rho(h)$
  	$\mathop{\oplus}_{\substack{x\in X}}$

$\mat(\rho_g)=(a_{ij}(g))_{\scriptsize \substack{1\leq i\leq d \\ 1\leq j\leq d}}$ et $\mat(\rho'_g)=(a'_{ij}(g))_{\scriptsize \substack{1\leq i'\leq d' \\ 1\leq j'\leq d'}}$



\[\int_a^b{\mathbb{R}^2}g(u, v)\dd{P_{XY}}(u, v)=\iint g(u,v) f_{XY}(u, v)\dd \lambda(u) \dd \lambda(v)\]
$$\lim_{x\to\infty} f(x)$$	
$$\iiiint_V \mu(t,u,v,w) \,dt\,du\,dv\,dw$$
$$\sum_{n=1}^{\infty} 2^{-n} = 1$$	
\begin{definition}
	Si $X$ et $Y$ sont 2 v.a. ou definit la \textsc{Covariance} entre $X$ et $Y$ comme
	$\cov(X,Y)\overset{\text{def}}{=}\E\left[(X-\E(X))(Y-\E(Y))\right]=\E(XY)-\E(X)\E(Y)$.
\end{definition}
\fi
\pagebreak

% \tableofcontents

% insert your code here
%\input{./algebra/main.tex}
%\input{./geometrie-differentielle/main.tex}
%\input{./probabilite/main.tex}
%\input{./analyse-fonctionnelle/main.tex}
% \input{./Analyse-convexe-et-dualite-en-optimisation/main.tex}
%\input{./tikz/main.tex}
%\input{./Theorie-du-distributions/main.tex}
%\input{./optimisation/mine.tex}
 \input{./modelisation/main.tex}

% yves.aubry@univ-tln.fr : algebra

\end{document}

%% !TEX encoding = UTF-8 Unicode
% !TEX TS-program = xelatex

\documentclass[french]{report}

%\usepackage[utf8]{inputenc}
%\usepackage[T1]{fontenc}
\usepackage{babel}


\newif\ifcomment
%\commenttrue # Show comments

\usepackage{physics}
\usepackage{amssymb}


\usepackage{amsthm}
% \usepackage{thmtools}
\usepackage{mathtools}
\usepackage{amsfonts}

\usepackage{color}

\usepackage{tikz}

\usepackage{geometry}
\geometry{a5paper, margin=0.1in, right=1cm}

\usepackage{dsfont}

\usepackage{graphicx}
\graphicspath{ {images/} }

\usepackage{faktor}

\usepackage{IEEEtrantools}
\usepackage{enumerate}   
\usepackage[PostScript=dvips]{"/Users/aware/Documents/Courses/diagrams"}


\newtheorem{theorem}{Théorème}[section]
\renewcommand{\thetheorem}{\arabic{theorem}}
\newtheorem{lemme}{Lemme}[section]
\renewcommand{\thelemme}{\arabic{lemme}}
\newtheorem{proposition}{Proposition}[section]
\renewcommand{\theproposition}{\arabic{proposition}}
\newtheorem{notations}{Notations}[section]
\newtheorem{problem}{Problème}[section]
\newtheorem{corollary}{Corollaire}[theorem]
\renewcommand{\thecorollary}{\arabic{corollary}}
\newtheorem{property}{Propriété}[section]
\newtheorem{objective}{Objectif}[section]

\theoremstyle{definition}
\newtheorem{definition}{Définition}[section]
\renewcommand{\thedefinition}{\arabic{definition}}
\newtheorem{exercise}{Exercice}[chapter]
\renewcommand{\theexercise}{\arabic{exercise}}
\newtheorem{example}{Exemple}[chapter]
\renewcommand{\theexample}{\arabic{example}}
\newtheorem*{solution}{Solution}
\newtheorem*{application}{Application}
\newtheorem*{notation}{Notation}
\newtheorem*{vocabulary}{Vocabulaire}
\newtheorem*{properties}{Propriétés}



\theoremstyle{remark}
\newtheorem*{remark}{Remarque}
\newtheorem*{rappel}{Rappel}


\usepackage{etoolbox}
\AtBeginEnvironment{exercise}{\small}
\AtBeginEnvironment{example}{\small}

\usepackage{cases}
\usepackage[red]{mypack}

\usepackage[framemethod=TikZ]{mdframed}

\definecolor{bg}{rgb}{0.4,0.25,0.95}
\definecolor{pagebg}{rgb}{0,0,0.5}
\surroundwithmdframed[
   topline=false,
   rightline=false,
   bottomline=false,
   leftmargin=\parindent,
   skipabove=8pt,
   skipbelow=8pt,
   linecolor=blue,
   innerbottommargin=10pt,
   % backgroundcolor=bg,font=\color{orange}\sffamily, fontcolor=white
]{definition}

\usepackage{empheq}
\usepackage[most]{tcolorbox}

\newtcbox{\mymath}[1][]{%
    nobeforeafter, math upper, tcbox raise base,
    enhanced, colframe=blue!30!black,
    colback=red!10, boxrule=1pt,
    #1}

\usepackage{unixode}


\DeclareMathOperator{\ord}{ord}
\DeclareMathOperator{\orb}{orb}
\DeclareMathOperator{\stab}{stab}
\DeclareMathOperator{\Stab}{stab}
\DeclareMathOperator{\ppcm}{ppcm}
\DeclareMathOperator{\conj}{Conj}
\DeclareMathOperator{\End}{End}
\DeclareMathOperator{\rot}{rot}
\DeclareMathOperator{\trs}{trace}
\DeclareMathOperator{\Ind}{Ind}
\DeclareMathOperator{\mat}{Mat}
\DeclareMathOperator{\id}{Id}
\DeclareMathOperator{\vect}{vect}
\DeclareMathOperator{\img}{img}
\DeclareMathOperator{\cov}{Cov}
\DeclareMathOperator{\dist}{dist}
\DeclareMathOperator{\irr}{Irr}
\DeclareMathOperator{\image}{Im}
\DeclareMathOperator{\pd}{\partial}
\DeclareMathOperator{\epi}{epi}
\DeclareMathOperator{\Argmin}{Argmin}
\DeclareMathOperator{\dom}{dom}
\DeclareMathOperator{\proj}{proj}
\DeclareMathOperator{\ctg}{ctg}
\DeclareMathOperator{\supp}{supp}
\DeclareMathOperator{\argmin}{argmin}
\DeclareMathOperator{\mult}{mult}
\DeclareMathOperator{\ch}{ch}
\DeclareMathOperator{\sh}{sh}
\DeclareMathOperator{\rang}{rang}
\DeclareMathOperator{\diam}{diam}
\DeclareMathOperator{\Epigraphe}{Epigraphe}




\usepackage{xcolor}
\everymath{\color{blue}}
%\everymath{\color[rgb]{0,1,1}}
%\pagecolor[rgb]{0,0,0.5}


\newcommand*{\pdtest}[3][]{\ensuremath{\frac{\partial^{#1} #2}{\partial #3}}}

\newcommand*{\deffunc}[6][]{\ensuremath{
\begin{array}{rcl}
#2 : #3 &\rightarrow& #4\\
#5 &\mapsto& #6
\end{array}
}}

\newcommand{\eqcolon}{\mathrel{\resizebox{\widthof{$\mathord{=}$}}{\height}{ $\!\!=\!\!\resizebox{1.2\width}{0.8\height}{\raisebox{0.23ex}{$\mathop{:}$}}\!\!$ }}}
\newcommand{\coloneq}{\mathrel{\resizebox{\widthof{$\mathord{=}$}}{\height}{ $\!\!\resizebox{1.2\width}{0.8\height}{\raisebox{0.23ex}{$\mathop{:}$}}\!\!=\!\!$ }}}
\newcommand{\eqcolonl}{\ensuremath{\mathrel{=\!\!\mathop{:}}}}
\newcommand{\coloneql}{\ensuremath{\mathrel{\mathop{:} \!\! =}}}
\newcommand{\vc}[1]{% inline column vector
  \left(\begin{smallmatrix}#1\end{smallmatrix}\right)%
}
\newcommand{\vr}[1]{% inline row vector
  \begin{smallmatrix}(\,#1\,)\end{smallmatrix}%
}
\makeatletter
\newcommand*{\defeq}{\ =\mathrel{\rlap{%
                     \raisebox{0.3ex}{$\m@th\cdot$}}%
                     \raisebox{-0.3ex}{$\m@th\cdot$}}%
                     }
\makeatother

\newcommand{\mathcircle}[1]{% inline row vector
 \overset{\circ}{#1}
}
\newcommand{\ulim}{% low limit
 \underline{\lim}
}
\newcommand{\ssi}{% iff
\iff
}
\newcommand{\ps}[2]{
\expval{#1 | #2}
}
\newcommand{\df}[1]{
\mqty{#1}
}
\newcommand{\n}[1]{
\norm{#1}
}
\newcommand{\sys}[1]{
\left\{\smqty{#1}\right.
}


\newcommand{\eqdef}{\ensuremath{\overset{\text{def}}=}}


\def\Circlearrowright{\ensuremath{%
  \rotatebox[origin=c]{230}{$\circlearrowright$}}}

\newcommand\ct[1]{\text{\rmfamily\upshape #1}}
\newcommand\question[1]{ {\color{red} ...!? \small #1}}
\newcommand\caz[1]{\left\{\begin{array} #1 \end{array}\right.}
\newcommand\const{\text{\rmfamily\upshape const}}
\newcommand\toP{ \overset{\pro}{\to}}
\newcommand\toPP{ \overset{\text{PP}}{\to}}
\newcommand{\oeq}{\mathrel{\text{\textcircled{$=$}}}}





\usepackage{xcolor}
% \usepackage[normalem]{ulem}
\usepackage{lipsum}
\makeatletter
% \newcommand\colorwave[1][blue]{\bgroup \markoverwith{\lower3.5\p@\hbox{\sixly \textcolor{#1}{\char58}}}\ULon}
%\font\sixly=lasy6 % does not re-load if already loaded, so no memory problem.

\newmdtheoremenv[
linewidth= 1pt,linecolor= blue,%
leftmargin=20,rightmargin=20,innertopmargin=0pt, innerrightmargin=40,%
tikzsetting = { draw=lightgray, line width = 0.3pt,dashed,%
dash pattern = on 15pt off 3pt},%
splittopskip=\topskip,skipbelow=\baselineskip,%
skipabove=\baselineskip,ntheorem,roundcorner=0pt,
% backgroundcolor=pagebg,font=\color{orange}\sffamily, fontcolor=white
]{examplebox}{Exemple}[section]



\newcommand\R{\mathbb{R}}
\newcommand\Z{\mathbb{Z}}
\newcommand\N{\mathbb{N}}
\newcommand\E{\mathbb{E}}
\newcommand\F{\mathcal{F}}
\newcommand\cH{\mathcal{H}}
\newcommand\V{\mathbb{V}}
\newcommand\dmo{ ^{-1} }
\newcommand\kapa{\kappa}
\newcommand\im{Im}
\newcommand\hs{\mathcal{H}}





\usepackage{soul}

\makeatletter
\newcommand*{\whiten}[1]{\llap{\textcolor{white}{{\the\SOUL@token}}\hspace{#1pt}}}
\DeclareRobustCommand*\myul{%
    \def\SOUL@everyspace{\underline{\space}\kern\z@}%
    \def\SOUL@everytoken{%
     \setbox0=\hbox{\the\SOUL@token}%
     \ifdim\dp0>\z@
        \raisebox{\dp0}{\underline{\phantom{\the\SOUL@token}}}%
        \whiten{1}\whiten{0}%
        \whiten{-1}\whiten{-2}%
        \llap{\the\SOUL@token}%
     \else
        \underline{\the\SOUL@token}%
     \fi}%
\SOUL@}
\makeatother

\newcommand*{\demp}{\fontfamily{lmtt}\selectfont}

\DeclareTextFontCommand{\textdemp}{\demp}

\begin{document}

\ifcomment
Multiline
comment
\fi
\ifcomment
\myul{Typesetting test}
% \color[rgb]{1,1,1}
$∑_i^n≠ 60º±∞π∆¬≈√j∫h≤≥µ$

$\CR \R\pro\ind\pro\gS\pro
\mqty[a&b\\c&d]$
$\pro\mathbb{P}$
$\dd{x}$

  \[
    \alpha(x)=\left\{
                \begin{array}{ll}
                  x\\
                  \frac{1}{1+e^{-kx}}\\
                  \frac{e^x-e^{-x}}{e^x+e^{-x}}
                \end{array}
              \right.
  \]

  $\expval{x}$
  
  $\chi_\rho(ghg\dmo)=\Tr(\rho_{ghg\dmo})=\Tr(\rho_g\circ\rho_h\circ\rho\dmo_g)=\Tr(\rho_h)\overset{\mbox{\scalebox{0.5}{$\Tr(AB)=\Tr(BA)$}}}{=}\chi_\rho(h)$
  	$\mathop{\oplus}_{\substack{x\in X}}$

$\mat(\rho_g)=(a_{ij}(g))_{\scriptsize \substack{1\leq i\leq d \\ 1\leq j\leq d}}$ et $\mat(\rho'_g)=(a'_{ij}(g))_{\scriptsize \substack{1\leq i'\leq d' \\ 1\leq j'\leq d'}}$



\[\int_a^b{\mathbb{R}^2}g(u, v)\dd{P_{XY}}(u, v)=\iint g(u,v) f_{XY}(u, v)\dd \lambda(u) \dd \lambda(v)\]
$$\lim_{x\to\infty} f(x)$$	
$$\iiiint_V \mu(t,u,v,w) \,dt\,du\,dv\,dw$$
$$\sum_{n=1}^{\infty} 2^{-n} = 1$$	
\begin{definition}
	Si $X$ et $Y$ sont 2 v.a. ou definit la \textsc{Covariance} entre $X$ et $Y$ comme
	$\cov(X,Y)\overset{\text{def}}{=}\E\left[(X-\E(X))(Y-\E(Y))\right]=\E(XY)-\E(X)\E(Y)$.
\end{definition}
\fi
\pagebreak

% \tableofcontents

% insert your code here
%\input{./algebra/main.tex}
%\input{./geometrie-differentielle/main.tex}
%\input{./probabilite/main.tex}
%\input{./analyse-fonctionnelle/main.tex}
% \input{./Analyse-convexe-et-dualite-en-optimisation/main.tex}
%\input{./tikz/main.tex}
%\input{./Theorie-du-distributions/main.tex}
%\input{./optimisation/mine.tex}
 \input{./modelisation/main.tex}

% yves.aubry@univ-tln.fr : algebra

\end{document}

% % !TEX encoding = UTF-8 Unicode
% !TEX TS-program = xelatex

\documentclass[french]{report}

%\usepackage[utf8]{inputenc}
%\usepackage[T1]{fontenc}
\usepackage{babel}


\newif\ifcomment
%\commenttrue # Show comments

\usepackage{physics}
\usepackage{amssymb}


\usepackage{amsthm}
% \usepackage{thmtools}
\usepackage{mathtools}
\usepackage{amsfonts}

\usepackage{color}

\usepackage{tikz}

\usepackage{geometry}
\geometry{a5paper, margin=0.1in, right=1cm}

\usepackage{dsfont}

\usepackage{graphicx}
\graphicspath{ {images/} }

\usepackage{faktor}

\usepackage{IEEEtrantools}
\usepackage{enumerate}   
\usepackage[PostScript=dvips]{"/Users/aware/Documents/Courses/diagrams"}


\newtheorem{theorem}{Théorème}[section]
\renewcommand{\thetheorem}{\arabic{theorem}}
\newtheorem{lemme}{Lemme}[section]
\renewcommand{\thelemme}{\arabic{lemme}}
\newtheorem{proposition}{Proposition}[section]
\renewcommand{\theproposition}{\arabic{proposition}}
\newtheorem{notations}{Notations}[section]
\newtheorem{problem}{Problème}[section]
\newtheorem{corollary}{Corollaire}[theorem]
\renewcommand{\thecorollary}{\arabic{corollary}}
\newtheorem{property}{Propriété}[section]
\newtheorem{objective}{Objectif}[section]

\theoremstyle{definition}
\newtheorem{definition}{Définition}[section]
\renewcommand{\thedefinition}{\arabic{definition}}
\newtheorem{exercise}{Exercice}[chapter]
\renewcommand{\theexercise}{\arabic{exercise}}
\newtheorem{example}{Exemple}[chapter]
\renewcommand{\theexample}{\arabic{example}}
\newtheorem*{solution}{Solution}
\newtheorem*{application}{Application}
\newtheorem*{notation}{Notation}
\newtheorem*{vocabulary}{Vocabulaire}
\newtheorem*{properties}{Propriétés}



\theoremstyle{remark}
\newtheorem*{remark}{Remarque}
\newtheorem*{rappel}{Rappel}


\usepackage{etoolbox}
\AtBeginEnvironment{exercise}{\small}
\AtBeginEnvironment{example}{\small}

\usepackage{cases}
\usepackage[red]{mypack}

\usepackage[framemethod=TikZ]{mdframed}

\definecolor{bg}{rgb}{0.4,0.25,0.95}
\definecolor{pagebg}{rgb}{0,0,0.5}
\surroundwithmdframed[
   topline=false,
   rightline=false,
   bottomline=false,
   leftmargin=\parindent,
   skipabove=8pt,
   skipbelow=8pt,
   linecolor=blue,
   innerbottommargin=10pt,
   % backgroundcolor=bg,font=\color{orange}\sffamily, fontcolor=white
]{definition}

\usepackage{empheq}
\usepackage[most]{tcolorbox}

\newtcbox{\mymath}[1][]{%
    nobeforeafter, math upper, tcbox raise base,
    enhanced, colframe=blue!30!black,
    colback=red!10, boxrule=1pt,
    #1}

\usepackage{unixode}


\DeclareMathOperator{\ord}{ord}
\DeclareMathOperator{\orb}{orb}
\DeclareMathOperator{\stab}{stab}
\DeclareMathOperator{\Stab}{stab}
\DeclareMathOperator{\ppcm}{ppcm}
\DeclareMathOperator{\conj}{Conj}
\DeclareMathOperator{\End}{End}
\DeclareMathOperator{\rot}{rot}
\DeclareMathOperator{\trs}{trace}
\DeclareMathOperator{\Ind}{Ind}
\DeclareMathOperator{\mat}{Mat}
\DeclareMathOperator{\id}{Id}
\DeclareMathOperator{\vect}{vect}
\DeclareMathOperator{\img}{img}
\DeclareMathOperator{\cov}{Cov}
\DeclareMathOperator{\dist}{dist}
\DeclareMathOperator{\irr}{Irr}
\DeclareMathOperator{\image}{Im}
\DeclareMathOperator{\pd}{\partial}
\DeclareMathOperator{\epi}{epi}
\DeclareMathOperator{\Argmin}{Argmin}
\DeclareMathOperator{\dom}{dom}
\DeclareMathOperator{\proj}{proj}
\DeclareMathOperator{\ctg}{ctg}
\DeclareMathOperator{\supp}{supp}
\DeclareMathOperator{\argmin}{argmin}
\DeclareMathOperator{\mult}{mult}
\DeclareMathOperator{\ch}{ch}
\DeclareMathOperator{\sh}{sh}
\DeclareMathOperator{\rang}{rang}
\DeclareMathOperator{\diam}{diam}
\DeclareMathOperator{\Epigraphe}{Epigraphe}




\usepackage{xcolor}
\everymath{\color{blue}}
%\everymath{\color[rgb]{0,1,1}}
%\pagecolor[rgb]{0,0,0.5}


\newcommand*{\pdtest}[3][]{\ensuremath{\frac{\partial^{#1} #2}{\partial #3}}}

\newcommand*{\deffunc}[6][]{\ensuremath{
\begin{array}{rcl}
#2 : #3 &\rightarrow& #4\\
#5 &\mapsto& #6
\end{array}
}}

\newcommand{\eqcolon}{\mathrel{\resizebox{\widthof{$\mathord{=}$}}{\height}{ $\!\!=\!\!\resizebox{1.2\width}{0.8\height}{\raisebox{0.23ex}{$\mathop{:}$}}\!\!$ }}}
\newcommand{\coloneq}{\mathrel{\resizebox{\widthof{$\mathord{=}$}}{\height}{ $\!\!\resizebox{1.2\width}{0.8\height}{\raisebox{0.23ex}{$\mathop{:}$}}\!\!=\!\!$ }}}
\newcommand{\eqcolonl}{\ensuremath{\mathrel{=\!\!\mathop{:}}}}
\newcommand{\coloneql}{\ensuremath{\mathrel{\mathop{:} \!\! =}}}
\newcommand{\vc}[1]{% inline column vector
  \left(\begin{smallmatrix}#1\end{smallmatrix}\right)%
}
\newcommand{\vr}[1]{% inline row vector
  \begin{smallmatrix}(\,#1\,)\end{smallmatrix}%
}
\makeatletter
\newcommand*{\defeq}{\ =\mathrel{\rlap{%
                     \raisebox{0.3ex}{$\m@th\cdot$}}%
                     \raisebox{-0.3ex}{$\m@th\cdot$}}%
                     }
\makeatother

\newcommand{\mathcircle}[1]{% inline row vector
 \overset{\circ}{#1}
}
\newcommand{\ulim}{% low limit
 \underline{\lim}
}
\newcommand{\ssi}{% iff
\iff
}
\newcommand{\ps}[2]{
\expval{#1 | #2}
}
\newcommand{\df}[1]{
\mqty{#1}
}
\newcommand{\n}[1]{
\norm{#1}
}
\newcommand{\sys}[1]{
\left\{\smqty{#1}\right.
}


\newcommand{\eqdef}{\ensuremath{\overset{\text{def}}=}}


\def\Circlearrowright{\ensuremath{%
  \rotatebox[origin=c]{230}{$\circlearrowright$}}}

\newcommand\ct[1]{\text{\rmfamily\upshape #1}}
\newcommand\question[1]{ {\color{red} ...!? \small #1}}
\newcommand\caz[1]{\left\{\begin{array} #1 \end{array}\right.}
\newcommand\const{\text{\rmfamily\upshape const}}
\newcommand\toP{ \overset{\pro}{\to}}
\newcommand\toPP{ \overset{\text{PP}}{\to}}
\newcommand{\oeq}{\mathrel{\text{\textcircled{$=$}}}}





\usepackage{xcolor}
% \usepackage[normalem]{ulem}
\usepackage{lipsum}
\makeatletter
% \newcommand\colorwave[1][blue]{\bgroup \markoverwith{\lower3.5\p@\hbox{\sixly \textcolor{#1}{\char58}}}\ULon}
%\font\sixly=lasy6 % does not re-load if already loaded, so no memory problem.

\newmdtheoremenv[
linewidth= 1pt,linecolor= blue,%
leftmargin=20,rightmargin=20,innertopmargin=0pt, innerrightmargin=40,%
tikzsetting = { draw=lightgray, line width = 0.3pt,dashed,%
dash pattern = on 15pt off 3pt},%
splittopskip=\topskip,skipbelow=\baselineskip,%
skipabove=\baselineskip,ntheorem,roundcorner=0pt,
% backgroundcolor=pagebg,font=\color{orange}\sffamily, fontcolor=white
]{examplebox}{Exemple}[section]



\newcommand\R{\mathbb{R}}
\newcommand\Z{\mathbb{Z}}
\newcommand\N{\mathbb{N}}
\newcommand\E{\mathbb{E}}
\newcommand\F{\mathcal{F}}
\newcommand\cH{\mathcal{H}}
\newcommand\V{\mathbb{V}}
\newcommand\dmo{ ^{-1} }
\newcommand\kapa{\kappa}
\newcommand\im{Im}
\newcommand\hs{\mathcal{H}}





\usepackage{soul}

\makeatletter
\newcommand*{\whiten}[1]{\llap{\textcolor{white}{{\the\SOUL@token}}\hspace{#1pt}}}
\DeclareRobustCommand*\myul{%
    \def\SOUL@everyspace{\underline{\space}\kern\z@}%
    \def\SOUL@everytoken{%
     \setbox0=\hbox{\the\SOUL@token}%
     \ifdim\dp0>\z@
        \raisebox{\dp0}{\underline{\phantom{\the\SOUL@token}}}%
        \whiten{1}\whiten{0}%
        \whiten{-1}\whiten{-2}%
        \llap{\the\SOUL@token}%
     \else
        \underline{\the\SOUL@token}%
     \fi}%
\SOUL@}
\makeatother

\newcommand*{\demp}{\fontfamily{lmtt}\selectfont}

\DeclareTextFontCommand{\textdemp}{\demp}

\begin{document}

\ifcomment
Multiline
comment
\fi
\ifcomment
\myul{Typesetting test}
% \color[rgb]{1,1,1}
$∑_i^n≠ 60º±∞π∆¬≈√j∫h≤≥µ$

$\CR \R\pro\ind\pro\gS\pro
\mqty[a&b\\c&d]$
$\pro\mathbb{P}$
$\dd{x}$

  \[
    \alpha(x)=\left\{
                \begin{array}{ll}
                  x\\
                  \frac{1}{1+e^{-kx}}\\
                  \frac{e^x-e^{-x}}{e^x+e^{-x}}
                \end{array}
              \right.
  \]

  $\expval{x}$
  
  $\chi_\rho(ghg\dmo)=\Tr(\rho_{ghg\dmo})=\Tr(\rho_g\circ\rho_h\circ\rho\dmo_g)=\Tr(\rho_h)\overset{\mbox{\scalebox{0.5}{$\Tr(AB)=\Tr(BA)$}}}{=}\chi_\rho(h)$
  	$\mathop{\oplus}_{\substack{x\in X}}$

$\mat(\rho_g)=(a_{ij}(g))_{\scriptsize \substack{1\leq i\leq d \\ 1\leq j\leq d}}$ et $\mat(\rho'_g)=(a'_{ij}(g))_{\scriptsize \substack{1\leq i'\leq d' \\ 1\leq j'\leq d'}}$



\[\int_a^b{\mathbb{R}^2}g(u, v)\dd{P_{XY}}(u, v)=\iint g(u,v) f_{XY}(u, v)\dd \lambda(u) \dd \lambda(v)\]
$$\lim_{x\to\infty} f(x)$$	
$$\iiiint_V \mu(t,u,v,w) \,dt\,du\,dv\,dw$$
$$\sum_{n=1}^{\infty} 2^{-n} = 1$$	
\begin{definition}
	Si $X$ et $Y$ sont 2 v.a. ou definit la \textsc{Covariance} entre $X$ et $Y$ comme
	$\cov(X,Y)\overset{\text{def}}{=}\E\left[(X-\E(X))(Y-\E(Y))\right]=\E(XY)-\E(X)\E(Y)$.
\end{definition}
\fi
\pagebreak

% \tableofcontents

% insert your code here
%\input{./algebra/main.tex}
%\input{./geometrie-differentielle/main.tex}
%\input{./probabilite/main.tex}
%\input{./analyse-fonctionnelle/main.tex}
% \input{./Analyse-convexe-et-dualite-en-optimisation/main.tex}
%\input{./tikz/main.tex}
%\input{./Theorie-du-distributions/main.tex}
%\input{./optimisation/mine.tex}
 \input{./modelisation/main.tex}

% yves.aubry@univ-tln.fr : algebra

\end{document}

%% !TEX encoding = UTF-8 Unicode
% !TEX TS-program = xelatex

\documentclass[french]{report}

%\usepackage[utf8]{inputenc}
%\usepackage[T1]{fontenc}
\usepackage{babel}


\newif\ifcomment
%\commenttrue # Show comments

\usepackage{physics}
\usepackage{amssymb}


\usepackage{amsthm}
% \usepackage{thmtools}
\usepackage{mathtools}
\usepackage{amsfonts}

\usepackage{color}

\usepackage{tikz}

\usepackage{geometry}
\geometry{a5paper, margin=0.1in, right=1cm}

\usepackage{dsfont}

\usepackage{graphicx}
\graphicspath{ {images/} }

\usepackage{faktor}

\usepackage{IEEEtrantools}
\usepackage{enumerate}   
\usepackage[PostScript=dvips]{"/Users/aware/Documents/Courses/diagrams"}


\newtheorem{theorem}{Théorème}[section]
\renewcommand{\thetheorem}{\arabic{theorem}}
\newtheorem{lemme}{Lemme}[section]
\renewcommand{\thelemme}{\arabic{lemme}}
\newtheorem{proposition}{Proposition}[section]
\renewcommand{\theproposition}{\arabic{proposition}}
\newtheorem{notations}{Notations}[section]
\newtheorem{problem}{Problème}[section]
\newtheorem{corollary}{Corollaire}[theorem]
\renewcommand{\thecorollary}{\arabic{corollary}}
\newtheorem{property}{Propriété}[section]
\newtheorem{objective}{Objectif}[section]

\theoremstyle{definition}
\newtheorem{definition}{Définition}[section]
\renewcommand{\thedefinition}{\arabic{definition}}
\newtheorem{exercise}{Exercice}[chapter]
\renewcommand{\theexercise}{\arabic{exercise}}
\newtheorem{example}{Exemple}[chapter]
\renewcommand{\theexample}{\arabic{example}}
\newtheorem*{solution}{Solution}
\newtheorem*{application}{Application}
\newtheorem*{notation}{Notation}
\newtheorem*{vocabulary}{Vocabulaire}
\newtheorem*{properties}{Propriétés}



\theoremstyle{remark}
\newtheorem*{remark}{Remarque}
\newtheorem*{rappel}{Rappel}


\usepackage{etoolbox}
\AtBeginEnvironment{exercise}{\small}
\AtBeginEnvironment{example}{\small}

\usepackage{cases}
\usepackage[red]{mypack}

\usepackage[framemethod=TikZ]{mdframed}

\definecolor{bg}{rgb}{0.4,0.25,0.95}
\definecolor{pagebg}{rgb}{0,0,0.5}
\surroundwithmdframed[
   topline=false,
   rightline=false,
   bottomline=false,
   leftmargin=\parindent,
   skipabove=8pt,
   skipbelow=8pt,
   linecolor=blue,
   innerbottommargin=10pt,
   % backgroundcolor=bg,font=\color{orange}\sffamily, fontcolor=white
]{definition}

\usepackage{empheq}
\usepackage[most]{tcolorbox}

\newtcbox{\mymath}[1][]{%
    nobeforeafter, math upper, tcbox raise base,
    enhanced, colframe=blue!30!black,
    colback=red!10, boxrule=1pt,
    #1}

\usepackage{unixode}


\DeclareMathOperator{\ord}{ord}
\DeclareMathOperator{\orb}{orb}
\DeclareMathOperator{\stab}{stab}
\DeclareMathOperator{\Stab}{stab}
\DeclareMathOperator{\ppcm}{ppcm}
\DeclareMathOperator{\conj}{Conj}
\DeclareMathOperator{\End}{End}
\DeclareMathOperator{\rot}{rot}
\DeclareMathOperator{\trs}{trace}
\DeclareMathOperator{\Ind}{Ind}
\DeclareMathOperator{\mat}{Mat}
\DeclareMathOperator{\id}{Id}
\DeclareMathOperator{\vect}{vect}
\DeclareMathOperator{\img}{img}
\DeclareMathOperator{\cov}{Cov}
\DeclareMathOperator{\dist}{dist}
\DeclareMathOperator{\irr}{Irr}
\DeclareMathOperator{\image}{Im}
\DeclareMathOperator{\pd}{\partial}
\DeclareMathOperator{\epi}{epi}
\DeclareMathOperator{\Argmin}{Argmin}
\DeclareMathOperator{\dom}{dom}
\DeclareMathOperator{\proj}{proj}
\DeclareMathOperator{\ctg}{ctg}
\DeclareMathOperator{\supp}{supp}
\DeclareMathOperator{\argmin}{argmin}
\DeclareMathOperator{\mult}{mult}
\DeclareMathOperator{\ch}{ch}
\DeclareMathOperator{\sh}{sh}
\DeclareMathOperator{\rang}{rang}
\DeclareMathOperator{\diam}{diam}
\DeclareMathOperator{\Epigraphe}{Epigraphe}




\usepackage{xcolor}
\everymath{\color{blue}}
%\everymath{\color[rgb]{0,1,1}}
%\pagecolor[rgb]{0,0,0.5}


\newcommand*{\pdtest}[3][]{\ensuremath{\frac{\partial^{#1} #2}{\partial #3}}}

\newcommand*{\deffunc}[6][]{\ensuremath{
\begin{array}{rcl}
#2 : #3 &\rightarrow& #4\\
#5 &\mapsto& #6
\end{array}
}}

\newcommand{\eqcolon}{\mathrel{\resizebox{\widthof{$\mathord{=}$}}{\height}{ $\!\!=\!\!\resizebox{1.2\width}{0.8\height}{\raisebox{0.23ex}{$\mathop{:}$}}\!\!$ }}}
\newcommand{\coloneq}{\mathrel{\resizebox{\widthof{$\mathord{=}$}}{\height}{ $\!\!\resizebox{1.2\width}{0.8\height}{\raisebox{0.23ex}{$\mathop{:}$}}\!\!=\!\!$ }}}
\newcommand{\eqcolonl}{\ensuremath{\mathrel{=\!\!\mathop{:}}}}
\newcommand{\coloneql}{\ensuremath{\mathrel{\mathop{:} \!\! =}}}
\newcommand{\vc}[1]{% inline column vector
  \left(\begin{smallmatrix}#1\end{smallmatrix}\right)%
}
\newcommand{\vr}[1]{% inline row vector
  \begin{smallmatrix}(\,#1\,)\end{smallmatrix}%
}
\makeatletter
\newcommand*{\defeq}{\ =\mathrel{\rlap{%
                     \raisebox{0.3ex}{$\m@th\cdot$}}%
                     \raisebox{-0.3ex}{$\m@th\cdot$}}%
                     }
\makeatother

\newcommand{\mathcircle}[1]{% inline row vector
 \overset{\circ}{#1}
}
\newcommand{\ulim}{% low limit
 \underline{\lim}
}
\newcommand{\ssi}{% iff
\iff
}
\newcommand{\ps}[2]{
\expval{#1 | #2}
}
\newcommand{\df}[1]{
\mqty{#1}
}
\newcommand{\n}[1]{
\norm{#1}
}
\newcommand{\sys}[1]{
\left\{\smqty{#1}\right.
}


\newcommand{\eqdef}{\ensuremath{\overset{\text{def}}=}}


\def\Circlearrowright{\ensuremath{%
  \rotatebox[origin=c]{230}{$\circlearrowright$}}}

\newcommand\ct[1]{\text{\rmfamily\upshape #1}}
\newcommand\question[1]{ {\color{red} ...!? \small #1}}
\newcommand\caz[1]{\left\{\begin{array} #1 \end{array}\right.}
\newcommand\const{\text{\rmfamily\upshape const}}
\newcommand\toP{ \overset{\pro}{\to}}
\newcommand\toPP{ \overset{\text{PP}}{\to}}
\newcommand{\oeq}{\mathrel{\text{\textcircled{$=$}}}}





\usepackage{xcolor}
% \usepackage[normalem]{ulem}
\usepackage{lipsum}
\makeatletter
% \newcommand\colorwave[1][blue]{\bgroup \markoverwith{\lower3.5\p@\hbox{\sixly \textcolor{#1}{\char58}}}\ULon}
%\font\sixly=lasy6 % does not re-load if already loaded, so no memory problem.

\newmdtheoremenv[
linewidth= 1pt,linecolor= blue,%
leftmargin=20,rightmargin=20,innertopmargin=0pt, innerrightmargin=40,%
tikzsetting = { draw=lightgray, line width = 0.3pt,dashed,%
dash pattern = on 15pt off 3pt},%
splittopskip=\topskip,skipbelow=\baselineskip,%
skipabove=\baselineskip,ntheorem,roundcorner=0pt,
% backgroundcolor=pagebg,font=\color{orange}\sffamily, fontcolor=white
]{examplebox}{Exemple}[section]



\newcommand\R{\mathbb{R}}
\newcommand\Z{\mathbb{Z}}
\newcommand\N{\mathbb{N}}
\newcommand\E{\mathbb{E}}
\newcommand\F{\mathcal{F}}
\newcommand\cH{\mathcal{H}}
\newcommand\V{\mathbb{V}}
\newcommand\dmo{ ^{-1} }
\newcommand\kapa{\kappa}
\newcommand\im{Im}
\newcommand\hs{\mathcal{H}}





\usepackage{soul}

\makeatletter
\newcommand*{\whiten}[1]{\llap{\textcolor{white}{{\the\SOUL@token}}\hspace{#1pt}}}
\DeclareRobustCommand*\myul{%
    \def\SOUL@everyspace{\underline{\space}\kern\z@}%
    \def\SOUL@everytoken{%
     \setbox0=\hbox{\the\SOUL@token}%
     \ifdim\dp0>\z@
        \raisebox{\dp0}{\underline{\phantom{\the\SOUL@token}}}%
        \whiten{1}\whiten{0}%
        \whiten{-1}\whiten{-2}%
        \llap{\the\SOUL@token}%
     \else
        \underline{\the\SOUL@token}%
     \fi}%
\SOUL@}
\makeatother

\newcommand*{\demp}{\fontfamily{lmtt}\selectfont}

\DeclareTextFontCommand{\textdemp}{\demp}

\begin{document}

\ifcomment
Multiline
comment
\fi
\ifcomment
\myul{Typesetting test}
% \color[rgb]{1,1,1}
$∑_i^n≠ 60º±∞π∆¬≈√j∫h≤≥µ$

$\CR \R\pro\ind\pro\gS\pro
\mqty[a&b\\c&d]$
$\pro\mathbb{P}$
$\dd{x}$

  \[
    \alpha(x)=\left\{
                \begin{array}{ll}
                  x\\
                  \frac{1}{1+e^{-kx}}\\
                  \frac{e^x-e^{-x}}{e^x+e^{-x}}
                \end{array}
              \right.
  \]

  $\expval{x}$
  
  $\chi_\rho(ghg\dmo)=\Tr(\rho_{ghg\dmo})=\Tr(\rho_g\circ\rho_h\circ\rho\dmo_g)=\Tr(\rho_h)\overset{\mbox{\scalebox{0.5}{$\Tr(AB)=\Tr(BA)$}}}{=}\chi_\rho(h)$
  	$\mathop{\oplus}_{\substack{x\in X}}$

$\mat(\rho_g)=(a_{ij}(g))_{\scriptsize \substack{1\leq i\leq d \\ 1\leq j\leq d}}$ et $\mat(\rho'_g)=(a'_{ij}(g))_{\scriptsize \substack{1\leq i'\leq d' \\ 1\leq j'\leq d'}}$



\[\int_a^b{\mathbb{R}^2}g(u, v)\dd{P_{XY}}(u, v)=\iint g(u,v) f_{XY}(u, v)\dd \lambda(u) \dd \lambda(v)\]
$$\lim_{x\to\infty} f(x)$$	
$$\iiiint_V \mu(t,u,v,w) \,dt\,du\,dv\,dw$$
$$\sum_{n=1}^{\infty} 2^{-n} = 1$$	
\begin{definition}
	Si $X$ et $Y$ sont 2 v.a. ou definit la \textsc{Covariance} entre $X$ et $Y$ comme
	$\cov(X,Y)\overset{\text{def}}{=}\E\left[(X-\E(X))(Y-\E(Y))\right]=\E(XY)-\E(X)\E(Y)$.
\end{definition}
\fi
\pagebreak

% \tableofcontents

% insert your code here
%\input{./algebra/main.tex}
%\input{./geometrie-differentielle/main.tex}
%\input{./probabilite/main.tex}
%\input{./analyse-fonctionnelle/main.tex}
% \input{./Analyse-convexe-et-dualite-en-optimisation/main.tex}
%\input{./tikz/main.tex}
%\input{./Theorie-du-distributions/main.tex}
%\input{./optimisation/mine.tex}
 \input{./modelisation/main.tex}

% yves.aubry@univ-tln.fr : algebra

\end{document}

%% !TEX encoding = UTF-8 Unicode
% !TEX TS-program = xelatex

\documentclass[french]{report}

%\usepackage[utf8]{inputenc}
%\usepackage[T1]{fontenc}
\usepackage{babel}


\newif\ifcomment
%\commenttrue # Show comments

\usepackage{physics}
\usepackage{amssymb}


\usepackage{amsthm}
% \usepackage{thmtools}
\usepackage{mathtools}
\usepackage{amsfonts}

\usepackage{color}

\usepackage{tikz}

\usepackage{geometry}
\geometry{a5paper, margin=0.1in, right=1cm}

\usepackage{dsfont}

\usepackage{graphicx}
\graphicspath{ {images/} }

\usepackage{faktor}

\usepackage{IEEEtrantools}
\usepackage{enumerate}   
\usepackage[PostScript=dvips]{"/Users/aware/Documents/Courses/diagrams"}


\newtheorem{theorem}{Théorème}[section]
\renewcommand{\thetheorem}{\arabic{theorem}}
\newtheorem{lemme}{Lemme}[section]
\renewcommand{\thelemme}{\arabic{lemme}}
\newtheorem{proposition}{Proposition}[section]
\renewcommand{\theproposition}{\arabic{proposition}}
\newtheorem{notations}{Notations}[section]
\newtheorem{problem}{Problème}[section]
\newtheorem{corollary}{Corollaire}[theorem]
\renewcommand{\thecorollary}{\arabic{corollary}}
\newtheorem{property}{Propriété}[section]
\newtheorem{objective}{Objectif}[section]

\theoremstyle{definition}
\newtheorem{definition}{Définition}[section]
\renewcommand{\thedefinition}{\arabic{definition}}
\newtheorem{exercise}{Exercice}[chapter]
\renewcommand{\theexercise}{\arabic{exercise}}
\newtheorem{example}{Exemple}[chapter]
\renewcommand{\theexample}{\arabic{example}}
\newtheorem*{solution}{Solution}
\newtheorem*{application}{Application}
\newtheorem*{notation}{Notation}
\newtheorem*{vocabulary}{Vocabulaire}
\newtheorem*{properties}{Propriétés}



\theoremstyle{remark}
\newtheorem*{remark}{Remarque}
\newtheorem*{rappel}{Rappel}


\usepackage{etoolbox}
\AtBeginEnvironment{exercise}{\small}
\AtBeginEnvironment{example}{\small}

\usepackage{cases}
\usepackage[red]{mypack}

\usepackage[framemethod=TikZ]{mdframed}

\definecolor{bg}{rgb}{0.4,0.25,0.95}
\definecolor{pagebg}{rgb}{0,0,0.5}
\surroundwithmdframed[
   topline=false,
   rightline=false,
   bottomline=false,
   leftmargin=\parindent,
   skipabove=8pt,
   skipbelow=8pt,
   linecolor=blue,
   innerbottommargin=10pt,
   % backgroundcolor=bg,font=\color{orange}\sffamily, fontcolor=white
]{definition}

\usepackage{empheq}
\usepackage[most]{tcolorbox}

\newtcbox{\mymath}[1][]{%
    nobeforeafter, math upper, tcbox raise base,
    enhanced, colframe=blue!30!black,
    colback=red!10, boxrule=1pt,
    #1}

\usepackage{unixode}


\DeclareMathOperator{\ord}{ord}
\DeclareMathOperator{\orb}{orb}
\DeclareMathOperator{\stab}{stab}
\DeclareMathOperator{\Stab}{stab}
\DeclareMathOperator{\ppcm}{ppcm}
\DeclareMathOperator{\conj}{Conj}
\DeclareMathOperator{\End}{End}
\DeclareMathOperator{\rot}{rot}
\DeclareMathOperator{\trs}{trace}
\DeclareMathOperator{\Ind}{Ind}
\DeclareMathOperator{\mat}{Mat}
\DeclareMathOperator{\id}{Id}
\DeclareMathOperator{\vect}{vect}
\DeclareMathOperator{\img}{img}
\DeclareMathOperator{\cov}{Cov}
\DeclareMathOperator{\dist}{dist}
\DeclareMathOperator{\irr}{Irr}
\DeclareMathOperator{\image}{Im}
\DeclareMathOperator{\pd}{\partial}
\DeclareMathOperator{\epi}{epi}
\DeclareMathOperator{\Argmin}{Argmin}
\DeclareMathOperator{\dom}{dom}
\DeclareMathOperator{\proj}{proj}
\DeclareMathOperator{\ctg}{ctg}
\DeclareMathOperator{\supp}{supp}
\DeclareMathOperator{\argmin}{argmin}
\DeclareMathOperator{\mult}{mult}
\DeclareMathOperator{\ch}{ch}
\DeclareMathOperator{\sh}{sh}
\DeclareMathOperator{\rang}{rang}
\DeclareMathOperator{\diam}{diam}
\DeclareMathOperator{\Epigraphe}{Epigraphe}




\usepackage{xcolor}
\everymath{\color{blue}}
%\everymath{\color[rgb]{0,1,1}}
%\pagecolor[rgb]{0,0,0.5}


\newcommand*{\pdtest}[3][]{\ensuremath{\frac{\partial^{#1} #2}{\partial #3}}}

\newcommand*{\deffunc}[6][]{\ensuremath{
\begin{array}{rcl}
#2 : #3 &\rightarrow& #4\\
#5 &\mapsto& #6
\end{array}
}}

\newcommand{\eqcolon}{\mathrel{\resizebox{\widthof{$\mathord{=}$}}{\height}{ $\!\!=\!\!\resizebox{1.2\width}{0.8\height}{\raisebox{0.23ex}{$\mathop{:}$}}\!\!$ }}}
\newcommand{\coloneq}{\mathrel{\resizebox{\widthof{$\mathord{=}$}}{\height}{ $\!\!\resizebox{1.2\width}{0.8\height}{\raisebox{0.23ex}{$\mathop{:}$}}\!\!=\!\!$ }}}
\newcommand{\eqcolonl}{\ensuremath{\mathrel{=\!\!\mathop{:}}}}
\newcommand{\coloneql}{\ensuremath{\mathrel{\mathop{:} \!\! =}}}
\newcommand{\vc}[1]{% inline column vector
  \left(\begin{smallmatrix}#1\end{smallmatrix}\right)%
}
\newcommand{\vr}[1]{% inline row vector
  \begin{smallmatrix}(\,#1\,)\end{smallmatrix}%
}
\makeatletter
\newcommand*{\defeq}{\ =\mathrel{\rlap{%
                     \raisebox{0.3ex}{$\m@th\cdot$}}%
                     \raisebox{-0.3ex}{$\m@th\cdot$}}%
                     }
\makeatother

\newcommand{\mathcircle}[1]{% inline row vector
 \overset{\circ}{#1}
}
\newcommand{\ulim}{% low limit
 \underline{\lim}
}
\newcommand{\ssi}{% iff
\iff
}
\newcommand{\ps}[2]{
\expval{#1 | #2}
}
\newcommand{\df}[1]{
\mqty{#1}
}
\newcommand{\n}[1]{
\norm{#1}
}
\newcommand{\sys}[1]{
\left\{\smqty{#1}\right.
}


\newcommand{\eqdef}{\ensuremath{\overset{\text{def}}=}}


\def\Circlearrowright{\ensuremath{%
  \rotatebox[origin=c]{230}{$\circlearrowright$}}}

\newcommand\ct[1]{\text{\rmfamily\upshape #1}}
\newcommand\question[1]{ {\color{red} ...!? \small #1}}
\newcommand\caz[1]{\left\{\begin{array} #1 \end{array}\right.}
\newcommand\const{\text{\rmfamily\upshape const}}
\newcommand\toP{ \overset{\pro}{\to}}
\newcommand\toPP{ \overset{\text{PP}}{\to}}
\newcommand{\oeq}{\mathrel{\text{\textcircled{$=$}}}}





\usepackage{xcolor}
% \usepackage[normalem]{ulem}
\usepackage{lipsum}
\makeatletter
% \newcommand\colorwave[1][blue]{\bgroup \markoverwith{\lower3.5\p@\hbox{\sixly \textcolor{#1}{\char58}}}\ULon}
%\font\sixly=lasy6 % does not re-load if already loaded, so no memory problem.

\newmdtheoremenv[
linewidth= 1pt,linecolor= blue,%
leftmargin=20,rightmargin=20,innertopmargin=0pt, innerrightmargin=40,%
tikzsetting = { draw=lightgray, line width = 0.3pt,dashed,%
dash pattern = on 15pt off 3pt},%
splittopskip=\topskip,skipbelow=\baselineskip,%
skipabove=\baselineskip,ntheorem,roundcorner=0pt,
% backgroundcolor=pagebg,font=\color{orange}\sffamily, fontcolor=white
]{examplebox}{Exemple}[section]



\newcommand\R{\mathbb{R}}
\newcommand\Z{\mathbb{Z}}
\newcommand\N{\mathbb{N}}
\newcommand\E{\mathbb{E}}
\newcommand\F{\mathcal{F}}
\newcommand\cH{\mathcal{H}}
\newcommand\V{\mathbb{V}}
\newcommand\dmo{ ^{-1} }
\newcommand\kapa{\kappa}
\newcommand\im{Im}
\newcommand\hs{\mathcal{H}}





\usepackage{soul}

\makeatletter
\newcommand*{\whiten}[1]{\llap{\textcolor{white}{{\the\SOUL@token}}\hspace{#1pt}}}
\DeclareRobustCommand*\myul{%
    \def\SOUL@everyspace{\underline{\space}\kern\z@}%
    \def\SOUL@everytoken{%
     \setbox0=\hbox{\the\SOUL@token}%
     \ifdim\dp0>\z@
        \raisebox{\dp0}{\underline{\phantom{\the\SOUL@token}}}%
        \whiten{1}\whiten{0}%
        \whiten{-1}\whiten{-2}%
        \llap{\the\SOUL@token}%
     \else
        \underline{\the\SOUL@token}%
     \fi}%
\SOUL@}
\makeatother

\newcommand*{\demp}{\fontfamily{lmtt}\selectfont}

\DeclareTextFontCommand{\textdemp}{\demp}

\begin{document}

\ifcomment
Multiline
comment
\fi
\ifcomment
\myul{Typesetting test}
% \color[rgb]{1,1,1}
$∑_i^n≠ 60º±∞π∆¬≈√j∫h≤≥µ$

$\CR \R\pro\ind\pro\gS\pro
\mqty[a&b\\c&d]$
$\pro\mathbb{P}$
$\dd{x}$

  \[
    \alpha(x)=\left\{
                \begin{array}{ll}
                  x\\
                  \frac{1}{1+e^{-kx}}\\
                  \frac{e^x-e^{-x}}{e^x+e^{-x}}
                \end{array}
              \right.
  \]

  $\expval{x}$
  
  $\chi_\rho(ghg\dmo)=\Tr(\rho_{ghg\dmo})=\Tr(\rho_g\circ\rho_h\circ\rho\dmo_g)=\Tr(\rho_h)\overset{\mbox{\scalebox{0.5}{$\Tr(AB)=\Tr(BA)$}}}{=}\chi_\rho(h)$
  	$\mathop{\oplus}_{\substack{x\in X}}$

$\mat(\rho_g)=(a_{ij}(g))_{\scriptsize \substack{1\leq i\leq d \\ 1\leq j\leq d}}$ et $\mat(\rho'_g)=(a'_{ij}(g))_{\scriptsize \substack{1\leq i'\leq d' \\ 1\leq j'\leq d'}}$



\[\int_a^b{\mathbb{R}^2}g(u, v)\dd{P_{XY}}(u, v)=\iint g(u,v) f_{XY}(u, v)\dd \lambda(u) \dd \lambda(v)\]
$$\lim_{x\to\infty} f(x)$$	
$$\iiiint_V \mu(t,u,v,w) \,dt\,du\,dv\,dw$$
$$\sum_{n=1}^{\infty} 2^{-n} = 1$$	
\begin{definition}
	Si $X$ et $Y$ sont 2 v.a. ou definit la \textsc{Covariance} entre $X$ et $Y$ comme
	$\cov(X,Y)\overset{\text{def}}{=}\E\left[(X-\E(X))(Y-\E(Y))\right]=\E(XY)-\E(X)\E(Y)$.
\end{definition}
\fi
\pagebreak

% \tableofcontents

% insert your code here
%\input{./algebra/main.tex}
%\input{./geometrie-differentielle/main.tex}
%\input{./probabilite/main.tex}
%\input{./analyse-fonctionnelle/main.tex}
% \input{./Analyse-convexe-et-dualite-en-optimisation/main.tex}
%\input{./tikz/main.tex}
%\input{./Theorie-du-distributions/main.tex}
%\input{./optimisation/mine.tex}
 \input{./modelisation/main.tex}

% yves.aubry@univ-tln.fr : algebra

\end{document}

%\input{./optimisation/mine.tex}
 % !TEX encoding = UTF-8 Unicode
% !TEX TS-program = xelatex

\documentclass[french]{report}

%\usepackage[utf8]{inputenc}
%\usepackage[T1]{fontenc}
\usepackage{babel}


\newif\ifcomment
%\commenttrue # Show comments

\usepackage{physics}
\usepackage{amssymb}


\usepackage{amsthm}
% \usepackage{thmtools}
\usepackage{mathtools}
\usepackage{amsfonts}

\usepackage{color}

\usepackage{tikz}

\usepackage{geometry}
\geometry{a5paper, margin=0.1in, right=1cm}

\usepackage{dsfont}

\usepackage{graphicx}
\graphicspath{ {images/} }

\usepackage{faktor}

\usepackage{IEEEtrantools}
\usepackage{enumerate}   
\usepackage[PostScript=dvips]{"/Users/aware/Documents/Courses/diagrams"}


\newtheorem{theorem}{Théorème}[section]
\renewcommand{\thetheorem}{\arabic{theorem}}
\newtheorem{lemme}{Lemme}[section]
\renewcommand{\thelemme}{\arabic{lemme}}
\newtheorem{proposition}{Proposition}[section]
\renewcommand{\theproposition}{\arabic{proposition}}
\newtheorem{notations}{Notations}[section]
\newtheorem{problem}{Problème}[section]
\newtheorem{corollary}{Corollaire}[theorem]
\renewcommand{\thecorollary}{\arabic{corollary}}
\newtheorem{property}{Propriété}[section]
\newtheorem{objective}{Objectif}[section]

\theoremstyle{definition}
\newtheorem{definition}{Définition}[section]
\renewcommand{\thedefinition}{\arabic{definition}}
\newtheorem{exercise}{Exercice}[chapter]
\renewcommand{\theexercise}{\arabic{exercise}}
\newtheorem{example}{Exemple}[chapter]
\renewcommand{\theexample}{\arabic{example}}
\newtheorem*{solution}{Solution}
\newtheorem*{application}{Application}
\newtheorem*{notation}{Notation}
\newtheorem*{vocabulary}{Vocabulaire}
\newtheorem*{properties}{Propriétés}



\theoremstyle{remark}
\newtheorem*{remark}{Remarque}
\newtheorem*{rappel}{Rappel}


\usepackage{etoolbox}
\AtBeginEnvironment{exercise}{\small}
\AtBeginEnvironment{example}{\small}

\usepackage{cases}
\usepackage[red]{mypack}

\usepackage[framemethod=TikZ]{mdframed}

\definecolor{bg}{rgb}{0.4,0.25,0.95}
\definecolor{pagebg}{rgb}{0,0,0.5}
\surroundwithmdframed[
   topline=false,
   rightline=false,
   bottomline=false,
   leftmargin=\parindent,
   skipabove=8pt,
   skipbelow=8pt,
   linecolor=blue,
   innerbottommargin=10pt,
   % backgroundcolor=bg,font=\color{orange}\sffamily, fontcolor=white
]{definition}

\usepackage{empheq}
\usepackage[most]{tcolorbox}

\newtcbox{\mymath}[1][]{%
    nobeforeafter, math upper, tcbox raise base,
    enhanced, colframe=blue!30!black,
    colback=red!10, boxrule=1pt,
    #1}

\usepackage{unixode}


\DeclareMathOperator{\ord}{ord}
\DeclareMathOperator{\orb}{orb}
\DeclareMathOperator{\stab}{stab}
\DeclareMathOperator{\Stab}{stab}
\DeclareMathOperator{\ppcm}{ppcm}
\DeclareMathOperator{\conj}{Conj}
\DeclareMathOperator{\End}{End}
\DeclareMathOperator{\rot}{rot}
\DeclareMathOperator{\trs}{trace}
\DeclareMathOperator{\Ind}{Ind}
\DeclareMathOperator{\mat}{Mat}
\DeclareMathOperator{\id}{Id}
\DeclareMathOperator{\vect}{vect}
\DeclareMathOperator{\img}{img}
\DeclareMathOperator{\cov}{Cov}
\DeclareMathOperator{\dist}{dist}
\DeclareMathOperator{\irr}{Irr}
\DeclareMathOperator{\image}{Im}
\DeclareMathOperator{\pd}{\partial}
\DeclareMathOperator{\epi}{epi}
\DeclareMathOperator{\Argmin}{Argmin}
\DeclareMathOperator{\dom}{dom}
\DeclareMathOperator{\proj}{proj}
\DeclareMathOperator{\ctg}{ctg}
\DeclareMathOperator{\supp}{supp}
\DeclareMathOperator{\argmin}{argmin}
\DeclareMathOperator{\mult}{mult}
\DeclareMathOperator{\ch}{ch}
\DeclareMathOperator{\sh}{sh}
\DeclareMathOperator{\rang}{rang}
\DeclareMathOperator{\diam}{diam}
\DeclareMathOperator{\Epigraphe}{Epigraphe}




\usepackage{xcolor}
\everymath{\color{blue}}
%\everymath{\color[rgb]{0,1,1}}
%\pagecolor[rgb]{0,0,0.5}


\newcommand*{\pdtest}[3][]{\ensuremath{\frac{\partial^{#1} #2}{\partial #3}}}

\newcommand*{\deffunc}[6][]{\ensuremath{
\begin{array}{rcl}
#2 : #3 &\rightarrow& #4\\
#5 &\mapsto& #6
\end{array}
}}

\newcommand{\eqcolon}{\mathrel{\resizebox{\widthof{$\mathord{=}$}}{\height}{ $\!\!=\!\!\resizebox{1.2\width}{0.8\height}{\raisebox{0.23ex}{$\mathop{:}$}}\!\!$ }}}
\newcommand{\coloneq}{\mathrel{\resizebox{\widthof{$\mathord{=}$}}{\height}{ $\!\!\resizebox{1.2\width}{0.8\height}{\raisebox{0.23ex}{$\mathop{:}$}}\!\!=\!\!$ }}}
\newcommand{\eqcolonl}{\ensuremath{\mathrel{=\!\!\mathop{:}}}}
\newcommand{\coloneql}{\ensuremath{\mathrel{\mathop{:} \!\! =}}}
\newcommand{\vc}[1]{% inline column vector
  \left(\begin{smallmatrix}#1\end{smallmatrix}\right)%
}
\newcommand{\vr}[1]{% inline row vector
  \begin{smallmatrix}(\,#1\,)\end{smallmatrix}%
}
\makeatletter
\newcommand*{\defeq}{\ =\mathrel{\rlap{%
                     \raisebox{0.3ex}{$\m@th\cdot$}}%
                     \raisebox{-0.3ex}{$\m@th\cdot$}}%
                     }
\makeatother

\newcommand{\mathcircle}[1]{% inline row vector
 \overset{\circ}{#1}
}
\newcommand{\ulim}{% low limit
 \underline{\lim}
}
\newcommand{\ssi}{% iff
\iff
}
\newcommand{\ps}[2]{
\expval{#1 | #2}
}
\newcommand{\df}[1]{
\mqty{#1}
}
\newcommand{\n}[1]{
\norm{#1}
}
\newcommand{\sys}[1]{
\left\{\smqty{#1}\right.
}


\newcommand{\eqdef}{\ensuremath{\overset{\text{def}}=}}


\def\Circlearrowright{\ensuremath{%
  \rotatebox[origin=c]{230}{$\circlearrowright$}}}

\newcommand\ct[1]{\text{\rmfamily\upshape #1}}
\newcommand\question[1]{ {\color{red} ...!? \small #1}}
\newcommand\caz[1]{\left\{\begin{array} #1 \end{array}\right.}
\newcommand\const{\text{\rmfamily\upshape const}}
\newcommand\toP{ \overset{\pro}{\to}}
\newcommand\toPP{ \overset{\text{PP}}{\to}}
\newcommand{\oeq}{\mathrel{\text{\textcircled{$=$}}}}





\usepackage{xcolor}
% \usepackage[normalem]{ulem}
\usepackage{lipsum}
\makeatletter
% \newcommand\colorwave[1][blue]{\bgroup \markoverwith{\lower3.5\p@\hbox{\sixly \textcolor{#1}{\char58}}}\ULon}
%\font\sixly=lasy6 % does not re-load if already loaded, so no memory problem.

\newmdtheoremenv[
linewidth= 1pt,linecolor= blue,%
leftmargin=20,rightmargin=20,innertopmargin=0pt, innerrightmargin=40,%
tikzsetting = { draw=lightgray, line width = 0.3pt,dashed,%
dash pattern = on 15pt off 3pt},%
splittopskip=\topskip,skipbelow=\baselineskip,%
skipabove=\baselineskip,ntheorem,roundcorner=0pt,
% backgroundcolor=pagebg,font=\color{orange}\sffamily, fontcolor=white
]{examplebox}{Exemple}[section]



\newcommand\R{\mathbb{R}}
\newcommand\Z{\mathbb{Z}}
\newcommand\N{\mathbb{N}}
\newcommand\E{\mathbb{E}}
\newcommand\F{\mathcal{F}}
\newcommand\cH{\mathcal{H}}
\newcommand\V{\mathbb{V}}
\newcommand\dmo{ ^{-1} }
\newcommand\kapa{\kappa}
\newcommand\im{Im}
\newcommand\hs{\mathcal{H}}





\usepackage{soul}

\makeatletter
\newcommand*{\whiten}[1]{\llap{\textcolor{white}{{\the\SOUL@token}}\hspace{#1pt}}}
\DeclareRobustCommand*\myul{%
    \def\SOUL@everyspace{\underline{\space}\kern\z@}%
    \def\SOUL@everytoken{%
     \setbox0=\hbox{\the\SOUL@token}%
     \ifdim\dp0>\z@
        \raisebox{\dp0}{\underline{\phantom{\the\SOUL@token}}}%
        \whiten{1}\whiten{0}%
        \whiten{-1}\whiten{-2}%
        \llap{\the\SOUL@token}%
     \else
        \underline{\the\SOUL@token}%
     \fi}%
\SOUL@}
\makeatother

\newcommand*{\demp}{\fontfamily{lmtt}\selectfont}

\DeclareTextFontCommand{\textdemp}{\demp}

\begin{document}

\ifcomment
Multiline
comment
\fi
\ifcomment
\myul{Typesetting test}
% \color[rgb]{1,1,1}
$∑_i^n≠ 60º±∞π∆¬≈√j∫h≤≥µ$

$\CR \R\pro\ind\pro\gS\pro
\mqty[a&b\\c&d]$
$\pro\mathbb{P}$
$\dd{x}$

  \[
    \alpha(x)=\left\{
                \begin{array}{ll}
                  x\\
                  \frac{1}{1+e^{-kx}}\\
                  \frac{e^x-e^{-x}}{e^x+e^{-x}}
                \end{array}
              \right.
  \]

  $\expval{x}$
  
  $\chi_\rho(ghg\dmo)=\Tr(\rho_{ghg\dmo})=\Tr(\rho_g\circ\rho_h\circ\rho\dmo_g)=\Tr(\rho_h)\overset{\mbox{\scalebox{0.5}{$\Tr(AB)=\Tr(BA)$}}}{=}\chi_\rho(h)$
  	$\mathop{\oplus}_{\substack{x\in X}}$

$\mat(\rho_g)=(a_{ij}(g))_{\scriptsize \substack{1\leq i\leq d \\ 1\leq j\leq d}}$ et $\mat(\rho'_g)=(a'_{ij}(g))_{\scriptsize \substack{1\leq i'\leq d' \\ 1\leq j'\leq d'}}$



\[\int_a^b{\mathbb{R}^2}g(u, v)\dd{P_{XY}}(u, v)=\iint g(u,v) f_{XY}(u, v)\dd \lambda(u) \dd \lambda(v)\]
$$\lim_{x\to\infty} f(x)$$	
$$\iiiint_V \mu(t,u,v,w) \,dt\,du\,dv\,dw$$
$$\sum_{n=1}^{\infty} 2^{-n} = 1$$	
\begin{definition}
	Si $X$ et $Y$ sont 2 v.a. ou definit la \textsc{Covariance} entre $X$ et $Y$ comme
	$\cov(X,Y)\overset{\text{def}}{=}\E\left[(X-\E(X))(Y-\E(Y))\right]=\E(XY)-\E(X)\E(Y)$.
\end{definition}
\fi
\pagebreak

% \tableofcontents

% insert your code here
%\input{./algebra/main.tex}
%\input{./geometrie-differentielle/main.tex}
%\input{./probabilite/main.tex}
%\input{./analyse-fonctionnelle/main.tex}
% \input{./Analyse-convexe-et-dualite-en-optimisation/main.tex}
%\input{./tikz/main.tex}
%\input{./Theorie-du-distributions/main.tex}
%\input{./optimisation/mine.tex}
 \input{./modelisation/main.tex}

% yves.aubry@univ-tln.fr : algebra

\end{document}


% yves.aubry@univ-tln.fr : algebra

\end{document}

% % !TEX encoding = UTF-8 Unicode
% !TEX TS-program = xelatex

\documentclass[french]{report}

%\usepackage[utf8]{inputenc}
%\usepackage[T1]{fontenc}
\usepackage{babel}


\newif\ifcomment
%\commenttrue # Show comments

\usepackage{physics}
\usepackage{amssymb}


\usepackage{amsthm}
% \usepackage{thmtools}
\usepackage{mathtools}
\usepackage{amsfonts}

\usepackage{color}

\usepackage{tikz}

\usepackage{geometry}
\geometry{a5paper, margin=0.1in, right=1cm}

\usepackage{dsfont}

\usepackage{graphicx}
\graphicspath{ {images/} }

\usepackage{faktor}

\usepackage{IEEEtrantools}
\usepackage{enumerate}   
\usepackage[PostScript=dvips]{"/Users/aware/Documents/Courses/diagrams"}


\newtheorem{theorem}{Théorème}[section]
\renewcommand{\thetheorem}{\arabic{theorem}}
\newtheorem{lemme}{Lemme}[section]
\renewcommand{\thelemme}{\arabic{lemme}}
\newtheorem{proposition}{Proposition}[section]
\renewcommand{\theproposition}{\arabic{proposition}}
\newtheorem{notations}{Notations}[section]
\newtheorem{problem}{Problème}[section]
\newtheorem{corollary}{Corollaire}[theorem]
\renewcommand{\thecorollary}{\arabic{corollary}}
\newtheorem{property}{Propriété}[section]
\newtheorem{objective}{Objectif}[section]

\theoremstyle{definition}
\newtheorem{definition}{Définition}[section]
\renewcommand{\thedefinition}{\arabic{definition}}
\newtheorem{exercise}{Exercice}[chapter]
\renewcommand{\theexercise}{\arabic{exercise}}
\newtheorem{example}{Exemple}[chapter]
\renewcommand{\theexample}{\arabic{example}}
\newtheorem*{solution}{Solution}
\newtheorem*{application}{Application}
\newtheorem*{notation}{Notation}
\newtheorem*{vocabulary}{Vocabulaire}
\newtheorem*{properties}{Propriétés}



\theoremstyle{remark}
\newtheorem*{remark}{Remarque}
\newtheorem*{rappel}{Rappel}


\usepackage{etoolbox}
\AtBeginEnvironment{exercise}{\small}
\AtBeginEnvironment{example}{\small}

\usepackage{cases}
\usepackage[red]{mypack}

\usepackage[framemethod=TikZ]{mdframed}

\definecolor{bg}{rgb}{0.4,0.25,0.95}
\definecolor{pagebg}{rgb}{0,0,0.5}
\surroundwithmdframed[
   topline=false,
   rightline=false,
   bottomline=false,
   leftmargin=\parindent,
   skipabove=8pt,
   skipbelow=8pt,
   linecolor=blue,
   innerbottommargin=10pt,
   % backgroundcolor=bg,font=\color{orange}\sffamily, fontcolor=white
]{definition}

\usepackage{empheq}
\usepackage[most]{tcolorbox}

\newtcbox{\mymath}[1][]{%
    nobeforeafter, math upper, tcbox raise base,
    enhanced, colframe=blue!30!black,
    colback=red!10, boxrule=1pt,
    #1}

\usepackage{unixode}


\DeclareMathOperator{\ord}{ord}
\DeclareMathOperator{\orb}{orb}
\DeclareMathOperator{\stab}{stab}
\DeclareMathOperator{\Stab}{stab}
\DeclareMathOperator{\ppcm}{ppcm}
\DeclareMathOperator{\conj}{Conj}
\DeclareMathOperator{\End}{End}
\DeclareMathOperator{\rot}{rot}
\DeclareMathOperator{\trs}{trace}
\DeclareMathOperator{\Ind}{Ind}
\DeclareMathOperator{\mat}{Mat}
\DeclareMathOperator{\id}{Id}
\DeclareMathOperator{\vect}{vect}
\DeclareMathOperator{\img}{img}
\DeclareMathOperator{\cov}{Cov}
\DeclareMathOperator{\dist}{dist}
\DeclareMathOperator{\irr}{Irr}
\DeclareMathOperator{\image}{Im}
\DeclareMathOperator{\pd}{\partial}
\DeclareMathOperator{\epi}{epi}
\DeclareMathOperator{\Argmin}{Argmin}
\DeclareMathOperator{\dom}{dom}
\DeclareMathOperator{\proj}{proj}
\DeclareMathOperator{\ctg}{ctg}
\DeclareMathOperator{\supp}{supp}
\DeclareMathOperator{\argmin}{argmin}
\DeclareMathOperator{\mult}{mult}
\DeclareMathOperator{\ch}{ch}
\DeclareMathOperator{\sh}{sh}
\DeclareMathOperator{\rang}{rang}
\DeclareMathOperator{\diam}{diam}
\DeclareMathOperator{\Epigraphe}{Epigraphe}




\usepackage{xcolor}
\everymath{\color{blue}}
%\everymath{\color[rgb]{0,1,1}}
%\pagecolor[rgb]{0,0,0.5}


\newcommand*{\pdtest}[3][]{\ensuremath{\frac{\partial^{#1} #2}{\partial #3}}}

\newcommand*{\deffunc}[6][]{\ensuremath{
\begin{array}{rcl}
#2 : #3 &\rightarrow& #4\\
#5 &\mapsto& #6
\end{array}
}}

\newcommand{\eqcolon}{\mathrel{\resizebox{\widthof{$\mathord{=}$}}{\height}{ $\!\!=\!\!\resizebox{1.2\width}{0.8\height}{\raisebox{0.23ex}{$\mathop{:}$}}\!\!$ }}}
\newcommand{\coloneq}{\mathrel{\resizebox{\widthof{$\mathord{=}$}}{\height}{ $\!\!\resizebox{1.2\width}{0.8\height}{\raisebox{0.23ex}{$\mathop{:}$}}\!\!=\!\!$ }}}
\newcommand{\eqcolonl}{\ensuremath{\mathrel{=\!\!\mathop{:}}}}
\newcommand{\coloneql}{\ensuremath{\mathrel{\mathop{:} \!\! =}}}
\newcommand{\vc}[1]{% inline column vector
  \left(\begin{smallmatrix}#1\end{smallmatrix}\right)%
}
\newcommand{\vr}[1]{% inline row vector
  \begin{smallmatrix}(\,#1\,)\end{smallmatrix}%
}
\makeatletter
\newcommand*{\defeq}{\ =\mathrel{\rlap{%
                     \raisebox{0.3ex}{$\m@th\cdot$}}%
                     \raisebox{-0.3ex}{$\m@th\cdot$}}%
                     }
\makeatother

\newcommand{\mathcircle}[1]{% inline row vector
 \overset{\circ}{#1}
}
\newcommand{\ulim}{% low limit
 \underline{\lim}
}
\newcommand{\ssi}{% iff
\iff
}
\newcommand{\ps}[2]{
\expval{#1 | #2}
}
\newcommand{\df}[1]{
\mqty{#1}
}
\newcommand{\n}[1]{
\norm{#1}
}
\newcommand{\sys}[1]{
\left\{\smqty{#1}\right.
}


\newcommand{\eqdef}{\ensuremath{\overset{\text{def}}=}}


\def\Circlearrowright{\ensuremath{%
  \rotatebox[origin=c]{230}{$\circlearrowright$}}}

\newcommand\ct[1]{\text{\rmfamily\upshape #1}}
\newcommand\question[1]{ {\color{red} ...!? \small #1}}
\newcommand\caz[1]{\left\{\begin{array} #1 \end{array}\right.}
\newcommand\const{\text{\rmfamily\upshape const}}
\newcommand\toP{ \overset{\pro}{\to}}
\newcommand\toPP{ \overset{\text{PP}}{\to}}
\newcommand{\oeq}{\mathrel{\text{\textcircled{$=$}}}}





\usepackage{xcolor}
% \usepackage[normalem]{ulem}
\usepackage{lipsum}
\makeatletter
% \newcommand\colorwave[1][blue]{\bgroup \markoverwith{\lower3.5\p@\hbox{\sixly \textcolor{#1}{\char58}}}\ULon}
%\font\sixly=lasy6 % does not re-load if already loaded, so no memory problem.

\newmdtheoremenv[
linewidth= 1pt,linecolor= blue,%
leftmargin=20,rightmargin=20,innertopmargin=0pt, innerrightmargin=40,%
tikzsetting = { draw=lightgray, line width = 0.3pt,dashed,%
dash pattern = on 15pt off 3pt},%
splittopskip=\topskip,skipbelow=\baselineskip,%
skipabove=\baselineskip,ntheorem,roundcorner=0pt,
% backgroundcolor=pagebg,font=\color{orange}\sffamily, fontcolor=white
]{examplebox}{Exemple}[section]



\newcommand\R{\mathbb{R}}
\newcommand\Z{\mathbb{Z}}
\newcommand\N{\mathbb{N}}
\newcommand\E{\mathbb{E}}
\newcommand\F{\mathcal{F}}
\newcommand\cH{\mathcal{H}}
\newcommand\V{\mathbb{V}}
\newcommand\dmo{ ^{-1} }
\newcommand\kapa{\kappa}
\newcommand\im{Im}
\newcommand\hs{\mathcal{H}}





\usepackage{soul}

\makeatletter
\newcommand*{\whiten}[1]{\llap{\textcolor{white}{{\the\SOUL@token}}\hspace{#1pt}}}
\DeclareRobustCommand*\myul{%
    \def\SOUL@everyspace{\underline{\space}\kern\z@}%
    \def\SOUL@everytoken{%
     \setbox0=\hbox{\the\SOUL@token}%
     \ifdim\dp0>\z@
        \raisebox{\dp0}{\underline{\phantom{\the\SOUL@token}}}%
        \whiten{1}\whiten{0}%
        \whiten{-1}\whiten{-2}%
        \llap{\the\SOUL@token}%
     \else
        \underline{\the\SOUL@token}%
     \fi}%
\SOUL@}
\makeatother

\newcommand*{\demp}{\fontfamily{lmtt}\selectfont}

\DeclareTextFontCommand{\textdemp}{\demp}

\begin{document}

\ifcomment
Multiline
comment
\fi
\ifcomment
\myul{Typesetting test}
% \color[rgb]{1,1,1}
$∑_i^n≠ 60º±∞π∆¬≈√j∫h≤≥µ$

$\CR \R\pro\ind\pro\gS\pro
\mqty[a&b\\c&d]$
$\pro\mathbb{P}$
$\dd{x}$

  \[
    \alpha(x)=\left\{
                \begin{array}{ll}
                  x\\
                  \frac{1}{1+e^{-kx}}\\
                  \frac{e^x-e^{-x}}{e^x+e^{-x}}
                \end{array}
              \right.
  \]

  $\expval{x}$
  
  $\chi_\rho(ghg\dmo)=\Tr(\rho_{ghg\dmo})=\Tr(\rho_g\circ\rho_h\circ\rho\dmo_g)=\Tr(\rho_h)\overset{\mbox{\scalebox{0.5}{$\Tr(AB)=\Tr(BA)$}}}{=}\chi_\rho(h)$
  	$\mathop{\oplus}_{\substack{x\in X}}$

$\mat(\rho_g)=(a_{ij}(g))_{\scriptsize \substack{1\leq i\leq d \\ 1\leq j\leq d}}$ et $\mat(\rho'_g)=(a'_{ij}(g))_{\scriptsize \substack{1\leq i'\leq d' \\ 1\leq j'\leq d'}}$



\[\int_a^b{\mathbb{R}^2}g(u, v)\dd{P_{XY}}(u, v)=\iint g(u,v) f_{XY}(u, v)\dd \lambda(u) \dd \lambda(v)\]
$$\lim_{x\to\infty} f(x)$$	
$$\iiiint_V \mu(t,u,v,w) \,dt\,du\,dv\,dw$$
$$\sum_{n=1}^{\infty} 2^{-n} = 1$$	
\begin{definition}
	Si $X$ et $Y$ sont 2 v.a. ou definit la \textsc{Covariance} entre $X$ et $Y$ comme
	$\cov(X,Y)\overset{\text{def}}{=}\E\left[(X-\E(X))(Y-\E(Y))\right]=\E(XY)-\E(X)\E(Y)$.
\end{definition}
\fi
\pagebreak

% \tableofcontents

% insert your code here
%% !TEX encoding = UTF-8 Unicode
% !TEX TS-program = xelatex

\documentclass[french]{report}

%\usepackage[utf8]{inputenc}
%\usepackage[T1]{fontenc}
\usepackage{babel}


\newif\ifcomment
%\commenttrue # Show comments

\usepackage{physics}
\usepackage{amssymb}


\usepackage{amsthm}
% \usepackage{thmtools}
\usepackage{mathtools}
\usepackage{amsfonts}

\usepackage{color}

\usepackage{tikz}

\usepackage{geometry}
\geometry{a5paper, margin=0.1in, right=1cm}

\usepackage{dsfont}

\usepackage{graphicx}
\graphicspath{ {images/} }

\usepackage{faktor}

\usepackage{IEEEtrantools}
\usepackage{enumerate}   
\usepackage[PostScript=dvips]{"/Users/aware/Documents/Courses/diagrams"}


\newtheorem{theorem}{Théorème}[section]
\renewcommand{\thetheorem}{\arabic{theorem}}
\newtheorem{lemme}{Lemme}[section]
\renewcommand{\thelemme}{\arabic{lemme}}
\newtheorem{proposition}{Proposition}[section]
\renewcommand{\theproposition}{\arabic{proposition}}
\newtheorem{notations}{Notations}[section]
\newtheorem{problem}{Problème}[section]
\newtheorem{corollary}{Corollaire}[theorem]
\renewcommand{\thecorollary}{\arabic{corollary}}
\newtheorem{property}{Propriété}[section]
\newtheorem{objective}{Objectif}[section]

\theoremstyle{definition}
\newtheorem{definition}{Définition}[section]
\renewcommand{\thedefinition}{\arabic{definition}}
\newtheorem{exercise}{Exercice}[chapter]
\renewcommand{\theexercise}{\arabic{exercise}}
\newtheorem{example}{Exemple}[chapter]
\renewcommand{\theexample}{\arabic{example}}
\newtheorem*{solution}{Solution}
\newtheorem*{application}{Application}
\newtheorem*{notation}{Notation}
\newtheorem*{vocabulary}{Vocabulaire}
\newtheorem*{properties}{Propriétés}



\theoremstyle{remark}
\newtheorem*{remark}{Remarque}
\newtheorem*{rappel}{Rappel}


\usepackage{etoolbox}
\AtBeginEnvironment{exercise}{\small}
\AtBeginEnvironment{example}{\small}

\usepackage{cases}
\usepackage[red]{mypack}

\usepackage[framemethod=TikZ]{mdframed}

\definecolor{bg}{rgb}{0.4,0.25,0.95}
\definecolor{pagebg}{rgb}{0,0,0.5}
\surroundwithmdframed[
   topline=false,
   rightline=false,
   bottomline=false,
   leftmargin=\parindent,
   skipabove=8pt,
   skipbelow=8pt,
   linecolor=blue,
   innerbottommargin=10pt,
   % backgroundcolor=bg,font=\color{orange}\sffamily, fontcolor=white
]{definition}

\usepackage{empheq}
\usepackage[most]{tcolorbox}

\newtcbox{\mymath}[1][]{%
    nobeforeafter, math upper, tcbox raise base,
    enhanced, colframe=blue!30!black,
    colback=red!10, boxrule=1pt,
    #1}

\usepackage{unixode}


\DeclareMathOperator{\ord}{ord}
\DeclareMathOperator{\orb}{orb}
\DeclareMathOperator{\stab}{stab}
\DeclareMathOperator{\Stab}{stab}
\DeclareMathOperator{\ppcm}{ppcm}
\DeclareMathOperator{\conj}{Conj}
\DeclareMathOperator{\End}{End}
\DeclareMathOperator{\rot}{rot}
\DeclareMathOperator{\trs}{trace}
\DeclareMathOperator{\Ind}{Ind}
\DeclareMathOperator{\mat}{Mat}
\DeclareMathOperator{\id}{Id}
\DeclareMathOperator{\vect}{vect}
\DeclareMathOperator{\img}{img}
\DeclareMathOperator{\cov}{Cov}
\DeclareMathOperator{\dist}{dist}
\DeclareMathOperator{\irr}{Irr}
\DeclareMathOperator{\image}{Im}
\DeclareMathOperator{\pd}{\partial}
\DeclareMathOperator{\epi}{epi}
\DeclareMathOperator{\Argmin}{Argmin}
\DeclareMathOperator{\dom}{dom}
\DeclareMathOperator{\proj}{proj}
\DeclareMathOperator{\ctg}{ctg}
\DeclareMathOperator{\supp}{supp}
\DeclareMathOperator{\argmin}{argmin}
\DeclareMathOperator{\mult}{mult}
\DeclareMathOperator{\ch}{ch}
\DeclareMathOperator{\sh}{sh}
\DeclareMathOperator{\rang}{rang}
\DeclareMathOperator{\diam}{diam}
\DeclareMathOperator{\Epigraphe}{Epigraphe}




\usepackage{xcolor}
\everymath{\color{blue}}
%\everymath{\color[rgb]{0,1,1}}
%\pagecolor[rgb]{0,0,0.5}


\newcommand*{\pdtest}[3][]{\ensuremath{\frac{\partial^{#1} #2}{\partial #3}}}

\newcommand*{\deffunc}[6][]{\ensuremath{
\begin{array}{rcl}
#2 : #3 &\rightarrow& #4\\
#5 &\mapsto& #6
\end{array}
}}

\newcommand{\eqcolon}{\mathrel{\resizebox{\widthof{$\mathord{=}$}}{\height}{ $\!\!=\!\!\resizebox{1.2\width}{0.8\height}{\raisebox{0.23ex}{$\mathop{:}$}}\!\!$ }}}
\newcommand{\coloneq}{\mathrel{\resizebox{\widthof{$\mathord{=}$}}{\height}{ $\!\!\resizebox{1.2\width}{0.8\height}{\raisebox{0.23ex}{$\mathop{:}$}}\!\!=\!\!$ }}}
\newcommand{\eqcolonl}{\ensuremath{\mathrel{=\!\!\mathop{:}}}}
\newcommand{\coloneql}{\ensuremath{\mathrel{\mathop{:} \!\! =}}}
\newcommand{\vc}[1]{% inline column vector
  \left(\begin{smallmatrix}#1\end{smallmatrix}\right)%
}
\newcommand{\vr}[1]{% inline row vector
  \begin{smallmatrix}(\,#1\,)\end{smallmatrix}%
}
\makeatletter
\newcommand*{\defeq}{\ =\mathrel{\rlap{%
                     \raisebox{0.3ex}{$\m@th\cdot$}}%
                     \raisebox{-0.3ex}{$\m@th\cdot$}}%
                     }
\makeatother

\newcommand{\mathcircle}[1]{% inline row vector
 \overset{\circ}{#1}
}
\newcommand{\ulim}{% low limit
 \underline{\lim}
}
\newcommand{\ssi}{% iff
\iff
}
\newcommand{\ps}[2]{
\expval{#1 | #2}
}
\newcommand{\df}[1]{
\mqty{#1}
}
\newcommand{\n}[1]{
\norm{#1}
}
\newcommand{\sys}[1]{
\left\{\smqty{#1}\right.
}


\newcommand{\eqdef}{\ensuremath{\overset{\text{def}}=}}


\def\Circlearrowright{\ensuremath{%
  \rotatebox[origin=c]{230}{$\circlearrowright$}}}

\newcommand\ct[1]{\text{\rmfamily\upshape #1}}
\newcommand\question[1]{ {\color{red} ...!? \small #1}}
\newcommand\caz[1]{\left\{\begin{array} #1 \end{array}\right.}
\newcommand\const{\text{\rmfamily\upshape const}}
\newcommand\toP{ \overset{\pro}{\to}}
\newcommand\toPP{ \overset{\text{PP}}{\to}}
\newcommand{\oeq}{\mathrel{\text{\textcircled{$=$}}}}





\usepackage{xcolor}
% \usepackage[normalem]{ulem}
\usepackage{lipsum}
\makeatletter
% \newcommand\colorwave[1][blue]{\bgroup \markoverwith{\lower3.5\p@\hbox{\sixly \textcolor{#1}{\char58}}}\ULon}
%\font\sixly=lasy6 % does not re-load if already loaded, so no memory problem.

\newmdtheoremenv[
linewidth= 1pt,linecolor= blue,%
leftmargin=20,rightmargin=20,innertopmargin=0pt, innerrightmargin=40,%
tikzsetting = { draw=lightgray, line width = 0.3pt,dashed,%
dash pattern = on 15pt off 3pt},%
splittopskip=\topskip,skipbelow=\baselineskip,%
skipabove=\baselineskip,ntheorem,roundcorner=0pt,
% backgroundcolor=pagebg,font=\color{orange}\sffamily, fontcolor=white
]{examplebox}{Exemple}[section]



\newcommand\R{\mathbb{R}}
\newcommand\Z{\mathbb{Z}}
\newcommand\N{\mathbb{N}}
\newcommand\E{\mathbb{E}}
\newcommand\F{\mathcal{F}}
\newcommand\cH{\mathcal{H}}
\newcommand\V{\mathbb{V}}
\newcommand\dmo{ ^{-1} }
\newcommand\kapa{\kappa}
\newcommand\im{Im}
\newcommand\hs{\mathcal{H}}





\usepackage{soul}

\makeatletter
\newcommand*{\whiten}[1]{\llap{\textcolor{white}{{\the\SOUL@token}}\hspace{#1pt}}}
\DeclareRobustCommand*\myul{%
    \def\SOUL@everyspace{\underline{\space}\kern\z@}%
    \def\SOUL@everytoken{%
     \setbox0=\hbox{\the\SOUL@token}%
     \ifdim\dp0>\z@
        \raisebox{\dp0}{\underline{\phantom{\the\SOUL@token}}}%
        \whiten{1}\whiten{0}%
        \whiten{-1}\whiten{-2}%
        \llap{\the\SOUL@token}%
     \else
        \underline{\the\SOUL@token}%
     \fi}%
\SOUL@}
\makeatother

\newcommand*{\demp}{\fontfamily{lmtt}\selectfont}

\DeclareTextFontCommand{\textdemp}{\demp}

\begin{document}

\ifcomment
Multiline
comment
\fi
\ifcomment
\myul{Typesetting test}
% \color[rgb]{1,1,1}
$∑_i^n≠ 60º±∞π∆¬≈√j∫h≤≥µ$

$\CR \R\pro\ind\pro\gS\pro
\mqty[a&b\\c&d]$
$\pro\mathbb{P}$
$\dd{x}$

  \[
    \alpha(x)=\left\{
                \begin{array}{ll}
                  x\\
                  \frac{1}{1+e^{-kx}}\\
                  \frac{e^x-e^{-x}}{e^x+e^{-x}}
                \end{array}
              \right.
  \]

  $\expval{x}$
  
  $\chi_\rho(ghg\dmo)=\Tr(\rho_{ghg\dmo})=\Tr(\rho_g\circ\rho_h\circ\rho\dmo_g)=\Tr(\rho_h)\overset{\mbox{\scalebox{0.5}{$\Tr(AB)=\Tr(BA)$}}}{=}\chi_\rho(h)$
  	$\mathop{\oplus}_{\substack{x\in X}}$

$\mat(\rho_g)=(a_{ij}(g))_{\scriptsize \substack{1\leq i\leq d \\ 1\leq j\leq d}}$ et $\mat(\rho'_g)=(a'_{ij}(g))_{\scriptsize \substack{1\leq i'\leq d' \\ 1\leq j'\leq d'}}$



\[\int_a^b{\mathbb{R}^2}g(u, v)\dd{P_{XY}}(u, v)=\iint g(u,v) f_{XY}(u, v)\dd \lambda(u) \dd \lambda(v)\]
$$\lim_{x\to\infty} f(x)$$	
$$\iiiint_V \mu(t,u,v,w) \,dt\,du\,dv\,dw$$
$$\sum_{n=1}^{\infty} 2^{-n} = 1$$	
\begin{definition}
	Si $X$ et $Y$ sont 2 v.a. ou definit la \textsc{Covariance} entre $X$ et $Y$ comme
	$\cov(X,Y)\overset{\text{def}}{=}\E\left[(X-\E(X))(Y-\E(Y))\right]=\E(XY)-\E(X)\E(Y)$.
\end{definition}
\fi
\pagebreak

% \tableofcontents

% insert your code here
%\input{./algebra/main.tex}
%\input{./geometrie-differentielle/main.tex}
%\input{./probabilite/main.tex}
%\input{./analyse-fonctionnelle/main.tex}
% \input{./Analyse-convexe-et-dualite-en-optimisation/main.tex}
%\input{./tikz/main.tex}
%\input{./Theorie-du-distributions/main.tex}
%\input{./optimisation/mine.tex}
 \input{./modelisation/main.tex}

% yves.aubry@univ-tln.fr : algebra

\end{document}

%% !TEX encoding = UTF-8 Unicode
% !TEX TS-program = xelatex

\documentclass[french]{report}

%\usepackage[utf8]{inputenc}
%\usepackage[T1]{fontenc}
\usepackage{babel}


\newif\ifcomment
%\commenttrue # Show comments

\usepackage{physics}
\usepackage{amssymb}


\usepackage{amsthm}
% \usepackage{thmtools}
\usepackage{mathtools}
\usepackage{amsfonts}

\usepackage{color}

\usepackage{tikz}

\usepackage{geometry}
\geometry{a5paper, margin=0.1in, right=1cm}

\usepackage{dsfont}

\usepackage{graphicx}
\graphicspath{ {images/} }

\usepackage{faktor}

\usepackage{IEEEtrantools}
\usepackage{enumerate}   
\usepackage[PostScript=dvips]{"/Users/aware/Documents/Courses/diagrams"}


\newtheorem{theorem}{Théorème}[section]
\renewcommand{\thetheorem}{\arabic{theorem}}
\newtheorem{lemme}{Lemme}[section]
\renewcommand{\thelemme}{\arabic{lemme}}
\newtheorem{proposition}{Proposition}[section]
\renewcommand{\theproposition}{\arabic{proposition}}
\newtheorem{notations}{Notations}[section]
\newtheorem{problem}{Problème}[section]
\newtheorem{corollary}{Corollaire}[theorem]
\renewcommand{\thecorollary}{\arabic{corollary}}
\newtheorem{property}{Propriété}[section]
\newtheorem{objective}{Objectif}[section]

\theoremstyle{definition}
\newtheorem{definition}{Définition}[section]
\renewcommand{\thedefinition}{\arabic{definition}}
\newtheorem{exercise}{Exercice}[chapter]
\renewcommand{\theexercise}{\arabic{exercise}}
\newtheorem{example}{Exemple}[chapter]
\renewcommand{\theexample}{\arabic{example}}
\newtheorem*{solution}{Solution}
\newtheorem*{application}{Application}
\newtheorem*{notation}{Notation}
\newtheorem*{vocabulary}{Vocabulaire}
\newtheorem*{properties}{Propriétés}



\theoremstyle{remark}
\newtheorem*{remark}{Remarque}
\newtheorem*{rappel}{Rappel}


\usepackage{etoolbox}
\AtBeginEnvironment{exercise}{\small}
\AtBeginEnvironment{example}{\small}

\usepackage{cases}
\usepackage[red]{mypack}

\usepackage[framemethod=TikZ]{mdframed}

\definecolor{bg}{rgb}{0.4,0.25,0.95}
\definecolor{pagebg}{rgb}{0,0,0.5}
\surroundwithmdframed[
   topline=false,
   rightline=false,
   bottomline=false,
   leftmargin=\parindent,
   skipabove=8pt,
   skipbelow=8pt,
   linecolor=blue,
   innerbottommargin=10pt,
   % backgroundcolor=bg,font=\color{orange}\sffamily, fontcolor=white
]{definition}

\usepackage{empheq}
\usepackage[most]{tcolorbox}

\newtcbox{\mymath}[1][]{%
    nobeforeafter, math upper, tcbox raise base,
    enhanced, colframe=blue!30!black,
    colback=red!10, boxrule=1pt,
    #1}

\usepackage{unixode}


\DeclareMathOperator{\ord}{ord}
\DeclareMathOperator{\orb}{orb}
\DeclareMathOperator{\stab}{stab}
\DeclareMathOperator{\Stab}{stab}
\DeclareMathOperator{\ppcm}{ppcm}
\DeclareMathOperator{\conj}{Conj}
\DeclareMathOperator{\End}{End}
\DeclareMathOperator{\rot}{rot}
\DeclareMathOperator{\trs}{trace}
\DeclareMathOperator{\Ind}{Ind}
\DeclareMathOperator{\mat}{Mat}
\DeclareMathOperator{\id}{Id}
\DeclareMathOperator{\vect}{vect}
\DeclareMathOperator{\img}{img}
\DeclareMathOperator{\cov}{Cov}
\DeclareMathOperator{\dist}{dist}
\DeclareMathOperator{\irr}{Irr}
\DeclareMathOperator{\image}{Im}
\DeclareMathOperator{\pd}{\partial}
\DeclareMathOperator{\epi}{epi}
\DeclareMathOperator{\Argmin}{Argmin}
\DeclareMathOperator{\dom}{dom}
\DeclareMathOperator{\proj}{proj}
\DeclareMathOperator{\ctg}{ctg}
\DeclareMathOperator{\supp}{supp}
\DeclareMathOperator{\argmin}{argmin}
\DeclareMathOperator{\mult}{mult}
\DeclareMathOperator{\ch}{ch}
\DeclareMathOperator{\sh}{sh}
\DeclareMathOperator{\rang}{rang}
\DeclareMathOperator{\diam}{diam}
\DeclareMathOperator{\Epigraphe}{Epigraphe}




\usepackage{xcolor}
\everymath{\color{blue}}
%\everymath{\color[rgb]{0,1,1}}
%\pagecolor[rgb]{0,0,0.5}


\newcommand*{\pdtest}[3][]{\ensuremath{\frac{\partial^{#1} #2}{\partial #3}}}

\newcommand*{\deffunc}[6][]{\ensuremath{
\begin{array}{rcl}
#2 : #3 &\rightarrow& #4\\
#5 &\mapsto& #6
\end{array}
}}

\newcommand{\eqcolon}{\mathrel{\resizebox{\widthof{$\mathord{=}$}}{\height}{ $\!\!=\!\!\resizebox{1.2\width}{0.8\height}{\raisebox{0.23ex}{$\mathop{:}$}}\!\!$ }}}
\newcommand{\coloneq}{\mathrel{\resizebox{\widthof{$\mathord{=}$}}{\height}{ $\!\!\resizebox{1.2\width}{0.8\height}{\raisebox{0.23ex}{$\mathop{:}$}}\!\!=\!\!$ }}}
\newcommand{\eqcolonl}{\ensuremath{\mathrel{=\!\!\mathop{:}}}}
\newcommand{\coloneql}{\ensuremath{\mathrel{\mathop{:} \!\! =}}}
\newcommand{\vc}[1]{% inline column vector
  \left(\begin{smallmatrix}#1\end{smallmatrix}\right)%
}
\newcommand{\vr}[1]{% inline row vector
  \begin{smallmatrix}(\,#1\,)\end{smallmatrix}%
}
\makeatletter
\newcommand*{\defeq}{\ =\mathrel{\rlap{%
                     \raisebox{0.3ex}{$\m@th\cdot$}}%
                     \raisebox{-0.3ex}{$\m@th\cdot$}}%
                     }
\makeatother

\newcommand{\mathcircle}[1]{% inline row vector
 \overset{\circ}{#1}
}
\newcommand{\ulim}{% low limit
 \underline{\lim}
}
\newcommand{\ssi}{% iff
\iff
}
\newcommand{\ps}[2]{
\expval{#1 | #2}
}
\newcommand{\df}[1]{
\mqty{#1}
}
\newcommand{\n}[1]{
\norm{#1}
}
\newcommand{\sys}[1]{
\left\{\smqty{#1}\right.
}


\newcommand{\eqdef}{\ensuremath{\overset{\text{def}}=}}


\def\Circlearrowright{\ensuremath{%
  \rotatebox[origin=c]{230}{$\circlearrowright$}}}

\newcommand\ct[1]{\text{\rmfamily\upshape #1}}
\newcommand\question[1]{ {\color{red} ...!? \small #1}}
\newcommand\caz[1]{\left\{\begin{array} #1 \end{array}\right.}
\newcommand\const{\text{\rmfamily\upshape const}}
\newcommand\toP{ \overset{\pro}{\to}}
\newcommand\toPP{ \overset{\text{PP}}{\to}}
\newcommand{\oeq}{\mathrel{\text{\textcircled{$=$}}}}





\usepackage{xcolor}
% \usepackage[normalem]{ulem}
\usepackage{lipsum}
\makeatletter
% \newcommand\colorwave[1][blue]{\bgroup \markoverwith{\lower3.5\p@\hbox{\sixly \textcolor{#1}{\char58}}}\ULon}
%\font\sixly=lasy6 % does not re-load if already loaded, so no memory problem.

\newmdtheoremenv[
linewidth= 1pt,linecolor= blue,%
leftmargin=20,rightmargin=20,innertopmargin=0pt, innerrightmargin=40,%
tikzsetting = { draw=lightgray, line width = 0.3pt,dashed,%
dash pattern = on 15pt off 3pt},%
splittopskip=\topskip,skipbelow=\baselineskip,%
skipabove=\baselineskip,ntheorem,roundcorner=0pt,
% backgroundcolor=pagebg,font=\color{orange}\sffamily, fontcolor=white
]{examplebox}{Exemple}[section]



\newcommand\R{\mathbb{R}}
\newcommand\Z{\mathbb{Z}}
\newcommand\N{\mathbb{N}}
\newcommand\E{\mathbb{E}}
\newcommand\F{\mathcal{F}}
\newcommand\cH{\mathcal{H}}
\newcommand\V{\mathbb{V}}
\newcommand\dmo{ ^{-1} }
\newcommand\kapa{\kappa}
\newcommand\im{Im}
\newcommand\hs{\mathcal{H}}





\usepackage{soul}

\makeatletter
\newcommand*{\whiten}[1]{\llap{\textcolor{white}{{\the\SOUL@token}}\hspace{#1pt}}}
\DeclareRobustCommand*\myul{%
    \def\SOUL@everyspace{\underline{\space}\kern\z@}%
    \def\SOUL@everytoken{%
     \setbox0=\hbox{\the\SOUL@token}%
     \ifdim\dp0>\z@
        \raisebox{\dp0}{\underline{\phantom{\the\SOUL@token}}}%
        \whiten{1}\whiten{0}%
        \whiten{-1}\whiten{-2}%
        \llap{\the\SOUL@token}%
     \else
        \underline{\the\SOUL@token}%
     \fi}%
\SOUL@}
\makeatother

\newcommand*{\demp}{\fontfamily{lmtt}\selectfont}

\DeclareTextFontCommand{\textdemp}{\demp}

\begin{document}

\ifcomment
Multiline
comment
\fi
\ifcomment
\myul{Typesetting test}
% \color[rgb]{1,1,1}
$∑_i^n≠ 60º±∞π∆¬≈√j∫h≤≥µ$

$\CR \R\pro\ind\pro\gS\pro
\mqty[a&b\\c&d]$
$\pro\mathbb{P}$
$\dd{x}$

  \[
    \alpha(x)=\left\{
                \begin{array}{ll}
                  x\\
                  \frac{1}{1+e^{-kx}}\\
                  \frac{e^x-e^{-x}}{e^x+e^{-x}}
                \end{array}
              \right.
  \]

  $\expval{x}$
  
  $\chi_\rho(ghg\dmo)=\Tr(\rho_{ghg\dmo})=\Tr(\rho_g\circ\rho_h\circ\rho\dmo_g)=\Tr(\rho_h)\overset{\mbox{\scalebox{0.5}{$\Tr(AB)=\Tr(BA)$}}}{=}\chi_\rho(h)$
  	$\mathop{\oplus}_{\substack{x\in X}}$

$\mat(\rho_g)=(a_{ij}(g))_{\scriptsize \substack{1\leq i\leq d \\ 1\leq j\leq d}}$ et $\mat(\rho'_g)=(a'_{ij}(g))_{\scriptsize \substack{1\leq i'\leq d' \\ 1\leq j'\leq d'}}$



\[\int_a^b{\mathbb{R}^2}g(u, v)\dd{P_{XY}}(u, v)=\iint g(u,v) f_{XY}(u, v)\dd \lambda(u) \dd \lambda(v)\]
$$\lim_{x\to\infty} f(x)$$	
$$\iiiint_V \mu(t,u,v,w) \,dt\,du\,dv\,dw$$
$$\sum_{n=1}^{\infty} 2^{-n} = 1$$	
\begin{definition}
	Si $X$ et $Y$ sont 2 v.a. ou definit la \textsc{Covariance} entre $X$ et $Y$ comme
	$\cov(X,Y)\overset{\text{def}}{=}\E\left[(X-\E(X))(Y-\E(Y))\right]=\E(XY)-\E(X)\E(Y)$.
\end{definition}
\fi
\pagebreak

% \tableofcontents

% insert your code here
%\input{./algebra/main.tex}
%\input{./geometrie-differentielle/main.tex}
%\input{./probabilite/main.tex}
%\input{./analyse-fonctionnelle/main.tex}
% \input{./Analyse-convexe-et-dualite-en-optimisation/main.tex}
%\input{./tikz/main.tex}
%\input{./Theorie-du-distributions/main.tex}
%\input{./optimisation/mine.tex}
 \input{./modelisation/main.tex}

% yves.aubry@univ-tln.fr : algebra

\end{document}

%% !TEX encoding = UTF-8 Unicode
% !TEX TS-program = xelatex

\documentclass[french]{report}

%\usepackage[utf8]{inputenc}
%\usepackage[T1]{fontenc}
\usepackage{babel}


\newif\ifcomment
%\commenttrue # Show comments

\usepackage{physics}
\usepackage{amssymb}


\usepackage{amsthm}
% \usepackage{thmtools}
\usepackage{mathtools}
\usepackage{amsfonts}

\usepackage{color}

\usepackage{tikz}

\usepackage{geometry}
\geometry{a5paper, margin=0.1in, right=1cm}

\usepackage{dsfont}

\usepackage{graphicx}
\graphicspath{ {images/} }

\usepackage{faktor}

\usepackage{IEEEtrantools}
\usepackage{enumerate}   
\usepackage[PostScript=dvips]{"/Users/aware/Documents/Courses/diagrams"}


\newtheorem{theorem}{Théorème}[section]
\renewcommand{\thetheorem}{\arabic{theorem}}
\newtheorem{lemme}{Lemme}[section]
\renewcommand{\thelemme}{\arabic{lemme}}
\newtheorem{proposition}{Proposition}[section]
\renewcommand{\theproposition}{\arabic{proposition}}
\newtheorem{notations}{Notations}[section]
\newtheorem{problem}{Problème}[section]
\newtheorem{corollary}{Corollaire}[theorem]
\renewcommand{\thecorollary}{\arabic{corollary}}
\newtheorem{property}{Propriété}[section]
\newtheorem{objective}{Objectif}[section]

\theoremstyle{definition}
\newtheorem{definition}{Définition}[section]
\renewcommand{\thedefinition}{\arabic{definition}}
\newtheorem{exercise}{Exercice}[chapter]
\renewcommand{\theexercise}{\arabic{exercise}}
\newtheorem{example}{Exemple}[chapter]
\renewcommand{\theexample}{\arabic{example}}
\newtheorem*{solution}{Solution}
\newtheorem*{application}{Application}
\newtheorem*{notation}{Notation}
\newtheorem*{vocabulary}{Vocabulaire}
\newtheorem*{properties}{Propriétés}



\theoremstyle{remark}
\newtheorem*{remark}{Remarque}
\newtheorem*{rappel}{Rappel}


\usepackage{etoolbox}
\AtBeginEnvironment{exercise}{\small}
\AtBeginEnvironment{example}{\small}

\usepackage{cases}
\usepackage[red]{mypack}

\usepackage[framemethod=TikZ]{mdframed}

\definecolor{bg}{rgb}{0.4,0.25,0.95}
\definecolor{pagebg}{rgb}{0,0,0.5}
\surroundwithmdframed[
   topline=false,
   rightline=false,
   bottomline=false,
   leftmargin=\parindent,
   skipabove=8pt,
   skipbelow=8pt,
   linecolor=blue,
   innerbottommargin=10pt,
   % backgroundcolor=bg,font=\color{orange}\sffamily, fontcolor=white
]{definition}

\usepackage{empheq}
\usepackage[most]{tcolorbox}

\newtcbox{\mymath}[1][]{%
    nobeforeafter, math upper, tcbox raise base,
    enhanced, colframe=blue!30!black,
    colback=red!10, boxrule=1pt,
    #1}

\usepackage{unixode}


\DeclareMathOperator{\ord}{ord}
\DeclareMathOperator{\orb}{orb}
\DeclareMathOperator{\stab}{stab}
\DeclareMathOperator{\Stab}{stab}
\DeclareMathOperator{\ppcm}{ppcm}
\DeclareMathOperator{\conj}{Conj}
\DeclareMathOperator{\End}{End}
\DeclareMathOperator{\rot}{rot}
\DeclareMathOperator{\trs}{trace}
\DeclareMathOperator{\Ind}{Ind}
\DeclareMathOperator{\mat}{Mat}
\DeclareMathOperator{\id}{Id}
\DeclareMathOperator{\vect}{vect}
\DeclareMathOperator{\img}{img}
\DeclareMathOperator{\cov}{Cov}
\DeclareMathOperator{\dist}{dist}
\DeclareMathOperator{\irr}{Irr}
\DeclareMathOperator{\image}{Im}
\DeclareMathOperator{\pd}{\partial}
\DeclareMathOperator{\epi}{epi}
\DeclareMathOperator{\Argmin}{Argmin}
\DeclareMathOperator{\dom}{dom}
\DeclareMathOperator{\proj}{proj}
\DeclareMathOperator{\ctg}{ctg}
\DeclareMathOperator{\supp}{supp}
\DeclareMathOperator{\argmin}{argmin}
\DeclareMathOperator{\mult}{mult}
\DeclareMathOperator{\ch}{ch}
\DeclareMathOperator{\sh}{sh}
\DeclareMathOperator{\rang}{rang}
\DeclareMathOperator{\diam}{diam}
\DeclareMathOperator{\Epigraphe}{Epigraphe}




\usepackage{xcolor}
\everymath{\color{blue}}
%\everymath{\color[rgb]{0,1,1}}
%\pagecolor[rgb]{0,0,0.5}


\newcommand*{\pdtest}[3][]{\ensuremath{\frac{\partial^{#1} #2}{\partial #3}}}

\newcommand*{\deffunc}[6][]{\ensuremath{
\begin{array}{rcl}
#2 : #3 &\rightarrow& #4\\
#5 &\mapsto& #6
\end{array}
}}

\newcommand{\eqcolon}{\mathrel{\resizebox{\widthof{$\mathord{=}$}}{\height}{ $\!\!=\!\!\resizebox{1.2\width}{0.8\height}{\raisebox{0.23ex}{$\mathop{:}$}}\!\!$ }}}
\newcommand{\coloneq}{\mathrel{\resizebox{\widthof{$\mathord{=}$}}{\height}{ $\!\!\resizebox{1.2\width}{0.8\height}{\raisebox{0.23ex}{$\mathop{:}$}}\!\!=\!\!$ }}}
\newcommand{\eqcolonl}{\ensuremath{\mathrel{=\!\!\mathop{:}}}}
\newcommand{\coloneql}{\ensuremath{\mathrel{\mathop{:} \!\! =}}}
\newcommand{\vc}[1]{% inline column vector
  \left(\begin{smallmatrix}#1\end{smallmatrix}\right)%
}
\newcommand{\vr}[1]{% inline row vector
  \begin{smallmatrix}(\,#1\,)\end{smallmatrix}%
}
\makeatletter
\newcommand*{\defeq}{\ =\mathrel{\rlap{%
                     \raisebox{0.3ex}{$\m@th\cdot$}}%
                     \raisebox{-0.3ex}{$\m@th\cdot$}}%
                     }
\makeatother

\newcommand{\mathcircle}[1]{% inline row vector
 \overset{\circ}{#1}
}
\newcommand{\ulim}{% low limit
 \underline{\lim}
}
\newcommand{\ssi}{% iff
\iff
}
\newcommand{\ps}[2]{
\expval{#1 | #2}
}
\newcommand{\df}[1]{
\mqty{#1}
}
\newcommand{\n}[1]{
\norm{#1}
}
\newcommand{\sys}[1]{
\left\{\smqty{#1}\right.
}


\newcommand{\eqdef}{\ensuremath{\overset{\text{def}}=}}


\def\Circlearrowright{\ensuremath{%
  \rotatebox[origin=c]{230}{$\circlearrowright$}}}

\newcommand\ct[1]{\text{\rmfamily\upshape #1}}
\newcommand\question[1]{ {\color{red} ...!? \small #1}}
\newcommand\caz[1]{\left\{\begin{array} #1 \end{array}\right.}
\newcommand\const{\text{\rmfamily\upshape const}}
\newcommand\toP{ \overset{\pro}{\to}}
\newcommand\toPP{ \overset{\text{PP}}{\to}}
\newcommand{\oeq}{\mathrel{\text{\textcircled{$=$}}}}





\usepackage{xcolor}
% \usepackage[normalem]{ulem}
\usepackage{lipsum}
\makeatletter
% \newcommand\colorwave[1][blue]{\bgroup \markoverwith{\lower3.5\p@\hbox{\sixly \textcolor{#1}{\char58}}}\ULon}
%\font\sixly=lasy6 % does not re-load if already loaded, so no memory problem.

\newmdtheoremenv[
linewidth= 1pt,linecolor= blue,%
leftmargin=20,rightmargin=20,innertopmargin=0pt, innerrightmargin=40,%
tikzsetting = { draw=lightgray, line width = 0.3pt,dashed,%
dash pattern = on 15pt off 3pt},%
splittopskip=\topskip,skipbelow=\baselineskip,%
skipabove=\baselineskip,ntheorem,roundcorner=0pt,
% backgroundcolor=pagebg,font=\color{orange}\sffamily, fontcolor=white
]{examplebox}{Exemple}[section]



\newcommand\R{\mathbb{R}}
\newcommand\Z{\mathbb{Z}}
\newcommand\N{\mathbb{N}}
\newcommand\E{\mathbb{E}}
\newcommand\F{\mathcal{F}}
\newcommand\cH{\mathcal{H}}
\newcommand\V{\mathbb{V}}
\newcommand\dmo{ ^{-1} }
\newcommand\kapa{\kappa}
\newcommand\im{Im}
\newcommand\hs{\mathcal{H}}





\usepackage{soul}

\makeatletter
\newcommand*{\whiten}[1]{\llap{\textcolor{white}{{\the\SOUL@token}}\hspace{#1pt}}}
\DeclareRobustCommand*\myul{%
    \def\SOUL@everyspace{\underline{\space}\kern\z@}%
    \def\SOUL@everytoken{%
     \setbox0=\hbox{\the\SOUL@token}%
     \ifdim\dp0>\z@
        \raisebox{\dp0}{\underline{\phantom{\the\SOUL@token}}}%
        \whiten{1}\whiten{0}%
        \whiten{-1}\whiten{-2}%
        \llap{\the\SOUL@token}%
     \else
        \underline{\the\SOUL@token}%
     \fi}%
\SOUL@}
\makeatother

\newcommand*{\demp}{\fontfamily{lmtt}\selectfont}

\DeclareTextFontCommand{\textdemp}{\demp}

\begin{document}

\ifcomment
Multiline
comment
\fi
\ifcomment
\myul{Typesetting test}
% \color[rgb]{1,1,1}
$∑_i^n≠ 60º±∞π∆¬≈√j∫h≤≥µ$

$\CR \R\pro\ind\pro\gS\pro
\mqty[a&b\\c&d]$
$\pro\mathbb{P}$
$\dd{x}$

  \[
    \alpha(x)=\left\{
                \begin{array}{ll}
                  x\\
                  \frac{1}{1+e^{-kx}}\\
                  \frac{e^x-e^{-x}}{e^x+e^{-x}}
                \end{array}
              \right.
  \]

  $\expval{x}$
  
  $\chi_\rho(ghg\dmo)=\Tr(\rho_{ghg\dmo})=\Tr(\rho_g\circ\rho_h\circ\rho\dmo_g)=\Tr(\rho_h)\overset{\mbox{\scalebox{0.5}{$\Tr(AB)=\Tr(BA)$}}}{=}\chi_\rho(h)$
  	$\mathop{\oplus}_{\substack{x\in X}}$

$\mat(\rho_g)=(a_{ij}(g))_{\scriptsize \substack{1\leq i\leq d \\ 1\leq j\leq d}}$ et $\mat(\rho'_g)=(a'_{ij}(g))_{\scriptsize \substack{1\leq i'\leq d' \\ 1\leq j'\leq d'}}$



\[\int_a^b{\mathbb{R}^2}g(u, v)\dd{P_{XY}}(u, v)=\iint g(u,v) f_{XY}(u, v)\dd \lambda(u) \dd \lambda(v)\]
$$\lim_{x\to\infty} f(x)$$	
$$\iiiint_V \mu(t,u,v,w) \,dt\,du\,dv\,dw$$
$$\sum_{n=1}^{\infty} 2^{-n} = 1$$	
\begin{definition}
	Si $X$ et $Y$ sont 2 v.a. ou definit la \textsc{Covariance} entre $X$ et $Y$ comme
	$\cov(X,Y)\overset{\text{def}}{=}\E\left[(X-\E(X))(Y-\E(Y))\right]=\E(XY)-\E(X)\E(Y)$.
\end{definition}
\fi
\pagebreak

% \tableofcontents

% insert your code here
%\input{./algebra/main.tex}
%\input{./geometrie-differentielle/main.tex}
%\input{./probabilite/main.tex}
%\input{./analyse-fonctionnelle/main.tex}
% \input{./Analyse-convexe-et-dualite-en-optimisation/main.tex}
%\input{./tikz/main.tex}
%\input{./Theorie-du-distributions/main.tex}
%\input{./optimisation/mine.tex}
 \input{./modelisation/main.tex}

% yves.aubry@univ-tln.fr : algebra

\end{document}

%% !TEX encoding = UTF-8 Unicode
% !TEX TS-program = xelatex

\documentclass[french]{report}

%\usepackage[utf8]{inputenc}
%\usepackage[T1]{fontenc}
\usepackage{babel}


\newif\ifcomment
%\commenttrue # Show comments

\usepackage{physics}
\usepackage{amssymb}


\usepackage{amsthm}
% \usepackage{thmtools}
\usepackage{mathtools}
\usepackage{amsfonts}

\usepackage{color}

\usepackage{tikz}

\usepackage{geometry}
\geometry{a5paper, margin=0.1in, right=1cm}

\usepackage{dsfont}

\usepackage{graphicx}
\graphicspath{ {images/} }

\usepackage{faktor}

\usepackage{IEEEtrantools}
\usepackage{enumerate}   
\usepackage[PostScript=dvips]{"/Users/aware/Documents/Courses/diagrams"}


\newtheorem{theorem}{Théorème}[section]
\renewcommand{\thetheorem}{\arabic{theorem}}
\newtheorem{lemme}{Lemme}[section]
\renewcommand{\thelemme}{\arabic{lemme}}
\newtheorem{proposition}{Proposition}[section]
\renewcommand{\theproposition}{\arabic{proposition}}
\newtheorem{notations}{Notations}[section]
\newtheorem{problem}{Problème}[section]
\newtheorem{corollary}{Corollaire}[theorem]
\renewcommand{\thecorollary}{\arabic{corollary}}
\newtheorem{property}{Propriété}[section]
\newtheorem{objective}{Objectif}[section]

\theoremstyle{definition}
\newtheorem{definition}{Définition}[section]
\renewcommand{\thedefinition}{\arabic{definition}}
\newtheorem{exercise}{Exercice}[chapter]
\renewcommand{\theexercise}{\arabic{exercise}}
\newtheorem{example}{Exemple}[chapter]
\renewcommand{\theexample}{\arabic{example}}
\newtheorem*{solution}{Solution}
\newtheorem*{application}{Application}
\newtheorem*{notation}{Notation}
\newtheorem*{vocabulary}{Vocabulaire}
\newtheorem*{properties}{Propriétés}



\theoremstyle{remark}
\newtheorem*{remark}{Remarque}
\newtheorem*{rappel}{Rappel}


\usepackage{etoolbox}
\AtBeginEnvironment{exercise}{\small}
\AtBeginEnvironment{example}{\small}

\usepackage{cases}
\usepackage[red]{mypack}

\usepackage[framemethod=TikZ]{mdframed}

\definecolor{bg}{rgb}{0.4,0.25,0.95}
\definecolor{pagebg}{rgb}{0,0,0.5}
\surroundwithmdframed[
   topline=false,
   rightline=false,
   bottomline=false,
   leftmargin=\parindent,
   skipabove=8pt,
   skipbelow=8pt,
   linecolor=blue,
   innerbottommargin=10pt,
   % backgroundcolor=bg,font=\color{orange}\sffamily, fontcolor=white
]{definition}

\usepackage{empheq}
\usepackage[most]{tcolorbox}

\newtcbox{\mymath}[1][]{%
    nobeforeafter, math upper, tcbox raise base,
    enhanced, colframe=blue!30!black,
    colback=red!10, boxrule=1pt,
    #1}

\usepackage{unixode}


\DeclareMathOperator{\ord}{ord}
\DeclareMathOperator{\orb}{orb}
\DeclareMathOperator{\stab}{stab}
\DeclareMathOperator{\Stab}{stab}
\DeclareMathOperator{\ppcm}{ppcm}
\DeclareMathOperator{\conj}{Conj}
\DeclareMathOperator{\End}{End}
\DeclareMathOperator{\rot}{rot}
\DeclareMathOperator{\trs}{trace}
\DeclareMathOperator{\Ind}{Ind}
\DeclareMathOperator{\mat}{Mat}
\DeclareMathOperator{\id}{Id}
\DeclareMathOperator{\vect}{vect}
\DeclareMathOperator{\img}{img}
\DeclareMathOperator{\cov}{Cov}
\DeclareMathOperator{\dist}{dist}
\DeclareMathOperator{\irr}{Irr}
\DeclareMathOperator{\image}{Im}
\DeclareMathOperator{\pd}{\partial}
\DeclareMathOperator{\epi}{epi}
\DeclareMathOperator{\Argmin}{Argmin}
\DeclareMathOperator{\dom}{dom}
\DeclareMathOperator{\proj}{proj}
\DeclareMathOperator{\ctg}{ctg}
\DeclareMathOperator{\supp}{supp}
\DeclareMathOperator{\argmin}{argmin}
\DeclareMathOperator{\mult}{mult}
\DeclareMathOperator{\ch}{ch}
\DeclareMathOperator{\sh}{sh}
\DeclareMathOperator{\rang}{rang}
\DeclareMathOperator{\diam}{diam}
\DeclareMathOperator{\Epigraphe}{Epigraphe}




\usepackage{xcolor}
\everymath{\color{blue}}
%\everymath{\color[rgb]{0,1,1}}
%\pagecolor[rgb]{0,0,0.5}


\newcommand*{\pdtest}[3][]{\ensuremath{\frac{\partial^{#1} #2}{\partial #3}}}

\newcommand*{\deffunc}[6][]{\ensuremath{
\begin{array}{rcl}
#2 : #3 &\rightarrow& #4\\
#5 &\mapsto& #6
\end{array}
}}

\newcommand{\eqcolon}{\mathrel{\resizebox{\widthof{$\mathord{=}$}}{\height}{ $\!\!=\!\!\resizebox{1.2\width}{0.8\height}{\raisebox{0.23ex}{$\mathop{:}$}}\!\!$ }}}
\newcommand{\coloneq}{\mathrel{\resizebox{\widthof{$\mathord{=}$}}{\height}{ $\!\!\resizebox{1.2\width}{0.8\height}{\raisebox{0.23ex}{$\mathop{:}$}}\!\!=\!\!$ }}}
\newcommand{\eqcolonl}{\ensuremath{\mathrel{=\!\!\mathop{:}}}}
\newcommand{\coloneql}{\ensuremath{\mathrel{\mathop{:} \!\! =}}}
\newcommand{\vc}[1]{% inline column vector
  \left(\begin{smallmatrix}#1\end{smallmatrix}\right)%
}
\newcommand{\vr}[1]{% inline row vector
  \begin{smallmatrix}(\,#1\,)\end{smallmatrix}%
}
\makeatletter
\newcommand*{\defeq}{\ =\mathrel{\rlap{%
                     \raisebox{0.3ex}{$\m@th\cdot$}}%
                     \raisebox{-0.3ex}{$\m@th\cdot$}}%
                     }
\makeatother

\newcommand{\mathcircle}[1]{% inline row vector
 \overset{\circ}{#1}
}
\newcommand{\ulim}{% low limit
 \underline{\lim}
}
\newcommand{\ssi}{% iff
\iff
}
\newcommand{\ps}[2]{
\expval{#1 | #2}
}
\newcommand{\df}[1]{
\mqty{#1}
}
\newcommand{\n}[1]{
\norm{#1}
}
\newcommand{\sys}[1]{
\left\{\smqty{#1}\right.
}


\newcommand{\eqdef}{\ensuremath{\overset{\text{def}}=}}


\def\Circlearrowright{\ensuremath{%
  \rotatebox[origin=c]{230}{$\circlearrowright$}}}

\newcommand\ct[1]{\text{\rmfamily\upshape #1}}
\newcommand\question[1]{ {\color{red} ...!? \small #1}}
\newcommand\caz[1]{\left\{\begin{array} #1 \end{array}\right.}
\newcommand\const{\text{\rmfamily\upshape const}}
\newcommand\toP{ \overset{\pro}{\to}}
\newcommand\toPP{ \overset{\text{PP}}{\to}}
\newcommand{\oeq}{\mathrel{\text{\textcircled{$=$}}}}





\usepackage{xcolor}
% \usepackage[normalem]{ulem}
\usepackage{lipsum}
\makeatletter
% \newcommand\colorwave[1][blue]{\bgroup \markoverwith{\lower3.5\p@\hbox{\sixly \textcolor{#1}{\char58}}}\ULon}
%\font\sixly=lasy6 % does not re-load if already loaded, so no memory problem.

\newmdtheoremenv[
linewidth= 1pt,linecolor= blue,%
leftmargin=20,rightmargin=20,innertopmargin=0pt, innerrightmargin=40,%
tikzsetting = { draw=lightgray, line width = 0.3pt,dashed,%
dash pattern = on 15pt off 3pt},%
splittopskip=\topskip,skipbelow=\baselineskip,%
skipabove=\baselineskip,ntheorem,roundcorner=0pt,
% backgroundcolor=pagebg,font=\color{orange}\sffamily, fontcolor=white
]{examplebox}{Exemple}[section]



\newcommand\R{\mathbb{R}}
\newcommand\Z{\mathbb{Z}}
\newcommand\N{\mathbb{N}}
\newcommand\E{\mathbb{E}}
\newcommand\F{\mathcal{F}}
\newcommand\cH{\mathcal{H}}
\newcommand\V{\mathbb{V}}
\newcommand\dmo{ ^{-1} }
\newcommand\kapa{\kappa}
\newcommand\im{Im}
\newcommand\hs{\mathcal{H}}





\usepackage{soul}

\makeatletter
\newcommand*{\whiten}[1]{\llap{\textcolor{white}{{\the\SOUL@token}}\hspace{#1pt}}}
\DeclareRobustCommand*\myul{%
    \def\SOUL@everyspace{\underline{\space}\kern\z@}%
    \def\SOUL@everytoken{%
     \setbox0=\hbox{\the\SOUL@token}%
     \ifdim\dp0>\z@
        \raisebox{\dp0}{\underline{\phantom{\the\SOUL@token}}}%
        \whiten{1}\whiten{0}%
        \whiten{-1}\whiten{-2}%
        \llap{\the\SOUL@token}%
     \else
        \underline{\the\SOUL@token}%
     \fi}%
\SOUL@}
\makeatother

\newcommand*{\demp}{\fontfamily{lmtt}\selectfont}

\DeclareTextFontCommand{\textdemp}{\demp}

\begin{document}

\ifcomment
Multiline
comment
\fi
\ifcomment
\myul{Typesetting test}
% \color[rgb]{1,1,1}
$∑_i^n≠ 60º±∞π∆¬≈√j∫h≤≥µ$

$\CR \R\pro\ind\pro\gS\pro
\mqty[a&b\\c&d]$
$\pro\mathbb{P}$
$\dd{x}$

  \[
    \alpha(x)=\left\{
                \begin{array}{ll}
                  x\\
                  \frac{1}{1+e^{-kx}}\\
                  \frac{e^x-e^{-x}}{e^x+e^{-x}}
                \end{array}
              \right.
  \]

  $\expval{x}$
  
  $\chi_\rho(ghg\dmo)=\Tr(\rho_{ghg\dmo})=\Tr(\rho_g\circ\rho_h\circ\rho\dmo_g)=\Tr(\rho_h)\overset{\mbox{\scalebox{0.5}{$\Tr(AB)=\Tr(BA)$}}}{=}\chi_\rho(h)$
  	$\mathop{\oplus}_{\substack{x\in X}}$

$\mat(\rho_g)=(a_{ij}(g))_{\scriptsize \substack{1\leq i\leq d \\ 1\leq j\leq d}}$ et $\mat(\rho'_g)=(a'_{ij}(g))_{\scriptsize \substack{1\leq i'\leq d' \\ 1\leq j'\leq d'}}$



\[\int_a^b{\mathbb{R}^2}g(u, v)\dd{P_{XY}}(u, v)=\iint g(u,v) f_{XY}(u, v)\dd \lambda(u) \dd \lambda(v)\]
$$\lim_{x\to\infty} f(x)$$	
$$\iiiint_V \mu(t,u,v,w) \,dt\,du\,dv\,dw$$
$$\sum_{n=1}^{\infty} 2^{-n} = 1$$	
\begin{definition}
	Si $X$ et $Y$ sont 2 v.a. ou definit la \textsc{Covariance} entre $X$ et $Y$ comme
	$\cov(X,Y)\overset{\text{def}}{=}\E\left[(X-\E(X))(Y-\E(Y))\right]=\E(XY)-\E(X)\E(Y)$.
\end{definition}
\fi
\pagebreak

% \tableofcontents

% insert your code here
%\input{./algebra/main.tex}
%\input{./geometrie-differentielle/main.tex}
%\input{./probabilite/main.tex}
%\input{./analyse-fonctionnelle/main.tex}
% \input{./Analyse-convexe-et-dualite-en-optimisation/main.tex}
%\input{./tikz/main.tex}
%\input{./Theorie-du-distributions/main.tex}
%\input{./optimisation/mine.tex}
 \input{./modelisation/main.tex}

% yves.aubry@univ-tln.fr : algebra

\end{document}

% % !TEX encoding = UTF-8 Unicode
% !TEX TS-program = xelatex

\documentclass[french]{report}

%\usepackage[utf8]{inputenc}
%\usepackage[T1]{fontenc}
\usepackage{babel}


\newif\ifcomment
%\commenttrue # Show comments

\usepackage{physics}
\usepackage{amssymb}


\usepackage{amsthm}
% \usepackage{thmtools}
\usepackage{mathtools}
\usepackage{amsfonts}

\usepackage{color}

\usepackage{tikz}

\usepackage{geometry}
\geometry{a5paper, margin=0.1in, right=1cm}

\usepackage{dsfont}

\usepackage{graphicx}
\graphicspath{ {images/} }

\usepackage{faktor}

\usepackage{IEEEtrantools}
\usepackage{enumerate}   
\usepackage[PostScript=dvips]{"/Users/aware/Documents/Courses/diagrams"}


\newtheorem{theorem}{Théorème}[section]
\renewcommand{\thetheorem}{\arabic{theorem}}
\newtheorem{lemme}{Lemme}[section]
\renewcommand{\thelemme}{\arabic{lemme}}
\newtheorem{proposition}{Proposition}[section]
\renewcommand{\theproposition}{\arabic{proposition}}
\newtheorem{notations}{Notations}[section]
\newtheorem{problem}{Problème}[section]
\newtheorem{corollary}{Corollaire}[theorem]
\renewcommand{\thecorollary}{\arabic{corollary}}
\newtheorem{property}{Propriété}[section]
\newtheorem{objective}{Objectif}[section]

\theoremstyle{definition}
\newtheorem{definition}{Définition}[section]
\renewcommand{\thedefinition}{\arabic{definition}}
\newtheorem{exercise}{Exercice}[chapter]
\renewcommand{\theexercise}{\arabic{exercise}}
\newtheorem{example}{Exemple}[chapter]
\renewcommand{\theexample}{\arabic{example}}
\newtheorem*{solution}{Solution}
\newtheorem*{application}{Application}
\newtheorem*{notation}{Notation}
\newtheorem*{vocabulary}{Vocabulaire}
\newtheorem*{properties}{Propriétés}



\theoremstyle{remark}
\newtheorem*{remark}{Remarque}
\newtheorem*{rappel}{Rappel}


\usepackage{etoolbox}
\AtBeginEnvironment{exercise}{\small}
\AtBeginEnvironment{example}{\small}

\usepackage{cases}
\usepackage[red]{mypack}

\usepackage[framemethod=TikZ]{mdframed}

\definecolor{bg}{rgb}{0.4,0.25,0.95}
\definecolor{pagebg}{rgb}{0,0,0.5}
\surroundwithmdframed[
   topline=false,
   rightline=false,
   bottomline=false,
   leftmargin=\parindent,
   skipabove=8pt,
   skipbelow=8pt,
   linecolor=blue,
   innerbottommargin=10pt,
   % backgroundcolor=bg,font=\color{orange}\sffamily, fontcolor=white
]{definition}

\usepackage{empheq}
\usepackage[most]{tcolorbox}

\newtcbox{\mymath}[1][]{%
    nobeforeafter, math upper, tcbox raise base,
    enhanced, colframe=blue!30!black,
    colback=red!10, boxrule=1pt,
    #1}

\usepackage{unixode}


\DeclareMathOperator{\ord}{ord}
\DeclareMathOperator{\orb}{orb}
\DeclareMathOperator{\stab}{stab}
\DeclareMathOperator{\Stab}{stab}
\DeclareMathOperator{\ppcm}{ppcm}
\DeclareMathOperator{\conj}{Conj}
\DeclareMathOperator{\End}{End}
\DeclareMathOperator{\rot}{rot}
\DeclareMathOperator{\trs}{trace}
\DeclareMathOperator{\Ind}{Ind}
\DeclareMathOperator{\mat}{Mat}
\DeclareMathOperator{\id}{Id}
\DeclareMathOperator{\vect}{vect}
\DeclareMathOperator{\img}{img}
\DeclareMathOperator{\cov}{Cov}
\DeclareMathOperator{\dist}{dist}
\DeclareMathOperator{\irr}{Irr}
\DeclareMathOperator{\image}{Im}
\DeclareMathOperator{\pd}{\partial}
\DeclareMathOperator{\epi}{epi}
\DeclareMathOperator{\Argmin}{Argmin}
\DeclareMathOperator{\dom}{dom}
\DeclareMathOperator{\proj}{proj}
\DeclareMathOperator{\ctg}{ctg}
\DeclareMathOperator{\supp}{supp}
\DeclareMathOperator{\argmin}{argmin}
\DeclareMathOperator{\mult}{mult}
\DeclareMathOperator{\ch}{ch}
\DeclareMathOperator{\sh}{sh}
\DeclareMathOperator{\rang}{rang}
\DeclareMathOperator{\diam}{diam}
\DeclareMathOperator{\Epigraphe}{Epigraphe}




\usepackage{xcolor}
\everymath{\color{blue}}
%\everymath{\color[rgb]{0,1,1}}
%\pagecolor[rgb]{0,0,0.5}


\newcommand*{\pdtest}[3][]{\ensuremath{\frac{\partial^{#1} #2}{\partial #3}}}

\newcommand*{\deffunc}[6][]{\ensuremath{
\begin{array}{rcl}
#2 : #3 &\rightarrow& #4\\
#5 &\mapsto& #6
\end{array}
}}

\newcommand{\eqcolon}{\mathrel{\resizebox{\widthof{$\mathord{=}$}}{\height}{ $\!\!=\!\!\resizebox{1.2\width}{0.8\height}{\raisebox{0.23ex}{$\mathop{:}$}}\!\!$ }}}
\newcommand{\coloneq}{\mathrel{\resizebox{\widthof{$\mathord{=}$}}{\height}{ $\!\!\resizebox{1.2\width}{0.8\height}{\raisebox{0.23ex}{$\mathop{:}$}}\!\!=\!\!$ }}}
\newcommand{\eqcolonl}{\ensuremath{\mathrel{=\!\!\mathop{:}}}}
\newcommand{\coloneql}{\ensuremath{\mathrel{\mathop{:} \!\! =}}}
\newcommand{\vc}[1]{% inline column vector
  \left(\begin{smallmatrix}#1\end{smallmatrix}\right)%
}
\newcommand{\vr}[1]{% inline row vector
  \begin{smallmatrix}(\,#1\,)\end{smallmatrix}%
}
\makeatletter
\newcommand*{\defeq}{\ =\mathrel{\rlap{%
                     \raisebox{0.3ex}{$\m@th\cdot$}}%
                     \raisebox{-0.3ex}{$\m@th\cdot$}}%
                     }
\makeatother

\newcommand{\mathcircle}[1]{% inline row vector
 \overset{\circ}{#1}
}
\newcommand{\ulim}{% low limit
 \underline{\lim}
}
\newcommand{\ssi}{% iff
\iff
}
\newcommand{\ps}[2]{
\expval{#1 | #2}
}
\newcommand{\df}[1]{
\mqty{#1}
}
\newcommand{\n}[1]{
\norm{#1}
}
\newcommand{\sys}[1]{
\left\{\smqty{#1}\right.
}


\newcommand{\eqdef}{\ensuremath{\overset{\text{def}}=}}


\def\Circlearrowright{\ensuremath{%
  \rotatebox[origin=c]{230}{$\circlearrowright$}}}

\newcommand\ct[1]{\text{\rmfamily\upshape #1}}
\newcommand\question[1]{ {\color{red} ...!? \small #1}}
\newcommand\caz[1]{\left\{\begin{array} #1 \end{array}\right.}
\newcommand\const{\text{\rmfamily\upshape const}}
\newcommand\toP{ \overset{\pro}{\to}}
\newcommand\toPP{ \overset{\text{PP}}{\to}}
\newcommand{\oeq}{\mathrel{\text{\textcircled{$=$}}}}





\usepackage{xcolor}
% \usepackage[normalem]{ulem}
\usepackage{lipsum}
\makeatletter
% \newcommand\colorwave[1][blue]{\bgroup \markoverwith{\lower3.5\p@\hbox{\sixly \textcolor{#1}{\char58}}}\ULon}
%\font\sixly=lasy6 % does not re-load if already loaded, so no memory problem.

\newmdtheoremenv[
linewidth= 1pt,linecolor= blue,%
leftmargin=20,rightmargin=20,innertopmargin=0pt, innerrightmargin=40,%
tikzsetting = { draw=lightgray, line width = 0.3pt,dashed,%
dash pattern = on 15pt off 3pt},%
splittopskip=\topskip,skipbelow=\baselineskip,%
skipabove=\baselineskip,ntheorem,roundcorner=0pt,
% backgroundcolor=pagebg,font=\color{orange}\sffamily, fontcolor=white
]{examplebox}{Exemple}[section]



\newcommand\R{\mathbb{R}}
\newcommand\Z{\mathbb{Z}}
\newcommand\N{\mathbb{N}}
\newcommand\E{\mathbb{E}}
\newcommand\F{\mathcal{F}}
\newcommand\cH{\mathcal{H}}
\newcommand\V{\mathbb{V}}
\newcommand\dmo{ ^{-1} }
\newcommand\kapa{\kappa}
\newcommand\im{Im}
\newcommand\hs{\mathcal{H}}





\usepackage{soul}

\makeatletter
\newcommand*{\whiten}[1]{\llap{\textcolor{white}{{\the\SOUL@token}}\hspace{#1pt}}}
\DeclareRobustCommand*\myul{%
    \def\SOUL@everyspace{\underline{\space}\kern\z@}%
    \def\SOUL@everytoken{%
     \setbox0=\hbox{\the\SOUL@token}%
     \ifdim\dp0>\z@
        \raisebox{\dp0}{\underline{\phantom{\the\SOUL@token}}}%
        \whiten{1}\whiten{0}%
        \whiten{-1}\whiten{-2}%
        \llap{\the\SOUL@token}%
     \else
        \underline{\the\SOUL@token}%
     \fi}%
\SOUL@}
\makeatother

\newcommand*{\demp}{\fontfamily{lmtt}\selectfont}

\DeclareTextFontCommand{\textdemp}{\demp}

\begin{document}

\ifcomment
Multiline
comment
\fi
\ifcomment
\myul{Typesetting test}
% \color[rgb]{1,1,1}
$∑_i^n≠ 60º±∞π∆¬≈√j∫h≤≥µ$

$\CR \R\pro\ind\pro\gS\pro
\mqty[a&b\\c&d]$
$\pro\mathbb{P}$
$\dd{x}$

  \[
    \alpha(x)=\left\{
                \begin{array}{ll}
                  x\\
                  \frac{1}{1+e^{-kx}}\\
                  \frac{e^x-e^{-x}}{e^x+e^{-x}}
                \end{array}
              \right.
  \]

  $\expval{x}$
  
  $\chi_\rho(ghg\dmo)=\Tr(\rho_{ghg\dmo})=\Tr(\rho_g\circ\rho_h\circ\rho\dmo_g)=\Tr(\rho_h)\overset{\mbox{\scalebox{0.5}{$\Tr(AB)=\Tr(BA)$}}}{=}\chi_\rho(h)$
  	$\mathop{\oplus}_{\substack{x\in X}}$

$\mat(\rho_g)=(a_{ij}(g))_{\scriptsize \substack{1\leq i\leq d \\ 1\leq j\leq d}}$ et $\mat(\rho'_g)=(a'_{ij}(g))_{\scriptsize \substack{1\leq i'\leq d' \\ 1\leq j'\leq d'}}$



\[\int_a^b{\mathbb{R}^2}g(u, v)\dd{P_{XY}}(u, v)=\iint g(u,v) f_{XY}(u, v)\dd \lambda(u) \dd \lambda(v)\]
$$\lim_{x\to\infty} f(x)$$	
$$\iiiint_V \mu(t,u,v,w) \,dt\,du\,dv\,dw$$
$$\sum_{n=1}^{\infty} 2^{-n} = 1$$	
\begin{definition}
	Si $X$ et $Y$ sont 2 v.a. ou definit la \textsc{Covariance} entre $X$ et $Y$ comme
	$\cov(X,Y)\overset{\text{def}}{=}\E\left[(X-\E(X))(Y-\E(Y))\right]=\E(XY)-\E(X)\E(Y)$.
\end{definition}
\fi
\pagebreak

% \tableofcontents

% insert your code here
%\input{./algebra/main.tex}
%\input{./geometrie-differentielle/main.tex}
%\input{./probabilite/main.tex}
%\input{./analyse-fonctionnelle/main.tex}
% \input{./Analyse-convexe-et-dualite-en-optimisation/main.tex}
%\input{./tikz/main.tex}
%\input{./Theorie-du-distributions/main.tex}
%\input{./optimisation/mine.tex}
 \input{./modelisation/main.tex}

% yves.aubry@univ-tln.fr : algebra

\end{document}

%% !TEX encoding = UTF-8 Unicode
% !TEX TS-program = xelatex

\documentclass[french]{report}

%\usepackage[utf8]{inputenc}
%\usepackage[T1]{fontenc}
\usepackage{babel}


\newif\ifcomment
%\commenttrue # Show comments

\usepackage{physics}
\usepackage{amssymb}


\usepackage{amsthm}
% \usepackage{thmtools}
\usepackage{mathtools}
\usepackage{amsfonts}

\usepackage{color}

\usepackage{tikz}

\usepackage{geometry}
\geometry{a5paper, margin=0.1in, right=1cm}

\usepackage{dsfont}

\usepackage{graphicx}
\graphicspath{ {images/} }

\usepackage{faktor}

\usepackage{IEEEtrantools}
\usepackage{enumerate}   
\usepackage[PostScript=dvips]{"/Users/aware/Documents/Courses/diagrams"}


\newtheorem{theorem}{Théorème}[section]
\renewcommand{\thetheorem}{\arabic{theorem}}
\newtheorem{lemme}{Lemme}[section]
\renewcommand{\thelemme}{\arabic{lemme}}
\newtheorem{proposition}{Proposition}[section]
\renewcommand{\theproposition}{\arabic{proposition}}
\newtheorem{notations}{Notations}[section]
\newtheorem{problem}{Problème}[section]
\newtheorem{corollary}{Corollaire}[theorem]
\renewcommand{\thecorollary}{\arabic{corollary}}
\newtheorem{property}{Propriété}[section]
\newtheorem{objective}{Objectif}[section]

\theoremstyle{definition}
\newtheorem{definition}{Définition}[section]
\renewcommand{\thedefinition}{\arabic{definition}}
\newtheorem{exercise}{Exercice}[chapter]
\renewcommand{\theexercise}{\arabic{exercise}}
\newtheorem{example}{Exemple}[chapter]
\renewcommand{\theexample}{\arabic{example}}
\newtheorem*{solution}{Solution}
\newtheorem*{application}{Application}
\newtheorem*{notation}{Notation}
\newtheorem*{vocabulary}{Vocabulaire}
\newtheorem*{properties}{Propriétés}



\theoremstyle{remark}
\newtheorem*{remark}{Remarque}
\newtheorem*{rappel}{Rappel}


\usepackage{etoolbox}
\AtBeginEnvironment{exercise}{\small}
\AtBeginEnvironment{example}{\small}

\usepackage{cases}
\usepackage[red]{mypack}

\usepackage[framemethod=TikZ]{mdframed}

\definecolor{bg}{rgb}{0.4,0.25,0.95}
\definecolor{pagebg}{rgb}{0,0,0.5}
\surroundwithmdframed[
   topline=false,
   rightline=false,
   bottomline=false,
   leftmargin=\parindent,
   skipabove=8pt,
   skipbelow=8pt,
   linecolor=blue,
   innerbottommargin=10pt,
   % backgroundcolor=bg,font=\color{orange}\sffamily, fontcolor=white
]{definition}

\usepackage{empheq}
\usepackage[most]{tcolorbox}

\newtcbox{\mymath}[1][]{%
    nobeforeafter, math upper, tcbox raise base,
    enhanced, colframe=blue!30!black,
    colback=red!10, boxrule=1pt,
    #1}

\usepackage{unixode}


\DeclareMathOperator{\ord}{ord}
\DeclareMathOperator{\orb}{orb}
\DeclareMathOperator{\stab}{stab}
\DeclareMathOperator{\Stab}{stab}
\DeclareMathOperator{\ppcm}{ppcm}
\DeclareMathOperator{\conj}{Conj}
\DeclareMathOperator{\End}{End}
\DeclareMathOperator{\rot}{rot}
\DeclareMathOperator{\trs}{trace}
\DeclareMathOperator{\Ind}{Ind}
\DeclareMathOperator{\mat}{Mat}
\DeclareMathOperator{\id}{Id}
\DeclareMathOperator{\vect}{vect}
\DeclareMathOperator{\img}{img}
\DeclareMathOperator{\cov}{Cov}
\DeclareMathOperator{\dist}{dist}
\DeclareMathOperator{\irr}{Irr}
\DeclareMathOperator{\image}{Im}
\DeclareMathOperator{\pd}{\partial}
\DeclareMathOperator{\epi}{epi}
\DeclareMathOperator{\Argmin}{Argmin}
\DeclareMathOperator{\dom}{dom}
\DeclareMathOperator{\proj}{proj}
\DeclareMathOperator{\ctg}{ctg}
\DeclareMathOperator{\supp}{supp}
\DeclareMathOperator{\argmin}{argmin}
\DeclareMathOperator{\mult}{mult}
\DeclareMathOperator{\ch}{ch}
\DeclareMathOperator{\sh}{sh}
\DeclareMathOperator{\rang}{rang}
\DeclareMathOperator{\diam}{diam}
\DeclareMathOperator{\Epigraphe}{Epigraphe}




\usepackage{xcolor}
\everymath{\color{blue}}
%\everymath{\color[rgb]{0,1,1}}
%\pagecolor[rgb]{0,0,0.5}


\newcommand*{\pdtest}[3][]{\ensuremath{\frac{\partial^{#1} #2}{\partial #3}}}

\newcommand*{\deffunc}[6][]{\ensuremath{
\begin{array}{rcl}
#2 : #3 &\rightarrow& #4\\
#5 &\mapsto& #6
\end{array}
}}

\newcommand{\eqcolon}{\mathrel{\resizebox{\widthof{$\mathord{=}$}}{\height}{ $\!\!=\!\!\resizebox{1.2\width}{0.8\height}{\raisebox{0.23ex}{$\mathop{:}$}}\!\!$ }}}
\newcommand{\coloneq}{\mathrel{\resizebox{\widthof{$\mathord{=}$}}{\height}{ $\!\!\resizebox{1.2\width}{0.8\height}{\raisebox{0.23ex}{$\mathop{:}$}}\!\!=\!\!$ }}}
\newcommand{\eqcolonl}{\ensuremath{\mathrel{=\!\!\mathop{:}}}}
\newcommand{\coloneql}{\ensuremath{\mathrel{\mathop{:} \!\! =}}}
\newcommand{\vc}[1]{% inline column vector
  \left(\begin{smallmatrix}#1\end{smallmatrix}\right)%
}
\newcommand{\vr}[1]{% inline row vector
  \begin{smallmatrix}(\,#1\,)\end{smallmatrix}%
}
\makeatletter
\newcommand*{\defeq}{\ =\mathrel{\rlap{%
                     \raisebox{0.3ex}{$\m@th\cdot$}}%
                     \raisebox{-0.3ex}{$\m@th\cdot$}}%
                     }
\makeatother

\newcommand{\mathcircle}[1]{% inline row vector
 \overset{\circ}{#1}
}
\newcommand{\ulim}{% low limit
 \underline{\lim}
}
\newcommand{\ssi}{% iff
\iff
}
\newcommand{\ps}[2]{
\expval{#1 | #2}
}
\newcommand{\df}[1]{
\mqty{#1}
}
\newcommand{\n}[1]{
\norm{#1}
}
\newcommand{\sys}[1]{
\left\{\smqty{#1}\right.
}


\newcommand{\eqdef}{\ensuremath{\overset{\text{def}}=}}


\def\Circlearrowright{\ensuremath{%
  \rotatebox[origin=c]{230}{$\circlearrowright$}}}

\newcommand\ct[1]{\text{\rmfamily\upshape #1}}
\newcommand\question[1]{ {\color{red} ...!? \small #1}}
\newcommand\caz[1]{\left\{\begin{array} #1 \end{array}\right.}
\newcommand\const{\text{\rmfamily\upshape const}}
\newcommand\toP{ \overset{\pro}{\to}}
\newcommand\toPP{ \overset{\text{PP}}{\to}}
\newcommand{\oeq}{\mathrel{\text{\textcircled{$=$}}}}





\usepackage{xcolor}
% \usepackage[normalem]{ulem}
\usepackage{lipsum}
\makeatletter
% \newcommand\colorwave[1][blue]{\bgroup \markoverwith{\lower3.5\p@\hbox{\sixly \textcolor{#1}{\char58}}}\ULon}
%\font\sixly=lasy6 % does not re-load if already loaded, so no memory problem.

\newmdtheoremenv[
linewidth= 1pt,linecolor= blue,%
leftmargin=20,rightmargin=20,innertopmargin=0pt, innerrightmargin=40,%
tikzsetting = { draw=lightgray, line width = 0.3pt,dashed,%
dash pattern = on 15pt off 3pt},%
splittopskip=\topskip,skipbelow=\baselineskip,%
skipabove=\baselineskip,ntheorem,roundcorner=0pt,
% backgroundcolor=pagebg,font=\color{orange}\sffamily, fontcolor=white
]{examplebox}{Exemple}[section]



\newcommand\R{\mathbb{R}}
\newcommand\Z{\mathbb{Z}}
\newcommand\N{\mathbb{N}}
\newcommand\E{\mathbb{E}}
\newcommand\F{\mathcal{F}}
\newcommand\cH{\mathcal{H}}
\newcommand\V{\mathbb{V}}
\newcommand\dmo{ ^{-1} }
\newcommand\kapa{\kappa}
\newcommand\im{Im}
\newcommand\hs{\mathcal{H}}





\usepackage{soul}

\makeatletter
\newcommand*{\whiten}[1]{\llap{\textcolor{white}{{\the\SOUL@token}}\hspace{#1pt}}}
\DeclareRobustCommand*\myul{%
    \def\SOUL@everyspace{\underline{\space}\kern\z@}%
    \def\SOUL@everytoken{%
     \setbox0=\hbox{\the\SOUL@token}%
     \ifdim\dp0>\z@
        \raisebox{\dp0}{\underline{\phantom{\the\SOUL@token}}}%
        \whiten{1}\whiten{0}%
        \whiten{-1}\whiten{-2}%
        \llap{\the\SOUL@token}%
     \else
        \underline{\the\SOUL@token}%
     \fi}%
\SOUL@}
\makeatother

\newcommand*{\demp}{\fontfamily{lmtt}\selectfont}

\DeclareTextFontCommand{\textdemp}{\demp}

\begin{document}

\ifcomment
Multiline
comment
\fi
\ifcomment
\myul{Typesetting test}
% \color[rgb]{1,1,1}
$∑_i^n≠ 60º±∞π∆¬≈√j∫h≤≥µ$

$\CR \R\pro\ind\pro\gS\pro
\mqty[a&b\\c&d]$
$\pro\mathbb{P}$
$\dd{x}$

  \[
    \alpha(x)=\left\{
                \begin{array}{ll}
                  x\\
                  \frac{1}{1+e^{-kx}}\\
                  \frac{e^x-e^{-x}}{e^x+e^{-x}}
                \end{array}
              \right.
  \]

  $\expval{x}$
  
  $\chi_\rho(ghg\dmo)=\Tr(\rho_{ghg\dmo})=\Tr(\rho_g\circ\rho_h\circ\rho\dmo_g)=\Tr(\rho_h)\overset{\mbox{\scalebox{0.5}{$\Tr(AB)=\Tr(BA)$}}}{=}\chi_\rho(h)$
  	$\mathop{\oplus}_{\substack{x\in X}}$

$\mat(\rho_g)=(a_{ij}(g))_{\scriptsize \substack{1\leq i\leq d \\ 1\leq j\leq d}}$ et $\mat(\rho'_g)=(a'_{ij}(g))_{\scriptsize \substack{1\leq i'\leq d' \\ 1\leq j'\leq d'}}$



\[\int_a^b{\mathbb{R}^2}g(u, v)\dd{P_{XY}}(u, v)=\iint g(u,v) f_{XY}(u, v)\dd \lambda(u) \dd \lambda(v)\]
$$\lim_{x\to\infty} f(x)$$	
$$\iiiint_V \mu(t,u,v,w) \,dt\,du\,dv\,dw$$
$$\sum_{n=1}^{\infty} 2^{-n} = 1$$	
\begin{definition}
	Si $X$ et $Y$ sont 2 v.a. ou definit la \textsc{Covariance} entre $X$ et $Y$ comme
	$\cov(X,Y)\overset{\text{def}}{=}\E\left[(X-\E(X))(Y-\E(Y))\right]=\E(XY)-\E(X)\E(Y)$.
\end{definition}
\fi
\pagebreak

% \tableofcontents

% insert your code here
%\input{./algebra/main.tex}
%\input{./geometrie-differentielle/main.tex}
%\input{./probabilite/main.tex}
%\input{./analyse-fonctionnelle/main.tex}
% \input{./Analyse-convexe-et-dualite-en-optimisation/main.tex}
%\input{./tikz/main.tex}
%\input{./Theorie-du-distributions/main.tex}
%\input{./optimisation/mine.tex}
 \input{./modelisation/main.tex}

% yves.aubry@univ-tln.fr : algebra

\end{document}

%% !TEX encoding = UTF-8 Unicode
% !TEX TS-program = xelatex

\documentclass[french]{report}

%\usepackage[utf8]{inputenc}
%\usepackage[T1]{fontenc}
\usepackage{babel}


\newif\ifcomment
%\commenttrue # Show comments

\usepackage{physics}
\usepackage{amssymb}


\usepackage{amsthm}
% \usepackage{thmtools}
\usepackage{mathtools}
\usepackage{amsfonts}

\usepackage{color}

\usepackage{tikz}

\usepackage{geometry}
\geometry{a5paper, margin=0.1in, right=1cm}

\usepackage{dsfont}

\usepackage{graphicx}
\graphicspath{ {images/} }

\usepackage{faktor}

\usepackage{IEEEtrantools}
\usepackage{enumerate}   
\usepackage[PostScript=dvips]{"/Users/aware/Documents/Courses/diagrams"}


\newtheorem{theorem}{Théorème}[section]
\renewcommand{\thetheorem}{\arabic{theorem}}
\newtheorem{lemme}{Lemme}[section]
\renewcommand{\thelemme}{\arabic{lemme}}
\newtheorem{proposition}{Proposition}[section]
\renewcommand{\theproposition}{\arabic{proposition}}
\newtheorem{notations}{Notations}[section]
\newtheorem{problem}{Problème}[section]
\newtheorem{corollary}{Corollaire}[theorem]
\renewcommand{\thecorollary}{\arabic{corollary}}
\newtheorem{property}{Propriété}[section]
\newtheorem{objective}{Objectif}[section]

\theoremstyle{definition}
\newtheorem{definition}{Définition}[section]
\renewcommand{\thedefinition}{\arabic{definition}}
\newtheorem{exercise}{Exercice}[chapter]
\renewcommand{\theexercise}{\arabic{exercise}}
\newtheorem{example}{Exemple}[chapter]
\renewcommand{\theexample}{\arabic{example}}
\newtheorem*{solution}{Solution}
\newtheorem*{application}{Application}
\newtheorem*{notation}{Notation}
\newtheorem*{vocabulary}{Vocabulaire}
\newtheorem*{properties}{Propriétés}



\theoremstyle{remark}
\newtheorem*{remark}{Remarque}
\newtheorem*{rappel}{Rappel}


\usepackage{etoolbox}
\AtBeginEnvironment{exercise}{\small}
\AtBeginEnvironment{example}{\small}

\usepackage{cases}
\usepackage[red]{mypack}

\usepackage[framemethod=TikZ]{mdframed}

\definecolor{bg}{rgb}{0.4,0.25,0.95}
\definecolor{pagebg}{rgb}{0,0,0.5}
\surroundwithmdframed[
   topline=false,
   rightline=false,
   bottomline=false,
   leftmargin=\parindent,
   skipabove=8pt,
   skipbelow=8pt,
   linecolor=blue,
   innerbottommargin=10pt,
   % backgroundcolor=bg,font=\color{orange}\sffamily, fontcolor=white
]{definition}

\usepackage{empheq}
\usepackage[most]{tcolorbox}

\newtcbox{\mymath}[1][]{%
    nobeforeafter, math upper, tcbox raise base,
    enhanced, colframe=blue!30!black,
    colback=red!10, boxrule=1pt,
    #1}

\usepackage{unixode}


\DeclareMathOperator{\ord}{ord}
\DeclareMathOperator{\orb}{orb}
\DeclareMathOperator{\stab}{stab}
\DeclareMathOperator{\Stab}{stab}
\DeclareMathOperator{\ppcm}{ppcm}
\DeclareMathOperator{\conj}{Conj}
\DeclareMathOperator{\End}{End}
\DeclareMathOperator{\rot}{rot}
\DeclareMathOperator{\trs}{trace}
\DeclareMathOperator{\Ind}{Ind}
\DeclareMathOperator{\mat}{Mat}
\DeclareMathOperator{\id}{Id}
\DeclareMathOperator{\vect}{vect}
\DeclareMathOperator{\img}{img}
\DeclareMathOperator{\cov}{Cov}
\DeclareMathOperator{\dist}{dist}
\DeclareMathOperator{\irr}{Irr}
\DeclareMathOperator{\image}{Im}
\DeclareMathOperator{\pd}{\partial}
\DeclareMathOperator{\epi}{epi}
\DeclareMathOperator{\Argmin}{Argmin}
\DeclareMathOperator{\dom}{dom}
\DeclareMathOperator{\proj}{proj}
\DeclareMathOperator{\ctg}{ctg}
\DeclareMathOperator{\supp}{supp}
\DeclareMathOperator{\argmin}{argmin}
\DeclareMathOperator{\mult}{mult}
\DeclareMathOperator{\ch}{ch}
\DeclareMathOperator{\sh}{sh}
\DeclareMathOperator{\rang}{rang}
\DeclareMathOperator{\diam}{diam}
\DeclareMathOperator{\Epigraphe}{Epigraphe}




\usepackage{xcolor}
\everymath{\color{blue}}
%\everymath{\color[rgb]{0,1,1}}
%\pagecolor[rgb]{0,0,0.5}


\newcommand*{\pdtest}[3][]{\ensuremath{\frac{\partial^{#1} #2}{\partial #3}}}

\newcommand*{\deffunc}[6][]{\ensuremath{
\begin{array}{rcl}
#2 : #3 &\rightarrow& #4\\
#5 &\mapsto& #6
\end{array}
}}

\newcommand{\eqcolon}{\mathrel{\resizebox{\widthof{$\mathord{=}$}}{\height}{ $\!\!=\!\!\resizebox{1.2\width}{0.8\height}{\raisebox{0.23ex}{$\mathop{:}$}}\!\!$ }}}
\newcommand{\coloneq}{\mathrel{\resizebox{\widthof{$\mathord{=}$}}{\height}{ $\!\!\resizebox{1.2\width}{0.8\height}{\raisebox{0.23ex}{$\mathop{:}$}}\!\!=\!\!$ }}}
\newcommand{\eqcolonl}{\ensuremath{\mathrel{=\!\!\mathop{:}}}}
\newcommand{\coloneql}{\ensuremath{\mathrel{\mathop{:} \!\! =}}}
\newcommand{\vc}[1]{% inline column vector
  \left(\begin{smallmatrix}#1\end{smallmatrix}\right)%
}
\newcommand{\vr}[1]{% inline row vector
  \begin{smallmatrix}(\,#1\,)\end{smallmatrix}%
}
\makeatletter
\newcommand*{\defeq}{\ =\mathrel{\rlap{%
                     \raisebox{0.3ex}{$\m@th\cdot$}}%
                     \raisebox{-0.3ex}{$\m@th\cdot$}}%
                     }
\makeatother

\newcommand{\mathcircle}[1]{% inline row vector
 \overset{\circ}{#1}
}
\newcommand{\ulim}{% low limit
 \underline{\lim}
}
\newcommand{\ssi}{% iff
\iff
}
\newcommand{\ps}[2]{
\expval{#1 | #2}
}
\newcommand{\df}[1]{
\mqty{#1}
}
\newcommand{\n}[1]{
\norm{#1}
}
\newcommand{\sys}[1]{
\left\{\smqty{#1}\right.
}


\newcommand{\eqdef}{\ensuremath{\overset{\text{def}}=}}


\def\Circlearrowright{\ensuremath{%
  \rotatebox[origin=c]{230}{$\circlearrowright$}}}

\newcommand\ct[1]{\text{\rmfamily\upshape #1}}
\newcommand\question[1]{ {\color{red} ...!? \small #1}}
\newcommand\caz[1]{\left\{\begin{array} #1 \end{array}\right.}
\newcommand\const{\text{\rmfamily\upshape const}}
\newcommand\toP{ \overset{\pro}{\to}}
\newcommand\toPP{ \overset{\text{PP}}{\to}}
\newcommand{\oeq}{\mathrel{\text{\textcircled{$=$}}}}





\usepackage{xcolor}
% \usepackage[normalem]{ulem}
\usepackage{lipsum}
\makeatletter
% \newcommand\colorwave[1][blue]{\bgroup \markoverwith{\lower3.5\p@\hbox{\sixly \textcolor{#1}{\char58}}}\ULon}
%\font\sixly=lasy6 % does not re-load if already loaded, so no memory problem.

\newmdtheoremenv[
linewidth= 1pt,linecolor= blue,%
leftmargin=20,rightmargin=20,innertopmargin=0pt, innerrightmargin=40,%
tikzsetting = { draw=lightgray, line width = 0.3pt,dashed,%
dash pattern = on 15pt off 3pt},%
splittopskip=\topskip,skipbelow=\baselineskip,%
skipabove=\baselineskip,ntheorem,roundcorner=0pt,
% backgroundcolor=pagebg,font=\color{orange}\sffamily, fontcolor=white
]{examplebox}{Exemple}[section]



\newcommand\R{\mathbb{R}}
\newcommand\Z{\mathbb{Z}}
\newcommand\N{\mathbb{N}}
\newcommand\E{\mathbb{E}}
\newcommand\F{\mathcal{F}}
\newcommand\cH{\mathcal{H}}
\newcommand\V{\mathbb{V}}
\newcommand\dmo{ ^{-1} }
\newcommand\kapa{\kappa}
\newcommand\im{Im}
\newcommand\hs{\mathcal{H}}





\usepackage{soul}

\makeatletter
\newcommand*{\whiten}[1]{\llap{\textcolor{white}{{\the\SOUL@token}}\hspace{#1pt}}}
\DeclareRobustCommand*\myul{%
    \def\SOUL@everyspace{\underline{\space}\kern\z@}%
    \def\SOUL@everytoken{%
     \setbox0=\hbox{\the\SOUL@token}%
     \ifdim\dp0>\z@
        \raisebox{\dp0}{\underline{\phantom{\the\SOUL@token}}}%
        \whiten{1}\whiten{0}%
        \whiten{-1}\whiten{-2}%
        \llap{\the\SOUL@token}%
     \else
        \underline{\the\SOUL@token}%
     \fi}%
\SOUL@}
\makeatother

\newcommand*{\demp}{\fontfamily{lmtt}\selectfont}

\DeclareTextFontCommand{\textdemp}{\demp}

\begin{document}

\ifcomment
Multiline
comment
\fi
\ifcomment
\myul{Typesetting test}
% \color[rgb]{1,1,1}
$∑_i^n≠ 60º±∞π∆¬≈√j∫h≤≥µ$

$\CR \R\pro\ind\pro\gS\pro
\mqty[a&b\\c&d]$
$\pro\mathbb{P}$
$\dd{x}$

  \[
    \alpha(x)=\left\{
                \begin{array}{ll}
                  x\\
                  \frac{1}{1+e^{-kx}}\\
                  \frac{e^x-e^{-x}}{e^x+e^{-x}}
                \end{array}
              \right.
  \]

  $\expval{x}$
  
  $\chi_\rho(ghg\dmo)=\Tr(\rho_{ghg\dmo})=\Tr(\rho_g\circ\rho_h\circ\rho\dmo_g)=\Tr(\rho_h)\overset{\mbox{\scalebox{0.5}{$\Tr(AB)=\Tr(BA)$}}}{=}\chi_\rho(h)$
  	$\mathop{\oplus}_{\substack{x\in X}}$

$\mat(\rho_g)=(a_{ij}(g))_{\scriptsize \substack{1\leq i\leq d \\ 1\leq j\leq d}}$ et $\mat(\rho'_g)=(a'_{ij}(g))_{\scriptsize \substack{1\leq i'\leq d' \\ 1\leq j'\leq d'}}$



\[\int_a^b{\mathbb{R}^2}g(u, v)\dd{P_{XY}}(u, v)=\iint g(u,v) f_{XY}(u, v)\dd \lambda(u) \dd \lambda(v)\]
$$\lim_{x\to\infty} f(x)$$	
$$\iiiint_V \mu(t,u,v,w) \,dt\,du\,dv\,dw$$
$$\sum_{n=1}^{\infty} 2^{-n} = 1$$	
\begin{definition}
	Si $X$ et $Y$ sont 2 v.a. ou definit la \textsc{Covariance} entre $X$ et $Y$ comme
	$\cov(X,Y)\overset{\text{def}}{=}\E\left[(X-\E(X))(Y-\E(Y))\right]=\E(XY)-\E(X)\E(Y)$.
\end{definition}
\fi
\pagebreak

% \tableofcontents

% insert your code here
%\input{./algebra/main.tex}
%\input{./geometrie-differentielle/main.tex}
%\input{./probabilite/main.tex}
%\input{./analyse-fonctionnelle/main.tex}
% \input{./Analyse-convexe-et-dualite-en-optimisation/main.tex}
%\input{./tikz/main.tex}
%\input{./Theorie-du-distributions/main.tex}
%\input{./optimisation/mine.tex}
 \input{./modelisation/main.tex}

% yves.aubry@univ-tln.fr : algebra

\end{document}

%\input{./optimisation/mine.tex}
 % !TEX encoding = UTF-8 Unicode
% !TEX TS-program = xelatex

\documentclass[french]{report}

%\usepackage[utf8]{inputenc}
%\usepackage[T1]{fontenc}
\usepackage{babel}


\newif\ifcomment
%\commenttrue # Show comments

\usepackage{physics}
\usepackage{amssymb}


\usepackage{amsthm}
% \usepackage{thmtools}
\usepackage{mathtools}
\usepackage{amsfonts}

\usepackage{color}

\usepackage{tikz}

\usepackage{geometry}
\geometry{a5paper, margin=0.1in, right=1cm}

\usepackage{dsfont}

\usepackage{graphicx}
\graphicspath{ {images/} }

\usepackage{faktor}

\usepackage{IEEEtrantools}
\usepackage{enumerate}   
\usepackage[PostScript=dvips]{"/Users/aware/Documents/Courses/diagrams"}


\newtheorem{theorem}{Théorème}[section]
\renewcommand{\thetheorem}{\arabic{theorem}}
\newtheorem{lemme}{Lemme}[section]
\renewcommand{\thelemme}{\arabic{lemme}}
\newtheorem{proposition}{Proposition}[section]
\renewcommand{\theproposition}{\arabic{proposition}}
\newtheorem{notations}{Notations}[section]
\newtheorem{problem}{Problème}[section]
\newtheorem{corollary}{Corollaire}[theorem]
\renewcommand{\thecorollary}{\arabic{corollary}}
\newtheorem{property}{Propriété}[section]
\newtheorem{objective}{Objectif}[section]

\theoremstyle{definition}
\newtheorem{definition}{Définition}[section]
\renewcommand{\thedefinition}{\arabic{definition}}
\newtheorem{exercise}{Exercice}[chapter]
\renewcommand{\theexercise}{\arabic{exercise}}
\newtheorem{example}{Exemple}[chapter]
\renewcommand{\theexample}{\arabic{example}}
\newtheorem*{solution}{Solution}
\newtheorem*{application}{Application}
\newtheorem*{notation}{Notation}
\newtheorem*{vocabulary}{Vocabulaire}
\newtheorem*{properties}{Propriétés}



\theoremstyle{remark}
\newtheorem*{remark}{Remarque}
\newtheorem*{rappel}{Rappel}


\usepackage{etoolbox}
\AtBeginEnvironment{exercise}{\small}
\AtBeginEnvironment{example}{\small}

\usepackage{cases}
\usepackage[red]{mypack}

\usepackage[framemethod=TikZ]{mdframed}

\definecolor{bg}{rgb}{0.4,0.25,0.95}
\definecolor{pagebg}{rgb}{0,0,0.5}
\surroundwithmdframed[
   topline=false,
   rightline=false,
   bottomline=false,
   leftmargin=\parindent,
   skipabove=8pt,
   skipbelow=8pt,
   linecolor=blue,
   innerbottommargin=10pt,
   % backgroundcolor=bg,font=\color{orange}\sffamily, fontcolor=white
]{definition}

\usepackage{empheq}
\usepackage[most]{tcolorbox}

\newtcbox{\mymath}[1][]{%
    nobeforeafter, math upper, tcbox raise base,
    enhanced, colframe=blue!30!black,
    colback=red!10, boxrule=1pt,
    #1}

\usepackage{unixode}


\DeclareMathOperator{\ord}{ord}
\DeclareMathOperator{\orb}{orb}
\DeclareMathOperator{\stab}{stab}
\DeclareMathOperator{\Stab}{stab}
\DeclareMathOperator{\ppcm}{ppcm}
\DeclareMathOperator{\conj}{Conj}
\DeclareMathOperator{\End}{End}
\DeclareMathOperator{\rot}{rot}
\DeclareMathOperator{\trs}{trace}
\DeclareMathOperator{\Ind}{Ind}
\DeclareMathOperator{\mat}{Mat}
\DeclareMathOperator{\id}{Id}
\DeclareMathOperator{\vect}{vect}
\DeclareMathOperator{\img}{img}
\DeclareMathOperator{\cov}{Cov}
\DeclareMathOperator{\dist}{dist}
\DeclareMathOperator{\irr}{Irr}
\DeclareMathOperator{\image}{Im}
\DeclareMathOperator{\pd}{\partial}
\DeclareMathOperator{\epi}{epi}
\DeclareMathOperator{\Argmin}{Argmin}
\DeclareMathOperator{\dom}{dom}
\DeclareMathOperator{\proj}{proj}
\DeclareMathOperator{\ctg}{ctg}
\DeclareMathOperator{\supp}{supp}
\DeclareMathOperator{\argmin}{argmin}
\DeclareMathOperator{\mult}{mult}
\DeclareMathOperator{\ch}{ch}
\DeclareMathOperator{\sh}{sh}
\DeclareMathOperator{\rang}{rang}
\DeclareMathOperator{\diam}{diam}
\DeclareMathOperator{\Epigraphe}{Epigraphe}




\usepackage{xcolor}
\everymath{\color{blue}}
%\everymath{\color[rgb]{0,1,1}}
%\pagecolor[rgb]{0,0,0.5}


\newcommand*{\pdtest}[3][]{\ensuremath{\frac{\partial^{#1} #2}{\partial #3}}}

\newcommand*{\deffunc}[6][]{\ensuremath{
\begin{array}{rcl}
#2 : #3 &\rightarrow& #4\\
#5 &\mapsto& #6
\end{array}
}}

\newcommand{\eqcolon}{\mathrel{\resizebox{\widthof{$\mathord{=}$}}{\height}{ $\!\!=\!\!\resizebox{1.2\width}{0.8\height}{\raisebox{0.23ex}{$\mathop{:}$}}\!\!$ }}}
\newcommand{\coloneq}{\mathrel{\resizebox{\widthof{$\mathord{=}$}}{\height}{ $\!\!\resizebox{1.2\width}{0.8\height}{\raisebox{0.23ex}{$\mathop{:}$}}\!\!=\!\!$ }}}
\newcommand{\eqcolonl}{\ensuremath{\mathrel{=\!\!\mathop{:}}}}
\newcommand{\coloneql}{\ensuremath{\mathrel{\mathop{:} \!\! =}}}
\newcommand{\vc}[1]{% inline column vector
  \left(\begin{smallmatrix}#1\end{smallmatrix}\right)%
}
\newcommand{\vr}[1]{% inline row vector
  \begin{smallmatrix}(\,#1\,)\end{smallmatrix}%
}
\makeatletter
\newcommand*{\defeq}{\ =\mathrel{\rlap{%
                     \raisebox{0.3ex}{$\m@th\cdot$}}%
                     \raisebox{-0.3ex}{$\m@th\cdot$}}%
                     }
\makeatother

\newcommand{\mathcircle}[1]{% inline row vector
 \overset{\circ}{#1}
}
\newcommand{\ulim}{% low limit
 \underline{\lim}
}
\newcommand{\ssi}{% iff
\iff
}
\newcommand{\ps}[2]{
\expval{#1 | #2}
}
\newcommand{\df}[1]{
\mqty{#1}
}
\newcommand{\n}[1]{
\norm{#1}
}
\newcommand{\sys}[1]{
\left\{\smqty{#1}\right.
}


\newcommand{\eqdef}{\ensuremath{\overset{\text{def}}=}}


\def\Circlearrowright{\ensuremath{%
  \rotatebox[origin=c]{230}{$\circlearrowright$}}}

\newcommand\ct[1]{\text{\rmfamily\upshape #1}}
\newcommand\question[1]{ {\color{red} ...!? \small #1}}
\newcommand\caz[1]{\left\{\begin{array} #1 \end{array}\right.}
\newcommand\const{\text{\rmfamily\upshape const}}
\newcommand\toP{ \overset{\pro}{\to}}
\newcommand\toPP{ \overset{\text{PP}}{\to}}
\newcommand{\oeq}{\mathrel{\text{\textcircled{$=$}}}}





\usepackage{xcolor}
% \usepackage[normalem]{ulem}
\usepackage{lipsum}
\makeatletter
% \newcommand\colorwave[1][blue]{\bgroup \markoverwith{\lower3.5\p@\hbox{\sixly \textcolor{#1}{\char58}}}\ULon}
%\font\sixly=lasy6 % does not re-load if already loaded, so no memory problem.

\newmdtheoremenv[
linewidth= 1pt,linecolor= blue,%
leftmargin=20,rightmargin=20,innertopmargin=0pt, innerrightmargin=40,%
tikzsetting = { draw=lightgray, line width = 0.3pt,dashed,%
dash pattern = on 15pt off 3pt},%
splittopskip=\topskip,skipbelow=\baselineskip,%
skipabove=\baselineskip,ntheorem,roundcorner=0pt,
% backgroundcolor=pagebg,font=\color{orange}\sffamily, fontcolor=white
]{examplebox}{Exemple}[section]



\newcommand\R{\mathbb{R}}
\newcommand\Z{\mathbb{Z}}
\newcommand\N{\mathbb{N}}
\newcommand\E{\mathbb{E}}
\newcommand\F{\mathcal{F}}
\newcommand\cH{\mathcal{H}}
\newcommand\V{\mathbb{V}}
\newcommand\dmo{ ^{-1} }
\newcommand\kapa{\kappa}
\newcommand\im{Im}
\newcommand\hs{\mathcal{H}}





\usepackage{soul}

\makeatletter
\newcommand*{\whiten}[1]{\llap{\textcolor{white}{{\the\SOUL@token}}\hspace{#1pt}}}
\DeclareRobustCommand*\myul{%
    \def\SOUL@everyspace{\underline{\space}\kern\z@}%
    \def\SOUL@everytoken{%
     \setbox0=\hbox{\the\SOUL@token}%
     \ifdim\dp0>\z@
        \raisebox{\dp0}{\underline{\phantom{\the\SOUL@token}}}%
        \whiten{1}\whiten{0}%
        \whiten{-1}\whiten{-2}%
        \llap{\the\SOUL@token}%
     \else
        \underline{\the\SOUL@token}%
     \fi}%
\SOUL@}
\makeatother

\newcommand*{\demp}{\fontfamily{lmtt}\selectfont}

\DeclareTextFontCommand{\textdemp}{\demp}

\begin{document}

\ifcomment
Multiline
comment
\fi
\ifcomment
\myul{Typesetting test}
% \color[rgb]{1,1,1}
$∑_i^n≠ 60º±∞π∆¬≈√j∫h≤≥µ$

$\CR \R\pro\ind\pro\gS\pro
\mqty[a&b\\c&d]$
$\pro\mathbb{P}$
$\dd{x}$

  \[
    \alpha(x)=\left\{
                \begin{array}{ll}
                  x\\
                  \frac{1}{1+e^{-kx}}\\
                  \frac{e^x-e^{-x}}{e^x+e^{-x}}
                \end{array}
              \right.
  \]

  $\expval{x}$
  
  $\chi_\rho(ghg\dmo)=\Tr(\rho_{ghg\dmo})=\Tr(\rho_g\circ\rho_h\circ\rho\dmo_g)=\Tr(\rho_h)\overset{\mbox{\scalebox{0.5}{$\Tr(AB)=\Tr(BA)$}}}{=}\chi_\rho(h)$
  	$\mathop{\oplus}_{\substack{x\in X}}$

$\mat(\rho_g)=(a_{ij}(g))_{\scriptsize \substack{1\leq i\leq d \\ 1\leq j\leq d}}$ et $\mat(\rho'_g)=(a'_{ij}(g))_{\scriptsize \substack{1\leq i'\leq d' \\ 1\leq j'\leq d'}}$



\[\int_a^b{\mathbb{R}^2}g(u, v)\dd{P_{XY}}(u, v)=\iint g(u,v) f_{XY}(u, v)\dd \lambda(u) \dd \lambda(v)\]
$$\lim_{x\to\infty} f(x)$$	
$$\iiiint_V \mu(t,u,v,w) \,dt\,du\,dv\,dw$$
$$\sum_{n=1}^{\infty} 2^{-n} = 1$$	
\begin{definition}
	Si $X$ et $Y$ sont 2 v.a. ou definit la \textsc{Covariance} entre $X$ et $Y$ comme
	$\cov(X,Y)\overset{\text{def}}{=}\E\left[(X-\E(X))(Y-\E(Y))\right]=\E(XY)-\E(X)\E(Y)$.
\end{definition}
\fi
\pagebreak

% \tableofcontents

% insert your code here
%\input{./algebra/main.tex}
%\input{./geometrie-differentielle/main.tex}
%\input{./probabilite/main.tex}
%\input{./analyse-fonctionnelle/main.tex}
% \input{./Analyse-convexe-et-dualite-en-optimisation/main.tex}
%\input{./tikz/main.tex}
%\input{./Theorie-du-distributions/main.tex}
%\input{./optimisation/mine.tex}
 \input{./modelisation/main.tex}

% yves.aubry@univ-tln.fr : algebra

\end{document}


% yves.aubry@univ-tln.fr : algebra

\end{document}

%% !TEX encoding = UTF-8 Unicode
% !TEX TS-program = xelatex

\documentclass[french]{report}

%\usepackage[utf8]{inputenc}
%\usepackage[T1]{fontenc}
\usepackage{babel}


\newif\ifcomment
%\commenttrue # Show comments

\usepackage{physics}
\usepackage{amssymb}


\usepackage{amsthm}
% \usepackage{thmtools}
\usepackage{mathtools}
\usepackage{amsfonts}

\usepackage{color}

\usepackage{tikz}

\usepackage{geometry}
\geometry{a5paper, margin=0.1in, right=1cm}

\usepackage{dsfont}

\usepackage{graphicx}
\graphicspath{ {images/} }

\usepackage{faktor}

\usepackage{IEEEtrantools}
\usepackage{enumerate}   
\usepackage[PostScript=dvips]{"/Users/aware/Documents/Courses/diagrams"}


\newtheorem{theorem}{Théorème}[section]
\renewcommand{\thetheorem}{\arabic{theorem}}
\newtheorem{lemme}{Lemme}[section]
\renewcommand{\thelemme}{\arabic{lemme}}
\newtheorem{proposition}{Proposition}[section]
\renewcommand{\theproposition}{\arabic{proposition}}
\newtheorem{notations}{Notations}[section]
\newtheorem{problem}{Problème}[section]
\newtheorem{corollary}{Corollaire}[theorem]
\renewcommand{\thecorollary}{\arabic{corollary}}
\newtheorem{property}{Propriété}[section]
\newtheorem{objective}{Objectif}[section]

\theoremstyle{definition}
\newtheorem{definition}{Définition}[section]
\renewcommand{\thedefinition}{\arabic{definition}}
\newtheorem{exercise}{Exercice}[chapter]
\renewcommand{\theexercise}{\arabic{exercise}}
\newtheorem{example}{Exemple}[chapter]
\renewcommand{\theexample}{\arabic{example}}
\newtheorem*{solution}{Solution}
\newtheorem*{application}{Application}
\newtheorem*{notation}{Notation}
\newtheorem*{vocabulary}{Vocabulaire}
\newtheorem*{properties}{Propriétés}



\theoremstyle{remark}
\newtheorem*{remark}{Remarque}
\newtheorem*{rappel}{Rappel}


\usepackage{etoolbox}
\AtBeginEnvironment{exercise}{\small}
\AtBeginEnvironment{example}{\small}

\usepackage{cases}
\usepackage[red]{mypack}

\usepackage[framemethod=TikZ]{mdframed}

\definecolor{bg}{rgb}{0.4,0.25,0.95}
\definecolor{pagebg}{rgb}{0,0,0.5}
\surroundwithmdframed[
   topline=false,
   rightline=false,
   bottomline=false,
   leftmargin=\parindent,
   skipabove=8pt,
   skipbelow=8pt,
   linecolor=blue,
   innerbottommargin=10pt,
   % backgroundcolor=bg,font=\color{orange}\sffamily, fontcolor=white
]{definition}

\usepackage{empheq}
\usepackage[most]{tcolorbox}

\newtcbox{\mymath}[1][]{%
    nobeforeafter, math upper, tcbox raise base,
    enhanced, colframe=blue!30!black,
    colback=red!10, boxrule=1pt,
    #1}

\usepackage{unixode}


\DeclareMathOperator{\ord}{ord}
\DeclareMathOperator{\orb}{orb}
\DeclareMathOperator{\stab}{stab}
\DeclareMathOperator{\Stab}{stab}
\DeclareMathOperator{\ppcm}{ppcm}
\DeclareMathOperator{\conj}{Conj}
\DeclareMathOperator{\End}{End}
\DeclareMathOperator{\rot}{rot}
\DeclareMathOperator{\trs}{trace}
\DeclareMathOperator{\Ind}{Ind}
\DeclareMathOperator{\mat}{Mat}
\DeclareMathOperator{\id}{Id}
\DeclareMathOperator{\vect}{vect}
\DeclareMathOperator{\img}{img}
\DeclareMathOperator{\cov}{Cov}
\DeclareMathOperator{\dist}{dist}
\DeclareMathOperator{\irr}{Irr}
\DeclareMathOperator{\image}{Im}
\DeclareMathOperator{\pd}{\partial}
\DeclareMathOperator{\epi}{epi}
\DeclareMathOperator{\Argmin}{Argmin}
\DeclareMathOperator{\dom}{dom}
\DeclareMathOperator{\proj}{proj}
\DeclareMathOperator{\ctg}{ctg}
\DeclareMathOperator{\supp}{supp}
\DeclareMathOperator{\argmin}{argmin}
\DeclareMathOperator{\mult}{mult}
\DeclareMathOperator{\ch}{ch}
\DeclareMathOperator{\sh}{sh}
\DeclareMathOperator{\rang}{rang}
\DeclareMathOperator{\diam}{diam}
\DeclareMathOperator{\Epigraphe}{Epigraphe}




\usepackage{xcolor}
\everymath{\color{blue}}
%\everymath{\color[rgb]{0,1,1}}
%\pagecolor[rgb]{0,0,0.5}


\newcommand*{\pdtest}[3][]{\ensuremath{\frac{\partial^{#1} #2}{\partial #3}}}

\newcommand*{\deffunc}[6][]{\ensuremath{
\begin{array}{rcl}
#2 : #3 &\rightarrow& #4\\
#5 &\mapsto& #6
\end{array}
}}

\newcommand{\eqcolon}{\mathrel{\resizebox{\widthof{$\mathord{=}$}}{\height}{ $\!\!=\!\!\resizebox{1.2\width}{0.8\height}{\raisebox{0.23ex}{$\mathop{:}$}}\!\!$ }}}
\newcommand{\coloneq}{\mathrel{\resizebox{\widthof{$\mathord{=}$}}{\height}{ $\!\!\resizebox{1.2\width}{0.8\height}{\raisebox{0.23ex}{$\mathop{:}$}}\!\!=\!\!$ }}}
\newcommand{\eqcolonl}{\ensuremath{\mathrel{=\!\!\mathop{:}}}}
\newcommand{\coloneql}{\ensuremath{\mathrel{\mathop{:} \!\! =}}}
\newcommand{\vc}[1]{% inline column vector
  \left(\begin{smallmatrix}#1\end{smallmatrix}\right)%
}
\newcommand{\vr}[1]{% inline row vector
  \begin{smallmatrix}(\,#1\,)\end{smallmatrix}%
}
\makeatletter
\newcommand*{\defeq}{\ =\mathrel{\rlap{%
                     \raisebox{0.3ex}{$\m@th\cdot$}}%
                     \raisebox{-0.3ex}{$\m@th\cdot$}}%
                     }
\makeatother

\newcommand{\mathcircle}[1]{% inline row vector
 \overset{\circ}{#1}
}
\newcommand{\ulim}{% low limit
 \underline{\lim}
}
\newcommand{\ssi}{% iff
\iff
}
\newcommand{\ps}[2]{
\expval{#1 | #2}
}
\newcommand{\df}[1]{
\mqty{#1}
}
\newcommand{\n}[1]{
\norm{#1}
}
\newcommand{\sys}[1]{
\left\{\smqty{#1}\right.
}


\newcommand{\eqdef}{\ensuremath{\overset{\text{def}}=}}


\def\Circlearrowright{\ensuremath{%
  \rotatebox[origin=c]{230}{$\circlearrowright$}}}

\newcommand\ct[1]{\text{\rmfamily\upshape #1}}
\newcommand\question[1]{ {\color{red} ...!? \small #1}}
\newcommand\caz[1]{\left\{\begin{array} #1 \end{array}\right.}
\newcommand\const{\text{\rmfamily\upshape const}}
\newcommand\toP{ \overset{\pro}{\to}}
\newcommand\toPP{ \overset{\text{PP}}{\to}}
\newcommand{\oeq}{\mathrel{\text{\textcircled{$=$}}}}





\usepackage{xcolor}
% \usepackage[normalem]{ulem}
\usepackage{lipsum}
\makeatletter
% \newcommand\colorwave[1][blue]{\bgroup \markoverwith{\lower3.5\p@\hbox{\sixly \textcolor{#1}{\char58}}}\ULon}
%\font\sixly=lasy6 % does not re-load if already loaded, so no memory problem.

\newmdtheoremenv[
linewidth= 1pt,linecolor= blue,%
leftmargin=20,rightmargin=20,innertopmargin=0pt, innerrightmargin=40,%
tikzsetting = { draw=lightgray, line width = 0.3pt,dashed,%
dash pattern = on 15pt off 3pt},%
splittopskip=\topskip,skipbelow=\baselineskip,%
skipabove=\baselineskip,ntheorem,roundcorner=0pt,
% backgroundcolor=pagebg,font=\color{orange}\sffamily, fontcolor=white
]{examplebox}{Exemple}[section]



\newcommand\R{\mathbb{R}}
\newcommand\Z{\mathbb{Z}}
\newcommand\N{\mathbb{N}}
\newcommand\E{\mathbb{E}}
\newcommand\F{\mathcal{F}}
\newcommand\cH{\mathcal{H}}
\newcommand\V{\mathbb{V}}
\newcommand\dmo{ ^{-1} }
\newcommand\kapa{\kappa}
\newcommand\im{Im}
\newcommand\hs{\mathcal{H}}





\usepackage{soul}

\makeatletter
\newcommand*{\whiten}[1]{\llap{\textcolor{white}{{\the\SOUL@token}}\hspace{#1pt}}}
\DeclareRobustCommand*\myul{%
    \def\SOUL@everyspace{\underline{\space}\kern\z@}%
    \def\SOUL@everytoken{%
     \setbox0=\hbox{\the\SOUL@token}%
     \ifdim\dp0>\z@
        \raisebox{\dp0}{\underline{\phantom{\the\SOUL@token}}}%
        \whiten{1}\whiten{0}%
        \whiten{-1}\whiten{-2}%
        \llap{\the\SOUL@token}%
     \else
        \underline{\the\SOUL@token}%
     \fi}%
\SOUL@}
\makeatother

\newcommand*{\demp}{\fontfamily{lmtt}\selectfont}

\DeclareTextFontCommand{\textdemp}{\demp}

\begin{document}

\ifcomment
Multiline
comment
\fi
\ifcomment
\myul{Typesetting test}
% \color[rgb]{1,1,1}
$∑_i^n≠ 60º±∞π∆¬≈√j∫h≤≥µ$

$\CR \R\pro\ind\pro\gS\pro
\mqty[a&b\\c&d]$
$\pro\mathbb{P}$
$\dd{x}$

  \[
    \alpha(x)=\left\{
                \begin{array}{ll}
                  x\\
                  \frac{1}{1+e^{-kx}}\\
                  \frac{e^x-e^{-x}}{e^x+e^{-x}}
                \end{array}
              \right.
  \]

  $\expval{x}$
  
  $\chi_\rho(ghg\dmo)=\Tr(\rho_{ghg\dmo})=\Tr(\rho_g\circ\rho_h\circ\rho\dmo_g)=\Tr(\rho_h)\overset{\mbox{\scalebox{0.5}{$\Tr(AB)=\Tr(BA)$}}}{=}\chi_\rho(h)$
  	$\mathop{\oplus}_{\substack{x\in X}}$

$\mat(\rho_g)=(a_{ij}(g))_{\scriptsize \substack{1\leq i\leq d \\ 1\leq j\leq d}}$ et $\mat(\rho'_g)=(a'_{ij}(g))_{\scriptsize \substack{1\leq i'\leq d' \\ 1\leq j'\leq d'}}$



\[\int_a^b{\mathbb{R}^2}g(u, v)\dd{P_{XY}}(u, v)=\iint g(u,v) f_{XY}(u, v)\dd \lambda(u) \dd \lambda(v)\]
$$\lim_{x\to\infty} f(x)$$	
$$\iiiint_V \mu(t,u,v,w) \,dt\,du\,dv\,dw$$
$$\sum_{n=1}^{\infty} 2^{-n} = 1$$	
\begin{definition}
	Si $X$ et $Y$ sont 2 v.a. ou definit la \textsc{Covariance} entre $X$ et $Y$ comme
	$\cov(X,Y)\overset{\text{def}}{=}\E\left[(X-\E(X))(Y-\E(Y))\right]=\E(XY)-\E(X)\E(Y)$.
\end{definition}
\fi
\pagebreak

% \tableofcontents

% insert your code here
%% !TEX encoding = UTF-8 Unicode
% !TEX TS-program = xelatex

\documentclass[french]{report}

%\usepackage[utf8]{inputenc}
%\usepackage[T1]{fontenc}
\usepackage{babel}


\newif\ifcomment
%\commenttrue # Show comments

\usepackage{physics}
\usepackage{amssymb}


\usepackage{amsthm}
% \usepackage{thmtools}
\usepackage{mathtools}
\usepackage{amsfonts}

\usepackage{color}

\usepackage{tikz}

\usepackage{geometry}
\geometry{a5paper, margin=0.1in, right=1cm}

\usepackage{dsfont}

\usepackage{graphicx}
\graphicspath{ {images/} }

\usepackage{faktor}

\usepackage{IEEEtrantools}
\usepackage{enumerate}   
\usepackage[PostScript=dvips]{"/Users/aware/Documents/Courses/diagrams"}


\newtheorem{theorem}{Théorème}[section]
\renewcommand{\thetheorem}{\arabic{theorem}}
\newtheorem{lemme}{Lemme}[section]
\renewcommand{\thelemme}{\arabic{lemme}}
\newtheorem{proposition}{Proposition}[section]
\renewcommand{\theproposition}{\arabic{proposition}}
\newtheorem{notations}{Notations}[section]
\newtheorem{problem}{Problème}[section]
\newtheorem{corollary}{Corollaire}[theorem]
\renewcommand{\thecorollary}{\arabic{corollary}}
\newtheorem{property}{Propriété}[section]
\newtheorem{objective}{Objectif}[section]

\theoremstyle{definition}
\newtheorem{definition}{Définition}[section]
\renewcommand{\thedefinition}{\arabic{definition}}
\newtheorem{exercise}{Exercice}[chapter]
\renewcommand{\theexercise}{\arabic{exercise}}
\newtheorem{example}{Exemple}[chapter]
\renewcommand{\theexample}{\arabic{example}}
\newtheorem*{solution}{Solution}
\newtheorem*{application}{Application}
\newtheorem*{notation}{Notation}
\newtheorem*{vocabulary}{Vocabulaire}
\newtheorem*{properties}{Propriétés}



\theoremstyle{remark}
\newtheorem*{remark}{Remarque}
\newtheorem*{rappel}{Rappel}


\usepackage{etoolbox}
\AtBeginEnvironment{exercise}{\small}
\AtBeginEnvironment{example}{\small}

\usepackage{cases}
\usepackage[red]{mypack}

\usepackage[framemethod=TikZ]{mdframed}

\definecolor{bg}{rgb}{0.4,0.25,0.95}
\definecolor{pagebg}{rgb}{0,0,0.5}
\surroundwithmdframed[
   topline=false,
   rightline=false,
   bottomline=false,
   leftmargin=\parindent,
   skipabove=8pt,
   skipbelow=8pt,
   linecolor=blue,
   innerbottommargin=10pt,
   % backgroundcolor=bg,font=\color{orange}\sffamily, fontcolor=white
]{definition}

\usepackage{empheq}
\usepackage[most]{tcolorbox}

\newtcbox{\mymath}[1][]{%
    nobeforeafter, math upper, tcbox raise base,
    enhanced, colframe=blue!30!black,
    colback=red!10, boxrule=1pt,
    #1}

\usepackage{unixode}


\DeclareMathOperator{\ord}{ord}
\DeclareMathOperator{\orb}{orb}
\DeclareMathOperator{\stab}{stab}
\DeclareMathOperator{\Stab}{stab}
\DeclareMathOperator{\ppcm}{ppcm}
\DeclareMathOperator{\conj}{Conj}
\DeclareMathOperator{\End}{End}
\DeclareMathOperator{\rot}{rot}
\DeclareMathOperator{\trs}{trace}
\DeclareMathOperator{\Ind}{Ind}
\DeclareMathOperator{\mat}{Mat}
\DeclareMathOperator{\id}{Id}
\DeclareMathOperator{\vect}{vect}
\DeclareMathOperator{\img}{img}
\DeclareMathOperator{\cov}{Cov}
\DeclareMathOperator{\dist}{dist}
\DeclareMathOperator{\irr}{Irr}
\DeclareMathOperator{\image}{Im}
\DeclareMathOperator{\pd}{\partial}
\DeclareMathOperator{\epi}{epi}
\DeclareMathOperator{\Argmin}{Argmin}
\DeclareMathOperator{\dom}{dom}
\DeclareMathOperator{\proj}{proj}
\DeclareMathOperator{\ctg}{ctg}
\DeclareMathOperator{\supp}{supp}
\DeclareMathOperator{\argmin}{argmin}
\DeclareMathOperator{\mult}{mult}
\DeclareMathOperator{\ch}{ch}
\DeclareMathOperator{\sh}{sh}
\DeclareMathOperator{\rang}{rang}
\DeclareMathOperator{\diam}{diam}
\DeclareMathOperator{\Epigraphe}{Epigraphe}




\usepackage{xcolor}
\everymath{\color{blue}}
%\everymath{\color[rgb]{0,1,1}}
%\pagecolor[rgb]{0,0,0.5}


\newcommand*{\pdtest}[3][]{\ensuremath{\frac{\partial^{#1} #2}{\partial #3}}}

\newcommand*{\deffunc}[6][]{\ensuremath{
\begin{array}{rcl}
#2 : #3 &\rightarrow& #4\\
#5 &\mapsto& #6
\end{array}
}}

\newcommand{\eqcolon}{\mathrel{\resizebox{\widthof{$\mathord{=}$}}{\height}{ $\!\!=\!\!\resizebox{1.2\width}{0.8\height}{\raisebox{0.23ex}{$\mathop{:}$}}\!\!$ }}}
\newcommand{\coloneq}{\mathrel{\resizebox{\widthof{$\mathord{=}$}}{\height}{ $\!\!\resizebox{1.2\width}{0.8\height}{\raisebox{0.23ex}{$\mathop{:}$}}\!\!=\!\!$ }}}
\newcommand{\eqcolonl}{\ensuremath{\mathrel{=\!\!\mathop{:}}}}
\newcommand{\coloneql}{\ensuremath{\mathrel{\mathop{:} \!\! =}}}
\newcommand{\vc}[1]{% inline column vector
  \left(\begin{smallmatrix}#1\end{smallmatrix}\right)%
}
\newcommand{\vr}[1]{% inline row vector
  \begin{smallmatrix}(\,#1\,)\end{smallmatrix}%
}
\makeatletter
\newcommand*{\defeq}{\ =\mathrel{\rlap{%
                     \raisebox{0.3ex}{$\m@th\cdot$}}%
                     \raisebox{-0.3ex}{$\m@th\cdot$}}%
                     }
\makeatother

\newcommand{\mathcircle}[1]{% inline row vector
 \overset{\circ}{#1}
}
\newcommand{\ulim}{% low limit
 \underline{\lim}
}
\newcommand{\ssi}{% iff
\iff
}
\newcommand{\ps}[2]{
\expval{#1 | #2}
}
\newcommand{\df}[1]{
\mqty{#1}
}
\newcommand{\n}[1]{
\norm{#1}
}
\newcommand{\sys}[1]{
\left\{\smqty{#1}\right.
}


\newcommand{\eqdef}{\ensuremath{\overset{\text{def}}=}}


\def\Circlearrowright{\ensuremath{%
  \rotatebox[origin=c]{230}{$\circlearrowright$}}}

\newcommand\ct[1]{\text{\rmfamily\upshape #1}}
\newcommand\question[1]{ {\color{red} ...!? \small #1}}
\newcommand\caz[1]{\left\{\begin{array} #1 \end{array}\right.}
\newcommand\const{\text{\rmfamily\upshape const}}
\newcommand\toP{ \overset{\pro}{\to}}
\newcommand\toPP{ \overset{\text{PP}}{\to}}
\newcommand{\oeq}{\mathrel{\text{\textcircled{$=$}}}}





\usepackage{xcolor}
% \usepackage[normalem]{ulem}
\usepackage{lipsum}
\makeatletter
% \newcommand\colorwave[1][blue]{\bgroup \markoverwith{\lower3.5\p@\hbox{\sixly \textcolor{#1}{\char58}}}\ULon}
%\font\sixly=lasy6 % does not re-load if already loaded, so no memory problem.

\newmdtheoremenv[
linewidth= 1pt,linecolor= blue,%
leftmargin=20,rightmargin=20,innertopmargin=0pt, innerrightmargin=40,%
tikzsetting = { draw=lightgray, line width = 0.3pt,dashed,%
dash pattern = on 15pt off 3pt},%
splittopskip=\topskip,skipbelow=\baselineskip,%
skipabove=\baselineskip,ntheorem,roundcorner=0pt,
% backgroundcolor=pagebg,font=\color{orange}\sffamily, fontcolor=white
]{examplebox}{Exemple}[section]



\newcommand\R{\mathbb{R}}
\newcommand\Z{\mathbb{Z}}
\newcommand\N{\mathbb{N}}
\newcommand\E{\mathbb{E}}
\newcommand\F{\mathcal{F}}
\newcommand\cH{\mathcal{H}}
\newcommand\V{\mathbb{V}}
\newcommand\dmo{ ^{-1} }
\newcommand\kapa{\kappa}
\newcommand\im{Im}
\newcommand\hs{\mathcal{H}}





\usepackage{soul}

\makeatletter
\newcommand*{\whiten}[1]{\llap{\textcolor{white}{{\the\SOUL@token}}\hspace{#1pt}}}
\DeclareRobustCommand*\myul{%
    \def\SOUL@everyspace{\underline{\space}\kern\z@}%
    \def\SOUL@everytoken{%
     \setbox0=\hbox{\the\SOUL@token}%
     \ifdim\dp0>\z@
        \raisebox{\dp0}{\underline{\phantom{\the\SOUL@token}}}%
        \whiten{1}\whiten{0}%
        \whiten{-1}\whiten{-2}%
        \llap{\the\SOUL@token}%
     \else
        \underline{\the\SOUL@token}%
     \fi}%
\SOUL@}
\makeatother

\newcommand*{\demp}{\fontfamily{lmtt}\selectfont}

\DeclareTextFontCommand{\textdemp}{\demp}

\begin{document}

\ifcomment
Multiline
comment
\fi
\ifcomment
\myul{Typesetting test}
% \color[rgb]{1,1,1}
$∑_i^n≠ 60º±∞π∆¬≈√j∫h≤≥µ$

$\CR \R\pro\ind\pro\gS\pro
\mqty[a&b\\c&d]$
$\pro\mathbb{P}$
$\dd{x}$

  \[
    \alpha(x)=\left\{
                \begin{array}{ll}
                  x\\
                  \frac{1}{1+e^{-kx}}\\
                  \frac{e^x-e^{-x}}{e^x+e^{-x}}
                \end{array}
              \right.
  \]

  $\expval{x}$
  
  $\chi_\rho(ghg\dmo)=\Tr(\rho_{ghg\dmo})=\Tr(\rho_g\circ\rho_h\circ\rho\dmo_g)=\Tr(\rho_h)\overset{\mbox{\scalebox{0.5}{$\Tr(AB)=\Tr(BA)$}}}{=}\chi_\rho(h)$
  	$\mathop{\oplus}_{\substack{x\in X}}$

$\mat(\rho_g)=(a_{ij}(g))_{\scriptsize \substack{1\leq i\leq d \\ 1\leq j\leq d}}$ et $\mat(\rho'_g)=(a'_{ij}(g))_{\scriptsize \substack{1\leq i'\leq d' \\ 1\leq j'\leq d'}}$



\[\int_a^b{\mathbb{R}^2}g(u, v)\dd{P_{XY}}(u, v)=\iint g(u,v) f_{XY}(u, v)\dd \lambda(u) \dd \lambda(v)\]
$$\lim_{x\to\infty} f(x)$$	
$$\iiiint_V \mu(t,u,v,w) \,dt\,du\,dv\,dw$$
$$\sum_{n=1}^{\infty} 2^{-n} = 1$$	
\begin{definition}
	Si $X$ et $Y$ sont 2 v.a. ou definit la \textsc{Covariance} entre $X$ et $Y$ comme
	$\cov(X,Y)\overset{\text{def}}{=}\E\left[(X-\E(X))(Y-\E(Y))\right]=\E(XY)-\E(X)\E(Y)$.
\end{definition}
\fi
\pagebreak

% \tableofcontents

% insert your code here
%\input{./algebra/main.tex}
%\input{./geometrie-differentielle/main.tex}
%\input{./probabilite/main.tex}
%\input{./analyse-fonctionnelle/main.tex}
% \input{./Analyse-convexe-et-dualite-en-optimisation/main.tex}
%\input{./tikz/main.tex}
%\input{./Theorie-du-distributions/main.tex}
%\input{./optimisation/mine.tex}
 \input{./modelisation/main.tex}

% yves.aubry@univ-tln.fr : algebra

\end{document}

%% !TEX encoding = UTF-8 Unicode
% !TEX TS-program = xelatex

\documentclass[french]{report}

%\usepackage[utf8]{inputenc}
%\usepackage[T1]{fontenc}
\usepackage{babel}


\newif\ifcomment
%\commenttrue # Show comments

\usepackage{physics}
\usepackage{amssymb}


\usepackage{amsthm}
% \usepackage{thmtools}
\usepackage{mathtools}
\usepackage{amsfonts}

\usepackage{color}

\usepackage{tikz}

\usepackage{geometry}
\geometry{a5paper, margin=0.1in, right=1cm}

\usepackage{dsfont}

\usepackage{graphicx}
\graphicspath{ {images/} }

\usepackage{faktor}

\usepackage{IEEEtrantools}
\usepackage{enumerate}   
\usepackage[PostScript=dvips]{"/Users/aware/Documents/Courses/diagrams"}


\newtheorem{theorem}{Théorème}[section]
\renewcommand{\thetheorem}{\arabic{theorem}}
\newtheorem{lemme}{Lemme}[section]
\renewcommand{\thelemme}{\arabic{lemme}}
\newtheorem{proposition}{Proposition}[section]
\renewcommand{\theproposition}{\arabic{proposition}}
\newtheorem{notations}{Notations}[section]
\newtheorem{problem}{Problème}[section]
\newtheorem{corollary}{Corollaire}[theorem]
\renewcommand{\thecorollary}{\arabic{corollary}}
\newtheorem{property}{Propriété}[section]
\newtheorem{objective}{Objectif}[section]

\theoremstyle{definition}
\newtheorem{definition}{Définition}[section]
\renewcommand{\thedefinition}{\arabic{definition}}
\newtheorem{exercise}{Exercice}[chapter]
\renewcommand{\theexercise}{\arabic{exercise}}
\newtheorem{example}{Exemple}[chapter]
\renewcommand{\theexample}{\arabic{example}}
\newtheorem*{solution}{Solution}
\newtheorem*{application}{Application}
\newtheorem*{notation}{Notation}
\newtheorem*{vocabulary}{Vocabulaire}
\newtheorem*{properties}{Propriétés}



\theoremstyle{remark}
\newtheorem*{remark}{Remarque}
\newtheorem*{rappel}{Rappel}


\usepackage{etoolbox}
\AtBeginEnvironment{exercise}{\small}
\AtBeginEnvironment{example}{\small}

\usepackage{cases}
\usepackage[red]{mypack}

\usepackage[framemethod=TikZ]{mdframed}

\definecolor{bg}{rgb}{0.4,0.25,0.95}
\definecolor{pagebg}{rgb}{0,0,0.5}
\surroundwithmdframed[
   topline=false,
   rightline=false,
   bottomline=false,
   leftmargin=\parindent,
   skipabove=8pt,
   skipbelow=8pt,
   linecolor=blue,
   innerbottommargin=10pt,
   % backgroundcolor=bg,font=\color{orange}\sffamily, fontcolor=white
]{definition}

\usepackage{empheq}
\usepackage[most]{tcolorbox}

\newtcbox{\mymath}[1][]{%
    nobeforeafter, math upper, tcbox raise base,
    enhanced, colframe=blue!30!black,
    colback=red!10, boxrule=1pt,
    #1}

\usepackage{unixode}


\DeclareMathOperator{\ord}{ord}
\DeclareMathOperator{\orb}{orb}
\DeclareMathOperator{\stab}{stab}
\DeclareMathOperator{\Stab}{stab}
\DeclareMathOperator{\ppcm}{ppcm}
\DeclareMathOperator{\conj}{Conj}
\DeclareMathOperator{\End}{End}
\DeclareMathOperator{\rot}{rot}
\DeclareMathOperator{\trs}{trace}
\DeclareMathOperator{\Ind}{Ind}
\DeclareMathOperator{\mat}{Mat}
\DeclareMathOperator{\id}{Id}
\DeclareMathOperator{\vect}{vect}
\DeclareMathOperator{\img}{img}
\DeclareMathOperator{\cov}{Cov}
\DeclareMathOperator{\dist}{dist}
\DeclareMathOperator{\irr}{Irr}
\DeclareMathOperator{\image}{Im}
\DeclareMathOperator{\pd}{\partial}
\DeclareMathOperator{\epi}{epi}
\DeclareMathOperator{\Argmin}{Argmin}
\DeclareMathOperator{\dom}{dom}
\DeclareMathOperator{\proj}{proj}
\DeclareMathOperator{\ctg}{ctg}
\DeclareMathOperator{\supp}{supp}
\DeclareMathOperator{\argmin}{argmin}
\DeclareMathOperator{\mult}{mult}
\DeclareMathOperator{\ch}{ch}
\DeclareMathOperator{\sh}{sh}
\DeclareMathOperator{\rang}{rang}
\DeclareMathOperator{\diam}{diam}
\DeclareMathOperator{\Epigraphe}{Epigraphe}




\usepackage{xcolor}
\everymath{\color{blue}}
%\everymath{\color[rgb]{0,1,1}}
%\pagecolor[rgb]{0,0,0.5}


\newcommand*{\pdtest}[3][]{\ensuremath{\frac{\partial^{#1} #2}{\partial #3}}}

\newcommand*{\deffunc}[6][]{\ensuremath{
\begin{array}{rcl}
#2 : #3 &\rightarrow& #4\\
#5 &\mapsto& #6
\end{array}
}}

\newcommand{\eqcolon}{\mathrel{\resizebox{\widthof{$\mathord{=}$}}{\height}{ $\!\!=\!\!\resizebox{1.2\width}{0.8\height}{\raisebox{0.23ex}{$\mathop{:}$}}\!\!$ }}}
\newcommand{\coloneq}{\mathrel{\resizebox{\widthof{$\mathord{=}$}}{\height}{ $\!\!\resizebox{1.2\width}{0.8\height}{\raisebox{0.23ex}{$\mathop{:}$}}\!\!=\!\!$ }}}
\newcommand{\eqcolonl}{\ensuremath{\mathrel{=\!\!\mathop{:}}}}
\newcommand{\coloneql}{\ensuremath{\mathrel{\mathop{:} \!\! =}}}
\newcommand{\vc}[1]{% inline column vector
  \left(\begin{smallmatrix}#1\end{smallmatrix}\right)%
}
\newcommand{\vr}[1]{% inline row vector
  \begin{smallmatrix}(\,#1\,)\end{smallmatrix}%
}
\makeatletter
\newcommand*{\defeq}{\ =\mathrel{\rlap{%
                     \raisebox{0.3ex}{$\m@th\cdot$}}%
                     \raisebox{-0.3ex}{$\m@th\cdot$}}%
                     }
\makeatother

\newcommand{\mathcircle}[1]{% inline row vector
 \overset{\circ}{#1}
}
\newcommand{\ulim}{% low limit
 \underline{\lim}
}
\newcommand{\ssi}{% iff
\iff
}
\newcommand{\ps}[2]{
\expval{#1 | #2}
}
\newcommand{\df}[1]{
\mqty{#1}
}
\newcommand{\n}[1]{
\norm{#1}
}
\newcommand{\sys}[1]{
\left\{\smqty{#1}\right.
}


\newcommand{\eqdef}{\ensuremath{\overset{\text{def}}=}}


\def\Circlearrowright{\ensuremath{%
  \rotatebox[origin=c]{230}{$\circlearrowright$}}}

\newcommand\ct[1]{\text{\rmfamily\upshape #1}}
\newcommand\question[1]{ {\color{red} ...!? \small #1}}
\newcommand\caz[1]{\left\{\begin{array} #1 \end{array}\right.}
\newcommand\const{\text{\rmfamily\upshape const}}
\newcommand\toP{ \overset{\pro}{\to}}
\newcommand\toPP{ \overset{\text{PP}}{\to}}
\newcommand{\oeq}{\mathrel{\text{\textcircled{$=$}}}}





\usepackage{xcolor}
% \usepackage[normalem]{ulem}
\usepackage{lipsum}
\makeatletter
% \newcommand\colorwave[1][blue]{\bgroup \markoverwith{\lower3.5\p@\hbox{\sixly \textcolor{#1}{\char58}}}\ULon}
%\font\sixly=lasy6 % does not re-load if already loaded, so no memory problem.

\newmdtheoremenv[
linewidth= 1pt,linecolor= blue,%
leftmargin=20,rightmargin=20,innertopmargin=0pt, innerrightmargin=40,%
tikzsetting = { draw=lightgray, line width = 0.3pt,dashed,%
dash pattern = on 15pt off 3pt},%
splittopskip=\topskip,skipbelow=\baselineskip,%
skipabove=\baselineskip,ntheorem,roundcorner=0pt,
% backgroundcolor=pagebg,font=\color{orange}\sffamily, fontcolor=white
]{examplebox}{Exemple}[section]



\newcommand\R{\mathbb{R}}
\newcommand\Z{\mathbb{Z}}
\newcommand\N{\mathbb{N}}
\newcommand\E{\mathbb{E}}
\newcommand\F{\mathcal{F}}
\newcommand\cH{\mathcal{H}}
\newcommand\V{\mathbb{V}}
\newcommand\dmo{ ^{-1} }
\newcommand\kapa{\kappa}
\newcommand\im{Im}
\newcommand\hs{\mathcal{H}}





\usepackage{soul}

\makeatletter
\newcommand*{\whiten}[1]{\llap{\textcolor{white}{{\the\SOUL@token}}\hspace{#1pt}}}
\DeclareRobustCommand*\myul{%
    \def\SOUL@everyspace{\underline{\space}\kern\z@}%
    \def\SOUL@everytoken{%
     \setbox0=\hbox{\the\SOUL@token}%
     \ifdim\dp0>\z@
        \raisebox{\dp0}{\underline{\phantom{\the\SOUL@token}}}%
        \whiten{1}\whiten{0}%
        \whiten{-1}\whiten{-2}%
        \llap{\the\SOUL@token}%
     \else
        \underline{\the\SOUL@token}%
     \fi}%
\SOUL@}
\makeatother

\newcommand*{\demp}{\fontfamily{lmtt}\selectfont}

\DeclareTextFontCommand{\textdemp}{\demp}

\begin{document}

\ifcomment
Multiline
comment
\fi
\ifcomment
\myul{Typesetting test}
% \color[rgb]{1,1,1}
$∑_i^n≠ 60º±∞π∆¬≈√j∫h≤≥µ$

$\CR \R\pro\ind\pro\gS\pro
\mqty[a&b\\c&d]$
$\pro\mathbb{P}$
$\dd{x}$

  \[
    \alpha(x)=\left\{
                \begin{array}{ll}
                  x\\
                  \frac{1}{1+e^{-kx}}\\
                  \frac{e^x-e^{-x}}{e^x+e^{-x}}
                \end{array}
              \right.
  \]

  $\expval{x}$
  
  $\chi_\rho(ghg\dmo)=\Tr(\rho_{ghg\dmo})=\Tr(\rho_g\circ\rho_h\circ\rho\dmo_g)=\Tr(\rho_h)\overset{\mbox{\scalebox{0.5}{$\Tr(AB)=\Tr(BA)$}}}{=}\chi_\rho(h)$
  	$\mathop{\oplus}_{\substack{x\in X}}$

$\mat(\rho_g)=(a_{ij}(g))_{\scriptsize \substack{1\leq i\leq d \\ 1\leq j\leq d}}$ et $\mat(\rho'_g)=(a'_{ij}(g))_{\scriptsize \substack{1\leq i'\leq d' \\ 1\leq j'\leq d'}}$



\[\int_a^b{\mathbb{R}^2}g(u, v)\dd{P_{XY}}(u, v)=\iint g(u,v) f_{XY}(u, v)\dd \lambda(u) \dd \lambda(v)\]
$$\lim_{x\to\infty} f(x)$$	
$$\iiiint_V \mu(t,u,v,w) \,dt\,du\,dv\,dw$$
$$\sum_{n=1}^{\infty} 2^{-n} = 1$$	
\begin{definition}
	Si $X$ et $Y$ sont 2 v.a. ou definit la \textsc{Covariance} entre $X$ et $Y$ comme
	$\cov(X,Y)\overset{\text{def}}{=}\E\left[(X-\E(X))(Y-\E(Y))\right]=\E(XY)-\E(X)\E(Y)$.
\end{definition}
\fi
\pagebreak

% \tableofcontents

% insert your code here
%\input{./algebra/main.tex}
%\input{./geometrie-differentielle/main.tex}
%\input{./probabilite/main.tex}
%\input{./analyse-fonctionnelle/main.tex}
% \input{./Analyse-convexe-et-dualite-en-optimisation/main.tex}
%\input{./tikz/main.tex}
%\input{./Theorie-du-distributions/main.tex}
%\input{./optimisation/mine.tex}
 \input{./modelisation/main.tex}

% yves.aubry@univ-tln.fr : algebra

\end{document}

%% !TEX encoding = UTF-8 Unicode
% !TEX TS-program = xelatex

\documentclass[french]{report}

%\usepackage[utf8]{inputenc}
%\usepackage[T1]{fontenc}
\usepackage{babel}


\newif\ifcomment
%\commenttrue # Show comments

\usepackage{physics}
\usepackage{amssymb}


\usepackage{amsthm}
% \usepackage{thmtools}
\usepackage{mathtools}
\usepackage{amsfonts}

\usepackage{color}

\usepackage{tikz}

\usepackage{geometry}
\geometry{a5paper, margin=0.1in, right=1cm}

\usepackage{dsfont}

\usepackage{graphicx}
\graphicspath{ {images/} }

\usepackage{faktor}

\usepackage{IEEEtrantools}
\usepackage{enumerate}   
\usepackage[PostScript=dvips]{"/Users/aware/Documents/Courses/diagrams"}


\newtheorem{theorem}{Théorème}[section]
\renewcommand{\thetheorem}{\arabic{theorem}}
\newtheorem{lemme}{Lemme}[section]
\renewcommand{\thelemme}{\arabic{lemme}}
\newtheorem{proposition}{Proposition}[section]
\renewcommand{\theproposition}{\arabic{proposition}}
\newtheorem{notations}{Notations}[section]
\newtheorem{problem}{Problème}[section]
\newtheorem{corollary}{Corollaire}[theorem]
\renewcommand{\thecorollary}{\arabic{corollary}}
\newtheorem{property}{Propriété}[section]
\newtheorem{objective}{Objectif}[section]

\theoremstyle{definition}
\newtheorem{definition}{Définition}[section]
\renewcommand{\thedefinition}{\arabic{definition}}
\newtheorem{exercise}{Exercice}[chapter]
\renewcommand{\theexercise}{\arabic{exercise}}
\newtheorem{example}{Exemple}[chapter]
\renewcommand{\theexample}{\arabic{example}}
\newtheorem*{solution}{Solution}
\newtheorem*{application}{Application}
\newtheorem*{notation}{Notation}
\newtheorem*{vocabulary}{Vocabulaire}
\newtheorem*{properties}{Propriétés}



\theoremstyle{remark}
\newtheorem*{remark}{Remarque}
\newtheorem*{rappel}{Rappel}


\usepackage{etoolbox}
\AtBeginEnvironment{exercise}{\small}
\AtBeginEnvironment{example}{\small}

\usepackage{cases}
\usepackage[red]{mypack}

\usepackage[framemethod=TikZ]{mdframed}

\definecolor{bg}{rgb}{0.4,0.25,0.95}
\definecolor{pagebg}{rgb}{0,0,0.5}
\surroundwithmdframed[
   topline=false,
   rightline=false,
   bottomline=false,
   leftmargin=\parindent,
   skipabove=8pt,
   skipbelow=8pt,
   linecolor=blue,
   innerbottommargin=10pt,
   % backgroundcolor=bg,font=\color{orange}\sffamily, fontcolor=white
]{definition}

\usepackage{empheq}
\usepackage[most]{tcolorbox}

\newtcbox{\mymath}[1][]{%
    nobeforeafter, math upper, tcbox raise base,
    enhanced, colframe=blue!30!black,
    colback=red!10, boxrule=1pt,
    #1}

\usepackage{unixode}


\DeclareMathOperator{\ord}{ord}
\DeclareMathOperator{\orb}{orb}
\DeclareMathOperator{\stab}{stab}
\DeclareMathOperator{\Stab}{stab}
\DeclareMathOperator{\ppcm}{ppcm}
\DeclareMathOperator{\conj}{Conj}
\DeclareMathOperator{\End}{End}
\DeclareMathOperator{\rot}{rot}
\DeclareMathOperator{\trs}{trace}
\DeclareMathOperator{\Ind}{Ind}
\DeclareMathOperator{\mat}{Mat}
\DeclareMathOperator{\id}{Id}
\DeclareMathOperator{\vect}{vect}
\DeclareMathOperator{\img}{img}
\DeclareMathOperator{\cov}{Cov}
\DeclareMathOperator{\dist}{dist}
\DeclareMathOperator{\irr}{Irr}
\DeclareMathOperator{\image}{Im}
\DeclareMathOperator{\pd}{\partial}
\DeclareMathOperator{\epi}{epi}
\DeclareMathOperator{\Argmin}{Argmin}
\DeclareMathOperator{\dom}{dom}
\DeclareMathOperator{\proj}{proj}
\DeclareMathOperator{\ctg}{ctg}
\DeclareMathOperator{\supp}{supp}
\DeclareMathOperator{\argmin}{argmin}
\DeclareMathOperator{\mult}{mult}
\DeclareMathOperator{\ch}{ch}
\DeclareMathOperator{\sh}{sh}
\DeclareMathOperator{\rang}{rang}
\DeclareMathOperator{\diam}{diam}
\DeclareMathOperator{\Epigraphe}{Epigraphe}




\usepackage{xcolor}
\everymath{\color{blue}}
%\everymath{\color[rgb]{0,1,1}}
%\pagecolor[rgb]{0,0,0.5}


\newcommand*{\pdtest}[3][]{\ensuremath{\frac{\partial^{#1} #2}{\partial #3}}}

\newcommand*{\deffunc}[6][]{\ensuremath{
\begin{array}{rcl}
#2 : #3 &\rightarrow& #4\\
#5 &\mapsto& #6
\end{array}
}}

\newcommand{\eqcolon}{\mathrel{\resizebox{\widthof{$\mathord{=}$}}{\height}{ $\!\!=\!\!\resizebox{1.2\width}{0.8\height}{\raisebox{0.23ex}{$\mathop{:}$}}\!\!$ }}}
\newcommand{\coloneq}{\mathrel{\resizebox{\widthof{$\mathord{=}$}}{\height}{ $\!\!\resizebox{1.2\width}{0.8\height}{\raisebox{0.23ex}{$\mathop{:}$}}\!\!=\!\!$ }}}
\newcommand{\eqcolonl}{\ensuremath{\mathrel{=\!\!\mathop{:}}}}
\newcommand{\coloneql}{\ensuremath{\mathrel{\mathop{:} \!\! =}}}
\newcommand{\vc}[1]{% inline column vector
  \left(\begin{smallmatrix}#1\end{smallmatrix}\right)%
}
\newcommand{\vr}[1]{% inline row vector
  \begin{smallmatrix}(\,#1\,)\end{smallmatrix}%
}
\makeatletter
\newcommand*{\defeq}{\ =\mathrel{\rlap{%
                     \raisebox{0.3ex}{$\m@th\cdot$}}%
                     \raisebox{-0.3ex}{$\m@th\cdot$}}%
                     }
\makeatother

\newcommand{\mathcircle}[1]{% inline row vector
 \overset{\circ}{#1}
}
\newcommand{\ulim}{% low limit
 \underline{\lim}
}
\newcommand{\ssi}{% iff
\iff
}
\newcommand{\ps}[2]{
\expval{#1 | #2}
}
\newcommand{\df}[1]{
\mqty{#1}
}
\newcommand{\n}[1]{
\norm{#1}
}
\newcommand{\sys}[1]{
\left\{\smqty{#1}\right.
}


\newcommand{\eqdef}{\ensuremath{\overset{\text{def}}=}}


\def\Circlearrowright{\ensuremath{%
  \rotatebox[origin=c]{230}{$\circlearrowright$}}}

\newcommand\ct[1]{\text{\rmfamily\upshape #1}}
\newcommand\question[1]{ {\color{red} ...!? \small #1}}
\newcommand\caz[1]{\left\{\begin{array} #1 \end{array}\right.}
\newcommand\const{\text{\rmfamily\upshape const}}
\newcommand\toP{ \overset{\pro}{\to}}
\newcommand\toPP{ \overset{\text{PP}}{\to}}
\newcommand{\oeq}{\mathrel{\text{\textcircled{$=$}}}}





\usepackage{xcolor}
% \usepackage[normalem]{ulem}
\usepackage{lipsum}
\makeatletter
% \newcommand\colorwave[1][blue]{\bgroup \markoverwith{\lower3.5\p@\hbox{\sixly \textcolor{#1}{\char58}}}\ULon}
%\font\sixly=lasy6 % does not re-load if already loaded, so no memory problem.

\newmdtheoremenv[
linewidth= 1pt,linecolor= blue,%
leftmargin=20,rightmargin=20,innertopmargin=0pt, innerrightmargin=40,%
tikzsetting = { draw=lightgray, line width = 0.3pt,dashed,%
dash pattern = on 15pt off 3pt},%
splittopskip=\topskip,skipbelow=\baselineskip,%
skipabove=\baselineskip,ntheorem,roundcorner=0pt,
% backgroundcolor=pagebg,font=\color{orange}\sffamily, fontcolor=white
]{examplebox}{Exemple}[section]



\newcommand\R{\mathbb{R}}
\newcommand\Z{\mathbb{Z}}
\newcommand\N{\mathbb{N}}
\newcommand\E{\mathbb{E}}
\newcommand\F{\mathcal{F}}
\newcommand\cH{\mathcal{H}}
\newcommand\V{\mathbb{V}}
\newcommand\dmo{ ^{-1} }
\newcommand\kapa{\kappa}
\newcommand\im{Im}
\newcommand\hs{\mathcal{H}}





\usepackage{soul}

\makeatletter
\newcommand*{\whiten}[1]{\llap{\textcolor{white}{{\the\SOUL@token}}\hspace{#1pt}}}
\DeclareRobustCommand*\myul{%
    \def\SOUL@everyspace{\underline{\space}\kern\z@}%
    \def\SOUL@everytoken{%
     \setbox0=\hbox{\the\SOUL@token}%
     \ifdim\dp0>\z@
        \raisebox{\dp0}{\underline{\phantom{\the\SOUL@token}}}%
        \whiten{1}\whiten{0}%
        \whiten{-1}\whiten{-2}%
        \llap{\the\SOUL@token}%
     \else
        \underline{\the\SOUL@token}%
     \fi}%
\SOUL@}
\makeatother

\newcommand*{\demp}{\fontfamily{lmtt}\selectfont}

\DeclareTextFontCommand{\textdemp}{\demp}

\begin{document}

\ifcomment
Multiline
comment
\fi
\ifcomment
\myul{Typesetting test}
% \color[rgb]{1,1,1}
$∑_i^n≠ 60º±∞π∆¬≈√j∫h≤≥µ$

$\CR \R\pro\ind\pro\gS\pro
\mqty[a&b\\c&d]$
$\pro\mathbb{P}$
$\dd{x}$

  \[
    \alpha(x)=\left\{
                \begin{array}{ll}
                  x\\
                  \frac{1}{1+e^{-kx}}\\
                  \frac{e^x-e^{-x}}{e^x+e^{-x}}
                \end{array}
              \right.
  \]

  $\expval{x}$
  
  $\chi_\rho(ghg\dmo)=\Tr(\rho_{ghg\dmo})=\Tr(\rho_g\circ\rho_h\circ\rho\dmo_g)=\Tr(\rho_h)\overset{\mbox{\scalebox{0.5}{$\Tr(AB)=\Tr(BA)$}}}{=}\chi_\rho(h)$
  	$\mathop{\oplus}_{\substack{x\in X}}$

$\mat(\rho_g)=(a_{ij}(g))_{\scriptsize \substack{1\leq i\leq d \\ 1\leq j\leq d}}$ et $\mat(\rho'_g)=(a'_{ij}(g))_{\scriptsize \substack{1\leq i'\leq d' \\ 1\leq j'\leq d'}}$



\[\int_a^b{\mathbb{R}^2}g(u, v)\dd{P_{XY}}(u, v)=\iint g(u,v) f_{XY}(u, v)\dd \lambda(u) \dd \lambda(v)\]
$$\lim_{x\to\infty} f(x)$$	
$$\iiiint_V \mu(t,u,v,w) \,dt\,du\,dv\,dw$$
$$\sum_{n=1}^{\infty} 2^{-n} = 1$$	
\begin{definition}
	Si $X$ et $Y$ sont 2 v.a. ou definit la \textsc{Covariance} entre $X$ et $Y$ comme
	$\cov(X,Y)\overset{\text{def}}{=}\E\left[(X-\E(X))(Y-\E(Y))\right]=\E(XY)-\E(X)\E(Y)$.
\end{definition}
\fi
\pagebreak

% \tableofcontents

% insert your code here
%\input{./algebra/main.tex}
%\input{./geometrie-differentielle/main.tex}
%\input{./probabilite/main.tex}
%\input{./analyse-fonctionnelle/main.tex}
% \input{./Analyse-convexe-et-dualite-en-optimisation/main.tex}
%\input{./tikz/main.tex}
%\input{./Theorie-du-distributions/main.tex}
%\input{./optimisation/mine.tex}
 \input{./modelisation/main.tex}

% yves.aubry@univ-tln.fr : algebra

\end{document}

%% !TEX encoding = UTF-8 Unicode
% !TEX TS-program = xelatex

\documentclass[french]{report}

%\usepackage[utf8]{inputenc}
%\usepackage[T1]{fontenc}
\usepackage{babel}


\newif\ifcomment
%\commenttrue # Show comments

\usepackage{physics}
\usepackage{amssymb}


\usepackage{amsthm}
% \usepackage{thmtools}
\usepackage{mathtools}
\usepackage{amsfonts}

\usepackage{color}

\usepackage{tikz}

\usepackage{geometry}
\geometry{a5paper, margin=0.1in, right=1cm}

\usepackage{dsfont}

\usepackage{graphicx}
\graphicspath{ {images/} }

\usepackage{faktor}

\usepackage{IEEEtrantools}
\usepackage{enumerate}   
\usepackage[PostScript=dvips]{"/Users/aware/Documents/Courses/diagrams"}


\newtheorem{theorem}{Théorème}[section]
\renewcommand{\thetheorem}{\arabic{theorem}}
\newtheorem{lemme}{Lemme}[section]
\renewcommand{\thelemme}{\arabic{lemme}}
\newtheorem{proposition}{Proposition}[section]
\renewcommand{\theproposition}{\arabic{proposition}}
\newtheorem{notations}{Notations}[section]
\newtheorem{problem}{Problème}[section]
\newtheorem{corollary}{Corollaire}[theorem]
\renewcommand{\thecorollary}{\arabic{corollary}}
\newtheorem{property}{Propriété}[section]
\newtheorem{objective}{Objectif}[section]

\theoremstyle{definition}
\newtheorem{definition}{Définition}[section]
\renewcommand{\thedefinition}{\arabic{definition}}
\newtheorem{exercise}{Exercice}[chapter]
\renewcommand{\theexercise}{\arabic{exercise}}
\newtheorem{example}{Exemple}[chapter]
\renewcommand{\theexample}{\arabic{example}}
\newtheorem*{solution}{Solution}
\newtheorem*{application}{Application}
\newtheorem*{notation}{Notation}
\newtheorem*{vocabulary}{Vocabulaire}
\newtheorem*{properties}{Propriétés}



\theoremstyle{remark}
\newtheorem*{remark}{Remarque}
\newtheorem*{rappel}{Rappel}


\usepackage{etoolbox}
\AtBeginEnvironment{exercise}{\small}
\AtBeginEnvironment{example}{\small}

\usepackage{cases}
\usepackage[red]{mypack}

\usepackage[framemethod=TikZ]{mdframed}

\definecolor{bg}{rgb}{0.4,0.25,0.95}
\definecolor{pagebg}{rgb}{0,0,0.5}
\surroundwithmdframed[
   topline=false,
   rightline=false,
   bottomline=false,
   leftmargin=\parindent,
   skipabove=8pt,
   skipbelow=8pt,
   linecolor=blue,
   innerbottommargin=10pt,
   % backgroundcolor=bg,font=\color{orange}\sffamily, fontcolor=white
]{definition}

\usepackage{empheq}
\usepackage[most]{tcolorbox}

\newtcbox{\mymath}[1][]{%
    nobeforeafter, math upper, tcbox raise base,
    enhanced, colframe=blue!30!black,
    colback=red!10, boxrule=1pt,
    #1}

\usepackage{unixode}


\DeclareMathOperator{\ord}{ord}
\DeclareMathOperator{\orb}{orb}
\DeclareMathOperator{\stab}{stab}
\DeclareMathOperator{\Stab}{stab}
\DeclareMathOperator{\ppcm}{ppcm}
\DeclareMathOperator{\conj}{Conj}
\DeclareMathOperator{\End}{End}
\DeclareMathOperator{\rot}{rot}
\DeclareMathOperator{\trs}{trace}
\DeclareMathOperator{\Ind}{Ind}
\DeclareMathOperator{\mat}{Mat}
\DeclareMathOperator{\id}{Id}
\DeclareMathOperator{\vect}{vect}
\DeclareMathOperator{\img}{img}
\DeclareMathOperator{\cov}{Cov}
\DeclareMathOperator{\dist}{dist}
\DeclareMathOperator{\irr}{Irr}
\DeclareMathOperator{\image}{Im}
\DeclareMathOperator{\pd}{\partial}
\DeclareMathOperator{\epi}{epi}
\DeclareMathOperator{\Argmin}{Argmin}
\DeclareMathOperator{\dom}{dom}
\DeclareMathOperator{\proj}{proj}
\DeclareMathOperator{\ctg}{ctg}
\DeclareMathOperator{\supp}{supp}
\DeclareMathOperator{\argmin}{argmin}
\DeclareMathOperator{\mult}{mult}
\DeclareMathOperator{\ch}{ch}
\DeclareMathOperator{\sh}{sh}
\DeclareMathOperator{\rang}{rang}
\DeclareMathOperator{\diam}{diam}
\DeclareMathOperator{\Epigraphe}{Epigraphe}




\usepackage{xcolor}
\everymath{\color{blue}}
%\everymath{\color[rgb]{0,1,1}}
%\pagecolor[rgb]{0,0,0.5}


\newcommand*{\pdtest}[3][]{\ensuremath{\frac{\partial^{#1} #2}{\partial #3}}}

\newcommand*{\deffunc}[6][]{\ensuremath{
\begin{array}{rcl}
#2 : #3 &\rightarrow& #4\\
#5 &\mapsto& #6
\end{array}
}}

\newcommand{\eqcolon}{\mathrel{\resizebox{\widthof{$\mathord{=}$}}{\height}{ $\!\!=\!\!\resizebox{1.2\width}{0.8\height}{\raisebox{0.23ex}{$\mathop{:}$}}\!\!$ }}}
\newcommand{\coloneq}{\mathrel{\resizebox{\widthof{$\mathord{=}$}}{\height}{ $\!\!\resizebox{1.2\width}{0.8\height}{\raisebox{0.23ex}{$\mathop{:}$}}\!\!=\!\!$ }}}
\newcommand{\eqcolonl}{\ensuremath{\mathrel{=\!\!\mathop{:}}}}
\newcommand{\coloneql}{\ensuremath{\mathrel{\mathop{:} \!\! =}}}
\newcommand{\vc}[1]{% inline column vector
  \left(\begin{smallmatrix}#1\end{smallmatrix}\right)%
}
\newcommand{\vr}[1]{% inline row vector
  \begin{smallmatrix}(\,#1\,)\end{smallmatrix}%
}
\makeatletter
\newcommand*{\defeq}{\ =\mathrel{\rlap{%
                     \raisebox{0.3ex}{$\m@th\cdot$}}%
                     \raisebox{-0.3ex}{$\m@th\cdot$}}%
                     }
\makeatother

\newcommand{\mathcircle}[1]{% inline row vector
 \overset{\circ}{#1}
}
\newcommand{\ulim}{% low limit
 \underline{\lim}
}
\newcommand{\ssi}{% iff
\iff
}
\newcommand{\ps}[2]{
\expval{#1 | #2}
}
\newcommand{\df}[1]{
\mqty{#1}
}
\newcommand{\n}[1]{
\norm{#1}
}
\newcommand{\sys}[1]{
\left\{\smqty{#1}\right.
}


\newcommand{\eqdef}{\ensuremath{\overset{\text{def}}=}}


\def\Circlearrowright{\ensuremath{%
  \rotatebox[origin=c]{230}{$\circlearrowright$}}}

\newcommand\ct[1]{\text{\rmfamily\upshape #1}}
\newcommand\question[1]{ {\color{red} ...!? \small #1}}
\newcommand\caz[1]{\left\{\begin{array} #1 \end{array}\right.}
\newcommand\const{\text{\rmfamily\upshape const}}
\newcommand\toP{ \overset{\pro}{\to}}
\newcommand\toPP{ \overset{\text{PP}}{\to}}
\newcommand{\oeq}{\mathrel{\text{\textcircled{$=$}}}}





\usepackage{xcolor}
% \usepackage[normalem]{ulem}
\usepackage{lipsum}
\makeatletter
% \newcommand\colorwave[1][blue]{\bgroup \markoverwith{\lower3.5\p@\hbox{\sixly \textcolor{#1}{\char58}}}\ULon}
%\font\sixly=lasy6 % does not re-load if already loaded, so no memory problem.

\newmdtheoremenv[
linewidth= 1pt,linecolor= blue,%
leftmargin=20,rightmargin=20,innertopmargin=0pt, innerrightmargin=40,%
tikzsetting = { draw=lightgray, line width = 0.3pt,dashed,%
dash pattern = on 15pt off 3pt},%
splittopskip=\topskip,skipbelow=\baselineskip,%
skipabove=\baselineskip,ntheorem,roundcorner=0pt,
% backgroundcolor=pagebg,font=\color{orange}\sffamily, fontcolor=white
]{examplebox}{Exemple}[section]



\newcommand\R{\mathbb{R}}
\newcommand\Z{\mathbb{Z}}
\newcommand\N{\mathbb{N}}
\newcommand\E{\mathbb{E}}
\newcommand\F{\mathcal{F}}
\newcommand\cH{\mathcal{H}}
\newcommand\V{\mathbb{V}}
\newcommand\dmo{ ^{-1} }
\newcommand\kapa{\kappa}
\newcommand\im{Im}
\newcommand\hs{\mathcal{H}}





\usepackage{soul}

\makeatletter
\newcommand*{\whiten}[1]{\llap{\textcolor{white}{{\the\SOUL@token}}\hspace{#1pt}}}
\DeclareRobustCommand*\myul{%
    \def\SOUL@everyspace{\underline{\space}\kern\z@}%
    \def\SOUL@everytoken{%
     \setbox0=\hbox{\the\SOUL@token}%
     \ifdim\dp0>\z@
        \raisebox{\dp0}{\underline{\phantom{\the\SOUL@token}}}%
        \whiten{1}\whiten{0}%
        \whiten{-1}\whiten{-2}%
        \llap{\the\SOUL@token}%
     \else
        \underline{\the\SOUL@token}%
     \fi}%
\SOUL@}
\makeatother

\newcommand*{\demp}{\fontfamily{lmtt}\selectfont}

\DeclareTextFontCommand{\textdemp}{\demp}

\begin{document}

\ifcomment
Multiline
comment
\fi
\ifcomment
\myul{Typesetting test}
% \color[rgb]{1,1,1}
$∑_i^n≠ 60º±∞π∆¬≈√j∫h≤≥µ$

$\CR \R\pro\ind\pro\gS\pro
\mqty[a&b\\c&d]$
$\pro\mathbb{P}$
$\dd{x}$

  \[
    \alpha(x)=\left\{
                \begin{array}{ll}
                  x\\
                  \frac{1}{1+e^{-kx}}\\
                  \frac{e^x-e^{-x}}{e^x+e^{-x}}
                \end{array}
              \right.
  \]

  $\expval{x}$
  
  $\chi_\rho(ghg\dmo)=\Tr(\rho_{ghg\dmo})=\Tr(\rho_g\circ\rho_h\circ\rho\dmo_g)=\Tr(\rho_h)\overset{\mbox{\scalebox{0.5}{$\Tr(AB)=\Tr(BA)$}}}{=}\chi_\rho(h)$
  	$\mathop{\oplus}_{\substack{x\in X}}$

$\mat(\rho_g)=(a_{ij}(g))_{\scriptsize \substack{1\leq i\leq d \\ 1\leq j\leq d}}$ et $\mat(\rho'_g)=(a'_{ij}(g))_{\scriptsize \substack{1\leq i'\leq d' \\ 1\leq j'\leq d'}}$



\[\int_a^b{\mathbb{R}^2}g(u, v)\dd{P_{XY}}(u, v)=\iint g(u,v) f_{XY}(u, v)\dd \lambda(u) \dd \lambda(v)\]
$$\lim_{x\to\infty} f(x)$$	
$$\iiiint_V \mu(t,u,v,w) \,dt\,du\,dv\,dw$$
$$\sum_{n=1}^{\infty} 2^{-n} = 1$$	
\begin{definition}
	Si $X$ et $Y$ sont 2 v.a. ou definit la \textsc{Covariance} entre $X$ et $Y$ comme
	$\cov(X,Y)\overset{\text{def}}{=}\E\left[(X-\E(X))(Y-\E(Y))\right]=\E(XY)-\E(X)\E(Y)$.
\end{definition}
\fi
\pagebreak

% \tableofcontents

% insert your code here
%\input{./algebra/main.tex}
%\input{./geometrie-differentielle/main.tex}
%\input{./probabilite/main.tex}
%\input{./analyse-fonctionnelle/main.tex}
% \input{./Analyse-convexe-et-dualite-en-optimisation/main.tex}
%\input{./tikz/main.tex}
%\input{./Theorie-du-distributions/main.tex}
%\input{./optimisation/mine.tex}
 \input{./modelisation/main.tex}

% yves.aubry@univ-tln.fr : algebra

\end{document}

% % !TEX encoding = UTF-8 Unicode
% !TEX TS-program = xelatex

\documentclass[french]{report}

%\usepackage[utf8]{inputenc}
%\usepackage[T1]{fontenc}
\usepackage{babel}


\newif\ifcomment
%\commenttrue # Show comments

\usepackage{physics}
\usepackage{amssymb}


\usepackage{amsthm}
% \usepackage{thmtools}
\usepackage{mathtools}
\usepackage{amsfonts}

\usepackage{color}

\usepackage{tikz}

\usepackage{geometry}
\geometry{a5paper, margin=0.1in, right=1cm}

\usepackage{dsfont}

\usepackage{graphicx}
\graphicspath{ {images/} }

\usepackage{faktor}

\usepackage{IEEEtrantools}
\usepackage{enumerate}   
\usepackage[PostScript=dvips]{"/Users/aware/Documents/Courses/diagrams"}


\newtheorem{theorem}{Théorème}[section]
\renewcommand{\thetheorem}{\arabic{theorem}}
\newtheorem{lemme}{Lemme}[section]
\renewcommand{\thelemme}{\arabic{lemme}}
\newtheorem{proposition}{Proposition}[section]
\renewcommand{\theproposition}{\arabic{proposition}}
\newtheorem{notations}{Notations}[section]
\newtheorem{problem}{Problème}[section]
\newtheorem{corollary}{Corollaire}[theorem]
\renewcommand{\thecorollary}{\arabic{corollary}}
\newtheorem{property}{Propriété}[section]
\newtheorem{objective}{Objectif}[section]

\theoremstyle{definition}
\newtheorem{definition}{Définition}[section]
\renewcommand{\thedefinition}{\arabic{definition}}
\newtheorem{exercise}{Exercice}[chapter]
\renewcommand{\theexercise}{\arabic{exercise}}
\newtheorem{example}{Exemple}[chapter]
\renewcommand{\theexample}{\arabic{example}}
\newtheorem*{solution}{Solution}
\newtheorem*{application}{Application}
\newtheorem*{notation}{Notation}
\newtheorem*{vocabulary}{Vocabulaire}
\newtheorem*{properties}{Propriétés}



\theoremstyle{remark}
\newtheorem*{remark}{Remarque}
\newtheorem*{rappel}{Rappel}


\usepackage{etoolbox}
\AtBeginEnvironment{exercise}{\small}
\AtBeginEnvironment{example}{\small}

\usepackage{cases}
\usepackage[red]{mypack}

\usepackage[framemethod=TikZ]{mdframed}

\definecolor{bg}{rgb}{0.4,0.25,0.95}
\definecolor{pagebg}{rgb}{0,0,0.5}
\surroundwithmdframed[
   topline=false,
   rightline=false,
   bottomline=false,
   leftmargin=\parindent,
   skipabove=8pt,
   skipbelow=8pt,
   linecolor=blue,
   innerbottommargin=10pt,
   % backgroundcolor=bg,font=\color{orange}\sffamily, fontcolor=white
]{definition}

\usepackage{empheq}
\usepackage[most]{tcolorbox}

\newtcbox{\mymath}[1][]{%
    nobeforeafter, math upper, tcbox raise base,
    enhanced, colframe=blue!30!black,
    colback=red!10, boxrule=1pt,
    #1}

\usepackage{unixode}


\DeclareMathOperator{\ord}{ord}
\DeclareMathOperator{\orb}{orb}
\DeclareMathOperator{\stab}{stab}
\DeclareMathOperator{\Stab}{stab}
\DeclareMathOperator{\ppcm}{ppcm}
\DeclareMathOperator{\conj}{Conj}
\DeclareMathOperator{\End}{End}
\DeclareMathOperator{\rot}{rot}
\DeclareMathOperator{\trs}{trace}
\DeclareMathOperator{\Ind}{Ind}
\DeclareMathOperator{\mat}{Mat}
\DeclareMathOperator{\id}{Id}
\DeclareMathOperator{\vect}{vect}
\DeclareMathOperator{\img}{img}
\DeclareMathOperator{\cov}{Cov}
\DeclareMathOperator{\dist}{dist}
\DeclareMathOperator{\irr}{Irr}
\DeclareMathOperator{\image}{Im}
\DeclareMathOperator{\pd}{\partial}
\DeclareMathOperator{\epi}{epi}
\DeclareMathOperator{\Argmin}{Argmin}
\DeclareMathOperator{\dom}{dom}
\DeclareMathOperator{\proj}{proj}
\DeclareMathOperator{\ctg}{ctg}
\DeclareMathOperator{\supp}{supp}
\DeclareMathOperator{\argmin}{argmin}
\DeclareMathOperator{\mult}{mult}
\DeclareMathOperator{\ch}{ch}
\DeclareMathOperator{\sh}{sh}
\DeclareMathOperator{\rang}{rang}
\DeclareMathOperator{\diam}{diam}
\DeclareMathOperator{\Epigraphe}{Epigraphe}




\usepackage{xcolor}
\everymath{\color{blue}}
%\everymath{\color[rgb]{0,1,1}}
%\pagecolor[rgb]{0,0,0.5}


\newcommand*{\pdtest}[3][]{\ensuremath{\frac{\partial^{#1} #2}{\partial #3}}}

\newcommand*{\deffunc}[6][]{\ensuremath{
\begin{array}{rcl}
#2 : #3 &\rightarrow& #4\\
#5 &\mapsto& #6
\end{array}
}}

\newcommand{\eqcolon}{\mathrel{\resizebox{\widthof{$\mathord{=}$}}{\height}{ $\!\!=\!\!\resizebox{1.2\width}{0.8\height}{\raisebox{0.23ex}{$\mathop{:}$}}\!\!$ }}}
\newcommand{\coloneq}{\mathrel{\resizebox{\widthof{$\mathord{=}$}}{\height}{ $\!\!\resizebox{1.2\width}{0.8\height}{\raisebox{0.23ex}{$\mathop{:}$}}\!\!=\!\!$ }}}
\newcommand{\eqcolonl}{\ensuremath{\mathrel{=\!\!\mathop{:}}}}
\newcommand{\coloneql}{\ensuremath{\mathrel{\mathop{:} \!\! =}}}
\newcommand{\vc}[1]{% inline column vector
  \left(\begin{smallmatrix}#1\end{smallmatrix}\right)%
}
\newcommand{\vr}[1]{% inline row vector
  \begin{smallmatrix}(\,#1\,)\end{smallmatrix}%
}
\makeatletter
\newcommand*{\defeq}{\ =\mathrel{\rlap{%
                     \raisebox{0.3ex}{$\m@th\cdot$}}%
                     \raisebox{-0.3ex}{$\m@th\cdot$}}%
                     }
\makeatother

\newcommand{\mathcircle}[1]{% inline row vector
 \overset{\circ}{#1}
}
\newcommand{\ulim}{% low limit
 \underline{\lim}
}
\newcommand{\ssi}{% iff
\iff
}
\newcommand{\ps}[2]{
\expval{#1 | #2}
}
\newcommand{\df}[1]{
\mqty{#1}
}
\newcommand{\n}[1]{
\norm{#1}
}
\newcommand{\sys}[1]{
\left\{\smqty{#1}\right.
}


\newcommand{\eqdef}{\ensuremath{\overset{\text{def}}=}}


\def\Circlearrowright{\ensuremath{%
  \rotatebox[origin=c]{230}{$\circlearrowright$}}}

\newcommand\ct[1]{\text{\rmfamily\upshape #1}}
\newcommand\question[1]{ {\color{red} ...!? \small #1}}
\newcommand\caz[1]{\left\{\begin{array} #1 \end{array}\right.}
\newcommand\const{\text{\rmfamily\upshape const}}
\newcommand\toP{ \overset{\pro}{\to}}
\newcommand\toPP{ \overset{\text{PP}}{\to}}
\newcommand{\oeq}{\mathrel{\text{\textcircled{$=$}}}}





\usepackage{xcolor}
% \usepackage[normalem]{ulem}
\usepackage{lipsum}
\makeatletter
% \newcommand\colorwave[1][blue]{\bgroup \markoverwith{\lower3.5\p@\hbox{\sixly \textcolor{#1}{\char58}}}\ULon}
%\font\sixly=lasy6 % does not re-load if already loaded, so no memory problem.

\newmdtheoremenv[
linewidth= 1pt,linecolor= blue,%
leftmargin=20,rightmargin=20,innertopmargin=0pt, innerrightmargin=40,%
tikzsetting = { draw=lightgray, line width = 0.3pt,dashed,%
dash pattern = on 15pt off 3pt},%
splittopskip=\topskip,skipbelow=\baselineskip,%
skipabove=\baselineskip,ntheorem,roundcorner=0pt,
% backgroundcolor=pagebg,font=\color{orange}\sffamily, fontcolor=white
]{examplebox}{Exemple}[section]



\newcommand\R{\mathbb{R}}
\newcommand\Z{\mathbb{Z}}
\newcommand\N{\mathbb{N}}
\newcommand\E{\mathbb{E}}
\newcommand\F{\mathcal{F}}
\newcommand\cH{\mathcal{H}}
\newcommand\V{\mathbb{V}}
\newcommand\dmo{ ^{-1} }
\newcommand\kapa{\kappa}
\newcommand\im{Im}
\newcommand\hs{\mathcal{H}}





\usepackage{soul}

\makeatletter
\newcommand*{\whiten}[1]{\llap{\textcolor{white}{{\the\SOUL@token}}\hspace{#1pt}}}
\DeclareRobustCommand*\myul{%
    \def\SOUL@everyspace{\underline{\space}\kern\z@}%
    \def\SOUL@everytoken{%
     \setbox0=\hbox{\the\SOUL@token}%
     \ifdim\dp0>\z@
        \raisebox{\dp0}{\underline{\phantom{\the\SOUL@token}}}%
        \whiten{1}\whiten{0}%
        \whiten{-1}\whiten{-2}%
        \llap{\the\SOUL@token}%
     \else
        \underline{\the\SOUL@token}%
     \fi}%
\SOUL@}
\makeatother

\newcommand*{\demp}{\fontfamily{lmtt}\selectfont}

\DeclareTextFontCommand{\textdemp}{\demp}

\begin{document}

\ifcomment
Multiline
comment
\fi
\ifcomment
\myul{Typesetting test}
% \color[rgb]{1,1,1}
$∑_i^n≠ 60º±∞π∆¬≈√j∫h≤≥µ$

$\CR \R\pro\ind\pro\gS\pro
\mqty[a&b\\c&d]$
$\pro\mathbb{P}$
$\dd{x}$

  \[
    \alpha(x)=\left\{
                \begin{array}{ll}
                  x\\
                  \frac{1}{1+e^{-kx}}\\
                  \frac{e^x-e^{-x}}{e^x+e^{-x}}
                \end{array}
              \right.
  \]

  $\expval{x}$
  
  $\chi_\rho(ghg\dmo)=\Tr(\rho_{ghg\dmo})=\Tr(\rho_g\circ\rho_h\circ\rho\dmo_g)=\Tr(\rho_h)\overset{\mbox{\scalebox{0.5}{$\Tr(AB)=\Tr(BA)$}}}{=}\chi_\rho(h)$
  	$\mathop{\oplus}_{\substack{x\in X}}$

$\mat(\rho_g)=(a_{ij}(g))_{\scriptsize \substack{1\leq i\leq d \\ 1\leq j\leq d}}$ et $\mat(\rho'_g)=(a'_{ij}(g))_{\scriptsize \substack{1\leq i'\leq d' \\ 1\leq j'\leq d'}}$



\[\int_a^b{\mathbb{R}^2}g(u, v)\dd{P_{XY}}(u, v)=\iint g(u,v) f_{XY}(u, v)\dd \lambda(u) \dd \lambda(v)\]
$$\lim_{x\to\infty} f(x)$$	
$$\iiiint_V \mu(t,u,v,w) \,dt\,du\,dv\,dw$$
$$\sum_{n=1}^{\infty} 2^{-n} = 1$$	
\begin{definition}
	Si $X$ et $Y$ sont 2 v.a. ou definit la \textsc{Covariance} entre $X$ et $Y$ comme
	$\cov(X,Y)\overset{\text{def}}{=}\E\left[(X-\E(X))(Y-\E(Y))\right]=\E(XY)-\E(X)\E(Y)$.
\end{definition}
\fi
\pagebreak

% \tableofcontents

% insert your code here
%\input{./algebra/main.tex}
%\input{./geometrie-differentielle/main.tex}
%\input{./probabilite/main.tex}
%\input{./analyse-fonctionnelle/main.tex}
% \input{./Analyse-convexe-et-dualite-en-optimisation/main.tex}
%\input{./tikz/main.tex}
%\input{./Theorie-du-distributions/main.tex}
%\input{./optimisation/mine.tex}
 \input{./modelisation/main.tex}

% yves.aubry@univ-tln.fr : algebra

\end{document}

%% !TEX encoding = UTF-8 Unicode
% !TEX TS-program = xelatex

\documentclass[french]{report}

%\usepackage[utf8]{inputenc}
%\usepackage[T1]{fontenc}
\usepackage{babel}


\newif\ifcomment
%\commenttrue # Show comments

\usepackage{physics}
\usepackage{amssymb}


\usepackage{amsthm}
% \usepackage{thmtools}
\usepackage{mathtools}
\usepackage{amsfonts}

\usepackage{color}

\usepackage{tikz}

\usepackage{geometry}
\geometry{a5paper, margin=0.1in, right=1cm}

\usepackage{dsfont}

\usepackage{graphicx}
\graphicspath{ {images/} }

\usepackage{faktor}

\usepackage{IEEEtrantools}
\usepackage{enumerate}   
\usepackage[PostScript=dvips]{"/Users/aware/Documents/Courses/diagrams"}


\newtheorem{theorem}{Théorème}[section]
\renewcommand{\thetheorem}{\arabic{theorem}}
\newtheorem{lemme}{Lemme}[section]
\renewcommand{\thelemme}{\arabic{lemme}}
\newtheorem{proposition}{Proposition}[section]
\renewcommand{\theproposition}{\arabic{proposition}}
\newtheorem{notations}{Notations}[section]
\newtheorem{problem}{Problème}[section]
\newtheorem{corollary}{Corollaire}[theorem]
\renewcommand{\thecorollary}{\arabic{corollary}}
\newtheorem{property}{Propriété}[section]
\newtheorem{objective}{Objectif}[section]

\theoremstyle{definition}
\newtheorem{definition}{Définition}[section]
\renewcommand{\thedefinition}{\arabic{definition}}
\newtheorem{exercise}{Exercice}[chapter]
\renewcommand{\theexercise}{\arabic{exercise}}
\newtheorem{example}{Exemple}[chapter]
\renewcommand{\theexample}{\arabic{example}}
\newtheorem*{solution}{Solution}
\newtheorem*{application}{Application}
\newtheorem*{notation}{Notation}
\newtheorem*{vocabulary}{Vocabulaire}
\newtheorem*{properties}{Propriétés}



\theoremstyle{remark}
\newtheorem*{remark}{Remarque}
\newtheorem*{rappel}{Rappel}


\usepackage{etoolbox}
\AtBeginEnvironment{exercise}{\small}
\AtBeginEnvironment{example}{\small}

\usepackage{cases}
\usepackage[red]{mypack}

\usepackage[framemethod=TikZ]{mdframed}

\definecolor{bg}{rgb}{0.4,0.25,0.95}
\definecolor{pagebg}{rgb}{0,0,0.5}
\surroundwithmdframed[
   topline=false,
   rightline=false,
   bottomline=false,
   leftmargin=\parindent,
   skipabove=8pt,
   skipbelow=8pt,
   linecolor=blue,
   innerbottommargin=10pt,
   % backgroundcolor=bg,font=\color{orange}\sffamily, fontcolor=white
]{definition}

\usepackage{empheq}
\usepackage[most]{tcolorbox}

\newtcbox{\mymath}[1][]{%
    nobeforeafter, math upper, tcbox raise base,
    enhanced, colframe=blue!30!black,
    colback=red!10, boxrule=1pt,
    #1}

\usepackage{unixode}


\DeclareMathOperator{\ord}{ord}
\DeclareMathOperator{\orb}{orb}
\DeclareMathOperator{\stab}{stab}
\DeclareMathOperator{\Stab}{stab}
\DeclareMathOperator{\ppcm}{ppcm}
\DeclareMathOperator{\conj}{Conj}
\DeclareMathOperator{\End}{End}
\DeclareMathOperator{\rot}{rot}
\DeclareMathOperator{\trs}{trace}
\DeclareMathOperator{\Ind}{Ind}
\DeclareMathOperator{\mat}{Mat}
\DeclareMathOperator{\id}{Id}
\DeclareMathOperator{\vect}{vect}
\DeclareMathOperator{\img}{img}
\DeclareMathOperator{\cov}{Cov}
\DeclareMathOperator{\dist}{dist}
\DeclareMathOperator{\irr}{Irr}
\DeclareMathOperator{\image}{Im}
\DeclareMathOperator{\pd}{\partial}
\DeclareMathOperator{\epi}{epi}
\DeclareMathOperator{\Argmin}{Argmin}
\DeclareMathOperator{\dom}{dom}
\DeclareMathOperator{\proj}{proj}
\DeclareMathOperator{\ctg}{ctg}
\DeclareMathOperator{\supp}{supp}
\DeclareMathOperator{\argmin}{argmin}
\DeclareMathOperator{\mult}{mult}
\DeclareMathOperator{\ch}{ch}
\DeclareMathOperator{\sh}{sh}
\DeclareMathOperator{\rang}{rang}
\DeclareMathOperator{\diam}{diam}
\DeclareMathOperator{\Epigraphe}{Epigraphe}




\usepackage{xcolor}
\everymath{\color{blue}}
%\everymath{\color[rgb]{0,1,1}}
%\pagecolor[rgb]{0,0,0.5}


\newcommand*{\pdtest}[3][]{\ensuremath{\frac{\partial^{#1} #2}{\partial #3}}}

\newcommand*{\deffunc}[6][]{\ensuremath{
\begin{array}{rcl}
#2 : #3 &\rightarrow& #4\\
#5 &\mapsto& #6
\end{array}
}}

\newcommand{\eqcolon}{\mathrel{\resizebox{\widthof{$\mathord{=}$}}{\height}{ $\!\!=\!\!\resizebox{1.2\width}{0.8\height}{\raisebox{0.23ex}{$\mathop{:}$}}\!\!$ }}}
\newcommand{\coloneq}{\mathrel{\resizebox{\widthof{$\mathord{=}$}}{\height}{ $\!\!\resizebox{1.2\width}{0.8\height}{\raisebox{0.23ex}{$\mathop{:}$}}\!\!=\!\!$ }}}
\newcommand{\eqcolonl}{\ensuremath{\mathrel{=\!\!\mathop{:}}}}
\newcommand{\coloneql}{\ensuremath{\mathrel{\mathop{:} \!\! =}}}
\newcommand{\vc}[1]{% inline column vector
  \left(\begin{smallmatrix}#1\end{smallmatrix}\right)%
}
\newcommand{\vr}[1]{% inline row vector
  \begin{smallmatrix}(\,#1\,)\end{smallmatrix}%
}
\makeatletter
\newcommand*{\defeq}{\ =\mathrel{\rlap{%
                     \raisebox{0.3ex}{$\m@th\cdot$}}%
                     \raisebox{-0.3ex}{$\m@th\cdot$}}%
                     }
\makeatother

\newcommand{\mathcircle}[1]{% inline row vector
 \overset{\circ}{#1}
}
\newcommand{\ulim}{% low limit
 \underline{\lim}
}
\newcommand{\ssi}{% iff
\iff
}
\newcommand{\ps}[2]{
\expval{#1 | #2}
}
\newcommand{\df}[1]{
\mqty{#1}
}
\newcommand{\n}[1]{
\norm{#1}
}
\newcommand{\sys}[1]{
\left\{\smqty{#1}\right.
}


\newcommand{\eqdef}{\ensuremath{\overset{\text{def}}=}}


\def\Circlearrowright{\ensuremath{%
  \rotatebox[origin=c]{230}{$\circlearrowright$}}}

\newcommand\ct[1]{\text{\rmfamily\upshape #1}}
\newcommand\question[1]{ {\color{red} ...!? \small #1}}
\newcommand\caz[1]{\left\{\begin{array} #1 \end{array}\right.}
\newcommand\const{\text{\rmfamily\upshape const}}
\newcommand\toP{ \overset{\pro}{\to}}
\newcommand\toPP{ \overset{\text{PP}}{\to}}
\newcommand{\oeq}{\mathrel{\text{\textcircled{$=$}}}}





\usepackage{xcolor}
% \usepackage[normalem]{ulem}
\usepackage{lipsum}
\makeatletter
% \newcommand\colorwave[1][blue]{\bgroup \markoverwith{\lower3.5\p@\hbox{\sixly \textcolor{#1}{\char58}}}\ULon}
%\font\sixly=lasy6 % does not re-load if already loaded, so no memory problem.

\newmdtheoremenv[
linewidth= 1pt,linecolor= blue,%
leftmargin=20,rightmargin=20,innertopmargin=0pt, innerrightmargin=40,%
tikzsetting = { draw=lightgray, line width = 0.3pt,dashed,%
dash pattern = on 15pt off 3pt},%
splittopskip=\topskip,skipbelow=\baselineskip,%
skipabove=\baselineskip,ntheorem,roundcorner=0pt,
% backgroundcolor=pagebg,font=\color{orange}\sffamily, fontcolor=white
]{examplebox}{Exemple}[section]



\newcommand\R{\mathbb{R}}
\newcommand\Z{\mathbb{Z}}
\newcommand\N{\mathbb{N}}
\newcommand\E{\mathbb{E}}
\newcommand\F{\mathcal{F}}
\newcommand\cH{\mathcal{H}}
\newcommand\V{\mathbb{V}}
\newcommand\dmo{ ^{-1} }
\newcommand\kapa{\kappa}
\newcommand\im{Im}
\newcommand\hs{\mathcal{H}}





\usepackage{soul}

\makeatletter
\newcommand*{\whiten}[1]{\llap{\textcolor{white}{{\the\SOUL@token}}\hspace{#1pt}}}
\DeclareRobustCommand*\myul{%
    \def\SOUL@everyspace{\underline{\space}\kern\z@}%
    \def\SOUL@everytoken{%
     \setbox0=\hbox{\the\SOUL@token}%
     \ifdim\dp0>\z@
        \raisebox{\dp0}{\underline{\phantom{\the\SOUL@token}}}%
        \whiten{1}\whiten{0}%
        \whiten{-1}\whiten{-2}%
        \llap{\the\SOUL@token}%
     \else
        \underline{\the\SOUL@token}%
     \fi}%
\SOUL@}
\makeatother

\newcommand*{\demp}{\fontfamily{lmtt}\selectfont}

\DeclareTextFontCommand{\textdemp}{\demp}

\begin{document}

\ifcomment
Multiline
comment
\fi
\ifcomment
\myul{Typesetting test}
% \color[rgb]{1,1,1}
$∑_i^n≠ 60º±∞π∆¬≈√j∫h≤≥µ$

$\CR \R\pro\ind\pro\gS\pro
\mqty[a&b\\c&d]$
$\pro\mathbb{P}$
$\dd{x}$

  \[
    \alpha(x)=\left\{
                \begin{array}{ll}
                  x\\
                  \frac{1}{1+e^{-kx}}\\
                  \frac{e^x-e^{-x}}{e^x+e^{-x}}
                \end{array}
              \right.
  \]

  $\expval{x}$
  
  $\chi_\rho(ghg\dmo)=\Tr(\rho_{ghg\dmo})=\Tr(\rho_g\circ\rho_h\circ\rho\dmo_g)=\Tr(\rho_h)\overset{\mbox{\scalebox{0.5}{$\Tr(AB)=\Tr(BA)$}}}{=}\chi_\rho(h)$
  	$\mathop{\oplus}_{\substack{x\in X}}$

$\mat(\rho_g)=(a_{ij}(g))_{\scriptsize \substack{1\leq i\leq d \\ 1\leq j\leq d}}$ et $\mat(\rho'_g)=(a'_{ij}(g))_{\scriptsize \substack{1\leq i'\leq d' \\ 1\leq j'\leq d'}}$



\[\int_a^b{\mathbb{R}^2}g(u, v)\dd{P_{XY}}(u, v)=\iint g(u,v) f_{XY}(u, v)\dd \lambda(u) \dd \lambda(v)\]
$$\lim_{x\to\infty} f(x)$$	
$$\iiiint_V \mu(t,u,v,w) \,dt\,du\,dv\,dw$$
$$\sum_{n=1}^{\infty} 2^{-n} = 1$$	
\begin{definition}
	Si $X$ et $Y$ sont 2 v.a. ou definit la \textsc{Covariance} entre $X$ et $Y$ comme
	$\cov(X,Y)\overset{\text{def}}{=}\E\left[(X-\E(X))(Y-\E(Y))\right]=\E(XY)-\E(X)\E(Y)$.
\end{definition}
\fi
\pagebreak

% \tableofcontents

% insert your code here
%\input{./algebra/main.tex}
%\input{./geometrie-differentielle/main.tex}
%\input{./probabilite/main.tex}
%\input{./analyse-fonctionnelle/main.tex}
% \input{./Analyse-convexe-et-dualite-en-optimisation/main.tex}
%\input{./tikz/main.tex}
%\input{./Theorie-du-distributions/main.tex}
%\input{./optimisation/mine.tex}
 \input{./modelisation/main.tex}

% yves.aubry@univ-tln.fr : algebra

\end{document}

%% !TEX encoding = UTF-8 Unicode
% !TEX TS-program = xelatex

\documentclass[french]{report}

%\usepackage[utf8]{inputenc}
%\usepackage[T1]{fontenc}
\usepackage{babel}


\newif\ifcomment
%\commenttrue # Show comments

\usepackage{physics}
\usepackage{amssymb}


\usepackage{amsthm}
% \usepackage{thmtools}
\usepackage{mathtools}
\usepackage{amsfonts}

\usepackage{color}

\usepackage{tikz}

\usepackage{geometry}
\geometry{a5paper, margin=0.1in, right=1cm}

\usepackage{dsfont}

\usepackage{graphicx}
\graphicspath{ {images/} }

\usepackage{faktor}

\usepackage{IEEEtrantools}
\usepackage{enumerate}   
\usepackage[PostScript=dvips]{"/Users/aware/Documents/Courses/diagrams"}


\newtheorem{theorem}{Théorème}[section]
\renewcommand{\thetheorem}{\arabic{theorem}}
\newtheorem{lemme}{Lemme}[section]
\renewcommand{\thelemme}{\arabic{lemme}}
\newtheorem{proposition}{Proposition}[section]
\renewcommand{\theproposition}{\arabic{proposition}}
\newtheorem{notations}{Notations}[section]
\newtheorem{problem}{Problème}[section]
\newtheorem{corollary}{Corollaire}[theorem]
\renewcommand{\thecorollary}{\arabic{corollary}}
\newtheorem{property}{Propriété}[section]
\newtheorem{objective}{Objectif}[section]

\theoremstyle{definition}
\newtheorem{definition}{Définition}[section]
\renewcommand{\thedefinition}{\arabic{definition}}
\newtheorem{exercise}{Exercice}[chapter]
\renewcommand{\theexercise}{\arabic{exercise}}
\newtheorem{example}{Exemple}[chapter]
\renewcommand{\theexample}{\arabic{example}}
\newtheorem*{solution}{Solution}
\newtheorem*{application}{Application}
\newtheorem*{notation}{Notation}
\newtheorem*{vocabulary}{Vocabulaire}
\newtheorem*{properties}{Propriétés}



\theoremstyle{remark}
\newtheorem*{remark}{Remarque}
\newtheorem*{rappel}{Rappel}


\usepackage{etoolbox}
\AtBeginEnvironment{exercise}{\small}
\AtBeginEnvironment{example}{\small}

\usepackage{cases}
\usepackage[red]{mypack}

\usepackage[framemethod=TikZ]{mdframed}

\definecolor{bg}{rgb}{0.4,0.25,0.95}
\definecolor{pagebg}{rgb}{0,0,0.5}
\surroundwithmdframed[
   topline=false,
   rightline=false,
   bottomline=false,
   leftmargin=\parindent,
   skipabove=8pt,
   skipbelow=8pt,
   linecolor=blue,
   innerbottommargin=10pt,
   % backgroundcolor=bg,font=\color{orange}\sffamily, fontcolor=white
]{definition}

\usepackage{empheq}
\usepackage[most]{tcolorbox}

\newtcbox{\mymath}[1][]{%
    nobeforeafter, math upper, tcbox raise base,
    enhanced, colframe=blue!30!black,
    colback=red!10, boxrule=1pt,
    #1}

\usepackage{unixode}


\DeclareMathOperator{\ord}{ord}
\DeclareMathOperator{\orb}{orb}
\DeclareMathOperator{\stab}{stab}
\DeclareMathOperator{\Stab}{stab}
\DeclareMathOperator{\ppcm}{ppcm}
\DeclareMathOperator{\conj}{Conj}
\DeclareMathOperator{\End}{End}
\DeclareMathOperator{\rot}{rot}
\DeclareMathOperator{\trs}{trace}
\DeclareMathOperator{\Ind}{Ind}
\DeclareMathOperator{\mat}{Mat}
\DeclareMathOperator{\id}{Id}
\DeclareMathOperator{\vect}{vect}
\DeclareMathOperator{\img}{img}
\DeclareMathOperator{\cov}{Cov}
\DeclareMathOperator{\dist}{dist}
\DeclareMathOperator{\irr}{Irr}
\DeclareMathOperator{\image}{Im}
\DeclareMathOperator{\pd}{\partial}
\DeclareMathOperator{\epi}{epi}
\DeclareMathOperator{\Argmin}{Argmin}
\DeclareMathOperator{\dom}{dom}
\DeclareMathOperator{\proj}{proj}
\DeclareMathOperator{\ctg}{ctg}
\DeclareMathOperator{\supp}{supp}
\DeclareMathOperator{\argmin}{argmin}
\DeclareMathOperator{\mult}{mult}
\DeclareMathOperator{\ch}{ch}
\DeclareMathOperator{\sh}{sh}
\DeclareMathOperator{\rang}{rang}
\DeclareMathOperator{\diam}{diam}
\DeclareMathOperator{\Epigraphe}{Epigraphe}




\usepackage{xcolor}
\everymath{\color{blue}}
%\everymath{\color[rgb]{0,1,1}}
%\pagecolor[rgb]{0,0,0.5}


\newcommand*{\pdtest}[3][]{\ensuremath{\frac{\partial^{#1} #2}{\partial #3}}}

\newcommand*{\deffunc}[6][]{\ensuremath{
\begin{array}{rcl}
#2 : #3 &\rightarrow& #4\\
#5 &\mapsto& #6
\end{array}
}}

\newcommand{\eqcolon}{\mathrel{\resizebox{\widthof{$\mathord{=}$}}{\height}{ $\!\!=\!\!\resizebox{1.2\width}{0.8\height}{\raisebox{0.23ex}{$\mathop{:}$}}\!\!$ }}}
\newcommand{\coloneq}{\mathrel{\resizebox{\widthof{$\mathord{=}$}}{\height}{ $\!\!\resizebox{1.2\width}{0.8\height}{\raisebox{0.23ex}{$\mathop{:}$}}\!\!=\!\!$ }}}
\newcommand{\eqcolonl}{\ensuremath{\mathrel{=\!\!\mathop{:}}}}
\newcommand{\coloneql}{\ensuremath{\mathrel{\mathop{:} \!\! =}}}
\newcommand{\vc}[1]{% inline column vector
  \left(\begin{smallmatrix}#1\end{smallmatrix}\right)%
}
\newcommand{\vr}[1]{% inline row vector
  \begin{smallmatrix}(\,#1\,)\end{smallmatrix}%
}
\makeatletter
\newcommand*{\defeq}{\ =\mathrel{\rlap{%
                     \raisebox{0.3ex}{$\m@th\cdot$}}%
                     \raisebox{-0.3ex}{$\m@th\cdot$}}%
                     }
\makeatother

\newcommand{\mathcircle}[1]{% inline row vector
 \overset{\circ}{#1}
}
\newcommand{\ulim}{% low limit
 \underline{\lim}
}
\newcommand{\ssi}{% iff
\iff
}
\newcommand{\ps}[2]{
\expval{#1 | #2}
}
\newcommand{\df}[1]{
\mqty{#1}
}
\newcommand{\n}[1]{
\norm{#1}
}
\newcommand{\sys}[1]{
\left\{\smqty{#1}\right.
}


\newcommand{\eqdef}{\ensuremath{\overset{\text{def}}=}}


\def\Circlearrowright{\ensuremath{%
  \rotatebox[origin=c]{230}{$\circlearrowright$}}}

\newcommand\ct[1]{\text{\rmfamily\upshape #1}}
\newcommand\question[1]{ {\color{red} ...!? \small #1}}
\newcommand\caz[1]{\left\{\begin{array} #1 \end{array}\right.}
\newcommand\const{\text{\rmfamily\upshape const}}
\newcommand\toP{ \overset{\pro}{\to}}
\newcommand\toPP{ \overset{\text{PP}}{\to}}
\newcommand{\oeq}{\mathrel{\text{\textcircled{$=$}}}}





\usepackage{xcolor}
% \usepackage[normalem]{ulem}
\usepackage{lipsum}
\makeatletter
% \newcommand\colorwave[1][blue]{\bgroup \markoverwith{\lower3.5\p@\hbox{\sixly \textcolor{#1}{\char58}}}\ULon}
%\font\sixly=lasy6 % does not re-load if already loaded, so no memory problem.

\newmdtheoremenv[
linewidth= 1pt,linecolor= blue,%
leftmargin=20,rightmargin=20,innertopmargin=0pt, innerrightmargin=40,%
tikzsetting = { draw=lightgray, line width = 0.3pt,dashed,%
dash pattern = on 15pt off 3pt},%
splittopskip=\topskip,skipbelow=\baselineskip,%
skipabove=\baselineskip,ntheorem,roundcorner=0pt,
% backgroundcolor=pagebg,font=\color{orange}\sffamily, fontcolor=white
]{examplebox}{Exemple}[section]



\newcommand\R{\mathbb{R}}
\newcommand\Z{\mathbb{Z}}
\newcommand\N{\mathbb{N}}
\newcommand\E{\mathbb{E}}
\newcommand\F{\mathcal{F}}
\newcommand\cH{\mathcal{H}}
\newcommand\V{\mathbb{V}}
\newcommand\dmo{ ^{-1} }
\newcommand\kapa{\kappa}
\newcommand\im{Im}
\newcommand\hs{\mathcal{H}}





\usepackage{soul}

\makeatletter
\newcommand*{\whiten}[1]{\llap{\textcolor{white}{{\the\SOUL@token}}\hspace{#1pt}}}
\DeclareRobustCommand*\myul{%
    \def\SOUL@everyspace{\underline{\space}\kern\z@}%
    \def\SOUL@everytoken{%
     \setbox0=\hbox{\the\SOUL@token}%
     \ifdim\dp0>\z@
        \raisebox{\dp0}{\underline{\phantom{\the\SOUL@token}}}%
        \whiten{1}\whiten{0}%
        \whiten{-1}\whiten{-2}%
        \llap{\the\SOUL@token}%
     \else
        \underline{\the\SOUL@token}%
     \fi}%
\SOUL@}
\makeatother

\newcommand*{\demp}{\fontfamily{lmtt}\selectfont}

\DeclareTextFontCommand{\textdemp}{\demp}

\begin{document}

\ifcomment
Multiline
comment
\fi
\ifcomment
\myul{Typesetting test}
% \color[rgb]{1,1,1}
$∑_i^n≠ 60º±∞π∆¬≈√j∫h≤≥µ$

$\CR \R\pro\ind\pro\gS\pro
\mqty[a&b\\c&d]$
$\pro\mathbb{P}$
$\dd{x}$

  \[
    \alpha(x)=\left\{
                \begin{array}{ll}
                  x\\
                  \frac{1}{1+e^{-kx}}\\
                  \frac{e^x-e^{-x}}{e^x+e^{-x}}
                \end{array}
              \right.
  \]

  $\expval{x}$
  
  $\chi_\rho(ghg\dmo)=\Tr(\rho_{ghg\dmo})=\Tr(\rho_g\circ\rho_h\circ\rho\dmo_g)=\Tr(\rho_h)\overset{\mbox{\scalebox{0.5}{$\Tr(AB)=\Tr(BA)$}}}{=}\chi_\rho(h)$
  	$\mathop{\oplus}_{\substack{x\in X}}$

$\mat(\rho_g)=(a_{ij}(g))_{\scriptsize \substack{1\leq i\leq d \\ 1\leq j\leq d}}$ et $\mat(\rho'_g)=(a'_{ij}(g))_{\scriptsize \substack{1\leq i'\leq d' \\ 1\leq j'\leq d'}}$



\[\int_a^b{\mathbb{R}^2}g(u, v)\dd{P_{XY}}(u, v)=\iint g(u,v) f_{XY}(u, v)\dd \lambda(u) \dd \lambda(v)\]
$$\lim_{x\to\infty} f(x)$$	
$$\iiiint_V \mu(t,u,v,w) \,dt\,du\,dv\,dw$$
$$\sum_{n=1}^{\infty} 2^{-n} = 1$$	
\begin{definition}
	Si $X$ et $Y$ sont 2 v.a. ou definit la \textsc{Covariance} entre $X$ et $Y$ comme
	$\cov(X,Y)\overset{\text{def}}{=}\E\left[(X-\E(X))(Y-\E(Y))\right]=\E(XY)-\E(X)\E(Y)$.
\end{definition}
\fi
\pagebreak

% \tableofcontents

% insert your code here
%\input{./algebra/main.tex}
%\input{./geometrie-differentielle/main.tex}
%\input{./probabilite/main.tex}
%\input{./analyse-fonctionnelle/main.tex}
% \input{./Analyse-convexe-et-dualite-en-optimisation/main.tex}
%\input{./tikz/main.tex}
%\input{./Theorie-du-distributions/main.tex}
%\input{./optimisation/mine.tex}
 \input{./modelisation/main.tex}

% yves.aubry@univ-tln.fr : algebra

\end{document}

%\input{./optimisation/mine.tex}
 % !TEX encoding = UTF-8 Unicode
% !TEX TS-program = xelatex

\documentclass[french]{report}

%\usepackage[utf8]{inputenc}
%\usepackage[T1]{fontenc}
\usepackage{babel}


\newif\ifcomment
%\commenttrue # Show comments

\usepackage{physics}
\usepackage{amssymb}


\usepackage{amsthm}
% \usepackage{thmtools}
\usepackage{mathtools}
\usepackage{amsfonts}

\usepackage{color}

\usepackage{tikz}

\usepackage{geometry}
\geometry{a5paper, margin=0.1in, right=1cm}

\usepackage{dsfont}

\usepackage{graphicx}
\graphicspath{ {images/} }

\usepackage{faktor}

\usepackage{IEEEtrantools}
\usepackage{enumerate}   
\usepackage[PostScript=dvips]{"/Users/aware/Documents/Courses/diagrams"}


\newtheorem{theorem}{Théorème}[section]
\renewcommand{\thetheorem}{\arabic{theorem}}
\newtheorem{lemme}{Lemme}[section]
\renewcommand{\thelemme}{\arabic{lemme}}
\newtheorem{proposition}{Proposition}[section]
\renewcommand{\theproposition}{\arabic{proposition}}
\newtheorem{notations}{Notations}[section]
\newtheorem{problem}{Problème}[section]
\newtheorem{corollary}{Corollaire}[theorem]
\renewcommand{\thecorollary}{\arabic{corollary}}
\newtheorem{property}{Propriété}[section]
\newtheorem{objective}{Objectif}[section]

\theoremstyle{definition}
\newtheorem{definition}{Définition}[section]
\renewcommand{\thedefinition}{\arabic{definition}}
\newtheorem{exercise}{Exercice}[chapter]
\renewcommand{\theexercise}{\arabic{exercise}}
\newtheorem{example}{Exemple}[chapter]
\renewcommand{\theexample}{\arabic{example}}
\newtheorem*{solution}{Solution}
\newtheorem*{application}{Application}
\newtheorem*{notation}{Notation}
\newtheorem*{vocabulary}{Vocabulaire}
\newtheorem*{properties}{Propriétés}



\theoremstyle{remark}
\newtheorem*{remark}{Remarque}
\newtheorem*{rappel}{Rappel}


\usepackage{etoolbox}
\AtBeginEnvironment{exercise}{\small}
\AtBeginEnvironment{example}{\small}

\usepackage{cases}
\usepackage[red]{mypack}

\usepackage[framemethod=TikZ]{mdframed}

\definecolor{bg}{rgb}{0.4,0.25,0.95}
\definecolor{pagebg}{rgb}{0,0,0.5}
\surroundwithmdframed[
   topline=false,
   rightline=false,
   bottomline=false,
   leftmargin=\parindent,
   skipabove=8pt,
   skipbelow=8pt,
   linecolor=blue,
   innerbottommargin=10pt,
   % backgroundcolor=bg,font=\color{orange}\sffamily, fontcolor=white
]{definition}

\usepackage{empheq}
\usepackage[most]{tcolorbox}

\newtcbox{\mymath}[1][]{%
    nobeforeafter, math upper, tcbox raise base,
    enhanced, colframe=blue!30!black,
    colback=red!10, boxrule=1pt,
    #1}

\usepackage{unixode}


\DeclareMathOperator{\ord}{ord}
\DeclareMathOperator{\orb}{orb}
\DeclareMathOperator{\stab}{stab}
\DeclareMathOperator{\Stab}{stab}
\DeclareMathOperator{\ppcm}{ppcm}
\DeclareMathOperator{\conj}{Conj}
\DeclareMathOperator{\End}{End}
\DeclareMathOperator{\rot}{rot}
\DeclareMathOperator{\trs}{trace}
\DeclareMathOperator{\Ind}{Ind}
\DeclareMathOperator{\mat}{Mat}
\DeclareMathOperator{\id}{Id}
\DeclareMathOperator{\vect}{vect}
\DeclareMathOperator{\img}{img}
\DeclareMathOperator{\cov}{Cov}
\DeclareMathOperator{\dist}{dist}
\DeclareMathOperator{\irr}{Irr}
\DeclareMathOperator{\image}{Im}
\DeclareMathOperator{\pd}{\partial}
\DeclareMathOperator{\epi}{epi}
\DeclareMathOperator{\Argmin}{Argmin}
\DeclareMathOperator{\dom}{dom}
\DeclareMathOperator{\proj}{proj}
\DeclareMathOperator{\ctg}{ctg}
\DeclareMathOperator{\supp}{supp}
\DeclareMathOperator{\argmin}{argmin}
\DeclareMathOperator{\mult}{mult}
\DeclareMathOperator{\ch}{ch}
\DeclareMathOperator{\sh}{sh}
\DeclareMathOperator{\rang}{rang}
\DeclareMathOperator{\diam}{diam}
\DeclareMathOperator{\Epigraphe}{Epigraphe}




\usepackage{xcolor}
\everymath{\color{blue}}
%\everymath{\color[rgb]{0,1,1}}
%\pagecolor[rgb]{0,0,0.5}


\newcommand*{\pdtest}[3][]{\ensuremath{\frac{\partial^{#1} #2}{\partial #3}}}

\newcommand*{\deffunc}[6][]{\ensuremath{
\begin{array}{rcl}
#2 : #3 &\rightarrow& #4\\
#5 &\mapsto& #6
\end{array}
}}

\newcommand{\eqcolon}{\mathrel{\resizebox{\widthof{$\mathord{=}$}}{\height}{ $\!\!=\!\!\resizebox{1.2\width}{0.8\height}{\raisebox{0.23ex}{$\mathop{:}$}}\!\!$ }}}
\newcommand{\coloneq}{\mathrel{\resizebox{\widthof{$\mathord{=}$}}{\height}{ $\!\!\resizebox{1.2\width}{0.8\height}{\raisebox{0.23ex}{$\mathop{:}$}}\!\!=\!\!$ }}}
\newcommand{\eqcolonl}{\ensuremath{\mathrel{=\!\!\mathop{:}}}}
\newcommand{\coloneql}{\ensuremath{\mathrel{\mathop{:} \!\! =}}}
\newcommand{\vc}[1]{% inline column vector
  \left(\begin{smallmatrix}#1\end{smallmatrix}\right)%
}
\newcommand{\vr}[1]{% inline row vector
  \begin{smallmatrix}(\,#1\,)\end{smallmatrix}%
}
\makeatletter
\newcommand*{\defeq}{\ =\mathrel{\rlap{%
                     \raisebox{0.3ex}{$\m@th\cdot$}}%
                     \raisebox{-0.3ex}{$\m@th\cdot$}}%
                     }
\makeatother

\newcommand{\mathcircle}[1]{% inline row vector
 \overset{\circ}{#1}
}
\newcommand{\ulim}{% low limit
 \underline{\lim}
}
\newcommand{\ssi}{% iff
\iff
}
\newcommand{\ps}[2]{
\expval{#1 | #2}
}
\newcommand{\df}[1]{
\mqty{#1}
}
\newcommand{\n}[1]{
\norm{#1}
}
\newcommand{\sys}[1]{
\left\{\smqty{#1}\right.
}


\newcommand{\eqdef}{\ensuremath{\overset{\text{def}}=}}


\def\Circlearrowright{\ensuremath{%
  \rotatebox[origin=c]{230}{$\circlearrowright$}}}

\newcommand\ct[1]{\text{\rmfamily\upshape #1}}
\newcommand\question[1]{ {\color{red} ...!? \small #1}}
\newcommand\caz[1]{\left\{\begin{array} #1 \end{array}\right.}
\newcommand\const{\text{\rmfamily\upshape const}}
\newcommand\toP{ \overset{\pro}{\to}}
\newcommand\toPP{ \overset{\text{PP}}{\to}}
\newcommand{\oeq}{\mathrel{\text{\textcircled{$=$}}}}





\usepackage{xcolor}
% \usepackage[normalem]{ulem}
\usepackage{lipsum}
\makeatletter
% \newcommand\colorwave[1][blue]{\bgroup \markoverwith{\lower3.5\p@\hbox{\sixly \textcolor{#1}{\char58}}}\ULon}
%\font\sixly=lasy6 % does not re-load if already loaded, so no memory problem.

\newmdtheoremenv[
linewidth= 1pt,linecolor= blue,%
leftmargin=20,rightmargin=20,innertopmargin=0pt, innerrightmargin=40,%
tikzsetting = { draw=lightgray, line width = 0.3pt,dashed,%
dash pattern = on 15pt off 3pt},%
splittopskip=\topskip,skipbelow=\baselineskip,%
skipabove=\baselineskip,ntheorem,roundcorner=0pt,
% backgroundcolor=pagebg,font=\color{orange}\sffamily, fontcolor=white
]{examplebox}{Exemple}[section]



\newcommand\R{\mathbb{R}}
\newcommand\Z{\mathbb{Z}}
\newcommand\N{\mathbb{N}}
\newcommand\E{\mathbb{E}}
\newcommand\F{\mathcal{F}}
\newcommand\cH{\mathcal{H}}
\newcommand\V{\mathbb{V}}
\newcommand\dmo{ ^{-1} }
\newcommand\kapa{\kappa}
\newcommand\im{Im}
\newcommand\hs{\mathcal{H}}





\usepackage{soul}

\makeatletter
\newcommand*{\whiten}[1]{\llap{\textcolor{white}{{\the\SOUL@token}}\hspace{#1pt}}}
\DeclareRobustCommand*\myul{%
    \def\SOUL@everyspace{\underline{\space}\kern\z@}%
    \def\SOUL@everytoken{%
     \setbox0=\hbox{\the\SOUL@token}%
     \ifdim\dp0>\z@
        \raisebox{\dp0}{\underline{\phantom{\the\SOUL@token}}}%
        \whiten{1}\whiten{0}%
        \whiten{-1}\whiten{-2}%
        \llap{\the\SOUL@token}%
     \else
        \underline{\the\SOUL@token}%
     \fi}%
\SOUL@}
\makeatother

\newcommand*{\demp}{\fontfamily{lmtt}\selectfont}

\DeclareTextFontCommand{\textdemp}{\demp}

\begin{document}

\ifcomment
Multiline
comment
\fi
\ifcomment
\myul{Typesetting test}
% \color[rgb]{1,1,1}
$∑_i^n≠ 60º±∞π∆¬≈√j∫h≤≥µ$

$\CR \R\pro\ind\pro\gS\pro
\mqty[a&b\\c&d]$
$\pro\mathbb{P}$
$\dd{x}$

  \[
    \alpha(x)=\left\{
                \begin{array}{ll}
                  x\\
                  \frac{1}{1+e^{-kx}}\\
                  \frac{e^x-e^{-x}}{e^x+e^{-x}}
                \end{array}
              \right.
  \]

  $\expval{x}$
  
  $\chi_\rho(ghg\dmo)=\Tr(\rho_{ghg\dmo})=\Tr(\rho_g\circ\rho_h\circ\rho\dmo_g)=\Tr(\rho_h)\overset{\mbox{\scalebox{0.5}{$\Tr(AB)=\Tr(BA)$}}}{=}\chi_\rho(h)$
  	$\mathop{\oplus}_{\substack{x\in X}}$

$\mat(\rho_g)=(a_{ij}(g))_{\scriptsize \substack{1\leq i\leq d \\ 1\leq j\leq d}}$ et $\mat(\rho'_g)=(a'_{ij}(g))_{\scriptsize \substack{1\leq i'\leq d' \\ 1\leq j'\leq d'}}$



\[\int_a^b{\mathbb{R}^2}g(u, v)\dd{P_{XY}}(u, v)=\iint g(u,v) f_{XY}(u, v)\dd \lambda(u) \dd \lambda(v)\]
$$\lim_{x\to\infty} f(x)$$	
$$\iiiint_V \mu(t,u,v,w) \,dt\,du\,dv\,dw$$
$$\sum_{n=1}^{\infty} 2^{-n} = 1$$	
\begin{definition}
	Si $X$ et $Y$ sont 2 v.a. ou definit la \textsc{Covariance} entre $X$ et $Y$ comme
	$\cov(X,Y)\overset{\text{def}}{=}\E\left[(X-\E(X))(Y-\E(Y))\right]=\E(XY)-\E(X)\E(Y)$.
\end{definition}
\fi
\pagebreak

% \tableofcontents

% insert your code here
%\input{./algebra/main.tex}
%\input{./geometrie-differentielle/main.tex}
%\input{./probabilite/main.tex}
%\input{./analyse-fonctionnelle/main.tex}
% \input{./Analyse-convexe-et-dualite-en-optimisation/main.tex}
%\input{./tikz/main.tex}
%\input{./Theorie-du-distributions/main.tex}
%\input{./optimisation/mine.tex}
 \input{./modelisation/main.tex}

% yves.aubry@univ-tln.fr : algebra

\end{document}


% yves.aubry@univ-tln.fr : algebra

\end{document}

%% !TEX encoding = UTF-8 Unicode
% !TEX TS-program = xelatex

\documentclass[french]{report}

%\usepackage[utf8]{inputenc}
%\usepackage[T1]{fontenc}
\usepackage{babel}


\newif\ifcomment
%\commenttrue # Show comments

\usepackage{physics}
\usepackage{amssymb}


\usepackage{amsthm}
% \usepackage{thmtools}
\usepackage{mathtools}
\usepackage{amsfonts}

\usepackage{color}

\usepackage{tikz}

\usepackage{geometry}
\geometry{a5paper, margin=0.1in, right=1cm}

\usepackage{dsfont}

\usepackage{graphicx}
\graphicspath{ {images/} }

\usepackage{faktor}

\usepackage{IEEEtrantools}
\usepackage{enumerate}   
\usepackage[PostScript=dvips]{"/Users/aware/Documents/Courses/diagrams"}


\newtheorem{theorem}{Théorème}[section]
\renewcommand{\thetheorem}{\arabic{theorem}}
\newtheorem{lemme}{Lemme}[section]
\renewcommand{\thelemme}{\arabic{lemme}}
\newtheorem{proposition}{Proposition}[section]
\renewcommand{\theproposition}{\arabic{proposition}}
\newtheorem{notations}{Notations}[section]
\newtheorem{problem}{Problème}[section]
\newtheorem{corollary}{Corollaire}[theorem]
\renewcommand{\thecorollary}{\arabic{corollary}}
\newtheorem{property}{Propriété}[section]
\newtheorem{objective}{Objectif}[section]

\theoremstyle{definition}
\newtheorem{definition}{Définition}[section]
\renewcommand{\thedefinition}{\arabic{definition}}
\newtheorem{exercise}{Exercice}[chapter]
\renewcommand{\theexercise}{\arabic{exercise}}
\newtheorem{example}{Exemple}[chapter]
\renewcommand{\theexample}{\arabic{example}}
\newtheorem*{solution}{Solution}
\newtheorem*{application}{Application}
\newtheorem*{notation}{Notation}
\newtheorem*{vocabulary}{Vocabulaire}
\newtheorem*{properties}{Propriétés}



\theoremstyle{remark}
\newtheorem*{remark}{Remarque}
\newtheorem*{rappel}{Rappel}


\usepackage{etoolbox}
\AtBeginEnvironment{exercise}{\small}
\AtBeginEnvironment{example}{\small}

\usepackage{cases}
\usepackage[red]{mypack}

\usepackage[framemethod=TikZ]{mdframed}

\definecolor{bg}{rgb}{0.4,0.25,0.95}
\definecolor{pagebg}{rgb}{0,0,0.5}
\surroundwithmdframed[
   topline=false,
   rightline=false,
   bottomline=false,
   leftmargin=\parindent,
   skipabove=8pt,
   skipbelow=8pt,
   linecolor=blue,
   innerbottommargin=10pt,
   % backgroundcolor=bg,font=\color{orange}\sffamily, fontcolor=white
]{definition}

\usepackage{empheq}
\usepackage[most]{tcolorbox}

\newtcbox{\mymath}[1][]{%
    nobeforeafter, math upper, tcbox raise base,
    enhanced, colframe=blue!30!black,
    colback=red!10, boxrule=1pt,
    #1}

\usepackage{unixode}


\DeclareMathOperator{\ord}{ord}
\DeclareMathOperator{\orb}{orb}
\DeclareMathOperator{\stab}{stab}
\DeclareMathOperator{\Stab}{stab}
\DeclareMathOperator{\ppcm}{ppcm}
\DeclareMathOperator{\conj}{Conj}
\DeclareMathOperator{\End}{End}
\DeclareMathOperator{\rot}{rot}
\DeclareMathOperator{\trs}{trace}
\DeclareMathOperator{\Ind}{Ind}
\DeclareMathOperator{\mat}{Mat}
\DeclareMathOperator{\id}{Id}
\DeclareMathOperator{\vect}{vect}
\DeclareMathOperator{\img}{img}
\DeclareMathOperator{\cov}{Cov}
\DeclareMathOperator{\dist}{dist}
\DeclareMathOperator{\irr}{Irr}
\DeclareMathOperator{\image}{Im}
\DeclareMathOperator{\pd}{\partial}
\DeclareMathOperator{\epi}{epi}
\DeclareMathOperator{\Argmin}{Argmin}
\DeclareMathOperator{\dom}{dom}
\DeclareMathOperator{\proj}{proj}
\DeclareMathOperator{\ctg}{ctg}
\DeclareMathOperator{\supp}{supp}
\DeclareMathOperator{\argmin}{argmin}
\DeclareMathOperator{\mult}{mult}
\DeclareMathOperator{\ch}{ch}
\DeclareMathOperator{\sh}{sh}
\DeclareMathOperator{\rang}{rang}
\DeclareMathOperator{\diam}{diam}
\DeclareMathOperator{\Epigraphe}{Epigraphe}




\usepackage{xcolor}
\everymath{\color{blue}}
%\everymath{\color[rgb]{0,1,1}}
%\pagecolor[rgb]{0,0,0.5}


\newcommand*{\pdtest}[3][]{\ensuremath{\frac{\partial^{#1} #2}{\partial #3}}}

\newcommand*{\deffunc}[6][]{\ensuremath{
\begin{array}{rcl}
#2 : #3 &\rightarrow& #4\\
#5 &\mapsto& #6
\end{array}
}}

\newcommand{\eqcolon}{\mathrel{\resizebox{\widthof{$\mathord{=}$}}{\height}{ $\!\!=\!\!\resizebox{1.2\width}{0.8\height}{\raisebox{0.23ex}{$\mathop{:}$}}\!\!$ }}}
\newcommand{\coloneq}{\mathrel{\resizebox{\widthof{$\mathord{=}$}}{\height}{ $\!\!\resizebox{1.2\width}{0.8\height}{\raisebox{0.23ex}{$\mathop{:}$}}\!\!=\!\!$ }}}
\newcommand{\eqcolonl}{\ensuremath{\mathrel{=\!\!\mathop{:}}}}
\newcommand{\coloneql}{\ensuremath{\mathrel{\mathop{:} \!\! =}}}
\newcommand{\vc}[1]{% inline column vector
  \left(\begin{smallmatrix}#1\end{smallmatrix}\right)%
}
\newcommand{\vr}[1]{% inline row vector
  \begin{smallmatrix}(\,#1\,)\end{smallmatrix}%
}
\makeatletter
\newcommand*{\defeq}{\ =\mathrel{\rlap{%
                     \raisebox{0.3ex}{$\m@th\cdot$}}%
                     \raisebox{-0.3ex}{$\m@th\cdot$}}%
                     }
\makeatother

\newcommand{\mathcircle}[1]{% inline row vector
 \overset{\circ}{#1}
}
\newcommand{\ulim}{% low limit
 \underline{\lim}
}
\newcommand{\ssi}{% iff
\iff
}
\newcommand{\ps}[2]{
\expval{#1 | #2}
}
\newcommand{\df}[1]{
\mqty{#1}
}
\newcommand{\n}[1]{
\norm{#1}
}
\newcommand{\sys}[1]{
\left\{\smqty{#1}\right.
}


\newcommand{\eqdef}{\ensuremath{\overset{\text{def}}=}}


\def\Circlearrowright{\ensuremath{%
  \rotatebox[origin=c]{230}{$\circlearrowright$}}}

\newcommand\ct[1]{\text{\rmfamily\upshape #1}}
\newcommand\question[1]{ {\color{red} ...!? \small #1}}
\newcommand\caz[1]{\left\{\begin{array} #1 \end{array}\right.}
\newcommand\const{\text{\rmfamily\upshape const}}
\newcommand\toP{ \overset{\pro}{\to}}
\newcommand\toPP{ \overset{\text{PP}}{\to}}
\newcommand{\oeq}{\mathrel{\text{\textcircled{$=$}}}}





\usepackage{xcolor}
% \usepackage[normalem]{ulem}
\usepackage{lipsum}
\makeatletter
% \newcommand\colorwave[1][blue]{\bgroup \markoverwith{\lower3.5\p@\hbox{\sixly \textcolor{#1}{\char58}}}\ULon}
%\font\sixly=lasy6 % does not re-load if already loaded, so no memory problem.

\newmdtheoremenv[
linewidth= 1pt,linecolor= blue,%
leftmargin=20,rightmargin=20,innertopmargin=0pt, innerrightmargin=40,%
tikzsetting = { draw=lightgray, line width = 0.3pt,dashed,%
dash pattern = on 15pt off 3pt},%
splittopskip=\topskip,skipbelow=\baselineskip,%
skipabove=\baselineskip,ntheorem,roundcorner=0pt,
% backgroundcolor=pagebg,font=\color{orange}\sffamily, fontcolor=white
]{examplebox}{Exemple}[section]



\newcommand\R{\mathbb{R}}
\newcommand\Z{\mathbb{Z}}
\newcommand\N{\mathbb{N}}
\newcommand\E{\mathbb{E}}
\newcommand\F{\mathcal{F}}
\newcommand\cH{\mathcal{H}}
\newcommand\V{\mathbb{V}}
\newcommand\dmo{ ^{-1} }
\newcommand\kapa{\kappa}
\newcommand\im{Im}
\newcommand\hs{\mathcal{H}}





\usepackage{soul}

\makeatletter
\newcommand*{\whiten}[1]{\llap{\textcolor{white}{{\the\SOUL@token}}\hspace{#1pt}}}
\DeclareRobustCommand*\myul{%
    \def\SOUL@everyspace{\underline{\space}\kern\z@}%
    \def\SOUL@everytoken{%
     \setbox0=\hbox{\the\SOUL@token}%
     \ifdim\dp0>\z@
        \raisebox{\dp0}{\underline{\phantom{\the\SOUL@token}}}%
        \whiten{1}\whiten{0}%
        \whiten{-1}\whiten{-2}%
        \llap{\the\SOUL@token}%
     \else
        \underline{\the\SOUL@token}%
     \fi}%
\SOUL@}
\makeatother

\newcommand*{\demp}{\fontfamily{lmtt}\selectfont}

\DeclareTextFontCommand{\textdemp}{\demp}

\begin{document}

\ifcomment
Multiline
comment
\fi
\ifcomment
\myul{Typesetting test}
% \color[rgb]{1,1,1}
$∑_i^n≠ 60º±∞π∆¬≈√j∫h≤≥µ$

$\CR \R\pro\ind\pro\gS\pro
\mqty[a&b\\c&d]$
$\pro\mathbb{P}$
$\dd{x}$

  \[
    \alpha(x)=\left\{
                \begin{array}{ll}
                  x\\
                  \frac{1}{1+e^{-kx}}\\
                  \frac{e^x-e^{-x}}{e^x+e^{-x}}
                \end{array}
              \right.
  \]

  $\expval{x}$
  
  $\chi_\rho(ghg\dmo)=\Tr(\rho_{ghg\dmo})=\Tr(\rho_g\circ\rho_h\circ\rho\dmo_g)=\Tr(\rho_h)\overset{\mbox{\scalebox{0.5}{$\Tr(AB)=\Tr(BA)$}}}{=}\chi_\rho(h)$
  	$\mathop{\oplus}_{\substack{x\in X}}$

$\mat(\rho_g)=(a_{ij}(g))_{\scriptsize \substack{1\leq i\leq d \\ 1\leq j\leq d}}$ et $\mat(\rho'_g)=(a'_{ij}(g))_{\scriptsize \substack{1\leq i'\leq d' \\ 1\leq j'\leq d'}}$



\[\int_a^b{\mathbb{R}^2}g(u, v)\dd{P_{XY}}(u, v)=\iint g(u,v) f_{XY}(u, v)\dd \lambda(u) \dd \lambda(v)\]
$$\lim_{x\to\infty} f(x)$$	
$$\iiiint_V \mu(t,u,v,w) \,dt\,du\,dv\,dw$$
$$\sum_{n=1}^{\infty} 2^{-n} = 1$$	
\begin{definition}
	Si $X$ et $Y$ sont 2 v.a. ou definit la \textsc{Covariance} entre $X$ et $Y$ comme
	$\cov(X,Y)\overset{\text{def}}{=}\E\left[(X-\E(X))(Y-\E(Y))\right]=\E(XY)-\E(X)\E(Y)$.
\end{definition}
\fi
\pagebreak

% \tableofcontents

% insert your code here
%% !TEX encoding = UTF-8 Unicode
% !TEX TS-program = xelatex

\documentclass[french]{report}

%\usepackage[utf8]{inputenc}
%\usepackage[T1]{fontenc}
\usepackage{babel}


\newif\ifcomment
%\commenttrue # Show comments

\usepackage{physics}
\usepackage{amssymb}


\usepackage{amsthm}
% \usepackage{thmtools}
\usepackage{mathtools}
\usepackage{amsfonts}

\usepackage{color}

\usepackage{tikz}

\usepackage{geometry}
\geometry{a5paper, margin=0.1in, right=1cm}

\usepackage{dsfont}

\usepackage{graphicx}
\graphicspath{ {images/} }

\usepackage{faktor}

\usepackage{IEEEtrantools}
\usepackage{enumerate}   
\usepackage[PostScript=dvips]{"/Users/aware/Documents/Courses/diagrams"}


\newtheorem{theorem}{Théorème}[section]
\renewcommand{\thetheorem}{\arabic{theorem}}
\newtheorem{lemme}{Lemme}[section]
\renewcommand{\thelemme}{\arabic{lemme}}
\newtheorem{proposition}{Proposition}[section]
\renewcommand{\theproposition}{\arabic{proposition}}
\newtheorem{notations}{Notations}[section]
\newtheorem{problem}{Problème}[section]
\newtheorem{corollary}{Corollaire}[theorem]
\renewcommand{\thecorollary}{\arabic{corollary}}
\newtheorem{property}{Propriété}[section]
\newtheorem{objective}{Objectif}[section]

\theoremstyle{definition}
\newtheorem{definition}{Définition}[section]
\renewcommand{\thedefinition}{\arabic{definition}}
\newtheorem{exercise}{Exercice}[chapter]
\renewcommand{\theexercise}{\arabic{exercise}}
\newtheorem{example}{Exemple}[chapter]
\renewcommand{\theexample}{\arabic{example}}
\newtheorem*{solution}{Solution}
\newtheorem*{application}{Application}
\newtheorem*{notation}{Notation}
\newtheorem*{vocabulary}{Vocabulaire}
\newtheorem*{properties}{Propriétés}



\theoremstyle{remark}
\newtheorem*{remark}{Remarque}
\newtheorem*{rappel}{Rappel}


\usepackage{etoolbox}
\AtBeginEnvironment{exercise}{\small}
\AtBeginEnvironment{example}{\small}

\usepackage{cases}
\usepackage[red]{mypack}

\usepackage[framemethod=TikZ]{mdframed}

\definecolor{bg}{rgb}{0.4,0.25,0.95}
\definecolor{pagebg}{rgb}{0,0,0.5}
\surroundwithmdframed[
   topline=false,
   rightline=false,
   bottomline=false,
   leftmargin=\parindent,
   skipabove=8pt,
   skipbelow=8pt,
   linecolor=blue,
   innerbottommargin=10pt,
   % backgroundcolor=bg,font=\color{orange}\sffamily, fontcolor=white
]{definition}

\usepackage{empheq}
\usepackage[most]{tcolorbox}

\newtcbox{\mymath}[1][]{%
    nobeforeafter, math upper, tcbox raise base,
    enhanced, colframe=blue!30!black,
    colback=red!10, boxrule=1pt,
    #1}

\usepackage{unixode}


\DeclareMathOperator{\ord}{ord}
\DeclareMathOperator{\orb}{orb}
\DeclareMathOperator{\stab}{stab}
\DeclareMathOperator{\Stab}{stab}
\DeclareMathOperator{\ppcm}{ppcm}
\DeclareMathOperator{\conj}{Conj}
\DeclareMathOperator{\End}{End}
\DeclareMathOperator{\rot}{rot}
\DeclareMathOperator{\trs}{trace}
\DeclareMathOperator{\Ind}{Ind}
\DeclareMathOperator{\mat}{Mat}
\DeclareMathOperator{\id}{Id}
\DeclareMathOperator{\vect}{vect}
\DeclareMathOperator{\img}{img}
\DeclareMathOperator{\cov}{Cov}
\DeclareMathOperator{\dist}{dist}
\DeclareMathOperator{\irr}{Irr}
\DeclareMathOperator{\image}{Im}
\DeclareMathOperator{\pd}{\partial}
\DeclareMathOperator{\epi}{epi}
\DeclareMathOperator{\Argmin}{Argmin}
\DeclareMathOperator{\dom}{dom}
\DeclareMathOperator{\proj}{proj}
\DeclareMathOperator{\ctg}{ctg}
\DeclareMathOperator{\supp}{supp}
\DeclareMathOperator{\argmin}{argmin}
\DeclareMathOperator{\mult}{mult}
\DeclareMathOperator{\ch}{ch}
\DeclareMathOperator{\sh}{sh}
\DeclareMathOperator{\rang}{rang}
\DeclareMathOperator{\diam}{diam}
\DeclareMathOperator{\Epigraphe}{Epigraphe}




\usepackage{xcolor}
\everymath{\color{blue}}
%\everymath{\color[rgb]{0,1,1}}
%\pagecolor[rgb]{0,0,0.5}


\newcommand*{\pdtest}[3][]{\ensuremath{\frac{\partial^{#1} #2}{\partial #3}}}

\newcommand*{\deffunc}[6][]{\ensuremath{
\begin{array}{rcl}
#2 : #3 &\rightarrow& #4\\
#5 &\mapsto& #6
\end{array}
}}

\newcommand{\eqcolon}{\mathrel{\resizebox{\widthof{$\mathord{=}$}}{\height}{ $\!\!=\!\!\resizebox{1.2\width}{0.8\height}{\raisebox{0.23ex}{$\mathop{:}$}}\!\!$ }}}
\newcommand{\coloneq}{\mathrel{\resizebox{\widthof{$\mathord{=}$}}{\height}{ $\!\!\resizebox{1.2\width}{0.8\height}{\raisebox{0.23ex}{$\mathop{:}$}}\!\!=\!\!$ }}}
\newcommand{\eqcolonl}{\ensuremath{\mathrel{=\!\!\mathop{:}}}}
\newcommand{\coloneql}{\ensuremath{\mathrel{\mathop{:} \!\! =}}}
\newcommand{\vc}[1]{% inline column vector
  \left(\begin{smallmatrix}#1\end{smallmatrix}\right)%
}
\newcommand{\vr}[1]{% inline row vector
  \begin{smallmatrix}(\,#1\,)\end{smallmatrix}%
}
\makeatletter
\newcommand*{\defeq}{\ =\mathrel{\rlap{%
                     \raisebox{0.3ex}{$\m@th\cdot$}}%
                     \raisebox{-0.3ex}{$\m@th\cdot$}}%
                     }
\makeatother

\newcommand{\mathcircle}[1]{% inline row vector
 \overset{\circ}{#1}
}
\newcommand{\ulim}{% low limit
 \underline{\lim}
}
\newcommand{\ssi}{% iff
\iff
}
\newcommand{\ps}[2]{
\expval{#1 | #2}
}
\newcommand{\df}[1]{
\mqty{#1}
}
\newcommand{\n}[1]{
\norm{#1}
}
\newcommand{\sys}[1]{
\left\{\smqty{#1}\right.
}


\newcommand{\eqdef}{\ensuremath{\overset{\text{def}}=}}


\def\Circlearrowright{\ensuremath{%
  \rotatebox[origin=c]{230}{$\circlearrowright$}}}

\newcommand\ct[1]{\text{\rmfamily\upshape #1}}
\newcommand\question[1]{ {\color{red} ...!? \small #1}}
\newcommand\caz[1]{\left\{\begin{array} #1 \end{array}\right.}
\newcommand\const{\text{\rmfamily\upshape const}}
\newcommand\toP{ \overset{\pro}{\to}}
\newcommand\toPP{ \overset{\text{PP}}{\to}}
\newcommand{\oeq}{\mathrel{\text{\textcircled{$=$}}}}





\usepackage{xcolor}
% \usepackage[normalem]{ulem}
\usepackage{lipsum}
\makeatletter
% \newcommand\colorwave[1][blue]{\bgroup \markoverwith{\lower3.5\p@\hbox{\sixly \textcolor{#1}{\char58}}}\ULon}
%\font\sixly=lasy6 % does not re-load if already loaded, so no memory problem.

\newmdtheoremenv[
linewidth= 1pt,linecolor= blue,%
leftmargin=20,rightmargin=20,innertopmargin=0pt, innerrightmargin=40,%
tikzsetting = { draw=lightgray, line width = 0.3pt,dashed,%
dash pattern = on 15pt off 3pt},%
splittopskip=\topskip,skipbelow=\baselineskip,%
skipabove=\baselineskip,ntheorem,roundcorner=0pt,
% backgroundcolor=pagebg,font=\color{orange}\sffamily, fontcolor=white
]{examplebox}{Exemple}[section]



\newcommand\R{\mathbb{R}}
\newcommand\Z{\mathbb{Z}}
\newcommand\N{\mathbb{N}}
\newcommand\E{\mathbb{E}}
\newcommand\F{\mathcal{F}}
\newcommand\cH{\mathcal{H}}
\newcommand\V{\mathbb{V}}
\newcommand\dmo{ ^{-1} }
\newcommand\kapa{\kappa}
\newcommand\im{Im}
\newcommand\hs{\mathcal{H}}





\usepackage{soul}

\makeatletter
\newcommand*{\whiten}[1]{\llap{\textcolor{white}{{\the\SOUL@token}}\hspace{#1pt}}}
\DeclareRobustCommand*\myul{%
    \def\SOUL@everyspace{\underline{\space}\kern\z@}%
    \def\SOUL@everytoken{%
     \setbox0=\hbox{\the\SOUL@token}%
     \ifdim\dp0>\z@
        \raisebox{\dp0}{\underline{\phantom{\the\SOUL@token}}}%
        \whiten{1}\whiten{0}%
        \whiten{-1}\whiten{-2}%
        \llap{\the\SOUL@token}%
     \else
        \underline{\the\SOUL@token}%
     \fi}%
\SOUL@}
\makeatother

\newcommand*{\demp}{\fontfamily{lmtt}\selectfont}

\DeclareTextFontCommand{\textdemp}{\demp}

\begin{document}

\ifcomment
Multiline
comment
\fi
\ifcomment
\myul{Typesetting test}
% \color[rgb]{1,1,1}
$∑_i^n≠ 60º±∞π∆¬≈√j∫h≤≥µ$

$\CR \R\pro\ind\pro\gS\pro
\mqty[a&b\\c&d]$
$\pro\mathbb{P}$
$\dd{x}$

  \[
    \alpha(x)=\left\{
                \begin{array}{ll}
                  x\\
                  \frac{1}{1+e^{-kx}}\\
                  \frac{e^x-e^{-x}}{e^x+e^{-x}}
                \end{array}
              \right.
  \]

  $\expval{x}$
  
  $\chi_\rho(ghg\dmo)=\Tr(\rho_{ghg\dmo})=\Tr(\rho_g\circ\rho_h\circ\rho\dmo_g)=\Tr(\rho_h)\overset{\mbox{\scalebox{0.5}{$\Tr(AB)=\Tr(BA)$}}}{=}\chi_\rho(h)$
  	$\mathop{\oplus}_{\substack{x\in X}}$

$\mat(\rho_g)=(a_{ij}(g))_{\scriptsize \substack{1\leq i\leq d \\ 1\leq j\leq d}}$ et $\mat(\rho'_g)=(a'_{ij}(g))_{\scriptsize \substack{1\leq i'\leq d' \\ 1\leq j'\leq d'}}$



\[\int_a^b{\mathbb{R}^2}g(u, v)\dd{P_{XY}}(u, v)=\iint g(u,v) f_{XY}(u, v)\dd \lambda(u) \dd \lambda(v)\]
$$\lim_{x\to\infty} f(x)$$	
$$\iiiint_V \mu(t,u,v,w) \,dt\,du\,dv\,dw$$
$$\sum_{n=1}^{\infty} 2^{-n} = 1$$	
\begin{definition}
	Si $X$ et $Y$ sont 2 v.a. ou definit la \textsc{Covariance} entre $X$ et $Y$ comme
	$\cov(X,Y)\overset{\text{def}}{=}\E\left[(X-\E(X))(Y-\E(Y))\right]=\E(XY)-\E(X)\E(Y)$.
\end{definition}
\fi
\pagebreak

% \tableofcontents

% insert your code here
%\input{./algebra/main.tex}
%\input{./geometrie-differentielle/main.tex}
%\input{./probabilite/main.tex}
%\input{./analyse-fonctionnelle/main.tex}
% \input{./Analyse-convexe-et-dualite-en-optimisation/main.tex}
%\input{./tikz/main.tex}
%\input{./Theorie-du-distributions/main.tex}
%\input{./optimisation/mine.tex}
 \input{./modelisation/main.tex}

% yves.aubry@univ-tln.fr : algebra

\end{document}

%% !TEX encoding = UTF-8 Unicode
% !TEX TS-program = xelatex

\documentclass[french]{report}

%\usepackage[utf8]{inputenc}
%\usepackage[T1]{fontenc}
\usepackage{babel}


\newif\ifcomment
%\commenttrue # Show comments

\usepackage{physics}
\usepackage{amssymb}


\usepackage{amsthm}
% \usepackage{thmtools}
\usepackage{mathtools}
\usepackage{amsfonts}

\usepackage{color}

\usepackage{tikz}

\usepackage{geometry}
\geometry{a5paper, margin=0.1in, right=1cm}

\usepackage{dsfont}

\usepackage{graphicx}
\graphicspath{ {images/} }

\usepackage{faktor}

\usepackage{IEEEtrantools}
\usepackage{enumerate}   
\usepackage[PostScript=dvips]{"/Users/aware/Documents/Courses/diagrams"}


\newtheorem{theorem}{Théorème}[section]
\renewcommand{\thetheorem}{\arabic{theorem}}
\newtheorem{lemme}{Lemme}[section]
\renewcommand{\thelemme}{\arabic{lemme}}
\newtheorem{proposition}{Proposition}[section]
\renewcommand{\theproposition}{\arabic{proposition}}
\newtheorem{notations}{Notations}[section]
\newtheorem{problem}{Problème}[section]
\newtheorem{corollary}{Corollaire}[theorem]
\renewcommand{\thecorollary}{\arabic{corollary}}
\newtheorem{property}{Propriété}[section]
\newtheorem{objective}{Objectif}[section]

\theoremstyle{definition}
\newtheorem{definition}{Définition}[section]
\renewcommand{\thedefinition}{\arabic{definition}}
\newtheorem{exercise}{Exercice}[chapter]
\renewcommand{\theexercise}{\arabic{exercise}}
\newtheorem{example}{Exemple}[chapter]
\renewcommand{\theexample}{\arabic{example}}
\newtheorem*{solution}{Solution}
\newtheorem*{application}{Application}
\newtheorem*{notation}{Notation}
\newtheorem*{vocabulary}{Vocabulaire}
\newtheorem*{properties}{Propriétés}



\theoremstyle{remark}
\newtheorem*{remark}{Remarque}
\newtheorem*{rappel}{Rappel}


\usepackage{etoolbox}
\AtBeginEnvironment{exercise}{\small}
\AtBeginEnvironment{example}{\small}

\usepackage{cases}
\usepackage[red]{mypack}

\usepackage[framemethod=TikZ]{mdframed}

\definecolor{bg}{rgb}{0.4,0.25,0.95}
\definecolor{pagebg}{rgb}{0,0,0.5}
\surroundwithmdframed[
   topline=false,
   rightline=false,
   bottomline=false,
   leftmargin=\parindent,
   skipabove=8pt,
   skipbelow=8pt,
   linecolor=blue,
   innerbottommargin=10pt,
   % backgroundcolor=bg,font=\color{orange}\sffamily, fontcolor=white
]{definition}

\usepackage{empheq}
\usepackage[most]{tcolorbox}

\newtcbox{\mymath}[1][]{%
    nobeforeafter, math upper, tcbox raise base,
    enhanced, colframe=blue!30!black,
    colback=red!10, boxrule=1pt,
    #1}

\usepackage{unixode}


\DeclareMathOperator{\ord}{ord}
\DeclareMathOperator{\orb}{orb}
\DeclareMathOperator{\stab}{stab}
\DeclareMathOperator{\Stab}{stab}
\DeclareMathOperator{\ppcm}{ppcm}
\DeclareMathOperator{\conj}{Conj}
\DeclareMathOperator{\End}{End}
\DeclareMathOperator{\rot}{rot}
\DeclareMathOperator{\trs}{trace}
\DeclareMathOperator{\Ind}{Ind}
\DeclareMathOperator{\mat}{Mat}
\DeclareMathOperator{\id}{Id}
\DeclareMathOperator{\vect}{vect}
\DeclareMathOperator{\img}{img}
\DeclareMathOperator{\cov}{Cov}
\DeclareMathOperator{\dist}{dist}
\DeclareMathOperator{\irr}{Irr}
\DeclareMathOperator{\image}{Im}
\DeclareMathOperator{\pd}{\partial}
\DeclareMathOperator{\epi}{epi}
\DeclareMathOperator{\Argmin}{Argmin}
\DeclareMathOperator{\dom}{dom}
\DeclareMathOperator{\proj}{proj}
\DeclareMathOperator{\ctg}{ctg}
\DeclareMathOperator{\supp}{supp}
\DeclareMathOperator{\argmin}{argmin}
\DeclareMathOperator{\mult}{mult}
\DeclareMathOperator{\ch}{ch}
\DeclareMathOperator{\sh}{sh}
\DeclareMathOperator{\rang}{rang}
\DeclareMathOperator{\diam}{diam}
\DeclareMathOperator{\Epigraphe}{Epigraphe}




\usepackage{xcolor}
\everymath{\color{blue}}
%\everymath{\color[rgb]{0,1,1}}
%\pagecolor[rgb]{0,0,0.5}


\newcommand*{\pdtest}[3][]{\ensuremath{\frac{\partial^{#1} #2}{\partial #3}}}

\newcommand*{\deffunc}[6][]{\ensuremath{
\begin{array}{rcl}
#2 : #3 &\rightarrow& #4\\
#5 &\mapsto& #6
\end{array}
}}

\newcommand{\eqcolon}{\mathrel{\resizebox{\widthof{$\mathord{=}$}}{\height}{ $\!\!=\!\!\resizebox{1.2\width}{0.8\height}{\raisebox{0.23ex}{$\mathop{:}$}}\!\!$ }}}
\newcommand{\coloneq}{\mathrel{\resizebox{\widthof{$\mathord{=}$}}{\height}{ $\!\!\resizebox{1.2\width}{0.8\height}{\raisebox{0.23ex}{$\mathop{:}$}}\!\!=\!\!$ }}}
\newcommand{\eqcolonl}{\ensuremath{\mathrel{=\!\!\mathop{:}}}}
\newcommand{\coloneql}{\ensuremath{\mathrel{\mathop{:} \!\! =}}}
\newcommand{\vc}[1]{% inline column vector
  \left(\begin{smallmatrix}#1\end{smallmatrix}\right)%
}
\newcommand{\vr}[1]{% inline row vector
  \begin{smallmatrix}(\,#1\,)\end{smallmatrix}%
}
\makeatletter
\newcommand*{\defeq}{\ =\mathrel{\rlap{%
                     \raisebox{0.3ex}{$\m@th\cdot$}}%
                     \raisebox{-0.3ex}{$\m@th\cdot$}}%
                     }
\makeatother

\newcommand{\mathcircle}[1]{% inline row vector
 \overset{\circ}{#1}
}
\newcommand{\ulim}{% low limit
 \underline{\lim}
}
\newcommand{\ssi}{% iff
\iff
}
\newcommand{\ps}[2]{
\expval{#1 | #2}
}
\newcommand{\df}[1]{
\mqty{#1}
}
\newcommand{\n}[1]{
\norm{#1}
}
\newcommand{\sys}[1]{
\left\{\smqty{#1}\right.
}


\newcommand{\eqdef}{\ensuremath{\overset{\text{def}}=}}


\def\Circlearrowright{\ensuremath{%
  \rotatebox[origin=c]{230}{$\circlearrowright$}}}

\newcommand\ct[1]{\text{\rmfamily\upshape #1}}
\newcommand\question[1]{ {\color{red} ...!? \small #1}}
\newcommand\caz[1]{\left\{\begin{array} #1 \end{array}\right.}
\newcommand\const{\text{\rmfamily\upshape const}}
\newcommand\toP{ \overset{\pro}{\to}}
\newcommand\toPP{ \overset{\text{PP}}{\to}}
\newcommand{\oeq}{\mathrel{\text{\textcircled{$=$}}}}





\usepackage{xcolor}
% \usepackage[normalem]{ulem}
\usepackage{lipsum}
\makeatletter
% \newcommand\colorwave[1][blue]{\bgroup \markoverwith{\lower3.5\p@\hbox{\sixly \textcolor{#1}{\char58}}}\ULon}
%\font\sixly=lasy6 % does not re-load if already loaded, so no memory problem.

\newmdtheoremenv[
linewidth= 1pt,linecolor= blue,%
leftmargin=20,rightmargin=20,innertopmargin=0pt, innerrightmargin=40,%
tikzsetting = { draw=lightgray, line width = 0.3pt,dashed,%
dash pattern = on 15pt off 3pt},%
splittopskip=\topskip,skipbelow=\baselineskip,%
skipabove=\baselineskip,ntheorem,roundcorner=0pt,
% backgroundcolor=pagebg,font=\color{orange}\sffamily, fontcolor=white
]{examplebox}{Exemple}[section]



\newcommand\R{\mathbb{R}}
\newcommand\Z{\mathbb{Z}}
\newcommand\N{\mathbb{N}}
\newcommand\E{\mathbb{E}}
\newcommand\F{\mathcal{F}}
\newcommand\cH{\mathcal{H}}
\newcommand\V{\mathbb{V}}
\newcommand\dmo{ ^{-1} }
\newcommand\kapa{\kappa}
\newcommand\im{Im}
\newcommand\hs{\mathcal{H}}





\usepackage{soul}

\makeatletter
\newcommand*{\whiten}[1]{\llap{\textcolor{white}{{\the\SOUL@token}}\hspace{#1pt}}}
\DeclareRobustCommand*\myul{%
    \def\SOUL@everyspace{\underline{\space}\kern\z@}%
    \def\SOUL@everytoken{%
     \setbox0=\hbox{\the\SOUL@token}%
     \ifdim\dp0>\z@
        \raisebox{\dp0}{\underline{\phantom{\the\SOUL@token}}}%
        \whiten{1}\whiten{0}%
        \whiten{-1}\whiten{-2}%
        \llap{\the\SOUL@token}%
     \else
        \underline{\the\SOUL@token}%
     \fi}%
\SOUL@}
\makeatother

\newcommand*{\demp}{\fontfamily{lmtt}\selectfont}

\DeclareTextFontCommand{\textdemp}{\demp}

\begin{document}

\ifcomment
Multiline
comment
\fi
\ifcomment
\myul{Typesetting test}
% \color[rgb]{1,1,1}
$∑_i^n≠ 60º±∞π∆¬≈√j∫h≤≥µ$

$\CR \R\pro\ind\pro\gS\pro
\mqty[a&b\\c&d]$
$\pro\mathbb{P}$
$\dd{x}$

  \[
    \alpha(x)=\left\{
                \begin{array}{ll}
                  x\\
                  \frac{1}{1+e^{-kx}}\\
                  \frac{e^x-e^{-x}}{e^x+e^{-x}}
                \end{array}
              \right.
  \]

  $\expval{x}$
  
  $\chi_\rho(ghg\dmo)=\Tr(\rho_{ghg\dmo})=\Tr(\rho_g\circ\rho_h\circ\rho\dmo_g)=\Tr(\rho_h)\overset{\mbox{\scalebox{0.5}{$\Tr(AB)=\Tr(BA)$}}}{=}\chi_\rho(h)$
  	$\mathop{\oplus}_{\substack{x\in X}}$

$\mat(\rho_g)=(a_{ij}(g))_{\scriptsize \substack{1\leq i\leq d \\ 1\leq j\leq d}}$ et $\mat(\rho'_g)=(a'_{ij}(g))_{\scriptsize \substack{1\leq i'\leq d' \\ 1\leq j'\leq d'}}$



\[\int_a^b{\mathbb{R}^2}g(u, v)\dd{P_{XY}}(u, v)=\iint g(u,v) f_{XY}(u, v)\dd \lambda(u) \dd \lambda(v)\]
$$\lim_{x\to\infty} f(x)$$	
$$\iiiint_V \mu(t,u,v,w) \,dt\,du\,dv\,dw$$
$$\sum_{n=1}^{\infty} 2^{-n} = 1$$	
\begin{definition}
	Si $X$ et $Y$ sont 2 v.a. ou definit la \textsc{Covariance} entre $X$ et $Y$ comme
	$\cov(X,Y)\overset{\text{def}}{=}\E\left[(X-\E(X))(Y-\E(Y))\right]=\E(XY)-\E(X)\E(Y)$.
\end{definition}
\fi
\pagebreak

% \tableofcontents

% insert your code here
%\input{./algebra/main.tex}
%\input{./geometrie-differentielle/main.tex}
%\input{./probabilite/main.tex}
%\input{./analyse-fonctionnelle/main.tex}
% \input{./Analyse-convexe-et-dualite-en-optimisation/main.tex}
%\input{./tikz/main.tex}
%\input{./Theorie-du-distributions/main.tex}
%\input{./optimisation/mine.tex}
 \input{./modelisation/main.tex}

% yves.aubry@univ-tln.fr : algebra

\end{document}

%% !TEX encoding = UTF-8 Unicode
% !TEX TS-program = xelatex

\documentclass[french]{report}

%\usepackage[utf8]{inputenc}
%\usepackage[T1]{fontenc}
\usepackage{babel}


\newif\ifcomment
%\commenttrue # Show comments

\usepackage{physics}
\usepackage{amssymb}


\usepackage{amsthm}
% \usepackage{thmtools}
\usepackage{mathtools}
\usepackage{amsfonts}

\usepackage{color}

\usepackage{tikz}

\usepackage{geometry}
\geometry{a5paper, margin=0.1in, right=1cm}

\usepackage{dsfont}

\usepackage{graphicx}
\graphicspath{ {images/} }

\usepackage{faktor}

\usepackage{IEEEtrantools}
\usepackage{enumerate}   
\usepackage[PostScript=dvips]{"/Users/aware/Documents/Courses/diagrams"}


\newtheorem{theorem}{Théorème}[section]
\renewcommand{\thetheorem}{\arabic{theorem}}
\newtheorem{lemme}{Lemme}[section]
\renewcommand{\thelemme}{\arabic{lemme}}
\newtheorem{proposition}{Proposition}[section]
\renewcommand{\theproposition}{\arabic{proposition}}
\newtheorem{notations}{Notations}[section]
\newtheorem{problem}{Problème}[section]
\newtheorem{corollary}{Corollaire}[theorem]
\renewcommand{\thecorollary}{\arabic{corollary}}
\newtheorem{property}{Propriété}[section]
\newtheorem{objective}{Objectif}[section]

\theoremstyle{definition}
\newtheorem{definition}{Définition}[section]
\renewcommand{\thedefinition}{\arabic{definition}}
\newtheorem{exercise}{Exercice}[chapter]
\renewcommand{\theexercise}{\arabic{exercise}}
\newtheorem{example}{Exemple}[chapter]
\renewcommand{\theexample}{\arabic{example}}
\newtheorem*{solution}{Solution}
\newtheorem*{application}{Application}
\newtheorem*{notation}{Notation}
\newtheorem*{vocabulary}{Vocabulaire}
\newtheorem*{properties}{Propriétés}



\theoremstyle{remark}
\newtheorem*{remark}{Remarque}
\newtheorem*{rappel}{Rappel}


\usepackage{etoolbox}
\AtBeginEnvironment{exercise}{\small}
\AtBeginEnvironment{example}{\small}

\usepackage{cases}
\usepackage[red]{mypack}

\usepackage[framemethod=TikZ]{mdframed}

\definecolor{bg}{rgb}{0.4,0.25,0.95}
\definecolor{pagebg}{rgb}{0,0,0.5}
\surroundwithmdframed[
   topline=false,
   rightline=false,
   bottomline=false,
   leftmargin=\parindent,
   skipabove=8pt,
   skipbelow=8pt,
   linecolor=blue,
   innerbottommargin=10pt,
   % backgroundcolor=bg,font=\color{orange}\sffamily, fontcolor=white
]{definition}

\usepackage{empheq}
\usepackage[most]{tcolorbox}

\newtcbox{\mymath}[1][]{%
    nobeforeafter, math upper, tcbox raise base,
    enhanced, colframe=blue!30!black,
    colback=red!10, boxrule=1pt,
    #1}

\usepackage{unixode}


\DeclareMathOperator{\ord}{ord}
\DeclareMathOperator{\orb}{orb}
\DeclareMathOperator{\stab}{stab}
\DeclareMathOperator{\Stab}{stab}
\DeclareMathOperator{\ppcm}{ppcm}
\DeclareMathOperator{\conj}{Conj}
\DeclareMathOperator{\End}{End}
\DeclareMathOperator{\rot}{rot}
\DeclareMathOperator{\trs}{trace}
\DeclareMathOperator{\Ind}{Ind}
\DeclareMathOperator{\mat}{Mat}
\DeclareMathOperator{\id}{Id}
\DeclareMathOperator{\vect}{vect}
\DeclareMathOperator{\img}{img}
\DeclareMathOperator{\cov}{Cov}
\DeclareMathOperator{\dist}{dist}
\DeclareMathOperator{\irr}{Irr}
\DeclareMathOperator{\image}{Im}
\DeclareMathOperator{\pd}{\partial}
\DeclareMathOperator{\epi}{epi}
\DeclareMathOperator{\Argmin}{Argmin}
\DeclareMathOperator{\dom}{dom}
\DeclareMathOperator{\proj}{proj}
\DeclareMathOperator{\ctg}{ctg}
\DeclareMathOperator{\supp}{supp}
\DeclareMathOperator{\argmin}{argmin}
\DeclareMathOperator{\mult}{mult}
\DeclareMathOperator{\ch}{ch}
\DeclareMathOperator{\sh}{sh}
\DeclareMathOperator{\rang}{rang}
\DeclareMathOperator{\diam}{diam}
\DeclareMathOperator{\Epigraphe}{Epigraphe}




\usepackage{xcolor}
\everymath{\color{blue}}
%\everymath{\color[rgb]{0,1,1}}
%\pagecolor[rgb]{0,0,0.5}


\newcommand*{\pdtest}[3][]{\ensuremath{\frac{\partial^{#1} #2}{\partial #3}}}

\newcommand*{\deffunc}[6][]{\ensuremath{
\begin{array}{rcl}
#2 : #3 &\rightarrow& #4\\
#5 &\mapsto& #6
\end{array}
}}

\newcommand{\eqcolon}{\mathrel{\resizebox{\widthof{$\mathord{=}$}}{\height}{ $\!\!=\!\!\resizebox{1.2\width}{0.8\height}{\raisebox{0.23ex}{$\mathop{:}$}}\!\!$ }}}
\newcommand{\coloneq}{\mathrel{\resizebox{\widthof{$\mathord{=}$}}{\height}{ $\!\!\resizebox{1.2\width}{0.8\height}{\raisebox{0.23ex}{$\mathop{:}$}}\!\!=\!\!$ }}}
\newcommand{\eqcolonl}{\ensuremath{\mathrel{=\!\!\mathop{:}}}}
\newcommand{\coloneql}{\ensuremath{\mathrel{\mathop{:} \!\! =}}}
\newcommand{\vc}[1]{% inline column vector
  \left(\begin{smallmatrix}#1\end{smallmatrix}\right)%
}
\newcommand{\vr}[1]{% inline row vector
  \begin{smallmatrix}(\,#1\,)\end{smallmatrix}%
}
\makeatletter
\newcommand*{\defeq}{\ =\mathrel{\rlap{%
                     \raisebox{0.3ex}{$\m@th\cdot$}}%
                     \raisebox{-0.3ex}{$\m@th\cdot$}}%
                     }
\makeatother

\newcommand{\mathcircle}[1]{% inline row vector
 \overset{\circ}{#1}
}
\newcommand{\ulim}{% low limit
 \underline{\lim}
}
\newcommand{\ssi}{% iff
\iff
}
\newcommand{\ps}[2]{
\expval{#1 | #2}
}
\newcommand{\df}[1]{
\mqty{#1}
}
\newcommand{\n}[1]{
\norm{#1}
}
\newcommand{\sys}[1]{
\left\{\smqty{#1}\right.
}


\newcommand{\eqdef}{\ensuremath{\overset{\text{def}}=}}


\def\Circlearrowright{\ensuremath{%
  \rotatebox[origin=c]{230}{$\circlearrowright$}}}

\newcommand\ct[1]{\text{\rmfamily\upshape #1}}
\newcommand\question[1]{ {\color{red} ...!? \small #1}}
\newcommand\caz[1]{\left\{\begin{array} #1 \end{array}\right.}
\newcommand\const{\text{\rmfamily\upshape const}}
\newcommand\toP{ \overset{\pro}{\to}}
\newcommand\toPP{ \overset{\text{PP}}{\to}}
\newcommand{\oeq}{\mathrel{\text{\textcircled{$=$}}}}





\usepackage{xcolor}
% \usepackage[normalem]{ulem}
\usepackage{lipsum}
\makeatletter
% \newcommand\colorwave[1][blue]{\bgroup \markoverwith{\lower3.5\p@\hbox{\sixly \textcolor{#1}{\char58}}}\ULon}
%\font\sixly=lasy6 % does not re-load if already loaded, so no memory problem.

\newmdtheoremenv[
linewidth= 1pt,linecolor= blue,%
leftmargin=20,rightmargin=20,innertopmargin=0pt, innerrightmargin=40,%
tikzsetting = { draw=lightgray, line width = 0.3pt,dashed,%
dash pattern = on 15pt off 3pt},%
splittopskip=\topskip,skipbelow=\baselineskip,%
skipabove=\baselineskip,ntheorem,roundcorner=0pt,
% backgroundcolor=pagebg,font=\color{orange}\sffamily, fontcolor=white
]{examplebox}{Exemple}[section]



\newcommand\R{\mathbb{R}}
\newcommand\Z{\mathbb{Z}}
\newcommand\N{\mathbb{N}}
\newcommand\E{\mathbb{E}}
\newcommand\F{\mathcal{F}}
\newcommand\cH{\mathcal{H}}
\newcommand\V{\mathbb{V}}
\newcommand\dmo{ ^{-1} }
\newcommand\kapa{\kappa}
\newcommand\im{Im}
\newcommand\hs{\mathcal{H}}





\usepackage{soul}

\makeatletter
\newcommand*{\whiten}[1]{\llap{\textcolor{white}{{\the\SOUL@token}}\hspace{#1pt}}}
\DeclareRobustCommand*\myul{%
    \def\SOUL@everyspace{\underline{\space}\kern\z@}%
    \def\SOUL@everytoken{%
     \setbox0=\hbox{\the\SOUL@token}%
     \ifdim\dp0>\z@
        \raisebox{\dp0}{\underline{\phantom{\the\SOUL@token}}}%
        \whiten{1}\whiten{0}%
        \whiten{-1}\whiten{-2}%
        \llap{\the\SOUL@token}%
     \else
        \underline{\the\SOUL@token}%
     \fi}%
\SOUL@}
\makeatother

\newcommand*{\demp}{\fontfamily{lmtt}\selectfont}

\DeclareTextFontCommand{\textdemp}{\demp}

\begin{document}

\ifcomment
Multiline
comment
\fi
\ifcomment
\myul{Typesetting test}
% \color[rgb]{1,1,1}
$∑_i^n≠ 60º±∞π∆¬≈√j∫h≤≥µ$

$\CR \R\pro\ind\pro\gS\pro
\mqty[a&b\\c&d]$
$\pro\mathbb{P}$
$\dd{x}$

  \[
    \alpha(x)=\left\{
                \begin{array}{ll}
                  x\\
                  \frac{1}{1+e^{-kx}}\\
                  \frac{e^x-e^{-x}}{e^x+e^{-x}}
                \end{array}
              \right.
  \]

  $\expval{x}$
  
  $\chi_\rho(ghg\dmo)=\Tr(\rho_{ghg\dmo})=\Tr(\rho_g\circ\rho_h\circ\rho\dmo_g)=\Tr(\rho_h)\overset{\mbox{\scalebox{0.5}{$\Tr(AB)=\Tr(BA)$}}}{=}\chi_\rho(h)$
  	$\mathop{\oplus}_{\substack{x\in X}}$

$\mat(\rho_g)=(a_{ij}(g))_{\scriptsize \substack{1\leq i\leq d \\ 1\leq j\leq d}}$ et $\mat(\rho'_g)=(a'_{ij}(g))_{\scriptsize \substack{1\leq i'\leq d' \\ 1\leq j'\leq d'}}$



\[\int_a^b{\mathbb{R}^2}g(u, v)\dd{P_{XY}}(u, v)=\iint g(u,v) f_{XY}(u, v)\dd \lambda(u) \dd \lambda(v)\]
$$\lim_{x\to\infty} f(x)$$	
$$\iiiint_V \mu(t,u,v,w) \,dt\,du\,dv\,dw$$
$$\sum_{n=1}^{\infty} 2^{-n} = 1$$	
\begin{definition}
	Si $X$ et $Y$ sont 2 v.a. ou definit la \textsc{Covariance} entre $X$ et $Y$ comme
	$\cov(X,Y)\overset{\text{def}}{=}\E\left[(X-\E(X))(Y-\E(Y))\right]=\E(XY)-\E(X)\E(Y)$.
\end{definition}
\fi
\pagebreak

% \tableofcontents

% insert your code here
%\input{./algebra/main.tex}
%\input{./geometrie-differentielle/main.tex}
%\input{./probabilite/main.tex}
%\input{./analyse-fonctionnelle/main.tex}
% \input{./Analyse-convexe-et-dualite-en-optimisation/main.tex}
%\input{./tikz/main.tex}
%\input{./Theorie-du-distributions/main.tex}
%\input{./optimisation/mine.tex}
 \input{./modelisation/main.tex}

% yves.aubry@univ-tln.fr : algebra

\end{document}

%% !TEX encoding = UTF-8 Unicode
% !TEX TS-program = xelatex

\documentclass[french]{report}

%\usepackage[utf8]{inputenc}
%\usepackage[T1]{fontenc}
\usepackage{babel}


\newif\ifcomment
%\commenttrue # Show comments

\usepackage{physics}
\usepackage{amssymb}


\usepackage{amsthm}
% \usepackage{thmtools}
\usepackage{mathtools}
\usepackage{amsfonts}

\usepackage{color}

\usepackage{tikz}

\usepackage{geometry}
\geometry{a5paper, margin=0.1in, right=1cm}

\usepackage{dsfont}

\usepackage{graphicx}
\graphicspath{ {images/} }

\usepackage{faktor}

\usepackage{IEEEtrantools}
\usepackage{enumerate}   
\usepackage[PostScript=dvips]{"/Users/aware/Documents/Courses/diagrams"}


\newtheorem{theorem}{Théorème}[section]
\renewcommand{\thetheorem}{\arabic{theorem}}
\newtheorem{lemme}{Lemme}[section]
\renewcommand{\thelemme}{\arabic{lemme}}
\newtheorem{proposition}{Proposition}[section]
\renewcommand{\theproposition}{\arabic{proposition}}
\newtheorem{notations}{Notations}[section]
\newtheorem{problem}{Problème}[section]
\newtheorem{corollary}{Corollaire}[theorem]
\renewcommand{\thecorollary}{\arabic{corollary}}
\newtheorem{property}{Propriété}[section]
\newtheorem{objective}{Objectif}[section]

\theoremstyle{definition}
\newtheorem{definition}{Définition}[section]
\renewcommand{\thedefinition}{\arabic{definition}}
\newtheorem{exercise}{Exercice}[chapter]
\renewcommand{\theexercise}{\arabic{exercise}}
\newtheorem{example}{Exemple}[chapter]
\renewcommand{\theexample}{\arabic{example}}
\newtheorem*{solution}{Solution}
\newtheorem*{application}{Application}
\newtheorem*{notation}{Notation}
\newtheorem*{vocabulary}{Vocabulaire}
\newtheorem*{properties}{Propriétés}



\theoremstyle{remark}
\newtheorem*{remark}{Remarque}
\newtheorem*{rappel}{Rappel}


\usepackage{etoolbox}
\AtBeginEnvironment{exercise}{\small}
\AtBeginEnvironment{example}{\small}

\usepackage{cases}
\usepackage[red]{mypack}

\usepackage[framemethod=TikZ]{mdframed}

\definecolor{bg}{rgb}{0.4,0.25,0.95}
\definecolor{pagebg}{rgb}{0,0,0.5}
\surroundwithmdframed[
   topline=false,
   rightline=false,
   bottomline=false,
   leftmargin=\parindent,
   skipabove=8pt,
   skipbelow=8pt,
   linecolor=blue,
   innerbottommargin=10pt,
   % backgroundcolor=bg,font=\color{orange}\sffamily, fontcolor=white
]{definition}

\usepackage{empheq}
\usepackage[most]{tcolorbox}

\newtcbox{\mymath}[1][]{%
    nobeforeafter, math upper, tcbox raise base,
    enhanced, colframe=blue!30!black,
    colback=red!10, boxrule=1pt,
    #1}

\usepackage{unixode}


\DeclareMathOperator{\ord}{ord}
\DeclareMathOperator{\orb}{orb}
\DeclareMathOperator{\stab}{stab}
\DeclareMathOperator{\Stab}{stab}
\DeclareMathOperator{\ppcm}{ppcm}
\DeclareMathOperator{\conj}{Conj}
\DeclareMathOperator{\End}{End}
\DeclareMathOperator{\rot}{rot}
\DeclareMathOperator{\trs}{trace}
\DeclareMathOperator{\Ind}{Ind}
\DeclareMathOperator{\mat}{Mat}
\DeclareMathOperator{\id}{Id}
\DeclareMathOperator{\vect}{vect}
\DeclareMathOperator{\img}{img}
\DeclareMathOperator{\cov}{Cov}
\DeclareMathOperator{\dist}{dist}
\DeclareMathOperator{\irr}{Irr}
\DeclareMathOperator{\image}{Im}
\DeclareMathOperator{\pd}{\partial}
\DeclareMathOperator{\epi}{epi}
\DeclareMathOperator{\Argmin}{Argmin}
\DeclareMathOperator{\dom}{dom}
\DeclareMathOperator{\proj}{proj}
\DeclareMathOperator{\ctg}{ctg}
\DeclareMathOperator{\supp}{supp}
\DeclareMathOperator{\argmin}{argmin}
\DeclareMathOperator{\mult}{mult}
\DeclareMathOperator{\ch}{ch}
\DeclareMathOperator{\sh}{sh}
\DeclareMathOperator{\rang}{rang}
\DeclareMathOperator{\diam}{diam}
\DeclareMathOperator{\Epigraphe}{Epigraphe}




\usepackage{xcolor}
\everymath{\color{blue}}
%\everymath{\color[rgb]{0,1,1}}
%\pagecolor[rgb]{0,0,0.5}


\newcommand*{\pdtest}[3][]{\ensuremath{\frac{\partial^{#1} #2}{\partial #3}}}

\newcommand*{\deffunc}[6][]{\ensuremath{
\begin{array}{rcl}
#2 : #3 &\rightarrow& #4\\
#5 &\mapsto& #6
\end{array}
}}

\newcommand{\eqcolon}{\mathrel{\resizebox{\widthof{$\mathord{=}$}}{\height}{ $\!\!=\!\!\resizebox{1.2\width}{0.8\height}{\raisebox{0.23ex}{$\mathop{:}$}}\!\!$ }}}
\newcommand{\coloneq}{\mathrel{\resizebox{\widthof{$\mathord{=}$}}{\height}{ $\!\!\resizebox{1.2\width}{0.8\height}{\raisebox{0.23ex}{$\mathop{:}$}}\!\!=\!\!$ }}}
\newcommand{\eqcolonl}{\ensuremath{\mathrel{=\!\!\mathop{:}}}}
\newcommand{\coloneql}{\ensuremath{\mathrel{\mathop{:} \!\! =}}}
\newcommand{\vc}[1]{% inline column vector
  \left(\begin{smallmatrix}#1\end{smallmatrix}\right)%
}
\newcommand{\vr}[1]{% inline row vector
  \begin{smallmatrix}(\,#1\,)\end{smallmatrix}%
}
\makeatletter
\newcommand*{\defeq}{\ =\mathrel{\rlap{%
                     \raisebox{0.3ex}{$\m@th\cdot$}}%
                     \raisebox{-0.3ex}{$\m@th\cdot$}}%
                     }
\makeatother

\newcommand{\mathcircle}[1]{% inline row vector
 \overset{\circ}{#1}
}
\newcommand{\ulim}{% low limit
 \underline{\lim}
}
\newcommand{\ssi}{% iff
\iff
}
\newcommand{\ps}[2]{
\expval{#1 | #2}
}
\newcommand{\df}[1]{
\mqty{#1}
}
\newcommand{\n}[1]{
\norm{#1}
}
\newcommand{\sys}[1]{
\left\{\smqty{#1}\right.
}


\newcommand{\eqdef}{\ensuremath{\overset{\text{def}}=}}


\def\Circlearrowright{\ensuremath{%
  \rotatebox[origin=c]{230}{$\circlearrowright$}}}

\newcommand\ct[1]{\text{\rmfamily\upshape #1}}
\newcommand\question[1]{ {\color{red} ...!? \small #1}}
\newcommand\caz[1]{\left\{\begin{array} #1 \end{array}\right.}
\newcommand\const{\text{\rmfamily\upshape const}}
\newcommand\toP{ \overset{\pro}{\to}}
\newcommand\toPP{ \overset{\text{PP}}{\to}}
\newcommand{\oeq}{\mathrel{\text{\textcircled{$=$}}}}





\usepackage{xcolor}
% \usepackage[normalem]{ulem}
\usepackage{lipsum}
\makeatletter
% \newcommand\colorwave[1][blue]{\bgroup \markoverwith{\lower3.5\p@\hbox{\sixly \textcolor{#1}{\char58}}}\ULon}
%\font\sixly=lasy6 % does not re-load if already loaded, so no memory problem.

\newmdtheoremenv[
linewidth= 1pt,linecolor= blue,%
leftmargin=20,rightmargin=20,innertopmargin=0pt, innerrightmargin=40,%
tikzsetting = { draw=lightgray, line width = 0.3pt,dashed,%
dash pattern = on 15pt off 3pt},%
splittopskip=\topskip,skipbelow=\baselineskip,%
skipabove=\baselineskip,ntheorem,roundcorner=0pt,
% backgroundcolor=pagebg,font=\color{orange}\sffamily, fontcolor=white
]{examplebox}{Exemple}[section]



\newcommand\R{\mathbb{R}}
\newcommand\Z{\mathbb{Z}}
\newcommand\N{\mathbb{N}}
\newcommand\E{\mathbb{E}}
\newcommand\F{\mathcal{F}}
\newcommand\cH{\mathcal{H}}
\newcommand\V{\mathbb{V}}
\newcommand\dmo{ ^{-1} }
\newcommand\kapa{\kappa}
\newcommand\im{Im}
\newcommand\hs{\mathcal{H}}





\usepackage{soul}

\makeatletter
\newcommand*{\whiten}[1]{\llap{\textcolor{white}{{\the\SOUL@token}}\hspace{#1pt}}}
\DeclareRobustCommand*\myul{%
    \def\SOUL@everyspace{\underline{\space}\kern\z@}%
    \def\SOUL@everytoken{%
     \setbox0=\hbox{\the\SOUL@token}%
     \ifdim\dp0>\z@
        \raisebox{\dp0}{\underline{\phantom{\the\SOUL@token}}}%
        \whiten{1}\whiten{0}%
        \whiten{-1}\whiten{-2}%
        \llap{\the\SOUL@token}%
     \else
        \underline{\the\SOUL@token}%
     \fi}%
\SOUL@}
\makeatother

\newcommand*{\demp}{\fontfamily{lmtt}\selectfont}

\DeclareTextFontCommand{\textdemp}{\demp}

\begin{document}

\ifcomment
Multiline
comment
\fi
\ifcomment
\myul{Typesetting test}
% \color[rgb]{1,1,1}
$∑_i^n≠ 60º±∞π∆¬≈√j∫h≤≥µ$

$\CR \R\pro\ind\pro\gS\pro
\mqty[a&b\\c&d]$
$\pro\mathbb{P}$
$\dd{x}$

  \[
    \alpha(x)=\left\{
                \begin{array}{ll}
                  x\\
                  \frac{1}{1+e^{-kx}}\\
                  \frac{e^x-e^{-x}}{e^x+e^{-x}}
                \end{array}
              \right.
  \]

  $\expval{x}$
  
  $\chi_\rho(ghg\dmo)=\Tr(\rho_{ghg\dmo})=\Tr(\rho_g\circ\rho_h\circ\rho\dmo_g)=\Tr(\rho_h)\overset{\mbox{\scalebox{0.5}{$\Tr(AB)=\Tr(BA)$}}}{=}\chi_\rho(h)$
  	$\mathop{\oplus}_{\substack{x\in X}}$

$\mat(\rho_g)=(a_{ij}(g))_{\scriptsize \substack{1\leq i\leq d \\ 1\leq j\leq d}}$ et $\mat(\rho'_g)=(a'_{ij}(g))_{\scriptsize \substack{1\leq i'\leq d' \\ 1\leq j'\leq d'}}$



\[\int_a^b{\mathbb{R}^2}g(u, v)\dd{P_{XY}}(u, v)=\iint g(u,v) f_{XY}(u, v)\dd \lambda(u) \dd \lambda(v)\]
$$\lim_{x\to\infty} f(x)$$	
$$\iiiint_V \mu(t,u,v,w) \,dt\,du\,dv\,dw$$
$$\sum_{n=1}^{\infty} 2^{-n} = 1$$	
\begin{definition}
	Si $X$ et $Y$ sont 2 v.a. ou definit la \textsc{Covariance} entre $X$ et $Y$ comme
	$\cov(X,Y)\overset{\text{def}}{=}\E\left[(X-\E(X))(Y-\E(Y))\right]=\E(XY)-\E(X)\E(Y)$.
\end{definition}
\fi
\pagebreak

% \tableofcontents

% insert your code here
%\input{./algebra/main.tex}
%\input{./geometrie-differentielle/main.tex}
%\input{./probabilite/main.tex}
%\input{./analyse-fonctionnelle/main.tex}
% \input{./Analyse-convexe-et-dualite-en-optimisation/main.tex}
%\input{./tikz/main.tex}
%\input{./Theorie-du-distributions/main.tex}
%\input{./optimisation/mine.tex}
 \input{./modelisation/main.tex}

% yves.aubry@univ-tln.fr : algebra

\end{document}

% % !TEX encoding = UTF-8 Unicode
% !TEX TS-program = xelatex

\documentclass[french]{report}

%\usepackage[utf8]{inputenc}
%\usepackage[T1]{fontenc}
\usepackage{babel}


\newif\ifcomment
%\commenttrue # Show comments

\usepackage{physics}
\usepackage{amssymb}


\usepackage{amsthm}
% \usepackage{thmtools}
\usepackage{mathtools}
\usepackage{amsfonts}

\usepackage{color}

\usepackage{tikz}

\usepackage{geometry}
\geometry{a5paper, margin=0.1in, right=1cm}

\usepackage{dsfont}

\usepackage{graphicx}
\graphicspath{ {images/} }

\usepackage{faktor}

\usepackage{IEEEtrantools}
\usepackage{enumerate}   
\usepackage[PostScript=dvips]{"/Users/aware/Documents/Courses/diagrams"}


\newtheorem{theorem}{Théorème}[section]
\renewcommand{\thetheorem}{\arabic{theorem}}
\newtheorem{lemme}{Lemme}[section]
\renewcommand{\thelemme}{\arabic{lemme}}
\newtheorem{proposition}{Proposition}[section]
\renewcommand{\theproposition}{\arabic{proposition}}
\newtheorem{notations}{Notations}[section]
\newtheorem{problem}{Problème}[section]
\newtheorem{corollary}{Corollaire}[theorem]
\renewcommand{\thecorollary}{\arabic{corollary}}
\newtheorem{property}{Propriété}[section]
\newtheorem{objective}{Objectif}[section]

\theoremstyle{definition}
\newtheorem{definition}{Définition}[section]
\renewcommand{\thedefinition}{\arabic{definition}}
\newtheorem{exercise}{Exercice}[chapter]
\renewcommand{\theexercise}{\arabic{exercise}}
\newtheorem{example}{Exemple}[chapter]
\renewcommand{\theexample}{\arabic{example}}
\newtheorem*{solution}{Solution}
\newtheorem*{application}{Application}
\newtheorem*{notation}{Notation}
\newtheorem*{vocabulary}{Vocabulaire}
\newtheorem*{properties}{Propriétés}



\theoremstyle{remark}
\newtheorem*{remark}{Remarque}
\newtheorem*{rappel}{Rappel}


\usepackage{etoolbox}
\AtBeginEnvironment{exercise}{\small}
\AtBeginEnvironment{example}{\small}

\usepackage{cases}
\usepackage[red]{mypack}

\usepackage[framemethod=TikZ]{mdframed}

\definecolor{bg}{rgb}{0.4,0.25,0.95}
\definecolor{pagebg}{rgb}{0,0,0.5}
\surroundwithmdframed[
   topline=false,
   rightline=false,
   bottomline=false,
   leftmargin=\parindent,
   skipabove=8pt,
   skipbelow=8pt,
   linecolor=blue,
   innerbottommargin=10pt,
   % backgroundcolor=bg,font=\color{orange}\sffamily, fontcolor=white
]{definition}

\usepackage{empheq}
\usepackage[most]{tcolorbox}

\newtcbox{\mymath}[1][]{%
    nobeforeafter, math upper, tcbox raise base,
    enhanced, colframe=blue!30!black,
    colback=red!10, boxrule=1pt,
    #1}

\usepackage{unixode}


\DeclareMathOperator{\ord}{ord}
\DeclareMathOperator{\orb}{orb}
\DeclareMathOperator{\stab}{stab}
\DeclareMathOperator{\Stab}{stab}
\DeclareMathOperator{\ppcm}{ppcm}
\DeclareMathOperator{\conj}{Conj}
\DeclareMathOperator{\End}{End}
\DeclareMathOperator{\rot}{rot}
\DeclareMathOperator{\trs}{trace}
\DeclareMathOperator{\Ind}{Ind}
\DeclareMathOperator{\mat}{Mat}
\DeclareMathOperator{\id}{Id}
\DeclareMathOperator{\vect}{vect}
\DeclareMathOperator{\img}{img}
\DeclareMathOperator{\cov}{Cov}
\DeclareMathOperator{\dist}{dist}
\DeclareMathOperator{\irr}{Irr}
\DeclareMathOperator{\image}{Im}
\DeclareMathOperator{\pd}{\partial}
\DeclareMathOperator{\epi}{epi}
\DeclareMathOperator{\Argmin}{Argmin}
\DeclareMathOperator{\dom}{dom}
\DeclareMathOperator{\proj}{proj}
\DeclareMathOperator{\ctg}{ctg}
\DeclareMathOperator{\supp}{supp}
\DeclareMathOperator{\argmin}{argmin}
\DeclareMathOperator{\mult}{mult}
\DeclareMathOperator{\ch}{ch}
\DeclareMathOperator{\sh}{sh}
\DeclareMathOperator{\rang}{rang}
\DeclareMathOperator{\diam}{diam}
\DeclareMathOperator{\Epigraphe}{Epigraphe}




\usepackage{xcolor}
\everymath{\color{blue}}
%\everymath{\color[rgb]{0,1,1}}
%\pagecolor[rgb]{0,0,0.5}


\newcommand*{\pdtest}[3][]{\ensuremath{\frac{\partial^{#1} #2}{\partial #3}}}

\newcommand*{\deffunc}[6][]{\ensuremath{
\begin{array}{rcl}
#2 : #3 &\rightarrow& #4\\
#5 &\mapsto& #6
\end{array}
}}

\newcommand{\eqcolon}{\mathrel{\resizebox{\widthof{$\mathord{=}$}}{\height}{ $\!\!=\!\!\resizebox{1.2\width}{0.8\height}{\raisebox{0.23ex}{$\mathop{:}$}}\!\!$ }}}
\newcommand{\coloneq}{\mathrel{\resizebox{\widthof{$\mathord{=}$}}{\height}{ $\!\!\resizebox{1.2\width}{0.8\height}{\raisebox{0.23ex}{$\mathop{:}$}}\!\!=\!\!$ }}}
\newcommand{\eqcolonl}{\ensuremath{\mathrel{=\!\!\mathop{:}}}}
\newcommand{\coloneql}{\ensuremath{\mathrel{\mathop{:} \!\! =}}}
\newcommand{\vc}[1]{% inline column vector
  \left(\begin{smallmatrix}#1\end{smallmatrix}\right)%
}
\newcommand{\vr}[1]{% inline row vector
  \begin{smallmatrix}(\,#1\,)\end{smallmatrix}%
}
\makeatletter
\newcommand*{\defeq}{\ =\mathrel{\rlap{%
                     \raisebox{0.3ex}{$\m@th\cdot$}}%
                     \raisebox{-0.3ex}{$\m@th\cdot$}}%
                     }
\makeatother

\newcommand{\mathcircle}[1]{% inline row vector
 \overset{\circ}{#1}
}
\newcommand{\ulim}{% low limit
 \underline{\lim}
}
\newcommand{\ssi}{% iff
\iff
}
\newcommand{\ps}[2]{
\expval{#1 | #2}
}
\newcommand{\df}[1]{
\mqty{#1}
}
\newcommand{\n}[1]{
\norm{#1}
}
\newcommand{\sys}[1]{
\left\{\smqty{#1}\right.
}


\newcommand{\eqdef}{\ensuremath{\overset{\text{def}}=}}


\def\Circlearrowright{\ensuremath{%
  \rotatebox[origin=c]{230}{$\circlearrowright$}}}

\newcommand\ct[1]{\text{\rmfamily\upshape #1}}
\newcommand\question[1]{ {\color{red} ...!? \small #1}}
\newcommand\caz[1]{\left\{\begin{array} #1 \end{array}\right.}
\newcommand\const{\text{\rmfamily\upshape const}}
\newcommand\toP{ \overset{\pro}{\to}}
\newcommand\toPP{ \overset{\text{PP}}{\to}}
\newcommand{\oeq}{\mathrel{\text{\textcircled{$=$}}}}





\usepackage{xcolor}
% \usepackage[normalem]{ulem}
\usepackage{lipsum}
\makeatletter
% \newcommand\colorwave[1][blue]{\bgroup \markoverwith{\lower3.5\p@\hbox{\sixly \textcolor{#1}{\char58}}}\ULon}
%\font\sixly=lasy6 % does not re-load if already loaded, so no memory problem.

\newmdtheoremenv[
linewidth= 1pt,linecolor= blue,%
leftmargin=20,rightmargin=20,innertopmargin=0pt, innerrightmargin=40,%
tikzsetting = { draw=lightgray, line width = 0.3pt,dashed,%
dash pattern = on 15pt off 3pt},%
splittopskip=\topskip,skipbelow=\baselineskip,%
skipabove=\baselineskip,ntheorem,roundcorner=0pt,
% backgroundcolor=pagebg,font=\color{orange}\sffamily, fontcolor=white
]{examplebox}{Exemple}[section]



\newcommand\R{\mathbb{R}}
\newcommand\Z{\mathbb{Z}}
\newcommand\N{\mathbb{N}}
\newcommand\E{\mathbb{E}}
\newcommand\F{\mathcal{F}}
\newcommand\cH{\mathcal{H}}
\newcommand\V{\mathbb{V}}
\newcommand\dmo{ ^{-1} }
\newcommand\kapa{\kappa}
\newcommand\im{Im}
\newcommand\hs{\mathcal{H}}





\usepackage{soul}

\makeatletter
\newcommand*{\whiten}[1]{\llap{\textcolor{white}{{\the\SOUL@token}}\hspace{#1pt}}}
\DeclareRobustCommand*\myul{%
    \def\SOUL@everyspace{\underline{\space}\kern\z@}%
    \def\SOUL@everytoken{%
     \setbox0=\hbox{\the\SOUL@token}%
     \ifdim\dp0>\z@
        \raisebox{\dp0}{\underline{\phantom{\the\SOUL@token}}}%
        \whiten{1}\whiten{0}%
        \whiten{-1}\whiten{-2}%
        \llap{\the\SOUL@token}%
     \else
        \underline{\the\SOUL@token}%
     \fi}%
\SOUL@}
\makeatother

\newcommand*{\demp}{\fontfamily{lmtt}\selectfont}

\DeclareTextFontCommand{\textdemp}{\demp}

\begin{document}

\ifcomment
Multiline
comment
\fi
\ifcomment
\myul{Typesetting test}
% \color[rgb]{1,1,1}
$∑_i^n≠ 60º±∞π∆¬≈√j∫h≤≥µ$

$\CR \R\pro\ind\pro\gS\pro
\mqty[a&b\\c&d]$
$\pro\mathbb{P}$
$\dd{x}$

  \[
    \alpha(x)=\left\{
                \begin{array}{ll}
                  x\\
                  \frac{1}{1+e^{-kx}}\\
                  \frac{e^x-e^{-x}}{e^x+e^{-x}}
                \end{array}
              \right.
  \]

  $\expval{x}$
  
  $\chi_\rho(ghg\dmo)=\Tr(\rho_{ghg\dmo})=\Tr(\rho_g\circ\rho_h\circ\rho\dmo_g)=\Tr(\rho_h)\overset{\mbox{\scalebox{0.5}{$\Tr(AB)=\Tr(BA)$}}}{=}\chi_\rho(h)$
  	$\mathop{\oplus}_{\substack{x\in X}}$

$\mat(\rho_g)=(a_{ij}(g))_{\scriptsize \substack{1\leq i\leq d \\ 1\leq j\leq d}}$ et $\mat(\rho'_g)=(a'_{ij}(g))_{\scriptsize \substack{1\leq i'\leq d' \\ 1\leq j'\leq d'}}$



\[\int_a^b{\mathbb{R}^2}g(u, v)\dd{P_{XY}}(u, v)=\iint g(u,v) f_{XY}(u, v)\dd \lambda(u) \dd \lambda(v)\]
$$\lim_{x\to\infty} f(x)$$	
$$\iiiint_V \mu(t,u,v,w) \,dt\,du\,dv\,dw$$
$$\sum_{n=1}^{\infty} 2^{-n} = 1$$	
\begin{definition}
	Si $X$ et $Y$ sont 2 v.a. ou definit la \textsc{Covariance} entre $X$ et $Y$ comme
	$\cov(X,Y)\overset{\text{def}}{=}\E\left[(X-\E(X))(Y-\E(Y))\right]=\E(XY)-\E(X)\E(Y)$.
\end{definition}
\fi
\pagebreak

% \tableofcontents

% insert your code here
%\input{./algebra/main.tex}
%\input{./geometrie-differentielle/main.tex}
%\input{./probabilite/main.tex}
%\input{./analyse-fonctionnelle/main.tex}
% \input{./Analyse-convexe-et-dualite-en-optimisation/main.tex}
%\input{./tikz/main.tex}
%\input{./Theorie-du-distributions/main.tex}
%\input{./optimisation/mine.tex}
 \input{./modelisation/main.tex}

% yves.aubry@univ-tln.fr : algebra

\end{document}

%% !TEX encoding = UTF-8 Unicode
% !TEX TS-program = xelatex

\documentclass[french]{report}

%\usepackage[utf8]{inputenc}
%\usepackage[T1]{fontenc}
\usepackage{babel}


\newif\ifcomment
%\commenttrue # Show comments

\usepackage{physics}
\usepackage{amssymb}


\usepackage{amsthm}
% \usepackage{thmtools}
\usepackage{mathtools}
\usepackage{amsfonts}

\usepackage{color}

\usepackage{tikz}

\usepackage{geometry}
\geometry{a5paper, margin=0.1in, right=1cm}

\usepackage{dsfont}

\usepackage{graphicx}
\graphicspath{ {images/} }

\usepackage{faktor}

\usepackage{IEEEtrantools}
\usepackage{enumerate}   
\usepackage[PostScript=dvips]{"/Users/aware/Documents/Courses/diagrams"}


\newtheorem{theorem}{Théorème}[section]
\renewcommand{\thetheorem}{\arabic{theorem}}
\newtheorem{lemme}{Lemme}[section]
\renewcommand{\thelemme}{\arabic{lemme}}
\newtheorem{proposition}{Proposition}[section]
\renewcommand{\theproposition}{\arabic{proposition}}
\newtheorem{notations}{Notations}[section]
\newtheorem{problem}{Problème}[section]
\newtheorem{corollary}{Corollaire}[theorem]
\renewcommand{\thecorollary}{\arabic{corollary}}
\newtheorem{property}{Propriété}[section]
\newtheorem{objective}{Objectif}[section]

\theoremstyle{definition}
\newtheorem{definition}{Définition}[section]
\renewcommand{\thedefinition}{\arabic{definition}}
\newtheorem{exercise}{Exercice}[chapter]
\renewcommand{\theexercise}{\arabic{exercise}}
\newtheorem{example}{Exemple}[chapter]
\renewcommand{\theexample}{\arabic{example}}
\newtheorem*{solution}{Solution}
\newtheorem*{application}{Application}
\newtheorem*{notation}{Notation}
\newtheorem*{vocabulary}{Vocabulaire}
\newtheorem*{properties}{Propriétés}



\theoremstyle{remark}
\newtheorem*{remark}{Remarque}
\newtheorem*{rappel}{Rappel}


\usepackage{etoolbox}
\AtBeginEnvironment{exercise}{\small}
\AtBeginEnvironment{example}{\small}

\usepackage{cases}
\usepackage[red]{mypack}

\usepackage[framemethod=TikZ]{mdframed}

\definecolor{bg}{rgb}{0.4,0.25,0.95}
\definecolor{pagebg}{rgb}{0,0,0.5}
\surroundwithmdframed[
   topline=false,
   rightline=false,
   bottomline=false,
   leftmargin=\parindent,
   skipabove=8pt,
   skipbelow=8pt,
   linecolor=blue,
   innerbottommargin=10pt,
   % backgroundcolor=bg,font=\color{orange}\sffamily, fontcolor=white
]{definition}

\usepackage{empheq}
\usepackage[most]{tcolorbox}

\newtcbox{\mymath}[1][]{%
    nobeforeafter, math upper, tcbox raise base,
    enhanced, colframe=blue!30!black,
    colback=red!10, boxrule=1pt,
    #1}

\usepackage{unixode}


\DeclareMathOperator{\ord}{ord}
\DeclareMathOperator{\orb}{orb}
\DeclareMathOperator{\stab}{stab}
\DeclareMathOperator{\Stab}{stab}
\DeclareMathOperator{\ppcm}{ppcm}
\DeclareMathOperator{\conj}{Conj}
\DeclareMathOperator{\End}{End}
\DeclareMathOperator{\rot}{rot}
\DeclareMathOperator{\trs}{trace}
\DeclareMathOperator{\Ind}{Ind}
\DeclareMathOperator{\mat}{Mat}
\DeclareMathOperator{\id}{Id}
\DeclareMathOperator{\vect}{vect}
\DeclareMathOperator{\img}{img}
\DeclareMathOperator{\cov}{Cov}
\DeclareMathOperator{\dist}{dist}
\DeclareMathOperator{\irr}{Irr}
\DeclareMathOperator{\image}{Im}
\DeclareMathOperator{\pd}{\partial}
\DeclareMathOperator{\epi}{epi}
\DeclareMathOperator{\Argmin}{Argmin}
\DeclareMathOperator{\dom}{dom}
\DeclareMathOperator{\proj}{proj}
\DeclareMathOperator{\ctg}{ctg}
\DeclareMathOperator{\supp}{supp}
\DeclareMathOperator{\argmin}{argmin}
\DeclareMathOperator{\mult}{mult}
\DeclareMathOperator{\ch}{ch}
\DeclareMathOperator{\sh}{sh}
\DeclareMathOperator{\rang}{rang}
\DeclareMathOperator{\diam}{diam}
\DeclareMathOperator{\Epigraphe}{Epigraphe}




\usepackage{xcolor}
\everymath{\color{blue}}
%\everymath{\color[rgb]{0,1,1}}
%\pagecolor[rgb]{0,0,0.5}


\newcommand*{\pdtest}[3][]{\ensuremath{\frac{\partial^{#1} #2}{\partial #3}}}

\newcommand*{\deffunc}[6][]{\ensuremath{
\begin{array}{rcl}
#2 : #3 &\rightarrow& #4\\
#5 &\mapsto& #6
\end{array}
}}

\newcommand{\eqcolon}{\mathrel{\resizebox{\widthof{$\mathord{=}$}}{\height}{ $\!\!=\!\!\resizebox{1.2\width}{0.8\height}{\raisebox{0.23ex}{$\mathop{:}$}}\!\!$ }}}
\newcommand{\coloneq}{\mathrel{\resizebox{\widthof{$\mathord{=}$}}{\height}{ $\!\!\resizebox{1.2\width}{0.8\height}{\raisebox{0.23ex}{$\mathop{:}$}}\!\!=\!\!$ }}}
\newcommand{\eqcolonl}{\ensuremath{\mathrel{=\!\!\mathop{:}}}}
\newcommand{\coloneql}{\ensuremath{\mathrel{\mathop{:} \!\! =}}}
\newcommand{\vc}[1]{% inline column vector
  \left(\begin{smallmatrix}#1\end{smallmatrix}\right)%
}
\newcommand{\vr}[1]{% inline row vector
  \begin{smallmatrix}(\,#1\,)\end{smallmatrix}%
}
\makeatletter
\newcommand*{\defeq}{\ =\mathrel{\rlap{%
                     \raisebox{0.3ex}{$\m@th\cdot$}}%
                     \raisebox{-0.3ex}{$\m@th\cdot$}}%
                     }
\makeatother

\newcommand{\mathcircle}[1]{% inline row vector
 \overset{\circ}{#1}
}
\newcommand{\ulim}{% low limit
 \underline{\lim}
}
\newcommand{\ssi}{% iff
\iff
}
\newcommand{\ps}[2]{
\expval{#1 | #2}
}
\newcommand{\df}[1]{
\mqty{#1}
}
\newcommand{\n}[1]{
\norm{#1}
}
\newcommand{\sys}[1]{
\left\{\smqty{#1}\right.
}


\newcommand{\eqdef}{\ensuremath{\overset{\text{def}}=}}


\def\Circlearrowright{\ensuremath{%
  \rotatebox[origin=c]{230}{$\circlearrowright$}}}

\newcommand\ct[1]{\text{\rmfamily\upshape #1}}
\newcommand\question[1]{ {\color{red} ...!? \small #1}}
\newcommand\caz[1]{\left\{\begin{array} #1 \end{array}\right.}
\newcommand\const{\text{\rmfamily\upshape const}}
\newcommand\toP{ \overset{\pro}{\to}}
\newcommand\toPP{ \overset{\text{PP}}{\to}}
\newcommand{\oeq}{\mathrel{\text{\textcircled{$=$}}}}





\usepackage{xcolor}
% \usepackage[normalem]{ulem}
\usepackage{lipsum}
\makeatletter
% \newcommand\colorwave[1][blue]{\bgroup \markoverwith{\lower3.5\p@\hbox{\sixly \textcolor{#1}{\char58}}}\ULon}
%\font\sixly=lasy6 % does not re-load if already loaded, so no memory problem.

\newmdtheoremenv[
linewidth= 1pt,linecolor= blue,%
leftmargin=20,rightmargin=20,innertopmargin=0pt, innerrightmargin=40,%
tikzsetting = { draw=lightgray, line width = 0.3pt,dashed,%
dash pattern = on 15pt off 3pt},%
splittopskip=\topskip,skipbelow=\baselineskip,%
skipabove=\baselineskip,ntheorem,roundcorner=0pt,
% backgroundcolor=pagebg,font=\color{orange}\sffamily, fontcolor=white
]{examplebox}{Exemple}[section]



\newcommand\R{\mathbb{R}}
\newcommand\Z{\mathbb{Z}}
\newcommand\N{\mathbb{N}}
\newcommand\E{\mathbb{E}}
\newcommand\F{\mathcal{F}}
\newcommand\cH{\mathcal{H}}
\newcommand\V{\mathbb{V}}
\newcommand\dmo{ ^{-1} }
\newcommand\kapa{\kappa}
\newcommand\im{Im}
\newcommand\hs{\mathcal{H}}





\usepackage{soul}

\makeatletter
\newcommand*{\whiten}[1]{\llap{\textcolor{white}{{\the\SOUL@token}}\hspace{#1pt}}}
\DeclareRobustCommand*\myul{%
    \def\SOUL@everyspace{\underline{\space}\kern\z@}%
    \def\SOUL@everytoken{%
     \setbox0=\hbox{\the\SOUL@token}%
     \ifdim\dp0>\z@
        \raisebox{\dp0}{\underline{\phantom{\the\SOUL@token}}}%
        \whiten{1}\whiten{0}%
        \whiten{-1}\whiten{-2}%
        \llap{\the\SOUL@token}%
     \else
        \underline{\the\SOUL@token}%
     \fi}%
\SOUL@}
\makeatother

\newcommand*{\demp}{\fontfamily{lmtt}\selectfont}

\DeclareTextFontCommand{\textdemp}{\demp}

\begin{document}

\ifcomment
Multiline
comment
\fi
\ifcomment
\myul{Typesetting test}
% \color[rgb]{1,1,1}
$∑_i^n≠ 60º±∞π∆¬≈√j∫h≤≥µ$

$\CR \R\pro\ind\pro\gS\pro
\mqty[a&b\\c&d]$
$\pro\mathbb{P}$
$\dd{x}$

  \[
    \alpha(x)=\left\{
                \begin{array}{ll}
                  x\\
                  \frac{1}{1+e^{-kx}}\\
                  \frac{e^x-e^{-x}}{e^x+e^{-x}}
                \end{array}
              \right.
  \]

  $\expval{x}$
  
  $\chi_\rho(ghg\dmo)=\Tr(\rho_{ghg\dmo})=\Tr(\rho_g\circ\rho_h\circ\rho\dmo_g)=\Tr(\rho_h)\overset{\mbox{\scalebox{0.5}{$\Tr(AB)=\Tr(BA)$}}}{=}\chi_\rho(h)$
  	$\mathop{\oplus}_{\substack{x\in X}}$

$\mat(\rho_g)=(a_{ij}(g))_{\scriptsize \substack{1\leq i\leq d \\ 1\leq j\leq d}}$ et $\mat(\rho'_g)=(a'_{ij}(g))_{\scriptsize \substack{1\leq i'\leq d' \\ 1\leq j'\leq d'}}$



\[\int_a^b{\mathbb{R}^2}g(u, v)\dd{P_{XY}}(u, v)=\iint g(u,v) f_{XY}(u, v)\dd \lambda(u) \dd \lambda(v)\]
$$\lim_{x\to\infty} f(x)$$	
$$\iiiint_V \mu(t,u,v,w) \,dt\,du\,dv\,dw$$
$$\sum_{n=1}^{\infty} 2^{-n} = 1$$	
\begin{definition}
	Si $X$ et $Y$ sont 2 v.a. ou definit la \textsc{Covariance} entre $X$ et $Y$ comme
	$\cov(X,Y)\overset{\text{def}}{=}\E\left[(X-\E(X))(Y-\E(Y))\right]=\E(XY)-\E(X)\E(Y)$.
\end{definition}
\fi
\pagebreak

% \tableofcontents

% insert your code here
%\input{./algebra/main.tex}
%\input{./geometrie-differentielle/main.tex}
%\input{./probabilite/main.tex}
%\input{./analyse-fonctionnelle/main.tex}
% \input{./Analyse-convexe-et-dualite-en-optimisation/main.tex}
%\input{./tikz/main.tex}
%\input{./Theorie-du-distributions/main.tex}
%\input{./optimisation/mine.tex}
 \input{./modelisation/main.tex}

% yves.aubry@univ-tln.fr : algebra

\end{document}

%% !TEX encoding = UTF-8 Unicode
% !TEX TS-program = xelatex

\documentclass[french]{report}

%\usepackage[utf8]{inputenc}
%\usepackage[T1]{fontenc}
\usepackage{babel}


\newif\ifcomment
%\commenttrue # Show comments

\usepackage{physics}
\usepackage{amssymb}


\usepackage{amsthm}
% \usepackage{thmtools}
\usepackage{mathtools}
\usepackage{amsfonts}

\usepackage{color}

\usepackage{tikz}

\usepackage{geometry}
\geometry{a5paper, margin=0.1in, right=1cm}

\usepackage{dsfont}

\usepackage{graphicx}
\graphicspath{ {images/} }

\usepackage{faktor}

\usepackage{IEEEtrantools}
\usepackage{enumerate}   
\usepackage[PostScript=dvips]{"/Users/aware/Documents/Courses/diagrams"}


\newtheorem{theorem}{Théorème}[section]
\renewcommand{\thetheorem}{\arabic{theorem}}
\newtheorem{lemme}{Lemme}[section]
\renewcommand{\thelemme}{\arabic{lemme}}
\newtheorem{proposition}{Proposition}[section]
\renewcommand{\theproposition}{\arabic{proposition}}
\newtheorem{notations}{Notations}[section]
\newtheorem{problem}{Problème}[section]
\newtheorem{corollary}{Corollaire}[theorem]
\renewcommand{\thecorollary}{\arabic{corollary}}
\newtheorem{property}{Propriété}[section]
\newtheorem{objective}{Objectif}[section]

\theoremstyle{definition}
\newtheorem{definition}{Définition}[section]
\renewcommand{\thedefinition}{\arabic{definition}}
\newtheorem{exercise}{Exercice}[chapter]
\renewcommand{\theexercise}{\arabic{exercise}}
\newtheorem{example}{Exemple}[chapter]
\renewcommand{\theexample}{\arabic{example}}
\newtheorem*{solution}{Solution}
\newtheorem*{application}{Application}
\newtheorem*{notation}{Notation}
\newtheorem*{vocabulary}{Vocabulaire}
\newtheorem*{properties}{Propriétés}



\theoremstyle{remark}
\newtheorem*{remark}{Remarque}
\newtheorem*{rappel}{Rappel}


\usepackage{etoolbox}
\AtBeginEnvironment{exercise}{\small}
\AtBeginEnvironment{example}{\small}

\usepackage{cases}
\usepackage[red]{mypack}

\usepackage[framemethod=TikZ]{mdframed}

\definecolor{bg}{rgb}{0.4,0.25,0.95}
\definecolor{pagebg}{rgb}{0,0,0.5}
\surroundwithmdframed[
   topline=false,
   rightline=false,
   bottomline=false,
   leftmargin=\parindent,
   skipabove=8pt,
   skipbelow=8pt,
   linecolor=blue,
   innerbottommargin=10pt,
   % backgroundcolor=bg,font=\color{orange}\sffamily, fontcolor=white
]{definition}

\usepackage{empheq}
\usepackage[most]{tcolorbox}

\newtcbox{\mymath}[1][]{%
    nobeforeafter, math upper, tcbox raise base,
    enhanced, colframe=blue!30!black,
    colback=red!10, boxrule=1pt,
    #1}

\usepackage{unixode}


\DeclareMathOperator{\ord}{ord}
\DeclareMathOperator{\orb}{orb}
\DeclareMathOperator{\stab}{stab}
\DeclareMathOperator{\Stab}{stab}
\DeclareMathOperator{\ppcm}{ppcm}
\DeclareMathOperator{\conj}{Conj}
\DeclareMathOperator{\End}{End}
\DeclareMathOperator{\rot}{rot}
\DeclareMathOperator{\trs}{trace}
\DeclareMathOperator{\Ind}{Ind}
\DeclareMathOperator{\mat}{Mat}
\DeclareMathOperator{\id}{Id}
\DeclareMathOperator{\vect}{vect}
\DeclareMathOperator{\img}{img}
\DeclareMathOperator{\cov}{Cov}
\DeclareMathOperator{\dist}{dist}
\DeclareMathOperator{\irr}{Irr}
\DeclareMathOperator{\image}{Im}
\DeclareMathOperator{\pd}{\partial}
\DeclareMathOperator{\epi}{epi}
\DeclareMathOperator{\Argmin}{Argmin}
\DeclareMathOperator{\dom}{dom}
\DeclareMathOperator{\proj}{proj}
\DeclareMathOperator{\ctg}{ctg}
\DeclareMathOperator{\supp}{supp}
\DeclareMathOperator{\argmin}{argmin}
\DeclareMathOperator{\mult}{mult}
\DeclareMathOperator{\ch}{ch}
\DeclareMathOperator{\sh}{sh}
\DeclareMathOperator{\rang}{rang}
\DeclareMathOperator{\diam}{diam}
\DeclareMathOperator{\Epigraphe}{Epigraphe}




\usepackage{xcolor}
\everymath{\color{blue}}
%\everymath{\color[rgb]{0,1,1}}
%\pagecolor[rgb]{0,0,0.5}


\newcommand*{\pdtest}[3][]{\ensuremath{\frac{\partial^{#1} #2}{\partial #3}}}

\newcommand*{\deffunc}[6][]{\ensuremath{
\begin{array}{rcl}
#2 : #3 &\rightarrow& #4\\
#5 &\mapsto& #6
\end{array}
}}

\newcommand{\eqcolon}{\mathrel{\resizebox{\widthof{$\mathord{=}$}}{\height}{ $\!\!=\!\!\resizebox{1.2\width}{0.8\height}{\raisebox{0.23ex}{$\mathop{:}$}}\!\!$ }}}
\newcommand{\coloneq}{\mathrel{\resizebox{\widthof{$\mathord{=}$}}{\height}{ $\!\!\resizebox{1.2\width}{0.8\height}{\raisebox{0.23ex}{$\mathop{:}$}}\!\!=\!\!$ }}}
\newcommand{\eqcolonl}{\ensuremath{\mathrel{=\!\!\mathop{:}}}}
\newcommand{\coloneql}{\ensuremath{\mathrel{\mathop{:} \!\! =}}}
\newcommand{\vc}[1]{% inline column vector
  \left(\begin{smallmatrix}#1\end{smallmatrix}\right)%
}
\newcommand{\vr}[1]{% inline row vector
  \begin{smallmatrix}(\,#1\,)\end{smallmatrix}%
}
\makeatletter
\newcommand*{\defeq}{\ =\mathrel{\rlap{%
                     \raisebox{0.3ex}{$\m@th\cdot$}}%
                     \raisebox{-0.3ex}{$\m@th\cdot$}}%
                     }
\makeatother

\newcommand{\mathcircle}[1]{% inline row vector
 \overset{\circ}{#1}
}
\newcommand{\ulim}{% low limit
 \underline{\lim}
}
\newcommand{\ssi}{% iff
\iff
}
\newcommand{\ps}[2]{
\expval{#1 | #2}
}
\newcommand{\df}[1]{
\mqty{#1}
}
\newcommand{\n}[1]{
\norm{#1}
}
\newcommand{\sys}[1]{
\left\{\smqty{#1}\right.
}


\newcommand{\eqdef}{\ensuremath{\overset{\text{def}}=}}


\def\Circlearrowright{\ensuremath{%
  \rotatebox[origin=c]{230}{$\circlearrowright$}}}

\newcommand\ct[1]{\text{\rmfamily\upshape #1}}
\newcommand\question[1]{ {\color{red} ...!? \small #1}}
\newcommand\caz[1]{\left\{\begin{array} #1 \end{array}\right.}
\newcommand\const{\text{\rmfamily\upshape const}}
\newcommand\toP{ \overset{\pro}{\to}}
\newcommand\toPP{ \overset{\text{PP}}{\to}}
\newcommand{\oeq}{\mathrel{\text{\textcircled{$=$}}}}





\usepackage{xcolor}
% \usepackage[normalem]{ulem}
\usepackage{lipsum}
\makeatletter
% \newcommand\colorwave[1][blue]{\bgroup \markoverwith{\lower3.5\p@\hbox{\sixly \textcolor{#1}{\char58}}}\ULon}
%\font\sixly=lasy6 % does not re-load if already loaded, so no memory problem.

\newmdtheoremenv[
linewidth= 1pt,linecolor= blue,%
leftmargin=20,rightmargin=20,innertopmargin=0pt, innerrightmargin=40,%
tikzsetting = { draw=lightgray, line width = 0.3pt,dashed,%
dash pattern = on 15pt off 3pt},%
splittopskip=\topskip,skipbelow=\baselineskip,%
skipabove=\baselineskip,ntheorem,roundcorner=0pt,
% backgroundcolor=pagebg,font=\color{orange}\sffamily, fontcolor=white
]{examplebox}{Exemple}[section]



\newcommand\R{\mathbb{R}}
\newcommand\Z{\mathbb{Z}}
\newcommand\N{\mathbb{N}}
\newcommand\E{\mathbb{E}}
\newcommand\F{\mathcal{F}}
\newcommand\cH{\mathcal{H}}
\newcommand\V{\mathbb{V}}
\newcommand\dmo{ ^{-1} }
\newcommand\kapa{\kappa}
\newcommand\im{Im}
\newcommand\hs{\mathcal{H}}





\usepackage{soul}

\makeatletter
\newcommand*{\whiten}[1]{\llap{\textcolor{white}{{\the\SOUL@token}}\hspace{#1pt}}}
\DeclareRobustCommand*\myul{%
    \def\SOUL@everyspace{\underline{\space}\kern\z@}%
    \def\SOUL@everytoken{%
     \setbox0=\hbox{\the\SOUL@token}%
     \ifdim\dp0>\z@
        \raisebox{\dp0}{\underline{\phantom{\the\SOUL@token}}}%
        \whiten{1}\whiten{0}%
        \whiten{-1}\whiten{-2}%
        \llap{\the\SOUL@token}%
     \else
        \underline{\the\SOUL@token}%
     \fi}%
\SOUL@}
\makeatother

\newcommand*{\demp}{\fontfamily{lmtt}\selectfont}

\DeclareTextFontCommand{\textdemp}{\demp}

\begin{document}

\ifcomment
Multiline
comment
\fi
\ifcomment
\myul{Typesetting test}
% \color[rgb]{1,1,1}
$∑_i^n≠ 60º±∞π∆¬≈√j∫h≤≥µ$

$\CR \R\pro\ind\pro\gS\pro
\mqty[a&b\\c&d]$
$\pro\mathbb{P}$
$\dd{x}$

  \[
    \alpha(x)=\left\{
                \begin{array}{ll}
                  x\\
                  \frac{1}{1+e^{-kx}}\\
                  \frac{e^x-e^{-x}}{e^x+e^{-x}}
                \end{array}
              \right.
  \]

  $\expval{x}$
  
  $\chi_\rho(ghg\dmo)=\Tr(\rho_{ghg\dmo})=\Tr(\rho_g\circ\rho_h\circ\rho\dmo_g)=\Tr(\rho_h)\overset{\mbox{\scalebox{0.5}{$\Tr(AB)=\Tr(BA)$}}}{=}\chi_\rho(h)$
  	$\mathop{\oplus}_{\substack{x\in X}}$

$\mat(\rho_g)=(a_{ij}(g))_{\scriptsize \substack{1\leq i\leq d \\ 1\leq j\leq d}}$ et $\mat(\rho'_g)=(a'_{ij}(g))_{\scriptsize \substack{1\leq i'\leq d' \\ 1\leq j'\leq d'}}$



\[\int_a^b{\mathbb{R}^2}g(u, v)\dd{P_{XY}}(u, v)=\iint g(u,v) f_{XY}(u, v)\dd \lambda(u) \dd \lambda(v)\]
$$\lim_{x\to\infty} f(x)$$	
$$\iiiint_V \mu(t,u,v,w) \,dt\,du\,dv\,dw$$
$$\sum_{n=1}^{\infty} 2^{-n} = 1$$	
\begin{definition}
	Si $X$ et $Y$ sont 2 v.a. ou definit la \textsc{Covariance} entre $X$ et $Y$ comme
	$\cov(X,Y)\overset{\text{def}}{=}\E\left[(X-\E(X))(Y-\E(Y))\right]=\E(XY)-\E(X)\E(Y)$.
\end{definition}
\fi
\pagebreak

% \tableofcontents

% insert your code here
%\input{./algebra/main.tex}
%\input{./geometrie-differentielle/main.tex}
%\input{./probabilite/main.tex}
%\input{./analyse-fonctionnelle/main.tex}
% \input{./Analyse-convexe-et-dualite-en-optimisation/main.tex}
%\input{./tikz/main.tex}
%\input{./Theorie-du-distributions/main.tex}
%\input{./optimisation/mine.tex}
 \input{./modelisation/main.tex}

% yves.aubry@univ-tln.fr : algebra

\end{document}

%\input{./optimisation/mine.tex}
 % !TEX encoding = UTF-8 Unicode
% !TEX TS-program = xelatex

\documentclass[french]{report}

%\usepackage[utf8]{inputenc}
%\usepackage[T1]{fontenc}
\usepackage{babel}


\newif\ifcomment
%\commenttrue # Show comments

\usepackage{physics}
\usepackage{amssymb}


\usepackage{amsthm}
% \usepackage{thmtools}
\usepackage{mathtools}
\usepackage{amsfonts}

\usepackage{color}

\usepackage{tikz}

\usepackage{geometry}
\geometry{a5paper, margin=0.1in, right=1cm}

\usepackage{dsfont}

\usepackage{graphicx}
\graphicspath{ {images/} }

\usepackage{faktor}

\usepackage{IEEEtrantools}
\usepackage{enumerate}   
\usepackage[PostScript=dvips]{"/Users/aware/Documents/Courses/diagrams"}


\newtheorem{theorem}{Théorème}[section]
\renewcommand{\thetheorem}{\arabic{theorem}}
\newtheorem{lemme}{Lemme}[section]
\renewcommand{\thelemme}{\arabic{lemme}}
\newtheorem{proposition}{Proposition}[section]
\renewcommand{\theproposition}{\arabic{proposition}}
\newtheorem{notations}{Notations}[section]
\newtheorem{problem}{Problème}[section]
\newtheorem{corollary}{Corollaire}[theorem]
\renewcommand{\thecorollary}{\arabic{corollary}}
\newtheorem{property}{Propriété}[section]
\newtheorem{objective}{Objectif}[section]

\theoremstyle{definition}
\newtheorem{definition}{Définition}[section]
\renewcommand{\thedefinition}{\arabic{definition}}
\newtheorem{exercise}{Exercice}[chapter]
\renewcommand{\theexercise}{\arabic{exercise}}
\newtheorem{example}{Exemple}[chapter]
\renewcommand{\theexample}{\arabic{example}}
\newtheorem*{solution}{Solution}
\newtheorem*{application}{Application}
\newtheorem*{notation}{Notation}
\newtheorem*{vocabulary}{Vocabulaire}
\newtheorem*{properties}{Propriétés}



\theoremstyle{remark}
\newtheorem*{remark}{Remarque}
\newtheorem*{rappel}{Rappel}


\usepackage{etoolbox}
\AtBeginEnvironment{exercise}{\small}
\AtBeginEnvironment{example}{\small}

\usepackage{cases}
\usepackage[red]{mypack}

\usepackage[framemethod=TikZ]{mdframed}

\definecolor{bg}{rgb}{0.4,0.25,0.95}
\definecolor{pagebg}{rgb}{0,0,0.5}
\surroundwithmdframed[
   topline=false,
   rightline=false,
   bottomline=false,
   leftmargin=\parindent,
   skipabove=8pt,
   skipbelow=8pt,
   linecolor=blue,
   innerbottommargin=10pt,
   % backgroundcolor=bg,font=\color{orange}\sffamily, fontcolor=white
]{definition}

\usepackage{empheq}
\usepackage[most]{tcolorbox}

\newtcbox{\mymath}[1][]{%
    nobeforeafter, math upper, tcbox raise base,
    enhanced, colframe=blue!30!black,
    colback=red!10, boxrule=1pt,
    #1}

\usepackage{unixode}


\DeclareMathOperator{\ord}{ord}
\DeclareMathOperator{\orb}{orb}
\DeclareMathOperator{\stab}{stab}
\DeclareMathOperator{\Stab}{stab}
\DeclareMathOperator{\ppcm}{ppcm}
\DeclareMathOperator{\conj}{Conj}
\DeclareMathOperator{\End}{End}
\DeclareMathOperator{\rot}{rot}
\DeclareMathOperator{\trs}{trace}
\DeclareMathOperator{\Ind}{Ind}
\DeclareMathOperator{\mat}{Mat}
\DeclareMathOperator{\id}{Id}
\DeclareMathOperator{\vect}{vect}
\DeclareMathOperator{\img}{img}
\DeclareMathOperator{\cov}{Cov}
\DeclareMathOperator{\dist}{dist}
\DeclareMathOperator{\irr}{Irr}
\DeclareMathOperator{\image}{Im}
\DeclareMathOperator{\pd}{\partial}
\DeclareMathOperator{\epi}{epi}
\DeclareMathOperator{\Argmin}{Argmin}
\DeclareMathOperator{\dom}{dom}
\DeclareMathOperator{\proj}{proj}
\DeclareMathOperator{\ctg}{ctg}
\DeclareMathOperator{\supp}{supp}
\DeclareMathOperator{\argmin}{argmin}
\DeclareMathOperator{\mult}{mult}
\DeclareMathOperator{\ch}{ch}
\DeclareMathOperator{\sh}{sh}
\DeclareMathOperator{\rang}{rang}
\DeclareMathOperator{\diam}{diam}
\DeclareMathOperator{\Epigraphe}{Epigraphe}




\usepackage{xcolor}
\everymath{\color{blue}}
%\everymath{\color[rgb]{0,1,1}}
%\pagecolor[rgb]{0,0,0.5}


\newcommand*{\pdtest}[3][]{\ensuremath{\frac{\partial^{#1} #2}{\partial #3}}}

\newcommand*{\deffunc}[6][]{\ensuremath{
\begin{array}{rcl}
#2 : #3 &\rightarrow& #4\\
#5 &\mapsto& #6
\end{array}
}}

\newcommand{\eqcolon}{\mathrel{\resizebox{\widthof{$\mathord{=}$}}{\height}{ $\!\!=\!\!\resizebox{1.2\width}{0.8\height}{\raisebox{0.23ex}{$\mathop{:}$}}\!\!$ }}}
\newcommand{\coloneq}{\mathrel{\resizebox{\widthof{$\mathord{=}$}}{\height}{ $\!\!\resizebox{1.2\width}{0.8\height}{\raisebox{0.23ex}{$\mathop{:}$}}\!\!=\!\!$ }}}
\newcommand{\eqcolonl}{\ensuremath{\mathrel{=\!\!\mathop{:}}}}
\newcommand{\coloneql}{\ensuremath{\mathrel{\mathop{:} \!\! =}}}
\newcommand{\vc}[1]{% inline column vector
  \left(\begin{smallmatrix}#1\end{smallmatrix}\right)%
}
\newcommand{\vr}[1]{% inline row vector
  \begin{smallmatrix}(\,#1\,)\end{smallmatrix}%
}
\makeatletter
\newcommand*{\defeq}{\ =\mathrel{\rlap{%
                     \raisebox{0.3ex}{$\m@th\cdot$}}%
                     \raisebox{-0.3ex}{$\m@th\cdot$}}%
                     }
\makeatother

\newcommand{\mathcircle}[1]{% inline row vector
 \overset{\circ}{#1}
}
\newcommand{\ulim}{% low limit
 \underline{\lim}
}
\newcommand{\ssi}{% iff
\iff
}
\newcommand{\ps}[2]{
\expval{#1 | #2}
}
\newcommand{\df}[1]{
\mqty{#1}
}
\newcommand{\n}[1]{
\norm{#1}
}
\newcommand{\sys}[1]{
\left\{\smqty{#1}\right.
}


\newcommand{\eqdef}{\ensuremath{\overset{\text{def}}=}}


\def\Circlearrowright{\ensuremath{%
  \rotatebox[origin=c]{230}{$\circlearrowright$}}}

\newcommand\ct[1]{\text{\rmfamily\upshape #1}}
\newcommand\question[1]{ {\color{red} ...!? \small #1}}
\newcommand\caz[1]{\left\{\begin{array} #1 \end{array}\right.}
\newcommand\const{\text{\rmfamily\upshape const}}
\newcommand\toP{ \overset{\pro}{\to}}
\newcommand\toPP{ \overset{\text{PP}}{\to}}
\newcommand{\oeq}{\mathrel{\text{\textcircled{$=$}}}}





\usepackage{xcolor}
% \usepackage[normalem]{ulem}
\usepackage{lipsum}
\makeatletter
% \newcommand\colorwave[1][blue]{\bgroup \markoverwith{\lower3.5\p@\hbox{\sixly \textcolor{#1}{\char58}}}\ULon}
%\font\sixly=lasy6 % does not re-load if already loaded, so no memory problem.

\newmdtheoremenv[
linewidth= 1pt,linecolor= blue,%
leftmargin=20,rightmargin=20,innertopmargin=0pt, innerrightmargin=40,%
tikzsetting = { draw=lightgray, line width = 0.3pt,dashed,%
dash pattern = on 15pt off 3pt},%
splittopskip=\topskip,skipbelow=\baselineskip,%
skipabove=\baselineskip,ntheorem,roundcorner=0pt,
% backgroundcolor=pagebg,font=\color{orange}\sffamily, fontcolor=white
]{examplebox}{Exemple}[section]



\newcommand\R{\mathbb{R}}
\newcommand\Z{\mathbb{Z}}
\newcommand\N{\mathbb{N}}
\newcommand\E{\mathbb{E}}
\newcommand\F{\mathcal{F}}
\newcommand\cH{\mathcal{H}}
\newcommand\V{\mathbb{V}}
\newcommand\dmo{ ^{-1} }
\newcommand\kapa{\kappa}
\newcommand\im{Im}
\newcommand\hs{\mathcal{H}}





\usepackage{soul}

\makeatletter
\newcommand*{\whiten}[1]{\llap{\textcolor{white}{{\the\SOUL@token}}\hspace{#1pt}}}
\DeclareRobustCommand*\myul{%
    \def\SOUL@everyspace{\underline{\space}\kern\z@}%
    \def\SOUL@everytoken{%
     \setbox0=\hbox{\the\SOUL@token}%
     \ifdim\dp0>\z@
        \raisebox{\dp0}{\underline{\phantom{\the\SOUL@token}}}%
        \whiten{1}\whiten{0}%
        \whiten{-1}\whiten{-2}%
        \llap{\the\SOUL@token}%
     \else
        \underline{\the\SOUL@token}%
     \fi}%
\SOUL@}
\makeatother

\newcommand*{\demp}{\fontfamily{lmtt}\selectfont}

\DeclareTextFontCommand{\textdemp}{\demp}

\begin{document}

\ifcomment
Multiline
comment
\fi
\ifcomment
\myul{Typesetting test}
% \color[rgb]{1,1,1}
$∑_i^n≠ 60º±∞π∆¬≈√j∫h≤≥µ$

$\CR \R\pro\ind\pro\gS\pro
\mqty[a&b\\c&d]$
$\pro\mathbb{P}$
$\dd{x}$

  \[
    \alpha(x)=\left\{
                \begin{array}{ll}
                  x\\
                  \frac{1}{1+e^{-kx}}\\
                  \frac{e^x-e^{-x}}{e^x+e^{-x}}
                \end{array}
              \right.
  \]

  $\expval{x}$
  
  $\chi_\rho(ghg\dmo)=\Tr(\rho_{ghg\dmo})=\Tr(\rho_g\circ\rho_h\circ\rho\dmo_g)=\Tr(\rho_h)\overset{\mbox{\scalebox{0.5}{$\Tr(AB)=\Tr(BA)$}}}{=}\chi_\rho(h)$
  	$\mathop{\oplus}_{\substack{x\in X}}$

$\mat(\rho_g)=(a_{ij}(g))_{\scriptsize \substack{1\leq i\leq d \\ 1\leq j\leq d}}$ et $\mat(\rho'_g)=(a'_{ij}(g))_{\scriptsize \substack{1\leq i'\leq d' \\ 1\leq j'\leq d'}}$



\[\int_a^b{\mathbb{R}^2}g(u, v)\dd{P_{XY}}(u, v)=\iint g(u,v) f_{XY}(u, v)\dd \lambda(u) \dd \lambda(v)\]
$$\lim_{x\to\infty} f(x)$$	
$$\iiiint_V \mu(t,u,v,w) \,dt\,du\,dv\,dw$$
$$\sum_{n=1}^{\infty} 2^{-n} = 1$$	
\begin{definition}
	Si $X$ et $Y$ sont 2 v.a. ou definit la \textsc{Covariance} entre $X$ et $Y$ comme
	$\cov(X,Y)\overset{\text{def}}{=}\E\left[(X-\E(X))(Y-\E(Y))\right]=\E(XY)-\E(X)\E(Y)$.
\end{definition}
\fi
\pagebreak

% \tableofcontents

% insert your code here
%\input{./algebra/main.tex}
%\input{./geometrie-differentielle/main.tex}
%\input{./probabilite/main.tex}
%\input{./analyse-fonctionnelle/main.tex}
% \input{./Analyse-convexe-et-dualite-en-optimisation/main.tex}
%\input{./tikz/main.tex}
%\input{./Theorie-du-distributions/main.tex}
%\input{./optimisation/mine.tex}
 \input{./modelisation/main.tex}

% yves.aubry@univ-tln.fr : algebra

\end{document}


% yves.aubry@univ-tln.fr : algebra

\end{document}

%\input{./optimisation/mine.tex}
 % !TEX encoding = UTF-8 Unicode
% !TEX TS-program = xelatex

\documentclass[french]{report}

%\usepackage[utf8]{inputenc}
%\usepackage[T1]{fontenc}
\usepackage{babel}


\newif\ifcomment
%\commenttrue # Show comments

\usepackage{physics}
\usepackage{amssymb}


\usepackage{amsthm}
% \usepackage{thmtools}
\usepackage{mathtools}
\usepackage{amsfonts}

\usepackage{color}

\usepackage{tikz}

\usepackage{geometry}
\geometry{a5paper, margin=0.1in, right=1cm}

\usepackage{dsfont}

\usepackage{graphicx}
\graphicspath{ {images/} }

\usepackage{faktor}

\usepackage{IEEEtrantools}
\usepackage{enumerate}   
\usepackage[PostScript=dvips]{"/Users/aware/Documents/Courses/diagrams"}


\newtheorem{theorem}{Théorème}[section]
\renewcommand{\thetheorem}{\arabic{theorem}}
\newtheorem{lemme}{Lemme}[section]
\renewcommand{\thelemme}{\arabic{lemme}}
\newtheorem{proposition}{Proposition}[section]
\renewcommand{\theproposition}{\arabic{proposition}}
\newtheorem{notations}{Notations}[section]
\newtheorem{problem}{Problème}[section]
\newtheorem{corollary}{Corollaire}[theorem]
\renewcommand{\thecorollary}{\arabic{corollary}}
\newtheorem{property}{Propriété}[section]
\newtheorem{objective}{Objectif}[section]

\theoremstyle{definition}
\newtheorem{definition}{Définition}[section]
\renewcommand{\thedefinition}{\arabic{definition}}
\newtheorem{exercise}{Exercice}[chapter]
\renewcommand{\theexercise}{\arabic{exercise}}
\newtheorem{example}{Exemple}[chapter]
\renewcommand{\theexample}{\arabic{example}}
\newtheorem*{solution}{Solution}
\newtheorem*{application}{Application}
\newtheorem*{notation}{Notation}
\newtheorem*{vocabulary}{Vocabulaire}
\newtheorem*{properties}{Propriétés}



\theoremstyle{remark}
\newtheorem*{remark}{Remarque}
\newtheorem*{rappel}{Rappel}


\usepackage{etoolbox}
\AtBeginEnvironment{exercise}{\small}
\AtBeginEnvironment{example}{\small}

\usepackage{cases}
\usepackage[red]{mypack}

\usepackage[framemethod=TikZ]{mdframed}

\definecolor{bg}{rgb}{0.4,0.25,0.95}
\definecolor{pagebg}{rgb}{0,0,0.5}
\surroundwithmdframed[
   topline=false,
   rightline=false,
   bottomline=false,
   leftmargin=\parindent,
   skipabove=8pt,
   skipbelow=8pt,
   linecolor=blue,
   innerbottommargin=10pt,
   % backgroundcolor=bg,font=\color{orange}\sffamily, fontcolor=white
]{definition}

\usepackage{empheq}
\usepackage[most]{tcolorbox}

\newtcbox{\mymath}[1][]{%
    nobeforeafter, math upper, tcbox raise base,
    enhanced, colframe=blue!30!black,
    colback=red!10, boxrule=1pt,
    #1}

\usepackage{unixode}


\DeclareMathOperator{\ord}{ord}
\DeclareMathOperator{\orb}{orb}
\DeclareMathOperator{\stab}{stab}
\DeclareMathOperator{\Stab}{stab}
\DeclareMathOperator{\ppcm}{ppcm}
\DeclareMathOperator{\conj}{Conj}
\DeclareMathOperator{\End}{End}
\DeclareMathOperator{\rot}{rot}
\DeclareMathOperator{\trs}{trace}
\DeclareMathOperator{\Ind}{Ind}
\DeclareMathOperator{\mat}{Mat}
\DeclareMathOperator{\id}{Id}
\DeclareMathOperator{\vect}{vect}
\DeclareMathOperator{\img}{img}
\DeclareMathOperator{\cov}{Cov}
\DeclareMathOperator{\dist}{dist}
\DeclareMathOperator{\irr}{Irr}
\DeclareMathOperator{\image}{Im}
\DeclareMathOperator{\pd}{\partial}
\DeclareMathOperator{\epi}{epi}
\DeclareMathOperator{\Argmin}{Argmin}
\DeclareMathOperator{\dom}{dom}
\DeclareMathOperator{\proj}{proj}
\DeclareMathOperator{\ctg}{ctg}
\DeclareMathOperator{\supp}{supp}
\DeclareMathOperator{\argmin}{argmin}
\DeclareMathOperator{\mult}{mult}
\DeclareMathOperator{\ch}{ch}
\DeclareMathOperator{\sh}{sh}
\DeclareMathOperator{\rang}{rang}
\DeclareMathOperator{\diam}{diam}
\DeclareMathOperator{\Epigraphe}{Epigraphe}




\usepackage{xcolor}
\everymath{\color{blue}}
%\everymath{\color[rgb]{0,1,1}}
%\pagecolor[rgb]{0,0,0.5}


\newcommand*{\pdtest}[3][]{\ensuremath{\frac{\partial^{#1} #2}{\partial #3}}}

\newcommand*{\deffunc}[6][]{\ensuremath{
\begin{array}{rcl}
#2 : #3 &\rightarrow& #4\\
#5 &\mapsto& #6
\end{array}
}}

\newcommand{\eqcolon}{\mathrel{\resizebox{\widthof{$\mathord{=}$}}{\height}{ $\!\!=\!\!\resizebox{1.2\width}{0.8\height}{\raisebox{0.23ex}{$\mathop{:}$}}\!\!$ }}}
\newcommand{\coloneq}{\mathrel{\resizebox{\widthof{$\mathord{=}$}}{\height}{ $\!\!\resizebox{1.2\width}{0.8\height}{\raisebox{0.23ex}{$\mathop{:}$}}\!\!=\!\!$ }}}
\newcommand{\eqcolonl}{\ensuremath{\mathrel{=\!\!\mathop{:}}}}
\newcommand{\coloneql}{\ensuremath{\mathrel{\mathop{:} \!\! =}}}
\newcommand{\vc}[1]{% inline column vector
  \left(\begin{smallmatrix}#1\end{smallmatrix}\right)%
}
\newcommand{\vr}[1]{% inline row vector
  \begin{smallmatrix}(\,#1\,)\end{smallmatrix}%
}
\makeatletter
\newcommand*{\defeq}{\ =\mathrel{\rlap{%
                     \raisebox{0.3ex}{$\m@th\cdot$}}%
                     \raisebox{-0.3ex}{$\m@th\cdot$}}%
                     }
\makeatother

\newcommand{\mathcircle}[1]{% inline row vector
 \overset{\circ}{#1}
}
\newcommand{\ulim}{% low limit
 \underline{\lim}
}
\newcommand{\ssi}{% iff
\iff
}
\newcommand{\ps}[2]{
\expval{#1 | #2}
}
\newcommand{\df}[1]{
\mqty{#1}
}
\newcommand{\n}[1]{
\norm{#1}
}
\newcommand{\sys}[1]{
\left\{\smqty{#1}\right.
}


\newcommand{\eqdef}{\ensuremath{\overset{\text{def}}=}}


\def\Circlearrowright{\ensuremath{%
  \rotatebox[origin=c]{230}{$\circlearrowright$}}}

\newcommand\ct[1]{\text{\rmfamily\upshape #1}}
\newcommand\question[1]{ {\color{red} ...!? \small #1}}
\newcommand\caz[1]{\left\{\begin{array} #1 \end{array}\right.}
\newcommand\const{\text{\rmfamily\upshape const}}
\newcommand\toP{ \overset{\pro}{\to}}
\newcommand\toPP{ \overset{\text{PP}}{\to}}
\newcommand{\oeq}{\mathrel{\text{\textcircled{$=$}}}}





\usepackage{xcolor}
% \usepackage[normalem]{ulem}
\usepackage{lipsum}
\makeatletter
% \newcommand\colorwave[1][blue]{\bgroup \markoverwith{\lower3.5\p@\hbox{\sixly \textcolor{#1}{\char58}}}\ULon}
%\font\sixly=lasy6 % does not re-load if already loaded, so no memory problem.

\newmdtheoremenv[
linewidth= 1pt,linecolor= blue,%
leftmargin=20,rightmargin=20,innertopmargin=0pt, innerrightmargin=40,%
tikzsetting = { draw=lightgray, line width = 0.3pt,dashed,%
dash pattern = on 15pt off 3pt},%
splittopskip=\topskip,skipbelow=\baselineskip,%
skipabove=\baselineskip,ntheorem,roundcorner=0pt,
% backgroundcolor=pagebg,font=\color{orange}\sffamily, fontcolor=white
]{examplebox}{Exemple}[section]



\newcommand\R{\mathbb{R}}
\newcommand\Z{\mathbb{Z}}
\newcommand\N{\mathbb{N}}
\newcommand\E{\mathbb{E}}
\newcommand\F{\mathcal{F}}
\newcommand\cH{\mathcal{H}}
\newcommand\V{\mathbb{V}}
\newcommand\dmo{ ^{-1} }
\newcommand\kapa{\kappa}
\newcommand\im{Im}
\newcommand\hs{\mathcal{H}}





\usepackage{soul}

\makeatletter
\newcommand*{\whiten}[1]{\llap{\textcolor{white}{{\the\SOUL@token}}\hspace{#1pt}}}
\DeclareRobustCommand*\myul{%
    \def\SOUL@everyspace{\underline{\space}\kern\z@}%
    \def\SOUL@everytoken{%
     \setbox0=\hbox{\the\SOUL@token}%
     \ifdim\dp0>\z@
        \raisebox{\dp0}{\underline{\phantom{\the\SOUL@token}}}%
        \whiten{1}\whiten{0}%
        \whiten{-1}\whiten{-2}%
        \llap{\the\SOUL@token}%
     \else
        \underline{\the\SOUL@token}%
     \fi}%
\SOUL@}
\makeatother

\newcommand*{\demp}{\fontfamily{lmtt}\selectfont}

\DeclareTextFontCommand{\textdemp}{\demp}

\begin{document}

\ifcomment
Multiline
comment
\fi
\ifcomment
\myul{Typesetting test}
% \color[rgb]{1,1,1}
$∑_i^n≠ 60º±∞π∆¬≈√j∫h≤≥µ$

$\CR \R\pro\ind\pro\gS\pro
\mqty[a&b\\c&d]$
$\pro\mathbb{P}$
$\dd{x}$

  \[
    \alpha(x)=\left\{
                \begin{array}{ll}
                  x\\
                  \frac{1}{1+e^{-kx}}\\
                  \frac{e^x-e^{-x}}{e^x+e^{-x}}
                \end{array}
              \right.
  \]

  $\expval{x}$
  
  $\chi_\rho(ghg\dmo)=\Tr(\rho_{ghg\dmo})=\Tr(\rho_g\circ\rho_h\circ\rho\dmo_g)=\Tr(\rho_h)\overset{\mbox{\scalebox{0.5}{$\Tr(AB)=\Tr(BA)$}}}{=}\chi_\rho(h)$
  	$\mathop{\oplus}_{\substack{x\in X}}$

$\mat(\rho_g)=(a_{ij}(g))_{\scriptsize \substack{1\leq i\leq d \\ 1\leq j\leq d}}$ et $\mat(\rho'_g)=(a'_{ij}(g))_{\scriptsize \substack{1\leq i'\leq d' \\ 1\leq j'\leq d'}}$



\[\int_a^b{\mathbb{R}^2}g(u, v)\dd{P_{XY}}(u, v)=\iint g(u,v) f_{XY}(u, v)\dd \lambda(u) \dd \lambda(v)\]
$$\lim_{x\to\infty} f(x)$$	
$$\iiiint_V \mu(t,u,v,w) \,dt\,du\,dv\,dw$$
$$\sum_{n=1}^{\infty} 2^{-n} = 1$$	
\begin{definition}
	Si $X$ et $Y$ sont 2 v.a. ou definit la \textsc{Covariance} entre $X$ et $Y$ comme
	$\cov(X,Y)\overset{\text{def}}{=}\E\left[(X-\E(X))(Y-\E(Y))\right]=\E(XY)-\E(X)\E(Y)$.
\end{definition}
\fi
\pagebreak

% \tableofcontents

% insert your code here
%% !TEX encoding = UTF-8 Unicode
% !TEX TS-program = xelatex

\documentclass[french]{report}

%\usepackage[utf8]{inputenc}
%\usepackage[T1]{fontenc}
\usepackage{babel}


\newif\ifcomment
%\commenttrue # Show comments

\usepackage{physics}
\usepackage{amssymb}


\usepackage{amsthm}
% \usepackage{thmtools}
\usepackage{mathtools}
\usepackage{amsfonts}

\usepackage{color}

\usepackage{tikz}

\usepackage{geometry}
\geometry{a5paper, margin=0.1in, right=1cm}

\usepackage{dsfont}

\usepackage{graphicx}
\graphicspath{ {images/} }

\usepackage{faktor}

\usepackage{IEEEtrantools}
\usepackage{enumerate}   
\usepackage[PostScript=dvips]{"/Users/aware/Documents/Courses/diagrams"}


\newtheorem{theorem}{Théorème}[section]
\renewcommand{\thetheorem}{\arabic{theorem}}
\newtheorem{lemme}{Lemme}[section]
\renewcommand{\thelemme}{\arabic{lemme}}
\newtheorem{proposition}{Proposition}[section]
\renewcommand{\theproposition}{\arabic{proposition}}
\newtheorem{notations}{Notations}[section]
\newtheorem{problem}{Problème}[section]
\newtheorem{corollary}{Corollaire}[theorem]
\renewcommand{\thecorollary}{\arabic{corollary}}
\newtheorem{property}{Propriété}[section]
\newtheorem{objective}{Objectif}[section]

\theoremstyle{definition}
\newtheorem{definition}{Définition}[section]
\renewcommand{\thedefinition}{\arabic{definition}}
\newtheorem{exercise}{Exercice}[chapter]
\renewcommand{\theexercise}{\arabic{exercise}}
\newtheorem{example}{Exemple}[chapter]
\renewcommand{\theexample}{\arabic{example}}
\newtheorem*{solution}{Solution}
\newtheorem*{application}{Application}
\newtheorem*{notation}{Notation}
\newtheorem*{vocabulary}{Vocabulaire}
\newtheorem*{properties}{Propriétés}



\theoremstyle{remark}
\newtheorem*{remark}{Remarque}
\newtheorem*{rappel}{Rappel}


\usepackage{etoolbox}
\AtBeginEnvironment{exercise}{\small}
\AtBeginEnvironment{example}{\small}

\usepackage{cases}
\usepackage[red]{mypack}

\usepackage[framemethod=TikZ]{mdframed}

\definecolor{bg}{rgb}{0.4,0.25,0.95}
\definecolor{pagebg}{rgb}{0,0,0.5}
\surroundwithmdframed[
   topline=false,
   rightline=false,
   bottomline=false,
   leftmargin=\parindent,
   skipabove=8pt,
   skipbelow=8pt,
   linecolor=blue,
   innerbottommargin=10pt,
   % backgroundcolor=bg,font=\color{orange}\sffamily, fontcolor=white
]{definition}

\usepackage{empheq}
\usepackage[most]{tcolorbox}

\newtcbox{\mymath}[1][]{%
    nobeforeafter, math upper, tcbox raise base,
    enhanced, colframe=blue!30!black,
    colback=red!10, boxrule=1pt,
    #1}

\usepackage{unixode}


\DeclareMathOperator{\ord}{ord}
\DeclareMathOperator{\orb}{orb}
\DeclareMathOperator{\stab}{stab}
\DeclareMathOperator{\Stab}{stab}
\DeclareMathOperator{\ppcm}{ppcm}
\DeclareMathOperator{\conj}{Conj}
\DeclareMathOperator{\End}{End}
\DeclareMathOperator{\rot}{rot}
\DeclareMathOperator{\trs}{trace}
\DeclareMathOperator{\Ind}{Ind}
\DeclareMathOperator{\mat}{Mat}
\DeclareMathOperator{\id}{Id}
\DeclareMathOperator{\vect}{vect}
\DeclareMathOperator{\img}{img}
\DeclareMathOperator{\cov}{Cov}
\DeclareMathOperator{\dist}{dist}
\DeclareMathOperator{\irr}{Irr}
\DeclareMathOperator{\image}{Im}
\DeclareMathOperator{\pd}{\partial}
\DeclareMathOperator{\epi}{epi}
\DeclareMathOperator{\Argmin}{Argmin}
\DeclareMathOperator{\dom}{dom}
\DeclareMathOperator{\proj}{proj}
\DeclareMathOperator{\ctg}{ctg}
\DeclareMathOperator{\supp}{supp}
\DeclareMathOperator{\argmin}{argmin}
\DeclareMathOperator{\mult}{mult}
\DeclareMathOperator{\ch}{ch}
\DeclareMathOperator{\sh}{sh}
\DeclareMathOperator{\rang}{rang}
\DeclareMathOperator{\diam}{diam}
\DeclareMathOperator{\Epigraphe}{Epigraphe}




\usepackage{xcolor}
\everymath{\color{blue}}
%\everymath{\color[rgb]{0,1,1}}
%\pagecolor[rgb]{0,0,0.5}


\newcommand*{\pdtest}[3][]{\ensuremath{\frac{\partial^{#1} #2}{\partial #3}}}

\newcommand*{\deffunc}[6][]{\ensuremath{
\begin{array}{rcl}
#2 : #3 &\rightarrow& #4\\
#5 &\mapsto& #6
\end{array}
}}

\newcommand{\eqcolon}{\mathrel{\resizebox{\widthof{$\mathord{=}$}}{\height}{ $\!\!=\!\!\resizebox{1.2\width}{0.8\height}{\raisebox{0.23ex}{$\mathop{:}$}}\!\!$ }}}
\newcommand{\coloneq}{\mathrel{\resizebox{\widthof{$\mathord{=}$}}{\height}{ $\!\!\resizebox{1.2\width}{0.8\height}{\raisebox{0.23ex}{$\mathop{:}$}}\!\!=\!\!$ }}}
\newcommand{\eqcolonl}{\ensuremath{\mathrel{=\!\!\mathop{:}}}}
\newcommand{\coloneql}{\ensuremath{\mathrel{\mathop{:} \!\! =}}}
\newcommand{\vc}[1]{% inline column vector
  \left(\begin{smallmatrix}#1\end{smallmatrix}\right)%
}
\newcommand{\vr}[1]{% inline row vector
  \begin{smallmatrix}(\,#1\,)\end{smallmatrix}%
}
\makeatletter
\newcommand*{\defeq}{\ =\mathrel{\rlap{%
                     \raisebox{0.3ex}{$\m@th\cdot$}}%
                     \raisebox{-0.3ex}{$\m@th\cdot$}}%
                     }
\makeatother

\newcommand{\mathcircle}[1]{% inline row vector
 \overset{\circ}{#1}
}
\newcommand{\ulim}{% low limit
 \underline{\lim}
}
\newcommand{\ssi}{% iff
\iff
}
\newcommand{\ps}[2]{
\expval{#1 | #2}
}
\newcommand{\df}[1]{
\mqty{#1}
}
\newcommand{\n}[1]{
\norm{#1}
}
\newcommand{\sys}[1]{
\left\{\smqty{#1}\right.
}


\newcommand{\eqdef}{\ensuremath{\overset{\text{def}}=}}


\def\Circlearrowright{\ensuremath{%
  \rotatebox[origin=c]{230}{$\circlearrowright$}}}

\newcommand\ct[1]{\text{\rmfamily\upshape #1}}
\newcommand\question[1]{ {\color{red} ...!? \small #1}}
\newcommand\caz[1]{\left\{\begin{array} #1 \end{array}\right.}
\newcommand\const{\text{\rmfamily\upshape const}}
\newcommand\toP{ \overset{\pro}{\to}}
\newcommand\toPP{ \overset{\text{PP}}{\to}}
\newcommand{\oeq}{\mathrel{\text{\textcircled{$=$}}}}





\usepackage{xcolor}
% \usepackage[normalem]{ulem}
\usepackage{lipsum}
\makeatletter
% \newcommand\colorwave[1][blue]{\bgroup \markoverwith{\lower3.5\p@\hbox{\sixly \textcolor{#1}{\char58}}}\ULon}
%\font\sixly=lasy6 % does not re-load if already loaded, so no memory problem.

\newmdtheoremenv[
linewidth= 1pt,linecolor= blue,%
leftmargin=20,rightmargin=20,innertopmargin=0pt, innerrightmargin=40,%
tikzsetting = { draw=lightgray, line width = 0.3pt,dashed,%
dash pattern = on 15pt off 3pt},%
splittopskip=\topskip,skipbelow=\baselineskip,%
skipabove=\baselineskip,ntheorem,roundcorner=0pt,
% backgroundcolor=pagebg,font=\color{orange}\sffamily, fontcolor=white
]{examplebox}{Exemple}[section]



\newcommand\R{\mathbb{R}}
\newcommand\Z{\mathbb{Z}}
\newcommand\N{\mathbb{N}}
\newcommand\E{\mathbb{E}}
\newcommand\F{\mathcal{F}}
\newcommand\cH{\mathcal{H}}
\newcommand\V{\mathbb{V}}
\newcommand\dmo{ ^{-1} }
\newcommand\kapa{\kappa}
\newcommand\im{Im}
\newcommand\hs{\mathcal{H}}





\usepackage{soul}

\makeatletter
\newcommand*{\whiten}[1]{\llap{\textcolor{white}{{\the\SOUL@token}}\hspace{#1pt}}}
\DeclareRobustCommand*\myul{%
    \def\SOUL@everyspace{\underline{\space}\kern\z@}%
    \def\SOUL@everytoken{%
     \setbox0=\hbox{\the\SOUL@token}%
     \ifdim\dp0>\z@
        \raisebox{\dp0}{\underline{\phantom{\the\SOUL@token}}}%
        \whiten{1}\whiten{0}%
        \whiten{-1}\whiten{-2}%
        \llap{\the\SOUL@token}%
     \else
        \underline{\the\SOUL@token}%
     \fi}%
\SOUL@}
\makeatother

\newcommand*{\demp}{\fontfamily{lmtt}\selectfont}

\DeclareTextFontCommand{\textdemp}{\demp}

\begin{document}

\ifcomment
Multiline
comment
\fi
\ifcomment
\myul{Typesetting test}
% \color[rgb]{1,1,1}
$∑_i^n≠ 60º±∞π∆¬≈√j∫h≤≥µ$

$\CR \R\pro\ind\pro\gS\pro
\mqty[a&b\\c&d]$
$\pro\mathbb{P}$
$\dd{x}$

  \[
    \alpha(x)=\left\{
                \begin{array}{ll}
                  x\\
                  \frac{1}{1+e^{-kx}}\\
                  \frac{e^x-e^{-x}}{e^x+e^{-x}}
                \end{array}
              \right.
  \]

  $\expval{x}$
  
  $\chi_\rho(ghg\dmo)=\Tr(\rho_{ghg\dmo})=\Tr(\rho_g\circ\rho_h\circ\rho\dmo_g)=\Tr(\rho_h)\overset{\mbox{\scalebox{0.5}{$\Tr(AB)=\Tr(BA)$}}}{=}\chi_\rho(h)$
  	$\mathop{\oplus}_{\substack{x\in X}}$

$\mat(\rho_g)=(a_{ij}(g))_{\scriptsize \substack{1\leq i\leq d \\ 1\leq j\leq d}}$ et $\mat(\rho'_g)=(a'_{ij}(g))_{\scriptsize \substack{1\leq i'\leq d' \\ 1\leq j'\leq d'}}$



\[\int_a^b{\mathbb{R}^2}g(u, v)\dd{P_{XY}}(u, v)=\iint g(u,v) f_{XY}(u, v)\dd \lambda(u) \dd \lambda(v)\]
$$\lim_{x\to\infty} f(x)$$	
$$\iiiint_V \mu(t,u,v,w) \,dt\,du\,dv\,dw$$
$$\sum_{n=1}^{\infty} 2^{-n} = 1$$	
\begin{definition}
	Si $X$ et $Y$ sont 2 v.a. ou definit la \textsc{Covariance} entre $X$ et $Y$ comme
	$\cov(X,Y)\overset{\text{def}}{=}\E\left[(X-\E(X))(Y-\E(Y))\right]=\E(XY)-\E(X)\E(Y)$.
\end{definition}
\fi
\pagebreak

% \tableofcontents

% insert your code here
%\input{./algebra/main.tex}
%\input{./geometrie-differentielle/main.tex}
%\input{./probabilite/main.tex}
%\input{./analyse-fonctionnelle/main.tex}
% \input{./Analyse-convexe-et-dualite-en-optimisation/main.tex}
%\input{./tikz/main.tex}
%\input{./Theorie-du-distributions/main.tex}
%\input{./optimisation/mine.tex}
 \input{./modelisation/main.tex}

% yves.aubry@univ-tln.fr : algebra

\end{document}

%% !TEX encoding = UTF-8 Unicode
% !TEX TS-program = xelatex

\documentclass[french]{report}

%\usepackage[utf8]{inputenc}
%\usepackage[T1]{fontenc}
\usepackage{babel}


\newif\ifcomment
%\commenttrue # Show comments

\usepackage{physics}
\usepackage{amssymb}


\usepackage{amsthm}
% \usepackage{thmtools}
\usepackage{mathtools}
\usepackage{amsfonts}

\usepackage{color}

\usepackage{tikz}

\usepackage{geometry}
\geometry{a5paper, margin=0.1in, right=1cm}

\usepackage{dsfont}

\usepackage{graphicx}
\graphicspath{ {images/} }

\usepackage{faktor}

\usepackage{IEEEtrantools}
\usepackage{enumerate}   
\usepackage[PostScript=dvips]{"/Users/aware/Documents/Courses/diagrams"}


\newtheorem{theorem}{Théorème}[section]
\renewcommand{\thetheorem}{\arabic{theorem}}
\newtheorem{lemme}{Lemme}[section]
\renewcommand{\thelemme}{\arabic{lemme}}
\newtheorem{proposition}{Proposition}[section]
\renewcommand{\theproposition}{\arabic{proposition}}
\newtheorem{notations}{Notations}[section]
\newtheorem{problem}{Problème}[section]
\newtheorem{corollary}{Corollaire}[theorem]
\renewcommand{\thecorollary}{\arabic{corollary}}
\newtheorem{property}{Propriété}[section]
\newtheorem{objective}{Objectif}[section]

\theoremstyle{definition}
\newtheorem{definition}{Définition}[section]
\renewcommand{\thedefinition}{\arabic{definition}}
\newtheorem{exercise}{Exercice}[chapter]
\renewcommand{\theexercise}{\arabic{exercise}}
\newtheorem{example}{Exemple}[chapter]
\renewcommand{\theexample}{\arabic{example}}
\newtheorem*{solution}{Solution}
\newtheorem*{application}{Application}
\newtheorem*{notation}{Notation}
\newtheorem*{vocabulary}{Vocabulaire}
\newtheorem*{properties}{Propriétés}



\theoremstyle{remark}
\newtheorem*{remark}{Remarque}
\newtheorem*{rappel}{Rappel}


\usepackage{etoolbox}
\AtBeginEnvironment{exercise}{\small}
\AtBeginEnvironment{example}{\small}

\usepackage{cases}
\usepackage[red]{mypack}

\usepackage[framemethod=TikZ]{mdframed}

\definecolor{bg}{rgb}{0.4,0.25,0.95}
\definecolor{pagebg}{rgb}{0,0,0.5}
\surroundwithmdframed[
   topline=false,
   rightline=false,
   bottomline=false,
   leftmargin=\parindent,
   skipabove=8pt,
   skipbelow=8pt,
   linecolor=blue,
   innerbottommargin=10pt,
   % backgroundcolor=bg,font=\color{orange}\sffamily, fontcolor=white
]{definition}

\usepackage{empheq}
\usepackage[most]{tcolorbox}

\newtcbox{\mymath}[1][]{%
    nobeforeafter, math upper, tcbox raise base,
    enhanced, colframe=blue!30!black,
    colback=red!10, boxrule=1pt,
    #1}

\usepackage{unixode}


\DeclareMathOperator{\ord}{ord}
\DeclareMathOperator{\orb}{orb}
\DeclareMathOperator{\stab}{stab}
\DeclareMathOperator{\Stab}{stab}
\DeclareMathOperator{\ppcm}{ppcm}
\DeclareMathOperator{\conj}{Conj}
\DeclareMathOperator{\End}{End}
\DeclareMathOperator{\rot}{rot}
\DeclareMathOperator{\trs}{trace}
\DeclareMathOperator{\Ind}{Ind}
\DeclareMathOperator{\mat}{Mat}
\DeclareMathOperator{\id}{Id}
\DeclareMathOperator{\vect}{vect}
\DeclareMathOperator{\img}{img}
\DeclareMathOperator{\cov}{Cov}
\DeclareMathOperator{\dist}{dist}
\DeclareMathOperator{\irr}{Irr}
\DeclareMathOperator{\image}{Im}
\DeclareMathOperator{\pd}{\partial}
\DeclareMathOperator{\epi}{epi}
\DeclareMathOperator{\Argmin}{Argmin}
\DeclareMathOperator{\dom}{dom}
\DeclareMathOperator{\proj}{proj}
\DeclareMathOperator{\ctg}{ctg}
\DeclareMathOperator{\supp}{supp}
\DeclareMathOperator{\argmin}{argmin}
\DeclareMathOperator{\mult}{mult}
\DeclareMathOperator{\ch}{ch}
\DeclareMathOperator{\sh}{sh}
\DeclareMathOperator{\rang}{rang}
\DeclareMathOperator{\diam}{diam}
\DeclareMathOperator{\Epigraphe}{Epigraphe}




\usepackage{xcolor}
\everymath{\color{blue}}
%\everymath{\color[rgb]{0,1,1}}
%\pagecolor[rgb]{0,0,0.5}


\newcommand*{\pdtest}[3][]{\ensuremath{\frac{\partial^{#1} #2}{\partial #3}}}

\newcommand*{\deffunc}[6][]{\ensuremath{
\begin{array}{rcl}
#2 : #3 &\rightarrow& #4\\
#5 &\mapsto& #6
\end{array}
}}

\newcommand{\eqcolon}{\mathrel{\resizebox{\widthof{$\mathord{=}$}}{\height}{ $\!\!=\!\!\resizebox{1.2\width}{0.8\height}{\raisebox{0.23ex}{$\mathop{:}$}}\!\!$ }}}
\newcommand{\coloneq}{\mathrel{\resizebox{\widthof{$\mathord{=}$}}{\height}{ $\!\!\resizebox{1.2\width}{0.8\height}{\raisebox{0.23ex}{$\mathop{:}$}}\!\!=\!\!$ }}}
\newcommand{\eqcolonl}{\ensuremath{\mathrel{=\!\!\mathop{:}}}}
\newcommand{\coloneql}{\ensuremath{\mathrel{\mathop{:} \!\! =}}}
\newcommand{\vc}[1]{% inline column vector
  \left(\begin{smallmatrix}#1\end{smallmatrix}\right)%
}
\newcommand{\vr}[1]{% inline row vector
  \begin{smallmatrix}(\,#1\,)\end{smallmatrix}%
}
\makeatletter
\newcommand*{\defeq}{\ =\mathrel{\rlap{%
                     \raisebox{0.3ex}{$\m@th\cdot$}}%
                     \raisebox{-0.3ex}{$\m@th\cdot$}}%
                     }
\makeatother

\newcommand{\mathcircle}[1]{% inline row vector
 \overset{\circ}{#1}
}
\newcommand{\ulim}{% low limit
 \underline{\lim}
}
\newcommand{\ssi}{% iff
\iff
}
\newcommand{\ps}[2]{
\expval{#1 | #2}
}
\newcommand{\df}[1]{
\mqty{#1}
}
\newcommand{\n}[1]{
\norm{#1}
}
\newcommand{\sys}[1]{
\left\{\smqty{#1}\right.
}


\newcommand{\eqdef}{\ensuremath{\overset{\text{def}}=}}


\def\Circlearrowright{\ensuremath{%
  \rotatebox[origin=c]{230}{$\circlearrowright$}}}

\newcommand\ct[1]{\text{\rmfamily\upshape #1}}
\newcommand\question[1]{ {\color{red} ...!? \small #1}}
\newcommand\caz[1]{\left\{\begin{array} #1 \end{array}\right.}
\newcommand\const{\text{\rmfamily\upshape const}}
\newcommand\toP{ \overset{\pro}{\to}}
\newcommand\toPP{ \overset{\text{PP}}{\to}}
\newcommand{\oeq}{\mathrel{\text{\textcircled{$=$}}}}





\usepackage{xcolor}
% \usepackage[normalem]{ulem}
\usepackage{lipsum}
\makeatletter
% \newcommand\colorwave[1][blue]{\bgroup \markoverwith{\lower3.5\p@\hbox{\sixly \textcolor{#1}{\char58}}}\ULon}
%\font\sixly=lasy6 % does not re-load if already loaded, so no memory problem.

\newmdtheoremenv[
linewidth= 1pt,linecolor= blue,%
leftmargin=20,rightmargin=20,innertopmargin=0pt, innerrightmargin=40,%
tikzsetting = { draw=lightgray, line width = 0.3pt,dashed,%
dash pattern = on 15pt off 3pt},%
splittopskip=\topskip,skipbelow=\baselineskip,%
skipabove=\baselineskip,ntheorem,roundcorner=0pt,
% backgroundcolor=pagebg,font=\color{orange}\sffamily, fontcolor=white
]{examplebox}{Exemple}[section]



\newcommand\R{\mathbb{R}}
\newcommand\Z{\mathbb{Z}}
\newcommand\N{\mathbb{N}}
\newcommand\E{\mathbb{E}}
\newcommand\F{\mathcal{F}}
\newcommand\cH{\mathcal{H}}
\newcommand\V{\mathbb{V}}
\newcommand\dmo{ ^{-1} }
\newcommand\kapa{\kappa}
\newcommand\im{Im}
\newcommand\hs{\mathcal{H}}





\usepackage{soul}

\makeatletter
\newcommand*{\whiten}[1]{\llap{\textcolor{white}{{\the\SOUL@token}}\hspace{#1pt}}}
\DeclareRobustCommand*\myul{%
    \def\SOUL@everyspace{\underline{\space}\kern\z@}%
    \def\SOUL@everytoken{%
     \setbox0=\hbox{\the\SOUL@token}%
     \ifdim\dp0>\z@
        \raisebox{\dp0}{\underline{\phantom{\the\SOUL@token}}}%
        \whiten{1}\whiten{0}%
        \whiten{-1}\whiten{-2}%
        \llap{\the\SOUL@token}%
     \else
        \underline{\the\SOUL@token}%
     \fi}%
\SOUL@}
\makeatother

\newcommand*{\demp}{\fontfamily{lmtt}\selectfont}

\DeclareTextFontCommand{\textdemp}{\demp}

\begin{document}

\ifcomment
Multiline
comment
\fi
\ifcomment
\myul{Typesetting test}
% \color[rgb]{1,1,1}
$∑_i^n≠ 60º±∞π∆¬≈√j∫h≤≥µ$

$\CR \R\pro\ind\pro\gS\pro
\mqty[a&b\\c&d]$
$\pro\mathbb{P}$
$\dd{x}$

  \[
    \alpha(x)=\left\{
                \begin{array}{ll}
                  x\\
                  \frac{1}{1+e^{-kx}}\\
                  \frac{e^x-e^{-x}}{e^x+e^{-x}}
                \end{array}
              \right.
  \]

  $\expval{x}$
  
  $\chi_\rho(ghg\dmo)=\Tr(\rho_{ghg\dmo})=\Tr(\rho_g\circ\rho_h\circ\rho\dmo_g)=\Tr(\rho_h)\overset{\mbox{\scalebox{0.5}{$\Tr(AB)=\Tr(BA)$}}}{=}\chi_\rho(h)$
  	$\mathop{\oplus}_{\substack{x\in X}}$

$\mat(\rho_g)=(a_{ij}(g))_{\scriptsize \substack{1\leq i\leq d \\ 1\leq j\leq d}}$ et $\mat(\rho'_g)=(a'_{ij}(g))_{\scriptsize \substack{1\leq i'\leq d' \\ 1\leq j'\leq d'}}$



\[\int_a^b{\mathbb{R}^2}g(u, v)\dd{P_{XY}}(u, v)=\iint g(u,v) f_{XY}(u, v)\dd \lambda(u) \dd \lambda(v)\]
$$\lim_{x\to\infty} f(x)$$	
$$\iiiint_V \mu(t,u,v,w) \,dt\,du\,dv\,dw$$
$$\sum_{n=1}^{\infty} 2^{-n} = 1$$	
\begin{definition}
	Si $X$ et $Y$ sont 2 v.a. ou definit la \textsc{Covariance} entre $X$ et $Y$ comme
	$\cov(X,Y)\overset{\text{def}}{=}\E\left[(X-\E(X))(Y-\E(Y))\right]=\E(XY)-\E(X)\E(Y)$.
\end{definition}
\fi
\pagebreak

% \tableofcontents

% insert your code here
%\input{./algebra/main.tex}
%\input{./geometrie-differentielle/main.tex}
%\input{./probabilite/main.tex}
%\input{./analyse-fonctionnelle/main.tex}
% \input{./Analyse-convexe-et-dualite-en-optimisation/main.tex}
%\input{./tikz/main.tex}
%\input{./Theorie-du-distributions/main.tex}
%\input{./optimisation/mine.tex}
 \input{./modelisation/main.tex}

% yves.aubry@univ-tln.fr : algebra

\end{document}

%% !TEX encoding = UTF-8 Unicode
% !TEX TS-program = xelatex

\documentclass[french]{report}

%\usepackage[utf8]{inputenc}
%\usepackage[T1]{fontenc}
\usepackage{babel}


\newif\ifcomment
%\commenttrue # Show comments

\usepackage{physics}
\usepackage{amssymb}


\usepackage{amsthm}
% \usepackage{thmtools}
\usepackage{mathtools}
\usepackage{amsfonts}

\usepackage{color}

\usepackage{tikz}

\usepackage{geometry}
\geometry{a5paper, margin=0.1in, right=1cm}

\usepackage{dsfont}

\usepackage{graphicx}
\graphicspath{ {images/} }

\usepackage{faktor}

\usepackage{IEEEtrantools}
\usepackage{enumerate}   
\usepackage[PostScript=dvips]{"/Users/aware/Documents/Courses/diagrams"}


\newtheorem{theorem}{Théorème}[section]
\renewcommand{\thetheorem}{\arabic{theorem}}
\newtheorem{lemme}{Lemme}[section]
\renewcommand{\thelemme}{\arabic{lemme}}
\newtheorem{proposition}{Proposition}[section]
\renewcommand{\theproposition}{\arabic{proposition}}
\newtheorem{notations}{Notations}[section]
\newtheorem{problem}{Problème}[section]
\newtheorem{corollary}{Corollaire}[theorem]
\renewcommand{\thecorollary}{\arabic{corollary}}
\newtheorem{property}{Propriété}[section]
\newtheorem{objective}{Objectif}[section]

\theoremstyle{definition}
\newtheorem{definition}{Définition}[section]
\renewcommand{\thedefinition}{\arabic{definition}}
\newtheorem{exercise}{Exercice}[chapter]
\renewcommand{\theexercise}{\arabic{exercise}}
\newtheorem{example}{Exemple}[chapter]
\renewcommand{\theexample}{\arabic{example}}
\newtheorem*{solution}{Solution}
\newtheorem*{application}{Application}
\newtheorem*{notation}{Notation}
\newtheorem*{vocabulary}{Vocabulaire}
\newtheorem*{properties}{Propriétés}



\theoremstyle{remark}
\newtheorem*{remark}{Remarque}
\newtheorem*{rappel}{Rappel}


\usepackage{etoolbox}
\AtBeginEnvironment{exercise}{\small}
\AtBeginEnvironment{example}{\small}

\usepackage{cases}
\usepackage[red]{mypack}

\usepackage[framemethod=TikZ]{mdframed}

\definecolor{bg}{rgb}{0.4,0.25,0.95}
\definecolor{pagebg}{rgb}{0,0,0.5}
\surroundwithmdframed[
   topline=false,
   rightline=false,
   bottomline=false,
   leftmargin=\parindent,
   skipabove=8pt,
   skipbelow=8pt,
   linecolor=blue,
   innerbottommargin=10pt,
   % backgroundcolor=bg,font=\color{orange}\sffamily, fontcolor=white
]{definition}

\usepackage{empheq}
\usepackage[most]{tcolorbox}

\newtcbox{\mymath}[1][]{%
    nobeforeafter, math upper, tcbox raise base,
    enhanced, colframe=blue!30!black,
    colback=red!10, boxrule=1pt,
    #1}

\usepackage{unixode}


\DeclareMathOperator{\ord}{ord}
\DeclareMathOperator{\orb}{orb}
\DeclareMathOperator{\stab}{stab}
\DeclareMathOperator{\Stab}{stab}
\DeclareMathOperator{\ppcm}{ppcm}
\DeclareMathOperator{\conj}{Conj}
\DeclareMathOperator{\End}{End}
\DeclareMathOperator{\rot}{rot}
\DeclareMathOperator{\trs}{trace}
\DeclareMathOperator{\Ind}{Ind}
\DeclareMathOperator{\mat}{Mat}
\DeclareMathOperator{\id}{Id}
\DeclareMathOperator{\vect}{vect}
\DeclareMathOperator{\img}{img}
\DeclareMathOperator{\cov}{Cov}
\DeclareMathOperator{\dist}{dist}
\DeclareMathOperator{\irr}{Irr}
\DeclareMathOperator{\image}{Im}
\DeclareMathOperator{\pd}{\partial}
\DeclareMathOperator{\epi}{epi}
\DeclareMathOperator{\Argmin}{Argmin}
\DeclareMathOperator{\dom}{dom}
\DeclareMathOperator{\proj}{proj}
\DeclareMathOperator{\ctg}{ctg}
\DeclareMathOperator{\supp}{supp}
\DeclareMathOperator{\argmin}{argmin}
\DeclareMathOperator{\mult}{mult}
\DeclareMathOperator{\ch}{ch}
\DeclareMathOperator{\sh}{sh}
\DeclareMathOperator{\rang}{rang}
\DeclareMathOperator{\diam}{diam}
\DeclareMathOperator{\Epigraphe}{Epigraphe}




\usepackage{xcolor}
\everymath{\color{blue}}
%\everymath{\color[rgb]{0,1,1}}
%\pagecolor[rgb]{0,0,0.5}


\newcommand*{\pdtest}[3][]{\ensuremath{\frac{\partial^{#1} #2}{\partial #3}}}

\newcommand*{\deffunc}[6][]{\ensuremath{
\begin{array}{rcl}
#2 : #3 &\rightarrow& #4\\
#5 &\mapsto& #6
\end{array}
}}

\newcommand{\eqcolon}{\mathrel{\resizebox{\widthof{$\mathord{=}$}}{\height}{ $\!\!=\!\!\resizebox{1.2\width}{0.8\height}{\raisebox{0.23ex}{$\mathop{:}$}}\!\!$ }}}
\newcommand{\coloneq}{\mathrel{\resizebox{\widthof{$\mathord{=}$}}{\height}{ $\!\!\resizebox{1.2\width}{0.8\height}{\raisebox{0.23ex}{$\mathop{:}$}}\!\!=\!\!$ }}}
\newcommand{\eqcolonl}{\ensuremath{\mathrel{=\!\!\mathop{:}}}}
\newcommand{\coloneql}{\ensuremath{\mathrel{\mathop{:} \!\! =}}}
\newcommand{\vc}[1]{% inline column vector
  \left(\begin{smallmatrix}#1\end{smallmatrix}\right)%
}
\newcommand{\vr}[1]{% inline row vector
  \begin{smallmatrix}(\,#1\,)\end{smallmatrix}%
}
\makeatletter
\newcommand*{\defeq}{\ =\mathrel{\rlap{%
                     \raisebox{0.3ex}{$\m@th\cdot$}}%
                     \raisebox{-0.3ex}{$\m@th\cdot$}}%
                     }
\makeatother

\newcommand{\mathcircle}[1]{% inline row vector
 \overset{\circ}{#1}
}
\newcommand{\ulim}{% low limit
 \underline{\lim}
}
\newcommand{\ssi}{% iff
\iff
}
\newcommand{\ps}[2]{
\expval{#1 | #2}
}
\newcommand{\df}[1]{
\mqty{#1}
}
\newcommand{\n}[1]{
\norm{#1}
}
\newcommand{\sys}[1]{
\left\{\smqty{#1}\right.
}


\newcommand{\eqdef}{\ensuremath{\overset{\text{def}}=}}


\def\Circlearrowright{\ensuremath{%
  \rotatebox[origin=c]{230}{$\circlearrowright$}}}

\newcommand\ct[1]{\text{\rmfamily\upshape #1}}
\newcommand\question[1]{ {\color{red} ...!? \small #1}}
\newcommand\caz[1]{\left\{\begin{array} #1 \end{array}\right.}
\newcommand\const{\text{\rmfamily\upshape const}}
\newcommand\toP{ \overset{\pro}{\to}}
\newcommand\toPP{ \overset{\text{PP}}{\to}}
\newcommand{\oeq}{\mathrel{\text{\textcircled{$=$}}}}





\usepackage{xcolor}
% \usepackage[normalem]{ulem}
\usepackage{lipsum}
\makeatletter
% \newcommand\colorwave[1][blue]{\bgroup \markoverwith{\lower3.5\p@\hbox{\sixly \textcolor{#1}{\char58}}}\ULon}
%\font\sixly=lasy6 % does not re-load if already loaded, so no memory problem.

\newmdtheoremenv[
linewidth= 1pt,linecolor= blue,%
leftmargin=20,rightmargin=20,innertopmargin=0pt, innerrightmargin=40,%
tikzsetting = { draw=lightgray, line width = 0.3pt,dashed,%
dash pattern = on 15pt off 3pt},%
splittopskip=\topskip,skipbelow=\baselineskip,%
skipabove=\baselineskip,ntheorem,roundcorner=0pt,
% backgroundcolor=pagebg,font=\color{orange}\sffamily, fontcolor=white
]{examplebox}{Exemple}[section]



\newcommand\R{\mathbb{R}}
\newcommand\Z{\mathbb{Z}}
\newcommand\N{\mathbb{N}}
\newcommand\E{\mathbb{E}}
\newcommand\F{\mathcal{F}}
\newcommand\cH{\mathcal{H}}
\newcommand\V{\mathbb{V}}
\newcommand\dmo{ ^{-1} }
\newcommand\kapa{\kappa}
\newcommand\im{Im}
\newcommand\hs{\mathcal{H}}





\usepackage{soul}

\makeatletter
\newcommand*{\whiten}[1]{\llap{\textcolor{white}{{\the\SOUL@token}}\hspace{#1pt}}}
\DeclareRobustCommand*\myul{%
    \def\SOUL@everyspace{\underline{\space}\kern\z@}%
    \def\SOUL@everytoken{%
     \setbox0=\hbox{\the\SOUL@token}%
     \ifdim\dp0>\z@
        \raisebox{\dp0}{\underline{\phantom{\the\SOUL@token}}}%
        \whiten{1}\whiten{0}%
        \whiten{-1}\whiten{-2}%
        \llap{\the\SOUL@token}%
     \else
        \underline{\the\SOUL@token}%
     \fi}%
\SOUL@}
\makeatother

\newcommand*{\demp}{\fontfamily{lmtt}\selectfont}

\DeclareTextFontCommand{\textdemp}{\demp}

\begin{document}

\ifcomment
Multiline
comment
\fi
\ifcomment
\myul{Typesetting test}
% \color[rgb]{1,1,1}
$∑_i^n≠ 60º±∞π∆¬≈√j∫h≤≥µ$

$\CR \R\pro\ind\pro\gS\pro
\mqty[a&b\\c&d]$
$\pro\mathbb{P}$
$\dd{x}$

  \[
    \alpha(x)=\left\{
                \begin{array}{ll}
                  x\\
                  \frac{1}{1+e^{-kx}}\\
                  \frac{e^x-e^{-x}}{e^x+e^{-x}}
                \end{array}
              \right.
  \]

  $\expval{x}$
  
  $\chi_\rho(ghg\dmo)=\Tr(\rho_{ghg\dmo})=\Tr(\rho_g\circ\rho_h\circ\rho\dmo_g)=\Tr(\rho_h)\overset{\mbox{\scalebox{0.5}{$\Tr(AB)=\Tr(BA)$}}}{=}\chi_\rho(h)$
  	$\mathop{\oplus}_{\substack{x\in X}}$

$\mat(\rho_g)=(a_{ij}(g))_{\scriptsize \substack{1\leq i\leq d \\ 1\leq j\leq d}}$ et $\mat(\rho'_g)=(a'_{ij}(g))_{\scriptsize \substack{1\leq i'\leq d' \\ 1\leq j'\leq d'}}$



\[\int_a^b{\mathbb{R}^2}g(u, v)\dd{P_{XY}}(u, v)=\iint g(u,v) f_{XY}(u, v)\dd \lambda(u) \dd \lambda(v)\]
$$\lim_{x\to\infty} f(x)$$	
$$\iiiint_V \mu(t,u,v,w) \,dt\,du\,dv\,dw$$
$$\sum_{n=1}^{\infty} 2^{-n} = 1$$	
\begin{definition}
	Si $X$ et $Y$ sont 2 v.a. ou definit la \textsc{Covariance} entre $X$ et $Y$ comme
	$\cov(X,Y)\overset{\text{def}}{=}\E\left[(X-\E(X))(Y-\E(Y))\right]=\E(XY)-\E(X)\E(Y)$.
\end{definition}
\fi
\pagebreak

% \tableofcontents

% insert your code here
%\input{./algebra/main.tex}
%\input{./geometrie-differentielle/main.tex}
%\input{./probabilite/main.tex}
%\input{./analyse-fonctionnelle/main.tex}
% \input{./Analyse-convexe-et-dualite-en-optimisation/main.tex}
%\input{./tikz/main.tex}
%\input{./Theorie-du-distributions/main.tex}
%\input{./optimisation/mine.tex}
 \input{./modelisation/main.tex}

% yves.aubry@univ-tln.fr : algebra

\end{document}

%% !TEX encoding = UTF-8 Unicode
% !TEX TS-program = xelatex

\documentclass[french]{report}

%\usepackage[utf8]{inputenc}
%\usepackage[T1]{fontenc}
\usepackage{babel}


\newif\ifcomment
%\commenttrue # Show comments

\usepackage{physics}
\usepackage{amssymb}


\usepackage{amsthm}
% \usepackage{thmtools}
\usepackage{mathtools}
\usepackage{amsfonts}

\usepackage{color}

\usepackage{tikz}

\usepackage{geometry}
\geometry{a5paper, margin=0.1in, right=1cm}

\usepackage{dsfont}

\usepackage{graphicx}
\graphicspath{ {images/} }

\usepackage{faktor}

\usepackage{IEEEtrantools}
\usepackage{enumerate}   
\usepackage[PostScript=dvips]{"/Users/aware/Documents/Courses/diagrams"}


\newtheorem{theorem}{Théorème}[section]
\renewcommand{\thetheorem}{\arabic{theorem}}
\newtheorem{lemme}{Lemme}[section]
\renewcommand{\thelemme}{\arabic{lemme}}
\newtheorem{proposition}{Proposition}[section]
\renewcommand{\theproposition}{\arabic{proposition}}
\newtheorem{notations}{Notations}[section]
\newtheorem{problem}{Problème}[section]
\newtheorem{corollary}{Corollaire}[theorem]
\renewcommand{\thecorollary}{\arabic{corollary}}
\newtheorem{property}{Propriété}[section]
\newtheorem{objective}{Objectif}[section]

\theoremstyle{definition}
\newtheorem{definition}{Définition}[section]
\renewcommand{\thedefinition}{\arabic{definition}}
\newtheorem{exercise}{Exercice}[chapter]
\renewcommand{\theexercise}{\arabic{exercise}}
\newtheorem{example}{Exemple}[chapter]
\renewcommand{\theexample}{\arabic{example}}
\newtheorem*{solution}{Solution}
\newtheorem*{application}{Application}
\newtheorem*{notation}{Notation}
\newtheorem*{vocabulary}{Vocabulaire}
\newtheorem*{properties}{Propriétés}



\theoremstyle{remark}
\newtheorem*{remark}{Remarque}
\newtheorem*{rappel}{Rappel}


\usepackage{etoolbox}
\AtBeginEnvironment{exercise}{\small}
\AtBeginEnvironment{example}{\small}

\usepackage{cases}
\usepackage[red]{mypack}

\usepackage[framemethod=TikZ]{mdframed}

\definecolor{bg}{rgb}{0.4,0.25,0.95}
\definecolor{pagebg}{rgb}{0,0,0.5}
\surroundwithmdframed[
   topline=false,
   rightline=false,
   bottomline=false,
   leftmargin=\parindent,
   skipabove=8pt,
   skipbelow=8pt,
   linecolor=blue,
   innerbottommargin=10pt,
   % backgroundcolor=bg,font=\color{orange}\sffamily, fontcolor=white
]{definition}

\usepackage{empheq}
\usepackage[most]{tcolorbox}

\newtcbox{\mymath}[1][]{%
    nobeforeafter, math upper, tcbox raise base,
    enhanced, colframe=blue!30!black,
    colback=red!10, boxrule=1pt,
    #1}

\usepackage{unixode}


\DeclareMathOperator{\ord}{ord}
\DeclareMathOperator{\orb}{orb}
\DeclareMathOperator{\stab}{stab}
\DeclareMathOperator{\Stab}{stab}
\DeclareMathOperator{\ppcm}{ppcm}
\DeclareMathOperator{\conj}{Conj}
\DeclareMathOperator{\End}{End}
\DeclareMathOperator{\rot}{rot}
\DeclareMathOperator{\trs}{trace}
\DeclareMathOperator{\Ind}{Ind}
\DeclareMathOperator{\mat}{Mat}
\DeclareMathOperator{\id}{Id}
\DeclareMathOperator{\vect}{vect}
\DeclareMathOperator{\img}{img}
\DeclareMathOperator{\cov}{Cov}
\DeclareMathOperator{\dist}{dist}
\DeclareMathOperator{\irr}{Irr}
\DeclareMathOperator{\image}{Im}
\DeclareMathOperator{\pd}{\partial}
\DeclareMathOperator{\epi}{epi}
\DeclareMathOperator{\Argmin}{Argmin}
\DeclareMathOperator{\dom}{dom}
\DeclareMathOperator{\proj}{proj}
\DeclareMathOperator{\ctg}{ctg}
\DeclareMathOperator{\supp}{supp}
\DeclareMathOperator{\argmin}{argmin}
\DeclareMathOperator{\mult}{mult}
\DeclareMathOperator{\ch}{ch}
\DeclareMathOperator{\sh}{sh}
\DeclareMathOperator{\rang}{rang}
\DeclareMathOperator{\diam}{diam}
\DeclareMathOperator{\Epigraphe}{Epigraphe}




\usepackage{xcolor}
\everymath{\color{blue}}
%\everymath{\color[rgb]{0,1,1}}
%\pagecolor[rgb]{0,0,0.5}


\newcommand*{\pdtest}[3][]{\ensuremath{\frac{\partial^{#1} #2}{\partial #3}}}

\newcommand*{\deffunc}[6][]{\ensuremath{
\begin{array}{rcl}
#2 : #3 &\rightarrow& #4\\
#5 &\mapsto& #6
\end{array}
}}

\newcommand{\eqcolon}{\mathrel{\resizebox{\widthof{$\mathord{=}$}}{\height}{ $\!\!=\!\!\resizebox{1.2\width}{0.8\height}{\raisebox{0.23ex}{$\mathop{:}$}}\!\!$ }}}
\newcommand{\coloneq}{\mathrel{\resizebox{\widthof{$\mathord{=}$}}{\height}{ $\!\!\resizebox{1.2\width}{0.8\height}{\raisebox{0.23ex}{$\mathop{:}$}}\!\!=\!\!$ }}}
\newcommand{\eqcolonl}{\ensuremath{\mathrel{=\!\!\mathop{:}}}}
\newcommand{\coloneql}{\ensuremath{\mathrel{\mathop{:} \!\! =}}}
\newcommand{\vc}[1]{% inline column vector
  \left(\begin{smallmatrix}#1\end{smallmatrix}\right)%
}
\newcommand{\vr}[1]{% inline row vector
  \begin{smallmatrix}(\,#1\,)\end{smallmatrix}%
}
\makeatletter
\newcommand*{\defeq}{\ =\mathrel{\rlap{%
                     \raisebox{0.3ex}{$\m@th\cdot$}}%
                     \raisebox{-0.3ex}{$\m@th\cdot$}}%
                     }
\makeatother

\newcommand{\mathcircle}[1]{% inline row vector
 \overset{\circ}{#1}
}
\newcommand{\ulim}{% low limit
 \underline{\lim}
}
\newcommand{\ssi}{% iff
\iff
}
\newcommand{\ps}[2]{
\expval{#1 | #2}
}
\newcommand{\df}[1]{
\mqty{#1}
}
\newcommand{\n}[1]{
\norm{#1}
}
\newcommand{\sys}[1]{
\left\{\smqty{#1}\right.
}


\newcommand{\eqdef}{\ensuremath{\overset{\text{def}}=}}


\def\Circlearrowright{\ensuremath{%
  \rotatebox[origin=c]{230}{$\circlearrowright$}}}

\newcommand\ct[1]{\text{\rmfamily\upshape #1}}
\newcommand\question[1]{ {\color{red} ...!? \small #1}}
\newcommand\caz[1]{\left\{\begin{array} #1 \end{array}\right.}
\newcommand\const{\text{\rmfamily\upshape const}}
\newcommand\toP{ \overset{\pro}{\to}}
\newcommand\toPP{ \overset{\text{PP}}{\to}}
\newcommand{\oeq}{\mathrel{\text{\textcircled{$=$}}}}





\usepackage{xcolor}
% \usepackage[normalem]{ulem}
\usepackage{lipsum}
\makeatletter
% \newcommand\colorwave[1][blue]{\bgroup \markoverwith{\lower3.5\p@\hbox{\sixly \textcolor{#1}{\char58}}}\ULon}
%\font\sixly=lasy6 % does not re-load if already loaded, so no memory problem.

\newmdtheoremenv[
linewidth= 1pt,linecolor= blue,%
leftmargin=20,rightmargin=20,innertopmargin=0pt, innerrightmargin=40,%
tikzsetting = { draw=lightgray, line width = 0.3pt,dashed,%
dash pattern = on 15pt off 3pt},%
splittopskip=\topskip,skipbelow=\baselineskip,%
skipabove=\baselineskip,ntheorem,roundcorner=0pt,
% backgroundcolor=pagebg,font=\color{orange}\sffamily, fontcolor=white
]{examplebox}{Exemple}[section]



\newcommand\R{\mathbb{R}}
\newcommand\Z{\mathbb{Z}}
\newcommand\N{\mathbb{N}}
\newcommand\E{\mathbb{E}}
\newcommand\F{\mathcal{F}}
\newcommand\cH{\mathcal{H}}
\newcommand\V{\mathbb{V}}
\newcommand\dmo{ ^{-1} }
\newcommand\kapa{\kappa}
\newcommand\im{Im}
\newcommand\hs{\mathcal{H}}





\usepackage{soul}

\makeatletter
\newcommand*{\whiten}[1]{\llap{\textcolor{white}{{\the\SOUL@token}}\hspace{#1pt}}}
\DeclareRobustCommand*\myul{%
    \def\SOUL@everyspace{\underline{\space}\kern\z@}%
    \def\SOUL@everytoken{%
     \setbox0=\hbox{\the\SOUL@token}%
     \ifdim\dp0>\z@
        \raisebox{\dp0}{\underline{\phantom{\the\SOUL@token}}}%
        \whiten{1}\whiten{0}%
        \whiten{-1}\whiten{-2}%
        \llap{\the\SOUL@token}%
     \else
        \underline{\the\SOUL@token}%
     \fi}%
\SOUL@}
\makeatother

\newcommand*{\demp}{\fontfamily{lmtt}\selectfont}

\DeclareTextFontCommand{\textdemp}{\demp}

\begin{document}

\ifcomment
Multiline
comment
\fi
\ifcomment
\myul{Typesetting test}
% \color[rgb]{1,1,1}
$∑_i^n≠ 60º±∞π∆¬≈√j∫h≤≥µ$

$\CR \R\pro\ind\pro\gS\pro
\mqty[a&b\\c&d]$
$\pro\mathbb{P}$
$\dd{x}$

  \[
    \alpha(x)=\left\{
                \begin{array}{ll}
                  x\\
                  \frac{1}{1+e^{-kx}}\\
                  \frac{e^x-e^{-x}}{e^x+e^{-x}}
                \end{array}
              \right.
  \]

  $\expval{x}$
  
  $\chi_\rho(ghg\dmo)=\Tr(\rho_{ghg\dmo})=\Tr(\rho_g\circ\rho_h\circ\rho\dmo_g)=\Tr(\rho_h)\overset{\mbox{\scalebox{0.5}{$\Tr(AB)=\Tr(BA)$}}}{=}\chi_\rho(h)$
  	$\mathop{\oplus}_{\substack{x\in X}}$

$\mat(\rho_g)=(a_{ij}(g))_{\scriptsize \substack{1\leq i\leq d \\ 1\leq j\leq d}}$ et $\mat(\rho'_g)=(a'_{ij}(g))_{\scriptsize \substack{1\leq i'\leq d' \\ 1\leq j'\leq d'}}$



\[\int_a^b{\mathbb{R}^2}g(u, v)\dd{P_{XY}}(u, v)=\iint g(u,v) f_{XY}(u, v)\dd \lambda(u) \dd \lambda(v)\]
$$\lim_{x\to\infty} f(x)$$	
$$\iiiint_V \mu(t,u,v,w) \,dt\,du\,dv\,dw$$
$$\sum_{n=1}^{\infty} 2^{-n} = 1$$	
\begin{definition}
	Si $X$ et $Y$ sont 2 v.a. ou definit la \textsc{Covariance} entre $X$ et $Y$ comme
	$\cov(X,Y)\overset{\text{def}}{=}\E\left[(X-\E(X))(Y-\E(Y))\right]=\E(XY)-\E(X)\E(Y)$.
\end{definition}
\fi
\pagebreak

% \tableofcontents

% insert your code here
%\input{./algebra/main.tex}
%\input{./geometrie-differentielle/main.tex}
%\input{./probabilite/main.tex}
%\input{./analyse-fonctionnelle/main.tex}
% \input{./Analyse-convexe-et-dualite-en-optimisation/main.tex}
%\input{./tikz/main.tex}
%\input{./Theorie-du-distributions/main.tex}
%\input{./optimisation/mine.tex}
 \input{./modelisation/main.tex}

% yves.aubry@univ-tln.fr : algebra

\end{document}

% % !TEX encoding = UTF-8 Unicode
% !TEX TS-program = xelatex

\documentclass[french]{report}

%\usepackage[utf8]{inputenc}
%\usepackage[T1]{fontenc}
\usepackage{babel}


\newif\ifcomment
%\commenttrue # Show comments

\usepackage{physics}
\usepackage{amssymb}


\usepackage{amsthm}
% \usepackage{thmtools}
\usepackage{mathtools}
\usepackage{amsfonts}

\usepackage{color}

\usepackage{tikz}

\usepackage{geometry}
\geometry{a5paper, margin=0.1in, right=1cm}

\usepackage{dsfont}

\usepackage{graphicx}
\graphicspath{ {images/} }

\usepackage{faktor}

\usepackage{IEEEtrantools}
\usepackage{enumerate}   
\usepackage[PostScript=dvips]{"/Users/aware/Documents/Courses/diagrams"}


\newtheorem{theorem}{Théorème}[section]
\renewcommand{\thetheorem}{\arabic{theorem}}
\newtheorem{lemme}{Lemme}[section]
\renewcommand{\thelemme}{\arabic{lemme}}
\newtheorem{proposition}{Proposition}[section]
\renewcommand{\theproposition}{\arabic{proposition}}
\newtheorem{notations}{Notations}[section]
\newtheorem{problem}{Problème}[section]
\newtheorem{corollary}{Corollaire}[theorem]
\renewcommand{\thecorollary}{\arabic{corollary}}
\newtheorem{property}{Propriété}[section]
\newtheorem{objective}{Objectif}[section]

\theoremstyle{definition}
\newtheorem{definition}{Définition}[section]
\renewcommand{\thedefinition}{\arabic{definition}}
\newtheorem{exercise}{Exercice}[chapter]
\renewcommand{\theexercise}{\arabic{exercise}}
\newtheorem{example}{Exemple}[chapter]
\renewcommand{\theexample}{\arabic{example}}
\newtheorem*{solution}{Solution}
\newtheorem*{application}{Application}
\newtheorem*{notation}{Notation}
\newtheorem*{vocabulary}{Vocabulaire}
\newtheorem*{properties}{Propriétés}



\theoremstyle{remark}
\newtheorem*{remark}{Remarque}
\newtheorem*{rappel}{Rappel}


\usepackage{etoolbox}
\AtBeginEnvironment{exercise}{\small}
\AtBeginEnvironment{example}{\small}

\usepackage{cases}
\usepackage[red]{mypack}

\usepackage[framemethod=TikZ]{mdframed}

\definecolor{bg}{rgb}{0.4,0.25,0.95}
\definecolor{pagebg}{rgb}{0,0,0.5}
\surroundwithmdframed[
   topline=false,
   rightline=false,
   bottomline=false,
   leftmargin=\parindent,
   skipabove=8pt,
   skipbelow=8pt,
   linecolor=blue,
   innerbottommargin=10pt,
   % backgroundcolor=bg,font=\color{orange}\sffamily, fontcolor=white
]{definition}

\usepackage{empheq}
\usepackage[most]{tcolorbox}

\newtcbox{\mymath}[1][]{%
    nobeforeafter, math upper, tcbox raise base,
    enhanced, colframe=blue!30!black,
    colback=red!10, boxrule=1pt,
    #1}

\usepackage{unixode}


\DeclareMathOperator{\ord}{ord}
\DeclareMathOperator{\orb}{orb}
\DeclareMathOperator{\stab}{stab}
\DeclareMathOperator{\Stab}{stab}
\DeclareMathOperator{\ppcm}{ppcm}
\DeclareMathOperator{\conj}{Conj}
\DeclareMathOperator{\End}{End}
\DeclareMathOperator{\rot}{rot}
\DeclareMathOperator{\trs}{trace}
\DeclareMathOperator{\Ind}{Ind}
\DeclareMathOperator{\mat}{Mat}
\DeclareMathOperator{\id}{Id}
\DeclareMathOperator{\vect}{vect}
\DeclareMathOperator{\img}{img}
\DeclareMathOperator{\cov}{Cov}
\DeclareMathOperator{\dist}{dist}
\DeclareMathOperator{\irr}{Irr}
\DeclareMathOperator{\image}{Im}
\DeclareMathOperator{\pd}{\partial}
\DeclareMathOperator{\epi}{epi}
\DeclareMathOperator{\Argmin}{Argmin}
\DeclareMathOperator{\dom}{dom}
\DeclareMathOperator{\proj}{proj}
\DeclareMathOperator{\ctg}{ctg}
\DeclareMathOperator{\supp}{supp}
\DeclareMathOperator{\argmin}{argmin}
\DeclareMathOperator{\mult}{mult}
\DeclareMathOperator{\ch}{ch}
\DeclareMathOperator{\sh}{sh}
\DeclareMathOperator{\rang}{rang}
\DeclareMathOperator{\diam}{diam}
\DeclareMathOperator{\Epigraphe}{Epigraphe}




\usepackage{xcolor}
\everymath{\color{blue}}
%\everymath{\color[rgb]{0,1,1}}
%\pagecolor[rgb]{0,0,0.5}


\newcommand*{\pdtest}[3][]{\ensuremath{\frac{\partial^{#1} #2}{\partial #3}}}

\newcommand*{\deffunc}[6][]{\ensuremath{
\begin{array}{rcl}
#2 : #3 &\rightarrow& #4\\
#5 &\mapsto& #6
\end{array}
}}

\newcommand{\eqcolon}{\mathrel{\resizebox{\widthof{$\mathord{=}$}}{\height}{ $\!\!=\!\!\resizebox{1.2\width}{0.8\height}{\raisebox{0.23ex}{$\mathop{:}$}}\!\!$ }}}
\newcommand{\coloneq}{\mathrel{\resizebox{\widthof{$\mathord{=}$}}{\height}{ $\!\!\resizebox{1.2\width}{0.8\height}{\raisebox{0.23ex}{$\mathop{:}$}}\!\!=\!\!$ }}}
\newcommand{\eqcolonl}{\ensuremath{\mathrel{=\!\!\mathop{:}}}}
\newcommand{\coloneql}{\ensuremath{\mathrel{\mathop{:} \!\! =}}}
\newcommand{\vc}[1]{% inline column vector
  \left(\begin{smallmatrix}#1\end{smallmatrix}\right)%
}
\newcommand{\vr}[1]{% inline row vector
  \begin{smallmatrix}(\,#1\,)\end{smallmatrix}%
}
\makeatletter
\newcommand*{\defeq}{\ =\mathrel{\rlap{%
                     \raisebox{0.3ex}{$\m@th\cdot$}}%
                     \raisebox{-0.3ex}{$\m@th\cdot$}}%
                     }
\makeatother

\newcommand{\mathcircle}[1]{% inline row vector
 \overset{\circ}{#1}
}
\newcommand{\ulim}{% low limit
 \underline{\lim}
}
\newcommand{\ssi}{% iff
\iff
}
\newcommand{\ps}[2]{
\expval{#1 | #2}
}
\newcommand{\df}[1]{
\mqty{#1}
}
\newcommand{\n}[1]{
\norm{#1}
}
\newcommand{\sys}[1]{
\left\{\smqty{#1}\right.
}


\newcommand{\eqdef}{\ensuremath{\overset{\text{def}}=}}


\def\Circlearrowright{\ensuremath{%
  \rotatebox[origin=c]{230}{$\circlearrowright$}}}

\newcommand\ct[1]{\text{\rmfamily\upshape #1}}
\newcommand\question[1]{ {\color{red} ...!? \small #1}}
\newcommand\caz[1]{\left\{\begin{array} #1 \end{array}\right.}
\newcommand\const{\text{\rmfamily\upshape const}}
\newcommand\toP{ \overset{\pro}{\to}}
\newcommand\toPP{ \overset{\text{PP}}{\to}}
\newcommand{\oeq}{\mathrel{\text{\textcircled{$=$}}}}





\usepackage{xcolor}
% \usepackage[normalem]{ulem}
\usepackage{lipsum}
\makeatletter
% \newcommand\colorwave[1][blue]{\bgroup \markoverwith{\lower3.5\p@\hbox{\sixly \textcolor{#1}{\char58}}}\ULon}
%\font\sixly=lasy6 % does not re-load if already loaded, so no memory problem.

\newmdtheoremenv[
linewidth= 1pt,linecolor= blue,%
leftmargin=20,rightmargin=20,innertopmargin=0pt, innerrightmargin=40,%
tikzsetting = { draw=lightgray, line width = 0.3pt,dashed,%
dash pattern = on 15pt off 3pt},%
splittopskip=\topskip,skipbelow=\baselineskip,%
skipabove=\baselineskip,ntheorem,roundcorner=0pt,
% backgroundcolor=pagebg,font=\color{orange}\sffamily, fontcolor=white
]{examplebox}{Exemple}[section]



\newcommand\R{\mathbb{R}}
\newcommand\Z{\mathbb{Z}}
\newcommand\N{\mathbb{N}}
\newcommand\E{\mathbb{E}}
\newcommand\F{\mathcal{F}}
\newcommand\cH{\mathcal{H}}
\newcommand\V{\mathbb{V}}
\newcommand\dmo{ ^{-1} }
\newcommand\kapa{\kappa}
\newcommand\im{Im}
\newcommand\hs{\mathcal{H}}





\usepackage{soul}

\makeatletter
\newcommand*{\whiten}[1]{\llap{\textcolor{white}{{\the\SOUL@token}}\hspace{#1pt}}}
\DeclareRobustCommand*\myul{%
    \def\SOUL@everyspace{\underline{\space}\kern\z@}%
    \def\SOUL@everytoken{%
     \setbox0=\hbox{\the\SOUL@token}%
     \ifdim\dp0>\z@
        \raisebox{\dp0}{\underline{\phantom{\the\SOUL@token}}}%
        \whiten{1}\whiten{0}%
        \whiten{-1}\whiten{-2}%
        \llap{\the\SOUL@token}%
     \else
        \underline{\the\SOUL@token}%
     \fi}%
\SOUL@}
\makeatother

\newcommand*{\demp}{\fontfamily{lmtt}\selectfont}

\DeclareTextFontCommand{\textdemp}{\demp}

\begin{document}

\ifcomment
Multiline
comment
\fi
\ifcomment
\myul{Typesetting test}
% \color[rgb]{1,1,1}
$∑_i^n≠ 60º±∞π∆¬≈√j∫h≤≥µ$

$\CR \R\pro\ind\pro\gS\pro
\mqty[a&b\\c&d]$
$\pro\mathbb{P}$
$\dd{x}$

  \[
    \alpha(x)=\left\{
                \begin{array}{ll}
                  x\\
                  \frac{1}{1+e^{-kx}}\\
                  \frac{e^x-e^{-x}}{e^x+e^{-x}}
                \end{array}
              \right.
  \]

  $\expval{x}$
  
  $\chi_\rho(ghg\dmo)=\Tr(\rho_{ghg\dmo})=\Tr(\rho_g\circ\rho_h\circ\rho\dmo_g)=\Tr(\rho_h)\overset{\mbox{\scalebox{0.5}{$\Tr(AB)=\Tr(BA)$}}}{=}\chi_\rho(h)$
  	$\mathop{\oplus}_{\substack{x\in X}}$

$\mat(\rho_g)=(a_{ij}(g))_{\scriptsize \substack{1\leq i\leq d \\ 1\leq j\leq d}}$ et $\mat(\rho'_g)=(a'_{ij}(g))_{\scriptsize \substack{1\leq i'\leq d' \\ 1\leq j'\leq d'}}$



\[\int_a^b{\mathbb{R}^2}g(u, v)\dd{P_{XY}}(u, v)=\iint g(u,v) f_{XY}(u, v)\dd \lambda(u) \dd \lambda(v)\]
$$\lim_{x\to\infty} f(x)$$	
$$\iiiint_V \mu(t,u,v,w) \,dt\,du\,dv\,dw$$
$$\sum_{n=1}^{\infty} 2^{-n} = 1$$	
\begin{definition}
	Si $X$ et $Y$ sont 2 v.a. ou definit la \textsc{Covariance} entre $X$ et $Y$ comme
	$\cov(X,Y)\overset{\text{def}}{=}\E\left[(X-\E(X))(Y-\E(Y))\right]=\E(XY)-\E(X)\E(Y)$.
\end{definition}
\fi
\pagebreak

% \tableofcontents

% insert your code here
%\input{./algebra/main.tex}
%\input{./geometrie-differentielle/main.tex}
%\input{./probabilite/main.tex}
%\input{./analyse-fonctionnelle/main.tex}
% \input{./Analyse-convexe-et-dualite-en-optimisation/main.tex}
%\input{./tikz/main.tex}
%\input{./Theorie-du-distributions/main.tex}
%\input{./optimisation/mine.tex}
 \input{./modelisation/main.tex}

% yves.aubry@univ-tln.fr : algebra

\end{document}

%% !TEX encoding = UTF-8 Unicode
% !TEX TS-program = xelatex

\documentclass[french]{report}

%\usepackage[utf8]{inputenc}
%\usepackage[T1]{fontenc}
\usepackage{babel}


\newif\ifcomment
%\commenttrue # Show comments

\usepackage{physics}
\usepackage{amssymb}


\usepackage{amsthm}
% \usepackage{thmtools}
\usepackage{mathtools}
\usepackage{amsfonts}

\usepackage{color}

\usepackage{tikz}

\usepackage{geometry}
\geometry{a5paper, margin=0.1in, right=1cm}

\usepackage{dsfont}

\usepackage{graphicx}
\graphicspath{ {images/} }

\usepackage{faktor}

\usepackage{IEEEtrantools}
\usepackage{enumerate}   
\usepackage[PostScript=dvips]{"/Users/aware/Documents/Courses/diagrams"}


\newtheorem{theorem}{Théorème}[section]
\renewcommand{\thetheorem}{\arabic{theorem}}
\newtheorem{lemme}{Lemme}[section]
\renewcommand{\thelemme}{\arabic{lemme}}
\newtheorem{proposition}{Proposition}[section]
\renewcommand{\theproposition}{\arabic{proposition}}
\newtheorem{notations}{Notations}[section]
\newtheorem{problem}{Problème}[section]
\newtheorem{corollary}{Corollaire}[theorem]
\renewcommand{\thecorollary}{\arabic{corollary}}
\newtheorem{property}{Propriété}[section]
\newtheorem{objective}{Objectif}[section]

\theoremstyle{definition}
\newtheorem{definition}{Définition}[section]
\renewcommand{\thedefinition}{\arabic{definition}}
\newtheorem{exercise}{Exercice}[chapter]
\renewcommand{\theexercise}{\arabic{exercise}}
\newtheorem{example}{Exemple}[chapter]
\renewcommand{\theexample}{\arabic{example}}
\newtheorem*{solution}{Solution}
\newtheorem*{application}{Application}
\newtheorem*{notation}{Notation}
\newtheorem*{vocabulary}{Vocabulaire}
\newtheorem*{properties}{Propriétés}



\theoremstyle{remark}
\newtheorem*{remark}{Remarque}
\newtheorem*{rappel}{Rappel}


\usepackage{etoolbox}
\AtBeginEnvironment{exercise}{\small}
\AtBeginEnvironment{example}{\small}

\usepackage{cases}
\usepackage[red]{mypack}

\usepackage[framemethod=TikZ]{mdframed}

\definecolor{bg}{rgb}{0.4,0.25,0.95}
\definecolor{pagebg}{rgb}{0,0,0.5}
\surroundwithmdframed[
   topline=false,
   rightline=false,
   bottomline=false,
   leftmargin=\parindent,
   skipabove=8pt,
   skipbelow=8pt,
   linecolor=blue,
   innerbottommargin=10pt,
   % backgroundcolor=bg,font=\color{orange}\sffamily, fontcolor=white
]{definition}

\usepackage{empheq}
\usepackage[most]{tcolorbox}

\newtcbox{\mymath}[1][]{%
    nobeforeafter, math upper, tcbox raise base,
    enhanced, colframe=blue!30!black,
    colback=red!10, boxrule=1pt,
    #1}

\usepackage{unixode}


\DeclareMathOperator{\ord}{ord}
\DeclareMathOperator{\orb}{orb}
\DeclareMathOperator{\stab}{stab}
\DeclareMathOperator{\Stab}{stab}
\DeclareMathOperator{\ppcm}{ppcm}
\DeclareMathOperator{\conj}{Conj}
\DeclareMathOperator{\End}{End}
\DeclareMathOperator{\rot}{rot}
\DeclareMathOperator{\trs}{trace}
\DeclareMathOperator{\Ind}{Ind}
\DeclareMathOperator{\mat}{Mat}
\DeclareMathOperator{\id}{Id}
\DeclareMathOperator{\vect}{vect}
\DeclareMathOperator{\img}{img}
\DeclareMathOperator{\cov}{Cov}
\DeclareMathOperator{\dist}{dist}
\DeclareMathOperator{\irr}{Irr}
\DeclareMathOperator{\image}{Im}
\DeclareMathOperator{\pd}{\partial}
\DeclareMathOperator{\epi}{epi}
\DeclareMathOperator{\Argmin}{Argmin}
\DeclareMathOperator{\dom}{dom}
\DeclareMathOperator{\proj}{proj}
\DeclareMathOperator{\ctg}{ctg}
\DeclareMathOperator{\supp}{supp}
\DeclareMathOperator{\argmin}{argmin}
\DeclareMathOperator{\mult}{mult}
\DeclareMathOperator{\ch}{ch}
\DeclareMathOperator{\sh}{sh}
\DeclareMathOperator{\rang}{rang}
\DeclareMathOperator{\diam}{diam}
\DeclareMathOperator{\Epigraphe}{Epigraphe}




\usepackage{xcolor}
\everymath{\color{blue}}
%\everymath{\color[rgb]{0,1,1}}
%\pagecolor[rgb]{0,0,0.5}


\newcommand*{\pdtest}[3][]{\ensuremath{\frac{\partial^{#1} #2}{\partial #3}}}

\newcommand*{\deffunc}[6][]{\ensuremath{
\begin{array}{rcl}
#2 : #3 &\rightarrow& #4\\
#5 &\mapsto& #6
\end{array}
}}

\newcommand{\eqcolon}{\mathrel{\resizebox{\widthof{$\mathord{=}$}}{\height}{ $\!\!=\!\!\resizebox{1.2\width}{0.8\height}{\raisebox{0.23ex}{$\mathop{:}$}}\!\!$ }}}
\newcommand{\coloneq}{\mathrel{\resizebox{\widthof{$\mathord{=}$}}{\height}{ $\!\!\resizebox{1.2\width}{0.8\height}{\raisebox{0.23ex}{$\mathop{:}$}}\!\!=\!\!$ }}}
\newcommand{\eqcolonl}{\ensuremath{\mathrel{=\!\!\mathop{:}}}}
\newcommand{\coloneql}{\ensuremath{\mathrel{\mathop{:} \!\! =}}}
\newcommand{\vc}[1]{% inline column vector
  \left(\begin{smallmatrix}#1\end{smallmatrix}\right)%
}
\newcommand{\vr}[1]{% inline row vector
  \begin{smallmatrix}(\,#1\,)\end{smallmatrix}%
}
\makeatletter
\newcommand*{\defeq}{\ =\mathrel{\rlap{%
                     \raisebox{0.3ex}{$\m@th\cdot$}}%
                     \raisebox{-0.3ex}{$\m@th\cdot$}}%
                     }
\makeatother

\newcommand{\mathcircle}[1]{% inline row vector
 \overset{\circ}{#1}
}
\newcommand{\ulim}{% low limit
 \underline{\lim}
}
\newcommand{\ssi}{% iff
\iff
}
\newcommand{\ps}[2]{
\expval{#1 | #2}
}
\newcommand{\df}[1]{
\mqty{#1}
}
\newcommand{\n}[1]{
\norm{#1}
}
\newcommand{\sys}[1]{
\left\{\smqty{#1}\right.
}


\newcommand{\eqdef}{\ensuremath{\overset{\text{def}}=}}


\def\Circlearrowright{\ensuremath{%
  \rotatebox[origin=c]{230}{$\circlearrowright$}}}

\newcommand\ct[1]{\text{\rmfamily\upshape #1}}
\newcommand\question[1]{ {\color{red} ...!? \small #1}}
\newcommand\caz[1]{\left\{\begin{array} #1 \end{array}\right.}
\newcommand\const{\text{\rmfamily\upshape const}}
\newcommand\toP{ \overset{\pro}{\to}}
\newcommand\toPP{ \overset{\text{PP}}{\to}}
\newcommand{\oeq}{\mathrel{\text{\textcircled{$=$}}}}





\usepackage{xcolor}
% \usepackage[normalem]{ulem}
\usepackage{lipsum}
\makeatletter
% \newcommand\colorwave[1][blue]{\bgroup \markoverwith{\lower3.5\p@\hbox{\sixly \textcolor{#1}{\char58}}}\ULon}
%\font\sixly=lasy6 % does not re-load if already loaded, so no memory problem.

\newmdtheoremenv[
linewidth= 1pt,linecolor= blue,%
leftmargin=20,rightmargin=20,innertopmargin=0pt, innerrightmargin=40,%
tikzsetting = { draw=lightgray, line width = 0.3pt,dashed,%
dash pattern = on 15pt off 3pt},%
splittopskip=\topskip,skipbelow=\baselineskip,%
skipabove=\baselineskip,ntheorem,roundcorner=0pt,
% backgroundcolor=pagebg,font=\color{orange}\sffamily, fontcolor=white
]{examplebox}{Exemple}[section]



\newcommand\R{\mathbb{R}}
\newcommand\Z{\mathbb{Z}}
\newcommand\N{\mathbb{N}}
\newcommand\E{\mathbb{E}}
\newcommand\F{\mathcal{F}}
\newcommand\cH{\mathcal{H}}
\newcommand\V{\mathbb{V}}
\newcommand\dmo{ ^{-1} }
\newcommand\kapa{\kappa}
\newcommand\im{Im}
\newcommand\hs{\mathcal{H}}





\usepackage{soul}

\makeatletter
\newcommand*{\whiten}[1]{\llap{\textcolor{white}{{\the\SOUL@token}}\hspace{#1pt}}}
\DeclareRobustCommand*\myul{%
    \def\SOUL@everyspace{\underline{\space}\kern\z@}%
    \def\SOUL@everytoken{%
     \setbox0=\hbox{\the\SOUL@token}%
     \ifdim\dp0>\z@
        \raisebox{\dp0}{\underline{\phantom{\the\SOUL@token}}}%
        \whiten{1}\whiten{0}%
        \whiten{-1}\whiten{-2}%
        \llap{\the\SOUL@token}%
     \else
        \underline{\the\SOUL@token}%
     \fi}%
\SOUL@}
\makeatother

\newcommand*{\demp}{\fontfamily{lmtt}\selectfont}

\DeclareTextFontCommand{\textdemp}{\demp}

\begin{document}

\ifcomment
Multiline
comment
\fi
\ifcomment
\myul{Typesetting test}
% \color[rgb]{1,1,1}
$∑_i^n≠ 60º±∞π∆¬≈√j∫h≤≥µ$

$\CR \R\pro\ind\pro\gS\pro
\mqty[a&b\\c&d]$
$\pro\mathbb{P}$
$\dd{x}$

  \[
    \alpha(x)=\left\{
                \begin{array}{ll}
                  x\\
                  \frac{1}{1+e^{-kx}}\\
                  \frac{e^x-e^{-x}}{e^x+e^{-x}}
                \end{array}
              \right.
  \]

  $\expval{x}$
  
  $\chi_\rho(ghg\dmo)=\Tr(\rho_{ghg\dmo})=\Tr(\rho_g\circ\rho_h\circ\rho\dmo_g)=\Tr(\rho_h)\overset{\mbox{\scalebox{0.5}{$\Tr(AB)=\Tr(BA)$}}}{=}\chi_\rho(h)$
  	$\mathop{\oplus}_{\substack{x\in X}}$

$\mat(\rho_g)=(a_{ij}(g))_{\scriptsize \substack{1\leq i\leq d \\ 1\leq j\leq d}}$ et $\mat(\rho'_g)=(a'_{ij}(g))_{\scriptsize \substack{1\leq i'\leq d' \\ 1\leq j'\leq d'}}$



\[\int_a^b{\mathbb{R}^2}g(u, v)\dd{P_{XY}}(u, v)=\iint g(u,v) f_{XY}(u, v)\dd \lambda(u) \dd \lambda(v)\]
$$\lim_{x\to\infty} f(x)$$	
$$\iiiint_V \mu(t,u,v,w) \,dt\,du\,dv\,dw$$
$$\sum_{n=1}^{\infty} 2^{-n} = 1$$	
\begin{definition}
	Si $X$ et $Y$ sont 2 v.a. ou definit la \textsc{Covariance} entre $X$ et $Y$ comme
	$\cov(X,Y)\overset{\text{def}}{=}\E\left[(X-\E(X))(Y-\E(Y))\right]=\E(XY)-\E(X)\E(Y)$.
\end{definition}
\fi
\pagebreak

% \tableofcontents

% insert your code here
%\input{./algebra/main.tex}
%\input{./geometrie-differentielle/main.tex}
%\input{./probabilite/main.tex}
%\input{./analyse-fonctionnelle/main.tex}
% \input{./Analyse-convexe-et-dualite-en-optimisation/main.tex}
%\input{./tikz/main.tex}
%\input{./Theorie-du-distributions/main.tex}
%\input{./optimisation/mine.tex}
 \input{./modelisation/main.tex}

% yves.aubry@univ-tln.fr : algebra

\end{document}

%% !TEX encoding = UTF-8 Unicode
% !TEX TS-program = xelatex

\documentclass[french]{report}

%\usepackage[utf8]{inputenc}
%\usepackage[T1]{fontenc}
\usepackage{babel}


\newif\ifcomment
%\commenttrue # Show comments

\usepackage{physics}
\usepackage{amssymb}


\usepackage{amsthm}
% \usepackage{thmtools}
\usepackage{mathtools}
\usepackage{amsfonts}

\usepackage{color}

\usepackage{tikz}

\usepackage{geometry}
\geometry{a5paper, margin=0.1in, right=1cm}

\usepackage{dsfont}

\usepackage{graphicx}
\graphicspath{ {images/} }

\usepackage{faktor}

\usepackage{IEEEtrantools}
\usepackage{enumerate}   
\usepackage[PostScript=dvips]{"/Users/aware/Documents/Courses/diagrams"}


\newtheorem{theorem}{Théorème}[section]
\renewcommand{\thetheorem}{\arabic{theorem}}
\newtheorem{lemme}{Lemme}[section]
\renewcommand{\thelemme}{\arabic{lemme}}
\newtheorem{proposition}{Proposition}[section]
\renewcommand{\theproposition}{\arabic{proposition}}
\newtheorem{notations}{Notations}[section]
\newtheorem{problem}{Problème}[section]
\newtheorem{corollary}{Corollaire}[theorem]
\renewcommand{\thecorollary}{\arabic{corollary}}
\newtheorem{property}{Propriété}[section]
\newtheorem{objective}{Objectif}[section]

\theoremstyle{definition}
\newtheorem{definition}{Définition}[section]
\renewcommand{\thedefinition}{\arabic{definition}}
\newtheorem{exercise}{Exercice}[chapter]
\renewcommand{\theexercise}{\arabic{exercise}}
\newtheorem{example}{Exemple}[chapter]
\renewcommand{\theexample}{\arabic{example}}
\newtheorem*{solution}{Solution}
\newtheorem*{application}{Application}
\newtheorem*{notation}{Notation}
\newtheorem*{vocabulary}{Vocabulaire}
\newtheorem*{properties}{Propriétés}



\theoremstyle{remark}
\newtheorem*{remark}{Remarque}
\newtheorem*{rappel}{Rappel}


\usepackage{etoolbox}
\AtBeginEnvironment{exercise}{\small}
\AtBeginEnvironment{example}{\small}

\usepackage{cases}
\usepackage[red]{mypack}

\usepackage[framemethod=TikZ]{mdframed}

\definecolor{bg}{rgb}{0.4,0.25,0.95}
\definecolor{pagebg}{rgb}{0,0,0.5}
\surroundwithmdframed[
   topline=false,
   rightline=false,
   bottomline=false,
   leftmargin=\parindent,
   skipabove=8pt,
   skipbelow=8pt,
   linecolor=blue,
   innerbottommargin=10pt,
   % backgroundcolor=bg,font=\color{orange}\sffamily, fontcolor=white
]{definition}

\usepackage{empheq}
\usepackage[most]{tcolorbox}

\newtcbox{\mymath}[1][]{%
    nobeforeafter, math upper, tcbox raise base,
    enhanced, colframe=blue!30!black,
    colback=red!10, boxrule=1pt,
    #1}

\usepackage{unixode}


\DeclareMathOperator{\ord}{ord}
\DeclareMathOperator{\orb}{orb}
\DeclareMathOperator{\stab}{stab}
\DeclareMathOperator{\Stab}{stab}
\DeclareMathOperator{\ppcm}{ppcm}
\DeclareMathOperator{\conj}{Conj}
\DeclareMathOperator{\End}{End}
\DeclareMathOperator{\rot}{rot}
\DeclareMathOperator{\trs}{trace}
\DeclareMathOperator{\Ind}{Ind}
\DeclareMathOperator{\mat}{Mat}
\DeclareMathOperator{\id}{Id}
\DeclareMathOperator{\vect}{vect}
\DeclareMathOperator{\img}{img}
\DeclareMathOperator{\cov}{Cov}
\DeclareMathOperator{\dist}{dist}
\DeclareMathOperator{\irr}{Irr}
\DeclareMathOperator{\image}{Im}
\DeclareMathOperator{\pd}{\partial}
\DeclareMathOperator{\epi}{epi}
\DeclareMathOperator{\Argmin}{Argmin}
\DeclareMathOperator{\dom}{dom}
\DeclareMathOperator{\proj}{proj}
\DeclareMathOperator{\ctg}{ctg}
\DeclareMathOperator{\supp}{supp}
\DeclareMathOperator{\argmin}{argmin}
\DeclareMathOperator{\mult}{mult}
\DeclareMathOperator{\ch}{ch}
\DeclareMathOperator{\sh}{sh}
\DeclareMathOperator{\rang}{rang}
\DeclareMathOperator{\diam}{diam}
\DeclareMathOperator{\Epigraphe}{Epigraphe}




\usepackage{xcolor}
\everymath{\color{blue}}
%\everymath{\color[rgb]{0,1,1}}
%\pagecolor[rgb]{0,0,0.5}


\newcommand*{\pdtest}[3][]{\ensuremath{\frac{\partial^{#1} #2}{\partial #3}}}

\newcommand*{\deffunc}[6][]{\ensuremath{
\begin{array}{rcl}
#2 : #3 &\rightarrow& #4\\
#5 &\mapsto& #6
\end{array}
}}

\newcommand{\eqcolon}{\mathrel{\resizebox{\widthof{$\mathord{=}$}}{\height}{ $\!\!=\!\!\resizebox{1.2\width}{0.8\height}{\raisebox{0.23ex}{$\mathop{:}$}}\!\!$ }}}
\newcommand{\coloneq}{\mathrel{\resizebox{\widthof{$\mathord{=}$}}{\height}{ $\!\!\resizebox{1.2\width}{0.8\height}{\raisebox{0.23ex}{$\mathop{:}$}}\!\!=\!\!$ }}}
\newcommand{\eqcolonl}{\ensuremath{\mathrel{=\!\!\mathop{:}}}}
\newcommand{\coloneql}{\ensuremath{\mathrel{\mathop{:} \!\! =}}}
\newcommand{\vc}[1]{% inline column vector
  \left(\begin{smallmatrix}#1\end{smallmatrix}\right)%
}
\newcommand{\vr}[1]{% inline row vector
  \begin{smallmatrix}(\,#1\,)\end{smallmatrix}%
}
\makeatletter
\newcommand*{\defeq}{\ =\mathrel{\rlap{%
                     \raisebox{0.3ex}{$\m@th\cdot$}}%
                     \raisebox{-0.3ex}{$\m@th\cdot$}}%
                     }
\makeatother

\newcommand{\mathcircle}[1]{% inline row vector
 \overset{\circ}{#1}
}
\newcommand{\ulim}{% low limit
 \underline{\lim}
}
\newcommand{\ssi}{% iff
\iff
}
\newcommand{\ps}[2]{
\expval{#1 | #2}
}
\newcommand{\df}[1]{
\mqty{#1}
}
\newcommand{\n}[1]{
\norm{#1}
}
\newcommand{\sys}[1]{
\left\{\smqty{#1}\right.
}


\newcommand{\eqdef}{\ensuremath{\overset{\text{def}}=}}


\def\Circlearrowright{\ensuremath{%
  \rotatebox[origin=c]{230}{$\circlearrowright$}}}

\newcommand\ct[1]{\text{\rmfamily\upshape #1}}
\newcommand\question[1]{ {\color{red} ...!? \small #1}}
\newcommand\caz[1]{\left\{\begin{array} #1 \end{array}\right.}
\newcommand\const{\text{\rmfamily\upshape const}}
\newcommand\toP{ \overset{\pro}{\to}}
\newcommand\toPP{ \overset{\text{PP}}{\to}}
\newcommand{\oeq}{\mathrel{\text{\textcircled{$=$}}}}





\usepackage{xcolor}
% \usepackage[normalem]{ulem}
\usepackage{lipsum}
\makeatletter
% \newcommand\colorwave[1][blue]{\bgroup \markoverwith{\lower3.5\p@\hbox{\sixly \textcolor{#1}{\char58}}}\ULon}
%\font\sixly=lasy6 % does not re-load if already loaded, so no memory problem.

\newmdtheoremenv[
linewidth= 1pt,linecolor= blue,%
leftmargin=20,rightmargin=20,innertopmargin=0pt, innerrightmargin=40,%
tikzsetting = { draw=lightgray, line width = 0.3pt,dashed,%
dash pattern = on 15pt off 3pt},%
splittopskip=\topskip,skipbelow=\baselineskip,%
skipabove=\baselineskip,ntheorem,roundcorner=0pt,
% backgroundcolor=pagebg,font=\color{orange}\sffamily, fontcolor=white
]{examplebox}{Exemple}[section]



\newcommand\R{\mathbb{R}}
\newcommand\Z{\mathbb{Z}}
\newcommand\N{\mathbb{N}}
\newcommand\E{\mathbb{E}}
\newcommand\F{\mathcal{F}}
\newcommand\cH{\mathcal{H}}
\newcommand\V{\mathbb{V}}
\newcommand\dmo{ ^{-1} }
\newcommand\kapa{\kappa}
\newcommand\im{Im}
\newcommand\hs{\mathcal{H}}





\usepackage{soul}

\makeatletter
\newcommand*{\whiten}[1]{\llap{\textcolor{white}{{\the\SOUL@token}}\hspace{#1pt}}}
\DeclareRobustCommand*\myul{%
    \def\SOUL@everyspace{\underline{\space}\kern\z@}%
    \def\SOUL@everytoken{%
     \setbox0=\hbox{\the\SOUL@token}%
     \ifdim\dp0>\z@
        \raisebox{\dp0}{\underline{\phantom{\the\SOUL@token}}}%
        \whiten{1}\whiten{0}%
        \whiten{-1}\whiten{-2}%
        \llap{\the\SOUL@token}%
     \else
        \underline{\the\SOUL@token}%
     \fi}%
\SOUL@}
\makeatother

\newcommand*{\demp}{\fontfamily{lmtt}\selectfont}

\DeclareTextFontCommand{\textdemp}{\demp}

\begin{document}

\ifcomment
Multiline
comment
\fi
\ifcomment
\myul{Typesetting test}
% \color[rgb]{1,1,1}
$∑_i^n≠ 60º±∞π∆¬≈√j∫h≤≥µ$

$\CR \R\pro\ind\pro\gS\pro
\mqty[a&b\\c&d]$
$\pro\mathbb{P}$
$\dd{x}$

  \[
    \alpha(x)=\left\{
                \begin{array}{ll}
                  x\\
                  \frac{1}{1+e^{-kx}}\\
                  \frac{e^x-e^{-x}}{e^x+e^{-x}}
                \end{array}
              \right.
  \]

  $\expval{x}$
  
  $\chi_\rho(ghg\dmo)=\Tr(\rho_{ghg\dmo})=\Tr(\rho_g\circ\rho_h\circ\rho\dmo_g)=\Tr(\rho_h)\overset{\mbox{\scalebox{0.5}{$\Tr(AB)=\Tr(BA)$}}}{=}\chi_\rho(h)$
  	$\mathop{\oplus}_{\substack{x\in X}}$

$\mat(\rho_g)=(a_{ij}(g))_{\scriptsize \substack{1\leq i\leq d \\ 1\leq j\leq d}}$ et $\mat(\rho'_g)=(a'_{ij}(g))_{\scriptsize \substack{1\leq i'\leq d' \\ 1\leq j'\leq d'}}$



\[\int_a^b{\mathbb{R}^2}g(u, v)\dd{P_{XY}}(u, v)=\iint g(u,v) f_{XY}(u, v)\dd \lambda(u) \dd \lambda(v)\]
$$\lim_{x\to\infty} f(x)$$	
$$\iiiint_V \mu(t,u,v,w) \,dt\,du\,dv\,dw$$
$$\sum_{n=1}^{\infty} 2^{-n} = 1$$	
\begin{definition}
	Si $X$ et $Y$ sont 2 v.a. ou definit la \textsc{Covariance} entre $X$ et $Y$ comme
	$\cov(X,Y)\overset{\text{def}}{=}\E\left[(X-\E(X))(Y-\E(Y))\right]=\E(XY)-\E(X)\E(Y)$.
\end{definition}
\fi
\pagebreak

% \tableofcontents

% insert your code here
%\input{./algebra/main.tex}
%\input{./geometrie-differentielle/main.tex}
%\input{./probabilite/main.tex}
%\input{./analyse-fonctionnelle/main.tex}
% \input{./Analyse-convexe-et-dualite-en-optimisation/main.tex}
%\input{./tikz/main.tex}
%\input{./Theorie-du-distributions/main.tex}
%\input{./optimisation/mine.tex}
 \input{./modelisation/main.tex}

% yves.aubry@univ-tln.fr : algebra

\end{document}

%\input{./optimisation/mine.tex}
 % !TEX encoding = UTF-8 Unicode
% !TEX TS-program = xelatex

\documentclass[french]{report}

%\usepackage[utf8]{inputenc}
%\usepackage[T1]{fontenc}
\usepackage{babel}


\newif\ifcomment
%\commenttrue # Show comments

\usepackage{physics}
\usepackage{amssymb}


\usepackage{amsthm}
% \usepackage{thmtools}
\usepackage{mathtools}
\usepackage{amsfonts}

\usepackage{color}

\usepackage{tikz}

\usepackage{geometry}
\geometry{a5paper, margin=0.1in, right=1cm}

\usepackage{dsfont}

\usepackage{graphicx}
\graphicspath{ {images/} }

\usepackage{faktor}

\usepackage{IEEEtrantools}
\usepackage{enumerate}   
\usepackage[PostScript=dvips]{"/Users/aware/Documents/Courses/diagrams"}


\newtheorem{theorem}{Théorème}[section]
\renewcommand{\thetheorem}{\arabic{theorem}}
\newtheorem{lemme}{Lemme}[section]
\renewcommand{\thelemme}{\arabic{lemme}}
\newtheorem{proposition}{Proposition}[section]
\renewcommand{\theproposition}{\arabic{proposition}}
\newtheorem{notations}{Notations}[section]
\newtheorem{problem}{Problème}[section]
\newtheorem{corollary}{Corollaire}[theorem]
\renewcommand{\thecorollary}{\arabic{corollary}}
\newtheorem{property}{Propriété}[section]
\newtheorem{objective}{Objectif}[section]

\theoremstyle{definition}
\newtheorem{definition}{Définition}[section]
\renewcommand{\thedefinition}{\arabic{definition}}
\newtheorem{exercise}{Exercice}[chapter]
\renewcommand{\theexercise}{\arabic{exercise}}
\newtheorem{example}{Exemple}[chapter]
\renewcommand{\theexample}{\arabic{example}}
\newtheorem*{solution}{Solution}
\newtheorem*{application}{Application}
\newtheorem*{notation}{Notation}
\newtheorem*{vocabulary}{Vocabulaire}
\newtheorem*{properties}{Propriétés}



\theoremstyle{remark}
\newtheorem*{remark}{Remarque}
\newtheorem*{rappel}{Rappel}


\usepackage{etoolbox}
\AtBeginEnvironment{exercise}{\small}
\AtBeginEnvironment{example}{\small}

\usepackage{cases}
\usepackage[red]{mypack}

\usepackage[framemethod=TikZ]{mdframed}

\definecolor{bg}{rgb}{0.4,0.25,0.95}
\definecolor{pagebg}{rgb}{0,0,0.5}
\surroundwithmdframed[
   topline=false,
   rightline=false,
   bottomline=false,
   leftmargin=\parindent,
   skipabove=8pt,
   skipbelow=8pt,
   linecolor=blue,
   innerbottommargin=10pt,
   % backgroundcolor=bg,font=\color{orange}\sffamily, fontcolor=white
]{definition}

\usepackage{empheq}
\usepackage[most]{tcolorbox}

\newtcbox{\mymath}[1][]{%
    nobeforeafter, math upper, tcbox raise base,
    enhanced, colframe=blue!30!black,
    colback=red!10, boxrule=1pt,
    #1}

\usepackage{unixode}


\DeclareMathOperator{\ord}{ord}
\DeclareMathOperator{\orb}{orb}
\DeclareMathOperator{\stab}{stab}
\DeclareMathOperator{\Stab}{stab}
\DeclareMathOperator{\ppcm}{ppcm}
\DeclareMathOperator{\conj}{Conj}
\DeclareMathOperator{\End}{End}
\DeclareMathOperator{\rot}{rot}
\DeclareMathOperator{\trs}{trace}
\DeclareMathOperator{\Ind}{Ind}
\DeclareMathOperator{\mat}{Mat}
\DeclareMathOperator{\id}{Id}
\DeclareMathOperator{\vect}{vect}
\DeclareMathOperator{\img}{img}
\DeclareMathOperator{\cov}{Cov}
\DeclareMathOperator{\dist}{dist}
\DeclareMathOperator{\irr}{Irr}
\DeclareMathOperator{\image}{Im}
\DeclareMathOperator{\pd}{\partial}
\DeclareMathOperator{\epi}{epi}
\DeclareMathOperator{\Argmin}{Argmin}
\DeclareMathOperator{\dom}{dom}
\DeclareMathOperator{\proj}{proj}
\DeclareMathOperator{\ctg}{ctg}
\DeclareMathOperator{\supp}{supp}
\DeclareMathOperator{\argmin}{argmin}
\DeclareMathOperator{\mult}{mult}
\DeclareMathOperator{\ch}{ch}
\DeclareMathOperator{\sh}{sh}
\DeclareMathOperator{\rang}{rang}
\DeclareMathOperator{\diam}{diam}
\DeclareMathOperator{\Epigraphe}{Epigraphe}




\usepackage{xcolor}
\everymath{\color{blue}}
%\everymath{\color[rgb]{0,1,1}}
%\pagecolor[rgb]{0,0,0.5}


\newcommand*{\pdtest}[3][]{\ensuremath{\frac{\partial^{#1} #2}{\partial #3}}}

\newcommand*{\deffunc}[6][]{\ensuremath{
\begin{array}{rcl}
#2 : #3 &\rightarrow& #4\\
#5 &\mapsto& #6
\end{array}
}}

\newcommand{\eqcolon}{\mathrel{\resizebox{\widthof{$\mathord{=}$}}{\height}{ $\!\!=\!\!\resizebox{1.2\width}{0.8\height}{\raisebox{0.23ex}{$\mathop{:}$}}\!\!$ }}}
\newcommand{\coloneq}{\mathrel{\resizebox{\widthof{$\mathord{=}$}}{\height}{ $\!\!\resizebox{1.2\width}{0.8\height}{\raisebox{0.23ex}{$\mathop{:}$}}\!\!=\!\!$ }}}
\newcommand{\eqcolonl}{\ensuremath{\mathrel{=\!\!\mathop{:}}}}
\newcommand{\coloneql}{\ensuremath{\mathrel{\mathop{:} \!\! =}}}
\newcommand{\vc}[1]{% inline column vector
  \left(\begin{smallmatrix}#1\end{smallmatrix}\right)%
}
\newcommand{\vr}[1]{% inline row vector
  \begin{smallmatrix}(\,#1\,)\end{smallmatrix}%
}
\makeatletter
\newcommand*{\defeq}{\ =\mathrel{\rlap{%
                     \raisebox{0.3ex}{$\m@th\cdot$}}%
                     \raisebox{-0.3ex}{$\m@th\cdot$}}%
                     }
\makeatother

\newcommand{\mathcircle}[1]{% inline row vector
 \overset{\circ}{#1}
}
\newcommand{\ulim}{% low limit
 \underline{\lim}
}
\newcommand{\ssi}{% iff
\iff
}
\newcommand{\ps}[2]{
\expval{#1 | #2}
}
\newcommand{\df}[1]{
\mqty{#1}
}
\newcommand{\n}[1]{
\norm{#1}
}
\newcommand{\sys}[1]{
\left\{\smqty{#1}\right.
}


\newcommand{\eqdef}{\ensuremath{\overset{\text{def}}=}}


\def\Circlearrowright{\ensuremath{%
  \rotatebox[origin=c]{230}{$\circlearrowright$}}}

\newcommand\ct[1]{\text{\rmfamily\upshape #1}}
\newcommand\question[1]{ {\color{red} ...!? \small #1}}
\newcommand\caz[1]{\left\{\begin{array} #1 \end{array}\right.}
\newcommand\const{\text{\rmfamily\upshape const}}
\newcommand\toP{ \overset{\pro}{\to}}
\newcommand\toPP{ \overset{\text{PP}}{\to}}
\newcommand{\oeq}{\mathrel{\text{\textcircled{$=$}}}}





\usepackage{xcolor}
% \usepackage[normalem]{ulem}
\usepackage{lipsum}
\makeatletter
% \newcommand\colorwave[1][blue]{\bgroup \markoverwith{\lower3.5\p@\hbox{\sixly \textcolor{#1}{\char58}}}\ULon}
%\font\sixly=lasy6 % does not re-load if already loaded, so no memory problem.

\newmdtheoremenv[
linewidth= 1pt,linecolor= blue,%
leftmargin=20,rightmargin=20,innertopmargin=0pt, innerrightmargin=40,%
tikzsetting = { draw=lightgray, line width = 0.3pt,dashed,%
dash pattern = on 15pt off 3pt},%
splittopskip=\topskip,skipbelow=\baselineskip,%
skipabove=\baselineskip,ntheorem,roundcorner=0pt,
% backgroundcolor=pagebg,font=\color{orange}\sffamily, fontcolor=white
]{examplebox}{Exemple}[section]



\newcommand\R{\mathbb{R}}
\newcommand\Z{\mathbb{Z}}
\newcommand\N{\mathbb{N}}
\newcommand\E{\mathbb{E}}
\newcommand\F{\mathcal{F}}
\newcommand\cH{\mathcal{H}}
\newcommand\V{\mathbb{V}}
\newcommand\dmo{ ^{-1} }
\newcommand\kapa{\kappa}
\newcommand\im{Im}
\newcommand\hs{\mathcal{H}}





\usepackage{soul}

\makeatletter
\newcommand*{\whiten}[1]{\llap{\textcolor{white}{{\the\SOUL@token}}\hspace{#1pt}}}
\DeclareRobustCommand*\myul{%
    \def\SOUL@everyspace{\underline{\space}\kern\z@}%
    \def\SOUL@everytoken{%
     \setbox0=\hbox{\the\SOUL@token}%
     \ifdim\dp0>\z@
        \raisebox{\dp0}{\underline{\phantom{\the\SOUL@token}}}%
        \whiten{1}\whiten{0}%
        \whiten{-1}\whiten{-2}%
        \llap{\the\SOUL@token}%
     \else
        \underline{\the\SOUL@token}%
     \fi}%
\SOUL@}
\makeatother

\newcommand*{\demp}{\fontfamily{lmtt}\selectfont}

\DeclareTextFontCommand{\textdemp}{\demp}

\begin{document}

\ifcomment
Multiline
comment
\fi
\ifcomment
\myul{Typesetting test}
% \color[rgb]{1,1,1}
$∑_i^n≠ 60º±∞π∆¬≈√j∫h≤≥µ$

$\CR \R\pro\ind\pro\gS\pro
\mqty[a&b\\c&d]$
$\pro\mathbb{P}$
$\dd{x}$

  \[
    \alpha(x)=\left\{
                \begin{array}{ll}
                  x\\
                  \frac{1}{1+e^{-kx}}\\
                  \frac{e^x-e^{-x}}{e^x+e^{-x}}
                \end{array}
              \right.
  \]

  $\expval{x}$
  
  $\chi_\rho(ghg\dmo)=\Tr(\rho_{ghg\dmo})=\Tr(\rho_g\circ\rho_h\circ\rho\dmo_g)=\Tr(\rho_h)\overset{\mbox{\scalebox{0.5}{$\Tr(AB)=\Tr(BA)$}}}{=}\chi_\rho(h)$
  	$\mathop{\oplus}_{\substack{x\in X}}$

$\mat(\rho_g)=(a_{ij}(g))_{\scriptsize \substack{1\leq i\leq d \\ 1\leq j\leq d}}$ et $\mat(\rho'_g)=(a'_{ij}(g))_{\scriptsize \substack{1\leq i'\leq d' \\ 1\leq j'\leq d'}}$



\[\int_a^b{\mathbb{R}^2}g(u, v)\dd{P_{XY}}(u, v)=\iint g(u,v) f_{XY}(u, v)\dd \lambda(u) \dd \lambda(v)\]
$$\lim_{x\to\infty} f(x)$$	
$$\iiiint_V \mu(t,u,v,w) \,dt\,du\,dv\,dw$$
$$\sum_{n=1}^{\infty} 2^{-n} = 1$$	
\begin{definition}
	Si $X$ et $Y$ sont 2 v.a. ou definit la \textsc{Covariance} entre $X$ et $Y$ comme
	$\cov(X,Y)\overset{\text{def}}{=}\E\left[(X-\E(X))(Y-\E(Y))\right]=\E(XY)-\E(X)\E(Y)$.
\end{definition}
\fi
\pagebreak

% \tableofcontents

% insert your code here
%\input{./algebra/main.tex}
%\input{./geometrie-differentielle/main.tex}
%\input{./probabilite/main.tex}
%\input{./analyse-fonctionnelle/main.tex}
% \input{./Analyse-convexe-et-dualite-en-optimisation/main.tex}
%\input{./tikz/main.tex}
%\input{./Theorie-du-distributions/main.tex}
%\input{./optimisation/mine.tex}
 \input{./modelisation/main.tex}

% yves.aubry@univ-tln.fr : algebra

\end{document}


% yves.aubry@univ-tln.fr : algebra

\end{document}


% yves.aubry@univ-tln.fr : algebra

\end{document}

%% !TEX encoding = UTF-8 Unicode
% !TEX TS-program = xelatex

\documentclass[french]{report}

%\usepackage[utf8]{inputenc}
%\usepackage[T1]{fontenc}
\usepackage{babel}


\newif\ifcomment
%\commenttrue # Show comments

\usepackage{physics}
\usepackage{amssymb}


\usepackage{amsthm}
% \usepackage{thmtools}
\usepackage{mathtools}
\usepackage{amsfonts}

\usepackage{color}

\usepackage{tikz}

\usepackage{geometry}
\geometry{a5paper, margin=0.1in, right=1cm}

\usepackage{dsfont}

\usepackage{graphicx}
\graphicspath{ {images/} }

\usepackage{faktor}

\usepackage{IEEEtrantools}
\usepackage{enumerate}   
\usepackage[PostScript=dvips]{"/Users/aware/Documents/Courses/diagrams"}


\newtheorem{theorem}{Théorème}[section]
\renewcommand{\thetheorem}{\arabic{theorem}}
\newtheorem{lemme}{Lemme}[section]
\renewcommand{\thelemme}{\arabic{lemme}}
\newtheorem{proposition}{Proposition}[section]
\renewcommand{\theproposition}{\arabic{proposition}}
\newtheorem{notations}{Notations}[section]
\newtheorem{problem}{Problème}[section]
\newtheorem{corollary}{Corollaire}[theorem]
\renewcommand{\thecorollary}{\arabic{corollary}}
\newtheorem{property}{Propriété}[section]
\newtheorem{objective}{Objectif}[section]

\theoremstyle{definition}
\newtheorem{definition}{Définition}[section]
\renewcommand{\thedefinition}{\arabic{definition}}
\newtheorem{exercise}{Exercice}[chapter]
\renewcommand{\theexercise}{\arabic{exercise}}
\newtheorem{example}{Exemple}[chapter]
\renewcommand{\theexample}{\arabic{example}}
\newtheorem*{solution}{Solution}
\newtheorem*{application}{Application}
\newtheorem*{notation}{Notation}
\newtheorem*{vocabulary}{Vocabulaire}
\newtheorem*{properties}{Propriétés}



\theoremstyle{remark}
\newtheorem*{remark}{Remarque}
\newtheorem*{rappel}{Rappel}


\usepackage{etoolbox}
\AtBeginEnvironment{exercise}{\small}
\AtBeginEnvironment{example}{\small}

\usepackage{cases}
\usepackage[red]{mypack}

\usepackage[framemethod=TikZ]{mdframed}

\definecolor{bg}{rgb}{0.4,0.25,0.95}
\definecolor{pagebg}{rgb}{0,0,0.5}
\surroundwithmdframed[
   topline=false,
   rightline=false,
   bottomline=false,
   leftmargin=\parindent,
   skipabove=8pt,
   skipbelow=8pt,
   linecolor=blue,
   innerbottommargin=10pt,
   % backgroundcolor=bg,font=\color{orange}\sffamily, fontcolor=white
]{definition}

\usepackage{empheq}
\usepackage[most]{tcolorbox}

\newtcbox{\mymath}[1][]{%
    nobeforeafter, math upper, tcbox raise base,
    enhanced, colframe=blue!30!black,
    colback=red!10, boxrule=1pt,
    #1}

\usepackage{unixode}


\DeclareMathOperator{\ord}{ord}
\DeclareMathOperator{\orb}{orb}
\DeclareMathOperator{\stab}{stab}
\DeclareMathOperator{\Stab}{stab}
\DeclareMathOperator{\ppcm}{ppcm}
\DeclareMathOperator{\conj}{Conj}
\DeclareMathOperator{\End}{End}
\DeclareMathOperator{\rot}{rot}
\DeclareMathOperator{\trs}{trace}
\DeclareMathOperator{\Ind}{Ind}
\DeclareMathOperator{\mat}{Mat}
\DeclareMathOperator{\id}{Id}
\DeclareMathOperator{\vect}{vect}
\DeclareMathOperator{\img}{img}
\DeclareMathOperator{\cov}{Cov}
\DeclareMathOperator{\dist}{dist}
\DeclareMathOperator{\irr}{Irr}
\DeclareMathOperator{\image}{Im}
\DeclareMathOperator{\pd}{\partial}
\DeclareMathOperator{\epi}{epi}
\DeclareMathOperator{\Argmin}{Argmin}
\DeclareMathOperator{\dom}{dom}
\DeclareMathOperator{\proj}{proj}
\DeclareMathOperator{\ctg}{ctg}
\DeclareMathOperator{\supp}{supp}
\DeclareMathOperator{\argmin}{argmin}
\DeclareMathOperator{\mult}{mult}
\DeclareMathOperator{\ch}{ch}
\DeclareMathOperator{\sh}{sh}
\DeclareMathOperator{\rang}{rang}
\DeclareMathOperator{\diam}{diam}
\DeclareMathOperator{\Epigraphe}{Epigraphe}




\usepackage{xcolor}
\everymath{\color{blue}}
%\everymath{\color[rgb]{0,1,1}}
%\pagecolor[rgb]{0,0,0.5}


\newcommand*{\pdtest}[3][]{\ensuremath{\frac{\partial^{#1} #2}{\partial #3}}}

\newcommand*{\deffunc}[6][]{\ensuremath{
\begin{array}{rcl}
#2 : #3 &\rightarrow& #4\\
#5 &\mapsto& #6
\end{array}
}}

\newcommand{\eqcolon}{\mathrel{\resizebox{\widthof{$\mathord{=}$}}{\height}{ $\!\!=\!\!\resizebox{1.2\width}{0.8\height}{\raisebox{0.23ex}{$\mathop{:}$}}\!\!$ }}}
\newcommand{\coloneq}{\mathrel{\resizebox{\widthof{$\mathord{=}$}}{\height}{ $\!\!\resizebox{1.2\width}{0.8\height}{\raisebox{0.23ex}{$\mathop{:}$}}\!\!=\!\!$ }}}
\newcommand{\eqcolonl}{\ensuremath{\mathrel{=\!\!\mathop{:}}}}
\newcommand{\coloneql}{\ensuremath{\mathrel{\mathop{:} \!\! =}}}
\newcommand{\vc}[1]{% inline column vector
  \left(\begin{smallmatrix}#1\end{smallmatrix}\right)%
}
\newcommand{\vr}[1]{% inline row vector
  \begin{smallmatrix}(\,#1\,)\end{smallmatrix}%
}
\makeatletter
\newcommand*{\defeq}{\ =\mathrel{\rlap{%
                     \raisebox{0.3ex}{$\m@th\cdot$}}%
                     \raisebox{-0.3ex}{$\m@th\cdot$}}%
                     }
\makeatother

\newcommand{\mathcircle}[1]{% inline row vector
 \overset{\circ}{#1}
}
\newcommand{\ulim}{% low limit
 \underline{\lim}
}
\newcommand{\ssi}{% iff
\iff
}
\newcommand{\ps}[2]{
\expval{#1 | #2}
}
\newcommand{\df}[1]{
\mqty{#1}
}
\newcommand{\n}[1]{
\norm{#1}
}
\newcommand{\sys}[1]{
\left\{\smqty{#1}\right.
}


\newcommand{\eqdef}{\ensuremath{\overset{\text{def}}=}}


\def\Circlearrowright{\ensuremath{%
  \rotatebox[origin=c]{230}{$\circlearrowright$}}}

\newcommand\ct[1]{\text{\rmfamily\upshape #1}}
\newcommand\question[1]{ {\color{red} ...!? \small #1}}
\newcommand\caz[1]{\left\{\begin{array} #1 \end{array}\right.}
\newcommand\const{\text{\rmfamily\upshape const}}
\newcommand\toP{ \overset{\pro}{\to}}
\newcommand\toPP{ \overset{\text{PP}}{\to}}
\newcommand{\oeq}{\mathrel{\text{\textcircled{$=$}}}}





\usepackage{xcolor}
% \usepackage[normalem]{ulem}
\usepackage{lipsum}
\makeatletter
% \newcommand\colorwave[1][blue]{\bgroup \markoverwith{\lower3.5\p@\hbox{\sixly \textcolor{#1}{\char58}}}\ULon}
%\font\sixly=lasy6 % does not re-load if already loaded, so no memory problem.

\newmdtheoremenv[
linewidth= 1pt,linecolor= blue,%
leftmargin=20,rightmargin=20,innertopmargin=0pt, innerrightmargin=40,%
tikzsetting = { draw=lightgray, line width = 0.3pt,dashed,%
dash pattern = on 15pt off 3pt},%
splittopskip=\topskip,skipbelow=\baselineskip,%
skipabove=\baselineskip,ntheorem,roundcorner=0pt,
% backgroundcolor=pagebg,font=\color{orange}\sffamily, fontcolor=white
]{examplebox}{Exemple}[section]



\newcommand\R{\mathbb{R}}
\newcommand\Z{\mathbb{Z}}
\newcommand\N{\mathbb{N}}
\newcommand\E{\mathbb{E}}
\newcommand\F{\mathcal{F}}
\newcommand\cH{\mathcal{H}}
\newcommand\V{\mathbb{V}}
\newcommand\dmo{ ^{-1} }
\newcommand\kapa{\kappa}
\newcommand\im{Im}
\newcommand\hs{\mathcal{H}}





\usepackage{soul}

\makeatletter
\newcommand*{\whiten}[1]{\llap{\textcolor{white}{{\the\SOUL@token}}\hspace{#1pt}}}
\DeclareRobustCommand*\myul{%
    \def\SOUL@everyspace{\underline{\space}\kern\z@}%
    \def\SOUL@everytoken{%
     \setbox0=\hbox{\the\SOUL@token}%
     \ifdim\dp0>\z@
        \raisebox{\dp0}{\underline{\phantom{\the\SOUL@token}}}%
        \whiten{1}\whiten{0}%
        \whiten{-1}\whiten{-2}%
        \llap{\the\SOUL@token}%
     \else
        \underline{\the\SOUL@token}%
     \fi}%
\SOUL@}
\makeatother

\newcommand*{\demp}{\fontfamily{lmtt}\selectfont}

\DeclareTextFontCommand{\textdemp}{\demp}

\begin{document}

\ifcomment
Multiline
comment
\fi
\ifcomment
\myul{Typesetting test}
% \color[rgb]{1,1,1}
$∑_i^n≠ 60º±∞π∆¬≈√j∫h≤≥µ$

$\CR \R\pro\ind\pro\gS\pro
\mqty[a&b\\c&d]$
$\pro\mathbb{P}$
$\dd{x}$

  \[
    \alpha(x)=\left\{
                \begin{array}{ll}
                  x\\
                  \frac{1}{1+e^{-kx}}\\
                  \frac{e^x-e^{-x}}{e^x+e^{-x}}
                \end{array}
              \right.
  \]

  $\expval{x}$
  
  $\chi_\rho(ghg\dmo)=\Tr(\rho_{ghg\dmo})=\Tr(\rho_g\circ\rho_h\circ\rho\dmo_g)=\Tr(\rho_h)\overset{\mbox{\scalebox{0.5}{$\Tr(AB)=\Tr(BA)$}}}{=}\chi_\rho(h)$
  	$\mathop{\oplus}_{\substack{x\in X}}$

$\mat(\rho_g)=(a_{ij}(g))_{\scriptsize \substack{1\leq i\leq d \\ 1\leq j\leq d}}$ et $\mat(\rho'_g)=(a'_{ij}(g))_{\scriptsize \substack{1\leq i'\leq d' \\ 1\leq j'\leq d'}}$



\[\int_a^b{\mathbb{R}^2}g(u, v)\dd{P_{XY}}(u, v)=\iint g(u,v) f_{XY}(u, v)\dd \lambda(u) \dd \lambda(v)\]
$$\lim_{x\to\infty} f(x)$$	
$$\iiiint_V \mu(t,u,v,w) \,dt\,du\,dv\,dw$$
$$\sum_{n=1}^{\infty} 2^{-n} = 1$$	
\begin{definition}
	Si $X$ et $Y$ sont 2 v.a. ou definit la \textsc{Covariance} entre $X$ et $Y$ comme
	$\cov(X,Y)\overset{\text{def}}{=}\E\left[(X-\E(X))(Y-\E(Y))\right]=\E(XY)-\E(X)\E(Y)$.
\end{definition}
\fi
\pagebreak

% \tableofcontents

% insert your code here
%% !TEX encoding = UTF-8 Unicode
% !TEX TS-program = xelatex

\documentclass[french]{report}

%\usepackage[utf8]{inputenc}
%\usepackage[T1]{fontenc}
\usepackage{babel}


\newif\ifcomment
%\commenttrue # Show comments

\usepackage{physics}
\usepackage{amssymb}


\usepackage{amsthm}
% \usepackage{thmtools}
\usepackage{mathtools}
\usepackage{amsfonts}

\usepackage{color}

\usepackage{tikz}

\usepackage{geometry}
\geometry{a5paper, margin=0.1in, right=1cm}

\usepackage{dsfont}

\usepackage{graphicx}
\graphicspath{ {images/} }

\usepackage{faktor}

\usepackage{IEEEtrantools}
\usepackage{enumerate}   
\usepackage[PostScript=dvips]{"/Users/aware/Documents/Courses/diagrams"}


\newtheorem{theorem}{Théorème}[section]
\renewcommand{\thetheorem}{\arabic{theorem}}
\newtheorem{lemme}{Lemme}[section]
\renewcommand{\thelemme}{\arabic{lemme}}
\newtheorem{proposition}{Proposition}[section]
\renewcommand{\theproposition}{\arabic{proposition}}
\newtheorem{notations}{Notations}[section]
\newtheorem{problem}{Problème}[section]
\newtheorem{corollary}{Corollaire}[theorem]
\renewcommand{\thecorollary}{\arabic{corollary}}
\newtheorem{property}{Propriété}[section]
\newtheorem{objective}{Objectif}[section]

\theoremstyle{definition}
\newtheorem{definition}{Définition}[section]
\renewcommand{\thedefinition}{\arabic{definition}}
\newtheorem{exercise}{Exercice}[chapter]
\renewcommand{\theexercise}{\arabic{exercise}}
\newtheorem{example}{Exemple}[chapter]
\renewcommand{\theexample}{\arabic{example}}
\newtheorem*{solution}{Solution}
\newtheorem*{application}{Application}
\newtheorem*{notation}{Notation}
\newtheorem*{vocabulary}{Vocabulaire}
\newtheorem*{properties}{Propriétés}



\theoremstyle{remark}
\newtheorem*{remark}{Remarque}
\newtheorem*{rappel}{Rappel}


\usepackage{etoolbox}
\AtBeginEnvironment{exercise}{\small}
\AtBeginEnvironment{example}{\small}

\usepackage{cases}
\usepackage[red]{mypack}

\usepackage[framemethod=TikZ]{mdframed}

\definecolor{bg}{rgb}{0.4,0.25,0.95}
\definecolor{pagebg}{rgb}{0,0,0.5}
\surroundwithmdframed[
   topline=false,
   rightline=false,
   bottomline=false,
   leftmargin=\parindent,
   skipabove=8pt,
   skipbelow=8pt,
   linecolor=blue,
   innerbottommargin=10pt,
   % backgroundcolor=bg,font=\color{orange}\sffamily, fontcolor=white
]{definition}

\usepackage{empheq}
\usepackage[most]{tcolorbox}

\newtcbox{\mymath}[1][]{%
    nobeforeafter, math upper, tcbox raise base,
    enhanced, colframe=blue!30!black,
    colback=red!10, boxrule=1pt,
    #1}

\usepackage{unixode}


\DeclareMathOperator{\ord}{ord}
\DeclareMathOperator{\orb}{orb}
\DeclareMathOperator{\stab}{stab}
\DeclareMathOperator{\Stab}{stab}
\DeclareMathOperator{\ppcm}{ppcm}
\DeclareMathOperator{\conj}{Conj}
\DeclareMathOperator{\End}{End}
\DeclareMathOperator{\rot}{rot}
\DeclareMathOperator{\trs}{trace}
\DeclareMathOperator{\Ind}{Ind}
\DeclareMathOperator{\mat}{Mat}
\DeclareMathOperator{\id}{Id}
\DeclareMathOperator{\vect}{vect}
\DeclareMathOperator{\img}{img}
\DeclareMathOperator{\cov}{Cov}
\DeclareMathOperator{\dist}{dist}
\DeclareMathOperator{\irr}{Irr}
\DeclareMathOperator{\image}{Im}
\DeclareMathOperator{\pd}{\partial}
\DeclareMathOperator{\epi}{epi}
\DeclareMathOperator{\Argmin}{Argmin}
\DeclareMathOperator{\dom}{dom}
\DeclareMathOperator{\proj}{proj}
\DeclareMathOperator{\ctg}{ctg}
\DeclareMathOperator{\supp}{supp}
\DeclareMathOperator{\argmin}{argmin}
\DeclareMathOperator{\mult}{mult}
\DeclareMathOperator{\ch}{ch}
\DeclareMathOperator{\sh}{sh}
\DeclareMathOperator{\rang}{rang}
\DeclareMathOperator{\diam}{diam}
\DeclareMathOperator{\Epigraphe}{Epigraphe}




\usepackage{xcolor}
\everymath{\color{blue}}
%\everymath{\color[rgb]{0,1,1}}
%\pagecolor[rgb]{0,0,0.5}


\newcommand*{\pdtest}[3][]{\ensuremath{\frac{\partial^{#1} #2}{\partial #3}}}

\newcommand*{\deffunc}[6][]{\ensuremath{
\begin{array}{rcl}
#2 : #3 &\rightarrow& #4\\
#5 &\mapsto& #6
\end{array}
}}

\newcommand{\eqcolon}{\mathrel{\resizebox{\widthof{$\mathord{=}$}}{\height}{ $\!\!=\!\!\resizebox{1.2\width}{0.8\height}{\raisebox{0.23ex}{$\mathop{:}$}}\!\!$ }}}
\newcommand{\coloneq}{\mathrel{\resizebox{\widthof{$\mathord{=}$}}{\height}{ $\!\!\resizebox{1.2\width}{0.8\height}{\raisebox{0.23ex}{$\mathop{:}$}}\!\!=\!\!$ }}}
\newcommand{\eqcolonl}{\ensuremath{\mathrel{=\!\!\mathop{:}}}}
\newcommand{\coloneql}{\ensuremath{\mathrel{\mathop{:} \!\! =}}}
\newcommand{\vc}[1]{% inline column vector
  \left(\begin{smallmatrix}#1\end{smallmatrix}\right)%
}
\newcommand{\vr}[1]{% inline row vector
  \begin{smallmatrix}(\,#1\,)\end{smallmatrix}%
}
\makeatletter
\newcommand*{\defeq}{\ =\mathrel{\rlap{%
                     \raisebox{0.3ex}{$\m@th\cdot$}}%
                     \raisebox{-0.3ex}{$\m@th\cdot$}}%
                     }
\makeatother

\newcommand{\mathcircle}[1]{% inline row vector
 \overset{\circ}{#1}
}
\newcommand{\ulim}{% low limit
 \underline{\lim}
}
\newcommand{\ssi}{% iff
\iff
}
\newcommand{\ps}[2]{
\expval{#1 | #2}
}
\newcommand{\df}[1]{
\mqty{#1}
}
\newcommand{\n}[1]{
\norm{#1}
}
\newcommand{\sys}[1]{
\left\{\smqty{#1}\right.
}


\newcommand{\eqdef}{\ensuremath{\overset{\text{def}}=}}


\def\Circlearrowright{\ensuremath{%
  \rotatebox[origin=c]{230}{$\circlearrowright$}}}

\newcommand\ct[1]{\text{\rmfamily\upshape #1}}
\newcommand\question[1]{ {\color{red} ...!? \small #1}}
\newcommand\caz[1]{\left\{\begin{array} #1 \end{array}\right.}
\newcommand\const{\text{\rmfamily\upshape const}}
\newcommand\toP{ \overset{\pro}{\to}}
\newcommand\toPP{ \overset{\text{PP}}{\to}}
\newcommand{\oeq}{\mathrel{\text{\textcircled{$=$}}}}





\usepackage{xcolor}
% \usepackage[normalem]{ulem}
\usepackage{lipsum}
\makeatletter
% \newcommand\colorwave[1][blue]{\bgroup \markoverwith{\lower3.5\p@\hbox{\sixly \textcolor{#1}{\char58}}}\ULon}
%\font\sixly=lasy6 % does not re-load if already loaded, so no memory problem.

\newmdtheoremenv[
linewidth= 1pt,linecolor= blue,%
leftmargin=20,rightmargin=20,innertopmargin=0pt, innerrightmargin=40,%
tikzsetting = { draw=lightgray, line width = 0.3pt,dashed,%
dash pattern = on 15pt off 3pt},%
splittopskip=\topskip,skipbelow=\baselineskip,%
skipabove=\baselineskip,ntheorem,roundcorner=0pt,
% backgroundcolor=pagebg,font=\color{orange}\sffamily, fontcolor=white
]{examplebox}{Exemple}[section]



\newcommand\R{\mathbb{R}}
\newcommand\Z{\mathbb{Z}}
\newcommand\N{\mathbb{N}}
\newcommand\E{\mathbb{E}}
\newcommand\F{\mathcal{F}}
\newcommand\cH{\mathcal{H}}
\newcommand\V{\mathbb{V}}
\newcommand\dmo{ ^{-1} }
\newcommand\kapa{\kappa}
\newcommand\im{Im}
\newcommand\hs{\mathcal{H}}





\usepackage{soul}

\makeatletter
\newcommand*{\whiten}[1]{\llap{\textcolor{white}{{\the\SOUL@token}}\hspace{#1pt}}}
\DeclareRobustCommand*\myul{%
    \def\SOUL@everyspace{\underline{\space}\kern\z@}%
    \def\SOUL@everytoken{%
     \setbox0=\hbox{\the\SOUL@token}%
     \ifdim\dp0>\z@
        \raisebox{\dp0}{\underline{\phantom{\the\SOUL@token}}}%
        \whiten{1}\whiten{0}%
        \whiten{-1}\whiten{-2}%
        \llap{\the\SOUL@token}%
     \else
        \underline{\the\SOUL@token}%
     \fi}%
\SOUL@}
\makeatother

\newcommand*{\demp}{\fontfamily{lmtt}\selectfont}

\DeclareTextFontCommand{\textdemp}{\demp}

\begin{document}

\ifcomment
Multiline
comment
\fi
\ifcomment
\myul{Typesetting test}
% \color[rgb]{1,1,1}
$∑_i^n≠ 60º±∞π∆¬≈√j∫h≤≥µ$

$\CR \R\pro\ind\pro\gS\pro
\mqty[a&b\\c&d]$
$\pro\mathbb{P}$
$\dd{x}$

  \[
    \alpha(x)=\left\{
                \begin{array}{ll}
                  x\\
                  \frac{1}{1+e^{-kx}}\\
                  \frac{e^x-e^{-x}}{e^x+e^{-x}}
                \end{array}
              \right.
  \]

  $\expval{x}$
  
  $\chi_\rho(ghg\dmo)=\Tr(\rho_{ghg\dmo})=\Tr(\rho_g\circ\rho_h\circ\rho\dmo_g)=\Tr(\rho_h)\overset{\mbox{\scalebox{0.5}{$\Tr(AB)=\Tr(BA)$}}}{=}\chi_\rho(h)$
  	$\mathop{\oplus}_{\substack{x\in X}}$

$\mat(\rho_g)=(a_{ij}(g))_{\scriptsize \substack{1\leq i\leq d \\ 1\leq j\leq d}}$ et $\mat(\rho'_g)=(a'_{ij}(g))_{\scriptsize \substack{1\leq i'\leq d' \\ 1\leq j'\leq d'}}$



\[\int_a^b{\mathbb{R}^2}g(u, v)\dd{P_{XY}}(u, v)=\iint g(u,v) f_{XY}(u, v)\dd \lambda(u) \dd \lambda(v)\]
$$\lim_{x\to\infty} f(x)$$	
$$\iiiint_V \mu(t,u,v,w) \,dt\,du\,dv\,dw$$
$$\sum_{n=1}^{\infty} 2^{-n} = 1$$	
\begin{definition}
	Si $X$ et $Y$ sont 2 v.a. ou definit la \textsc{Covariance} entre $X$ et $Y$ comme
	$\cov(X,Y)\overset{\text{def}}{=}\E\left[(X-\E(X))(Y-\E(Y))\right]=\E(XY)-\E(X)\E(Y)$.
\end{definition}
\fi
\pagebreak

% \tableofcontents

% insert your code here
%% !TEX encoding = UTF-8 Unicode
% !TEX TS-program = xelatex

\documentclass[french]{report}

%\usepackage[utf8]{inputenc}
%\usepackage[T1]{fontenc}
\usepackage{babel}


\newif\ifcomment
%\commenttrue # Show comments

\usepackage{physics}
\usepackage{amssymb}


\usepackage{amsthm}
% \usepackage{thmtools}
\usepackage{mathtools}
\usepackage{amsfonts}

\usepackage{color}

\usepackage{tikz}

\usepackage{geometry}
\geometry{a5paper, margin=0.1in, right=1cm}

\usepackage{dsfont}

\usepackage{graphicx}
\graphicspath{ {images/} }

\usepackage{faktor}

\usepackage{IEEEtrantools}
\usepackage{enumerate}   
\usepackage[PostScript=dvips]{"/Users/aware/Documents/Courses/diagrams"}


\newtheorem{theorem}{Théorème}[section]
\renewcommand{\thetheorem}{\arabic{theorem}}
\newtheorem{lemme}{Lemme}[section]
\renewcommand{\thelemme}{\arabic{lemme}}
\newtheorem{proposition}{Proposition}[section]
\renewcommand{\theproposition}{\arabic{proposition}}
\newtheorem{notations}{Notations}[section]
\newtheorem{problem}{Problème}[section]
\newtheorem{corollary}{Corollaire}[theorem]
\renewcommand{\thecorollary}{\arabic{corollary}}
\newtheorem{property}{Propriété}[section]
\newtheorem{objective}{Objectif}[section]

\theoremstyle{definition}
\newtheorem{definition}{Définition}[section]
\renewcommand{\thedefinition}{\arabic{definition}}
\newtheorem{exercise}{Exercice}[chapter]
\renewcommand{\theexercise}{\arabic{exercise}}
\newtheorem{example}{Exemple}[chapter]
\renewcommand{\theexample}{\arabic{example}}
\newtheorem*{solution}{Solution}
\newtheorem*{application}{Application}
\newtheorem*{notation}{Notation}
\newtheorem*{vocabulary}{Vocabulaire}
\newtheorem*{properties}{Propriétés}



\theoremstyle{remark}
\newtheorem*{remark}{Remarque}
\newtheorem*{rappel}{Rappel}


\usepackage{etoolbox}
\AtBeginEnvironment{exercise}{\small}
\AtBeginEnvironment{example}{\small}

\usepackage{cases}
\usepackage[red]{mypack}

\usepackage[framemethod=TikZ]{mdframed}

\definecolor{bg}{rgb}{0.4,0.25,0.95}
\definecolor{pagebg}{rgb}{0,0,0.5}
\surroundwithmdframed[
   topline=false,
   rightline=false,
   bottomline=false,
   leftmargin=\parindent,
   skipabove=8pt,
   skipbelow=8pt,
   linecolor=blue,
   innerbottommargin=10pt,
   % backgroundcolor=bg,font=\color{orange}\sffamily, fontcolor=white
]{definition}

\usepackage{empheq}
\usepackage[most]{tcolorbox}

\newtcbox{\mymath}[1][]{%
    nobeforeafter, math upper, tcbox raise base,
    enhanced, colframe=blue!30!black,
    colback=red!10, boxrule=1pt,
    #1}

\usepackage{unixode}


\DeclareMathOperator{\ord}{ord}
\DeclareMathOperator{\orb}{orb}
\DeclareMathOperator{\stab}{stab}
\DeclareMathOperator{\Stab}{stab}
\DeclareMathOperator{\ppcm}{ppcm}
\DeclareMathOperator{\conj}{Conj}
\DeclareMathOperator{\End}{End}
\DeclareMathOperator{\rot}{rot}
\DeclareMathOperator{\trs}{trace}
\DeclareMathOperator{\Ind}{Ind}
\DeclareMathOperator{\mat}{Mat}
\DeclareMathOperator{\id}{Id}
\DeclareMathOperator{\vect}{vect}
\DeclareMathOperator{\img}{img}
\DeclareMathOperator{\cov}{Cov}
\DeclareMathOperator{\dist}{dist}
\DeclareMathOperator{\irr}{Irr}
\DeclareMathOperator{\image}{Im}
\DeclareMathOperator{\pd}{\partial}
\DeclareMathOperator{\epi}{epi}
\DeclareMathOperator{\Argmin}{Argmin}
\DeclareMathOperator{\dom}{dom}
\DeclareMathOperator{\proj}{proj}
\DeclareMathOperator{\ctg}{ctg}
\DeclareMathOperator{\supp}{supp}
\DeclareMathOperator{\argmin}{argmin}
\DeclareMathOperator{\mult}{mult}
\DeclareMathOperator{\ch}{ch}
\DeclareMathOperator{\sh}{sh}
\DeclareMathOperator{\rang}{rang}
\DeclareMathOperator{\diam}{diam}
\DeclareMathOperator{\Epigraphe}{Epigraphe}




\usepackage{xcolor}
\everymath{\color{blue}}
%\everymath{\color[rgb]{0,1,1}}
%\pagecolor[rgb]{0,0,0.5}


\newcommand*{\pdtest}[3][]{\ensuremath{\frac{\partial^{#1} #2}{\partial #3}}}

\newcommand*{\deffunc}[6][]{\ensuremath{
\begin{array}{rcl}
#2 : #3 &\rightarrow& #4\\
#5 &\mapsto& #6
\end{array}
}}

\newcommand{\eqcolon}{\mathrel{\resizebox{\widthof{$\mathord{=}$}}{\height}{ $\!\!=\!\!\resizebox{1.2\width}{0.8\height}{\raisebox{0.23ex}{$\mathop{:}$}}\!\!$ }}}
\newcommand{\coloneq}{\mathrel{\resizebox{\widthof{$\mathord{=}$}}{\height}{ $\!\!\resizebox{1.2\width}{0.8\height}{\raisebox{0.23ex}{$\mathop{:}$}}\!\!=\!\!$ }}}
\newcommand{\eqcolonl}{\ensuremath{\mathrel{=\!\!\mathop{:}}}}
\newcommand{\coloneql}{\ensuremath{\mathrel{\mathop{:} \!\! =}}}
\newcommand{\vc}[1]{% inline column vector
  \left(\begin{smallmatrix}#1\end{smallmatrix}\right)%
}
\newcommand{\vr}[1]{% inline row vector
  \begin{smallmatrix}(\,#1\,)\end{smallmatrix}%
}
\makeatletter
\newcommand*{\defeq}{\ =\mathrel{\rlap{%
                     \raisebox{0.3ex}{$\m@th\cdot$}}%
                     \raisebox{-0.3ex}{$\m@th\cdot$}}%
                     }
\makeatother

\newcommand{\mathcircle}[1]{% inline row vector
 \overset{\circ}{#1}
}
\newcommand{\ulim}{% low limit
 \underline{\lim}
}
\newcommand{\ssi}{% iff
\iff
}
\newcommand{\ps}[2]{
\expval{#1 | #2}
}
\newcommand{\df}[1]{
\mqty{#1}
}
\newcommand{\n}[1]{
\norm{#1}
}
\newcommand{\sys}[1]{
\left\{\smqty{#1}\right.
}


\newcommand{\eqdef}{\ensuremath{\overset{\text{def}}=}}


\def\Circlearrowright{\ensuremath{%
  \rotatebox[origin=c]{230}{$\circlearrowright$}}}

\newcommand\ct[1]{\text{\rmfamily\upshape #1}}
\newcommand\question[1]{ {\color{red} ...!? \small #1}}
\newcommand\caz[1]{\left\{\begin{array} #1 \end{array}\right.}
\newcommand\const{\text{\rmfamily\upshape const}}
\newcommand\toP{ \overset{\pro}{\to}}
\newcommand\toPP{ \overset{\text{PP}}{\to}}
\newcommand{\oeq}{\mathrel{\text{\textcircled{$=$}}}}





\usepackage{xcolor}
% \usepackage[normalem]{ulem}
\usepackage{lipsum}
\makeatletter
% \newcommand\colorwave[1][blue]{\bgroup \markoverwith{\lower3.5\p@\hbox{\sixly \textcolor{#1}{\char58}}}\ULon}
%\font\sixly=lasy6 % does not re-load if already loaded, so no memory problem.

\newmdtheoremenv[
linewidth= 1pt,linecolor= blue,%
leftmargin=20,rightmargin=20,innertopmargin=0pt, innerrightmargin=40,%
tikzsetting = { draw=lightgray, line width = 0.3pt,dashed,%
dash pattern = on 15pt off 3pt},%
splittopskip=\topskip,skipbelow=\baselineskip,%
skipabove=\baselineskip,ntheorem,roundcorner=0pt,
% backgroundcolor=pagebg,font=\color{orange}\sffamily, fontcolor=white
]{examplebox}{Exemple}[section]



\newcommand\R{\mathbb{R}}
\newcommand\Z{\mathbb{Z}}
\newcommand\N{\mathbb{N}}
\newcommand\E{\mathbb{E}}
\newcommand\F{\mathcal{F}}
\newcommand\cH{\mathcal{H}}
\newcommand\V{\mathbb{V}}
\newcommand\dmo{ ^{-1} }
\newcommand\kapa{\kappa}
\newcommand\im{Im}
\newcommand\hs{\mathcal{H}}





\usepackage{soul}

\makeatletter
\newcommand*{\whiten}[1]{\llap{\textcolor{white}{{\the\SOUL@token}}\hspace{#1pt}}}
\DeclareRobustCommand*\myul{%
    \def\SOUL@everyspace{\underline{\space}\kern\z@}%
    \def\SOUL@everytoken{%
     \setbox0=\hbox{\the\SOUL@token}%
     \ifdim\dp0>\z@
        \raisebox{\dp0}{\underline{\phantom{\the\SOUL@token}}}%
        \whiten{1}\whiten{0}%
        \whiten{-1}\whiten{-2}%
        \llap{\the\SOUL@token}%
     \else
        \underline{\the\SOUL@token}%
     \fi}%
\SOUL@}
\makeatother

\newcommand*{\demp}{\fontfamily{lmtt}\selectfont}

\DeclareTextFontCommand{\textdemp}{\demp}

\begin{document}

\ifcomment
Multiline
comment
\fi
\ifcomment
\myul{Typesetting test}
% \color[rgb]{1,1,1}
$∑_i^n≠ 60º±∞π∆¬≈√j∫h≤≥µ$

$\CR \R\pro\ind\pro\gS\pro
\mqty[a&b\\c&d]$
$\pro\mathbb{P}$
$\dd{x}$

  \[
    \alpha(x)=\left\{
                \begin{array}{ll}
                  x\\
                  \frac{1}{1+e^{-kx}}\\
                  \frac{e^x-e^{-x}}{e^x+e^{-x}}
                \end{array}
              \right.
  \]

  $\expval{x}$
  
  $\chi_\rho(ghg\dmo)=\Tr(\rho_{ghg\dmo})=\Tr(\rho_g\circ\rho_h\circ\rho\dmo_g)=\Tr(\rho_h)\overset{\mbox{\scalebox{0.5}{$\Tr(AB)=\Tr(BA)$}}}{=}\chi_\rho(h)$
  	$\mathop{\oplus}_{\substack{x\in X}}$

$\mat(\rho_g)=(a_{ij}(g))_{\scriptsize \substack{1\leq i\leq d \\ 1\leq j\leq d}}$ et $\mat(\rho'_g)=(a'_{ij}(g))_{\scriptsize \substack{1\leq i'\leq d' \\ 1\leq j'\leq d'}}$



\[\int_a^b{\mathbb{R}^2}g(u, v)\dd{P_{XY}}(u, v)=\iint g(u,v) f_{XY}(u, v)\dd \lambda(u) \dd \lambda(v)\]
$$\lim_{x\to\infty} f(x)$$	
$$\iiiint_V \mu(t,u,v,w) \,dt\,du\,dv\,dw$$
$$\sum_{n=1}^{\infty} 2^{-n} = 1$$	
\begin{definition}
	Si $X$ et $Y$ sont 2 v.a. ou definit la \textsc{Covariance} entre $X$ et $Y$ comme
	$\cov(X,Y)\overset{\text{def}}{=}\E\left[(X-\E(X))(Y-\E(Y))\right]=\E(XY)-\E(X)\E(Y)$.
\end{definition}
\fi
\pagebreak

% \tableofcontents

% insert your code here
%\input{./algebra/main.tex}
%\input{./geometrie-differentielle/main.tex}
%\input{./probabilite/main.tex}
%\input{./analyse-fonctionnelle/main.tex}
% \input{./Analyse-convexe-et-dualite-en-optimisation/main.tex}
%\input{./tikz/main.tex}
%\input{./Theorie-du-distributions/main.tex}
%\input{./optimisation/mine.tex}
 \input{./modelisation/main.tex}

% yves.aubry@univ-tln.fr : algebra

\end{document}

%% !TEX encoding = UTF-8 Unicode
% !TEX TS-program = xelatex

\documentclass[french]{report}

%\usepackage[utf8]{inputenc}
%\usepackage[T1]{fontenc}
\usepackage{babel}


\newif\ifcomment
%\commenttrue # Show comments

\usepackage{physics}
\usepackage{amssymb}


\usepackage{amsthm}
% \usepackage{thmtools}
\usepackage{mathtools}
\usepackage{amsfonts}

\usepackage{color}

\usepackage{tikz}

\usepackage{geometry}
\geometry{a5paper, margin=0.1in, right=1cm}

\usepackage{dsfont}

\usepackage{graphicx}
\graphicspath{ {images/} }

\usepackage{faktor}

\usepackage{IEEEtrantools}
\usepackage{enumerate}   
\usepackage[PostScript=dvips]{"/Users/aware/Documents/Courses/diagrams"}


\newtheorem{theorem}{Théorème}[section]
\renewcommand{\thetheorem}{\arabic{theorem}}
\newtheorem{lemme}{Lemme}[section]
\renewcommand{\thelemme}{\arabic{lemme}}
\newtheorem{proposition}{Proposition}[section]
\renewcommand{\theproposition}{\arabic{proposition}}
\newtheorem{notations}{Notations}[section]
\newtheorem{problem}{Problème}[section]
\newtheorem{corollary}{Corollaire}[theorem]
\renewcommand{\thecorollary}{\arabic{corollary}}
\newtheorem{property}{Propriété}[section]
\newtheorem{objective}{Objectif}[section]

\theoremstyle{definition}
\newtheorem{definition}{Définition}[section]
\renewcommand{\thedefinition}{\arabic{definition}}
\newtheorem{exercise}{Exercice}[chapter]
\renewcommand{\theexercise}{\arabic{exercise}}
\newtheorem{example}{Exemple}[chapter]
\renewcommand{\theexample}{\arabic{example}}
\newtheorem*{solution}{Solution}
\newtheorem*{application}{Application}
\newtheorem*{notation}{Notation}
\newtheorem*{vocabulary}{Vocabulaire}
\newtheorem*{properties}{Propriétés}



\theoremstyle{remark}
\newtheorem*{remark}{Remarque}
\newtheorem*{rappel}{Rappel}


\usepackage{etoolbox}
\AtBeginEnvironment{exercise}{\small}
\AtBeginEnvironment{example}{\small}

\usepackage{cases}
\usepackage[red]{mypack}

\usepackage[framemethod=TikZ]{mdframed}

\definecolor{bg}{rgb}{0.4,0.25,0.95}
\definecolor{pagebg}{rgb}{0,0,0.5}
\surroundwithmdframed[
   topline=false,
   rightline=false,
   bottomline=false,
   leftmargin=\parindent,
   skipabove=8pt,
   skipbelow=8pt,
   linecolor=blue,
   innerbottommargin=10pt,
   % backgroundcolor=bg,font=\color{orange}\sffamily, fontcolor=white
]{definition}

\usepackage{empheq}
\usepackage[most]{tcolorbox}

\newtcbox{\mymath}[1][]{%
    nobeforeafter, math upper, tcbox raise base,
    enhanced, colframe=blue!30!black,
    colback=red!10, boxrule=1pt,
    #1}

\usepackage{unixode}


\DeclareMathOperator{\ord}{ord}
\DeclareMathOperator{\orb}{orb}
\DeclareMathOperator{\stab}{stab}
\DeclareMathOperator{\Stab}{stab}
\DeclareMathOperator{\ppcm}{ppcm}
\DeclareMathOperator{\conj}{Conj}
\DeclareMathOperator{\End}{End}
\DeclareMathOperator{\rot}{rot}
\DeclareMathOperator{\trs}{trace}
\DeclareMathOperator{\Ind}{Ind}
\DeclareMathOperator{\mat}{Mat}
\DeclareMathOperator{\id}{Id}
\DeclareMathOperator{\vect}{vect}
\DeclareMathOperator{\img}{img}
\DeclareMathOperator{\cov}{Cov}
\DeclareMathOperator{\dist}{dist}
\DeclareMathOperator{\irr}{Irr}
\DeclareMathOperator{\image}{Im}
\DeclareMathOperator{\pd}{\partial}
\DeclareMathOperator{\epi}{epi}
\DeclareMathOperator{\Argmin}{Argmin}
\DeclareMathOperator{\dom}{dom}
\DeclareMathOperator{\proj}{proj}
\DeclareMathOperator{\ctg}{ctg}
\DeclareMathOperator{\supp}{supp}
\DeclareMathOperator{\argmin}{argmin}
\DeclareMathOperator{\mult}{mult}
\DeclareMathOperator{\ch}{ch}
\DeclareMathOperator{\sh}{sh}
\DeclareMathOperator{\rang}{rang}
\DeclareMathOperator{\diam}{diam}
\DeclareMathOperator{\Epigraphe}{Epigraphe}




\usepackage{xcolor}
\everymath{\color{blue}}
%\everymath{\color[rgb]{0,1,1}}
%\pagecolor[rgb]{0,0,0.5}


\newcommand*{\pdtest}[3][]{\ensuremath{\frac{\partial^{#1} #2}{\partial #3}}}

\newcommand*{\deffunc}[6][]{\ensuremath{
\begin{array}{rcl}
#2 : #3 &\rightarrow& #4\\
#5 &\mapsto& #6
\end{array}
}}

\newcommand{\eqcolon}{\mathrel{\resizebox{\widthof{$\mathord{=}$}}{\height}{ $\!\!=\!\!\resizebox{1.2\width}{0.8\height}{\raisebox{0.23ex}{$\mathop{:}$}}\!\!$ }}}
\newcommand{\coloneq}{\mathrel{\resizebox{\widthof{$\mathord{=}$}}{\height}{ $\!\!\resizebox{1.2\width}{0.8\height}{\raisebox{0.23ex}{$\mathop{:}$}}\!\!=\!\!$ }}}
\newcommand{\eqcolonl}{\ensuremath{\mathrel{=\!\!\mathop{:}}}}
\newcommand{\coloneql}{\ensuremath{\mathrel{\mathop{:} \!\! =}}}
\newcommand{\vc}[1]{% inline column vector
  \left(\begin{smallmatrix}#1\end{smallmatrix}\right)%
}
\newcommand{\vr}[1]{% inline row vector
  \begin{smallmatrix}(\,#1\,)\end{smallmatrix}%
}
\makeatletter
\newcommand*{\defeq}{\ =\mathrel{\rlap{%
                     \raisebox{0.3ex}{$\m@th\cdot$}}%
                     \raisebox{-0.3ex}{$\m@th\cdot$}}%
                     }
\makeatother

\newcommand{\mathcircle}[1]{% inline row vector
 \overset{\circ}{#1}
}
\newcommand{\ulim}{% low limit
 \underline{\lim}
}
\newcommand{\ssi}{% iff
\iff
}
\newcommand{\ps}[2]{
\expval{#1 | #2}
}
\newcommand{\df}[1]{
\mqty{#1}
}
\newcommand{\n}[1]{
\norm{#1}
}
\newcommand{\sys}[1]{
\left\{\smqty{#1}\right.
}


\newcommand{\eqdef}{\ensuremath{\overset{\text{def}}=}}


\def\Circlearrowright{\ensuremath{%
  \rotatebox[origin=c]{230}{$\circlearrowright$}}}

\newcommand\ct[1]{\text{\rmfamily\upshape #1}}
\newcommand\question[1]{ {\color{red} ...!? \small #1}}
\newcommand\caz[1]{\left\{\begin{array} #1 \end{array}\right.}
\newcommand\const{\text{\rmfamily\upshape const}}
\newcommand\toP{ \overset{\pro}{\to}}
\newcommand\toPP{ \overset{\text{PP}}{\to}}
\newcommand{\oeq}{\mathrel{\text{\textcircled{$=$}}}}





\usepackage{xcolor}
% \usepackage[normalem]{ulem}
\usepackage{lipsum}
\makeatletter
% \newcommand\colorwave[1][blue]{\bgroup \markoverwith{\lower3.5\p@\hbox{\sixly \textcolor{#1}{\char58}}}\ULon}
%\font\sixly=lasy6 % does not re-load if already loaded, so no memory problem.

\newmdtheoremenv[
linewidth= 1pt,linecolor= blue,%
leftmargin=20,rightmargin=20,innertopmargin=0pt, innerrightmargin=40,%
tikzsetting = { draw=lightgray, line width = 0.3pt,dashed,%
dash pattern = on 15pt off 3pt},%
splittopskip=\topskip,skipbelow=\baselineskip,%
skipabove=\baselineskip,ntheorem,roundcorner=0pt,
% backgroundcolor=pagebg,font=\color{orange}\sffamily, fontcolor=white
]{examplebox}{Exemple}[section]



\newcommand\R{\mathbb{R}}
\newcommand\Z{\mathbb{Z}}
\newcommand\N{\mathbb{N}}
\newcommand\E{\mathbb{E}}
\newcommand\F{\mathcal{F}}
\newcommand\cH{\mathcal{H}}
\newcommand\V{\mathbb{V}}
\newcommand\dmo{ ^{-1} }
\newcommand\kapa{\kappa}
\newcommand\im{Im}
\newcommand\hs{\mathcal{H}}





\usepackage{soul}

\makeatletter
\newcommand*{\whiten}[1]{\llap{\textcolor{white}{{\the\SOUL@token}}\hspace{#1pt}}}
\DeclareRobustCommand*\myul{%
    \def\SOUL@everyspace{\underline{\space}\kern\z@}%
    \def\SOUL@everytoken{%
     \setbox0=\hbox{\the\SOUL@token}%
     \ifdim\dp0>\z@
        \raisebox{\dp0}{\underline{\phantom{\the\SOUL@token}}}%
        \whiten{1}\whiten{0}%
        \whiten{-1}\whiten{-2}%
        \llap{\the\SOUL@token}%
     \else
        \underline{\the\SOUL@token}%
     \fi}%
\SOUL@}
\makeatother

\newcommand*{\demp}{\fontfamily{lmtt}\selectfont}

\DeclareTextFontCommand{\textdemp}{\demp}

\begin{document}

\ifcomment
Multiline
comment
\fi
\ifcomment
\myul{Typesetting test}
% \color[rgb]{1,1,1}
$∑_i^n≠ 60º±∞π∆¬≈√j∫h≤≥µ$

$\CR \R\pro\ind\pro\gS\pro
\mqty[a&b\\c&d]$
$\pro\mathbb{P}$
$\dd{x}$

  \[
    \alpha(x)=\left\{
                \begin{array}{ll}
                  x\\
                  \frac{1}{1+e^{-kx}}\\
                  \frac{e^x-e^{-x}}{e^x+e^{-x}}
                \end{array}
              \right.
  \]

  $\expval{x}$
  
  $\chi_\rho(ghg\dmo)=\Tr(\rho_{ghg\dmo})=\Tr(\rho_g\circ\rho_h\circ\rho\dmo_g)=\Tr(\rho_h)\overset{\mbox{\scalebox{0.5}{$\Tr(AB)=\Tr(BA)$}}}{=}\chi_\rho(h)$
  	$\mathop{\oplus}_{\substack{x\in X}}$

$\mat(\rho_g)=(a_{ij}(g))_{\scriptsize \substack{1\leq i\leq d \\ 1\leq j\leq d}}$ et $\mat(\rho'_g)=(a'_{ij}(g))_{\scriptsize \substack{1\leq i'\leq d' \\ 1\leq j'\leq d'}}$



\[\int_a^b{\mathbb{R}^2}g(u, v)\dd{P_{XY}}(u, v)=\iint g(u,v) f_{XY}(u, v)\dd \lambda(u) \dd \lambda(v)\]
$$\lim_{x\to\infty} f(x)$$	
$$\iiiint_V \mu(t,u,v,w) \,dt\,du\,dv\,dw$$
$$\sum_{n=1}^{\infty} 2^{-n} = 1$$	
\begin{definition}
	Si $X$ et $Y$ sont 2 v.a. ou definit la \textsc{Covariance} entre $X$ et $Y$ comme
	$\cov(X,Y)\overset{\text{def}}{=}\E\left[(X-\E(X))(Y-\E(Y))\right]=\E(XY)-\E(X)\E(Y)$.
\end{definition}
\fi
\pagebreak

% \tableofcontents

% insert your code here
%\input{./algebra/main.tex}
%\input{./geometrie-differentielle/main.tex}
%\input{./probabilite/main.tex}
%\input{./analyse-fonctionnelle/main.tex}
% \input{./Analyse-convexe-et-dualite-en-optimisation/main.tex}
%\input{./tikz/main.tex}
%\input{./Theorie-du-distributions/main.tex}
%\input{./optimisation/mine.tex}
 \input{./modelisation/main.tex}

% yves.aubry@univ-tln.fr : algebra

\end{document}

%% !TEX encoding = UTF-8 Unicode
% !TEX TS-program = xelatex

\documentclass[french]{report}

%\usepackage[utf8]{inputenc}
%\usepackage[T1]{fontenc}
\usepackage{babel}


\newif\ifcomment
%\commenttrue # Show comments

\usepackage{physics}
\usepackage{amssymb}


\usepackage{amsthm}
% \usepackage{thmtools}
\usepackage{mathtools}
\usepackage{amsfonts}

\usepackage{color}

\usepackage{tikz}

\usepackage{geometry}
\geometry{a5paper, margin=0.1in, right=1cm}

\usepackage{dsfont}

\usepackage{graphicx}
\graphicspath{ {images/} }

\usepackage{faktor}

\usepackage{IEEEtrantools}
\usepackage{enumerate}   
\usepackage[PostScript=dvips]{"/Users/aware/Documents/Courses/diagrams"}


\newtheorem{theorem}{Théorème}[section]
\renewcommand{\thetheorem}{\arabic{theorem}}
\newtheorem{lemme}{Lemme}[section]
\renewcommand{\thelemme}{\arabic{lemme}}
\newtheorem{proposition}{Proposition}[section]
\renewcommand{\theproposition}{\arabic{proposition}}
\newtheorem{notations}{Notations}[section]
\newtheorem{problem}{Problème}[section]
\newtheorem{corollary}{Corollaire}[theorem]
\renewcommand{\thecorollary}{\arabic{corollary}}
\newtheorem{property}{Propriété}[section]
\newtheorem{objective}{Objectif}[section]

\theoremstyle{definition}
\newtheorem{definition}{Définition}[section]
\renewcommand{\thedefinition}{\arabic{definition}}
\newtheorem{exercise}{Exercice}[chapter]
\renewcommand{\theexercise}{\arabic{exercise}}
\newtheorem{example}{Exemple}[chapter]
\renewcommand{\theexample}{\arabic{example}}
\newtheorem*{solution}{Solution}
\newtheorem*{application}{Application}
\newtheorem*{notation}{Notation}
\newtheorem*{vocabulary}{Vocabulaire}
\newtheorem*{properties}{Propriétés}



\theoremstyle{remark}
\newtheorem*{remark}{Remarque}
\newtheorem*{rappel}{Rappel}


\usepackage{etoolbox}
\AtBeginEnvironment{exercise}{\small}
\AtBeginEnvironment{example}{\small}

\usepackage{cases}
\usepackage[red]{mypack}

\usepackage[framemethod=TikZ]{mdframed}

\definecolor{bg}{rgb}{0.4,0.25,0.95}
\definecolor{pagebg}{rgb}{0,0,0.5}
\surroundwithmdframed[
   topline=false,
   rightline=false,
   bottomline=false,
   leftmargin=\parindent,
   skipabove=8pt,
   skipbelow=8pt,
   linecolor=blue,
   innerbottommargin=10pt,
   % backgroundcolor=bg,font=\color{orange}\sffamily, fontcolor=white
]{definition}

\usepackage{empheq}
\usepackage[most]{tcolorbox}

\newtcbox{\mymath}[1][]{%
    nobeforeafter, math upper, tcbox raise base,
    enhanced, colframe=blue!30!black,
    colback=red!10, boxrule=1pt,
    #1}

\usepackage{unixode}


\DeclareMathOperator{\ord}{ord}
\DeclareMathOperator{\orb}{orb}
\DeclareMathOperator{\stab}{stab}
\DeclareMathOperator{\Stab}{stab}
\DeclareMathOperator{\ppcm}{ppcm}
\DeclareMathOperator{\conj}{Conj}
\DeclareMathOperator{\End}{End}
\DeclareMathOperator{\rot}{rot}
\DeclareMathOperator{\trs}{trace}
\DeclareMathOperator{\Ind}{Ind}
\DeclareMathOperator{\mat}{Mat}
\DeclareMathOperator{\id}{Id}
\DeclareMathOperator{\vect}{vect}
\DeclareMathOperator{\img}{img}
\DeclareMathOperator{\cov}{Cov}
\DeclareMathOperator{\dist}{dist}
\DeclareMathOperator{\irr}{Irr}
\DeclareMathOperator{\image}{Im}
\DeclareMathOperator{\pd}{\partial}
\DeclareMathOperator{\epi}{epi}
\DeclareMathOperator{\Argmin}{Argmin}
\DeclareMathOperator{\dom}{dom}
\DeclareMathOperator{\proj}{proj}
\DeclareMathOperator{\ctg}{ctg}
\DeclareMathOperator{\supp}{supp}
\DeclareMathOperator{\argmin}{argmin}
\DeclareMathOperator{\mult}{mult}
\DeclareMathOperator{\ch}{ch}
\DeclareMathOperator{\sh}{sh}
\DeclareMathOperator{\rang}{rang}
\DeclareMathOperator{\diam}{diam}
\DeclareMathOperator{\Epigraphe}{Epigraphe}




\usepackage{xcolor}
\everymath{\color{blue}}
%\everymath{\color[rgb]{0,1,1}}
%\pagecolor[rgb]{0,0,0.5}


\newcommand*{\pdtest}[3][]{\ensuremath{\frac{\partial^{#1} #2}{\partial #3}}}

\newcommand*{\deffunc}[6][]{\ensuremath{
\begin{array}{rcl}
#2 : #3 &\rightarrow& #4\\
#5 &\mapsto& #6
\end{array}
}}

\newcommand{\eqcolon}{\mathrel{\resizebox{\widthof{$\mathord{=}$}}{\height}{ $\!\!=\!\!\resizebox{1.2\width}{0.8\height}{\raisebox{0.23ex}{$\mathop{:}$}}\!\!$ }}}
\newcommand{\coloneq}{\mathrel{\resizebox{\widthof{$\mathord{=}$}}{\height}{ $\!\!\resizebox{1.2\width}{0.8\height}{\raisebox{0.23ex}{$\mathop{:}$}}\!\!=\!\!$ }}}
\newcommand{\eqcolonl}{\ensuremath{\mathrel{=\!\!\mathop{:}}}}
\newcommand{\coloneql}{\ensuremath{\mathrel{\mathop{:} \!\! =}}}
\newcommand{\vc}[1]{% inline column vector
  \left(\begin{smallmatrix}#1\end{smallmatrix}\right)%
}
\newcommand{\vr}[1]{% inline row vector
  \begin{smallmatrix}(\,#1\,)\end{smallmatrix}%
}
\makeatletter
\newcommand*{\defeq}{\ =\mathrel{\rlap{%
                     \raisebox{0.3ex}{$\m@th\cdot$}}%
                     \raisebox{-0.3ex}{$\m@th\cdot$}}%
                     }
\makeatother

\newcommand{\mathcircle}[1]{% inline row vector
 \overset{\circ}{#1}
}
\newcommand{\ulim}{% low limit
 \underline{\lim}
}
\newcommand{\ssi}{% iff
\iff
}
\newcommand{\ps}[2]{
\expval{#1 | #2}
}
\newcommand{\df}[1]{
\mqty{#1}
}
\newcommand{\n}[1]{
\norm{#1}
}
\newcommand{\sys}[1]{
\left\{\smqty{#1}\right.
}


\newcommand{\eqdef}{\ensuremath{\overset{\text{def}}=}}


\def\Circlearrowright{\ensuremath{%
  \rotatebox[origin=c]{230}{$\circlearrowright$}}}

\newcommand\ct[1]{\text{\rmfamily\upshape #1}}
\newcommand\question[1]{ {\color{red} ...!? \small #1}}
\newcommand\caz[1]{\left\{\begin{array} #1 \end{array}\right.}
\newcommand\const{\text{\rmfamily\upshape const}}
\newcommand\toP{ \overset{\pro}{\to}}
\newcommand\toPP{ \overset{\text{PP}}{\to}}
\newcommand{\oeq}{\mathrel{\text{\textcircled{$=$}}}}





\usepackage{xcolor}
% \usepackage[normalem]{ulem}
\usepackage{lipsum}
\makeatletter
% \newcommand\colorwave[1][blue]{\bgroup \markoverwith{\lower3.5\p@\hbox{\sixly \textcolor{#1}{\char58}}}\ULon}
%\font\sixly=lasy6 % does not re-load if already loaded, so no memory problem.

\newmdtheoremenv[
linewidth= 1pt,linecolor= blue,%
leftmargin=20,rightmargin=20,innertopmargin=0pt, innerrightmargin=40,%
tikzsetting = { draw=lightgray, line width = 0.3pt,dashed,%
dash pattern = on 15pt off 3pt},%
splittopskip=\topskip,skipbelow=\baselineskip,%
skipabove=\baselineskip,ntheorem,roundcorner=0pt,
% backgroundcolor=pagebg,font=\color{orange}\sffamily, fontcolor=white
]{examplebox}{Exemple}[section]



\newcommand\R{\mathbb{R}}
\newcommand\Z{\mathbb{Z}}
\newcommand\N{\mathbb{N}}
\newcommand\E{\mathbb{E}}
\newcommand\F{\mathcal{F}}
\newcommand\cH{\mathcal{H}}
\newcommand\V{\mathbb{V}}
\newcommand\dmo{ ^{-1} }
\newcommand\kapa{\kappa}
\newcommand\im{Im}
\newcommand\hs{\mathcal{H}}





\usepackage{soul}

\makeatletter
\newcommand*{\whiten}[1]{\llap{\textcolor{white}{{\the\SOUL@token}}\hspace{#1pt}}}
\DeclareRobustCommand*\myul{%
    \def\SOUL@everyspace{\underline{\space}\kern\z@}%
    \def\SOUL@everytoken{%
     \setbox0=\hbox{\the\SOUL@token}%
     \ifdim\dp0>\z@
        \raisebox{\dp0}{\underline{\phantom{\the\SOUL@token}}}%
        \whiten{1}\whiten{0}%
        \whiten{-1}\whiten{-2}%
        \llap{\the\SOUL@token}%
     \else
        \underline{\the\SOUL@token}%
     \fi}%
\SOUL@}
\makeatother

\newcommand*{\demp}{\fontfamily{lmtt}\selectfont}

\DeclareTextFontCommand{\textdemp}{\demp}

\begin{document}

\ifcomment
Multiline
comment
\fi
\ifcomment
\myul{Typesetting test}
% \color[rgb]{1,1,1}
$∑_i^n≠ 60º±∞π∆¬≈√j∫h≤≥µ$

$\CR \R\pro\ind\pro\gS\pro
\mqty[a&b\\c&d]$
$\pro\mathbb{P}$
$\dd{x}$

  \[
    \alpha(x)=\left\{
                \begin{array}{ll}
                  x\\
                  \frac{1}{1+e^{-kx}}\\
                  \frac{e^x-e^{-x}}{e^x+e^{-x}}
                \end{array}
              \right.
  \]

  $\expval{x}$
  
  $\chi_\rho(ghg\dmo)=\Tr(\rho_{ghg\dmo})=\Tr(\rho_g\circ\rho_h\circ\rho\dmo_g)=\Tr(\rho_h)\overset{\mbox{\scalebox{0.5}{$\Tr(AB)=\Tr(BA)$}}}{=}\chi_\rho(h)$
  	$\mathop{\oplus}_{\substack{x\in X}}$

$\mat(\rho_g)=(a_{ij}(g))_{\scriptsize \substack{1\leq i\leq d \\ 1\leq j\leq d}}$ et $\mat(\rho'_g)=(a'_{ij}(g))_{\scriptsize \substack{1\leq i'\leq d' \\ 1\leq j'\leq d'}}$



\[\int_a^b{\mathbb{R}^2}g(u, v)\dd{P_{XY}}(u, v)=\iint g(u,v) f_{XY}(u, v)\dd \lambda(u) \dd \lambda(v)\]
$$\lim_{x\to\infty} f(x)$$	
$$\iiiint_V \mu(t,u,v,w) \,dt\,du\,dv\,dw$$
$$\sum_{n=1}^{\infty} 2^{-n} = 1$$	
\begin{definition}
	Si $X$ et $Y$ sont 2 v.a. ou definit la \textsc{Covariance} entre $X$ et $Y$ comme
	$\cov(X,Y)\overset{\text{def}}{=}\E\left[(X-\E(X))(Y-\E(Y))\right]=\E(XY)-\E(X)\E(Y)$.
\end{definition}
\fi
\pagebreak

% \tableofcontents

% insert your code here
%\input{./algebra/main.tex}
%\input{./geometrie-differentielle/main.tex}
%\input{./probabilite/main.tex}
%\input{./analyse-fonctionnelle/main.tex}
% \input{./Analyse-convexe-et-dualite-en-optimisation/main.tex}
%\input{./tikz/main.tex}
%\input{./Theorie-du-distributions/main.tex}
%\input{./optimisation/mine.tex}
 \input{./modelisation/main.tex}

% yves.aubry@univ-tln.fr : algebra

\end{document}

%% !TEX encoding = UTF-8 Unicode
% !TEX TS-program = xelatex

\documentclass[french]{report}

%\usepackage[utf8]{inputenc}
%\usepackage[T1]{fontenc}
\usepackage{babel}


\newif\ifcomment
%\commenttrue # Show comments

\usepackage{physics}
\usepackage{amssymb}


\usepackage{amsthm}
% \usepackage{thmtools}
\usepackage{mathtools}
\usepackage{amsfonts}

\usepackage{color}

\usepackage{tikz}

\usepackage{geometry}
\geometry{a5paper, margin=0.1in, right=1cm}

\usepackage{dsfont}

\usepackage{graphicx}
\graphicspath{ {images/} }

\usepackage{faktor}

\usepackage{IEEEtrantools}
\usepackage{enumerate}   
\usepackage[PostScript=dvips]{"/Users/aware/Documents/Courses/diagrams"}


\newtheorem{theorem}{Théorème}[section]
\renewcommand{\thetheorem}{\arabic{theorem}}
\newtheorem{lemme}{Lemme}[section]
\renewcommand{\thelemme}{\arabic{lemme}}
\newtheorem{proposition}{Proposition}[section]
\renewcommand{\theproposition}{\arabic{proposition}}
\newtheorem{notations}{Notations}[section]
\newtheorem{problem}{Problème}[section]
\newtheorem{corollary}{Corollaire}[theorem]
\renewcommand{\thecorollary}{\arabic{corollary}}
\newtheorem{property}{Propriété}[section]
\newtheorem{objective}{Objectif}[section]

\theoremstyle{definition}
\newtheorem{definition}{Définition}[section]
\renewcommand{\thedefinition}{\arabic{definition}}
\newtheorem{exercise}{Exercice}[chapter]
\renewcommand{\theexercise}{\arabic{exercise}}
\newtheorem{example}{Exemple}[chapter]
\renewcommand{\theexample}{\arabic{example}}
\newtheorem*{solution}{Solution}
\newtheorem*{application}{Application}
\newtheorem*{notation}{Notation}
\newtheorem*{vocabulary}{Vocabulaire}
\newtheorem*{properties}{Propriétés}



\theoremstyle{remark}
\newtheorem*{remark}{Remarque}
\newtheorem*{rappel}{Rappel}


\usepackage{etoolbox}
\AtBeginEnvironment{exercise}{\small}
\AtBeginEnvironment{example}{\small}

\usepackage{cases}
\usepackage[red]{mypack}

\usepackage[framemethod=TikZ]{mdframed}

\definecolor{bg}{rgb}{0.4,0.25,0.95}
\definecolor{pagebg}{rgb}{0,0,0.5}
\surroundwithmdframed[
   topline=false,
   rightline=false,
   bottomline=false,
   leftmargin=\parindent,
   skipabove=8pt,
   skipbelow=8pt,
   linecolor=blue,
   innerbottommargin=10pt,
   % backgroundcolor=bg,font=\color{orange}\sffamily, fontcolor=white
]{definition}

\usepackage{empheq}
\usepackage[most]{tcolorbox}

\newtcbox{\mymath}[1][]{%
    nobeforeafter, math upper, tcbox raise base,
    enhanced, colframe=blue!30!black,
    colback=red!10, boxrule=1pt,
    #1}

\usepackage{unixode}


\DeclareMathOperator{\ord}{ord}
\DeclareMathOperator{\orb}{orb}
\DeclareMathOperator{\stab}{stab}
\DeclareMathOperator{\Stab}{stab}
\DeclareMathOperator{\ppcm}{ppcm}
\DeclareMathOperator{\conj}{Conj}
\DeclareMathOperator{\End}{End}
\DeclareMathOperator{\rot}{rot}
\DeclareMathOperator{\trs}{trace}
\DeclareMathOperator{\Ind}{Ind}
\DeclareMathOperator{\mat}{Mat}
\DeclareMathOperator{\id}{Id}
\DeclareMathOperator{\vect}{vect}
\DeclareMathOperator{\img}{img}
\DeclareMathOperator{\cov}{Cov}
\DeclareMathOperator{\dist}{dist}
\DeclareMathOperator{\irr}{Irr}
\DeclareMathOperator{\image}{Im}
\DeclareMathOperator{\pd}{\partial}
\DeclareMathOperator{\epi}{epi}
\DeclareMathOperator{\Argmin}{Argmin}
\DeclareMathOperator{\dom}{dom}
\DeclareMathOperator{\proj}{proj}
\DeclareMathOperator{\ctg}{ctg}
\DeclareMathOperator{\supp}{supp}
\DeclareMathOperator{\argmin}{argmin}
\DeclareMathOperator{\mult}{mult}
\DeclareMathOperator{\ch}{ch}
\DeclareMathOperator{\sh}{sh}
\DeclareMathOperator{\rang}{rang}
\DeclareMathOperator{\diam}{diam}
\DeclareMathOperator{\Epigraphe}{Epigraphe}




\usepackage{xcolor}
\everymath{\color{blue}}
%\everymath{\color[rgb]{0,1,1}}
%\pagecolor[rgb]{0,0,0.5}


\newcommand*{\pdtest}[3][]{\ensuremath{\frac{\partial^{#1} #2}{\partial #3}}}

\newcommand*{\deffunc}[6][]{\ensuremath{
\begin{array}{rcl}
#2 : #3 &\rightarrow& #4\\
#5 &\mapsto& #6
\end{array}
}}

\newcommand{\eqcolon}{\mathrel{\resizebox{\widthof{$\mathord{=}$}}{\height}{ $\!\!=\!\!\resizebox{1.2\width}{0.8\height}{\raisebox{0.23ex}{$\mathop{:}$}}\!\!$ }}}
\newcommand{\coloneq}{\mathrel{\resizebox{\widthof{$\mathord{=}$}}{\height}{ $\!\!\resizebox{1.2\width}{0.8\height}{\raisebox{0.23ex}{$\mathop{:}$}}\!\!=\!\!$ }}}
\newcommand{\eqcolonl}{\ensuremath{\mathrel{=\!\!\mathop{:}}}}
\newcommand{\coloneql}{\ensuremath{\mathrel{\mathop{:} \!\! =}}}
\newcommand{\vc}[1]{% inline column vector
  \left(\begin{smallmatrix}#1\end{smallmatrix}\right)%
}
\newcommand{\vr}[1]{% inline row vector
  \begin{smallmatrix}(\,#1\,)\end{smallmatrix}%
}
\makeatletter
\newcommand*{\defeq}{\ =\mathrel{\rlap{%
                     \raisebox{0.3ex}{$\m@th\cdot$}}%
                     \raisebox{-0.3ex}{$\m@th\cdot$}}%
                     }
\makeatother

\newcommand{\mathcircle}[1]{% inline row vector
 \overset{\circ}{#1}
}
\newcommand{\ulim}{% low limit
 \underline{\lim}
}
\newcommand{\ssi}{% iff
\iff
}
\newcommand{\ps}[2]{
\expval{#1 | #2}
}
\newcommand{\df}[1]{
\mqty{#1}
}
\newcommand{\n}[1]{
\norm{#1}
}
\newcommand{\sys}[1]{
\left\{\smqty{#1}\right.
}


\newcommand{\eqdef}{\ensuremath{\overset{\text{def}}=}}


\def\Circlearrowright{\ensuremath{%
  \rotatebox[origin=c]{230}{$\circlearrowright$}}}

\newcommand\ct[1]{\text{\rmfamily\upshape #1}}
\newcommand\question[1]{ {\color{red} ...!? \small #1}}
\newcommand\caz[1]{\left\{\begin{array} #1 \end{array}\right.}
\newcommand\const{\text{\rmfamily\upshape const}}
\newcommand\toP{ \overset{\pro}{\to}}
\newcommand\toPP{ \overset{\text{PP}}{\to}}
\newcommand{\oeq}{\mathrel{\text{\textcircled{$=$}}}}





\usepackage{xcolor}
% \usepackage[normalem]{ulem}
\usepackage{lipsum}
\makeatletter
% \newcommand\colorwave[1][blue]{\bgroup \markoverwith{\lower3.5\p@\hbox{\sixly \textcolor{#1}{\char58}}}\ULon}
%\font\sixly=lasy6 % does not re-load if already loaded, so no memory problem.

\newmdtheoremenv[
linewidth= 1pt,linecolor= blue,%
leftmargin=20,rightmargin=20,innertopmargin=0pt, innerrightmargin=40,%
tikzsetting = { draw=lightgray, line width = 0.3pt,dashed,%
dash pattern = on 15pt off 3pt},%
splittopskip=\topskip,skipbelow=\baselineskip,%
skipabove=\baselineskip,ntheorem,roundcorner=0pt,
% backgroundcolor=pagebg,font=\color{orange}\sffamily, fontcolor=white
]{examplebox}{Exemple}[section]



\newcommand\R{\mathbb{R}}
\newcommand\Z{\mathbb{Z}}
\newcommand\N{\mathbb{N}}
\newcommand\E{\mathbb{E}}
\newcommand\F{\mathcal{F}}
\newcommand\cH{\mathcal{H}}
\newcommand\V{\mathbb{V}}
\newcommand\dmo{ ^{-1} }
\newcommand\kapa{\kappa}
\newcommand\im{Im}
\newcommand\hs{\mathcal{H}}





\usepackage{soul}

\makeatletter
\newcommand*{\whiten}[1]{\llap{\textcolor{white}{{\the\SOUL@token}}\hspace{#1pt}}}
\DeclareRobustCommand*\myul{%
    \def\SOUL@everyspace{\underline{\space}\kern\z@}%
    \def\SOUL@everytoken{%
     \setbox0=\hbox{\the\SOUL@token}%
     \ifdim\dp0>\z@
        \raisebox{\dp0}{\underline{\phantom{\the\SOUL@token}}}%
        \whiten{1}\whiten{0}%
        \whiten{-1}\whiten{-2}%
        \llap{\the\SOUL@token}%
     \else
        \underline{\the\SOUL@token}%
     \fi}%
\SOUL@}
\makeatother

\newcommand*{\demp}{\fontfamily{lmtt}\selectfont}

\DeclareTextFontCommand{\textdemp}{\demp}

\begin{document}

\ifcomment
Multiline
comment
\fi
\ifcomment
\myul{Typesetting test}
% \color[rgb]{1,1,1}
$∑_i^n≠ 60º±∞π∆¬≈√j∫h≤≥µ$

$\CR \R\pro\ind\pro\gS\pro
\mqty[a&b\\c&d]$
$\pro\mathbb{P}$
$\dd{x}$

  \[
    \alpha(x)=\left\{
                \begin{array}{ll}
                  x\\
                  \frac{1}{1+e^{-kx}}\\
                  \frac{e^x-e^{-x}}{e^x+e^{-x}}
                \end{array}
              \right.
  \]

  $\expval{x}$
  
  $\chi_\rho(ghg\dmo)=\Tr(\rho_{ghg\dmo})=\Tr(\rho_g\circ\rho_h\circ\rho\dmo_g)=\Tr(\rho_h)\overset{\mbox{\scalebox{0.5}{$\Tr(AB)=\Tr(BA)$}}}{=}\chi_\rho(h)$
  	$\mathop{\oplus}_{\substack{x\in X}}$

$\mat(\rho_g)=(a_{ij}(g))_{\scriptsize \substack{1\leq i\leq d \\ 1\leq j\leq d}}$ et $\mat(\rho'_g)=(a'_{ij}(g))_{\scriptsize \substack{1\leq i'\leq d' \\ 1\leq j'\leq d'}}$



\[\int_a^b{\mathbb{R}^2}g(u, v)\dd{P_{XY}}(u, v)=\iint g(u,v) f_{XY}(u, v)\dd \lambda(u) \dd \lambda(v)\]
$$\lim_{x\to\infty} f(x)$$	
$$\iiiint_V \mu(t,u,v,w) \,dt\,du\,dv\,dw$$
$$\sum_{n=1}^{\infty} 2^{-n} = 1$$	
\begin{definition}
	Si $X$ et $Y$ sont 2 v.a. ou definit la \textsc{Covariance} entre $X$ et $Y$ comme
	$\cov(X,Y)\overset{\text{def}}{=}\E\left[(X-\E(X))(Y-\E(Y))\right]=\E(XY)-\E(X)\E(Y)$.
\end{definition}
\fi
\pagebreak

% \tableofcontents

% insert your code here
%\input{./algebra/main.tex}
%\input{./geometrie-differentielle/main.tex}
%\input{./probabilite/main.tex}
%\input{./analyse-fonctionnelle/main.tex}
% \input{./Analyse-convexe-et-dualite-en-optimisation/main.tex}
%\input{./tikz/main.tex}
%\input{./Theorie-du-distributions/main.tex}
%\input{./optimisation/mine.tex}
 \input{./modelisation/main.tex}

% yves.aubry@univ-tln.fr : algebra

\end{document}

% % !TEX encoding = UTF-8 Unicode
% !TEX TS-program = xelatex

\documentclass[french]{report}

%\usepackage[utf8]{inputenc}
%\usepackage[T1]{fontenc}
\usepackage{babel}


\newif\ifcomment
%\commenttrue # Show comments

\usepackage{physics}
\usepackage{amssymb}


\usepackage{amsthm}
% \usepackage{thmtools}
\usepackage{mathtools}
\usepackage{amsfonts}

\usepackage{color}

\usepackage{tikz}

\usepackage{geometry}
\geometry{a5paper, margin=0.1in, right=1cm}

\usepackage{dsfont}

\usepackage{graphicx}
\graphicspath{ {images/} }

\usepackage{faktor}

\usepackage{IEEEtrantools}
\usepackage{enumerate}   
\usepackage[PostScript=dvips]{"/Users/aware/Documents/Courses/diagrams"}


\newtheorem{theorem}{Théorème}[section]
\renewcommand{\thetheorem}{\arabic{theorem}}
\newtheorem{lemme}{Lemme}[section]
\renewcommand{\thelemme}{\arabic{lemme}}
\newtheorem{proposition}{Proposition}[section]
\renewcommand{\theproposition}{\arabic{proposition}}
\newtheorem{notations}{Notations}[section]
\newtheorem{problem}{Problème}[section]
\newtheorem{corollary}{Corollaire}[theorem]
\renewcommand{\thecorollary}{\arabic{corollary}}
\newtheorem{property}{Propriété}[section]
\newtheorem{objective}{Objectif}[section]

\theoremstyle{definition}
\newtheorem{definition}{Définition}[section]
\renewcommand{\thedefinition}{\arabic{definition}}
\newtheorem{exercise}{Exercice}[chapter]
\renewcommand{\theexercise}{\arabic{exercise}}
\newtheorem{example}{Exemple}[chapter]
\renewcommand{\theexample}{\arabic{example}}
\newtheorem*{solution}{Solution}
\newtheorem*{application}{Application}
\newtheorem*{notation}{Notation}
\newtheorem*{vocabulary}{Vocabulaire}
\newtheorem*{properties}{Propriétés}



\theoremstyle{remark}
\newtheorem*{remark}{Remarque}
\newtheorem*{rappel}{Rappel}


\usepackage{etoolbox}
\AtBeginEnvironment{exercise}{\small}
\AtBeginEnvironment{example}{\small}

\usepackage{cases}
\usepackage[red]{mypack}

\usepackage[framemethod=TikZ]{mdframed}

\definecolor{bg}{rgb}{0.4,0.25,0.95}
\definecolor{pagebg}{rgb}{0,0,0.5}
\surroundwithmdframed[
   topline=false,
   rightline=false,
   bottomline=false,
   leftmargin=\parindent,
   skipabove=8pt,
   skipbelow=8pt,
   linecolor=blue,
   innerbottommargin=10pt,
   % backgroundcolor=bg,font=\color{orange}\sffamily, fontcolor=white
]{definition}

\usepackage{empheq}
\usepackage[most]{tcolorbox}

\newtcbox{\mymath}[1][]{%
    nobeforeafter, math upper, tcbox raise base,
    enhanced, colframe=blue!30!black,
    colback=red!10, boxrule=1pt,
    #1}

\usepackage{unixode}


\DeclareMathOperator{\ord}{ord}
\DeclareMathOperator{\orb}{orb}
\DeclareMathOperator{\stab}{stab}
\DeclareMathOperator{\Stab}{stab}
\DeclareMathOperator{\ppcm}{ppcm}
\DeclareMathOperator{\conj}{Conj}
\DeclareMathOperator{\End}{End}
\DeclareMathOperator{\rot}{rot}
\DeclareMathOperator{\trs}{trace}
\DeclareMathOperator{\Ind}{Ind}
\DeclareMathOperator{\mat}{Mat}
\DeclareMathOperator{\id}{Id}
\DeclareMathOperator{\vect}{vect}
\DeclareMathOperator{\img}{img}
\DeclareMathOperator{\cov}{Cov}
\DeclareMathOperator{\dist}{dist}
\DeclareMathOperator{\irr}{Irr}
\DeclareMathOperator{\image}{Im}
\DeclareMathOperator{\pd}{\partial}
\DeclareMathOperator{\epi}{epi}
\DeclareMathOperator{\Argmin}{Argmin}
\DeclareMathOperator{\dom}{dom}
\DeclareMathOperator{\proj}{proj}
\DeclareMathOperator{\ctg}{ctg}
\DeclareMathOperator{\supp}{supp}
\DeclareMathOperator{\argmin}{argmin}
\DeclareMathOperator{\mult}{mult}
\DeclareMathOperator{\ch}{ch}
\DeclareMathOperator{\sh}{sh}
\DeclareMathOperator{\rang}{rang}
\DeclareMathOperator{\diam}{diam}
\DeclareMathOperator{\Epigraphe}{Epigraphe}




\usepackage{xcolor}
\everymath{\color{blue}}
%\everymath{\color[rgb]{0,1,1}}
%\pagecolor[rgb]{0,0,0.5}


\newcommand*{\pdtest}[3][]{\ensuremath{\frac{\partial^{#1} #2}{\partial #3}}}

\newcommand*{\deffunc}[6][]{\ensuremath{
\begin{array}{rcl}
#2 : #3 &\rightarrow& #4\\
#5 &\mapsto& #6
\end{array}
}}

\newcommand{\eqcolon}{\mathrel{\resizebox{\widthof{$\mathord{=}$}}{\height}{ $\!\!=\!\!\resizebox{1.2\width}{0.8\height}{\raisebox{0.23ex}{$\mathop{:}$}}\!\!$ }}}
\newcommand{\coloneq}{\mathrel{\resizebox{\widthof{$\mathord{=}$}}{\height}{ $\!\!\resizebox{1.2\width}{0.8\height}{\raisebox{0.23ex}{$\mathop{:}$}}\!\!=\!\!$ }}}
\newcommand{\eqcolonl}{\ensuremath{\mathrel{=\!\!\mathop{:}}}}
\newcommand{\coloneql}{\ensuremath{\mathrel{\mathop{:} \!\! =}}}
\newcommand{\vc}[1]{% inline column vector
  \left(\begin{smallmatrix}#1\end{smallmatrix}\right)%
}
\newcommand{\vr}[1]{% inline row vector
  \begin{smallmatrix}(\,#1\,)\end{smallmatrix}%
}
\makeatletter
\newcommand*{\defeq}{\ =\mathrel{\rlap{%
                     \raisebox{0.3ex}{$\m@th\cdot$}}%
                     \raisebox{-0.3ex}{$\m@th\cdot$}}%
                     }
\makeatother

\newcommand{\mathcircle}[1]{% inline row vector
 \overset{\circ}{#1}
}
\newcommand{\ulim}{% low limit
 \underline{\lim}
}
\newcommand{\ssi}{% iff
\iff
}
\newcommand{\ps}[2]{
\expval{#1 | #2}
}
\newcommand{\df}[1]{
\mqty{#1}
}
\newcommand{\n}[1]{
\norm{#1}
}
\newcommand{\sys}[1]{
\left\{\smqty{#1}\right.
}


\newcommand{\eqdef}{\ensuremath{\overset{\text{def}}=}}


\def\Circlearrowright{\ensuremath{%
  \rotatebox[origin=c]{230}{$\circlearrowright$}}}

\newcommand\ct[1]{\text{\rmfamily\upshape #1}}
\newcommand\question[1]{ {\color{red} ...!? \small #1}}
\newcommand\caz[1]{\left\{\begin{array} #1 \end{array}\right.}
\newcommand\const{\text{\rmfamily\upshape const}}
\newcommand\toP{ \overset{\pro}{\to}}
\newcommand\toPP{ \overset{\text{PP}}{\to}}
\newcommand{\oeq}{\mathrel{\text{\textcircled{$=$}}}}





\usepackage{xcolor}
% \usepackage[normalem]{ulem}
\usepackage{lipsum}
\makeatletter
% \newcommand\colorwave[1][blue]{\bgroup \markoverwith{\lower3.5\p@\hbox{\sixly \textcolor{#1}{\char58}}}\ULon}
%\font\sixly=lasy6 % does not re-load if already loaded, so no memory problem.

\newmdtheoremenv[
linewidth= 1pt,linecolor= blue,%
leftmargin=20,rightmargin=20,innertopmargin=0pt, innerrightmargin=40,%
tikzsetting = { draw=lightgray, line width = 0.3pt,dashed,%
dash pattern = on 15pt off 3pt},%
splittopskip=\topskip,skipbelow=\baselineskip,%
skipabove=\baselineskip,ntheorem,roundcorner=0pt,
% backgroundcolor=pagebg,font=\color{orange}\sffamily, fontcolor=white
]{examplebox}{Exemple}[section]



\newcommand\R{\mathbb{R}}
\newcommand\Z{\mathbb{Z}}
\newcommand\N{\mathbb{N}}
\newcommand\E{\mathbb{E}}
\newcommand\F{\mathcal{F}}
\newcommand\cH{\mathcal{H}}
\newcommand\V{\mathbb{V}}
\newcommand\dmo{ ^{-1} }
\newcommand\kapa{\kappa}
\newcommand\im{Im}
\newcommand\hs{\mathcal{H}}





\usepackage{soul}

\makeatletter
\newcommand*{\whiten}[1]{\llap{\textcolor{white}{{\the\SOUL@token}}\hspace{#1pt}}}
\DeclareRobustCommand*\myul{%
    \def\SOUL@everyspace{\underline{\space}\kern\z@}%
    \def\SOUL@everytoken{%
     \setbox0=\hbox{\the\SOUL@token}%
     \ifdim\dp0>\z@
        \raisebox{\dp0}{\underline{\phantom{\the\SOUL@token}}}%
        \whiten{1}\whiten{0}%
        \whiten{-1}\whiten{-2}%
        \llap{\the\SOUL@token}%
     \else
        \underline{\the\SOUL@token}%
     \fi}%
\SOUL@}
\makeatother

\newcommand*{\demp}{\fontfamily{lmtt}\selectfont}

\DeclareTextFontCommand{\textdemp}{\demp}

\begin{document}

\ifcomment
Multiline
comment
\fi
\ifcomment
\myul{Typesetting test}
% \color[rgb]{1,1,1}
$∑_i^n≠ 60º±∞π∆¬≈√j∫h≤≥µ$

$\CR \R\pro\ind\pro\gS\pro
\mqty[a&b\\c&d]$
$\pro\mathbb{P}$
$\dd{x}$

  \[
    \alpha(x)=\left\{
                \begin{array}{ll}
                  x\\
                  \frac{1}{1+e^{-kx}}\\
                  \frac{e^x-e^{-x}}{e^x+e^{-x}}
                \end{array}
              \right.
  \]

  $\expval{x}$
  
  $\chi_\rho(ghg\dmo)=\Tr(\rho_{ghg\dmo})=\Tr(\rho_g\circ\rho_h\circ\rho\dmo_g)=\Tr(\rho_h)\overset{\mbox{\scalebox{0.5}{$\Tr(AB)=\Tr(BA)$}}}{=}\chi_\rho(h)$
  	$\mathop{\oplus}_{\substack{x\in X}}$

$\mat(\rho_g)=(a_{ij}(g))_{\scriptsize \substack{1\leq i\leq d \\ 1\leq j\leq d}}$ et $\mat(\rho'_g)=(a'_{ij}(g))_{\scriptsize \substack{1\leq i'\leq d' \\ 1\leq j'\leq d'}}$



\[\int_a^b{\mathbb{R}^2}g(u, v)\dd{P_{XY}}(u, v)=\iint g(u,v) f_{XY}(u, v)\dd \lambda(u) \dd \lambda(v)\]
$$\lim_{x\to\infty} f(x)$$	
$$\iiiint_V \mu(t,u,v,w) \,dt\,du\,dv\,dw$$
$$\sum_{n=1}^{\infty} 2^{-n} = 1$$	
\begin{definition}
	Si $X$ et $Y$ sont 2 v.a. ou definit la \textsc{Covariance} entre $X$ et $Y$ comme
	$\cov(X,Y)\overset{\text{def}}{=}\E\left[(X-\E(X))(Y-\E(Y))\right]=\E(XY)-\E(X)\E(Y)$.
\end{definition}
\fi
\pagebreak

% \tableofcontents

% insert your code here
%\input{./algebra/main.tex}
%\input{./geometrie-differentielle/main.tex}
%\input{./probabilite/main.tex}
%\input{./analyse-fonctionnelle/main.tex}
% \input{./Analyse-convexe-et-dualite-en-optimisation/main.tex}
%\input{./tikz/main.tex}
%\input{./Theorie-du-distributions/main.tex}
%\input{./optimisation/mine.tex}
 \input{./modelisation/main.tex}

% yves.aubry@univ-tln.fr : algebra

\end{document}

%% !TEX encoding = UTF-8 Unicode
% !TEX TS-program = xelatex

\documentclass[french]{report}

%\usepackage[utf8]{inputenc}
%\usepackage[T1]{fontenc}
\usepackage{babel}


\newif\ifcomment
%\commenttrue # Show comments

\usepackage{physics}
\usepackage{amssymb}


\usepackage{amsthm}
% \usepackage{thmtools}
\usepackage{mathtools}
\usepackage{amsfonts}

\usepackage{color}

\usepackage{tikz}

\usepackage{geometry}
\geometry{a5paper, margin=0.1in, right=1cm}

\usepackage{dsfont}

\usepackage{graphicx}
\graphicspath{ {images/} }

\usepackage{faktor}

\usepackage{IEEEtrantools}
\usepackage{enumerate}   
\usepackage[PostScript=dvips]{"/Users/aware/Documents/Courses/diagrams"}


\newtheorem{theorem}{Théorème}[section]
\renewcommand{\thetheorem}{\arabic{theorem}}
\newtheorem{lemme}{Lemme}[section]
\renewcommand{\thelemme}{\arabic{lemme}}
\newtheorem{proposition}{Proposition}[section]
\renewcommand{\theproposition}{\arabic{proposition}}
\newtheorem{notations}{Notations}[section]
\newtheorem{problem}{Problème}[section]
\newtheorem{corollary}{Corollaire}[theorem]
\renewcommand{\thecorollary}{\arabic{corollary}}
\newtheorem{property}{Propriété}[section]
\newtheorem{objective}{Objectif}[section]

\theoremstyle{definition}
\newtheorem{definition}{Définition}[section]
\renewcommand{\thedefinition}{\arabic{definition}}
\newtheorem{exercise}{Exercice}[chapter]
\renewcommand{\theexercise}{\arabic{exercise}}
\newtheorem{example}{Exemple}[chapter]
\renewcommand{\theexample}{\arabic{example}}
\newtheorem*{solution}{Solution}
\newtheorem*{application}{Application}
\newtheorem*{notation}{Notation}
\newtheorem*{vocabulary}{Vocabulaire}
\newtheorem*{properties}{Propriétés}



\theoremstyle{remark}
\newtheorem*{remark}{Remarque}
\newtheorem*{rappel}{Rappel}


\usepackage{etoolbox}
\AtBeginEnvironment{exercise}{\small}
\AtBeginEnvironment{example}{\small}

\usepackage{cases}
\usepackage[red]{mypack}

\usepackage[framemethod=TikZ]{mdframed}

\definecolor{bg}{rgb}{0.4,0.25,0.95}
\definecolor{pagebg}{rgb}{0,0,0.5}
\surroundwithmdframed[
   topline=false,
   rightline=false,
   bottomline=false,
   leftmargin=\parindent,
   skipabove=8pt,
   skipbelow=8pt,
   linecolor=blue,
   innerbottommargin=10pt,
   % backgroundcolor=bg,font=\color{orange}\sffamily, fontcolor=white
]{definition}

\usepackage{empheq}
\usepackage[most]{tcolorbox}

\newtcbox{\mymath}[1][]{%
    nobeforeafter, math upper, tcbox raise base,
    enhanced, colframe=blue!30!black,
    colback=red!10, boxrule=1pt,
    #1}

\usepackage{unixode}


\DeclareMathOperator{\ord}{ord}
\DeclareMathOperator{\orb}{orb}
\DeclareMathOperator{\stab}{stab}
\DeclareMathOperator{\Stab}{stab}
\DeclareMathOperator{\ppcm}{ppcm}
\DeclareMathOperator{\conj}{Conj}
\DeclareMathOperator{\End}{End}
\DeclareMathOperator{\rot}{rot}
\DeclareMathOperator{\trs}{trace}
\DeclareMathOperator{\Ind}{Ind}
\DeclareMathOperator{\mat}{Mat}
\DeclareMathOperator{\id}{Id}
\DeclareMathOperator{\vect}{vect}
\DeclareMathOperator{\img}{img}
\DeclareMathOperator{\cov}{Cov}
\DeclareMathOperator{\dist}{dist}
\DeclareMathOperator{\irr}{Irr}
\DeclareMathOperator{\image}{Im}
\DeclareMathOperator{\pd}{\partial}
\DeclareMathOperator{\epi}{epi}
\DeclareMathOperator{\Argmin}{Argmin}
\DeclareMathOperator{\dom}{dom}
\DeclareMathOperator{\proj}{proj}
\DeclareMathOperator{\ctg}{ctg}
\DeclareMathOperator{\supp}{supp}
\DeclareMathOperator{\argmin}{argmin}
\DeclareMathOperator{\mult}{mult}
\DeclareMathOperator{\ch}{ch}
\DeclareMathOperator{\sh}{sh}
\DeclareMathOperator{\rang}{rang}
\DeclareMathOperator{\diam}{diam}
\DeclareMathOperator{\Epigraphe}{Epigraphe}




\usepackage{xcolor}
\everymath{\color{blue}}
%\everymath{\color[rgb]{0,1,1}}
%\pagecolor[rgb]{0,0,0.5}


\newcommand*{\pdtest}[3][]{\ensuremath{\frac{\partial^{#1} #2}{\partial #3}}}

\newcommand*{\deffunc}[6][]{\ensuremath{
\begin{array}{rcl}
#2 : #3 &\rightarrow& #4\\
#5 &\mapsto& #6
\end{array}
}}

\newcommand{\eqcolon}{\mathrel{\resizebox{\widthof{$\mathord{=}$}}{\height}{ $\!\!=\!\!\resizebox{1.2\width}{0.8\height}{\raisebox{0.23ex}{$\mathop{:}$}}\!\!$ }}}
\newcommand{\coloneq}{\mathrel{\resizebox{\widthof{$\mathord{=}$}}{\height}{ $\!\!\resizebox{1.2\width}{0.8\height}{\raisebox{0.23ex}{$\mathop{:}$}}\!\!=\!\!$ }}}
\newcommand{\eqcolonl}{\ensuremath{\mathrel{=\!\!\mathop{:}}}}
\newcommand{\coloneql}{\ensuremath{\mathrel{\mathop{:} \!\! =}}}
\newcommand{\vc}[1]{% inline column vector
  \left(\begin{smallmatrix}#1\end{smallmatrix}\right)%
}
\newcommand{\vr}[1]{% inline row vector
  \begin{smallmatrix}(\,#1\,)\end{smallmatrix}%
}
\makeatletter
\newcommand*{\defeq}{\ =\mathrel{\rlap{%
                     \raisebox{0.3ex}{$\m@th\cdot$}}%
                     \raisebox{-0.3ex}{$\m@th\cdot$}}%
                     }
\makeatother

\newcommand{\mathcircle}[1]{% inline row vector
 \overset{\circ}{#1}
}
\newcommand{\ulim}{% low limit
 \underline{\lim}
}
\newcommand{\ssi}{% iff
\iff
}
\newcommand{\ps}[2]{
\expval{#1 | #2}
}
\newcommand{\df}[1]{
\mqty{#1}
}
\newcommand{\n}[1]{
\norm{#1}
}
\newcommand{\sys}[1]{
\left\{\smqty{#1}\right.
}


\newcommand{\eqdef}{\ensuremath{\overset{\text{def}}=}}


\def\Circlearrowright{\ensuremath{%
  \rotatebox[origin=c]{230}{$\circlearrowright$}}}

\newcommand\ct[1]{\text{\rmfamily\upshape #1}}
\newcommand\question[1]{ {\color{red} ...!? \small #1}}
\newcommand\caz[1]{\left\{\begin{array} #1 \end{array}\right.}
\newcommand\const{\text{\rmfamily\upshape const}}
\newcommand\toP{ \overset{\pro}{\to}}
\newcommand\toPP{ \overset{\text{PP}}{\to}}
\newcommand{\oeq}{\mathrel{\text{\textcircled{$=$}}}}





\usepackage{xcolor}
% \usepackage[normalem]{ulem}
\usepackage{lipsum}
\makeatletter
% \newcommand\colorwave[1][blue]{\bgroup \markoverwith{\lower3.5\p@\hbox{\sixly \textcolor{#1}{\char58}}}\ULon}
%\font\sixly=lasy6 % does not re-load if already loaded, so no memory problem.

\newmdtheoremenv[
linewidth= 1pt,linecolor= blue,%
leftmargin=20,rightmargin=20,innertopmargin=0pt, innerrightmargin=40,%
tikzsetting = { draw=lightgray, line width = 0.3pt,dashed,%
dash pattern = on 15pt off 3pt},%
splittopskip=\topskip,skipbelow=\baselineskip,%
skipabove=\baselineskip,ntheorem,roundcorner=0pt,
% backgroundcolor=pagebg,font=\color{orange}\sffamily, fontcolor=white
]{examplebox}{Exemple}[section]



\newcommand\R{\mathbb{R}}
\newcommand\Z{\mathbb{Z}}
\newcommand\N{\mathbb{N}}
\newcommand\E{\mathbb{E}}
\newcommand\F{\mathcal{F}}
\newcommand\cH{\mathcal{H}}
\newcommand\V{\mathbb{V}}
\newcommand\dmo{ ^{-1} }
\newcommand\kapa{\kappa}
\newcommand\im{Im}
\newcommand\hs{\mathcal{H}}





\usepackage{soul}

\makeatletter
\newcommand*{\whiten}[1]{\llap{\textcolor{white}{{\the\SOUL@token}}\hspace{#1pt}}}
\DeclareRobustCommand*\myul{%
    \def\SOUL@everyspace{\underline{\space}\kern\z@}%
    \def\SOUL@everytoken{%
     \setbox0=\hbox{\the\SOUL@token}%
     \ifdim\dp0>\z@
        \raisebox{\dp0}{\underline{\phantom{\the\SOUL@token}}}%
        \whiten{1}\whiten{0}%
        \whiten{-1}\whiten{-2}%
        \llap{\the\SOUL@token}%
     \else
        \underline{\the\SOUL@token}%
     \fi}%
\SOUL@}
\makeatother

\newcommand*{\demp}{\fontfamily{lmtt}\selectfont}

\DeclareTextFontCommand{\textdemp}{\demp}

\begin{document}

\ifcomment
Multiline
comment
\fi
\ifcomment
\myul{Typesetting test}
% \color[rgb]{1,1,1}
$∑_i^n≠ 60º±∞π∆¬≈√j∫h≤≥µ$

$\CR \R\pro\ind\pro\gS\pro
\mqty[a&b\\c&d]$
$\pro\mathbb{P}$
$\dd{x}$

  \[
    \alpha(x)=\left\{
                \begin{array}{ll}
                  x\\
                  \frac{1}{1+e^{-kx}}\\
                  \frac{e^x-e^{-x}}{e^x+e^{-x}}
                \end{array}
              \right.
  \]

  $\expval{x}$
  
  $\chi_\rho(ghg\dmo)=\Tr(\rho_{ghg\dmo})=\Tr(\rho_g\circ\rho_h\circ\rho\dmo_g)=\Tr(\rho_h)\overset{\mbox{\scalebox{0.5}{$\Tr(AB)=\Tr(BA)$}}}{=}\chi_\rho(h)$
  	$\mathop{\oplus}_{\substack{x\in X}}$

$\mat(\rho_g)=(a_{ij}(g))_{\scriptsize \substack{1\leq i\leq d \\ 1\leq j\leq d}}$ et $\mat(\rho'_g)=(a'_{ij}(g))_{\scriptsize \substack{1\leq i'\leq d' \\ 1\leq j'\leq d'}}$



\[\int_a^b{\mathbb{R}^2}g(u, v)\dd{P_{XY}}(u, v)=\iint g(u,v) f_{XY}(u, v)\dd \lambda(u) \dd \lambda(v)\]
$$\lim_{x\to\infty} f(x)$$	
$$\iiiint_V \mu(t,u,v,w) \,dt\,du\,dv\,dw$$
$$\sum_{n=1}^{\infty} 2^{-n} = 1$$	
\begin{definition}
	Si $X$ et $Y$ sont 2 v.a. ou definit la \textsc{Covariance} entre $X$ et $Y$ comme
	$\cov(X,Y)\overset{\text{def}}{=}\E\left[(X-\E(X))(Y-\E(Y))\right]=\E(XY)-\E(X)\E(Y)$.
\end{definition}
\fi
\pagebreak

% \tableofcontents

% insert your code here
%\input{./algebra/main.tex}
%\input{./geometrie-differentielle/main.tex}
%\input{./probabilite/main.tex}
%\input{./analyse-fonctionnelle/main.tex}
% \input{./Analyse-convexe-et-dualite-en-optimisation/main.tex}
%\input{./tikz/main.tex}
%\input{./Theorie-du-distributions/main.tex}
%\input{./optimisation/mine.tex}
 \input{./modelisation/main.tex}

% yves.aubry@univ-tln.fr : algebra

\end{document}

%% !TEX encoding = UTF-8 Unicode
% !TEX TS-program = xelatex

\documentclass[french]{report}

%\usepackage[utf8]{inputenc}
%\usepackage[T1]{fontenc}
\usepackage{babel}


\newif\ifcomment
%\commenttrue # Show comments

\usepackage{physics}
\usepackage{amssymb}


\usepackage{amsthm}
% \usepackage{thmtools}
\usepackage{mathtools}
\usepackage{amsfonts}

\usepackage{color}

\usepackage{tikz}

\usepackage{geometry}
\geometry{a5paper, margin=0.1in, right=1cm}

\usepackage{dsfont}

\usepackage{graphicx}
\graphicspath{ {images/} }

\usepackage{faktor}

\usepackage{IEEEtrantools}
\usepackage{enumerate}   
\usepackage[PostScript=dvips]{"/Users/aware/Documents/Courses/diagrams"}


\newtheorem{theorem}{Théorème}[section]
\renewcommand{\thetheorem}{\arabic{theorem}}
\newtheorem{lemme}{Lemme}[section]
\renewcommand{\thelemme}{\arabic{lemme}}
\newtheorem{proposition}{Proposition}[section]
\renewcommand{\theproposition}{\arabic{proposition}}
\newtheorem{notations}{Notations}[section]
\newtheorem{problem}{Problème}[section]
\newtheorem{corollary}{Corollaire}[theorem]
\renewcommand{\thecorollary}{\arabic{corollary}}
\newtheorem{property}{Propriété}[section]
\newtheorem{objective}{Objectif}[section]

\theoremstyle{definition}
\newtheorem{definition}{Définition}[section]
\renewcommand{\thedefinition}{\arabic{definition}}
\newtheorem{exercise}{Exercice}[chapter]
\renewcommand{\theexercise}{\arabic{exercise}}
\newtheorem{example}{Exemple}[chapter]
\renewcommand{\theexample}{\arabic{example}}
\newtheorem*{solution}{Solution}
\newtheorem*{application}{Application}
\newtheorem*{notation}{Notation}
\newtheorem*{vocabulary}{Vocabulaire}
\newtheorem*{properties}{Propriétés}



\theoremstyle{remark}
\newtheorem*{remark}{Remarque}
\newtheorem*{rappel}{Rappel}


\usepackage{etoolbox}
\AtBeginEnvironment{exercise}{\small}
\AtBeginEnvironment{example}{\small}

\usepackage{cases}
\usepackage[red]{mypack}

\usepackage[framemethod=TikZ]{mdframed}

\definecolor{bg}{rgb}{0.4,0.25,0.95}
\definecolor{pagebg}{rgb}{0,0,0.5}
\surroundwithmdframed[
   topline=false,
   rightline=false,
   bottomline=false,
   leftmargin=\parindent,
   skipabove=8pt,
   skipbelow=8pt,
   linecolor=blue,
   innerbottommargin=10pt,
   % backgroundcolor=bg,font=\color{orange}\sffamily, fontcolor=white
]{definition}

\usepackage{empheq}
\usepackage[most]{tcolorbox}

\newtcbox{\mymath}[1][]{%
    nobeforeafter, math upper, tcbox raise base,
    enhanced, colframe=blue!30!black,
    colback=red!10, boxrule=1pt,
    #1}

\usepackage{unixode}


\DeclareMathOperator{\ord}{ord}
\DeclareMathOperator{\orb}{orb}
\DeclareMathOperator{\stab}{stab}
\DeclareMathOperator{\Stab}{stab}
\DeclareMathOperator{\ppcm}{ppcm}
\DeclareMathOperator{\conj}{Conj}
\DeclareMathOperator{\End}{End}
\DeclareMathOperator{\rot}{rot}
\DeclareMathOperator{\trs}{trace}
\DeclareMathOperator{\Ind}{Ind}
\DeclareMathOperator{\mat}{Mat}
\DeclareMathOperator{\id}{Id}
\DeclareMathOperator{\vect}{vect}
\DeclareMathOperator{\img}{img}
\DeclareMathOperator{\cov}{Cov}
\DeclareMathOperator{\dist}{dist}
\DeclareMathOperator{\irr}{Irr}
\DeclareMathOperator{\image}{Im}
\DeclareMathOperator{\pd}{\partial}
\DeclareMathOperator{\epi}{epi}
\DeclareMathOperator{\Argmin}{Argmin}
\DeclareMathOperator{\dom}{dom}
\DeclareMathOperator{\proj}{proj}
\DeclareMathOperator{\ctg}{ctg}
\DeclareMathOperator{\supp}{supp}
\DeclareMathOperator{\argmin}{argmin}
\DeclareMathOperator{\mult}{mult}
\DeclareMathOperator{\ch}{ch}
\DeclareMathOperator{\sh}{sh}
\DeclareMathOperator{\rang}{rang}
\DeclareMathOperator{\diam}{diam}
\DeclareMathOperator{\Epigraphe}{Epigraphe}




\usepackage{xcolor}
\everymath{\color{blue}}
%\everymath{\color[rgb]{0,1,1}}
%\pagecolor[rgb]{0,0,0.5}


\newcommand*{\pdtest}[3][]{\ensuremath{\frac{\partial^{#1} #2}{\partial #3}}}

\newcommand*{\deffunc}[6][]{\ensuremath{
\begin{array}{rcl}
#2 : #3 &\rightarrow& #4\\
#5 &\mapsto& #6
\end{array}
}}

\newcommand{\eqcolon}{\mathrel{\resizebox{\widthof{$\mathord{=}$}}{\height}{ $\!\!=\!\!\resizebox{1.2\width}{0.8\height}{\raisebox{0.23ex}{$\mathop{:}$}}\!\!$ }}}
\newcommand{\coloneq}{\mathrel{\resizebox{\widthof{$\mathord{=}$}}{\height}{ $\!\!\resizebox{1.2\width}{0.8\height}{\raisebox{0.23ex}{$\mathop{:}$}}\!\!=\!\!$ }}}
\newcommand{\eqcolonl}{\ensuremath{\mathrel{=\!\!\mathop{:}}}}
\newcommand{\coloneql}{\ensuremath{\mathrel{\mathop{:} \!\! =}}}
\newcommand{\vc}[1]{% inline column vector
  \left(\begin{smallmatrix}#1\end{smallmatrix}\right)%
}
\newcommand{\vr}[1]{% inline row vector
  \begin{smallmatrix}(\,#1\,)\end{smallmatrix}%
}
\makeatletter
\newcommand*{\defeq}{\ =\mathrel{\rlap{%
                     \raisebox{0.3ex}{$\m@th\cdot$}}%
                     \raisebox{-0.3ex}{$\m@th\cdot$}}%
                     }
\makeatother

\newcommand{\mathcircle}[1]{% inline row vector
 \overset{\circ}{#1}
}
\newcommand{\ulim}{% low limit
 \underline{\lim}
}
\newcommand{\ssi}{% iff
\iff
}
\newcommand{\ps}[2]{
\expval{#1 | #2}
}
\newcommand{\df}[1]{
\mqty{#1}
}
\newcommand{\n}[1]{
\norm{#1}
}
\newcommand{\sys}[1]{
\left\{\smqty{#1}\right.
}


\newcommand{\eqdef}{\ensuremath{\overset{\text{def}}=}}


\def\Circlearrowright{\ensuremath{%
  \rotatebox[origin=c]{230}{$\circlearrowright$}}}

\newcommand\ct[1]{\text{\rmfamily\upshape #1}}
\newcommand\question[1]{ {\color{red} ...!? \small #1}}
\newcommand\caz[1]{\left\{\begin{array} #1 \end{array}\right.}
\newcommand\const{\text{\rmfamily\upshape const}}
\newcommand\toP{ \overset{\pro}{\to}}
\newcommand\toPP{ \overset{\text{PP}}{\to}}
\newcommand{\oeq}{\mathrel{\text{\textcircled{$=$}}}}





\usepackage{xcolor}
% \usepackage[normalem]{ulem}
\usepackage{lipsum}
\makeatletter
% \newcommand\colorwave[1][blue]{\bgroup \markoverwith{\lower3.5\p@\hbox{\sixly \textcolor{#1}{\char58}}}\ULon}
%\font\sixly=lasy6 % does not re-load if already loaded, so no memory problem.

\newmdtheoremenv[
linewidth= 1pt,linecolor= blue,%
leftmargin=20,rightmargin=20,innertopmargin=0pt, innerrightmargin=40,%
tikzsetting = { draw=lightgray, line width = 0.3pt,dashed,%
dash pattern = on 15pt off 3pt},%
splittopskip=\topskip,skipbelow=\baselineskip,%
skipabove=\baselineskip,ntheorem,roundcorner=0pt,
% backgroundcolor=pagebg,font=\color{orange}\sffamily, fontcolor=white
]{examplebox}{Exemple}[section]



\newcommand\R{\mathbb{R}}
\newcommand\Z{\mathbb{Z}}
\newcommand\N{\mathbb{N}}
\newcommand\E{\mathbb{E}}
\newcommand\F{\mathcal{F}}
\newcommand\cH{\mathcal{H}}
\newcommand\V{\mathbb{V}}
\newcommand\dmo{ ^{-1} }
\newcommand\kapa{\kappa}
\newcommand\im{Im}
\newcommand\hs{\mathcal{H}}





\usepackage{soul}

\makeatletter
\newcommand*{\whiten}[1]{\llap{\textcolor{white}{{\the\SOUL@token}}\hspace{#1pt}}}
\DeclareRobustCommand*\myul{%
    \def\SOUL@everyspace{\underline{\space}\kern\z@}%
    \def\SOUL@everytoken{%
     \setbox0=\hbox{\the\SOUL@token}%
     \ifdim\dp0>\z@
        \raisebox{\dp0}{\underline{\phantom{\the\SOUL@token}}}%
        \whiten{1}\whiten{0}%
        \whiten{-1}\whiten{-2}%
        \llap{\the\SOUL@token}%
     \else
        \underline{\the\SOUL@token}%
     \fi}%
\SOUL@}
\makeatother

\newcommand*{\demp}{\fontfamily{lmtt}\selectfont}

\DeclareTextFontCommand{\textdemp}{\demp}

\begin{document}

\ifcomment
Multiline
comment
\fi
\ifcomment
\myul{Typesetting test}
% \color[rgb]{1,1,1}
$∑_i^n≠ 60º±∞π∆¬≈√j∫h≤≥µ$

$\CR \R\pro\ind\pro\gS\pro
\mqty[a&b\\c&d]$
$\pro\mathbb{P}$
$\dd{x}$

  \[
    \alpha(x)=\left\{
                \begin{array}{ll}
                  x\\
                  \frac{1}{1+e^{-kx}}\\
                  \frac{e^x-e^{-x}}{e^x+e^{-x}}
                \end{array}
              \right.
  \]

  $\expval{x}$
  
  $\chi_\rho(ghg\dmo)=\Tr(\rho_{ghg\dmo})=\Tr(\rho_g\circ\rho_h\circ\rho\dmo_g)=\Tr(\rho_h)\overset{\mbox{\scalebox{0.5}{$\Tr(AB)=\Tr(BA)$}}}{=}\chi_\rho(h)$
  	$\mathop{\oplus}_{\substack{x\in X}}$

$\mat(\rho_g)=(a_{ij}(g))_{\scriptsize \substack{1\leq i\leq d \\ 1\leq j\leq d}}$ et $\mat(\rho'_g)=(a'_{ij}(g))_{\scriptsize \substack{1\leq i'\leq d' \\ 1\leq j'\leq d'}}$



\[\int_a^b{\mathbb{R}^2}g(u, v)\dd{P_{XY}}(u, v)=\iint g(u,v) f_{XY}(u, v)\dd \lambda(u) \dd \lambda(v)\]
$$\lim_{x\to\infty} f(x)$$	
$$\iiiint_V \mu(t,u,v,w) \,dt\,du\,dv\,dw$$
$$\sum_{n=1}^{\infty} 2^{-n} = 1$$	
\begin{definition}
	Si $X$ et $Y$ sont 2 v.a. ou definit la \textsc{Covariance} entre $X$ et $Y$ comme
	$\cov(X,Y)\overset{\text{def}}{=}\E\left[(X-\E(X))(Y-\E(Y))\right]=\E(XY)-\E(X)\E(Y)$.
\end{definition}
\fi
\pagebreak

% \tableofcontents

% insert your code here
%\input{./algebra/main.tex}
%\input{./geometrie-differentielle/main.tex}
%\input{./probabilite/main.tex}
%\input{./analyse-fonctionnelle/main.tex}
% \input{./Analyse-convexe-et-dualite-en-optimisation/main.tex}
%\input{./tikz/main.tex}
%\input{./Theorie-du-distributions/main.tex}
%\input{./optimisation/mine.tex}
 \input{./modelisation/main.tex}

% yves.aubry@univ-tln.fr : algebra

\end{document}

%\input{./optimisation/mine.tex}
 % !TEX encoding = UTF-8 Unicode
% !TEX TS-program = xelatex

\documentclass[french]{report}

%\usepackage[utf8]{inputenc}
%\usepackage[T1]{fontenc}
\usepackage{babel}


\newif\ifcomment
%\commenttrue # Show comments

\usepackage{physics}
\usepackage{amssymb}


\usepackage{amsthm}
% \usepackage{thmtools}
\usepackage{mathtools}
\usepackage{amsfonts}

\usepackage{color}

\usepackage{tikz}

\usepackage{geometry}
\geometry{a5paper, margin=0.1in, right=1cm}

\usepackage{dsfont}

\usepackage{graphicx}
\graphicspath{ {images/} }

\usepackage{faktor}

\usepackage{IEEEtrantools}
\usepackage{enumerate}   
\usepackage[PostScript=dvips]{"/Users/aware/Documents/Courses/diagrams"}


\newtheorem{theorem}{Théorème}[section]
\renewcommand{\thetheorem}{\arabic{theorem}}
\newtheorem{lemme}{Lemme}[section]
\renewcommand{\thelemme}{\arabic{lemme}}
\newtheorem{proposition}{Proposition}[section]
\renewcommand{\theproposition}{\arabic{proposition}}
\newtheorem{notations}{Notations}[section]
\newtheorem{problem}{Problème}[section]
\newtheorem{corollary}{Corollaire}[theorem]
\renewcommand{\thecorollary}{\arabic{corollary}}
\newtheorem{property}{Propriété}[section]
\newtheorem{objective}{Objectif}[section]

\theoremstyle{definition}
\newtheorem{definition}{Définition}[section]
\renewcommand{\thedefinition}{\arabic{definition}}
\newtheorem{exercise}{Exercice}[chapter]
\renewcommand{\theexercise}{\arabic{exercise}}
\newtheorem{example}{Exemple}[chapter]
\renewcommand{\theexample}{\arabic{example}}
\newtheorem*{solution}{Solution}
\newtheorem*{application}{Application}
\newtheorem*{notation}{Notation}
\newtheorem*{vocabulary}{Vocabulaire}
\newtheorem*{properties}{Propriétés}



\theoremstyle{remark}
\newtheorem*{remark}{Remarque}
\newtheorem*{rappel}{Rappel}


\usepackage{etoolbox}
\AtBeginEnvironment{exercise}{\small}
\AtBeginEnvironment{example}{\small}

\usepackage{cases}
\usepackage[red]{mypack}

\usepackage[framemethod=TikZ]{mdframed}

\definecolor{bg}{rgb}{0.4,0.25,0.95}
\definecolor{pagebg}{rgb}{0,0,0.5}
\surroundwithmdframed[
   topline=false,
   rightline=false,
   bottomline=false,
   leftmargin=\parindent,
   skipabove=8pt,
   skipbelow=8pt,
   linecolor=blue,
   innerbottommargin=10pt,
   % backgroundcolor=bg,font=\color{orange}\sffamily, fontcolor=white
]{definition}

\usepackage{empheq}
\usepackage[most]{tcolorbox}

\newtcbox{\mymath}[1][]{%
    nobeforeafter, math upper, tcbox raise base,
    enhanced, colframe=blue!30!black,
    colback=red!10, boxrule=1pt,
    #1}

\usepackage{unixode}


\DeclareMathOperator{\ord}{ord}
\DeclareMathOperator{\orb}{orb}
\DeclareMathOperator{\stab}{stab}
\DeclareMathOperator{\Stab}{stab}
\DeclareMathOperator{\ppcm}{ppcm}
\DeclareMathOperator{\conj}{Conj}
\DeclareMathOperator{\End}{End}
\DeclareMathOperator{\rot}{rot}
\DeclareMathOperator{\trs}{trace}
\DeclareMathOperator{\Ind}{Ind}
\DeclareMathOperator{\mat}{Mat}
\DeclareMathOperator{\id}{Id}
\DeclareMathOperator{\vect}{vect}
\DeclareMathOperator{\img}{img}
\DeclareMathOperator{\cov}{Cov}
\DeclareMathOperator{\dist}{dist}
\DeclareMathOperator{\irr}{Irr}
\DeclareMathOperator{\image}{Im}
\DeclareMathOperator{\pd}{\partial}
\DeclareMathOperator{\epi}{epi}
\DeclareMathOperator{\Argmin}{Argmin}
\DeclareMathOperator{\dom}{dom}
\DeclareMathOperator{\proj}{proj}
\DeclareMathOperator{\ctg}{ctg}
\DeclareMathOperator{\supp}{supp}
\DeclareMathOperator{\argmin}{argmin}
\DeclareMathOperator{\mult}{mult}
\DeclareMathOperator{\ch}{ch}
\DeclareMathOperator{\sh}{sh}
\DeclareMathOperator{\rang}{rang}
\DeclareMathOperator{\diam}{diam}
\DeclareMathOperator{\Epigraphe}{Epigraphe}




\usepackage{xcolor}
\everymath{\color{blue}}
%\everymath{\color[rgb]{0,1,1}}
%\pagecolor[rgb]{0,0,0.5}


\newcommand*{\pdtest}[3][]{\ensuremath{\frac{\partial^{#1} #2}{\partial #3}}}

\newcommand*{\deffunc}[6][]{\ensuremath{
\begin{array}{rcl}
#2 : #3 &\rightarrow& #4\\
#5 &\mapsto& #6
\end{array}
}}

\newcommand{\eqcolon}{\mathrel{\resizebox{\widthof{$\mathord{=}$}}{\height}{ $\!\!=\!\!\resizebox{1.2\width}{0.8\height}{\raisebox{0.23ex}{$\mathop{:}$}}\!\!$ }}}
\newcommand{\coloneq}{\mathrel{\resizebox{\widthof{$\mathord{=}$}}{\height}{ $\!\!\resizebox{1.2\width}{0.8\height}{\raisebox{0.23ex}{$\mathop{:}$}}\!\!=\!\!$ }}}
\newcommand{\eqcolonl}{\ensuremath{\mathrel{=\!\!\mathop{:}}}}
\newcommand{\coloneql}{\ensuremath{\mathrel{\mathop{:} \!\! =}}}
\newcommand{\vc}[1]{% inline column vector
  \left(\begin{smallmatrix}#1\end{smallmatrix}\right)%
}
\newcommand{\vr}[1]{% inline row vector
  \begin{smallmatrix}(\,#1\,)\end{smallmatrix}%
}
\makeatletter
\newcommand*{\defeq}{\ =\mathrel{\rlap{%
                     \raisebox{0.3ex}{$\m@th\cdot$}}%
                     \raisebox{-0.3ex}{$\m@th\cdot$}}%
                     }
\makeatother

\newcommand{\mathcircle}[1]{% inline row vector
 \overset{\circ}{#1}
}
\newcommand{\ulim}{% low limit
 \underline{\lim}
}
\newcommand{\ssi}{% iff
\iff
}
\newcommand{\ps}[2]{
\expval{#1 | #2}
}
\newcommand{\df}[1]{
\mqty{#1}
}
\newcommand{\n}[1]{
\norm{#1}
}
\newcommand{\sys}[1]{
\left\{\smqty{#1}\right.
}


\newcommand{\eqdef}{\ensuremath{\overset{\text{def}}=}}


\def\Circlearrowright{\ensuremath{%
  \rotatebox[origin=c]{230}{$\circlearrowright$}}}

\newcommand\ct[1]{\text{\rmfamily\upshape #1}}
\newcommand\question[1]{ {\color{red} ...!? \small #1}}
\newcommand\caz[1]{\left\{\begin{array} #1 \end{array}\right.}
\newcommand\const{\text{\rmfamily\upshape const}}
\newcommand\toP{ \overset{\pro}{\to}}
\newcommand\toPP{ \overset{\text{PP}}{\to}}
\newcommand{\oeq}{\mathrel{\text{\textcircled{$=$}}}}





\usepackage{xcolor}
% \usepackage[normalem]{ulem}
\usepackage{lipsum}
\makeatletter
% \newcommand\colorwave[1][blue]{\bgroup \markoverwith{\lower3.5\p@\hbox{\sixly \textcolor{#1}{\char58}}}\ULon}
%\font\sixly=lasy6 % does not re-load if already loaded, so no memory problem.

\newmdtheoremenv[
linewidth= 1pt,linecolor= blue,%
leftmargin=20,rightmargin=20,innertopmargin=0pt, innerrightmargin=40,%
tikzsetting = { draw=lightgray, line width = 0.3pt,dashed,%
dash pattern = on 15pt off 3pt},%
splittopskip=\topskip,skipbelow=\baselineskip,%
skipabove=\baselineskip,ntheorem,roundcorner=0pt,
% backgroundcolor=pagebg,font=\color{orange}\sffamily, fontcolor=white
]{examplebox}{Exemple}[section]



\newcommand\R{\mathbb{R}}
\newcommand\Z{\mathbb{Z}}
\newcommand\N{\mathbb{N}}
\newcommand\E{\mathbb{E}}
\newcommand\F{\mathcal{F}}
\newcommand\cH{\mathcal{H}}
\newcommand\V{\mathbb{V}}
\newcommand\dmo{ ^{-1} }
\newcommand\kapa{\kappa}
\newcommand\im{Im}
\newcommand\hs{\mathcal{H}}





\usepackage{soul}

\makeatletter
\newcommand*{\whiten}[1]{\llap{\textcolor{white}{{\the\SOUL@token}}\hspace{#1pt}}}
\DeclareRobustCommand*\myul{%
    \def\SOUL@everyspace{\underline{\space}\kern\z@}%
    \def\SOUL@everytoken{%
     \setbox0=\hbox{\the\SOUL@token}%
     \ifdim\dp0>\z@
        \raisebox{\dp0}{\underline{\phantom{\the\SOUL@token}}}%
        \whiten{1}\whiten{0}%
        \whiten{-1}\whiten{-2}%
        \llap{\the\SOUL@token}%
     \else
        \underline{\the\SOUL@token}%
     \fi}%
\SOUL@}
\makeatother

\newcommand*{\demp}{\fontfamily{lmtt}\selectfont}

\DeclareTextFontCommand{\textdemp}{\demp}

\begin{document}

\ifcomment
Multiline
comment
\fi
\ifcomment
\myul{Typesetting test}
% \color[rgb]{1,1,1}
$∑_i^n≠ 60º±∞π∆¬≈√j∫h≤≥µ$

$\CR \R\pro\ind\pro\gS\pro
\mqty[a&b\\c&d]$
$\pro\mathbb{P}$
$\dd{x}$

  \[
    \alpha(x)=\left\{
                \begin{array}{ll}
                  x\\
                  \frac{1}{1+e^{-kx}}\\
                  \frac{e^x-e^{-x}}{e^x+e^{-x}}
                \end{array}
              \right.
  \]

  $\expval{x}$
  
  $\chi_\rho(ghg\dmo)=\Tr(\rho_{ghg\dmo})=\Tr(\rho_g\circ\rho_h\circ\rho\dmo_g)=\Tr(\rho_h)\overset{\mbox{\scalebox{0.5}{$\Tr(AB)=\Tr(BA)$}}}{=}\chi_\rho(h)$
  	$\mathop{\oplus}_{\substack{x\in X}}$

$\mat(\rho_g)=(a_{ij}(g))_{\scriptsize \substack{1\leq i\leq d \\ 1\leq j\leq d}}$ et $\mat(\rho'_g)=(a'_{ij}(g))_{\scriptsize \substack{1\leq i'\leq d' \\ 1\leq j'\leq d'}}$



\[\int_a^b{\mathbb{R}^2}g(u, v)\dd{P_{XY}}(u, v)=\iint g(u,v) f_{XY}(u, v)\dd \lambda(u) \dd \lambda(v)\]
$$\lim_{x\to\infty} f(x)$$	
$$\iiiint_V \mu(t,u,v,w) \,dt\,du\,dv\,dw$$
$$\sum_{n=1}^{\infty} 2^{-n} = 1$$	
\begin{definition}
	Si $X$ et $Y$ sont 2 v.a. ou definit la \textsc{Covariance} entre $X$ et $Y$ comme
	$\cov(X,Y)\overset{\text{def}}{=}\E\left[(X-\E(X))(Y-\E(Y))\right]=\E(XY)-\E(X)\E(Y)$.
\end{definition}
\fi
\pagebreak

% \tableofcontents

% insert your code here
%\input{./algebra/main.tex}
%\input{./geometrie-differentielle/main.tex}
%\input{./probabilite/main.tex}
%\input{./analyse-fonctionnelle/main.tex}
% \input{./Analyse-convexe-et-dualite-en-optimisation/main.tex}
%\input{./tikz/main.tex}
%\input{./Theorie-du-distributions/main.tex}
%\input{./optimisation/mine.tex}
 \input{./modelisation/main.tex}

% yves.aubry@univ-tln.fr : algebra

\end{document}


% yves.aubry@univ-tln.fr : algebra

\end{document}

%% !TEX encoding = UTF-8 Unicode
% !TEX TS-program = xelatex

\documentclass[french]{report}

%\usepackage[utf8]{inputenc}
%\usepackage[T1]{fontenc}
\usepackage{babel}


\newif\ifcomment
%\commenttrue # Show comments

\usepackage{physics}
\usepackage{amssymb}


\usepackage{amsthm}
% \usepackage{thmtools}
\usepackage{mathtools}
\usepackage{amsfonts}

\usepackage{color}

\usepackage{tikz}

\usepackage{geometry}
\geometry{a5paper, margin=0.1in, right=1cm}

\usepackage{dsfont}

\usepackage{graphicx}
\graphicspath{ {images/} }

\usepackage{faktor}

\usepackage{IEEEtrantools}
\usepackage{enumerate}   
\usepackage[PostScript=dvips]{"/Users/aware/Documents/Courses/diagrams"}


\newtheorem{theorem}{Théorème}[section]
\renewcommand{\thetheorem}{\arabic{theorem}}
\newtheorem{lemme}{Lemme}[section]
\renewcommand{\thelemme}{\arabic{lemme}}
\newtheorem{proposition}{Proposition}[section]
\renewcommand{\theproposition}{\arabic{proposition}}
\newtheorem{notations}{Notations}[section]
\newtheorem{problem}{Problème}[section]
\newtheorem{corollary}{Corollaire}[theorem]
\renewcommand{\thecorollary}{\arabic{corollary}}
\newtheorem{property}{Propriété}[section]
\newtheorem{objective}{Objectif}[section]

\theoremstyle{definition}
\newtheorem{definition}{Définition}[section]
\renewcommand{\thedefinition}{\arabic{definition}}
\newtheorem{exercise}{Exercice}[chapter]
\renewcommand{\theexercise}{\arabic{exercise}}
\newtheorem{example}{Exemple}[chapter]
\renewcommand{\theexample}{\arabic{example}}
\newtheorem*{solution}{Solution}
\newtheorem*{application}{Application}
\newtheorem*{notation}{Notation}
\newtheorem*{vocabulary}{Vocabulaire}
\newtheorem*{properties}{Propriétés}



\theoremstyle{remark}
\newtheorem*{remark}{Remarque}
\newtheorem*{rappel}{Rappel}


\usepackage{etoolbox}
\AtBeginEnvironment{exercise}{\small}
\AtBeginEnvironment{example}{\small}

\usepackage{cases}
\usepackage[red]{mypack}

\usepackage[framemethod=TikZ]{mdframed}

\definecolor{bg}{rgb}{0.4,0.25,0.95}
\definecolor{pagebg}{rgb}{0,0,0.5}
\surroundwithmdframed[
   topline=false,
   rightline=false,
   bottomline=false,
   leftmargin=\parindent,
   skipabove=8pt,
   skipbelow=8pt,
   linecolor=blue,
   innerbottommargin=10pt,
   % backgroundcolor=bg,font=\color{orange}\sffamily, fontcolor=white
]{definition}

\usepackage{empheq}
\usepackage[most]{tcolorbox}

\newtcbox{\mymath}[1][]{%
    nobeforeafter, math upper, tcbox raise base,
    enhanced, colframe=blue!30!black,
    colback=red!10, boxrule=1pt,
    #1}

\usepackage{unixode}


\DeclareMathOperator{\ord}{ord}
\DeclareMathOperator{\orb}{orb}
\DeclareMathOperator{\stab}{stab}
\DeclareMathOperator{\Stab}{stab}
\DeclareMathOperator{\ppcm}{ppcm}
\DeclareMathOperator{\conj}{Conj}
\DeclareMathOperator{\End}{End}
\DeclareMathOperator{\rot}{rot}
\DeclareMathOperator{\trs}{trace}
\DeclareMathOperator{\Ind}{Ind}
\DeclareMathOperator{\mat}{Mat}
\DeclareMathOperator{\id}{Id}
\DeclareMathOperator{\vect}{vect}
\DeclareMathOperator{\img}{img}
\DeclareMathOperator{\cov}{Cov}
\DeclareMathOperator{\dist}{dist}
\DeclareMathOperator{\irr}{Irr}
\DeclareMathOperator{\image}{Im}
\DeclareMathOperator{\pd}{\partial}
\DeclareMathOperator{\epi}{epi}
\DeclareMathOperator{\Argmin}{Argmin}
\DeclareMathOperator{\dom}{dom}
\DeclareMathOperator{\proj}{proj}
\DeclareMathOperator{\ctg}{ctg}
\DeclareMathOperator{\supp}{supp}
\DeclareMathOperator{\argmin}{argmin}
\DeclareMathOperator{\mult}{mult}
\DeclareMathOperator{\ch}{ch}
\DeclareMathOperator{\sh}{sh}
\DeclareMathOperator{\rang}{rang}
\DeclareMathOperator{\diam}{diam}
\DeclareMathOperator{\Epigraphe}{Epigraphe}




\usepackage{xcolor}
\everymath{\color{blue}}
%\everymath{\color[rgb]{0,1,1}}
%\pagecolor[rgb]{0,0,0.5}


\newcommand*{\pdtest}[3][]{\ensuremath{\frac{\partial^{#1} #2}{\partial #3}}}

\newcommand*{\deffunc}[6][]{\ensuremath{
\begin{array}{rcl}
#2 : #3 &\rightarrow& #4\\
#5 &\mapsto& #6
\end{array}
}}

\newcommand{\eqcolon}{\mathrel{\resizebox{\widthof{$\mathord{=}$}}{\height}{ $\!\!=\!\!\resizebox{1.2\width}{0.8\height}{\raisebox{0.23ex}{$\mathop{:}$}}\!\!$ }}}
\newcommand{\coloneq}{\mathrel{\resizebox{\widthof{$\mathord{=}$}}{\height}{ $\!\!\resizebox{1.2\width}{0.8\height}{\raisebox{0.23ex}{$\mathop{:}$}}\!\!=\!\!$ }}}
\newcommand{\eqcolonl}{\ensuremath{\mathrel{=\!\!\mathop{:}}}}
\newcommand{\coloneql}{\ensuremath{\mathrel{\mathop{:} \!\! =}}}
\newcommand{\vc}[1]{% inline column vector
  \left(\begin{smallmatrix}#1\end{smallmatrix}\right)%
}
\newcommand{\vr}[1]{% inline row vector
  \begin{smallmatrix}(\,#1\,)\end{smallmatrix}%
}
\makeatletter
\newcommand*{\defeq}{\ =\mathrel{\rlap{%
                     \raisebox{0.3ex}{$\m@th\cdot$}}%
                     \raisebox{-0.3ex}{$\m@th\cdot$}}%
                     }
\makeatother

\newcommand{\mathcircle}[1]{% inline row vector
 \overset{\circ}{#1}
}
\newcommand{\ulim}{% low limit
 \underline{\lim}
}
\newcommand{\ssi}{% iff
\iff
}
\newcommand{\ps}[2]{
\expval{#1 | #2}
}
\newcommand{\df}[1]{
\mqty{#1}
}
\newcommand{\n}[1]{
\norm{#1}
}
\newcommand{\sys}[1]{
\left\{\smqty{#1}\right.
}


\newcommand{\eqdef}{\ensuremath{\overset{\text{def}}=}}


\def\Circlearrowright{\ensuremath{%
  \rotatebox[origin=c]{230}{$\circlearrowright$}}}

\newcommand\ct[1]{\text{\rmfamily\upshape #1}}
\newcommand\question[1]{ {\color{red} ...!? \small #1}}
\newcommand\caz[1]{\left\{\begin{array} #1 \end{array}\right.}
\newcommand\const{\text{\rmfamily\upshape const}}
\newcommand\toP{ \overset{\pro}{\to}}
\newcommand\toPP{ \overset{\text{PP}}{\to}}
\newcommand{\oeq}{\mathrel{\text{\textcircled{$=$}}}}





\usepackage{xcolor}
% \usepackage[normalem]{ulem}
\usepackage{lipsum}
\makeatletter
% \newcommand\colorwave[1][blue]{\bgroup \markoverwith{\lower3.5\p@\hbox{\sixly \textcolor{#1}{\char58}}}\ULon}
%\font\sixly=lasy6 % does not re-load if already loaded, so no memory problem.

\newmdtheoremenv[
linewidth= 1pt,linecolor= blue,%
leftmargin=20,rightmargin=20,innertopmargin=0pt, innerrightmargin=40,%
tikzsetting = { draw=lightgray, line width = 0.3pt,dashed,%
dash pattern = on 15pt off 3pt},%
splittopskip=\topskip,skipbelow=\baselineskip,%
skipabove=\baselineskip,ntheorem,roundcorner=0pt,
% backgroundcolor=pagebg,font=\color{orange}\sffamily, fontcolor=white
]{examplebox}{Exemple}[section]



\newcommand\R{\mathbb{R}}
\newcommand\Z{\mathbb{Z}}
\newcommand\N{\mathbb{N}}
\newcommand\E{\mathbb{E}}
\newcommand\F{\mathcal{F}}
\newcommand\cH{\mathcal{H}}
\newcommand\V{\mathbb{V}}
\newcommand\dmo{ ^{-1} }
\newcommand\kapa{\kappa}
\newcommand\im{Im}
\newcommand\hs{\mathcal{H}}





\usepackage{soul}

\makeatletter
\newcommand*{\whiten}[1]{\llap{\textcolor{white}{{\the\SOUL@token}}\hspace{#1pt}}}
\DeclareRobustCommand*\myul{%
    \def\SOUL@everyspace{\underline{\space}\kern\z@}%
    \def\SOUL@everytoken{%
     \setbox0=\hbox{\the\SOUL@token}%
     \ifdim\dp0>\z@
        \raisebox{\dp0}{\underline{\phantom{\the\SOUL@token}}}%
        \whiten{1}\whiten{0}%
        \whiten{-1}\whiten{-2}%
        \llap{\the\SOUL@token}%
     \else
        \underline{\the\SOUL@token}%
     \fi}%
\SOUL@}
\makeatother

\newcommand*{\demp}{\fontfamily{lmtt}\selectfont}

\DeclareTextFontCommand{\textdemp}{\demp}

\begin{document}

\ifcomment
Multiline
comment
\fi
\ifcomment
\myul{Typesetting test}
% \color[rgb]{1,1,1}
$∑_i^n≠ 60º±∞π∆¬≈√j∫h≤≥µ$

$\CR \R\pro\ind\pro\gS\pro
\mqty[a&b\\c&d]$
$\pro\mathbb{P}$
$\dd{x}$

  \[
    \alpha(x)=\left\{
                \begin{array}{ll}
                  x\\
                  \frac{1}{1+e^{-kx}}\\
                  \frac{e^x-e^{-x}}{e^x+e^{-x}}
                \end{array}
              \right.
  \]

  $\expval{x}$
  
  $\chi_\rho(ghg\dmo)=\Tr(\rho_{ghg\dmo})=\Tr(\rho_g\circ\rho_h\circ\rho\dmo_g)=\Tr(\rho_h)\overset{\mbox{\scalebox{0.5}{$\Tr(AB)=\Tr(BA)$}}}{=}\chi_\rho(h)$
  	$\mathop{\oplus}_{\substack{x\in X}}$

$\mat(\rho_g)=(a_{ij}(g))_{\scriptsize \substack{1\leq i\leq d \\ 1\leq j\leq d}}$ et $\mat(\rho'_g)=(a'_{ij}(g))_{\scriptsize \substack{1\leq i'\leq d' \\ 1\leq j'\leq d'}}$



\[\int_a^b{\mathbb{R}^2}g(u, v)\dd{P_{XY}}(u, v)=\iint g(u,v) f_{XY}(u, v)\dd \lambda(u) \dd \lambda(v)\]
$$\lim_{x\to\infty} f(x)$$	
$$\iiiint_V \mu(t,u,v,w) \,dt\,du\,dv\,dw$$
$$\sum_{n=1}^{\infty} 2^{-n} = 1$$	
\begin{definition}
	Si $X$ et $Y$ sont 2 v.a. ou definit la \textsc{Covariance} entre $X$ et $Y$ comme
	$\cov(X,Y)\overset{\text{def}}{=}\E\left[(X-\E(X))(Y-\E(Y))\right]=\E(XY)-\E(X)\E(Y)$.
\end{definition}
\fi
\pagebreak

% \tableofcontents

% insert your code here
%% !TEX encoding = UTF-8 Unicode
% !TEX TS-program = xelatex

\documentclass[french]{report}

%\usepackage[utf8]{inputenc}
%\usepackage[T1]{fontenc}
\usepackage{babel}


\newif\ifcomment
%\commenttrue # Show comments

\usepackage{physics}
\usepackage{amssymb}


\usepackage{amsthm}
% \usepackage{thmtools}
\usepackage{mathtools}
\usepackage{amsfonts}

\usepackage{color}

\usepackage{tikz}

\usepackage{geometry}
\geometry{a5paper, margin=0.1in, right=1cm}

\usepackage{dsfont}

\usepackage{graphicx}
\graphicspath{ {images/} }

\usepackage{faktor}

\usepackage{IEEEtrantools}
\usepackage{enumerate}   
\usepackage[PostScript=dvips]{"/Users/aware/Documents/Courses/diagrams"}


\newtheorem{theorem}{Théorème}[section]
\renewcommand{\thetheorem}{\arabic{theorem}}
\newtheorem{lemme}{Lemme}[section]
\renewcommand{\thelemme}{\arabic{lemme}}
\newtheorem{proposition}{Proposition}[section]
\renewcommand{\theproposition}{\arabic{proposition}}
\newtheorem{notations}{Notations}[section]
\newtheorem{problem}{Problème}[section]
\newtheorem{corollary}{Corollaire}[theorem]
\renewcommand{\thecorollary}{\arabic{corollary}}
\newtheorem{property}{Propriété}[section]
\newtheorem{objective}{Objectif}[section]

\theoremstyle{definition}
\newtheorem{definition}{Définition}[section]
\renewcommand{\thedefinition}{\arabic{definition}}
\newtheorem{exercise}{Exercice}[chapter]
\renewcommand{\theexercise}{\arabic{exercise}}
\newtheorem{example}{Exemple}[chapter]
\renewcommand{\theexample}{\arabic{example}}
\newtheorem*{solution}{Solution}
\newtheorem*{application}{Application}
\newtheorem*{notation}{Notation}
\newtheorem*{vocabulary}{Vocabulaire}
\newtheorem*{properties}{Propriétés}



\theoremstyle{remark}
\newtheorem*{remark}{Remarque}
\newtheorem*{rappel}{Rappel}


\usepackage{etoolbox}
\AtBeginEnvironment{exercise}{\small}
\AtBeginEnvironment{example}{\small}

\usepackage{cases}
\usepackage[red]{mypack}

\usepackage[framemethod=TikZ]{mdframed}

\definecolor{bg}{rgb}{0.4,0.25,0.95}
\definecolor{pagebg}{rgb}{0,0,0.5}
\surroundwithmdframed[
   topline=false,
   rightline=false,
   bottomline=false,
   leftmargin=\parindent,
   skipabove=8pt,
   skipbelow=8pt,
   linecolor=blue,
   innerbottommargin=10pt,
   % backgroundcolor=bg,font=\color{orange}\sffamily, fontcolor=white
]{definition}

\usepackage{empheq}
\usepackage[most]{tcolorbox}

\newtcbox{\mymath}[1][]{%
    nobeforeafter, math upper, tcbox raise base,
    enhanced, colframe=blue!30!black,
    colback=red!10, boxrule=1pt,
    #1}

\usepackage{unixode}


\DeclareMathOperator{\ord}{ord}
\DeclareMathOperator{\orb}{orb}
\DeclareMathOperator{\stab}{stab}
\DeclareMathOperator{\Stab}{stab}
\DeclareMathOperator{\ppcm}{ppcm}
\DeclareMathOperator{\conj}{Conj}
\DeclareMathOperator{\End}{End}
\DeclareMathOperator{\rot}{rot}
\DeclareMathOperator{\trs}{trace}
\DeclareMathOperator{\Ind}{Ind}
\DeclareMathOperator{\mat}{Mat}
\DeclareMathOperator{\id}{Id}
\DeclareMathOperator{\vect}{vect}
\DeclareMathOperator{\img}{img}
\DeclareMathOperator{\cov}{Cov}
\DeclareMathOperator{\dist}{dist}
\DeclareMathOperator{\irr}{Irr}
\DeclareMathOperator{\image}{Im}
\DeclareMathOperator{\pd}{\partial}
\DeclareMathOperator{\epi}{epi}
\DeclareMathOperator{\Argmin}{Argmin}
\DeclareMathOperator{\dom}{dom}
\DeclareMathOperator{\proj}{proj}
\DeclareMathOperator{\ctg}{ctg}
\DeclareMathOperator{\supp}{supp}
\DeclareMathOperator{\argmin}{argmin}
\DeclareMathOperator{\mult}{mult}
\DeclareMathOperator{\ch}{ch}
\DeclareMathOperator{\sh}{sh}
\DeclareMathOperator{\rang}{rang}
\DeclareMathOperator{\diam}{diam}
\DeclareMathOperator{\Epigraphe}{Epigraphe}




\usepackage{xcolor}
\everymath{\color{blue}}
%\everymath{\color[rgb]{0,1,1}}
%\pagecolor[rgb]{0,0,0.5}


\newcommand*{\pdtest}[3][]{\ensuremath{\frac{\partial^{#1} #2}{\partial #3}}}

\newcommand*{\deffunc}[6][]{\ensuremath{
\begin{array}{rcl}
#2 : #3 &\rightarrow& #4\\
#5 &\mapsto& #6
\end{array}
}}

\newcommand{\eqcolon}{\mathrel{\resizebox{\widthof{$\mathord{=}$}}{\height}{ $\!\!=\!\!\resizebox{1.2\width}{0.8\height}{\raisebox{0.23ex}{$\mathop{:}$}}\!\!$ }}}
\newcommand{\coloneq}{\mathrel{\resizebox{\widthof{$\mathord{=}$}}{\height}{ $\!\!\resizebox{1.2\width}{0.8\height}{\raisebox{0.23ex}{$\mathop{:}$}}\!\!=\!\!$ }}}
\newcommand{\eqcolonl}{\ensuremath{\mathrel{=\!\!\mathop{:}}}}
\newcommand{\coloneql}{\ensuremath{\mathrel{\mathop{:} \!\! =}}}
\newcommand{\vc}[1]{% inline column vector
  \left(\begin{smallmatrix}#1\end{smallmatrix}\right)%
}
\newcommand{\vr}[1]{% inline row vector
  \begin{smallmatrix}(\,#1\,)\end{smallmatrix}%
}
\makeatletter
\newcommand*{\defeq}{\ =\mathrel{\rlap{%
                     \raisebox{0.3ex}{$\m@th\cdot$}}%
                     \raisebox{-0.3ex}{$\m@th\cdot$}}%
                     }
\makeatother

\newcommand{\mathcircle}[1]{% inline row vector
 \overset{\circ}{#1}
}
\newcommand{\ulim}{% low limit
 \underline{\lim}
}
\newcommand{\ssi}{% iff
\iff
}
\newcommand{\ps}[2]{
\expval{#1 | #2}
}
\newcommand{\df}[1]{
\mqty{#1}
}
\newcommand{\n}[1]{
\norm{#1}
}
\newcommand{\sys}[1]{
\left\{\smqty{#1}\right.
}


\newcommand{\eqdef}{\ensuremath{\overset{\text{def}}=}}


\def\Circlearrowright{\ensuremath{%
  \rotatebox[origin=c]{230}{$\circlearrowright$}}}

\newcommand\ct[1]{\text{\rmfamily\upshape #1}}
\newcommand\question[1]{ {\color{red} ...!? \small #1}}
\newcommand\caz[1]{\left\{\begin{array} #1 \end{array}\right.}
\newcommand\const{\text{\rmfamily\upshape const}}
\newcommand\toP{ \overset{\pro}{\to}}
\newcommand\toPP{ \overset{\text{PP}}{\to}}
\newcommand{\oeq}{\mathrel{\text{\textcircled{$=$}}}}





\usepackage{xcolor}
% \usepackage[normalem]{ulem}
\usepackage{lipsum}
\makeatletter
% \newcommand\colorwave[1][blue]{\bgroup \markoverwith{\lower3.5\p@\hbox{\sixly \textcolor{#1}{\char58}}}\ULon}
%\font\sixly=lasy6 % does not re-load if already loaded, so no memory problem.

\newmdtheoremenv[
linewidth= 1pt,linecolor= blue,%
leftmargin=20,rightmargin=20,innertopmargin=0pt, innerrightmargin=40,%
tikzsetting = { draw=lightgray, line width = 0.3pt,dashed,%
dash pattern = on 15pt off 3pt},%
splittopskip=\topskip,skipbelow=\baselineskip,%
skipabove=\baselineskip,ntheorem,roundcorner=0pt,
% backgroundcolor=pagebg,font=\color{orange}\sffamily, fontcolor=white
]{examplebox}{Exemple}[section]



\newcommand\R{\mathbb{R}}
\newcommand\Z{\mathbb{Z}}
\newcommand\N{\mathbb{N}}
\newcommand\E{\mathbb{E}}
\newcommand\F{\mathcal{F}}
\newcommand\cH{\mathcal{H}}
\newcommand\V{\mathbb{V}}
\newcommand\dmo{ ^{-1} }
\newcommand\kapa{\kappa}
\newcommand\im{Im}
\newcommand\hs{\mathcal{H}}





\usepackage{soul}

\makeatletter
\newcommand*{\whiten}[1]{\llap{\textcolor{white}{{\the\SOUL@token}}\hspace{#1pt}}}
\DeclareRobustCommand*\myul{%
    \def\SOUL@everyspace{\underline{\space}\kern\z@}%
    \def\SOUL@everytoken{%
     \setbox0=\hbox{\the\SOUL@token}%
     \ifdim\dp0>\z@
        \raisebox{\dp0}{\underline{\phantom{\the\SOUL@token}}}%
        \whiten{1}\whiten{0}%
        \whiten{-1}\whiten{-2}%
        \llap{\the\SOUL@token}%
     \else
        \underline{\the\SOUL@token}%
     \fi}%
\SOUL@}
\makeatother

\newcommand*{\demp}{\fontfamily{lmtt}\selectfont}

\DeclareTextFontCommand{\textdemp}{\demp}

\begin{document}

\ifcomment
Multiline
comment
\fi
\ifcomment
\myul{Typesetting test}
% \color[rgb]{1,1,1}
$∑_i^n≠ 60º±∞π∆¬≈√j∫h≤≥µ$

$\CR \R\pro\ind\pro\gS\pro
\mqty[a&b\\c&d]$
$\pro\mathbb{P}$
$\dd{x}$

  \[
    \alpha(x)=\left\{
                \begin{array}{ll}
                  x\\
                  \frac{1}{1+e^{-kx}}\\
                  \frac{e^x-e^{-x}}{e^x+e^{-x}}
                \end{array}
              \right.
  \]

  $\expval{x}$
  
  $\chi_\rho(ghg\dmo)=\Tr(\rho_{ghg\dmo})=\Tr(\rho_g\circ\rho_h\circ\rho\dmo_g)=\Tr(\rho_h)\overset{\mbox{\scalebox{0.5}{$\Tr(AB)=\Tr(BA)$}}}{=}\chi_\rho(h)$
  	$\mathop{\oplus}_{\substack{x\in X}}$

$\mat(\rho_g)=(a_{ij}(g))_{\scriptsize \substack{1\leq i\leq d \\ 1\leq j\leq d}}$ et $\mat(\rho'_g)=(a'_{ij}(g))_{\scriptsize \substack{1\leq i'\leq d' \\ 1\leq j'\leq d'}}$



\[\int_a^b{\mathbb{R}^2}g(u, v)\dd{P_{XY}}(u, v)=\iint g(u,v) f_{XY}(u, v)\dd \lambda(u) \dd \lambda(v)\]
$$\lim_{x\to\infty} f(x)$$	
$$\iiiint_V \mu(t,u,v,w) \,dt\,du\,dv\,dw$$
$$\sum_{n=1}^{\infty} 2^{-n} = 1$$	
\begin{definition}
	Si $X$ et $Y$ sont 2 v.a. ou definit la \textsc{Covariance} entre $X$ et $Y$ comme
	$\cov(X,Y)\overset{\text{def}}{=}\E\left[(X-\E(X))(Y-\E(Y))\right]=\E(XY)-\E(X)\E(Y)$.
\end{definition}
\fi
\pagebreak

% \tableofcontents

% insert your code here
%\input{./algebra/main.tex}
%\input{./geometrie-differentielle/main.tex}
%\input{./probabilite/main.tex}
%\input{./analyse-fonctionnelle/main.tex}
% \input{./Analyse-convexe-et-dualite-en-optimisation/main.tex}
%\input{./tikz/main.tex}
%\input{./Theorie-du-distributions/main.tex}
%\input{./optimisation/mine.tex}
 \input{./modelisation/main.tex}

% yves.aubry@univ-tln.fr : algebra

\end{document}

%% !TEX encoding = UTF-8 Unicode
% !TEX TS-program = xelatex

\documentclass[french]{report}

%\usepackage[utf8]{inputenc}
%\usepackage[T1]{fontenc}
\usepackage{babel}


\newif\ifcomment
%\commenttrue # Show comments

\usepackage{physics}
\usepackage{amssymb}


\usepackage{amsthm}
% \usepackage{thmtools}
\usepackage{mathtools}
\usepackage{amsfonts}

\usepackage{color}

\usepackage{tikz}

\usepackage{geometry}
\geometry{a5paper, margin=0.1in, right=1cm}

\usepackage{dsfont}

\usepackage{graphicx}
\graphicspath{ {images/} }

\usepackage{faktor}

\usepackage{IEEEtrantools}
\usepackage{enumerate}   
\usepackage[PostScript=dvips]{"/Users/aware/Documents/Courses/diagrams"}


\newtheorem{theorem}{Théorème}[section]
\renewcommand{\thetheorem}{\arabic{theorem}}
\newtheorem{lemme}{Lemme}[section]
\renewcommand{\thelemme}{\arabic{lemme}}
\newtheorem{proposition}{Proposition}[section]
\renewcommand{\theproposition}{\arabic{proposition}}
\newtheorem{notations}{Notations}[section]
\newtheorem{problem}{Problème}[section]
\newtheorem{corollary}{Corollaire}[theorem]
\renewcommand{\thecorollary}{\arabic{corollary}}
\newtheorem{property}{Propriété}[section]
\newtheorem{objective}{Objectif}[section]

\theoremstyle{definition}
\newtheorem{definition}{Définition}[section]
\renewcommand{\thedefinition}{\arabic{definition}}
\newtheorem{exercise}{Exercice}[chapter]
\renewcommand{\theexercise}{\arabic{exercise}}
\newtheorem{example}{Exemple}[chapter]
\renewcommand{\theexample}{\arabic{example}}
\newtheorem*{solution}{Solution}
\newtheorem*{application}{Application}
\newtheorem*{notation}{Notation}
\newtheorem*{vocabulary}{Vocabulaire}
\newtheorem*{properties}{Propriétés}



\theoremstyle{remark}
\newtheorem*{remark}{Remarque}
\newtheorem*{rappel}{Rappel}


\usepackage{etoolbox}
\AtBeginEnvironment{exercise}{\small}
\AtBeginEnvironment{example}{\small}

\usepackage{cases}
\usepackage[red]{mypack}

\usepackage[framemethod=TikZ]{mdframed}

\definecolor{bg}{rgb}{0.4,0.25,0.95}
\definecolor{pagebg}{rgb}{0,0,0.5}
\surroundwithmdframed[
   topline=false,
   rightline=false,
   bottomline=false,
   leftmargin=\parindent,
   skipabove=8pt,
   skipbelow=8pt,
   linecolor=blue,
   innerbottommargin=10pt,
   % backgroundcolor=bg,font=\color{orange}\sffamily, fontcolor=white
]{definition}

\usepackage{empheq}
\usepackage[most]{tcolorbox}

\newtcbox{\mymath}[1][]{%
    nobeforeafter, math upper, tcbox raise base,
    enhanced, colframe=blue!30!black,
    colback=red!10, boxrule=1pt,
    #1}

\usepackage{unixode}


\DeclareMathOperator{\ord}{ord}
\DeclareMathOperator{\orb}{orb}
\DeclareMathOperator{\stab}{stab}
\DeclareMathOperator{\Stab}{stab}
\DeclareMathOperator{\ppcm}{ppcm}
\DeclareMathOperator{\conj}{Conj}
\DeclareMathOperator{\End}{End}
\DeclareMathOperator{\rot}{rot}
\DeclareMathOperator{\trs}{trace}
\DeclareMathOperator{\Ind}{Ind}
\DeclareMathOperator{\mat}{Mat}
\DeclareMathOperator{\id}{Id}
\DeclareMathOperator{\vect}{vect}
\DeclareMathOperator{\img}{img}
\DeclareMathOperator{\cov}{Cov}
\DeclareMathOperator{\dist}{dist}
\DeclareMathOperator{\irr}{Irr}
\DeclareMathOperator{\image}{Im}
\DeclareMathOperator{\pd}{\partial}
\DeclareMathOperator{\epi}{epi}
\DeclareMathOperator{\Argmin}{Argmin}
\DeclareMathOperator{\dom}{dom}
\DeclareMathOperator{\proj}{proj}
\DeclareMathOperator{\ctg}{ctg}
\DeclareMathOperator{\supp}{supp}
\DeclareMathOperator{\argmin}{argmin}
\DeclareMathOperator{\mult}{mult}
\DeclareMathOperator{\ch}{ch}
\DeclareMathOperator{\sh}{sh}
\DeclareMathOperator{\rang}{rang}
\DeclareMathOperator{\diam}{diam}
\DeclareMathOperator{\Epigraphe}{Epigraphe}




\usepackage{xcolor}
\everymath{\color{blue}}
%\everymath{\color[rgb]{0,1,1}}
%\pagecolor[rgb]{0,0,0.5}


\newcommand*{\pdtest}[3][]{\ensuremath{\frac{\partial^{#1} #2}{\partial #3}}}

\newcommand*{\deffunc}[6][]{\ensuremath{
\begin{array}{rcl}
#2 : #3 &\rightarrow& #4\\
#5 &\mapsto& #6
\end{array}
}}

\newcommand{\eqcolon}{\mathrel{\resizebox{\widthof{$\mathord{=}$}}{\height}{ $\!\!=\!\!\resizebox{1.2\width}{0.8\height}{\raisebox{0.23ex}{$\mathop{:}$}}\!\!$ }}}
\newcommand{\coloneq}{\mathrel{\resizebox{\widthof{$\mathord{=}$}}{\height}{ $\!\!\resizebox{1.2\width}{0.8\height}{\raisebox{0.23ex}{$\mathop{:}$}}\!\!=\!\!$ }}}
\newcommand{\eqcolonl}{\ensuremath{\mathrel{=\!\!\mathop{:}}}}
\newcommand{\coloneql}{\ensuremath{\mathrel{\mathop{:} \!\! =}}}
\newcommand{\vc}[1]{% inline column vector
  \left(\begin{smallmatrix}#1\end{smallmatrix}\right)%
}
\newcommand{\vr}[1]{% inline row vector
  \begin{smallmatrix}(\,#1\,)\end{smallmatrix}%
}
\makeatletter
\newcommand*{\defeq}{\ =\mathrel{\rlap{%
                     \raisebox{0.3ex}{$\m@th\cdot$}}%
                     \raisebox{-0.3ex}{$\m@th\cdot$}}%
                     }
\makeatother

\newcommand{\mathcircle}[1]{% inline row vector
 \overset{\circ}{#1}
}
\newcommand{\ulim}{% low limit
 \underline{\lim}
}
\newcommand{\ssi}{% iff
\iff
}
\newcommand{\ps}[2]{
\expval{#1 | #2}
}
\newcommand{\df}[1]{
\mqty{#1}
}
\newcommand{\n}[1]{
\norm{#1}
}
\newcommand{\sys}[1]{
\left\{\smqty{#1}\right.
}


\newcommand{\eqdef}{\ensuremath{\overset{\text{def}}=}}


\def\Circlearrowright{\ensuremath{%
  \rotatebox[origin=c]{230}{$\circlearrowright$}}}

\newcommand\ct[1]{\text{\rmfamily\upshape #1}}
\newcommand\question[1]{ {\color{red} ...!? \small #1}}
\newcommand\caz[1]{\left\{\begin{array} #1 \end{array}\right.}
\newcommand\const{\text{\rmfamily\upshape const}}
\newcommand\toP{ \overset{\pro}{\to}}
\newcommand\toPP{ \overset{\text{PP}}{\to}}
\newcommand{\oeq}{\mathrel{\text{\textcircled{$=$}}}}





\usepackage{xcolor}
% \usepackage[normalem]{ulem}
\usepackage{lipsum}
\makeatletter
% \newcommand\colorwave[1][blue]{\bgroup \markoverwith{\lower3.5\p@\hbox{\sixly \textcolor{#1}{\char58}}}\ULon}
%\font\sixly=lasy6 % does not re-load if already loaded, so no memory problem.

\newmdtheoremenv[
linewidth= 1pt,linecolor= blue,%
leftmargin=20,rightmargin=20,innertopmargin=0pt, innerrightmargin=40,%
tikzsetting = { draw=lightgray, line width = 0.3pt,dashed,%
dash pattern = on 15pt off 3pt},%
splittopskip=\topskip,skipbelow=\baselineskip,%
skipabove=\baselineskip,ntheorem,roundcorner=0pt,
% backgroundcolor=pagebg,font=\color{orange}\sffamily, fontcolor=white
]{examplebox}{Exemple}[section]



\newcommand\R{\mathbb{R}}
\newcommand\Z{\mathbb{Z}}
\newcommand\N{\mathbb{N}}
\newcommand\E{\mathbb{E}}
\newcommand\F{\mathcal{F}}
\newcommand\cH{\mathcal{H}}
\newcommand\V{\mathbb{V}}
\newcommand\dmo{ ^{-1} }
\newcommand\kapa{\kappa}
\newcommand\im{Im}
\newcommand\hs{\mathcal{H}}





\usepackage{soul}

\makeatletter
\newcommand*{\whiten}[1]{\llap{\textcolor{white}{{\the\SOUL@token}}\hspace{#1pt}}}
\DeclareRobustCommand*\myul{%
    \def\SOUL@everyspace{\underline{\space}\kern\z@}%
    \def\SOUL@everytoken{%
     \setbox0=\hbox{\the\SOUL@token}%
     \ifdim\dp0>\z@
        \raisebox{\dp0}{\underline{\phantom{\the\SOUL@token}}}%
        \whiten{1}\whiten{0}%
        \whiten{-1}\whiten{-2}%
        \llap{\the\SOUL@token}%
     \else
        \underline{\the\SOUL@token}%
     \fi}%
\SOUL@}
\makeatother

\newcommand*{\demp}{\fontfamily{lmtt}\selectfont}

\DeclareTextFontCommand{\textdemp}{\demp}

\begin{document}

\ifcomment
Multiline
comment
\fi
\ifcomment
\myul{Typesetting test}
% \color[rgb]{1,1,1}
$∑_i^n≠ 60º±∞π∆¬≈√j∫h≤≥µ$

$\CR \R\pro\ind\pro\gS\pro
\mqty[a&b\\c&d]$
$\pro\mathbb{P}$
$\dd{x}$

  \[
    \alpha(x)=\left\{
                \begin{array}{ll}
                  x\\
                  \frac{1}{1+e^{-kx}}\\
                  \frac{e^x-e^{-x}}{e^x+e^{-x}}
                \end{array}
              \right.
  \]

  $\expval{x}$
  
  $\chi_\rho(ghg\dmo)=\Tr(\rho_{ghg\dmo})=\Tr(\rho_g\circ\rho_h\circ\rho\dmo_g)=\Tr(\rho_h)\overset{\mbox{\scalebox{0.5}{$\Tr(AB)=\Tr(BA)$}}}{=}\chi_\rho(h)$
  	$\mathop{\oplus}_{\substack{x\in X}}$

$\mat(\rho_g)=(a_{ij}(g))_{\scriptsize \substack{1\leq i\leq d \\ 1\leq j\leq d}}$ et $\mat(\rho'_g)=(a'_{ij}(g))_{\scriptsize \substack{1\leq i'\leq d' \\ 1\leq j'\leq d'}}$



\[\int_a^b{\mathbb{R}^2}g(u, v)\dd{P_{XY}}(u, v)=\iint g(u,v) f_{XY}(u, v)\dd \lambda(u) \dd \lambda(v)\]
$$\lim_{x\to\infty} f(x)$$	
$$\iiiint_V \mu(t,u,v,w) \,dt\,du\,dv\,dw$$
$$\sum_{n=1}^{\infty} 2^{-n} = 1$$	
\begin{definition}
	Si $X$ et $Y$ sont 2 v.a. ou definit la \textsc{Covariance} entre $X$ et $Y$ comme
	$\cov(X,Y)\overset{\text{def}}{=}\E\left[(X-\E(X))(Y-\E(Y))\right]=\E(XY)-\E(X)\E(Y)$.
\end{definition}
\fi
\pagebreak

% \tableofcontents

% insert your code here
%\input{./algebra/main.tex}
%\input{./geometrie-differentielle/main.tex}
%\input{./probabilite/main.tex}
%\input{./analyse-fonctionnelle/main.tex}
% \input{./Analyse-convexe-et-dualite-en-optimisation/main.tex}
%\input{./tikz/main.tex}
%\input{./Theorie-du-distributions/main.tex}
%\input{./optimisation/mine.tex}
 \input{./modelisation/main.tex}

% yves.aubry@univ-tln.fr : algebra

\end{document}

%% !TEX encoding = UTF-8 Unicode
% !TEX TS-program = xelatex

\documentclass[french]{report}

%\usepackage[utf8]{inputenc}
%\usepackage[T1]{fontenc}
\usepackage{babel}


\newif\ifcomment
%\commenttrue # Show comments

\usepackage{physics}
\usepackage{amssymb}


\usepackage{amsthm}
% \usepackage{thmtools}
\usepackage{mathtools}
\usepackage{amsfonts}

\usepackage{color}

\usepackage{tikz}

\usepackage{geometry}
\geometry{a5paper, margin=0.1in, right=1cm}

\usepackage{dsfont}

\usepackage{graphicx}
\graphicspath{ {images/} }

\usepackage{faktor}

\usepackage{IEEEtrantools}
\usepackage{enumerate}   
\usepackage[PostScript=dvips]{"/Users/aware/Documents/Courses/diagrams"}


\newtheorem{theorem}{Théorème}[section]
\renewcommand{\thetheorem}{\arabic{theorem}}
\newtheorem{lemme}{Lemme}[section]
\renewcommand{\thelemme}{\arabic{lemme}}
\newtheorem{proposition}{Proposition}[section]
\renewcommand{\theproposition}{\arabic{proposition}}
\newtheorem{notations}{Notations}[section]
\newtheorem{problem}{Problème}[section]
\newtheorem{corollary}{Corollaire}[theorem]
\renewcommand{\thecorollary}{\arabic{corollary}}
\newtheorem{property}{Propriété}[section]
\newtheorem{objective}{Objectif}[section]

\theoremstyle{definition}
\newtheorem{definition}{Définition}[section]
\renewcommand{\thedefinition}{\arabic{definition}}
\newtheorem{exercise}{Exercice}[chapter]
\renewcommand{\theexercise}{\arabic{exercise}}
\newtheorem{example}{Exemple}[chapter]
\renewcommand{\theexample}{\arabic{example}}
\newtheorem*{solution}{Solution}
\newtheorem*{application}{Application}
\newtheorem*{notation}{Notation}
\newtheorem*{vocabulary}{Vocabulaire}
\newtheorem*{properties}{Propriétés}



\theoremstyle{remark}
\newtheorem*{remark}{Remarque}
\newtheorem*{rappel}{Rappel}


\usepackage{etoolbox}
\AtBeginEnvironment{exercise}{\small}
\AtBeginEnvironment{example}{\small}

\usepackage{cases}
\usepackage[red]{mypack}

\usepackage[framemethod=TikZ]{mdframed}

\definecolor{bg}{rgb}{0.4,0.25,0.95}
\definecolor{pagebg}{rgb}{0,0,0.5}
\surroundwithmdframed[
   topline=false,
   rightline=false,
   bottomline=false,
   leftmargin=\parindent,
   skipabove=8pt,
   skipbelow=8pt,
   linecolor=blue,
   innerbottommargin=10pt,
   % backgroundcolor=bg,font=\color{orange}\sffamily, fontcolor=white
]{definition}

\usepackage{empheq}
\usepackage[most]{tcolorbox}

\newtcbox{\mymath}[1][]{%
    nobeforeafter, math upper, tcbox raise base,
    enhanced, colframe=blue!30!black,
    colback=red!10, boxrule=1pt,
    #1}

\usepackage{unixode}


\DeclareMathOperator{\ord}{ord}
\DeclareMathOperator{\orb}{orb}
\DeclareMathOperator{\stab}{stab}
\DeclareMathOperator{\Stab}{stab}
\DeclareMathOperator{\ppcm}{ppcm}
\DeclareMathOperator{\conj}{Conj}
\DeclareMathOperator{\End}{End}
\DeclareMathOperator{\rot}{rot}
\DeclareMathOperator{\trs}{trace}
\DeclareMathOperator{\Ind}{Ind}
\DeclareMathOperator{\mat}{Mat}
\DeclareMathOperator{\id}{Id}
\DeclareMathOperator{\vect}{vect}
\DeclareMathOperator{\img}{img}
\DeclareMathOperator{\cov}{Cov}
\DeclareMathOperator{\dist}{dist}
\DeclareMathOperator{\irr}{Irr}
\DeclareMathOperator{\image}{Im}
\DeclareMathOperator{\pd}{\partial}
\DeclareMathOperator{\epi}{epi}
\DeclareMathOperator{\Argmin}{Argmin}
\DeclareMathOperator{\dom}{dom}
\DeclareMathOperator{\proj}{proj}
\DeclareMathOperator{\ctg}{ctg}
\DeclareMathOperator{\supp}{supp}
\DeclareMathOperator{\argmin}{argmin}
\DeclareMathOperator{\mult}{mult}
\DeclareMathOperator{\ch}{ch}
\DeclareMathOperator{\sh}{sh}
\DeclareMathOperator{\rang}{rang}
\DeclareMathOperator{\diam}{diam}
\DeclareMathOperator{\Epigraphe}{Epigraphe}




\usepackage{xcolor}
\everymath{\color{blue}}
%\everymath{\color[rgb]{0,1,1}}
%\pagecolor[rgb]{0,0,0.5}


\newcommand*{\pdtest}[3][]{\ensuremath{\frac{\partial^{#1} #2}{\partial #3}}}

\newcommand*{\deffunc}[6][]{\ensuremath{
\begin{array}{rcl}
#2 : #3 &\rightarrow& #4\\
#5 &\mapsto& #6
\end{array}
}}

\newcommand{\eqcolon}{\mathrel{\resizebox{\widthof{$\mathord{=}$}}{\height}{ $\!\!=\!\!\resizebox{1.2\width}{0.8\height}{\raisebox{0.23ex}{$\mathop{:}$}}\!\!$ }}}
\newcommand{\coloneq}{\mathrel{\resizebox{\widthof{$\mathord{=}$}}{\height}{ $\!\!\resizebox{1.2\width}{0.8\height}{\raisebox{0.23ex}{$\mathop{:}$}}\!\!=\!\!$ }}}
\newcommand{\eqcolonl}{\ensuremath{\mathrel{=\!\!\mathop{:}}}}
\newcommand{\coloneql}{\ensuremath{\mathrel{\mathop{:} \!\! =}}}
\newcommand{\vc}[1]{% inline column vector
  \left(\begin{smallmatrix}#1\end{smallmatrix}\right)%
}
\newcommand{\vr}[1]{% inline row vector
  \begin{smallmatrix}(\,#1\,)\end{smallmatrix}%
}
\makeatletter
\newcommand*{\defeq}{\ =\mathrel{\rlap{%
                     \raisebox{0.3ex}{$\m@th\cdot$}}%
                     \raisebox{-0.3ex}{$\m@th\cdot$}}%
                     }
\makeatother

\newcommand{\mathcircle}[1]{% inline row vector
 \overset{\circ}{#1}
}
\newcommand{\ulim}{% low limit
 \underline{\lim}
}
\newcommand{\ssi}{% iff
\iff
}
\newcommand{\ps}[2]{
\expval{#1 | #2}
}
\newcommand{\df}[1]{
\mqty{#1}
}
\newcommand{\n}[1]{
\norm{#1}
}
\newcommand{\sys}[1]{
\left\{\smqty{#1}\right.
}


\newcommand{\eqdef}{\ensuremath{\overset{\text{def}}=}}


\def\Circlearrowright{\ensuremath{%
  \rotatebox[origin=c]{230}{$\circlearrowright$}}}

\newcommand\ct[1]{\text{\rmfamily\upshape #1}}
\newcommand\question[1]{ {\color{red} ...!? \small #1}}
\newcommand\caz[1]{\left\{\begin{array} #1 \end{array}\right.}
\newcommand\const{\text{\rmfamily\upshape const}}
\newcommand\toP{ \overset{\pro}{\to}}
\newcommand\toPP{ \overset{\text{PP}}{\to}}
\newcommand{\oeq}{\mathrel{\text{\textcircled{$=$}}}}





\usepackage{xcolor}
% \usepackage[normalem]{ulem}
\usepackage{lipsum}
\makeatletter
% \newcommand\colorwave[1][blue]{\bgroup \markoverwith{\lower3.5\p@\hbox{\sixly \textcolor{#1}{\char58}}}\ULon}
%\font\sixly=lasy6 % does not re-load if already loaded, so no memory problem.

\newmdtheoremenv[
linewidth= 1pt,linecolor= blue,%
leftmargin=20,rightmargin=20,innertopmargin=0pt, innerrightmargin=40,%
tikzsetting = { draw=lightgray, line width = 0.3pt,dashed,%
dash pattern = on 15pt off 3pt},%
splittopskip=\topskip,skipbelow=\baselineskip,%
skipabove=\baselineskip,ntheorem,roundcorner=0pt,
% backgroundcolor=pagebg,font=\color{orange}\sffamily, fontcolor=white
]{examplebox}{Exemple}[section]



\newcommand\R{\mathbb{R}}
\newcommand\Z{\mathbb{Z}}
\newcommand\N{\mathbb{N}}
\newcommand\E{\mathbb{E}}
\newcommand\F{\mathcal{F}}
\newcommand\cH{\mathcal{H}}
\newcommand\V{\mathbb{V}}
\newcommand\dmo{ ^{-1} }
\newcommand\kapa{\kappa}
\newcommand\im{Im}
\newcommand\hs{\mathcal{H}}





\usepackage{soul}

\makeatletter
\newcommand*{\whiten}[1]{\llap{\textcolor{white}{{\the\SOUL@token}}\hspace{#1pt}}}
\DeclareRobustCommand*\myul{%
    \def\SOUL@everyspace{\underline{\space}\kern\z@}%
    \def\SOUL@everytoken{%
     \setbox0=\hbox{\the\SOUL@token}%
     \ifdim\dp0>\z@
        \raisebox{\dp0}{\underline{\phantom{\the\SOUL@token}}}%
        \whiten{1}\whiten{0}%
        \whiten{-1}\whiten{-2}%
        \llap{\the\SOUL@token}%
     \else
        \underline{\the\SOUL@token}%
     \fi}%
\SOUL@}
\makeatother

\newcommand*{\demp}{\fontfamily{lmtt}\selectfont}

\DeclareTextFontCommand{\textdemp}{\demp}

\begin{document}

\ifcomment
Multiline
comment
\fi
\ifcomment
\myul{Typesetting test}
% \color[rgb]{1,1,1}
$∑_i^n≠ 60º±∞π∆¬≈√j∫h≤≥µ$

$\CR \R\pro\ind\pro\gS\pro
\mqty[a&b\\c&d]$
$\pro\mathbb{P}$
$\dd{x}$

  \[
    \alpha(x)=\left\{
                \begin{array}{ll}
                  x\\
                  \frac{1}{1+e^{-kx}}\\
                  \frac{e^x-e^{-x}}{e^x+e^{-x}}
                \end{array}
              \right.
  \]

  $\expval{x}$
  
  $\chi_\rho(ghg\dmo)=\Tr(\rho_{ghg\dmo})=\Tr(\rho_g\circ\rho_h\circ\rho\dmo_g)=\Tr(\rho_h)\overset{\mbox{\scalebox{0.5}{$\Tr(AB)=\Tr(BA)$}}}{=}\chi_\rho(h)$
  	$\mathop{\oplus}_{\substack{x\in X}}$

$\mat(\rho_g)=(a_{ij}(g))_{\scriptsize \substack{1\leq i\leq d \\ 1\leq j\leq d}}$ et $\mat(\rho'_g)=(a'_{ij}(g))_{\scriptsize \substack{1\leq i'\leq d' \\ 1\leq j'\leq d'}}$



\[\int_a^b{\mathbb{R}^2}g(u, v)\dd{P_{XY}}(u, v)=\iint g(u,v) f_{XY}(u, v)\dd \lambda(u) \dd \lambda(v)\]
$$\lim_{x\to\infty} f(x)$$	
$$\iiiint_V \mu(t,u,v,w) \,dt\,du\,dv\,dw$$
$$\sum_{n=1}^{\infty} 2^{-n} = 1$$	
\begin{definition}
	Si $X$ et $Y$ sont 2 v.a. ou definit la \textsc{Covariance} entre $X$ et $Y$ comme
	$\cov(X,Y)\overset{\text{def}}{=}\E\left[(X-\E(X))(Y-\E(Y))\right]=\E(XY)-\E(X)\E(Y)$.
\end{definition}
\fi
\pagebreak

% \tableofcontents

% insert your code here
%\input{./algebra/main.tex}
%\input{./geometrie-differentielle/main.tex}
%\input{./probabilite/main.tex}
%\input{./analyse-fonctionnelle/main.tex}
% \input{./Analyse-convexe-et-dualite-en-optimisation/main.tex}
%\input{./tikz/main.tex}
%\input{./Theorie-du-distributions/main.tex}
%\input{./optimisation/mine.tex}
 \input{./modelisation/main.tex}

% yves.aubry@univ-tln.fr : algebra

\end{document}

%% !TEX encoding = UTF-8 Unicode
% !TEX TS-program = xelatex

\documentclass[french]{report}

%\usepackage[utf8]{inputenc}
%\usepackage[T1]{fontenc}
\usepackage{babel}


\newif\ifcomment
%\commenttrue # Show comments

\usepackage{physics}
\usepackage{amssymb}


\usepackage{amsthm}
% \usepackage{thmtools}
\usepackage{mathtools}
\usepackage{amsfonts}

\usepackage{color}

\usepackage{tikz}

\usepackage{geometry}
\geometry{a5paper, margin=0.1in, right=1cm}

\usepackage{dsfont}

\usepackage{graphicx}
\graphicspath{ {images/} }

\usepackage{faktor}

\usepackage{IEEEtrantools}
\usepackage{enumerate}   
\usepackage[PostScript=dvips]{"/Users/aware/Documents/Courses/diagrams"}


\newtheorem{theorem}{Théorème}[section]
\renewcommand{\thetheorem}{\arabic{theorem}}
\newtheorem{lemme}{Lemme}[section]
\renewcommand{\thelemme}{\arabic{lemme}}
\newtheorem{proposition}{Proposition}[section]
\renewcommand{\theproposition}{\arabic{proposition}}
\newtheorem{notations}{Notations}[section]
\newtheorem{problem}{Problème}[section]
\newtheorem{corollary}{Corollaire}[theorem]
\renewcommand{\thecorollary}{\arabic{corollary}}
\newtheorem{property}{Propriété}[section]
\newtheorem{objective}{Objectif}[section]

\theoremstyle{definition}
\newtheorem{definition}{Définition}[section]
\renewcommand{\thedefinition}{\arabic{definition}}
\newtheorem{exercise}{Exercice}[chapter]
\renewcommand{\theexercise}{\arabic{exercise}}
\newtheorem{example}{Exemple}[chapter]
\renewcommand{\theexample}{\arabic{example}}
\newtheorem*{solution}{Solution}
\newtheorem*{application}{Application}
\newtheorem*{notation}{Notation}
\newtheorem*{vocabulary}{Vocabulaire}
\newtheorem*{properties}{Propriétés}



\theoremstyle{remark}
\newtheorem*{remark}{Remarque}
\newtheorem*{rappel}{Rappel}


\usepackage{etoolbox}
\AtBeginEnvironment{exercise}{\small}
\AtBeginEnvironment{example}{\small}

\usepackage{cases}
\usepackage[red]{mypack}

\usepackage[framemethod=TikZ]{mdframed}

\definecolor{bg}{rgb}{0.4,0.25,0.95}
\definecolor{pagebg}{rgb}{0,0,0.5}
\surroundwithmdframed[
   topline=false,
   rightline=false,
   bottomline=false,
   leftmargin=\parindent,
   skipabove=8pt,
   skipbelow=8pt,
   linecolor=blue,
   innerbottommargin=10pt,
   % backgroundcolor=bg,font=\color{orange}\sffamily, fontcolor=white
]{definition}

\usepackage{empheq}
\usepackage[most]{tcolorbox}

\newtcbox{\mymath}[1][]{%
    nobeforeafter, math upper, tcbox raise base,
    enhanced, colframe=blue!30!black,
    colback=red!10, boxrule=1pt,
    #1}

\usepackage{unixode}


\DeclareMathOperator{\ord}{ord}
\DeclareMathOperator{\orb}{orb}
\DeclareMathOperator{\stab}{stab}
\DeclareMathOperator{\Stab}{stab}
\DeclareMathOperator{\ppcm}{ppcm}
\DeclareMathOperator{\conj}{Conj}
\DeclareMathOperator{\End}{End}
\DeclareMathOperator{\rot}{rot}
\DeclareMathOperator{\trs}{trace}
\DeclareMathOperator{\Ind}{Ind}
\DeclareMathOperator{\mat}{Mat}
\DeclareMathOperator{\id}{Id}
\DeclareMathOperator{\vect}{vect}
\DeclareMathOperator{\img}{img}
\DeclareMathOperator{\cov}{Cov}
\DeclareMathOperator{\dist}{dist}
\DeclareMathOperator{\irr}{Irr}
\DeclareMathOperator{\image}{Im}
\DeclareMathOperator{\pd}{\partial}
\DeclareMathOperator{\epi}{epi}
\DeclareMathOperator{\Argmin}{Argmin}
\DeclareMathOperator{\dom}{dom}
\DeclareMathOperator{\proj}{proj}
\DeclareMathOperator{\ctg}{ctg}
\DeclareMathOperator{\supp}{supp}
\DeclareMathOperator{\argmin}{argmin}
\DeclareMathOperator{\mult}{mult}
\DeclareMathOperator{\ch}{ch}
\DeclareMathOperator{\sh}{sh}
\DeclareMathOperator{\rang}{rang}
\DeclareMathOperator{\diam}{diam}
\DeclareMathOperator{\Epigraphe}{Epigraphe}




\usepackage{xcolor}
\everymath{\color{blue}}
%\everymath{\color[rgb]{0,1,1}}
%\pagecolor[rgb]{0,0,0.5}


\newcommand*{\pdtest}[3][]{\ensuremath{\frac{\partial^{#1} #2}{\partial #3}}}

\newcommand*{\deffunc}[6][]{\ensuremath{
\begin{array}{rcl}
#2 : #3 &\rightarrow& #4\\
#5 &\mapsto& #6
\end{array}
}}

\newcommand{\eqcolon}{\mathrel{\resizebox{\widthof{$\mathord{=}$}}{\height}{ $\!\!=\!\!\resizebox{1.2\width}{0.8\height}{\raisebox{0.23ex}{$\mathop{:}$}}\!\!$ }}}
\newcommand{\coloneq}{\mathrel{\resizebox{\widthof{$\mathord{=}$}}{\height}{ $\!\!\resizebox{1.2\width}{0.8\height}{\raisebox{0.23ex}{$\mathop{:}$}}\!\!=\!\!$ }}}
\newcommand{\eqcolonl}{\ensuremath{\mathrel{=\!\!\mathop{:}}}}
\newcommand{\coloneql}{\ensuremath{\mathrel{\mathop{:} \!\! =}}}
\newcommand{\vc}[1]{% inline column vector
  \left(\begin{smallmatrix}#1\end{smallmatrix}\right)%
}
\newcommand{\vr}[1]{% inline row vector
  \begin{smallmatrix}(\,#1\,)\end{smallmatrix}%
}
\makeatletter
\newcommand*{\defeq}{\ =\mathrel{\rlap{%
                     \raisebox{0.3ex}{$\m@th\cdot$}}%
                     \raisebox{-0.3ex}{$\m@th\cdot$}}%
                     }
\makeatother

\newcommand{\mathcircle}[1]{% inline row vector
 \overset{\circ}{#1}
}
\newcommand{\ulim}{% low limit
 \underline{\lim}
}
\newcommand{\ssi}{% iff
\iff
}
\newcommand{\ps}[2]{
\expval{#1 | #2}
}
\newcommand{\df}[1]{
\mqty{#1}
}
\newcommand{\n}[1]{
\norm{#1}
}
\newcommand{\sys}[1]{
\left\{\smqty{#1}\right.
}


\newcommand{\eqdef}{\ensuremath{\overset{\text{def}}=}}


\def\Circlearrowright{\ensuremath{%
  \rotatebox[origin=c]{230}{$\circlearrowright$}}}

\newcommand\ct[1]{\text{\rmfamily\upshape #1}}
\newcommand\question[1]{ {\color{red} ...!? \small #1}}
\newcommand\caz[1]{\left\{\begin{array} #1 \end{array}\right.}
\newcommand\const{\text{\rmfamily\upshape const}}
\newcommand\toP{ \overset{\pro}{\to}}
\newcommand\toPP{ \overset{\text{PP}}{\to}}
\newcommand{\oeq}{\mathrel{\text{\textcircled{$=$}}}}





\usepackage{xcolor}
% \usepackage[normalem]{ulem}
\usepackage{lipsum}
\makeatletter
% \newcommand\colorwave[1][blue]{\bgroup \markoverwith{\lower3.5\p@\hbox{\sixly \textcolor{#1}{\char58}}}\ULon}
%\font\sixly=lasy6 % does not re-load if already loaded, so no memory problem.

\newmdtheoremenv[
linewidth= 1pt,linecolor= blue,%
leftmargin=20,rightmargin=20,innertopmargin=0pt, innerrightmargin=40,%
tikzsetting = { draw=lightgray, line width = 0.3pt,dashed,%
dash pattern = on 15pt off 3pt},%
splittopskip=\topskip,skipbelow=\baselineskip,%
skipabove=\baselineskip,ntheorem,roundcorner=0pt,
% backgroundcolor=pagebg,font=\color{orange}\sffamily, fontcolor=white
]{examplebox}{Exemple}[section]



\newcommand\R{\mathbb{R}}
\newcommand\Z{\mathbb{Z}}
\newcommand\N{\mathbb{N}}
\newcommand\E{\mathbb{E}}
\newcommand\F{\mathcal{F}}
\newcommand\cH{\mathcal{H}}
\newcommand\V{\mathbb{V}}
\newcommand\dmo{ ^{-1} }
\newcommand\kapa{\kappa}
\newcommand\im{Im}
\newcommand\hs{\mathcal{H}}





\usepackage{soul}

\makeatletter
\newcommand*{\whiten}[1]{\llap{\textcolor{white}{{\the\SOUL@token}}\hspace{#1pt}}}
\DeclareRobustCommand*\myul{%
    \def\SOUL@everyspace{\underline{\space}\kern\z@}%
    \def\SOUL@everytoken{%
     \setbox0=\hbox{\the\SOUL@token}%
     \ifdim\dp0>\z@
        \raisebox{\dp0}{\underline{\phantom{\the\SOUL@token}}}%
        \whiten{1}\whiten{0}%
        \whiten{-1}\whiten{-2}%
        \llap{\the\SOUL@token}%
     \else
        \underline{\the\SOUL@token}%
     \fi}%
\SOUL@}
\makeatother

\newcommand*{\demp}{\fontfamily{lmtt}\selectfont}

\DeclareTextFontCommand{\textdemp}{\demp}

\begin{document}

\ifcomment
Multiline
comment
\fi
\ifcomment
\myul{Typesetting test}
% \color[rgb]{1,1,1}
$∑_i^n≠ 60º±∞π∆¬≈√j∫h≤≥µ$

$\CR \R\pro\ind\pro\gS\pro
\mqty[a&b\\c&d]$
$\pro\mathbb{P}$
$\dd{x}$

  \[
    \alpha(x)=\left\{
                \begin{array}{ll}
                  x\\
                  \frac{1}{1+e^{-kx}}\\
                  \frac{e^x-e^{-x}}{e^x+e^{-x}}
                \end{array}
              \right.
  \]

  $\expval{x}$
  
  $\chi_\rho(ghg\dmo)=\Tr(\rho_{ghg\dmo})=\Tr(\rho_g\circ\rho_h\circ\rho\dmo_g)=\Tr(\rho_h)\overset{\mbox{\scalebox{0.5}{$\Tr(AB)=\Tr(BA)$}}}{=}\chi_\rho(h)$
  	$\mathop{\oplus}_{\substack{x\in X}}$

$\mat(\rho_g)=(a_{ij}(g))_{\scriptsize \substack{1\leq i\leq d \\ 1\leq j\leq d}}$ et $\mat(\rho'_g)=(a'_{ij}(g))_{\scriptsize \substack{1\leq i'\leq d' \\ 1\leq j'\leq d'}}$



\[\int_a^b{\mathbb{R}^2}g(u, v)\dd{P_{XY}}(u, v)=\iint g(u,v) f_{XY}(u, v)\dd \lambda(u) \dd \lambda(v)\]
$$\lim_{x\to\infty} f(x)$$	
$$\iiiint_V \mu(t,u,v,w) \,dt\,du\,dv\,dw$$
$$\sum_{n=1}^{\infty} 2^{-n} = 1$$	
\begin{definition}
	Si $X$ et $Y$ sont 2 v.a. ou definit la \textsc{Covariance} entre $X$ et $Y$ comme
	$\cov(X,Y)\overset{\text{def}}{=}\E\left[(X-\E(X))(Y-\E(Y))\right]=\E(XY)-\E(X)\E(Y)$.
\end{definition}
\fi
\pagebreak

% \tableofcontents

% insert your code here
%\input{./algebra/main.tex}
%\input{./geometrie-differentielle/main.tex}
%\input{./probabilite/main.tex}
%\input{./analyse-fonctionnelle/main.tex}
% \input{./Analyse-convexe-et-dualite-en-optimisation/main.tex}
%\input{./tikz/main.tex}
%\input{./Theorie-du-distributions/main.tex}
%\input{./optimisation/mine.tex}
 \input{./modelisation/main.tex}

% yves.aubry@univ-tln.fr : algebra

\end{document}

% % !TEX encoding = UTF-8 Unicode
% !TEX TS-program = xelatex

\documentclass[french]{report}

%\usepackage[utf8]{inputenc}
%\usepackage[T1]{fontenc}
\usepackage{babel}


\newif\ifcomment
%\commenttrue # Show comments

\usepackage{physics}
\usepackage{amssymb}


\usepackage{amsthm}
% \usepackage{thmtools}
\usepackage{mathtools}
\usepackage{amsfonts}

\usepackage{color}

\usepackage{tikz}

\usepackage{geometry}
\geometry{a5paper, margin=0.1in, right=1cm}

\usepackage{dsfont}

\usepackage{graphicx}
\graphicspath{ {images/} }

\usepackage{faktor}

\usepackage{IEEEtrantools}
\usepackage{enumerate}   
\usepackage[PostScript=dvips]{"/Users/aware/Documents/Courses/diagrams"}


\newtheorem{theorem}{Théorème}[section]
\renewcommand{\thetheorem}{\arabic{theorem}}
\newtheorem{lemme}{Lemme}[section]
\renewcommand{\thelemme}{\arabic{lemme}}
\newtheorem{proposition}{Proposition}[section]
\renewcommand{\theproposition}{\arabic{proposition}}
\newtheorem{notations}{Notations}[section]
\newtheorem{problem}{Problème}[section]
\newtheorem{corollary}{Corollaire}[theorem]
\renewcommand{\thecorollary}{\arabic{corollary}}
\newtheorem{property}{Propriété}[section]
\newtheorem{objective}{Objectif}[section]

\theoremstyle{definition}
\newtheorem{definition}{Définition}[section]
\renewcommand{\thedefinition}{\arabic{definition}}
\newtheorem{exercise}{Exercice}[chapter]
\renewcommand{\theexercise}{\arabic{exercise}}
\newtheorem{example}{Exemple}[chapter]
\renewcommand{\theexample}{\arabic{example}}
\newtheorem*{solution}{Solution}
\newtheorem*{application}{Application}
\newtheorem*{notation}{Notation}
\newtheorem*{vocabulary}{Vocabulaire}
\newtheorem*{properties}{Propriétés}



\theoremstyle{remark}
\newtheorem*{remark}{Remarque}
\newtheorem*{rappel}{Rappel}


\usepackage{etoolbox}
\AtBeginEnvironment{exercise}{\small}
\AtBeginEnvironment{example}{\small}

\usepackage{cases}
\usepackage[red]{mypack}

\usepackage[framemethod=TikZ]{mdframed}

\definecolor{bg}{rgb}{0.4,0.25,0.95}
\definecolor{pagebg}{rgb}{0,0,0.5}
\surroundwithmdframed[
   topline=false,
   rightline=false,
   bottomline=false,
   leftmargin=\parindent,
   skipabove=8pt,
   skipbelow=8pt,
   linecolor=blue,
   innerbottommargin=10pt,
   % backgroundcolor=bg,font=\color{orange}\sffamily, fontcolor=white
]{definition}

\usepackage{empheq}
\usepackage[most]{tcolorbox}

\newtcbox{\mymath}[1][]{%
    nobeforeafter, math upper, tcbox raise base,
    enhanced, colframe=blue!30!black,
    colback=red!10, boxrule=1pt,
    #1}

\usepackage{unixode}


\DeclareMathOperator{\ord}{ord}
\DeclareMathOperator{\orb}{orb}
\DeclareMathOperator{\stab}{stab}
\DeclareMathOperator{\Stab}{stab}
\DeclareMathOperator{\ppcm}{ppcm}
\DeclareMathOperator{\conj}{Conj}
\DeclareMathOperator{\End}{End}
\DeclareMathOperator{\rot}{rot}
\DeclareMathOperator{\trs}{trace}
\DeclareMathOperator{\Ind}{Ind}
\DeclareMathOperator{\mat}{Mat}
\DeclareMathOperator{\id}{Id}
\DeclareMathOperator{\vect}{vect}
\DeclareMathOperator{\img}{img}
\DeclareMathOperator{\cov}{Cov}
\DeclareMathOperator{\dist}{dist}
\DeclareMathOperator{\irr}{Irr}
\DeclareMathOperator{\image}{Im}
\DeclareMathOperator{\pd}{\partial}
\DeclareMathOperator{\epi}{epi}
\DeclareMathOperator{\Argmin}{Argmin}
\DeclareMathOperator{\dom}{dom}
\DeclareMathOperator{\proj}{proj}
\DeclareMathOperator{\ctg}{ctg}
\DeclareMathOperator{\supp}{supp}
\DeclareMathOperator{\argmin}{argmin}
\DeclareMathOperator{\mult}{mult}
\DeclareMathOperator{\ch}{ch}
\DeclareMathOperator{\sh}{sh}
\DeclareMathOperator{\rang}{rang}
\DeclareMathOperator{\diam}{diam}
\DeclareMathOperator{\Epigraphe}{Epigraphe}




\usepackage{xcolor}
\everymath{\color{blue}}
%\everymath{\color[rgb]{0,1,1}}
%\pagecolor[rgb]{0,0,0.5}


\newcommand*{\pdtest}[3][]{\ensuremath{\frac{\partial^{#1} #2}{\partial #3}}}

\newcommand*{\deffunc}[6][]{\ensuremath{
\begin{array}{rcl}
#2 : #3 &\rightarrow& #4\\
#5 &\mapsto& #6
\end{array}
}}

\newcommand{\eqcolon}{\mathrel{\resizebox{\widthof{$\mathord{=}$}}{\height}{ $\!\!=\!\!\resizebox{1.2\width}{0.8\height}{\raisebox{0.23ex}{$\mathop{:}$}}\!\!$ }}}
\newcommand{\coloneq}{\mathrel{\resizebox{\widthof{$\mathord{=}$}}{\height}{ $\!\!\resizebox{1.2\width}{0.8\height}{\raisebox{0.23ex}{$\mathop{:}$}}\!\!=\!\!$ }}}
\newcommand{\eqcolonl}{\ensuremath{\mathrel{=\!\!\mathop{:}}}}
\newcommand{\coloneql}{\ensuremath{\mathrel{\mathop{:} \!\! =}}}
\newcommand{\vc}[1]{% inline column vector
  \left(\begin{smallmatrix}#1\end{smallmatrix}\right)%
}
\newcommand{\vr}[1]{% inline row vector
  \begin{smallmatrix}(\,#1\,)\end{smallmatrix}%
}
\makeatletter
\newcommand*{\defeq}{\ =\mathrel{\rlap{%
                     \raisebox{0.3ex}{$\m@th\cdot$}}%
                     \raisebox{-0.3ex}{$\m@th\cdot$}}%
                     }
\makeatother

\newcommand{\mathcircle}[1]{% inline row vector
 \overset{\circ}{#1}
}
\newcommand{\ulim}{% low limit
 \underline{\lim}
}
\newcommand{\ssi}{% iff
\iff
}
\newcommand{\ps}[2]{
\expval{#1 | #2}
}
\newcommand{\df}[1]{
\mqty{#1}
}
\newcommand{\n}[1]{
\norm{#1}
}
\newcommand{\sys}[1]{
\left\{\smqty{#1}\right.
}


\newcommand{\eqdef}{\ensuremath{\overset{\text{def}}=}}


\def\Circlearrowright{\ensuremath{%
  \rotatebox[origin=c]{230}{$\circlearrowright$}}}

\newcommand\ct[1]{\text{\rmfamily\upshape #1}}
\newcommand\question[1]{ {\color{red} ...!? \small #1}}
\newcommand\caz[1]{\left\{\begin{array} #1 \end{array}\right.}
\newcommand\const{\text{\rmfamily\upshape const}}
\newcommand\toP{ \overset{\pro}{\to}}
\newcommand\toPP{ \overset{\text{PP}}{\to}}
\newcommand{\oeq}{\mathrel{\text{\textcircled{$=$}}}}





\usepackage{xcolor}
% \usepackage[normalem]{ulem}
\usepackage{lipsum}
\makeatletter
% \newcommand\colorwave[1][blue]{\bgroup \markoverwith{\lower3.5\p@\hbox{\sixly \textcolor{#1}{\char58}}}\ULon}
%\font\sixly=lasy6 % does not re-load if already loaded, so no memory problem.

\newmdtheoremenv[
linewidth= 1pt,linecolor= blue,%
leftmargin=20,rightmargin=20,innertopmargin=0pt, innerrightmargin=40,%
tikzsetting = { draw=lightgray, line width = 0.3pt,dashed,%
dash pattern = on 15pt off 3pt},%
splittopskip=\topskip,skipbelow=\baselineskip,%
skipabove=\baselineskip,ntheorem,roundcorner=0pt,
% backgroundcolor=pagebg,font=\color{orange}\sffamily, fontcolor=white
]{examplebox}{Exemple}[section]



\newcommand\R{\mathbb{R}}
\newcommand\Z{\mathbb{Z}}
\newcommand\N{\mathbb{N}}
\newcommand\E{\mathbb{E}}
\newcommand\F{\mathcal{F}}
\newcommand\cH{\mathcal{H}}
\newcommand\V{\mathbb{V}}
\newcommand\dmo{ ^{-1} }
\newcommand\kapa{\kappa}
\newcommand\im{Im}
\newcommand\hs{\mathcal{H}}





\usepackage{soul}

\makeatletter
\newcommand*{\whiten}[1]{\llap{\textcolor{white}{{\the\SOUL@token}}\hspace{#1pt}}}
\DeclareRobustCommand*\myul{%
    \def\SOUL@everyspace{\underline{\space}\kern\z@}%
    \def\SOUL@everytoken{%
     \setbox0=\hbox{\the\SOUL@token}%
     \ifdim\dp0>\z@
        \raisebox{\dp0}{\underline{\phantom{\the\SOUL@token}}}%
        \whiten{1}\whiten{0}%
        \whiten{-1}\whiten{-2}%
        \llap{\the\SOUL@token}%
     \else
        \underline{\the\SOUL@token}%
     \fi}%
\SOUL@}
\makeatother

\newcommand*{\demp}{\fontfamily{lmtt}\selectfont}

\DeclareTextFontCommand{\textdemp}{\demp}

\begin{document}

\ifcomment
Multiline
comment
\fi
\ifcomment
\myul{Typesetting test}
% \color[rgb]{1,1,1}
$∑_i^n≠ 60º±∞π∆¬≈√j∫h≤≥µ$

$\CR \R\pro\ind\pro\gS\pro
\mqty[a&b\\c&d]$
$\pro\mathbb{P}$
$\dd{x}$

  \[
    \alpha(x)=\left\{
                \begin{array}{ll}
                  x\\
                  \frac{1}{1+e^{-kx}}\\
                  \frac{e^x-e^{-x}}{e^x+e^{-x}}
                \end{array}
              \right.
  \]

  $\expval{x}$
  
  $\chi_\rho(ghg\dmo)=\Tr(\rho_{ghg\dmo})=\Tr(\rho_g\circ\rho_h\circ\rho\dmo_g)=\Tr(\rho_h)\overset{\mbox{\scalebox{0.5}{$\Tr(AB)=\Tr(BA)$}}}{=}\chi_\rho(h)$
  	$\mathop{\oplus}_{\substack{x\in X}}$

$\mat(\rho_g)=(a_{ij}(g))_{\scriptsize \substack{1\leq i\leq d \\ 1\leq j\leq d}}$ et $\mat(\rho'_g)=(a'_{ij}(g))_{\scriptsize \substack{1\leq i'\leq d' \\ 1\leq j'\leq d'}}$



\[\int_a^b{\mathbb{R}^2}g(u, v)\dd{P_{XY}}(u, v)=\iint g(u,v) f_{XY}(u, v)\dd \lambda(u) \dd \lambda(v)\]
$$\lim_{x\to\infty} f(x)$$	
$$\iiiint_V \mu(t,u,v,w) \,dt\,du\,dv\,dw$$
$$\sum_{n=1}^{\infty} 2^{-n} = 1$$	
\begin{definition}
	Si $X$ et $Y$ sont 2 v.a. ou definit la \textsc{Covariance} entre $X$ et $Y$ comme
	$\cov(X,Y)\overset{\text{def}}{=}\E\left[(X-\E(X))(Y-\E(Y))\right]=\E(XY)-\E(X)\E(Y)$.
\end{definition}
\fi
\pagebreak

% \tableofcontents

% insert your code here
%\input{./algebra/main.tex}
%\input{./geometrie-differentielle/main.tex}
%\input{./probabilite/main.tex}
%\input{./analyse-fonctionnelle/main.tex}
% \input{./Analyse-convexe-et-dualite-en-optimisation/main.tex}
%\input{./tikz/main.tex}
%\input{./Theorie-du-distributions/main.tex}
%\input{./optimisation/mine.tex}
 \input{./modelisation/main.tex}

% yves.aubry@univ-tln.fr : algebra

\end{document}

%% !TEX encoding = UTF-8 Unicode
% !TEX TS-program = xelatex

\documentclass[french]{report}

%\usepackage[utf8]{inputenc}
%\usepackage[T1]{fontenc}
\usepackage{babel}


\newif\ifcomment
%\commenttrue # Show comments

\usepackage{physics}
\usepackage{amssymb}


\usepackage{amsthm}
% \usepackage{thmtools}
\usepackage{mathtools}
\usepackage{amsfonts}

\usepackage{color}

\usepackage{tikz}

\usepackage{geometry}
\geometry{a5paper, margin=0.1in, right=1cm}

\usepackage{dsfont}

\usepackage{graphicx}
\graphicspath{ {images/} }

\usepackage{faktor}

\usepackage{IEEEtrantools}
\usepackage{enumerate}   
\usepackage[PostScript=dvips]{"/Users/aware/Documents/Courses/diagrams"}


\newtheorem{theorem}{Théorème}[section]
\renewcommand{\thetheorem}{\arabic{theorem}}
\newtheorem{lemme}{Lemme}[section]
\renewcommand{\thelemme}{\arabic{lemme}}
\newtheorem{proposition}{Proposition}[section]
\renewcommand{\theproposition}{\arabic{proposition}}
\newtheorem{notations}{Notations}[section]
\newtheorem{problem}{Problème}[section]
\newtheorem{corollary}{Corollaire}[theorem]
\renewcommand{\thecorollary}{\arabic{corollary}}
\newtheorem{property}{Propriété}[section]
\newtheorem{objective}{Objectif}[section]

\theoremstyle{definition}
\newtheorem{definition}{Définition}[section]
\renewcommand{\thedefinition}{\arabic{definition}}
\newtheorem{exercise}{Exercice}[chapter]
\renewcommand{\theexercise}{\arabic{exercise}}
\newtheorem{example}{Exemple}[chapter]
\renewcommand{\theexample}{\arabic{example}}
\newtheorem*{solution}{Solution}
\newtheorem*{application}{Application}
\newtheorem*{notation}{Notation}
\newtheorem*{vocabulary}{Vocabulaire}
\newtheorem*{properties}{Propriétés}



\theoremstyle{remark}
\newtheorem*{remark}{Remarque}
\newtheorem*{rappel}{Rappel}


\usepackage{etoolbox}
\AtBeginEnvironment{exercise}{\small}
\AtBeginEnvironment{example}{\small}

\usepackage{cases}
\usepackage[red]{mypack}

\usepackage[framemethod=TikZ]{mdframed}

\definecolor{bg}{rgb}{0.4,0.25,0.95}
\definecolor{pagebg}{rgb}{0,0,0.5}
\surroundwithmdframed[
   topline=false,
   rightline=false,
   bottomline=false,
   leftmargin=\parindent,
   skipabove=8pt,
   skipbelow=8pt,
   linecolor=blue,
   innerbottommargin=10pt,
   % backgroundcolor=bg,font=\color{orange}\sffamily, fontcolor=white
]{definition}

\usepackage{empheq}
\usepackage[most]{tcolorbox}

\newtcbox{\mymath}[1][]{%
    nobeforeafter, math upper, tcbox raise base,
    enhanced, colframe=blue!30!black,
    colback=red!10, boxrule=1pt,
    #1}

\usepackage{unixode}


\DeclareMathOperator{\ord}{ord}
\DeclareMathOperator{\orb}{orb}
\DeclareMathOperator{\stab}{stab}
\DeclareMathOperator{\Stab}{stab}
\DeclareMathOperator{\ppcm}{ppcm}
\DeclareMathOperator{\conj}{Conj}
\DeclareMathOperator{\End}{End}
\DeclareMathOperator{\rot}{rot}
\DeclareMathOperator{\trs}{trace}
\DeclareMathOperator{\Ind}{Ind}
\DeclareMathOperator{\mat}{Mat}
\DeclareMathOperator{\id}{Id}
\DeclareMathOperator{\vect}{vect}
\DeclareMathOperator{\img}{img}
\DeclareMathOperator{\cov}{Cov}
\DeclareMathOperator{\dist}{dist}
\DeclareMathOperator{\irr}{Irr}
\DeclareMathOperator{\image}{Im}
\DeclareMathOperator{\pd}{\partial}
\DeclareMathOperator{\epi}{epi}
\DeclareMathOperator{\Argmin}{Argmin}
\DeclareMathOperator{\dom}{dom}
\DeclareMathOperator{\proj}{proj}
\DeclareMathOperator{\ctg}{ctg}
\DeclareMathOperator{\supp}{supp}
\DeclareMathOperator{\argmin}{argmin}
\DeclareMathOperator{\mult}{mult}
\DeclareMathOperator{\ch}{ch}
\DeclareMathOperator{\sh}{sh}
\DeclareMathOperator{\rang}{rang}
\DeclareMathOperator{\diam}{diam}
\DeclareMathOperator{\Epigraphe}{Epigraphe}




\usepackage{xcolor}
\everymath{\color{blue}}
%\everymath{\color[rgb]{0,1,1}}
%\pagecolor[rgb]{0,0,0.5}


\newcommand*{\pdtest}[3][]{\ensuremath{\frac{\partial^{#1} #2}{\partial #3}}}

\newcommand*{\deffunc}[6][]{\ensuremath{
\begin{array}{rcl}
#2 : #3 &\rightarrow& #4\\
#5 &\mapsto& #6
\end{array}
}}

\newcommand{\eqcolon}{\mathrel{\resizebox{\widthof{$\mathord{=}$}}{\height}{ $\!\!=\!\!\resizebox{1.2\width}{0.8\height}{\raisebox{0.23ex}{$\mathop{:}$}}\!\!$ }}}
\newcommand{\coloneq}{\mathrel{\resizebox{\widthof{$\mathord{=}$}}{\height}{ $\!\!\resizebox{1.2\width}{0.8\height}{\raisebox{0.23ex}{$\mathop{:}$}}\!\!=\!\!$ }}}
\newcommand{\eqcolonl}{\ensuremath{\mathrel{=\!\!\mathop{:}}}}
\newcommand{\coloneql}{\ensuremath{\mathrel{\mathop{:} \!\! =}}}
\newcommand{\vc}[1]{% inline column vector
  \left(\begin{smallmatrix}#1\end{smallmatrix}\right)%
}
\newcommand{\vr}[1]{% inline row vector
  \begin{smallmatrix}(\,#1\,)\end{smallmatrix}%
}
\makeatletter
\newcommand*{\defeq}{\ =\mathrel{\rlap{%
                     \raisebox{0.3ex}{$\m@th\cdot$}}%
                     \raisebox{-0.3ex}{$\m@th\cdot$}}%
                     }
\makeatother

\newcommand{\mathcircle}[1]{% inline row vector
 \overset{\circ}{#1}
}
\newcommand{\ulim}{% low limit
 \underline{\lim}
}
\newcommand{\ssi}{% iff
\iff
}
\newcommand{\ps}[2]{
\expval{#1 | #2}
}
\newcommand{\df}[1]{
\mqty{#1}
}
\newcommand{\n}[1]{
\norm{#1}
}
\newcommand{\sys}[1]{
\left\{\smqty{#1}\right.
}


\newcommand{\eqdef}{\ensuremath{\overset{\text{def}}=}}


\def\Circlearrowright{\ensuremath{%
  \rotatebox[origin=c]{230}{$\circlearrowright$}}}

\newcommand\ct[1]{\text{\rmfamily\upshape #1}}
\newcommand\question[1]{ {\color{red} ...!? \small #1}}
\newcommand\caz[1]{\left\{\begin{array} #1 \end{array}\right.}
\newcommand\const{\text{\rmfamily\upshape const}}
\newcommand\toP{ \overset{\pro}{\to}}
\newcommand\toPP{ \overset{\text{PP}}{\to}}
\newcommand{\oeq}{\mathrel{\text{\textcircled{$=$}}}}





\usepackage{xcolor}
% \usepackage[normalem]{ulem}
\usepackage{lipsum}
\makeatletter
% \newcommand\colorwave[1][blue]{\bgroup \markoverwith{\lower3.5\p@\hbox{\sixly \textcolor{#1}{\char58}}}\ULon}
%\font\sixly=lasy6 % does not re-load if already loaded, so no memory problem.

\newmdtheoremenv[
linewidth= 1pt,linecolor= blue,%
leftmargin=20,rightmargin=20,innertopmargin=0pt, innerrightmargin=40,%
tikzsetting = { draw=lightgray, line width = 0.3pt,dashed,%
dash pattern = on 15pt off 3pt},%
splittopskip=\topskip,skipbelow=\baselineskip,%
skipabove=\baselineskip,ntheorem,roundcorner=0pt,
% backgroundcolor=pagebg,font=\color{orange}\sffamily, fontcolor=white
]{examplebox}{Exemple}[section]



\newcommand\R{\mathbb{R}}
\newcommand\Z{\mathbb{Z}}
\newcommand\N{\mathbb{N}}
\newcommand\E{\mathbb{E}}
\newcommand\F{\mathcal{F}}
\newcommand\cH{\mathcal{H}}
\newcommand\V{\mathbb{V}}
\newcommand\dmo{ ^{-1} }
\newcommand\kapa{\kappa}
\newcommand\im{Im}
\newcommand\hs{\mathcal{H}}





\usepackage{soul}

\makeatletter
\newcommand*{\whiten}[1]{\llap{\textcolor{white}{{\the\SOUL@token}}\hspace{#1pt}}}
\DeclareRobustCommand*\myul{%
    \def\SOUL@everyspace{\underline{\space}\kern\z@}%
    \def\SOUL@everytoken{%
     \setbox0=\hbox{\the\SOUL@token}%
     \ifdim\dp0>\z@
        \raisebox{\dp0}{\underline{\phantom{\the\SOUL@token}}}%
        \whiten{1}\whiten{0}%
        \whiten{-1}\whiten{-2}%
        \llap{\the\SOUL@token}%
     \else
        \underline{\the\SOUL@token}%
     \fi}%
\SOUL@}
\makeatother

\newcommand*{\demp}{\fontfamily{lmtt}\selectfont}

\DeclareTextFontCommand{\textdemp}{\demp}

\begin{document}

\ifcomment
Multiline
comment
\fi
\ifcomment
\myul{Typesetting test}
% \color[rgb]{1,1,1}
$∑_i^n≠ 60º±∞π∆¬≈√j∫h≤≥µ$

$\CR \R\pro\ind\pro\gS\pro
\mqty[a&b\\c&d]$
$\pro\mathbb{P}$
$\dd{x}$

  \[
    \alpha(x)=\left\{
                \begin{array}{ll}
                  x\\
                  \frac{1}{1+e^{-kx}}\\
                  \frac{e^x-e^{-x}}{e^x+e^{-x}}
                \end{array}
              \right.
  \]

  $\expval{x}$
  
  $\chi_\rho(ghg\dmo)=\Tr(\rho_{ghg\dmo})=\Tr(\rho_g\circ\rho_h\circ\rho\dmo_g)=\Tr(\rho_h)\overset{\mbox{\scalebox{0.5}{$\Tr(AB)=\Tr(BA)$}}}{=}\chi_\rho(h)$
  	$\mathop{\oplus}_{\substack{x\in X}}$

$\mat(\rho_g)=(a_{ij}(g))_{\scriptsize \substack{1\leq i\leq d \\ 1\leq j\leq d}}$ et $\mat(\rho'_g)=(a'_{ij}(g))_{\scriptsize \substack{1\leq i'\leq d' \\ 1\leq j'\leq d'}}$



\[\int_a^b{\mathbb{R}^2}g(u, v)\dd{P_{XY}}(u, v)=\iint g(u,v) f_{XY}(u, v)\dd \lambda(u) \dd \lambda(v)\]
$$\lim_{x\to\infty} f(x)$$	
$$\iiiint_V \mu(t,u,v,w) \,dt\,du\,dv\,dw$$
$$\sum_{n=1}^{\infty} 2^{-n} = 1$$	
\begin{definition}
	Si $X$ et $Y$ sont 2 v.a. ou definit la \textsc{Covariance} entre $X$ et $Y$ comme
	$\cov(X,Y)\overset{\text{def}}{=}\E\left[(X-\E(X))(Y-\E(Y))\right]=\E(XY)-\E(X)\E(Y)$.
\end{definition}
\fi
\pagebreak

% \tableofcontents

% insert your code here
%\input{./algebra/main.tex}
%\input{./geometrie-differentielle/main.tex}
%\input{./probabilite/main.tex}
%\input{./analyse-fonctionnelle/main.tex}
% \input{./Analyse-convexe-et-dualite-en-optimisation/main.tex}
%\input{./tikz/main.tex}
%\input{./Theorie-du-distributions/main.tex}
%\input{./optimisation/mine.tex}
 \input{./modelisation/main.tex}

% yves.aubry@univ-tln.fr : algebra

\end{document}

%% !TEX encoding = UTF-8 Unicode
% !TEX TS-program = xelatex

\documentclass[french]{report}

%\usepackage[utf8]{inputenc}
%\usepackage[T1]{fontenc}
\usepackage{babel}


\newif\ifcomment
%\commenttrue # Show comments

\usepackage{physics}
\usepackage{amssymb}


\usepackage{amsthm}
% \usepackage{thmtools}
\usepackage{mathtools}
\usepackage{amsfonts}

\usepackage{color}

\usepackage{tikz}

\usepackage{geometry}
\geometry{a5paper, margin=0.1in, right=1cm}

\usepackage{dsfont}

\usepackage{graphicx}
\graphicspath{ {images/} }

\usepackage{faktor}

\usepackage{IEEEtrantools}
\usepackage{enumerate}   
\usepackage[PostScript=dvips]{"/Users/aware/Documents/Courses/diagrams"}


\newtheorem{theorem}{Théorème}[section]
\renewcommand{\thetheorem}{\arabic{theorem}}
\newtheorem{lemme}{Lemme}[section]
\renewcommand{\thelemme}{\arabic{lemme}}
\newtheorem{proposition}{Proposition}[section]
\renewcommand{\theproposition}{\arabic{proposition}}
\newtheorem{notations}{Notations}[section]
\newtheorem{problem}{Problème}[section]
\newtheorem{corollary}{Corollaire}[theorem]
\renewcommand{\thecorollary}{\arabic{corollary}}
\newtheorem{property}{Propriété}[section]
\newtheorem{objective}{Objectif}[section]

\theoremstyle{definition}
\newtheorem{definition}{Définition}[section]
\renewcommand{\thedefinition}{\arabic{definition}}
\newtheorem{exercise}{Exercice}[chapter]
\renewcommand{\theexercise}{\arabic{exercise}}
\newtheorem{example}{Exemple}[chapter]
\renewcommand{\theexample}{\arabic{example}}
\newtheorem*{solution}{Solution}
\newtheorem*{application}{Application}
\newtheorem*{notation}{Notation}
\newtheorem*{vocabulary}{Vocabulaire}
\newtheorem*{properties}{Propriétés}



\theoremstyle{remark}
\newtheorem*{remark}{Remarque}
\newtheorem*{rappel}{Rappel}


\usepackage{etoolbox}
\AtBeginEnvironment{exercise}{\small}
\AtBeginEnvironment{example}{\small}

\usepackage{cases}
\usepackage[red]{mypack}

\usepackage[framemethod=TikZ]{mdframed}

\definecolor{bg}{rgb}{0.4,0.25,0.95}
\definecolor{pagebg}{rgb}{0,0,0.5}
\surroundwithmdframed[
   topline=false,
   rightline=false,
   bottomline=false,
   leftmargin=\parindent,
   skipabove=8pt,
   skipbelow=8pt,
   linecolor=blue,
   innerbottommargin=10pt,
   % backgroundcolor=bg,font=\color{orange}\sffamily, fontcolor=white
]{definition}

\usepackage{empheq}
\usepackage[most]{tcolorbox}

\newtcbox{\mymath}[1][]{%
    nobeforeafter, math upper, tcbox raise base,
    enhanced, colframe=blue!30!black,
    colback=red!10, boxrule=1pt,
    #1}

\usepackage{unixode}


\DeclareMathOperator{\ord}{ord}
\DeclareMathOperator{\orb}{orb}
\DeclareMathOperator{\stab}{stab}
\DeclareMathOperator{\Stab}{stab}
\DeclareMathOperator{\ppcm}{ppcm}
\DeclareMathOperator{\conj}{Conj}
\DeclareMathOperator{\End}{End}
\DeclareMathOperator{\rot}{rot}
\DeclareMathOperator{\trs}{trace}
\DeclareMathOperator{\Ind}{Ind}
\DeclareMathOperator{\mat}{Mat}
\DeclareMathOperator{\id}{Id}
\DeclareMathOperator{\vect}{vect}
\DeclareMathOperator{\img}{img}
\DeclareMathOperator{\cov}{Cov}
\DeclareMathOperator{\dist}{dist}
\DeclareMathOperator{\irr}{Irr}
\DeclareMathOperator{\image}{Im}
\DeclareMathOperator{\pd}{\partial}
\DeclareMathOperator{\epi}{epi}
\DeclareMathOperator{\Argmin}{Argmin}
\DeclareMathOperator{\dom}{dom}
\DeclareMathOperator{\proj}{proj}
\DeclareMathOperator{\ctg}{ctg}
\DeclareMathOperator{\supp}{supp}
\DeclareMathOperator{\argmin}{argmin}
\DeclareMathOperator{\mult}{mult}
\DeclareMathOperator{\ch}{ch}
\DeclareMathOperator{\sh}{sh}
\DeclareMathOperator{\rang}{rang}
\DeclareMathOperator{\diam}{diam}
\DeclareMathOperator{\Epigraphe}{Epigraphe}




\usepackage{xcolor}
\everymath{\color{blue}}
%\everymath{\color[rgb]{0,1,1}}
%\pagecolor[rgb]{0,0,0.5}


\newcommand*{\pdtest}[3][]{\ensuremath{\frac{\partial^{#1} #2}{\partial #3}}}

\newcommand*{\deffunc}[6][]{\ensuremath{
\begin{array}{rcl}
#2 : #3 &\rightarrow& #4\\
#5 &\mapsto& #6
\end{array}
}}

\newcommand{\eqcolon}{\mathrel{\resizebox{\widthof{$\mathord{=}$}}{\height}{ $\!\!=\!\!\resizebox{1.2\width}{0.8\height}{\raisebox{0.23ex}{$\mathop{:}$}}\!\!$ }}}
\newcommand{\coloneq}{\mathrel{\resizebox{\widthof{$\mathord{=}$}}{\height}{ $\!\!\resizebox{1.2\width}{0.8\height}{\raisebox{0.23ex}{$\mathop{:}$}}\!\!=\!\!$ }}}
\newcommand{\eqcolonl}{\ensuremath{\mathrel{=\!\!\mathop{:}}}}
\newcommand{\coloneql}{\ensuremath{\mathrel{\mathop{:} \!\! =}}}
\newcommand{\vc}[1]{% inline column vector
  \left(\begin{smallmatrix}#1\end{smallmatrix}\right)%
}
\newcommand{\vr}[1]{% inline row vector
  \begin{smallmatrix}(\,#1\,)\end{smallmatrix}%
}
\makeatletter
\newcommand*{\defeq}{\ =\mathrel{\rlap{%
                     \raisebox{0.3ex}{$\m@th\cdot$}}%
                     \raisebox{-0.3ex}{$\m@th\cdot$}}%
                     }
\makeatother

\newcommand{\mathcircle}[1]{% inline row vector
 \overset{\circ}{#1}
}
\newcommand{\ulim}{% low limit
 \underline{\lim}
}
\newcommand{\ssi}{% iff
\iff
}
\newcommand{\ps}[2]{
\expval{#1 | #2}
}
\newcommand{\df}[1]{
\mqty{#1}
}
\newcommand{\n}[1]{
\norm{#1}
}
\newcommand{\sys}[1]{
\left\{\smqty{#1}\right.
}


\newcommand{\eqdef}{\ensuremath{\overset{\text{def}}=}}


\def\Circlearrowright{\ensuremath{%
  \rotatebox[origin=c]{230}{$\circlearrowright$}}}

\newcommand\ct[1]{\text{\rmfamily\upshape #1}}
\newcommand\question[1]{ {\color{red} ...!? \small #1}}
\newcommand\caz[1]{\left\{\begin{array} #1 \end{array}\right.}
\newcommand\const{\text{\rmfamily\upshape const}}
\newcommand\toP{ \overset{\pro}{\to}}
\newcommand\toPP{ \overset{\text{PP}}{\to}}
\newcommand{\oeq}{\mathrel{\text{\textcircled{$=$}}}}





\usepackage{xcolor}
% \usepackage[normalem]{ulem}
\usepackage{lipsum}
\makeatletter
% \newcommand\colorwave[1][blue]{\bgroup \markoverwith{\lower3.5\p@\hbox{\sixly \textcolor{#1}{\char58}}}\ULon}
%\font\sixly=lasy6 % does not re-load if already loaded, so no memory problem.

\newmdtheoremenv[
linewidth= 1pt,linecolor= blue,%
leftmargin=20,rightmargin=20,innertopmargin=0pt, innerrightmargin=40,%
tikzsetting = { draw=lightgray, line width = 0.3pt,dashed,%
dash pattern = on 15pt off 3pt},%
splittopskip=\topskip,skipbelow=\baselineskip,%
skipabove=\baselineskip,ntheorem,roundcorner=0pt,
% backgroundcolor=pagebg,font=\color{orange}\sffamily, fontcolor=white
]{examplebox}{Exemple}[section]



\newcommand\R{\mathbb{R}}
\newcommand\Z{\mathbb{Z}}
\newcommand\N{\mathbb{N}}
\newcommand\E{\mathbb{E}}
\newcommand\F{\mathcal{F}}
\newcommand\cH{\mathcal{H}}
\newcommand\V{\mathbb{V}}
\newcommand\dmo{ ^{-1} }
\newcommand\kapa{\kappa}
\newcommand\im{Im}
\newcommand\hs{\mathcal{H}}





\usepackage{soul}

\makeatletter
\newcommand*{\whiten}[1]{\llap{\textcolor{white}{{\the\SOUL@token}}\hspace{#1pt}}}
\DeclareRobustCommand*\myul{%
    \def\SOUL@everyspace{\underline{\space}\kern\z@}%
    \def\SOUL@everytoken{%
     \setbox0=\hbox{\the\SOUL@token}%
     \ifdim\dp0>\z@
        \raisebox{\dp0}{\underline{\phantom{\the\SOUL@token}}}%
        \whiten{1}\whiten{0}%
        \whiten{-1}\whiten{-2}%
        \llap{\the\SOUL@token}%
     \else
        \underline{\the\SOUL@token}%
     \fi}%
\SOUL@}
\makeatother

\newcommand*{\demp}{\fontfamily{lmtt}\selectfont}

\DeclareTextFontCommand{\textdemp}{\demp}

\begin{document}

\ifcomment
Multiline
comment
\fi
\ifcomment
\myul{Typesetting test}
% \color[rgb]{1,1,1}
$∑_i^n≠ 60º±∞π∆¬≈√j∫h≤≥µ$

$\CR \R\pro\ind\pro\gS\pro
\mqty[a&b\\c&d]$
$\pro\mathbb{P}$
$\dd{x}$

  \[
    \alpha(x)=\left\{
                \begin{array}{ll}
                  x\\
                  \frac{1}{1+e^{-kx}}\\
                  \frac{e^x-e^{-x}}{e^x+e^{-x}}
                \end{array}
              \right.
  \]

  $\expval{x}$
  
  $\chi_\rho(ghg\dmo)=\Tr(\rho_{ghg\dmo})=\Tr(\rho_g\circ\rho_h\circ\rho\dmo_g)=\Tr(\rho_h)\overset{\mbox{\scalebox{0.5}{$\Tr(AB)=\Tr(BA)$}}}{=}\chi_\rho(h)$
  	$\mathop{\oplus}_{\substack{x\in X}}$

$\mat(\rho_g)=(a_{ij}(g))_{\scriptsize \substack{1\leq i\leq d \\ 1\leq j\leq d}}$ et $\mat(\rho'_g)=(a'_{ij}(g))_{\scriptsize \substack{1\leq i'\leq d' \\ 1\leq j'\leq d'}}$



\[\int_a^b{\mathbb{R}^2}g(u, v)\dd{P_{XY}}(u, v)=\iint g(u,v) f_{XY}(u, v)\dd \lambda(u) \dd \lambda(v)\]
$$\lim_{x\to\infty} f(x)$$	
$$\iiiint_V \mu(t,u,v,w) \,dt\,du\,dv\,dw$$
$$\sum_{n=1}^{\infty} 2^{-n} = 1$$	
\begin{definition}
	Si $X$ et $Y$ sont 2 v.a. ou definit la \textsc{Covariance} entre $X$ et $Y$ comme
	$\cov(X,Y)\overset{\text{def}}{=}\E\left[(X-\E(X))(Y-\E(Y))\right]=\E(XY)-\E(X)\E(Y)$.
\end{definition}
\fi
\pagebreak

% \tableofcontents

% insert your code here
%\input{./algebra/main.tex}
%\input{./geometrie-differentielle/main.tex}
%\input{./probabilite/main.tex}
%\input{./analyse-fonctionnelle/main.tex}
% \input{./Analyse-convexe-et-dualite-en-optimisation/main.tex}
%\input{./tikz/main.tex}
%\input{./Theorie-du-distributions/main.tex}
%\input{./optimisation/mine.tex}
 \input{./modelisation/main.tex}

% yves.aubry@univ-tln.fr : algebra

\end{document}

%\input{./optimisation/mine.tex}
 % !TEX encoding = UTF-8 Unicode
% !TEX TS-program = xelatex

\documentclass[french]{report}

%\usepackage[utf8]{inputenc}
%\usepackage[T1]{fontenc}
\usepackage{babel}


\newif\ifcomment
%\commenttrue # Show comments

\usepackage{physics}
\usepackage{amssymb}


\usepackage{amsthm}
% \usepackage{thmtools}
\usepackage{mathtools}
\usepackage{amsfonts}

\usepackage{color}

\usepackage{tikz}

\usepackage{geometry}
\geometry{a5paper, margin=0.1in, right=1cm}

\usepackage{dsfont}

\usepackage{graphicx}
\graphicspath{ {images/} }

\usepackage{faktor}

\usepackage{IEEEtrantools}
\usepackage{enumerate}   
\usepackage[PostScript=dvips]{"/Users/aware/Documents/Courses/diagrams"}


\newtheorem{theorem}{Théorème}[section]
\renewcommand{\thetheorem}{\arabic{theorem}}
\newtheorem{lemme}{Lemme}[section]
\renewcommand{\thelemme}{\arabic{lemme}}
\newtheorem{proposition}{Proposition}[section]
\renewcommand{\theproposition}{\arabic{proposition}}
\newtheorem{notations}{Notations}[section]
\newtheorem{problem}{Problème}[section]
\newtheorem{corollary}{Corollaire}[theorem]
\renewcommand{\thecorollary}{\arabic{corollary}}
\newtheorem{property}{Propriété}[section]
\newtheorem{objective}{Objectif}[section]

\theoremstyle{definition}
\newtheorem{definition}{Définition}[section]
\renewcommand{\thedefinition}{\arabic{definition}}
\newtheorem{exercise}{Exercice}[chapter]
\renewcommand{\theexercise}{\arabic{exercise}}
\newtheorem{example}{Exemple}[chapter]
\renewcommand{\theexample}{\arabic{example}}
\newtheorem*{solution}{Solution}
\newtheorem*{application}{Application}
\newtheorem*{notation}{Notation}
\newtheorem*{vocabulary}{Vocabulaire}
\newtheorem*{properties}{Propriétés}



\theoremstyle{remark}
\newtheorem*{remark}{Remarque}
\newtheorem*{rappel}{Rappel}


\usepackage{etoolbox}
\AtBeginEnvironment{exercise}{\small}
\AtBeginEnvironment{example}{\small}

\usepackage{cases}
\usepackage[red]{mypack}

\usepackage[framemethod=TikZ]{mdframed}

\definecolor{bg}{rgb}{0.4,0.25,0.95}
\definecolor{pagebg}{rgb}{0,0,0.5}
\surroundwithmdframed[
   topline=false,
   rightline=false,
   bottomline=false,
   leftmargin=\parindent,
   skipabove=8pt,
   skipbelow=8pt,
   linecolor=blue,
   innerbottommargin=10pt,
   % backgroundcolor=bg,font=\color{orange}\sffamily, fontcolor=white
]{definition}

\usepackage{empheq}
\usepackage[most]{tcolorbox}

\newtcbox{\mymath}[1][]{%
    nobeforeafter, math upper, tcbox raise base,
    enhanced, colframe=blue!30!black,
    colback=red!10, boxrule=1pt,
    #1}

\usepackage{unixode}


\DeclareMathOperator{\ord}{ord}
\DeclareMathOperator{\orb}{orb}
\DeclareMathOperator{\stab}{stab}
\DeclareMathOperator{\Stab}{stab}
\DeclareMathOperator{\ppcm}{ppcm}
\DeclareMathOperator{\conj}{Conj}
\DeclareMathOperator{\End}{End}
\DeclareMathOperator{\rot}{rot}
\DeclareMathOperator{\trs}{trace}
\DeclareMathOperator{\Ind}{Ind}
\DeclareMathOperator{\mat}{Mat}
\DeclareMathOperator{\id}{Id}
\DeclareMathOperator{\vect}{vect}
\DeclareMathOperator{\img}{img}
\DeclareMathOperator{\cov}{Cov}
\DeclareMathOperator{\dist}{dist}
\DeclareMathOperator{\irr}{Irr}
\DeclareMathOperator{\image}{Im}
\DeclareMathOperator{\pd}{\partial}
\DeclareMathOperator{\epi}{epi}
\DeclareMathOperator{\Argmin}{Argmin}
\DeclareMathOperator{\dom}{dom}
\DeclareMathOperator{\proj}{proj}
\DeclareMathOperator{\ctg}{ctg}
\DeclareMathOperator{\supp}{supp}
\DeclareMathOperator{\argmin}{argmin}
\DeclareMathOperator{\mult}{mult}
\DeclareMathOperator{\ch}{ch}
\DeclareMathOperator{\sh}{sh}
\DeclareMathOperator{\rang}{rang}
\DeclareMathOperator{\diam}{diam}
\DeclareMathOperator{\Epigraphe}{Epigraphe}




\usepackage{xcolor}
\everymath{\color{blue}}
%\everymath{\color[rgb]{0,1,1}}
%\pagecolor[rgb]{0,0,0.5}


\newcommand*{\pdtest}[3][]{\ensuremath{\frac{\partial^{#1} #2}{\partial #3}}}

\newcommand*{\deffunc}[6][]{\ensuremath{
\begin{array}{rcl}
#2 : #3 &\rightarrow& #4\\
#5 &\mapsto& #6
\end{array}
}}

\newcommand{\eqcolon}{\mathrel{\resizebox{\widthof{$\mathord{=}$}}{\height}{ $\!\!=\!\!\resizebox{1.2\width}{0.8\height}{\raisebox{0.23ex}{$\mathop{:}$}}\!\!$ }}}
\newcommand{\coloneq}{\mathrel{\resizebox{\widthof{$\mathord{=}$}}{\height}{ $\!\!\resizebox{1.2\width}{0.8\height}{\raisebox{0.23ex}{$\mathop{:}$}}\!\!=\!\!$ }}}
\newcommand{\eqcolonl}{\ensuremath{\mathrel{=\!\!\mathop{:}}}}
\newcommand{\coloneql}{\ensuremath{\mathrel{\mathop{:} \!\! =}}}
\newcommand{\vc}[1]{% inline column vector
  \left(\begin{smallmatrix}#1\end{smallmatrix}\right)%
}
\newcommand{\vr}[1]{% inline row vector
  \begin{smallmatrix}(\,#1\,)\end{smallmatrix}%
}
\makeatletter
\newcommand*{\defeq}{\ =\mathrel{\rlap{%
                     \raisebox{0.3ex}{$\m@th\cdot$}}%
                     \raisebox{-0.3ex}{$\m@th\cdot$}}%
                     }
\makeatother

\newcommand{\mathcircle}[1]{% inline row vector
 \overset{\circ}{#1}
}
\newcommand{\ulim}{% low limit
 \underline{\lim}
}
\newcommand{\ssi}{% iff
\iff
}
\newcommand{\ps}[2]{
\expval{#1 | #2}
}
\newcommand{\df}[1]{
\mqty{#1}
}
\newcommand{\n}[1]{
\norm{#1}
}
\newcommand{\sys}[1]{
\left\{\smqty{#1}\right.
}


\newcommand{\eqdef}{\ensuremath{\overset{\text{def}}=}}


\def\Circlearrowright{\ensuremath{%
  \rotatebox[origin=c]{230}{$\circlearrowright$}}}

\newcommand\ct[1]{\text{\rmfamily\upshape #1}}
\newcommand\question[1]{ {\color{red} ...!? \small #1}}
\newcommand\caz[1]{\left\{\begin{array} #1 \end{array}\right.}
\newcommand\const{\text{\rmfamily\upshape const}}
\newcommand\toP{ \overset{\pro}{\to}}
\newcommand\toPP{ \overset{\text{PP}}{\to}}
\newcommand{\oeq}{\mathrel{\text{\textcircled{$=$}}}}





\usepackage{xcolor}
% \usepackage[normalem]{ulem}
\usepackage{lipsum}
\makeatletter
% \newcommand\colorwave[1][blue]{\bgroup \markoverwith{\lower3.5\p@\hbox{\sixly \textcolor{#1}{\char58}}}\ULon}
%\font\sixly=lasy6 % does not re-load if already loaded, so no memory problem.

\newmdtheoremenv[
linewidth= 1pt,linecolor= blue,%
leftmargin=20,rightmargin=20,innertopmargin=0pt, innerrightmargin=40,%
tikzsetting = { draw=lightgray, line width = 0.3pt,dashed,%
dash pattern = on 15pt off 3pt},%
splittopskip=\topskip,skipbelow=\baselineskip,%
skipabove=\baselineskip,ntheorem,roundcorner=0pt,
% backgroundcolor=pagebg,font=\color{orange}\sffamily, fontcolor=white
]{examplebox}{Exemple}[section]



\newcommand\R{\mathbb{R}}
\newcommand\Z{\mathbb{Z}}
\newcommand\N{\mathbb{N}}
\newcommand\E{\mathbb{E}}
\newcommand\F{\mathcal{F}}
\newcommand\cH{\mathcal{H}}
\newcommand\V{\mathbb{V}}
\newcommand\dmo{ ^{-1} }
\newcommand\kapa{\kappa}
\newcommand\im{Im}
\newcommand\hs{\mathcal{H}}





\usepackage{soul}

\makeatletter
\newcommand*{\whiten}[1]{\llap{\textcolor{white}{{\the\SOUL@token}}\hspace{#1pt}}}
\DeclareRobustCommand*\myul{%
    \def\SOUL@everyspace{\underline{\space}\kern\z@}%
    \def\SOUL@everytoken{%
     \setbox0=\hbox{\the\SOUL@token}%
     \ifdim\dp0>\z@
        \raisebox{\dp0}{\underline{\phantom{\the\SOUL@token}}}%
        \whiten{1}\whiten{0}%
        \whiten{-1}\whiten{-2}%
        \llap{\the\SOUL@token}%
     \else
        \underline{\the\SOUL@token}%
     \fi}%
\SOUL@}
\makeatother

\newcommand*{\demp}{\fontfamily{lmtt}\selectfont}

\DeclareTextFontCommand{\textdemp}{\demp}

\begin{document}

\ifcomment
Multiline
comment
\fi
\ifcomment
\myul{Typesetting test}
% \color[rgb]{1,1,1}
$∑_i^n≠ 60º±∞π∆¬≈√j∫h≤≥µ$

$\CR \R\pro\ind\pro\gS\pro
\mqty[a&b\\c&d]$
$\pro\mathbb{P}$
$\dd{x}$

  \[
    \alpha(x)=\left\{
                \begin{array}{ll}
                  x\\
                  \frac{1}{1+e^{-kx}}\\
                  \frac{e^x-e^{-x}}{e^x+e^{-x}}
                \end{array}
              \right.
  \]

  $\expval{x}$
  
  $\chi_\rho(ghg\dmo)=\Tr(\rho_{ghg\dmo})=\Tr(\rho_g\circ\rho_h\circ\rho\dmo_g)=\Tr(\rho_h)\overset{\mbox{\scalebox{0.5}{$\Tr(AB)=\Tr(BA)$}}}{=}\chi_\rho(h)$
  	$\mathop{\oplus}_{\substack{x\in X}}$

$\mat(\rho_g)=(a_{ij}(g))_{\scriptsize \substack{1\leq i\leq d \\ 1\leq j\leq d}}$ et $\mat(\rho'_g)=(a'_{ij}(g))_{\scriptsize \substack{1\leq i'\leq d' \\ 1\leq j'\leq d'}}$



\[\int_a^b{\mathbb{R}^2}g(u, v)\dd{P_{XY}}(u, v)=\iint g(u,v) f_{XY}(u, v)\dd \lambda(u) \dd \lambda(v)\]
$$\lim_{x\to\infty} f(x)$$	
$$\iiiint_V \mu(t,u,v,w) \,dt\,du\,dv\,dw$$
$$\sum_{n=1}^{\infty} 2^{-n} = 1$$	
\begin{definition}
	Si $X$ et $Y$ sont 2 v.a. ou definit la \textsc{Covariance} entre $X$ et $Y$ comme
	$\cov(X,Y)\overset{\text{def}}{=}\E\left[(X-\E(X))(Y-\E(Y))\right]=\E(XY)-\E(X)\E(Y)$.
\end{definition}
\fi
\pagebreak

% \tableofcontents

% insert your code here
%\input{./algebra/main.tex}
%\input{./geometrie-differentielle/main.tex}
%\input{./probabilite/main.tex}
%\input{./analyse-fonctionnelle/main.tex}
% \input{./Analyse-convexe-et-dualite-en-optimisation/main.tex}
%\input{./tikz/main.tex}
%\input{./Theorie-du-distributions/main.tex}
%\input{./optimisation/mine.tex}
 \input{./modelisation/main.tex}

% yves.aubry@univ-tln.fr : algebra

\end{document}


% yves.aubry@univ-tln.fr : algebra

\end{document}

%% !TEX encoding = UTF-8 Unicode
% !TEX TS-program = xelatex

\documentclass[french]{report}

%\usepackage[utf8]{inputenc}
%\usepackage[T1]{fontenc}
\usepackage{babel}


\newif\ifcomment
%\commenttrue # Show comments

\usepackage{physics}
\usepackage{amssymb}


\usepackage{amsthm}
% \usepackage{thmtools}
\usepackage{mathtools}
\usepackage{amsfonts}

\usepackage{color}

\usepackage{tikz}

\usepackage{geometry}
\geometry{a5paper, margin=0.1in, right=1cm}

\usepackage{dsfont}

\usepackage{graphicx}
\graphicspath{ {images/} }

\usepackage{faktor}

\usepackage{IEEEtrantools}
\usepackage{enumerate}   
\usepackage[PostScript=dvips]{"/Users/aware/Documents/Courses/diagrams"}


\newtheorem{theorem}{Théorème}[section]
\renewcommand{\thetheorem}{\arabic{theorem}}
\newtheorem{lemme}{Lemme}[section]
\renewcommand{\thelemme}{\arabic{lemme}}
\newtheorem{proposition}{Proposition}[section]
\renewcommand{\theproposition}{\arabic{proposition}}
\newtheorem{notations}{Notations}[section]
\newtheorem{problem}{Problème}[section]
\newtheorem{corollary}{Corollaire}[theorem]
\renewcommand{\thecorollary}{\arabic{corollary}}
\newtheorem{property}{Propriété}[section]
\newtheorem{objective}{Objectif}[section]

\theoremstyle{definition}
\newtheorem{definition}{Définition}[section]
\renewcommand{\thedefinition}{\arabic{definition}}
\newtheorem{exercise}{Exercice}[chapter]
\renewcommand{\theexercise}{\arabic{exercise}}
\newtheorem{example}{Exemple}[chapter]
\renewcommand{\theexample}{\arabic{example}}
\newtheorem*{solution}{Solution}
\newtheorem*{application}{Application}
\newtheorem*{notation}{Notation}
\newtheorem*{vocabulary}{Vocabulaire}
\newtheorem*{properties}{Propriétés}



\theoremstyle{remark}
\newtheorem*{remark}{Remarque}
\newtheorem*{rappel}{Rappel}


\usepackage{etoolbox}
\AtBeginEnvironment{exercise}{\small}
\AtBeginEnvironment{example}{\small}

\usepackage{cases}
\usepackage[red]{mypack}

\usepackage[framemethod=TikZ]{mdframed}

\definecolor{bg}{rgb}{0.4,0.25,0.95}
\definecolor{pagebg}{rgb}{0,0,0.5}
\surroundwithmdframed[
   topline=false,
   rightline=false,
   bottomline=false,
   leftmargin=\parindent,
   skipabove=8pt,
   skipbelow=8pt,
   linecolor=blue,
   innerbottommargin=10pt,
   % backgroundcolor=bg,font=\color{orange}\sffamily, fontcolor=white
]{definition}

\usepackage{empheq}
\usepackage[most]{tcolorbox}

\newtcbox{\mymath}[1][]{%
    nobeforeafter, math upper, tcbox raise base,
    enhanced, colframe=blue!30!black,
    colback=red!10, boxrule=1pt,
    #1}

\usepackage{unixode}


\DeclareMathOperator{\ord}{ord}
\DeclareMathOperator{\orb}{orb}
\DeclareMathOperator{\stab}{stab}
\DeclareMathOperator{\Stab}{stab}
\DeclareMathOperator{\ppcm}{ppcm}
\DeclareMathOperator{\conj}{Conj}
\DeclareMathOperator{\End}{End}
\DeclareMathOperator{\rot}{rot}
\DeclareMathOperator{\trs}{trace}
\DeclareMathOperator{\Ind}{Ind}
\DeclareMathOperator{\mat}{Mat}
\DeclareMathOperator{\id}{Id}
\DeclareMathOperator{\vect}{vect}
\DeclareMathOperator{\img}{img}
\DeclareMathOperator{\cov}{Cov}
\DeclareMathOperator{\dist}{dist}
\DeclareMathOperator{\irr}{Irr}
\DeclareMathOperator{\image}{Im}
\DeclareMathOperator{\pd}{\partial}
\DeclareMathOperator{\epi}{epi}
\DeclareMathOperator{\Argmin}{Argmin}
\DeclareMathOperator{\dom}{dom}
\DeclareMathOperator{\proj}{proj}
\DeclareMathOperator{\ctg}{ctg}
\DeclareMathOperator{\supp}{supp}
\DeclareMathOperator{\argmin}{argmin}
\DeclareMathOperator{\mult}{mult}
\DeclareMathOperator{\ch}{ch}
\DeclareMathOperator{\sh}{sh}
\DeclareMathOperator{\rang}{rang}
\DeclareMathOperator{\diam}{diam}
\DeclareMathOperator{\Epigraphe}{Epigraphe}




\usepackage{xcolor}
\everymath{\color{blue}}
%\everymath{\color[rgb]{0,1,1}}
%\pagecolor[rgb]{0,0,0.5}


\newcommand*{\pdtest}[3][]{\ensuremath{\frac{\partial^{#1} #2}{\partial #3}}}

\newcommand*{\deffunc}[6][]{\ensuremath{
\begin{array}{rcl}
#2 : #3 &\rightarrow& #4\\
#5 &\mapsto& #6
\end{array}
}}

\newcommand{\eqcolon}{\mathrel{\resizebox{\widthof{$\mathord{=}$}}{\height}{ $\!\!=\!\!\resizebox{1.2\width}{0.8\height}{\raisebox{0.23ex}{$\mathop{:}$}}\!\!$ }}}
\newcommand{\coloneq}{\mathrel{\resizebox{\widthof{$\mathord{=}$}}{\height}{ $\!\!\resizebox{1.2\width}{0.8\height}{\raisebox{0.23ex}{$\mathop{:}$}}\!\!=\!\!$ }}}
\newcommand{\eqcolonl}{\ensuremath{\mathrel{=\!\!\mathop{:}}}}
\newcommand{\coloneql}{\ensuremath{\mathrel{\mathop{:} \!\! =}}}
\newcommand{\vc}[1]{% inline column vector
  \left(\begin{smallmatrix}#1\end{smallmatrix}\right)%
}
\newcommand{\vr}[1]{% inline row vector
  \begin{smallmatrix}(\,#1\,)\end{smallmatrix}%
}
\makeatletter
\newcommand*{\defeq}{\ =\mathrel{\rlap{%
                     \raisebox{0.3ex}{$\m@th\cdot$}}%
                     \raisebox{-0.3ex}{$\m@th\cdot$}}%
                     }
\makeatother

\newcommand{\mathcircle}[1]{% inline row vector
 \overset{\circ}{#1}
}
\newcommand{\ulim}{% low limit
 \underline{\lim}
}
\newcommand{\ssi}{% iff
\iff
}
\newcommand{\ps}[2]{
\expval{#1 | #2}
}
\newcommand{\df}[1]{
\mqty{#1}
}
\newcommand{\n}[1]{
\norm{#1}
}
\newcommand{\sys}[1]{
\left\{\smqty{#1}\right.
}


\newcommand{\eqdef}{\ensuremath{\overset{\text{def}}=}}


\def\Circlearrowright{\ensuremath{%
  \rotatebox[origin=c]{230}{$\circlearrowright$}}}

\newcommand\ct[1]{\text{\rmfamily\upshape #1}}
\newcommand\question[1]{ {\color{red} ...!? \small #1}}
\newcommand\caz[1]{\left\{\begin{array} #1 \end{array}\right.}
\newcommand\const{\text{\rmfamily\upshape const}}
\newcommand\toP{ \overset{\pro}{\to}}
\newcommand\toPP{ \overset{\text{PP}}{\to}}
\newcommand{\oeq}{\mathrel{\text{\textcircled{$=$}}}}





\usepackage{xcolor}
% \usepackage[normalem]{ulem}
\usepackage{lipsum}
\makeatletter
% \newcommand\colorwave[1][blue]{\bgroup \markoverwith{\lower3.5\p@\hbox{\sixly \textcolor{#1}{\char58}}}\ULon}
%\font\sixly=lasy6 % does not re-load if already loaded, so no memory problem.

\newmdtheoremenv[
linewidth= 1pt,linecolor= blue,%
leftmargin=20,rightmargin=20,innertopmargin=0pt, innerrightmargin=40,%
tikzsetting = { draw=lightgray, line width = 0.3pt,dashed,%
dash pattern = on 15pt off 3pt},%
splittopskip=\topskip,skipbelow=\baselineskip,%
skipabove=\baselineskip,ntheorem,roundcorner=0pt,
% backgroundcolor=pagebg,font=\color{orange}\sffamily, fontcolor=white
]{examplebox}{Exemple}[section]



\newcommand\R{\mathbb{R}}
\newcommand\Z{\mathbb{Z}}
\newcommand\N{\mathbb{N}}
\newcommand\E{\mathbb{E}}
\newcommand\F{\mathcal{F}}
\newcommand\cH{\mathcal{H}}
\newcommand\V{\mathbb{V}}
\newcommand\dmo{ ^{-1} }
\newcommand\kapa{\kappa}
\newcommand\im{Im}
\newcommand\hs{\mathcal{H}}





\usepackage{soul}

\makeatletter
\newcommand*{\whiten}[1]{\llap{\textcolor{white}{{\the\SOUL@token}}\hspace{#1pt}}}
\DeclareRobustCommand*\myul{%
    \def\SOUL@everyspace{\underline{\space}\kern\z@}%
    \def\SOUL@everytoken{%
     \setbox0=\hbox{\the\SOUL@token}%
     \ifdim\dp0>\z@
        \raisebox{\dp0}{\underline{\phantom{\the\SOUL@token}}}%
        \whiten{1}\whiten{0}%
        \whiten{-1}\whiten{-2}%
        \llap{\the\SOUL@token}%
     \else
        \underline{\the\SOUL@token}%
     \fi}%
\SOUL@}
\makeatother

\newcommand*{\demp}{\fontfamily{lmtt}\selectfont}

\DeclareTextFontCommand{\textdemp}{\demp}

\begin{document}

\ifcomment
Multiline
comment
\fi
\ifcomment
\myul{Typesetting test}
% \color[rgb]{1,1,1}
$∑_i^n≠ 60º±∞π∆¬≈√j∫h≤≥µ$

$\CR \R\pro\ind\pro\gS\pro
\mqty[a&b\\c&d]$
$\pro\mathbb{P}$
$\dd{x}$

  \[
    \alpha(x)=\left\{
                \begin{array}{ll}
                  x\\
                  \frac{1}{1+e^{-kx}}\\
                  \frac{e^x-e^{-x}}{e^x+e^{-x}}
                \end{array}
              \right.
  \]

  $\expval{x}$
  
  $\chi_\rho(ghg\dmo)=\Tr(\rho_{ghg\dmo})=\Tr(\rho_g\circ\rho_h\circ\rho\dmo_g)=\Tr(\rho_h)\overset{\mbox{\scalebox{0.5}{$\Tr(AB)=\Tr(BA)$}}}{=}\chi_\rho(h)$
  	$\mathop{\oplus}_{\substack{x\in X}}$

$\mat(\rho_g)=(a_{ij}(g))_{\scriptsize \substack{1\leq i\leq d \\ 1\leq j\leq d}}$ et $\mat(\rho'_g)=(a'_{ij}(g))_{\scriptsize \substack{1\leq i'\leq d' \\ 1\leq j'\leq d'}}$



\[\int_a^b{\mathbb{R}^2}g(u, v)\dd{P_{XY}}(u, v)=\iint g(u,v) f_{XY}(u, v)\dd \lambda(u) \dd \lambda(v)\]
$$\lim_{x\to\infty} f(x)$$	
$$\iiiint_V \mu(t,u,v,w) \,dt\,du\,dv\,dw$$
$$\sum_{n=1}^{\infty} 2^{-n} = 1$$	
\begin{definition}
	Si $X$ et $Y$ sont 2 v.a. ou definit la \textsc{Covariance} entre $X$ et $Y$ comme
	$\cov(X,Y)\overset{\text{def}}{=}\E\left[(X-\E(X))(Y-\E(Y))\right]=\E(XY)-\E(X)\E(Y)$.
\end{definition}
\fi
\pagebreak

% \tableofcontents

% insert your code here
%% !TEX encoding = UTF-8 Unicode
% !TEX TS-program = xelatex

\documentclass[french]{report}

%\usepackage[utf8]{inputenc}
%\usepackage[T1]{fontenc}
\usepackage{babel}


\newif\ifcomment
%\commenttrue # Show comments

\usepackage{physics}
\usepackage{amssymb}


\usepackage{amsthm}
% \usepackage{thmtools}
\usepackage{mathtools}
\usepackage{amsfonts}

\usepackage{color}

\usepackage{tikz}

\usepackage{geometry}
\geometry{a5paper, margin=0.1in, right=1cm}

\usepackage{dsfont}

\usepackage{graphicx}
\graphicspath{ {images/} }

\usepackage{faktor}

\usepackage{IEEEtrantools}
\usepackage{enumerate}   
\usepackage[PostScript=dvips]{"/Users/aware/Documents/Courses/diagrams"}


\newtheorem{theorem}{Théorème}[section]
\renewcommand{\thetheorem}{\arabic{theorem}}
\newtheorem{lemme}{Lemme}[section]
\renewcommand{\thelemme}{\arabic{lemme}}
\newtheorem{proposition}{Proposition}[section]
\renewcommand{\theproposition}{\arabic{proposition}}
\newtheorem{notations}{Notations}[section]
\newtheorem{problem}{Problème}[section]
\newtheorem{corollary}{Corollaire}[theorem]
\renewcommand{\thecorollary}{\arabic{corollary}}
\newtheorem{property}{Propriété}[section]
\newtheorem{objective}{Objectif}[section]

\theoremstyle{definition}
\newtheorem{definition}{Définition}[section]
\renewcommand{\thedefinition}{\arabic{definition}}
\newtheorem{exercise}{Exercice}[chapter]
\renewcommand{\theexercise}{\arabic{exercise}}
\newtheorem{example}{Exemple}[chapter]
\renewcommand{\theexample}{\arabic{example}}
\newtheorem*{solution}{Solution}
\newtheorem*{application}{Application}
\newtheorem*{notation}{Notation}
\newtheorem*{vocabulary}{Vocabulaire}
\newtheorem*{properties}{Propriétés}



\theoremstyle{remark}
\newtheorem*{remark}{Remarque}
\newtheorem*{rappel}{Rappel}


\usepackage{etoolbox}
\AtBeginEnvironment{exercise}{\small}
\AtBeginEnvironment{example}{\small}

\usepackage{cases}
\usepackage[red]{mypack}

\usepackage[framemethod=TikZ]{mdframed}

\definecolor{bg}{rgb}{0.4,0.25,0.95}
\definecolor{pagebg}{rgb}{0,0,0.5}
\surroundwithmdframed[
   topline=false,
   rightline=false,
   bottomline=false,
   leftmargin=\parindent,
   skipabove=8pt,
   skipbelow=8pt,
   linecolor=blue,
   innerbottommargin=10pt,
   % backgroundcolor=bg,font=\color{orange}\sffamily, fontcolor=white
]{definition}

\usepackage{empheq}
\usepackage[most]{tcolorbox}

\newtcbox{\mymath}[1][]{%
    nobeforeafter, math upper, tcbox raise base,
    enhanced, colframe=blue!30!black,
    colback=red!10, boxrule=1pt,
    #1}

\usepackage{unixode}


\DeclareMathOperator{\ord}{ord}
\DeclareMathOperator{\orb}{orb}
\DeclareMathOperator{\stab}{stab}
\DeclareMathOperator{\Stab}{stab}
\DeclareMathOperator{\ppcm}{ppcm}
\DeclareMathOperator{\conj}{Conj}
\DeclareMathOperator{\End}{End}
\DeclareMathOperator{\rot}{rot}
\DeclareMathOperator{\trs}{trace}
\DeclareMathOperator{\Ind}{Ind}
\DeclareMathOperator{\mat}{Mat}
\DeclareMathOperator{\id}{Id}
\DeclareMathOperator{\vect}{vect}
\DeclareMathOperator{\img}{img}
\DeclareMathOperator{\cov}{Cov}
\DeclareMathOperator{\dist}{dist}
\DeclareMathOperator{\irr}{Irr}
\DeclareMathOperator{\image}{Im}
\DeclareMathOperator{\pd}{\partial}
\DeclareMathOperator{\epi}{epi}
\DeclareMathOperator{\Argmin}{Argmin}
\DeclareMathOperator{\dom}{dom}
\DeclareMathOperator{\proj}{proj}
\DeclareMathOperator{\ctg}{ctg}
\DeclareMathOperator{\supp}{supp}
\DeclareMathOperator{\argmin}{argmin}
\DeclareMathOperator{\mult}{mult}
\DeclareMathOperator{\ch}{ch}
\DeclareMathOperator{\sh}{sh}
\DeclareMathOperator{\rang}{rang}
\DeclareMathOperator{\diam}{diam}
\DeclareMathOperator{\Epigraphe}{Epigraphe}




\usepackage{xcolor}
\everymath{\color{blue}}
%\everymath{\color[rgb]{0,1,1}}
%\pagecolor[rgb]{0,0,0.5}


\newcommand*{\pdtest}[3][]{\ensuremath{\frac{\partial^{#1} #2}{\partial #3}}}

\newcommand*{\deffunc}[6][]{\ensuremath{
\begin{array}{rcl}
#2 : #3 &\rightarrow& #4\\
#5 &\mapsto& #6
\end{array}
}}

\newcommand{\eqcolon}{\mathrel{\resizebox{\widthof{$\mathord{=}$}}{\height}{ $\!\!=\!\!\resizebox{1.2\width}{0.8\height}{\raisebox{0.23ex}{$\mathop{:}$}}\!\!$ }}}
\newcommand{\coloneq}{\mathrel{\resizebox{\widthof{$\mathord{=}$}}{\height}{ $\!\!\resizebox{1.2\width}{0.8\height}{\raisebox{0.23ex}{$\mathop{:}$}}\!\!=\!\!$ }}}
\newcommand{\eqcolonl}{\ensuremath{\mathrel{=\!\!\mathop{:}}}}
\newcommand{\coloneql}{\ensuremath{\mathrel{\mathop{:} \!\! =}}}
\newcommand{\vc}[1]{% inline column vector
  \left(\begin{smallmatrix}#1\end{smallmatrix}\right)%
}
\newcommand{\vr}[1]{% inline row vector
  \begin{smallmatrix}(\,#1\,)\end{smallmatrix}%
}
\makeatletter
\newcommand*{\defeq}{\ =\mathrel{\rlap{%
                     \raisebox{0.3ex}{$\m@th\cdot$}}%
                     \raisebox{-0.3ex}{$\m@th\cdot$}}%
                     }
\makeatother

\newcommand{\mathcircle}[1]{% inline row vector
 \overset{\circ}{#1}
}
\newcommand{\ulim}{% low limit
 \underline{\lim}
}
\newcommand{\ssi}{% iff
\iff
}
\newcommand{\ps}[2]{
\expval{#1 | #2}
}
\newcommand{\df}[1]{
\mqty{#1}
}
\newcommand{\n}[1]{
\norm{#1}
}
\newcommand{\sys}[1]{
\left\{\smqty{#1}\right.
}


\newcommand{\eqdef}{\ensuremath{\overset{\text{def}}=}}


\def\Circlearrowright{\ensuremath{%
  \rotatebox[origin=c]{230}{$\circlearrowright$}}}

\newcommand\ct[1]{\text{\rmfamily\upshape #1}}
\newcommand\question[1]{ {\color{red} ...!? \small #1}}
\newcommand\caz[1]{\left\{\begin{array} #1 \end{array}\right.}
\newcommand\const{\text{\rmfamily\upshape const}}
\newcommand\toP{ \overset{\pro}{\to}}
\newcommand\toPP{ \overset{\text{PP}}{\to}}
\newcommand{\oeq}{\mathrel{\text{\textcircled{$=$}}}}





\usepackage{xcolor}
% \usepackage[normalem]{ulem}
\usepackage{lipsum}
\makeatletter
% \newcommand\colorwave[1][blue]{\bgroup \markoverwith{\lower3.5\p@\hbox{\sixly \textcolor{#1}{\char58}}}\ULon}
%\font\sixly=lasy6 % does not re-load if already loaded, so no memory problem.

\newmdtheoremenv[
linewidth= 1pt,linecolor= blue,%
leftmargin=20,rightmargin=20,innertopmargin=0pt, innerrightmargin=40,%
tikzsetting = { draw=lightgray, line width = 0.3pt,dashed,%
dash pattern = on 15pt off 3pt},%
splittopskip=\topskip,skipbelow=\baselineskip,%
skipabove=\baselineskip,ntheorem,roundcorner=0pt,
% backgroundcolor=pagebg,font=\color{orange}\sffamily, fontcolor=white
]{examplebox}{Exemple}[section]



\newcommand\R{\mathbb{R}}
\newcommand\Z{\mathbb{Z}}
\newcommand\N{\mathbb{N}}
\newcommand\E{\mathbb{E}}
\newcommand\F{\mathcal{F}}
\newcommand\cH{\mathcal{H}}
\newcommand\V{\mathbb{V}}
\newcommand\dmo{ ^{-1} }
\newcommand\kapa{\kappa}
\newcommand\im{Im}
\newcommand\hs{\mathcal{H}}





\usepackage{soul}

\makeatletter
\newcommand*{\whiten}[1]{\llap{\textcolor{white}{{\the\SOUL@token}}\hspace{#1pt}}}
\DeclareRobustCommand*\myul{%
    \def\SOUL@everyspace{\underline{\space}\kern\z@}%
    \def\SOUL@everytoken{%
     \setbox0=\hbox{\the\SOUL@token}%
     \ifdim\dp0>\z@
        \raisebox{\dp0}{\underline{\phantom{\the\SOUL@token}}}%
        \whiten{1}\whiten{0}%
        \whiten{-1}\whiten{-2}%
        \llap{\the\SOUL@token}%
     \else
        \underline{\the\SOUL@token}%
     \fi}%
\SOUL@}
\makeatother

\newcommand*{\demp}{\fontfamily{lmtt}\selectfont}

\DeclareTextFontCommand{\textdemp}{\demp}

\begin{document}

\ifcomment
Multiline
comment
\fi
\ifcomment
\myul{Typesetting test}
% \color[rgb]{1,1,1}
$∑_i^n≠ 60º±∞π∆¬≈√j∫h≤≥µ$

$\CR \R\pro\ind\pro\gS\pro
\mqty[a&b\\c&d]$
$\pro\mathbb{P}$
$\dd{x}$

  \[
    \alpha(x)=\left\{
                \begin{array}{ll}
                  x\\
                  \frac{1}{1+e^{-kx}}\\
                  \frac{e^x-e^{-x}}{e^x+e^{-x}}
                \end{array}
              \right.
  \]

  $\expval{x}$
  
  $\chi_\rho(ghg\dmo)=\Tr(\rho_{ghg\dmo})=\Tr(\rho_g\circ\rho_h\circ\rho\dmo_g)=\Tr(\rho_h)\overset{\mbox{\scalebox{0.5}{$\Tr(AB)=\Tr(BA)$}}}{=}\chi_\rho(h)$
  	$\mathop{\oplus}_{\substack{x\in X}}$

$\mat(\rho_g)=(a_{ij}(g))_{\scriptsize \substack{1\leq i\leq d \\ 1\leq j\leq d}}$ et $\mat(\rho'_g)=(a'_{ij}(g))_{\scriptsize \substack{1\leq i'\leq d' \\ 1\leq j'\leq d'}}$



\[\int_a^b{\mathbb{R}^2}g(u, v)\dd{P_{XY}}(u, v)=\iint g(u,v) f_{XY}(u, v)\dd \lambda(u) \dd \lambda(v)\]
$$\lim_{x\to\infty} f(x)$$	
$$\iiiint_V \mu(t,u,v,w) \,dt\,du\,dv\,dw$$
$$\sum_{n=1}^{\infty} 2^{-n} = 1$$	
\begin{definition}
	Si $X$ et $Y$ sont 2 v.a. ou definit la \textsc{Covariance} entre $X$ et $Y$ comme
	$\cov(X,Y)\overset{\text{def}}{=}\E\left[(X-\E(X))(Y-\E(Y))\right]=\E(XY)-\E(X)\E(Y)$.
\end{definition}
\fi
\pagebreak

% \tableofcontents

% insert your code here
%\input{./algebra/main.tex}
%\input{./geometrie-differentielle/main.tex}
%\input{./probabilite/main.tex}
%\input{./analyse-fonctionnelle/main.tex}
% \input{./Analyse-convexe-et-dualite-en-optimisation/main.tex}
%\input{./tikz/main.tex}
%\input{./Theorie-du-distributions/main.tex}
%\input{./optimisation/mine.tex}
 \input{./modelisation/main.tex}

% yves.aubry@univ-tln.fr : algebra

\end{document}

%% !TEX encoding = UTF-8 Unicode
% !TEX TS-program = xelatex

\documentclass[french]{report}

%\usepackage[utf8]{inputenc}
%\usepackage[T1]{fontenc}
\usepackage{babel}


\newif\ifcomment
%\commenttrue # Show comments

\usepackage{physics}
\usepackage{amssymb}


\usepackage{amsthm}
% \usepackage{thmtools}
\usepackage{mathtools}
\usepackage{amsfonts}

\usepackage{color}

\usepackage{tikz}

\usepackage{geometry}
\geometry{a5paper, margin=0.1in, right=1cm}

\usepackage{dsfont}

\usepackage{graphicx}
\graphicspath{ {images/} }

\usepackage{faktor}

\usepackage{IEEEtrantools}
\usepackage{enumerate}   
\usepackage[PostScript=dvips]{"/Users/aware/Documents/Courses/diagrams"}


\newtheorem{theorem}{Théorème}[section]
\renewcommand{\thetheorem}{\arabic{theorem}}
\newtheorem{lemme}{Lemme}[section]
\renewcommand{\thelemme}{\arabic{lemme}}
\newtheorem{proposition}{Proposition}[section]
\renewcommand{\theproposition}{\arabic{proposition}}
\newtheorem{notations}{Notations}[section]
\newtheorem{problem}{Problème}[section]
\newtheorem{corollary}{Corollaire}[theorem]
\renewcommand{\thecorollary}{\arabic{corollary}}
\newtheorem{property}{Propriété}[section]
\newtheorem{objective}{Objectif}[section]

\theoremstyle{definition}
\newtheorem{definition}{Définition}[section]
\renewcommand{\thedefinition}{\arabic{definition}}
\newtheorem{exercise}{Exercice}[chapter]
\renewcommand{\theexercise}{\arabic{exercise}}
\newtheorem{example}{Exemple}[chapter]
\renewcommand{\theexample}{\arabic{example}}
\newtheorem*{solution}{Solution}
\newtheorem*{application}{Application}
\newtheorem*{notation}{Notation}
\newtheorem*{vocabulary}{Vocabulaire}
\newtheorem*{properties}{Propriétés}



\theoremstyle{remark}
\newtheorem*{remark}{Remarque}
\newtheorem*{rappel}{Rappel}


\usepackage{etoolbox}
\AtBeginEnvironment{exercise}{\small}
\AtBeginEnvironment{example}{\small}

\usepackage{cases}
\usepackage[red]{mypack}

\usepackage[framemethod=TikZ]{mdframed}

\definecolor{bg}{rgb}{0.4,0.25,0.95}
\definecolor{pagebg}{rgb}{0,0,0.5}
\surroundwithmdframed[
   topline=false,
   rightline=false,
   bottomline=false,
   leftmargin=\parindent,
   skipabove=8pt,
   skipbelow=8pt,
   linecolor=blue,
   innerbottommargin=10pt,
   % backgroundcolor=bg,font=\color{orange}\sffamily, fontcolor=white
]{definition}

\usepackage{empheq}
\usepackage[most]{tcolorbox}

\newtcbox{\mymath}[1][]{%
    nobeforeafter, math upper, tcbox raise base,
    enhanced, colframe=blue!30!black,
    colback=red!10, boxrule=1pt,
    #1}

\usepackage{unixode}


\DeclareMathOperator{\ord}{ord}
\DeclareMathOperator{\orb}{orb}
\DeclareMathOperator{\stab}{stab}
\DeclareMathOperator{\Stab}{stab}
\DeclareMathOperator{\ppcm}{ppcm}
\DeclareMathOperator{\conj}{Conj}
\DeclareMathOperator{\End}{End}
\DeclareMathOperator{\rot}{rot}
\DeclareMathOperator{\trs}{trace}
\DeclareMathOperator{\Ind}{Ind}
\DeclareMathOperator{\mat}{Mat}
\DeclareMathOperator{\id}{Id}
\DeclareMathOperator{\vect}{vect}
\DeclareMathOperator{\img}{img}
\DeclareMathOperator{\cov}{Cov}
\DeclareMathOperator{\dist}{dist}
\DeclareMathOperator{\irr}{Irr}
\DeclareMathOperator{\image}{Im}
\DeclareMathOperator{\pd}{\partial}
\DeclareMathOperator{\epi}{epi}
\DeclareMathOperator{\Argmin}{Argmin}
\DeclareMathOperator{\dom}{dom}
\DeclareMathOperator{\proj}{proj}
\DeclareMathOperator{\ctg}{ctg}
\DeclareMathOperator{\supp}{supp}
\DeclareMathOperator{\argmin}{argmin}
\DeclareMathOperator{\mult}{mult}
\DeclareMathOperator{\ch}{ch}
\DeclareMathOperator{\sh}{sh}
\DeclareMathOperator{\rang}{rang}
\DeclareMathOperator{\diam}{diam}
\DeclareMathOperator{\Epigraphe}{Epigraphe}




\usepackage{xcolor}
\everymath{\color{blue}}
%\everymath{\color[rgb]{0,1,1}}
%\pagecolor[rgb]{0,0,0.5}


\newcommand*{\pdtest}[3][]{\ensuremath{\frac{\partial^{#1} #2}{\partial #3}}}

\newcommand*{\deffunc}[6][]{\ensuremath{
\begin{array}{rcl}
#2 : #3 &\rightarrow& #4\\
#5 &\mapsto& #6
\end{array}
}}

\newcommand{\eqcolon}{\mathrel{\resizebox{\widthof{$\mathord{=}$}}{\height}{ $\!\!=\!\!\resizebox{1.2\width}{0.8\height}{\raisebox{0.23ex}{$\mathop{:}$}}\!\!$ }}}
\newcommand{\coloneq}{\mathrel{\resizebox{\widthof{$\mathord{=}$}}{\height}{ $\!\!\resizebox{1.2\width}{0.8\height}{\raisebox{0.23ex}{$\mathop{:}$}}\!\!=\!\!$ }}}
\newcommand{\eqcolonl}{\ensuremath{\mathrel{=\!\!\mathop{:}}}}
\newcommand{\coloneql}{\ensuremath{\mathrel{\mathop{:} \!\! =}}}
\newcommand{\vc}[1]{% inline column vector
  \left(\begin{smallmatrix}#1\end{smallmatrix}\right)%
}
\newcommand{\vr}[1]{% inline row vector
  \begin{smallmatrix}(\,#1\,)\end{smallmatrix}%
}
\makeatletter
\newcommand*{\defeq}{\ =\mathrel{\rlap{%
                     \raisebox{0.3ex}{$\m@th\cdot$}}%
                     \raisebox{-0.3ex}{$\m@th\cdot$}}%
                     }
\makeatother

\newcommand{\mathcircle}[1]{% inline row vector
 \overset{\circ}{#1}
}
\newcommand{\ulim}{% low limit
 \underline{\lim}
}
\newcommand{\ssi}{% iff
\iff
}
\newcommand{\ps}[2]{
\expval{#1 | #2}
}
\newcommand{\df}[1]{
\mqty{#1}
}
\newcommand{\n}[1]{
\norm{#1}
}
\newcommand{\sys}[1]{
\left\{\smqty{#1}\right.
}


\newcommand{\eqdef}{\ensuremath{\overset{\text{def}}=}}


\def\Circlearrowright{\ensuremath{%
  \rotatebox[origin=c]{230}{$\circlearrowright$}}}

\newcommand\ct[1]{\text{\rmfamily\upshape #1}}
\newcommand\question[1]{ {\color{red} ...!? \small #1}}
\newcommand\caz[1]{\left\{\begin{array} #1 \end{array}\right.}
\newcommand\const{\text{\rmfamily\upshape const}}
\newcommand\toP{ \overset{\pro}{\to}}
\newcommand\toPP{ \overset{\text{PP}}{\to}}
\newcommand{\oeq}{\mathrel{\text{\textcircled{$=$}}}}





\usepackage{xcolor}
% \usepackage[normalem]{ulem}
\usepackage{lipsum}
\makeatletter
% \newcommand\colorwave[1][blue]{\bgroup \markoverwith{\lower3.5\p@\hbox{\sixly \textcolor{#1}{\char58}}}\ULon}
%\font\sixly=lasy6 % does not re-load if already loaded, so no memory problem.

\newmdtheoremenv[
linewidth= 1pt,linecolor= blue,%
leftmargin=20,rightmargin=20,innertopmargin=0pt, innerrightmargin=40,%
tikzsetting = { draw=lightgray, line width = 0.3pt,dashed,%
dash pattern = on 15pt off 3pt},%
splittopskip=\topskip,skipbelow=\baselineskip,%
skipabove=\baselineskip,ntheorem,roundcorner=0pt,
% backgroundcolor=pagebg,font=\color{orange}\sffamily, fontcolor=white
]{examplebox}{Exemple}[section]



\newcommand\R{\mathbb{R}}
\newcommand\Z{\mathbb{Z}}
\newcommand\N{\mathbb{N}}
\newcommand\E{\mathbb{E}}
\newcommand\F{\mathcal{F}}
\newcommand\cH{\mathcal{H}}
\newcommand\V{\mathbb{V}}
\newcommand\dmo{ ^{-1} }
\newcommand\kapa{\kappa}
\newcommand\im{Im}
\newcommand\hs{\mathcal{H}}





\usepackage{soul}

\makeatletter
\newcommand*{\whiten}[1]{\llap{\textcolor{white}{{\the\SOUL@token}}\hspace{#1pt}}}
\DeclareRobustCommand*\myul{%
    \def\SOUL@everyspace{\underline{\space}\kern\z@}%
    \def\SOUL@everytoken{%
     \setbox0=\hbox{\the\SOUL@token}%
     \ifdim\dp0>\z@
        \raisebox{\dp0}{\underline{\phantom{\the\SOUL@token}}}%
        \whiten{1}\whiten{0}%
        \whiten{-1}\whiten{-2}%
        \llap{\the\SOUL@token}%
     \else
        \underline{\the\SOUL@token}%
     \fi}%
\SOUL@}
\makeatother

\newcommand*{\demp}{\fontfamily{lmtt}\selectfont}

\DeclareTextFontCommand{\textdemp}{\demp}

\begin{document}

\ifcomment
Multiline
comment
\fi
\ifcomment
\myul{Typesetting test}
% \color[rgb]{1,1,1}
$∑_i^n≠ 60º±∞π∆¬≈√j∫h≤≥µ$

$\CR \R\pro\ind\pro\gS\pro
\mqty[a&b\\c&d]$
$\pro\mathbb{P}$
$\dd{x}$

  \[
    \alpha(x)=\left\{
                \begin{array}{ll}
                  x\\
                  \frac{1}{1+e^{-kx}}\\
                  \frac{e^x-e^{-x}}{e^x+e^{-x}}
                \end{array}
              \right.
  \]

  $\expval{x}$
  
  $\chi_\rho(ghg\dmo)=\Tr(\rho_{ghg\dmo})=\Tr(\rho_g\circ\rho_h\circ\rho\dmo_g)=\Tr(\rho_h)\overset{\mbox{\scalebox{0.5}{$\Tr(AB)=\Tr(BA)$}}}{=}\chi_\rho(h)$
  	$\mathop{\oplus}_{\substack{x\in X}}$

$\mat(\rho_g)=(a_{ij}(g))_{\scriptsize \substack{1\leq i\leq d \\ 1\leq j\leq d}}$ et $\mat(\rho'_g)=(a'_{ij}(g))_{\scriptsize \substack{1\leq i'\leq d' \\ 1\leq j'\leq d'}}$



\[\int_a^b{\mathbb{R}^2}g(u, v)\dd{P_{XY}}(u, v)=\iint g(u,v) f_{XY}(u, v)\dd \lambda(u) \dd \lambda(v)\]
$$\lim_{x\to\infty} f(x)$$	
$$\iiiint_V \mu(t,u,v,w) \,dt\,du\,dv\,dw$$
$$\sum_{n=1}^{\infty} 2^{-n} = 1$$	
\begin{definition}
	Si $X$ et $Y$ sont 2 v.a. ou definit la \textsc{Covariance} entre $X$ et $Y$ comme
	$\cov(X,Y)\overset{\text{def}}{=}\E\left[(X-\E(X))(Y-\E(Y))\right]=\E(XY)-\E(X)\E(Y)$.
\end{definition}
\fi
\pagebreak

% \tableofcontents

% insert your code here
%\input{./algebra/main.tex}
%\input{./geometrie-differentielle/main.tex}
%\input{./probabilite/main.tex}
%\input{./analyse-fonctionnelle/main.tex}
% \input{./Analyse-convexe-et-dualite-en-optimisation/main.tex}
%\input{./tikz/main.tex}
%\input{./Theorie-du-distributions/main.tex}
%\input{./optimisation/mine.tex}
 \input{./modelisation/main.tex}

% yves.aubry@univ-tln.fr : algebra

\end{document}

%% !TEX encoding = UTF-8 Unicode
% !TEX TS-program = xelatex

\documentclass[french]{report}

%\usepackage[utf8]{inputenc}
%\usepackage[T1]{fontenc}
\usepackage{babel}


\newif\ifcomment
%\commenttrue # Show comments

\usepackage{physics}
\usepackage{amssymb}


\usepackage{amsthm}
% \usepackage{thmtools}
\usepackage{mathtools}
\usepackage{amsfonts}

\usepackage{color}

\usepackage{tikz}

\usepackage{geometry}
\geometry{a5paper, margin=0.1in, right=1cm}

\usepackage{dsfont}

\usepackage{graphicx}
\graphicspath{ {images/} }

\usepackage{faktor}

\usepackage{IEEEtrantools}
\usepackage{enumerate}   
\usepackage[PostScript=dvips]{"/Users/aware/Documents/Courses/diagrams"}


\newtheorem{theorem}{Théorème}[section]
\renewcommand{\thetheorem}{\arabic{theorem}}
\newtheorem{lemme}{Lemme}[section]
\renewcommand{\thelemme}{\arabic{lemme}}
\newtheorem{proposition}{Proposition}[section]
\renewcommand{\theproposition}{\arabic{proposition}}
\newtheorem{notations}{Notations}[section]
\newtheorem{problem}{Problème}[section]
\newtheorem{corollary}{Corollaire}[theorem]
\renewcommand{\thecorollary}{\arabic{corollary}}
\newtheorem{property}{Propriété}[section]
\newtheorem{objective}{Objectif}[section]

\theoremstyle{definition}
\newtheorem{definition}{Définition}[section]
\renewcommand{\thedefinition}{\arabic{definition}}
\newtheorem{exercise}{Exercice}[chapter]
\renewcommand{\theexercise}{\arabic{exercise}}
\newtheorem{example}{Exemple}[chapter]
\renewcommand{\theexample}{\arabic{example}}
\newtheorem*{solution}{Solution}
\newtheorem*{application}{Application}
\newtheorem*{notation}{Notation}
\newtheorem*{vocabulary}{Vocabulaire}
\newtheorem*{properties}{Propriétés}



\theoremstyle{remark}
\newtheorem*{remark}{Remarque}
\newtheorem*{rappel}{Rappel}


\usepackage{etoolbox}
\AtBeginEnvironment{exercise}{\small}
\AtBeginEnvironment{example}{\small}

\usepackage{cases}
\usepackage[red]{mypack}

\usepackage[framemethod=TikZ]{mdframed}

\definecolor{bg}{rgb}{0.4,0.25,0.95}
\definecolor{pagebg}{rgb}{0,0,0.5}
\surroundwithmdframed[
   topline=false,
   rightline=false,
   bottomline=false,
   leftmargin=\parindent,
   skipabove=8pt,
   skipbelow=8pt,
   linecolor=blue,
   innerbottommargin=10pt,
   % backgroundcolor=bg,font=\color{orange}\sffamily, fontcolor=white
]{definition}

\usepackage{empheq}
\usepackage[most]{tcolorbox}

\newtcbox{\mymath}[1][]{%
    nobeforeafter, math upper, tcbox raise base,
    enhanced, colframe=blue!30!black,
    colback=red!10, boxrule=1pt,
    #1}

\usepackage{unixode}


\DeclareMathOperator{\ord}{ord}
\DeclareMathOperator{\orb}{orb}
\DeclareMathOperator{\stab}{stab}
\DeclareMathOperator{\Stab}{stab}
\DeclareMathOperator{\ppcm}{ppcm}
\DeclareMathOperator{\conj}{Conj}
\DeclareMathOperator{\End}{End}
\DeclareMathOperator{\rot}{rot}
\DeclareMathOperator{\trs}{trace}
\DeclareMathOperator{\Ind}{Ind}
\DeclareMathOperator{\mat}{Mat}
\DeclareMathOperator{\id}{Id}
\DeclareMathOperator{\vect}{vect}
\DeclareMathOperator{\img}{img}
\DeclareMathOperator{\cov}{Cov}
\DeclareMathOperator{\dist}{dist}
\DeclareMathOperator{\irr}{Irr}
\DeclareMathOperator{\image}{Im}
\DeclareMathOperator{\pd}{\partial}
\DeclareMathOperator{\epi}{epi}
\DeclareMathOperator{\Argmin}{Argmin}
\DeclareMathOperator{\dom}{dom}
\DeclareMathOperator{\proj}{proj}
\DeclareMathOperator{\ctg}{ctg}
\DeclareMathOperator{\supp}{supp}
\DeclareMathOperator{\argmin}{argmin}
\DeclareMathOperator{\mult}{mult}
\DeclareMathOperator{\ch}{ch}
\DeclareMathOperator{\sh}{sh}
\DeclareMathOperator{\rang}{rang}
\DeclareMathOperator{\diam}{diam}
\DeclareMathOperator{\Epigraphe}{Epigraphe}




\usepackage{xcolor}
\everymath{\color{blue}}
%\everymath{\color[rgb]{0,1,1}}
%\pagecolor[rgb]{0,0,0.5}


\newcommand*{\pdtest}[3][]{\ensuremath{\frac{\partial^{#1} #2}{\partial #3}}}

\newcommand*{\deffunc}[6][]{\ensuremath{
\begin{array}{rcl}
#2 : #3 &\rightarrow& #4\\
#5 &\mapsto& #6
\end{array}
}}

\newcommand{\eqcolon}{\mathrel{\resizebox{\widthof{$\mathord{=}$}}{\height}{ $\!\!=\!\!\resizebox{1.2\width}{0.8\height}{\raisebox{0.23ex}{$\mathop{:}$}}\!\!$ }}}
\newcommand{\coloneq}{\mathrel{\resizebox{\widthof{$\mathord{=}$}}{\height}{ $\!\!\resizebox{1.2\width}{0.8\height}{\raisebox{0.23ex}{$\mathop{:}$}}\!\!=\!\!$ }}}
\newcommand{\eqcolonl}{\ensuremath{\mathrel{=\!\!\mathop{:}}}}
\newcommand{\coloneql}{\ensuremath{\mathrel{\mathop{:} \!\! =}}}
\newcommand{\vc}[1]{% inline column vector
  \left(\begin{smallmatrix}#1\end{smallmatrix}\right)%
}
\newcommand{\vr}[1]{% inline row vector
  \begin{smallmatrix}(\,#1\,)\end{smallmatrix}%
}
\makeatletter
\newcommand*{\defeq}{\ =\mathrel{\rlap{%
                     \raisebox{0.3ex}{$\m@th\cdot$}}%
                     \raisebox{-0.3ex}{$\m@th\cdot$}}%
                     }
\makeatother

\newcommand{\mathcircle}[1]{% inline row vector
 \overset{\circ}{#1}
}
\newcommand{\ulim}{% low limit
 \underline{\lim}
}
\newcommand{\ssi}{% iff
\iff
}
\newcommand{\ps}[2]{
\expval{#1 | #2}
}
\newcommand{\df}[1]{
\mqty{#1}
}
\newcommand{\n}[1]{
\norm{#1}
}
\newcommand{\sys}[1]{
\left\{\smqty{#1}\right.
}


\newcommand{\eqdef}{\ensuremath{\overset{\text{def}}=}}


\def\Circlearrowright{\ensuremath{%
  \rotatebox[origin=c]{230}{$\circlearrowright$}}}

\newcommand\ct[1]{\text{\rmfamily\upshape #1}}
\newcommand\question[1]{ {\color{red} ...!? \small #1}}
\newcommand\caz[1]{\left\{\begin{array} #1 \end{array}\right.}
\newcommand\const{\text{\rmfamily\upshape const}}
\newcommand\toP{ \overset{\pro}{\to}}
\newcommand\toPP{ \overset{\text{PP}}{\to}}
\newcommand{\oeq}{\mathrel{\text{\textcircled{$=$}}}}





\usepackage{xcolor}
% \usepackage[normalem]{ulem}
\usepackage{lipsum}
\makeatletter
% \newcommand\colorwave[1][blue]{\bgroup \markoverwith{\lower3.5\p@\hbox{\sixly \textcolor{#1}{\char58}}}\ULon}
%\font\sixly=lasy6 % does not re-load if already loaded, so no memory problem.

\newmdtheoremenv[
linewidth= 1pt,linecolor= blue,%
leftmargin=20,rightmargin=20,innertopmargin=0pt, innerrightmargin=40,%
tikzsetting = { draw=lightgray, line width = 0.3pt,dashed,%
dash pattern = on 15pt off 3pt},%
splittopskip=\topskip,skipbelow=\baselineskip,%
skipabove=\baselineskip,ntheorem,roundcorner=0pt,
% backgroundcolor=pagebg,font=\color{orange}\sffamily, fontcolor=white
]{examplebox}{Exemple}[section]



\newcommand\R{\mathbb{R}}
\newcommand\Z{\mathbb{Z}}
\newcommand\N{\mathbb{N}}
\newcommand\E{\mathbb{E}}
\newcommand\F{\mathcal{F}}
\newcommand\cH{\mathcal{H}}
\newcommand\V{\mathbb{V}}
\newcommand\dmo{ ^{-1} }
\newcommand\kapa{\kappa}
\newcommand\im{Im}
\newcommand\hs{\mathcal{H}}





\usepackage{soul}

\makeatletter
\newcommand*{\whiten}[1]{\llap{\textcolor{white}{{\the\SOUL@token}}\hspace{#1pt}}}
\DeclareRobustCommand*\myul{%
    \def\SOUL@everyspace{\underline{\space}\kern\z@}%
    \def\SOUL@everytoken{%
     \setbox0=\hbox{\the\SOUL@token}%
     \ifdim\dp0>\z@
        \raisebox{\dp0}{\underline{\phantom{\the\SOUL@token}}}%
        \whiten{1}\whiten{0}%
        \whiten{-1}\whiten{-2}%
        \llap{\the\SOUL@token}%
     \else
        \underline{\the\SOUL@token}%
     \fi}%
\SOUL@}
\makeatother

\newcommand*{\demp}{\fontfamily{lmtt}\selectfont}

\DeclareTextFontCommand{\textdemp}{\demp}

\begin{document}

\ifcomment
Multiline
comment
\fi
\ifcomment
\myul{Typesetting test}
% \color[rgb]{1,1,1}
$∑_i^n≠ 60º±∞π∆¬≈√j∫h≤≥µ$

$\CR \R\pro\ind\pro\gS\pro
\mqty[a&b\\c&d]$
$\pro\mathbb{P}$
$\dd{x}$

  \[
    \alpha(x)=\left\{
                \begin{array}{ll}
                  x\\
                  \frac{1}{1+e^{-kx}}\\
                  \frac{e^x-e^{-x}}{e^x+e^{-x}}
                \end{array}
              \right.
  \]

  $\expval{x}$
  
  $\chi_\rho(ghg\dmo)=\Tr(\rho_{ghg\dmo})=\Tr(\rho_g\circ\rho_h\circ\rho\dmo_g)=\Tr(\rho_h)\overset{\mbox{\scalebox{0.5}{$\Tr(AB)=\Tr(BA)$}}}{=}\chi_\rho(h)$
  	$\mathop{\oplus}_{\substack{x\in X}}$

$\mat(\rho_g)=(a_{ij}(g))_{\scriptsize \substack{1\leq i\leq d \\ 1\leq j\leq d}}$ et $\mat(\rho'_g)=(a'_{ij}(g))_{\scriptsize \substack{1\leq i'\leq d' \\ 1\leq j'\leq d'}}$



\[\int_a^b{\mathbb{R}^2}g(u, v)\dd{P_{XY}}(u, v)=\iint g(u,v) f_{XY}(u, v)\dd \lambda(u) \dd \lambda(v)\]
$$\lim_{x\to\infty} f(x)$$	
$$\iiiint_V \mu(t,u,v,w) \,dt\,du\,dv\,dw$$
$$\sum_{n=1}^{\infty} 2^{-n} = 1$$	
\begin{definition}
	Si $X$ et $Y$ sont 2 v.a. ou definit la \textsc{Covariance} entre $X$ et $Y$ comme
	$\cov(X,Y)\overset{\text{def}}{=}\E\left[(X-\E(X))(Y-\E(Y))\right]=\E(XY)-\E(X)\E(Y)$.
\end{definition}
\fi
\pagebreak

% \tableofcontents

% insert your code here
%\input{./algebra/main.tex}
%\input{./geometrie-differentielle/main.tex}
%\input{./probabilite/main.tex}
%\input{./analyse-fonctionnelle/main.tex}
% \input{./Analyse-convexe-et-dualite-en-optimisation/main.tex}
%\input{./tikz/main.tex}
%\input{./Theorie-du-distributions/main.tex}
%\input{./optimisation/mine.tex}
 \input{./modelisation/main.tex}

% yves.aubry@univ-tln.fr : algebra

\end{document}

%% !TEX encoding = UTF-8 Unicode
% !TEX TS-program = xelatex

\documentclass[french]{report}

%\usepackage[utf8]{inputenc}
%\usepackage[T1]{fontenc}
\usepackage{babel}


\newif\ifcomment
%\commenttrue # Show comments

\usepackage{physics}
\usepackage{amssymb}


\usepackage{amsthm}
% \usepackage{thmtools}
\usepackage{mathtools}
\usepackage{amsfonts}

\usepackage{color}

\usepackage{tikz}

\usepackage{geometry}
\geometry{a5paper, margin=0.1in, right=1cm}

\usepackage{dsfont}

\usepackage{graphicx}
\graphicspath{ {images/} }

\usepackage{faktor}

\usepackage{IEEEtrantools}
\usepackage{enumerate}   
\usepackage[PostScript=dvips]{"/Users/aware/Documents/Courses/diagrams"}


\newtheorem{theorem}{Théorème}[section]
\renewcommand{\thetheorem}{\arabic{theorem}}
\newtheorem{lemme}{Lemme}[section]
\renewcommand{\thelemme}{\arabic{lemme}}
\newtheorem{proposition}{Proposition}[section]
\renewcommand{\theproposition}{\arabic{proposition}}
\newtheorem{notations}{Notations}[section]
\newtheorem{problem}{Problème}[section]
\newtheorem{corollary}{Corollaire}[theorem]
\renewcommand{\thecorollary}{\arabic{corollary}}
\newtheorem{property}{Propriété}[section]
\newtheorem{objective}{Objectif}[section]

\theoremstyle{definition}
\newtheorem{definition}{Définition}[section]
\renewcommand{\thedefinition}{\arabic{definition}}
\newtheorem{exercise}{Exercice}[chapter]
\renewcommand{\theexercise}{\arabic{exercise}}
\newtheorem{example}{Exemple}[chapter]
\renewcommand{\theexample}{\arabic{example}}
\newtheorem*{solution}{Solution}
\newtheorem*{application}{Application}
\newtheorem*{notation}{Notation}
\newtheorem*{vocabulary}{Vocabulaire}
\newtheorem*{properties}{Propriétés}



\theoremstyle{remark}
\newtheorem*{remark}{Remarque}
\newtheorem*{rappel}{Rappel}


\usepackage{etoolbox}
\AtBeginEnvironment{exercise}{\small}
\AtBeginEnvironment{example}{\small}

\usepackage{cases}
\usepackage[red]{mypack}

\usepackage[framemethod=TikZ]{mdframed}

\definecolor{bg}{rgb}{0.4,0.25,0.95}
\definecolor{pagebg}{rgb}{0,0,0.5}
\surroundwithmdframed[
   topline=false,
   rightline=false,
   bottomline=false,
   leftmargin=\parindent,
   skipabove=8pt,
   skipbelow=8pt,
   linecolor=blue,
   innerbottommargin=10pt,
   % backgroundcolor=bg,font=\color{orange}\sffamily, fontcolor=white
]{definition}

\usepackage{empheq}
\usepackage[most]{tcolorbox}

\newtcbox{\mymath}[1][]{%
    nobeforeafter, math upper, tcbox raise base,
    enhanced, colframe=blue!30!black,
    colback=red!10, boxrule=1pt,
    #1}

\usepackage{unixode}


\DeclareMathOperator{\ord}{ord}
\DeclareMathOperator{\orb}{orb}
\DeclareMathOperator{\stab}{stab}
\DeclareMathOperator{\Stab}{stab}
\DeclareMathOperator{\ppcm}{ppcm}
\DeclareMathOperator{\conj}{Conj}
\DeclareMathOperator{\End}{End}
\DeclareMathOperator{\rot}{rot}
\DeclareMathOperator{\trs}{trace}
\DeclareMathOperator{\Ind}{Ind}
\DeclareMathOperator{\mat}{Mat}
\DeclareMathOperator{\id}{Id}
\DeclareMathOperator{\vect}{vect}
\DeclareMathOperator{\img}{img}
\DeclareMathOperator{\cov}{Cov}
\DeclareMathOperator{\dist}{dist}
\DeclareMathOperator{\irr}{Irr}
\DeclareMathOperator{\image}{Im}
\DeclareMathOperator{\pd}{\partial}
\DeclareMathOperator{\epi}{epi}
\DeclareMathOperator{\Argmin}{Argmin}
\DeclareMathOperator{\dom}{dom}
\DeclareMathOperator{\proj}{proj}
\DeclareMathOperator{\ctg}{ctg}
\DeclareMathOperator{\supp}{supp}
\DeclareMathOperator{\argmin}{argmin}
\DeclareMathOperator{\mult}{mult}
\DeclareMathOperator{\ch}{ch}
\DeclareMathOperator{\sh}{sh}
\DeclareMathOperator{\rang}{rang}
\DeclareMathOperator{\diam}{diam}
\DeclareMathOperator{\Epigraphe}{Epigraphe}




\usepackage{xcolor}
\everymath{\color{blue}}
%\everymath{\color[rgb]{0,1,1}}
%\pagecolor[rgb]{0,0,0.5}


\newcommand*{\pdtest}[3][]{\ensuremath{\frac{\partial^{#1} #2}{\partial #3}}}

\newcommand*{\deffunc}[6][]{\ensuremath{
\begin{array}{rcl}
#2 : #3 &\rightarrow& #4\\
#5 &\mapsto& #6
\end{array}
}}

\newcommand{\eqcolon}{\mathrel{\resizebox{\widthof{$\mathord{=}$}}{\height}{ $\!\!=\!\!\resizebox{1.2\width}{0.8\height}{\raisebox{0.23ex}{$\mathop{:}$}}\!\!$ }}}
\newcommand{\coloneq}{\mathrel{\resizebox{\widthof{$\mathord{=}$}}{\height}{ $\!\!\resizebox{1.2\width}{0.8\height}{\raisebox{0.23ex}{$\mathop{:}$}}\!\!=\!\!$ }}}
\newcommand{\eqcolonl}{\ensuremath{\mathrel{=\!\!\mathop{:}}}}
\newcommand{\coloneql}{\ensuremath{\mathrel{\mathop{:} \!\! =}}}
\newcommand{\vc}[1]{% inline column vector
  \left(\begin{smallmatrix}#1\end{smallmatrix}\right)%
}
\newcommand{\vr}[1]{% inline row vector
  \begin{smallmatrix}(\,#1\,)\end{smallmatrix}%
}
\makeatletter
\newcommand*{\defeq}{\ =\mathrel{\rlap{%
                     \raisebox{0.3ex}{$\m@th\cdot$}}%
                     \raisebox{-0.3ex}{$\m@th\cdot$}}%
                     }
\makeatother

\newcommand{\mathcircle}[1]{% inline row vector
 \overset{\circ}{#1}
}
\newcommand{\ulim}{% low limit
 \underline{\lim}
}
\newcommand{\ssi}{% iff
\iff
}
\newcommand{\ps}[2]{
\expval{#1 | #2}
}
\newcommand{\df}[1]{
\mqty{#1}
}
\newcommand{\n}[1]{
\norm{#1}
}
\newcommand{\sys}[1]{
\left\{\smqty{#1}\right.
}


\newcommand{\eqdef}{\ensuremath{\overset{\text{def}}=}}


\def\Circlearrowright{\ensuremath{%
  \rotatebox[origin=c]{230}{$\circlearrowright$}}}

\newcommand\ct[1]{\text{\rmfamily\upshape #1}}
\newcommand\question[1]{ {\color{red} ...!? \small #1}}
\newcommand\caz[1]{\left\{\begin{array} #1 \end{array}\right.}
\newcommand\const{\text{\rmfamily\upshape const}}
\newcommand\toP{ \overset{\pro}{\to}}
\newcommand\toPP{ \overset{\text{PP}}{\to}}
\newcommand{\oeq}{\mathrel{\text{\textcircled{$=$}}}}





\usepackage{xcolor}
% \usepackage[normalem]{ulem}
\usepackage{lipsum}
\makeatletter
% \newcommand\colorwave[1][blue]{\bgroup \markoverwith{\lower3.5\p@\hbox{\sixly \textcolor{#1}{\char58}}}\ULon}
%\font\sixly=lasy6 % does not re-load if already loaded, so no memory problem.

\newmdtheoremenv[
linewidth= 1pt,linecolor= blue,%
leftmargin=20,rightmargin=20,innertopmargin=0pt, innerrightmargin=40,%
tikzsetting = { draw=lightgray, line width = 0.3pt,dashed,%
dash pattern = on 15pt off 3pt},%
splittopskip=\topskip,skipbelow=\baselineskip,%
skipabove=\baselineskip,ntheorem,roundcorner=0pt,
% backgroundcolor=pagebg,font=\color{orange}\sffamily, fontcolor=white
]{examplebox}{Exemple}[section]



\newcommand\R{\mathbb{R}}
\newcommand\Z{\mathbb{Z}}
\newcommand\N{\mathbb{N}}
\newcommand\E{\mathbb{E}}
\newcommand\F{\mathcal{F}}
\newcommand\cH{\mathcal{H}}
\newcommand\V{\mathbb{V}}
\newcommand\dmo{ ^{-1} }
\newcommand\kapa{\kappa}
\newcommand\im{Im}
\newcommand\hs{\mathcal{H}}





\usepackage{soul}

\makeatletter
\newcommand*{\whiten}[1]{\llap{\textcolor{white}{{\the\SOUL@token}}\hspace{#1pt}}}
\DeclareRobustCommand*\myul{%
    \def\SOUL@everyspace{\underline{\space}\kern\z@}%
    \def\SOUL@everytoken{%
     \setbox0=\hbox{\the\SOUL@token}%
     \ifdim\dp0>\z@
        \raisebox{\dp0}{\underline{\phantom{\the\SOUL@token}}}%
        \whiten{1}\whiten{0}%
        \whiten{-1}\whiten{-2}%
        \llap{\the\SOUL@token}%
     \else
        \underline{\the\SOUL@token}%
     \fi}%
\SOUL@}
\makeatother

\newcommand*{\demp}{\fontfamily{lmtt}\selectfont}

\DeclareTextFontCommand{\textdemp}{\demp}

\begin{document}

\ifcomment
Multiline
comment
\fi
\ifcomment
\myul{Typesetting test}
% \color[rgb]{1,1,1}
$∑_i^n≠ 60º±∞π∆¬≈√j∫h≤≥µ$

$\CR \R\pro\ind\pro\gS\pro
\mqty[a&b\\c&d]$
$\pro\mathbb{P}$
$\dd{x}$

  \[
    \alpha(x)=\left\{
                \begin{array}{ll}
                  x\\
                  \frac{1}{1+e^{-kx}}\\
                  \frac{e^x-e^{-x}}{e^x+e^{-x}}
                \end{array}
              \right.
  \]

  $\expval{x}$
  
  $\chi_\rho(ghg\dmo)=\Tr(\rho_{ghg\dmo})=\Tr(\rho_g\circ\rho_h\circ\rho\dmo_g)=\Tr(\rho_h)\overset{\mbox{\scalebox{0.5}{$\Tr(AB)=\Tr(BA)$}}}{=}\chi_\rho(h)$
  	$\mathop{\oplus}_{\substack{x\in X}}$

$\mat(\rho_g)=(a_{ij}(g))_{\scriptsize \substack{1\leq i\leq d \\ 1\leq j\leq d}}$ et $\mat(\rho'_g)=(a'_{ij}(g))_{\scriptsize \substack{1\leq i'\leq d' \\ 1\leq j'\leq d'}}$



\[\int_a^b{\mathbb{R}^2}g(u, v)\dd{P_{XY}}(u, v)=\iint g(u,v) f_{XY}(u, v)\dd \lambda(u) \dd \lambda(v)\]
$$\lim_{x\to\infty} f(x)$$	
$$\iiiint_V \mu(t,u,v,w) \,dt\,du\,dv\,dw$$
$$\sum_{n=1}^{\infty} 2^{-n} = 1$$	
\begin{definition}
	Si $X$ et $Y$ sont 2 v.a. ou definit la \textsc{Covariance} entre $X$ et $Y$ comme
	$\cov(X,Y)\overset{\text{def}}{=}\E\left[(X-\E(X))(Y-\E(Y))\right]=\E(XY)-\E(X)\E(Y)$.
\end{definition}
\fi
\pagebreak

% \tableofcontents

% insert your code here
%\input{./algebra/main.tex}
%\input{./geometrie-differentielle/main.tex}
%\input{./probabilite/main.tex}
%\input{./analyse-fonctionnelle/main.tex}
% \input{./Analyse-convexe-et-dualite-en-optimisation/main.tex}
%\input{./tikz/main.tex}
%\input{./Theorie-du-distributions/main.tex}
%\input{./optimisation/mine.tex}
 \input{./modelisation/main.tex}

% yves.aubry@univ-tln.fr : algebra

\end{document}

% % !TEX encoding = UTF-8 Unicode
% !TEX TS-program = xelatex

\documentclass[french]{report}

%\usepackage[utf8]{inputenc}
%\usepackage[T1]{fontenc}
\usepackage{babel}


\newif\ifcomment
%\commenttrue # Show comments

\usepackage{physics}
\usepackage{amssymb}


\usepackage{amsthm}
% \usepackage{thmtools}
\usepackage{mathtools}
\usepackage{amsfonts}

\usepackage{color}

\usepackage{tikz}

\usepackage{geometry}
\geometry{a5paper, margin=0.1in, right=1cm}

\usepackage{dsfont}

\usepackage{graphicx}
\graphicspath{ {images/} }

\usepackage{faktor}

\usepackage{IEEEtrantools}
\usepackage{enumerate}   
\usepackage[PostScript=dvips]{"/Users/aware/Documents/Courses/diagrams"}


\newtheorem{theorem}{Théorème}[section]
\renewcommand{\thetheorem}{\arabic{theorem}}
\newtheorem{lemme}{Lemme}[section]
\renewcommand{\thelemme}{\arabic{lemme}}
\newtheorem{proposition}{Proposition}[section]
\renewcommand{\theproposition}{\arabic{proposition}}
\newtheorem{notations}{Notations}[section]
\newtheorem{problem}{Problème}[section]
\newtheorem{corollary}{Corollaire}[theorem]
\renewcommand{\thecorollary}{\arabic{corollary}}
\newtheorem{property}{Propriété}[section]
\newtheorem{objective}{Objectif}[section]

\theoremstyle{definition}
\newtheorem{definition}{Définition}[section]
\renewcommand{\thedefinition}{\arabic{definition}}
\newtheorem{exercise}{Exercice}[chapter]
\renewcommand{\theexercise}{\arabic{exercise}}
\newtheorem{example}{Exemple}[chapter]
\renewcommand{\theexample}{\arabic{example}}
\newtheorem*{solution}{Solution}
\newtheorem*{application}{Application}
\newtheorem*{notation}{Notation}
\newtheorem*{vocabulary}{Vocabulaire}
\newtheorem*{properties}{Propriétés}



\theoremstyle{remark}
\newtheorem*{remark}{Remarque}
\newtheorem*{rappel}{Rappel}


\usepackage{etoolbox}
\AtBeginEnvironment{exercise}{\small}
\AtBeginEnvironment{example}{\small}

\usepackage{cases}
\usepackage[red]{mypack}

\usepackage[framemethod=TikZ]{mdframed}

\definecolor{bg}{rgb}{0.4,0.25,0.95}
\definecolor{pagebg}{rgb}{0,0,0.5}
\surroundwithmdframed[
   topline=false,
   rightline=false,
   bottomline=false,
   leftmargin=\parindent,
   skipabove=8pt,
   skipbelow=8pt,
   linecolor=blue,
   innerbottommargin=10pt,
   % backgroundcolor=bg,font=\color{orange}\sffamily, fontcolor=white
]{definition}

\usepackage{empheq}
\usepackage[most]{tcolorbox}

\newtcbox{\mymath}[1][]{%
    nobeforeafter, math upper, tcbox raise base,
    enhanced, colframe=blue!30!black,
    colback=red!10, boxrule=1pt,
    #1}

\usepackage{unixode}


\DeclareMathOperator{\ord}{ord}
\DeclareMathOperator{\orb}{orb}
\DeclareMathOperator{\stab}{stab}
\DeclareMathOperator{\Stab}{stab}
\DeclareMathOperator{\ppcm}{ppcm}
\DeclareMathOperator{\conj}{Conj}
\DeclareMathOperator{\End}{End}
\DeclareMathOperator{\rot}{rot}
\DeclareMathOperator{\trs}{trace}
\DeclareMathOperator{\Ind}{Ind}
\DeclareMathOperator{\mat}{Mat}
\DeclareMathOperator{\id}{Id}
\DeclareMathOperator{\vect}{vect}
\DeclareMathOperator{\img}{img}
\DeclareMathOperator{\cov}{Cov}
\DeclareMathOperator{\dist}{dist}
\DeclareMathOperator{\irr}{Irr}
\DeclareMathOperator{\image}{Im}
\DeclareMathOperator{\pd}{\partial}
\DeclareMathOperator{\epi}{epi}
\DeclareMathOperator{\Argmin}{Argmin}
\DeclareMathOperator{\dom}{dom}
\DeclareMathOperator{\proj}{proj}
\DeclareMathOperator{\ctg}{ctg}
\DeclareMathOperator{\supp}{supp}
\DeclareMathOperator{\argmin}{argmin}
\DeclareMathOperator{\mult}{mult}
\DeclareMathOperator{\ch}{ch}
\DeclareMathOperator{\sh}{sh}
\DeclareMathOperator{\rang}{rang}
\DeclareMathOperator{\diam}{diam}
\DeclareMathOperator{\Epigraphe}{Epigraphe}




\usepackage{xcolor}
\everymath{\color{blue}}
%\everymath{\color[rgb]{0,1,1}}
%\pagecolor[rgb]{0,0,0.5}


\newcommand*{\pdtest}[3][]{\ensuremath{\frac{\partial^{#1} #2}{\partial #3}}}

\newcommand*{\deffunc}[6][]{\ensuremath{
\begin{array}{rcl}
#2 : #3 &\rightarrow& #4\\
#5 &\mapsto& #6
\end{array}
}}

\newcommand{\eqcolon}{\mathrel{\resizebox{\widthof{$\mathord{=}$}}{\height}{ $\!\!=\!\!\resizebox{1.2\width}{0.8\height}{\raisebox{0.23ex}{$\mathop{:}$}}\!\!$ }}}
\newcommand{\coloneq}{\mathrel{\resizebox{\widthof{$\mathord{=}$}}{\height}{ $\!\!\resizebox{1.2\width}{0.8\height}{\raisebox{0.23ex}{$\mathop{:}$}}\!\!=\!\!$ }}}
\newcommand{\eqcolonl}{\ensuremath{\mathrel{=\!\!\mathop{:}}}}
\newcommand{\coloneql}{\ensuremath{\mathrel{\mathop{:} \!\! =}}}
\newcommand{\vc}[1]{% inline column vector
  \left(\begin{smallmatrix}#1\end{smallmatrix}\right)%
}
\newcommand{\vr}[1]{% inline row vector
  \begin{smallmatrix}(\,#1\,)\end{smallmatrix}%
}
\makeatletter
\newcommand*{\defeq}{\ =\mathrel{\rlap{%
                     \raisebox{0.3ex}{$\m@th\cdot$}}%
                     \raisebox{-0.3ex}{$\m@th\cdot$}}%
                     }
\makeatother

\newcommand{\mathcircle}[1]{% inline row vector
 \overset{\circ}{#1}
}
\newcommand{\ulim}{% low limit
 \underline{\lim}
}
\newcommand{\ssi}{% iff
\iff
}
\newcommand{\ps}[2]{
\expval{#1 | #2}
}
\newcommand{\df}[1]{
\mqty{#1}
}
\newcommand{\n}[1]{
\norm{#1}
}
\newcommand{\sys}[1]{
\left\{\smqty{#1}\right.
}


\newcommand{\eqdef}{\ensuremath{\overset{\text{def}}=}}


\def\Circlearrowright{\ensuremath{%
  \rotatebox[origin=c]{230}{$\circlearrowright$}}}

\newcommand\ct[1]{\text{\rmfamily\upshape #1}}
\newcommand\question[1]{ {\color{red} ...!? \small #1}}
\newcommand\caz[1]{\left\{\begin{array} #1 \end{array}\right.}
\newcommand\const{\text{\rmfamily\upshape const}}
\newcommand\toP{ \overset{\pro}{\to}}
\newcommand\toPP{ \overset{\text{PP}}{\to}}
\newcommand{\oeq}{\mathrel{\text{\textcircled{$=$}}}}





\usepackage{xcolor}
% \usepackage[normalem]{ulem}
\usepackage{lipsum}
\makeatletter
% \newcommand\colorwave[1][blue]{\bgroup \markoverwith{\lower3.5\p@\hbox{\sixly \textcolor{#1}{\char58}}}\ULon}
%\font\sixly=lasy6 % does not re-load if already loaded, so no memory problem.

\newmdtheoremenv[
linewidth= 1pt,linecolor= blue,%
leftmargin=20,rightmargin=20,innertopmargin=0pt, innerrightmargin=40,%
tikzsetting = { draw=lightgray, line width = 0.3pt,dashed,%
dash pattern = on 15pt off 3pt},%
splittopskip=\topskip,skipbelow=\baselineskip,%
skipabove=\baselineskip,ntheorem,roundcorner=0pt,
% backgroundcolor=pagebg,font=\color{orange}\sffamily, fontcolor=white
]{examplebox}{Exemple}[section]



\newcommand\R{\mathbb{R}}
\newcommand\Z{\mathbb{Z}}
\newcommand\N{\mathbb{N}}
\newcommand\E{\mathbb{E}}
\newcommand\F{\mathcal{F}}
\newcommand\cH{\mathcal{H}}
\newcommand\V{\mathbb{V}}
\newcommand\dmo{ ^{-1} }
\newcommand\kapa{\kappa}
\newcommand\im{Im}
\newcommand\hs{\mathcal{H}}





\usepackage{soul}

\makeatletter
\newcommand*{\whiten}[1]{\llap{\textcolor{white}{{\the\SOUL@token}}\hspace{#1pt}}}
\DeclareRobustCommand*\myul{%
    \def\SOUL@everyspace{\underline{\space}\kern\z@}%
    \def\SOUL@everytoken{%
     \setbox0=\hbox{\the\SOUL@token}%
     \ifdim\dp0>\z@
        \raisebox{\dp0}{\underline{\phantom{\the\SOUL@token}}}%
        \whiten{1}\whiten{0}%
        \whiten{-1}\whiten{-2}%
        \llap{\the\SOUL@token}%
     \else
        \underline{\the\SOUL@token}%
     \fi}%
\SOUL@}
\makeatother

\newcommand*{\demp}{\fontfamily{lmtt}\selectfont}

\DeclareTextFontCommand{\textdemp}{\demp}

\begin{document}

\ifcomment
Multiline
comment
\fi
\ifcomment
\myul{Typesetting test}
% \color[rgb]{1,1,1}
$∑_i^n≠ 60º±∞π∆¬≈√j∫h≤≥µ$

$\CR \R\pro\ind\pro\gS\pro
\mqty[a&b\\c&d]$
$\pro\mathbb{P}$
$\dd{x}$

  \[
    \alpha(x)=\left\{
                \begin{array}{ll}
                  x\\
                  \frac{1}{1+e^{-kx}}\\
                  \frac{e^x-e^{-x}}{e^x+e^{-x}}
                \end{array}
              \right.
  \]

  $\expval{x}$
  
  $\chi_\rho(ghg\dmo)=\Tr(\rho_{ghg\dmo})=\Tr(\rho_g\circ\rho_h\circ\rho\dmo_g)=\Tr(\rho_h)\overset{\mbox{\scalebox{0.5}{$\Tr(AB)=\Tr(BA)$}}}{=}\chi_\rho(h)$
  	$\mathop{\oplus}_{\substack{x\in X}}$

$\mat(\rho_g)=(a_{ij}(g))_{\scriptsize \substack{1\leq i\leq d \\ 1\leq j\leq d}}$ et $\mat(\rho'_g)=(a'_{ij}(g))_{\scriptsize \substack{1\leq i'\leq d' \\ 1\leq j'\leq d'}}$



\[\int_a^b{\mathbb{R}^2}g(u, v)\dd{P_{XY}}(u, v)=\iint g(u,v) f_{XY}(u, v)\dd \lambda(u) \dd \lambda(v)\]
$$\lim_{x\to\infty} f(x)$$	
$$\iiiint_V \mu(t,u,v,w) \,dt\,du\,dv\,dw$$
$$\sum_{n=1}^{\infty} 2^{-n} = 1$$	
\begin{definition}
	Si $X$ et $Y$ sont 2 v.a. ou definit la \textsc{Covariance} entre $X$ et $Y$ comme
	$\cov(X,Y)\overset{\text{def}}{=}\E\left[(X-\E(X))(Y-\E(Y))\right]=\E(XY)-\E(X)\E(Y)$.
\end{definition}
\fi
\pagebreak

% \tableofcontents

% insert your code here
%\input{./algebra/main.tex}
%\input{./geometrie-differentielle/main.tex}
%\input{./probabilite/main.tex}
%\input{./analyse-fonctionnelle/main.tex}
% \input{./Analyse-convexe-et-dualite-en-optimisation/main.tex}
%\input{./tikz/main.tex}
%\input{./Theorie-du-distributions/main.tex}
%\input{./optimisation/mine.tex}
 \input{./modelisation/main.tex}

% yves.aubry@univ-tln.fr : algebra

\end{document}

%% !TEX encoding = UTF-8 Unicode
% !TEX TS-program = xelatex

\documentclass[french]{report}

%\usepackage[utf8]{inputenc}
%\usepackage[T1]{fontenc}
\usepackage{babel}


\newif\ifcomment
%\commenttrue # Show comments

\usepackage{physics}
\usepackage{amssymb}


\usepackage{amsthm}
% \usepackage{thmtools}
\usepackage{mathtools}
\usepackage{amsfonts}

\usepackage{color}

\usepackage{tikz}

\usepackage{geometry}
\geometry{a5paper, margin=0.1in, right=1cm}

\usepackage{dsfont}

\usepackage{graphicx}
\graphicspath{ {images/} }

\usepackage{faktor}

\usepackage{IEEEtrantools}
\usepackage{enumerate}   
\usepackage[PostScript=dvips]{"/Users/aware/Documents/Courses/diagrams"}


\newtheorem{theorem}{Théorème}[section]
\renewcommand{\thetheorem}{\arabic{theorem}}
\newtheorem{lemme}{Lemme}[section]
\renewcommand{\thelemme}{\arabic{lemme}}
\newtheorem{proposition}{Proposition}[section]
\renewcommand{\theproposition}{\arabic{proposition}}
\newtheorem{notations}{Notations}[section]
\newtheorem{problem}{Problème}[section]
\newtheorem{corollary}{Corollaire}[theorem]
\renewcommand{\thecorollary}{\arabic{corollary}}
\newtheorem{property}{Propriété}[section]
\newtheorem{objective}{Objectif}[section]

\theoremstyle{definition}
\newtheorem{definition}{Définition}[section]
\renewcommand{\thedefinition}{\arabic{definition}}
\newtheorem{exercise}{Exercice}[chapter]
\renewcommand{\theexercise}{\arabic{exercise}}
\newtheorem{example}{Exemple}[chapter]
\renewcommand{\theexample}{\arabic{example}}
\newtheorem*{solution}{Solution}
\newtheorem*{application}{Application}
\newtheorem*{notation}{Notation}
\newtheorem*{vocabulary}{Vocabulaire}
\newtheorem*{properties}{Propriétés}



\theoremstyle{remark}
\newtheorem*{remark}{Remarque}
\newtheorem*{rappel}{Rappel}


\usepackage{etoolbox}
\AtBeginEnvironment{exercise}{\small}
\AtBeginEnvironment{example}{\small}

\usepackage{cases}
\usepackage[red]{mypack}

\usepackage[framemethod=TikZ]{mdframed}

\definecolor{bg}{rgb}{0.4,0.25,0.95}
\definecolor{pagebg}{rgb}{0,0,0.5}
\surroundwithmdframed[
   topline=false,
   rightline=false,
   bottomline=false,
   leftmargin=\parindent,
   skipabove=8pt,
   skipbelow=8pt,
   linecolor=blue,
   innerbottommargin=10pt,
   % backgroundcolor=bg,font=\color{orange}\sffamily, fontcolor=white
]{definition}

\usepackage{empheq}
\usepackage[most]{tcolorbox}

\newtcbox{\mymath}[1][]{%
    nobeforeafter, math upper, tcbox raise base,
    enhanced, colframe=blue!30!black,
    colback=red!10, boxrule=1pt,
    #1}

\usepackage{unixode}


\DeclareMathOperator{\ord}{ord}
\DeclareMathOperator{\orb}{orb}
\DeclareMathOperator{\stab}{stab}
\DeclareMathOperator{\Stab}{stab}
\DeclareMathOperator{\ppcm}{ppcm}
\DeclareMathOperator{\conj}{Conj}
\DeclareMathOperator{\End}{End}
\DeclareMathOperator{\rot}{rot}
\DeclareMathOperator{\trs}{trace}
\DeclareMathOperator{\Ind}{Ind}
\DeclareMathOperator{\mat}{Mat}
\DeclareMathOperator{\id}{Id}
\DeclareMathOperator{\vect}{vect}
\DeclareMathOperator{\img}{img}
\DeclareMathOperator{\cov}{Cov}
\DeclareMathOperator{\dist}{dist}
\DeclareMathOperator{\irr}{Irr}
\DeclareMathOperator{\image}{Im}
\DeclareMathOperator{\pd}{\partial}
\DeclareMathOperator{\epi}{epi}
\DeclareMathOperator{\Argmin}{Argmin}
\DeclareMathOperator{\dom}{dom}
\DeclareMathOperator{\proj}{proj}
\DeclareMathOperator{\ctg}{ctg}
\DeclareMathOperator{\supp}{supp}
\DeclareMathOperator{\argmin}{argmin}
\DeclareMathOperator{\mult}{mult}
\DeclareMathOperator{\ch}{ch}
\DeclareMathOperator{\sh}{sh}
\DeclareMathOperator{\rang}{rang}
\DeclareMathOperator{\diam}{diam}
\DeclareMathOperator{\Epigraphe}{Epigraphe}




\usepackage{xcolor}
\everymath{\color{blue}}
%\everymath{\color[rgb]{0,1,1}}
%\pagecolor[rgb]{0,0,0.5}


\newcommand*{\pdtest}[3][]{\ensuremath{\frac{\partial^{#1} #2}{\partial #3}}}

\newcommand*{\deffunc}[6][]{\ensuremath{
\begin{array}{rcl}
#2 : #3 &\rightarrow& #4\\
#5 &\mapsto& #6
\end{array}
}}

\newcommand{\eqcolon}{\mathrel{\resizebox{\widthof{$\mathord{=}$}}{\height}{ $\!\!=\!\!\resizebox{1.2\width}{0.8\height}{\raisebox{0.23ex}{$\mathop{:}$}}\!\!$ }}}
\newcommand{\coloneq}{\mathrel{\resizebox{\widthof{$\mathord{=}$}}{\height}{ $\!\!\resizebox{1.2\width}{0.8\height}{\raisebox{0.23ex}{$\mathop{:}$}}\!\!=\!\!$ }}}
\newcommand{\eqcolonl}{\ensuremath{\mathrel{=\!\!\mathop{:}}}}
\newcommand{\coloneql}{\ensuremath{\mathrel{\mathop{:} \!\! =}}}
\newcommand{\vc}[1]{% inline column vector
  \left(\begin{smallmatrix}#1\end{smallmatrix}\right)%
}
\newcommand{\vr}[1]{% inline row vector
  \begin{smallmatrix}(\,#1\,)\end{smallmatrix}%
}
\makeatletter
\newcommand*{\defeq}{\ =\mathrel{\rlap{%
                     \raisebox{0.3ex}{$\m@th\cdot$}}%
                     \raisebox{-0.3ex}{$\m@th\cdot$}}%
                     }
\makeatother

\newcommand{\mathcircle}[1]{% inline row vector
 \overset{\circ}{#1}
}
\newcommand{\ulim}{% low limit
 \underline{\lim}
}
\newcommand{\ssi}{% iff
\iff
}
\newcommand{\ps}[2]{
\expval{#1 | #2}
}
\newcommand{\df}[1]{
\mqty{#1}
}
\newcommand{\n}[1]{
\norm{#1}
}
\newcommand{\sys}[1]{
\left\{\smqty{#1}\right.
}


\newcommand{\eqdef}{\ensuremath{\overset{\text{def}}=}}


\def\Circlearrowright{\ensuremath{%
  \rotatebox[origin=c]{230}{$\circlearrowright$}}}

\newcommand\ct[1]{\text{\rmfamily\upshape #1}}
\newcommand\question[1]{ {\color{red} ...!? \small #1}}
\newcommand\caz[1]{\left\{\begin{array} #1 \end{array}\right.}
\newcommand\const{\text{\rmfamily\upshape const}}
\newcommand\toP{ \overset{\pro}{\to}}
\newcommand\toPP{ \overset{\text{PP}}{\to}}
\newcommand{\oeq}{\mathrel{\text{\textcircled{$=$}}}}





\usepackage{xcolor}
% \usepackage[normalem]{ulem}
\usepackage{lipsum}
\makeatletter
% \newcommand\colorwave[1][blue]{\bgroup \markoverwith{\lower3.5\p@\hbox{\sixly \textcolor{#1}{\char58}}}\ULon}
%\font\sixly=lasy6 % does not re-load if already loaded, so no memory problem.

\newmdtheoremenv[
linewidth= 1pt,linecolor= blue,%
leftmargin=20,rightmargin=20,innertopmargin=0pt, innerrightmargin=40,%
tikzsetting = { draw=lightgray, line width = 0.3pt,dashed,%
dash pattern = on 15pt off 3pt},%
splittopskip=\topskip,skipbelow=\baselineskip,%
skipabove=\baselineskip,ntheorem,roundcorner=0pt,
% backgroundcolor=pagebg,font=\color{orange}\sffamily, fontcolor=white
]{examplebox}{Exemple}[section]



\newcommand\R{\mathbb{R}}
\newcommand\Z{\mathbb{Z}}
\newcommand\N{\mathbb{N}}
\newcommand\E{\mathbb{E}}
\newcommand\F{\mathcal{F}}
\newcommand\cH{\mathcal{H}}
\newcommand\V{\mathbb{V}}
\newcommand\dmo{ ^{-1} }
\newcommand\kapa{\kappa}
\newcommand\im{Im}
\newcommand\hs{\mathcal{H}}





\usepackage{soul}

\makeatletter
\newcommand*{\whiten}[1]{\llap{\textcolor{white}{{\the\SOUL@token}}\hspace{#1pt}}}
\DeclareRobustCommand*\myul{%
    \def\SOUL@everyspace{\underline{\space}\kern\z@}%
    \def\SOUL@everytoken{%
     \setbox0=\hbox{\the\SOUL@token}%
     \ifdim\dp0>\z@
        \raisebox{\dp0}{\underline{\phantom{\the\SOUL@token}}}%
        \whiten{1}\whiten{0}%
        \whiten{-1}\whiten{-2}%
        \llap{\the\SOUL@token}%
     \else
        \underline{\the\SOUL@token}%
     \fi}%
\SOUL@}
\makeatother

\newcommand*{\demp}{\fontfamily{lmtt}\selectfont}

\DeclareTextFontCommand{\textdemp}{\demp}

\begin{document}

\ifcomment
Multiline
comment
\fi
\ifcomment
\myul{Typesetting test}
% \color[rgb]{1,1,1}
$∑_i^n≠ 60º±∞π∆¬≈√j∫h≤≥µ$

$\CR \R\pro\ind\pro\gS\pro
\mqty[a&b\\c&d]$
$\pro\mathbb{P}$
$\dd{x}$

  \[
    \alpha(x)=\left\{
                \begin{array}{ll}
                  x\\
                  \frac{1}{1+e^{-kx}}\\
                  \frac{e^x-e^{-x}}{e^x+e^{-x}}
                \end{array}
              \right.
  \]

  $\expval{x}$
  
  $\chi_\rho(ghg\dmo)=\Tr(\rho_{ghg\dmo})=\Tr(\rho_g\circ\rho_h\circ\rho\dmo_g)=\Tr(\rho_h)\overset{\mbox{\scalebox{0.5}{$\Tr(AB)=\Tr(BA)$}}}{=}\chi_\rho(h)$
  	$\mathop{\oplus}_{\substack{x\in X}}$

$\mat(\rho_g)=(a_{ij}(g))_{\scriptsize \substack{1\leq i\leq d \\ 1\leq j\leq d}}$ et $\mat(\rho'_g)=(a'_{ij}(g))_{\scriptsize \substack{1\leq i'\leq d' \\ 1\leq j'\leq d'}}$



\[\int_a^b{\mathbb{R}^2}g(u, v)\dd{P_{XY}}(u, v)=\iint g(u,v) f_{XY}(u, v)\dd \lambda(u) \dd \lambda(v)\]
$$\lim_{x\to\infty} f(x)$$	
$$\iiiint_V \mu(t,u,v,w) \,dt\,du\,dv\,dw$$
$$\sum_{n=1}^{\infty} 2^{-n} = 1$$	
\begin{definition}
	Si $X$ et $Y$ sont 2 v.a. ou definit la \textsc{Covariance} entre $X$ et $Y$ comme
	$\cov(X,Y)\overset{\text{def}}{=}\E\left[(X-\E(X))(Y-\E(Y))\right]=\E(XY)-\E(X)\E(Y)$.
\end{definition}
\fi
\pagebreak

% \tableofcontents

% insert your code here
%\input{./algebra/main.tex}
%\input{./geometrie-differentielle/main.tex}
%\input{./probabilite/main.tex}
%\input{./analyse-fonctionnelle/main.tex}
% \input{./Analyse-convexe-et-dualite-en-optimisation/main.tex}
%\input{./tikz/main.tex}
%\input{./Theorie-du-distributions/main.tex}
%\input{./optimisation/mine.tex}
 \input{./modelisation/main.tex}

% yves.aubry@univ-tln.fr : algebra

\end{document}

%% !TEX encoding = UTF-8 Unicode
% !TEX TS-program = xelatex

\documentclass[french]{report}

%\usepackage[utf8]{inputenc}
%\usepackage[T1]{fontenc}
\usepackage{babel}


\newif\ifcomment
%\commenttrue # Show comments

\usepackage{physics}
\usepackage{amssymb}


\usepackage{amsthm}
% \usepackage{thmtools}
\usepackage{mathtools}
\usepackage{amsfonts}

\usepackage{color}

\usepackage{tikz}

\usepackage{geometry}
\geometry{a5paper, margin=0.1in, right=1cm}

\usepackage{dsfont}

\usepackage{graphicx}
\graphicspath{ {images/} }

\usepackage{faktor}

\usepackage{IEEEtrantools}
\usepackage{enumerate}   
\usepackage[PostScript=dvips]{"/Users/aware/Documents/Courses/diagrams"}


\newtheorem{theorem}{Théorème}[section]
\renewcommand{\thetheorem}{\arabic{theorem}}
\newtheorem{lemme}{Lemme}[section]
\renewcommand{\thelemme}{\arabic{lemme}}
\newtheorem{proposition}{Proposition}[section]
\renewcommand{\theproposition}{\arabic{proposition}}
\newtheorem{notations}{Notations}[section]
\newtheorem{problem}{Problème}[section]
\newtheorem{corollary}{Corollaire}[theorem]
\renewcommand{\thecorollary}{\arabic{corollary}}
\newtheorem{property}{Propriété}[section]
\newtheorem{objective}{Objectif}[section]

\theoremstyle{definition}
\newtheorem{definition}{Définition}[section]
\renewcommand{\thedefinition}{\arabic{definition}}
\newtheorem{exercise}{Exercice}[chapter]
\renewcommand{\theexercise}{\arabic{exercise}}
\newtheorem{example}{Exemple}[chapter]
\renewcommand{\theexample}{\arabic{example}}
\newtheorem*{solution}{Solution}
\newtheorem*{application}{Application}
\newtheorem*{notation}{Notation}
\newtheorem*{vocabulary}{Vocabulaire}
\newtheorem*{properties}{Propriétés}



\theoremstyle{remark}
\newtheorem*{remark}{Remarque}
\newtheorem*{rappel}{Rappel}


\usepackage{etoolbox}
\AtBeginEnvironment{exercise}{\small}
\AtBeginEnvironment{example}{\small}

\usepackage{cases}
\usepackage[red]{mypack}

\usepackage[framemethod=TikZ]{mdframed}

\definecolor{bg}{rgb}{0.4,0.25,0.95}
\definecolor{pagebg}{rgb}{0,0,0.5}
\surroundwithmdframed[
   topline=false,
   rightline=false,
   bottomline=false,
   leftmargin=\parindent,
   skipabove=8pt,
   skipbelow=8pt,
   linecolor=blue,
   innerbottommargin=10pt,
   % backgroundcolor=bg,font=\color{orange}\sffamily, fontcolor=white
]{definition}

\usepackage{empheq}
\usepackage[most]{tcolorbox}

\newtcbox{\mymath}[1][]{%
    nobeforeafter, math upper, tcbox raise base,
    enhanced, colframe=blue!30!black,
    colback=red!10, boxrule=1pt,
    #1}

\usepackage{unixode}


\DeclareMathOperator{\ord}{ord}
\DeclareMathOperator{\orb}{orb}
\DeclareMathOperator{\stab}{stab}
\DeclareMathOperator{\Stab}{stab}
\DeclareMathOperator{\ppcm}{ppcm}
\DeclareMathOperator{\conj}{Conj}
\DeclareMathOperator{\End}{End}
\DeclareMathOperator{\rot}{rot}
\DeclareMathOperator{\trs}{trace}
\DeclareMathOperator{\Ind}{Ind}
\DeclareMathOperator{\mat}{Mat}
\DeclareMathOperator{\id}{Id}
\DeclareMathOperator{\vect}{vect}
\DeclareMathOperator{\img}{img}
\DeclareMathOperator{\cov}{Cov}
\DeclareMathOperator{\dist}{dist}
\DeclareMathOperator{\irr}{Irr}
\DeclareMathOperator{\image}{Im}
\DeclareMathOperator{\pd}{\partial}
\DeclareMathOperator{\epi}{epi}
\DeclareMathOperator{\Argmin}{Argmin}
\DeclareMathOperator{\dom}{dom}
\DeclareMathOperator{\proj}{proj}
\DeclareMathOperator{\ctg}{ctg}
\DeclareMathOperator{\supp}{supp}
\DeclareMathOperator{\argmin}{argmin}
\DeclareMathOperator{\mult}{mult}
\DeclareMathOperator{\ch}{ch}
\DeclareMathOperator{\sh}{sh}
\DeclareMathOperator{\rang}{rang}
\DeclareMathOperator{\diam}{diam}
\DeclareMathOperator{\Epigraphe}{Epigraphe}




\usepackage{xcolor}
\everymath{\color{blue}}
%\everymath{\color[rgb]{0,1,1}}
%\pagecolor[rgb]{0,0,0.5}


\newcommand*{\pdtest}[3][]{\ensuremath{\frac{\partial^{#1} #2}{\partial #3}}}

\newcommand*{\deffunc}[6][]{\ensuremath{
\begin{array}{rcl}
#2 : #3 &\rightarrow& #4\\
#5 &\mapsto& #6
\end{array}
}}

\newcommand{\eqcolon}{\mathrel{\resizebox{\widthof{$\mathord{=}$}}{\height}{ $\!\!=\!\!\resizebox{1.2\width}{0.8\height}{\raisebox{0.23ex}{$\mathop{:}$}}\!\!$ }}}
\newcommand{\coloneq}{\mathrel{\resizebox{\widthof{$\mathord{=}$}}{\height}{ $\!\!\resizebox{1.2\width}{0.8\height}{\raisebox{0.23ex}{$\mathop{:}$}}\!\!=\!\!$ }}}
\newcommand{\eqcolonl}{\ensuremath{\mathrel{=\!\!\mathop{:}}}}
\newcommand{\coloneql}{\ensuremath{\mathrel{\mathop{:} \!\! =}}}
\newcommand{\vc}[1]{% inline column vector
  \left(\begin{smallmatrix}#1\end{smallmatrix}\right)%
}
\newcommand{\vr}[1]{% inline row vector
  \begin{smallmatrix}(\,#1\,)\end{smallmatrix}%
}
\makeatletter
\newcommand*{\defeq}{\ =\mathrel{\rlap{%
                     \raisebox{0.3ex}{$\m@th\cdot$}}%
                     \raisebox{-0.3ex}{$\m@th\cdot$}}%
                     }
\makeatother

\newcommand{\mathcircle}[1]{% inline row vector
 \overset{\circ}{#1}
}
\newcommand{\ulim}{% low limit
 \underline{\lim}
}
\newcommand{\ssi}{% iff
\iff
}
\newcommand{\ps}[2]{
\expval{#1 | #2}
}
\newcommand{\df}[1]{
\mqty{#1}
}
\newcommand{\n}[1]{
\norm{#1}
}
\newcommand{\sys}[1]{
\left\{\smqty{#1}\right.
}


\newcommand{\eqdef}{\ensuremath{\overset{\text{def}}=}}


\def\Circlearrowright{\ensuremath{%
  \rotatebox[origin=c]{230}{$\circlearrowright$}}}

\newcommand\ct[1]{\text{\rmfamily\upshape #1}}
\newcommand\question[1]{ {\color{red} ...!? \small #1}}
\newcommand\caz[1]{\left\{\begin{array} #1 \end{array}\right.}
\newcommand\const{\text{\rmfamily\upshape const}}
\newcommand\toP{ \overset{\pro}{\to}}
\newcommand\toPP{ \overset{\text{PP}}{\to}}
\newcommand{\oeq}{\mathrel{\text{\textcircled{$=$}}}}





\usepackage{xcolor}
% \usepackage[normalem]{ulem}
\usepackage{lipsum}
\makeatletter
% \newcommand\colorwave[1][blue]{\bgroup \markoverwith{\lower3.5\p@\hbox{\sixly \textcolor{#1}{\char58}}}\ULon}
%\font\sixly=lasy6 % does not re-load if already loaded, so no memory problem.

\newmdtheoremenv[
linewidth= 1pt,linecolor= blue,%
leftmargin=20,rightmargin=20,innertopmargin=0pt, innerrightmargin=40,%
tikzsetting = { draw=lightgray, line width = 0.3pt,dashed,%
dash pattern = on 15pt off 3pt},%
splittopskip=\topskip,skipbelow=\baselineskip,%
skipabove=\baselineskip,ntheorem,roundcorner=0pt,
% backgroundcolor=pagebg,font=\color{orange}\sffamily, fontcolor=white
]{examplebox}{Exemple}[section]



\newcommand\R{\mathbb{R}}
\newcommand\Z{\mathbb{Z}}
\newcommand\N{\mathbb{N}}
\newcommand\E{\mathbb{E}}
\newcommand\F{\mathcal{F}}
\newcommand\cH{\mathcal{H}}
\newcommand\V{\mathbb{V}}
\newcommand\dmo{ ^{-1} }
\newcommand\kapa{\kappa}
\newcommand\im{Im}
\newcommand\hs{\mathcal{H}}





\usepackage{soul}

\makeatletter
\newcommand*{\whiten}[1]{\llap{\textcolor{white}{{\the\SOUL@token}}\hspace{#1pt}}}
\DeclareRobustCommand*\myul{%
    \def\SOUL@everyspace{\underline{\space}\kern\z@}%
    \def\SOUL@everytoken{%
     \setbox0=\hbox{\the\SOUL@token}%
     \ifdim\dp0>\z@
        \raisebox{\dp0}{\underline{\phantom{\the\SOUL@token}}}%
        \whiten{1}\whiten{0}%
        \whiten{-1}\whiten{-2}%
        \llap{\the\SOUL@token}%
     \else
        \underline{\the\SOUL@token}%
     \fi}%
\SOUL@}
\makeatother

\newcommand*{\demp}{\fontfamily{lmtt}\selectfont}

\DeclareTextFontCommand{\textdemp}{\demp}

\begin{document}

\ifcomment
Multiline
comment
\fi
\ifcomment
\myul{Typesetting test}
% \color[rgb]{1,1,1}
$∑_i^n≠ 60º±∞π∆¬≈√j∫h≤≥µ$

$\CR \R\pro\ind\pro\gS\pro
\mqty[a&b\\c&d]$
$\pro\mathbb{P}$
$\dd{x}$

  \[
    \alpha(x)=\left\{
                \begin{array}{ll}
                  x\\
                  \frac{1}{1+e^{-kx}}\\
                  \frac{e^x-e^{-x}}{e^x+e^{-x}}
                \end{array}
              \right.
  \]

  $\expval{x}$
  
  $\chi_\rho(ghg\dmo)=\Tr(\rho_{ghg\dmo})=\Tr(\rho_g\circ\rho_h\circ\rho\dmo_g)=\Tr(\rho_h)\overset{\mbox{\scalebox{0.5}{$\Tr(AB)=\Tr(BA)$}}}{=}\chi_\rho(h)$
  	$\mathop{\oplus}_{\substack{x\in X}}$

$\mat(\rho_g)=(a_{ij}(g))_{\scriptsize \substack{1\leq i\leq d \\ 1\leq j\leq d}}$ et $\mat(\rho'_g)=(a'_{ij}(g))_{\scriptsize \substack{1\leq i'\leq d' \\ 1\leq j'\leq d'}}$



\[\int_a^b{\mathbb{R}^2}g(u, v)\dd{P_{XY}}(u, v)=\iint g(u,v) f_{XY}(u, v)\dd \lambda(u) \dd \lambda(v)\]
$$\lim_{x\to\infty} f(x)$$	
$$\iiiint_V \mu(t,u,v,w) \,dt\,du\,dv\,dw$$
$$\sum_{n=1}^{\infty} 2^{-n} = 1$$	
\begin{definition}
	Si $X$ et $Y$ sont 2 v.a. ou definit la \textsc{Covariance} entre $X$ et $Y$ comme
	$\cov(X,Y)\overset{\text{def}}{=}\E\left[(X-\E(X))(Y-\E(Y))\right]=\E(XY)-\E(X)\E(Y)$.
\end{definition}
\fi
\pagebreak

% \tableofcontents

% insert your code here
%\input{./algebra/main.tex}
%\input{./geometrie-differentielle/main.tex}
%\input{./probabilite/main.tex}
%\input{./analyse-fonctionnelle/main.tex}
% \input{./Analyse-convexe-et-dualite-en-optimisation/main.tex}
%\input{./tikz/main.tex}
%\input{./Theorie-du-distributions/main.tex}
%\input{./optimisation/mine.tex}
 \input{./modelisation/main.tex}

% yves.aubry@univ-tln.fr : algebra

\end{document}

%\input{./optimisation/mine.tex}
 % !TEX encoding = UTF-8 Unicode
% !TEX TS-program = xelatex

\documentclass[french]{report}

%\usepackage[utf8]{inputenc}
%\usepackage[T1]{fontenc}
\usepackage{babel}


\newif\ifcomment
%\commenttrue # Show comments

\usepackage{physics}
\usepackage{amssymb}


\usepackage{amsthm}
% \usepackage{thmtools}
\usepackage{mathtools}
\usepackage{amsfonts}

\usepackage{color}

\usepackage{tikz}

\usepackage{geometry}
\geometry{a5paper, margin=0.1in, right=1cm}

\usepackage{dsfont}

\usepackage{graphicx}
\graphicspath{ {images/} }

\usepackage{faktor}

\usepackage{IEEEtrantools}
\usepackage{enumerate}   
\usepackage[PostScript=dvips]{"/Users/aware/Documents/Courses/diagrams"}


\newtheorem{theorem}{Théorème}[section]
\renewcommand{\thetheorem}{\arabic{theorem}}
\newtheorem{lemme}{Lemme}[section]
\renewcommand{\thelemme}{\arabic{lemme}}
\newtheorem{proposition}{Proposition}[section]
\renewcommand{\theproposition}{\arabic{proposition}}
\newtheorem{notations}{Notations}[section]
\newtheorem{problem}{Problème}[section]
\newtheorem{corollary}{Corollaire}[theorem]
\renewcommand{\thecorollary}{\arabic{corollary}}
\newtheorem{property}{Propriété}[section]
\newtheorem{objective}{Objectif}[section]

\theoremstyle{definition}
\newtheorem{definition}{Définition}[section]
\renewcommand{\thedefinition}{\arabic{definition}}
\newtheorem{exercise}{Exercice}[chapter]
\renewcommand{\theexercise}{\arabic{exercise}}
\newtheorem{example}{Exemple}[chapter]
\renewcommand{\theexample}{\arabic{example}}
\newtheorem*{solution}{Solution}
\newtheorem*{application}{Application}
\newtheorem*{notation}{Notation}
\newtheorem*{vocabulary}{Vocabulaire}
\newtheorem*{properties}{Propriétés}



\theoremstyle{remark}
\newtheorem*{remark}{Remarque}
\newtheorem*{rappel}{Rappel}


\usepackage{etoolbox}
\AtBeginEnvironment{exercise}{\small}
\AtBeginEnvironment{example}{\small}

\usepackage{cases}
\usepackage[red]{mypack}

\usepackage[framemethod=TikZ]{mdframed}

\definecolor{bg}{rgb}{0.4,0.25,0.95}
\definecolor{pagebg}{rgb}{0,0,0.5}
\surroundwithmdframed[
   topline=false,
   rightline=false,
   bottomline=false,
   leftmargin=\parindent,
   skipabove=8pt,
   skipbelow=8pt,
   linecolor=blue,
   innerbottommargin=10pt,
   % backgroundcolor=bg,font=\color{orange}\sffamily, fontcolor=white
]{definition}

\usepackage{empheq}
\usepackage[most]{tcolorbox}

\newtcbox{\mymath}[1][]{%
    nobeforeafter, math upper, tcbox raise base,
    enhanced, colframe=blue!30!black,
    colback=red!10, boxrule=1pt,
    #1}

\usepackage{unixode}


\DeclareMathOperator{\ord}{ord}
\DeclareMathOperator{\orb}{orb}
\DeclareMathOperator{\stab}{stab}
\DeclareMathOperator{\Stab}{stab}
\DeclareMathOperator{\ppcm}{ppcm}
\DeclareMathOperator{\conj}{Conj}
\DeclareMathOperator{\End}{End}
\DeclareMathOperator{\rot}{rot}
\DeclareMathOperator{\trs}{trace}
\DeclareMathOperator{\Ind}{Ind}
\DeclareMathOperator{\mat}{Mat}
\DeclareMathOperator{\id}{Id}
\DeclareMathOperator{\vect}{vect}
\DeclareMathOperator{\img}{img}
\DeclareMathOperator{\cov}{Cov}
\DeclareMathOperator{\dist}{dist}
\DeclareMathOperator{\irr}{Irr}
\DeclareMathOperator{\image}{Im}
\DeclareMathOperator{\pd}{\partial}
\DeclareMathOperator{\epi}{epi}
\DeclareMathOperator{\Argmin}{Argmin}
\DeclareMathOperator{\dom}{dom}
\DeclareMathOperator{\proj}{proj}
\DeclareMathOperator{\ctg}{ctg}
\DeclareMathOperator{\supp}{supp}
\DeclareMathOperator{\argmin}{argmin}
\DeclareMathOperator{\mult}{mult}
\DeclareMathOperator{\ch}{ch}
\DeclareMathOperator{\sh}{sh}
\DeclareMathOperator{\rang}{rang}
\DeclareMathOperator{\diam}{diam}
\DeclareMathOperator{\Epigraphe}{Epigraphe}




\usepackage{xcolor}
\everymath{\color{blue}}
%\everymath{\color[rgb]{0,1,1}}
%\pagecolor[rgb]{0,0,0.5}


\newcommand*{\pdtest}[3][]{\ensuremath{\frac{\partial^{#1} #2}{\partial #3}}}

\newcommand*{\deffunc}[6][]{\ensuremath{
\begin{array}{rcl}
#2 : #3 &\rightarrow& #4\\
#5 &\mapsto& #6
\end{array}
}}

\newcommand{\eqcolon}{\mathrel{\resizebox{\widthof{$\mathord{=}$}}{\height}{ $\!\!=\!\!\resizebox{1.2\width}{0.8\height}{\raisebox{0.23ex}{$\mathop{:}$}}\!\!$ }}}
\newcommand{\coloneq}{\mathrel{\resizebox{\widthof{$\mathord{=}$}}{\height}{ $\!\!\resizebox{1.2\width}{0.8\height}{\raisebox{0.23ex}{$\mathop{:}$}}\!\!=\!\!$ }}}
\newcommand{\eqcolonl}{\ensuremath{\mathrel{=\!\!\mathop{:}}}}
\newcommand{\coloneql}{\ensuremath{\mathrel{\mathop{:} \!\! =}}}
\newcommand{\vc}[1]{% inline column vector
  \left(\begin{smallmatrix}#1\end{smallmatrix}\right)%
}
\newcommand{\vr}[1]{% inline row vector
  \begin{smallmatrix}(\,#1\,)\end{smallmatrix}%
}
\makeatletter
\newcommand*{\defeq}{\ =\mathrel{\rlap{%
                     \raisebox{0.3ex}{$\m@th\cdot$}}%
                     \raisebox{-0.3ex}{$\m@th\cdot$}}%
                     }
\makeatother

\newcommand{\mathcircle}[1]{% inline row vector
 \overset{\circ}{#1}
}
\newcommand{\ulim}{% low limit
 \underline{\lim}
}
\newcommand{\ssi}{% iff
\iff
}
\newcommand{\ps}[2]{
\expval{#1 | #2}
}
\newcommand{\df}[1]{
\mqty{#1}
}
\newcommand{\n}[1]{
\norm{#1}
}
\newcommand{\sys}[1]{
\left\{\smqty{#1}\right.
}


\newcommand{\eqdef}{\ensuremath{\overset{\text{def}}=}}


\def\Circlearrowright{\ensuremath{%
  \rotatebox[origin=c]{230}{$\circlearrowright$}}}

\newcommand\ct[1]{\text{\rmfamily\upshape #1}}
\newcommand\question[1]{ {\color{red} ...!? \small #1}}
\newcommand\caz[1]{\left\{\begin{array} #1 \end{array}\right.}
\newcommand\const{\text{\rmfamily\upshape const}}
\newcommand\toP{ \overset{\pro}{\to}}
\newcommand\toPP{ \overset{\text{PP}}{\to}}
\newcommand{\oeq}{\mathrel{\text{\textcircled{$=$}}}}





\usepackage{xcolor}
% \usepackage[normalem]{ulem}
\usepackage{lipsum}
\makeatletter
% \newcommand\colorwave[1][blue]{\bgroup \markoverwith{\lower3.5\p@\hbox{\sixly \textcolor{#1}{\char58}}}\ULon}
%\font\sixly=lasy6 % does not re-load if already loaded, so no memory problem.

\newmdtheoremenv[
linewidth= 1pt,linecolor= blue,%
leftmargin=20,rightmargin=20,innertopmargin=0pt, innerrightmargin=40,%
tikzsetting = { draw=lightgray, line width = 0.3pt,dashed,%
dash pattern = on 15pt off 3pt},%
splittopskip=\topskip,skipbelow=\baselineskip,%
skipabove=\baselineskip,ntheorem,roundcorner=0pt,
% backgroundcolor=pagebg,font=\color{orange}\sffamily, fontcolor=white
]{examplebox}{Exemple}[section]



\newcommand\R{\mathbb{R}}
\newcommand\Z{\mathbb{Z}}
\newcommand\N{\mathbb{N}}
\newcommand\E{\mathbb{E}}
\newcommand\F{\mathcal{F}}
\newcommand\cH{\mathcal{H}}
\newcommand\V{\mathbb{V}}
\newcommand\dmo{ ^{-1} }
\newcommand\kapa{\kappa}
\newcommand\im{Im}
\newcommand\hs{\mathcal{H}}





\usepackage{soul}

\makeatletter
\newcommand*{\whiten}[1]{\llap{\textcolor{white}{{\the\SOUL@token}}\hspace{#1pt}}}
\DeclareRobustCommand*\myul{%
    \def\SOUL@everyspace{\underline{\space}\kern\z@}%
    \def\SOUL@everytoken{%
     \setbox0=\hbox{\the\SOUL@token}%
     \ifdim\dp0>\z@
        \raisebox{\dp0}{\underline{\phantom{\the\SOUL@token}}}%
        \whiten{1}\whiten{0}%
        \whiten{-1}\whiten{-2}%
        \llap{\the\SOUL@token}%
     \else
        \underline{\the\SOUL@token}%
     \fi}%
\SOUL@}
\makeatother

\newcommand*{\demp}{\fontfamily{lmtt}\selectfont}

\DeclareTextFontCommand{\textdemp}{\demp}

\begin{document}

\ifcomment
Multiline
comment
\fi
\ifcomment
\myul{Typesetting test}
% \color[rgb]{1,1,1}
$∑_i^n≠ 60º±∞π∆¬≈√j∫h≤≥µ$

$\CR \R\pro\ind\pro\gS\pro
\mqty[a&b\\c&d]$
$\pro\mathbb{P}$
$\dd{x}$

  \[
    \alpha(x)=\left\{
                \begin{array}{ll}
                  x\\
                  \frac{1}{1+e^{-kx}}\\
                  \frac{e^x-e^{-x}}{e^x+e^{-x}}
                \end{array}
              \right.
  \]

  $\expval{x}$
  
  $\chi_\rho(ghg\dmo)=\Tr(\rho_{ghg\dmo})=\Tr(\rho_g\circ\rho_h\circ\rho\dmo_g)=\Tr(\rho_h)\overset{\mbox{\scalebox{0.5}{$\Tr(AB)=\Tr(BA)$}}}{=}\chi_\rho(h)$
  	$\mathop{\oplus}_{\substack{x\in X}}$

$\mat(\rho_g)=(a_{ij}(g))_{\scriptsize \substack{1\leq i\leq d \\ 1\leq j\leq d}}$ et $\mat(\rho'_g)=(a'_{ij}(g))_{\scriptsize \substack{1\leq i'\leq d' \\ 1\leq j'\leq d'}}$



\[\int_a^b{\mathbb{R}^2}g(u, v)\dd{P_{XY}}(u, v)=\iint g(u,v) f_{XY}(u, v)\dd \lambda(u) \dd \lambda(v)\]
$$\lim_{x\to\infty} f(x)$$	
$$\iiiint_V \mu(t,u,v,w) \,dt\,du\,dv\,dw$$
$$\sum_{n=1}^{\infty} 2^{-n} = 1$$	
\begin{definition}
	Si $X$ et $Y$ sont 2 v.a. ou definit la \textsc{Covariance} entre $X$ et $Y$ comme
	$\cov(X,Y)\overset{\text{def}}{=}\E\left[(X-\E(X))(Y-\E(Y))\right]=\E(XY)-\E(X)\E(Y)$.
\end{definition}
\fi
\pagebreak

% \tableofcontents

% insert your code here
%\input{./algebra/main.tex}
%\input{./geometrie-differentielle/main.tex}
%\input{./probabilite/main.tex}
%\input{./analyse-fonctionnelle/main.tex}
% \input{./Analyse-convexe-et-dualite-en-optimisation/main.tex}
%\input{./tikz/main.tex}
%\input{./Theorie-du-distributions/main.tex}
%\input{./optimisation/mine.tex}
 \input{./modelisation/main.tex}

% yves.aubry@univ-tln.fr : algebra

\end{document}


% yves.aubry@univ-tln.fr : algebra

\end{document}

%% !TEX encoding = UTF-8 Unicode
% !TEX TS-program = xelatex

\documentclass[french]{report}

%\usepackage[utf8]{inputenc}
%\usepackage[T1]{fontenc}
\usepackage{babel}


\newif\ifcomment
%\commenttrue # Show comments

\usepackage{physics}
\usepackage{amssymb}


\usepackage{amsthm}
% \usepackage{thmtools}
\usepackage{mathtools}
\usepackage{amsfonts}

\usepackage{color}

\usepackage{tikz}

\usepackage{geometry}
\geometry{a5paper, margin=0.1in, right=1cm}

\usepackage{dsfont}

\usepackage{graphicx}
\graphicspath{ {images/} }

\usepackage{faktor}

\usepackage{IEEEtrantools}
\usepackage{enumerate}   
\usepackage[PostScript=dvips]{"/Users/aware/Documents/Courses/diagrams"}


\newtheorem{theorem}{Théorème}[section]
\renewcommand{\thetheorem}{\arabic{theorem}}
\newtheorem{lemme}{Lemme}[section]
\renewcommand{\thelemme}{\arabic{lemme}}
\newtheorem{proposition}{Proposition}[section]
\renewcommand{\theproposition}{\arabic{proposition}}
\newtheorem{notations}{Notations}[section]
\newtheorem{problem}{Problème}[section]
\newtheorem{corollary}{Corollaire}[theorem]
\renewcommand{\thecorollary}{\arabic{corollary}}
\newtheorem{property}{Propriété}[section]
\newtheorem{objective}{Objectif}[section]

\theoremstyle{definition}
\newtheorem{definition}{Définition}[section]
\renewcommand{\thedefinition}{\arabic{definition}}
\newtheorem{exercise}{Exercice}[chapter]
\renewcommand{\theexercise}{\arabic{exercise}}
\newtheorem{example}{Exemple}[chapter]
\renewcommand{\theexample}{\arabic{example}}
\newtheorem*{solution}{Solution}
\newtheorem*{application}{Application}
\newtheorem*{notation}{Notation}
\newtheorem*{vocabulary}{Vocabulaire}
\newtheorem*{properties}{Propriétés}



\theoremstyle{remark}
\newtheorem*{remark}{Remarque}
\newtheorem*{rappel}{Rappel}


\usepackage{etoolbox}
\AtBeginEnvironment{exercise}{\small}
\AtBeginEnvironment{example}{\small}

\usepackage{cases}
\usepackage[red]{mypack}

\usepackage[framemethod=TikZ]{mdframed}

\definecolor{bg}{rgb}{0.4,0.25,0.95}
\definecolor{pagebg}{rgb}{0,0,0.5}
\surroundwithmdframed[
   topline=false,
   rightline=false,
   bottomline=false,
   leftmargin=\parindent,
   skipabove=8pt,
   skipbelow=8pt,
   linecolor=blue,
   innerbottommargin=10pt,
   % backgroundcolor=bg,font=\color{orange}\sffamily, fontcolor=white
]{definition}

\usepackage{empheq}
\usepackage[most]{tcolorbox}

\newtcbox{\mymath}[1][]{%
    nobeforeafter, math upper, tcbox raise base,
    enhanced, colframe=blue!30!black,
    colback=red!10, boxrule=1pt,
    #1}

\usepackage{unixode}


\DeclareMathOperator{\ord}{ord}
\DeclareMathOperator{\orb}{orb}
\DeclareMathOperator{\stab}{stab}
\DeclareMathOperator{\Stab}{stab}
\DeclareMathOperator{\ppcm}{ppcm}
\DeclareMathOperator{\conj}{Conj}
\DeclareMathOperator{\End}{End}
\DeclareMathOperator{\rot}{rot}
\DeclareMathOperator{\trs}{trace}
\DeclareMathOperator{\Ind}{Ind}
\DeclareMathOperator{\mat}{Mat}
\DeclareMathOperator{\id}{Id}
\DeclareMathOperator{\vect}{vect}
\DeclareMathOperator{\img}{img}
\DeclareMathOperator{\cov}{Cov}
\DeclareMathOperator{\dist}{dist}
\DeclareMathOperator{\irr}{Irr}
\DeclareMathOperator{\image}{Im}
\DeclareMathOperator{\pd}{\partial}
\DeclareMathOperator{\epi}{epi}
\DeclareMathOperator{\Argmin}{Argmin}
\DeclareMathOperator{\dom}{dom}
\DeclareMathOperator{\proj}{proj}
\DeclareMathOperator{\ctg}{ctg}
\DeclareMathOperator{\supp}{supp}
\DeclareMathOperator{\argmin}{argmin}
\DeclareMathOperator{\mult}{mult}
\DeclareMathOperator{\ch}{ch}
\DeclareMathOperator{\sh}{sh}
\DeclareMathOperator{\rang}{rang}
\DeclareMathOperator{\diam}{diam}
\DeclareMathOperator{\Epigraphe}{Epigraphe}




\usepackage{xcolor}
\everymath{\color{blue}}
%\everymath{\color[rgb]{0,1,1}}
%\pagecolor[rgb]{0,0,0.5}


\newcommand*{\pdtest}[3][]{\ensuremath{\frac{\partial^{#1} #2}{\partial #3}}}

\newcommand*{\deffunc}[6][]{\ensuremath{
\begin{array}{rcl}
#2 : #3 &\rightarrow& #4\\
#5 &\mapsto& #6
\end{array}
}}

\newcommand{\eqcolon}{\mathrel{\resizebox{\widthof{$\mathord{=}$}}{\height}{ $\!\!=\!\!\resizebox{1.2\width}{0.8\height}{\raisebox{0.23ex}{$\mathop{:}$}}\!\!$ }}}
\newcommand{\coloneq}{\mathrel{\resizebox{\widthof{$\mathord{=}$}}{\height}{ $\!\!\resizebox{1.2\width}{0.8\height}{\raisebox{0.23ex}{$\mathop{:}$}}\!\!=\!\!$ }}}
\newcommand{\eqcolonl}{\ensuremath{\mathrel{=\!\!\mathop{:}}}}
\newcommand{\coloneql}{\ensuremath{\mathrel{\mathop{:} \!\! =}}}
\newcommand{\vc}[1]{% inline column vector
  \left(\begin{smallmatrix}#1\end{smallmatrix}\right)%
}
\newcommand{\vr}[1]{% inline row vector
  \begin{smallmatrix}(\,#1\,)\end{smallmatrix}%
}
\makeatletter
\newcommand*{\defeq}{\ =\mathrel{\rlap{%
                     \raisebox{0.3ex}{$\m@th\cdot$}}%
                     \raisebox{-0.3ex}{$\m@th\cdot$}}%
                     }
\makeatother

\newcommand{\mathcircle}[1]{% inline row vector
 \overset{\circ}{#1}
}
\newcommand{\ulim}{% low limit
 \underline{\lim}
}
\newcommand{\ssi}{% iff
\iff
}
\newcommand{\ps}[2]{
\expval{#1 | #2}
}
\newcommand{\df}[1]{
\mqty{#1}
}
\newcommand{\n}[1]{
\norm{#1}
}
\newcommand{\sys}[1]{
\left\{\smqty{#1}\right.
}


\newcommand{\eqdef}{\ensuremath{\overset{\text{def}}=}}


\def\Circlearrowright{\ensuremath{%
  \rotatebox[origin=c]{230}{$\circlearrowright$}}}

\newcommand\ct[1]{\text{\rmfamily\upshape #1}}
\newcommand\question[1]{ {\color{red} ...!? \small #1}}
\newcommand\caz[1]{\left\{\begin{array} #1 \end{array}\right.}
\newcommand\const{\text{\rmfamily\upshape const}}
\newcommand\toP{ \overset{\pro}{\to}}
\newcommand\toPP{ \overset{\text{PP}}{\to}}
\newcommand{\oeq}{\mathrel{\text{\textcircled{$=$}}}}





\usepackage{xcolor}
% \usepackage[normalem]{ulem}
\usepackage{lipsum}
\makeatletter
% \newcommand\colorwave[1][blue]{\bgroup \markoverwith{\lower3.5\p@\hbox{\sixly \textcolor{#1}{\char58}}}\ULon}
%\font\sixly=lasy6 % does not re-load if already loaded, so no memory problem.

\newmdtheoremenv[
linewidth= 1pt,linecolor= blue,%
leftmargin=20,rightmargin=20,innertopmargin=0pt, innerrightmargin=40,%
tikzsetting = { draw=lightgray, line width = 0.3pt,dashed,%
dash pattern = on 15pt off 3pt},%
splittopskip=\topskip,skipbelow=\baselineskip,%
skipabove=\baselineskip,ntheorem,roundcorner=0pt,
% backgroundcolor=pagebg,font=\color{orange}\sffamily, fontcolor=white
]{examplebox}{Exemple}[section]



\newcommand\R{\mathbb{R}}
\newcommand\Z{\mathbb{Z}}
\newcommand\N{\mathbb{N}}
\newcommand\E{\mathbb{E}}
\newcommand\F{\mathcal{F}}
\newcommand\cH{\mathcal{H}}
\newcommand\V{\mathbb{V}}
\newcommand\dmo{ ^{-1} }
\newcommand\kapa{\kappa}
\newcommand\im{Im}
\newcommand\hs{\mathcal{H}}





\usepackage{soul}

\makeatletter
\newcommand*{\whiten}[1]{\llap{\textcolor{white}{{\the\SOUL@token}}\hspace{#1pt}}}
\DeclareRobustCommand*\myul{%
    \def\SOUL@everyspace{\underline{\space}\kern\z@}%
    \def\SOUL@everytoken{%
     \setbox0=\hbox{\the\SOUL@token}%
     \ifdim\dp0>\z@
        \raisebox{\dp0}{\underline{\phantom{\the\SOUL@token}}}%
        \whiten{1}\whiten{0}%
        \whiten{-1}\whiten{-2}%
        \llap{\the\SOUL@token}%
     \else
        \underline{\the\SOUL@token}%
     \fi}%
\SOUL@}
\makeatother

\newcommand*{\demp}{\fontfamily{lmtt}\selectfont}

\DeclareTextFontCommand{\textdemp}{\demp}

\begin{document}

\ifcomment
Multiline
comment
\fi
\ifcomment
\myul{Typesetting test}
% \color[rgb]{1,1,1}
$∑_i^n≠ 60º±∞π∆¬≈√j∫h≤≥µ$

$\CR \R\pro\ind\pro\gS\pro
\mqty[a&b\\c&d]$
$\pro\mathbb{P}$
$\dd{x}$

  \[
    \alpha(x)=\left\{
                \begin{array}{ll}
                  x\\
                  \frac{1}{1+e^{-kx}}\\
                  \frac{e^x-e^{-x}}{e^x+e^{-x}}
                \end{array}
              \right.
  \]

  $\expval{x}$
  
  $\chi_\rho(ghg\dmo)=\Tr(\rho_{ghg\dmo})=\Tr(\rho_g\circ\rho_h\circ\rho\dmo_g)=\Tr(\rho_h)\overset{\mbox{\scalebox{0.5}{$\Tr(AB)=\Tr(BA)$}}}{=}\chi_\rho(h)$
  	$\mathop{\oplus}_{\substack{x\in X}}$

$\mat(\rho_g)=(a_{ij}(g))_{\scriptsize \substack{1\leq i\leq d \\ 1\leq j\leq d}}$ et $\mat(\rho'_g)=(a'_{ij}(g))_{\scriptsize \substack{1\leq i'\leq d' \\ 1\leq j'\leq d'}}$



\[\int_a^b{\mathbb{R}^2}g(u, v)\dd{P_{XY}}(u, v)=\iint g(u,v) f_{XY}(u, v)\dd \lambda(u) \dd \lambda(v)\]
$$\lim_{x\to\infty} f(x)$$	
$$\iiiint_V \mu(t,u,v,w) \,dt\,du\,dv\,dw$$
$$\sum_{n=1}^{\infty} 2^{-n} = 1$$	
\begin{definition}
	Si $X$ et $Y$ sont 2 v.a. ou definit la \textsc{Covariance} entre $X$ et $Y$ comme
	$\cov(X,Y)\overset{\text{def}}{=}\E\left[(X-\E(X))(Y-\E(Y))\right]=\E(XY)-\E(X)\E(Y)$.
\end{definition}
\fi
\pagebreak

% \tableofcontents

% insert your code here
%% !TEX encoding = UTF-8 Unicode
% !TEX TS-program = xelatex

\documentclass[french]{report}

%\usepackage[utf8]{inputenc}
%\usepackage[T1]{fontenc}
\usepackage{babel}


\newif\ifcomment
%\commenttrue # Show comments

\usepackage{physics}
\usepackage{amssymb}


\usepackage{amsthm}
% \usepackage{thmtools}
\usepackage{mathtools}
\usepackage{amsfonts}

\usepackage{color}

\usepackage{tikz}

\usepackage{geometry}
\geometry{a5paper, margin=0.1in, right=1cm}

\usepackage{dsfont}

\usepackage{graphicx}
\graphicspath{ {images/} }

\usepackage{faktor}

\usepackage{IEEEtrantools}
\usepackage{enumerate}   
\usepackage[PostScript=dvips]{"/Users/aware/Documents/Courses/diagrams"}


\newtheorem{theorem}{Théorème}[section]
\renewcommand{\thetheorem}{\arabic{theorem}}
\newtheorem{lemme}{Lemme}[section]
\renewcommand{\thelemme}{\arabic{lemme}}
\newtheorem{proposition}{Proposition}[section]
\renewcommand{\theproposition}{\arabic{proposition}}
\newtheorem{notations}{Notations}[section]
\newtheorem{problem}{Problème}[section]
\newtheorem{corollary}{Corollaire}[theorem]
\renewcommand{\thecorollary}{\arabic{corollary}}
\newtheorem{property}{Propriété}[section]
\newtheorem{objective}{Objectif}[section]

\theoremstyle{definition}
\newtheorem{definition}{Définition}[section]
\renewcommand{\thedefinition}{\arabic{definition}}
\newtheorem{exercise}{Exercice}[chapter]
\renewcommand{\theexercise}{\arabic{exercise}}
\newtheorem{example}{Exemple}[chapter]
\renewcommand{\theexample}{\arabic{example}}
\newtheorem*{solution}{Solution}
\newtheorem*{application}{Application}
\newtheorem*{notation}{Notation}
\newtheorem*{vocabulary}{Vocabulaire}
\newtheorem*{properties}{Propriétés}



\theoremstyle{remark}
\newtheorem*{remark}{Remarque}
\newtheorem*{rappel}{Rappel}


\usepackage{etoolbox}
\AtBeginEnvironment{exercise}{\small}
\AtBeginEnvironment{example}{\small}

\usepackage{cases}
\usepackage[red]{mypack}

\usepackage[framemethod=TikZ]{mdframed}

\definecolor{bg}{rgb}{0.4,0.25,0.95}
\definecolor{pagebg}{rgb}{0,0,0.5}
\surroundwithmdframed[
   topline=false,
   rightline=false,
   bottomline=false,
   leftmargin=\parindent,
   skipabove=8pt,
   skipbelow=8pt,
   linecolor=blue,
   innerbottommargin=10pt,
   % backgroundcolor=bg,font=\color{orange}\sffamily, fontcolor=white
]{definition}

\usepackage{empheq}
\usepackage[most]{tcolorbox}

\newtcbox{\mymath}[1][]{%
    nobeforeafter, math upper, tcbox raise base,
    enhanced, colframe=blue!30!black,
    colback=red!10, boxrule=1pt,
    #1}

\usepackage{unixode}


\DeclareMathOperator{\ord}{ord}
\DeclareMathOperator{\orb}{orb}
\DeclareMathOperator{\stab}{stab}
\DeclareMathOperator{\Stab}{stab}
\DeclareMathOperator{\ppcm}{ppcm}
\DeclareMathOperator{\conj}{Conj}
\DeclareMathOperator{\End}{End}
\DeclareMathOperator{\rot}{rot}
\DeclareMathOperator{\trs}{trace}
\DeclareMathOperator{\Ind}{Ind}
\DeclareMathOperator{\mat}{Mat}
\DeclareMathOperator{\id}{Id}
\DeclareMathOperator{\vect}{vect}
\DeclareMathOperator{\img}{img}
\DeclareMathOperator{\cov}{Cov}
\DeclareMathOperator{\dist}{dist}
\DeclareMathOperator{\irr}{Irr}
\DeclareMathOperator{\image}{Im}
\DeclareMathOperator{\pd}{\partial}
\DeclareMathOperator{\epi}{epi}
\DeclareMathOperator{\Argmin}{Argmin}
\DeclareMathOperator{\dom}{dom}
\DeclareMathOperator{\proj}{proj}
\DeclareMathOperator{\ctg}{ctg}
\DeclareMathOperator{\supp}{supp}
\DeclareMathOperator{\argmin}{argmin}
\DeclareMathOperator{\mult}{mult}
\DeclareMathOperator{\ch}{ch}
\DeclareMathOperator{\sh}{sh}
\DeclareMathOperator{\rang}{rang}
\DeclareMathOperator{\diam}{diam}
\DeclareMathOperator{\Epigraphe}{Epigraphe}




\usepackage{xcolor}
\everymath{\color{blue}}
%\everymath{\color[rgb]{0,1,1}}
%\pagecolor[rgb]{0,0,0.5}


\newcommand*{\pdtest}[3][]{\ensuremath{\frac{\partial^{#1} #2}{\partial #3}}}

\newcommand*{\deffunc}[6][]{\ensuremath{
\begin{array}{rcl}
#2 : #3 &\rightarrow& #4\\
#5 &\mapsto& #6
\end{array}
}}

\newcommand{\eqcolon}{\mathrel{\resizebox{\widthof{$\mathord{=}$}}{\height}{ $\!\!=\!\!\resizebox{1.2\width}{0.8\height}{\raisebox{0.23ex}{$\mathop{:}$}}\!\!$ }}}
\newcommand{\coloneq}{\mathrel{\resizebox{\widthof{$\mathord{=}$}}{\height}{ $\!\!\resizebox{1.2\width}{0.8\height}{\raisebox{0.23ex}{$\mathop{:}$}}\!\!=\!\!$ }}}
\newcommand{\eqcolonl}{\ensuremath{\mathrel{=\!\!\mathop{:}}}}
\newcommand{\coloneql}{\ensuremath{\mathrel{\mathop{:} \!\! =}}}
\newcommand{\vc}[1]{% inline column vector
  \left(\begin{smallmatrix}#1\end{smallmatrix}\right)%
}
\newcommand{\vr}[1]{% inline row vector
  \begin{smallmatrix}(\,#1\,)\end{smallmatrix}%
}
\makeatletter
\newcommand*{\defeq}{\ =\mathrel{\rlap{%
                     \raisebox{0.3ex}{$\m@th\cdot$}}%
                     \raisebox{-0.3ex}{$\m@th\cdot$}}%
                     }
\makeatother

\newcommand{\mathcircle}[1]{% inline row vector
 \overset{\circ}{#1}
}
\newcommand{\ulim}{% low limit
 \underline{\lim}
}
\newcommand{\ssi}{% iff
\iff
}
\newcommand{\ps}[2]{
\expval{#1 | #2}
}
\newcommand{\df}[1]{
\mqty{#1}
}
\newcommand{\n}[1]{
\norm{#1}
}
\newcommand{\sys}[1]{
\left\{\smqty{#1}\right.
}


\newcommand{\eqdef}{\ensuremath{\overset{\text{def}}=}}


\def\Circlearrowright{\ensuremath{%
  \rotatebox[origin=c]{230}{$\circlearrowright$}}}

\newcommand\ct[1]{\text{\rmfamily\upshape #1}}
\newcommand\question[1]{ {\color{red} ...!? \small #1}}
\newcommand\caz[1]{\left\{\begin{array} #1 \end{array}\right.}
\newcommand\const{\text{\rmfamily\upshape const}}
\newcommand\toP{ \overset{\pro}{\to}}
\newcommand\toPP{ \overset{\text{PP}}{\to}}
\newcommand{\oeq}{\mathrel{\text{\textcircled{$=$}}}}





\usepackage{xcolor}
% \usepackage[normalem]{ulem}
\usepackage{lipsum}
\makeatletter
% \newcommand\colorwave[1][blue]{\bgroup \markoverwith{\lower3.5\p@\hbox{\sixly \textcolor{#1}{\char58}}}\ULon}
%\font\sixly=lasy6 % does not re-load if already loaded, so no memory problem.

\newmdtheoremenv[
linewidth= 1pt,linecolor= blue,%
leftmargin=20,rightmargin=20,innertopmargin=0pt, innerrightmargin=40,%
tikzsetting = { draw=lightgray, line width = 0.3pt,dashed,%
dash pattern = on 15pt off 3pt},%
splittopskip=\topskip,skipbelow=\baselineskip,%
skipabove=\baselineskip,ntheorem,roundcorner=0pt,
% backgroundcolor=pagebg,font=\color{orange}\sffamily, fontcolor=white
]{examplebox}{Exemple}[section]



\newcommand\R{\mathbb{R}}
\newcommand\Z{\mathbb{Z}}
\newcommand\N{\mathbb{N}}
\newcommand\E{\mathbb{E}}
\newcommand\F{\mathcal{F}}
\newcommand\cH{\mathcal{H}}
\newcommand\V{\mathbb{V}}
\newcommand\dmo{ ^{-1} }
\newcommand\kapa{\kappa}
\newcommand\im{Im}
\newcommand\hs{\mathcal{H}}





\usepackage{soul}

\makeatletter
\newcommand*{\whiten}[1]{\llap{\textcolor{white}{{\the\SOUL@token}}\hspace{#1pt}}}
\DeclareRobustCommand*\myul{%
    \def\SOUL@everyspace{\underline{\space}\kern\z@}%
    \def\SOUL@everytoken{%
     \setbox0=\hbox{\the\SOUL@token}%
     \ifdim\dp0>\z@
        \raisebox{\dp0}{\underline{\phantom{\the\SOUL@token}}}%
        \whiten{1}\whiten{0}%
        \whiten{-1}\whiten{-2}%
        \llap{\the\SOUL@token}%
     \else
        \underline{\the\SOUL@token}%
     \fi}%
\SOUL@}
\makeatother

\newcommand*{\demp}{\fontfamily{lmtt}\selectfont}

\DeclareTextFontCommand{\textdemp}{\demp}

\begin{document}

\ifcomment
Multiline
comment
\fi
\ifcomment
\myul{Typesetting test}
% \color[rgb]{1,1,1}
$∑_i^n≠ 60º±∞π∆¬≈√j∫h≤≥µ$

$\CR \R\pro\ind\pro\gS\pro
\mqty[a&b\\c&d]$
$\pro\mathbb{P}$
$\dd{x}$

  \[
    \alpha(x)=\left\{
                \begin{array}{ll}
                  x\\
                  \frac{1}{1+e^{-kx}}\\
                  \frac{e^x-e^{-x}}{e^x+e^{-x}}
                \end{array}
              \right.
  \]

  $\expval{x}$
  
  $\chi_\rho(ghg\dmo)=\Tr(\rho_{ghg\dmo})=\Tr(\rho_g\circ\rho_h\circ\rho\dmo_g)=\Tr(\rho_h)\overset{\mbox{\scalebox{0.5}{$\Tr(AB)=\Tr(BA)$}}}{=}\chi_\rho(h)$
  	$\mathop{\oplus}_{\substack{x\in X}}$

$\mat(\rho_g)=(a_{ij}(g))_{\scriptsize \substack{1\leq i\leq d \\ 1\leq j\leq d}}$ et $\mat(\rho'_g)=(a'_{ij}(g))_{\scriptsize \substack{1\leq i'\leq d' \\ 1\leq j'\leq d'}}$



\[\int_a^b{\mathbb{R}^2}g(u, v)\dd{P_{XY}}(u, v)=\iint g(u,v) f_{XY}(u, v)\dd \lambda(u) \dd \lambda(v)\]
$$\lim_{x\to\infty} f(x)$$	
$$\iiiint_V \mu(t,u,v,w) \,dt\,du\,dv\,dw$$
$$\sum_{n=1}^{\infty} 2^{-n} = 1$$	
\begin{definition}
	Si $X$ et $Y$ sont 2 v.a. ou definit la \textsc{Covariance} entre $X$ et $Y$ comme
	$\cov(X,Y)\overset{\text{def}}{=}\E\left[(X-\E(X))(Y-\E(Y))\right]=\E(XY)-\E(X)\E(Y)$.
\end{definition}
\fi
\pagebreak

% \tableofcontents

% insert your code here
%\input{./algebra/main.tex}
%\input{./geometrie-differentielle/main.tex}
%\input{./probabilite/main.tex}
%\input{./analyse-fonctionnelle/main.tex}
% \input{./Analyse-convexe-et-dualite-en-optimisation/main.tex}
%\input{./tikz/main.tex}
%\input{./Theorie-du-distributions/main.tex}
%\input{./optimisation/mine.tex}
 \input{./modelisation/main.tex}

% yves.aubry@univ-tln.fr : algebra

\end{document}

%% !TEX encoding = UTF-8 Unicode
% !TEX TS-program = xelatex

\documentclass[french]{report}

%\usepackage[utf8]{inputenc}
%\usepackage[T1]{fontenc}
\usepackage{babel}


\newif\ifcomment
%\commenttrue # Show comments

\usepackage{physics}
\usepackage{amssymb}


\usepackage{amsthm}
% \usepackage{thmtools}
\usepackage{mathtools}
\usepackage{amsfonts}

\usepackage{color}

\usepackage{tikz}

\usepackage{geometry}
\geometry{a5paper, margin=0.1in, right=1cm}

\usepackage{dsfont}

\usepackage{graphicx}
\graphicspath{ {images/} }

\usepackage{faktor}

\usepackage{IEEEtrantools}
\usepackage{enumerate}   
\usepackage[PostScript=dvips]{"/Users/aware/Documents/Courses/diagrams"}


\newtheorem{theorem}{Théorème}[section]
\renewcommand{\thetheorem}{\arabic{theorem}}
\newtheorem{lemme}{Lemme}[section]
\renewcommand{\thelemme}{\arabic{lemme}}
\newtheorem{proposition}{Proposition}[section]
\renewcommand{\theproposition}{\arabic{proposition}}
\newtheorem{notations}{Notations}[section]
\newtheorem{problem}{Problème}[section]
\newtheorem{corollary}{Corollaire}[theorem]
\renewcommand{\thecorollary}{\arabic{corollary}}
\newtheorem{property}{Propriété}[section]
\newtheorem{objective}{Objectif}[section]

\theoremstyle{definition}
\newtheorem{definition}{Définition}[section]
\renewcommand{\thedefinition}{\arabic{definition}}
\newtheorem{exercise}{Exercice}[chapter]
\renewcommand{\theexercise}{\arabic{exercise}}
\newtheorem{example}{Exemple}[chapter]
\renewcommand{\theexample}{\arabic{example}}
\newtheorem*{solution}{Solution}
\newtheorem*{application}{Application}
\newtheorem*{notation}{Notation}
\newtheorem*{vocabulary}{Vocabulaire}
\newtheorem*{properties}{Propriétés}



\theoremstyle{remark}
\newtheorem*{remark}{Remarque}
\newtheorem*{rappel}{Rappel}


\usepackage{etoolbox}
\AtBeginEnvironment{exercise}{\small}
\AtBeginEnvironment{example}{\small}

\usepackage{cases}
\usepackage[red]{mypack}

\usepackage[framemethod=TikZ]{mdframed}

\definecolor{bg}{rgb}{0.4,0.25,0.95}
\definecolor{pagebg}{rgb}{0,0,0.5}
\surroundwithmdframed[
   topline=false,
   rightline=false,
   bottomline=false,
   leftmargin=\parindent,
   skipabove=8pt,
   skipbelow=8pt,
   linecolor=blue,
   innerbottommargin=10pt,
   % backgroundcolor=bg,font=\color{orange}\sffamily, fontcolor=white
]{definition}

\usepackage{empheq}
\usepackage[most]{tcolorbox}

\newtcbox{\mymath}[1][]{%
    nobeforeafter, math upper, tcbox raise base,
    enhanced, colframe=blue!30!black,
    colback=red!10, boxrule=1pt,
    #1}

\usepackage{unixode}


\DeclareMathOperator{\ord}{ord}
\DeclareMathOperator{\orb}{orb}
\DeclareMathOperator{\stab}{stab}
\DeclareMathOperator{\Stab}{stab}
\DeclareMathOperator{\ppcm}{ppcm}
\DeclareMathOperator{\conj}{Conj}
\DeclareMathOperator{\End}{End}
\DeclareMathOperator{\rot}{rot}
\DeclareMathOperator{\trs}{trace}
\DeclareMathOperator{\Ind}{Ind}
\DeclareMathOperator{\mat}{Mat}
\DeclareMathOperator{\id}{Id}
\DeclareMathOperator{\vect}{vect}
\DeclareMathOperator{\img}{img}
\DeclareMathOperator{\cov}{Cov}
\DeclareMathOperator{\dist}{dist}
\DeclareMathOperator{\irr}{Irr}
\DeclareMathOperator{\image}{Im}
\DeclareMathOperator{\pd}{\partial}
\DeclareMathOperator{\epi}{epi}
\DeclareMathOperator{\Argmin}{Argmin}
\DeclareMathOperator{\dom}{dom}
\DeclareMathOperator{\proj}{proj}
\DeclareMathOperator{\ctg}{ctg}
\DeclareMathOperator{\supp}{supp}
\DeclareMathOperator{\argmin}{argmin}
\DeclareMathOperator{\mult}{mult}
\DeclareMathOperator{\ch}{ch}
\DeclareMathOperator{\sh}{sh}
\DeclareMathOperator{\rang}{rang}
\DeclareMathOperator{\diam}{diam}
\DeclareMathOperator{\Epigraphe}{Epigraphe}




\usepackage{xcolor}
\everymath{\color{blue}}
%\everymath{\color[rgb]{0,1,1}}
%\pagecolor[rgb]{0,0,0.5}


\newcommand*{\pdtest}[3][]{\ensuremath{\frac{\partial^{#1} #2}{\partial #3}}}

\newcommand*{\deffunc}[6][]{\ensuremath{
\begin{array}{rcl}
#2 : #3 &\rightarrow& #4\\
#5 &\mapsto& #6
\end{array}
}}

\newcommand{\eqcolon}{\mathrel{\resizebox{\widthof{$\mathord{=}$}}{\height}{ $\!\!=\!\!\resizebox{1.2\width}{0.8\height}{\raisebox{0.23ex}{$\mathop{:}$}}\!\!$ }}}
\newcommand{\coloneq}{\mathrel{\resizebox{\widthof{$\mathord{=}$}}{\height}{ $\!\!\resizebox{1.2\width}{0.8\height}{\raisebox{0.23ex}{$\mathop{:}$}}\!\!=\!\!$ }}}
\newcommand{\eqcolonl}{\ensuremath{\mathrel{=\!\!\mathop{:}}}}
\newcommand{\coloneql}{\ensuremath{\mathrel{\mathop{:} \!\! =}}}
\newcommand{\vc}[1]{% inline column vector
  \left(\begin{smallmatrix}#1\end{smallmatrix}\right)%
}
\newcommand{\vr}[1]{% inline row vector
  \begin{smallmatrix}(\,#1\,)\end{smallmatrix}%
}
\makeatletter
\newcommand*{\defeq}{\ =\mathrel{\rlap{%
                     \raisebox{0.3ex}{$\m@th\cdot$}}%
                     \raisebox{-0.3ex}{$\m@th\cdot$}}%
                     }
\makeatother

\newcommand{\mathcircle}[1]{% inline row vector
 \overset{\circ}{#1}
}
\newcommand{\ulim}{% low limit
 \underline{\lim}
}
\newcommand{\ssi}{% iff
\iff
}
\newcommand{\ps}[2]{
\expval{#1 | #2}
}
\newcommand{\df}[1]{
\mqty{#1}
}
\newcommand{\n}[1]{
\norm{#1}
}
\newcommand{\sys}[1]{
\left\{\smqty{#1}\right.
}


\newcommand{\eqdef}{\ensuremath{\overset{\text{def}}=}}


\def\Circlearrowright{\ensuremath{%
  \rotatebox[origin=c]{230}{$\circlearrowright$}}}

\newcommand\ct[1]{\text{\rmfamily\upshape #1}}
\newcommand\question[1]{ {\color{red} ...!? \small #1}}
\newcommand\caz[1]{\left\{\begin{array} #1 \end{array}\right.}
\newcommand\const{\text{\rmfamily\upshape const}}
\newcommand\toP{ \overset{\pro}{\to}}
\newcommand\toPP{ \overset{\text{PP}}{\to}}
\newcommand{\oeq}{\mathrel{\text{\textcircled{$=$}}}}





\usepackage{xcolor}
% \usepackage[normalem]{ulem}
\usepackage{lipsum}
\makeatletter
% \newcommand\colorwave[1][blue]{\bgroup \markoverwith{\lower3.5\p@\hbox{\sixly \textcolor{#1}{\char58}}}\ULon}
%\font\sixly=lasy6 % does not re-load if already loaded, so no memory problem.

\newmdtheoremenv[
linewidth= 1pt,linecolor= blue,%
leftmargin=20,rightmargin=20,innertopmargin=0pt, innerrightmargin=40,%
tikzsetting = { draw=lightgray, line width = 0.3pt,dashed,%
dash pattern = on 15pt off 3pt},%
splittopskip=\topskip,skipbelow=\baselineskip,%
skipabove=\baselineskip,ntheorem,roundcorner=0pt,
% backgroundcolor=pagebg,font=\color{orange}\sffamily, fontcolor=white
]{examplebox}{Exemple}[section]



\newcommand\R{\mathbb{R}}
\newcommand\Z{\mathbb{Z}}
\newcommand\N{\mathbb{N}}
\newcommand\E{\mathbb{E}}
\newcommand\F{\mathcal{F}}
\newcommand\cH{\mathcal{H}}
\newcommand\V{\mathbb{V}}
\newcommand\dmo{ ^{-1} }
\newcommand\kapa{\kappa}
\newcommand\im{Im}
\newcommand\hs{\mathcal{H}}





\usepackage{soul}

\makeatletter
\newcommand*{\whiten}[1]{\llap{\textcolor{white}{{\the\SOUL@token}}\hspace{#1pt}}}
\DeclareRobustCommand*\myul{%
    \def\SOUL@everyspace{\underline{\space}\kern\z@}%
    \def\SOUL@everytoken{%
     \setbox0=\hbox{\the\SOUL@token}%
     \ifdim\dp0>\z@
        \raisebox{\dp0}{\underline{\phantom{\the\SOUL@token}}}%
        \whiten{1}\whiten{0}%
        \whiten{-1}\whiten{-2}%
        \llap{\the\SOUL@token}%
     \else
        \underline{\the\SOUL@token}%
     \fi}%
\SOUL@}
\makeatother

\newcommand*{\demp}{\fontfamily{lmtt}\selectfont}

\DeclareTextFontCommand{\textdemp}{\demp}

\begin{document}

\ifcomment
Multiline
comment
\fi
\ifcomment
\myul{Typesetting test}
% \color[rgb]{1,1,1}
$∑_i^n≠ 60º±∞π∆¬≈√j∫h≤≥µ$

$\CR \R\pro\ind\pro\gS\pro
\mqty[a&b\\c&d]$
$\pro\mathbb{P}$
$\dd{x}$

  \[
    \alpha(x)=\left\{
                \begin{array}{ll}
                  x\\
                  \frac{1}{1+e^{-kx}}\\
                  \frac{e^x-e^{-x}}{e^x+e^{-x}}
                \end{array}
              \right.
  \]

  $\expval{x}$
  
  $\chi_\rho(ghg\dmo)=\Tr(\rho_{ghg\dmo})=\Tr(\rho_g\circ\rho_h\circ\rho\dmo_g)=\Tr(\rho_h)\overset{\mbox{\scalebox{0.5}{$\Tr(AB)=\Tr(BA)$}}}{=}\chi_\rho(h)$
  	$\mathop{\oplus}_{\substack{x\in X}}$

$\mat(\rho_g)=(a_{ij}(g))_{\scriptsize \substack{1\leq i\leq d \\ 1\leq j\leq d}}$ et $\mat(\rho'_g)=(a'_{ij}(g))_{\scriptsize \substack{1\leq i'\leq d' \\ 1\leq j'\leq d'}}$



\[\int_a^b{\mathbb{R}^2}g(u, v)\dd{P_{XY}}(u, v)=\iint g(u,v) f_{XY}(u, v)\dd \lambda(u) \dd \lambda(v)\]
$$\lim_{x\to\infty} f(x)$$	
$$\iiiint_V \mu(t,u,v,w) \,dt\,du\,dv\,dw$$
$$\sum_{n=1}^{\infty} 2^{-n} = 1$$	
\begin{definition}
	Si $X$ et $Y$ sont 2 v.a. ou definit la \textsc{Covariance} entre $X$ et $Y$ comme
	$\cov(X,Y)\overset{\text{def}}{=}\E\left[(X-\E(X))(Y-\E(Y))\right]=\E(XY)-\E(X)\E(Y)$.
\end{definition}
\fi
\pagebreak

% \tableofcontents

% insert your code here
%\input{./algebra/main.tex}
%\input{./geometrie-differentielle/main.tex}
%\input{./probabilite/main.tex}
%\input{./analyse-fonctionnelle/main.tex}
% \input{./Analyse-convexe-et-dualite-en-optimisation/main.tex}
%\input{./tikz/main.tex}
%\input{./Theorie-du-distributions/main.tex}
%\input{./optimisation/mine.tex}
 \input{./modelisation/main.tex}

% yves.aubry@univ-tln.fr : algebra

\end{document}

%% !TEX encoding = UTF-8 Unicode
% !TEX TS-program = xelatex

\documentclass[french]{report}

%\usepackage[utf8]{inputenc}
%\usepackage[T1]{fontenc}
\usepackage{babel}


\newif\ifcomment
%\commenttrue # Show comments

\usepackage{physics}
\usepackage{amssymb}


\usepackage{amsthm}
% \usepackage{thmtools}
\usepackage{mathtools}
\usepackage{amsfonts}

\usepackage{color}

\usepackage{tikz}

\usepackage{geometry}
\geometry{a5paper, margin=0.1in, right=1cm}

\usepackage{dsfont}

\usepackage{graphicx}
\graphicspath{ {images/} }

\usepackage{faktor}

\usepackage{IEEEtrantools}
\usepackage{enumerate}   
\usepackage[PostScript=dvips]{"/Users/aware/Documents/Courses/diagrams"}


\newtheorem{theorem}{Théorème}[section]
\renewcommand{\thetheorem}{\arabic{theorem}}
\newtheorem{lemme}{Lemme}[section]
\renewcommand{\thelemme}{\arabic{lemme}}
\newtheorem{proposition}{Proposition}[section]
\renewcommand{\theproposition}{\arabic{proposition}}
\newtheorem{notations}{Notations}[section]
\newtheorem{problem}{Problème}[section]
\newtheorem{corollary}{Corollaire}[theorem]
\renewcommand{\thecorollary}{\arabic{corollary}}
\newtheorem{property}{Propriété}[section]
\newtheorem{objective}{Objectif}[section]

\theoremstyle{definition}
\newtheorem{definition}{Définition}[section]
\renewcommand{\thedefinition}{\arabic{definition}}
\newtheorem{exercise}{Exercice}[chapter]
\renewcommand{\theexercise}{\arabic{exercise}}
\newtheorem{example}{Exemple}[chapter]
\renewcommand{\theexample}{\arabic{example}}
\newtheorem*{solution}{Solution}
\newtheorem*{application}{Application}
\newtheorem*{notation}{Notation}
\newtheorem*{vocabulary}{Vocabulaire}
\newtheorem*{properties}{Propriétés}



\theoremstyle{remark}
\newtheorem*{remark}{Remarque}
\newtheorem*{rappel}{Rappel}


\usepackage{etoolbox}
\AtBeginEnvironment{exercise}{\small}
\AtBeginEnvironment{example}{\small}

\usepackage{cases}
\usepackage[red]{mypack}

\usepackage[framemethod=TikZ]{mdframed}

\definecolor{bg}{rgb}{0.4,0.25,0.95}
\definecolor{pagebg}{rgb}{0,0,0.5}
\surroundwithmdframed[
   topline=false,
   rightline=false,
   bottomline=false,
   leftmargin=\parindent,
   skipabove=8pt,
   skipbelow=8pt,
   linecolor=blue,
   innerbottommargin=10pt,
   % backgroundcolor=bg,font=\color{orange}\sffamily, fontcolor=white
]{definition}

\usepackage{empheq}
\usepackage[most]{tcolorbox}

\newtcbox{\mymath}[1][]{%
    nobeforeafter, math upper, tcbox raise base,
    enhanced, colframe=blue!30!black,
    colback=red!10, boxrule=1pt,
    #1}

\usepackage{unixode}


\DeclareMathOperator{\ord}{ord}
\DeclareMathOperator{\orb}{orb}
\DeclareMathOperator{\stab}{stab}
\DeclareMathOperator{\Stab}{stab}
\DeclareMathOperator{\ppcm}{ppcm}
\DeclareMathOperator{\conj}{Conj}
\DeclareMathOperator{\End}{End}
\DeclareMathOperator{\rot}{rot}
\DeclareMathOperator{\trs}{trace}
\DeclareMathOperator{\Ind}{Ind}
\DeclareMathOperator{\mat}{Mat}
\DeclareMathOperator{\id}{Id}
\DeclareMathOperator{\vect}{vect}
\DeclareMathOperator{\img}{img}
\DeclareMathOperator{\cov}{Cov}
\DeclareMathOperator{\dist}{dist}
\DeclareMathOperator{\irr}{Irr}
\DeclareMathOperator{\image}{Im}
\DeclareMathOperator{\pd}{\partial}
\DeclareMathOperator{\epi}{epi}
\DeclareMathOperator{\Argmin}{Argmin}
\DeclareMathOperator{\dom}{dom}
\DeclareMathOperator{\proj}{proj}
\DeclareMathOperator{\ctg}{ctg}
\DeclareMathOperator{\supp}{supp}
\DeclareMathOperator{\argmin}{argmin}
\DeclareMathOperator{\mult}{mult}
\DeclareMathOperator{\ch}{ch}
\DeclareMathOperator{\sh}{sh}
\DeclareMathOperator{\rang}{rang}
\DeclareMathOperator{\diam}{diam}
\DeclareMathOperator{\Epigraphe}{Epigraphe}




\usepackage{xcolor}
\everymath{\color{blue}}
%\everymath{\color[rgb]{0,1,1}}
%\pagecolor[rgb]{0,0,0.5}


\newcommand*{\pdtest}[3][]{\ensuremath{\frac{\partial^{#1} #2}{\partial #3}}}

\newcommand*{\deffunc}[6][]{\ensuremath{
\begin{array}{rcl}
#2 : #3 &\rightarrow& #4\\
#5 &\mapsto& #6
\end{array}
}}

\newcommand{\eqcolon}{\mathrel{\resizebox{\widthof{$\mathord{=}$}}{\height}{ $\!\!=\!\!\resizebox{1.2\width}{0.8\height}{\raisebox{0.23ex}{$\mathop{:}$}}\!\!$ }}}
\newcommand{\coloneq}{\mathrel{\resizebox{\widthof{$\mathord{=}$}}{\height}{ $\!\!\resizebox{1.2\width}{0.8\height}{\raisebox{0.23ex}{$\mathop{:}$}}\!\!=\!\!$ }}}
\newcommand{\eqcolonl}{\ensuremath{\mathrel{=\!\!\mathop{:}}}}
\newcommand{\coloneql}{\ensuremath{\mathrel{\mathop{:} \!\! =}}}
\newcommand{\vc}[1]{% inline column vector
  \left(\begin{smallmatrix}#1\end{smallmatrix}\right)%
}
\newcommand{\vr}[1]{% inline row vector
  \begin{smallmatrix}(\,#1\,)\end{smallmatrix}%
}
\makeatletter
\newcommand*{\defeq}{\ =\mathrel{\rlap{%
                     \raisebox{0.3ex}{$\m@th\cdot$}}%
                     \raisebox{-0.3ex}{$\m@th\cdot$}}%
                     }
\makeatother

\newcommand{\mathcircle}[1]{% inline row vector
 \overset{\circ}{#1}
}
\newcommand{\ulim}{% low limit
 \underline{\lim}
}
\newcommand{\ssi}{% iff
\iff
}
\newcommand{\ps}[2]{
\expval{#1 | #2}
}
\newcommand{\df}[1]{
\mqty{#1}
}
\newcommand{\n}[1]{
\norm{#1}
}
\newcommand{\sys}[1]{
\left\{\smqty{#1}\right.
}


\newcommand{\eqdef}{\ensuremath{\overset{\text{def}}=}}


\def\Circlearrowright{\ensuremath{%
  \rotatebox[origin=c]{230}{$\circlearrowright$}}}

\newcommand\ct[1]{\text{\rmfamily\upshape #1}}
\newcommand\question[1]{ {\color{red} ...!? \small #1}}
\newcommand\caz[1]{\left\{\begin{array} #1 \end{array}\right.}
\newcommand\const{\text{\rmfamily\upshape const}}
\newcommand\toP{ \overset{\pro}{\to}}
\newcommand\toPP{ \overset{\text{PP}}{\to}}
\newcommand{\oeq}{\mathrel{\text{\textcircled{$=$}}}}





\usepackage{xcolor}
% \usepackage[normalem]{ulem}
\usepackage{lipsum}
\makeatletter
% \newcommand\colorwave[1][blue]{\bgroup \markoverwith{\lower3.5\p@\hbox{\sixly \textcolor{#1}{\char58}}}\ULon}
%\font\sixly=lasy6 % does not re-load if already loaded, so no memory problem.

\newmdtheoremenv[
linewidth= 1pt,linecolor= blue,%
leftmargin=20,rightmargin=20,innertopmargin=0pt, innerrightmargin=40,%
tikzsetting = { draw=lightgray, line width = 0.3pt,dashed,%
dash pattern = on 15pt off 3pt},%
splittopskip=\topskip,skipbelow=\baselineskip,%
skipabove=\baselineskip,ntheorem,roundcorner=0pt,
% backgroundcolor=pagebg,font=\color{orange}\sffamily, fontcolor=white
]{examplebox}{Exemple}[section]



\newcommand\R{\mathbb{R}}
\newcommand\Z{\mathbb{Z}}
\newcommand\N{\mathbb{N}}
\newcommand\E{\mathbb{E}}
\newcommand\F{\mathcal{F}}
\newcommand\cH{\mathcal{H}}
\newcommand\V{\mathbb{V}}
\newcommand\dmo{ ^{-1} }
\newcommand\kapa{\kappa}
\newcommand\im{Im}
\newcommand\hs{\mathcal{H}}





\usepackage{soul}

\makeatletter
\newcommand*{\whiten}[1]{\llap{\textcolor{white}{{\the\SOUL@token}}\hspace{#1pt}}}
\DeclareRobustCommand*\myul{%
    \def\SOUL@everyspace{\underline{\space}\kern\z@}%
    \def\SOUL@everytoken{%
     \setbox0=\hbox{\the\SOUL@token}%
     \ifdim\dp0>\z@
        \raisebox{\dp0}{\underline{\phantom{\the\SOUL@token}}}%
        \whiten{1}\whiten{0}%
        \whiten{-1}\whiten{-2}%
        \llap{\the\SOUL@token}%
     \else
        \underline{\the\SOUL@token}%
     \fi}%
\SOUL@}
\makeatother

\newcommand*{\demp}{\fontfamily{lmtt}\selectfont}

\DeclareTextFontCommand{\textdemp}{\demp}

\begin{document}

\ifcomment
Multiline
comment
\fi
\ifcomment
\myul{Typesetting test}
% \color[rgb]{1,1,1}
$∑_i^n≠ 60º±∞π∆¬≈√j∫h≤≥µ$

$\CR \R\pro\ind\pro\gS\pro
\mqty[a&b\\c&d]$
$\pro\mathbb{P}$
$\dd{x}$

  \[
    \alpha(x)=\left\{
                \begin{array}{ll}
                  x\\
                  \frac{1}{1+e^{-kx}}\\
                  \frac{e^x-e^{-x}}{e^x+e^{-x}}
                \end{array}
              \right.
  \]

  $\expval{x}$
  
  $\chi_\rho(ghg\dmo)=\Tr(\rho_{ghg\dmo})=\Tr(\rho_g\circ\rho_h\circ\rho\dmo_g)=\Tr(\rho_h)\overset{\mbox{\scalebox{0.5}{$\Tr(AB)=\Tr(BA)$}}}{=}\chi_\rho(h)$
  	$\mathop{\oplus}_{\substack{x\in X}}$

$\mat(\rho_g)=(a_{ij}(g))_{\scriptsize \substack{1\leq i\leq d \\ 1\leq j\leq d}}$ et $\mat(\rho'_g)=(a'_{ij}(g))_{\scriptsize \substack{1\leq i'\leq d' \\ 1\leq j'\leq d'}}$



\[\int_a^b{\mathbb{R}^2}g(u, v)\dd{P_{XY}}(u, v)=\iint g(u,v) f_{XY}(u, v)\dd \lambda(u) \dd \lambda(v)\]
$$\lim_{x\to\infty} f(x)$$	
$$\iiiint_V \mu(t,u,v,w) \,dt\,du\,dv\,dw$$
$$\sum_{n=1}^{\infty} 2^{-n} = 1$$	
\begin{definition}
	Si $X$ et $Y$ sont 2 v.a. ou definit la \textsc{Covariance} entre $X$ et $Y$ comme
	$\cov(X,Y)\overset{\text{def}}{=}\E\left[(X-\E(X))(Y-\E(Y))\right]=\E(XY)-\E(X)\E(Y)$.
\end{definition}
\fi
\pagebreak

% \tableofcontents

% insert your code here
%\input{./algebra/main.tex}
%\input{./geometrie-differentielle/main.tex}
%\input{./probabilite/main.tex}
%\input{./analyse-fonctionnelle/main.tex}
% \input{./Analyse-convexe-et-dualite-en-optimisation/main.tex}
%\input{./tikz/main.tex}
%\input{./Theorie-du-distributions/main.tex}
%\input{./optimisation/mine.tex}
 \input{./modelisation/main.tex}

% yves.aubry@univ-tln.fr : algebra

\end{document}

%% !TEX encoding = UTF-8 Unicode
% !TEX TS-program = xelatex

\documentclass[french]{report}

%\usepackage[utf8]{inputenc}
%\usepackage[T1]{fontenc}
\usepackage{babel}


\newif\ifcomment
%\commenttrue # Show comments

\usepackage{physics}
\usepackage{amssymb}


\usepackage{amsthm}
% \usepackage{thmtools}
\usepackage{mathtools}
\usepackage{amsfonts}

\usepackage{color}

\usepackage{tikz}

\usepackage{geometry}
\geometry{a5paper, margin=0.1in, right=1cm}

\usepackage{dsfont}

\usepackage{graphicx}
\graphicspath{ {images/} }

\usepackage{faktor}

\usepackage{IEEEtrantools}
\usepackage{enumerate}   
\usepackage[PostScript=dvips]{"/Users/aware/Documents/Courses/diagrams"}


\newtheorem{theorem}{Théorème}[section]
\renewcommand{\thetheorem}{\arabic{theorem}}
\newtheorem{lemme}{Lemme}[section]
\renewcommand{\thelemme}{\arabic{lemme}}
\newtheorem{proposition}{Proposition}[section]
\renewcommand{\theproposition}{\arabic{proposition}}
\newtheorem{notations}{Notations}[section]
\newtheorem{problem}{Problème}[section]
\newtheorem{corollary}{Corollaire}[theorem]
\renewcommand{\thecorollary}{\arabic{corollary}}
\newtheorem{property}{Propriété}[section]
\newtheorem{objective}{Objectif}[section]

\theoremstyle{definition}
\newtheorem{definition}{Définition}[section]
\renewcommand{\thedefinition}{\arabic{definition}}
\newtheorem{exercise}{Exercice}[chapter]
\renewcommand{\theexercise}{\arabic{exercise}}
\newtheorem{example}{Exemple}[chapter]
\renewcommand{\theexample}{\arabic{example}}
\newtheorem*{solution}{Solution}
\newtheorem*{application}{Application}
\newtheorem*{notation}{Notation}
\newtheorem*{vocabulary}{Vocabulaire}
\newtheorem*{properties}{Propriétés}



\theoremstyle{remark}
\newtheorem*{remark}{Remarque}
\newtheorem*{rappel}{Rappel}


\usepackage{etoolbox}
\AtBeginEnvironment{exercise}{\small}
\AtBeginEnvironment{example}{\small}

\usepackage{cases}
\usepackage[red]{mypack}

\usepackage[framemethod=TikZ]{mdframed}

\definecolor{bg}{rgb}{0.4,0.25,0.95}
\definecolor{pagebg}{rgb}{0,0,0.5}
\surroundwithmdframed[
   topline=false,
   rightline=false,
   bottomline=false,
   leftmargin=\parindent,
   skipabove=8pt,
   skipbelow=8pt,
   linecolor=blue,
   innerbottommargin=10pt,
   % backgroundcolor=bg,font=\color{orange}\sffamily, fontcolor=white
]{definition}

\usepackage{empheq}
\usepackage[most]{tcolorbox}

\newtcbox{\mymath}[1][]{%
    nobeforeafter, math upper, tcbox raise base,
    enhanced, colframe=blue!30!black,
    colback=red!10, boxrule=1pt,
    #1}

\usepackage{unixode}


\DeclareMathOperator{\ord}{ord}
\DeclareMathOperator{\orb}{orb}
\DeclareMathOperator{\stab}{stab}
\DeclareMathOperator{\Stab}{stab}
\DeclareMathOperator{\ppcm}{ppcm}
\DeclareMathOperator{\conj}{Conj}
\DeclareMathOperator{\End}{End}
\DeclareMathOperator{\rot}{rot}
\DeclareMathOperator{\trs}{trace}
\DeclareMathOperator{\Ind}{Ind}
\DeclareMathOperator{\mat}{Mat}
\DeclareMathOperator{\id}{Id}
\DeclareMathOperator{\vect}{vect}
\DeclareMathOperator{\img}{img}
\DeclareMathOperator{\cov}{Cov}
\DeclareMathOperator{\dist}{dist}
\DeclareMathOperator{\irr}{Irr}
\DeclareMathOperator{\image}{Im}
\DeclareMathOperator{\pd}{\partial}
\DeclareMathOperator{\epi}{epi}
\DeclareMathOperator{\Argmin}{Argmin}
\DeclareMathOperator{\dom}{dom}
\DeclareMathOperator{\proj}{proj}
\DeclareMathOperator{\ctg}{ctg}
\DeclareMathOperator{\supp}{supp}
\DeclareMathOperator{\argmin}{argmin}
\DeclareMathOperator{\mult}{mult}
\DeclareMathOperator{\ch}{ch}
\DeclareMathOperator{\sh}{sh}
\DeclareMathOperator{\rang}{rang}
\DeclareMathOperator{\diam}{diam}
\DeclareMathOperator{\Epigraphe}{Epigraphe}




\usepackage{xcolor}
\everymath{\color{blue}}
%\everymath{\color[rgb]{0,1,1}}
%\pagecolor[rgb]{0,0,0.5}


\newcommand*{\pdtest}[3][]{\ensuremath{\frac{\partial^{#1} #2}{\partial #3}}}

\newcommand*{\deffunc}[6][]{\ensuremath{
\begin{array}{rcl}
#2 : #3 &\rightarrow& #4\\
#5 &\mapsto& #6
\end{array}
}}

\newcommand{\eqcolon}{\mathrel{\resizebox{\widthof{$\mathord{=}$}}{\height}{ $\!\!=\!\!\resizebox{1.2\width}{0.8\height}{\raisebox{0.23ex}{$\mathop{:}$}}\!\!$ }}}
\newcommand{\coloneq}{\mathrel{\resizebox{\widthof{$\mathord{=}$}}{\height}{ $\!\!\resizebox{1.2\width}{0.8\height}{\raisebox{0.23ex}{$\mathop{:}$}}\!\!=\!\!$ }}}
\newcommand{\eqcolonl}{\ensuremath{\mathrel{=\!\!\mathop{:}}}}
\newcommand{\coloneql}{\ensuremath{\mathrel{\mathop{:} \!\! =}}}
\newcommand{\vc}[1]{% inline column vector
  \left(\begin{smallmatrix}#1\end{smallmatrix}\right)%
}
\newcommand{\vr}[1]{% inline row vector
  \begin{smallmatrix}(\,#1\,)\end{smallmatrix}%
}
\makeatletter
\newcommand*{\defeq}{\ =\mathrel{\rlap{%
                     \raisebox{0.3ex}{$\m@th\cdot$}}%
                     \raisebox{-0.3ex}{$\m@th\cdot$}}%
                     }
\makeatother

\newcommand{\mathcircle}[1]{% inline row vector
 \overset{\circ}{#1}
}
\newcommand{\ulim}{% low limit
 \underline{\lim}
}
\newcommand{\ssi}{% iff
\iff
}
\newcommand{\ps}[2]{
\expval{#1 | #2}
}
\newcommand{\df}[1]{
\mqty{#1}
}
\newcommand{\n}[1]{
\norm{#1}
}
\newcommand{\sys}[1]{
\left\{\smqty{#1}\right.
}


\newcommand{\eqdef}{\ensuremath{\overset{\text{def}}=}}


\def\Circlearrowright{\ensuremath{%
  \rotatebox[origin=c]{230}{$\circlearrowright$}}}

\newcommand\ct[1]{\text{\rmfamily\upshape #1}}
\newcommand\question[1]{ {\color{red} ...!? \small #1}}
\newcommand\caz[1]{\left\{\begin{array} #1 \end{array}\right.}
\newcommand\const{\text{\rmfamily\upshape const}}
\newcommand\toP{ \overset{\pro}{\to}}
\newcommand\toPP{ \overset{\text{PP}}{\to}}
\newcommand{\oeq}{\mathrel{\text{\textcircled{$=$}}}}





\usepackage{xcolor}
% \usepackage[normalem]{ulem}
\usepackage{lipsum}
\makeatletter
% \newcommand\colorwave[1][blue]{\bgroup \markoverwith{\lower3.5\p@\hbox{\sixly \textcolor{#1}{\char58}}}\ULon}
%\font\sixly=lasy6 % does not re-load if already loaded, so no memory problem.

\newmdtheoremenv[
linewidth= 1pt,linecolor= blue,%
leftmargin=20,rightmargin=20,innertopmargin=0pt, innerrightmargin=40,%
tikzsetting = { draw=lightgray, line width = 0.3pt,dashed,%
dash pattern = on 15pt off 3pt},%
splittopskip=\topskip,skipbelow=\baselineskip,%
skipabove=\baselineskip,ntheorem,roundcorner=0pt,
% backgroundcolor=pagebg,font=\color{orange}\sffamily, fontcolor=white
]{examplebox}{Exemple}[section]



\newcommand\R{\mathbb{R}}
\newcommand\Z{\mathbb{Z}}
\newcommand\N{\mathbb{N}}
\newcommand\E{\mathbb{E}}
\newcommand\F{\mathcal{F}}
\newcommand\cH{\mathcal{H}}
\newcommand\V{\mathbb{V}}
\newcommand\dmo{ ^{-1} }
\newcommand\kapa{\kappa}
\newcommand\im{Im}
\newcommand\hs{\mathcal{H}}





\usepackage{soul}

\makeatletter
\newcommand*{\whiten}[1]{\llap{\textcolor{white}{{\the\SOUL@token}}\hspace{#1pt}}}
\DeclareRobustCommand*\myul{%
    \def\SOUL@everyspace{\underline{\space}\kern\z@}%
    \def\SOUL@everytoken{%
     \setbox0=\hbox{\the\SOUL@token}%
     \ifdim\dp0>\z@
        \raisebox{\dp0}{\underline{\phantom{\the\SOUL@token}}}%
        \whiten{1}\whiten{0}%
        \whiten{-1}\whiten{-2}%
        \llap{\the\SOUL@token}%
     \else
        \underline{\the\SOUL@token}%
     \fi}%
\SOUL@}
\makeatother

\newcommand*{\demp}{\fontfamily{lmtt}\selectfont}

\DeclareTextFontCommand{\textdemp}{\demp}

\begin{document}

\ifcomment
Multiline
comment
\fi
\ifcomment
\myul{Typesetting test}
% \color[rgb]{1,1,1}
$∑_i^n≠ 60º±∞π∆¬≈√j∫h≤≥µ$

$\CR \R\pro\ind\pro\gS\pro
\mqty[a&b\\c&d]$
$\pro\mathbb{P}$
$\dd{x}$

  \[
    \alpha(x)=\left\{
                \begin{array}{ll}
                  x\\
                  \frac{1}{1+e^{-kx}}\\
                  \frac{e^x-e^{-x}}{e^x+e^{-x}}
                \end{array}
              \right.
  \]

  $\expval{x}$
  
  $\chi_\rho(ghg\dmo)=\Tr(\rho_{ghg\dmo})=\Tr(\rho_g\circ\rho_h\circ\rho\dmo_g)=\Tr(\rho_h)\overset{\mbox{\scalebox{0.5}{$\Tr(AB)=\Tr(BA)$}}}{=}\chi_\rho(h)$
  	$\mathop{\oplus}_{\substack{x\in X}}$

$\mat(\rho_g)=(a_{ij}(g))_{\scriptsize \substack{1\leq i\leq d \\ 1\leq j\leq d}}$ et $\mat(\rho'_g)=(a'_{ij}(g))_{\scriptsize \substack{1\leq i'\leq d' \\ 1\leq j'\leq d'}}$



\[\int_a^b{\mathbb{R}^2}g(u, v)\dd{P_{XY}}(u, v)=\iint g(u,v) f_{XY}(u, v)\dd \lambda(u) \dd \lambda(v)\]
$$\lim_{x\to\infty} f(x)$$	
$$\iiiint_V \mu(t,u,v,w) \,dt\,du\,dv\,dw$$
$$\sum_{n=1}^{\infty} 2^{-n} = 1$$	
\begin{definition}
	Si $X$ et $Y$ sont 2 v.a. ou definit la \textsc{Covariance} entre $X$ et $Y$ comme
	$\cov(X,Y)\overset{\text{def}}{=}\E\left[(X-\E(X))(Y-\E(Y))\right]=\E(XY)-\E(X)\E(Y)$.
\end{definition}
\fi
\pagebreak

% \tableofcontents

% insert your code here
%\input{./algebra/main.tex}
%\input{./geometrie-differentielle/main.tex}
%\input{./probabilite/main.tex}
%\input{./analyse-fonctionnelle/main.tex}
% \input{./Analyse-convexe-et-dualite-en-optimisation/main.tex}
%\input{./tikz/main.tex}
%\input{./Theorie-du-distributions/main.tex}
%\input{./optimisation/mine.tex}
 \input{./modelisation/main.tex}

% yves.aubry@univ-tln.fr : algebra

\end{document}

% % !TEX encoding = UTF-8 Unicode
% !TEX TS-program = xelatex

\documentclass[french]{report}

%\usepackage[utf8]{inputenc}
%\usepackage[T1]{fontenc}
\usepackage{babel}


\newif\ifcomment
%\commenttrue # Show comments

\usepackage{physics}
\usepackage{amssymb}


\usepackage{amsthm}
% \usepackage{thmtools}
\usepackage{mathtools}
\usepackage{amsfonts}

\usepackage{color}

\usepackage{tikz}

\usepackage{geometry}
\geometry{a5paper, margin=0.1in, right=1cm}

\usepackage{dsfont}

\usepackage{graphicx}
\graphicspath{ {images/} }

\usepackage{faktor}

\usepackage{IEEEtrantools}
\usepackage{enumerate}   
\usepackage[PostScript=dvips]{"/Users/aware/Documents/Courses/diagrams"}


\newtheorem{theorem}{Théorème}[section]
\renewcommand{\thetheorem}{\arabic{theorem}}
\newtheorem{lemme}{Lemme}[section]
\renewcommand{\thelemme}{\arabic{lemme}}
\newtheorem{proposition}{Proposition}[section]
\renewcommand{\theproposition}{\arabic{proposition}}
\newtheorem{notations}{Notations}[section]
\newtheorem{problem}{Problème}[section]
\newtheorem{corollary}{Corollaire}[theorem]
\renewcommand{\thecorollary}{\arabic{corollary}}
\newtheorem{property}{Propriété}[section]
\newtheorem{objective}{Objectif}[section]

\theoremstyle{definition}
\newtheorem{definition}{Définition}[section]
\renewcommand{\thedefinition}{\arabic{definition}}
\newtheorem{exercise}{Exercice}[chapter]
\renewcommand{\theexercise}{\arabic{exercise}}
\newtheorem{example}{Exemple}[chapter]
\renewcommand{\theexample}{\arabic{example}}
\newtheorem*{solution}{Solution}
\newtheorem*{application}{Application}
\newtheorem*{notation}{Notation}
\newtheorem*{vocabulary}{Vocabulaire}
\newtheorem*{properties}{Propriétés}



\theoremstyle{remark}
\newtheorem*{remark}{Remarque}
\newtheorem*{rappel}{Rappel}


\usepackage{etoolbox}
\AtBeginEnvironment{exercise}{\small}
\AtBeginEnvironment{example}{\small}

\usepackage{cases}
\usepackage[red]{mypack}

\usepackage[framemethod=TikZ]{mdframed}

\definecolor{bg}{rgb}{0.4,0.25,0.95}
\definecolor{pagebg}{rgb}{0,0,0.5}
\surroundwithmdframed[
   topline=false,
   rightline=false,
   bottomline=false,
   leftmargin=\parindent,
   skipabove=8pt,
   skipbelow=8pt,
   linecolor=blue,
   innerbottommargin=10pt,
   % backgroundcolor=bg,font=\color{orange}\sffamily, fontcolor=white
]{definition}

\usepackage{empheq}
\usepackage[most]{tcolorbox}

\newtcbox{\mymath}[1][]{%
    nobeforeafter, math upper, tcbox raise base,
    enhanced, colframe=blue!30!black,
    colback=red!10, boxrule=1pt,
    #1}

\usepackage{unixode}


\DeclareMathOperator{\ord}{ord}
\DeclareMathOperator{\orb}{orb}
\DeclareMathOperator{\stab}{stab}
\DeclareMathOperator{\Stab}{stab}
\DeclareMathOperator{\ppcm}{ppcm}
\DeclareMathOperator{\conj}{Conj}
\DeclareMathOperator{\End}{End}
\DeclareMathOperator{\rot}{rot}
\DeclareMathOperator{\trs}{trace}
\DeclareMathOperator{\Ind}{Ind}
\DeclareMathOperator{\mat}{Mat}
\DeclareMathOperator{\id}{Id}
\DeclareMathOperator{\vect}{vect}
\DeclareMathOperator{\img}{img}
\DeclareMathOperator{\cov}{Cov}
\DeclareMathOperator{\dist}{dist}
\DeclareMathOperator{\irr}{Irr}
\DeclareMathOperator{\image}{Im}
\DeclareMathOperator{\pd}{\partial}
\DeclareMathOperator{\epi}{epi}
\DeclareMathOperator{\Argmin}{Argmin}
\DeclareMathOperator{\dom}{dom}
\DeclareMathOperator{\proj}{proj}
\DeclareMathOperator{\ctg}{ctg}
\DeclareMathOperator{\supp}{supp}
\DeclareMathOperator{\argmin}{argmin}
\DeclareMathOperator{\mult}{mult}
\DeclareMathOperator{\ch}{ch}
\DeclareMathOperator{\sh}{sh}
\DeclareMathOperator{\rang}{rang}
\DeclareMathOperator{\diam}{diam}
\DeclareMathOperator{\Epigraphe}{Epigraphe}




\usepackage{xcolor}
\everymath{\color{blue}}
%\everymath{\color[rgb]{0,1,1}}
%\pagecolor[rgb]{0,0,0.5}


\newcommand*{\pdtest}[3][]{\ensuremath{\frac{\partial^{#1} #2}{\partial #3}}}

\newcommand*{\deffunc}[6][]{\ensuremath{
\begin{array}{rcl}
#2 : #3 &\rightarrow& #4\\
#5 &\mapsto& #6
\end{array}
}}

\newcommand{\eqcolon}{\mathrel{\resizebox{\widthof{$\mathord{=}$}}{\height}{ $\!\!=\!\!\resizebox{1.2\width}{0.8\height}{\raisebox{0.23ex}{$\mathop{:}$}}\!\!$ }}}
\newcommand{\coloneq}{\mathrel{\resizebox{\widthof{$\mathord{=}$}}{\height}{ $\!\!\resizebox{1.2\width}{0.8\height}{\raisebox{0.23ex}{$\mathop{:}$}}\!\!=\!\!$ }}}
\newcommand{\eqcolonl}{\ensuremath{\mathrel{=\!\!\mathop{:}}}}
\newcommand{\coloneql}{\ensuremath{\mathrel{\mathop{:} \!\! =}}}
\newcommand{\vc}[1]{% inline column vector
  \left(\begin{smallmatrix}#1\end{smallmatrix}\right)%
}
\newcommand{\vr}[1]{% inline row vector
  \begin{smallmatrix}(\,#1\,)\end{smallmatrix}%
}
\makeatletter
\newcommand*{\defeq}{\ =\mathrel{\rlap{%
                     \raisebox{0.3ex}{$\m@th\cdot$}}%
                     \raisebox{-0.3ex}{$\m@th\cdot$}}%
                     }
\makeatother

\newcommand{\mathcircle}[1]{% inline row vector
 \overset{\circ}{#1}
}
\newcommand{\ulim}{% low limit
 \underline{\lim}
}
\newcommand{\ssi}{% iff
\iff
}
\newcommand{\ps}[2]{
\expval{#1 | #2}
}
\newcommand{\df}[1]{
\mqty{#1}
}
\newcommand{\n}[1]{
\norm{#1}
}
\newcommand{\sys}[1]{
\left\{\smqty{#1}\right.
}


\newcommand{\eqdef}{\ensuremath{\overset{\text{def}}=}}


\def\Circlearrowright{\ensuremath{%
  \rotatebox[origin=c]{230}{$\circlearrowright$}}}

\newcommand\ct[1]{\text{\rmfamily\upshape #1}}
\newcommand\question[1]{ {\color{red} ...!? \small #1}}
\newcommand\caz[1]{\left\{\begin{array} #1 \end{array}\right.}
\newcommand\const{\text{\rmfamily\upshape const}}
\newcommand\toP{ \overset{\pro}{\to}}
\newcommand\toPP{ \overset{\text{PP}}{\to}}
\newcommand{\oeq}{\mathrel{\text{\textcircled{$=$}}}}





\usepackage{xcolor}
% \usepackage[normalem]{ulem}
\usepackage{lipsum}
\makeatletter
% \newcommand\colorwave[1][blue]{\bgroup \markoverwith{\lower3.5\p@\hbox{\sixly \textcolor{#1}{\char58}}}\ULon}
%\font\sixly=lasy6 % does not re-load if already loaded, so no memory problem.

\newmdtheoremenv[
linewidth= 1pt,linecolor= blue,%
leftmargin=20,rightmargin=20,innertopmargin=0pt, innerrightmargin=40,%
tikzsetting = { draw=lightgray, line width = 0.3pt,dashed,%
dash pattern = on 15pt off 3pt},%
splittopskip=\topskip,skipbelow=\baselineskip,%
skipabove=\baselineskip,ntheorem,roundcorner=0pt,
% backgroundcolor=pagebg,font=\color{orange}\sffamily, fontcolor=white
]{examplebox}{Exemple}[section]



\newcommand\R{\mathbb{R}}
\newcommand\Z{\mathbb{Z}}
\newcommand\N{\mathbb{N}}
\newcommand\E{\mathbb{E}}
\newcommand\F{\mathcal{F}}
\newcommand\cH{\mathcal{H}}
\newcommand\V{\mathbb{V}}
\newcommand\dmo{ ^{-1} }
\newcommand\kapa{\kappa}
\newcommand\im{Im}
\newcommand\hs{\mathcal{H}}





\usepackage{soul}

\makeatletter
\newcommand*{\whiten}[1]{\llap{\textcolor{white}{{\the\SOUL@token}}\hspace{#1pt}}}
\DeclareRobustCommand*\myul{%
    \def\SOUL@everyspace{\underline{\space}\kern\z@}%
    \def\SOUL@everytoken{%
     \setbox0=\hbox{\the\SOUL@token}%
     \ifdim\dp0>\z@
        \raisebox{\dp0}{\underline{\phantom{\the\SOUL@token}}}%
        \whiten{1}\whiten{0}%
        \whiten{-1}\whiten{-2}%
        \llap{\the\SOUL@token}%
     \else
        \underline{\the\SOUL@token}%
     \fi}%
\SOUL@}
\makeatother

\newcommand*{\demp}{\fontfamily{lmtt}\selectfont}

\DeclareTextFontCommand{\textdemp}{\demp}

\begin{document}

\ifcomment
Multiline
comment
\fi
\ifcomment
\myul{Typesetting test}
% \color[rgb]{1,1,1}
$∑_i^n≠ 60º±∞π∆¬≈√j∫h≤≥µ$

$\CR \R\pro\ind\pro\gS\pro
\mqty[a&b\\c&d]$
$\pro\mathbb{P}$
$\dd{x}$

  \[
    \alpha(x)=\left\{
                \begin{array}{ll}
                  x\\
                  \frac{1}{1+e^{-kx}}\\
                  \frac{e^x-e^{-x}}{e^x+e^{-x}}
                \end{array}
              \right.
  \]

  $\expval{x}$
  
  $\chi_\rho(ghg\dmo)=\Tr(\rho_{ghg\dmo})=\Tr(\rho_g\circ\rho_h\circ\rho\dmo_g)=\Tr(\rho_h)\overset{\mbox{\scalebox{0.5}{$\Tr(AB)=\Tr(BA)$}}}{=}\chi_\rho(h)$
  	$\mathop{\oplus}_{\substack{x\in X}}$

$\mat(\rho_g)=(a_{ij}(g))_{\scriptsize \substack{1\leq i\leq d \\ 1\leq j\leq d}}$ et $\mat(\rho'_g)=(a'_{ij}(g))_{\scriptsize \substack{1\leq i'\leq d' \\ 1\leq j'\leq d'}}$



\[\int_a^b{\mathbb{R}^2}g(u, v)\dd{P_{XY}}(u, v)=\iint g(u,v) f_{XY}(u, v)\dd \lambda(u) \dd \lambda(v)\]
$$\lim_{x\to\infty} f(x)$$	
$$\iiiint_V \mu(t,u,v,w) \,dt\,du\,dv\,dw$$
$$\sum_{n=1}^{\infty} 2^{-n} = 1$$	
\begin{definition}
	Si $X$ et $Y$ sont 2 v.a. ou definit la \textsc{Covariance} entre $X$ et $Y$ comme
	$\cov(X,Y)\overset{\text{def}}{=}\E\left[(X-\E(X))(Y-\E(Y))\right]=\E(XY)-\E(X)\E(Y)$.
\end{definition}
\fi
\pagebreak

% \tableofcontents

% insert your code here
%\input{./algebra/main.tex}
%\input{./geometrie-differentielle/main.tex}
%\input{./probabilite/main.tex}
%\input{./analyse-fonctionnelle/main.tex}
% \input{./Analyse-convexe-et-dualite-en-optimisation/main.tex}
%\input{./tikz/main.tex}
%\input{./Theorie-du-distributions/main.tex}
%\input{./optimisation/mine.tex}
 \input{./modelisation/main.tex}

% yves.aubry@univ-tln.fr : algebra

\end{document}

%% !TEX encoding = UTF-8 Unicode
% !TEX TS-program = xelatex

\documentclass[french]{report}

%\usepackage[utf8]{inputenc}
%\usepackage[T1]{fontenc}
\usepackage{babel}


\newif\ifcomment
%\commenttrue # Show comments

\usepackage{physics}
\usepackage{amssymb}


\usepackage{amsthm}
% \usepackage{thmtools}
\usepackage{mathtools}
\usepackage{amsfonts}

\usepackage{color}

\usepackage{tikz}

\usepackage{geometry}
\geometry{a5paper, margin=0.1in, right=1cm}

\usepackage{dsfont}

\usepackage{graphicx}
\graphicspath{ {images/} }

\usepackage{faktor}

\usepackage{IEEEtrantools}
\usepackage{enumerate}   
\usepackage[PostScript=dvips]{"/Users/aware/Documents/Courses/diagrams"}


\newtheorem{theorem}{Théorème}[section]
\renewcommand{\thetheorem}{\arabic{theorem}}
\newtheorem{lemme}{Lemme}[section]
\renewcommand{\thelemme}{\arabic{lemme}}
\newtheorem{proposition}{Proposition}[section]
\renewcommand{\theproposition}{\arabic{proposition}}
\newtheorem{notations}{Notations}[section]
\newtheorem{problem}{Problème}[section]
\newtheorem{corollary}{Corollaire}[theorem]
\renewcommand{\thecorollary}{\arabic{corollary}}
\newtheorem{property}{Propriété}[section]
\newtheorem{objective}{Objectif}[section]

\theoremstyle{definition}
\newtheorem{definition}{Définition}[section]
\renewcommand{\thedefinition}{\arabic{definition}}
\newtheorem{exercise}{Exercice}[chapter]
\renewcommand{\theexercise}{\arabic{exercise}}
\newtheorem{example}{Exemple}[chapter]
\renewcommand{\theexample}{\arabic{example}}
\newtheorem*{solution}{Solution}
\newtheorem*{application}{Application}
\newtheorem*{notation}{Notation}
\newtheorem*{vocabulary}{Vocabulaire}
\newtheorem*{properties}{Propriétés}



\theoremstyle{remark}
\newtheorem*{remark}{Remarque}
\newtheorem*{rappel}{Rappel}


\usepackage{etoolbox}
\AtBeginEnvironment{exercise}{\small}
\AtBeginEnvironment{example}{\small}

\usepackage{cases}
\usepackage[red]{mypack}

\usepackage[framemethod=TikZ]{mdframed}

\definecolor{bg}{rgb}{0.4,0.25,0.95}
\definecolor{pagebg}{rgb}{0,0,0.5}
\surroundwithmdframed[
   topline=false,
   rightline=false,
   bottomline=false,
   leftmargin=\parindent,
   skipabove=8pt,
   skipbelow=8pt,
   linecolor=blue,
   innerbottommargin=10pt,
   % backgroundcolor=bg,font=\color{orange}\sffamily, fontcolor=white
]{definition}

\usepackage{empheq}
\usepackage[most]{tcolorbox}

\newtcbox{\mymath}[1][]{%
    nobeforeafter, math upper, tcbox raise base,
    enhanced, colframe=blue!30!black,
    colback=red!10, boxrule=1pt,
    #1}

\usepackage{unixode}


\DeclareMathOperator{\ord}{ord}
\DeclareMathOperator{\orb}{orb}
\DeclareMathOperator{\stab}{stab}
\DeclareMathOperator{\Stab}{stab}
\DeclareMathOperator{\ppcm}{ppcm}
\DeclareMathOperator{\conj}{Conj}
\DeclareMathOperator{\End}{End}
\DeclareMathOperator{\rot}{rot}
\DeclareMathOperator{\trs}{trace}
\DeclareMathOperator{\Ind}{Ind}
\DeclareMathOperator{\mat}{Mat}
\DeclareMathOperator{\id}{Id}
\DeclareMathOperator{\vect}{vect}
\DeclareMathOperator{\img}{img}
\DeclareMathOperator{\cov}{Cov}
\DeclareMathOperator{\dist}{dist}
\DeclareMathOperator{\irr}{Irr}
\DeclareMathOperator{\image}{Im}
\DeclareMathOperator{\pd}{\partial}
\DeclareMathOperator{\epi}{epi}
\DeclareMathOperator{\Argmin}{Argmin}
\DeclareMathOperator{\dom}{dom}
\DeclareMathOperator{\proj}{proj}
\DeclareMathOperator{\ctg}{ctg}
\DeclareMathOperator{\supp}{supp}
\DeclareMathOperator{\argmin}{argmin}
\DeclareMathOperator{\mult}{mult}
\DeclareMathOperator{\ch}{ch}
\DeclareMathOperator{\sh}{sh}
\DeclareMathOperator{\rang}{rang}
\DeclareMathOperator{\diam}{diam}
\DeclareMathOperator{\Epigraphe}{Epigraphe}




\usepackage{xcolor}
\everymath{\color{blue}}
%\everymath{\color[rgb]{0,1,1}}
%\pagecolor[rgb]{0,0,0.5}


\newcommand*{\pdtest}[3][]{\ensuremath{\frac{\partial^{#1} #2}{\partial #3}}}

\newcommand*{\deffunc}[6][]{\ensuremath{
\begin{array}{rcl}
#2 : #3 &\rightarrow& #4\\
#5 &\mapsto& #6
\end{array}
}}

\newcommand{\eqcolon}{\mathrel{\resizebox{\widthof{$\mathord{=}$}}{\height}{ $\!\!=\!\!\resizebox{1.2\width}{0.8\height}{\raisebox{0.23ex}{$\mathop{:}$}}\!\!$ }}}
\newcommand{\coloneq}{\mathrel{\resizebox{\widthof{$\mathord{=}$}}{\height}{ $\!\!\resizebox{1.2\width}{0.8\height}{\raisebox{0.23ex}{$\mathop{:}$}}\!\!=\!\!$ }}}
\newcommand{\eqcolonl}{\ensuremath{\mathrel{=\!\!\mathop{:}}}}
\newcommand{\coloneql}{\ensuremath{\mathrel{\mathop{:} \!\! =}}}
\newcommand{\vc}[1]{% inline column vector
  \left(\begin{smallmatrix}#1\end{smallmatrix}\right)%
}
\newcommand{\vr}[1]{% inline row vector
  \begin{smallmatrix}(\,#1\,)\end{smallmatrix}%
}
\makeatletter
\newcommand*{\defeq}{\ =\mathrel{\rlap{%
                     \raisebox{0.3ex}{$\m@th\cdot$}}%
                     \raisebox{-0.3ex}{$\m@th\cdot$}}%
                     }
\makeatother

\newcommand{\mathcircle}[1]{% inline row vector
 \overset{\circ}{#1}
}
\newcommand{\ulim}{% low limit
 \underline{\lim}
}
\newcommand{\ssi}{% iff
\iff
}
\newcommand{\ps}[2]{
\expval{#1 | #2}
}
\newcommand{\df}[1]{
\mqty{#1}
}
\newcommand{\n}[1]{
\norm{#1}
}
\newcommand{\sys}[1]{
\left\{\smqty{#1}\right.
}


\newcommand{\eqdef}{\ensuremath{\overset{\text{def}}=}}


\def\Circlearrowright{\ensuremath{%
  \rotatebox[origin=c]{230}{$\circlearrowright$}}}

\newcommand\ct[1]{\text{\rmfamily\upshape #1}}
\newcommand\question[1]{ {\color{red} ...!? \small #1}}
\newcommand\caz[1]{\left\{\begin{array} #1 \end{array}\right.}
\newcommand\const{\text{\rmfamily\upshape const}}
\newcommand\toP{ \overset{\pro}{\to}}
\newcommand\toPP{ \overset{\text{PP}}{\to}}
\newcommand{\oeq}{\mathrel{\text{\textcircled{$=$}}}}





\usepackage{xcolor}
% \usepackage[normalem]{ulem}
\usepackage{lipsum}
\makeatletter
% \newcommand\colorwave[1][blue]{\bgroup \markoverwith{\lower3.5\p@\hbox{\sixly \textcolor{#1}{\char58}}}\ULon}
%\font\sixly=lasy6 % does not re-load if already loaded, so no memory problem.

\newmdtheoremenv[
linewidth= 1pt,linecolor= blue,%
leftmargin=20,rightmargin=20,innertopmargin=0pt, innerrightmargin=40,%
tikzsetting = { draw=lightgray, line width = 0.3pt,dashed,%
dash pattern = on 15pt off 3pt},%
splittopskip=\topskip,skipbelow=\baselineskip,%
skipabove=\baselineskip,ntheorem,roundcorner=0pt,
% backgroundcolor=pagebg,font=\color{orange}\sffamily, fontcolor=white
]{examplebox}{Exemple}[section]



\newcommand\R{\mathbb{R}}
\newcommand\Z{\mathbb{Z}}
\newcommand\N{\mathbb{N}}
\newcommand\E{\mathbb{E}}
\newcommand\F{\mathcal{F}}
\newcommand\cH{\mathcal{H}}
\newcommand\V{\mathbb{V}}
\newcommand\dmo{ ^{-1} }
\newcommand\kapa{\kappa}
\newcommand\im{Im}
\newcommand\hs{\mathcal{H}}





\usepackage{soul}

\makeatletter
\newcommand*{\whiten}[1]{\llap{\textcolor{white}{{\the\SOUL@token}}\hspace{#1pt}}}
\DeclareRobustCommand*\myul{%
    \def\SOUL@everyspace{\underline{\space}\kern\z@}%
    \def\SOUL@everytoken{%
     \setbox0=\hbox{\the\SOUL@token}%
     \ifdim\dp0>\z@
        \raisebox{\dp0}{\underline{\phantom{\the\SOUL@token}}}%
        \whiten{1}\whiten{0}%
        \whiten{-1}\whiten{-2}%
        \llap{\the\SOUL@token}%
     \else
        \underline{\the\SOUL@token}%
     \fi}%
\SOUL@}
\makeatother

\newcommand*{\demp}{\fontfamily{lmtt}\selectfont}

\DeclareTextFontCommand{\textdemp}{\demp}

\begin{document}

\ifcomment
Multiline
comment
\fi
\ifcomment
\myul{Typesetting test}
% \color[rgb]{1,1,1}
$∑_i^n≠ 60º±∞π∆¬≈√j∫h≤≥µ$

$\CR \R\pro\ind\pro\gS\pro
\mqty[a&b\\c&d]$
$\pro\mathbb{P}$
$\dd{x}$

  \[
    \alpha(x)=\left\{
                \begin{array}{ll}
                  x\\
                  \frac{1}{1+e^{-kx}}\\
                  \frac{e^x-e^{-x}}{e^x+e^{-x}}
                \end{array}
              \right.
  \]

  $\expval{x}$
  
  $\chi_\rho(ghg\dmo)=\Tr(\rho_{ghg\dmo})=\Tr(\rho_g\circ\rho_h\circ\rho\dmo_g)=\Tr(\rho_h)\overset{\mbox{\scalebox{0.5}{$\Tr(AB)=\Tr(BA)$}}}{=}\chi_\rho(h)$
  	$\mathop{\oplus}_{\substack{x\in X}}$

$\mat(\rho_g)=(a_{ij}(g))_{\scriptsize \substack{1\leq i\leq d \\ 1\leq j\leq d}}$ et $\mat(\rho'_g)=(a'_{ij}(g))_{\scriptsize \substack{1\leq i'\leq d' \\ 1\leq j'\leq d'}}$



\[\int_a^b{\mathbb{R}^2}g(u, v)\dd{P_{XY}}(u, v)=\iint g(u,v) f_{XY}(u, v)\dd \lambda(u) \dd \lambda(v)\]
$$\lim_{x\to\infty} f(x)$$	
$$\iiiint_V \mu(t,u,v,w) \,dt\,du\,dv\,dw$$
$$\sum_{n=1}^{\infty} 2^{-n} = 1$$	
\begin{definition}
	Si $X$ et $Y$ sont 2 v.a. ou definit la \textsc{Covariance} entre $X$ et $Y$ comme
	$\cov(X,Y)\overset{\text{def}}{=}\E\left[(X-\E(X))(Y-\E(Y))\right]=\E(XY)-\E(X)\E(Y)$.
\end{definition}
\fi
\pagebreak

% \tableofcontents

% insert your code here
%\input{./algebra/main.tex}
%\input{./geometrie-differentielle/main.tex}
%\input{./probabilite/main.tex}
%\input{./analyse-fonctionnelle/main.tex}
% \input{./Analyse-convexe-et-dualite-en-optimisation/main.tex}
%\input{./tikz/main.tex}
%\input{./Theorie-du-distributions/main.tex}
%\input{./optimisation/mine.tex}
 \input{./modelisation/main.tex}

% yves.aubry@univ-tln.fr : algebra

\end{document}

%% !TEX encoding = UTF-8 Unicode
% !TEX TS-program = xelatex

\documentclass[french]{report}

%\usepackage[utf8]{inputenc}
%\usepackage[T1]{fontenc}
\usepackage{babel}


\newif\ifcomment
%\commenttrue # Show comments

\usepackage{physics}
\usepackage{amssymb}


\usepackage{amsthm}
% \usepackage{thmtools}
\usepackage{mathtools}
\usepackage{amsfonts}

\usepackage{color}

\usepackage{tikz}

\usepackage{geometry}
\geometry{a5paper, margin=0.1in, right=1cm}

\usepackage{dsfont}

\usepackage{graphicx}
\graphicspath{ {images/} }

\usepackage{faktor}

\usepackage{IEEEtrantools}
\usepackage{enumerate}   
\usepackage[PostScript=dvips]{"/Users/aware/Documents/Courses/diagrams"}


\newtheorem{theorem}{Théorème}[section]
\renewcommand{\thetheorem}{\arabic{theorem}}
\newtheorem{lemme}{Lemme}[section]
\renewcommand{\thelemme}{\arabic{lemme}}
\newtheorem{proposition}{Proposition}[section]
\renewcommand{\theproposition}{\arabic{proposition}}
\newtheorem{notations}{Notations}[section]
\newtheorem{problem}{Problème}[section]
\newtheorem{corollary}{Corollaire}[theorem]
\renewcommand{\thecorollary}{\arabic{corollary}}
\newtheorem{property}{Propriété}[section]
\newtheorem{objective}{Objectif}[section]

\theoremstyle{definition}
\newtheorem{definition}{Définition}[section]
\renewcommand{\thedefinition}{\arabic{definition}}
\newtheorem{exercise}{Exercice}[chapter]
\renewcommand{\theexercise}{\arabic{exercise}}
\newtheorem{example}{Exemple}[chapter]
\renewcommand{\theexample}{\arabic{example}}
\newtheorem*{solution}{Solution}
\newtheorem*{application}{Application}
\newtheorem*{notation}{Notation}
\newtheorem*{vocabulary}{Vocabulaire}
\newtheorem*{properties}{Propriétés}



\theoremstyle{remark}
\newtheorem*{remark}{Remarque}
\newtheorem*{rappel}{Rappel}


\usepackage{etoolbox}
\AtBeginEnvironment{exercise}{\small}
\AtBeginEnvironment{example}{\small}

\usepackage{cases}
\usepackage[red]{mypack}

\usepackage[framemethod=TikZ]{mdframed}

\definecolor{bg}{rgb}{0.4,0.25,0.95}
\definecolor{pagebg}{rgb}{0,0,0.5}
\surroundwithmdframed[
   topline=false,
   rightline=false,
   bottomline=false,
   leftmargin=\parindent,
   skipabove=8pt,
   skipbelow=8pt,
   linecolor=blue,
   innerbottommargin=10pt,
   % backgroundcolor=bg,font=\color{orange}\sffamily, fontcolor=white
]{definition}

\usepackage{empheq}
\usepackage[most]{tcolorbox}

\newtcbox{\mymath}[1][]{%
    nobeforeafter, math upper, tcbox raise base,
    enhanced, colframe=blue!30!black,
    colback=red!10, boxrule=1pt,
    #1}

\usepackage{unixode}


\DeclareMathOperator{\ord}{ord}
\DeclareMathOperator{\orb}{orb}
\DeclareMathOperator{\stab}{stab}
\DeclareMathOperator{\Stab}{stab}
\DeclareMathOperator{\ppcm}{ppcm}
\DeclareMathOperator{\conj}{Conj}
\DeclareMathOperator{\End}{End}
\DeclareMathOperator{\rot}{rot}
\DeclareMathOperator{\trs}{trace}
\DeclareMathOperator{\Ind}{Ind}
\DeclareMathOperator{\mat}{Mat}
\DeclareMathOperator{\id}{Id}
\DeclareMathOperator{\vect}{vect}
\DeclareMathOperator{\img}{img}
\DeclareMathOperator{\cov}{Cov}
\DeclareMathOperator{\dist}{dist}
\DeclareMathOperator{\irr}{Irr}
\DeclareMathOperator{\image}{Im}
\DeclareMathOperator{\pd}{\partial}
\DeclareMathOperator{\epi}{epi}
\DeclareMathOperator{\Argmin}{Argmin}
\DeclareMathOperator{\dom}{dom}
\DeclareMathOperator{\proj}{proj}
\DeclareMathOperator{\ctg}{ctg}
\DeclareMathOperator{\supp}{supp}
\DeclareMathOperator{\argmin}{argmin}
\DeclareMathOperator{\mult}{mult}
\DeclareMathOperator{\ch}{ch}
\DeclareMathOperator{\sh}{sh}
\DeclareMathOperator{\rang}{rang}
\DeclareMathOperator{\diam}{diam}
\DeclareMathOperator{\Epigraphe}{Epigraphe}




\usepackage{xcolor}
\everymath{\color{blue}}
%\everymath{\color[rgb]{0,1,1}}
%\pagecolor[rgb]{0,0,0.5}


\newcommand*{\pdtest}[3][]{\ensuremath{\frac{\partial^{#1} #2}{\partial #3}}}

\newcommand*{\deffunc}[6][]{\ensuremath{
\begin{array}{rcl}
#2 : #3 &\rightarrow& #4\\
#5 &\mapsto& #6
\end{array}
}}

\newcommand{\eqcolon}{\mathrel{\resizebox{\widthof{$\mathord{=}$}}{\height}{ $\!\!=\!\!\resizebox{1.2\width}{0.8\height}{\raisebox{0.23ex}{$\mathop{:}$}}\!\!$ }}}
\newcommand{\coloneq}{\mathrel{\resizebox{\widthof{$\mathord{=}$}}{\height}{ $\!\!\resizebox{1.2\width}{0.8\height}{\raisebox{0.23ex}{$\mathop{:}$}}\!\!=\!\!$ }}}
\newcommand{\eqcolonl}{\ensuremath{\mathrel{=\!\!\mathop{:}}}}
\newcommand{\coloneql}{\ensuremath{\mathrel{\mathop{:} \!\! =}}}
\newcommand{\vc}[1]{% inline column vector
  \left(\begin{smallmatrix}#1\end{smallmatrix}\right)%
}
\newcommand{\vr}[1]{% inline row vector
  \begin{smallmatrix}(\,#1\,)\end{smallmatrix}%
}
\makeatletter
\newcommand*{\defeq}{\ =\mathrel{\rlap{%
                     \raisebox{0.3ex}{$\m@th\cdot$}}%
                     \raisebox{-0.3ex}{$\m@th\cdot$}}%
                     }
\makeatother

\newcommand{\mathcircle}[1]{% inline row vector
 \overset{\circ}{#1}
}
\newcommand{\ulim}{% low limit
 \underline{\lim}
}
\newcommand{\ssi}{% iff
\iff
}
\newcommand{\ps}[2]{
\expval{#1 | #2}
}
\newcommand{\df}[1]{
\mqty{#1}
}
\newcommand{\n}[1]{
\norm{#1}
}
\newcommand{\sys}[1]{
\left\{\smqty{#1}\right.
}


\newcommand{\eqdef}{\ensuremath{\overset{\text{def}}=}}


\def\Circlearrowright{\ensuremath{%
  \rotatebox[origin=c]{230}{$\circlearrowright$}}}

\newcommand\ct[1]{\text{\rmfamily\upshape #1}}
\newcommand\question[1]{ {\color{red} ...!? \small #1}}
\newcommand\caz[1]{\left\{\begin{array} #1 \end{array}\right.}
\newcommand\const{\text{\rmfamily\upshape const}}
\newcommand\toP{ \overset{\pro}{\to}}
\newcommand\toPP{ \overset{\text{PP}}{\to}}
\newcommand{\oeq}{\mathrel{\text{\textcircled{$=$}}}}





\usepackage{xcolor}
% \usepackage[normalem]{ulem}
\usepackage{lipsum}
\makeatletter
% \newcommand\colorwave[1][blue]{\bgroup \markoverwith{\lower3.5\p@\hbox{\sixly \textcolor{#1}{\char58}}}\ULon}
%\font\sixly=lasy6 % does not re-load if already loaded, so no memory problem.

\newmdtheoremenv[
linewidth= 1pt,linecolor= blue,%
leftmargin=20,rightmargin=20,innertopmargin=0pt, innerrightmargin=40,%
tikzsetting = { draw=lightgray, line width = 0.3pt,dashed,%
dash pattern = on 15pt off 3pt},%
splittopskip=\topskip,skipbelow=\baselineskip,%
skipabove=\baselineskip,ntheorem,roundcorner=0pt,
% backgroundcolor=pagebg,font=\color{orange}\sffamily, fontcolor=white
]{examplebox}{Exemple}[section]



\newcommand\R{\mathbb{R}}
\newcommand\Z{\mathbb{Z}}
\newcommand\N{\mathbb{N}}
\newcommand\E{\mathbb{E}}
\newcommand\F{\mathcal{F}}
\newcommand\cH{\mathcal{H}}
\newcommand\V{\mathbb{V}}
\newcommand\dmo{ ^{-1} }
\newcommand\kapa{\kappa}
\newcommand\im{Im}
\newcommand\hs{\mathcal{H}}





\usepackage{soul}

\makeatletter
\newcommand*{\whiten}[1]{\llap{\textcolor{white}{{\the\SOUL@token}}\hspace{#1pt}}}
\DeclareRobustCommand*\myul{%
    \def\SOUL@everyspace{\underline{\space}\kern\z@}%
    \def\SOUL@everytoken{%
     \setbox0=\hbox{\the\SOUL@token}%
     \ifdim\dp0>\z@
        \raisebox{\dp0}{\underline{\phantom{\the\SOUL@token}}}%
        \whiten{1}\whiten{0}%
        \whiten{-1}\whiten{-2}%
        \llap{\the\SOUL@token}%
     \else
        \underline{\the\SOUL@token}%
     \fi}%
\SOUL@}
\makeatother

\newcommand*{\demp}{\fontfamily{lmtt}\selectfont}

\DeclareTextFontCommand{\textdemp}{\demp}

\begin{document}

\ifcomment
Multiline
comment
\fi
\ifcomment
\myul{Typesetting test}
% \color[rgb]{1,1,1}
$∑_i^n≠ 60º±∞π∆¬≈√j∫h≤≥µ$

$\CR \R\pro\ind\pro\gS\pro
\mqty[a&b\\c&d]$
$\pro\mathbb{P}$
$\dd{x}$

  \[
    \alpha(x)=\left\{
                \begin{array}{ll}
                  x\\
                  \frac{1}{1+e^{-kx}}\\
                  \frac{e^x-e^{-x}}{e^x+e^{-x}}
                \end{array}
              \right.
  \]

  $\expval{x}$
  
  $\chi_\rho(ghg\dmo)=\Tr(\rho_{ghg\dmo})=\Tr(\rho_g\circ\rho_h\circ\rho\dmo_g)=\Tr(\rho_h)\overset{\mbox{\scalebox{0.5}{$\Tr(AB)=\Tr(BA)$}}}{=}\chi_\rho(h)$
  	$\mathop{\oplus}_{\substack{x\in X}}$

$\mat(\rho_g)=(a_{ij}(g))_{\scriptsize \substack{1\leq i\leq d \\ 1\leq j\leq d}}$ et $\mat(\rho'_g)=(a'_{ij}(g))_{\scriptsize \substack{1\leq i'\leq d' \\ 1\leq j'\leq d'}}$



\[\int_a^b{\mathbb{R}^2}g(u, v)\dd{P_{XY}}(u, v)=\iint g(u,v) f_{XY}(u, v)\dd \lambda(u) \dd \lambda(v)\]
$$\lim_{x\to\infty} f(x)$$	
$$\iiiint_V \mu(t,u,v,w) \,dt\,du\,dv\,dw$$
$$\sum_{n=1}^{\infty} 2^{-n} = 1$$	
\begin{definition}
	Si $X$ et $Y$ sont 2 v.a. ou definit la \textsc{Covariance} entre $X$ et $Y$ comme
	$\cov(X,Y)\overset{\text{def}}{=}\E\left[(X-\E(X))(Y-\E(Y))\right]=\E(XY)-\E(X)\E(Y)$.
\end{definition}
\fi
\pagebreak

% \tableofcontents

% insert your code here
%\input{./algebra/main.tex}
%\input{./geometrie-differentielle/main.tex}
%\input{./probabilite/main.tex}
%\input{./analyse-fonctionnelle/main.tex}
% \input{./Analyse-convexe-et-dualite-en-optimisation/main.tex}
%\input{./tikz/main.tex}
%\input{./Theorie-du-distributions/main.tex}
%\input{./optimisation/mine.tex}
 \input{./modelisation/main.tex}

% yves.aubry@univ-tln.fr : algebra

\end{document}

%\input{./optimisation/mine.tex}
 % !TEX encoding = UTF-8 Unicode
% !TEX TS-program = xelatex

\documentclass[french]{report}

%\usepackage[utf8]{inputenc}
%\usepackage[T1]{fontenc}
\usepackage{babel}


\newif\ifcomment
%\commenttrue # Show comments

\usepackage{physics}
\usepackage{amssymb}


\usepackage{amsthm}
% \usepackage{thmtools}
\usepackage{mathtools}
\usepackage{amsfonts}

\usepackage{color}

\usepackage{tikz}

\usepackage{geometry}
\geometry{a5paper, margin=0.1in, right=1cm}

\usepackage{dsfont}

\usepackage{graphicx}
\graphicspath{ {images/} }

\usepackage{faktor}

\usepackage{IEEEtrantools}
\usepackage{enumerate}   
\usepackage[PostScript=dvips]{"/Users/aware/Documents/Courses/diagrams"}


\newtheorem{theorem}{Théorème}[section]
\renewcommand{\thetheorem}{\arabic{theorem}}
\newtheorem{lemme}{Lemme}[section]
\renewcommand{\thelemme}{\arabic{lemme}}
\newtheorem{proposition}{Proposition}[section]
\renewcommand{\theproposition}{\arabic{proposition}}
\newtheorem{notations}{Notations}[section]
\newtheorem{problem}{Problème}[section]
\newtheorem{corollary}{Corollaire}[theorem]
\renewcommand{\thecorollary}{\arabic{corollary}}
\newtheorem{property}{Propriété}[section]
\newtheorem{objective}{Objectif}[section]

\theoremstyle{definition}
\newtheorem{definition}{Définition}[section]
\renewcommand{\thedefinition}{\arabic{definition}}
\newtheorem{exercise}{Exercice}[chapter]
\renewcommand{\theexercise}{\arabic{exercise}}
\newtheorem{example}{Exemple}[chapter]
\renewcommand{\theexample}{\arabic{example}}
\newtheorem*{solution}{Solution}
\newtheorem*{application}{Application}
\newtheorem*{notation}{Notation}
\newtheorem*{vocabulary}{Vocabulaire}
\newtheorem*{properties}{Propriétés}



\theoremstyle{remark}
\newtheorem*{remark}{Remarque}
\newtheorem*{rappel}{Rappel}


\usepackage{etoolbox}
\AtBeginEnvironment{exercise}{\small}
\AtBeginEnvironment{example}{\small}

\usepackage{cases}
\usepackage[red]{mypack}

\usepackage[framemethod=TikZ]{mdframed}

\definecolor{bg}{rgb}{0.4,0.25,0.95}
\definecolor{pagebg}{rgb}{0,0,0.5}
\surroundwithmdframed[
   topline=false,
   rightline=false,
   bottomline=false,
   leftmargin=\parindent,
   skipabove=8pt,
   skipbelow=8pt,
   linecolor=blue,
   innerbottommargin=10pt,
   % backgroundcolor=bg,font=\color{orange}\sffamily, fontcolor=white
]{definition}

\usepackage{empheq}
\usepackage[most]{tcolorbox}

\newtcbox{\mymath}[1][]{%
    nobeforeafter, math upper, tcbox raise base,
    enhanced, colframe=blue!30!black,
    colback=red!10, boxrule=1pt,
    #1}

\usepackage{unixode}


\DeclareMathOperator{\ord}{ord}
\DeclareMathOperator{\orb}{orb}
\DeclareMathOperator{\stab}{stab}
\DeclareMathOperator{\Stab}{stab}
\DeclareMathOperator{\ppcm}{ppcm}
\DeclareMathOperator{\conj}{Conj}
\DeclareMathOperator{\End}{End}
\DeclareMathOperator{\rot}{rot}
\DeclareMathOperator{\trs}{trace}
\DeclareMathOperator{\Ind}{Ind}
\DeclareMathOperator{\mat}{Mat}
\DeclareMathOperator{\id}{Id}
\DeclareMathOperator{\vect}{vect}
\DeclareMathOperator{\img}{img}
\DeclareMathOperator{\cov}{Cov}
\DeclareMathOperator{\dist}{dist}
\DeclareMathOperator{\irr}{Irr}
\DeclareMathOperator{\image}{Im}
\DeclareMathOperator{\pd}{\partial}
\DeclareMathOperator{\epi}{epi}
\DeclareMathOperator{\Argmin}{Argmin}
\DeclareMathOperator{\dom}{dom}
\DeclareMathOperator{\proj}{proj}
\DeclareMathOperator{\ctg}{ctg}
\DeclareMathOperator{\supp}{supp}
\DeclareMathOperator{\argmin}{argmin}
\DeclareMathOperator{\mult}{mult}
\DeclareMathOperator{\ch}{ch}
\DeclareMathOperator{\sh}{sh}
\DeclareMathOperator{\rang}{rang}
\DeclareMathOperator{\diam}{diam}
\DeclareMathOperator{\Epigraphe}{Epigraphe}




\usepackage{xcolor}
\everymath{\color{blue}}
%\everymath{\color[rgb]{0,1,1}}
%\pagecolor[rgb]{0,0,0.5}


\newcommand*{\pdtest}[3][]{\ensuremath{\frac{\partial^{#1} #2}{\partial #3}}}

\newcommand*{\deffunc}[6][]{\ensuremath{
\begin{array}{rcl}
#2 : #3 &\rightarrow& #4\\
#5 &\mapsto& #6
\end{array}
}}

\newcommand{\eqcolon}{\mathrel{\resizebox{\widthof{$\mathord{=}$}}{\height}{ $\!\!=\!\!\resizebox{1.2\width}{0.8\height}{\raisebox{0.23ex}{$\mathop{:}$}}\!\!$ }}}
\newcommand{\coloneq}{\mathrel{\resizebox{\widthof{$\mathord{=}$}}{\height}{ $\!\!\resizebox{1.2\width}{0.8\height}{\raisebox{0.23ex}{$\mathop{:}$}}\!\!=\!\!$ }}}
\newcommand{\eqcolonl}{\ensuremath{\mathrel{=\!\!\mathop{:}}}}
\newcommand{\coloneql}{\ensuremath{\mathrel{\mathop{:} \!\! =}}}
\newcommand{\vc}[1]{% inline column vector
  \left(\begin{smallmatrix}#1\end{smallmatrix}\right)%
}
\newcommand{\vr}[1]{% inline row vector
  \begin{smallmatrix}(\,#1\,)\end{smallmatrix}%
}
\makeatletter
\newcommand*{\defeq}{\ =\mathrel{\rlap{%
                     \raisebox{0.3ex}{$\m@th\cdot$}}%
                     \raisebox{-0.3ex}{$\m@th\cdot$}}%
                     }
\makeatother

\newcommand{\mathcircle}[1]{% inline row vector
 \overset{\circ}{#1}
}
\newcommand{\ulim}{% low limit
 \underline{\lim}
}
\newcommand{\ssi}{% iff
\iff
}
\newcommand{\ps}[2]{
\expval{#1 | #2}
}
\newcommand{\df}[1]{
\mqty{#1}
}
\newcommand{\n}[1]{
\norm{#1}
}
\newcommand{\sys}[1]{
\left\{\smqty{#1}\right.
}


\newcommand{\eqdef}{\ensuremath{\overset{\text{def}}=}}


\def\Circlearrowright{\ensuremath{%
  \rotatebox[origin=c]{230}{$\circlearrowright$}}}

\newcommand\ct[1]{\text{\rmfamily\upshape #1}}
\newcommand\question[1]{ {\color{red} ...!? \small #1}}
\newcommand\caz[1]{\left\{\begin{array} #1 \end{array}\right.}
\newcommand\const{\text{\rmfamily\upshape const}}
\newcommand\toP{ \overset{\pro}{\to}}
\newcommand\toPP{ \overset{\text{PP}}{\to}}
\newcommand{\oeq}{\mathrel{\text{\textcircled{$=$}}}}





\usepackage{xcolor}
% \usepackage[normalem]{ulem}
\usepackage{lipsum}
\makeatletter
% \newcommand\colorwave[1][blue]{\bgroup \markoverwith{\lower3.5\p@\hbox{\sixly \textcolor{#1}{\char58}}}\ULon}
%\font\sixly=lasy6 % does not re-load if already loaded, so no memory problem.

\newmdtheoremenv[
linewidth= 1pt,linecolor= blue,%
leftmargin=20,rightmargin=20,innertopmargin=0pt, innerrightmargin=40,%
tikzsetting = { draw=lightgray, line width = 0.3pt,dashed,%
dash pattern = on 15pt off 3pt},%
splittopskip=\topskip,skipbelow=\baselineskip,%
skipabove=\baselineskip,ntheorem,roundcorner=0pt,
% backgroundcolor=pagebg,font=\color{orange}\sffamily, fontcolor=white
]{examplebox}{Exemple}[section]



\newcommand\R{\mathbb{R}}
\newcommand\Z{\mathbb{Z}}
\newcommand\N{\mathbb{N}}
\newcommand\E{\mathbb{E}}
\newcommand\F{\mathcal{F}}
\newcommand\cH{\mathcal{H}}
\newcommand\V{\mathbb{V}}
\newcommand\dmo{ ^{-1} }
\newcommand\kapa{\kappa}
\newcommand\im{Im}
\newcommand\hs{\mathcal{H}}





\usepackage{soul}

\makeatletter
\newcommand*{\whiten}[1]{\llap{\textcolor{white}{{\the\SOUL@token}}\hspace{#1pt}}}
\DeclareRobustCommand*\myul{%
    \def\SOUL@everyspace{\underline{\space}\kern\z@}%
    \def\SOUL@everytoken{%
     \setbox0=\hbox{\the\SOUL@token}%
     \ifdim\dp0>\z@
        \raisebox{\dp0}{\underline{\phantom{\the\SOUL@token}}}%
        \whiten{1}\whiten{0}%
        \whiten{-1}\whiten{-2}%
        \llap{\the\SOUL@token}%
     \else
        \underline{\the\SOUL@token}%
     \fi}%
\SOUL@}
\makeatother

\newcommand*{\demp}{\fontfamily{lmtt}\selectfont}

\DeclareTextFontCommand{\textdemp}{\demp}

\begin{document}

\ifcomment
Multiline
comment
\fi
\ifcomment
\myul{Typesetting test}
% \color[rgb]{1,1,1}
$∑_i^n≠ 60º±∞π∆¬≈√j∫h≤≥µ$

$\CR \R\pro\ind\pro\gS\pro
\mqty[a&b\\c&d]$
$\pro\mathbb{P}$
$\dd{x}$

  \[
    \alpha(x)=\left\{
                \begin{array}{ll}
                  x\\
                  \frac{1}{1+e^{-kx}}\\
                  \frac{e^x-e^{-x}}{e^x+e^{-x}}
                \end{array}
              \right.
  \]

  $\expval{x}$
  
  $\chi_\rho(ghg\dmo)=\Tr(\rho_{ghg\dmo})=\Tr(\rho_g\circ\rho_h\circ\rho\dmo_g)=\Tr(\rho_h)\overset{\mbox{\scalebox{0.5}{$\Tr(AB)=\Tr(BA)$}}}{=}\chi_\rho(h)$
  	$\mathop{\oplus}_{\substack{x\in X}}$

$\mat(\rho_g)=(a_{ij}(g))_{\scriptsize \substack{1\leq i\leq d \\ 1\leq j\leq d}}$ et $\mat(\rho'_g)=(a'_{ij}(g))_{\scriptsize \substack{1\leq i'\leq d' \\ 1\leq j'\leq d'}}$



\[\int_a^b{\mathbb{R}^2}g(u, v)\dd{P_{XY}}(u, v)=\iint g(u,v) f_{XY}(u, v)\dd \lambda(u) \dd \lambda(v)\]
$$\lim_{x\to\infty} f(x)$$	
$$\iiiint_V \mu(t,u,v,w) \,dt\,du\,dv\,dw$$
$$\sum_{n=1}^{\infty} 2^{-n} = 1$$	
\begin{definition}
	Si $X$ et $Y$ sont 2 v.a. ou definit la \textsc{Covariance} entre $X$ et $Y$ comme
	$\cov(X,Y)\overset{\text{def}}{=}\E\left[(X-\E(X))(Y-\E(Y))\right]=\E(XY)-\E(X)\E(Y)$.
\end{definition}
\fi
\pagebreak

% \tableofcontents

% insert your code here
%\input{./algebra/main.tex}
%\input{./geometrie-differentielle/main.tex}
%\input{./probabilite/main.tex}
%\input{./analyse-fonctionnelle/main.tex}
% \input{./Analyse-convexe-et-dualite-en-optimisation/main.tex}
%\input{./tikz/main.tex}
%\input{./Theorie-du-distributions/main.tex}
%\input{./optimisation/mine.tex}
 \input{./modelisation/main.tex}

% yves.aubry@univ-tln.fr : algebra

\end{document}


% yves.aubry@univ-tln.fr : algebra

\end{document}

% % !TEX encoding = UTF-8 Unicode
% !TEX TS-program = xelatex

\documentclass[french]{report}

%\usepackage[utf8]{inputenc}
%\usepackage[T1]{fontenc}
\usepackage{babel}


\newif\ifcomment
%\commenttrue # Show comments

\usepackage{physics}
\usepackage{amssymb}


\usepackage{amsthm}
% \usepackage{thmtools}
\usepackage{mathtools}
\usepackage{amsfonts}

\usepackage{color}

\usepackage{tikz}

\usepackage{geometry}
\geometry{a5paper, margin=0.1in, right=1cm}

\usepackage{dsfont}

\usepackage{graphicx}
\graphicspath{ {images/} }

\usepackage{faktor}

\usepackage{IEEEtrantools}
\usepackage{enumerate}   
\usepackage[PostScript=dvips]{"/Users/aware/Documents/Courses/diagrams"}


\newtheorem{theorem}{Théorème}[section]
\renewcommand{\thetheorem}{\arabic{theorem}}
\newtheorem{lemme}{Lemme}[section]
\renewcommand{\thelemme}{\arabic{lemme}}
\newtheorem{proposition}{Proposition}[section]
\renewcommand{\theproposition}{\arabic{proposition}}
\newtheorem{notations}{Notations}[section]
\newtheorem{problem}{Problème}[section]
\newtheorem{corollary}{Corollaire}[theorem]
\renewcommand{\thecorollary}{\arabic{corollary}}
\newtheorem{property}{Propriété}[section]
\newtheorem{objective}{Objectif}[section]

\theoremstyle{definition}
\newtheorem{definition}{Définition}[section]
\renewcommand{\thedefinition}{\arabic{definition}}
\newtheorem{exercise}{Exercice}[chapter]
\renewcommand{\theexercise}{\arabic{exercise}}
\newtheorem{example}{Exemple}[chapter]
\renewcommand{\theexample}{\arabic{example}}
\newtheorem*{solution}{Solution}
\newtheorem*{application}{Application}
\newtheorem*{notation}{Notation}
\newtheorem*{vocabulary}{Vocabulaire}
\newtheorem*{properties}{Propriétés}



\theoremstyle{remark}
\newtheorem*{remark}{Remarque}
\newtheorem*{rappel}{Rappel}


\usepackage{etoolbox}
\AtBeginEnvironment{exercise}{\small}
\AtBeginEnvironment{example}{\small}

\usepackage{cases}
\usepackage[red]{mypack}

\usepackage[framemethod=TikZ]{mdframed}

\definecolor{bg}{rgb}{0.4,0.25,0.95}
\definecolor{pagebg}{rgb}{0,0,0.5}
\surroundwithmdframed[
   topline=false,
   rightline=false,
   bottomline=false,
   leftmargin=\parindent,
   skipabove=8pt,
   skipbelow=8pt,
   linecolor=blue,
   innerbottommargin=10pt,
   % backgroundcolor=bg,font=\color{orange}\sffamily, fontcolor=white
]{definition}

\usepackage{empheq}
\usepackage[most]{tcolorbox}

\newtcbox{\mymath}[1][]{%
    nobeforeafter, math upper, tcbox raise base,
    enhanced, colframe=blue!30!black,
    colback=red!10, boxrule=1pt,
    #1}

\usepackage{unixode}


\DeclareMathOperator{\ord}{ord}
\DeclareMathOperator{\orb}{orb}
\DeclareMathOperator{\stab}{stab}
\DeclareMathOperator{\Stab}{stab}
\DeclareMathOperator{\ppcm}{ppcm}
\DeclareMathOperator{\conj}{Conj}
\DeclareMathOperator{\End}{End}
\DeclareMathOperator{\rot}{rot}
\DeclareMathOperator{\trs}{trace}
\DeclareMathOperator{\Ind}{Ind}
\DeclareMathOperator{\mat}{Mat}
\DeclareMathOperator{\id}{Id}
\DeclareMathOperator{\vect}{vect}
\DeclareMathOperator{\img}{img}
\DeclareMathOperator{\cov}{Cov}
\DeclareMathOperator{\dist}{dist}
\DeclareMathOperator{\irr}{Irr}
\DeclareMathOperator{\image}{Im}
\DeclareMathOperator{\pd}{\partial}
\DeclareMathOperator{\epi}{epi}
\DeclareMathOperator{\Argmin}{Argmin}
\DeclareMathOperator{\dom}{dom}
\DeclareMathOperator{\proj}{proj}
\DeclareMathOperator{\ctg}{ctg}
\DeclareMathOperator{\supp}{supp}
\DeclareMathOperator{\argmin}{argmin}
\DeclareMathOperator{\mult}{mult}
\DeclareMathOperator{\ch}{ch}
\DeclareMathOperator{\sh}{sh}
\DeclareMathOperator{\rang}{rang}
\DeclareMathOperator{\diam}{diam}
\DeclareMathOperator{\Epigraphe}{Epigraphe}




\usepackage{xcolor}
\everymath{\color{blue}}
%\everymath{\color[rgb]{0,1,1}}
%\pagecolor[rgb]{0,0,0.5}


\newcommand*{\pdtest}[3][]{\ensuremath{\frac{\partial^{#1} #2}{\partial #3}}}

\newcommand*{\deffunc}[6][]{\ensuremath{
\begin{array}{rcl}
#2 : #3 &\rightarrow& #4\\
#5 &\mapsto& #6
\end{array}
}}

\newcommand{\eqcolon}{\mathrel{\resizebox{\widthof{$\mathord{=}$}}{\height}{ $\!\!=\!\!\resizebox{1.2\width}{0.8\height}{\raisebox{0.23ex}{$\mathop{:}$}}\!\!$ }}}
\newcommand{\coloneq}{\mathrel{\resizebox{\widthof{$\mathord{=}$}}{\height}{ $\!\!\resizebox{1.2\width}{0.8\height}{\raisebox{0.23ex}{$\mathop{:}$}}\!\!=\!\!$ }}}
\newcommand{\eqcolonl}{\ensuremath{\mathrel{=\!\!\mathop{:}}}}
\newcommand{\coloneql}{\ensuremath{\mathrel{\mathop{:} \!\! =}}}
\newcommand{\vc}[1]{% inline column vector
  \left(\begin{smallmatrix}#1\end{smallmatrix}\right)%
}
\newcommand{\vr}[1]{% inline row vector
  \begin{smallmatrix}(\,#1\,)\end{smallmatrix}%
}
\makeatletter
\newcommand*{\defeq}{\ =\mathrel{\rlap{%
                     \raisebox{0.3ex}{$\m@th\cdot$}}%
                     \raisebox{-0.3ex}{$\m@th\cdot$}}%
                     }
\makeatother

\newcommand{\mathcircle}[1]{% inline row vector
 \overset{\circ}{#1}
}
\newcommand{\ulim}{% low limit
 \underline{\lim}
}
\newcommand{\ssi}{% iff
\iff
}
\newcommand{\ps}[2]{
\expval{#1 | #2}
}
\newcommand{\df}[1]{
\mqty{#1}
}
\newcommand{\n}[1]{
\norm{#1}
}
\newcommand{\sys}[1]{
\left\{\smqty{#1}\right.
}


\newcommand{\eqdef}{\ensuremath{\overset{\text{def}}=}}


\def\Circlearrowright{\ensuremath{%
  \rotatebox[origin=c]{230}{$\circlearrowright$}}}

\newcommand\ct[1]{\text{\rmfamily\upshape #1}}
\newcommand\question[1]{ {\color{red} ...!? \small #1}}
\newcommand\caz[1]{\left\{\begin{array} #1 \end{array}\right.}
\newcommand\const{\text{\rmfamily\upshape const}}
\newcommand\toP{ \overset{\pro}{\to}}
\newcommand\toPP{ \overset{\text{PP}}{\to}}
\newcommand{\oeq}{\mathrel{\text{\textcircled{$=$}}}}





\usepackage{xcolor}
% \usepackage[normalem]{ulem}
\usepackage{lipsum}
\makeatletter
% \newcommand\colorwave[1][blue]{\bgroup \markoverwith{\lower3.5\p@\hbox{\sixly \textcolor{#1}{\char58}}}\ULon}
%\font\sixly=lasy6 % does not re-load if already loaded, so no memory problem.

\newmdtheoremenv[
linewidth= 1pt,linecolor= blue,%
leftmargin=20,rightmargin=20,innertopmargin=0pt, innerrightmargin=40,%
tikzsetting = { draw=lightgray, line width = 0.3pt,dashed,%
dash pattern = on 15pt off 3pt},%
splittopskip=\topskip,skipbelow=\baselineskip,%
skipabove=\baselineskip,ntheorem,roundcorner=0pt,
% backgroundcolor=pagebg,font=\color{orange}\sffamily, fontcolor=white
]{examplebox}{Exemple}[section]



\newcommand\R{\mathbb{R}}
\newcommand\Z{\mathbb{Z}}
\newcommand\N{\mathbb{N}}
\newcommand\E{\mathbb{E}}
\newcommand\F{\mathcal{F}}
\newcommand\cH{\mathcal{H}}
\newcommand\V{\mathbb{V}}
\newcommand\dmo{ ^{-1} }
\newcommand\kapa{\kappa}
\newcommand\im{Im}
\newcommand\hs{\mathcal{H}}





\usepackage{soul}

\makeatletter
\newcommand*{\whiten}[1]{\llap{\textcolor{white}{{\the\SOUL@token}}\hspace{#1pt}}}
\DeclareRobustCommand*\myul{%
    \def\SOUL@everyspace{\underline{\space}\kern\z@}%
    \def\SOUL@everytoken{%
     \setbox0=\hbox{\the\SOUL@token}%
     \ifdim\dp0>\z@
        \raisebox{\dp0}{\underline{\phantom{\the\SOUL@token}}}%
        \whiten{1}\whiten{0}%
        \whiten{-1}\whiten{-2}%
        \llap{\the\SOUL@token}%
     \else
        \underline{\the\SOUL@token}%
     \fi}%
\SOUL@}
\makeatother

\newcommand*{\demp}{\fontfamily{lmtt}\selectfont}

\DeclareTextFontCommand{\textdemp}{\demp}

\begin{document}

\ifcomment
Multiline
comment
\fi
\ifcomment
\myul{Typesetting test}
% \color[rgb]{1,1,1}
$∑_i^n≠ 60º±∞π∆¬≈√j∫h≤≥µ$

$\CR \R\pro\ind\pro\gS\pro
\mqty[a&b\\c&d]$
$\pro\mathbb{P}$
$\dd{x}$

  \[
    \alpha(x)=\left\{
                \begin{array}{ll}
                  x\\
                  \frac{1}{1+e^{-kx}}\\
                  \frac{e^x-e^{-x}}{e^x+e^{-x}}
                \end{array}
              \right.
  \]

  $\expval{x}$
  
  $\chi_\rho(ghg\dmo)=\Tr(\rho_{ghg\dmo})=\Tr(\rho_g\circ\rho_h\circ\rho\dmo_g)=\Tr(\rho_h)\overset{\mbox{\scalebox{0.5}{$\Tr(AB)=\Tr(BA)$}}}{=}\chi_\rho(h)$
  	$\mathop{\oplus}_{\substack{x\in X}}$

$\mat(\rho_g)=(a_{ij}(g))_{\scriptsize \substack{1\leq i\leq d \\ 1\leq j\leq d}}$ et $\mat(\rho'_g)=(a'_{ij}(g))_{\scriptsize \substack{1\leq i'\leq d' \\ 1\leq j'\leq d'}}$



\[\int_a^b{\mathbb{R}^2}g(u, v)\dd{P_{XY}}(u, v)=\iint g(u,v) f_{XY}(u, v)\dd \lambda(u) \dd \lambda(v)\]
$$\lim_{x\to\infty} f(x)$$	
$$\iiiint_V \mu(t,u,v,w) \,dt\,du\,dv\,dw$$
$$\sum_{n=1}^{\infty} 2^{-n} = 1$$	
\begin{definition}
	Si $X$ et $Y$ sont 2 v.a. ou definit la \textsc{Covariance} entre $X$ et $Y$ comme
	$\cov(X,Y)\overset{\text{def}}{=}\E\left[(X-\E(X))(Y-\E(Y))\right]=\E(XY)-\E(X)\E(Y)$.
\end{definition}
\fi
\pagebreak

% \tableofcontents

% insert your code here
%% !TEX encoding = UTF-8 Unicode
% !TEX TS-program = xelatex

\documentclass[french]{report}

%\usepackage[utf8]{inputenc}
%\usepackage[T1]{fontenc}
\usepackage{babel}


\newif\ifcomment
%\commenttrue # Show comments

\usepackage{physics}
\usepackage{amssymb}


\usepackage{amsthm}
% \usepackage{thmtools}
\usepackage{mathtools}
\usepackage{amsfonts}

\usepackage{color}

\usepackage{tikz}

\usepackage{geometry}
\geometry{a5paper, margin=0.1in, right=1cm}

\usepackage{dsfont}

\usepackage{graphicx}
\graphicspath{ {images/} }

\usepackage{faktor}

\usepackage{IEEEtrantools}
\usepackage{enumerate}   
\usepackage[PostScript=dvips]{"/Users/aware/Documents/Courses/diagrams"}


\newtheorem{theorem}{Théorème}[section]
\renewcommand{\thetheorem}{\arabic{theorem}}
\newtheorem{lemme}{Lemme}[section]
\renewcommand{\thelemme}{\arabic{lemme}}
\newtheorem{proposition}{Proposition}[section]
\renewcommand{\theproposition}{\arabic{proposition}}
\newtheorem{notations}{Notations}[section]
\newtheorem{problem}{Problème}[section]
\newtheorem{corollary}{Corollaire}[theorem]
\renewcommand{\thecorollary}{\arabic{corollary}}
\newtheorem{property}{Propriété}[section]
\newtheorem{objective}{Objectif}[section]

\theoremstyle{definition}
\newtheorem{definition}{Définition}[section]
\renewcommand{\thedefinition}{\arabic{definition}}
\newtheorem{exercise}{Exercice}[chapter]
\renewcommand{\theexercise}{\arabic{exercise}}
\newtheorem{example}{Exemple}[chapter]
\renewcommand{\theexample}{\arabic{example}}
\newtheorem*{solution}{Solution}
\newtheorem*{application}{Application}
\newtheorem*{notation}{Notation}
\newtheorem*{vocabulary}{Vocabulaire}
\newtheorem*{properties}{Propriétés}



\theoremstyle{remark}
\newtheorem*{remark}{Remarque}
\newtheorem*{rappel}{Rappel}


\usepackage{etoolbox}
\AtBeginEnvironment{exercise}{\small}
\AtBeginEnvironment{example}{\small}

\usepackage{cases}
\usepackage[red]{mypack}

\usepackage[framemethod=TikZ]{mdframed}

\definecolor{bg}{rgb}{0.4,0.25,0.95}
\definecolor{pagebg}{rgb}{0,0,0.5}
\surroundwithmdframed[
   topline=false,
   rightline=false,
   bottomline=false,
   leftmargin=\parindent,
   skipabove=8pt,
   skipbelow=8pt,
   linecolor=blue,
   innerbottommargin=10pt,
   % backgroundcolor=bg,font=\color{orange}\sffamily, fontcolor=white
]{definition}

\usepackage{empheq}
\usepackage[most]{tcolorbox}

\newtcbox{\mymath}[1][]{%
    nobeforeafter, math upper, tcbox raise base,
    enhanced, colframe=blue!30!black,
    colback=red!10, boxrule=1pt,
    #1}

\usepackage{unixode}


\DeclareMathOperator{\ord}{ord}
\DeclareMathOperator{\orb}{orb}
\DeclareMathOperator{\stab}{stab}
\DeclareMathOperator{\Stab}{stab}
\DeclareMathOperator{\ppcm}{ppcm}
\DeclareMathOperator{\conj}{Conj}
\DeclareMathOperator{\End}{End}
\DeclareMathOperator{\rot}{rot}
\DeclareMathOperator{\trs}{trace}
\DeclareMathOperator{\Ind}{Ind}
\DeclareMathOperator{\mat}{Mat}
\DeclareMathOperator{\id}{Id}
\DeclareMathOperator{\vect}{vect}
\DeclareMathOperator{\img}{img}
\DeclareMathOperator{\cov}{Cov}
\DeclareMathOperator{\dist}{dist}
\DeclareMathOperator{\irr}{Irr}
\DeclareMathOperator{\image}{Im}
\DeclareMathOperator{\pd}{\partial}
\DeclareMathOperator{\epi}{epi}
\DeclareMathOperator{\Argmin}{Argmin}
\DeclareMathOperator{\dom}{dom}
\DeclareMathOperator{\proj}{proj}
\DeclareMathOperator{\ctg}{ctg}
\DeclareMathOperator{\supp}{supp}
\DeclareMathOperator{\argmin}{argmin}
\DeclareMathOperator{\mult}{mult}
\DeclareMathOperator{\ch}{ch}
\DeclareMathOperator{\sh}{sh}
\DeclareMathOperator{\rang}{rang}
\DeclareMathOperator{\diam}{diam}
\DeclareMathOperator{\Epigraphe}{Epigraphe}




\usepackage{xcolor}
\everymath{\color{blue}}
%\everymath{\color[rgb]{0,1,1}}
%\pagecolor[rgb]{0,0,0.5}


\newcommand*{\pdtest}[3][]{\ensuremath{\frac{\partial^{#1} #2}{\partial #3}}}

\newcommand*{\deffunc}[6][]{\ensuremath{
\begin{array}{rcl}
#2 : #3 &\rightarrow& #4\\
#5 &\mapsto& #6
\end{array}
}}

\newcommand{\eqcolon}{\mathrel{\resizebox{\widthof{$\mathord{=}$}}{\height}{ $\!\!=\!\!\resizebox{1.2\width}{0.8\height}{\raisebox{0.23ex}{$\mathop{:}$}}\!\!$ }}}
\newcommand{\coloneq}{\mathrel{\resizebox{\widthof{$\mathord{=}$}}{\height}{ $\!\!\resizebox{1.2\width}{0.8\height}{\raisebox{0.23ex}{$\mathop{:}$}}\!\!=\!\!$ }}}
\newcommand{\eqcolonl}{\ensuremath{\mathrel{=\!\!\mathop{:}}}}
\newcommand{\coloneql}{\ensuremath{\mathrel{\mathop{:} \!\! =}}}
\newcommand{\vc}[1]{% inline column vector
  \left(\begin{smallmatrix}#1\end{smallmatrix}\right)%
}
\newcommand{\vr}[1]{% inline row vector
  \begin{smallmatrix}(\,#1\,)\end{smallmatrix}%
}
\makeatletter
\newcommand*{\defeq}{\ =\mathrel{\rlap{%
                     \raisebox{0.3ex}{$\m@th\cdot$}}%
                     \raisebox{-0.3ex}{$\m@th\cdot$}}%
                     }
\makeatother

\newcommand{\mathcircle}[1]{% inline row vector
 \overset{\circ}{#1}
}
\newcommand{\ulim}{% low limit
 \underline{\lim}
}
\newcommand{\ssi}{% iff
\iff
}
\newcommand{\ps}[2]{
\expval{#1 | #2}
}
\newcommand{\df}[1]{
\mqty{#1}
}
\newcommand{\n}[1]{
\norm{#1}
}
\newcommand{\sys}[1]{
\left\{\smqty{#1}\right.
}


\newcommand{\eqdef}{\ensuremath{\overset{\text{def}}=}}


\def\Circlearrowright{\ensuremath{%
  \rotatebox[origin=c]{230}{$\circlearrowright$}}}

\newcommand\ct[1]{\text{\rmfamily\upshape #1}}
\newcommand\question[1]{ {\color{red} ...!? \small #1}}
\newcommand\caz[1]{\left\{\begin{array} #1 \end{array}\right.}
\newcommand\const{\text{\rmfamily\upshape const}}
\newcommand\toP{ \overset{\pro}{\to}}
\newcommand\toPP{ \overset{\text{PP}}{\to}}
\newcommand{\oeq}{\mathrel{\text{\textcircled{$=$}}}}





\usepackage{xcolor}
% \usepackage[normalem]{ulem}
\usepackage{lipsum}
\makeatletter
% \newcommand\colorwave[1][blue]{\bgroup \markoverwith{\lower3.5\p@\hbox{\sixly \textcolor{#1}{\char58}}}\ULon}
%\font\sixly=lasy6 % does not re-load if already loaded, so no memory problem.

\newmdtheoremenv[
linewidth= 1pt,linecolor= blue,%
leftmargin=20,rightmargin=20,innertopmargin=0pt, innerrightmargin=40,%
tikzsetting = { draw=lightgray, line width = 0.3pt,dashed,%
dash pattern = on 15pt off 3pt},%
splittopskip=\topskip,skipbelow=\baselineskip,%
skipabove=\baselineskip,ntheorem,roundcorner=0pt,
% backgroundcolor=pagebg,font=\color{orange}\sffamily, fontcolor=white
]{examplebox}{Exemple}[section]



\newcommand\R{\mathbb{R}}
\newcommand\Z{\mathbb{Z}}
\newcommand\N{\mathbb{N}}
\newcommand\E{\mathbb{E}}
\newcommand\F{\mathcal{F}}
\newcommand\cH{\mathcal{H}}
\newcommand\V{\mathbb{V}}
\newcommand\dmo{ ^{-1} }
\newcommand\kapa{\kappa}
\newcommand\im{Im}
\newcommand\hs{\mathcal{H}}





\usepackage{soul}

\makeatletter
\newcommand*{\whiten}[1]{\llap{\textcolor{white}{{\the\SOUL@token}}\hspace{#1pt}}}
\DeclareRobustCommand*\myul{%
    \def\SOUL@everyspace{\underline{\space}\kern\z@}%
    \def\SOUL@everytoken{%
     \setbox0=\hbox{\the\SOUL@token}%
     \ifdim\dp0>\z@
        \raisebox{\dp0}{\underline{\phantom{\the\SOUL@token}}}%
        \whiten{1}\whiten{0}%
        \whiten{-1}\whiten{-2}%
        \llap{\the\SOUL@token}%
     \else
        \underline{\the\SOUL@token}%
     \fi}%
\SOUL@}
\makeatother

\newcommand*{\demp}{\fontfamily{lmtt}\selectfont}

\DeclareTextFontCommand{\textdemp}{\demp}

\begin{document}

\ifcomment
Multiline
comment
\fi
\ifcomment
\myul{Typesetting test}
% \color[rgb]{1,1,1}
$∑_i^n≠ 60º±∞π∆¬≈√j∫h≤≥µ$

$\CR \R\pro\ind\pro\gS\pro
\mqty[a&b\\c&d]$
$\pro\mathbb{P}$
$\dd{x}$

  \[
    \alpha(x)=\left\{
                \begin{array}{ll}
                  x\\
                  \frac{1}{1+e^{-kx}}\\
                  \frac{e^x-e^{-x}}{e^x+e^{-x}}
                \end{array}
              \right.
  \]

  $\expval{x}$
  
  $\chi_\rho(ghg\dmo)=\Tr(\rho_{ghg\dmo})=\Tr(\rho_g\circ\rho_h\circ\rho\dmo_g)=\Tr(\rho_h)\overset{\mbox{\scalebox{0.5}{$\Tr(AB)=\Tr(BA)$}}}{=}\chi_\rho(h)$
  	$\mathop{\oplus}_{\substack{x\in X}}$

$\mat(\rho_g)=(a_{ij}(g))_{\scriptsize \substack{1\leq i\leq d \\ 1\leq j\leq d}}$ et $\mat(\rho'_g)=(a'_{ij}(g))_{\scriptsize \substack{1\leq i'\leq d' \\ 1\leq j'\leq d'}}$



\[\int_a^b{\mathbb{R}^2}g(u, v)\dd{P_{XY}}(u, v)=\iint g(u,v) f_{XY}(u, v)\dd \lambda(u) \dd \lambda(v)\]
$$\lim_{x\to\infty} f(x)$$	
$$\iiiint_V \mu(t,u,v,w) \,dt\,du\,dv\,dw$$
$$\sum_{n=1}^{\infty} 2^{-n} = 1$$	
\begin{definition}
	Si $X$ et $Y$ sont 2 v.a. ou definit la \textsc{Covariance} entre $X$ et $Y$ comme
	$\cov(X,Y)\overset{\text{def}}{=}\E\left[(X-\E(X))(Y-\E(Y))\right]=\E(XY)-\E(X)\E(Y)$.
\end{definition}
\fi
\pagebreak

% \tableofcontents

% insert your code here
%\input{./algebra/main.tex}
%\input{./geometrie-differentielle/main.tex}
%\input{./probabilite/main.tex}
%\input{./analyse-fonctionnelle/main.tex}
% \input{./Analyse-convexe-et-dualite-en-optimisation/main.tex}
%\input{./tikz/main.tex}
%\input{./Theorie-du-distributions/main.tex}
%\input{./optimisation/mine.tex}
 \input{./modelisation/main.tex}

% yves.aubry@univ-tln.fr : algebra

\end{document}

%% !TEX encoding = UTF-8 Unicode
% !TEX TS-program = xelatex

\documentclass[french]{report}

%\usepackage[utf8]{inputenc}
%\usepackage[T1]{fontenc}
\usepackage{babel}


\newif\ifcomment
%\commenttrue # Show comments

\usepackage{physics}
\usepackage{amssymb}


\usepackage{amsthm}
% \usepackage{thmtools}
\usepackage{mathtools}
\usepackage{amsfonts}

\usepackage{color}

\usepackage{tikz}

\usepackage{geometry}
\geometry{a5paper, margin=0.1in, right=1cm}

\usepackage{dsfont}

\usepackage{graphicx}
\graphicspath{ {images/} }

\usepackage{faktor}

\usepackage{IEEEtrantools}
\usepackage{enumerate}   
\usepackage[PostScript=dvips]{"/Users/aware/Documents/Courses/diagrams"}


\newtheorem{theorem}{Théorème}[section]
\renewcommand{\thetheorem}{\arabic{theorem}}
\newtheorem{lemme}{Lemme}[section]
\renewcommand{\thelemme}{\arabic{lemme}}
\newtheorem{proposition}{Proposition}[section]
\renewcommand{\theproposition}{\arabic{proposition}}
\newtheorem{notations}{Notations}[section]
\newtheorem{problem}{Problème}[section]
\newtheorem{corollary}{Corollaire}[theorem]
\renewcommand{\thecorollary}{\arabic{corollary}}
\newtheorem{property}{Propriété}[section]
\newtheorem{objective}{Objectif}[section]

\theoremstyle{definition}
\newtheorem{definition}{Définition}[section]
\renewcommand{\thedefinition}{\arabic{definition}}
\newtheorem{exercise}{Exercice}[chapter]
\renewcommand{\theexercise}{\arabic{exercise}}
\newtheorem{example}{Exemple}[chapter]
\renewcommand{\theexample}{\arabic{example}}
\newtheorem*{solution}{Solution}
\newtheorem*{application}{Application}
\newtheorem*{notation}{Notation}
\newtheorem*{vocabulary}{Vocabulaire}
\newtheorem*{properties}{Propriétés}



\theoremstyle{remark}
\newtheorem*{remark}{Remarque}
\newtheorem*{rappel}{Rappel}


\usepackage{etoolbox}
\AtBeginEnvironment{exercise}{\small}
\AtBeginEnvironment{example}{\small}

\usepackage{cases}
\usepackage[red]{mypack}

\usepackage[framemethod=TikZ]{mdframed}

\definecolor{bg}{rgb}{0.4,0.25,0.95}
\definecolor{pagebg}{rgb}{0,0,0.5}
\surroundwithmdframed[
   topline=false,
   rightline=false,
   bottomline=false,
   leftmargin=\parindent,
   skipabove=8pt,
   skipbelow=8pt,
   linecolor=blue,
   innerbottommargin=10pt,
   % backgroundcolor=bg,font=\color{orange}\sffamily, fontcolor=white
]{definition}

\usepackage{empheq}
\usepackage[most]{tcolorbox}

\newtcbox{\mymath}[1][]{%
    nobeforeafter, math upper, tcbox raise base,
    enhanced, colframe=blue!30!black,
    colback=red!10, boxrule=1pt,
    #1}

\usepackage{unixode}


\DeclareMathOperator{\ord}{ord}
\DeclareMathOperator{\orb}{orb}
\DeclareMathOperator{\stab}{stab}
\DeclareMathOperator{\Stab}{stab}
\DeclareMathOperator{\ppcm}{ppcm}
\DeclareMathOperator{\conj}{Conj}
\DeclareMathOperator{\End}{End}
\DeclareMathOperator{\rot}{rot}
\DeclareMathOperator{\trs}{trace}
\DeclareMathOperator{\Ind}{Ind}
\DeclareMathOperator{\mat}{Mat}
\DeclareMathOperator{\id}{Id}
\DeclareMathOperator{\vect}{vect}
\DeclareMathOperator{\img}{img}
\DeclareMathOperator{\cov}{Cov}
\DeclareMathOperator{\dist}{dist}
\DeclareMathOperator{\irr}{Irr}
\DeclareMathOperator{\image}{Im}
\DeclareMathOperator{\pd}{\partial}
\DeclareMathOperator{\epi}{epi}
\DeclareMathOperator{\Argmin}{Argmin}
\DeclareMathOperator{\dom}{dom}
\DeclareMathOperator{\proj}{proj}
\DeclareMathOperator{\ctg}{ctg}
\DeclareMathOperator{\supp}{supp}
\DeclareMathOperator{\argmin}{argmin}
\DeclareMathOperator{\mult}{mult}
\DeclareMathOperator{\ch}{ch}
\DeclareMathOperator{\sh}{sh}
\DeclareMathOperator{\rang}{rang}
\DeclareMathOperator{\diam}{diam}
\DeclareMathOperator{\Epigraphe}{Epigraphe}




\usepackage{xcolor}
\everymath{\color{blue}}
%\everymath{\color[rgb]{0,1,1}}
%\pagecolor[rgb]{0,0,0.5}


\newcommand*{\pdtest}[3][]{\ensuremath{\frac{\partial^{#1} #2}{\partial #3}}}

\newcommand*{\deffunc}[6][]{\ensuremath{
\begin{array}{rcl}
#2 : #3 &\rightarrow& #4\\
#5 &\mapsto& #6
\end{array}
}}

\newcommand{\eqcolon}{\mathrel{\resizebox{\widthof{$\mathord{=}$}}{\height}{ $\!\!=\!\!\resizebox{1.2\width}{0.8\height}{\raisebox{0.23ex}{$\mathop{:}$}}\!\!$ }}}
\newcommand{\coloneq}{\mathrel{\resizebox{\widthof{$\mathord{=}$}}{\height}{ $\!\!\resizebox{1.2\width}{0.8\height}{\raisebox{0.23ex}{$\mathop{:}$}}\!\!=\!\!$ }}}
\newcommand{\eqcolonl}{\ensuremath{\mathrel{=\!\!\mathop{:}}}}
\newcommand{\coloneql}{\ensuremath{\mathrel{\mathop{:} \!\! =}}}
\newcommand{\vc}[1]{% inline column vector
  \left(\begin{smallmatrix}#1\end{smallmatrix}\right)%
}
\newcommand{\vr}[1]{% inline row vector
  \begin{smallmatrix}(\,#1\,)\end{smallmatrix}%
}
\makeatletter
\newcommand*{\defeq}{\ =\mathrel{\rlap{%
                     \raisebox{0.3ex}{$\m@th\cdot$}}%
                     \raisebox{-0.3ex}{$\m@th\cdot$}}%
                     }
\makeatother

\newcommand{\mathcircle}[1]{% inline row vector
 \overset{\circ}{#1}
}
\newcommand{\ulim}{% low limit
 \underline{\lim}
}
\newcommand{\ssi}{% iff
\iff
}
\newcommand{\ps}[2]{
\expval{#1 | #2}
}
\newcommand{\df}[1]{
\mqty{#1}
}
\newcommand{\n}[1]{
\norm{#1}
}
\newcommand{\sys}[1]{
\left\{\smqty{#1}\right.
}


\newcommand{\eqdef}{\ensuremath{\overset{\text{def}}=}}


\def\Circlearrowright{\ensuremath{%
  \rotatebox[origin=c]{230}{$\circlearrowright$}}}

\newcommand\ct[1]{\text{\rmfamily\upshape #1}}
\newcommand\question[1]{ {\color{red} ...!? \small #1}}
\newcommand\caz[1]{\left\{\begin{array} #1 \end{array}\right.}
\newcommand\const{\text{\rmfamily\upshape const}}
\newcommand\toP{ \overset{\pro}{\to}}
\newcommand\toPP{ \overset{\text{PP}}{\to}}
\newcommand{\oeq}{\mathrel{\text{\textcircled{$=$}}}}





\usepackage{xcolor}
% \usepackage[normalem]{ulem}
\usepackage{lipsum}
\makeatletter
% \newcommand\colorwave[1][blue]{\bgroup \markoverwith{\lower3.5\p@\hbox{\sixly \textcolor{#1}{\char58}}}\ULon}
%\font\sixly=lasy6 % does not re-load if already loaded, so no memory problem.

\newmdtheoremenv[
linewidth= 1pt,linecolor= blue,%
leftmargin=20,rightmargin=20,innertopmargin=0pt, innerrightmargin=40,%
tikzsetting = { draw=lightgray, line width = 0.3pt,dashed,%
dash pattern = on 15pt off 3pt},%
splittopskip=\topskip,skipbelow=\baselineskip,%
skipabove=\baselineskip,ntheorem,roundcorner=0pt,
% backgroundcolor=pagebg,font=\color{orange}\sffamily, fontcolor=white
]{examplebox}{Exemple}[section]



\newcommand\R{\mathbb{R}}
\newcommand\Z{\mathbb{Z}}
\newcommand\N{\mathbb{N}}
\newcommand\E{\mathbb{E}}
\newcommand\F{\mathcal{F}}
\newcommand\cH{\mathcal{H}}
\newcommand\V{\mathbb{V}}
\newcommand\dmo{ ^{-1} }
\newcommand\kapa{\kappa}
\newcommand\im{Im}
\newcommand\hs{\mathcal{H}}





\usepackage{soul}

\makeatletter
\newcommand*{\whiten}[1]{\llap{\textcolor{white}{{\the\SOUL@token}}\hspace{#1pt}}}
\DeclareRobustCommand*\myul{%
    \def\SOUL@everyspace{\underline{\space}\kern\z@}%
    \def\SOUL@everytoken{%
     \setbox0=\hbox{\the\SOUL@token}%
     \ifdim\dp0>\z@
        \raisebox{\dp0}{\underline{\phantom{\the\SOUL@token}}}%
        \whiten{1}\whiten{0}%
        \whiten{-1}\whiten{-2}%
        \llap{\the\SOUL@token}%
     \else
        \underline{\the\SOUL@token}%
     \fi}%
\SOUL@}
\makeatother

\newcommand*{\demp}{\fontfamily{lmtt}\selectfont}

\DeclareTextFontCommand{\textdemp}{\demp}

\begin{document}

\ifcomment
Multiline
comment
\fi
\ifcomment
\myul{Typesetting test}
% \color[rgb]{1,1,1}
$∑_i^n≠ 60º±∞π∆¬≈√j∫h≤≥µ$

$\CR \R\pro\ind\pro\gS\pro
\mqty[a&b\\c&d]$
$\pro\mathbb{P}$
$\dd{x}$

  \[
    \alpha(x)=\left\{
                \begin{array}{ll}
                  x\\
                  \frac{1}{1+e^{-kx}}\\
                  \frac{e^x-e^{-x}}{e^x+e^{-x}}
                \end{array}
              \right.
  \]

  $\expval{x}$
  
  $\chi_\rho(ghg\dmo)=\Tr(\rho_{ghg\dmo})=\Tr(\rho_g\circ\rho_h\circ\rho\dmo_g)=\Tr(\rho_h)\overset{\mbox{\scalebox{0.5}{$\Tr(AB)=\Tr(BA)$}}}{=}\chi_\rho(h)$
  	$\mathop{\oplus}_{\substack{x\in X}}$

$\mat(\rho_g)=(a_{ij}(g))_{\scriptsize \substack{1\leq i\leq d \\ 1\leq j\leq d}}$ et $\mat(\rho'_g)=(a'_{ij}(g))_{\scriptsize \substack{1\leq i'\leq d' \\ 1\leq j'\leq d'}}$



\[\int_a^b{\mathbb{R}^2}g(u, v)\dd{P_{XY}}(u, v)=\iint g(u,v) f_{XY}(u, v)\dd \lambda(u) \dd \lambda(v)\]
$$\lim_{x\to\infty} f(x)$$	
$$\iiiint_V \mu(t,u,v,w) \,dt\,du\,dv\,dw$$
$$\sum_{n=1}^{\infty} 2^{-n} = 1$$	
\begin{definition}
	Si $X$ et $Y$ sont 2 v.a. ou definit la \textsc{Covariance} entre $X$ et $Y$ comme
	$\cov(X,Y)\overset{\text{def}}{=}\E\left[(X-\E(X))(Y-\E(Y))\right]=\E(XY)-\E(X)\E(Y)$.
\end{definition}
\fi
\pagebreak

% \tableofcontents

% insert your code here
%\input{./algebra/main.tex}
%\input{./geometrie-differentielle/main.tex}
%\input{./probabilite/main.tex}
%\input{./analyse-fonctionnelle/main.tex}
% \input{./Analyse-convexe-et-dualite-en-optimisation/main.tex}
%\input{./tikz/main.tex}
%\input{./Theorie-du-distributions/main.tex}
%\input{./optimisation/mine.tex}
 \input{./modelisation/main.tex}

% yves.aubry@univ-tln.fr : algebra

\end{document}

%% !TEX encoding = UTF-8 Unicode
% !TEX TS-program = xelatex

\documentclass[french]{report}

%\usepackage[utf8]{inputenc}
%\usepackage[T1]{fontenc}
\usepackage{babel}


\newif\ifcomment
%\commenttrue # Show comments

\usepackage{physics}
\usepackage{amssymb}


\usepackage{amsthm}
% \usepackage{thmtools}
\usepackage{mathtools}
\usepackage{amsfonts}

\usepackage{color}

\usepackage{tikz}

\usepackage{geometry}
\geometry{a5paper, margin=0.1in, right=1cm}

\usepackage{dsfont}

\usepackage{graphicx}
\graphicspath{ {images/} }

\usepackage{faktor}

\usepackage{IEEEtrantools}
\usepackage{enumerate}   
\usepackage[PostScript=dvips]{"/Users/aware/Documents/Courses/diagrams"}


\newtheorem{theorem}{Théorème}[section]
\renewcommand{\thetheorem}{\arabic{theorem}}
\newtheorem{lemme}{Lemme}[section]
\renewcommand{\thelemme}{\arabic{lemme}}
\newtheorem{proposition}{Proposition}[section]
\renewcommand{\theproposition}{\arabic{proposition}}
\newtheorem{notations}{Notations}[section]
\newtheorem{problem}{Problème}[section]
\newtheorem{corollary}{Corollaire}[theorem]
\renewcommand{\thecorollary}{\arabic{corollary}}
\newtheorem{property}{Propriété}[section]
\newtheorem{objective}{Objectif}[section]

\theoremstyle{definition}
\newtheorem{definition}{Définition}[section]
\renewcommand{\thedefinition}{\arabic{definition}}
\newtheorem{exercise}{Exercice}[chapter]
\renewcommand{\theexercise}{\arabic{exercise}}
\newtheorem{example}{Exemple}[chapter]
\renewcommand{\theexample}{\arabic{example}}
\newtheorem*{solution}{Solution}
\newtheorem*{application}{Application}
\newtheorem*{notation}{Notation}
\newtheorem*{vocabulary}{Vocabulaire}
\newtheorem*{properties}{Propriétés}



\theoremstyle{remark}
\newtheorem*{remark}{Remarque}
\newtheorem*{rappel}{Rappel}


\usepackage{etoolbox}
\AtBeginEnvironment{exercise}{\small}
\AtBeginEnvironment{example}{\small}

\usepackage{cases}
\usepackage[red]{mypack}

\usepackage[framemethod=TikZ]{mdframed}

\definecolor{bg}{rgb}{0.4,0.25,0.95}
\definecolor{pagebg}{rgb}{0,0,0.5}
\surroundwithmdframed[
   topline=false,
   rightline=false,
   bottomline=false,
   leftmargin=\parindent,
   skipabove=8pt,
   skipbelow=8pt,
   linecolor=blue,
   innerbottommargin=10pt,
   % backgroundcolor=bg,font=\color{orange}\sffamily, fontcolor=white
]{definition}

\usepackage{empheq}
\usepackage[most]{tcolorbox}

\newtcbox{\mymath}[1][]{%
    nobeforeafter, math upper, tcbox raise base,
    enhanced, colframe=blue!30!black,
    colback=red!10, boxrule=1pt,
    #1}

\usepackage{unixode}


\DeclareMathOperator{\ord}{ord}
\DeclareMathOperator{\orb}{orb}
\DeclareMathOperator{\stab}{stab}
\DeclareMathOperator{\Stab}{stab}
\DeclareMathOperator{\ppcm}{ppcm}
\DeclareMathOperator{\conj}{Conj}
\DeclareMathOperator{\End}{End}
\DeclareMathOperator{\rot}{rot}
\DeclareMathOperator{\trs}{trace}
\DeclareMathOperator{\Ind}{Ind}
\DeclareMathOperator{\mat}{Mat}
\DeclareMathOperator{\id}{Id}
\DeclareMathOperator{\vect}{vect}
\DeclareMathOperator{\img}{img}
\DeclareMathOperator{\cov}{Cov}
\DeclareMathOperator{\dist}{dist}
\DeclareMathOperator{\irr}{Irr}
\DeclareMathOperator{\image}{Im}
\DeclareMathOperator{\pd}{\partial}
\DeclareMathOperator{\epi}{epi}
\DeclareMathOperator{\Argmin}{Argmin}
\DeclareMathOperator{\dom}{dom}
\DeclareMathOperator{\proj}{proj}
\DeclareMathOperator{\ctg}{ctg}
\DeclareMathOperator{\supp}{supp}
\DeclareMathOperator{\argmin}{argmin}
\DeclareMathOperator{\mult}{mult}
\DeclareMathOperator{\ch}{ch}
\DeclareMathOperator{\sh}{sh}
\DeclareMathOperator{\rang}{rang}
\DeclareMathOperator{\diam}{diam}
\DeclareMathOperator{\Epigraphe}{Epigraphe}




\usepackage{xcolor}
\everymath{\color{blue}}
%\everymath{\color[rgb]{0,1,1}}
%\pagecolor[rgb]{0,0,0.5}


\newcommand*{\pdtest}[3][]{\ensuremath{\frac{\partial^{#1} #2}{\partial #3}}}

\newcommand*{\deffunc}[6][]{\ensuremath{
\begin{array}{rcl}
#2 : #3 &\rightarrow& #4\\
#5 &\mapsto& #6
\end{array}
}}

\newcommand{\eqcolon}{\mathrel{\resizebox{\widthof{$\mathord{=}$}}{\height}{ $\!\!=\!\!\resizebox{1.2\width}{0.8\height}{\raisebox{0.23ex}{$\mathop{:}$}}\!\!$ }}}
\newcommand{\coloneq}{\mathrel{\resizebox{\widthof{$\mathord{=}$}}{\height}{ $\!\!\resizebox{1.2\width}{0.8\height}{\raisebox{0.23ex}{$\mathop{:}$}}\!\!=\!\!$ }}}
\newcommand{\eqcolonl}{\ensuremath{\mathrel{=\!\!\mathop{:}}}}
\newcommand{\coloneql}{\ensuremath{\mathrel{\mathop{:} \!\! =}}}
\newcommand{\vc}[1]{% inline column vector
  \left(\begin{smallmatrix}#1\end{smallmatrix}\right)%
}
\newcommand{\vr}[1]{% inline row vector
  \begin{smallmatrix}(\,#1\,)\end{smallmatrix}%
}
\makeatletter
\newcommand*{\defeq}{\ =\mathrel{\rlap{%
                     \raisebox{0.3ex}{$\m@th\cdot$}}%
                     \raisebox{-0.3ex}{$\m@th\cdot$}}%
                     }
\makeatother

\newcommand{\mathcircle}[1]{% inline row vector
 \overset{\circ}{#1}
}
\newcommand{\ulim}{% low limit
 \underline{\lim}
}
\newcommand{\ssi}{% iff
\iff
}
\newcommand{\ps}[2]{
\expval{#1 | #2}
}
\newcommand{\df}[1]{
\mqty{#1}
}
\newcommand{\n}[1]{
\norm{#1}
}
\newcommand{\sys}[1]{
\left\{\smqty{#1}\right.
}


\newcommand{\eqdef}{\ensuremath{\overset{\text{def}}=}}


\def\Circlearrowright{\ensuremath{%
  \rotatebox[origin=c]{230}{$\circlearrowright$}}}

\newcommand\ct[1]{\text{\rmfamily\upshape #1}}
\newcommand\question[1]{ {\color{red} ...!? \small #1}}
\newcommand\caz[1]{\left\{\begin{array} #1 \end{array}\right.}
\newcommand\const{\text{\rmfamily\upshape const}}
\newcommand\toP{ \overset{\pro}{\to}}
\newcommand\toPP{ \overset{\text{PP}}{\to}}
\newcommand{\oeq}{\mathrel{\text{\textcircled{$=$}}}}





\usepackage{xcolor}
% \usepackage[normalem]{ulem}
\usepackage{lipsum}
\makeatletter
% \newcommand\colorwave[1][blue]{\bgroup \markoverwith{\lower3.5\p@\hbox{\sixly \textcolor{#1}{\char58}}}\ULon}
%\font\sixly=lasy6 % does not re-load if already loaded, so no memory problem.

\newmdtheoremenv[
linewidth= 1pt,linecolor= blue,%
leftmargin=20,rightmargin=20,innertopmargin=0pt, innerrightmargin=40,%
tikzsetting = { draw=lightgray, line width = 0.3pt,dashed,%
dash pattern = on 15pt off 3pt},%
splittopskip=\topskip,skipbelow=\baselineskip,%
skipabove=\baselineskip,ntheorem,roundcorner=0pt,
% backgroundcolor=pagebg,font=\color{orange}\sffamily, fontcolor=white
]{examplebox}{Exemple}[section]



\newcommand\R{\mathbb{R}}
\newcommand\Z{\mathbb{Z}}
\newcommand\N{\mathbb{N}}
\newcommand\E{\mathbb{E}}
\newcommand\F{\mathcal{F}}
\newcommand\cH{\mathcal{H}}
\newcommand\V{\mathbb{V}}
\newcommand\dmo{ ^{-1} }
\newcommand\kapa{\kappa}
\newcommand\im{Im}
\newcommand\hs{\mathcal{H}}





\usepackage{soul}

\makeatletter
\newcommand*{\whiten}[1]{\llap{\textcolor{white}{{\the\SOUL@token}}\hspace{#1pt}}}
\DeclareRobustCommand*\myul{%
    \def\SOUL@everyspace{\underline{\space}\kern\z@}%
    \def\SOUL@everytoken{%
     \setbox0=\hbox{\the\SOUL@token}%
     \ifdim\dp0>\z@
        \raisebox{\dp0}{\underline{\phantom{\the\SOUL@token}}}%
        \whiten{1}\whiten{0}%
        \whiten{-1}\whiten{-2}%
        \llap{\the\SOUL@token}%
     \else
        \underline{\the\SOUL@token}%
     \fi}%
\SOUL@}
\makeatother

\newcommand*{\demp}{\fontfamily{lmtt}\selectfont}

\DeclareTextFontCommand{\textdemp}{\demp}

\begin{document}

\ifcomment
Multiline
comment
\fi
\ifcomment
\myul{Typesetting test}
% \color[rgb]{1,1,1}
$∑_i^n≠ 60º±∞π∆¬≈√j∫h≤≥µ$

$\CR \R\pro\ind\pro\gS\pro
\mqty[a&b\\c&d]$
$\pro\mathbb{P}$
$\dd{x}$

  \[
    \alpha(x)=\left\{
                \begin{array}{ll}
                  x\\
                  \frac{1}{1+e^{-kx}}\\
                  \frac{e^x-e^{-x}}{e^x+e^{-x}}
                \end{array}
              \right.
  \]

  $\expval{x}$
  
  $\chi_\rho(ghg\dmo)=\Tr(\rho_{ghg\dmo})=\Tr(\rho_g\circ\rho_h\circ\rho\dmo_g)=\Tr(\rho_h)\overset{\mbox{\scalebox{0.5}{$\Tr(AB)=\Tr(BA)$}}}{=}\chi_\rho(h)$
  	$\mathop{\oplus}_{\substack{x\in X}}$

$\mat(\rho_g)=(a_{ij}(g))_{\scriptsize \substack{1\leq i\leq d \\ 1\leq j\leq d}}$ et $\mat(\rho'_g)=(a'_{ij}(g))_{\scriptsize \substack{1\leq i'\leq d' \\ 1\leq j'\leq d'}}$



\[\int_a^b{\mathbb{R}^2}g(u, v)\dd{P_{XY}}(u, v)=\iint g(u,v) f_{XY}(u, v)\dd \lambda(u) \dd \lambda(v)\]
$$\lim_{x\to\infty} f(x)$$	
$$\iiiint_V \mu(t,u,v,w) \,dt\,du\,dv\,dw$$
$$\sum_{n=1}^{\infty} 2^{-n} = 1$$	
\begin{definition}
	Si $X$ et $Y$ sont 2 v.a. ou definit la \textsc{Covariance} entre $X$ et $Y$ comme
	$\cov(X,Y)\overset{\text{def}}{=}\E\left[(X-\E(X))(Y-\E(Y))\right]=\E(XY)-\E(X)\E(Y)$.
\end{definition}
\fi
\pagebreak

% \tableofcontents

% insert your code here
%\input{./algebra/main.tex}
%\input{./geometrie-differentielle/main.tex}
%\input{./probabilite/main.tex}
%\input{./analyse-fonctionnelle/main.tex}
% \input{./Analyse-convexe-et-dualite-en-optimisation/main.tex}
%\input{./tikz/main.tex}
%\input{./Theorie-du-distributions/main.tex}
%\input{./optimisation/mine.tex}
 \input{./modelisation/main.tex}

% yves.aubry@univ-tln.fr : algebra

\end{document}

%% !TEX encoding = UTF-8 Unicode
% !TEX TS-program = xelatex

\documentclass[french]{report}

%\usepackage[utf8]{inputenc}
%\usepackage[T1]{fontenc}
\usepackage{babel}


\newif\ifcomment
%\commenttrue # Show comments

\usepackage{physics}
\usepackage{amssymb}


\usepackage{amsthm}
% \usepackage{thmtools}
\usepackage{mathtools}
\usepackage{amsfonts}

\usepackage{color}

\usepackage{tikz}

\usepackage{geometry}
\geometry{a5paper, margin=0.1in, right=1cm}

\usepackage{dsfont}

\usepackage{graphicx}
\graphicspath{ {images/} }

\usepackage{faktor}

\usepackage{IEEEtrantools}
\usepackage{enumerate}   
\usepackage[PostScript=dvips]{"/Users/aware/Documents/Courses/diagrams"}


\newtheorem{theorem}{Théorème}[section]
\renewcommand{\thetheorem}{\arabic{theorem}}
\newtheorem{lemme}{Lemme}[section]
\renewcommand{\thelemme}{\arabic{lemme}}
\newtheorem{proposition}{Proposition}[section]
\renewcommand{\theproposition}{\arabic{proposition}}
\newtheorem{notations}{Notations}[section]
\newtheorem{problem}{Problème}[section]
\newtheorem{corollary}{Corollaire}[theorem]
\renewcommand{\thecorollary}{\arabic{corollary}}
\newtheorem{property}{Propriété}[section]
\newtheorem{objective}{Objectif}[section]

\theoremstyle{definition}
\newtheorem{definition}{Définition}[section]
\renewcommand{\thedefinition}{\arabic{definition}}
\newtheorem{exercise}{Exercice}[chapter]
\renewcommand{\theexercise}{\arabic{exercise}}
\newtheorem{example}{Exemple}[chapter]
\renewcommand{\theexample}{\arabic{example}}
\newtheorem*{solution}{Solution}
\newtheorem*{application}{Application}
\newtheorem*{notation}{Notation}
\newtheorem*{vocabulary}{Vocabulaire}
\newtheorem*{properties}{Propriétés}



\theoremstyle{remark}
\newtheorem*{remark}{Remarque}
\newtheorem*{rappel}{Rappel}


\usepackage{etoolbox}
\AtBeginEnvironment{exercise}{\small}
\AtBeginEnvironment{example}{\small}

\usepackage{cases}
\usepackage[red]{mypack}

\usepackage[framemethod=TikZ]{mdframed}

\definecolor{bg}{rgb}{0.4,0.25,0.95}
\definecolor{pagebg}{rgb}{0,0,0.5}
\surroundwithmdframed[
   topline=false,
   rightline=false,
   bottomline=false,
   leftmargin=\parindent,
   skipabove=8pt,
   skipbelow=8pt,
   linecolor=blue,
   innerbottommargin=10pt,
   % backgroundcolor=bg,font=\color{orange}\sffamily, fontcolor=white
]{definition}

\usepackage{empheq}
\usepackage[most]{tcolorbox}

\newtcbox{\mymath}[1][]{%
    nobeforeafter, math upper, tcbox raise base,
    enhanced, colframe=blue!30!black,
    colback=red!10, boxrule=1pt,
    #1}

\usepackage{unixode}


\DeclareMathOperator{\ord}{ord}
\DeclareMathOperator{\orb}{orb}
\DeclareMathOperator{\stab}{stab}
\DeclareMathOperator{\Stab}{stab}
\DeclareMathOperator{\ppcm}{ppcm}
\DeclareMathOperator{\conj}{Conj}
\DeclareMathOperator{\End}{End}
\DeclareMathOperator{\rot}{rot}
\DeclareMathOperator{\trs}{trace}
\DeclareMathOperator{\Ind}{Ind}
\DeclareMathOperator{\mat}{Mat}
\DeclareMathOperator{\id}{Id}
\DeclareMathOperator{\vect}{vect}
\DeclareMathOperator{\img}{img}
\DeclareMathOperator{\cov}{Cov}
\DeclareMathOperator{\dist}{dist}
\DeclareMathOperator{\irr}{Irr}
\DeclareMathOperator{\image}{Im}
\DeclareMathOperator{\pd}{\partial}
\DeclareMathOperator{\epi}{epi}
\DeclareMathOperator{\Argmin}{Argmin}
\DeclareMathOperator{\dom}{dom}
\DeclareMathOperator{\proj}{proj}
\DeclareMathOperator{\ctg}{ctg}
\DeclareMathOperator{\supp}{supp}
\DeclareMathOperator{\argmin}{argmin}
\DeclareMathOperator{\mult}{mult}
\DeclareMathOperator{\ch}{ch}
\DeclareMathOperator{\sh}{sh}
\DeclareMathOperator{\rang}{rang}
\DeclareMathOperator{\diam}{diam}
\DeclareMathOperator{\Epigraphe}{Epigraphe}




\usepackage{xcolor}
\everymath{\color{blue}}
%\everymath{\color[rgb]{0,1,1}}
%\pagecolor[rgb]{0,0,0.5}


\newcommand*{\pdtest}[3][]{\ensuremath{\frac{\partial^{#1} #2}{\partial #3}}}

\newcommand*{\deffunc}[6][]{\ensuremath{
\begin{array}{rcl}
#2 : #3 &\rightarrow& #4\\
#5 &\mapsto& #6
\end{array}
}}

\newcommand{\eqcolon}{\mathrel{\resizebox{\widthof{$\mathord{=}$}}{\height}{ $\!\!=\!\!\resizebox{1.2\width}{0.8\height}{\raisebox{0.23ex}{$\mathop{:}$}}\!\!$ }}}
\newcommand{\coloneq}{\mathrel{\resizebox{\widthof{$\mathord{=}$}}{\height}{ $\!\!\resizebox{1.2\width}{0.8\height}{\raisebox{0.23ex}{$\mathop{:}$}}\!\!=\!\!$ }}}
\newcommand{\eqcolonl}{\ensuremath{\mathrel{=\!\!\mathop{:}}}}
\newcommand{\coloneql}{\ensuremath{\mathrel{\mathop{:} \!\! =}}}
\newcommand{\vc}[1]{% inline column vector
  \left(\begin{smallmatrix}#1\end{smallmatrix}\right)%
}
\newcommand{\vr}[1]{% inline row vector
  \begin{smallmatrix}(\,#1\,)\end{smallmatrix}%
}
\makeatletter
\newcommand*{\defeq}{\ =\mathrel{\rlap{%
                     \raisebox{0.3ex}{$\m@th\cdot$}}%
                     \raisebox{-0.3ex}{$\m@th\cdot$}}%
                     }
\makeatother

\newcommand{\mathcircle}[1]{% inline row vector
 \overset{\circ}{#1}
}
\newcommand{\ulim}{% low limit
 \underline{\lim}
}
\newcommand{\ssi}{% iff
\iff
}
\newcommand{\ps}[2]{
\expval{#1 | #2}
}
\newcommand{\df}[1]{
\mqty{#1}
}
\newcommand{\n}[1]{
\norm{#1}
}
\newcommand{\sys}[1]{
\left\{\smqty{#1}\right.
}


\newcommand{\eqdef}{\ensuremath{\overset{\text{def}}=}}


\def\Circlearrowright{\ensuremath{%
  \rotatebox[origin=c]{230}{$\circlearrowright$}}}

\newcommand\ct[1]{\text{\rmfamily\upshape #1}}
\newcommand\question[1]{ {\color{red} ...!? \small #1}}
\newcommand\caz[1]{\left\{\begin{array} #1 \end{array}\right.}
\newcommand\const{\text{\rmfamily\upshape const}}
\newcommand\toP{ \overset{\pro}{\to}}
\newcommand\toPP{ \overset{\text{PP}}{\to}}
\newcommand{\oeq}{\mathrel{\text{\textcircled{$=$}}}}





\usepackage{xcolor}
% \usepackage[normalem]{ulem}
\usepackage{lipsum}
\makeatletter
% \newcommand\colorwave[1][blue]{\bgroup \markoverwith{\lower3.5\p@\hbox{\sixly \textcolor{#1}{\char58}}}\ULon}
%\font\sixly=lasy6 % does not re-load if already loaded, so no memory problem.

\newmdtheoremenv[
linewidth= 1pt,linecolor= blue,%
leftmargin=20,rightmargin=20,innertopmargin=0pt, innerrightmargin=40,%
tikzsetting = { draw=lightgray, line width = 0.3pt,dashed,%
dash pattern = on 15pt off 3pt},%
splittopskip=\topskip,skipbelow=\baselineskip,%
skipabove=\baselineskip,ntheorem,roundcorner=0pt,
% backgroundcolor=pagebg,font=\color{orange}\sffamily, fontcolor=white
]{examplebox}{Exemple}[section]



\newcommand\R{\mathbb{R}}
\newcommand\Z{\mathbb{Z}}
\newcommand\N{\mathbb{N}}
\newcommand\E{\mathbb{E}}
\newcommand\F{\mathcal{F}}
\newcommand\cH{\mathcal{H}}
\newcommand\V{\mathbb{V}}
\newcommand\dmo{ ^{-1} }
\newcommand\kapa{\kappa}
\newcommand\im{Im}
\newcommand\hs{\mathcal{H}}





\usepackage{soul}

\makeatletter
\newcommand*{\whiten}[1]{\llap{\textcolor{white}{{\the\SOUL@token}}\hspace{#1pt}}}
\DeclareRobustCommand*\myul{%
    \def\SOUL@everyspace{\underline{\space}\kern\z@}%
    \def\SOUL@everytoken{%
     \setbox0=\hbox{\the\SOUL@token}%
     \ifdim\dp0>\z@
        \raisebox{\dp0}{\underline{\phantom{\the\SOUL@token}}}%
        \whiten{1}\whiten{0}%
        \whiten{-1}\whiten{-2}%
        \llap{\the\SOUL@token}%
     \else
        \underline{\the\SOUL@token}%
     \fi}%
\SOUL@}
\makeatother

\newcommand*{\demp}{\fontfamily{lmtt}\selectfont}

\DeclareTextFontCommand{\textdemp}{\demp}

\begin{document}

\ifcomment
Multiline
comment
\fi
\ifcomment
\myul{Typesetting test}
% \color[rgb]{1,1,1}
$∑_i^n≠ 60º±∞π∆¬≈√j∫h≤≥µ$

$\CR \R\pro\ind\pro\gS\pro
\mqty[a&b\\c&d]$
$\pro\mathbb{P}$
$\dd{x}$

  \[
    \alpha(x)=\left\{
                \begin{array}{ll}
                  x\\
                  \frac{1}{1+e^{-kx}}\\
                  \frac{e^x-e^{-x}}{e^x+e^{-x}}
                \end{array}
              \right.
  \]

  $\expval{x}$
  
  $\chi_\rho(ghg\dmo)=\Tr(\rho_{ghg\dmo})=\Tr(\rho_g\circ\rho_h\circ\rho\dmo_g)=\Tr(\rho_h)\overset{\mbox{\scalebox{0.5}{$\Tr(AB)=\Tr(BA)$}}}{=}\chi_\rho(h)$
  	$\mathop{\oplus}_{\substack{x\in X}}$

$\mat(\rho_g)=(a_{ij}(g))_{\scriptsize \substack{1\leq i\leq d \\ 1\leq j\leq d}}$ et $\mat(\rho'_g)=(a'_{ij}(g))_{\scriptsize \substack{1\leq i'\leq d' \\ 1\leq j'\leq d'}}$



\[\int_a^b{\mathbb{R}^2}g(u, v)\dd{P_{XY}}(u, v)=\iint g(u,v) f_{XY}(u, v)\dd \lambda(u) \dd \lambda(v)\]
$$\lim_{x\to\infty} f(x)$$	
$$\iiiint_V \mu(t,u,v,w) \,dt\,du\,dv\,dw$$
$$\sum_{n=1}^{\infty} 2^{-n} = 1$$	
\begin{definition}
	Si $X$ et $Y$ sont 2 v.a. ou definit la \textsc{Covariance} entre $X$ et $Y$ comme
	$\cov(X,Y)\overset{\text{def}}{=}\E\left[(X-\E(X))(Y-\E(Y))\right]=\E(XY)-\E(X)\E(Y)$.
\end{definition}
\fi
\pagebreak

% \tableofcontents

% insert your code here
%\input{./algebra/main.tex}
%\input{./geometrie-differentielle/main.tex}
%\input{./probabilite/main.tex}
%\input{./analyse-fonctionnelle/main.tex}
% \input{./Analyse-convexe-et-dualite-en-optimisation/main.tex}
%\input{./tikz/main.tex}
%\input{./Theorie-du-distributions/main.tex}
%\input{./optimisation/mine.tex}
 \input{./modelisation/main.tex}

% yves.aubry@univ-tln.fr : algebra

\end{document}

% % !TEX encoding = UTF-8 Unicode
% !TEX TS-program = xelatex

\documentclass[french]{report}

%\usepackage[utf8]{inputenc}
%\usepackage[T1]{fontenc}
\usepackage{babel}


\newif\ifcomment
%\commenttrue # Show comments

\usepackage{physics}
\usepackage{amssymb}


\usepackage{amsthm}
% \usepackage{thmtools}
\usepackage{mathtools}
\usepackage{amsfonts}

\usepackage{color}

\usepackage{tikz}

\usepackage{geometry}
\geometry{a5paper, margin=0.1in, right=1cm}

\usepackage{dsfont}

\usepackage{graphicx}
\graphicspath{ {images/} }

\usepackage{faktor}

\usepackage{IEEEtrantools}
\usepackage{enumerate}   
\usepackage[PostScript=dvips]{"/Users/aware/Documents/Courses/diagrams"}


\newtheorem{theorem}{Théorème}[section]
\renewcommand{\thetheorem}{\arabic{theorem}}
\newtheorem{lemme}{Lemme}[section]
\renewcommand{\thelemme}{\arabic{lemme}}
\newtheorem{proposition}{Proposition}[section]
\renewcommand{\theproposition}{\arabic{proposition}}
\newtheorem{notations}{Notations}[section]
\newtheorem{problem}{Problème}[section]
\newtheorem{corollary}{Corollaire}[theorem]
\renewcommand{\thecorollary}{\arabic{corollary}}
\newtheorem{property}{Propriété}[section]
\newtheorem{objective}{Objectif}[section]

\theoremstyle{definition}
\newtheorem{definition}{Définition}[section]
\renewcommand{\thedefinition}{\arabic{definition}}
\newtheorem{exercise}{Exercice}[chapter]
\renewcommand{\theexercise}{\arabic{exercise}}
\newtheorem{example}{Exemple}[chapter]
\renewcommand{\theexample}{\arabic{example}}
\newtheorem*{solution}{Solution}
\newtheorem*{application}{Application}
\newtheorem*{notation}{Notation}
\newtheorem*{vocabulary}{Vocabulaire}
\newtheorem*{properties}{Propriétés}



\theoremstyle{remark}
\newtheorem*{remark}{Remarque}
\newtheorem*{rappel}{Rappel}


\usepackage{etoolbox}
\AtBeginEnvironment{exercise}{\small}
\AtBeginEnvironment{example}{\small}

\usepackage{cases}
\usepackage[red]{mypack}

\usepackage[framemethod=TikZ]{mdframed}

\definecolor{bg}{rgb}{0.4,0.25,0.95}
\definecolor{pagebg}{rgb}{0,0,0.5}
\surroundwithmdframed[
   topline=false,
   rightline=false,
   bottomline=false,
   leftmargin=\parindent,
   skipabove=8pt,
   skipbelow=8pt,
   linecolor=blue,
   innerbottommargin=10pt,
   % backgroundcolor=bg,font=\color{orange}\sffamily, fontcolor=white
]{definition}

\usepackage{empheq}
\usepackage[most]{tcolorbox}

\newtcbox{\mymath}[1][]{%
    nobeforeafter, math upper, tcbox raise base,
    enhanced, colframe=blue!30!black,
    colback=red!10, boxrule=1pt,
    #1}

\usepackage{unixode}


\DeclareMathOperator{\ord}{ord}
\DeclareMathOperator{\orb}{orb}
\DeclareMathOperator{\stab}{stab}
\DeclareMathOperator{\Stab}{stab}
\DeclareMathOperator{\ppcm}{ppcm}
\DeclareMathOperator{\conj}{Conj}
\DeclareMathOperator{\End}{End}
\DeclareMathOperator{\rot}{rot}
\DeclareMathOperator{\trs}{trace}
\DeclareMathOperator{\Ind}{Ind}
\DeclareMathOperator{\mat}{Mat}
\DeclareMathOperator{\id}{Id}
\DeclareMathOperator{\vect}{vect}
\DeclareMathOperator{\img}{img}
\DeclareMathOperator{\cov}{Cov}
\DeclareMathOperator{\dist}{dist}
\DeclareMathOperator{\irr}{Irr}
\DeclareMathOperator{\image}{Im}
\DeclareMathOperator{\pd}{\partial}
\DeclareMathOperator{\epi}{epi}
\DeclareMathOperator{\Argmin}{Argmin}
\DeclareMathOperator{\dom}{dom}
\DeclareMathOperator{\proj}{proj}
\DeclareMathOperator{\ctg}{ctg}
\DeclareMathOperator{\supp}{supp}
\DeclareMathOperator{\argmin}{argmin}
\DeclareMathOperator{\mult}{mult}
\DeclareMathOperator{\ch}{ch}
\DeclareMathOperator{\sh}{sh}
\DeclareMathOperator{\rang}{rang}
\DeclareMathOperator{\diam}{diam}
\DeclareMathOperator{\Epigraphe}{Epigraphe}




\usepackage{xcolor}
\everymath{\color{blue}}
%\everymath{\color[rgb]{0,1,1}}
%\pagecolor[rgb]{0,0,0.5}


\newcommand*{\pdtest}[3][]{\ensuremath{\frac{\partial^{#1} #2}{\partial #3}}}

\newcommand*{\deffunc}[6][]{\ensuremath{
\begin{array}{rcl}
#2 : #3 &\rightarrow& #4\\
#5 &\mapsto& #6
\end{array}
}}

\newcommand{\eqcolon}{\mathrel{\resizebox{\widthof{$\mathord{=}$}}{\height}{ $\!\!=\!\!\resizebox{1.2\width}{0.8\height}{\raisebox{0.23ex}{$\mathop{:}$}}\!\!$ }}}
\newcommand{\coloneq}{\mathrel{\resizebox{\widthof{$\mathord{=}$}}{\height}{ $\!\!\resizebox{1.2\width}{0.8\height}{\raisebox{0.23ex}{$\mathop{:}$}}\!\!=\!\!$ }}}
\newcommand{\eqcolonl}{\ensuremath{\mathrel{=\!\!\mathop{:}}}}
\newcommand{\coloneql}{\ensuremath{\mathrel{\mathop{:} \!\! =}}}
\newcommand{\vc}[1]{% inline column vector
  \left(\begin{smallmatrix}#1\end{smallmatrix}\right)%
}
\newcommand{\vr}[1]{% inline row vector
  \begin{smallmatrix}(\,#1\,)\end{smallmatrix}%
}
\makeatletter
\newcommand*{\defeq}{\ =\mathrel{\rlap{%
                     \raisebox{0.3ex}{$\m@th\cdot$}}%
                     \raisebox{-0.3ex}{$\m@th\cdot$}}%
                     }
\makeatother

\newcommand{\mathcircle}[1]{% inline row vector
 \overset{\circ}{#1}
}
\newcommand{\ulim}{% low limit
 \underline{\lim}
}
\newcommand{\ssi}{% iff
\iff
}
\newcommand{\ps}[2]{
\expval{#1 | #2}
}
\newcommand{\df}[1]{
\mqty{#1}
}
\newcommand{\n}[1]{
\norm{#1}
}
\newcommand{\sys}[1]{
\left\{\smqty{#1}\right.
}


\newcommand{\eqdef}{\ensuremath{\overset{\text{def}}=}}


\def\Circlearrowright{\ensuremath{%
  \rotatebox[origin=c]{230}{$\circlearrowright$}}}

\newcommand\ct[1]{\text{\rmfamily\upshape #1}}
\newcommand\question[1]{ {\color{red} ...!? \small #1}}
\newcommand\caz[1]{\left\{\begin{array} #1 \end{array}\right.}
\newcommand\const{\text{\rmfamily\upshape const}}
\newcommand\toP{ \overset{\pro}{\to}}
\newcommand\toPP{ \overset{\text{PP}}{\to}}
\newcommand{\oeq}{\mathrel{\text{\textcircled{$=$}}}}





\usepackage{xcolor}
% \usepackage[normalem]{ulem}
\usepackage{lipsum}
\makeatletter
% \newcommand\colorwave[1][blue]{\bgroup \markoverwith{\lower3.5\p@\hbox{\sixly \textcolor{#1}{\char58}}}\ULon}
%\font\sixly=lasy6 % does not re-load if already loaded, so no memory problem.

\newmdtheoremenv[
linewidth= 1pt,linecolor= blue,%
leftmargin=20,rightmargin=20,innertopmargin=0pt, innerrightmargin=40,%
tikzsetting = { draw=lightgray, line width = 0.3pt,dashed,%
dash pattern = on 15pt off 3pt},%
splittopskip=\topskip,skipbelow=\baselineskip,%
skipabove=\baselineskip,ntheorem,roundcorner=0pt,
% backgroundcolor=pagebg,font=\color{orange}\sffamily, fontcolor=white
]{examplebox}{Exemple}[section]



\newcommand\R{\mathbb{R}}
\newcommand\Z{\mathbb{Z}}
\newcommand\N{\mathbb{N}}
\newcommand\E{\mathbb{E}}
\newcommand\F{\mathcal{F}}
\newcommand\cH{\mathcal{H}}
\newcommand\V{\mathbb{V}}
\newcommand\dmo{ ^{-1} }
\newcommand\kapa{\kappa}
\newcommand\im{Im}
\newcommand\hs{\mathcal{H}}





\usepackage{soul}

\makeatletter
\newcommand*{\whiten}[1]{\llap{\textcolor{white}{{\the\SOUL@token}}\hspace{#1pt}}}
\DeclareRobustCommand*\myul{%
    \def\SOUL@everyspace{\underline{\space}\kern\z@}%
    \def\SOUL@everytoken{%
     \setbox0=\hbox{\the\SOUL@token}%
     \ifdim\dp0>\z@
        \raisebox{\dp0}{\underline{\phantom{\the\SOUL@token}}}%
        \whiten{1}\whiten{0}%
        \whiten{-1}\whiten{-2}%
        \llap{\the\SOUL@token}%
     \else
        \underline{\the\SOUL@token}%
     \fi}%
\SOUL@}
\makeatother

\newcommand*{\demp}{\fontfamily{lmtt}\selectfont}

\DeclareTextFontCommand{\textdemp}{\demp}

\begin{document}

\ifcomment
Multiline
comment
\fi
\ifcomment
\myul{Typesetting test}
% \color[rgb]{1,1,1}
$∑_i^n≠ 60º±∞π∆¬≈√j∫h≤≥µ$

$\CR \R\pro\ind\pro\gS\pro
\mqty[a&b\\c&d]$
$\pro\mathbb{P}$
$\dd{x}$

  \[
    \alpha(x)=\left\{
                \begin{array}{ll}
                  x\\
                  \frac{1}{1+e^{-kx}}\\
                  \frac{e^x-e^{-x}}{e^x+e^{-x}}
                \end{array}
              \right.
  \]

  $\expval{x}$
  
  $\chi_\rho(ghg\dmo)=\Tr(\rho_{ghg\dmo})=\Tr(\rho_g\circ\rho_h\circ\rho\dmo_g)=\Tr(\rho_h)\overset{\mbox{\scalebox{0.5}{$\Tr(AB)=\Tr(BA)$}}}{=}\chi_\rho(h)$
  	$\mathop{\oplus}_{\substack{x\in X}}$

$\mat(\rho_g)=(a_{ij}(g))_{\scriptsize \substack{1\leq i\leq d \\ 1\leq j\leq d}}$ et $\mat(\rho'_g)=(a'_{ij}(g))_{\scriptsize \substack{1\leq i'\leq d' \\ 1\leq j'\leq d'}}$



\[\int_a^b{\mathbb{R}^2}g(u, v)\dd{P_{XY}}(u, v)=\iint g(u,v) f_{XY}(u, v)\dd \lambda(u) \dd \lambda(v)\]
$$\lim_{x\to\infty} f(x)$$	
$$\iiiint_V \mu(t,u,v,w) \,dt\,du\,dv\,dw$$
$$\sum_{n=1}^{\infty} 2^{-n} = 1$$	
\begin{definition}
	Si $X$ et $Y$ sont 2 v.a. ou definit la \textsc{Covariance} entre $X$ et $Y$ comme
	$\cov(X,Y)\overset{\text{def}}{=}\E\left[(X-\E(X))(Y-\E(Y))\right]=\E(XY)-\E(X)\E(Y)$.
\end{definition}
\fi
\pagebreak

% \tableofcontents

% insert your code here
%\input{./algebra/main.tex}
%\input{./geometrie-differentielle/main.tex}
%\input{./probabilite/main.tex}
%\input{./analyse-fonctionnelle/main.tex}
% \input{./Analyse-convexe-et-dualite-en-optimisation/main.tex}
%\input{./tikz/main.tex}
%\input{./Theorie-du-distributions/main.tex}
%\input{./optimisation/mine.tex}
 \input{./modelisation/main.tex}

% yves.aubry@univ-tln.fr : algebra

\end{document}

%% !TEX encoding = UTF-8 Unicode
% !TEX TS-program = xelatex

\documentclass[french]{report}

%\usepackage[utf8]{inputenc}
%\usepackage[T1]{fontenc}
\usepackage{babel}


\newif\ifcomment
%\commenttrue # Show comments

\usepackage{physics}
\usepackage{amssymb}


\usepackage{amsthm}
% \usepackage{thmtools}
\usepackage{mathtools}
\usepackage{amsfonts}

\usepackage{color}

\usepackage{tikz}

\usepackage{geometry}
\geometry{a5paper, margin=0.1in, right=1cm}

\usepackage{dsfont}

\usepackage{graphicx}
\graphicspath{ {images/} }

\usepackage{faktor}

\usepackage{IEEEtrantools}
\usepackage{enumerate}   
\usepackage[PostScript=dvips]{"/Users/aware/Documents/Courses/diagrams"}


\newtheorem{theorem}{Théorème}[section]
\renewcommand{\thetheorem}{\arabic{theorem}}
\newtheorem{lemme}{Lemme}[section]
\renewcommand{\thelemme}{\arabic{lemme}}
\newtheorem{proposition}{Proposition}[section]
\renewcommand{\theproposition}{\arabic{proposition}}
\newtheorem{notations}{Notations}[section]
\newtheorem{problem}{Problème}[section]
\newtheorem{corollary}{Corollaire}[theorem]
\renewcommand{\thecorollary}{\arabic{corollary}}
\newtheorem{property}{Propriété}[section]
\newtheorem{objective}{Objectif}[section]

\theoremstyle{definition}
\newtheorem{definition}{Définition}[section]
\renewcommand{\thedefinition}{\arabic{definition}}
\newtheorem{exercise}{Exercice}[chapter]
\renewcommand{\theexercise}{\arabic{exercise}}
\newtheorem{example}{Exemple}[chapter]
\renewcommand{\theexample}{\arabic{example}}
\newtheorem*{solution}{Solution}
\newtheorem*{application}{Application}
\newtheorem*{notation}{Notation}
\newtheorem*{vocabulary}{Vocabulaire}
\newtheorem*{properties}{Propriétés}



\theoremstyle{remark}
\newtheorem*{remark}{Remarque}
\newtheorem*{rappel}{Rappel}


\usepackage{etoolbox}
\AtBeginEnvironment{exercise}{\small}
\AtBeginEnvironment{example}{\small}

\usepackage{cases}
\usepackage[red]{mypack}

\usepackage[framemethod=TikZ]{mdframed}

\definecolor{bg}{rgb}{0.4,0.25,0.95}
\definecolor{pagebg}{rgb}{0,0,0.5}
\surroundwithmdframed[
   topline=false,
   rightline=false,
   bottomline=false,
   leftmargin=\parindent,
   skipabove=8pt,
   skipbelow=8pt,
   linecolor=blue,
   innerbottommargin=10pt,
   % backgroundcolor=bg,font=\color{orange}\sffamily, fontcolor=white
]{definition}

\usepackage{empheq}
\usepackage[most]{tcolorbox}

\newtcbox{\mymath}[1][]{%
    nobeforeafter, math upper, tcbox raise base,
    enhanced, colframe=blue!30!black,
    colback=red!10, boxrule=1pt,
    #1}

\usepackage{unixode}


\DeclareMathOperator{\ord}{ord}
\DeclareMathOperator{\orb}{orb}
\DeclareMathOperator{\stab}{stab}
\DeclareMathOperator{\Stab}{stab}
\DeclareMathOperator{\ppcm}{ppcm}
\DeclareMathOperator{\conj}{Conj}
\DeclareMathOperator{\End}{End}
\DeclareMathOperator{\rot}{rot}
\DeclareMathOperator{\trs}{trace}
\DeclareMathOperator{\Ind}{Ind}
\DeclareMathOperator{\mat}{Mat}
\DeclareMathOperator{\id}{Id}
\DeclareMathOperator{\vect}{vect}
\DeclareMathOperator{\img}{img}
\DeclareMathOperator{\cov}{Cov}
\DeclareMathOperator{\dist}{dist}
\DeclareMathOperator{\irr}{Irr}
\DeclareMathOperator{\image}{Im}
\DeclareMathOperator{\pd}{\partial}
\DeclareMathOperator{\epi}{epi}
\DeclareMathOperator{\Argmin}{Argmin}
\DeclareMathOperator{\dom}{dom}
\DeclareMathOperator{\proj}{proj}
\DeclareMathOperator{\ctg}{ctg}
\DeclareMathOperator{\supp}{supp}
\DeclareMathOperator{\argmin}{argmin}
\DeclareMathOperator{\mult}{mult}
\DeclareMathOperator{\ch}{ch}
\DeclareMathOperator{\sh}{sh}
\DeclareMathOperator{\rang}{rang}
\DeclareMathOperator{\diam}{diam}
\DeclareMathOperator{\Epigraphe}{Epigraphe}




\usepackage{xcolor}
\everymath{\color{blue}}
%\everymath{\color[rgb]{0,1,1}}
%\pagecolor[rgb]{0,0,0.5}


\newcommand*{\pdtest}[3][]{\ensuremath{\frac{\partial^{#1} #2}{\partial #3}}}

\newcommand*{\deffunc}[6][]{\ensuremath{
\begin{array}{rcl}
#2 : #3 &\rightarrow& #4\\
#5 &\mapsto& #6
\end{array}
}}

\newcommand{\eqcolon}{\mathrel{\resizebox{\widthof{$\mathord{=}$}}{\height}{ $\!\!=\!\!\resizebox{1.2\width}{0.8\height}{\raisebox{0.23ex}{$\mathop{:}$}}\!\!$ }}}
\newcommand{\coloneq}{\mathrel{\resizebox{\widthof{$\mathord{=}$}}{\height}{ $\!\!\resizebox{1.2\width}{0.8\height}{\raisebox{0.23ex}{$\mathop{:}$}}\!\!=\!\!$ }}}
\newcommand{\eqcolonl}{\ensuremath{\mathrel{=\!\!\mathop{:}}}}
\newcommand{\coloneql}{\ensuremath{\mathrel{\mathop{:} \!\! =}}}
\newcommand{\vc}[1]{% inline column vector
  \left(\begin{smallmatrix}#1\end{smallmatrix}\right)%
}
\newcommand{\vr}[1]{% inline row vector
  \begin{smallmatrix}(\,#1\,)\end{smallmatrix}%
}
\makeatletter
\newcommand*{\defeq}{\ =\mathrel{\rlap{%
                     \raisebox{0.3ex}{$\m@th\cdot$}}%
                     \raisebox{-0.3ex}{$\m@th\cdot$}}%
                     }
\makeatother

\newcommand{\mathcircle}[1]{% inline row vector
 \overset{\circ}{#1}
}
\newcommand{\ulim}{% low limit
 \underline{\lim}
}
\newcommand{\ssi}{% iff
\iff
}
\newcommand{\ps}[2]{
\expval{#1 | #2}
}
\newcommand{\df}[1]{
\mqty{#1}
}
\newcommand{\n}[1]{
\norm{#1}
}
\newcommand{\sys}[1]{
\left\{\smqty{#1}\right.
}


\newcommand{\eqdef}{\ensuremath{\overset{\text{def}}=}}


\def\Circlearrowright{\ensuremath{%
  \rotatebox[origin=c]{230}{$\circlearrowright$}}}

\newcommand\ct[1]{\text{\rmfamily\upshape #1}}
\newcommand\question[1]{ {\color{red} ...!? \small #1}}
\newcommand\caz[1]{\left\{\begin{array} #1 \end{array}\right.}
\newcommand\const{\text{\rmfamily\upshape const}}
\newcommand\toP{ \overset{\pro}{\to}}
\newcommand\toPP{ \overset{\text{PP}}{\to}}
\newcommand{\oeq}{\mathrel{\text{\textcircled{$=$}}}}





\usepackage{xcolor}
% \usepackage[normalem]{ulem}
\usepackage{lipsum}
\makeatletter
% \newcommand\colorwave[1][blue]{\bgroup \markoverwith{\lower3.5\p@\hbox{\sixly \textcolor{#1}{\char58}}}\ULon}
%\font\sixly=lasy6 % does not re-load if already loaded, so no memory problem.

\newmdtheoremenv[
linewidth= 1pt,linecolor= blue,%
leftmargin=20,rightmargin=20,innertopmargin=0pt, innerrightmargin=40,%
tikzsetting = { draw=lightgray, line width = 0.3pt,dashed,%
dash pattern = on 15pt off 3pt},%
splittopskip=\topskip,skipbelow=\baselineskip,%
skipabove=\baselineskip,ntheorem,roundcorner=0pt,
% backgroundcolor=pagebg,font=\color{orange}\sffamily, fontcolor=white
]{examplebox}{Exemple}[section]



\newcommand\R{\mathbb{R}}
\newcommand\Z{\mathbb{Z}}
\newcommand\N{\mathbb{N}}
\newcommand\E{\mathbb{E}}
\newcommand\F{\mathcal{F}}
\newcommand\cH{\mathcal{H}}
\newcommand\V{\mathbb{V}}
\newcommand\dmo{ ^{-1} }
\newcommand\kapa{\kappa}
\newcommand\im{Im}
\newcommand\hs{\mathcal{H}}





\usepackage{soul}

\makeatletter
\newcommand*{\whiten}[1]{\llap{\textcolor{white}{{\the\SOUL@token}}\hspace{#1pt}}}
\DeclareRobustCommand*\myul{%
    \def\SOUL@everyspace{\underline{\space}\kern\z@}%
    \def\SOUL@everytoken{%
     \setbox0=\hbox{\the\SOUL@token}%
     \ifdim\dp0>\z@
        \raisebox{\dp0}{\underline{\phantom{\the\SOUL@token}}}%
        \whiten{1}\whiten{0}%
        \whiten{-1}\whiten{-2}%
        \llap{\the\SOUL@token}%
     \else
        \underline{\the\SOUL@token}%
     \fi}%
\SOUL@}
\makeatother

\newcommand*{\demp}{\fontfamily{lmtt}\selectfont}

\DeclareTextFontCommand{\textdemp}{\demp}

\begin{document}

\ifcomment
Multiline
comment
\fi
\ifcomment
\myul{Typesetting test}
% \color[rgb]{1,1,1}
$∑_i^n≠ 60º±∞π∆¬≈√j∫h≤≥µ$

$\CR \R\pro\ind\pro\gS\pro
\mqty[a&b\\c&d]$
$\pro\mathbb{P}$
$\dd{x}$

  \[
    \alpha(x)=\left\{
                \begin{array}{ll}
                  x\\
                  \frac{1}{1+e^{-kx}}\\
                  \frac{e^x-e^{-x}}{e^x+e^{-x}}
                \end{array}
              \right.
  \]

  $\expval{x}$
  
  $\chi_\rho(ghg\dmo)=\Tr(\rho_{ghg\dmo})=\Tr(\rho_g\circ\rho_h\circ\rho\dmo_g)=\Tr(\rho_h)\overset{\mbox{\scalebox{0.5}{$\Tr(AB)=\Tr(BA)$}}}{=}\chi_\rho(h)$
  	$\mathop{\oplus}_{\substack{x\in X}}$

$\mat(\rho_g)=(a_{ij}(g))_{\scriptsize \substack{1\leq i\leq d \\ 1\leq j\leq d}}$ et $\mat(\rho'_g)=(a'_{ij}(g))_{\scriptsize \substack{1\leq i'\leq d' \\ 1\leq j'\leq d'}}$



\[\int_a^b{\mathbb{R}^2}g(u, v)\dd{P_{XY}}(u, v)=\iint g(u,v) f_{XY}(u, v)\dd \lambda(u) \dd \lambda(v)\]
$$\lim_{x\to\infty} f(x)$$	
$$\iiiint_V \mu(t,u,v,w) \,dt\,du\,dv\,dw$$
$$\sum_{n=1}^{\infty} 2^{-n} = 1$$	
\begin{definition}
	Si $X$ et $Y$ sont 2 v.a. ou definit la \textsc{Covariance} entre $X$ et $Y$ comme
	$\cov(X,Y)\overset{\text{def}}{=}\E\left[(X-\E(X))(Y-\E(Y))\right]=\E(XY)-\E(X)\E(Y)$.
\end{definition}
\fi
\pagebreak

% \tableofcontents

% insert your code here
%\input{./algebra/main.tex}
%\input{./geometrie-differentielle/main.tex}
%\input{./probabilite/main.tex}
%\input{./analyse-fonctionnelle/main.tex}
% \input{./Analyse-convexe-et-dualite-en-optimisation/main.tex}
%\input{./tikz/main.tex}
%\input{./Theorie-du-distributions/main.tex}
%\input{./optimisation/mine.tex}
 \input{./modelisation/main.tex}

% yves.aubry@univ-tln.fr : algebra

\end{document}

%% !TEX encoding = UTF-8 Unicode
% !TEX TS-program = xelatex

\documentclass[french]{report}

%\usepackage[utf8]{inputenc}
%\usepackage[T1]{fontenc}
\usepackage{babel}


\newif\ifcomment
%\commenttrue # Show comments

\usepackage{physics}
\usepackage{amssymb}


\usepackage{amsthm}
% \usepackage{thmtools}
\usepackage{mathtools}
\usepackage{amsfonts}

\usepackage{color}

\usepackage{tikz}

\usepackage{geometry}
\geometry{a5paper, margin=0.1in, right=1cm}

\usepackage{dsfont}

\usepackage{graphicx}
\graphicspath{ {images/} }

\usepackage{faktor}

\usepackage{IEEEtrantools}
\usepackage{enumerate}   
\usepackage[PostScript=dvips]{"/Users/aware/Documents/Courses/diagrams"}


\newtheorem{theorem}{Théorème}[section]
\renewcommand{\thetheorem}{\arabic{theorem}}
\newtheorem{lemme}{Lemme}[section]
\renewcommand{\thelemme}{\arabic{lemme}}
\newtheorem{proposition}{Proposition}[section]
\renewcommand{\theproposition}{\arabic{proposition}}
\newtheorem{notations}{Notations}[section]
\newtheorem{problem}{Problème}[section]
\newtheorem{corollary}{Corollaire}[theorem]
\renewcommand{\thecorollary}{\arabic{corollary}}
\newtheorem{property}{Propriété}[section]
\newtheorem{objective}{Objectif}[section]

\theoremstyle{definition}
\newtheorem{definition}{Définition}[section]
\renewcommand{\thedefinition}{\arabic{definition}}
\newtheorem{exercise}{Exercice}[chapter]
\renewcommand{\theexercise}{\arabic{exercise}}
\newtheorem{example}{Exemple}[chapter]
\renewcommand{\theexample}{\arabic{example}}
\newtheorem*{solution}{Solution}
\newtheorem*{application}{Application}
\newtheorem*{notation}{Notation}
\newtheorem*{vocabulary}{Vocabulaire}
\newtheorem*{properties}{Propriétés}



\theoremstyle{remark}
\newtheorem*{remark}{Remarque}
\newtheorem*{rappel}{Rappel}


\usepackage{etoolbox}
\AtBeginEnvironment{exercise}{\small}
\AtBeginEnvironment{example}{\small}

\usepackage{cases}
\usepackage[red]{mypack}

\usepackage[framemethod=TikZ]{mdframed}

\definecolor{bg}{rgb}{0.4,0.25,0.95}
\definecolor{pagebg}{rgb}{0,0,0.5}
\surroundwithmdframed[
   topline=false,
   rightline=false,
   bottomline=false,
   leftmargin=\parindent,
   skipabove=8pt,
   skipbelow=8pt,
   linecolor=blue,
   innerbottommargin=10pt,
   % backgroundcolor=bg,font=\color{orange}\sffamily, fontcolor=white
]{definition}

\usepackage{empheq}
\usepackage[most]{tcolorbox}

\newtcbox{\mymath}[1][]{%
    nobeforeafter, math upper, tcbox raise base,
    enhanced, colframe=blue!30!black,
    colback=red!10, boxrule=1pt,
    #1}

\usepackage{unixode}


\DeclareMathOperator{\ord}{ord}
\DeclareMathOperator{\orb}{orb}
\DeclareMathOperator{\stab}{stab}
\DeclareMathOperator{\Stab}{stab}
\DeclareMathOperator{\ppcm}{ppcm}
\DeclareMathOperator{\conj}{Conj}
\DeclareMathOperator{\End}{End}
\DeclareMathOperator{\rot}{rot}
\DeclareMathOperator{\trs}{trace}
\DeclareMathOperator{\Ind}{Ind}
\DeclareMathOperator{\mat}{Mat}
\DeclareMathOperator{\id}{Id}
\DeclareMathOperator{\vect}{vect}
\DeclareMathOperator{\img}{img}
\DeclareMathOperator{\cov}{Cov}
\DeclareMathOperator{\dist}{dist}
\DeclareMathOperator{\irr}{Irr}
\DeclareMathOperator{\image}{Im}
\DeclareMathOperator{\pd}{\partial}
\DeclareMathOperator{\epi}{epi}
\DeclareMathOperator{\Argmin}{Argmin}
\DeclareMathOperator{\dom}{dom}
\DeclareMathOperator{\proj}{proj}
\DeclareMathOperator{\ctg}{ctg}
\DeclareMathOperator{\supp}{supp}
\DeclareMathOperator{\argmin}{argmin}
\DeclareMathOperator{\mult}{mult}
\DeclareMathOperator{\ch}{ch}
\DeclareMathOperator{\sh}{sh}
\DeclareMathOperator{\rang}{rang}
\DeclareMathOperator{\diam}{diam}
\DeclareMathOperator{\Epigraphe}{Epigraphe}




\usepackage{xcolor}
\everymath{\color{blue}}
%\everymath{\color[rgb]{0,1,1}}
%\pagecolor[rgb]{0,0,0.5}


\newcommand*{\pdtest}[3][]{\ensuremath{\frac{\partial^{#1} #2}{\partial #3}}}

\newcommand*{\deffunc}[6][]{\ensuremath{
\begin{array}{rcl}
#2 : #3 &\rightarrow& #4\\
#5 &\mapsto& #6
\end{array}
}}

\newcommand{\eqcolon}{\mathrel{\resizebox{\widthof{$\mathord{=}$}}{\height}{ $\!\!=\!\!\resizebox{1.2\width}{0.8\height}{\raisebox{0.23ex}{$\mathop{:}$}}\!\!$ }}}
\newcommand{\coloneq}{\mathrel{\resizebox{\widthof{$\mathord{=}$}}{\height}{ $\!\!\resizebox{1.2\width}{0.8\height}{\raisebox{0.23ex}{$\mathop{:}$}}\!\!=\!\!$ }}}
\newcommand{\eqcolonl}{\ensuremath{\mathrel{=\!\!\mathop{:}}}}
\newcommand{\coloneql}{\ensuremath{\mathrel{\mathop{:} \!\! =}}}
\newcommand{\vc}[1]{% inline column vector
  \left(\begin{smallmatrix}#1\end{smallmatrix}\right)%
}
\newcommand{\vr}[1]{% inline row vector
  \begin{smallmatrix}(\,#1\,)\end{smallmatrix}%
}
\makeatletter
\newcommand*{\defeq}{\ =\mathrel{\rlap{%
                     \raisebox{0.3ex}{$\m@th\cdot$}}%
                     \raisebox{-0.3ex}{$\m@th\cdot$}}%
                     }
\makeatother

\newcommand{\mathcircle}[1]{% inline row vector
 \overset{\circ}{#1}
}
\newcommand{\ulim}{% low limit
 \underline{\lim}
}
\newcommand{\ssi}{% iff
\iff
}
\newcommand{\ps}[2]{
\expval{#1 | #2}
}
\newcommand{\df}[1]{
\mqty{#1}
}
\newcommand{\n}[1]{
\norm{#1}
}
\newcommand{\sys}[1]{
\left\{\smqty{#1}\right.
}


\newcommand{\eqdef}{\ensuremath{\overset{\text{def}}=}}


\def\Circlearrowright{\ensuremath{%
  \rotatebox[origin=c]{230}{$\circlearrowright$}}}

\newcommand\ct[1]{\text{\rmfamily\upshape #1}}
\newcommand\question[1]{ {\color{red} ...!? \small #1}}
\newcommand\caz[1]{\left\{\begin{array} #1 \end{array}\right.}
\newcommand\const{\text{\rmfamily\upshape const}}
\newcommand\toP{ \overset{\pro}{\to}}
\newcommand\toPP{ \overset{\text{PP}}{\to}}
\newcommand{\oeq}{\mathrel{\text{\textcircled{$=$}}}}





\usepackage{xcolor}
% \usepackage[normalem]{ulem}
\usepackage{lipsum}
\makeatletter
% \newcommand\colorwave[1][blue]{\bgroup \markoverwith{\lower3.5\p@\hbox{\sixly \textcolor{#1}{\char58}}}\ULon}
%\font\sixly=lasy6 % does not re-load if already loaded, so no memory problem.

\newmdtheoremenv[
linewidth= 1pt,linecolor= blue,%
leftmargin=20,rightmargin=20,innertopmargin=0pt, innerrightmargin=40,%
tikzsetting = { draw=lightgray, line width = 0.3pt,dashed,%
dash pattern = on 15pt off 3pt},%
splittopskip=\topskip,skipbelow=\baselineskip,%
skipabove=\baselineskip,ntheorem,roundcorner=0pt,
% backgroundcolor=pagebg,font=\color{orange}\sffamily, fontcolor=white
]{examplebox}{Exemple}[section]



\newcommand\R{\mathbb{R}}
\newcommand\Z{\mathbb{Z}}
\newcommand\N{\mathbb{N}}
\newcommand\E{\mathbb{E}}
\newcommand\F{\mathcal{F}}
\newcommand\cH{\mathcal{H}}
\newcommand\V{\mathbb{V}}
\newcommand\dmo{ ^{-1} }
\newcommand\kapa{\kappa}
\newcommand\im{Im}
\newcommand\hs{\mathcal{H}}





\usepackage{soul}

\makeatletter
\newcommand*{\whiten}[1]{\llap{\textcolor{white}{{\the\SOUL@token}}\hspace{#1pt}}}
\DeclareRobustCommand*\myul{%
    \def\SOUL@everyspace{\underline{\space}\kern\z@}%
    \def\SOUL@everytoken{%
     \setbox0=\hbox{\the\SOUL@token}%
     \ifdim\dp0>\z@
        \raisebox{\dp0}{\underline{\phantom{\the\SOUL@token}}}%
        \whiten{1}\whiten{0}%
        \whiten{-1}\whiten{-2}%
        \llap{\the\SOUL@token}%
     \else
        \underline{\the\SOUL@token}%
     \fi}%
\SOUL@}
\makeatother

\newcommand*{\demp}{\fontfamily{lmtt}\selectfont}

\DeclareTextFontCommand{\textdemp}{\demp}

\begin{document}

\ifcomment
Multiline
comment
\fi
\ifcomment
\myul{Typesetting test}
% \color[rgb]{1,1,1}
$∑_i^n≠ 60º±∞π∆¬≈√j∫h≤≥µ$

$\CR \R\pro\ind\pro\gS\pro
\mqty[a&b\\c&d]$
$\pro\mathbb{P}$
$\dd{x}$

  \[
    \alpha(x)=\left\{
                \begin{array}{ll}
                  x\\
                  \frac{1}{1+e^{-kx}}\\
                  \frac{e^x-e^{-x}}{e^x+e^{-x}}
                \end{array}
              \right.
  \]

  $\expval{x}$
  
  $\chi_\rho(ghg\dmo)=\Tr(\rho_{ghg\dmo})=\Tr(\rho_g\circ\rho_h\circ\rho\dmo_g)=\Tr(\rho_h)\overset{\mbox{\scalebox{0.5}{$\Tr(AB)=\Tr(BA)$}}}{=}\chi_\rho(h)$
  	$\mathop{\oplus}_{\substack{x\in X}}$

$\mat(\rho_g)=(a_{ij}(g))_{\scriptsize \substack{1\leq i\leq d \\ 1\leq j\leq d}}$ et $\mat(\rho'_g)=(a'_{ij}(g))_{\scriptsize \substack{1\leq i'\leq d' \\ 1\leq j'\leq d'}}$



\[\int_a^b{\mathbb{R}^2}g(u, v)\dd{P_{XY}}(u, v)=\iint g(u,v) f_{XY}(u, v)\dd \lambda(u) \dd \lambda(v)\]
$$\lim_{x\to\infty} f(x)$$	
$$\iiiint_V \mu(t,u,v,w) \,dt\,du\,dv\,dw$$
$$\sum_{n=1}^{\infty} 2^{-n} = 1$$	
\begin{definition}
	Si $X$ et $Y$ sont 2 v.a. ou definit la \textsc{Covariance} entre $X$ et $Y$ comme
	$\cov(X,Y)\overset{\text{def}}{=}\E\left[(X-\E(X))(Y-\E(Y))\right]=\E(XY)-\E(X)\E(Y)$.
\end{definition}
\fi
\pagebreak

% \tableofcontents

% insert your code here
%\input{./algebra/main.tex}
%\input{./geometrie-differentielle/main.tex}
%\input{./probabilite/main.tex}
%\input{./analyse-fonctionnelle/main.tex}
% \input{./Analyse-convexe-et-dualite-en-optimisation/main.tex}
%\input{./tikz/main.tex}
%\input{./Theorie-du-distributions/main.tex}
%\input{./optimisation/mine.tex}
 \input{./modelisation/main.tex}

% yves.aubry@univ-tln.fr : algebra

\end{document}

%\input{./optimisation/mine.tex}
 % !TEX encoding = UTF-8 Unicode
% !TEX TS-program = xelatex

\documentclass[french]{report}

%\usepackage[utf8]{inputenc}
%\usepackage[T1]{fontenc}
\usepackage{babel}


\newif\ifcomment
%\commenttrue # Show comments

\usepackage{physics}
\usepackage{amssymb}


\usepackage{amsthm}
% \usepackage{thmtools}
\usepackage{mathtools}
\usepackage{amsfonts}

\usepackage{color}

\usepackage{tikz}

\usepackage{geometry}
\geometry{a5paper, margin=0.1in, right=1cm}

\usepackage{dsfont}

\usepackage{graphicx}
\graphicspath{ {images/} }

\usepackage{faktor}

\usepackage{IEEEtrantools}
\usepackage{enumerate}   
\usepackage[PostScript=dvips]{"/Users/aware/Documents/Courses/diagrams"}


\newtheorem{theorem}{Théorème}[section]
\renewcommand{\thetheorem}{\arabic{theorem}}
\newtheorem{lemme}{Lemme}[section]
\renewcommand{\thelemme}{\arabic{lemme}}
\newtheorem{proposition}{Proposition}[section]
\renewcommand{\theproposition}{\arabic{proposition}}
\newtheorem{notations}{Notations}[section]
\newtheorem{problem}{Problème}[section]
\newtheorem{corollary}{Corollaire}[theorem]
\renewcommand{\thecorollary}{\arabic{corollary}}
\newtheorem{property}{Propriété}[section]
\newtheorem{objective}{Objectif}[section]

\theoremstyle{definition}
\newtheorem{definition}{Définition}[section]
\renewcommand{\thedefinition}{\arabic{definition}}
\newtheorem{exercise}{Exercice}[chapter]
\renewcommand{\theexercise}{\arabic{exercise}}
\newtheorem{example}{Exemple}[chapter]
\renewcommand{\theexample}{\arabic{example}}
\newtheorem*{solution}{Solution}
\newtheorem*{application}{Application}
\newtheorem*{notation}{Notation}
\newtheorem*{vocabulary}{Vocabulaire}
\newtheorem*{properties}{Propriétés}



\theoremstyle{remark}
\newtheorem*{remark}{Remarque}
\newtheorem*{rappel}{Rappel}


\usepackage{etoolbox}
\AtBeginEnvironment{exercise}{\small}
\AtBeginEnvironment{example}{\small}

\usepackage{cases}
\usepackage[red]{mypack}

\usepackage[framemethod=TikZ]{mdframed}

\definecolor{bg}{rgb}{0.4,0.25,0.95}
\definecolor{pagebg}{rgb}{0,0,0.5}
\surroundwithmdframed[
   topline=false,
   rightline=false,
   bottomline=false,
   leftmargin=\parindent,
   skipabove=8pt,
   skipbelow=8pt,
   linecolor=blue,
   innerbottommargin=10pt,
   % backgroundcolor=bg,font=\color{orange}\sffamily, fontcolor=white
]{definition}

\usepackage{empheq}
\usepackage[most]{tcolorbox}

\newtcbox{\mymath}[1][]{%
    nobeforeafter, math upper, tcbox raise base,
    enhanced, colframe=blue!30!black,
    colback=red!10, boxrule=1pt,
    #1}

\usepackage{unixode}


\DeclareMathOperator{\ord}{ord}
\DeclareMathOperator{\orb}{orb}
\DeclareMathOperator{\stab}{stab}
\DeclareMathOperator{\Stab}{stab}
\DeclareMathOperator{\ppcm}{ppcm}
\DeclareMathOperator{\conj}{Conj}
\DeclareMathOperator{\End}{End}
\DeclareMathOperator{\rot}{rot}
\DeclareMathOperator{\trs}{trace}
\DeclareMathOperator{\Ind}{Ind}
\DeclareMathOperator{\mat}{Mat}
\DeclareMathOperator{\id}{Id}
\DeclareMathOperator{\vect}{vect}
\DeclareMathOperator{\img}{img}
\DeclareMathOperator{\cov}{Cov}
\DeclareMathOperator{\dist}{dist}
\DeclareMathOperator{\irr}{Irr}
\DeclareMathOperator{\image}{Im}
\DeclareMathOperator{\pd}{\partial}
\DeclareMathOperator{\epi}{epi}
\DeclareMathOperator{\Argmin}{Argmin}
\DeclareMathOperator{\dom}{dom}
\DeclareMathOperator{\proj}{proj}
\DeclareMathOperator{\ctg}{ctg}
\DeclareMathOperator{\supp}{supp}
\DeclareMathOperator{\argmin}{argmin}
\DeclareMathOperator{\mult}{mult}
\DeclareMathOperator{\ch}{ch}
\DeclareMathOperator{\sh}{sh}
\DeclareMathOperator{\rang}{rang}
\DeclareMathOperator{\diam}{diam}
\DeclareMathOperator{\Epigraphe}{Epigraphe}




\usepackage{xcolor}
\everymath{\color{blue}}
%\everymath{\color[rgb]{0,1,1}}
%\pagecolor[rgb]{0,0,0.5}


\newcommand*{\pdtest}[3][]{\ensuremath{\frac{\partial^{#1} #2}{\partial #3}}}

\newcommand*{\deffunc}[6][]{\ensuremath{
\begin{array}{rcl}
#2 : #3 &\rightarrow& #4\\
#5 &\mapsto& #6
\end{array}
}}

\newcommand{\eqcolon}{\mathrel{\resizebox{\widthof{$\mathord{=}$}}{\height}{ $\!\!=\!\!\resizebox{1.2\width}{0.8\height}{\raisebox{0.23ex}{$\mathop{:}$}}\!\!$ }}}
\newcommand{\coloneq}{\mathrel{\resizebox{\widthof{$\mathord{=}$}}{\height}{ $\!\!\resizebox{1.2\width}{0.8\height}{\raisebox{0.23ex}{$\mathop{:}$}}\!\!=\!\!$ }}}
\newcommand{\eqcolonl}{\ensuremath{\mathrel{=\!\!\mathop{:}}}}
\newcommand{\coloneql}{\ensuremath{\mathrel{\mathop{:} \!\! =}}}
\newcommand{\vc}[1]{% inline column vector
  \left(\begin{smallmatrix}#1\end{smallmatrix}\right)%
}
\newcommand{\vr}[1]{% inline row vector
  \begin{smallmatrix}(\,#1\,)\end{smallmatrix}%
}
\makeatletter
\newcommand*{\defeq}{\ =\mathrel{\rlap{%
                     \raisebox{0.3ex}{$\m@th\cdot$}}%
                     \raisebox{-0.3ex}{$\m@th\cdot$}}%
                     }
\makeatother

\newcommand{\mathcircle}[1]{% inline row vector
 \overset{\circ}{#1}
}
\newcommand{\ulim}{% low limit
 \underline{\lim}
}
\newcommand{\ssi}{% iff
\iff
}
\newcommand{\ps}[2]{
\expval{#1 | #2}
}
\newcommand{\df}[1]{
\mqty{#1}
}
\newcommand{\n}[1]{
\norm{#1}
}
\newcommand{\sys}[1]{
\left\{\smqty{#1}\right.
}


\newcommand{\eqdef}{\ensuremath{\overset{\text{def}}=}}


\def\Circlearrowright{\ensuremath{%
  \rotatebox[origin=c]{230}{$\circlearrowright$}}}

\newcommand\ct[1]{\text{\rmfamily\upshape #1}}
\newcommand\question[1]{ {\color{red} ...!? \small #1}}
\newcommand\caz[1]{\left\{\begin{array} #1 \end{array}\right.}
\newcommand\const{\text{\rmfamily\upshape const}}
\newcommand\toP{ \overset{\pro}{\to}}
\newcommand\toPP{ \overset{\text{PP}}{\to}}
\newcommand{\oeq}{\mathrel{\text{\textcircled{$=$}}}}





\usepackage{xcolor}
% \usepackage[normalem]{ulem}
\usepackage{lipsum}
\makeatletter
% \newcommand\colorwave[1][blue]{\bgroup \markoverwith{\lower3.5\p@\hbox{\sixly \textcolor{#1}{\char58}}}\ULon}
%\font\sixly=lasy6 % does not re-load if already loaded, so no memory problem.

\newmdtheoremenv[
linewidth= 1pt,linecolor= blue,%
leftmargin=20,rightmargin=20,innertopmargin=0pt, innerrightmargin=40,%
tikzsetting = { draw=lightgray, line width = 0.3pt,dashed,%
dash pattern = on 15pt off 3pt},%
splittopskip=\topskip,skipbelow=\baselineskip,%
skipabove=\baselineskip,ntheorem,roundcorner=0pt,
% backgroundcolor=pagebg,font=\color{orange}\sffamily, fontcolor=white
]{examplebox}{Exemple}[section]



\newcommand\R{\mathbb{R}}
\newcommand\Z{\mathbb{Z}}
\newcommand\N{\mathbb{N}}
\newcommand\E{\mathbb{E}}
\newcommand\F{\mathcal{F}}
\newcommand\cH{\mathcal{H}}
\newcommand\V{\mathbb{V}}
\newcommand\dmo{ ^{-1} }
\newcommand\kapa{\kappa}
\newcommand\im{Im}
\newcommand\hs{\mathcal{H}}





\usepackage{soul}

\makeatletter
\newcommand*{\whiten}[1]{\llap{\textcolor{white}{{\the\SOUL@token}}\hspace{#1pt}}}
\DeclareRobustCommand*\myul{%
    \def\SOUL@everyspace{\underline{\space}\kern\z@}%
    \def\SOUL@everytoken{%
     \setbox0=\hbox{\the\SOUL@token}%
     \ifdim\dp0>\z@
        \raisebox{\dp0}{\underline{\phantom{\the\SOUL@token}}}%
        \whiten{1}\whiten{0}%
        \whiten{-1}\whiten{-2}%
        \llap{\the\SOUL@token}%
     \else
        \underline{\the\SOUL@token}%
     \fi}%
\SOUL@}
\makeatother

\newcommand*{\demp}{\fontfamily{lmtt}\selectfont}

\DeclareTextFontCommand{\textdemp}{\demp}

\begin{document}

\ifcomment
Multiline
comment
\fi
\ifcomment
\myul{Typesetting test}
% \color[rgb]{1,1,1}
$∑_i^n≠ 60º±∞π∆¬≈√j∫h≤≥µ$

$\CR \R\pro\ind\pro\gS\pro
\mqty[a&b\\c&d]$
$\pro\mathbb{P}$
$\dd{x}$

  \[
    \alpha(x)=\left\{
                \begin{array}{ll}
                  x\\
                  \frac{1}{1+e^{-kx}}\\
                  \frac{e^x-e^{-x}}{e^x+e^{-x}}
                \end{array}
              \right.
  \]

  $\expval{x}$
  
  $\chi_\rho(ghg\dmo)=\Tr(\rho_{ghg\dmo})=\Tr(\rho_g\circ\rho_h\circ\rho\dmo_g)=\Tr(\rho_h)\overset{\mbox{\scalebox{0.5}{$\Tr(AB)=\Tr(BA)$}}}{=}\chi_\rho(h)$
  	$\mathop{\oplus}_{\substack{x\in X}}$

$\mat(\rho_g)=(a_{ij}(g))_{\scriptsize \substack{1\leq i\leq d \\ 1\leq j\leq d}}$ et $\mat(\rho'_g)=(a'_{ij}(g))_{\scriptsize \substack{1\leq i'\leq d' \\ 1\leq j'\leq d'}}$



\[\int_a^b{\mathbb{R}^2}g(u, v)\dd{P_{XY}}(u, v)=\iint g(u,v) f_{XY}(u, v)\dd \lambda(u) \dd \lambda(v)\]
$$\lim_{x\to\infty} f(x)$$	
$$\iiiint_V \mu(t,u,v,w) \,dt\,du\,dv\,dw$$
$$\sum_{n=1}^{\infty} 2^{-n} = 1$$	
\begin{definition}
	Si $X$ et $Y$ sont 2 v.a. ou definit la \textsc{Covariance} entre $X$ et $Y$ comme
	$\cov(X,Y)\overset{\text{def}}{=}\E\left[(X-\E(X))(Y-\E(Y))\right]=\E(XY)-\E(X)\E(Y)$.
\end{definition}
\fi
\pagebreak

% \tableofcontents

% insert your code here
%\input{./algebra/main.tex}
%\input{./geometrie-differentielle/main.tex}
%\input{./probabilite/main.tex}
%\input{./analyse-fonctionnelle/main.tex}
% \input{./Analyse-convexe-et-dualite-en-optimisation/main.tex}
%\input{./tikz/main.tex}
%\input{./Theorie-du-distributions/main.tex}
%\input{./optimisation/mine.tex}
 \input{./modelisation/main.tex}

% yves.aubry@univ-tln.fr : algebra

\end{document}


% yves.aubry@univ-tln.fr : algebra

\end{document}

%% !TEX encoding = UTF-8 Unicode
% !TEX TS-program = xelatex

\documentclass[french]{report}

%\usepackage[utf8]{inputenc}
%\usepackage[T1]{fontenc}
\usepackage{babel}


\newif\ifcomment
%\commenttrue # Show comments

\usepackage{physics}
\usepackage{amssymb}


\usepackage{amsthm}
% \usepackage{thmtools}
\usepackage{mathtools}
\usepackage{amsfonts}

\usepackage{color}

\usepackage{tikz}

\usepackage{geometry}
\geometry{a5paper, margin=0.1in, right=1cm}

\usepackage{dsfont}

\usepackage{graphicx}
\graphicspath{ {images/} }

\usepackage{faktor}

\usepackage{IEEEtrantools}
\usepackage{enumerate}   
\usepackage[PostScript=dvips]{"/Users/aware/Documents/Courses/diagrams"}


\newtheorem{theorem}{Théorème}[section]
\renewcommand{\thetheorem}{\arabic{theorem}}
\newtheorem{lemme}{Lemme}[section]
\renewcommand{\thelemme}{\arabic{lemme}}
\newtheorem{proposition}{Proposition}[section]
\renewcommand{\theproposition}{\arabic{proposition}}
\newtheorem{notations}{Notations}[section]
\newtheorem{problem}{Problème}[section]
\newtheorem{corollary}{Corollaire}[theorem]
\renewcommand{\thecorollary}{\arabic{corollary}}
\newtheorem{property}{Propriété}[section]
\newtheorem{objective}{Objectif}[section]

\theoremstyle{definition}
\newtheorem{definition}{Définition}[section]
\renewcommand{\thedefinition}{\arabic{definition}}
\newtheorem{exercise}{Exercice}[chapter]
\renewcommand{\theexercise}{\arabic{exercise}}
\newtheorem{example}{Exemple}[chapter]
\renewcommand{\theexample}{\arabic{example}}
\newtheorem*{solution}{Solution}
\newtheorem*{application}{Application}
\newtheorem*{notation}{Notation}
\newtheorem*{vocabulary}{Vocabulaire}
\newtheorem*{properties}{Propriétés}



\theoremstyle{remark}
\newtheorem*{remark}{Remarque}
\newtheorem*{rappel}{Rappel}


\usepackage{etoolbox}
\AtBeginEnvironment{exercise}{\small}
\AtBeginEnvironment{example}{\small}

\usepackage{cases}
\usepackage[red]{mypack}

\usepackage[framemethod=TikZ]{mdframed}

\definecolor{bg}{rgb}{0.4,0.25,0.95}
\definecolor{pagebg}{rgb}{0,0,0.5}
\surroundwithmdframed[
   topline=false,
   rightline=false,
   bottomline=false,
   leftmargin=\parindent,
   skipabove=8pt,
   skipbelow=8pt,
   linecolor=blue,
   innerbottommargin=10pt,
   % backgroundcolor=bg,font=\color{orange}\sffamily, fontcolor=white
]{definition}

\usepackage{empheq}
\usepackage[most]{tcolorbox}

\newtcbox{\mymath}[1][]{%
    nobeforeafter, math upper, tcbox raise base,
    enhanced, colframe=blue!30!black,
    colback=red!10, boxrule=1pt,
    #1}

\usepackage{unixode}


\DeclareMathOperator{\ord}{ord}
\DeclareMathOperator{\orb}{orb}
\DeclareMathOperator{\stab}{stab}
\DeclareMathOperator{\Stab}{stab}
\DeclareMathOperator{\ppcm}{ppcm}
\DeclareMathOperator{\conj}{Conj}
\DeclareMathOperator{\End}{End}
\DeclareMathOperator{\rot}{rot}
\DeclareMathOperator{\trs}{trace}
\DeclareMathOperator{\Ind}{Ind}
\DeclareMathOperator{\mat}{Mat}
\DeclareMathOperator{\id}{Id}
\DeclareMathOperator{\vect}{vect}
\DeclareMathOperator{\img}{img}
\DeclareMathOperator{\cov}{Cov}
\DeclareMathOperator{\dist}{dist}
\DeclareMathOperator{\irr}{Irr}
\DeclareMathOperator{\image}{Im}
\DeclareMathOperator{\pd}{\partial}
\DeclareMathOperator{\epi}{epi}
\DeclareMathOperator{\Argmin}{Argmin}
\DeclareMathOperator{\dom}{dom}
\DeclareMathOperator{\proj}{proj}
\DeclareMathOperator{\ctg}{ctg}
\DeclareMathOperator{\supp}{supp}
\DeclareMathOperator{\argmin}{argmin}
\DeclareMathOperator{\mult}{mult}
\DeclareMathOperator{\ch}{ch}
\DeclareMathOperator{\sh}{sh}
\DeclareMathOperator{\rang}{rang}
\DeclareMathOperator{\diam}{diam}
\DeclareMathOperator{\Epigraphe}{Epigraphe}




\usepackage{xcolor}
\everymath{\color{blue}}
%\everymath{\color[rgb]{0,1,1}}
%\pagecolor[rgb]{0,0,0.5}


\newcommand*{\pdtest}[3][]{\ensuremath{\frac{\partial^{#1} #2}{\partial #3}}}

\newcommand*{\deffunc}[6][]{\ensuremath{
\begin{array}{rcl}
#2 : #3 &\rightarrow& #4\\
#5 &\mapsto& #6
\end{array}
}}

\newcommand{\eqcolon}{\mathrel{\resizebox{\widthof{$\mathord{=}$}}{\height}{ $\!\!=\!\!\resizebox{1.2\width}{0.8\height}{\raisebox{0.23ex}{$\mathop{:}$}}\!\!$ }}}
\newcommand{\coloneq}{\mathrel{\resizebox{\widthof{$\mathord{=}$}}{\height}{ $\!\!\resizebox{1.2\width}{0.8\height}{\raisebox{0.23ex}{$\mathop{:}$}}\!\!=\!\!$ }}}
\newcommand{\eqcolonl}{\ensuremath{\mathrel{=\!\!\mathop{:}}}}
\newcommand{\coloneql}{\ensuremath{\mathrel{\mathop{:} \!\! =}}}
\newcommand{\vc}[1]{% inline column vector
  \left(\begin{smallmatrix}#1\end{smallmatrix}\right)%
}
\newcommand{\vr}[1]{% inline row vector
  \begin{smallmatrix}(\,#1\,)\end{smallmatrix}%
}
\makeatletter
\newcommand*{\defeq}{\ =\mathrel{\rlap{%
                     \raisebox{0.3ex}{$\m@th\cdot$}}%
                     \raisebox{-0.3ex}{$\m@th\cdot$}}%
                     }
\makeatother

\newcommand{\mathcircle}[1]{% inline row vector
 \overset{\circ}{#1}
}
\newcommand{\ulim}{% low limit
 \underline{\lim}
}
\newcommand{\ssi}{% iff
\iff
}
\newcommand{\ps}[2]{
\expval{#1 | #2}
}
\newcommand{\df}[1]{
\mqty{#1}
}
\newcommand{\n}[1]{
\norm{#1}
}
\newcommand{\sys}[1]{
\left\{\smqty{#1}\right.
}


\newcommand{\eqdef}{\ensuremath{\overset{\text{def}}=}}


\def\Circlearrowright{\ensuremath{%
  \rotatebox[origin=c]{230}{$\circlearrowright$}}}

\newcommand\ct[1]{\text{\rmfamily\upshape #1}}
\newcommand\question[1]{ {\color{red} ...!? \small #1}}
\newcommand\caz[1]{\left\{\begin{array} #1 \end{array}\right.}
\newcommand\const{\text{\rmfamily\upshape const}}
\newcommand\toP{ \overset{\pro}{\to}}
\newcommand\toPP{ \overset{\text{PP}}{\to}}
\newcommand{\oeq}{\mathrel{\text{\textcircled{$=$}}}}





\usepackage{xcolor}
% \usepackage[normalem]{ulem}
\usepackage{lipsum}
\makeatletter
% \newcommand\colorwave[1][blue]{\bgroup \markoverwith{\lower3.5\p@\hbox{\sixly \textcolor{#1}{\char58}}}\ULon}
%\font\sixly=lasy6 % does not re-load if already loaded, so no memory problem.

\newmdtheoremenv[
linewidth= 1pt,linecolor= blue,%
leftmargin=20,rightmargin=20,innertopmargin=0pt, innerrightmargin=40,%
tikzsetting = { draw=lightgray, line width = 0.3pt,dashed,%
dash pattern = on 15pt off 3pt},%
splittopskip=\topskip,skipbelow=\baselineskip,%
skipabove=\baselineskip,ntheorem,roundcorner=0pt,
% backgroundcolor=pagebg,font=\color{orange}\sffamily, fontcolor=white
]{examplebox}{Exemple}[section]



\newcommand\R{\mathbb{R}}
\newcommand\Z{\mathbb{Z}}
\newcommand\N{\mathbb{N}}
\newcommand\E{\mathbb{E}}
\newcommand\F{\mathcal{F}}
\newcommand\cH{\mathcal{H}}
\newcommand\V{\mathbb{V}}
\newcommand\dmo{ ^{-1} }
\newcommand\kapa{\kappa}
\newcommand\im{Im}
\newcommand\hs{\mathcal{H}}





\usepackage{soul}

\makeatletter
\newcommand*{\whiten}[1]{\llap{\textcolor{white}{{\the\SOUL@token}}\hspace{#1pt}}}
\DeclareRobustCommand*\myul{%
    \def\SOUL@everyspace{\underline{\space}\kern\z@}%
    \def\SOUL@everytoken{%
     \setbox0=\hbox{\the\SOUL@token}%
     \ifdim\dp0>\z@
        \raisebox{\dp0}{\underline{\phantom{\the\SOUL@token}}}%
        \whiten{1}\whiten{0}%
        \whiten{-1}\whiten{-2}%
        \llap{\the\SOUL@token}%
     \else
        \underline{\the\SOUL@token}%
     \fi}%
\SOUL@}
\makeatother

\newcommand*{\demp}{\fontfamily{lmtt}\selectfont}

\DeclareTextFontCommand{\textdemp}{\demp}

\begin{document}

\ifcomment
Multiline
comment
\fi
\ifcomment
\myul{Typesetting test}
% \color[rgb]{1,1,1}
$∑_i^n≠ 60º±∞π∆¬≈√j∫h≤≥µ$

$\CR \R\pro\ind\pro\gS\pro
\mqty[a&b\\c&d]$
$\pro\mathbb{P}$
$\dd{x}$

  \[
    \alpha(x)=\left\{
                \begin{array}{ll}
                  x\\
                  \frac{1}{1+e^{-kx}}\\
                  \frac{e^x-e^{-x}}{e^x+e^{-x}}
                \end{array}
              \right.
  \]

  $\expval{x}$
  
  $\chi_\rho(ghg\dmo)=\Tr(\rho_{ghg\dmo})=\Tr(\rho_g\circ\rho_h\circ\rho\dmo_g)=\Tr(\rho_h)\overset{\mbox{\scalebox{0.5}{$\Tr(AB)=\Tr(BA)$}}}{=}\chi_\rho(h)$
  	$\mathop{\oplus}_{\substack{x\in X}}$

$\mat(\rho_g)=(a_{ij}(g))_{\scriptsize \substack{1\leq i\leq d \\ 1\leq j\leq d}}$ et $\mat(\rho'_g)=(a'_{ij}(g))_{\scriptsize \substack{1\leq i'\leq d' \\ 1\leq j'\leq d'}}$



\[\int_a^b{\mathbb{R}^2}g(u, v)\dd{P_{XY}}(u, v)=\iint g(u,v) f_{XY}(u, v)\dd \lambda(u) \dd \lambda(v)\]
$$\lim_{x\to\infty} f(x)$$	
$$\iiiint_V \mu(t,u,v,w) \,dt\,du\,dv\,dw$$
$$\sum_{n=1}^{\infty} 2^{-n} = 1$$	
\begin{definition}
	Si $X$ et $Y$ sont 2 v.a. ou definit la \textsc{Covariance} entre $X$ et $Y$ comme
	$\cov(X,Y)\overset{\text{def}}{=}\E\left[(X-\E(X))(Y-\E(Y))\right]=\E(XY)-\E(X)\E(Y)$.
\end{definition}
\fi
\pagebreak

% \tableofcontents

% insert your code here
%% !TEX encoding = UTF-8 Unicode
% !TEX TS-program = xelatex

\documentclass[french]{report}

%\usepackage[utf8]{inputenc}
%\usepackage[T1]{fontenc}
\usepackage{babel}


\newif\ifcomment
%\commenttrue # Show comments

\usepackage{physics}
\usepackage{amssymb}


\usepackage{amsthm}
% \usepackage{thmtools}
\usepackage{mathtools}
\usepackage{amsfonts}

\usepackage{color}

\usepackage{tikz}

\usepackage{geometry}
\geometry{a5paper, margin=0.1in, right=1cm}

\usepackage{dsfont}

\usepackage{graphicx}
\graphicspath{ {images/} }

\usepackage{faktor}

\usepackage{IEEEtrantools}
\usepackage{enumerate}   
\usepackage[PostScript=dvips]{"/Users/aware/Documents/Courses/diagrams"}


\newtheorem{theorem}{Théorème}[section]
\renewcommand{\thetheorem}{\arabic{theorem}}
\newtheorem{lemme}{Lemme}[section]
\renewcommand{\thelemme}{\arabic{lemme}}
\newtheorem{proposition}{Proposition}[section]
\renewcommand{\theproposition}{\arabic{proposition}}
\newtheorem{notations}{Notations}[section]
\newtheorem{problem}{Problème}[section]
\newtheorem{corollary}{Corollaire}[theorem]
\renewcommand{\thecorollary}{\arabic{corollary}}
\newtheorem{property}{Propriété}[section]
\newtheorem{objective}{Objectif}[section]

\theoremstyle{definition}
\newtheorem{definition}{Définition}[section]
\renewcommand{\thedefinition}{\arabic{definition}}
\newtheorem{exercise}{Exercice}[chapter]
\renewcommand{\theexercise}{\arabic{exercise}}
\newtheorem{example}{Exemple}[chapter]
\renewcommand{\theexample}{\arabic{example}}
\newtheorem*{solution}{Solution}
\newtheorem*{application}{Application}
\newtheorem*{notation}{Notation}
\newtheorem*{vocabulary}{Vocabulaire}
\newtheorem*{properties}{Propriétés}



\theoremstyle{remark}
\newtheorem*{remark}{Remarque}
\newtheorem*{rappel}{Rappel}


\usepackage{etoolbox}
\AtBeginEnvironment{exercise}{\small}
\AtBeginEnvironment{example}{\small}

\usepackage{cases}
\usepackage[red]{mypack}

\usepackage[framemethod=TikZ]{mdframed}

\definecolor{bg}{rgb}{0.4,0.25,0.95}
\definecolor{pagebg}{rgb}{0,0,0.5}
\surroundwithmdframed[
   topline=false,
   rightline=false,
   bottomline=false,
   leftmargin=\parindent,
   skipabove=8pt,
   skipbelow=8pt,
   linecolor=blue,
   innerbottommargin=10pt,
   % backgroundcolor=bg,font=\color{orange}\sffamily, fontcolor=white
]{definition}

\usepackage{empheq}
\usepackage[most]{tcolorbox}

\newtcbox{\mymath}[1][]{%
    nobeforeafter, math upper, tcbox raise base,
    enhanced, colframe=blue!30!black,
    colback=red!10, boxrule=1pt,
    #1}

\usepackage{unixode}


\DeclareMathOperator{\ord}{ord}
\DeclareMathOperator{\orb}{orb}
\DeclareMathOperator{\stab}{stab}
\DeclareMathOperator{\Stab}{stab}
\DeclareMathOperator{\ppcm}{ppcm}
\DeclareMathOperator{\conj}{Conj}
\DeclareMathOperator{\End}{End}
\DeclareMathOperator{\rot}{rot}
\DeclareMathOperator{\trs}{trace}
\DeclareMathOperator{\Ind}{Ind}
\DeclareMathOperator{\mat}{Mat}
\DeclareMathOperator{\id}{Id}
\DeclareMathOperator{\vect}{vect}
\DeclareMathOperator{\img}{img}
\DeclareMathOperator{\cov}{Cov}
\DeclareMathOperator{\dist}{dist}
\DeclareMathOperator{\irr}{Irr}
\DeclareMathOperator{\image}{Im}
\DeclareMathOperator{\pd}{\partial}
\DeclareMathOperator{\epi}{epi}
\DeclareMathOperator{\Argmin}{Argmin}
\DeclareMathOperator{\dom}{dom}
\DeclareMathOperator{\proj}{proj}
\DeclareMathOperator{\ctg}{ctg}
\DeclareMathOperator{\supp}{supp}
\DeclareMathOperator{\argmin}{argmin}
\DeclareMathOperator{\mult}{mult}
\DeclareMathOperator{\ch}{ch}
\DeclareMathOperator{\sh}{sh}
\DeclareMathOperator{\rang}{rang}
\DeclareMathOperator{\diam}{diam}
\DeclareMathOperator{\Epigraphe}{Epigraphe}




\usepackage{xcolor}
\everymath{\color{blue}}
%\everymath{\color[rgb]{0,1,1}}
%\pagecolor[rgb]{0,0,0.5}


\newcommand*{\pdtest}[3][]{\ensuremath{\frac{\partial^{#1} #2}{\partial #3}}}

\newcommand*{\deffunc}[6][]{\ensuremath{
\begin{array}{rcl}
#2 : #3 &\rightarrow& #4\\
#5 &\mapsto& #6
\end{array}
}}

\newcommand{\eqcolon}{\mathrel{\resizebox{\widthof{$\mathord{=}$}}{\height}{ $\!\!=\!\!\resizebox{1.2\width}{0.8\height}{\raisebox{0.23ex}{$\mathop{:}$}}\!\!$ }}}
\newcommand{\coloneq}{\mathrel{\resizebox{\widthof{$\mathord{=}$}}{\height}{ $\!\!\resizebox{1.2\width}{0.8\height}{\raisebox{0.23ex}{$\mathop{:}$}}\!\!=\!\!$ }}}
\newcommand{\eqcolonl}{\ensuremath{\mathrel{=\!\!\mathop{:}}}}
\newcommand{\coloneql}{\ensuremath{\mathrel{\mathop{:} \!\! =}}}
\newcommand{\vc}[1]{% inline column vector
  \left(\begin{smallmatrix}#1\end{smallmatrix}\right)%
}
\newcommand{\vr}[1]{% inline row vector
  \begin{smallmatrix}(\,#1\,)\end{smallmatrix}%
}
\makeatletter
\newcommand*{\defeq}{\ =\mathrel{\rlap{%
                     \raisebox{0.3ex}{$\m@th\cdot$}}%
                     \raisebox{-0.3ex}{$\m@th\cdot$}}%
                     }
\makeatother

\newcommand{\mathcircle}[1]{% inline row vector
 \overset{\circ}{#1}
}
\newcommand{\ulim}{% low limit
 \underline{\lim}
}
\newcommand{\ssi}{% iff
\iff
}
\newcommand{\ps}[2]{
\expval{#1 | #2}
}
\newcommand{\df}[1]{
\mqty{#1}
}
\newcommand{\n}[1]{
\norm{#1}
}
\newcommand{\sys}[1]{
\left\{\smqty{#1}\right.
}


\newcommand{\eqdef}{\ensuremath{\overset{\text{def}}=}}


\def\Circlearrowright{\ensuremath{%
  \rotatebox[origin=c]{230}{$\circlearrowright$}}}

\newcommand\ct[1]{\text{\rmfamily\upshape #1}}
\newcommand\question[1]{ {\color{red} ...!? \small #1}}
\newcommand\caz[1]{\left\{\begin{array} #1 \end{array}\right.}
\newcommand\const{\text{\rmfamily\upshape const}}
\newcommand\toP{ \overset{\pro}{\to}}
\newcommand\toPP{ \overset{\text{PP}}{\to}}
\newcommand{\oeq}{\mathrel{\text{\textcircled{$=$}}}}





\usepackage{xcolor}
% \usepackage[normalem]{ulem}
\usepackage{lipsum}
\makeatletter
% \newcommand\colorwave[1][blue]{\bgroup \markoverwith{\lower3.5\p@\hbox{\sixly \textcolor{#1}{\char58}}}\ULon}
%\font\sixly=lasy6 % does not re-load if already loaded, so no memory problem.

\newmdtheoremenv[
linewidth= 1pt,linecolor= blue,%
leftmargin=20,rightmargin=20,innertopmargin=0pt, innerrightmargin=40,%
tikzsetting = { draw=lightgray, line width = 0.3pt,dashed,%
dash pattern = on 15pt off 3pt},%
splittopskip=\topskip,skipbelow=\baselineskip,%
skipabove=\baselineskip,ntheorem,roundcorner=0pt,
% backgroundcolor=pagebg,font=\color{orange}\sffamily, fontcolor=white
]{examplebox}{Exemple}[section]



\newcommand\R{\mathbb{R}}
\newcommand\Z{\mathbb{Z}}
\newcommand\N{\mathbb{N}}
\newcommand\E{\mathbb{E}}
\newcommand\F{\mathcal{F}}
\newcommand\cH{\mathcal{H}}
\newcommand\V{\mathbb{V}}
\newcommand\dmo{ ^{-1} }
\newcommand\kapa{\kappa}
\newcommand\im{Im}
\newcommand\hs{\mathcal{H}}





\usepackage{soul}

\makeatletter
\newcommand*{\whiten}[1]{\llap{\textcolor{white}{{\the\SOUL@token}}\hspace{#1pt}}}
\DeclareRobustCommand*\myul{%
    \def\SOUL@everyspace{\underline{\space}\kern\z@}%
    \def\SOUL@everytoken{%
     \setbox0=\hbox{\the\SOUL@token}%
     \ifdim\dp0>\z@
        \raisebox{\dp0}{\underline{\phantom{\the\SOUL@token}}}%
        \whiten{1}\whiten{0}%
        \whiten{-1}\whiten{-2}%
        \llap{\the\SOUL@token}%
     \else
        \underline{\the\SOUL@token}%
     \fi}%
\SOUL@}
\makeatother

\newcommand*{\demp}{\fontfamily{lmtt}\selectfont}

\DeclareTextFontCommand{\textdemp}{\demp}

\begin{document}

\ifcomment
Multiline
comment
\fi
\ifcomment
\myul{Typesetting test}
% \color[rgb]{1,1,1}
$∑_i^n≠ 60º±∞π∆¬≈√j∫h≤≥µ$

$\CR \R\pro\ind\pro\gS\pro
\mqty[a&b\\c&d]$
$\pro\mathbb{P}$
$\dd{x}$

  \[
    \alpha(x)=\left\{
                \begin{array}{ll}
                  x\\
                  \frac{1}{1+e^{-kx}}\\
                  \frac{e^x-e^{-x}}{e^x+e^{-x}}
                \end{array}
              \right.
  \]

  $\expval{x}$
  
  $\chi_\rho(ghg\dmo)=\Tr(\rho_{ghg\dmo})=\Tr(\rho_g\circ\rho_h\circ\rho\dmo_g)=\Tr(\rho_h)\overset{\mbox{\scalebox{0.5}{$\Tr(AB)=\Tr(BA)$}}}{=}\chi_\rho(h)$
  	$\mathop{\oplus}_{\substack{x\in X}}$

$\mat(\rho_g)=(a_{ij}(g))_{\scriptsize \substack{1\leq i\leq d \\ 1\leq j\leq d}}$ et $\mat(\rho'_g)=(a'_{ij}(g))_{\scriptsize \substack{1\leq i'\leq d' \\ 1\leq j'\leq d'}}$



\[\int_a^b{\mathbb{R}^2}g(u, v)\dd{P_{XY}}(u, v)=\iint g(u,v) f_{XY}(u, v)\dd \lambda(u) \dd \lambda(v)\]
$$\lim_{x\to\infty} f(x)$$	
$$\iiiint_V \mu(t,u,v,w) \,dt\,du\,dv\,dw$$
$$\sum_{n=1}^{\infty} 2^{-n} = 1$$	
\begin{definition}
	Si $X$ et $Y$ sont 2 v.a. ou definit la \textsc{Covariance} entre $X$ et $Y$ comme
	$\cov(X,Y)\overset{\text{def}}{=}\E\left[(X-\E(X))(Y-\E(Y))\right]=\E(XY)-\E(X)\E(Y)$.
\end{definition}
\fi
\pagebreak

% \tableofcontents

% insert your code here
%\input{./algebra/main.tex}
%\input{./geometrie-differentielle/main.tex}
%\input{./probabilite/main.tex}
%\input{./analyse-fonctionnelle/main.tex}
% \input{./Analyse-convexe-et-dualite-en-optimisation/main.tex}
%\input{./tikz/main.tex}
%\input{./Theorie-du-distributions/main.tex}
%\input{./optimisation/mine.tex}
 \input{./modelisation/main.tex}

% yves.aubry@univ-tln.fr : algebra

\end{document}

%% !TEX encoding = UTF-8 Unicode
% !TEX TS-program = xelatex

\documentclass[french]{report}

%\usepackage[utf8]{inputenc}
%\usepackage[T1]{fontenc}
\usepackage{babel}


\newif\ifcomment
%\commenttrue # Show comments

\usepackage{physics}
\usepackage{amssymb}


\usepackage{amsthm}
% \usepackage{thmtools}
\usepackage{mathtools}
\usepackage{amsfonts}

\usepackage{color}

\usepackage{tikz}

\usepackage{geometry}
\geometry{a5paper, margin=0.1in, right=1cm}

\usepackage{dsfont}

\usepackage{graphicx}
\graphicspath{ {images/} }

\usepackage{faktor}

\usepackage{IEEEtrantools}
\usepackage{enumerate}   
\usepackage[PostScript=dvips]{"/Users/aware/Documents/Courses/diagrams"}


\newtheorem{theorem}{Théorème}[section]
\renewcommand{\thetheorem}{\arabic{theorem}}
\newtheorem{lemme}{Lemme}[section]
\renewcommand{\thelemme}{\arabic{lemme}}
\newtheorem{proposition}{Proposition}[section]
\renewcommand{\theproposition}{\arabic{proposition}}
\newtheorem{notations}{Notations}[section]
\newtheorem{problem}{Problème}[section]
\newtheorem{corollary}{Corollaire}[theorem]
\renewcommand{\thecorollary}{\arabic{corollary}}
\newtheorem{property}{Propriété}[section]
\newtheorem{objective}{Objectif}[section]

\theoremstyle{definition}
\newtheorem{definition}{Définition}[section]
\renewcommand{\thedefinition}{\arabic{definition}}
\newtheorem{exercise}{Exercice}[chapter]
\renewcommand{\theexercise}{\arabic{exercise}}
\newtheorem{example}{Exemple}[chapter]
\renewcommand{\theexample}{\arabic{example}}
\newtheorem*{solution}{Solution}
\newtheorem*{application}{Application}
\newtheorem*{notation}{Notation}
\newtheorem*{vocabulary}{Vocabulaire}
\newtheorem*{properties}{Propriétés}



\theoremstyle{remark}
\newtheorem*{remark}{Remarque}
\newtheorem*{rappel}{Rappel}


\usepackage{etoolbox}
\AtBeginEnvironment{exercise}{\small}
\AtBeginEnvironment{example}{\small}

\usepackage{cases}
\usepackage[red]{mypack}

\usepackage[framemethod=TikZ]{mdframed}

\definecolor{bg}{rgb}{0.4,0.25,0.95}
\definecolor{pagebg}{rgb}{0,0,0.5}
\surroundwithmdframed[
   topline=false,
   rightline=false,
   bottomline=false,
   leftmargin=\parindent,
   skipabove=8pt,
   skipbelow=8pt,
   linecolor=blue,
   innerbottommargin=10pt,
   % backgroundcolor=bg,font=\color{orange}\sffamily, fontcolor=white
]{definition}

\usepackage{empheq}
\usepackage[most]{tcolorbox}

\newtcbox{\mymath}[1][]{%
    nobeforeafter, math upper, tcbox raise base,
    enhanced, colframe=blue!30!black,
    colback=red!10, boxrule=1pt,
    #1}

\usepackage{unixode}


\DeclareMathOperator{\ord}{ord}
\DeclareMathOperator{\orb}{orb}
\DeclareMathOperator{\stab}{stab}
\DeclareMathOperator{\Stab}{stab}
\DeclareMathOperator{\ppcm}{ppcm}
\DeclareMathOperator{\conj}{Conj}
\DeclareMathOperator{\End}{End}
\DeclareMathOperator{\rot}{rot}
\DeclareMathOperator{\trs}{trace}
\DeclareMathOperator{\Ind}{Ind}
\DeclareMathOperator{\mat}{Mat}
\DeclareMathOperator{\id}{Id}
\DeclareMathOperator{\vect}{vect}
\DeclareMathOperator{\img}{img}
\DeclareMathOperator{\cov}{Cov}
\DeclareMathOperator{\dist}{dist}
\DeclareMathOperator{\irr}{Irr}
\DeclareMathOperator{\image}{Im}
\DeclareMathOperator{\pd}{\partial}
\DeclareMathOperator{\epi}{epi}
\DeclareMathOperator{\Argmin}{Argmin}
\DeclareMathOperator{\dom}{dom}
\DeclareMathOperator{\proj}{proj}
\DeclareMathOperator{\ctg}{ctg}
\DeclareMathOperator{\supp}{supp}
\DeclareMathOperator{\argmin}{argmin}
\DeclareMathOperator{\mult}{mult}
\DeclareMathOperator{\ch}{ch}
\DeclareMathOperator{\sh}{sh}
\DeclareMathOperator{\rang}{rang}
\DeclareMathOperator{\diam}{diam}
\DeclareMathOperator{\Epigraphe}{Epigraphe}




\usepackage{xcolor}
\everymath{\color{blue}}
%\everymath{\color[rgb]{0,1,1}}
%\pagecolor[rgb]{0,0,0.5}


\newcommand*{\pdtest}[3][]{\ensuremath{\frac{\partial^{#1} #2}{\partial #3}}}

\newcommand*{\deffunc}[6][]{\ensuremath{
\begin{array}{rcl}
#2 : #3 &\rightarrow& #4\\
#5 &\mapsto& #6
\end{array}
}}

\newcommand{\eqcolon}{\mathrel{\resizebox{\widthof{$\mathord{=}$}}{\height}{ $\!\!=\!\!\resizebox{1.2\width}{0.8\height}{\raisebox{0.23ex}{$\mathop{:}$}}\!\!$ }}}
\newcommand{\coloneq}{\mathrel{\resizebox{\widthof{$\mathord{=}$}}{\height}{ $\!\!\resizebox{1.2\width}{0.8\height}{\raisebox{0.23ex}{$\mathop{:}$}}\!\!=\!\!$ }}}
\newcommand{\eqcolonl}{\ensuremath{\mathrel{=\!\!\mathop{:}}}}
\newcommand{\coloneql}{\ensuremath{\mathrel{\mathop{:} \!\! =}}}
\newcommand{\vc}[1]{% inline column vector
  \left(\begin{smallmatrix}#1\end{smallmatrix}\right)%
}
\newcommand{\vr}[1]{% inline row vector
  \begin{smallmatrix}(\,#1\,)\end{smallmatrix}%
}
\makeatletter
\newcommand*{\defeq}{\ =\mathrel{\rlap{%
                     \raisebox{0.3ex}{$\m@th\cdot$}}%
                     \raisebox{-0.3ex}{$\m@th\cdot$}}%
                     }
\makeatother

\newcommand{\mathcircle}[1]{% inline row vector
 \overset{\circ}{#1}
}
\newcommand{\ulim}{% low limit
 \underline{\lim}
}
\newcommand{\ssi}{% iff
\iff
}
\newcommand{\ps}[2]{
\expval{#1 | #2}
}
\newcommand{\df}[1]{
\mqty{#1}
}
\newcommand{\n}[1]{
\norm{#1}
}
\newcommand{\sys}[1]{
\left\{\smqty{#1}\right.
}


\newcommand{\eqdef}{\ensuremath{\overset{\text{def}}=}}


\def\Circlearrowright{\ensuremath{%
  \rotatebox[origin=c]{230}{$\circlearrowright$}}}

\newcommand\ct[1]{\text{\rmfamily\upshape #1}}
\newcommand\question[1]{ {\color{red} ...!? \small #1}}
\newcommand\caz[1]{\left\{\begin{array} #1 \end{array}\right.}
\newcommand\const{\text{\rmfamily\upshape const}}
\newcommand\toP{ \overset{\pro}{\to}}
\newcommand\toPP{ \overset{\text{PP}}{\to}}
\newcommand{\oeq}{\mathrel{\text{\textcircled{$=$}}}}





\usepackage{xcolor}
% \usepackage[normalem]{ulem}
\usepackage{lipsum}
\makeatletter
% \newcommand\colorwave[1][blue]{\bgroup \markoverwith{\lower3.5\p@\hbox{\sixly \textcolor{#1}{\char58}}}\ULon}
%\font\sixly=lasy6 % does not re-load if already loaded, so no memory problem.

\newmdtheoremenv[
linewidth= 1pt,linecolor= blue,%
leftmargin=20,rightmargin=20,innertopmargin=0pt, innerrightmargin=40,%
tikzsetting = { draw=lightgray, line width = 0.3pt,dashed,%
dash pattern = on 15pt off 3pt},%
splittopskip=\topskip,skipbelow=\baselineskip,%
skipabove=\baselineskip,ntheorem,roundcorner=0pt,
% backgroundcolor=pagebg,font=\color{orange}\sffamily, fontcolor=white
]{examplebox}{Exemple}[section]



\newcommand\R{\mathbb{R}}
\newcommand\Z{\mathbb{Z}}
\newcommand\N{\mathbb{N}}
\newcommand\E{\mathbb{E}}
\newcommand\F{\mathcal{F}}
\newcommand\cH{\mathcal{H}}
\newcommand\V{\mathbb{V}}
\newcommand\dmo{ ^{-1} }
\newcommand\kapa{\kappa}
\newcommand\im{Im}
\newcommand\hs{\mathcal{H}}





\usepackage{soul}

\makeatletter
\newcommand*{\whiten}[1]{\llap{\textcolor{white}{{\the\SOUL@token}}\hspace{#1pt}}}
\DeclareRobustCommand*\myul{%
    \def\SOUL@everyspace{\underline{\space}\kern\z@}%
    \def\SOUL@everytoken{%
     \setbox0=\hbox{\the\SOUL@token}%
     \ifdim\dp0>\z@
        \raisebox{\dp0}{\underline{\phantom{\the\SOUL@token}}}%
        \whiten{1}\whiten{0}%
        \whiten{-1}\whiten{-2}%
        \llap{\the\SOUL@token}%
     \else
        \underline{\the\SOUL@token}%
     \fi}%
\SOUL@}
\makeatother

\newcommand*{\demp}{\fontfamily{lmtt}\selectfont}

\DeclareTextFontCommand{\textdemp}{\demp}

\begin{document}

\ifcomment
Multiline
comment
\fi
\ifcomment
\myul{Typesetting test}
% \color[rgb]{1,1,1}
$∑_i^n≠ 60º±∞π∆¬≈√j∫h≤≥µ$

$\CR \R\pro\ind\pro\gS\pro
\mqty[a&b\\c&d]$
$\pro\mathbb{P}$
$\dd{x}$

  \[
    \alpha(x)=\left\{
                \begin{array}{ll}
                  x\\
                  \frac{1}{1+e^{-kx}}\\
                  \frac{e^x-e^{-x}}{e^x+e^{-x}}
                \end{array}
              \right.
  \]

  $\expval{x}$
  
  $\chi_\rho(ghg\dmo)=\Tr(\rho_{ghg\dmo})=\Tr(\rho_g\circ\rho_h\circ\rho\dmo_g)=\Tr(\rho_h)\overset{\mbox{\scalebox{0.5}{$\Tr(AB)=\Tr(BA)$}}}{=}\chi_\rho(h)$
  	$\mathop{\oplus}_{\substack{x\in X}}$

$\mat(\rho_g)=(a_{ij}(g))_{\scriptsize \substack{1\leq i\leq d \\ 1\leq j\leq d}}$ et $\mat(\rho'_g)=(a'_{ij}(g))_{\scriptsize \substack{1\leq i'\leq d' \\ 1\leq j'\leq d'}}$



\[\int_a^b{\mathbb{R}^2}g(u, v)\dd{P_{XY}}(u, v)=\iint g(u,v) f_{XY}(u, v)\dd \lambda(u) \dd \lambda(v)\]
$$\lim_{x\to\infty} f(x)$$	
$$\iiiint_V \mu(t,u,v,w) \,dt\,du\,dv\,dw$$
$$\sum_{n=1}^{\infty} 2^{-n} = 1$$	
\begin{definition}
	Si $X$ et $Y$ sont 2 v.a. ou definit la \textsc{Covariance} entre $X$ et $Y$ comme
	$\cov(X,Y)\overset{\text{def}}{=}\E\left[(X-\E(X))(Y-\E(Y))\right]=\E(XY)-\E(X)\E(Y)$.
\end{definition}
\fi
\pagebreak

% \tableofcontents

% insert your code here
%\input{./algebra/main.tex}
%\input{./geometrie-differentielle/main.tex}
%\input{./probabilite/main.tex}
%\input{./analyse-fonctionnelle/main.tex}
% \input{./Analyse-convexe-et-dualite-en-optimisation/main.tex}
%\input{./tikz/main.tex}
%\input{./Theorie-du-distributions/main.tex}
%\input{./optimisation/mine.tex}
 \input{./modelisation/main.tex}

% yves.aubry@univ-tln.fr : algebra

\end{document}

%% !TEX encoding = UTF-8 Unicode
% !TEX TS-program = xelatex

\documentclass[french]{report}

%\usepackage[utf8]{inputenc}
%\usepackage[T1]{fontenc}
\usepackage{babel}


\newif\ifcomment
%\commenttrue # Show comments

\usepackage{physics}
\usepackage{amssymb}


\usepackage{amsthm}
% \usepackage{thmtools}
\usepackage{mathtools}
\usepackage{amsfonts}

\usepackage{color}

\usepackage{tikz}

\usepackage{geometry}
\geometry{a5paper, margin=0.1in, right=1cm}

\usepackage{dsfont}

\usepackage{graphicx}
\graphicspath{ {images/} }

\usepackage{faktor}

\usepackage{IEEEtrantools}
\usepackage{enumerate}   
\usepackage[PostScript=dvips]{"/Users/aware/Documents/Courses/diagrams"}


\newtheorem{theorem}{Théorème}[section]
\renewcommand{\thetheorem}{\arabic{theorem}}
\newtheorem{lemme}{Lemme}[section]
\renewcommand{\thelemme}{\arabic{lemme}}
\newtheorem{proposition}{Proposition}[section]
\renewcommand{\theproposition}{\arabic{proposition}}
\newtheorem{notations}{Notations}[section]
\newtheorem{problem}{Problème}[section]
\newtheorem{corollary}{Corollaire}[theorem]
\renewcommand{\thecorollary}{\arabic{corollary}}
\newtheorem{property}{Propriété}[section]
\newtheorem{objective}{Objectif}[section]

\theoremstyle{definition}
\newtheorem{definition}{Définition}[section]
\renewcommand{\thedefinition}{\arabic{definition}}
\newtheorem{exercise}{Exercice}[chapter]
\renewcommand{\theexercise}{\arabic{exercise}}
\newtheorem{example}{Exemple}[chapter]
\renewcommand{\theexample}{\arabic{example}}
\newtheorem*{solution}{Solution}
\newtheorem*{application}{Application}
\newtheorem*{notation}{Notation}
\newtheorem*{vocabulary}{Vocabulaire}
\newtheorem*{properties}{Propriétés}



\theoremstyle{remark}
\newtheorem*{remark}{Remarque}
\newtheorem*{rappel}{Rappel}


\usepackage{etoolbox}
\AtBeginEnvironment{exercise}{\small}
\AtBeginEnvironment{example}{\small}

\usepackage{cases}
\usepackage[red]{mypack}

\usepackage[framemethod=TikZ]{mdframed}

\definecolor{bg}{rgb}{0.4,0.25,0.95}
\definecolor{pagebg}{rgb}{0,0,0.5}
\surroundwithmdframed[
   topline=false,
   rightline=false,
   bottomline=false,
   leftmargin=\parindent,
   skipabove=8pt,
   skipbelow=8pt,
   linecolor=blue,
   innerbottommargin=10pt,
   % backgroundcolor=bg,font=\color{orange}\sffamily, fontcolor=white
]{definition}

\usepackage{empheq}
\usepackage[most]{tcolorbox}

\newtcbox{\mymath}[1][]{%
    nobeforeafter, math upper, tcbox raise base,
    enhanced, colframe=blue!30!black,
    colback=red!10, boxrule=1pt,
    #1}

\usepackage{unixode}


\DeclareMathOperator{\ord}{ord}
\DeclareMathOperator{\orb}{orb}
\DeclareMathOperator{\stab}{stab}
\DeclareMathOperator{\Stab}{stab}
\DeclareMathOperator{\ppcm}{ppcm}
\DeclareMathOperator{\conj}{Conj}
\DeclareMathOperator{\End}{End}
\DeclareMathOperator{\rot}{rot}
\DeclareMathOperator{\trs}{trace}
\DeclareMathOperator{\Ind}{Ind}
\DeclareMathOperator{\mat}{Mat}
\DeclareMathOperator{\id}{Id}
\DeclareMathOperator{\vect}{vect}
\DeclareMathOperator{\img}{img}
\DeclareMathOperator{\cov}{Cov}
\DeclareMathOperator{\dist}{dist}
\DeclareMathOperator{\irr}{Irr}
\DeclareMathOperator{\image}{Im}
\DeclareMathOperator{\pd}{\partial}
\DeclareMathOperator{\epi}{epi}
\DeclareMathOperator{\Argmin}{Argmin}
\DeclareMathOperator{\dom}{dom}
\DeclareMathOperator{\proj}{proj}
\DeclareMathOperator{\ctg}{ctg}
\DeclareMathOperator{\supp}{supp}
\DeclareMathOperator{\argmin}{argmin}
\DeclareMathOperator{\mult}{mult}
\DeclareMathOperator{\ch}{ch}
\DeclareMathOperator{\sh}{sh}
\DeclareMathOperator{\rang}{rang}
\DeclareMathOperator{\diam}{diam}
\DeclareMathOperator{\Epigraphe}{Epigraphe}




\usepackage{xcolor}
\everymath{\color{blue}}
%\everymath{\color[rgb]{0,1,1}}
%\pagecolor[rgb]{0,0,0.5}


\newcommand*{\pdtest}[3][]{\ensuremath{\frac{\partial^{#1} #2}{\partial #3}}}

\newcommand*{\deffunc}[6][]{\ensuremath{
\begin{array}{rcl}
#2 : #3 &\rightarrow& #4\\
#5 &\mapsto& #6
\end{array}
}}

\newcommand{\eqcolon}{\mathrel{\resizebox{\widthof{$\mathord{=}$}}{\height}{ $\!\!=\!\!\resizebox{1.2\width}{0.8\height}{\raisebox{0.23ex}{$\mathop{:}$}}\!\!$ }}}
\newcommand{\coloneq}{\mathrel{\resizebox{\widthof{$\mathord{=}$}}{\height}{ $\!\!\resizebox{1.2\width}{0.8\height}{\raisebox{0.23ex}{$\mathop{:}$}}\!\!=\!\!$ }}}
\newcommand{\eqcolonl}{\ensuremath{\mathrel{=\!\!\mathop{:}}}}
\newcommand{\coloneql}{\ensuremath{\mathrel{\mathop{:} \!\! =}}}
\newcommand{\vc}[1]{% inline column vector
  \left(\begin{smallmatrix}#1\end{smallmatrix}\right)%
}
\newcommand{\vr}[1]{% inline row vector
  \begin{smallmatrix}(\,#1\,)\end{smallmatrix}%
}
\makeatletter
\newcommand*{\defeq}{\ =\mathrel{\rlap{%
                     \raisebox{0.3ex}{$\m@th\cdot$}}%
                     \raisebox{-0.3ex}{$\m@th\cdot$}}%
                     }
\makeatother

\newcommand{\mathcircle}[1]{% inline row vector
 \overset{\circ}{#1}
}
\newcommand{\ulim}{% low limit
 \underline{\lim}
}
\newcommand{\ssi}{% iff
\iff
}
\newcommand{\ps}[2]{
\expval{#1 | #2}
}
\newcommand{\df}[1]{
\mqty{#1}
}
\newcommand{\n}[1]{
\norm{#1}
}
\newcommand{\sys}[1]{
\left\{\smqty{#1}\right.
}


\newcommand{\eqdef}{\ensuremath{\overset{\text{def}}=}}


\def\Circlearrowright{\ensuremath{%
  \rotatebox[origin=c]{230}{$\circlearrowright$}}}

\newcommand\ct[1]{\text{\rmfamily\upshape #1}}
\newcommand\question[1]{ {\color{red} ...!? \small #1}}
\newcommand\caz[1]{\left\{\begin{array} #1 \end{array}\right.}
\newcommand\const{\text{\rmfamily\upshape const}}
\newcommand\toP{ \overset{\pro}{\to}}
\newcommand\toPP{ \overset{\text{PP}}{\to}}
\newcommand{\oeq}{\mathrel{\text{\textcircled{$=$}}}}





\usepackage{xcolor}
% \usepackage[normalem]{ulem}
\usepackage{lipsum}
\makeatletter
% \newcommand\colorwave[1][blue]{\bgroup \markoverwith{\lower3.5\p@\hbox{\sixly \textcolor{#1}{\char58}}}\ULon}
%\font\sixly=lasy6 % does not re-load if already loaded, so no memory problem.

\newmdtheoremenv[
linewidth= 1pt,linecolor= blue,%
leftmargin=20,rightmargin=20,innertopmargin=0pt, innerrightmargin=40,%
tikzsetting = { draw=lightgray, line width = 0.3pt,dashed,%
dash pattern = on 15pt off 3pt},%
splittopskip=\topskip,skipbelow=\baselineskip,%
skipabove=\baselineskip,ntheorem,roundcorner=0pt,
% backgroundcolor=pagebg,font=\color{orange}\sffamily, fontcolor=white
]{examplebox}{Exemple}[section]



\newcommand\R{\mathbb{R}}
\newcommand\Z{\mathbb{Z}}
\newcommand\N{\mathbb{N}}
\newcommand\E{\mathbb{E}}
\newcommand\F{\mathcal{F}}
\newcommand\cH{\mathcal{H}}
\newcommand\V{\mathbb{V}}
\newcommand\dmo{ ^{-1} }
\newcommand\kapa{\kappa}
\newcommand\im{Im}
\newcommand\hs{\mathcal{H}}





\usepackage{soul}

\makeatletter
\newcommand*{\whiten}[1]{\llap{\textcolor{white}{{\the\SOUL@token}}\hspace{#1pt}}}
\DeclareRobustCommand*\myul{%
    \def\SOUL@everyspace{\underline{\space}\kern\z@}%
    \def\SOUL@everytoken{%
     \setbox0=\hbox{\the\SOUL@token}%
     \ifdim\dp0>\z@
        \raisebox{\dp0}{\underline{\phantom{\the\SOUL@token}}}%
        \whiten{1}\whiten{0}%
        \whiten{-1}\whiten{-2}%
        \llap{\the\SOUL@token}%
     \else
        \underline{\the\SOUL@token}%
     \fi}%
\SOUL@}
\makeatother

\newcommand*{\demp}{\fontfamily{lmtt}\selectfont}

\DeclareTextFontCommand{\textdemp}{\demp}

\begin{document}

\ifcomment
Multiline
comment
\fi
\ifcomment
\myul{Typesetting test}
% \color[rgb]{1,1,1}
$∑_i^n≠ 60º±∞π∆¬≈√j∫h≤≥µ$

$\CR \R\pro\ind\pro\gS\pro
\mqty[a&b\\c&d]$
$\pro\mathbb{P}$
$\dd{x}$

  \[
    \alpha(x)=\left\{
                \begin{array}{ll}
                  x\\
                  \frac{1}{1+e^{-kx}}\\
                  \frac{e^x-e^{-x}}{e^x+e^{-x}}
                \end{array}
              \right.
  \]

  $\expval{x}$
  
  $\chi_\rho(ghg\dmo)=\Tr(\rho_{ghg\dmo})=\Tr(\rho_g\circ\rho_h\circ\rho\dmo_g)=\Tr(\rho_h)\overset{\mbox{\scalebox{0.5}{$\Tr(AB)=\Tr(BA)$}}}{=}\chi_\rho(h)$
  	$\mathop{\oplus}_{\substack{x\in X}}$

$\mat(\rho_g)=(a_{ij}(g))_{\scriptsize \substack{1\leq i\leq d \\ 1\leq j\leq d}}$ et $\mat(\rho'_g)=(a'_{ij}(g))_{\scriptsize \substack{1\leq i'\leq d' \\ 1\leq j'\leq d'}}$



\[\int_a^b{\mathbb{R}^2}g(u, v)\dd{P_{XY}}(u, v)=\iint g(u,v) f_{XY}(u, v)\dd \lambda(u) \dd \lambda(v)\]
$$\lim_{x\to\infty} f(x)$$	
$$\iiiint_V \mu(t,u,v,w) \,dt\,du\,dv\,dw$$
$$\sum_{n=1}^{\infty} 2^{-n} = 1$$	
\begin{definition}
	Si $X$ et $Y$ sont 2 v.a. ou definit la \textsc{Covariance} entre $X$ et $Y$ comme
	$\cov(X,Y)\overset{\text{def}}{=}\E\left[(X-\E(X))(Y-\E(Y))\right]=\E(XY)-\E(X)\E(Y)$.
\end{definition}
\fi
\pagebreak

% \tableofcontents

% insert your code here
%\input{./algebra/main.tex}
%\input{./geometrie-differentielle/main.tex}
%\input{./probabilite/main.tex}
%\input{./analyse-fonctionnelle/main.tex}
% \input{./Analyse-convexe-et-dualite-en-optimisation/main.tex}
%\input{./tikz/main.tex}
%\input{./Theorie-du-distributions/main.tex}
%\input{./optimisation/mine.tex}
 \input{./modelisation/main.tex}

% yves.aubry@univ-tln.fr : algebra

\end{document}

%% !TEX encoding = UTF-8 Unicode
% !TEX TS-program = xelatex

\documentclass[french]{report}

%\usepackage[utf8]{inputenc}
%\usepackage[T1]{fontenc}
\usepackage{babel}


\newif\ifcomment
%\commenttrue # Show comments

\usepackage{physics}
\usepackage{amssymb}


\usepackage{amsthm}
% \usepackage{thmtools}
\usepackage{mathtools}
\usepackage{amsfonts}

\usepackage{color}

\usepackage{tikz}

\usepackage{geometry}
\geometry{a5paper, margin=0.1in, right=1cm}

\usepackage{dsfont}

\usepackage{graphicx}
\graphicspath{ {images/} }

\usepackage{faktor}

\usepackage{IEEEtrantools}
\usepackage{enumerate}   
\usepackage[PostScript=dvips]{"/Users/aware/Documents/Courses/diagrams"}


\newtheorem{theorem}{Théorème}[section]
\renewcommand{\thetheorem}{\arabic{theorem}}
\newtheorem{lemme}{Lemme}[section]
\renewcommand{\thelemme}{\arabic{lemme}}
\newtheorem{proposition}{Proposition}[section]
\renewcommand{\theproposition}{\arabic{proposition}}
\newtheorem{notations}{Notations}[section]
\newtheorem{problem}{Problème}[section]
\newtheorem{corollary}{Corollaire}[theorem]
\renewcommand{\thecorollary}{\arabic{corollary}}
\newtheorem{property}{Propriété}[section]
\newtheorem{objective}{Objectif}[section]

\theoremstyle{definition}
\newtheorem{definition}{Définition}[section]
\renewcommand{\thedefinition}{\arabic{definition}}
\newtheorem{exercise}{Exercice}[chapter]
\renewcommand{\theexercise}{\arabic{exercise}}
\newtheorem{example}{Exemple}[chapter]
\renewcommand{\theexample}{\arabic{example}}
\newtheorem*{solution}{Solution}
\newtheorem*{application}{Application}
\newtheorem*{notation}{Notation}
\newtheorem*{vocabulary}{Vocabulaire}
\newtheorem*{properties}{Propriétés}



\theoremstyle{remark}
\newtheorem*{remark}{Remarque}
\newtheorem*{rappel}{Rappel}


\usepackage{etoolbox}
\AtBeginEnvironment{exercise}{\small}
\AtBeginEnvironment{example}{\small}

\usepackage{cases}
\usepackage[red]{mypack}

\usepackage[framemethod=TikZ]{mdframed}

\definecolor{bg}{rgb}{0.4,0.25,0.95}
\definecolor{pagebg}{rgb}{0,0,0.5}
\surroundwithmdframed[
   topline=false,
   rightline=false,
   bottomline=false,
   leftmargin=\parindent,
   skipabove=8pt,
   skipbelow=8pt,
   linecolor=blue,
   innerbottommargin=10pt,
   % backgroundcolor=bg,font=\color{orange}\sffamily, fontcolor=white
]{definition}

\usepackage{empheq}
\usepackage[most]{tcolorbox}

\newtcbox{\mymath}[1][]{%
    nobeforeafter, math upper, tcbox raise base,
    enhanced, colframe=blue!30!black,
    colback=red!10, boxrule=1pt,
    #1}

\usepackage{unixode}


\DeclareMathOperator{\ord}{ord}
\DeclareMathOperator{\orb}{orb}
\DeclareMathOperator{\stab}{stab}
\DeclareMathOperator{\Stab}{stab}
\DeclareMathOperator{\ppcm}{ppcm}
\DeclareMathOperator{\conj}{Conj}
\DeclareMathOperator{\End}{End}
\DeclareMathOperator{\rot}{rot}
\DeclareMathOperator{\trs}{trace}
\DeclareMathOperator{\Ind}{Ind}
\DeclareMathOperator{\mat}{Mat}
\DeclareMathOperator{\id}{Id}
\DeclareMathOperator{\vect}{vect}
\DeclareMathOperator{\img}{img}
\DeclareMathOperator{\cov}{Cov}
\DeclareMathOperator{\dist}{dist}
\DeclareMathOperator{\irr}{Irr}
\DeclareMathOperator{\image}{Im}
\DeclareMathOperator{\pd}{\partial}
\DeclareMathOperator{\epi}{epi}
\DeclareMathOperator{\Argmin}{Argmin}
\DeclareMathOperator{\dom}{dom}
\DeclareMathOperator{\proj}{proj}
\DeclareMathOperator{\ctg}{ctg}
\DeclareMathOperator{\supp}{supp}
\DeclareMathOperator{\argmin}{argmin}
\DeclareMathOperator{\mult}{mult}
\DeclareMathOperator{\ch}{ch}
\DeclareMathOperator{\sh}{sh}
\DeclareMathOperator{\rang}{rang}
\DeclareMathOperator{\diam}{diam}
\DeclareMathOperator{\Epigraphe}{Epigraphe}




\usepackage{xcolor}
\everymath{\color{blue}}
%\everymath{\color[rgb]{0,1,1}}
%\pagecolor[rgb]{0,0,0.5}


\newcommand*{\pdtest}[3][]{\ensuremath{\frac{\partial^{#1} #2}{\partial #3}}}

\newcommand*{\deffunc}[6][]{\ensuremath{
\begin{array}{rcl}
#2 : #3 &\rightarrow& #4\\
#5 &\mapsto& #6
\end{array}
}}

\newcommand{\eqcolon}{\mathrel{\resizebox{\widthof{$\mathord{=}$}}{\height}{ $\!\!=\!\!\resizebox{1.2\width}{0.8\height}{\raisebox{0.23ex}{$\mathop{:}$}}\!\!$ }}}
\newcommand{\coloneq}{\mathrel{\resizebox{\widthof{$\mathord{=}$}}{\height}{ $\!\!\resizebox{1.2\width}{0.8\height}{\raisebox{0.23ex}{$\mathop{:}$}}\!\!=\!\!$ }}}
\newcommand{\eqcolonl}{\ensuremath{\mathrel{=\!\!\mathop{:}}}}
\newcommand{\coloneql}{\ensuremath{\mathrel{\mathop{:} \!\! =}}}
\newcommand{\vc}[1]{% inline column vector
  \left(\begin{smallmatrix}#1\end{smallmatrix}\right)%
}
\newcommand{\vr}[1]{% inline row vector
  \begin{smallmatrix}(\,#1\,)\end{smallmatrix}%
}
\makeatletter
\newcommand*{\defeq}{\ =\mathrel{\rlap{%
                     \raisebox{0.3ex}{$\m@th\cdot$}}%
                     \raisebox{-0.3ex}{$\m@th\cdot$}}%
                     }
\makeatother

\newcommand{\mathcircle}[1]{% inline row vector
 \overset{\circ}{#1}
}
\newcommand{\ulim}{% low limit
 \underline{\lim}
}
\newcommand{\ssi}{% iff
\iff
}
\newcommand{\ps}[2]{
\expval{#1 | #2}
}
\newcommand{\df}[1]{
\mqty{#1}
}
\newcommand{\n}[1]{
\norm{#1}
}
\newcommand{\sys}[1]{
\left\{\smqty{#1}\right.
}


\newcommand{\eqdef}{\ensuremath{\overset{\text{def}}=}}


\def\Circlearrowright{\ensuremath{%
  \rotatebox[origin=c]{230}{$\circlearrowright$}}}

\newcommand\ct[1]{\text{\rmfamily\upshape #1}}
\newcommand\question[1]{ {\color{red} ...!? \small #1}}
\newcommand\caz[1]{\left\{\begin{array} #1 \end{array}\right.}
\newcommand\const{\text{\rmfamily\upshape const}}
\newcommand\toP{ \overset{\pro}{\to}}
\newcommand\toPP{ \overset{\text{PP}}{\to}}
\newcommand{\oeq}{\mathrel{\text{\textcircled{$=$}}}}





\usepackage{xcolor}
% \usepackage[normalem]{ulem}
\usepackage{lipsum}
\makeatletter
% \newcommand\colorwave[1][blue]{\bgroup \markoverwith{\lower3.5\p@\hbox{\sixly \textcolor{#1}{\char58}}}\ULon}
%\font\sixly=lasy6 % does not re-load if already loaded, so no memory problem.

\newmdtheoremenv[
linewidth= 1pt,linecolor= blue,%
leftmargin=20,rightmargin=20,innertopmargin=0pt, innerrightmargin=40,%
tikzsetting = { draw=lightgray, line width = 0.3pt,dashed,%
dash pattern = on 15pt off 3pt},%
splittopskip=\topskip,skipbelow=\baselineskip,%
skipabove=\baselineskip,ntheorem,roundcorner=0pt,
% backgroundcolor=pagebg,font=\color{orange}\sffamily, fontcolor=white
]{examplebox}{Exemple}[section]



\newcommand\R{\mathbb{R}}
\newcommand\Z{\mathbb{Z}}
\newcommand\N{\mathbb{N}}
\newcommand\E{\mathbb{E}}
\newcommand\F{\mathcal{F}}
\newcommand\cH{\mathcal{H}}
\newcommand\V{\mathbb{V}}
\newcommand\dmo{ ^{-1} }
\newcommand\kapa{\kappa}
\newcommand\im{Im}
\newcommand\hs{\mathcal{H}}





\usepackage{soul}

\makeatletter
\newcommand*{\whiten}[1]{\llap{\textcolor{white}{{\the\SOUL@token}}\hspace{#1pt}}}
\DeclareRobustCommand*\myul{%
    \def\SOUL@everyspace{\underline{\space}\kern\z@}%
    \def\SOUL@everytoken{%
     \setbox0=\hbox{\the\SOUL@token}%
     \ifdim\dp0>\z@
        \raisebox{\dp0}{\underline{\phantom{\the\SOUL@token}}}%
        \whiten{1}\whiten{0}%
        \whiten{-1}\whiten{-2}%
        \llap{\the\SOUL@token}%
     \else
        \underline{\the\SOUL@token}%
     \fi}%
\SOUL@}
\makeatother

\newcommand*{\demp}{\fontfamily{lmtt}\selectfont}

\DeclareTextFontCommand{\textdemp}{\demp}

\begin{document}

\ifcomment
Multiline
comment
\fi
\ifcomment
\myul{Typesetting test}
% \color[rgb]{1,1,1}
$∑_i^n≠ 60º±∞π∆¬≈√j∫h≤≥µ$

$\CR \R\pro\ind\pro\gS\pro
\mqty[a&b\\c&d]$
$\pro\mathbb{P}$
$\dd{x}$

  \[
    \alpha(x)=\left\{
                \begin{array}{ll}
                  x\\
                  \frac{1}{1+e^{-kx}}\\
                  \frac{e^x-e^{-x}}{e^x+e^{-x}}
                \end{array}
              \right.
  \]

  $\expval{x}$
  
  $\chi_\rho(ghg\dmo)=\Tr(\rho_{ghg\dmo})=\Tr(\rho_g\circ\rho_h\circ\rho\dmo_g)=\Tr(\rho_h)\overset{\mbox{\scalebox{0.5}{$\Tr(AB)=\Tr(BA)$}}}{=}\chi_\rho(h)$
  	$\mathop{\oplus}_{\substack{x\in X}}$

$\mat(\rho_g)=(a_{ij}(g))_{\scriptsize \substack{1\leq i\leq d \\ 1\leq j\leq d}}$ et $\mat(\rho'_g)=(a'_{ij}(g))_{\scriptsize \substack{1\leq i'\leq d' \\ 1\leq j'\leq d'}}$



\[\int_a^b{\mathbb{R}^2}g(u, v)\dd{P_{XY}}(u, v)=\iint g(u,v) f_{XY}(u, v)\dd \lambda(u) \dd \lambda(v)\]
$$\lim_{x\to\infty} f(x)$$	
$$\iiiint_V \mu(t,u,v,w) \,dt\,du\,dv\,dw$$
$$\sum_{n=1}^{\infty} 2^{-n} = 1$$	
\begin{definition}
	Si $X$ et $Y$ sont 2 v.a. ou definit la \textsc{Covariance} entre $X$ et $Y$ comme
	$\cov(X,Y)\overset{\text{def}}{=}\E\left[(X-\E(X))(Y-\E(Y))\right]=\E(XY)-\E(X)\E(Y)$.
\end{definition}
\fi
\pagebreak

% \tableofcontents

% insert your code here
%\input{./algebra/main.tex}
%\input{./geometrie-differentielle/main.tex}
%\input{./probabilite/main.tex}
%\input{./analyse-fonctionnelle/main.tex}
% \input{./Analyse-convexe-et-dualite-en-optimisation/main.tex}
%\input{./tikz/main.tex}
%\input{./Theorie-du-distributions/main.tex}
%\input{./optimisation/mine.tex}
 \input{./modelisation/main.tex}

% yves.aubry@univ-tln.fr : algebra

\end{document}

% % !TEX encoding = UTF-8 Unicode
% !TEX TS-program = xelatex

\documentclass[french]{report}

%\usepackage[utf8]{inputenc}
%\usepackage[T1]{fontenc}
\usepackage{babel}


\newif\ifcomment
%\commenttrue # Show comments

\usepackage{physics}
\usepackage{amssymb}


\usepackage{amsthm}
% \usepackage{thmtools}
\usepackage{mathtools}
\usepackage{amsfonts}

\usepackage{color}

\usepackage{tikz}

\usepackage{geometry}
\geometry{a5paper, margin=0.1in, right=1cm}

\usepackage{dsfont}

\usepackage{graphicx}
\graphicspath{ {images/} }

\usepackage{faktor}

\usepackage{IEEEtrantools}
\usepackage{enumerate}   
\usepackage[PostScript=dvips]{"/Users/aware/Documents/Courses/diagrams"}


\newtheorem{theorem}{Théorème}[section]
\renewcommand{\thetheorem}{\arabic{theorem}}
\newtheorem{lemme}{Lemme}[section]
\renewcommand{\thelemme}{\arabic{lemme}}
\newtheorem{proposition}{Proposition}[section]
\renewcommand{\theproposition}{\arabic{proposition}}
\newtheorem{notations}{Notations}[section]
\newtheorem{problem}{Problème}[section]
\newtheorem{corollary}{Corollaire}[theorem]
\renewcommand{\thecorollary}{\arabic{corollary}}
\newtheorem{property}{Propriété}[section]
\newtheorem{objective}{Objectif}[section]

\theoremstyle{definition}
\newtheorem{definition}{Définition}[section]
\renewcommand{\thedefinition}{\arabic{definition}}
\newtheorem{exercise}{Exercice}[chapter]
\renewcommand{\theexercise}{\arabic{exercise}}
\newtheorem{example}{Exemple}[chapter]
\renewcommand{\theexample}{\arabic{example}}
\newtheorem*{solution}{Solution}
\newtheorem*{application}{Application}
\newtheorem*{notation}{Notation}
\newtheorem*{vocabulary}{Vocabulaire}
\newtheorem*{properties}{Propriétés}



\theoremstyle{remark}
\newtheorem*{remark}{Remarque}
\newtheorem*{rappel}{Rappel}


\usepackage{etoolbox}
\AtBeginEnvironment{exercise}{\small}
\AtBeginEnvironment{example}{\small}

\usepackage{cases}
\usepackage[red]{mypack}

\usepackage[framemethod=TikZ]{mdframed}

\definecolor{bg}{rgb}{0.4,0.25,0.95}
\definecolor{pagebg}{rgb}{0,0,0.5}
\surroundwithmdframed[
   topline=false,
   rightline=false,
   bottomline=false,
   leftmargin=\parindent,
   skipabove=8pt,
   skipbelow=8pt,
   linecolor=blue,
   innerbottommargin=10pt,
   % backgroundcolor=bg,font=\color{orange}\sffamily, fontcolor=white
]{definition}

\usepackage{empheq}
\usepackage[most]{tcolorbox}

\newtcbox{\mymath}[1][]{%
    nobeforeafter, math upper, tcbox raise base,
    enhanced, colframe=blue!30!black,
    colback=red!10, boxrule=1pt,
    #1}

\usepackage{unixode}


\DeclareMathOperator{\ord}{ord}
\DeclareMathOperator{\orb}{orb}
\DeclareMathOperator{\stab}{stab}
\DeclareMathOperator{\Stab}{stab}
\DeclareMathOperator{\ppcm}{ppcm}
\DeclareMathOperator{\conj}{Conj}
\DeclareMathOperator{\End}{End}
\DeclareMathOperator{\rot}{rot}
\DeclareMathOperator{\trs}{trace}
\DeclareMathOperator{\Ind}{Ind}
\DeclareMathOperator{\mat}{Mat}
\DeclareMathOperator{\id}{Id}
\DeclareMathOperator{\vect}{vect}
\DeclareMathOperator{\img}{img}
\DeclareMathOperator{\cov}{Cov}
\DeclareMathOperator{\dist}{dist}
\DeclareMathOperator{\irr}{Irr}
\DeclareMathOperator{\image}{Im}
\DeclareMathOperator{\pd}{\partial}
\DeclareMathOperator{\epi}{epi}
\DeclareMathOperator{\Argmin}{Argmin}
\DeclareMathOperator{\dom}{dom}
\DeclareMathOperator{\proj}{proj}
\DeclareMathOperator{\ctg}{ctg}
\DeclareMathOperator{\supp}{supp}
\DeclareMathOperator{\argmin}{argmin}
\DeclareMathOperator{\mult}{mult}
\DeclareMathOperator{\ch}{ch}
\DeclareMathOperator{\sh}{sh}
\DeclareMathOperator{\rang}{rang}
\DeclareMathOperator{\diam}{diam}
\DeclareMathOperator{\Epigraphe}{Epigraphe}




\usepackage{xcolor}
\everymath{\color{blue}}
%\everymath{\color[rgb]{0,1,1}}
%\pagecolor[rgb]{0,0,0.5}


\newcommand*{\pdtest}[3][]{\ensuremath{\frac{\partial^{#1} #2}{\partial #3}}}

\newcommand*{\deffunc}[6][]{\ensuremath{
\begin{array}{rcl}
#2 : #3 &\rightarrow& #4\\
#5 &\mapsto& #6
\end{array}
}}

\newcommand{\eqcolon}{\mathrel{\resizebox{\widthof{$\mathord{=}$}}{\height}{ $\!\!=\!\!\resizebox{1.2\width}{0.8\height}{\raisebox{0.23ex}{$\mathop{:}$}}\!\!$ }}}
\newcommand{\coloneq}{\mathrel{\resizebox{\widthof{$\mathord{=}$}}{\height}{ $\!\!\resizebox{1.2\width}{0.8\height}{\raisebox{0.23ex}{$\mathop{:}$}}\!\!=\!\!$ }}}
\newcommand{\eqcolonl}{\ensuremath{\mathrel{=\!\!\mathop{:}}}}
\newcommand{\coloneql}{\ensuremath{\mathrel{\mathop{:} \!\! =}}}
\newcommand{\vc}[1]{% inline column vector
  \left(\begin{smallmatrix}#1\end{smallmatrix}\right)%
}
\newcommand{\vr}[1]{% inline row vector
  \begin{smallmatrix}(\,#1\,)\end{smallmatrix}%
}
\makeatletter
\newcommand*{\defeq}{\ =\mathrel{\rlap{%
                     \raisebox{0.3ex}{$\m@th\cdot$}}%
                     \raisebox{-0.3ex}{$\m@th\cdot$}}%
                     }
\makeatother

\newcommand{\mathcircle}[1]{% inline row vector
 \overset{\circ}{#1}
}
\newcommand{\ulim}{% low limit
 \underline{\lim}
}
\newcommand{\ssi}{% iff
\iff
}
\newcommand{\ps}[2]{
\expval{#1 | #2}
}
\newcommand{\df}[1]{
\mqty{#1}
}
\newcommand{\n}[1]{
\norm{#1}
}
\newcommand{\sys}[1]{
\left\{\smqty{#1}\right.
}


\newcommand{\eqdef}{\ensuremath{\overset{\text{def}}=}}


\def\Circlearrowright{\ensuremath{%
  \rotatebox[origin=c]{230}{$\circlearrowright$}}}

\newcommand\ct[1]{\text{\rmfamily\upshape #1}}
\newcommand\question[1]{ {\color{red} ...!? \small #1}}
\newcommand\caz[1]{\left\{\begin{array} #1 \end{array}\right.}
\newcommand\const{\text{\rmfamily\upshape const}}
\newcommand\toP{ \overset{\pro}{\to}}
\newcommand\toPP{ \overset{\text{PP}}{\to}}
\newcommand{\oeq}{\mathrel{\text{\textcircled{$=$}}}}





\usepackage{xcolor}
% \usepackage[normalem]{ulem}
\usepackage{lipsum}
\makeatletter
% \newcommand\colorwave[1][blue]{\bgroup \markoverwith{\lower3.5\p@\hbox{\sixly \textcolor{#1}{\char58}}}\ULon}
%\font\sixly=lasy6 % does not re-load if already loaded, so no memory problem.

\newmdtheoremenv[
linewidth= 1pt,linecolor= blue,%
leftmargin=20,rightmargin=20,innertopmargin=0pt, innerrightmargin=40,%
tikzsetting = { draw=lightgray, line width = 0.3pt,dashed,%
dash pattern = on 15pt off 3pt},%
splittopskip=\topskip,skipbelow=\baselineskip,%
skipabove=\baselineskip,ntheorem,roundcorner=0pt,
% backgroundcolor=pagebg,font=\color{orange}\sffamily, fontcolor=white
]{examplebox}{Exemple}[section]



\newcommand\R{\mathbb{R}}
\newcommand\Z{\mathbb{Z}}
\newcommand\N{\mathbb{N}}
\newcommand\E{\mathbb{E}}
\newcommand\F{\mathcal{F}}
\newcommand\cH{\mathcal{H}}
\newcommand\V{\mathbb{V}}
\newcommand\dmo{ ^{-1} }
\newcommand\kapa{\kappa}
\newcommand\im{Im}
\newcommand\hs{\mathcal{H}}





\usepackage{soul}

\makeatletter
\newcommand*{\whiten}[1]{\llap{\textcolor{white}{{\the\SOUL@token}}\hspace{#1pt}}}
\DeclareRobustCommand*\myul{%
    \def\SOUL@everyspace{\underline{\space}\kern\z@}%
    \def\SOUL@everytoken{%
     \setbox0=\hbox{\the\SOUL@token}%
     \ifdim\dp0>\z@
        \raisebox{\dp0}{\underline{\phantom{\the\SOUL@token}}}%
        \whiten{1}\whiten{0}%
        \whiten{-1}\whiten{-2}%
        \llap{\the\SOUL@token}%
     \else
        \underline{\the\SOUL@token}%
     \fi}%
\SOUL@}
\makeatother

\newcommand*{\demp}{\fontfamily{lmtt}\selectfont}

\DeclareTextFontCommand{\textdemp}{\demp}

\begin{document}

\ifcomment
Multiline
comment
\fi
\ifcomment
\myul{Typesetting test}
% \color[rgb]{1,1,1}
$∑_i^n≠ 60º±∞π∆¬≈√j∫h≤≥µ$

$\CR \R\pro\ind\pro\gS\pro
\mqty[a&b\\c&d]$
$\pro\mathbb{P}$
$\dd{x}$

  \[
    \alpha(x)=\left\{
                \begin{array}{ll}
                  x\\
                  \frac{1}{1+e^{-kx}}\\
                  \frac{e^x-e^{-x}}{e^x+e^{-x}}
                \end{array}
              \right.
  \]

  $\expval{x}$
  
  $\chi_\rho(ghg\dmo)=\Tr(\rho_{ghg\dmo})=\Tr(\rho_g\circ\rho_h\circ\rho\dmo_g)=\Tr(\rho_h)\overset{\mbox{\scalebox{0.5}{$\Tr(AB)=\Tr(BA)$}}}{=}\chi_\rho(h)$
  	$\mathop{\oplus}_{\substack{x\in X}}$

$\mat(\rho_g)=(a_{ij}(g))_{\scriptsize \substack{1\leq i\leq d \\ 1\leq j\leq d}}$ et $\mat(\rho'_g)=(a'_{ij}(g))_{\scriptsize \substack{1\leq i'\leq d' \\ 1\leq j'\leq d'}}$



\[\int_a^b{\mathbb{R}^2}g(u, v)\dd{P_{XY}}(u, v)=\iint g(u,v) f_{XY}(u, v)\dd \lambda(u) \dd \lambda(v)\]
$$\lim_{x\to\infty} f(x)$$	
$$\iiiint_V \mu(t,u,v,w) \,dt\,du\,dv\,dw$$
$$\sum_{n=1}^{\infty} 2^{-n} = 1$$	
\begin{definition}
	Si $X$ et $Y$ sont 2 v.a. ou definit la \textsc{Covariance} entre $X$ et $Y$ comme
	$\cov(X,Y)\overset{\text{def}}{=}\E\left[(X-\E(X))(Y-\E(Y))\right]=\E(XY)-\E(X)\E(Y)$.
\end{definition}
\fi
\pagebreak

% \tableofcontents

% insert your code here
%\input{./algebra/main.tex}
%\input{./geometrie-differentielle/main.tex}
%\input{./probabilite/main.tex}
%\input{./analyse-fonctionnelle/main.tex}
% \input{./Analyse-convexe-et-dualite-en-optimisation/main.tex}
%\input{./tikz/main.tex}
%\input{./Theorie-du-distributions/main.tex}
%\input{./optimisation/mine.tex}
 \input{./modelisation/main.tex}

% yves.aubry@univ-tln.fr : algebra

\end{document}

%% !TEX encoding = UTF-8 Unicode
% !TEX TS-program = xelatex

\documentclass[french]{report}

%\usepackage[utf8]{inputenc}
%\usepackage[T1]{fontenc}
\usepackage{babel}


\newif\ifcomment
%\commenttrue # Show comments

\usepackage{physics}
\usepackage{amssymb}


\usepackage{amsthm}
% \usepackage{thmtools}
\usepackage{mathtools}
\usepackage{amsfonts}

\usepackage{color}

\usepackage{tikz}

\usepackage{geometry}
\geometry{a5paper, margin=0.1in, right=1cm}

\usepackage{dsfont}

\usepackage{graphicx}
\graphicspath{ {images/} }

\usepackage{faktor}

\usepackage{IEEEtrantools}
\usepackage{enumerate}   
\usepackage[PostScript=dvips]{"/Users/aware/Documents/Courses/diagrams"}


\newtheorem{theorem}{Théorème}[section]
\renewcommand{\thetheorem}{\arabic{theorem}}
\newtheorem{lemme}{Lemme}[section]
\renewcommand{\thelemme}{\arabic{lemme}}
\newtheorem{proposition}{Proposition}[section]
\renewcommand{\theproposition}{\arabic{proposition}}
\newtheorem{notations}{Notations}[section]
\newtheorem{problem}{Problème}[section]
\newtheorem{corollary}{Corollaire}[theorem]
\renewcommand{\thecorollary}{\arabic{corollary}}
\newtheorem{property}{Propriété}[section]
\newtheorem{objective}{Objectif}[section]

\theoremstyle{definition}
\newtheorem{definition}{Définition}[section]
\renewcommand{\thedefinition}{\arabic{definition}}
\newtheorem{exercise}{Exercice}[chapter]
\renewcommand{\theexercise}{\arabic{exercise}}
\newtheorem{example}{Exemple}[chapter]
\renewcommand{\theexample}{\arabic{example}}
\newtheorem*{solution}{Solution}
\newtheorem*{application}{Application}
\newtheorem*{notation}{Notation}
\newtheorem*{vocabulary}{Vocabulaire}
\newtheorem*{properties}{Propriétés}



\theoremstyle{remark}
\newtheorem*{remark}{Remarque}
\newtheorem*{rappel}{Rappel}


\usepackage{etoolbox}
\AtBeginEnvironment{exercise}{\small}
\AtBeginEnvironment{example}{\small}

\usepackage{cases}
\usepackage[red]{mypack}

\usepackage[framemethod=TikZ]{mdframed}

\definecolor{bg}{rgb}{0.4,0.25,0.95}
\definecolor{pagebg}{rgb}{0,0,0.5}
\surroundwithmdframed[
   topline=false,
   rightline=false,
   bottomline=false,
   leftmargin=\parindent,
   skipabove=8pt,
   skipbelow=8pt,
   linecolor=blue,
   innerbottommargin=10pt,
   % backgroundcolor=bg,font=\color{orange}\sffamily, fontcolor=white
]{definition}

\usepackage{empheq}
\usepackage[most]{tcolorbox}

\newtcbox{\mymath}[1][]{%
    nobeforeafter, math upper, tcbox raise base,
    enhanced, colframe=blue!30!black,
    colback=red!10, boxrule=1pt,
    #1}

\usepackage{unixode}


\DeclareMathOperator{\ord}{ord}
\DeclareMathOperator{\orb}{orb}
\DeclareMathOperator{\stab}{stab}
\DeclareMathOperator{\Stab}{stab}
\DeclareMathOperator{\ppcm}{ppcm}
\DeclareMathOperator{\conj}{Conj}
\DeclareMathOperator{\End}{End}
\DeclareMathOperator{\rot}{rot}
\DeclareMathOperator{\trs}{trace}
\DeclareMathOperator{\Ind}{Ind}
\DeclareMathOperator{\mat}{Mat}
\DeclareMathOperator{\id}{Id}
\DeclareMathOperator{\vect}{vect}
\DeclareMathOperator{\img}{img}
\DeclareMathOperator{\cov}{Cov}
\DeclareMathOperator{\dist}{dist}
\DeclareMathOperator{\irr}{Irr}
\DeclareMathOperator{\image}{Im}
\DeclareMathOperator{\pd}{\partial}
\DeclareMathOperator{\epi}{epi}
\DeclareMathOperator{\Argmin}{Argmin}
\DeclareMathOperator{\dom}{dom}
\DeclareMathOperator{\proj}{proj}
\DeclareMathOperator{\ctg}{ctg}
\DeclareMathOperator{\supp}{supp}
\DeclareMathOperator{\argmin}{argmin}
\DeclareMathOperator{\mult}{mult}
\DeclareMathOperator{\ch}{ch}
\DeclareMathOperator{\sh}{sh}
\DeclareMathOperator{\rang}{rang}
\DeclareMathOperator{\diam}{diam}
\DeclareMathOperator{\Epigraphe}{Epigraphe}




\usepackage{xcolor}
\everymath{\color{blue}}
%\everymath{\color[rgb]{0,1,1}}
%\pagecolor[rgb]{0,0,0.5}


\newcommand*{\pdtest}[3][]{\ensuremath{\frac{\partial^{#1} #2}{\partial #3}}}

\newcommand*{\deffunc}[6][]{\ensuremath{
\begin{array}{rcl}
#2 : #3 &\rightarrow& #4\\
#5 &\mapsto& #6
\end{array}
}}

\newcommand{\eqcolon}{\mathrel{\resizebox{\widthof{$\mathord{=}$}}{\height}{ $\!\!=\!\!\resizebox{1.2\width}{0.8\height}{\raisebox{0.23ex}{$\mathop{:}$}}\!\!$ }}}
\newcommand{\coloneq}{\mathrel{\resizebox{\widthof{$\mathord{=}$}}{\height}{ $\!\!\resizebox{1.2\width}{0.8\height}{\raisebox{0.23ex}{$\mathop{:}$}}\!\!=\!\!$ }}}
\newcommand{\eqcolonl}{\ensuremath{\mathrel{=\!\!\mathop{:}}}}
\newcommand{\coloneql}{\ensuremath{\mathrel{\mathop{:} \!\! =}}}
\newcommand{\vc}[1]{% inline column vector
  \left(\begin{smallmatrix}#1\end{smallmatrix}\right)%
}
\newcommand{\vr}[1]{% inline row vector
  \begin{smallmatrix}(\,#1\,)\end{smallmatrix}%
}
\makeatletter
\newcommand*{\defeq}{\ =\mathrel{\rlap{%
                     \raisebox{0.3ex}{$\m@th\cdot$}}%
                     \raisebox{-0.3ex}{$\m@th\cdot$}}%
                     }
\makeatother

\newcommand{\mathcircle}[1]{% inline row vector
 \overset{\circ}{#1}
}
\newcommand{\ulim}{% low limit
 \underline{\lim}
}
\newcommand{\ssi}{% iff
\iff
}
\newcommand{\ps}[2]{
\expval{#1 | #2}
}
\newcommand{\df}[1]{
\mqty{#1}
}
\newcommand{\n}[1]{
\norm{#1}
}
\newcommand{\sys}[1]{
\left\{\smqty{#1}\right.
}


\newcommand{\eqdef}{\ensuremath{\overset{\text{def}}=}}


\def\Circlearrowright{\ensuremath{%
  \rotatebox[origin=c]{230}{$\circlearrowright$}}}

\newcommand\ct[1]{\text{\rmfamily\upshape #1}}
\newcommand\question[1]{ {\color{red} ...!? \small #1}}
\newcommand\caz[1]{\left\{\begin{array} #1 \end{array}\right.}
\newcommand\const{\text{\rmfamily\upshape const}}
\newcommand\toP{ \overset{\pro}{\to}}
\newcommand\toPP{ \overset{\text{PP}}{\to}}
\newcommand{\oeq}{\mathrel{\text{\textcircled{$=$}}}}





\usepackage{xcolor}
% \usepackage[normalem]{ulem}
\usepackage{lipsum}
\makeatletter
% \newcommand\colorwave[1][blue]{\bgroup \markoverwith{\lower3.5\p@\hbox{\sixly \textcolor{#1}{\char58}}}\ULon}
%\font\sixly=lasy6 % does not re-load if already loaded, so no memory problem.

\newmdtheoremenv[
linewidth= 1pt,linecolor= blue,%
leftmargin=20,rightmargin=20,innertopmargin=0pt, innerrightmargin=40,%
tikzsetting = { draw=lightgray, line width = 0.3pt,dashed,%
dash pattern = on 15pt off 3pt},%
splittopskip=\topskip,skipbelow=\baselineskip,%
skipabove=\baselineskip,ntheorem,roundcorner=0pt,
% backgroundcolor=pagebg,font=\color{orange}\sffamily, fontcolor=white
]{examplebox}{Exemple}[section]



\newcommand\R{\mathbb{R}}
\newcommand\Z{\mathbb{Z}}
\newcommand\N{\mathbb{N}}
\newcommand\E{\mathbb{E}}
\newcommand\F{\mathcal{F}}
\newcommand\cH{\mathcal{H}}
\newcommand\V{\mathbb{V}}
\newcommand\dmo{ ^{-1} }
\newcommand\kapa{\kappa}
\newcommand\im{Im}
\newcommand\hs{\mathcal{H}}





\usepackage{soul}

\makeatletter
\newcommand*{\whiten}[1]{\llap{\textcolor{white}{{\the\SOUL@token}}\hspace{#1pt}}}
\DeclareRobustCommand*\myul{%
    \def\SOUL@everyspace{\underline{\space}\kern\z@}%
    \def\SOUL@everytoken{%
     \setbox0=\hbox{\the\SOUL@token}%
     \ifdim\dp0>\z@
        \raisebox{\dp0}{\underline{\phantom{\the\SOUL@token}}}%
        \whiten{1}\whiten{0}%
        \whiten{-1}\whiten{-2}%
        \llap{\the\SOUL@token}%
     \else
        \underline{\the\SOUL@token}%
     \fi}%
\SOUL@}
\makeatother

\newcommand*{\demp}{\fontfamily{lmtt}\selectfont}

\DeclareTextFontCommand{\textdemp}{\demp}

\begin{document}

\ifcomment
Multiline
comment
\fi
\ifcomment
\myul{Typesetting test}
% \color[rgb]{1,1,1}
$∑_i^n≠ 60º±∞π∆¬≈√j∫h≤≥µ$

$\CR \R\pro\ind\pro\gS\pro
\mqty[a&b\\c&d]$
$\pro\mathbb{P}$
$\dd{x}$

  \[
    \alpha(x)=\left\{
                \begin{array}{ll}
                  x\\
                  \frac{1}{1+e^{-kx}}\\
                  \frac{e^x-e^{-x}}{e^x+e^{-x}}
                \end{array}
              \right.
  \]

  $\expval{x}$
  
  $\chi_\rho(ghg\dmo)=\Tr(\rho_{ghg\dmo})=\Tr(\rho_g\circ\rho_h\circ\rho\dmo_g)=\Tr(\rho_h)\overset{\mbox{\scalebox{0.5}{$\Tr(AB)=\Tr(BA)$}}}{=}\chi_\rho(h)$
  	$\mathop{\oplus}_{\substack{x\in X}}$

$\mat(\rho_g)=(a_{ij}(g))_{\scriptsize \substack{1\leq i\leq d \\ 1\leq j\leq d}}$ et $\mat(\rho'_g)=(a'_{ij}(g))_{\scriptsize \substack{1\leq i'\leq d' \\ 1\leq j'\leq d'}}$



\[\int_a^b{\mathbb{R}^2}g(u, v)\dd{P_{XY}}(u, v)=\iint g(u,v) f_{XY}(u, v)\dd \lambda(u) \dd \lambda(v)\]
$$\lim_{x\to\infty} f(x)$$	
$$\iiiint_V \mu(t,u,v,w) \,dt\,du\,dv\,dw$$
$$\sum_{n=1}^{\infty} 2^{-n} = 1$$	
\begin{definition}
	Si $X$ et $Y$ sont 2 v.a. ou definit la \textsc{Covariance} entre $X$ et $Y$ comme
	$\cov(X,Y)\overset{\text{def}}{=}\E\left[(X-\E(X))(Y-\E(Y))\right]=\E(XY)-\E(X)\E(Y)$.
\end{definition}
\fi
\pagebreak

% \tableofcontents

% insert your code here
%\input{./algebra/main.tex}
%\input{./geometrie-differentielle/main.tex}
%\input{./probabilite/main.tex}
%\input{./analyse-fonctionnelle/main.tex}
% \input{./Analyse-convexe-et-dualite-en-optimisation/main.tex}
%\input{./tikz/main.tex}
%\input{./Theorie-du-distributions/main.tex}
%\input{./optimisation/mine.tex}
 \input{./modelisation/main.tex}

% yves.aubry@univ-tln.fr : algebra

\end{document}

%% !TEX encoding = UTF-8 Unicode
% !TEX TS-program = xelatex

\documentclass[french]{report}

%\usepackage[utf8]{inputenc}
%\usepackage[T1]{fontenc}
\usepackage{babel}


\newif\ifcomment
%\commenttrue # Show comments

\usepackage{physics}
\usepackage{amssymb}


\usepackage{amsthm}
% \usepackage{thmtools}
\usepackage{mathtools}
\usepackage{amsfonts}

\usepackage{color}

\usepackage{tikz}

\usepackage{geometry}
\geometry{a5paper, margin=0.1in, right=1cm}

\usepackage{dsfont}

\usepackage{graphicx}
\graphicspath{ {images/} }

\usepackage{faktor}

\usepackage{IEEEtrantools}
\usepackage{enumerate}   
\usepackage[PostScript=dvips]{"/Users/aware/Documents/Courses/diagrams"}


\newtheorem{theorem}{Théorème}[section]
\renewcommand{\thetheorem}{\arabic{theorem}}
\newtheorem{lemme}{Lemme}[section]
\renewcommand{\thelemme}{\arabic{lemme}}
\newtheorem{proposition}{Proposition}[section]
\renewcommand{\theproposition}{\arabic{proposition}}
\newtheorem{notations}{Notations}[section]
\newtheorem{problem}{Problème}[section]
\newtheorem{corollary}{Corollaire}[theorem]
\renewcommand{\thecorollary}{\arabic{corollary}}
\newtheorem{property}{Propriété}[section]
\newtheorem{objective}{Objectif}[section]

\theoremstyle{definition}
\newtheorem{definition}{Définition}[section]
\renewcommand{\thedefinition}{\arabic{definition}}
\newtheorem{exercise}{Exercice}[chapter]
\renewcommand{\theexercise}{\arabic{exercise}}
\newtheorem{example}{Exemple}[chapter]
\renewcommand{\theexample}{\arabic{example}}
\newtheorem*{solution}{Solution}
\newtheorem*{application}{Application}
\newtheorem*{notation}{Notation}
\newtheorem*{vocabulary}{Vocabulaire}
\newtheorem*{properties}{Propriétés}



\theoremstyle{remark}
\newtheorem*{remark}{Remarque}
\newtheorem*{rappel}{Rappel}


\usepackage{etoolbox}
\AtBeginEnvironment{exercise}{\small}
\AtBeginEnvironment{example}{\small}

\usepackage{cases}
\usepackage[red]{mypack}

\usepackage[framemethod=TikZ]{mdframed}

\definecolor{bg}{rgb}{0.4,0.25,0.95}
\definecolor{pagebg}{rgb}{0,0,0.5}
\surroundwithmdframed[
   topline=false,
   rightline=false,
   bottomline=false,
   leftmargin=\parindent,
   skipabove=8pt,
   skipbelow=8pt,
   linecolor=blue,
   innerbottommargin=10pt,
   % backgroundcolor=bg,font=\color{orange}\sffamily, fontcolor=white
]{definition}

\usepackage{empheq}
\usepackage[most]{tcolorbox}

\newtcbox{\mymath}[1][]{%
    nobeforeafter, math upper, tcbox raise base,
    enhanced, colframe=blue!30!black,
    colback=red!10, boxrule=1pt,
    #1}

\usepackage{unixode}


\DeclareMathOperator{\ord}{ord}
\DeclareMathOperator{\orb}{orb}
\DeclareMathOperator{\stab}{stab}
\DeclareMathOperator{\Stab}{stab}
\DeclareMathOperator{\ppcm}{ppcm}
\DeclareMathOperator{\conj}{Conj}
\DeclareMathOperator{\End}{End}
\DeclareMathOperator{\rot}{rot}
\DeclareMathOperator{\trs}{trace}
\DeclareMathOperator{\Ind}{Ind}
\DeclareMathOperator{\mat}{Mat}
\DeclareMathOperator{\id}{Id}
\DeclareMathOperator{\vect}{vect}
\DeclareMathOperator{\img}{img}
\DeclareMathOperator{\cov}{Cov}
\DeclareMathOperator{\dist}{dist}
\DeclareMathOperator{\irr}{Irr}
\DeclareMathOperator{\image}{Im}
\DeclareMathOperator{\pd}{\partial}
\DeclareMathOperator{\epi}{epi}
\DeclareMathOperator{\Argmin}{Argmin}
\DeclareMathOperator{\dom}{dom}
\DeclareMathOperator{\proj}{proj}
\DeclareMathOperator{\ctg}{ctg}
\DeclareMathOperator{\supp}{supp}
\DeclareMathOperator{\argmin}{argmin}
\DeclareMathOperator{\mult}{mult}
\DeclareMathOperator{\ch}{ch}
\DeclareMathOperator{\sh}{sh}
\DeclareMathOperator{\rang}{rang}
\DeclareMathOperator{\diam}{diam}
\DeclareMathOperator{\Epigraphe}{Epigraphe}




\usepackage{xcolor}
\everymath{\color{blue}}
%\everymath{\color[rgb]{0,1,1}}
%\pagecolor[rgb]{0,0,0.5}


\newcommand*{\pdtest}[3][]{\ensuremath{\frac{\partial^{#1} #2}{\partial #3}}}

\newcommand*{\deffunc}[6][]{\ensuremath{
\begin{array}{rcl}
#2 : #3 &\rightarrow& #4\\
#5 &\mapsto& #6
\end{array}
}}

\newcommand{\eqcolon}{\mathrel{\resizebox{\widthof{$\mathord{=}$}}{\height}{ $\!\!=\!\!\resizebox{1.2\width}{0.8\height}{\raisebox{0.23ex}{$\mathop{:}$}}\!\!$ }}}
\newcommand{\coloneq}{\mathrel{\resizebox{\widthof{$\mathord{=}$}}{\height}{ $\!\!\resizebox{1.2\width}{0.8\height}{\raisebox{0.23ex}{$\mathop{:}$}}\!\!=\!\!$ }}}
\newcommand{\eqcolonl}{\ensuremath{\mathrel{=\!\!\mathop{:}}}}
\newcommand{\coloneql}{\ensuremath{\mathrel{\mathop{:} \!\! =}}}
\newcommand{\vc}[1]{% inline column vector
  \left(\begin{smallmatrix}#1\end{smallmatrix}\right)%
}
\newcommand{\vr}[1]{% inline row vector
  \begin{smallmatrix}(\,#1\,)\end{smallmatrix}%
}
\makeatletter
\newcommand*{\defeq}{\ =\mathrel{\rlap{%
                     \raisebox{0.3ex}{$\m@th\cdot$}}%
                     \raisebox{-0.3ex}{$\m@th\cdot$}}%
                     }
\makeatother

\newcommand{\mathcircle}[1]{% inline row vector
 \overset{\circ}{#1}
}
\newcommand{\ulim}{% low limit
 \underline{\lim}
}
\newcommand{\ssi}{% iff
\iff
}
\newcommand{\ps}[2]{
\expval{#1 | #2}
}
\newcommand{\df}[1]{
\mqty{#1}
}
\newcommand{\n}[1]{
\norm{#1}
}
\newcommand{\sys}[1]{
\left\{\smqty{#1}\right.
}


\newcommand{\eqdef}{\ensuremath{\overset{\text{def}}=}}


\def\Circlearrowright{\ensuremath{%
  \rotatebox[origin=c]{230}{$\circlearrowright$}}}

\newcommand\ct[1]{\text{\rmfamily\upshape #1}}
\newcommand\question[1]{ {\color{red} ...!? \small #1}}
\newcommand\caz[1]{\left\{\begin{array} #1 \end{array}\right.}
\newcommand\const{\text{\rmfamily\upshape const}}
\newcommand\toP{ \overset{\pro}{\to}}
\newcommand\toPP{ \overset{\text{PP}}{\to}}
\newcommand{\oeq}{\mathrel{\text{\textcircled{$=$}}}}





\usepackage{xcolor}
% \usepackage[normalem]{ulem}
\usepackage{lipsum}
\makeatletter
% \newcommand\colorwave[1][blue]{\bgroup \markoverwith{\lower3.5\p@\hbox{\sixly \textcolor{#1}{\char58}}}\ULon}
%\font\sixly=lasy6 % does not re-load if already loaded, so no memory problem.

\newmdtheoremenv[
linewidth= 1pt,linecolor= blue,%
leftmargin=20,rightmargin=20,innertopmargin=0pt, innerrightmargin=40,%
tikzsetting = { draw=lightgray, line width = 0.3pt,dashed,%
dash pattern = on 15pt off 3pt},%
splittopskip=\topskip,skipbelow=\baselineskip,%
skipabove=\baselineskip,ntheorem,roundcorner=0pt,
% backgroundcolor=pagebg,font=\color{orange}\sffamily, fontcolor=white
]{examplebox}{Exemple}[section]



\newcommand\R{\mathbb{R}}
\newcommand\Z{\mathbb{Z}}
\newcommand\N{\mathbb{N}}
\newcommand\E{\mathbb{E}}
\newcommand\F{\mathcal{F}}
\newcommand\cH{\mathcal{H}}
\newcommand\V{\mathbb{V}}
\newcommand\dmo{ ^{-1} }
\newcommand\kapa{\kappa}
\newcommand\im{Im}
\newcommand\hs{\mathcal{H}}





\usepackage{soul}

\makeatletter
\newcommand*{\whiten}[1]{\llap{\textcolor{white}{{\the\SOUL@token}}\hspace{#1pt}}}
\DeclareRobustCommand*\myul{%
    \def\SOUL@everyspace{\underline{\space}\kern\z@}%
    \def\SOUL@everytoken{%
     \setbox0=\hbox{\the\SOUL@token}%
     \ifdim\dp0>\z@
        \raisebox{\dp0}{\underline{\phantom{\the\SOUL@token}}}%
        \whiten{1}\whiten{0}%
        \whiten{-1}\whiten{-2}%
        \llap{\the\SOUL@token}%
     \else
        \underline{\the\SOUL@token}%
     \fi}%
\SOUL@}
\makeatother

\newcommand*{\demp}{\fontfamily{lmtt}\selectfont}

\DeclareTextFontCommand{\textdemp}{\demp}

\begin{document}

\ifcomment
Multiline
comment
\fi
\ifcomment
\myul{Typesetting test}
% \color[rgb]{1,1,1}
$∑_i^n≠ 60º±∞π∆¬≈√j∫h≤≥µ$

$\CR \R\pro\ind\pro\gS\pro
\mqty[a&b\\c&d]$
$\pro\mathbb{P}$
$\dd{x}$

  \[
    \alpha(x)=\left\{
                \begin{array}{ll}
                  x\\
                  \frac{1}{1+e^{-kx}}\\
                  \frac{e^x-e^{-x}}{e^x+e^{-x}}
                \end{array}
              \right.
  \]

  $\expval{x}$
  
  $\chi_\rho(ghg\dmo)=\Tr(\rho_{ghg\dmo})=\Tr(\rho_g\circ\rho_h\circ\rho\dmo_g)=\Tr(\rho_h)\overset{\mbox{\scalebox{0.5}{$\Tr(AB)=\Tr(BA)$}}}{=}\chi_\rho(h)$
  	$\mathop{\oplus}_{\substack{x\in X}}$

$\mat(\rho_g)=(a_{ij}(g))_{\scriptsize \substack{1\leq i\leq d \\ 1\leq j\leq d}}$ et $\mat(\rho'_g)=(a'_{ij}(g))_{\scriptsize \substack{1\leq i'\leq d' \\ 1\leq j'\leq d'}}$



\[\int_a^b{\mathbb{R}^2}g(u, v)\dd{P_{XY}}(u, v)=\iint g(u,v) f_{XY}(u, v)\dd \lambda(u) \dd \lambda(v)\]
$$\lim_{x\to\infty} f(x)$$	
$$\iiiint_V \mu(t,u,v,w) \,dt\,du\,dv\,dw$$
$$\sum_{n=1}^{\infty} 2^{-n} = 1$$	
\begin{definition}
	Si $X$ et $Y$ sont 2 v.a. ou definit la \textsc{Covariance} entre $X$ et $Y$ comme
	$\cov(X,Y)\overset{\text{def}}{=}\E\left[(X-\E(X))(Y-\E(Y))\right]=\E(XY)-\E(X)\E(Y)$.
\end{definition}
\fi
\pagebreak

% \tableofcontents

% insert your code here
%\input{./algebra/main.tex}
%\input{./geometrie-differentielle/main.tex}
%\input{./probabilite/main.tex}
%\input{./analyse-fonctionnelle/main.tex}
% \input{./Analyse-convexe-et-dualite-en-optimisation/main.tex}
%\input{./tikz/main.tex}
%\input{./Theorie-du-distributions/main.tex}
%\input{./optimisation/mine.tex}
 \input{./modelisation/main.tex}

% yves.aubry@univ-tln.fr : algebra

\end{document}

%\input{./optimisation/mine.tex}
 % !TEX encoding = UTF-8 Unicode
% !TEX TS-program = xelatex

\documentclass[french]{report}

%\usepackage[utf8]{inputenc}
%\usepackage[T1]{fontenc}
\usepackage{babel}


\newif\ifcomment
%\commenttrue # Show comments

\usepackage{physics}
\usepackage{amssymb}


\usepackage{amsthm}
% \usepackage{thmtools}
\usepackage{mathtools}
\usepackage{amsfonts}

\usepackage{color}

\usepackage{tikz}

\usepackage{geometry}
\geometry{a5paper, margin=0.1in, right=1cm}

\usepackage{dsfont}

\usepackage{graphicx}
\graphicspath{ {images/} }

\usepackage{faktor}

\usepackage{IEEEtrantools}
\usepackage{enumerate}   
\usepackage[PostScript=dvips]{"/Users/aware/Documents/Courses/diagrams"}


\newtheorem{theorem}{Théorème}[section]
\renewcommand{\thetheorem}{\arabic{theorem}}
\newtheorem{lemme}{Lemme}[section]
\renewcommand{\thelemme}{\arabic{lemme}}
\newtheorem{proposition}{Proposition}[section]
\renewcommand{\theproposition}{\arabic{proposition}}
\newtheorem{notations}{Notations}[section]
\newtheorem{problem}{Problème}[section]
\newtheorem{corollary}{Corollaire}[theorem]
\renewcommand{\thecorollary}{\arabic{corollary}}
\newtheorem{property}{Propriété}[section]
\newtheorem{objective}{Objectif}[section]

\theoremstyle{definition}
\newtheorem{definition}{Définition}[section]
\renewcommand{\thedefinition}{\arabic{definition}}
\newtheorem{exercise}{Exercice}[chapter]
\renewcommand{\theexercise}{\arabic{exercise}}
\newtheorem{example}{Exemple}[chapter]
\renewcommand{\theexample}{\arabic{example}}
\newtheorem*{solution}{Solution}
\newtheorem*{application}{Application}
\newtheorem*{notation}{Notation}
\newtheorem*{vocabulary}{Vocabulaire}
\newtheorem*{properties}{Propriétés}



\theoremstyle{remark}
\newtheorem*{remark}{Remarque}
\newtheorem*{rappel}{Rappel}


\usepackage{etoolbox}
\AtBeginEnvironment{exercise}{\small}
\AtBeginEnvironment{example}{\small}

\usepackage{cases}
\usepackage[red]{mypack}

\usepackage[framemethod=TikZ]{mdframed}

\definecolor{bg}{rgb}{0.4,0.25,0.95}
\definecolor{pagebg}{rgb}{0,0,0.5}
\surroundwithmdframed[
   topline=false,
   rightline=false,
   bottomline=false,
   leftmargin=\parindent,
   skipabove=8pt,
   skipbelow=8pt,
   linecolor=blue,
   innerbottommargin=10pt,
   % backgroundcolor=bg,font=\color{orange}\sffamily, fontcolor=white
]{definition}

\usepackage{empheq}
\usepackage[most]{tcolorbox}

\newtcbox{\mymath}[1][]{%
    nobeforeafter, math upper, tcbox raise base,
    enhanced, colframe=blue!30!black,
    colback=red!10, boxrule=1pt,
    #1}

\usepackage{unixode}


\DeclareMathOperator{\ord}{ord}
\DeclareMathOperator{\orb}{orb}
\DeclareMathOperator{\stab}{stab}
\DeclareMathOperator{\Stab}{stab}
\DeclareMathOperator{\ppcm}{ppcm}
\DeclareMathOperator{\conj}{Conj}
\DeclareMathOperator{\End}{End}
\DeclareMathOperator{\rot}{rot}
\DeclareMathOperator{\trs}{trace}
\DeclareMathOperator{\Ind}{Ind}
\DeclareMathOperator{\mat}{Mat}
\DeclareMathOperator{\id}{Id}
\DeclareMathOperator{\vect}{vect}
\DeclareMathOperator{\img}{img}
\DeclareMathOperator{\cov}{Cov}
\DeclareMathOperator{\dist}{dist}
\DeclareMathOperator{\irr}{Irr}
\DeclareMathOperator{\image}{Im}
\DeclareMathOperator{\pd}{\partial}
\DeclareMathOperator{\epi}{epi}
\DeclareMathOperator{\Argmin}{Argmin}
\DeclareMathOperator{\dom}{dom}
\DeclareMathOperator{\proj}{proj}
\DeclareMathOperator{\ctg}{ctg}
\DeclareMathOperator{\supp}{supp}
\DeclareMathOperator{\argmin}{argmin}
\DeclareMathOperator{\mult}{mult}
\DeclareMathOperator{\ch}{ch}
\DeclareMathOperator{\sh}{sh}
\DeclareMathOperator{\rang}{rang}
\DeclareMathOperator{\diam}{diam}
\DeclareMathOperator{\Epigraphe}{Epigraphe}




\usepackage{xcolor}
\everymath{\color{blue}}
%\everymath{\color[rgb]{0,1,1}}
%\pagecolor[rgb]{0,0,0.5}


\newcommand*{\pdtest}[3][]{\ensuremath{\frac{\partial^{#1} #2}{\partial #3}}}

\newcommand*{\deffunc}[6][]{\ensuremath{
\begin{array}{rcl}
#2 : #3 &\rightarrow& #4\\
#5 &\mapsto& #6
\end{array}
}}

\newcommand{\eqcolon}{\mathrel{\resizebox{\widthof{$\mathord{=}$}}{\height}{ $\!\!=\!\!\resizebox{1.2\width}{0.8\height}{\raisebox{0.23ex}{$\mathop{:}$}}\!\!$ }}}
\newcommand{\coloneq}{\mathrel{\resizebox{\widthof{$\mathord{=}$}}{\height}{ $\!\!\resizebox{1.2\width}{0.8\height}{\raisebox{0.23ex}{$\mathop{:}$}}\!\!=\!\!$ }}}
\newcommand{\eqcolonl}{\ensuremath{\mathrel{=\!\!\mathop{:}}}}
\newcommand{\coloneql}{\ensuremath{\mathrel{\mathop{:} \!\! =}}}
\newcommand{\vc}[1]{% inline column vector
  \left(\begin{smallmatrix}#1\end{smallmatrix}\right)%
}
\newcommand{\vr}[1]{% inline row vector
  \begin{smallmatrix}(\,#1\,)\end{smallmatrix}%
}
\makeatletter
\newcommand*{\defeq}{\ =\mathrel{\rlap{%
                     \raisebox{0.3ex}{$\m@th\cdot$}}%
                     \raisebox{-0.3ex}{$\m@th\cdot$}}%
                     }
\makeatother

\newcommand{\mathcircle}[1]{% inline row vector
 \overset{\circ}{#1}
}
\newcommand{\ulim}{% low limit
 \underline{\lim}
}
\newcommand{\ssi}{% iff
\iff
}
\newcommand{\ps}[2]{
\expval{#1 | #2}
}
\newcommand{\df}[1]{
\mqty{#1}
}
\newcommand{\n}[1]{
\norm{#1}
}
\newcommand{\sys}[1]{
\left\{\smqty{#1}\right.
}


\newcommand{\eqdef}{\ensuremath{\overset{\text{def}}=}}


\def\Circlearrowright{\ensuremath{%
  \rotatebox[origin=c]{230}{$\circlearrowright$}}}

\newcommand\ct[1]{\text{\rmfamily\upshape #1}}
\newcommand\question[1]{ {\color{red} ...!? \small #1}}
\newcommand\caz[1]{\left\{\begin{array} #1 \end{array}\right.}
\newcommand\const{\text{\rmfamily\upshape const}}
\newcommand\toP{ \overset{\pro}{\to}}
\newcommand\toPP{ \overset{\text{PP}}{\to}}
\newcommand{\oeq}{\mathrel{\text{\textcircled{$=$}}}}





\usepackage{xcolor}
% \usepackage[normalem]{ulem}
\usepackage{lipsum}
\makeatletter
% \newcommand\colorwave[1][blue]{\bgroup \markoverwith{\lower3.5\p@\hbox{\sixly \textcolor{#1}{\char58}}}\ULon}
%\font\sixly=lasy6 % does not re-load if already loaded, so no memory problem.

\newmdtheoremenv[
linewidth= 1pt,linecolor= blue,%
leftmargin=20,rightmargin=20,innertopmargin=0pt, innerrightmargin=40,%
tikzsetting = { draw=lightgray, line width = 0.3pt,dashed,%
dash pattern = on 15pt off 3pt},%
splittopskip=\topskip,skipbelow=\baselineskip,%
skipabove=\baselineskip,ntheorem,roundcorner=0pt,
% backgroundcolor=pagebg,font=\color{orange}\sffamily, fontcolor=white
]{examplebox}{Exemple}[section]



\newcommand\R{\mathbb{R}}
\newcommand\Z{\mathbb{Z}}
\newcommand\N{\mathbb{N}}
\newcommand\E{\mathbb{E}}
\newcommand\F{\mathcal{F}}
\newcommand\cH{\mathcal{H}}
\newcommand\V{\mathbb{V}}
\newcommand\dmo{ ^{-1} }
\newcommand\kapa{\kappa}
\newcommand\im{Im}
\newcommand\hs{\mathcal{H}}





\usepackage{soul}

\makeatletter
\newcommand*{\whiten}[1]{\llap{\textcolor{white}{{\the\SOUL@token}}\hspace{#1pt}}}
\DeclareRobustCommand*\myul{%
    \def\SOUL@everyspace{\underline{\space}\kern\z@}%
    \def\SOUL@everytoken{%
     \setbox0=\hbox{\the\SOUL@token}%
     \ifdim\dp0>\z@
        \raisebox{\dp0}{\underline{\phantom{\the\SOUL@token}}}%
        \whiten{1}\whiten{0}%
        \whiten{-1}\whiten{-2}%
        \llap{\the\SOUL@token}%
     \else
        \underline{\the\SOUL@token}%
     \fi}%
\SOUL@}
\makeatother

\newcommand*{\demp}{\fontfamily{lmtt}\selectfont}

\DeclareTextFontCommand{\textdemp}{\demp}

\begin{document}

\ifcomment
Multiline
comment
\fi
\ifcomment
\myul{Typesetting test}
% \color[rgb]{1,1,1}
$∑_i^n≠ 60º±∞π∆¬≈√j∫h≤≥µ$

$\CR \R\pro\ind\pro\gS\pro
\mqty[a&b\\c&d]$
$\pro\mathbb{P}$
$\dd{x}$

  \[
    \alpha(x)=\left\{
                \begin{array}{ll}
                  x\\
                  \frac{1}{1+e^{-kx}}\\
                  \frac{e^x-e^{-x}}{e^x+e^{-x}}
                \end{array}
              \right.
  \]

  $\expval{x}$
  
  $\chi_\rho(ghg\dmo)=\Tr(\rho_{ghg\dmo})=\Tr(\rho_g\circ\rho_h\circ\rho\dmo_g)=\Tr(\rho_h)\overset{\mbox{\scalebox{0.5}{$\Tr(AB)=\Tr(BA)$}}}{=}\chi_\rho(h)$
  	$\mathop{\oplus}_{\substack{x\in X}}$

$\mat(\rho_g)=(a_{ij}(g))_{\scriptsize \substack{1\leq i\leq d \\ 1\leq j\leq d}}$ et $\mat(\rho'_g)=(a'_{ij}(g))_{\scriptsize \substack{1\leq i'\leq d' \\ 1\leq j'\leq d'}}$



\[\int_a^b{\mathbb{R}^2}g(u, v)\dd{P_{XY}}(u, v)=\iint g(u,v) f_{XY}(u, v)\dd \lambda(u) \dd \lambda(v)\]
$$\lim_{x\to\infty} f(x)$$	
$$\iiiint_V \mu(t,u,v,w) \,dt\,du\,dv\,dw$$
$$\sum_{n=1}^{\infty} 2^{-n} = 1$$	
\begin{definition}
	Si $X$ et $Y$ sont 2 v.a. ou definit la \textsc{Covariance} entre $X$ et $Y$ comme
	$\cov(X,Y)\overset{\text{def}}{=}\E\left[(X-\E(X))(Y-\E(Y))\right]=\E(XY)-\E(X)\E(Y)$.
\end{definition}
\fi
\pagebreak

% \tableofcontents

% insert your code here
%\input{./algebra/main.tex}
%\input{./geometrie-differentielle/main.tex}
%\input{./probabilite/main.tex}
%\input{./analyse-fonctionnelle/main.tex}
% \input{./Analyse-convexe-et-dualite-en-optimisation/main.tex}
%\input{./tikz/main.tex}
%\input{./Theorie-du-distributions/main.tex}
%\input{./optimisation/mine.tex}
 \input{./modelisation/main.tex}

% yves.aubry@univ-tln.fr : algebra

\end{document}


% yves.aubry@univ-tln.fr : algebra

\end{document}

%% !TEX encoding = UTF-8 Unicode
% !TEX TS-program = xelatex

\documentclass[french]{report}

%\usepackage[utf8]{inputenc}
%\usepackage[T1]{fontenc}
\usepackage{babel}


\newif\ifcomment
%\commenttrue # Show comments

\usepackage{physics}
\usepackage{amssymb}


\usepackage{amsthm}
% \usepackage{thmtools}
\usepackage{mathtools}
\usepackage{amsfonts}

\usepackage{color}

\usepackage{tikz}

\usepackage{geometry}
\geometry{a5paper, margin=0.1in, right=1cm}

\usepackage{dsfont}

\usepackage{graphicx}
\graphicspath{ {images/} }

\usepackage{faktor}

\usepackage{IEEEtrantools}
\usepackage{enumerate}   
\usepackage[PostScript=dvips]{"/Users/aware/Documents/Courses/diagrams"}


\newtheorem{theorem}{Théorème}[section]
\renewcommand{\thetheorem}{\arabic{theorem}}
\newtheorem{lemme}{Lemme}[section]
\renewcommand{\thelemme}{\arabic{lemme}}
\newtheorem{proposition}{Proposition}[section]
\renewcommand{\theproposition}{\arabic{proposition}}
\newtheorem{notations}{Notations}[section]
\newtheorem{problem}{Problème}[section]
\newtheorem{corollary}{Corollaire}[theorem]
\renewcommand{\thecorollary}{\arabic{corollary}}
\newtheorem{property}{Propriété}[section]
\newtheorem{objective}{Objectif}[section]

\theoremstyle{definition}
\newtheorem{definition}{Définition}[section]
\renewcommand{\thedefinition}{\arabic{definition}}
\newtheorem{exercise}{Exercice}[chapter]
\renewcommand{\theexercise}{\arabic{exercise}}
\newtheorem{example}{Exemple}[chapter]
\renewcommand{\theexample}{\arabic{example}}
\newtheorem*{solution}{Solution}
\newtheorem*{application}{Application}
\newtheorem*{notation}{Notation}
\newtheorem*{vocabulary}{Vocabulaire}
\newtheorem*{properties}{Propriétés}



\theoremstyle{remark}
\newtheorem*{remark}{Remarque}
\newtheorem*{rappel}{Rappel}


\usepackage{etoolbox}
\AtBeginEnvironment{exercise}{\small}
\AtBeginEnvironment{example}{\small}

\usepackage{cases}
\usepackage[red]{mypack}

\usepackage[framemethod=TikZ]{mdframed}

\definecolor{bg}{rgb}{0.4,0.25,0.95}
\definecolor{pagebg}{rgb}{0,0,0.5}
\surroundwithmdframed[
   topline=false,
   rightline=false,
   bottomline=false,
   leftmargin=\parindent,
   skipabove=8pt,
   skipbelow=8pt,
   linecolor=blue,
   innerbottommargin=10pt,
   % backgroundcolor=bg,font=\color{orange}\sffamily, fontcolor=white
]{definition}

\usepackage{empheq}
\usepackage[most]{tcolorbox}

\newtcbox{\mymath}[1][]{%
    nobeforeafter, math upper, tcbox raise base,
    enhanced, colframe=blue!30!black,
    colback=red!10, boxrule=1pt,
    #1}

\usepackage{unixode}


\DeclareMathOperator{\ord}{ord}
\DeclareMathOperator{\orb}{orb}
\DeclareMathOperator{\stab}{stab}
\DeclareMathOperator{\Stab}{stab}
\DeclareMathOperator{\ppcm}{ppcm}
\DeclareMathOperator{\conj}{Conj}
\DeclareMathOperator{\End}{End}
\DeclareMathOperator{\rot}{rot}
\DeclareMathOperator{\trs}{trace}
\DeclareMathOperator{\Ind}{Ind}
\DeclareMathOperator{\mat}{Mat}
\DeclareMathOperator{\id}{Id}
\DeclareMathOperator{\vect}{vect}
\DeclareMathOperator{\img}{img}
\DeclareMathOperator{\cov}{Cov}
\DeclareMathOperator{\dist}{dist}
\DeclareMathOperator{\irr}{Irr}
\DeclareMathOperator{\image}{Im}
\DeclareMathOperator{\pd}{\partial}
\DeclareMathOperator{\epi}{epi}
\DeclareMathOperator{\Argmin}{Argmin}
\DeclareMathOperator{\dom}{dom}
\DeclareMathOperator{\proj}{proj}
\DeclareMathOperator{\ctg}{ctg}
\DeclareMathOperator{\supp}{supp}
\DeclareMathOperator{\argmin}{argmin}
\DeclareMathOperator{\mult}{mult}
\DeclareMathOperator{\ch}{ch}
\DeclareMathOperator{\sh}{sh}
\DeclareMathOperator{\rang}{rang}
\DeclareMathOperator{\diam}{diam}
\DeclareMathOperator{\Epigraphe}{Epigraphe}




\usepackage{xcolor}
\everymath{\color{blue}}
%\everymath{\color[rgb]{0,1,1}}
%\pagecolor[rgb]{0,0,0.5}


\newcommand*{\pdtest}[3][]{\ensuremath{\frac{\partial^{#1} #2}{\partial #3}}}

\newcommand*{\deffunc}[6][]{\ensuremath{
\begin{array}{rcl}
#2 : #3 &\rightarrow& #4\\
#5 &\mapsto& #6
\end{array}
}}

\newcommand{\eqcolon}{\mathrel{\resizebox{\widthof{$\mathord{=}$}}{\height}{ $\!\!=\!\!\resizebox{1.2\width}{0.8\height}{\raisebox{0.23ex}{$\mathop{:}$}}\!\!$ }}}
\newcommand{\coloneq}{\mathrel{\resizebox{\widthof{$\mathord{=}$}}{\height}{ $\!\!\resizebox{1.2\width}{0.8\height}{\raisebox{0.23ex}{$\mathop{:}$}}\!\!=\!\!$ }}}
\newcommand{\eqcolonl}{\ensuremath{\mathrel{=\!\!\mathop{:}}}}
\newcommand{\coloneql}{\ensuremath{\mathrel{\mathop{:} \!\! =}}}
\newcommand{\vc}[1]{% inline column vector
  \left(\begin{smallmatrix}#1\end{smallmatrix}\right)%
}
\newcommand{\vr}[1]{% inline row vector
  \begin{smallmatrix}(\,#1\,)\end{smallmatrix}%
}
\makeatletter
\newcommand*{\defeq}{\ =\mathrel{\rlap{%
                     \raisebox{0.3ex}{$\m@th\cdot$}}%
                     \raisebox{-0.3ex}{$\m@th\cdot$}}%
                     }
\makeatother

\newcommand{\mathcircle}[1]{% inline row vector
 \overset{\circ}{#1}
}
\newcommand{\ulim}{% low limit
 \underline{\lim}
}
\newcommand{\ssi}{% iff
\iff
}
\newcommand{\ps}[2]{
\expval{#1 | #2}
}
\newcommand{\df}[1]{
\mqty{#1}
}
\newcommand{\n}[1]{
\norm{#1}
}
\newcommand{\sys}[1]{
\left\{\smqty{#1}\right.
}


\newcommand{\eqdef}{\ensuremath{\overset{\text{def}}=}}


\def\Circlearrowright{\ensuremath{%
  \rotatebox[origin=c]{230}{$\circlearrowright$}}}

\newcommand\ct[1]{\text{\rmfamily\upshape #1}}
\newcommand\question[1]{ {\color{red} ...!? \small #1}}
\newcommand\caz[1]{\left\{\begin{array} #1 \end{array}\right.}
\newcommand\const{\text{\rmfamily\upshape const}}
\newcommand\toP{ \overset{\pro}{\to}}
\newcommand\toPP{ \overset{\text{PP}}{\to}}
\newcommand{\oeq}{\mathrel{\text{\textcircled{$=$}}}}





\usepackage{xcolor}
% \usepackage[normalem]{ulem}
\usepackage{lipsum}
\makeatletter
% \newcommand\colorwave[1][blue]{\bgroup \markoverwith{\lower3.5\p@\hbox{\sixly \textcolor{#1}{\char58}}}\ULon}
%\font\sixly=lasy6 % does not re-load if already loaded, so no memory problem.

\newmdtheoremenv[
linewidth= 1pt,linecolor= blue,%
leftmargin=20,rightmargin=20,innertopmargin=0pt, innerrightmargin=40,%
tikzsetting = { draw=lightgray, line width = 0.3pt,dashed,%
dash pattern = on 15pt off 3pt},%
splittopskip=\topskip,skipbelow=\baselineskip,%
skipabove=\baselineskip,ntheorem,roundcorner=0pt,
% backgroundcolor=pagebg,font=\color{orange}\sffamily, fontcolor=white
]{examplebox}{Exemple}[section]



\newcommand\R{\mathbb{R}}
\newcommand\Z{\mathbb{Z}}
\newcommand\N{\mathbb{N}}
\newcommand\E{\mathbb{E}}
\newcommand\F{\mathcal{F}}
\newcommand\cH{\mathcal{H}}
\newcommand\V{\mathbb{V}}
\newcommand\dmo{ ^{-1} }
\newcommand\kapa{\kappa}
\newcommand\im{Im}
\newcommand\hs{\mathcal{H}}





\usepackage{soul}

\makeatletter
\newcommand*{\whiten}[1]{\llap{\textcolor{white}{{\the\SOUL@token}}\hspace{#1pt}}}
\DeclareRobustCommand*\myul{%
    \def\SOUL@everyspace{\underline{\space}\kern\z@}%
    \def\SOUL@everytoken{%
     \setbox0=\hbox{\the\SOUL@token}%
     \ifdim\dp0>\z@
        \raisebox{\dp0}{\underline{\phantom{\the\SOUL@token}}}%
        \whiten{1}\whiten{0}%
        \whiten{-1}\whiten{-2}%
        \llap{\the\SOUL@token}%
     \else
        \underline{\the\SOUL@token}%
     \fi}%
\SOUL@}
\makeatother

\newcommand*{\demp}{\fontfamily{lmtt}\selectfont}

\DeclareTextFontCommand{\textdemp}{\demp}

\begin{document}

\ifcomment
Multiline
comment
\fi
\ifcomment
\myul{Typesetting test}
% \color[rgb]{1,1,1}
$∑_i^n≠ 60º±∞π∆¬≈√j∫h≤≥µ$

$\CR \R\pro\ind\pro\gS\pro
\mqty[a&b\\c&d]$
$\pro\mathbb{P}$
$\dd{x}$

  \[
    \alpha(x)=\left\{
                \begin{array}{ll}
                  x\\
                  \frac{1}{1+e^{-kx}}\\
                  \frac{e^x-e^{-x}}{e^x+e^{-x}}
                \end{array}
              \right.
  \]

  $\expval{x}$
  
  $\chi_\rho(ghg\dmo)=\Tr(\rho_{ghg\dmo})=\Tr(\rho_g\circ\rho_h\circ\rho\dmo_g)=\Tr(\rho_h)\overset{\mbox{\scalebox{0.5}{$\Tr(AB)=\Tr(BA)$}}}{=}\chi_\rho(h)$
  	$\mathop{\oplus}_{\substack{x\in X}}$

$\mat(\rho_g)=(a_{ij}(g))_{\scriptsize \substack{1\leq i\leq d \\ 1\leq j\leq d}}$ et $\mat(\rho'_g)=(a'_{ij}(g))_{\scriptsize \substack{1\leq i'\leq d' \\ 1\leq j'\leq d'}}$



\[\int_a^b{\mathbb{R}^2}g(u, v)\dd{P_{XY}}(u, v)=\iint g(u,v) f_{XY}(u, v)\dd \lambda(u) \dd \lambda(v)\]
$$\lim_{x\to\infty} f(x)$$	
$$\iiiint_V \mu(t,u,v,w) \,dt\,du\,dv\,dw$$
$$\sum_{n=1}^{\infty} 2^{-n} = 1$$	
\begin{definition}
	Si $X$ et $Y$ sont 2 v.a. ou definit la \textsc{Covariance} entre $X$ et $Y$ comme
	$\cov(X,Y)\overset{\text{def}}{=}\E\left[(X-\E(X))(Y-\E(Y))\right]=\E(XY)-\E(X)\E(Y)$.
\end{definition}
\fi
\pagebreak

% \tableofcontents

% insert your code here
%% !TEX encoding = UTF-8 Unicode
% !TEX TS-program = xelatex

\documentclass[french]{report}

%\usepackage[utf8]{inputenc}
%\usepackage[T1]{fontenc}
\usepackage{babel}


\newif\ifcomment
%\commenttrue # Show comments

\usepackage{physics}
\usepackage{amssymb}


\usepackage{amsthm}
% \usepackage{thmtools}
\usepackage{mathtools}
\usepackage{amsfonts}

\usepackage{color}

\usepackage{tikz}

\usepackage{geometry}
\geometry{a5paper, margin=0.1in, right=1cm}

\usepackage{dsfont}

\usepackage{graphicx}
\graphicspath{ {images/} }

\usepackage{faktor}

\usepackage{IEEEtrantools}
\usepackage{enumerate}   
\usepackage[PostScript=dvips]{"/Users/aware/Documents/Courses/diagrams"}


\newtheorem{theorem}{Théorème}[section]
\renewcommand{\thetheorem}{\arabic{theorem}}
\newtheorem{lemme}{Lemme}[section]
\renewcommand{\thelemme}{\arabic{lemme}}
\newtheorem{proposition}{Proposition}[section]
\renewcommand{\theproposition}{\arabic{proposition}}
\newtheorem{notations}{Notations}[section]
\newtheorem{problem}{Problème}[section]
\newtheorem{corollary}{Corollaire}[theorem]
\renewcommand{\thecorollary}{\arabic{corollary}}
\newtheorem{property}{Propriété}[section]
\newtheorem{objective}{Objectif}[section]

\theoremstyle{definition}
\newtheorem{definition}{Définition}[section]
\renewcommand{\thedefinition}{\arabic{definition}}
\newtheorem{exercise}{Exercice}[chapter]
\renewcommand{\theexercise}{\arabic{exercise}}
\newtheorem{example}{Exemple}[chapter]
\renewcommand{\theexample}{\arabic{example}}
\newtheorem*{solution}{Solution}
\newtheorem*{application}{Application}
\newtheorem*{notation}{Notation}
\newtheorem*{vocabulary}{Vocabulaire}
\newtheorem*{properties}{Propriétés}



\theoremstyle{remark}
\newtheorem*{remark}{Remarque}
\newtheorem*{rappel}{Rappel}


\usepackage{etoolbox}
\AtBeginEnvironment{exercise}{\small}
\AtBeginEnvironment{example}{\small}

\usepackage{cases}
\usepackage[red]{mypack}

\usepackage[framemethod=TikZ]{mdframed}

\definecolor{bg}{rgb}{0.4,0.25,0.95}
\definecolor{pagebg}{rgb}{0,0,0.5}
\surroundwithmdframed[
   topline=false,
   rightline=false,
   bottomline=false,
   leftmargin=\parindent,
   skipabove=8pt,
   skipbelow=8pt,
   linecolor=blue,
   innerbottommargin=10pt,
   % backgroundcolor=bg,font=\color{orange}\sffamily, fontcolor=white
]{definition}

\usepackage{empheq}
\usepackage[most]{tcolorbox}

\newtcbox{\mymath}[1][]{%
    nobeforeafter, math upper, tcbox raise base,
    enhanced, colframe=blue!30!black,
    colback=red!10, boxrule=1pt,
    #1}

\usepackage{unixode}


\DeclareMathOperator{\ord}{ord}
\DeclareMathOperator{\orb}{orb}
\DeclareMathOperator{\stab}{stab}
\DeclareMathOperator{\Stab}{stab}
\DeclareMathOperator{\ppcm}{ppcm}
\DeclareMathOperator{\conj}{Conj}
\DeclareMathOperator{\End}{End}
\DeclareMathOperator{\rot}{rot}
\DeclareMathOperator{\trs}{trace}
\DeclareMathOperator{\Ind}{Ind}
\DeclareMathOperator{\mat}{Mat}
\DeclareMathOperator{\id}{Id}
\DeclareMathOperator{\vect}{vect}
\DeclareMathOperator{\img}{img}
\DeclareMathOperator{\cov}{Cov}
\DeclareMathOperator{\dist}{dist}
\DeclareMathOperator{\irr}{Irr}
\DeclareMathOperator{\image}{Im}
\DeclareMathOperator{\pd}{\partial}
\DeclareMathOperator{\epi}{epi}
\DeclareMathOperator{\Argmin}{Argmin}
\DeclareMathOperator{\dom}{dom}
\DeclareMathOperator{\proj}{proj}
\DeclareMathOperator{\ctg}{ctg}
\DeclareMathOperator{\supp}{supp}
\DeclareMathOperator{\argmin}{argmin}
\DeclareMathOperator{\mult}{mult}
\DeclareMathOperator{\ch}{ch}
\DeclareMathOperator{\sh}{sh}
\DeclareMathOperator{\rang}{rang}
\DeclareMathOperator{\diam}{diam}
\DeclareMathOperator{\Epigraphe}{Epigraphe}




\usepackage{xcolor}
\everymath{\color{blue}}
%\everymath{\color[rgb]{0,1,1}}
%\pagecolor[rgb]{0,0,0.5}


\newcommand*{\pdtest}[3][]{\ensuremath{\frac{\partial^{#1} #2}{\partial #3}}}

\newcommand*{\deffunc}[6][]{\ensuremath{
\begin{array}{rcl}
#2 : #3 &\rightarrow& #4\\
#5 &\mapsto& #6
\end{array}
}}

\newcommand{\eqcolon}{\mathrel{\resizebox{\widthof{$\mathord{=}$}}{\height}{ $\!\!=\!\!\resizebox{1.2\width}{0.8\height}{\raisebox{0.23ex}{$\mathop{:}$}}\!\!$ }}}
\newcommand{\coloneq}{\mathrel{\resizebox{\widthof{$\mathord{=}$}}{\height}{ $\!\!\resizebox{1.2\width}{0.8\height}{\raisebox{0.23ex}{$\mathop{:}$}}\!\!=\!\!$ }}}
\newcommand{\eqcolonl}{\ensuremath{\mathrel{=\!\!\mathop{:}}}}
\newcommand{\coloneql}{\ensuremath{\mathrel{\mathop{:} \!\! =}}}
\newcommand{\vc}[1]{% inline column vector
  \left(\begin{smallmatrix}#1\end{smallmatrix}\right)%
}
\newcommand{\vr}[1]{% inline row vector
  \begin{smallmatrix}(\,#1\,)\end{smallmatrix}%
}
\makeatletter
\newcommand*{\defeq}{\ =\mathrel{\rlap{%
                     \raisebox{0.3ex}{$\m@th\cdot$}}%
                     \raisebox{-0.3ex}{$\m@th\cdot$}}%
                     }
\makeatother

\newcommand{\mathcircle}[1]{% inline row vector
 \overset{\circ}{#1}
}
\newcommand{\ulim}{% low limit
 \underline{\lim}
}
\newcommand{\ssi}{% iff
\iff
}
\newcommand{\ps}[2]{
\expval{#1 | #2}
}
\newcommand{\df}[1]{
\mqty{#1}
}
\newcommand{\n}[1]{
\norm{#1}
}
\newcommand{\sys}[1]{
\left\{\smqty{#1}\right.
}


\newcommand{\eqdef}{\ensuremath{\overset{\text{def}}=}}


\def\Circlearrowright{\ensuremath{%
  \rotatebox[origin=c]{230}{$\circlearrowright$}}}

\newcommand\ct[1]{\text{\rmfamily\upshape #1}}
\newcommand\question[1]{ {\color{red} ...!? \small #1}}
\newcommand\caz[1]{\left\{\begin{array} #1 \end{array}\right.}
\newcommand\const{\text{\rmfamily\upshape const}}
\newcommand\toP{ \overset{\pro}{\to}}
\newcommand\toPP{ \overset{\text{PP}}{\to}}
\newcommand{\oeq}{\mathrel{\text{\textcircled{$=$}}}}





\usepackage{xcolor}
% \usepackage[normalem]{ulem}
\usepackage{lipsum}
\makeatletter
% \newcommand\colorwave[1][blue]{\bgroup \markoverwith{\lower3.5\p@\hbox{\sixly \textcolor{#1}{\char58}}}\ULon}
%\font\sixly=lasy6 % does not re-load if already loaded, so no memory problem.

\newmdtheoremenv[
linewidth= 1pt,linecolor= blue,%
leftmargin=20,rightmargin=20,innertopmargin=0pt, innerrightmargin=40,%
tikzsetting = { draw=lightgray, line width = 0.3pt,dashed,%
dash pattern = on 15pt off 3pt},%
splittopskip=\topskip,skipbelow=\baselineskip,%
skipabove=\baselineskip,ntheorem,roundcorner=0pt,
% backgroundcolor=pagebg,font=\color{orange}\sffamily, fontcolor=white
]{examplebox}{Exemple}[section]



\newcommand\R{\mathbb{R}}
\newcommand\Z{\mathbb{Z}}
\newcommand\N{\mathbb{N}}
\newcommand\E{\mathbb{E}}
\newcommand\F{\mathcal{F}}
\newcommand\cH{\mathcal{H}}
\newcommand\V{\mathbb{V}}
\newcommand\dmo{ ^{-1} }
\newcommand\kapa{\kappa}
\newcommand\im{Im}
\newcommand\hs{\mathcal{H}}





\usepackage{soul}

\makeatletter
\newcommand*{\whiten}[1]{\llap{\textcolor{white}{{\the\SOUL@token}}\hspace{#1pt}}}
\DeclareRobustCommand*\myul{%
    \def\SOUL@everyspace{\underline{\space}\kern\z@}%
    \def\SOUL@everytoken{%
     \setbox0=\hbox{\the\SOUL@token}%
     \ifdim\dp0>\z@
        \raisebox{\dp0}{\underline{\phantom{\the\SOUL@token}}}%
        \whiten{1}\whiten{0}%
        \whiten{-1}\whiten{-2}%
        \llap{\the\SOUL@token}%
     \else
        \underline{\the\SOUL@token}%
     \fi}%
\SOUL@}
\makeatother

\newcommand*{\demp}{\fontfamily{lmtt}\selectfont}

\DeclareTextFontCommand{\textdemp}{\demp}

\begin{document}

\ifcomment
Multiline
comment
\fi
\ifcomment
\myul{Typesetting test}
% \color[rgb]{1,1,1}
$∑_i^n≠ 60º±∞π∆¬≈√j∫h≤≥µ$

$\CR \R\pro\ind\pro\gS\pro
\mqty[a&b\\c&d]$
$\pro\mathbb{P}$
$\dd{x}$

  \[
    \alpha(x)=\left\{
                \begin{array}{ll}
                  x\\
                  \frac{1}{1+e^{-kx}}\\
                  \frac{e^x-e^{-x}}{e^x+e^{-x}}
                \end{array}
              \right.
  \]

  $\expval{x}$
  
  $\chi_\rho(ghg\dmo)=\Tr(\rho_{ghg\dmo})=\Tr(\rho_g\circ\rho_h\circ\rho\dmo_g)=\Tr(\rho_h)\overset{\mbox{\scalebox{0.5}{$\Tr(AB)=\Tr(BA)$}}}{=}\chi_\rho(h)$
  	$\mathop{\oplus}_{\substack{x\in X}}$

$\mat(\rho_g)=(a_{ij}(g))_{\scriptsize \substack{1\leq i\leq d \\ 1\leq j\leq d}}$ et $\mat(\rho'_g)=(a'_{ij}(g))_{\scriptsize \substack{1\leq i'\leq d' \\ 1\leq j'\leq d'}}$



\[\int_a^b{\mathbb{R}^2}g(u, v)\dd{P_{XY}}(u, v)=\iint g(u,v) f_{XY}(u, v)\dd \lambda(u) \dd \lambda(v)\]
$$\lim_{x\to\infty} f(x)$$	
$$\iiiint_V \mu(t,u,v,w) \,dt\,du\,dv\,dw$$
$$\sum_{n=1}^{\infty} 2^{-n} = 1$$	
\begin{definition}
	Si $X$ et $Y$ sont 2 v.a. ou definit la \textsc{Covariance} entre $X$ et $Y$ comme
	$\cov(X,Y)\overset{\text{def}}{=}\E\left[(X-\E(X))(Y-\E(Y))\right]=\E(XY)-\E(X)\E(Y)$.
\end{definition}
\fi
\pagebreak

% \tableofcontents

% insert your code here
%\input{./algebra/main.tex}
%\input{./geometrie-differentielle/main.tex}
%\input{./probabilite/main.tex}
%\input{./analyse-fonctionnelle/main.tex}
% \input{./Analyse-convexe-et-dualite-en-optimisation/main.tex}
%\input{./tikz/main.tex}
%\input{./Theorie-du-distributions/main.tex}
%\input{./optimisation/mine.tex}
 \input{./modelisation/main.tex}

% yves.aubry@univ-tln.fr : algebra

\end{document}

%% !TEX encoding = UTF-8 Unicode
% !TEX TS-program = xelatex

\documentclass[french]{report}

%\usepackage[utf8]{inputenc}
%\usepackage[T1]{fontenc}
\usepackage{babel}


\newif\ifcomment
%\commenttrue # Show comments

\usepackage{physics}
\usepackage{amssymb}


\usepackage{amsthm}
% \usepackage{thmtools}
\usepackage{mathtools}
\usepackage{amsfonts}

\usepackage{color}

\usepackage{tikz}

\usepackage{geometry}
\geometry{a5paper, margin=0.1in, right=1cm}

\usepackage{dsfont}

\usepackage{graphicx}
\graphicspath{ {images/} }

\usepackage{faktor}

\usepackage{IEEEtrantools}
\usepackage{enumerate}   
\usepackage[PostScript=dvips]{"/Users/aware/Documents/Courses/diagrams"}


\newtheorem{theorem}{Théorème}[section]
\renewcommand{\thetheorem}{\arabic{theorem}}
\newtheorem{lemme}{Lemme}[section]
\renewcommand{\thelemme}{\arabic{lemme}}
\newtheorem{proposition}{Proposition}[section]
\renewcommand{\theproposition}{\arabic{proposition}}
\newtheorem{notations}{Notations}[section]
\newtheorem{problem}{Problème}[section]
\newtheorem{corollary}{Corollaire}[theorem]
\renewcommand{\thecorollary}{\arabic{corollary}}
\newtheorem{property}{Propriété}[section]
\newtheorem{objective}{Objectif}[section]

\theoremstyle{definition}
\newtheorem{definition}{Définition}[section]
\renewcommand{\thedefinition}{\arabic{definition}}
\newtheorem{exercise}{Exercice}[chapter]
\renewcommand{\theexercise}{\arabic{exercise}}
\newtheorem{example}{Exemple}[chapter]
\renewcommand{\theexample}{\arabic{example}}
\newtheorem*{solution}{Solution}
\newtheorem*{application}{Application}
\newtheorem*{notation}{Notation}
\newtheorem*{vocabulary}{Vocabulaire}
\newtheorem*{properties}{Propriétés}



\theoremstyle{remark}
\newtheorem*{remark}{Remarque}
\newtheorem*{rappel}{Rappel}


\usepackage{etoolbox}
\AtBeginEnvironment{exercise}{\small}
\AtBeginEnvironment{example}{\small}

\usepackage{cases}
\usepackage[red]{mypack}

\usepackage[framemethod=TikZ]{mdframed}

\definecolor{bg}{rgb}{0.4,0.25,0.95}
\definecolor{pagebg}{rgb}{0,0,0.5}
\surroundwithmdframed[
   topline=false,
   rightline=false,
   bottomline=false,
   leftmargin=\parindent,
   skipabove=8pt,
   skipbelow=8pt,
   linecolor=blue,
   innerbottommargin=10pt,
   % backgroundcolor=bg,font=\color{orange}\sffamily, fontcolor=white
]{definition}

\usepackage{empheq}
\usepackage[most]{tcolorbox}

\newtcbox{\mymath}[1][]{%
    nobeforeafter, math upper, tcbox raise base,
    enhanced, colframe=blue!30!black,
    colback=red!10, boxrule=1pt,
    #1}

\usepackage{unixode}


\DeclareMathOperator{\ord}{ord}
\DeclareMathOperator{\orb}{orb}
\DeclareMathOperator{\stab}{stab}
\DeclareMathOperator{\Stab}{stab}
\DeclareMathOperator{\ppcm}{ppcm}
\DeclareMathOperator{\conj}{Conj}
\DeclareMathOperator{\End}{End}
\DeclareMathOperator{\rot}{rot}
\DeclareMathOperator{\trs}{trace}
\DeclareMathOperator{\Ind}{Ind}
\DeclareMathOperator{\mat}{Mat}
\DeclareMathOperator{\id}{Id}
\DeclareMathOperator{\vect}{vect}
\DeclareMathOperator{\img}{img}
\DeclareMathOperator{\cov}{Cov}
\DeclareMathOperator{\dist}{dist}
\DeclareMathOperator{\irr}{Irr}
\DeclareMathOperator{\image}{Im}
\DeclareMathOperator{\pd}{\partial}
\DeclareMathOperator{\epi}{epi}
\DeclareMathOperator{\Argmin}{Argmin}
\DeclareMathOperator{\dom}{dom}
\DeclareMathOperator{\proj}{proj}
\DeclareMathOperator{\ctg}{ctg}
\DeclareMathOperator{\supp}{supp}
\DeclareMathOperator{\argmin}{argmin}
\DeclareMathOperator{\mult}{mult}
\DeclareMathOperator{\ch}{ch}
\DeclareMathOperator{\sh}{sh}
\DeclareMathOperator{\rang}{rang}
\DeclareMathOperator{\diam}{diam}
\DeclareMathOperator{\Epigraphe}{Epigraphe}




\usepackage{xcolor}
\everymath{\color{blue}}
%\everymath{\color[rgb]{0,1,1}}
%\pagecolor[rgb]{0,0,0.5}


\newcommand*{\pdtest}[3][]{\ensuremath{\frac{\partial^{#1} #2}{\partial #3}}}

\newcommand*{\deffunc}[6][]{\ensuremath{
\begin{array}{rcl}
#2 : #3 &\rightarrow& #4\\
#5 &\mapsto& #6
\end{array}
}}

\newcommand{\eqcolon}{\mathrel{\resizebox{\widthof{$\mathord{=}$}}{\height}{ $\!\!=\!\!\resizebox{1.2\width}{0.8\height}{\raisebox{0.23ex}{$\mathop{:}$}}\!\!$ }}}
\newcommand{\coloneq}{\mathrel{\resizebox{\widthof{$\mathord{=}$}}{\height}{ $\!\!\resizebox{1.2\width}{0.8\height}{\raisebox{0.23ex}{$\mathop{:}$}}\!\!=\!\!$ }}}
\newcommand{\eqcolonl}{\ensuremath{\mathrel{=\!\!\mathop{:}}}}
\newcommand{\coloneql}{\ensuremath{\mathrel{\mathop{:} \!\! =}}}
\newcommand{\vc}[1]{% inline column vector
  \left(\begin{smallmatrix}#1\end{smallmatrix}\right)%
}
\newcommand{\vr}[1]{% inline row vector
  \begin{smallmatrix}(\,#1\,)\end{smallmatrix}%
}
\makeatletter
\newcommand*{\defeq}{\ =\mathrel{\rlap{%
                     \raisebox{0.3ex}{$\m@th\cdot$}}%
                     \raisebox{-0.3ex}{$\m@th\cdot$}}%
                     }
\makeatother

\newcommand{\mathcircle}[1]{% inline row vector
 \overset{\circ}{#1}
}
\newcommand{\ulim}{% low limit
 \underline{\lim}
}
\newcommand{\ssi}{% iff
\iff
}
\newcommand{\ps}[2]{
\expval{#1 | #2}
}
\newcommand{\df}[1]{
\mqty{#1}
}
\newcommand{\n}[1]{
\norm{#1}
}
\newcommand{\sys}[1]{
\left\{\smqty{#1}\right.
}


\newcommand{\eqdef}{\ensuremath{\overset{\text{def}}=}}


\def\Circlearrowright{\ensuremath{%
  \rotatebox[origin=c]{230}{$\circlearrowright$}}}

\newcommand\ct[1]{\text{\rmfamily\upshape #1}}
\newcommand\question[1]{ {\color{red} ...!? \small #1}}
\newcommand\caz[1]{\left\{\begin{array} #1 \end{array}\right.}
\newcommand\const{\text{\rmfamily\upshape const}}
\newcommand\toP{ \overset{\pro}{\to}}
\newcommand\toPP{ \overset{\text{PP}}{\to}}
\newcommand{\oeq}{\mathrel{\text{\textcircled{$=$}}}}





\usepackage{xcolor}
% \usepackage[normalem]{ulem}
\usepackage{lipsum}
\makeatletter
% \newcommand\colorwave[1][blue]{\bgroup \markoverwith{\lower3.5\p@\hbox{\sixly \textcolor{#1}{\char58}}}\ULon}
%\font\sixly=lasy6 % does not re-load if already loaded, so no memory problem.

\newmdtheoremenv[
linewidth= 1pt,linecolor= blue,%
leftmargin=20,rightmargin=20,innertopmargin=0pt, innerrightmargin=40,%
tikzsetting = { draw=lightgray, line width = 0.3pt,dashed,%
dash pattern = on 15pt off 3pt},%
splittopskip=\topskip,skipbelow=\baselineskip,%
skipabove=\baselineskip,ntheorem,roundcorner=0pt,
% backgroundcolor=pagebg,font=\color{orange}\sffamily, fontcolor=white
]{examplebox}{Exemple}[section]



\newcommand\R{\mathbb{R}}
\newcommand\Z{\mathbb{Z}}
\newcommand\N{\mathbb{N}}
\newcommand\E{\mathbb{E}}
\newcommand\F{\mathcal{F}}
\newcommand\cH{\mathcal{H}}
\newcommand\V{\mathbb{V}}
\newcommand\dmo{ ^{-1} }
\newcommand\kapa{\kappa}
\newcommand\im{Im}
\newcommand\hs{\mathcal{H}}





\usepackage{soul}

\makeatletter
\newcommand*{\whiten}[1]{\llap{\textcolor{white}{{\the\SOUL@token}}\hspace{#1pt}}}
\DeclareRobustCommand*\myul{%
    \def\SOUL@everyspace{\underline{\space}\kern\z@}%
    \def\SOUL@everytoken{%
     \setbox0=\hbox{\the\SOUL@token}%
     \ifdim\dp0>\z@
        \raisebox{\dp0}{\underline{\phantom{\the\SOUL@token}}}%
        \whiten{1}\whiten{0}%
        \whiten{-1}\whiten{-2}%
        \llap{\the\SOUL@token}%
     \else
        \underline{\the\SOUL@token}%
     \fi}%
\SOUL@}
\makeatother

\newcommand*{\demp}{\fontfamily{lmtt}\selectfont}

\DeclareTextFontCommand{\textdemp}{\demp}

\begin{document}

\ifcomment
Multiline
comment
\fi
\ifcomment
\myul{Typesetting test}
% \color[rgb]{1,1,1}
$∑_i^n≠ 60º±∞π∆¬≈√j∫h≤≥µ$

$\CR \R\pro\ind\pro\gS\pro
\mqty[a&b\\c&d]$
$\pro\mathbb{P}$
$\dd{x}$

  \[
    \alpha(x)=\left\{
                \begin{array}{ll}
                  x\\
                  \frac{1}{1+e^{-kx}}\\
                  \frac{e^x-e^{-x}}{e^x+e^{-x}}
                \end{array}
              \right.
  \]

  $\expval{x}$
  
  $\chi_\rho(ghg\dmo)=\Tr(\rho_{ghg\dmo})=\Tr(\rho_g\circ\rho_h\circ\rho\dmo_g)=\Tr(\rho_h)\overset{\mbox{\scalebox{0.5}{$\Tr(AB)=\Tr(BA)$}}}{=}\chi_\rho(h)$
  	$\mathop{\oplus}_{\substack{x\in X}}$

$\mat(\rho_g)=(a_{ij}(g))_{\scriptsize \substack{1\leq i\leq d \\ 1\leq j\leq d}}$ et $\mat(\rho'_g)=(a'_{ij}(g))_{\scriptsize \substack{1\leq i'\leq d' \\ 1\leq j'\leq d'}}$



\[\int_a^b{\mathbb{R}^2}g(u, v)\dd{P_{XY}}(u, v)=\iint g(u,v) f_{XY}(u, v)\dd \lambda(u) \dd \lambda(v)\]
$$\lim_{x\to\infty} f(x)$$	
$$\iiiint_V \mu(t,u,v,w) \,dt\,du\,dv\,dw$$
$$\sum_{n=1}^{\infty} 2^{-n} = 1$$	
\begin{definition}
	Si $X$ et $Y$ sont 2 v.a. ou definit la \textsc{Covariance} entre $X$ et $Y$ comme
	$\cov(X,Y)\overset{\text{def}}{=}\E\left[(X-\E(X))(Y-\E(Y))\right]=\E(XY)-\E(X)\E(Y)$.
\end{definition}
\fi
\pagebreak

% \tableofcontents

% insert your code here
%\input{./algebra/main.tex}
%\input{./geometrie-differentielle/main.tex}
%\input{./probabilite/main.tex}
%\input{./analyse-fonctionnelle/main.tex}
% \input{./Analyse-convexe-et-dualite-en-optimisation/main.tex}
%\input{./tikz/main.tex}
%\input{./Theorie-du-distributions/main.tex}
%\input{./optimisation/mine.tex}
 \input{./modelisation/main.tex}

% yves.aubry@univ-tln.fr : algebra

\end{document}

%% !TEX encoding = UTF-8 Unicode
% !TEX TS-program = xelatex

\documentclass[french]{report}

%\usepackage[utf8]{inputenc}
%\usepackage[T1]{fontenc}
\usepackage{babel}


\newif\ifcomment
%\commenttrue # Show comments

\usepackage{physics}
\usepackage{amssymb}


\usepackage{amsthm}
% \usepackage{thmtools}
\usepackage{mathtools}
\usepackage{amsfonts}

\usepackage{color}

\usepackage{tikz}

\usepackage{geometry}
\geometry{a5paper, margin=0.1in, right=1cm}

\usepackage{dsfont}

\usepackage{graphicx}
\graphicspath{ {images/} }

\usepackage{faktor}

\usepackage{IEEEtrantools}
\usepackage{enumerate}   
\usepackage[PostScript=dvips]{"/Users/aware/Documents/Courses/diagrams"}


\newtheorem{theorem}{Théorème}[section]
\renewcommand{\thetheorem}{\arabic{theorem}}
\newtheorem{lemme}{Lemme}[section]
\renewcommand{\thelemme}{\arabic{lemme}}
\newtheorem{proposition}{Proposition}[section]
\renewcommand{\theproposition}{\arabic{proposition}}
\newtheorem{notations}{Notations}[section]
\newtheorem{problem}{Problème}[section]
\newtheorem{corollary}{Corollaire}[theorem]
\renewcommand{\thecorollary}{\arabic{corollary}}
\newtheorem{property}{Propriété}[section]
\newtheorem{objective}{Objectif}[section]

\theoremstyle{definition}
\newtheorem{definition}{Définition}[section]
\renewcommand{\thedefinition}{\arabic{definition}}
\newtheorem{exercise}{Exercice}[chapter]
\renewcommand{\theexercise}{\arabic{exercise}}
\newtheorem{example}{Exemple}[chapter]
\renewcommand{\theexample}{\arabic{example}}
\newtheorem*{solution}{Solution}
\newtheorem*{application}{Application}
\newtheorem*{notation}{Notation}
\newtheorem*{vocabulary}{Vocabulaire}
\newtheorem*{properties}{Propriétés}



\theoremstyle{remark}
\newtheorem*{remark}{Remarque}
\newtheorem*{rappel}{Rappel}


\usepackage{etoolbox}
\AtBeginEnvironment{exercise}{\small}
\AtBeginEnvironment{example}{\small}

\usepackage{cases}
\usepackage[red]{mypack}

\usepackage[framemethod=TikZ]{mdframed}

\definecolor{bg}{rgb}{0.4,0.25,0.95}
\definecolor{pagebg}{rgb}{0,0,0.5}
\surroundwithmdframed[
   topline=false,
   rightline=false,
   bottomline=false,
   leftmargin=\parindent,
   skipabove=8pt,
   skipbelow=8pt,
   linecolor=blue,
   innerbottommargin=10pt,
   % backgroundcolor=bg,font=\color{orange}\sffamily, fontcolor=white
]{definition}

\usepackage{empheq}
\usepackage[most]{tcolorbox}

\newtcbox{\mymath}[1][]{%
    nobeforeafter, math upper, tcbox raise base,
    enhanced, colframe=blue!30!black,
    colback=red!10, boxrule=1pt,
    #1}

\usepackage{unixode}


\DeclareMathOperator{\ord}{ord}
\DeclareMathOperator{\orb}{orb}
\DeclareMathOperator{\stab}{stab}
\DeclareMathOperator{\Stab}{stab}
\DeclareMathOperator{\ppcm}{ppcm}
\DeclareMathOperator{\conj}{Conj}
\DeclareMathOperator{\End}{End}
\DeclareMathOperator{\rot}{rot}
\DeclareMathOperator{\trs}{trace}
\DeclareMathOperator{\Ind}{Ind}
\DeclareMathOperator{\mat}{Mat}
\DeclareMathOperator{\id}{Id}
\DeclareMathOperator{\vect}{vect}
\DeclareMathOperator{\img}{img}
\DeclareMathOperator{\cov}{Cov}
\DeclareMathOperator{\dist}{dist}
\DeclareMathOperator{\irr}{Irr}
\DeclareMathOperator{\image}{Im}
\DeclareMathOperator{\pd}{\partial}
\DeclareMathOperator{\epi}{epi}
\DeclareMathOperator{\Argmin}{Argmin}
\DeclareMathOperator{\dom}{dom}
\DeclareMathOperator{\proj}{proj}
\DeclareMathOperator{\ctg}{ctg}
\DeclareMathOperator{\supp}{supp}
\DeclareMathOperator{\argmin}{argmin}
\DeclareMathOperator{\mult}{mult}
\DeclareMathOperator{\ch}{ch}
\DeclareMathOperator{\sh}{sh}
\DeclareMathOperator{\rang}{rang}
\DeclareMathOperator{\diam}{diam}
\DeclareMathOperator{\Epigraphe}{Epigraphe}




\usepackage{xcolor}
\everymath{\color{blue}}
%\everymath{\color[rgb]{0,1,1}}
%\pagecolor[rgb]{0,0,0.5}


\newcommand*{\pdtest}[3][]{\ensuremath{\frac{\partial^{#1} #2}{\partial #3}}}

\newcommand*{\deffunc}[6][]{\ensuremath{
\begin{array}{rcl}
#2 : #3 &\rightarrow& #4\\
#5 &\mapsto& #6
\end{array}
}}

\newcommand{\eqcolon}{\mathrel{\resizebox{\widthof{$\mathord{=}$}}{\height}{ $\!\!=\!\!\resizebox{1.2\width}{0.8\height}{\raisebox{0.23ex}{$\mathop{:}$}}\!\!$ }}}
\newcommand{\coloneq}{\mathrel{\resizebox{\widthof{$\mathord{=}$}}{\height}{ $\!\!\resizebox{1.2\width}{0.8\height}{\raisebox{0.23ex}{$\mathop{:}$}}\!\!=\!\!$ }}}
\newcommand{\eqcolonl}{\ensuremath{\mathrel{=\!\!\mathop{:}}}}
\newcommand{\coloneql}{\ensuremath{\mathrel{\mathop{:} \!\! =}}}
\newcommand{\vc}[1]{% inline column vector
  \left(\begin{smallmatrix}#1\end{smallmatrix}\right)%
}
\newcommand{\vr}[1]{% inline row vector
  \begin{smallmatrix}(\,#1\,)\end{smallmatrix}%
}
\makeatletter
\newcommand*{\defeq}{\ =\mathrel{\rlap{%
                     \raisebox{0.3ex}{$\m@th\cdot$}}%
                     \raisebox{-0.3ex}{$\m@th\cdot$}}%
                     }
\makeatother

\newcommand{\mathcircle}[1]{% inline row vector
 \overset{\circ}{#1}
}
\newcommand{\ulim}{% low limit
 \underline{\lim}
}
\newcommand{\ssi}{% iff
\iff
}
\newcommand{\ps}[2]{
\expval{#1 | #2}
}
\newcommand{\df}[1]{
\mqty{#1}
}
\newcommand{\n}[1]{
\norm{#1}
}
\newcommand{\sys}[1]{
\left\{\smqty{#1}\right.
}


\newcommand{\eqdef}{\ensuremath{\overset{\text{def}}=}}


\def\Circlearrowright{\ensuremath{%
  \rotatebox[origin=c]{230}{$\circlearrowright$}}}

\newcommand\ct[1]{\text{\rmfamily\upshape #1}}
\newcommand\question[1]{ {\color{red} ...!? \small #1}}
\newcommand\caz[1]{\left\{\begin{array} #1 \end{array}\right.}
\newcommand\const{\text{\rmfamily\upshape const}}
\newcommand\toP{ \overset{\pro}{\to}}
\newcommand\toPP{ \overset{\text{PP}}{\to}}
\newcommand{\oeq}{\mathrel{\text{\textcircled{$=$}}}}





\usepackage{xcolor}
% \usepackage[normalem]{ulem}
\usepackage{lipsum}
\makeatletter
% \newcommand\colorwave[1][blue]{\bgroup \markoverwith{\lower3.5\p@\hbox{\sixly \textcolor{#1}{\char58}}}\ULon}
%\font\sixly=lasy6 % does not re-load if already loaded, so no memory problem.

\newmdtheoremenv[
linewidth= 1pt,linecolor= blue,%
leftmargin=20,rightmargin=20,innertopmargin=0pt, innerrightmargin=40,%
tikzsetting = { draw=lightgray, line width = 0.3pt,dashed,%
dash pattern = on 15pt off 3pt},%
splittopskip=\topskip,skipbelow=\baselineskip,%
skipabove=\baselineskip,ntheorem,roundcorner=0pt,
% backgroundcolor=pagebg,font=\color{orange}\sffamily, fontcolor=white
]{examplebox}{Exemple}[section]



\newcommand\R{\mathbb{R}}
\newcommand\Z{\mathbb{Z}}
\newcommand\N{\mathbb{N}}
\newcommand\E{\mathbb{E}}
\newcommand\F{\mathcal{F}}
\newcommand\cH{\mathcal{H}}
\newcommand\V{\mathbb{V}}
\newcommand\dmo{ ^{-1} }
\newcommand\kapa{\kappa}
\newcommand\im{Im}
\newcommand\hs{\mathcal{H}}





\usepackage{soul}

\makeatletter
\newcommand*{\whiten}[1]{\llap{\textcolor{white}{{\the\SOUL@token}}\hspace{#1pt}}}
\DeclareRobustCommand*\myul{%
    \def\SOUL@everyspace{\underline{\space}\kern\z@}%
    \def\SOUL@everytoken{%
     \setbox0=\hbox{\the\SOUL@token}%
     \ifdim\dp0>\z@
        \raisebox{\dp0}{\underline{\phantom{\the\SOUL@token}}}%
        \whiten{1}\whiten{0}%
        \whiten{-1}\whiten{-2}%
        \llap{\the\SOUL@token}%
     \else
        \underline{\the\SOUL@token}%
     \fi}%
\SOUL@}
\makeatother

\newcommand*{\demp}{\fontfamily{lmtt}\selectfont}

\DeclareTextFontCommand{\textdemp}{\demp}

\begin{document}

\ifcomment
Multiline
comment
\fi
\ifcomment
\myul{Typesetting test}
% \color[rgb]{1,1,1}
$∑_i^n≠ 60º±∞π∆¬≈√j∫h≤≥µ$

$\CR \R\pro\ind\pro\gS\pro
\mqty[a&b\\c&d]$
$\pro\mathbb{P}$
$\dd{x}$

  \[
    \alpha(x)=\left\{
                \begin{array}{ll}
                  x\\
                  \frac{1}{1+e^{-kx}}\\
                  \frac{e^x-e^{-x}}{e^x+e^{-x}}
                \end{array}
              \right.
  \]

  $\expval{x}$
  
  $\chi_\rho(ghg\dmo)=\Tr(\rho_{ghg\dmo})=\Tr(\rho_g\circ\rho_h\circ\rho\dmo_g)=\Tr(\rho_h)\overset{\mbox{\scalebox{0.5}{$\Tr(AB)=\Tr(BA)$}}}{=}\chi_\rho(h)$
  	$\mathop{\oplus}_{\substack{x\in X}}$

$\mat(\rho_g)=(a_{ij}(g))_{\scriptsize \substack{1\leq i\leq d \\ 1\leq j\leq d}}$ et $\mat(\rho'_g)=(a'_{ij}(g))_{\scriptsize \substack{1\leq i'\leq d' \\ 1\leq j'\leq d'}}$



\[\int_a^b{\mathbb{R}^2}g(u, v)\dd{P_{XY}}(u, v)=\iint g(u,v) f_{XY}(u, v)\dd \lambda(u) \dd \lambda(v)\]
$$\lim_{x\to\infty} f(x)$$	
$$\iiiint_V \mu(t,u,v,w) \,dt\,du\,dv\,dw$$
$$\sum_{n=1}^{\infty} 2^{-n} = 1$$	
\begin{definition}
	Si $X$ et $Y$ sont 2 v.a. ou definit la \textsc{Covariance} entre $X$ et $Y$ comme
	$\cov(X,Y)\overset{\text{def}}{=}\E\left[(X-\E(X))(Y-\E(Y))\right]=\E(XY)-\E(X)\E(Y)$.
\end{definition}
\fi
\pagebreak

% \tableofcontents

% insert your code here
%\input{./algebra/main.tex}
%\input{./geometrie-differentielle/main.tex}
%\input{./probabilite/main.tex}
%\input{./analyse-fonctionnelle/main.tex}
% \input{./Analyse-convexe-et-dualite-en-optimisation/main.tex}
%\input{./tikz/main.tex}
%\input{./Theorie-du-distributions/main.tex}
%\input{./optimisation/mine.tex}
 \input{./modelisation/main.tex}

% yves.aubry@univ-tln.fr : algebra

\end{document}

%% !TEX encoding = UTF-8 Unicode
% !TEX TS-program = xelatex

\documentclass[french]{report}

%\usepackage[utf8]{inputenc}
%\usepackage[T1]{fontenc}
\usepackage{babel}


\newif\ifcomment
%\commenttrue # Show comments

\usepackage{physics}
\usepackage{amssymb}


\usepackage{amsthm}
% \usepackage{thmtools}
\usepackage{mathtools}
\usepackage{amsfonts}

\usepackage{color}

\usepackage{tikz}

\usepackage{geometry}
\geometry{a5paper, margin=0.1in, right=1cm}

\usepackage{dsfont}

\usepackage{graphicx}
\graphicspath{ {images/} }

\usepackage{faktor}

\usepackage{IEEEtrantools}
\usepackage{enumerate}   
\usepackage[PostScript=dvips]{"/Users/aware/Documents/Courses/diagrams"}


\newtheorem{theorem}{Théorème}[section]
\renewcommand{\thetheorem}{\arabic{theorem}}
\newtheorem{lemme}{Lemme}[section]
\renewcommand{\thelemme}{\arabic{lemme}}
\newtheorem{proposition}{Proposition}[section]
\renewcommand{\theproposition}{\arabic{proposition}}
\newtheorem{notations}{Notations}[section]
\newtheorem{problem}{Problème}[section]
\newtheorem{corollary}{Corollaire}[theorem]
\renewcommand{\thecorollary}{\arabic{corollary}}
\newtheorem{property}{Propriété}[section]
\newtheorem{objective}{Objectif}[section]

\theoremstyle{definition}
\newtheorem{definition}{Définition}[section]
\renewcommand{\thedefinition}{\arabic{definition}}
\newtheorem{exercise}{Exercice}[chapter]
\renewcommand{\theexercise}{\arabic{exercise}}
\newtheorem{example}{Exemple}[chapter]
\renewcommand{\theexample}{\arabic{example}}
\newtheorem*{solution}{Solution}
\newtheorem*{application}{Application}
\newtheorem*{notation}{Notation}
\newtheorem*{vocabulary}{Vocabulaire}
\newtheorem*{properties}{Propriétés}



\theoremstyle{remark}
\newtheorem*{remark}{Remarque}
\newtheorem*{rappel}{Rappel}


\usepackage{etoolbox}
\AtBeginEnvironment{exercise}{\small}
\AtBeginEnvironment{example}{\small}

\usepackage{cases}
\usepackage[red]{mypack}

\usepackage[framemethod=TikZ]{mdframed}

\definecolor{bg}{rgb}{0.4,0.25,0.95}
\definecolor{pagebg}{rgb}{0,0,0.5}
\surroundwithmdframed[
   topline=false,
   rightline=false,
   bottomline=false,
   leftmargin=\parindent,
   skipabove=8pt,
   skipbelow=8pt,
   linecolor=blue,
   innerbottommargin=10pt,
   % backgroundcolor=bg,font=\color{orange}\sffamily, fontcolor=white
]{definition}

\usepackage{empheq}
\usepackage[most]{tcolorbox}

\newtcbox{\mymath}[1][]{%
    nobeforeafter, math upper, tcbox raise base,
    enhanced, colframe=blue!30!black,
    colback=red!10, boxrule=1pt,
    #1}

\usepackage{unixode}


\DeclareMathOperator{\ord}{ord}
\DeclareMathOperator{\orb}{orb}
\DeclareMathOperator{\stab}{stab}
\DeclareMathOperator{\Stab}{stab}
\DeclareMathOperator{\ppcm}{ppcm}
\DeclareMathOperator{\conj}{Conj}
\DeclareMathOperator{\End}{End}
\DeclareMathOperator{\rot}{rot}
\DeclareMathOperator{\trs}{trace}
\DeclareMathOperator{\Ind}{Ind}
\DeclareMathOperator{\mat}{Mat}
\DeclareMathOperator{\id}{Id}
\DeclareMathOperator{\vect}{vect}
\DeclareMathOperator{\img}{img}
\DeclareMathOperator{\cov}{Cov}
\DeclareMathOperator{\dist}{dist}
\DeclareMathOperator{\irr}{Irr}
\DeclareMathOperator{\image}{Im}
\DeclareMathOperator{\pd}{\partial}
\DeclareMathOperator{\epi}{epi}
\DeclareMathOperator{\Argmin}{Argmin}
\DeclareMathOperator{\dom}{dom}
\DeclareMathOperator{\proj}{proj}
\DeclareMathOperator{\ctg}{ctg}
\DeclareMathOperator{\supp}{supp}
\DeclareMathOperator{\argmin}{argmin}
\DeclareMathOperator{\mult}{mult}
\DeclareMathOperator{\ch}{ch}
\DeclareMathOperator{\sh}{sh}
\DeclareMathOperator{\rang}{rang}
\DeclareMathOperator{\diam}{diam}
\DeclareMathOperator{\Epigraphe}{Epigraphe}




\usepackage{xcolor}
\everymath{\color{blue}}
%\everymath{\color[rgb]{0,1,1}}
%\pagecolor[rgb]{0,0,0.5}


\newcommand*{\pdtest}[3][]{\ensuremath{\frac{\partial^{#1} #2}{\partial #3}}}

\newcommand*{\deffunc}[6][]{\ensuremath{
\begin{array}{rcl}
#2 : #3 &\rightarrow& #4\\
#5 &\mapsto& #6
\end{array}
}}

\newcommand{\eqcolon}{\mathrel{\resizebox{\widthof{$\mathord{=}$}}{\height}{ $\!\!=\!\!\resizebox{1.2\width}{0.8\height}{\raisebox{0.23ex}{$\mathop{:}$}}\!\!$ }}}
\newcommand{\coloneq}{\mathrel{\resizebox{\widthof{$\mathord{=}$}}{\height}{ $\!\!\resizebox{1.2\width}{0.8\height}{\raisebox{0.23ex}{$\mathop{:}$}}\!\!=\!\!$ }}}
\newcommand{\eqcolonl}{\ensuremath{\mathrel{=\!\!\mathop{:}}}}
\newcommand{\coloneql}{\ensuremath{\mathrel{\mathop{:} \!\! =}}}
\newcommand{\vc}[1]{% inline column vector
  \left(\begin{smallmatrix}#1\end{smallmatrix}\right)%
}
\newcommand{\vr}[1]{% inline row vector
  \begin{smallmatrix}(\,#1\,)\end{smallmatrix}%
}
\makeatletter
\newcommand*{\defeq}{\ =\mathrel{\rlap{%
                     \raisebox{0.3ex}{$\m@th\cdot$}}%
                     \raisebox{-0.3ex}{$\m@th\cdot$}}%
                     }
\makeatother

\newcommand{\mathcircle}[1]{% inline row vector
 \overset{\circ}{#1}
}
\newcommand{\ulim}{% low limit
 \underline{\lim}
}
\newcommand{\ssi}{% iff
\iff
}
\newcommand{\ps}[2]{
\expval{#1 | #2}
}
\newcommand{\df}[1]{
\mqty{#1}
}
\newcommand{\n}[1]{
\norm{#1}
}
\newcommand{\sys}[1]{
\left\{\smqty{#1}\right.
}


\newcommand{\eqdef}{\ensuremath{\overset{\text{def}}=}}


\def\Circlearrowright{\ensuremath{%
  \rotatebox[origin=c]{230}{$\circlearrowright$}}}

\newcommand\ct[1]{\text{\rmfamily\upshape #1}}
\newcommand\question[1]{ {\color{red} ...!? \small #1}}
\newcommand\caz[1]{\left\{\begin{array} #1 \end{array}\right.}
\newcommand\const{\text{\rmfamily\upshape const}}
\newcommand\toP{ \overset{\pro}{\to}}
\newcommand\toPP{ \overset{\text{PP}}{\to}}
\newcommand{\oeq}{\mathrel{\text{\textcircled{$=$}}}}





\usepackage{xcolor}
% \usepackage[normalem]{ulem}
\usepackage{lipsum}
\makeatletter
% \newcommand\colorwave[1][blue]{\bgroup \markoverwith{\lower3.5\p@\hbox{\sixly \textcolor{#1}{\char58}}}\ULon}
%\font\sixly=lasy6 % does not re-load if already loaded, so no memory problem.

\newmdtheoremenv[
linewidth= 1pt,linecolor= blue,%
leftmargin=20,rightmargin=20,innertopmargin=0pt, innerrightmargin=40,%
tikzsetting = { draw=lightgray, line width = 0.3pt,dashed,%
dash pattern = on 15pt off 3pt},%
splittopskip=\topskip,skipbelow=\baselineskip,%
skipabove=\baselineskip,ntheorem,roundcorner=0pt,
% backgroundcolor=pagebg,font=\color{orange}\sffamily, fontcolor=white
]{examplebox}{Exemple}[section]



\newcommand\R{\mathbb{R}}
\newcommand\Z{\mathbb{Z}}
\newcommand\N{\mathbb{N}}
\newcommand\E{\mathbb{E}}
\newcommand\F{\mathcal{F}}
\newcommand\cH{\mathcal{H}}
\newcommand\V{\mathbb{V}}
\newcommand\dmo{ ^{-1} }
\newcommand\kapa{\kappa}
\newcommand\im{Im}
\newcommand\hs{\mathcal{H}}





\usepackage{soul}

\makeatletter
\newcommand*{\whiten}[1]{\llap{\textcolor{white}{{\the\SOUL@token}}\hspace{#1pt}}}
\DeclareRobustCommand*\myul{%
    \def\SOUL@everyspace{\underline{\space}\kern\z@}%
    \def\SOUL@everytoken{%
     \setbox0=\hbox{\the\SOUL@token}%
     \ifdim\dp0>\z@
        \raisebox{\dp0}{\underline{\phantom{\the\SOUL@token}}}%
        \whiten{1}\whiten{0}%
        \whiten{-1}\whiten{-2}%
        \llap{\the\SOUL@token}%
     \else
        \underline{\the\SOUL@token}%
     \fi}%
\SOUL@}
\makeatother

\newcommand*{\demp}{\fontfamily{lmtt}\selectfont}

\DeclareTextFontCommand{\textdemp}{\demp}

\begin{document}

\ifcomment
Multiline
comment
\fi
\ifcomment
\myul{Typesetting test}
% \color[rgb]{1,1,1}
$∑_i^n≠ 60º±∞π∆¬≈√j∫h≤≥µ$

$\CR \R\pro\ind\pro\gS\pro
\mqty[a&b\\c&d]$
$\pro\mathbb{P}$
$\dd{x}$

  \[
    \alpha(x)=\left\{
                \begin{array}{ll}
                  x\\
                  \frac{1}{1+e^{-kx}}\\
                  \frac{e^x-e^{-x}}{e^x+e^{-x}}
                \end{array}
              \right.
  \]

  $\expval{x}$
  
  $\chi_\rho(ghg\dmo)=\Tr(\rho_{ghg\dmo})=\Tr(\rho_g\circ\rho_h\circ\rho\dmo_g)=\Tr(\rho_h)\overset{\mbox{\scalebox{0.5}{$\Tr(AB)=\Tr(BA)$}}}{=}\chi_\rho(h)$
  	$\mathop{\oplus}_{\substack{x\in X}}$

$\mat(\rho_g)=(a_{ij}(g))_{\scriptsize \substack{1\leq i\leq d \\ 1\leq j\leq d}}$ et $\mat(\rho'_g)=(a'_{ij}(g))_{\scriptsize \substack{1\leq i'\leq d' \\ 1\leq j'\leq d'}}$



\[\int_a^b{\mathbb{R}^2}g(u, v)\dd{P_{XY}}(u, v)=\iint g(u,v) f_{XY}(u, v)\dd \lambda(u) \dd \lambda(v)\]
$$\lim_{x\to\infty} f(x)$$	
$$\iiiint_V \mu(t,u,v,w) \,dt\,du\,dv\,dw$$
$$\sum_{n=1}^{\infty} 2^{-n} = 1$$	
\begin{definition}
	Si $X$ et $Y$ sont 2 v.a. ou definit la \textsc{Covariance} entre $X$ et $Y$ comme
	$\cov(X,Y)\overset{\text{def}}{=}\E\left[(X-\E(X))(Y-\E(Y))\right]=\E(XY)-\E(X)\E(Y)$.
\end{definition}
\fi
\pagebreak

% \tableofcontents

% insert your code here
%\input{./algebra/main.tex}
%\input{./geometrie-differentielle/main.tex}
%\input{./probabilite/main.tex}
%\input{./analyse-fonctionnelle/main.tex}
% \input{./Analyse-convexe-et-dualite-en-optimisation/main.tex}
%\input{./tikz/main.tex}
%\input{./Theorie-du-distributions/main.tex}
%\input{./optimisation/mine.tex}
 \input{./modelisation/main.tex}

% yves.aubry@univ-tln.fr : algebra

\end{document}

% % !TEX encoding = UTF-8 Unicode
% !TEX TS-program = xelatex

\documentclass[french]{report}

%\usepackage[utf8]{inputenc}
%\usepackage[T1]{fontenc}
\usepackage{babel}


\newif\ifcomment
%\commenttrue # Show comments

\usepackage{physics}
\usepackage{amssymb}


\usepackage{amsthm}
% \usepackage{thmtools}
\usepackage{mathtools}
\usepackage{amsfonts}

\usepackage{color}

\usepackage{tikz}

\usepackage{geometry}
\geometry{a5paper, margin=0.1in, right=1cm}

\usepackage{dsfont}

\usepackage{graphicx}
\graphicspath{ {images/} }

\usepackage{faktor}

\usepackage{IEEEtrantools}
\usepackage{enumerate}   
\usepackage[PostScript=dvips]{"/Users/aware/Documents/Courses/diagrams"}


\newtheorem{theorem}{Théorème}[section]
\renewcommand{\thetheorem}{\arabic{theorem}}
\newtheorem{lemme}{Lemme}[section]
\renewcommand{\thelemme}{\arabic{lemme}}
\newtheorem{proposition}{Proposition}[section]
\renewcommand{\theproposition}{\arabic{proposition}}
\newtheorem{notations}{Notations}[section]
\newtheorem{problem}{Problème}[section]
\newtheorem{corollary}{Corollaire}[theorem]
\renewcommand{\thecorollary}{\arabic{corollary}}
\newtheorem{property}{Propriété}[section]
\newtheorem{objective}{Objectif}[section]

\theoremstyle{definition}
\newtheorem{definition}{Définition}[section]
\renewcommand{\thedefinition}{\arabic{definition}}
\newtheorem{exercise}{Exercice}[chapter]
\renewcommand{\theexercise}{\arabic{exercise}}
\newtheorem{example}{Exemple}[chapter]
\renewcommand{\theexample}{\arabic{example}}
\newtheorem*{solution}{Solution}
\newtheorem*{application}{Application}
\newtheorem*{notation}{Notation}
\newtheorem*{vocabulary}{Vocabulaire}
\newtheorem*{properties}{Propriétés}



\theoremstyle{remark}
\newtheorem*{remark}{Remarque}
\newtheorem*{rappel}{Rappel}


\usepackage{etoolbox}
\AtBeginEnvironment{exercise}{\small}
\AtBeginEnvironment{example}{\small}

\usepackage{cases}
\usepackage[red]{mypack}

\usepackage[framemethod=TikZ]{mdframed}

\definecolor{bg}{rgb}{0.4,0.25,0.95}
\definecolor{pagebg}{rgb}{0,0,0.5}
\surroundwithmdframed[
   topline=false,
   rightline=false,
   bottomline=false,
   leftmargin=\parindent,
   skipabove=8pt,
   skipbelow=8pt,
   linecolor=blue,
   innerbottommargin=10pt,
   % backgroundcolor=bg,font=\color{orange}\sffamily, fontcolor=white
]{definition}

\usepackage{empheq}
\usepackage[most]{tcolorbox}

\newtcbox{\mymath}[1][]{%
    nobeforeafter, math upper, tcbox raise base,
    enhanced, colframe=blue!30!black,
    colback=red!10, boxrule=1pt,
    #1}

\usepackage{unixode}


\DeclareMathOperator{\ord}{ord}
\DeclareMathOperator{\orb}{orb}
\DeclareMathOperator{\stab}{stab}
\DeclareMathOperator{\Stab}{stab}
\DeclareMathOperator{\ppcm}{ppcm}
\DeclareMathOperator{\conj}{Conj}
\DeclareMathOperator{\End}{End}
\DeclareMathOperator{\rot}{rot}
\DeclareMathOperator{\trs}{trace}
\DeclareMathOperator{\Ind}{Ind}
\DeclareMathOperator{\mat}{Mat}
\DeclareMathOperator{\id}{Id}
\DeclareMathOperator{\vect}{vect}
\DeclareMathOperator{\img}{img}
\DeclareMathOperator{\cov}{Cov}
\DeclareMathOperator{\dist}{dist}
\DeclareMathOperator{\irr}{Irr}
\DeclareMathOperator{\image}{Im}
\DeclareMathOperator{\pd}{\partial}
\DeclareMathOperator{\epi}{epi}
\DeclareMathOperator{\Argmin}{Argmin}
\DeclareMathOperator{\dom}{dom}
\DeclareMathOperator{\proj}{proj}
\DeclareMathOperator{\ctg}{ctg}
\DeclareMathOperator{\supp}{supp}
\DeclareMathOperator{\argmin}{argmin}
\DeclareMathOperator{\mult}{mult}
\DeclareMathOperator{\ch}{ch}
\DeclareMathOperator{\sh}{sh}
\DeclareMathOperator{\rang}{rang}
\DeclareMathOperator{\diam}{diam}
\DeclareMathOperator{\Epigraphe}{Epigraphe}




\usepackage{xcolor}
\everymath{\color{blue}}
%\everymath{\color[rgb]{0,1,1}}
%\pagecolor[rgb]{0,0,0.5}


\newcommand*{\pdtest}[3][]{\ensuremath{\frac{\partial^{#1} #2}{\partial #3}}}

\newcommand*{\deffunc}[6][]{\ensuremath{
\begin{array}{rcl}
#2 : #3 &\rightarrow& #4\\
#5 &\mapsto& #6
\end{array}
}}

\newcommand{\eqcolon}{\mathrel{\resizebox{\widthof{$\mathord{=}$}}{\height}{ $\!\!=\!\!\resizebox{1.2\width}{0.8\height}{\raisebox{0.23ex}{$\mathop{:}$}}\!\!$ }}}
\newcommand{\coloneq}{\mathrel{\resizebox{\widthof{$\mathord{=}$}}{\height}{ $\!\!\resizebox{1.2\width}{0.8\height}{\raisebox{0.23ex}{$\mathop{:}$}}\!\!=\!\!$ }}}
\newcommand{\eqcolonl}{\ensuremath{\mathrel{=\!\!\mathop{:}}}}
\newcommand{\coloneql}{\ensuremath{\mathrel{\mathop{:} \!\! =}}}
\newcommand{\vc}[1]{% inline column vector
  \left(\begin{smallmatrix}#1\end{smallmatrix}\right)%
}
\newcommand{\vr}[1]{% inline row vector
  \begin{smallmatrix}(\,#1\,)\end{smallmatrix}%
}
\makeatletter
\newcommand*{\defeq}{\ =\mathrel{\rlap{%
                     \raisebox{0.3ex}{$\m@th\cdot$}}%
                     \raisebox{-0.3ex}{$\m@th\cdot$}}%
                     }
\makeatother

\newcommand{\mathcircle}[1]{% inline row vector
 \overset{\circ}{#1}
}
\newcommand{\ulim}{% low limit
 \underline{\lim}
}
\newcommand{\ssi}{% iff
\iff
}
\newcommand{\ps}[2]{
\expval{#1 | #2}
}
\newcommand{\df}[1]{
\mqty{#1}
}
\newcommand{\n}[1]{
\norm{#1}
}
\newcommand{\sys}[1]{
\left\{\smqty{#1}\right.
}


\newcommand{\eqdef}{\ensuremath{\overset{\text{def}}=}}


\def\Circlearrowright{\ensuremath{%
  \rotatebox[origin=c]{230}{$\circlearrowright$}}}

\newcommand\ct[1]{\text{\rmfamily\upshape #1}}
\newcommand\question[1]{ {\color{red} ...!? \small #1}}
\newcommand\caz[1]{\left\{\begin{array} #1 \end{array}\right.}
\newcommand\const{\text{\rmfamily\upshape const}}
\newcommand\toP{ \overset{\pro}{\to}}
\newcommand\toPP{ \overset{\text{PP}}{\to}}
\newcommand{\oeq}{\mathrel{\text{\textcircled{$=$}}}}





\usepackage{xcolor}
% \usepackage[normalem]{ulem}
\usepackage{lipsum}
\makeatletter
% \newcommand\colorwave[1][blue]{\bgroup \markoverwith{\lower3.5\p@\hbox{\sixly \textcolor{#1}{\char58}}}\ULon}
%\font\sixly=lasy6 % does not re-load if already loaded, so no memory problem.

\newmdtheoremenv[
linewidth= 1pt,linecolor= blue,%
leftmargin=20,rightmargin=20,innertopmargin=0pt, innerrightmargin=40,%
tikzsetting = { draw=lightgray, line width = 0.3pt,dashed,%
dash pattern = on 15pt off 3pt},%
splittopskip=\topskip,skipbelow=\baselineskip,%
skipabove=\baselineskip,ntheorem,roundcorner=0pt,
% backgroundcolor=pagebg,font=\color{orange}\sffamily, fontcolor=white
]{examplebox}{Exemple}[section]



\newcommand\R{\mathbb{R}}
\newcommand\Z{\mathbb{Z}}
\newcommand\N{\mathbb{N}}
\newcommand\E{\mathbb{E}}
\newcommand\F{\mathcal{F}}
\newcommand\cH{\mathcal{H}}
\newcommand\V{\mathbb{V}}
\newcommand\dmo{ ^{-1} }
\newcommand\kapa{\kappa}
\newcommand\im{Im}
\newcommand\hs{\mathcal{H}}





\usepackage{soul}

\makeatletter
\newcommand*{\whiten}[1]{\llap{\textcolor{white}{{\the\SOUL@token}}\hspace{#1pt}}}
\DeclareRobustCommand*\myul{%
    \def\SOUL@everyspace{\underline{\space}\kern\z@}%
    \def\SOUL@everytoken{%
     \setbox0=\hbox{\the\SOUL@token}%
     \ifdim\dp0>\z@
        \raisebox{\dp0}{\underline{\phantom{\the\SOUL@token}}}%
        \whiten{1}\whiten{0}%
        \whiten{-1}\whiten{-2}%
        \llap{\the\SOUL@token}%
     \else
        \underline{\the\SOUL@token}%
     \fi}%
\SOUL@}
\makeatother

\newcommand*{\demp}{\fontfamily{lmtt}\selectfont}

\DeclareTextFontCommand{\textdemp}{\demp}

\begin{document}

\ifcomment
Multiline
comment
\fi
\ifcomment
\myul{Typesetting test}
% \color[rgb]{1,1,1}
$∑_i^n≠ 60º±∞π∆¬≈√j∫h≤≥µ$

$\CR \R\pro\ind\pro\gS\pro
\mqty[a&b\\c&d]$
$\pro\mathbb{P}$
$\dd{x}$

  \[
    \alpha(x)=\left\{
                \begin{array}{ll}
                  x\\
                  \frac{1}{1+e^{-kx}}\\
                  \frac{e^x-e^{-x}}{e^x+e^{-x}}
                \end{array}
              \right.
  \]

  $\expval{x}$
  
  $\chi_\rho(ghg\dmo)=\Tr(\rho_{ghg\dmo})=\Tr(\rho_g\circ\rho_h\circ\rho\dmo_g)=\Tr(\rho_h)\overset{\mbox{\scalebox{0.5}{$\Tr(AB)=\Tr(BA)$}}}{=}\chi_\rho(h)$
  	$\mathop{\oplus}_{\substack{x\in X}}$

$\mat(\rho_g)=(a_{ij}(g))_{\scriptsize \substack{1\leq i\leq d \\ 1\leq j\leq d}}$ et $\mat(\rho'_g)=(a'_{ij}(g))_{\scriptsize \substack{1\leq i'\leq d' \\ 1\leq j'\leq d'}}$



\[\int_a^b{\mathbb{R}^2}g(u, v)\dd{P_{XY}}(u, v)=\iint g(u,v) f_{XY}(u, v)\dd \lambda(u) \dd \lambda(v)\]
$$\lim_{x\to\infty} f(x)$$	
$$\iiiint_V \mu(t,u,v,w) \,dt\,du\,dv\,dw$$
$$\sum_{n=1}^{\infty} 2^{-n} = 1$$	
\begin{definition}
	Si $X$ et $Y$ sont 2 v.a. ou definit la \textsc{Covariance} entre $X$ et $Y$ comme
	$\cov(X,Y)\overset{\text{def}}{=}\E\left[(X-\E(X))(Y-\E(Y))\right]=\E(XY)-\E(X)\E(Y)$.
\end{definition}
\fi
\pagebreak

% \tableofcontents

% insert your code here
%\input{./algebra/main.tex}
%\input{./geometrie-differentielle/main.tex}
%\input{./probabilite/main.tex}
%\input{./analyse-fonctionnelle/main.tex}
% \input{./Analyse-convexe-et-dualite-en-optimisation/main.tex}
%\input{./tikz/main.tex}
%\input{./Theorie-du-distributions/main.tex}
%\input{./optimisation/mine.tex}
 \input{./modelisation/main.tex}

% yves.aubry@univ-tln.fr : algebra

\end{document}

%% !TEX encoding = UTF-8 Unicode
% !TEX TS-program = xelatex

\documentclass[french]{report}

%\usepackage[utf8]{inputenc}
%\usepackage[T1]{fontenc}
\usepackage{babel}


\newif\ifcomment
%\commenttrue # Show comments

\usepackage{physics}
\usepackage{amssymb}


\usepackage{amsthm}
% \usepackage{thmtools}
\usepackage{mathtools}
\usepackage{amsfonts}

\usepackage{color}

\usepackage{tikz}

\usepackage{geometry}
\geometry{a5paper, margin=0.1in, right=1cm}

\usepackage{dsfont}

\usepackage{graphicx}
\graphicspath{ {images/} }

\usepackage{faktor}

\usepackage{IEEEtrantools}
\usepackage{enumerate}   
\usepackage[PostScript=dvips]{"/Users/aware/Documents/Courses/diagrams"}


\newtheorem{theorem}{Théorème}[section]
\renewcommand{\thetheorem}{\arabic{theorem}}
\newtheorem{lemme}{Lemme}[section]
\renewcommand{\thelemme}{\arabic{lemme}}
\newtheorem{proposition}{Proposition}[section]
\renewcommand{\theproposition}{\arabic{proposition}}
\newtheorem{notations}{Notations}[section]
\newtheorem{problem}{Problème}[section]
\newtheorem{corollary}{Corollaire}[theorem]
\renewcommand{\thecorollary}{\arabic{corollary}}
\newtheorem{property}{Propriété}[section]
\newtheorem{objective}{Objectif}[section]

\theoremstyle{definition}
\newtheorem{definition}{Définition}[section]
\renewcommand{\thedefinition}{\arabic{definition}}
\newtheorem{exercise}{Exercice}[chapter]
\renewcommand{\theexercise}{\arabic{exercise}}
\newtheorem{example}{Exemple}[chapter]
\renewcommand{\theexample}{\arabic{example}}
\newtheorem*{solution}{Solution}
\newtheorem*{application}{Application}
\newtheorem*{notation}{Notation}
\newtheorem*{vocabulary}{Vocabulaire}
\newtheorem*{properties}{Propriétés}



\theoremstyle{remark}
\newtheorem*{remark}{Remarque}
\newtheorem*{rappel}{Rappel}


\usepackage{etoolbox}
\AtBeginEnvironment{exercise}{\small}
\AtBeginEnvironment{example}{\small}

\usepackage{cases}
\usepackage[red]{mypack}

\usepackage[framemethod=TikZ]{mdframed}

\definecolor{bg}{rgb}{0.4,0.25,0.95}
\definecolor{pagebg}{rgb}{0,0,0.5}
\surroundwithmdframed[
   topline=false,
   rightline=false,
   bottomline=false,
   leftmargin=\parindent,
   skipabove=8pt,
   skipbelow=8pt,
   linecolor=blue,
   innerbottommargin=10pt,
   % backgroundcolor=bg,font=\color{orange}\sffamily, fontcolor=white
]{definition}

\usepackage{empheq}
\usepackage[most]{tcolorbox}

\newtcbox{\mymath}[1][]{%
    nobeforeafter, math upper, tcbox raise base,
    enhanced, colframe=blue!30!black,
    colback=red!10, boxrule=1pt,
    #1}

\usepackage{unixode}


\DeclareMathOperator{\ord}{ord}
\DeclareMathOperator{\orb}{orb}
\DeclareMathOperator{\stab}{stab}
\DeclareMathOperator{\Stab}{stab}
\DeclareMathOperator{\ppcm}{ppcm}
\DeclareMathOperator{\conj}{Conj}
\DeclareMathOperator{\End}{End}
\DeclareMathOperator{\rot}{rot}
\DeclareMathOperator{\trs}{trace}
\DeclareMathOperator{\Ind}{Ind}
\DeclareMathOperator{\mat}{Mat}
\DeclareMathOperator{\id}{Id}
\DeclareMathOperator{\vect}{vect}
\DeclareMathOperator{\img}{img}
\DeclareMathOperator{\cov}{Cov}
\DeclareMathOperator{\dist}{dist}
\DeclareMathOperator{\irr}{Irr}
\DeclareMathOperator{\image}{Im}
\DeclareMathOperator{\pd}{\partial}
\DeclareMathOperator{\epi}{epi}
\DeclareMathOperator{\Argmin}{Argmin}
\DeclareMathOperator{\dom}{dom}
\DeclareMathOperator{\proj}{proj}
\DeclareMathOperator{\ctg}{ctg}
\DeclareMathOperator{\supp}{supp}
\DeclareMathOperator{\argmin}{argmin}
\DeclareMathOperator{\mult}{mult}
\DeclareMathOperator{\ch}{ch}
\DeclareMathOperator{\sh}{sh}
\DeclareMathOperator{\rang}{rang}
\DeclareMathOperator{\diam}{diam}
\DeclareMathOperator{\Epigraphe}{Epigraphe}




\usepackage{xcolor}
\everymath{\color{blue}}
%\everymath{\color[rgb]{0,1,1}}
%\pagecolor[rgb]{0,0,0.5}


\newcommand*{\pdtest}[3][]{\ensuremath{\frac{\partial^{#1} #2}{\partial #3}}}

\newcommand*{\deffunc}[6][]{\ensuremath{
\begin{array}{rcl}
#2 : #3 &\rightarrow& #4\\
#5 &\mapsto& #6
\end{array}
}}

\newcommand{\eqcolon}{\mathrel{\resizebox{\widthof{$\mathord{=}$}}{\height}{ $\!\!=\!\!\resizebox{1.2\width}{0.8\height}{\raisebox{0.23ex}{$\mathop{:}$}}\!\!$ }}}
\newcommand{\coloneq}{\mathrel{\resizebox{\widthof{$\mathord{=}$}}{\height}{ $\!\!\resizebox{1.2\width}{0.8\height}{\raisebox{0.23ex}{$\mathop{:}$}}\!\!=\!\!$ }}}
\newcommand{\eqcolonl}{\ensuremath{\mathrel{=\!\!\mathop{:}}}}
\newcommand{\coloneql}{\ensuremath{\mathrel{\mathop{:} \!\! =}}}
\newcommand{\vc}[1]{% inline column vector
  \left(\begin{smallmatrix}#1\end{smallmatrix}\right)%
}
\newcommand{\vr}[1]{% inline row vector
  \begin{smallmatrix}(\,#1\,)\end{smallmatrix}%
}
\makeatletter
\newcommand*{\defeq}{\ =\mathrel{\rlap{%
                     \raisebox{0.3ex}{$\m@th\cdot$}}%
                     \raisebox{-0.3ex}{$\m@th\cdot$}}%
                     }
\makeatother

\newcommand{\mathcircle}[1]{% inline row vector
 \overset{\circ}{#1}
}
\newcommand{\ulim}{% low limit
 \underline{\lim}
}
\newcommand{\ssi}{% iff
\iff
}
\newcommand{\ps}[2]{
\expval{#1 | #2}
}
\newcommand{\df}[1]{
\mqty{#1}
}
\newcommand{\n}[1]{
\norm{#1}
}
\newcommand{\sys}[1]{
\left\{\smqty{#1}\right.
}


\newcommand{\eqdef}{\ensuremath{\overset{\text{def}}=}}


\def\Circlearrowright{\ensuremath{%
  \rotatebox[origin=c]{230}{$\circlearrowright$}}}

\newcommand\ct[1]{\text{\rmfamily\upshape #1}}
\newcommand\question[1]{ {\color{red} ...!? \small #1}}
\newcommand\caz[1]{\left\{\begin{array} #1 \end{array}\right.}
\newcommand\const{\text{\rmfamily\upshape const}}
\newcommand\toP{ \overset{\pro}{\to}}
\newcommand\toPP{ \overset{\text{PP}}{\to}}
\newcommand{\oeq}{\mathrel{\text{\textcircled{$=$}}}}





\usepackage{xcolor}
% \usepackage[normalem]{ulem}
\usepackage{lipsum}
\makeatletter
% \newcommand\colorwave[1][blue]{\bgroup \markoverwith{\lower3.5\p@\hbox{\sixly \textcolor{#1}{\char58}}}\ULon}
%\font\sixly=lasy6 % does not re-load if already loaded, so no memory problem.

\newmdtheoremenv[
linewidth= 1pt,linecolor= blue,%
leftmargin=20,rightmargin=20,innertopmargin=0pt, innerrightmargin=40,%
tikzsetting = { draw=lightgray, line width = 0.3pt,dashed,%
dash pattern = on 15pt off 3pt},%
splittopskip=\topskip,skipbelow=\baselineskip,%
skipabove=\baselineskip,ntheorem,roundcorner=0pt,
% backgroundcolor=pagebg,font=\color{orange}\sffamily, fontcolor=white
]{examplebox}{Exemple}[section]



\newcommand\R{\mathbb{R}}
\newcommand\Z{\mathbb{Z}}
\newcommand\N{\mathbb{N}}
\newcommand\E{\mathbb{E}}
\newcommand\F{\mathcal{F}}
\newcommand\cH{\mathcal{H}}
\newcommand\V{\mathbb{V}}
\newcommand\dmo{ ^{-1} }
\newcommand\kapa{\kappa}
\newcommand\im{Im}
\newcommand\hs{\mathcal{H}}





\usepackage{soul}

\makeatletter
\newcommand*{\whiten}[1]{\llap{\textcolor{white}{{\the\SOUL@token}}\hspace{#1pt}}}
\DeclareRobustCommand*\myul{%
    \def\SOUL@everyspace{\underline{\space}\kern\z@}%
    \def\SOUL@everytoken{%
     \setbox0=\hbox{\the\SOUL@token}%
     \ifdim\dp0>\z@
        \raisebox{\dp0}{\underline{\phantom{\the\SOUL@token}}}%
        \whiten{1}\whiten{0}%
        \whiten{-1}\whiten{-2}%
        \llap{\the\SOUL@token}%
     \else
        \underline{\the\SOUL@token}%
     \fi}%
\SOUL@}
\makeatother

\newcommand*{\demp}{\fontfamily{lmtt}\selectfont}

\DeclareTextFontCommand{\textdemp}{\demp}

\begin{document}

\ifcomment
Multiline
comment
\fi
\ifcomment
\myul{Typesetting test}
% \color[rgb]{1,1,1}
$∑_i^n≠ 60º±∞π∆¬≈√j∫h≤≥µ$

$\CR \R\pro\ind\pro\gS\pro
\mqty[a&b\\c&d]$
$\pro\mathbb{P}$
$\dd{x}$

  \[
    \alpha(x)=\left\{
                \begin{array}{ll}
                  x\\
                  \frac{1}{1+e^{-kx}}\\
                  \frac{e^x-e^{-x}}{e^x+e^{-x}}
                \end{array}
              \right.
  \]

  $\expval{x}$
  
  $\chi_\rho(ghg\dmo)=\Tr(\rho_{ghg\dmo})=\Tr(\rho_g\circ\rho_h\circ\rho\dmo_g)=\Tr(\rho_h)\overset{\mbox{\scalebox{0.5}{$\Tr(AB)=\Tr(BA)$}}}{=}\chi_\rho(h)$
  	$\mathop{\oplus}_{\substack{x\in X}}$

$\mat(\rho_g)=(a_{ij}(g))_{\scriptsize \substack{1\leq i\leq d \\ 1\leq j\leq d}}$ et $\mat(\rho'_g)=(a'_{ij}(g))_{\scriptsize \substack{1\leq i'\leq d' \\ 1\leq j'\leq d'}}$



\[\int_a^b{\mathbb{R}^2}g(u, v)\dd{P_{XY}}(u, v)=\iint g(u,v) f_{XY}(u, v)\dd \lambda(u) \dd \lambda(v)\]
$$\lim_{x\to\infty} f(x)$$	
$$\iiiint_V \mu(t,u,v,w) \,dt\,du\,dv\,dw$$
$$\sum_{n=1}^{\infty} 2^{-n} = 1$$	
\begin{definition}
	Si $X$ et $Y$ sont 2 v.a. ou definit la \textsc{Covariance} entre $X$ et $Y$ comme
	$\cov(X,Y)\overset{\text{def}}{=}\E\left[(X-\E(X))(Y-\E(Y))\right]=\E(XY)-\E(X)\E(Y)$.
\end{definition}
\fi
\pagebreak

% \tableofcontents

% insert your code here
%\input{./algebra/main.tex}
%\input{./geometrie-differentielle/main.tex}
%\input{./probabilite/main.tex}
%\input{./analyse-fonctionnelle/main.tex}
% \input{./Analyse-convexe-et-dualite-en-optimisation/main.tex}
%\input{./tikz/main.tex}
%\input{./Theorie-du-distributions/main.tex}
%\input{./optimisation/mine.tex}
 \input{./modelisation/main.tex}

% yves.aubry@univ-tln.fr : algebra

\end{document}

%% !TEX encoding = UTF-8 Unicode
% !TEX TS-program = xelatex

\documentclass[french]{report}

%\usepackage[utf8]{inputenc}
%\usepackage[T1]{fontenc}
\usepackage{babel}


\newif\ifcomment
%\commenttrue # Show comments

\usepackage{physics}
\usepackage{amssymb}


\usepackage{amsthm}
% \usepackage{thmtools}
\usepackage{mathtools}
\usepackage{amsfonts}

\usepackage{color}

\usepackage{tikz}

\usepackage{geometry}
\geometry{a5paper, margin=0.1in, right=1cm}

\usepackage{dsfont}

\usepackage{graphicx}
\graphicspath{ {images/} }

\usepackage{faktor}

\usepackage{IEEEtrantools}
\usepackage{enumerate}   
\usepackage[PostScript=dvips]{"/Users/aware/Documents/Courses/diagrams"}


\newtheorem{theorem}{Théorème}[section]
\renewcommand{\thetheorem}{\arabic{theorem}}
\newtheorem{lemme}{Lemme}[section]
\renewcommand{\thelemme}{\arabic{lemme}}
\newtheorem{proposition}{Proposition}[section]
\renewcommand{\theproposition}{\arabic{proposition}}
\newtheorem{notations}{Notations}[section]
\newtheorem{problem}{Problème}[section]
\newtheorem{corollary}{Corollaire}[theorem]
\renewcommand{\thecorollary}{\arabic{corollary}}
\newtheorem{property}{Propriété}[section]
\newtheorem{objective}{Objectif}[section]

\theoremstyle{definition}
\newtheorem{definition}{Définition}[section]
\renewcommand{\thedefinition}{\arabic{definition}}
\newtheorem{exercise}{Exercice}[chapter]
\renewcommand{\theexercise}{\arabic{exercise}}
\newtheorem{example}{Exemple}[chapter]
\renewcommand{\theexample}{\arabic{example}}
\newtheorem*{solution}{Solution}
\newtheorem*{application}{Application}
\newtheorem*{notation}{Notation}
\newtheorem*{vocabulary}{Vocabulaire}
\newtheorem*{properties}{Propriétés}



\theoremstyle{remark}
\newtheorem*{remark}{Remarque}
\newtheorem*{rappel}{Rappel}


\usepackage{etoolbox}
\AtBeginEnvironment{exercise}{\small}
\AtBeginEnvironment{example}{\small}

\usepackage{cases}
\usepackage[red]{mypack}

\usepackage[framemethod=TikZ]{mdframed}

\definecolor{bg}{rgb}{0.4,0.25,0.95}
\definecolor{pagebg}{rgb}{0,0,0.5}
\surroundwithmdframed[
   topline=false,
   rightline=false,
   bottomline=false,
   leftmargin=\parindent,
   skipabove=8pt,
   skipbelow=8pt,
   linecolor=blue,
   innerbottommargin=10pt,
   % backgroundcolor=bg,font=\color{orange}\sffamily, fontcolor=white
]{definition}

\usepackage{empheq}
\usepackage[most]{tcolorbox}

\newtcbox{\mymath}[1][]{%
    nobeforeafter, math upper, tcbox raise base,
    enhanced, colframe=blue!30!black,
    colback=red!10, boxrule=1pt,
    #1}

\usepackage{unixode}


\DeclareMathOperator{\ord}{ord}
\DeclareMathOperator{\orb}{orb}
\DeclareMathOperator{\stab}{stab}
\DeclareMathOperator{\Stab}{stab}
\DeclareMathOperator{\ppcm}{ppcm}
\DeclareMathOperator{\conj}{Conj}
\DeclareMathOperator{\End}{End}
\DeclareMathOperator{\rot}{rot}
\DeclareMathOperator{\trs}{trace}
\DeclareMathOperator{\Ind}{Ind}
\DeclareMathOperator{\mat}{Mat}
\DeclareMathOperator{\id}{Id}
\DeclareMathOperator{\vect}{vect}
\DeclareMathOperator{\img}{img}
\DeclareMathOperator{\cov}{Cov}
\DeclareMathOperator{\dist}{dist}
\DeclareMathOperator{\irr}{Irr}
\DeclareMathOperator{\image}{Im}
\DeclareMathOperator{\pd}{\partial}
\DeclareMathOperator{\epi}{epi}
\DeclareMathOperator{\Argmin}{Argmin}
\DeclareMathOperator{\dom}{dom}
\DeclareMathOperator{\proj}{proj}
\DeclareMathOperator{\ctg}{ctg}
\DeclareMathOperator{\supp}{supp}
\DeclareMathOperator{\argmin}{argmin}
\DeclareMathOperator{\mult}{mult}
\DeclareMathOperator{\ch}{ch}
\DeclareMathOperator{\sh}{sh}
\DeclareMathOperator{\rang}{rang}
\DeclareMathOperator{\diam}{diam}
\DeclareMathOperator{\Epigraphe}{Epigraphe}




\usepackage{xcolor}
\everymath{\color{blue}}
%\everymath{\color[rgb]{0,1,1}}
%\pagecolor[rgb]{0,0,0.5}


\newcommand*{\pdtest}[3][]{\ensuremath{\frac{\partial^{#1} #2}{\partial #3}}}

\newcommand*{\deffunc}[6][]{\ensuremath{
\begin{array}{rcl}
#2 : #3 &\rightarrow& #4\\
#5 &\mapsto& #6
\end{array}
}}

\newcommand{\eqcolon}{\mathrel{\resizebox{\widthof{$\mathord{=}$}}{\height}{ $\!\!=\!\!\resizebox{1.2\width}{0.8\height}{\raisebox{0.23ex}{$\mathop{:}$}}\!\!$ }}}
\newcommand{\coloneq}{\mathrel{\resizebox{\widthof{$\mathord{=}$}}{\height}{ $\!\!\resizebox{1.2\width}{0.8\height}{\raisebox{0.23ex}{$\mathop{:}$}}\!\!=\!\!$ }}}
\newcommand{\eqcolonl}{\ensuremath{\mathrel{=\!\!\mathop{:}}}}
\newcommand{\coloneql}{\ensuremath{\mathrel{\mathop{:} \!\! =}}}
\newcommand{\vc}[1]{% inline column vector
  \left(\begin{smallmatrix}#1\end{smallmatrix}\right)%
}
\newcommand{\vr}[1]{% inline row vector
  \begin{smallmatrix}(\,#1\,)\end{smallmatrix}%
}
\makeatletter
\newcommand*{\defeq}{\ =\mathrel{\rlap{%
                     \raisebox{0.3ex}{$\m@th\cdot$}}%
                     \raisebox{-0.3ex}{$\m@th\cdot$}}%
                     }
\makeatother

\newcommand{\mathcircle}[1]{% inline row vector
 \overset{\circ}{#1}
}
\newcommand{\ulim}{% low limit
 \underline{\lim}
}
\newcommand{\ssi}{% iff
\iff
}
\newcommand{\ps}[2]{
\expval{#1 | #2}
}
\newcommand{\df}[1]{
\mqty{#1}
}
\newcommand{\n}[1]{
\norm{#1}
}
\newcommand{\sys}[1]{
\left\{\smqty{#1}\right.
}


\newcommand{\eqdef}{\ensuremath{\overset{\text{def}}=}}


\def\Circlearrowright{\ensuremath{%
  \rotatebox[origin=c]{230}{$\circlearrowright$}}}

\newcommand\ct[1]{\text{\rmfamily\upshape #1}}
\newcommand\question[1]{ {\color{red} ...!? \small #1}}
\newcommand\caz[1]{\left\{\begin{array} #1 \end{array}\right.}
\newcommand\const{\text{\rmfamily\upshape const}}
\newcommand\toP{ \overset{\pro}{\to}}
\newcommand\toPP{ \overset{\text{PP}}{\to}}
\newcommand{\oeq}{\mathrel{\text{\textcircled{$=$}}}}





\usepackage{xcolor}
% \usepackage[normalem]{ulem}
\usepackage{lipsum}
\makeatletter
% \newcommand\colorwave[1][blue]{\bgroup \markoverwith{\lower3.5\p@\hbox{\sixly \textcolor{#1}{\char58}}}\ULon}
%\font\sixly=lasy6 % does not re-load if already loaded, so no memory problem.

\newmdtheoremenv[
linewidth= 1pt,linecolor= blue,%
leftmargin=20,rightmargin=20,innertopmargin=0pt, innerrightmargin=40,%
tikzsetting = { draw=lightgray, line width = 0.3pt,dashed,%
dash pattern = on 15pt off 3pt},%
splittopskip=\topskip,skipbelow=\baselineskip,%
skipabove=\baselineskip,ntheorem,roundcorner=0pt,
% backgroundcolor=pagebg,font=\color{orange}\sffamily, fontcolor=white
]{examplebox}{Exemple}[section]



\newcommand\R{\mathbb{R}}
\newcommand\Z{\mathbb{Z}}
\newcommand\N{\mathbb{N}}
\newcommand\E{\mathbb{E}}
\newcommand\F{\mathcal{F}}
\newcommand\cH{\mathcal{H}}
\newcommand\V{\mathbb{V}}
\newcommand\dmo{ ^{-1} }
\newcommand\kapa{\kappa}
\newcommand\im{Im}
\newcommand\hs{\mathcal{H}}





\usepackage{soul}

\makeatletter
\newcommand*{\whiten}[1]{\llap{\textcolor{white}{{\the\SOUL@token}}\hspace{#1pt}}}
\DeclareRobustCommand*\myul{%
    \def\SOUL@everyspace{\underline{\space}\kern\z@}%
    \def\SOUL@everytoken{%
     \setbox0=\hbox{\the\SOUL@token}%
     \ifdim\dp0>\z@
        \raisebox{\dp0}{\underline{\phantom{\the\SOUL@token}}}%
        \whiten{1}\whiten{0}%
        \whiten{-1}\whiten{-2}%
        \llap{\the\SOUL@token}%
     \else
        \underline{\the\SOUL@token}%
     \fi}%
\SOUL@}
\makeatother

\newcommand*{\demp}{\fontfamily{lmtt}\selectfont}

\DeclareTextFontCommand{\textdemp}{\demp}

\begin{document}

\ifcomment
Multiline
comment
\fi
\ifcomment
\myul{Typesetting test}
% \color[rgb]{1,1,1}
$∑_i^n≠ 60º±∞π∆¬≈√j∫h≤≥µ$

$\CR \R\pro\ind\pro\gS\pro
\mqty[a&b\\c&d]$
$\pro\mathbb{P}$
$\dd{x}$

  \[
    \alpha(x)=\left\{
                \begin{array}{ll}
                  x\\
                  \frac{1}{1+e^{-kx}}\\
                  \frac{e^x-e^{-x}}{e^x+e^{-x}}
                \end{array}
              \right.
  \]

  $\expval{x}$
  
  $\chi_\rho(ghg\dmo)=\Tr(\rho_{ghg\dmo})=\Tr(\rho_g\circ\rho_h\circ\rho\dmo_g)=\Tr(\rho_h)\overset{\mbox{\scalebox{0.5}{$\Tr(AB)=\Tr(BA)$}}}{=}\chi_\rho(h)$
  	$\mathop{\oplus}_{\substack{x\in X}}$

$\mat(\rho_g)=(a_{ij}(g))_{\scriptsize \substack{1\leq i\leq d \\ 1\leq j\leq d}}$ et $\mat(\rho'_g)=(a'_{ij}(g))_{\scriptsize \substack{1\leq i'\leq d' \\ 1\leq j'\leq d'}}$



\[\int_a^b{\mathbb{R}^2}g(u, v)\dd{P_{XY}}(u, v)=\iint g(u,v) f_{XY}(u, v)\dd \lambda(u) \dd \lambda(v)\]
$$\lim_{x\to\infty} f(x)$$	
$$\iiiint_V \mu(t,u,v,w) \,dt\,du\,dv\,dw$$
$$\sum_{n=1}^{\infty} 2^{-n} = 1$$	
\begin{definition}
	Si $X$ et $Y$ sont 2 v.a. ou definit la \textsc{Covariance} entre $X$ et $Y$ comme
	$\cov(X,Y)\overset{\text{def}}{=}\E\left[(X-\E(X))(Y-\E(Y))\right]=\E(XY)-\E(X)\E(Y)$.
\end{definition}
\fi
\pagebreak

% \tableofcontents

% insert your code here
%\input{./algebra/main.tex}
%\input{./geometrie-differentielle/main.tex}
%\input{./probabilite/main.tex}
%\input{./analyse-fonctionnelle/main.tex}
% \input{./Analyse-convexe-et-dualite-en-optimisation/main.tex}
%\input{./tikz/main.tex}
%\input{./Theorie-du-distributions/main.tex}
%\input{./optimisation/mine.tex}
 \input{./modelisation/main.tex}

% yves.aubry@univ-tln.fr : algebra

\end{document}

%\input{./optimisation/mine.tex}
 % !TEX encoding = UTF-8 Unicode
% !TEX TS-program = xelatex

\documentclass[french]{report}

%\usepackage[utf8]{inputenc}
%\usepackage[T1]{fontenc}
\usepackage{babel}


\newif\ifcomment
%\commenttrue # Show comments

\usepackage{physics}
\usepackage{amssymb}


\usepackage{amsthm}
% \usepackage{thmtools}
\usepackage{mathtools}
\usepackage{amsfonts}

\usepackage{color}

\usepackage{tikz}

\usepackage{geometry}
\geometry{a5paper, margin=0.1in, right=1cm}

\usepackage{dsfont}

\usepackage{graphicx}
\graphicspath{ {images/} }

\usepackage{faktor}

\usepackage{IEEEtrantools}
\usepackage{enumerate}   
\usepackage[PostScript=dvips]{"/Users/aware/Documents/Courses/diagrams"}


\newtheorem{theorem}{Théorème}[section]
\renewcommand{\thetheorem}{\arabic{theorem}}
\newtheorem{lemme}{Lemme}[section]
\renewcommand{\thelemme}{\arabic{lemme}}
\newtheorem{proposition}{Proposition}[section]
\renewcommand{\theproposition}{\arabic{proposition}}
\newtheorem{notations}{Notations}[section]
\newtheorem{problem}{Problème}[section]
\newtheorem{corollary}{Corollaire}[theorem]
\renewcommand{\thecorollary}{\arabic{corollary}}
\newtheorem{property}{Propriété}[section]
\newtheorem{objective}{Objectif}[section]

\theoremstyle{definition}
\newtheorem{definition}{Définition}[section]
\renewcommand{\thedefinition}{\arabic{definition}}
\newtheorem{exercise}{Exercice}[chapter]
\renewcommand{\theexercise}{\arabic{exercise}}
\newtheorem{example}{Exemple}[chapter]
\renewcommand{\theexample}{\arabic{example}}
\newtheorem*{solution}{Solution}
\newtheorem*{application}{Application}
\newtheorem*{notation}{Notation}
\newtheorem*{vocabulary}{Vocabulaire}
\newtheorem*{properties}{Propriétés}



\theoremstyle{remark}
\newtheorem*{remark}{Remarque}
\newtheorem*{rappel}{Rappel}


\usepackage{etoolbox}
\AtBeginEnvironment{exercise}{\small}
\AtBeginEnvironment{example}{\small}

\usepackage{cases}
\usepackage[red]{mypack}

\usepackage[framemethod=TikZ]{mdframed}

\definecolor{bg}{rgb}{0.4,0.25,0.95}
\definecolor{pagebg}{rgb}{0,0,0.5}
\surroundwithmdframed[
   topline=false,
   rightline=false,
   bottomline=false,
   leftmargin=\parindent,
   skipabove=8pt,
   skipbelow=8pt,
   linecolor=blue,
   innerbottommargin=10pt,
   % backgroundcolor=bg,font=\color{orange}\sffamily, fontcolor=white
]{definition}

\usepackage{empheq}
\usepackage[most]{tcolorbox}

\newtcbox{\mymath}[1][]{%
    nobeforeafter, math upper, tcbox raise base,
    enhanced, colframe=blue!30!black,
    colback=red!10, boxrule=1pt,
    #1}

\usepackage{unixode}


\DeclareMathOperator{\ord}{ord}
\DeclareMathOperator{\orb}{orb}
\DeclareMathOperator{\stab}{stab}
\DeclareMathOperator{\Stab}{stab}
\DeclareMathOperator{\ppcm}{ppcm}
\DeclareMathOperator{\conj}{Conj}
\DeclareMathOperator{\End}{End}
\DeclareMathOperator{\rot}{rot}
\DeclareMathOperator{\trs}{trace}
\DeclareMathOperator{\Ind}{Ind}
\DeclareMathOperator{\mat}{Mat}
\DeclareMathOperator{\id}{Id}
\DeclareMathOperator{\vect}{vect}
\DeclareMathOperator{\img}{img}
\DeclareMathOperator{\cov}{Cov}
\DeclareMathOperator{\dist}{dist}
\DeclareMathOperator{\irr}{Irr}
\DeclareMathOperator{\image}{Im}
\DeclareMathOperator{\pd}{\partial}
\DeclareMathOperator{\epi}{epi}
\DeclareMathOperator{\Argmin}{Argmin}
\DeclareMathOperator{\dom}{dom}
\DeclareMathOperator{\proj}{proj}
\DeclareMathOperator{\ctg}{ctg}
\DeclareMathOperator{\supp}{supp}
\DeclareMathOperator{\argmin}{argmin}
\DeclareMathOperator{\mult}{mult}
\DeclareMathOperator{\ch}{ch}
\DeclareMathOperator{\sh}{sh}
\DeclareMathOperator{\rang}{rang}
\DeclareMathOperator{\diam}{diam}
\DeclareMathOperator{\Epigraphe}{Epigraphe}




\usepackage{xcolor}
\everymath{\color{blue}}
%\everymath{\color[rgb]{0,1,1}}
%\pagecolor[rgb]{0,0,0.5}


\newcommand*{\pdtest}[3][]{\ensuremath{\frac{\partial^{#1} #2}{\partial #3}}}

\newcommand*{\deffunc}[6][]{\ensuremath{
\begin{array}{rcl}
#2 : #3 &\rightarrow& #4\\
#5 &\mapsto& #6
\end{array}
}}

\newcommand{\eqcolon}{\mathrel{\resizebox{\widthof{$\mathord{=}$}}{\height}{ $\!\!=\!\!\resizebox{1.2\width}{0.8\height}{\raisebox{0.23ex}{$\mathop{:}$}}\!\!$ }}}
\newcommand{\coloneq}{\mathrel{\resizebox{\widthof{$\mathord{=}$}}{\height}{ $\!\!\resizebox{1.2\width}{0.8\height}{\raisebox{0.23ex}{$\mathop{:}$}}\!\!=\!\!$ }}}
\newcommand{\eqcolonl}{\ensuremath{\mathrel{=\!\!\mathop{:}}}}
\newcommand{\coloneql}{\ensuremath{\mathrel{\mathop{:} \!\! =}}}
\newcommand{\vc}[1]{% inline column vector
  \left(\begin{smallmatrix}#1\end{smallmatrix}\right)%
}
\newcommand{\vr}[1]{% inline row vector
  \begin{smallmatrix}(\,#1\,)\end{smallmatrix}%
}
\makeatletter
\newcommand*{\defeq}{\ =\mathrel{\rlap{%
                     \raisebox{0.3ex}{$\m@th\cdot$}}%
                     \raisebox{-0.3ex}{$\m@th\cdot$}}%
                     }
\makeatother

\newcommand{\mathcircle}[1]{% inline row vector
 \overset{\circ}{#1}
}
\newcommand{\ulim}{% low limit
 \underline{\lim}
}
\newcommand{\ssi}{% iff
\iff
}
\newcommand{\ps}[2]{
\expval{#1 | #2}
}
\newcommand{\df}[1]{
\mqty{#1}
}
\newcommand{\n}[1]{
\norm{#1}
}
\newcommand{\sys}[1]{
\left\{\smqty{#1}\right.
}


\newcommand{\eqdef}{\ensuremath{\overset{\text{def}}=}}


\def\Circlearrowright{\ensuremath{%
  \rotatebox[origin=c]{230}{$\circlearrowright$}}}

\newcommand\ct[1]{\text{\rmfamily\upshape #1}}
\newcommand\question[1]{ {\color{red} ...!? \small #1}}
\newcommand\caz[1]{\left\{\begin{array} #1 \end{array}\right.}
\newcommand\const{\text{\rmfamily\upshape const}}
\newcommand\toP{ \overset{\pro}{\to}}
\newcommand\toPP{ \overset{\text{PP}}{\to}}
\newcommand{\oeq}{\mathrel{\text{\textcircled{$=$}}}}





\usepackage{xcolor}
% \usepackage[normalem]{ulem}
\usepackage{lipsum}
\makeatletter
% \newcommand\colorwave[1][blue]{\bgroup \markoverwith{\lower3.5\p@\hbox{\sixly \textcolor{#1}{\char58}}}\ULon}
%\font\sixly=lasy6 % does not re-load if already loaded, so no memory problem.

\newmdtheoremenv[
linewidth= 1pt,linecolor= blue,%
leftmargin=20,rightmargin=20,innertopmargin=0pt, innerrightmargin=40,%
tikzsetting = { draw=lightgray, line width = 0.3pt,dashed,%
dash pattern = on 15pt off 3pt},%
splittopskip=\topskip,skipbelow=\baselineskip,%
skipabove=\baselineskip,ntheorem,roundcorner=0pt,
% backgroundcolor=pagebg,font=\color{orange}\sffamily, fontcolor=white
]{examplebox}{Exemple}[section]



\newcommand\R{\mathbb{R}}
\newcommand\Z{\mathbb{Z}}
\newcommand\N{\mathbb{N}}
\newcommand\E{\mathbb{E}}
\newcommand\F{\mathcal{F}}
\newcommand\cH{\mathcal{H}}
\newcommand\V{\mathbb{V}}
\newcommand\dmo{ ^{-1} }
\newcommand\kapa{\kappa}
\newcommand\im{Im}
\newcommand\hs{\mathcal{H}}





\usepackage{soul}

\makeatletter
\newcommand*{\whiten}[1]{\llap{\textcolor{white}{{\the\SOUL@token}}\hspace{#1pt}}}
\DeclareRobustCommand*\myul{%
    \def\SOUL@everyspace{\underline{\space}\kern\z@}%
    \def\SOUL@everytoken{%
     \setbox0=\hbox{\the\SOUL@token}%
     \ifdim\dp0>\z@
        \raisebox{\dp0}{\underline{\phantom{\the\SOUL@token}}}%
        \whiten{1}\whiten{0}%
        \whiten{-1}\whiten{-2}%
        \llap{\the\SOUL@token}%
     \else
        \underline{\the\SOUL@token}%
     \fi}%
\SOUL@}
\makeatother

\newcommand*{\demp}{\fontfamily{lmtt}\selectfont}

\DeclareTextFontCommand{\textdemp}{\demp}

\begin{document}

\ifcomment
Multiline
comment
\fi
\ifcomment
\myul{Typesetting test}
% \color[rgb]{1,1,1}
$∑_i^n≠ 60º±∞π∆¬≈√j∫h≤≥µ$

$\CR \R\pro\ind\pro\gS\pro
\mqty[a&b\\c&d]$
$\pro\mathbb{P}$
$\dd{x}$

  \[
    \alpha(x)=\left\{
                \begin{array}{ll}
                  x\\
                  \frac{1}{1+e^{-kx}}\\
                  \frac{e^x-e^{-x}}{e^x+e^{-x}}
                \end{array}
              \right.
  \]

  $\expval{x}$
  
  $\chi_\rho(ghg\dmo)=\Tr(\rho_{ghg\dmo})=\Tr(\rho_g\circ\rho_h\circ\rho\dmo_g)=\Tr(\rho_h)\overset{\mbox{\scalebox{0.5}{$\Tr(AB)=\Tr(BA)$}}}{=}\chi_\rho(h)$
  	$\mathop{\oplus}_{\substack{x\in X}}$

$\mat(\rho_g)=(a_{ij}(g))_{\scriptsize \substack{1\leq i\leq d \\ 1\leq j\leq d}}$ et $\mat(\rho'_g)=(a'_{ij}(g))_{\scriptsize \substack{1\leq i'\leq d' \\ 1\leq j'\leq d'}}$



\[\int_a^b{\mathbb{R}^2}g(u, v)\dd{P_{XY}}(u, v)=\iint g(u,v) f_{XY}(u, v)\dd \lambda(u) \dd \lambda(v)\]
$$\lim_{x\to\infty} f(x)$$	
$$\iiiint_V \mu(t,u,v,w) \,dt\,du\,dv\,dw$$
$$\sum_{n=1}^{\infty} 2^{-n} = 1$$	
\begin{definition}
	Si $X$ et $Y$ sont 2 v.a. ou definit la \textsc{Covariance} entre $X$ et $Y$ comme
	$\cov(X,Y)\overset{\text{def}}{=}\E\left[(X-\E(X))(Y-\E(Y))\right]=\E(XY)-\E(X)\E(Y)$.
\end{definition}
\fi
\pagebreak

% \tableofcontents

% insert your code here
%\input{./algebra/main.tex}
%\input{./geometrie-differentielle/main.tex}
%\input{./probabilite/main.tex}
%\input{./analyse-fonctionnelle/main.tex}
% \input{./Analyse-convexe-et-dualite-en-optimisation/main.tex}
%\input{./tikz/main.tex}
%\input{./Theorie-du-distributions/main.tex}
%\input{./optimisation/mine.tex}
 \input{./modelisation/main.tex}

% yves.aubry@univ-tln.fr : algebra

\end{document}


% yves.aubry@univ-tln.fr : algebra

\end{document}

%\input{./optimisation/mine.tex}
 % !TEX encoding = UTF-8 Unicode
% !TEX TS-program = xelatex

\documentclass[french]{report}

%\usepackage[utf8]{inputenc}
%\usepackage[T1]{fontenc}
\usepackage{babel}


\newif\ifcomment
%\commenttrue # Show comments

\usepackage{physics}
\usepackage{amssymb}


\usepackage{amsthm}
% \usepackage{thmtools}
\usepackage{mathtools}
\usepackage{amsfonts}

\usepackage{color}

\usepackage{tikz}

\usepackage{geometry}
\geometry{a5paper, margin=0.1in, right=1cm}

\usepackage{dsfont}

\usepackage{graphicx}
\graphicspath{ {images/} }

\usepackage{faktor}

\usepackage{IEEEtrantools}
\usepackage{enumerate}   
\usepackage[PostScript=dvips]{"/Users/aware/Documents/Courses/diagrams"}


\newtheorem{theorem}{Théorème}[section]
\renewcommand{\thetheorem}{\arabic{theorem}}
\newtheorem{lemme}{Lemme}[section]
\renewcommand{\thelemme}{\arabic{lemme}}
\newtheorem{proposition}{Proposition}[section]
\renewcommand{\theproposition}{\arabic{proposition}}
\newtheorem{notations}{Notations}[section]
\newtheorem{problem}{Problème}[section]
\newtheorem{corollary}{Corollaire}[theorem]
\renewcommand{\thecorollary}{\arabic{corollary}}
\newtheorem{property}{Propriété}[section]
\newtheorem{objective}{Objectif}[section]

\theoremstyle{definition}
\newtheorem{definition}{Définition}[section]
\renewcommand{\thedefinition}{\arabic{definition}}
\newtheorem{exercise}{Exercice}[chapter]
\renewcommand{\theexercise}{\arabic{exercise}}
\newtheorem{example}{Exemple}[chapter]
\renewcommand{\theexample}{\arabic{example}}
\newtheorem*{solution}{Solution}
\newtheorem*{application}{Application}
\newtheorem*{notation}{Notation}
\newtheorem*{vocabulary}{Vocabulaire}
\newtheorem*{properties}{Propriétés}



\theoremstyle{remark}
\newtheorem*{remark}{Remarque}
\newtheorem*{rappel}{Rappel}


\usepackage{etoolbox}
\AtBeginEnvironment{exercise}{\small}
\AtBeginEnvironment{example}{\small}

\usepackage{cases}
\usepackage[red]{mypack}

\usepackage[framemethod=TikZ]{mdframed}

\definecolor{bg}{rgb}{0.4,0.25,0.95}
\definecolor{pagebg}{rgb}{0,0,0.5}
\surroundwithmdframed[
   topline=false,
   rightline=false,
   bottomline=false,
   leftmargin=\parindent,
   skipabove=8pt,
   skipbelow=8pt,
   linecolor=blue,
   innerbottommargin=10pt,
   % backgroundcolor=bg,font=\color{orange}\sffamily, fontcolor=white
]{definition}

\usepackage{empheq}
\usepackage[most]{tcolorbox}

\newtcbox{\mymath}[1][]{%
    nobeforeafter, math upper, tcbox raise base,
    enhanced, colframe=blue!30!black,
    colback=red!10, boxrule=1pt,
    #1}

\usepackage{unixode}


\DeclareMathOperator{\ord}{ord}
\DeclareMathOperator{\orb}{orb}
\DeclareMathOperator{\stab}{stab}
\DeclareMathOperator{\Stab}{stab}
\DeclareMathOperator{\ppcm}{ppcm}
\DeclareMathOperator{\conj}{Conj}
\DeclareMathOperator{\End}{End}
\DeclareMathOperator{\rot}{rot}
\DeclareMathOperator{\trs}{trace}
\DeclareMathOperator{\Ind}{Ind}
\DeclareMathOperator{\mat}{Mat}
\DeclareMathOperator{\id}{Id}
\DeclareMathOperator{\vect}{vect}
\DeclareMathOperator{\img}{img}
\DeclareMathOperator{\cov}{Cov}
\DeclareMathOperator{\dist}{dist}
\DeclareMathOperator{\irr}{Irr}
\DeclareMathOperator{\image}{Im}
\DeclareMathOperator{\pd}{\partial}
\DeclareMathOperator{\epi}{epi}
\DeclareMathOperator{\Argmin}{Argmin}
\DeclareMathOperator{\dom}{dom}
\DeclareMathOperator{\proj}{proj}
\DeclareMathOperator{\ctg}{ctg}
\DeclareMathOperator{\supp}{supp}
\DeclareMathOperator{\argmin}{argmin}
\DeclareMathOperator{\mult}{mult}
\DeclareMathOperator{\ch}{ch}
\DeclareMathOperator{\sh}{sh}
\DeclareMathOperator{\rang}{rang}
\DeclareMathOperator{\diam}{diam}
\DeclareMathOperator{\Epigraphe}{Epigraphe}




\usepackage{xcolor}
\everymath{\color{blue}}
%\everymath{\color[rgb]{0,1,1}}
%\pagecolor[rgb]{0,0,0.5}


\newcommand*{\pdtest}[3][]{\ensuremath{\frac{\partial^{#1} #2}{\partial #3}}}

\newcommand*{\deffunc}[6][]{\ensuremath{
\begin{array}{rcl}
#2 : #3 &\rightarrow& #4\\
#5 &\mapsto& #6
\end{array}
}}

\newcommand{\eqcolon}{\mathrel{\resizebox{\widthof{$\mathord{=}$}}{\height}{ $\!\!=\!\!\resizebox{1.2\width}{0.8\height}{\raisebox{0.23ex}{$\mathop{:}$}}\!\!$ }}}
\newcommand{\coloneq}{\mathrel{\resizebox{\widthof{$\mathord{=}$}}{\height}{ $\!\!\resizebox{1.2\width}{0.8\height}{\raisebox{0.23ex}{$\mathop{:}$}}\!\!=\!\!$ }}}
\newcommand{\eqcolonl}{\ensuremath{\mathrel{=\!\!\mathop{:}}}}
\newcommand{\coloneql}{\ensuremath{\mathrel{\mathop{:} \!\! =}}}
\newcommand{\vc}[1]{% inline column vector
  \left(\begin{smallmatrix}#1\end{smallmatrix}\right)%
}
\newcommand{\vr}[1]{% inline row vector
  \begin{smallmatrix}(\,#1\,)\end{smallmatrix}%
}
\makeatletter
\newcommand*{\defeq}{\ =\mathrel{\rlap{%
                     \raisebox{0.3ex}{$\m@th\cdot$}}%
                     \raisebox{-0.3ex}{$\m@th\cdot$}}%
                     }
\makeatother

\newcommand{\mathcircle}[1]{% inline row vector
 \overset{\circ}{#1}
}
\newcommand{\ulim}{% low limit
 \underline{\lim}
}
\newcommand{\ssi}{% iff
\iff
}
\newcommand{\ps}[2]{
\expval{#1 | #2}
}
\newcommand{\df}[1]{
\mqty{#1}
}
\newcommand{\n}[1]{
\norm{#1}
}
\newcommand{\sys}[1]{
\left\{\smqty{#1}\right.
}


\newcommand{\eqdef}{\ensuremath{\overset{\text{def}}=}}


\def\Circlearrowright{\ensuremath{%
  \rotatebox[origin=c]{230}{$\circlearrowright$}}}

\newcommand\ct[1]{\text{\rmfamily\upshape #1}}
\newcommand\question[1]{ {\color{red} ...!? \small #1}}
\newcommand\caz[1]{\left\{\begin{array} #1 \end{array}\right.}
\newcommand\const{\text{\rmfamily\upshape const}}
\newcommand\toP{ \overset{\pro}{\to}}
\newcommand\toPP{ \overset{\text{PP}}{\to}}
\newcommand{\oeq}{\mathrel{\text{\textcircled{$=$}}}}





\usepackage{xcolor}
% \usepackage[normalem]{ulem}
\usepackage{lipsum}
\makeatletter
% \newcommand\colorwave[1][blue]{\bgroup \markoverwith{\lower3.5\p@\hbox{\sixly \textcolor{#1}{\char58}}}\ULon}
%\font\sixly=lasy6 % does not re-load if already loaded, so no memory problem.

\newmdtheoremenv[
linewidth= 1pt,linecolor= blue,%
leftmargin=20,rightmargin=20,innertopmargin=0pt, innerrightmargin=40,%
tikzsetting = { draw=lightgray, line width = 0.3pt,dashed,%
dash pattern = on 15pt off 3pt},%
splittopskip=\topskip,skipbelow=\baselineskip,%
skipabove=\baselineskip,ntheorem,roundcorner=0pt,
% backgroundcolor=pagebg,font=\color{orange}\sffamily, fontcolor=white
]{examplebox}{Exemple}[section]



\newcommand\R{\mathbb{R}}
\newcommand\Z{\mathbb{Z}}
\newcommand\N{\mathbb{N}}
\newcommand\E{\mathbb{E}}
\newcommand\F{\mathcal{F}}
\newcommand\cH{\mathcal{H}}
\newcommand\V{\mathbb{V}}
\newcommand\dmo{ ^{-1} }
\newcommand\kapa{\kappa}
\newcommand\im{Im}
\newcommand\hs{\mathcal{H}}





\usepackage{soul}

\makeatletter
\newcommand*{\whiten}[1]{\llap{\textcolor{white}{{\the\SOUL@token}}\hspace{#1pt}}}
\DeclareRobustCommand*\myul{%
    \def\SOUL@everyspace{\underline{\space}\kern\z@}%
    \def\SOUL@everytoken{%
     \setbox0=\hbox{\the\SOUL@token}%
     \ifdim\dp0>\z@
        \raisebox{\dp0}{\underline{\phantom{\the\SOUL@token}}}%
        \whiten{1}\whiten{0}%
        \whiten{-1}\whiten{-2}%
        \llap{\the\SOUL@token}%
     \else
        \underline{\the\SOUL@token}%
     \fi}%
\SOUL@}
\makeatother

\newcommand*{\demp}{\fontfamily{lmtt}\selectfont}

\DeclareTextFontCommand{\textdemp}{\demp}

\begin{document}

\ifcomment
Multiline
comment
\fi
\ifcomment
\myul{Typesetting test}
% \color[rgb]{1,1,1}
$∑_i^n≠ 60º±∞π∆¬≈√j∫h≤≥µ$

$\CR \R\pro\ind\pro\gS\pro
\mqty[a&b\\c&d]$
$\pro\mathbb{P}$
$\dd{x}$

  \[
    \alpha(x)=\left\{
                \begin{array}{ll}
                  x\\
                  \frac{1}{1+e^{-kx}}\\
                  \frac{e^x-e^{-x}}{e^x+e^{-x}}
                \end{array}
              \right.
  \]

  $\expval{x}$
  
  $\chi_\rho(ghg\dmo)=\Tr(\rho_{ghg\dmo})=\Tr(\rho_g\circ\rho_h\circ\rho\dmo_g)=\Tr(\rho_h)\overset{\mbox{\scalebox{0.5}{$\Tr(AB)=\Tr(BA)$}}}{=}\chi_\rho(h)$
  	$\mathop{\oplus}_{\substack{x\in X}}$

$\mat(\rho_g)=(a_{ij}(g))_{\scriptsize \substack{1\leq i\leq d \\ 1\leq j\leq d}}$ et $\mat(\rho'_g)=(a'_{ij}(g))_{\scriptsize \substack{1\leq i'\leq d' \\ 1\leq j'\leq d'}}$



\[\int_a^b{\mathbb{R}^2}g(u, v)\dd{P_{XY}}(u, v)=\iint g(u,v) f_{XY}(u, v)\dd \lambda(u) \dd \lambda(v)\]
$$\lim_{x\to\infty} f(x)$$	
$$\iiiint_V \mu(t,u,v,w) \,dt\,du\,dv\,dw$$
$$\sum_{n=1}^{\infty} 2^{-n} = 1$$	
\begin{definition}
	Si $X$ et $Y$ sont 2 v.a. ou definit la \textsc{Covariance} entre $X$ et $Y$ comme
	$\cov(X,Y)\overset{\text{def}}{=}\E\left[(X-\E(X))(Y-\E(Y))\right]=\E(XY)-\E(X)\E(Y)$.
\end{definition}
\fi
\pagebreak

% \tableofcontents

% insert your code here
%% !TEX encoding = UTF-8 Unicode
% !TEX TS-program = xelatex

\documentclass[french]{report}

%\usepackage[utf8]{inputenc}
%\usepackage[T1]{fontenc}
\usepackage{babel}


\newif\ifcomment
%\commenttrue # Show comments

\usepackage{physics}
\usepackage{amssymb}


\usepackage{amsthm}
% \usepackage{thmtools}
\usepackage{mathtools}
\usepackage{amsfonts}

\usepackage{color}

\usepackage{tikz}

\usepackage{geometry}
\geometry{a5paper, margin=0.1in, right=1cm}

\usepackage{dsfont}

\usepackage{graphicx}
\graphicspath{ {images/} }

\usepackage{faktor}

\usepackage{IEEEtrantools}
\usepackage{enumerate}   
\usepackage[PostScript=dvips]{"/Users/aware/Documents/Courses/diagrams"}


\newtheorem{theorem}{Théorème}[section]
\renewcommand{\thetheorem}{\arabic{theorem}}
\newtheorem{lemme}{Lemme}[section]
\renewcommand{\thelemme}{\arabic{lemme}}
\newtheorem{proposition}{Proposition}[section]
\renewcommand{\theproposition}{\arabic{proposition}}
\newtheorem{notations}{Notations}[section]
\newtheorem{problem}{Problème}[section]
\newtheorem{corollary}{Corollaire}[theorem]
\renewcommand{\thecorollary}{\arabic{corollary}}
\newtheorem{property}{Propriété}[section]
\newtheorem{objective}{Objectif}[section]

\theoremstyle{definition}
\newtheorem{definition}{Définition}[section]
\renewcommand{\thedefinition}{\arabic{definition}}
\newtheorem{exercise}{Exercice}[chapter]
\renewcommand{\theexercise}{\arabic{exercise}}
\newtheorem{example}{Exemple}[chapter]
\renewcommand{\theexample}{\arabic{example}}
\newtheorem*{solution}{Solution}
\newtheorem*{application}{Application}
\newtheorem*{notation}{Notation}
\newtheorem*{vocabulary}{Vocabulaire}
\newtheorem*{properties}{Propriétés}



\theoremstyle{remark}
\newtheorem*{remark}{Remarque}
\newtheorem*{rappel}{Rappel}


\usepackage{etoolbox}
\AtBeginEnvironment{exercise}{\small}
\AtBeginEnvironment{example}{\small}

\usepackage{cases}
\usepackage[red]{mypack}

\usepackage[framemethod=TikZ]{mdframed}

\definecolor{bg}{rgb}{0.4,0.25,0.95}
\definecolor{pagebg}{rgb}{0,0,0.5}
\surroundwithmdframed[
   topline=false,
   rightline=false,
   bottomline=false,
   leftmargin=\parindent,
   skipabove=8pt,
   skipbelow=8pt,
   linecolor=blue,
   innerbottommargin=10pt,
   % backgroundcolor=bg,font=\color{orange}\sffamily, fontcolor=white
]{definition}

\usepackage{empheq}
\usepackage[most]{tcolorbox}

\newtcbox{\mymath}[1][]{%
    nobeforeafter, math upper, tcbox raise base,
    enhanced, colframe=blue!30!black,
    colback=red!10, boxrule=1pt,
    #1}

\usepackage{unixode}


\DeclareMathOperator{\ord}{ord}
\DeclareMathOperator{\orb}{orb}
\DeclareMathOperator{\stab}{stab}
\DeclareMathOperator{\Stab}{stab}
\DeclareMathOperator{\ppcm}{ppcm}
\DeclareMathOperator{\conj}{Conj}
\DeclareMathOperator{\End}{End}
\DeclareMathOperator{\rot}{rot}
\DeclareMathOperator{\trs}{trace}
\DeclareMathOperator{\Ind}{Ind}
\DeclareMathOperator{\mat}{Mat}
\DeclareMathOperator{\id}{Id}
\DeclareMathOperator{\vect}{vect}
\DeclareMathOperator{\img}{img}
\DeclareMathOperator{\cov}{Cov}
\DeclareMathOperator{\dist}{dist}
\DeclareMathOperator{\irr}{Irr}
\DeclareMathOperator{\image}{Im}
\DeclareMathOperator{\pd}{\partial}
\DeclareMathOperator{\epi}{epi}
\DeclareMathOperator{\Argmin}{Argmin}
\DeclareMathOperator{\dom}{dom}
\DeclareMathOperator{\proj}{proj}
\DeclareMathOperator{\ctg}{ctg}
\DeclareMathOperator{\supp}{supp}
\DeclareMathOperator{\argmin}{argmin}
\DeclareMathOperator{\mult}{mult}
\DeclareMathOperator{\ch}{ch}
\DeclareMathOperator{\sh}{sh}
\DeclareMathOperator{\rang}{rang}
\DeclareMathOperator{\diam}{diam}
\DeclareMathOperator{\Epigraphe}{Epigraphe}




\usepackage{xcolor}
\everymath{\color{blue}}
%\everymath{\color[rgb]{0,1,1}}
%\pagecolor[rgb]{0,0,0.5}


\newcommand*{\pdtest}[3][]{\ensuremath{\frac{\partial^{#1} #2}{\partial #3}}}

\newcommand*{\deffunc}[6][]{\ensuremath{
\begin{array}{rcl}
#2 : #3 &\rightarrow& #4\\
#5 &\mapsto& #6
\end{array}
}}

\newcommand{\eqcolon}{\mathrel{\resizebox{\widthof{$\mathord{=}$}}{\height}{ $\!\!=\!\!\resizebox{1.2\width}{0.8\height}{\raisebox{0.23ex}{$\mathop{:}$}}\!\!$ }}}
\newcommand{\coloneq}{\mathrel{\resizebox{\widthof{$\mathord{=}$}}{\height}{ $\!\!\resizebox{1.2\width}{0.8\height}{\raisebox{0.23ex}{$\mathop{:}$}}\!\!=\!\!$ }}}
\newcommand{\eqcolonl}{\ensuremath{\mathrel{=\!\!\mathop{:}}}}
\newcommand{\coloneql}{\ensuremath{\mathrel{\mathop{:} \!\! =}}}
\newcommand{\vc}[1]{% inline column vector
  \left(\begin{smallmatrix}#1\end{smallmatrix}\right)%
}
\newcommand{\vr}[1]{% inline row vector
  \begin{smallmatrix}(\,#1\,)\end{smallmatrix}%
}
\makeatletter
\newcommand*{\defeq}{\ =\mathrel{\rlap{%
                     \raisebox{0.3ex}{$\m@th\cdot$}}%
                     \raisebox{-0.3ex}{$\m@th\cdot$}}%
                     }
\makeatother

\newcommand{\mathcircle}[1]{% inline row vector
 \overset{\circ}{#1}
}
\newcommand{\ulim}{% low limit
 \underline{\lim}
}
\newcommand{\ssi}{% iff
\iff
}
\newcommand{\ps}[2]{
\expval{#1 | #2}
}
\newcommand{\df}[1]{
\mqty{#1}
}
\newcommand{\n}[1]{
\norm{#1}
}
\newcommand{\sys}[1]{
\left\{\smqty{#1}\right.
}


\newcommand{\eqdef}{\ensuremath{\overset{\text{def}}=}}


\def\Circlearrowright{\ensuremath{%
  \rotatebox[origin=c]{230}{$\circlearrowright$}}}

\newcommand\ct[1]{\text{\rmfamily\upshape #1}}
\newcommand\question[1]{ {\color{red} ...!? \small #1}}
\newcommand\caz[1]{\left\{\begin{array} #1 \end{array}\right.}
\newcommand\const{\text{\rmfamily\upshape const}}
\newcommand\toP{ \overset{\pro}{\to}}
\newcommand\toPP{ \overset{\text{PP}}{\to}}
\newcommand{\oeq}{\mathrel{\text{\textcircled{$=$}}}}





\usepackage{xcolor}
% \usepackage[normalem]{ulem}
\usepackage{lipsum}
\makeatletter
% \newcommand\colorwave[1][blue]{\bgroup \markoverwith{\lower3.5\p@\hbox{\sixly \textcolor{#1}{\char58}}}\ULon}
%\font\sixly=lasy6 % does not re-load if already loaded, so no memory problem.

\newmdtheoremenv[
linewidth= 1pt,linecolor= blue,%
leftmargin=20,rightmargin=20,innertopmargin=0pt, innerrightmargin=40,%
tikzsetting = { draw=lightgray, line width = 0.3pt,dashed,%
dash pattern = on 15pt off 3pt},%
splittopskip=\topskip,skipbelow=\baselineskip,%
skipabove=\baselineskip,ntheorem,roundcorner=0pt,
% backgroundcolor=pagebg,font=\color{orange}\sffamily, fontcolor=white
]{examplebox}{Exemple}[section]



\newcommand\R{\mathbb{R}}
\newcommand\Z{\mathbb{Z}}
\newcommand\N{\mathbb{N}}
\newcommand\E{\mathbb{E}}
\newcommand\F{\mathcal{F}}
\newcommand\cH{\mathcal{H}}
\newcommand\V{\mathbb{V}}
\newcommand\dmo{ ^{-1} }
\newcommand\kapa{\kappa}
\newcommand\im{Im}
\newcommand\hs{\mathcal{H}}





\usepackage{soul}

\makeatletter
\newcommand*{\whiten}[1]{\llap{\textcolor{white}{{\the\SOUL@token}}\hspace{#1pt}}}
\DeclareRobustCommand*\myul{%
    \def\SOUL@everyspace{\underline{\space}\kern\z@}%
    \def\SOUL@everytoken{%
     \setbox0=\hbox{\the\SOUL@token}%
     \ifdim\dp0>\z@
        \raisebox{\dp0}{\underline{\phantom{\the\SOUL@token}}}%
        \whiten{1}\whiten{0}%
        \whiten{-1}\whiten{-2}%
        \llap{\the\SOUL@token}%
     \else
        \underline{\the\SOUL@token}%
     \fi}%
\SOUL@}
\makeatother

\newcommand*{\demp}{\fontfamily{lmtt}\selectfont}

\DeclareTextFontCommand{\textdemp}{\demp}

\begin{document}

\ifcomment
Multiline
comment
\fi
\ifcomment
\myul{Typesetting test}
% \color[rgb]{1,1,1}
$∑_i^n≠ 60º±∞π∆¬≈√j∫h≤≥µ$

$\CR \R\pro\ind\pro\gS\pro
\mqty[a&b\\c&d]$
$\pro\mathbb{P}$
$\dd{x}$

  \[
    \alpha(x)=\left\{
                \begin{array}{ll}
                  x\\
                  \frac{1}{1+e^{-kx}}\\
                  \frac{e^x-e^{-x}}{e^x+e^{-x}}
                \end{array}
              \right.
  \]

  $\expval{x}$
  
  $\chi_\rho(ghg\dmo)=\Tr(\rho_{ghg\dmo})=\Tr(\rho_g\circ\rho_h\circ\rho\dmo_g)=\Tr(\rho_h)\overset{\mbox{\scalebox{0.5}{$\Tr(AB)=\Tr(BA)$}}}{=}\chi_\rho(h)$
  	$\mathop{\oplus}_{\substack{x\in X}}$

$\mat(\rho_g)=(a_{ij}(g))_{\scriptsize \substack{1\leq i\leq d \\ 1\leq j\leq d}}$ et $\mat(\rho'_g)=(a'_{ij}(g))_{\scriptsize \substack{1\leq i'\leq d' \\ 1\leq j'\leq d'}}$



\[\int_a^b{\mathbb{R}^2}g(u, v)\dd{P_{XY}}(u, v)=\iint g(u,v) f_{XY}(u, v)\dd \lambda(u) \dd \lambda(v)\]
$$\lim_{x\to\infty} f(x)$$	
$$\iiiint_V \mu(t,u,v,w) \,dt\,du\,dv\,dw$$
$$\sum_{n=1}^{\infty} 2^{-n} = 1$$	
\begin{definition}
	Si $X$ et $Y$ sont 2 v.a. ou definit la \textsc{Covariance} entre $X$ et $Y$ comme
	$\cov(X,Y)\overset{\text{def}}{=}\E\left[(X-\E(X))(Y-\E(Y))\right]=\E(XY)-\E(X)\E(Y)$.
\end{definition}
\fi
\pagebreak

% \tableofcontents

% insert your code here
%\input{./algebra/main.tex}
%\input{./geometrie-differentielle/main.tex}
%\input{./probabilite/main.tex}
%\input{./analyse-fonctionnelle/main.tex}
% \input{./Analyse-convexe-et-dualite-en-optimisation/main.tex}
%\input{./tikz/main.tex}
%\input{./Theorie-du-distributions/main.tex}
%\input{./optimisation/mine.tex}
 \input{./modelisation/main.tex}

% yves.aubry@univ-tln.fr : algebra

\end{document}

%% !TEX encoding = UTF-8 Unicode
% !TEX TS-program = xelatex

\documentclass[french]{report}

%\usepackage[utf8]{inputenc}
%\usepackage[T1]{fontenc}
\usepackage{babel}


\newif\ifcomment
%\commenttrue # Show comments

\usepackage{physics}
\usepackage{amssymb}


\usepackage{amsthm}
% \usepackage{thmtools}
\usepackage{mathtools}
\usepackage{amsfonts}

\usepackage{color}

\usepackage{tikz}

\usepackage{geometry}
\geometry{a5paper, margin=0.1in, right=1cm}

\usepackage{dsfont}

\usepackage{graphicx}
\graphicspath{ {images/} }

\usepackage{faktor}

\usepackage{IEEEtrantools}
\usepackage{enumerate}   
\usepackage[PostScript=dvips]{"/Users/aware/Documents/Courses/diagrams"}


\newtheorem{theorem}{Théorème}[section]
\renewcommand{\thetheorem}{\arabic{theorem}}
\newtheorem{lemme}{Lemme}[section]
\renewcommand{\thelemme}{\arabic{lemme}}
\newtheorem{proposition}{Proposition}[section]
\renewcommand{\theproposition}{\arabic{proposition}}
\newtheorem{notations}{Notations}[section]
\newtheorem{problem}{Problème}[section]
\newtheorem{corollary}{Corollaire}[theorem]
\renewcommand{\thecorollary}{\arabic{corollary}}
\newtheorem{property}{Propriété}[section]
\newtheorem{objective}{Objectif}[section]

\theoremstyle{definition}
\newtheorem{definition}{Définition}[section]
\renewcommand{\thedefinition}{\arabic{definition}}
\newtheorem{exercise}{Exercice}[chapter]
\renewcommand{\theexercise}{\arabic{exercise}}
\newtheorem{example}{Exemple}[chapter]
\renewcommand{\theexample}{\arabic{example}}
\newtheorem*{solution}{Solution}
\newtheorem*{application}{Application}
\newtheorem*{notation}{Notation}
\newtheorem*{vocabulary}{Vocabulaire}
\newtheorem*{properties}{Propriétés}



\theoremstyle{remark}
\newtheorem*{remark}{Remarque}
\newtheorem*{rappel}{Rappel}


\usepackage{etoolbox}
\AtBeginEnvironment{exercise}{\small}
\AtBeginEnvironment{example}{\small}

\usepackage{cases}
\usepackage[red]{mypack}

\usepackage[framemethod=TikZ]{mdframed}

\definecolor{bg}{rgb}{0.4,0.25,0.95}
\definecolor{pagebg}{rgb}{0,0,0.5}
\surroundwithmdframed[
   topline=false,
   rightline=false,
   bottomline=false,
   leftmargin=\parindent,
   skipabove=8pt,
   skipbelow=8pt,
   linecolor=blue,
   innerbottommargin=10pt,
   % backgroundcolor=bg,font=\color{orange}\sffamily, fontcolor=white
]{definition}

\usepackage{empheq}
\usepackage[most]{tcolorbox}

\newtcbox{\mymath}[1][]{%
    nobeforeafter, math upper, tcbox raise base,
    enhanced, colframe=blue!30!black,
    colback=red!10, boxrule=1pt,
    #1}

\usepackage{unixode}


\DeclareMathOperator{\ord}{ord}
\DeclareMathOperator{\orb}{orb}
\DeclareMathOperator{\stab}{stab}
\DeclareMathOperator{\Stab}{stab}
\DeclareMathOperator{\ppcm}{ppcm}
\DeclareMathOperator{\conj}{Conj}
\DeclareMathOperator{\End}{End}
\DeclareMathOperator{\rot}{rot}
\DeclareMathOperator{\trs}{trace}
\DeclareMathOperator{\Ind}{Ind}
\DeclareMathOperator{\mat}{Mat}
\DeclareMathOperator{\id}{Id}
\DeclareMathOperator{\vect}{vect}
\DeclareMathOperator{\img}{img}
\DeclareMathOperator{\cov}{Cov}
\DeclareMathOperator{\dist}{dist}
\DeclareMathOperator{\irr}{Irr}
\DeclareMathOperator{\image}{Im}
\DeclareMathOperator{\pd}{\partial}
\DeclareMathOperator{\epi}{epi}
\DeclareMathOperator{\Argmin}{Argmin}
\DeclareMathOperator{\dom}{dom}
\DeclareMathOperator{\proj}{proj}
\DeclareMathOperator{\ctg}{ctg}
\DeclareMathOperator{\supp}{supp}
\DeclareMathOperator{\argmin}{argmin}
\DeclareMathOperator{\mult}{mult}
\DeclareMathOperator{\ch}{ch}
\DeclareMathOperator{\sh}{sh}
\DeclareMathOperator{\rang}{rang}
\DeclareMathOperator{\diam}{diam}
\DeclareMathOperator{\Epigraphe}{Epigraphe}




\usepackage{xcolor}
\everymath{\color{blue}}
%\everymath{\color[rgb]{0,1,1}}
%\pagecolor[rgb]{0,0,0.5}


\newcommand*{\pdtest}[3][]{\ensuremath{\frac{\partial^{#1} #2}{\partial #3}}}

\newcommand*{\deffunc}[6][]{\ensuremath{
\begin{array}{rcl}
#2 : #3 &\rightarrow& #4\\
#5 &\mapsto& #6
\end{array}
}}

\newcommand{\eqcolon}{\mathrel{\resizebox{\widthof{$\mathord{=}$}}{\height}{ $\!\!=\!\!\resizebox{1.2\width}{0.8\height}{\raisebox{0.23ex}{$\mathop{:}$}}\!\!$ }}}
\newcommand{\coloneq}{\mathrel{\resizebox{\widthof{$\mathord{=}$}}{\height}{ $\!\!\resizebox{1.2\width}{0.8\height}{\raisebox{0.23ex}{$\mathop{:}$}}\!\!=\!\!$ }}}
\newcommand{\eqcolonl}{\ensuremath{\mathrel{=\!\!\mathop{:}}}}
\newcommand{\coloneql}{\ensuremath{\mathrel{\mathop{:} \!\! =}}}
\newcommand{\vc}[1]{% inline column vector
  \left(\begin{smallmatrix}#1\end{smallmatrix}\right)%
}
\newcommand{\vr}[1]{% inline row vector
  \begin{smallmatrix}(\,#1\,)\end{smallmatrix}%
}
\makeatletter
\newcommand*{\defeq}{\ =\mathrel{\rlap{%
                     \raisebox{0.3ex}{$\m@th\cdot$}}%
                     \raisebox{-0.3ex}{$\m@th\cdot$}}%
                     }
\makeatother

\newcommand{\mathcircle}[1]{% inline row vector
 \overset{\circ}{#1}
}
\newcommand{\ulim}{% low limit
 \underline{\lim}
}
\newcommand{\ssi}{% iff
\iff
}
\newcommand{\ps}[2]{
\expval{#1 | #2}
}
\newcommand{\df}[1]{
\mqty{#1}
}
\newcommand{\n}[1]{
\norm{#1}
}
\newcommand{\sys}[1]{
\left\{\smqty{#1}\right.
}


\newcommand{\eqdef}{\ensuremath{\overset{\text{def}}=}}


\def\Circlearrowright{\ensuremath{%
  \rotatebox[origin=c]{230}{$\circlearrowright$}}}

\newcommand\ct[1]{\text{\rmfamily\upshape #1}}
\newcommand\question[1]{ {\color{red} ...!? \small #1}}
\newcommand\caz[1]{\left\{\begin{array} #1 \end{array}\right.}
\newcommand\const{\text{\rmfamily\upshape const}}
\newcommand\toP{ \overset{\pro}{\to}}
\newcommand\toPP{ \overset{\text{PP}}{\to}}
\newcommand{\oeq}{\mathrel{\text{\textcircled{$=$}}}}





\usepackage{xcolor}
% \usepackage[normalem]{ulem}
\usepackage{lipsum}
\makeatletter
% \newcommand\colorwave[1][blue]{\bgroup \markoverwith{\lower3.5\p@\hbox{\sixly \textcolor{#1}{\char58}}}\ULon}
%\font\sixly=lasy6 % does not re-load if already loaded, so no memory problem.

\newmdtheoremenv[
linewidth= 1pt,linecolor= blue,%
leftmargin=20,rightmargin=20,innertopmargin=0pt, innerrightmargin=40,%
tikzsetting = { draw=lightgray, line width = 0.3pt,dashed,%
dash pattern = on 15pt off 3pt},%
splittopskip=\topskip,skipbelow=\baselineskip,%
skipabove=\baselineskip,ntheorem,roundcorner=0pt,
% backgroundcolor=pagebg,font=\color{orange}\sffamily, fontcolor=white
]{examplebox}{Exemple}[section]



\newcommand\R{\mathbb{R}}
\newcommand\Z{\mathbb{Z}}
\newcommand\N{\mathbb{N}}
\newcommand\E{\mathbb{E}}
\newcommand\F{\mathcal{F}}
\newcommand\cH{\mathcal{H}}
\newcommand\V{\mathbb{V}}
\newcommand\dmo{ ^{-1} }
\newcommand\kapa{\kappa}
\newcommand\im{Im}
\newcommand\hs{\mathcal{H}}





\usepackage{soul}

\makeatletter
\newcommand*{\whiten}[1]{\llap{\textcolor{white}{{\the\SOUL@token}}\hspace{#1pt}}}
\DeclareRobustCommand*\myul{%
    \def\SOUL@everyspace{\underline{\space}\kern\z@}%
    \def\SOUL@everytoken{%
     \setbox0=\hbox{\the\SOUL@token}%
     \ifdim\dp0>\z@
        \raisebox{\dp0}{\underline{\phantom{\the\SOUL@token}}}%
        \whiten{1}\whiten{0}%
        \whiten{-1}\whiten{-2}%
        \llap{\the\SOUL@token}%
     \else
        \underline{\the\SOUL@token}%
     \fi}%
\SOUL@}
\makeatother

\newcommand*{\demp}{\fontfamily{lmtt}\selectfont}

\DeclareTextFontCommand{\textdemp}{\demp}

\begin{document}

\ifcomment
Multiline
comment
\fi
\ifcomment
\myul{Typesetting test}
% \color[rgb]{1,1,1}
$∑_i^n≠ 60º±∞π∆¬≈√j∫h≤≥µ$

$\CR \R\pro\ind\pro\gS\pro
\mqty[a&b\\c&d]$
$\pro\mathbb{P}$
$\dd{x}$

  \[
    \alpha(x)=\left\{
                \begin{array}{ll}
                  x\\
                  \frac{1}{1+e^{-kx}}\\
                  \frac{e^x-e^{-x}}{e^x+e^{-x}}
                \end{array}
              \right.
  \]

  $\expval{x}$
  
  $\chi_\rho(ghg\dmo)=\Tr(\rho_{ghg\dmo})=\Tr(\rho_g\circ\rho_h\circ\rho\dmo_g)=\Tr(\rho_h)\overset{\mbox{\scalebox{0.5}{$\Tr(AB)=\Tr(BA)$}}}{=}\chi_\rho(h)$
  	$\mathop{\oplus}_{\substack{x\in X}}$

$\mat(\rho_g)=(a_{ij}(g))_{\scriptsize \substack{1\leq i\leq d \\ 1\leq j\leq d}}$ et $\mat(\rho'_g)=(a'_{ij}(g))_{\scriptsize \substack{1\leq i'\leq d' \\ 1\leq j'\leq d'}}$



\[\int_a^b{\mathbb{R}^2}g(u, v)\dd{P_{XY}}(u, v)=\iint g(u,v) f_{XY}(u, v)\dd \lambda(u) \dd \lambda(v)\]
$$\lim_{x\to\infty} f(x)$$	
$$\iiiint_V \mu(t,u,v,w) \,dt\,du\,dv\,dw$$
$$\sum_{n=1}^{\infty} 2^{-n} = 1$$	
\begin{definition}
	Si $X$ et $Y$ sont 2 v.a. ou definit la \textsc{Covariance} entre $X$ et $Y$ comme
	$\cov(X,Y)\overset{\text{def}}{=}\E\left[(X-\E(X))(Y-\E(Y))\right]=\E(XY)-\E(X)\E(Y)$.
\end{definition}
\fi
\pagebreak

% \tableofcontents

% insert your code here
%\input{./algebra/main.tex}
%\input{./geometrie-differentielle/main.tex}
%\input{./probabilite/main.tex}
%\input{./analyse-fonctionnelle/main.tex}
% \input{./Analyse-convexe-et-dualite-en-optimisation/main.tex}
%\input{./tikz/main.tex}
%\input{./Theorie-du-distributions/main.tex}
%\input{./optimisation/mine.tex}
 \input{./modelisation/main.tex}

% yves.aubry@univ-tln.fr : algebra

\end{document}

%% !TEX encoding = UTF-8 Unicode
% !TEX TS-program = xelatex

\documentclass[french]{report}

%\usepackage[utf8]{inputenc}
%\usepackage[T1]{fontenc}
\usepackage{babel}


\newif\ifcomment
%\commenttrue # Show comments

\usepackage{physics}
\usepackage{amssymb}


\usepackage{amsthm}
% \usepackage{thmtools}
\usepackage{mathtools}
\usepackage{amsfonts}

\usepackage{color}

\usepackage{tikz}

\usepackage{geometry}
\geometry{a5paper, margin=0.1in, right=1cm}

\usepackage{dsfont}

\usepackage{graphicx}
\graphicspath{ {images/} }

\usepackage{faktor}

\usepackage{IEEEtrantools}
\usepackage{enumerate}   
\usepackage[PostScript=dvips]{"/Users/aware/Documents/Courses/diagrams"}


\newtheorem{theorem}{Théorème}[section]
\renewcommand{\thetheorem}{\arabic{theorem}}
\newtheorem{lemme}{Lemme}[section]
\renewcommand{\thelemme}{\arabic{lemme}}
\newtheorem{proposition}{Proposition}[section]
\renewcommand{\theproposition}{\arabic{proposition}}
\newtheorem{notations}{Notations}[section]
\newtheorem{problem}{Problème}[section]
\newtheorem{corollary}{Corollaire}[theorem]
\renewcommand{\thecorollary}{\arabic{corollary}}
\newtheorem{property}{Propriété}[section]
\newtheorem{objective}{Objectif}[section]

\theoremstyle{definition}
\newtheorem{definition}{Définition}[section]
\renewcommand{\thedefinition}{\arabic{definition}}
\newtheorem{exercise}{Exercice}[chapter]
\renewcommand{\theexercise}{\arabic{exercise}}
\newtheorem{example}{Exemple}[chapter]
\renewcommand{\theexample}{\arabic{example}}
\newtheorem*{solution}{Solution}
\newtheorem*{application}{Application}
\newtheorem*{notation}{Notation}
\newtheorem*{vocabulary}{Vocabulaire}
\newtheorem*{properties}{Propriétés}



\theoremstyle{remark}
\newtheorem*{remark}{Remarque}
\newtheorem*{rappel}{Rappel}


\usepackage{etoolbox}
\AtBeginEnvironment{exercise}{\small}
\AtBeginEnvironment{example}{\small}

\usepackage{cases}
\usepackage[red]{mypack}

\usepackage[framemethod=TikZ]{mdframed}

\definecolor{bg}{rgb}{0.4,0.25,0.95}
\definecolor{pagebg}{rgb}{0,0,0.5}
\surroundwithmdframed[
   topline=false,
   rightline=false,
   bottomline=false,
   leftmargin=\parindent,
   skipabove=8pt,
   skipbelow=8pt,
   linecolor=blue,
   innerbottommargin=10pt,
   % backgroundcolor=bg,font=\color{orange}\sffamily, fontcolor=white
]{definition}

\usepackage{empheq}
\usepackage[most]{tcolorbox}

\newtcbox{\mymath}[1][]{%
    nobeforeafter, math upper, tcbox raise base,
    enhanced, colframe=blue!30!black,
    colback=red!10, boxrule=1pt,
    #1}

\usepackage{unixode}


\DeclareMathOperator{\ord}{ord}
\DeclareMathOperator{\orb}{orb}
\DeclareMathOperator{\stab}{stab}
\DeclareMathOperator{\Stab}{stab}
\DeclareMathOperator{\ppcm}{ppcm}
\DeclareMathOperator{\conj}{Conj}
\DeclareMathOperator{\End}{End}
\DeclareMathOperator{\rot}{rot}
\DeclareMathOperator{\trs}{trace}
\DeclareMathOperator{\Ind}{Ind}
\DeclareMathOperator{\mat}{Mat}
\DeclareMathOperator{\id}{Id}
\DeclareMathOperator{\vect}{vect}
\DeclareMathOperator{\img}{img}
\DeclareMathOperator{\cov}{Cov}
\DeclareMathOperator{\dist}{dist}
\DeclareMathOperator{\irr}{Irr}
\DeclareMathOperator{\image}{Im}
\DeclareMathOperator{\pd}{\partial}
\DeclareMathOperator{\epi}{epi}
\DeclareMathOperator{\Argmin}{Argmin}
\DeclareMathOperator{\dom}{dom}
\DeclareMathOperator{\proj}{proj}
\DeclareMathOperator{\ctg}{ctg}
\DeclareMathOperator{\supp}{supp}
\DeclareMathOperator{\argmin}{argmin}
\DeclareMathOperator{\mult}{mult}
\DeclareMathOperator{\ch}{ch}
\DeclareMathOperator{\sh}{sh}
\DeclareMathOperator{\rang}{rang}
\DeclareMathOperator{\diam}{diam}
\DeclareMathOperator{\Epigraphe}{Epigraphe}




\usepackage{xcolor}
\everymath{\color{blue}}
%\everymath{\color[rgb]{0,1,1}}
%\pagecolor[rgb]{0,0,0.5}


\newcommand*{\pdtest}[3][]{\ensuremath{\frac{\partial^{#1} #2}{\partial #3}}}

\newcommand*{\deffunc}[6][]{\ensuremath{
\begin{array}{rcl}
#2 : #3 &\rightarrow& #4\\
#5 &\mapsto& #6
\end{array}
}}

\newcommand{\eqcolon}{\mathrel{\resizebox{\widthof{$\mathord{=}$}}{\height}{ $\!\!=\!\!\resizebox{1.2\width}{0.8\height}{\raisebox{0.23ex}{$\mathop{:}$}}\!\!$ }}}
\newcommand{\coloneq}{\mathrel{\resizebox{\widthof{$\mathord{=}$}}{\height}{ $\!\!\resizebox{1.2\width}{0.8\height}{\raisebox{0.23ex}{$\mathop{:}$}}\!\!=\!\!$ }}}
\newcommand{\eqcolonl}{\ensuremath{\mathrel{=\!\!\mathop{:}}}}
\newcommand{\coloneql}{\ensuremath{\mathrel{\mathop{:} \!\! =}}}
\newcommand{\vc}[1]{% inline column vector
  \left(\begin{smallmatrix}#1\end{smallmatrix}\right)%
}
\newcommand{\vr}[1]{% inline row vector
  \begin{smallmatrix}(\,#1\,)\end{smallmatrix}%
}
\makeatletter
\newcommand*{\defeq}{\ =\mathrel{\rlap{%
                     \raisebox{0.3ex}{$\m@th\cdot$}}%
                     \raisebox{-0.3ex}{$\m@th\cdot$}}%
                     }
\makeatother

\newcommand{\mathcircle}[1]{% inline row vector
 \overset{\circ}{#1}
}
\newcommand{\ulim}{% low limit
 \underline{\lim}
}
\newcommand{\ssi}{% iff
\iff
}
\newcommand{\ps}[2]{
\expval{#1 | #2}
}
\newcommand{\df}[1]{
\mqty{#1}
}
\newcommand{\n}[1]{
\norm{#1}
}
\newcommand{\sys}[1]{
\left\{\smqty{#1}\right.
}


\newcommand{\eqdef}{\ensuremath{\overset{\text{def}}=}}


\def\Circlearrowright{\ensuremath{%
  \rotatebox[origin=c]{230}{$\circlearrowright$}}}

\newcommand\ct[1]{\text{\rmfamily\upshape #1}}
\newcommand\question[1]{ {\color{red} ...!? \small #1}}
\newcommand\caz[1]{\left\{\begin{array} #1 \end{array}\right.}
\newcommand\const{\text{\rmfamily\upshape const}}
\newcommand\toP{ \overset{\pro}{\to}}
\newcommand\toPP{ \overset{\text{PP}}{\to}}
\newcommand{\oeq}{\mathrel{\text{\textcircled{$=$}}}}





\usepackage{xcolor}
% \usepackage[normalem]{ulem}
\usepackage{lipsum}
\makeatletter
% \newcommand\colorwave[1][blue]{\bgroup \markoverwith{\lower3.5\p@\hbox{\sixly \textcolor{#1}{\char58}}}\ULon}
%\font\sixly=lasy6 % does not re-load if already loaded, so no memory problem.

\newmdtheoremenv[
linewidth= 1pt,linecolor= blue,%
leftmargin=20,rightmargin=20,innertopmargin=0pt, innerrightmargin=40,%
tikzsetting = { draw=lightgray, line width = 0.3pt,dashed,%
dash pattern = on 15pt off 3pt},%
splittopskip=\topskip,skipbelow=\baselineskip,%
skipabove=\baselineskip,ntheorem,roundcorner=0pt,
% backgroundcolor=pagebg,font=\color{orange}\sffamily, fontcolor=white
]{examplebox}{Exemple}[section]



\newcommand\R{\mathbb{R}}
\newcommand\Z{\mathbb{Z}}
\newcommand\N{\mathbb{N}}
\newcommand\E{\mathbb{E}}
\newcommand\F{\mathcal{F}}
\newcommand\cH{\mathcal{H}}
\newcommand\V{\mathbb{V}}
\newcommand\dmo{ ^{-1} }
\newcommand\kapa{\kappa}
\newcommand\im{Im}
\newcommand\hs{\mathcal{H}}





\usepackage{soul}

\makeatletter
\newcommand*{\whiten}[1]{\llap{\textcolor{white}{{\the\SOUL@token}}\hspace{#1pt}}}
\DeclareRobustCommand*\myul{%
    \def\SOUL@everyspace{\underline{\space}\kern\z@}%
    \def\SOUL@everytoken{%
     \setbox0=\hbox{\the\SOUL@token}%
     \ifdim\dp0>\z@
        \raisebox{\dp0}{\underline{\phantom{\the\SOUL@token}}}%
        \whiten{1}\whiten{0}%
        \whiten{-1}\whiten{-2}%
        \llap{\the\SOUL@token}%
     \else
        \underline{\the\SOUL@token}%
     \fi}%
\SOUL@}
\makeatother

\newcommand*{\demp}{\fontfamily{lmtt}\selectfont}

\DeclareTextFontCommand{\textdemp}{\demp}

\begin{document}

\ifcomment
Multiline
comment
\fi
\ifcomment
\myul{Typesetting test}
% \color[rgb]{1,1,1}
$∑_i^n≠ 60º±∞π∆¬≈√j∫h≤≥µ$

$\CR \R\pro\ind\pro\gS\pro
\mqty[a&b\\c&d]$
$\pro\mathbb{P}$
$\dd{x}$

  \[
    \alpha(x)=\left\{
                \begin{array}{ll}
                  x\\
                  \frac{1}{1+e^{-kx}}\\
                  \frac{e^x-e^{-x}}{e^x+e^{-x}}
                \end{array}
              \right.
  \]

  $\expval{x}$
  
  $\chi_\rho(ghg\dmo)=\Tr(\rho_{ghg\dmo})=\Tr(\rho_g\circ\rho_h\circ\rho\dmo_g)=\Tr(\rho_h)\overset{\mbox{\scalebox{0.5}{$\Tr(AB)=\Tr(BA)$}}}{=}\chi_\rho(h)$
  	$\mathop{\oplus}_{\substack{x\in X}}$

$\mat(\rho_g)=(a_{ij}(g))_{\scriptsize \substack{1\leq i\leq d \\ 1\leq j\leq d}}$ et $\mat(\rho'_g)=(a'_{ij}(g))_{\scriptsize \substack{1\leq i'\leq d' \\ 1\leq j'\leq d'}}$



\[\int_a^b{\mathbb{R}^2}g(u, v)\dd{P_{XY}}(u, v)=\iint g(u,v) f_{XY}(u, v)\dd \lambda(u) \dd \lambda(v)\]
$$\lim_{x\to\infty} f(x)$$	
$$\iiiint_V \mu(t,u,v,w) \,dt\,du\,dv\,dw$$
$$\sum_{n=1}^{\infty} 2^{-n} = 1$$	
\begin{definition}
	Si $X$ et $Y$ sont 2 v.a. ou definit la \textsc{Covariance} entre $X$ et $Y$ comme
	$\cov(X,Y)\overset{\text{def}}{=}\E\left[(X-\E(X))(Y-\E(Y))\right]=\E(XY)-\E(X)\E(Y)$.
\end{definition}
\fi
\pagebreak

% \tableofcontents

% insert your code here
%\input{./algebra/main.tex}
%\input{./geometrie-differentielle/main.tex}
%\input{./probabilite/main.tex}
%\input{./analyse-fonctionnelle/main.tex}
% \input{./Analyse-convexe-et-dualite-en-optimisation/main.tex}
%\input{./tikz/main.tex}
%\input{./Theorie-du-distributions/main.tex}
%\input{./optimisation/mine.tex}
 \input{./modelisation/main.tex}

% yves.aubry@univ-tln.fr : algebra

\end{document}

%% !TEX encoding = UTF-8 Unicode
% !TEX TS-program = xelatex

\documentclass[french]{report}

%\usepackage[utf8]{inputenc}
%\usepackage[T1]{fontenc}
\usepackage{babel}


\newif\ifcomment
%\commenttrue # Show comments

\usepackage{physics}
\usepackage{amssymb}


\usepackage{amsthm}
% \usepackage{thmtools}
\usepackage{mathtools}
\usepackage{amsfonts}

\usepackage{color}

\usepackage{tikz}

\usepackage{geometry}
\geometry{a5paper, margin=0.1in, right=1cm}

\usepackage{dsfont}

\usepackage{graphicx}
\graphicspath{ {images/} }

\usepackage{faktor}

\usepackage{IEEEtrantools}
\usepackage{enumerate}   
\usepackage[PostScript=dvips]{"/Users/aware/Documents/Courses/diagrams"}


\newtheorem{theorem}{Théorème}[section]
\renewcommand{\thetheorem}{\arabic{theorem}}
\newtheorem{lemme}{Lemme}[section]
\renewcommand{\thelemme}{\arabic{lemme}}
\newtheorem{proposition}{Proposition}[section]
\renewcommand{\theproposition}{\arabic{proposition}}
\newtheorem{notations}{Notations}[section]
\newtheorem{problem}{Problème}[section]
\newtheorem{corollary}{Corollaire}[theorem]
\renewcommand{\thecorollary}{\arabic{corollary}}
\newtheorem{property}{Propriété}[section]
\newtheorem{objective}{Objectif}[section]

\theoremstyle{definition}
\newtheorem{definition}{Définition}[section]
\renewcommand{\thedefinition}{\arabic{definition}}
\newtheorem{exercise}{Exercice}[chapter]
\renewcommand{\theexercise}{\arabic{exercise}}
\newtheorem{example}{Exemple}[chapter]
\renewcommand{\theexample}{\arabic{example}}
\newtheorem*{solution}{Solution}
\newtheorem*{application}{Application}
\newtheorem*{notation}{Notation}
\newtheorem*{vocabulary}{Vocabulaire}
\newtheorem*{properties}{Propriétés}



\theoremstyle{remark}
\newtheorem*{remark}{Remarque}
\newtheorem*{rappel}{Rappel}


\usepackage{etoolbox}
\AtBeginEnvironment{exercise}{\small}
\AtBeginEnvironment{example}{\small}

\usepackage{cases}
\usepackage[red]{mypack}

\usepackage[framemethod=TikZ]{mdframed}

\definecolor{bg}{rgb}{0.4,0.25,0.95}
\definecolor{pagebg}{rgb}{0,0,0.5}
\surroundwithmdframed[
   topline=false,
   rightline=false,
   bottomline=false,
   leftmargin=\parindent,
   skipabove=8pt,
   skipbelow=8pt,
   linecolor=blue,
   innerbottommargin=10pt,
   % backgroundcolor=bg,font=\color{orange}\sffamily, fontcolor=white
]{definition}

\usepackage{empheq}
\usepackage[most]{tcolorbox}

\newtcbox{\mymath}[1][]{%
    nobeforeafter, math upper, tcbox raise base,
    enhanced, colframe=blue!30!black,
    colback=red!10, boxrule=1pt,
    #1}

\usepackage{unixode}


\DeclareMathOperator{\ord}{ord}
\DeclareMathOperator{\orb}{orb}
\DeclareMathOperator{\stab}{stab}
\DeclareMathOperator{\Stab}{stab}
\DeclareMathOperator{\ppcm}{ppcm}
\DeclareMathOperator{\conj}{Conj}
\DeclareMathOperator{\End}{End}
\DeclareMathOperator{\rot}{rot}
\DeclareMathOperator{\trs}{trace}
\DeclareMathOperator{\Ind}{Ind}
\DeclareMathOperator{\mat}{Mat}
\DeclareMathOperator{\id}{Id}
\DeclareMathOperator{\vect}{vect}
\DeclareMathOperator{\img}{img}
\DeclareMathOperator{\cov}{Cov}
\DeclareMathOperator{\dist}{dist}
\DeclareMathOperator{\irr}{Irr}
\DeclareMathOperator{\image}{Im}
\DeclareMathOperator{\pd}{\partial}
\DeclareMathOperator{\epi}{epi}
\DeclareMathOperator{\Argmin}{Argmin}
\DeclareMathOperator{\dom}{dom}
\DeclareMathOperator{\proj}{proj}
\DeclareMathOperator{\ctg}{ctg}
\DeclareMathOperator{\supp}{supp}
\DeclareMathOperator{\argmin}{argmin}
\DeclareMathOperator{\mult}{mult}
\DeclareMathOperator{\ch}{ch}
\DeclareMathOperator{\sh}{sh}
\DeclareMathOperator{\rang}{rang}
\DeclareMathOperator{\diam}{diam}
\DeclareMathOperator{\Epigraphe}{Epigraphe}




\usepackage{xcolor}
\everymath{\color{blue}}
%\everymath{\color[rgb]{0,1,1}}
%\pagecolor[rgb]{0,0,0.5}


\newcommand*{\pdtest}[3][]{\ensuremath{\frac{\partial^{#1} #2}{\partial #3}}}

\newcommand*{\deffunc}[6][]{\ensuremath{
\begin{array}{rcl}
#2 : #3 &\rightarrow& #4\\
#5 &\mapsto& #6
\end{array}
}}

\newcommand{\eqcolon}{\mathrel{\resizebox{\widthof{$\mathord{=}$}}{\height}{ $\!\!=\!\!\resizebox{1.2\width}{0.8\height}{\raisebox{0.23ex}{$\mathop{:}$}}\!\!$ }}}
\newcommand{\coloneq}{\mathrel{\resizebox{\widthof{$\mathord{=}$}}{\height}{ $\!\!\resizebox{1.2\width}{0.8\height}{\raisebox{0.23ex}{$\mathop{:}$}}\!\!=\!\!$ }}}
\newcommand{\eqcolonl}{\ensuremath{\mathrel{=\!\!\mathop{:}}}}
\newcommand{\coloneql}{\ensuremath{\mathrel{\mathop{:} \!\! =}}}
\newcommand{\vc}[1]{% inline column vector
  \left(\begin{smallmatrix}#1\end{smallmatrix}\right)%
}
\newcommand{\vr}[1]{% inline row vector
  \begin{smallmatrix}(\,#1\,)\end{smallmatrix}%
}
\makeatletter
\newcommand*{\defeq}{\ =\mathrel{\rlap{%
                     \raisebox{0.3ex}{$\m@th\cdot$}}%
                     \raisebox{-0.3ex}{$\m@th\cdot$}}%
                     }
\makeatother

\newcommand{\mathcircle}[1]{% inline row vector
 \overset{\circ}{#1}
}
\newcommand{\ulim}{% low limit
 \underline{\lim}
}
\newcommand{\ssi}{% iff
\iff
}
\newcommand{\ps}[2]{
\expval{#1 | #2}
}
\newcommand{\df}[1]{
\mqty{#1}
}
\newcommand{\n}[1]{
\norm{#1}
}
\newcommand{\sys}[1]{
\left\{\smqty{#1}\right.
}


\newcommand{\eqdef}{\ensuremath{\overset{\text{def}}=}}


\def\Circlearrowright{\ensuremath{%
  \rotatebox[origin=c]{230}{$\circlearrowright$}}}

\newcommand\ct[1]{\text{\rmfamily\upshape #1}}
\newcommand\question[1]{ {\color{red} ...!? \small #1}}
\newcommand\caz[1]{\left\{\begin{array} #1 \end{array}\right.}
\newcommand\const{\text{\rmfamily\upshape const}}
\newcommand\toP{ \overset{\pro}{\to}}
\newcommand\toPP{ \overset{\text{PP}}{\to}}
\newcommand{\oeq}{\mathrel{\text{\textcircled{$=$}}}}





\usepackage{xcolor}
% \usepackage[normalem]{ulem}
\usepackage{lipsum}
\makeatletter
% \newcommand\colorwave[1][blue]{\bgroup \markoverwith{\lower3.5\p@\hbox{\sixly \textcolor{#1}{\char58}}}\ULon}
%\font\sixly=lasy6 % does not re-load if already loaded, so no memory problem.

\newmdtheoremenv[
linewidth= 1pt,linecolor= blue,%
leftmargin=20,rightmargin=20,innertopmargin=0pt, innerrightmargin=40,%
tikzsetting = { draw=lightgray, line width = 0.3pt,dashed,%
dash pattern = on 15pt off 3pt},%
splittopskip=\topskip,skipbelow=\baselineskip,%
skipabove=\baselineskip,ntheorem,roundcorner=0pt,
% backgroundcolor=pagebg,font=\color{orange}\sffamily, fontcolor=white
]{examplebox}{Exemple}[section]



\newcommand\R{\mathbb{R}}
\newcommand\Z{\mathbb{Z}}
\newcommand\N{\mathbb{N}}
\newcommand\E{\mathbb{E}}
\newcommand\F{\mathcal{F}}
\newcommand\cH{\mathcal{H}}
\newcommand\V{\mathbb{V}}
\newcommand\dmo{ ^{-1} }
\newcommand\kapa{\kappa}
\newcommand\im{Im}
\newcommand\hs{\mathcal{H}}





\usepackage{soul}

\makeatletter
\newcommand*{\whiten}[1]{\llap{\textcolor{white}{{\the\SOUL@token}}\hspace{#1pt}}}
\DeclareRobustCommand*\myul{%
    \def\SOUL@everyspace{\underline{\space}\kern\z@}%
    \def\SOUL@everytoken{%
     \setbox0=\hbox{\the\SOUL@token}%
     \ifdim\dp0>\z@
        \raisebox{\dp0}{\underline{\phantom{\the\SOUL@token}}}%
        \whiten{1}\whiten{0}%
        \whiten{-1}\whiten{-2}%
        \llap{\the\SOUL@token}%
     \else
        \underline{\the\SOUL@token}%
     \fi}%
\SOUL@}
\makeatother

\newcommand*{\demp}{\fontfamily{lmtt}\selectfont}

\DeclareTextFontCommand{\textdemp}{\demp}

\begin{document}

\ifcomment
Multiline
comment
\fi
\ifcomment
\myul{Typesetting test}
% \color[rgb]{1,1,1}
$∑_i^n≠ 60º±∞π∆¬≈√j∫h≤≥µ$

$\CR \R\pro\ind\pro\gS\pro
\mqty[a&b\\c&d]$
$\pro\mathbb{P}$
$\dd{x}$

  \[
    \alpha(x)=\left\{
                \begin{array}{ll}
                  x\\
                  \frac{1}{1+e^{-kx}}\\
                  \frac{e^x-e^{-x}}{e^x+e^{-x}}
                \end{array}
              \right.
  \]

  $\expval{x}$
  
  $\chi_\rho(ghg\dmo)=\Tr(\rho_{ghg\dmo})=\Tr(\rho_g\circ\rho_h\circ\rho\dmo_g)=\Tr(\rho_h)\overset{\mbox{\scalebox{0.5}{$\Tr(AB)=\Tr(BA)$}}}{=}\chi_\rho(h)$
  	$\mathop{\oplus}_{\substack{x\in X}}$

$\mat(\rho_g)=(a_{ij}(g))_{\scriptsize \substack{1\leq i\leq d \\ 1\leq j\leq d}}$ et $\mat(\rho'_g)=(a'_{ij}(g))_{\scriptsize \substack{1\leq i'\leq d' \\ 1\leq j'\leq d'}}$



\[\int_a^b{\mathbb{R}^2}g(u, v)\dd{P_{XY}}(u, v)=\iint g(u,v) f_{XY}(u, v)\dd \lambda(u) \dd \lambda(v)\]
$$\lim_{x\to\infty} f(x)$$	
$$\iiiint_V \mu(t,u,v,w) \,dt\,du\,dv\,dw$$
$$\sum_{n=1}^{\infty} 2^{-n} = 1$$	
\begin{definition}
	Si $X$ et $Y$ sont 2 v.a. ou definit la \textsc{Covariance} entre $X$ et $Y$ comme
	$\cov(X,Y)\overset{\text{def}}{=}\E\left[(X-\E(X))(Y-\E(Y))\right]=\E(XY)-\E(X)\E(Y)$.
\end{definition}
\fi
\pagebreak

% \tableofcontents

% insert your code here
%\input{./algebra/main.tex}
%\input{./geometrie-differentielle/main.tex}
%\input{./probabilite/main.tex}
%\input{./analyse-fonctionnelle/main.tex}
% \input{./Analyse-convexe-et-dualite-en-optimisation/main.tex}
%\input{./tikz/main.tex}
%\input{./Theorie-du-distributions/main.tex}
%\input{./optimisation/mine.tex}
 \input{./modelisation/main.tex}

% yves.aubry@univ-tln.fr : algebra

\end{document}

% % !TEX encoding = UTF-8 Unicode
% !TEX TS-program = xelatex

\documentclass[french]{report}

%\usepackage[utf8]{inputenc}
%\usepackage[T1]{fontenc}
\usepackage{babel}


\newif\ifcomment
%\commenttrue # Show comments

\usepackage{physics}
\usepackage{amssymb}


\usepackage{amsthm}
% \usepackage{thmtools}
\usepackage{mathtools}
\usepackage{amsfonts}

\usepackage{color}

\usepackage{tikz}

\usepackage{geometry}
\geometry{a5paper, margin=0.1in, right=1cm}

\usepackage{dsfont}

\usepackage{graphicx}
\graphicspath{ {images/} }

\usepackage{faktor}

\usepackage{IEEEtrantools}
\usepackage{enumerate}   
\usepackage[PostScript=dvips]{"/Users/aware/Documents/Courses/diagrams"}


\newtheorem{theorem}{Théorème}[section]
\renewcommand{\thetheorem}{\arabic{theorem}}
\newtheorem{lemme}{Lemme}[section]
\renewcommand{\thelemme}{\arabic{lemme}}
\newtheorem{proposition}{Proposition}[section]
\renewcommand{\theproposition}{\arabic{proposition}}
\newtheorem{notations}{Notations}[section]
\newtheorem{problem}{Problème}[section]
\newtheorem{corollary}{Corollaire}[theorem]
\renewcommand{\thecorollary}{\arabic{corollary}}
\newtheorem{property}{Propriété}[section]
\newtheorem{objective}{Objectif}[section]

\theoremstyle{definition}
\newtheorem{definition}{Définition}[section]
\renewcommand{\thedefinition}{\arabic{definition}}
\newtheorem{exercise}{Exercice}[chapter]
\renewcommand{\theexercise}{\arabic{exercise}}
\newtheorem{example}{Exemple}[chapter]
\renewcommand{\theexample}{\arabic{example}}
\newtheorem*{solution}{Solution}
\newtheorem*{application}{Application}
\newtheorem*{notation}{Notation}
\newtheorem*{vocabulary}{Vocabulaire}
\newtheorem*{properties}{Propriétés}



\theoremstyle{remark}
\newtheorem*{remark}{Remarque}
\newtheorem*{rappel}{Rappel}


\usepackage{etoolbox}
\AtBeginEnvironment{exercise}{\small}
\AtBeginEnvironment{example}{\small}

\usepackage{cases}
\usepackage[red]{mypack}

\usepackage[framemethod=TikZ]{mdframed}

\definecolor{bg}{rgb}{0.4,0.25,0.95}
\definecolor{pagebg}{rgb}{0,0,0.5}
\surroundwithmdframed[
   topline=false,
   rightline=false,
   bottomline=false,
   leftmargin=\parindent,
   skipabove=8pt,
   skipbelow=8pt,
   linecolor=blue,
   innerbottommargin=10pt,
   % backgroundcolor=bg,font=\color{orange}\sffamily, fontcolor=white
]{definition}

\usepackage{empheq}
\usepackage[most]{tcolorbox}

\newtcbox{\mymath}[1][]{%
    nobeforeafter, math upper, tcbox raise base,
    enhanced, colframe=blue!30!black,
    colback=red!10, boxrule=1pt,
    #1}

\usepackage{unixode}


\DeclareMathOperator{\ord}{ord}
\DeclareMathOperator{\orb}{orb}
\DeclareMathOperator{\stab}{stab}
\DeclareMathOperator{\Stab}{stab}
\DeclareMathOperator{\ppcm}{ppcm}
\DeclareMathOperator{\conj}{Conj}
\DeclareMathOperator{\End}{End}
\DeclareMathOperator{\rot}{rot}
\DeclareMathOperator{\trs}{trace}
\DeclareMathOperator{\Ind}{Ind}
\DeclareMathOperator{\mat}{Mat}
\DeclareMathOperator{\id}{Id}
\DeclareMathOperator{\vect}{vect}
\DeclareMathOperator{\img}{img}
\DeclareMathOperator{\cov}{Cov}
\DeclareMathOperator{\dist}{dist}
\DeclareMathOperator{\irr}{Irr}
\DeclareMathOperator{\image}{Im}
\DeclareMathOperator{\pd}{\partial}
\DeclareMathOperator{\epi}{epi}
\DeclareMathOperator{\Argmin}{Argmin}
\DeclareMathOperator{\dom}{dom}
\DeclareMathOperator{\proj}{proj}
\DeclareMathOperator{\ctg}{ctg}
\DeclareMathOperator{\supp}{supp}
\DeclareMathOperator{\argmin}{argmin}
\DeclareMathOperator{\mult}{mult}
\DeclareMathOperator{\ch}{ch}
\DeclareMathOperator{\sh}{sh}
\DeclareMathOperator{\rang}{rang}
\DeclareMathOperator{\diam}{diam}
\DeclareMathOperator{\Epigraphe}{Epigraphe}




\usepackage{xcolor}
\everymath{\color{blue}}
%\everymath{\color[rgb]{0,1,1}}
%\pagecolor[rgb]{0,0,0.5}


\newcommand*{\pdtest}[3][]{\ensuremath{\frac{\partial^{#1} #2}{\partial #3}}}

\newcommand*{\deffunc}[6][]{\ensuremath{
\begin{array}{rcl}
#2 : #3 &\rightarrow& #4\\
#5 &\mapsto& #6
\end{array}
}}

\newcommand{\eqcolon}{\mathrel{\resizebox{\widthof{$\mathord{=}$}}{\height}{ $\!\!=\!\!\resizebox{1.2\width}{0.8\height}{\raisebox{0.23ex}{$\mathop{:}$}}\!\!$ }}}
\newcommand{\coloneq}{\mathrel{\resizebox{\widthof{$\mathord{=}$}}{\height}{ $\!\!\resizebox{1.2\width}{0.8\height}{\raisebox{0.23ex}{$\mathop{:}$}}\!\!=\!\!$ }}}
\newcommand{\eqcolonl}{\ensuremath{\mathrel{=\!\!\mathop{:}}}}
\newcommand{\coloneql}{\ensuremath{\mathrel{\mathop{:} \!\! =}}}
\newcommand{\vc}[1]{% inline column vector
  \left(\begin{smallmatrix}#1\end{smallmatrix}\right)%
}
\newcommand{\vr}[1]{% inline row vector
  \begin{smallmatrix}(\,#1\,)\end{smallmatrix}%
}
\makeatletter
\newcommand*{\defeq}{\ =\mathrel{\rlap{%
                     \raisebox{0.3ex}{$\m@th\cdot$}}%
                     \raisebox{-0.3ex}{$\m@th\cdot$}}%
                     }
\makeatother

\newcommand{\mathcircle}[1]{% inline row vector
 \overset{\circ}{#1}
}
\newcommand{\ulim}{% low limit
 \underline{\lim}
}
\newcommand{\ssi}{% iff
\iff
}
\newcommand{\ps}[2]{
\expval{#1 | #2}
}
\newcommand{\df}[1]{
\mqty{#1}
}
\newcommand{\n}[1]{
\norm{#1}
}
\newcommand{\sys}[1]{
\left\{\smqty{#1}\right.
}


\newcommand{\eqdef}{\ensuremath{\overset{\text{def}}=}}


\def\Circlearrowright{\ensuremath{%
  \rotatebox[origin=c]{230}{$\circlearrowright$}}}

\newcommand\ct[1]{\text{\rmfamily\upshape #1}}
\newcommand\question[1]{ {\color{red} ...!? \small #1}}
\newcommand\caz[1]{\left\{\begin{array} #1 \end{array}\right.}
\newcommand\const{\text{\rmfamily\upshape const}}
\newcommand\toP{ \overset{\pro}{\to}}
\newcommand\toPP{ \overset{\text{PP}}{\to}}
\newcommand{\oeq}{\mathrel{\text{\textcircled{$=$}}}}





\usepackage{xcolor}
% \usepackage[normalem]{ulem}
\usepackage{lipsum}
\makeatletter
% \newcommand\colorwave[1][blue]{\bgroup \markoverwith{\lower3.5\p@\hbox{\sixly \textcolor{#1}{\char58}}}\ULon}
%\font\sixly=lasy6 % does not re-load if already loaded, so no memory problem.

\newmdtheoremenv[
linewidth= 1pt,linecolor= blue,%
leftmargin=20,rightmargin=20,innertopmargin=0pt, innerrightmargin=40,%
tikzsetting = { draw=lightgray, line width = 0.3pt,dashed,%
dash pattern = on 15pt off 3pt},%
splittopskip=\topskip,skipbelow=\baselineskip,%
skipabove=\baselineskip,ntheorem,roundcorner=0pt,
% backgroundcolor=pagebg,font=\color{orange}\sffamily, fontcolor=white
]{examplebox}{Exemple}[section]



\newcommand\R{\mathbb{R}}
\newcommand\Z{\mathbb{Z}}
\newcommand\N{\mathbb{N}}
\newcommand\E{\mathbb{E}}
\newcommand\F{\mathcal{F}}
\newcommand\cH{\mathcal{H}}
\newcommand\V{\mathbb{V}}
\newcommand\dmo{ ^{-1} }
\newcommand\kapa{\kappa}
\newcommand\im{Im}
\newcommand\hs{\mathcal{H}}





\usepackage{soul}

\makeatletter
\newcommand*{\whiten}[1]{\llap{\textcolor{white}{{\the\SOUL@token}}\hspace{#1pt}}}
\DeclareRobustCommand*\myul{%
    \def\SOUL@everyspace{\underline{\space}\kern\z@}%
    \def\SOUL@everytoken{%
     \setbox0=\hbox{\the\SOUL@token}%
     \ifdim\dp0>\z@
        \raisebox{\dp0}{\underline{\phantom{\the\SOUL@token}}}%
        \whiten{1}\whiten{0}%
        \whiten{-1}\whiten{-2}%
        \llap{\the\SOUL@token}%
     \else
        \underline{\the\SOUL@token}%
     \fi}%
\SOUL@}
\makeatother

\newcommand*{\demp}{\fontfamily{lmtt}\selectfont}

\DeclareTextFontCommand{\textdemp}{\demp}

\begin{document}

\ifcomment
Multiline
comment
\fi
\ifcomment
\myul{Typesetting test}
% \color[rgb]{1,1,1}
$∑_i^n≠ 60º±∞π∆¬≈√j∫h≤≥µ$

$\CR \R\pro\ind\pro\gS\pro
\mqty[a&b\\c&d]$
$\pro\mathbb{P}$
$\dd{x}$

  \[
    \alpha(x)=\left\{
                \begin{array}{ll}
                  x\\
                  \frac{1}{1+e^{-kx}}\\
                  \frac{e^x-e^{-x}}{e^x+e^{-x}}
                \end{array}
              \right.
  \]

  $\expval{x}$
  
  $\chi_\rho(ghg\dmo)=\Tr(\rho_{ghg\dmo})=\Tr(\rho_g\circ\rho_h\circ\rho\dmo_g)=\Tr(\rho_h)\overset{\mbox{\scalebox{0.5}{$\Tr(AB)=\Tr(BA)$}}}{=}\chi_\rho(h)$
  	$\mathop{\oplus}_{\substack{x\in X}}$

$\mat(\rho_g)=(a_{ij}(g))_{\scriptsize \substack{1\leq i\leq d \\ 1\leq j\leq d}}$ et $\mat(\rho'_g)=(a'_{ij}(g))_{\scriptsize \substack{1\leq i'\leq d' \\ 1\leq j'\leq d'}}$



\[\int_a^b{\mathbb{R}^2}g(u, v)\dd{P_{XY}}(u, v)=\iint g(u,v) f_{XY}(u, v)\dd \lambda(u) \dd \lambda(v)\]
$$\lim_{x\to\infty} f(x)$$	
$$\iiiint_V \mu(t,u,v,w) \,dt\,du\,dv\,dw$$
$$\sum_{n=1}^{\infty} 2^{-n} = 1$$	
\begin{definition}
	Si $X$ et $Y$ sont 2 v.a. ou definit la \textsc{Covariance} entre $X$ et $Y$ comme
	$\cov(X,Y)\overset{\text{def}}{=}\E\left[(X-\E(X))(Y-\E(Y))\right]=\E(XY)-\E(X)\E(Y)$.
\end{definition}
\fi
\pagebreak

% \tableofcontents

% insert your code here
%\input{./algebra/main.tex}
%\input{./geometrie-differentielle/main.tex}
%\input{./probabilite/main.tex}
%\input{./analyse-fonctionnelle/main.tex}
% \input{./Analyse-convexe-et-dualite-en-optimisation/main.tex}
%\input{./tikz/main.tex}
%\input{./Theorie-du-distributions/main.tex}
%\input{./optimisation/mine.tex}
 \input{./modelisation/main.tex}

% yves.aubry@univ-tln.fr : algebra

\end{document}

%% !TEX encoding = UTF-8 Unicode
% !TEX TS-program = xelatex

\documentclass[french]{report}

%\usepackage[utf8]{inputenc}
%\usepackage[T1]{fontenc}
\usepackage{babel}


\newif\ifcomment
%\commenttrue # Show comments

\usepackage{physics}
\usepackage{amssymb}


\usepackage{amsthm}
% \usepackage{thmtools}
\usepackage{mathtools}
\usepackage{amsfonts}

\usepackage{color}

\usepackage{tikz}

\usepackage{geometry}
\geometry{a5paper, margin=0.1in, right=1cm}

\usepackage{dsfont}

\usepackage{graphicx}
\graphicspath{ {images/} }

\usepackage{faktor}

\usepackage{IEEEtrantools}
\usepackage{enumerate}   
\usepackage[PostScript=dvips]{"/Users/aware/Documents/Courses/diagrams"}


\newtheorem{theorem}{Théorème}[section]
\renewcommand{\thetheorem}{\arabic{theorem}}
\newtheorem{lemme}{Lemme}[section]
\renewcommand{\thelemme}{\arabic{lemme}}
\newtheorem{proposition}{Proposition}[section]
\renewcommand{\theproposition}{\arabic{proposition}}
\newtheorem{notations}{Notations}[section]
\newtheorem{problem}{Problème}[section]
\newtheorem{corollary}{Corollaire}[theorem]
\renewcommand{\thecorollary}{\arabic{corollary}}
\newtheorem{property}{Propriété}[section]
\newtheorem{objective}{Objectif}[section]

\theoremstyle{definition}
\newtheorem{definition}{Définition}[section]
\renewcommand{\thedefinition}{\arabic{definition}}
\newtheorem{exercise}{Exercice}[chapter]
\renewcommand{\theexercise}{\arabic{exercise}}
\newtheorem{example}{Exemple}[chapter]
\renewcommand{\theexample}{\arabic{example}}
\newtheorem*{solution}{Solution}
\newtheorem*{application}{Application}
\newtheorem*{notation}{Notation}
\newtheorem*{vocabulary}{Vocabulaire}
\newtheorem*{properties}{Propriétés}



\theoremstyle{remark}
\newtheorem*{remark}{Remarque}
\newtheorem*{rappel}{Rappel}


\usepackage{etoolbox}
\AtBeginEnvironment{exercise}{\small}
\AtBeginEnvironment{example}{\small}

\usepackage{cases}
\usepackage[red]{mypack}

\usepackage[framemethod=TikZ]{mdframed}

\definecolor{bg}{rgb}{0.4,0.25,0.95}
\definecolor{pagebg}{rgb}{0,0,0.5}
\surroundwithmdframed[
   topline=false,
   rightline=false,
   bottomline=false,
   leftmargin=\parindent,
   skipabove=8pt,
   skipbelow=8pt,
   linecolor=blue,
   innerbottommargin=10pt,
   % backgroundcolor=bg,font=\color{orange}\sffamily, fontcolor=white
]{definition}

\usepackage{empheq}
\usepackage[most]{tcolorbox}

\newtcbox{\mymath}[1][]{%
    nobeforeafter, math upper, tcbox raise base,
    enhanced, colframe=blue!30!black,
    colback=red!10, boxrule=1pt,
    #1}

\usepackage{unixode}


\DeclareMathOperator{\ord}{ord}
\DeclareMathOperator{\orb}{orb}
\DeclareMathOperator{\stab}{stab}
\DeclareMathOperator{\Stab}{stab}
\DeclareMathOperator{\ppcm}{ppcm}
\DeclareMathOperator{\conj}{Conj}
\DeclareMathOperator{\End}{End}
\DeclareMathOperator{\rot}{rot}
\DeclareMathOperator{\trs}{trace}
\DeclareMathOperator{\Ind}{Ind}
\DeclareMathOperator{\mat}{Mat}
\DeclareMathOperator{\id}{Id}
\DeclareMathOperator{\vect}{vect}
\DeclareMathOperator{\img}{img}
\DeclareMathOperator{\cov}{Cov}
\DeclareMathOperator{\dist}{dist}
\DeclareMathOperator{\irr}{Irr}
\DeclareMathOperator{\image}{Im}
\DeclareMathOperator{\pd}{\partial}
\DeclareMathOperator{\epi}{epi}
\DeclareMathOperator{\Argmin}{Argmin}
\DeclareMathOperator{\dom}{dom}
\DeclareMathOperator{\proj}{proj}
\DeclareMathOperator{\ctg}{ctg}
\DeclareMathOperator{\supp}{supp}
\DeclareMathOperator{\argmin}{argmin}
\DeclareMathOperator{\mult}{mult}
\DeclareMathOperator{\ch}{ch}
\DeclareMathOperator{\sh}{sh}
\DeclareMathOperator{\rang}{rang}
\DeclareMathOperator{\diam}{diam}
\DeclareMathOperator{\Epigraphe}{Epigraphe}




\usepackage{xcolor}
\everymath{\color{blue}}
%\everymath{\color[rgb]{0,1,1}}
%\pagecolor[rgb]{0,0,0.5}


\newcommand*{\pdtest}[3][]{\ensuremath{\frac{\partial^{#1} #2}{\partial #3}}}

\newcommand*{\deffunc}[6][]{\ensuremath{
\begin{array}{rcl}
#2 : #3 &\rightarrow& #4\\
#5 &\mapsto& #6
\end{array}
}}

\newcommand{\eqcolon}{\mathrel{\resizebox{\widthof{$\mathord{=}$}}{\height}{ $\!\!=\!\!\resizebox{1.2\width}{0.8\height}{\raisebox{0.23ex}{$\mathop{:}$}}\!\!$ }}}
\newcommand{\coloneq}{\mathrel{\resizebox{\widthof{$\mathord{=}$}}{\height}{ $\!\!\resizebox{1.2\width}{0.8\height}{\raisebox{0.23ex}{$\mathop{:}$}}\!\!=\!\!$ }}}
\newcommand{\eqcolonl}{\ensuremath{\mathrel{=\!\!\mathop{:}}}}
\newcommand{\coloneql}{\ensuremath{\mathrel{\mathop{:} \!\! =}}}
\newcommand{\vc}[1]{% inline column vector
  \left(\begin{smallmatrix}#1\end{smallmatrix}\right)%
}
\newcommand{\vr}[1]{% inline row vector
  \begin{smallmatrix}(\,#1\,)\end{smallmatrix}%
}
\makeatletter
\newcommand*{\defeq}{\ =\mathrel{\rlap{%
                     \raisebox{0.3ex}{$\m@th\cdot$}}%
                     \raisebox{-0.3ex}{$\m@th\cdot$}}%
                     }
\makeatother

\newcommand{\mathcircle}[1]{% inline row vector
 \overset{\circ}{#1}
}
\newcommand{\ulim}{% low limit
 \underline{\lim}
}
\newcommand{\ssi}{% iff
\iff
}
\newcommand{\ps}[2]{
\expval{#1 | #2}
}
\newcommand{\df}[1]{
\mqty{#1}
}
\newcommand{\n}[1]{
\norm{#1}
}
\newcommand{\sys}[1]{
\left\{\smqty{#1}\right.
}


\newcommand{\eqdef}{\ensuremath{\overset{\text{def}}=}}


\def\Circlearrowright{\ensuremath{%
  \rotatebox[origin=c]{230}{$\circlearrowright$}}}

\newcommand\ct[1]{\text{\rmfamily\upshape #1}}
\newcommand\question[1]{ {\color{red} ...!? \small #1}}
\newcommand\caz[1]{\left\{\begin{array} #1 \end{array}\right.}
\newcommand\const{\text{\rmfamily\upshape const}}
\newcommand\toP{ \overset{\pro}{\to}}
\newcommand\toPP{ \overset{\text{PP}}{\to}}
\newcommand{\oeq}{\mathrel{\text{\textcircled{$=$}}}}





\usepackage{xcolor}
% \usepackage[normalem]{ulem}
\usepackage{lipsum}
\makeatletter
% \newcommand\colorwave[1][blue]{\bgroup \markoverwith{\lower3.5\p@\hbox{\sixly \textcolor{#1}{\char58}}}\ULon}
%\font\sixly=lasy6 % does not re-load if already loaded, so no memory problem.

\newmdtheoremenv[
linewidth= 1pt,linecolor= blue,%
leftmargin=20,rightmargin=20,innertopmargin=0pt, innerrightmargin=40,%
tikzsetting = { draw=lightgray, line width = 0.3pt,dashed,%
dash pattern = on 15pt off 3pt},%
splittopskip=\topskip,skipbelow=\baselineskip,%
skipabove=\baselineskip,ntheorem,roundcorner=0pt,
% backgroundcolor=pagebg,font=\color{orange}\sffamily, fontcolor=white
]{examplebox}{Exemple}[section]



\newcommand\R{\mathbb{R}}
\newcommand\Z{\mathbb{Z}}
\newcommand\N{\mathbb{N}}
\newcommand\E{\mathbb{E}}
\newcommand\F{\mathcal{F}}
\newcommand\cH{\mathcal{H}}
\newcommand\V{\mathbb{V}}
\newcommand\dmo{ ^{-1} }
\newcommand\kapa{\kappa}
\newcommand\im{Im}
\newcommand\hs{\mathcal{H}}





\usepackage{soul}

\makeatletter
\newcommand*{\whiten}[1]{\llap{\textcolor{white}{{\the\SOUL@token}}\hspace{#1pt}}}
\DeclareRobustCommand*\myul{%
    \def\SOUL@everyspace{\underline{\space}\kern\z@}%
    \def\SOUL@everytoken{%
     \setbox0=\hbox{\the\SOUL@token}%
     \ifdim\dp0>\z@
        \raisebox{\dp0}{\underline{\phantom{\the\SOUL@token}}}%
        \whiten{1}\whiten{0}%
        \whiten{-1}\whiten{-2}%
        \llap{\the\SOUL@token}%
     \else
        \underline{\the\SOUL@token}%
     \fi}%
\SOUL@}
\makeatother

\newcommand*{\demp}{\fontfamily{lmtt}\selectfont}

\DeclareTextFontCommand{\textdemp}{\demp}

\begin{document}

\ifcomment
Multiline
comment
\fi
\ifcomment
\myul{Typesetting test}
% \color[rgb]{1,1,1}
$∑_i^n≠ 60º±∞π∆¬≈√j∫h≤≥µ$

$\CR \R\pro\ind\pro\gS\pro
\mqty[a&b\\c&d]$
$\pro\mathbb{P}$
$\dd{x}$

  \[
    \alpha(x)=\left\{
                \begin{array}{ll}
                  x\\
                  \frac{1}{1+e^{-kx}}\\
                  \frac{e^x-e^{-x}}{e^x+e^{-x}}
                \end{array}
              \right.
  \]

  $\expval{x}$
  
  $\chi_\rho(ghg\dmo)=\Tr(\rho_{ghg\dmo})=\Tr(\rho_g\circ\rho_h\circ\rho\dmo_g)=\Tr(\rho_h)\overset{\mbox{\scalebox{0.5}{$\Tr(AB)=\Tr(BA)$}}}{=}\chi_\rho(h)$
  	$\mathop{\oplus}_{\substack{x\in X}}$

$\mat(\rho_g)=(a_{ij}(g))_{\scriptsize \substack{1\leq i\leq d \\ 1\leq j\leq d}}$ et $\mat(\rho'_g)=(a'_{ij}(g))_{\scriptsize \substack{1\leq i'\leq d' \\ 1\leq j'\leq d'}}$



\[\int_a^b{\mathbb{R}^2}g(u, v)\dd{P_{XY}}(u, v)=\iint g(u,v) f_{XY}(u, v)\dd \lambda(u) \dd \lambda(v)\]
$$\lim_{x\to\infty} f(x)$$	
$$\iiiint_V \mu(t,u,v,w) \,dt\,du\,dv\,dw$$
$$\sum_{n=1}^{\infty} 2^{-n} = 1$$	
\begin{definition}
	Si $X$ et $Y$ sont 2 v.a. ou definit la \textsc{Covariance} entre $X$ et $Y$ comme
	$\cov(X,Y)\overset{\text{def}}{=}\E\left[(X-\E(X))(Y-\E(Y))\right]=\E(XY)-\E(X)\E(Y)$.
\end{definition}
\fi
\pagebreak

% \tableofcontents

% insert your code here
%\input{./algebra/main.tex}
%\input{./geometrie-differentielle/main.tex}
%\input{./probabilite/main.tex}
%\input{./analyse-fonctionnelle/main.tex}
% \input{./Analyse-convexe-et-dualite-en-optimisation/main.tex}
%\input{./tikz/main.tex}
%\input{./Theorie-du-distributions/main.tex}
%\input{./optimisation/mine.tex}
 \input{./modelisation/main.tex}

% yves.aubry@univ-tln.fr : algebra

\end{document}

%% !TEX encoding = UTF-8 Unicode
% !TEX TS-program = xelatex

\documentclass[french]{report}

%\usepackage[utf8]{inputenc}
%\usepackage[T1]{fontenc}
\usepackage{babel}


\newif\ifcomment
%\commenttrue # Show comments

\usepackage{physics}
\usepackage{amssymb}


\usepackage{amsthm}
% \usepackage{thmtools}
\usepackage{mathtools}
\usepackage{amsfonts}

\usepackage{color}

\usepackage{tikz}

\usepackage{geometry}
\geometry{a5paper, margin=0.1in, right=1cm}

\usepackage{dsfont}

\usepackage{graphicx}
\graphicspath{ {images/} }

\usepackage{faktor}

\usepackage{IEEEtrantools}
\usepackage{enumerate}   
\usepackage[PostScript=dvips]{"/Users/aware/Documents/Courses/diagrams"}


\newtheorem{theorem}{Théorème}[section]
\renewcommand{\thetheorem}{\arabic{theorem}}
\newtheorem{lemme}{Lemme}[section]
\renewcommand{\thelemme}{\arabic{lemme}}
\newtheorem{proposition}{Proposition}[section]
\renewcommand{\theproposition}{\arabic{proposition}}
\newtheorem{notations}{Notations}[section]
\newtheorem{problem}{Problème}[section]
\newtheorem{corollary}{Corollaire}[theorem]
\renewcommand{\thecorollary}{\arabic{corollary}}
\newtheorem{property}{Propriété}[section]
\newtheorem{objective}{Objectif}[section]

\theoremstyle{definition}
\newtheorem{definition}{Définition}[section]
\renewcommand{\thedefinition}{\arabic{definition}}
\newtheorem{exercise}{Exercice}[chapter]
\renewcommand{\theexercise}{\arabic{exercise}}
\newtheorem{example}{Exemple}[chapter]
\renewcommand{\theexample}{\arabic{example}}
\newtheorem*{solution}{Solution}
\newtheorem*{application}{Application}
\newtheorem*{notation}{Notation}
\newtheorem*{vocabulary}{Vocabulaire}
\newtheorem*{properties}{Propriétés}



\theoremstyle{remark}
\newtheorem*{remark}{Remarque}
\newtheorem*{rappel}{Rappel}


\usepackage{etoolbox}
\AtBeginEnvironment{exercise}{\small}
\AtBeginEnvironment{example}{\small}

\usepackage{cases}
\usepackage[red]{mypack}

\usepackage[framemethod=TikZ]{mdframed}

\definecolor{bg}{rgb}{0.4,0.25,0.95}
\definecolor{pagebg}{rgb}{0,0,0.5}
\surroundwithmdframed[
   topline=false,
   rightline=false,
   bottomline=false,
   leftmargin=\parindent,
   skipabove=8pt,
   skipbelow=8pt,
   linecolor=blue,
   innerbottommargin=10pt,
   % backgroundcolor=bg,font=\color{orange}\sffamily, fontcolor=white
]{definition}

\usepackage{empheq}
\usepackage[most]{tcolorbox}

\newtcbox{\mymath}[1][]{%
    nobeforeafter, math upper, tcbox raise base,
    enhanced, colframe=blue!30!black,
    colback=red!10, boxrule=1pt,
    #1}

\usepackage{unixode}


\DeclareMathOperator{\ord}{ord}
\DeclareMathOperator{\orb}{orb}
\DeclareMathOperator{\stab}{stab}
\DeclareMathOperator{\Stab}{stab}
\DeclareMathOperator{\ppcm}{ppcm}
\DeclareMathOperator{\conj}{Conj}
\DeclareMathOperator{\End}{End}
\DeclareMathOperator{\rot}{rot}
\DeclareMathOperator{\trs}{trace}
\DeclareMathOperator{\Ind}{Ind}
\DeclareMathOperator{\mat}{Mat}
\DeclareMathOperator{\id}{Id}
\DeclareMathOperator{\vect}{vect}
\DeclareMathOperator{\img}{img}
\DeclareMathOperator{\cov}{Cov}
\DeclareMathOperator{\dist}{dist}
\DeclareMathOperator{\irr}{Irr}
\DeclareMathOperator{\image}{Im}
\DeclareMathOperator{\pd}{\partial}
\DeclareMathOperator{\epi}{epi}
\DeclareMathOperator{\Argmin}{Argmin}
\DeclareMathOperator{\dom}{dom}
\DeclareMathOperator{\proj}{proj}
\DeclareMathOperator{\ctg}{ctg}
\DeclareMathOperator{\supp}{supp}
\DeclareMathOperator{\argmin}{argmin}
\DeclareMathOperator{\mult}{mult}
\DeclareMathOperator{\ch}{ch}
\DeclareMathOperator{\sh}{sh}
\DeclareMathOperator{\rang}{rang}
\DeclareMathOperator{\diam}{diam}
\DeclareMathOperator{\Epigraphe}{Epigraphe}




\usepackage{xcolor}
\everymath{\color{blue}}
%\everymath{\color[rgb]{0,1,1}}
%\pagecolor[rgb]{0,0,0.5}


\newcommand*{\pdtest}[3][]{\ensuremath{\frac{\partial^{#1} #2}{\partial #3}}}

\newcommand*{\deffunc}[6][]{\ensuremath{
\begin{array}{rcl}
#2 : #3 &\rightarrow& #4\\
#5 &\mapsto& #6
\end{array}
}}

\newcommand{\eqcolon}{\mathrel{\resizebox{\widthof{$\mathord{=}$}}{\height}{ $\!\!=\!\!\resizebox{1.2\width}{0.8\height}{\raisebox{0.23ex}{$\mathop{:}$}}\!\!$ }}}
\newcommand{\coloneq}{\mathrel{\resizebox{\widthof{$\mathord{=}$}}{\height}{ $\!\!\resizebox{1.2\width}{0.8\height}{\raisebox{0.23ex}{$\mathop{:}$}}\!\!=\!\!$ }}}
\newcommand{\eqcolonl}{\ensuremath{\mathrel{=\!\!\mathop{:}}}}
\newcommand{\coloneql}{\ensuremath{\mathrel{\mathop{:} \!\! =}}}
\newcommand{\vc}[1]{% inline column vector
  \left(\begin{smallmatrix}#1\end{smallmatrix}\right)%
}
\newcommand{\vr}[1]{% inline row vector
  \begin{smallmatrix}(\,#1\,)\end{smallmatrix}%
}
\makeatletter
\newcommand*{\defeq}{\ =\mathrel{\rlap{%
                     \raisebox{0.3ex}{$\m@th\cdot$}}%
                     \raisebox{-0.3ex}{$\m@th\cdot$}}%
                     }
\makeatother

\newcommand{\mathcircle}[1]{% inline row vector
 \overset{\circ}{#1}
}
\newcommand{\ulim}{% low limit
 \underline{\lim}
}
\newcommand{\ssi}{% iff
\iff
}
\newcommand{\ps}[2]{
\expval{#1 | #2}
}
\newcommand{\df}[1]{
\mqty{#1}
}
\newcommand{\n}[1]{
\norm{#1}
}
\newcommand{\sys}[1]{
\left\{\smqty{#1}\right.
}


\newcommand{\eqdef}{\ensuremath{\overset{\text{def}}=}}


\def\Circlearrowright{\ensuremath{%
  \rotatebox[origin=c]{230}{$\circlearrowright$}}}

\newcommand\ct[1]{\text{\rmfamily\upshape #1}}
\newcommand\question[1]{ {\color{red} ...!? \small #1}}
\newcommand\caz[1]{\left\{\begin{array} #1 \end{array}\right.}
\newcommand\const{\text{\rmfamily\upshape const}}
\newcommand\toP{ \overset{\pro}{\to}}
\newcommand\toPP{ \overset{\text{PP}}{\to}}
\newcommand{\oeq}{\mathrel{\text{\textcircled{$=$}}}}





\usepackage{xcolor}
% \usepackage[normalem]{ulem}
\usepackage{lipsum}
\makeatletter
% \newcommand\colorwave[1][blue]{\bgroup \markoverwith{\lower3.5\p@\hbox{\sixly \textcolor{#1}{\char58}}}\ULon}
%\font\sixly=lasy6 % does not re-load if already loaded, so no memory problem.

\newmdtheoremenv[
linewidth= 1pt,linecolor= blue,%
leftmargin=20,rightmargin=20,innertopmargin=0pt, innerrightmargin=40,%
tikzsetting = { draw=lightgray, line width = 0.3pt,dashed,%
dash pattern = on 15pt off 3pt},%
splittopskip=\topskip,skipbelow=\baselineskip,%
skipabove=\baselineskip,ntheorem,roundcorner=0pt,
% backgroundcolor=pagebg,font=\color{orange}\sffamily, fontcolor=white
]{examplebox}{Exemple}[section]



\newcommand\R{\mathbb{R}}
\newcommand\Z{\mathbb{Z}}
\newcommand\N{\mathbb{N}}
\newcommand\E{\mathbb{E}}
\newcommand\F{\mathcal{F}}
\newcommand\cH{\mathcal{H}}
\newcommand\V{\mathbb{V}}
\newcommand\dmo{ ^{-1} }
\newcommand\kapa{\kappa}
\newcommand\im{Im}
\newcommand\hs{\mathcal{H}}





\usepackage{soul}

\makeatletter
\newcommand*{\whiten}[1]{\llap{\textcolor{white}{{\the\SOUL@token}}\hspace{#1pt}}}
\DeclareRobustCommand*\myul{%
    \def\SOUL@everyspace{\underline{\space}\kern\z@}%
    \def\SOUL@everytoken{%
     \setbox0=\hbox{\the\SOUL@token}%
     \ifdim\dp0>\z@
        \raisebox{\dp0}{\underline{\phantom{\the\SOUL@token}}}%
        \whiten{1}\whiten{0}%
        \whiten{-1}\whiten{-2}%
        \llap{\the\SOUL@token}%
     \else
        \underline{\the\SOUL@token}%
     \fi}%
\SOUL@}
\makeatother

\newcommand*{\demp}{\fontfamily{lmtt}\selectfont}

\DeclareTextFontCommand{\textdemp}{\demp}

\begin{document}

\ifcomment
Multiline
comment
\fi
\ifcomment
\myul{Typesetting test}
% \color[rgb]{1,1,1}
$∑_i^n≠ 60º±∞π∆¬≈√j∫h≤≥µ$

$\CR \R\pro\ind\pro\gS\pro
\mqty[a&b\\c&d]$
$\pro\mathbb{P}$
$\dd{x}$

  \[
    \alpha(x)=\left\{
                \begin{array}{ll}
                  x\\
                  \frac{1}{1+e^{-kx}}\\
                  \frac{e^x-e^{-x}}{e^x+e^{-x}}
                \end{array}
              \right.
  \]

  $\expval{x}$
  
  $\chi_\rho(ghg\dmo)=\Tr(\rho_{ghg\dmo})=\Tr(\rho_g\circ\rho_h\circ\rho\dmo_g)=\Tr(\rho_h)\overset{\mbox{\scalebox{0.5}{$\Tr(AB)=\Tr(BA)$}}}{=}\chi_\rho(h)$
  	$\mathop{\oplus}_{\substack{x\in X}}$

$\mat(\rho_g)=(a_{ij}(g))_{\scriptsize \substack{1\leq i\leq d \\ 1\leq j\leq d}}$ et $\mat(\rho'_g)=(a'_{ij}(g))_{\scriptsize \substack{1\leq i'\leq d' \\ 1\leq j'\leq d'}}$



\[\int_a^b{\mathbb{R}^2}g(u, v)\dd{P_{XY}}(u, v)=\iint g(u,v) f_{XY}(u, v)\dd \lambda(u) \dd \lambda(v)\]
$$\lim_{x\to\infty} f(x)$$	
$$\iiiint_V \mu(t,u,v,w) \,dt\,du\,dv\,dw$$
$$\sum_{n=1}^{\infty} 2^{-n} = 1$$	
\begin{definition}
	Si $X$ et $Y$ sont 2 v.a. ou definit la \textsc{Covariance} entre $X$ et $Y$ comme
	$\cov(X,Y)\overset{\text{def}}{=}\E\left[(X-\E(X))(Y-\E(Y))\right]=\E(XY)-\E(X)\E(Y)$.
\end{definition}
\fi
\pagebreak

% \tableofcontents

% insert your code here
%\input{./algebra/main.tex}
%\input{./geometrie-differentielle/main.tex}
%\input{./probabilite/main.tex}
%\input{./analyse-fonctionnelle/main.tex}
% \input{./Analyse-convexe-et-dualite-en-optimisation/main.tex}
%\input{./tikz/main.tex}
%\input{./Theorie-du-distributions/main.tex}
%\input{./optimisation/mine.tex}
 \input{./modelisation/main.tex}

% yves.aubry@univ-tln.fr : algebra

\end{document}

%\input{./optimisation/mine.tex}
 % !TEX encoding = UTF-8 Unicode
% !TEX TS-program = xelatex

\documentclass[french]{report}

%\usepackage[utf8]{inputenc}
%\usepackage[T1]{fontenc}
\usepackage{babel}


\newif\ifcomment
%\commenttrue # Show comments

\usepackage{physics}
\usepackage{amssymb}


\usepackage{amsthm}
% \usepackage{thmtools}
\usepackage{mathtools}
\usepackage{amsfonts}

\usepackage{color}

\usepackage{tikz}

\usepackage{geometry}
\geometry{a5paper, margin=0.1in, right=1cm}

\usepackage{dsfont}

\usepackage{graphicx}
\graphicspath{ {images/} }

\usepackage{faktor}

\usepackage{IEEEtrantools}
\usepackage{enumerate}   
\usepackage[PostScript=dvips]{"/Users/aware/Documents/Courses/diagrams"}


\newtheorem{theorem}{Théorème}[section]
\renewcommand{\thetheorem}{\arabic{theorem}}
\newtheorem{lemme}{Lemme}[section]
\renewcommand{\thelemme}{\arabic{lemme}}
\newtheorem{proposition}{Proposition}[section]
\renewcommand{\theproposition}{\arabic{proposition}}
\newtheorem{notations}{Notations}[section]
\newtheorem{problem}{Problème}[section]
\newtheorem{corollary}{Corollaire}[theorem]
\renewcommand{\thecorollary}{\arabic{corollary}}
\newtheorem{property}{Propriété}[section]
\newtheorem{objective}{Objectif}[section]

\theoremstyle{definition}
\newtheorem{definition}{Définition}[section]
\renewcommand{\thedefinition}{\arabic{definition}}
\newtheorem{exercise}{Exercice}[chapter]
\renewcommand{\theexercise}{\arabic{exercise}}
\newtheorem{example}{Exemple}[chapter]
\renewcommand{\theexample}{\arabic{example}}
\newtheorem*{solution}{Solution}
\newtheorem*{application}{Application}
\newtheorem*{notation}{Notation}
\newtheorem*{vocabulary}{Vocabulaire}
\newtheorem*{properties}{Propriétés}



\theoremstyle{remark}
\newtheorem*{remark}{Remarque}
\newtheorem*{rappel}{Rappel}


\usepackage{etoolbox}
\AtBeginEnvironment{exercise}{\small}
\AtBeginEnvironment{example}{\small}

\usepackage{cases}
\usepackage[red]{mypack}

\usepackage[framemethod=TikZ]{mdframed}

\definecolor{bg}{rgb}{0.4,0.25,0.95}
\definecolor{pagebg}{rgb}{0,0,0.5}
\surroundwithmdframed[
   topline=false,
   rightline=false,
   bottomline=false,
   leftmargin=\parindent,
   skipabove=8pt,
   skipbelow=8pt,
   linecolor=blue,
   innerbottommargin=10pt,
   % backgroundcolor=bg,font=\color{orange}\sffamily, fontcolor=white
]{definition}

\usepackage{empheq}
\usepackage[most]{tcolorbox}

\newtcbox{\mymath}[1][]{%
    nobeforeafter, math upper, tcbox raise base,
    enhanced, colframe=blue!30!black,
    colback=red!10, boxrule=1pt,
    #1}

\usepackage{unixode}


\DeclareMathOperator{\ord}{ord}
\DeclareMathOperator{\orb}{orb}
\DeclareMathOperator{\stab}{stab}
\DeclareMathOperator{\Stab}{stab}
\DeclareMathOperator{\ppcm}{ppcm}
\DeclareMathOperator{\conj}{Conj}
\DeclareMathOperator{\End}{End}
\DeclareMathOperator{\rot}{rot}
\DeclareMathOperator{\trs}{trace}
\DeclareMathOperator{\Ind}{Ind}
\DeclareMathOperator{\mat}{Mat}
\DeclareMathOperator{\id}{Id}
\DeclareMathOperator{\vect}{vect}
\DeclareMathOperator{\img}{img}
\DeclareMathOperator{\cov}{Cov}
\DeclareMathOperator{\dist}{dist}
\DeclareMathOperator{\irr}{Irr}
\DeclareMathOperator{\image}{Im}
\DeclareMathOperator{\pd}{\partial}
\DeclareMathOperator{\epi}{epi}
\DeclareMathOperator{\Argmin}{Argmin}
\DeclareMathOperator{\dom}{dom}
\DeclareMathOperator{\proj}{proj}
\DeclareMathOperator{\ctg}{ctg}
\DeclareMathOperator{\supp}{supp}
\DeclareMathOperator{\argmin}{argmin}
\DeclareMathOperator{\mult}{mult}
\DeclareMathOperator{\ch}{ch}
\DeclareMathOperator{\sh}{sh}
\DeclareMathOperator{\rang}{rang}
\DeclareMathOperator{\diam}{diam}
\DeclareMathOperator{\Epigraphe}{Epigraphe}




\usepackage{xcolor}
\everymath{\color{blue}}
%\everymath{\color[rgb]{0,1,1}}
%\pagecolor[rgb]{0,0,0.5}


\newcommand*{\pdtest}[3][]{\ensuremath{\frac{\partial^{#1} #2}{\partial #3}}}

\newcommand*{\deffunc}[6][]{\ensuremath{
\begin{array}{rcl}
#2 : #3 &\rightarrow& #4\\
#5 &\mapsto& #6
\end{array}
}}

\newcommand{\eqcolon}{\mathrel{\resizebox{\widthof{$\mathord{=}$}}{\height}{ $\!\!=\!\!\resizebox{1.2\width}{0.8\height}{\raisebox{0.23ex}{$\mathop{:}$}}\!\!$ }}}
\newcommand{\coloneq}{\mathrel{\resizebox{\widthof{$\mathord{=}$}}{\height}{ $\!\!\resizebox{1.2\width}{0.8\height}{\raisebox{0.23ex}{$\mathop{:}$}}\!\!=\!\!$ }}}
\newcommand{\eqcolonl}{\ensuremath{\mathrel{=\!\!\mathop{:}}}}
\newcommand{\coloneql}{\ensuremath{\mathrel{\mathop{:} \!\! =}}}
\newcommand{\vc}[1]{% inline column vector
  \left(\begin{smallmatrix}#1\end{smallmatrix}\right)%
}
\newcommand{\vr}[1]{% inline row vector
  \begin{smallmatrix}(\,#1\,)\end{smallmatrix}%
}
\makeatletter
\newcommand*{\defeq}{\ =\mathrel{\rlap{%
                     \raisebox{0.3ex}{$\m@th\cdot$}}%
                     \raisebox{-0.3ex}{$\m@th\cdot$}}%
                     }
\makeatother

\newcommand{\mathcircle}[1]{% inline row vector
 \overset{\circ}{#1}
}
\newcommand{\ulim}{% low limit
 \underline{\lim}
}
\newcommand{\ssi}{% iff
\iff
}
\newcommand{\ps}[2]{
\expval{#1 | #2}
}
\newcommand{\df}[1]{
\mqty{#1}
}
\newcommand{\n}[1]{
\norm{#1}
}
\newcommand{\sys}[1]{
\left\{\smqty{#1}\right.
}


\newcommand{\eqdef}{\ensuremath{\overset{\text{def}}=}}


\def\Circlearrowright{\ensuremath{%
  \rotatebox[origin=c]{230}{$\circlearrowright$}}}

\newcommand\ct[1]{\text{\rmfamily\upshape #1}}
\newcommand\question[1]{ {\color{red} ...!? \small #1}}
\newcommand\caz[1]{\left\{\begin{array} #1 \end{array}\right.}
\newcommand\const{\text{\rmfamily\upshape const}}
\newcommand\toP{ \overset{\pro}{\to}}
\newcommand\toPP{ \overset{\text{PP}}{\to}}
\newcommand{\oeq}{\mathrel{\text{\textcircled{$=$}}}}





\usepackage{xcolor}
% \usepackage[normalem]{ulem}
\usepackage{lipsum}
\makeatletter
% \newcommand\colorwave[1][blue]{\bgroup \markoverwith{\lower3.5\p@\hbox{\sixly \textcolor{#1}{\char58}}}\ULon}
%\font\sixly=lasy6 % does not re-load if already loaded, so no memory problem.

\newmdtheoremenv[
linewidth= 1pt,linecolor= blue,%
leftmargin=20,rightmargin=20,innertopmargin=0pt, innerrightmargin=40,%
tikzsetting = { draw=lightgray, line width = 0.3pt,dashed,%
dash pattern = on 15pt off 3pt},%
splittopskip=\topskip,skipbelow=\baselineskip,%
skipabove=\baselineskip,ntheorem,roundcorner=0pt,
% backgroundcolor=pagebg,font=\color{orange}\sffamily, fontcolor=white
]{examplebox}{Exemple}[section]



\newcommand\R{\mathbb{R}}
\newcommand\Z{\mathbb{Z}}
\newcommand\N{\mathbb{N}}
\newcommand\E{\mathbb{E}}
\newcommand\F{\mathcal{F}}
\newcommand\cH{\mathcal{H}}
\newcommand\V{\mathbb{V}}
\newcommand\dmo{ ^{-1} }
\newcommand\kapa{\kappa}
\newcommand\im{Im}
\newcommand\hs{\mathcal{H}}





\usepackage{soul}

\makeatletter
\newcommand*{\whiten}[1]{\llap{\textcolor{white}{{\the\SOUL@token}}\hspace{#1pt}}}
\DeclareRobustCommand*\myul{%
    \def\SOUL@everyspace{\underline{\space}\kern\z@}%
    \def\SOUL@everytoken{%
     \setbox0=\hbox{\the\SOUL@token}%
     \ifdim\dp0>\z@
        \raisebox{\dp0}{\underline{\phantom{\the\SOUL@token}}}%
        \whiten{1}\whiten{0}%
        \whiten{-1}\whiten{-2}%
        \llap{\the\SOUL@token}%
     \else
        \underline{\the\SOUL@token}%
     \fi}%
\SOUL@}
\makeatother

\newcommand*{\demp}{\fontfamily{lmtt}\selectfont}

\DeclareTextFontCommand{\textdemp}{\demp}

\begin{document}

\ifcomment
Multiline
comment
\fi
\ifcomment
\myul{Typesetting test}
% \color[rgb]{1,1,1}
$∑_i^n≠ 60º±∞π∆¬≈√j∫h≤≥µ$

$\CR \R\pro\ind\pro\gS\pro
\mqty[a&b\\c&d]$
$\pro\mathbb{P}$
$\dd{x}$

  \[
    \alpha(x)=\left\{
                \begin{array}{ll}
                  x\\
                  \frac{1}{1+e^{-kx}}\\
                  \frac{e^x-e^{-x}}{e^x+e^{-x}}
                \end{array}
              \right.
  \]

  $\expval{x}$
  
  $\chi_\rho(ghg\dmo)=\Tr(\rho_{ghg\dmo})=\Tr(\rho_g\circ\rho_h\circ\rho\dmo_g)=\Tr(\rho_h)\overset{\mbox{\scalebox{0.5}{$\Tr(AB)=\Tr(BA)$}}}{=}\chi_\rho(h)$
  	$\mathop{\oplus}_{\substack{x\in X}}$

$\mat(\rho_g)=(a_{ij}(g))_{\scriptsize \substack{1\leq i\leq d \\ 1\leq j\leq d}}$ et $\mat(\rho'_g)=(a'_{ij}(g))_{\scriptsize \substack{1\leq i'\leq d' \\ 1\leq j'\leq d'}}$



\[\int_a^b{\mathbb{R}^2}g(u, v)\dd{P_{XY}}(u, v)=\iint g(u,v) f_{XY}(u, v)\dd \lambda(u) \dd \lambda(v)\]
$$\lim_{x\to\infty} f(x)$$	
$$\iiiint_V \mu(t,u,v,w) \,dt\,du\,dv\,dw$$
$$\sum_{n=1}^{\infty} 2^{-n} = 1$$	
\begin{definition}
	Si $X$ et $Y$ sont 2 v.a. ou definit la \textsc{Covariance} entre $X$ et $Y$ comme
	$\cov(X,Y)\overset{\text{def}}{=}\E\left[(X-\E(X))(Y-\E(Y))\right]=\E(XY)-\E(X)\E(Y)$.
\end{definition}
\fi
\pagebreak

% \tableofcontents

% insert your code here
%\input{./algebra/main.tex}
%\input{./geometrie-differentielle/main.tex}
%\input{./probabilite/main.tex}
%\input{./analyse-fonctionnelle/main.tex}
% \input{./Analyse-convexe-et-dualite-en-optimisation/main.tex}
%\input{./tikz/main.tex}
%\input{./Theorie-du-distributions/main.tex}
%\input{./optimisation/mine.tex}
 \input{./modelisation/main.tex}

% yves.aubry@univ-tln.fr : algebra

\end{document}


% yves.aubry@univ-tln.fr : algebra

\end{document}


% yves.aubry@univ-tln.fr : algebra

\end{document}

% % !TEX encoding = UTF-8 Unicode
% !TEX TS-program = xelatex

\documentclass[french]{report}

%\usepackage[utf8]{inputenc}
%\usepackage[T1]{fontenc}
\usepackage{babel}


\newif\ifcomment
%\commenttrue # Show comments

\usepackage{physics}
\usepackage{amssymb}


\usepackage{amsthm}
% \usepackage{thmtools}
\usepackage{mathtools}
\usepackage{amsfonts}

\usepackage{color}

\usepackage{tikz}

\usepackage{geometry}
\geometry{a5paper, margin=0.1in, right=1cm}

\usepackage{dsfont}

\usepackage{graphicx}
\graphicspath{ {images/} }

\usepackage{faktor}

\usepackage{IEEEtrantools}
\usepackage{enumerate}   
\usepackage[PostScript=dvips]{"/Users/aware/Documents/Courses/diagrams"}


\newtheorem{theorem}{Théorème}[section]
\renewcommand{\thetheorem}{\arabic{theorem}}
\newtheorem{lemme}{Lemme}[section]
\renewcommand{\thelemme}{\arabic{lemme}}
\newtheorem{proposition}{Proposition}[section]
\renewcommand{\theproposition}{\arabic{proposition}}
\newtheorem{notations}{Notations}[section]
\newtheorem{problem}{Problème}[section]
\newtheorem{corollary}{Corollaire}[theorem]
\renewcommand{\thecorollary}{\arabic{corollary}}
\newtheorem{property}{Propriété}[section]
\newtheorem{objective}{Objectif}[section]

\theoremstyle{definition}
\newtheorem{definition}{Définition}[section]
\renewcommand{\thedefinition}{\arabic{definition}}
\newtheorem{exercise}{Exercice}[chapter]
\renewcommand{\theexercise}{\arabic{exercise}}
\newtheorem{example}{Exemple}[chapter]
\renewcommand{\theexample}{\arabic{example}}
\newtheorem*{solution}{Solution}
\newtheorem*{application}{Application}
\newtheorem*{notation}{Notation}
\newtheorem*{vocabulary}{Vocabulaire}
\newtheorem*{properties}{Propriétés}



\theoremstyle{remark}
\newtheorem*{remark}{Remarque}
\newtheorem*{rappel}{Rappel}


\usepackage{etoolbox}
\AtBeginEnvironment{exercise}{\small}
\AtBeginEnvironment{example}{\small}

\usepackage{cases}
\usepackage[red]{mypack}

\usepackage[framemethod=TikZ]{mdframed}

\definecolor{bg}{rgb}{0.4,0.25,0.95}
\definecolor{pagebg}{rgb}{0,0,0.5}
\surroundwithmdframed[
   topline=false,
   rightline=false,
   bottomline=false,
   leftmargin=\parindent,
   skipabove=8pt,
   skipbelow=8pt,
   linecolor=blue,
   innerbottommargin=10pt,
   % backgroundcolor=bg,font=\color{orange}\sffamily, fontcolor=white
]{definition}

\usepackage{empheq}
\usepackage[most]{tcolorbox}

\newtcbox{\mymath}[1][]{%
    nobeforeafter, math upper, tcbox raise base,
    enhanced, colframe=blue!30!black,
    colback=red!10, boxrule=1pt,
    #1}

\usepackage{unixode}


\DeclareMathOperator{\ord}{ord}
\DeclareMathOperator{\orb}{orb}
\DeclareMathOperator{\stab}{stab}
\DeclareMathOperator{\Stab}{stab}
\DeclareMathOperator{\ppcm}{ppcm}
\DeclareMathOperator{\conj}{Conj}
\DeclareMathOperator{\End}{End}
\DeclareMathOperator{\rot}{rot}
\DeclareMathOperator{\trs}{trace}
\DeclareMathOperator{\Ind}{Ind}
\DeclareMathOperator{\mat}{Mat}
\DeclareMathOperator{\id}{Id}
\DeclareMathOperator{\vect}{vect}
\DeclareMathOperator{\img}{img}
\DeclareMathOperator{\cov}{Cov}
\DeclareMathOperator{\dist}{dist}
\DeclareMathOperator{\irr}{Irr}
\DeclareMathOperator{\image}{Im}
\DeclareMathOperator{\pd}{\partial}
\DeclareMathOperator{\epi}{epi}
\DeclareMathOperator{\Argmin}{Argmin}
\DeclareMathOperator{\dom}{dom}
\DeclareMathOperator{\proj}{proj}
\DeclareMathOperator{\ctg}{ctg}
\DeclareMathOperator{\supp}{supp}
\DeclareMathOperator{\argmin}{argmin}
\DeclareMathOperator{\mult}{mult}
\DeclareMathOperator{\ch}{ch}
\DeclareMathOperator{\sh}{sh}
\DeclareMathOperator{\rang}{rang}
\DeclareMathOperator{\diam}{diam}
\DeclareMathOperator{\Epigraphe}{Epigraphe}




\usepackage{xcolor}
\everymath{\color{blue}}
%\everymath{\color[rgb]{0,1,1}}
%\pagecolor[rgb]{0,0,0.5}


\newcommand*{\pdtest}[3][]{\ensuremath{\frac{\partial^{#1} #2}{\partial #3}}}

\newcommand*{\deffunc}[6][]{\ensuremath{
\begin{array}{rcl}
#2 : #3 &\rightarrow& #4\\
#5 &\mapsto& #6
\end{array}
}}

\newcommand{\eqcolon}{\mathrel{\resizebox{\widthof{$\mathord{=}$}}{\height}{ $\!\!=\!\!\resizebox{1.2\width}{0.8\height}{\raisebox{0.23ex}{$\mathop{:}$}}\!\!$ }}}
\newcommand{\coloneq}{\mathrel{\resizebox{\widthof{$\mathord{=}$}}{\height}{ $\!\!\resizebox{1.2\width}{0.8\height}{\raisebox{0.23ex}{$\mathop{:}$}}\!\!=\!\!$ }}}
\newcommand{\eqcolonl}{\ensuremath{\mathrel{=\!\!\mathop{:}}}}
\newcommand{\coloneql}{\ensuremath{\mathrel{\mathop{:} \!\! =}}}
\newcommand{\vc}[1]{% inline column vector
  \left(\begin{smallmatrix}#1\end{smallmatrix}\right)%
}
\newcommand{\vr}[1]{% inline row vector
  \begin{smallmatrix}(\,#1\,)\end{smallmatrix}%
}
\makeatletter
\newcommand*{\defeq}{\ =\mathrel{\rlap{%
                     \raisebox{0.3ex}{$\m@th\cdot$}}%
                     \raisebox{-0.3ex}{$\m@th\cdot$}}%
                     }
\makeatother

\newcommand{\mathcircle}[1]{% inline row vector
 \overset{\circ}{#1}
}
\newcommand{\ulim}{% low limit
 \underline{\lim}
}
\newcommand{\ssi}{% iff
\iff
}
\newcommand{\ps}[2]{
\expval{#1 | #2}
}
\newcommand{\df}[1]{
\mqty{#1}
}
\newcommand{\n}[1]{
\norm{#1}
}
\newcommand{\sys}[1]{
\left\{\smqty{#1}\right.
}


\newcommand{\eqdef}{\ensuremath{\overset{\text{def}}=}}


\def\Circlearrowright{\ensuremath{%
  \rotatebox[origin=c]{230}{$\circlearrowright$}}}

\newcommand\ct[1]{\text{\rmfamily\upshape #1}}
\newcommand\question[1]{ {\color{red} ...!? \small #1}}
\newcommand\caz[1]{\left\{\begin{array} #1 \end{array}\right.}
\newcommand\const{\text{\rmfamily\upshape const}}
\newcommand\toP{ \overset{\pro}{\to}}
\newcommand\toPP{ \overset{\text{PP}}{\to}}
\newcommand{\oeq}{\mathrel{\text{\textcircled{$=$}}}}





\usepackage{xcolor}
% \usepackage[normalem]{ulem}
\usepackage{lipsum}
\makeatletter
% \newcommand\colorwave[1][blue]{\bgroup \markoverwith{\lower3.5\p@\hbox{\sixly \textcolor{#1}{\char58}}}\ULon}
%\font\sixly=lasy6 % does not re-load if already loaded, so no memory problem.

\newmdtheoremenv[
linewidth= 1pt,linecolor= blue,%
leftmargin=20,rightmargin=20,innertopmargin=0pt, innerrightmargin=40,%
tikzsetting = { draw=lightgray, line width = 0.3pt,dashed,%
dash pattern = on 15pt off 3pt},%
splittopskip=\topskip,skipbelow=\baselineskip,%
skipabove=\baselineskip,ntheorem,roundcorner=0pt,
% backgroundcolor=pagebg,font=\color{orange}\sffamily, fontcolor=white
]{examplebox}{Exemple}[section]



\newcommand\R{\mathbb{R}}
\newcommand\Z{\mathbb{Z}}
\newcommand\N{\mathbb{N}}
\newcommand\E{\mathbb{E}}
\newcommand\F{\mathcal{F}}
\newcommand\cH{\mathcal{H}}
\newcommand\V{\mathbb{V}}
\newcommand\dmo{ ^{-1} }
\newcommand\kapa{\kappa}
\newcommand\im{Im}
\newcommand\hs{\mathcal{H}}





\usepackage{soul}

\makeatletter
\newcommand*{\whiten}[1]{\llap{\textcolor{white}{{\the\SOUL@token}}\hspace{#1pt}}}
\DeclareRobustCommand*\myul{%
    \def\SOUL@everyspace{\underline{\space}\kern\z@}%
    \def\SOUL@everytoken{%
     \setbox0=\hbox{\the\SOUL@token}%
     \ifdim\dp0>\z@
        \raisebox{\dp0}{\underline{\phantom{\the\SOUL@token}}}%
        \whiten{1}\whiten{0}%
        \whiten{-1}\whiten{-2}%
        \llap{\the\SOUL@token}%
     \else
        \underline{\the\SOUL@token}%
     \fi}%
\SOUL@}
\makeatother

\newcommand*{\demp}{\fontfamily{lmtt}\selectfont}

\DeclareTextFontCommand{\textdemp}{\demp}

\begin{document}

\ifcomment
Multiline
comment
\fi
\ifcomment
\myul{Typesetting test}
% \color[rgb]{1,1,1}
$∑_i^n≠ 60º±∞π∆¬≈√j∫h≤≥µ$

$\CR \R\pro\ind\pro\gS\pro
\mqty[a&b\\c&d]$
$\pro\mathbb{P}$
$\dd{x}$

  \[
    \alpha(x)=\left\{
                \begin{array}{ll}
                  x\\
                  \frac{1}{1+e^{-kx}}\\
                  \frac{e^x-e^{-x}}{e^x+e^{-x}}
                \end{array}
              \right.
  \]

  $\expval{x}$
  
  $\chi_\rho(ghg\dmo)=\Tr(\rho_{ghg\dmo})=\Tr(\rho_g\circ\rho_h\circ\rho\dmo_g)=\Tr(\rho_h)\overset{\mbox{\scalebox{0.5}{$\Tr(AB)=\Tr(BA)$}}}{=}\chi_\rho(h)$
  	$\mathop{\oplus}_{\substack{x\in X}}$

$\mat(\rho_g)=(a_{ij}(g))_{\scriptsize \substack{1\leq i\leq d \\ 1\leq j\leq d}}$ et $\mat(\rho'_g)=(a'_{ij}(g))_{\scriptsize \substack{1\leq i'\leq d' \\ 1\leq j'\leq d'}}$



\[\int_a^b{\mathbb{R}^2}g(u, v)\dd{P_{XY}}(u, v)=\iint g(u,v) f_{XY}(u, v)\dd \lambda(u) \dd \lambda(v)\]
$$\lim_{x\to\infty} f(x)$$	
$$\iiiint_V \mu(t,u,v,w) \,dt\,du\,dv\,dw$$
$$\sum_{n=1}^{\infty} 2^{-n} = 1$$	
\begin{definition}
	Si $X$ et $Y$ sont 2 v.a. ou definit la \textsc{Covariance} entre $X$ et $Y$ comme
	$\cov(X,Y)\overset{\text{def}}{=}\E\left[(X-\E(X))(Y-\E(Y))\right]=\E(XY)-\E(X)\E(Y)$.
\end{definition}
\fi
\pagebreak

% \tableofcontents

% insert your code here
%% !TEX encoding = UTF-8 Unicode
% !TEX TS-program = xelatex

\documentclass[french]{report}

%\usepackage[utf8]{inputenc}
%\usepackage[T1]{fontenc}
\usepackage{babel}


\newif\ifcomment
%\commenttrue # Show comments

\usepackage{physics}
\usepackage{amssymb}


\usepackage{amsthm}
% \usepackage{thmtools}
\usepackage{mathtools}
\usepackage{amsfonts}

\usepackage{color}

\usepackage{tikz}

\usepackage{geometry}
\geometry{a5paper, margin=0.1in, right=1cm}

\usepackage{dsfont}

\usepackage{graphicx}
\graphicspath{ {images/} }

\usepackage{faktor}

\usepackage{IEEEtrantools}
\usepackage{enumerate}   
\usepackage[PostScript=dvips]{"/Users/aware/Documents/Courses/diagrams"}


\newtheorem{theorem}{Théorème}[section]
\renewcommand{\thetheorem}{\arabic{theorem}}
\newtheorem{lemme}{Lemme}[section]
\renewcommand{\thelemme}{\arabic{lemme}}
\newtheorem{proposition}{Proposition}[section]
\renewcommand{\theproposition}{\arabic{proposition}}
\newtheorem{notations}{Notations}[section]
\newtheorem{problem}{Problème}[section]
\newtheorem{corollary}{Corollaire}[theorem]
\renewcommand{\thecorollary}{\arabic{corollary}}
\newtheorem{property}{Propriété}[section]
\newtheorem{objective}{Objectif}[section]

\theoremstyle{definition}
\newtheorem{definition}{Définition}[section]
\renewcommand{\thedefinition}{\arabic{definition}}
\newtheorem{exercise}{Exercice}[chapter]
\renewcommand{\theexercise}{\arabic{exercise}}
\newtheorem{example}{Exemple}[chapter]
\renewcommand{\theexample}{\arabic{example}}
\newtheorem*{solution}{Solution}
\newtheorem*{application}{Application}
\newtheorem*{notation}{Notation}
\newtheorem*{vocabulary}{Vocabulaire}
\newtheorem*{properties}{Propriétés}



\theoremstyle{remark}
\newtheorem*{remark}{Remarque}
\newtheorem*{rappel}{Rappel}


\usepackage{etoolbox}
\AtBeginEnvironment{exercise}{\small}
\AtBeginEnvironment{example}{\small}

\usepackage{cases}
\usepackage[red]{mypack}

\usepackage[framemethod=TikZ]{mdframed}

\definecolor{bg}{rgb}{0.4,0.25,0.95}
\definecolor{pagebg}{rgb}{0,0,0.5}
\surroundwithmdframed[
   topline=false,
   rightline=false,
   bottomline=false,
   leftmargin=\parindent,
   skipabove=8pt,
   skipbelow=8pt,
   linecolor=blue,
   innerbottommargin=10pt,
   % backgroundcolor=bg,font=\color{orange}\sffamily, fontcolor=white
]{definition}

\usepackage{empheq}
\usepackage[most]{tcolorbox}

\newtcbox{\mymath}[1][]{%
    nobeforeafter, math upper, tcbox raise base,
    enhanced, colframe=blue!30!black,
    colback=red!10, boxrule=1pt,
    #1}

\usepackage{unixode}


\DeclareMathOperator{\ord}{ord}
\DeclareMathOperator{\orb}{orb}
\DeclareMathOperator{\stab}{stab}
\DeclareMathOperator{\Stab}{stab}
\DeclareMathOperator{\ppcm}{ppcm}
\DeclareMathOperator{\conj}{Conj}
\DeclareMathOperator{\End}{End}
\DeclareMathOperator{\rot}{rot}
\DeclareMathOperator{\trs}{trace}
\DeclareMathOperator{\Ind}{Ind}
\DeclareMathOperator{\mat}{Mat}
\DeclareMathOperator{\id}{Id}
\DeclareMathOperator{\vect}{vect}
\DeclareMathOperator{\img}{img}
\DeclareMathOperator{\cov}{Cov}
\DeclareMathOperator{\dist}{dist}
\DeclareMathOperator{\irr}{Irr}
\DeclareMathOperator{\image}{Im}
\DeclareMathOperator{\pd}{\partial}
\DeclareMathOperator{\epi}{epi}
\DeclareMathOperator{\Argmin}{Argmin}
\DeclareMathOperator{\dom}{dom}
\DeclareMathOperator{\proj}{proj}
\DeclareMathOperator{\ctg}{ctg}
\DeclareMathOperator{\supp}{supp}
\DeclareMathOperator{\argmin}{argmin}
\DeclareMathOperator{\mult}{mult}
\DeclareMathOperator{\ch}{ch}
\DeclareMathOperator{\sh}{sh}
\DeclareMathOperator{\rang}{rang}
\DeclareMathOperator{\diam}{diam}
\DeclareMathOperator{\Epigraphe}{Epigraphe}




\usepackage{xcolor}
\everymath{\color{blue}}
%\everymath{\color[rgb]{0,1,1}}
%\pagecolor[rgb]{0,0,0.5}


\newcommand*{\pdtest}[3][]{\ensuremath{\frac{\partial^{#1} #2}{\partial #3}}}

\newcommand*{\deffunc}[6][]{\ensuremath{
\begin{array}{rcl}
#2 : #3 &\rightarrow& #4\\
#5 &\mapsto& #6
\end{array}
}}

\newcommand{\eqcolon}{\mathrel{\resizebox{\widthof{$\mathord{=}$}}{\height}{ $\!\!=\!\!\resizebox{1.2\width}{0.8\height}{\raisebox{0.23ex}{$\mathop{:}$}}\!\!$ }}}
\newcommand{\coloneq}{\mathrel{\resizebox{\widthof{$\mathord{=}$}}{\height}{ $\!\!\resizebox{1.2\width}{0.8\height}{\raisebox{0.23ex}{$\mathop{:}$}}\!\!=\!\!$ }}}
\newcommand{\eqcolonl}{\ensuremath{\mathrel{=\!\!\mathop{:}}}}
\newcommand{\coloneql}{\ensuremath{\mathrel{\mathop{:} \!\! =}}}
\newcommand{\vc}[1]{% inline column vector
  \left(\begin{smallmatrix}#1\end{smallmatrix}\right)%
}
\newcommand{\vr}[1]{% inline row vector
  \begin{smallmatrix}(\,#1\,)\end{smallmatrix}%
}
\makeatletter
\newcommand*{\defeq}{\ =\mathrel{\rlap{%
                     \raisebox{0.3ex}{$\m@th\cdot$}}%
                     \raisebox{-0.3ex}{$\m@th\cdot$}}%
                     }
\makeatother

\newcommand{\mathcircle}[1]{% inline row vector
 \overset{\circ}{#1}
}
\newcommand{\ulim}{% low limit
 \underline{\lim}
}
\newcommand{\ssi}{% iff
\iff
}
\newcommand{\ps}[2]{
\expval{#1 | #2}
}
\newcommand{\df}[1]{
\mqty{#1}
}
\newcommand{\n}[1]{
\norm{#1}
}
\newcommand{\sys}[1]{
\left\{\smqty{#1}\right.
}


\newcommand{\eqdef}{\ensuremath{\overset{\text{def}}=}}


\def\Circlearrowright{\ensuremath{%
  \rotatebox[origin=c]{230}{$\circlearrowright$}}}

\newcommand\ct[1]{\text{\rmfamily\upshape #1}}
\newcommand\question[1]{ {\color{red} ...!? \small #1}}
\newcommand\caz[1]{\left\{\begin{array} #1 \end{array}\right.}
\newcommand\const{\text{\rmfamily\upshape const}}
\newcommand\toP{ \overset{\pro}{\to}}
\newcommand\toPP{ \overset{\text{PP}}{\to}}
\newcommand{\oeq}{\mathrel{\text{\textcircled{$=$}}}}





\usepackage{xcolor}
% \usepackage[normalem]{ulem}
\usepackage{lipsum}
\makeatletter
% \newcommand\colorwave[1][blue]{\bgroup \markoverwith{\lower3.5\p@\hbox{\sixly \textcolor{#1}{\char58}}}\ULon}
%\font\sixly=lasy6 % does not re-load if already loaded, so no memory problem.

\newmdtheoremenv[
linewidth= 1pt,linecolor= blue,%
leftmargin=20,rightmargin=20,innertopmargin=0pt, innerrightmargin=40,%
tikzsetting = { draw=lightgray, line width = 0.3pt,dashed,%
dash pattern = on 15pt off 3pt},%
splittopskip=\topskip,skipbelow=\baselineskip,%
skipabove=\baselineskip,ntheorem,roundcorner=0pt,
% backgroundcolor=pagebg,font=\color{orange}\sffamily, fontcolor=white
]{examplebox}{Exemple}[section]



\newcommand\R{\mathbb{R}}
\newcommand\Z{\mathbb{Z}}
\newcommand\N{\mathbb{N}}
\newcommand\E{\mathbb{E}}
\newcommand\F{\mathcal{F}}
\newcommand\cH{\mathcal{H}}
\newcommand\V{\mathbb{V}}
\newcommand\dmo{ ^{-1} }
\newcommand\kapa{\kappa}
\newcommand\im{Im}
\newcommand\hs{\mathcal{H}}





\usepackage{soul}

\makeatletter
\newcommand*{\whiten}[1]{\llap{\textcolor{white}{{\the\SOUL@token}}\hspace{#1pt}}}
\DeclareRobustCommand*\myul{%
    \def\SOUL@everyspace{\underline{\space}\kern\z@}%
    \def\SOUL@everytoken{%
     \setbox0=\hbox{\the\SOUL@token}%
     \ifdim\dp0>\z@
        \raisebox{\dp0}{\underline{\phantom{\the\SOUL@token}}}%
        \whiten{1}\whiten{0}%
        \whiten{-1}\whiten{-2}%
        \llap{\the\SOUL@token}%
     \else
        \underline{\the\SOUL@token}%
     \fi}%
\SOUL@}
\makeatother

\newcommand*{\demp}{\fontfamily{lmtt}\selectfont}

\DeclareTextFontCommand{\textdemp}{\demp}

\begin{document}

\ifcomment
Multiline
comment
\fi
\ifcomment
\myul{Typesetting test}
% \color[rgb]{1,1,1}
$∑_i^n≠ 60º±∞π∆¬≈√j∫h≤≥µ$

$\CR \R\pro\ind\pro\gS\pro
\mqty[a&b\\c&d]$
$\pro\mathbb{P}$
$\dd{x}$

  \[
    \alpha(x)=\left\{
                \begin{array}{ll}
                  x\\
                  \frac{1}{1+e^{-kx}}\\
                  \frac{e^x-e^{-x}}{e^x+e^{-x}}
                \end{array}
              \right.
  \]

  $\expval{x}$
  
  $\chi_\rho(ghg\dmo)=\Tr(\rho_{ghg\dmo})=\Tr(\rho_g\circ\rho_h\circ\rho\dmo_g)=\Tr(\rho_h)\overset{\mbox{\scalebox{0.5}{$\Tr(AB)=\Tr(BA)$}}}{=}\chi_\rho(h)$
  	$\mathop{\oplus}_{\substack{x\in X}}$

$\mat(\rho_g)=(a_{ij}(g))_{\scriptsize \substack{1\leq i\leq d \\ 1\leq j\leq d}}$ et $\mat(\rho'_g)=(a'_{ij}(g))_{\scriptsize \substack{1\leq i'\leq d' \\ 1\leq j'\leq d'}}$



\[\int_a^b{\mathbb{R}^2}g(u, v)\dd{P_{XY}}(u, v)=\iint g(u,v) f_{XY}(u, v)\dd \lambda(u) \dd \lambda(v)\]
$$\lim_{x\to\infty} f(x)$$	
$$\iiiint_V \mu(t,u,v,w) \,dt\,du\,dv\,dw$$
$$\sum_{n=1}^{\infty} 2^{-n} = 1$$	
\begin{definition}
	Si $X$ et $Y$ sont 2 v.a. ou definit la \textsc{Covariance} entre $X$ et $Y$ comme
	$\cov(X,Y)\overset{\text{def}}{=}\E\left[(X-\E(X))(Y-\E(Y))\right]=\E(XY)-\E(X)\E(Y)$.
\end{definition}
\fi
\pagebreak

% \tableofcontents

% insert your code here
%% !TEX encoding = UTF-8 Unicode
% !TEX TS-program = xelatex

\documentclass[french]{report}

%\usepackage[utf8]{inputenc}
%\usepackage[T1]{fontenc}
\usepackage{babel}


\newif\ifcomment
%\commenttrue # Show comments

\usepackage{physics}
\usepackage{amssymb}


\usepackage{amsthm}
% \usepackage{thmtools}
\usepackage{mathtools}
\usepackage{amsfonts}

\usepackage{color}

\usepackage{tikz}

\usepackage{geometry}
\geometry{a5paper, margin=0.1in, right=1cm}

\usepackage{dsfont}

\usepackage{graphicx}
\graphicspath{ {images/} }

\usepackage{faktor}

\usepackage{IEEEtrantools}
\usepackage{enumerate}   
\usepackage[PostScript=dvips]{"/Users/aware/Documents/Courses/diagrams"}


\newtheorem{theorem}{Théorème}[section]
\renewcommand{\thetheorem}{\arabic{theorem}}
\newtheorem{lemme}{Lemme}[section]
\renewcommand{\thelemme}{\arabic{lemme}}
\newtheorem{proposition}{Proposition}[section]
\renewcommand{\theproposition}{\arabic{proposition}}
\newtheorem{notations}{Notations}[section]
\newtheorem{problem}{Problème}[section]
\newtheorem{corollary}{Corollaire}[theorem]
\renewcommand{\thecorollary}{\arabic{corollary}}
\newtheorem{property}{Propriété}[section]
\newtheorem{objective}{Objectif}[section]

\theoremstyle{definition}
\newtheorem{definition}{Définition}[section]
\renewcommand{\thedefinition}{\arabic{definition}}
\newtheorem{exercise}{Exercice}[chapter]
\renewcommand{\theexercise}{\arabic{exercise}}
\newtheorem{example}{Exemple}[chapter]
\renewcommand{\theexample}{\arabic{example}}
\newtheorem*{solution}{Solution}
\newtheorem*{application}{Application}
\newtheorem*{notation}{Notation}
\newtheorem*{vocabulary}{Vocabulaire}
\newtheorem*{properties}{Propriétés}



\theoremstyle{remark}
\newtheorem*{remark}{Remarque}
\newtheorem*{rappel}{Rappel}


\usepackage{etoolbox}
\AtBeginEnvironment{exercise}{\small}
\AtBeginEnvironment{example}{\small}

\usepackage{cases}
\usepackage[red]{mypack}

\usepackage[framemethod=TikZ]{mdframed}

\definecolor{bg}{rgb}{0.4,0.25,0.95}
\definecolor{pagebg}{rgb}{0,0,0.5}
\surroundwithmdframed[
   topline=false,
   rightline=false,
   bottomline=false,
   leftmargin=\parindent,
   skipabove=8pt,
   skipbelow=8pt,
   linecolor=blue,
   innerbottommargin=10pt,
   % backgroundcolor=bg,font=\color{orange}\sffamily, fontcolor=white
]{definition}

\usepackage{empheq}
\usepackage[most]{tcolorbox}

\newtcbox{\mymath}[1][]{%
    nobeforeafter, math upper, tcbox raise base,
    enhanced, colframe=blue!30!black,
    colback=red!10, boxrule=1pt,
    #1}

\usepackage{unixode}


\DeclareMathOperator{\ord}{ord}
\DeclareMathOperator{\orb}{orb}
\DeclareMathOperator{\stab}{stab}
\DeclareMathOperator{\Stab}{stab}
\DeclareMathOperator{\ppcm}{ppcm}
\DeclareMathOperator{\conj}{Conj}
\DeclareMathOperator{\End}{End}
\DeclareMathOperator{\rot}{rot}
\DeclareMathOperator{\trs}{trace}
\DeclareMathOperator{\Ind}{Ind}
\DeclareMathOperator{\mat}{Mat}
\DeclareMathOperator{\id}{Id}
\DeclareMathOperator{\vect}{vect}
\DeclareMathOperator{\img}{img}
\DeclareMathOperator{\cov}{Cov}
\DeclareMathOperator{\dist}{dist}
\DeclareMathOperator{\irr}{Irr}
\DeclareMathOperator{\image}{Im}
\DeclareMathOperator{\pd}{\partial}
\DeclareMathOperator{\epi}{epi}
\DeclareMathOperator{\Argmin}{Argmin}
\DeclareMathOperator{\dom}{dom}
\DeclareMathOperator{\proj}{proj}
\DeclareMathOperator{\ctg}{ctg}
\DeclareMathOperator{\supp}{supp}
\DeclareMathOperator{\argmin}{argmin}
\DeclareMathOperator{\mult}{mult}
\DeclareMathOperator{\ch}{ch}
\DeclareMathOperator{\sh}{sh}
\DeclareMathOperator{\rang}{rang}
\DeclareMathOperator{\diam}{diam}
\DeclareMathOperator{\Epigraphe}{Epigraphe}




\usepackage{xcolor}
\everymath{\color{blue}}
%\everymath{\color[rgb]{0,1,1}}
%\pagecolor[rgb]{0,0,0.5}


\newcommand*{\pdtest}[3][]{\ensuremath{\frac{\partial^{#1} #2}{\partial #3}}}

\newcommand*{\deffunc}[6][]{\ensuremath{
\begin{array}{rcl}
#2 : #3 &\rightarrow& #4\\
#5 &\mapsto& #6
\end{array}
}}

\newcommand{\eqcolon}{\mathrel{\resizebox{\widthof{$\mathord{=}$}}{\height}{ $\!\!=\!\!\resizebox{1.2\width}{0.8\height}{\raisebox{0.23ex}{$\mathop{:}$}}\!\!$ }}}
\newcommand{\coloneq}{\mathrel{\resizebox{\widthof{$\mathord{=}$}}{\height}{ $\!\!\resizebox{1.2\width}{0.8\height}{\raisebox{0.23ex}{$\mathop{:}$}}\!\!=\!\!$ }}}
\newcommand{\eqcolonl}{\ensuremath{\mathrel{=\!\!\mathop{:}}}}
\newcommand{\coloneql}{\ensuremath{\mathrel{\mathop{:} \!\! =}}}
\newcommand{\vc}[1]{% inline column vector
  \left(\begin{smallmatrix}#1\end{smallmatrix}\right)%
}
\newcommand{\vr}[1]{% inline row vector
  \begin{smallmatrix}(\,#1\,)\end{smallmatrix}%
}
\makeatletter
\newcommand*{\defeq}{\ =\mathrel{\rlap{%
                     \raisebox{0.3ex}{$\m@th\cdot$}}%
                     \raisebox{-0.3ex}{$\m@th\cdot$}}%
                     }
\makeatother

\newcommand{\mathcircle}[1]{% inline row vector
 \overset{\circ}{#1}
}
\newcommand{\ulim}{% low limit
 \underline{\lim}
}
\newcommand{\ssi}{% iff
\iff
}
\newcommand{\ps}[2]{
\expval{#1 | #2}
}
\newcommand{\df}[1]{
\mqty{#1}
}
\newcommand{\n}[1]{
\norm{#1}
}
\newcommand{\sys}[1]{
\left\{\smqty{#1}\right.
}


\newcommand{\eqdef}{\ensuremath{\overset{\text{def}}=}}


\def\Circlearrowright{\ensuremath{%
  \rotatebox[origin=c]{230}{$\circlearrowright$}}}

\newcommand\ct[1]{\text{\rmfamily\upshape #1}}
\newcommand\question[1]{ {\color{red} ...!? \small #1}}
\newcommand\caz[1]{\left\{\begin{array} #1 \end{array}\right.}
\newcommand\const{\text{\rmfamily\upshape const}}
\newcommand\toP{ \overset{\pro}{\to}}
\newcommand\toPP{ \overset{\text{PP}}{\to}}
\newcommand{\oeq}{\mathrel{\text{\textcircled{$=$}}}}





\usepackage{xcolor}
% \usepackage[normalem]{ulem}
\usepackage{lipsum}
\makeatletter
% \newcommand\colorwave[1][blue]{\bgroup \markoverwith{\lower3.5\p@\hbox{\sixly \textcolor{#1}{\char58}}}\ULon}
%\font\sixly=lasy6 % does not re-load if already loaded, so no memory problem.

\newmdtheoremenv[
linewidth= 1pt,linecolor= blue,%
leftmargin=20,rightmargin=20,innertopmargin=0pt, innerrightmargin=40,%
tikzsetting = { draw=lightgray, line width = 0.3pt,dashed,%
dash pattern = on 15pt off 3pt},%
splittopskip=\topskip,skipbelow=\baselineskip,%
skipabove=\baselineskip,ntheorem,roundcorner=0pt,
% backgroundcolor=pagebg,font=\color{orange}\sffamily, fontcolor=white
]{examplebox}{Exemple}[section]



\newcommand\R{\mathbb{R}}
\newcommand\Z{\mathbb{Z}}
\newcommand\N{\mathbb{N}}
\newcommand\E{\mathbb{E}}
\newcommand\F{\mathcal{F}}
\newcommand\cH{\mathcal{H}}
\newcommand\V{\mathbb{V}}
\newcommand\dmo{ ^{-1} }
\newcommand\kapa{\kappa}
\newcommand\im{Im}
\newcommand\hs{\mathcal{H}}





\usepackage{soul}

\makeatletter
\newcommand*{\whiten}[1]{\llap{\textcolor{white}{{\the\SOUL@token}}\hspace{#1pt}}}
\DeclareRobustCommand*\myul{%
    \def\SOUL@everyspace{\underline{\space}\kern\z@}%
    \def\SOUL@everytoken{%
     \setbox0=\hbox{\the\SOUL@token}%
     \ifdim\dp0>\z@
        \raisebox{\dp0}{\underline{\phantom{\the\SOUL@token}}}%
        \whiten{1}\whiten{0}%
        \whiten{-1}\whiten{-2}%
        \llap{\the\SOUL@token}%
     \else
        \underline{\the\SOUL@token}%
     \fi}%
\SOUL@}
\makeatother

\newcommand*{\demp}{\fontfamily{lmtt}\selectfont}

\DeclareTextFontCommand{\textdemp}{\demp}

\begin{document}

\ifcomment
Multiline
comment
\fi
\ifcomment
\myul{Typesetting test}
% \color[rgb]{1,1,1}
$∑_i^n≠ 60º±∞π∆¬≈√j∫h≤≥µ$

$\CR \R\pro\ind\pro\gS\pro
\mqty[a&b\\c&d]$
$\pro\mathbb{P}$
$\dd{x}$

  \[
    \alpha(x)=\left\{
                \begin{array}{ll}
                  x\\
                  \frac{1}{1+e^{-kx}}\\
                  \frac{e^x-e^{-x}}{e^x+e^{-x}}
                \end{array}
              \right.
  \]

  $\expval{x}$
  
  $\chi_\rho(ghg\dmo)=\Tr(\rho_{ghg\dmo})=\Tr(\rho_g\circ\rho_h\circ\rho\dmo_g)=\Tr(\rho_h)\overset{\mbox{\scalebox{0.5}{$\Tr(AB)=\Tr(BA)$}}}{=}\chi_\rho(h)$
  	$\mathop{\oplus}_{\substack{x\in X}}$

$\mat(\rho_g)=(a_{ij}(g))_{\scriptsize \substack{1\leq i\leq d \\ 1\leq j\leq d}}$ et $\mat(\rho'_g)=(a'_{ij}(g))_{\scriptsize \substack{1\leq i'\leq d' \\ 1\leq j'\leq d'}}$



\[\int_a^b{\mathbb{R}^2}g(u, v)\dd{P_{XY}}(u, v)=\iint g(u,v) f_{XY}(u, v)\dd \lambda(u) \dd \lambda(v)\]
$$\lim_{x\to\infty} f(x)$$	
$$\iiiint_V \mu(t,u,v,w) \,dt\,du\,dv\,dw$$
$$\sum_{n=1}^{\infty} 2^{-n} = 1$$	
\begin{definition}
	Si $X$ et $Y$ sont 2 v.a. ou definit la \textsc{Covariance} entre $X$ et $Y$ comme
	$\cov(X,Y)\overset{\text{def}}{=}\E\left[(X-\E(X))(Y-\E(Y))\right]=\E(XY)-\E(X)\E(Y)$.
\end{definition}
\fi
\pagebreak

% \tableofcontents

% insert your code here
%\input{./algebra/main.tex}
%\input{./geometrie-differentielle/main.tex}
%\input{./probabilite/main.tex}
%\input{./analyse-fonctionnelle/main.tex}
% \input{./Analyse-convexe-et-dualite-en-optimisation/main.tex}
%\input{./tikz/main.tex}
%\input{./Theorie-du-distributions/main.tex}
%\input{./optimisation/mine.tex}
 \input{./modelisation/main.tex}

% yves.aubry@univ-tln.fr : algebra

\end{document}

%% !TEX encoding = UTF-8 Unicode
% !TEX TS-program = xelatex

\documentclass[french]{report}

%\usepackage[utf8]{inputenc}
%\usepackage[T1]{fontenc}
\usepackage{babel}


\newif\ifcomment
%\commenttrue # Show comments

\usepackage{physics}
\usepackage{amssymb}


\usepackage{amsthm}
% \usepackage{thmtools}
\usepackage{mathtools}
\usepackage{amsfonts}

\usepackage{color}

\usepackage{tikz}

\usepackage{geometry}
\geometry{a5paper, margin=0.1in, right=1cm}

\usepackage{dsfont}

\usepackage{graphicx}
\graphicspath{ {images/} }

\usepackage{faktor}

\usepackage{IEEEtrantools}
\usepackage{enumerate}   
\usepackage[PostScript=dvips]{"/Users/aware/Documents/Courses/diagrams"}


\newtheorem{theorem}{Théorème}[section]
\renewcommand{\thetheorem}{\arabic{theorem}}
\newtheorem{lemme}{Lemme}[section]
\renewcommand{\thelemme}{\arabic{lemme}}
\newtheorem{proposition}{Proposition}[section]
\renewcommand{\theproposition}{\arabic{proposition}}
\newtheorem{notations}{Notations}[section]
\newtheorem{problem}{Problème}[section]
\newtheorem{corollary}{Corollaire}[theorem]
\renewcommand{\thecorollary}{\arabic{corollary}}
\newtheorem{property}{Propriété}[section]
\newtheorem{objective}{Objectif}[section]

\theoremstyle{definition}
\newtheorem{definition}{Définition}[section]
\renewcommand{\thedefinition}{\arabic{definition}}
\newtheorem{exercise}{Exercice}[chapter]
\renewcommand{\theexercise}{\arabic{exercise}}
\newtheorem{example}{Exemple}[chapter]
\renewcommand{\theexample}{\arabic{example}}
\newtheorem*{solution}{Solution}
\newtheorem*{application}{Application}
\newtheorem*{notation}{Notation}
\newtheorem*{vocabulary}{Vocabulaire}
\newtheorem*{properties}{Propriétés}



\theoremstyle{remark}
\newtheorem*{remark}{Remarque}
\newtheorem*{rappel}{Rappel}


\usepackage{etoolbox}
\AtBeginEnvironment{exercise}{\small}
\AtBeginEnvironment{example}{\small}

\usepackage{cases}
\usepackage[red]{mypack}

\usepackage[framemethod=TikZ]{mdframed}

\definecolor{bg}{rgb}{0.4,0.25,0.95}
\definecolor{pagebg}{rgb}{0,0,0.5}
\surroundwithmdframed[
   topline=false,
   rightline=false,
   bottomline=false,
   leftmargin=\parindent,
   skipabove=8pt,
   skipbelow=8pt,
   linecolor=blue,
   innerbottommargin=10pt,
   % backgroundcolor=bg,font=\color{orange}\sffamily, fontcolor=white
]{definition}

\usepackage{empheq}
\usepackage[most]{tcolorbox}

\newtcbox{\mymath}[1][]{%
    nobeforeafter, math upper, tcbox raise base,
    enhanced, colframe=blue!30!black,
    colback=red!10, boxrule=1pt,
    #1}

\usepackage{unixode}


\DeclareMathOperator{\ord}{ord}
\DeclareMathOperator{\orb}{orb}
\DeclareMathOperator{\stab}{stab}
\DeclareMathOperator{\Stab}{stab}
\DeclareMathOperator{\ppcm}{ppcm}
\DeclareMathOperator{\conj}{Conj}
\DeclareMathOperator{\End}{End}
\DeclareMathOperator{\rot}{rot}
\DeclareMathOperator{\trs}{trace}
\DeclareMathOperator{\Ind}{Ind}
\DeclareMathOperator{\mat}{Mat}
\DeclareMathOperator{\id}{Id}
\DeclareMathOperator{\vect}{vect}
\DeclareMathOperator{\img}{img}
\DeclareMathOperator{\cov}{Cov}
\DeclareMathOperator{\dist}{dist}
\DeclareMathOperator{\irr}{Irr}
\DeclareMathOperator{\image}{Im}
\DeclareMathOperator{\pd}{\partial}
\DeclareMathOperator{\epi}{epi}
\DeclareMathOperator{\Argmin}{Argmin}
\DeclareMathOperator{\dom}{dom}
\DeclareMathOperator{\proj}{proj}
\DeclareMathOperator{\ctg}{ctg}
\DeclareMathOperator{\supp}{supp}
\DeclareMathOperator{\argmin}{argmin}
\DeclareMathOperator{\mult}{mult}
\DeclareMathOperator{\ch}{ch}
\DeclareMathOperator{\sh}{sh}
\DeclareMathOperator{\rang}{rang}
\DeclareMathOperator{\diam}{diam}
\DeclareMathOperator{\Epigraphe}{Epigraphe}




\usepackage{xcolor}
\everymath{\color{blue}}
%\everymath{\color[rgb]{0,1,1}}
%\pagecolor[rgb]{0,0,0.5}


\newcommand*{\pdtest}[3][]{\ensuremath{\frac{\partial^{#1} #2}{\partial #3}}}

\newcommand*{\deffunc}[6][]{\ensuremath{
\begin{array}{rcl}
#2 : #3 &\rightarrow& #4\\
#5 &\mapsto& #6
\end{array}
}}

\newcommand{\eqcolon}{\mathrel{\resizebox{\widthof{$\mathord{=}$}}{\height}{ $\!\!=\!\!\resizebox{1.2\width}{0.8\height}{\raisebox{0.23ex}{$\mathop{:}$}}\!\!$ }}}
\newcommand{\coloneq}{\mathrel{\resizebox{\widthof{$\mathord{=}$}}{\height}{ $\!\!\resizebox{1.2\width}{0.8\height}{\raisebox{0.23ex}{$\mathop{:}$}}\!\!=\!\!$ }}}
\newcommand{\eqcolonl}{\ensuremath{\mathrel{=\!\!\mathop{:}}}}
\newcommand{\coloneql}{\ensuremath{\mathrel{\mathop{:} \!\! =}}}
\newcommand{\vc}[1]{% inline column vector
  \left(\begin{smallmatrix}#1\end{smallmatrix}\right)%
}
\newcommand{\vr}[1]{% inline row vector
  \begin{smallmatrix}(\,#1\,)\end{smallmatrix}%
}
\makeatletter
\newcommand*{\defeq}{\ =\mathrel{\rlap{%
                     \raisebox{0.3ex}{$\m@th\cdot$}}%
                     \raisebox{-0.3ex}{$\m@th\cdot$}}%
                     }
\makeatother

\newcommand{\mathcircle}[1]{% inline row vector
 \overset{\circ}{#1}
}
\newcommand{\ulim}{% low limit
 \underline{\lim}
}
\newcommand{\ssi}{% iff
\iff
}
\newcommand{\ps}[2]{
\expval{#1 | #2}
}
\newcommand{\df}[1]{
\mqty{#1}
}
\newcommand{\n}[1]{
\norm{#1}
}
\newcommand{\sys}[1]{
\left\{\smqty{#1}\right.
}


\newcommand{\eqdef}{\ensuremath{\overset{\text{def}}=}}


\def\Circlearrowright{\ensuremath{%
  \rotatebox[origin=c]{230}{$\circlearrowright$}}}

\newcommand\ct[1]{\text{\rmfamily\upshape #1}}
\newcommand\question[1]{ {\color{red} ...!? \small #1}}
\newcommand\caz[1]{\left\{\begin{array} #1 \end{array}\right.}
\newcommand\const{\text{\rmfamily\upshape const}}
\newcommand\toP{ \overset{\pro}{\to}}
\newcommand\toPP{ \overset{\text{PP}}{\to}}
\newcommand{\oeq}{\mathrel{\text{\textcircled{$=$}}}}





\usepackage{xcolor}
% \usepackage[normalem]{ulem}
\usepackage{lipsum}
\makeatletter
% \newcommand\colorwave[1][blue]{\bgroup \markoverwith{\lower3.5\p@\hbox{\sixly \textcolor{#1}{\char58}}}\ULon}
%\font\sixly=lasy6 % does not re-load if already loaded, so no memory problem.

\newmdtheoremenv[
linewidth= 1pt,linecolor= blue,%
leftmargin=20,rightmargin=20,innertopmargin=0pt, innerrightmargin=40,%
tikzsetting = { draw=lightgray, line width = 0.3pt,dashed,%
dash pattern = on 15pt off 3pt},%
splittopskip=\topskip,skipbelow=\baselineskip,%
skipabove=\baselineskip,ntheorem,roundcorner=0pt,
% backgroundcolor=pagebg,font=\color{orange}\sffamily, fontcolor=white
]{examplebox}{Exemple}[section]



\newcommand\R{\mathbb{R}}
\newcommand\Z{\mathbb{Z}}
\newcommand\N{\mathbb{N}}
\newcommand\E{\mathbb{E}}
\newcommand\F{\mathcal{F}}
\newcommand\cH{\mathcal{H}}
\newcommand\V{\mathbb{V}}
\newcommand\dmo{ ^{-1} }
\newcommand\kapa{\kappa}
\newcommand\im{Im}
\newcommand\hs{\mathcal{H}}





\usepackage{soul}

\makeatletter
\newcommand*{\whiten}[1]{\llap{\textcolor{white}{{\the\SOUL@token}}\hspace{#1pt}}}
\DeclareRobustCommand*\myul{%
    \def\SOUL@everyspace{\underline{\space}\kern\z@}%
    \def\SOUL@everytoken{%
     \setbox0=\hbox{\the\SOUL@token}%
     \ifdim\dp0>\z@
        \raisebox{\dp0}{\underline{\phantom{\the\SOUL@token}}}%
        \whiten{1}\whiten{0}%
        \whiten{-1}\whiten{-2}%
        \llap{\the\SOUL@token}%
     \else
        \underline{\the\SOUL@token}%
     \fi}%
\SOUL@}
\makeatother

\newcommand*{\demp}{\fontfamily{lmtt}\selectfont}

\DeclareTextFontCommand{\textdemp}{\demp}

\begin{document}

\ifcomment
Multiline
comment
\fi
\ifcomment
\myul{Typesetting test}
% \color[rgb]{1,1,1}
$∑_i^n≠ 60º±∞π∆¬≈√j∫h≤≥µ$

$\CR \R\pro\ind\pro\gS\pro
\mqty[a&b\\c&d]$
$\pro\mathbb{P}$
$\dd{x}$

  \[
    \alpha(x)=\left\{
                \begin{array}{ll}
                  x\\
                  \frac{1}{1+e^{-kx}}\\
                  \frac{e^x-e^{-x}}{e^x+e^{-x}}
                \end{array}
              \right.
  \]

  $\expval{x}$
  
  $\chi_\rho(ghg\dmo)=\Tr(\rho_{ghg\dmo})=\Tr(\rho_g\circ\rho_h\circ\rho\dmo_g)=\Tr(\rho_h)\overset{\mbox{\scalebox{0.5}{$\Tr(AB)=\Tr(BA)$}}}{=}\chi_\rho(h)$
  	$\mathop{\oplus}_{\substack{x\in X}}$

$\mat(\rho_g)=(a_{ij}(g))_{\scriptsize \substack{1\leq i\leq d \\ 1\leq j\leq d}}$ et $\mat(\rho'_g)=(a'_{ij}(g))_{\scriptsize \substack{1\leq i'\leq d' \\ 1\leq j'\leq d'}}$



\[\int_a^b{\mathbb{R}^2}g(u, v)\dd{P_{XY}}(u, v)=\iint g(u,v) f_{XY}(u, v)\dd \lambda(u) \dd \lambda(v)\]
$$\lim_{x\to\infty} f(x)$$	
$$\iiiint_V \mu(t,u,v,w) \,dt\,du\,dv\,dw$$
$$\sum_{n=1}^{\infty} 2^{-n} = 1$$	
\begin{definition}
	Si $X$ et $Y$ sont 2 v.a. ou definit la \textsc{Covariance} entre $X$ et $Y$ comme
	$\cov(X,Y)\overset{\text{def}}{=}\E\left[(X-\E(X))(Y-\E(Y))\right]=\E(XY)-\E(X)\E(Y)$.
\end{definition}
\fi
\pagebreak

% \tableofcontents

% insert your code here
%\input{./algebra/main.tex}
%\input{./geometrie-differentielle/main.tex}
%\input{./probabilite/main.tex}
%\input{./analyse-fonctionnelle/main.tex}
% \input{./Analyse-convexe-et-dualite-en-optimisation/main.tex}
%\input{./tikz/main.tex}
%\input{./Theorie-du-distributions/main.tex}
%\input{./optimisation/mine.tex}
 \input{./modelisation/main.tex}

% yves.aubry@univ-tln.fr : algebra

\end{document}

%% !TEX encoding = UTF-8 Unicode
% !TEX TS-program = xelatex

\documentclass[french]{report}

%\usepackage[utf8]{inputenc}
%\usepackage[T1]{fontenc}
\usepackage{babel}


\newif\ifcomment
%\commenttrue # Show comments

\usepackage{physics}
\usepackage{amssymb}


\usepackage{amsthm}
% \usepackage{thmtools}
\usepackage{mathtools}
\usepackage{amsfonts}

\usepackage{color}

\usepackage{tikz}

\usepackage{geometry}
\geometry{a5paper, margin=0.1in, right=1cm}

\usepackage{dsfont}

\usepackage{graphicx}
\graphicspath{ {images/} }

\usepackage{faktor}

\usepackage{IEEEtrantools}
\usepackage{enumerate}   
\usepackage[PostScript=dvips]{"/Users/aware/Documents/Courses/diagrams"}


\newtheorem{theorem}{Théorème}[section]
\renewcommand{\thetheorem}{\arabic{theorem}}
\newtheorem{lemme}{Lemme}[section]
\renewcommand{\thelemme}{\arabic{lemme}}
\newtheorem{proposition}{Proposition}[section]
\renewcommand{\theproposition}{\arabic{proposition}}
\newtheorem{notations}{Notations}[section]
\newtheorem{problem}{Problème}[section]
\newtheorem{corollary}{Corollaire}[theorem]
\renewcommand{\thecorollary}{\arabic{corollary}}
\newtheorem{property}{Propriété}[section]
\newtheorem{objective}{Objectif}[section]

\theoremstyle{definition}
\newtheorem{definition}{Définition}[section]
\renewcommand{\thedefinition}{\arabic{definition}}
\newtheorem{exercise}{Exercice}[chapter]
\renewcommand{\theexercise}{\arabic{exercise}}
\newtheorem{example}{Exemple}[chapter]
\renewcommand{\theexample}{\arabic{example}}
\newtheorem*{solution}{Solution}
\newtheorem*{application}{Application}
\newtheorem*{notation}{Notation}
\newtheorem*{vocabulary}{Vocabulaire}
\newtheorem*{properties}{Propriétés}



\theoremstyle{remark}
\newtheorem*{remark}{Remarque}
\newtheorem*{rappel}{Rappel}


\usepackage{etoolbox}
\AtBeginEnvironment{exercise}{\small}
\AtBeginEnvironment{example}{\small}

\usepackage{cases}
\usepackage[red]{mypack}

\usepackage[framemethod=TikZ]{mdframed}

\definecolor{bg}{rgb}{0.4,0.25,0.95}
\definecolor{pagebg}{rgb}{0,0,0.5}
\surroundwithmdframed[
   topline=false,
   rightline=false,
   bottomline=false,
   leftmargin=\parindent,
   skipabove=8pt,
   skipbelow=8pt,
   linecolor=blue,
   innerbottommargin=10pt,
   % backgroundcolor=bg,font=\color{orange}\sffamily, fontcolor=white
]{definition}

\usepackage{empheq}
\usepackage[most]{tcolorbox}

\newtcbox{\mymath}[1][]{%
    nobeforeafter, math upper, tcbox raise base,
    enhanced, colframe=blue!30!black,
    colback=red!10, boxrule=1pt,
    #1}

\usepackage{unixode}


\DeclareMathOperator{\ord}{ord}
\DeclareMathOperator{\orb}{orb}
\DeclareMathOperator{\stab}{stab}
\DeclareMathOperator{\Stab}{stab}
\DeclareMathOperator{\ppcm}{ppcm}
\DeclareMathOperator{\conj}{Conj}
\DeclareMathOperator{\End}{End}
\DeclareMathOperator{\rot}{rot}
\DeclareMathOperator{\trs}{trace}
\DeclareMathOperator{\Ind}{Ind}
\DeclareMathOperator{\mat}{Mat}
\DeclareMathOperator{\id}{Id}
\DeclareMathOperator{\vect}{vect}
\DeclareMathOperator{\img}{img}
\DeclareMathOperator{\cov}{Cov}
\DeclareMathOperator{\dist}{dist}
\DeclareMathOperator{\irr}{Irr}
\DeclareMathOperator{\image}{Im}
\DeclareMathOperator{\pd}{\partial}
\DeclareMathOperator{\epi}{epi}
\DeclareMathOperator{\Argmin}{Argmin}
\DeclareMathOperator{\dom}{dom}
\DeclareMathOperator{\proj}{proj}
\DeclareMathOperator{\ctg}{ctg}
\DeclareMathOperator{\supp}{supp}
\DeclareMathOperator{\argmin}{argmin}
\DeclareMathOperator{\mult}{mult}
\DeclareMathOperator{\ch}{ch}
\DeclareMathOperator{\sh}{sh}
\DeclareMathOperator{\rang}{rang}
\DeclareMathOperator{\diam}{diam}
\DeclareMathOperator{\Epigraphe}{Epigraphe}




\usepackage{xcolor}
\everymath{\color{blue}}
%\everymath{\color[rgb]{0,1,1}}
%\pagecolor[rgb]{0,0,0.5}


\newcommand*{\pdtest}[3][]{\ensuremath{\frac{\partial^{#1} #2}{\partial #3}}}

\newcommand*{\deffunc}[6][]{\ensuremath{
\begin{array}{rcl}
#2 : #3 &\rightarrow& #4\\
#5 &\mapsto& #6
\end{array}
}}

\newcommand{\eqcolon}{\mathrel{\resizebox{\widthof{$\mathord{=}$}}{\height}{ $\!\!=\!\!\resizebox{1.2\width}{0.8\height}{\raisebox{0.23ex}{$\mathop{:}$}}\!\!$ }}}
\newcommand{\coloneq}{\mathrel{\resizebox{\widthof{$\mathord{=}$}}{\height}{ $\!\!\resizebox{1.2\width}{0.8\height}{\raisebox{0.23ex}{$\mathop{:}$}}\!\!=\!\!$ }}}
\newcommand{\eqcolonl}{\ensuremath{\mathrel{=\!\!\mathop{:}}}}
\newcommand{\coloneql}{\ensuremath{\mathrel{\mathop{:} \!\! =}}}
\newcommand{\vc}[1]{% inline column vector
  \left(\begin{smallmatrix}#1\end{smallmatrix}\right)%
}
\newcommand{\vr}[1]{% inline row vector
  \begin{smallmatrix}(\,#1\,)\end{smallmatrix}%
}
\makeatletter
\newcommand*{\defeq}{\ =\mathrel{\rlap{%
                     \raisebox{0.3ex}{$\m@th\cdot$}}%
                     \raisebox{-0.3ex}{$\m@th\cdot$}}%
                     }
\makeatother

\newcommand{\mathcircle}[1]{% inline row vector
 \overset{\circ}{#1}
}
\newcommand{\ulim}{% low limit
 \underline{\lim}
}
\newcommand{\ssi}{% iff
\iff
}
\newcommand{\ps}[2]{
\expval{#1 | #2}
}
\newcommand{\df}[1]{
\mqty{#1}
}
\newcommand{\n}[1]{
\norm{#1}
}
\newcommand{\sys}[1]{
\left\{\smqty{#1}\right.
}


\newcommand{\eqdef}{\ensuremath{\overset{\text{def}}=}}


\def\Circlearrowright{\ensuremath{%
  \rotatebox[origin=c]{230}{$\circlearrowright$}}}

\newcommand\ct[1]{\text{\rmfamily\upshape #1}}
\newcommand\question[1]{ {\color{red} ...!? \small #1}}
\newcommand\caz[1]{\left\{\begin{array} #1 \end{array}\right.}
\newcommand\const{\text{\rmfamily\upshape const}}
\newcommand\toP{ \overset{\pro}{\to}}
\newcommand\toPP{ \overset{\text{PP}}{\to}}
\newcommand{\oeq}{\mathrel{\text{\textcircled{$=$}}}}





\usepackage{xcolor}
% \usepackage[normalem]{ulem}
\usepackage{lipsum}
\makeatletter
% \newcommand\colorwave[1][blue]{\bgroup \markoverwith{\lower3.5\p@\hbox{\sixly \textcolor{#1}{\char58}}}\ULon}
%\font\sixly=lasy6 % does not re-load if already loaded, so no memory problem.

\newmdtheoremenv[
linewidth= 1pt,linecolor= blue,%
leftmargin=20,rightmargin=20,innertopmargin=0pt, innerrightmargin=40,%
tikzsetting = { draw=lightgray, line width = 0.3pt,dashed,%
dash pattern = on 15pt off 3pt},%
splittopskip=\topskip,skipbelow=\baselineskip,%
skipabove=\baselineskip,ntheorem,roundcorner=0pt,
% backgroundcolor=pagebg,font=\color{orange}\sffamily, fontcolor=white
]{examplebox}{Exemple}[section]



\newcommand\R{\mathbb{R}}
\newcommand\Z{\mathbb{Z}}
\newcommand\N{\mathbb{N}}
\newcommand\E{\mathbb{E}}
\newcommand\F{\mathcal{F}}
\newcommand\cH{\mathcal{H}}
\newcommand\V{\mathbb{V}}
\newcommand\dmo{ ^{-1} }
\newcommand\kapa{\kappa}
\newcommand\im{Im}
\newcommand\hs{\mathcal{H}}





\usepackage{soul}

\makeatletter
\newcommand*{\whiten}[1]{\llap{\textcolor{white}{{\the\SOUL@token}}\hspace{#1pt}}}
\DeclareRobustCommand*\myul{%
    \def\SOUL@everyspace{\underline{\space}\kern\z@}%
    \def\SOUL@everytoken{%
     \setbox0=\hbox{\the\SOUL@token}%
     \ifdim\dp0>\z@
        \raisebox{\dp0}{\underline{\phantom{\the\SOUL@token}}}%
        \whiten{1}\whiten{0}%
        \whiten{-1}\whiten{-2}%
        \llap{\the\SOUL@token}%
     \else
        \underline{\the\SOUL@token}%
     \fi}%
\SOUL@}
\makeatother

\newcommand*{\demp}{\fontfamily{lmtt}\selectfont}

\DeclareTextFontCommand{\textdemp}{\demp}

\begin{document}

\ifcomment
Multiline
comment
\fi
\ifcomment
\myul{Typesetting test}
% \color[rgb]{1,1,1}
$∑_i^n≠ 60º±∞π∆¬≈√j∫h≤≥µ$

$\CR \R\pro\ind\pro\gS\pro
\mqty[a&b\\c&d]$
$\pro\mathbb{P}$
$\dd{x}$

  \[
    \alpha(x)=\left\{
                \begin{array}{ll}
                  x\\
                  \frac{1}{1+e^{-kx}}\\
                  \frac{e^x-e^{-x}}{e^x+e^{-x}}
                \end{array}
              \right.
  \]

  $\expval{x}$
  
  $\chi_\rho(ghg\dmo)=\Tr(\rho_{ghg\dmo})=\Tr(\rho_g\circ\rho_h\circ\rho\dmo_g)=\Tr(\rho_h)\overset{\mbox{\scalebox{0.5}{$\Tr(AB)=\Tr(BA)$}}}{=}\chi_\rho(h)$
  	$\mathop{\oplus}_{\substack{x\in X}}$

$\mat(\rho_g)=(a_{ij}(g))_{\scriptsize \substack{1\leq i\leq d \\ 1\leq j\leq d}}$ et $\mat(\rho'_g)=(a'_{ij}(g))_{\scriptsize \substack{1\leq i'\leq d' \\ 1\leq j'\leq d'}}$



\[\int_a^b{\mathbb{R}^2}g(u, v)\dd{P_{XY}}(u, v)=\iint g(u,v) f_{XY}(u, v)\dd \lambda(u) \dd \lambda(v)\]
$$\lim_{x\to\infty} f(x)$$	
$$\iiiint_V \mu(t,u,v,w) \,dt\,du\,dv\,dw$$
$$\sum_{n=1}^{\infty} 2^{-n} = 1$$	
\begin{definition}
	Si $X$ et $Y$ sont 2 v.a. ou definit la \textsc{Covariance} entre $X$ et $Y$ comme
	$\cov(X,Y)\overset{\text{def}}{=}\E\left[(X-\E(X))(Y-\E(Y))\right]=\E(XY)-\E(X)\E(Y)$.
\end{definition}
\fi
\pagebreak

% \tableofcontents

% insert your code here
%\input{./algebra/main.tex}
%\input{./geometrie-differentielle/main.tex}
%\input{./probabilite/main.tex}
%\input{./analyse-fonctionnelle/main.tex}
% \input{./Analyse-convexe-et-dualite-en-optimisation/main.tex}
%\input{./tikz/main.tex}
%\input{./Theorie-du-distributions/main.tex}
%\input{./optimisation/mine.tex}
 \input{./modelisation/main.tex}

% yves.aubry@univ-tln.fr : algebra

\end{document}

%% !TEX encoding = UTF-8 Unicode
% !TEX TS-program = xelatex

\documentclass[french]{report}

%\usepackage[utf8]{inputenc}
%\usepackage[T1]{fontenc}
\usepackage{babel}


\newif\ifcomment
%\commenttrue # Show comments

\usepackage{physics}
\usepackage{amssymb}


\usepackage{amsthm}
% \usepackage{thmtools}
\usepackage{mathtools}
\usepackage{amsfonts}

\usepackage{color}

\usepackage{tikz}

\usepackage{geometry}
\geometry{a5paper, margin=0.1in, right=1cm}

\usepackage{dsfont}

\usepackage{graphicx}
\graphicspath{ {images/} }

\usepackage{faktor}

\usepackage{IEEEtrantools}
\usepackage{enumerate}   
\usepackage[PostScript=dvips]{"/Users/aware/Documents/Courses/diagrams"}


\newtheorem{theorem}{Théorème}[section]
\renewcommand{\thetheorem}{\arabic{theorem}}
\newtheorem{lemme}{Lemme}[section]
\renewcommand{\thelemme}{\arabic{lemme}}
\newtheorem{proposition}{Proposition}[section]
\renewcommand{\theproposition}{\arabic{proposition}}
\newtheorem{notations}{Notations}[section]
\newtheorem{problem}{Problème}[section]
\newtheorem{corollary}{Corollaire}[theorem]
\renewcommand{\thecorollary}{\arabic{corollary}}
\newtheorem{property}{Propriété}[section]
\newtheorem{objective}{Objectif}[section]

\theoremstyle{definition}
\newtheorem{definition}{Définition}[section]
\renewcommand{\thedefinition}{\arabic{definition}}
\newtheorem{exercise}{Exercice}[chapter]
\renewcommand{\theexercise}{\arabic{exercise}}
\newtheorem{example}{Exemple}[chapter]
\renewcommand{\theexample}{\arabic{example}}
\newtheorem*{solution}{Solution}
\newtheorem*{application}{Application}
\newtheorem*{notation}{Notation}
\newtheorem*{vocabulary}{Vocabulaire}
\newtheorem*{properties}{Propriétés}



\theoremstyle{remark}
\newtheorem*{remark}{Remarque}
\newtheorem*{rappel}{Rappel}


\usepackage{etoolbox}
\AtBeginEnvironment{exercise}{\small}
\AtBeginEnvironment{example}{\small}

\usepackage{cases}
\usepackage[red]{mypack}

\usepackage[framemethod=TikZ]{mdframed}

\definecolor{bg}{rgb}{0.4,0.25,0.95}
\definecolor{pagebg}{rgb}{0,0,0.5}
\surroundwithmdframed[
   topline=false,
   rightline=false,
   bottomline=false,
   leftmargin=\parindent,
   skipabove=8pt,
   skipbelow=8pt,
   linecolor=blue,
   innerbottommargin=10pt,
   % backgroundcolor=bg,font=\color{orange}\sffamily, fontcolor=white
]{definition}

\usepackage{empheq}
\usepackage[most]{tcolorbox}

\newtcbox{\mymath}[1][]{%
    nobeforeafter, math upper, tcbox raise base,
    enhanced, colframe=blue!30!black,
    colback=red!10, boxrule=1pt,
    #1}

\usepackage{unixode}


\DeclareMathOperator{\ord}{ord}
\DeclareMathOperator{\orb}{orb}
\DeclareMathOperator{\stab}{stab}
\DeclareMathOperator{\Stab}{stab}
\DeclareMathOperator{\ppcm}{ppcm}
\DeclareMathOperator{\conj}{Conj}
\DeclareMathOperator{\End}{End}
\DeclareMathOperator{\rot}{rot}
\DeclareMathOperator{\trs}{trace}
\DeclareMathOperator{\Ind}{Ind}
\DeclareMathOperator{\mat}{Mat}
\DeclareMathOperator{\id}{Id}
\DeclareMathOperator{\vect}{vect}
\DeclareMathOperator{\img}{img}
\DeclareMathOperator{\cov}{Cov}
\DeclareMathOperator{\dist}{dist}
\DeclareMathOperator{\irr}{Irr}
\DeclareMathOperator{\image}{Im}
\DeclareMathOperator{\pd}{\partial}
\DeclareMathOperator{\epi}{epi}
\DeclareMathOperator{\Argmin}{Argmin}
\DeclareMathOperator{\dom}{dom}
\DeclareMathOperator{\proj}{proj}
\DeclareMathOperator{\ctg}{ctg}
\DeclareMathOperator{\supp}{supp}
\DeclareMathOperator{\argmin}{argmin}
\DeclareMathOperator{\mult}{mult}
\DeclareMathOperator{\ch}{ch}
\DeclareMathOperator{\sh}{sh}
\DeclareMathOperator{\rang}{rang}
\DeclareMathOperator{\diam}{diam}
\DeclareMathOperator{\Epigraphe}{Epigraphe}




\usepackage{xcolor}
\everymath{\color{blue}}
%\everymath{\color[rgb]{0,1,1}}
%\pagecolor[rgb]{0,0,0.5}


\newcommand*{\pdtest}[3][]{\ensuremath{\frac{\partial^{#1} #2}{\partial #3}}}

\newcommand*{\deffunc}[6][]{\ensuremath{
\begin{array}{rcl}
#2 : #3 &\rightarrow& #4\\
#5 &\mapsto& #6
\end{array}
}}

\newcommand{\eqcolon}{\mathrel{\resizebox{\widthof{$\mathord{=}$}}{\height}{ $\!\!=\!\!\resizebox{1.2\width}{0.8\height}{\raisebox{0.23ex}{$\mathop{:}$}}\!\!$ }}}
\newcommand{\coloneq}{\mathrel{\resizebox{\widthof{$\mathord{=}$}}{\height}{ $\!\!\resizebox{1.2\width}{0.8\height}{\raisebox{0.23ex}{$\mathop{:}$}}\!\!=\!\!$ }}}
\newcommand{\eqcolonl}{\ensuremath{\mathrel{=\!\!\mathop{:}}}}
\newcommand{\coloneql}{\ensuremath{\mathrel{\mathop{:} \!\! =}}}
\newcommand{\vc}[1]{% inline column vector
  \left(\begin{smallmatrix}#1\end{smallmatrix}\right)%
}
\newcommand{\vr}[1]{% inline row vector
  \begin{smallmatrix}(\,#1\,)\end{smallmatrix}%
}
\makeatletter
\newcommand*{\defeq}{\ =\mathrel{\rlap{%
                     \raisebox{0.3ex}{$\m@th\cdot$}}%
                     \raisebox{-0.3ex}{$\m@th\cdot$}}%
                     }
\makeatother

\newcommand{\mathcircle}[1]{% inline row vector
 \overset{\circ}{#1}
}
\newcommand{\ulim}{% low limit
 \underline{\lim}
}
\newcommand{\ssi}{% iff
\iff
}
\newcommand{\ps}[2]{
\expval{#1 | #2}
}
\newcommand{\df}[1]{
\mqty{#1}
}
\newcommand{\n}[1]{
\norm{#1}
}
\newcommand{\sys}[1]{
\left\{\smqty{#1}\right.
}


\newcommand{\eqdef}{\ensuremath{\overset{\text{def}}=}}


\def\Circlearrowright{\ensuremath{%
  \rotatebox[origin=c]{230}{$\circlearrowright$}}}

\newcommand\ct[1]{\text{\rmfamily\upshape #1}}
\newcommand\question[1]{ {\color{red} ...!? \small #1}}
\newcommand\caz[1]{\left\{\begin{array} #1 \end{array}\right.}
\newcommand\const{\text{\rmfamily\upshape const}}
\newcommand\toP{ \overset{\pro}{\to}}
\newcommand\toPP{ \overset{\text{PP}}{\to}}
\newcommand{\oeq}{\mathrel{\text{\textcircled{$=$}}}}





\usepackage{xcolor}
% \usepackage[normalem]{ulem}
\usepackage{lipsum}
\makeatletter
% \newcommand\colorwave[1][blue]{\bgroup \markoverwith{\lower3.5\p@\hbox{\sixly \textcolor{#1}{\char58}}}\ULon}
%\font\sixly=lasy6 % does not re-load if already loaded, so no memory problem.

\newmdtheoremenv[
linewidth= 1pt,linecolor= blue,%
leftmargin=20,rightmargin=20,innertopmargin=0pt, innerrightmargin=40,%
tikzsetting = { draw=lightgray, line width = 0.3pt,dashed,%
dash pattern = on 15pt off 3pt},%
splittopskip=\topskip,skipbelow=\baselineskip,%
skipabove=\baselineskip,ntheorem,roundcorner=0pt,
% backgroundcolor=pagebg,font=\color{orange}\sffamily, fontcolor=white
]{examplebox}{Exemple}[section]



\newcommand\R{\mathbb{R}}
\newcommand\Z{\mathbb{Z}}
\newcommand\N{\mathbb{N}}
\newcommand\E{\mathbb{E}}
\newcommand\F{\mathcal{F}}
\newcommand\cH{\mathcal{H}}
\newcommand\V{\mathbb{V}}
\newcommand\dmo{ ^{-1} }
\newcommand\kapa{\kappa}
\newcommand\im{Im}
\newcommand\hs{\mathcal{H}}





\usepackage{soul}

\makeatletter
\newcommand*{\whiten}[1]{\llap{\textcolor{white}{{\the\SOUL@token}}\hspace{#1pt}}}
\DeclareRobustCommand*\myul{%
    \def\SOUL@everyspace{\underline{\space}\kern\z@}%
    \def\SOUL@everytoken{%
     \setbox0=\hbox{\the\SOUL@token}%
     \ifdim\dp0>\z@
        \raisebox{\dp0}{\underline{\phantom{\the\SOUL@token}}}%
        \whiten{1}\whiten{0}%
        \whiten{-1}\whiten{-2}%
        \llap{\the\SOUL@token}%
     \else
        \underline{\the\SOUL@token}%
     \fi}%
\SOUL@}
\makeatother

\newcommand*{\demp}{\fontfamily{lmtt}\selectfont}

\DeclareTextFontCommand{\textdemp}{\demp}

\begin{document}

\ifcomment
Multiline
comment
\fi
\ifcomment
\myul{Typesetting test}
% \color[rgb]{1,1,1}
$∑_i^n≠ 60º±∞π∆¬≈√j∫h≤≥µ$

$\CR \R\pro\ind\pro\gS\pro
\mqty[a&b\\c&d]$
$\pro\mathbb{P}$
$\dd{x}$

  \[
    \alpha(x)=\left\{
                \begin{array}{ll}
                  x\\
                  \frac{1}{1+e^{-kx}}\\
                  \frac{e^x-e^{-x}}{e^x+e^{-x}}
                \end{array}
              \right.
  \]

  $\expval{x}$
  
  $\chi_\rho(ghg\dmo)=\Tr(\rho_{ghg\dmo})=\Tr(\rho_g\circ\rho_h\circ\rho\dmo_g)=\Tr(\rho_h)\overset{\mbox{\scalebox{0.5}{$\Tr(AB)=\Tr(BA)$}}}{=}\chi_\rho(h)$
  	$\mathop{\oplus}_{\substack{x\in X}}$

$\mat(\rho_g)=(a_{ij}(g))_{\scriptsize \substack{1\leq i\leq d \\ 1\leq j\leq d}}$ et $\mat(\rho'_g)=(a'_{ij}(g))_{\scriptsize \substack{1\leq i'\leq d' \\ 1\leq j'\leq d'}}$



\[\int_a^b{\mathbb{R}^2}g(u, v)\dd{P_{XY}}(u, v)=\iint g(u,v) f_{XY}(u, v)\dd \lambda(u) \dd \lambda(v)\]
$$\lim_{x\to\infty} f(x)$$	
$$\iiiint_V \mu(t,u,v,w) \,dt\,du\,dv\,dw$$
$$\sum_{n=1}^{\infty} 2^{-n} = 1$$	
\begin{definition}
	Si $X$ et $Y$ sont 2 v.a. ou definit la \textsc{Covariance} entre $X$ et $Y$ comme
	$\cov(X,Y)\overset{\text{def}}{=}\E\left[(X-\E(X))(Y-\E(Y))\right]=\E(XY)-\E(X)\E(Y)$.
\end{definition}
\fi
\pagebreak

% \tableofcontents

% insert your code here
%\input{./algebra/main.tex}
%\input{./geometrie-differentielle/main.tex}
%\input{./probabilite/main.tex}
%\input{./analyse-fonctionnelle/main.tex}
% \input{./Analyse-convexe-et-dualite-en-optimisation/main.tex}
%\input{./tikz/main.tex}
%\input{./Theorie-du-distributions/main.tex}
%\input{./optimisation/mine.tex}
 \input{./modelisation/main.tex}

% yves.aubry@univ-tln.fr : algebra

\end{document}

% % !TEX encoding = UTF-8 Unicode
% !TEX TS-program = xelatex

\documentclass[french]{report}

%\usepackage[utf8]{inputenc}
%\usepackage[T1]{fontenc}
\usepackage{babel}


\newif\ifcomment
%\commenttrue # Show comments

\usepackage{physics}
\usepackage{amssymb}


\usepackage{amsthm}
% \usepackage{thmtools}
\usepackage{mathtools}
\usepackage{amsfonts}

\usepackage{color}

\usepackage{tikz}

\usepackage{geometry}
\geometry{a5paper, margin=0.1in, right=1cm}

\usepackage{dsfont}

\usepackage{graphicx}
\graphicspath{ {images/} }

\usepackage{faktor}

\usepackage{IEEEtrantools}
\usepackage{enumerate}   
\usepackage[PostScript=dvips]{"/Users/aware/Documents/Courses/diagrams"}


\newtheorem{theorem}{Théorème}[section]
\renewcommand{\thetheorem}{\arabic{theorem}}
\newtheorem{lemme}{Lemme}[section]
\renewcommand{\thelemme}{\arabic{lemme}}
\newtheorem{proposition}{Proposition}[section]
\renewcommand{\theproposition}{\arabic{proposition}}
\newtheorem{notations}{Notations}[section]
\newtheorem{problem}{Problème}[section]
\newtheorem{corollary}{Corollaire}[theorem]
\renewcommand{\thecorollary}{\arabic{corollary}}
\newtheorem{property}{Propriété}[section]
\newtheorem{objective}{Objectif}[section]

\theoremstyle{definition}
\newtheorem{definition}{Définition}[section]
\renewcommand{\thedefinition}{\arabic{definition}}
\newtheorem{exercise}{Exercice}[chapter]
\renewcommand{\theexercise}{\arabic{exercise}}
\newtheorem{example}{Exemple}[chapter]
\renewcommand{\theexample}{\arabic{example}}
\newtheorem*{solution}{Solution}
\newtheorem*{application}{Application}
\newtheorem*{notation}{Notation}
\newtheorem*{vocabulary}{Vocabulaire}
\newtheorem*{properties}{Propriétés}



\theoremstyle{remark}
\newtheorem*{remark}{Remarque}
\newtheorem*{rappel}{Rappel}


\usepackage{etoolbox}
\AtBeginEnvironment{exercise}{\small}
\AtBeginEnvironment{example}{\small}

\usepackage{cases}
\usepackage[red]{mypack}

\usepackage[framemethod=TikZ]{mdframed}

\definecolor{bg}{rgb}{0.4,0.25,0.95}
\definecolor{pagebg}{rgb}{0,0,0.5}
\surroundwithmdframed[
   topline=false,
   rightline=false,
   bottomline=false,
   leftmargin=\parindent,
   skipabove=8pt,
   skipbelow=8pt,
   linecolor=blue,
   innerbottommargin=10pt,
   % backgroundcolor=bg,font=\color{orange}\sffamily, fontcolor=white
]{definition}

\usepackage{empheq}
\usepackage[most]{tcolorbox}

\newtcbox{\mymath}[1][]{%
    nobeforeafter, math upper, tcbox raise base,
    enhanced, colframe=blue!30!black,
    colback=red!10, boxrule=1pt,
    #1}

\usepackage{unixode}


\DeclareMathOperator{\ord}{ord}
\DeclareMathOperator{\orb}{orb}
\DeclareMathOperator{\stab}{stab}
\DeclareMathOperator{\Stab}{stab}
\DeclareMathOperator{\ppcm}{ppcm}
\DeclareMathOperator{\conj}{Conj}
\DeclareMathOperator{\End}{End}
\DeclareMathOperator{\rot}{rot}
\DeclareMathOperator{\trs}{trace}
\DeclareMathOperator{\Ind}{Ind}
\DeclareMathOperator{\mat}{Mat}
\DeclareMathOperator{\id}{Id}
\DeclareMathOperator{\vect}{vect}
\DeclareMathOperator{\img}{img}
\DeclareMathOperator{\cov}{Cov}
\DeclareMathOperator{\dist}{dist}
\DeclareMathOperator{\irr}{Irr}
\DeclareMathOperator{\image}{Im}
\DeclareMathOperator{\pd}{\partial}
\DeclareMathOperator{\epi}{epi}
\DeclareMathOperator{\Argmin}{Argmin}
\DeclareMathOperator{\dom}{dom}
\DeclareMathOperator{\proj}{proj}
\DeclareMathOperator{\ctg}{ctg}
\DeclareMathOperator{\supp}{supp}
\DeclareMathOperator{\argmin}{argmin}
\DeclareMathOperator{\mult}{mult}
\DeclareMathOperator{\ch}{ch}
\DeclareMathOperator{\sh}{sh}
\DeclareMathOperator{\rang}{rang}
\DeclareMathOperator{\diam}{diam}
\DeclareMathOperator{\Epigraphe}{Epigraphe}




\usepackage{xcolor}
\everymath{\color{blue}}
%\everymath{\color[rgb]{0,1,1}}
%\pagecolor[rgb]{0,0,0.5}


\newcommand*{\pdtest}[3][]{\ensuremath{\frac{\partial^{#1} #2}{\partial #3}}}

\newcommand*{\deffunc}[6][]{\ensuremath{
\begin{array}{rcl}
#2 : #3 &\rightarrow& #4\\
#5 &\mapsto& #6
\end{array}
}}

\newcommand{\eqcolon}{\mathrel{\resizebox{\widthof{$\mathord{=}$}}{\height}{ $\!\!=\!\!\resizebox{1.2\width}{0.8\height}{\raisebox{0.23ex}{$\mathop{:}$}}\!\!$ }}}
\newcommand{\coloneq}{\mathrel{\resizebox{\widthof{$\mathord{=}$}}{\height}{ $\!\!\resizebox{1.2\width}{0.8\height}{\raisebox{0.23ex}{$\mathop{:}$}}\!\!=\!\!$ }}}
\newcommand{\eqcolonl}{\ensuremath{\mathrel{=\!\!\mathop{:}}}}
\newcommand{\coloneql}{\ensuremath{\mathrel{\mathop{:} \!\! =}}}
\newcommand{\vc}[1]{% inline column vector
  \left(\begin{smallmatrix}#1\end{smallmatrix}\right)%
}
\newcommand{\vr}[1]{% inline row vector
  \begin{smallmatrix}(\,#1\,)\end{smallmatrix}%
}
\makeatletter
\newcommand*{\defeq}{\ =\mathrel{\rlap{%
                     \raisebox{0.3ex}{$\m@th\cdot$}}%
                     \raisebox{-0.3ex}{$\m@th\cdot$}}%
                     }
\makeatother

\newcommand{\mathcircle}[1]{% inline row vector
 \overset{\circ}{#1}
}
\newcommand{\ulim}{% low limit
 \underline{\lim}
}
\newcommand{\ssi}{% iff
\iff
}
\newcommand{\ps}[2]{
\expval{#1 | #2}
}
\newcommand{\df}[1]{
\mqty{#1}
}
\newcommand{\n}[1]{
\norm{#1}
}
\newcommand{\sys}[1]{
\left\{\smqty{#1}\right.
}


\newcommand{\eqdef}{\ensuremath{\overset{\text{def}}=}}


\def\Circlearrowright{\ensuremath{%
  \rotatebox[origin=c]{230}{$\circlearrowright$}}}

\newcommand\ct[1]{\text{\rmfamily\upshape #1}}
\newcommand\question[1]{ {\color{red} ...!? \small #1}}
\newcommand\caz[1]{\left\{\begin{array} #1 \end{array}\right.}
\newcommand\const{\text{\rmfamily\upshape const}}
\newcommand\toP{ \overset{\pro}{\to}}
\newcommand\toPP{ \overset{\text{PP}}{\to}}
\newcommand{\oeq}{\mathrel{\text{\textcircled{$=$}}}}





\usepackage{xcolor}
% \usepackage[normalem]{ulem}
\usepackage{lipsum}
\makeatletter
% \newcommand\colorwave[1][blue]{\bgroup \markoverwith{\lower3.5\p@\hbox{\sixly \textcolor{#1}{\char58}}}\ULon}
%\font\sixly=lasy6 % does not re-load if already loaded, so no memory problem.

\newmdtheoremenv[
linewidth= 1pt,linecolor= blue,%
leftmargin=20,rightmargin=20,innertopmargin=0pt, innerrightmargin=40,%
tikzsetting = { draw=lightgray, line width = 0.3pt,dashed,%
dash pattern = on 15pt off 3pt},%
splittopskip=\topskip,skipbelow=\baselineskip,%
skipabove=\baselineskip,ntheorem,roundcorner=0pt,
% backgroundcolor=pagebg,font=\color{orange}\sffamily, fontcolor=white
]{examplebox}{Exemple}[section]



\newcommand\R{\mathbb{R}}
\newcommand\Z{\mathbb{Z}}
\newcommand\N{\mathbb{N}}
\newcommand\E{\mathbb{E}}
\newcommand\F{\mathcal{F}}
\newcommand\cH{\mathcal{H}}
\newcommand\V{\mathbb{V}}
\newcommand\dmo{ ^{-1} }
\newcommand\kapa{\kappa}
\newcommand\im{Im}
\newcommand\hs{\mathcal{H}}





\usepackage{soul}

\makeatletter
\newcommand*{\whiten}[1]{\llap{\textcolor{white}{{\the\SOUL@token}}\hspace{#1pt}}}
\DeclareRobustCommand*\myul{%
    \def\SOUL@everyspace{\underline{\space}\kern\z@}%
    \def\SOUL@everytoken{%
     \setbox0=\hbox{\the\SOUL@token}%
     \ifdim\dp0>\z@
        \raisebox{\dp0}{\underline{\phantom{\the\SOUL@token}}}%
        \whiten{1}\whiten{0}%
        \whiten{-1}\whiten{-2}%
        \llap{\the\SOUL@token}%
     \else
        \underline{\the\SOUL@token}%
     \fi}%
\SOUL@}
\makeatother

\newcommand*{\demp}{\fontfamily{lmtt}\selectfont}

\DeclareTextFontCommand{\textdemp}{\demp}

\begin{document}

\ifcomment
Multiline
comment
\fi
\ifcomment
\myul{Typesetting test}
% \color[rgb]{1,1,1}
$∑_i^n≠ 60º±∞π∆¬≈√j∫h≤≥µ$

$\CR \R\pro\ind\pro\gS\pro
\mqty[a&b\\c&d]$
$\pro\mathbb{P}$
$\dd{x}$

  \[
    \alpha(x)=\left\{
                \begin{array}{ll}
                  x\\
                  \frac{1}{1+e^{-kx}}\\
                  \frac{e^x-e^{-x}}{e^x+e^{-x}}
                \end{array}
              \right.
  \]

  $\expval{x}$
  
  $\chi_\rho(ghg\dmo)=\Tr(\rho_{ghg\dmo})=\Tr(\rho_g\circ\rho_h\circ\rho\dmo_g)=\Tr(\rho_h)\overset{\mbox{\scalebox{0.5}{$\Tr(AB)=\Tr(BA)$}}}{=}\chi_\rho(h)$
  	$\mathop{\oplus}_{\substack{x\in X}}$

$\mat(\rho_g)=(a_{ij}(g))_{\scriptsize \substack{1\leq i\leq d \\ 1\leq j\leq d}}$ et $\mat(\rho'_g)=(a'_{ij}(g))_{\scriptsize \substack{1\leq i'\leq d' \\ 1\leq j'\leq d'}}$



\[\int_a^b{\mathbb{R}^2}g(u, v)\dd{P_{XY}}(u, v)=\iint g(u,v) f_{XY}(u, v)\dd \lambda(u) \dd \lambda(v)\]
$$\lim_{x\to\infty} f(x)$$	
$$\iiiint_V \mu(t,u,v,w) \,dt\,du\,dv\,dw$$
$$\sum_{n=1}^{\infty} 2^{-n} = 1$$	
\begin{definition}
	Si $X$ et $Y$ sont 2 v.a. ou definit la \textsc{Covariance} entre $X$ et $Y$ comme
	$\cov(X,Y)\overset{\text{def}}{=}\E\left[(X-\E(X))(Y-\E(Y))\right]=\E(XY)-\E(X)\E(Y)$.
\end{definition}
\fi
\pagebreak

% \tableofcontents

% insert your code here
%\input{./algebra/main.tex}
%\input{./geometrie-differentielle/main.tex}
%\input{./probabilite/main.tex}
%\input{./analyse-fonctionnelle/main.tex}
% \input{./Analyse-convexe-et-dualite-en-optimisation/main.tex}
%\input{./tikz/main.tex}
%\input{./Theorie-du-distributions/main.tex}
%\input{./optimisation/mine.tex}
 \input{./modelisation/main.tex}

% yves.aubry@univ-tln.fr : algebra

\end{document}

%% !TEX encoding = UTF-8 Unicode
% !TEX TS-program = xelatex

\documentclass[french]{report}

%\usepackage[utf8]{inputenc}
%\usepackage[T1]{fontenc}
\usepackage{babel}


\newif\ifcomment
%\commenttrue # Show comments

\usepackage{physics}
\usepackage{amssymb}


\usepackage{amsthm}
% \usepackage{thmtools}
\usepackage{mathtools}
\usepackage{amsfonts}

\usepackage{color}

\usepackage{tikz}

\usepackage{geometry}
\geometry{a5paper, margin=0.1in, right=1cm}

\usepackage{dsfont}

\usepackage{graphicx}
\graphicspath{ {images/} }

\usepackage{faktor}

\usepackage{IEEEtrantools}
\usepackage{enumerate}   
\usepackage[PostScript=dvips]{"/Users/aware/Documents/Courses/diagrams"}


\newtheorem{theorem}{Théorème}[section]
\renewcommand{\thetheorem}{\arabic{theorem}}
\newtheorem{lemme}{Lemme}[section]
\renewcommand{\thelemme}{\arabic{lemme}}
\newtheorem{proposition}{Proposition}[section]
\renewcommand{\theproposition}{\arabic{proposition}}
\newtheorem{notations}{Notations}[section]
\newtheorem{problem}{Problème}[section]
\newtheorem{corollary}{Corollaire}[theorem]
\renewcommand{\thecorollary}{\arabic{corollary}}
\newtheorem{property}{Propriété}[section]
\newtheorem{objective}{Objectif}[section]

\theoremstyle{definition}
\newtheorem{definition}{Définition}[section]
\renewcommand{\thedefinition}{\arabic{definition}}
\newtheorem{exercise}{Exercice}[chapter]
\renewcommand{\theexercise}{\arabic{exercise}}
\newtheorem{example}{Exemple}[chapter]
\renewcommand{\theexample}{\arabic{example}}
\newtheorem*{solution}{Solution}
\newtheorem*{application}{Application}
\newtheorem*{notation}{Notation}
\newtheorem*{vocabulary}{Vocabulaire}
\newtheorem*{properties}{Propriétés}



\theoremstyle{remark}
\newtheorem*{remark}{Remarque}
\newtheorem*{rappel}{Rappel}


\usepackage{etoolbox}
\AtBeginEnvironment{exercise}{\small}
\AtBeginEnvironment{example}{\small}

\usepackage{cases}
\usepackage[red]{mypack}

\usepackage[framemethod=TikZ]{mdframed}

\definecolor{bg}{rgb}{0.4,0.25,0.95}
\definecolor{pagebg}{rgb}{0,0,0.5}
\surroundwithmdframed[
   topline=false,
   rightline=false,
   bottomline=false,
   leftmargin=\parindent,
   skipabove=8pt,
   skipbelow=8pt,
   linecolor=blue,
   innerbottommargin=10pt,
   % backgroundcolor=bg,font=\color{orange}\sffamily, fontcolor=white
]{definition}

\usepackage{empheq}
\usepackage[most]{tcolorbox}

\newtcbox{\mymath}[1][]{%
    nobeforeafter, math upper, tcbox raise base,
    enhanced, colframe=blue!30!black,
    colback=red!10, boxrule=1pt,
    #1}

\usepackage{unixode}


\DeclareMathOperator{\ord}{ord}
\DeclareMathOperator{\orb}{orb}
\DeclareMathOperator{\stab}{stab}
\DeclareMathOperator{\Stab}{stab}
\DeclareMathOperator{\ppcm}{ppcm}
\DeclareMathOperator{\conj}{Conj}
\DeclareMathOperator{\End}{End}
\DeclareMathOperator{\rot}{rot}
\DeclareMathOperator{\trs}{trace}
\DeclareMathOperator{\Ind}{Ind}
\DeclareMathOperator{\mat}{Mat}
\DeclareMathOperator{\id}{Id}
\DeclareMathOperator{\vect}{vect}
\DeclareMathOperator{\img}{img}
\DeclareMathOperator{\cov}{Cov}
\DeclareMathOperator{\dist}{dist}
\DeclareMathOperator{\irr}{Irr}
\DeclareMathOperator{\image}{Im}
\DeclareMathOperator{\pd}{\partial}
\DeclareMathOperator{\epi}{epi}
\DeclareMathOperator{\Argmin}{Argmin}
\DeclareMathOperator{\dom}{dom}
\DeclareMathOperator{\proj}{proj}
\DeclareMathOperator{\ctg}{ctg}
\DeclareMathOperator{\supp}{supp}
\DeclareMathOperator{\argmin}{argmin}
\DeclareMathOperator{\mult}{mult}
\DeclareMathOperator{\ch}{ch}
\DeclareMathOperator{\sh}{sh}
\DeclareMathOperator{\rang}{rang}
\DeclareMathOperator{\diam}{diam}
\DeclareMathOperator{\Epigraphe}{Epigraphe}




\usepackage{xcolor}
\everymath{\color{blue}}
%\everymath{\color[rgb]{0,1,1}}
%\pagecolor[rgb]{0,0,0.5}


\newcommand*{\pdtest}[3][]{\ensuremath{\frac{\partial^{#1} #2}{\partial #3}}}

\newcommand*{\deffunc}[6][]{\ensuremath{
\begin{array}{rcl}
#2 : #3 &\rightarrow& #4\\
#5 &\mapsto& #6
\end{array}
}}

\newcommand{\eqcolon}{\mathrel{\resizebox{\widthof{$\mathord{=}$}}{\height}{ $\!\!=\!\!\resizebox{1.2\width}{0.8\height}{\raisebox{0.23ex}{$\mathop{:}$}}\!\!$ }}}
\newcommand{\coloneq}{\mathrel{\resizebox{\widthof{$\mathord{=}$}}{\height}{ $\!\!\resizebox{1.2\width}{0.8\height}{\raisebox{0.23ex}{$\mathop{:}$}}\!\!=\!\!$ }}}
\newcommand{\eqcolonl}{\ensuremath{\mathrel{=\!\!\mathop{:}}}}
\newcommand{\coloneql}{\ensuremath{\mathrel{\mathop{:} \!\! =}}}
\newcommand{\vc}[1]{% inline column vector
  \left(\begin{smallmatrix}#1\end{smallmatrix}\right)%
}
\newcommand{\vr}[1]{% inline row vector
  \begin{smallmatrix}(\,#1\,)\end{smallmatrix}%
}
\makeatletter
\newcommand*{\defeq}{\ =\mathrel{\rlap{%
                     \raisebox{0.3ex}{$\m@th\cdot$}}%
                     \raisebox{-0.3ex}{$\m@th\cdot$}}%
                     }
\makeatother

\newcommand{\mathcircle}[1]{% inline row vector
 \overset{\circ}{#1}
}
\newcommand{\ulim}{% low limit
 \underline{\lim}
}
\newcommand{\ssi}{% iff
\iff
}
\newcommand{\ps}[2]{
\expval{#1 | #2}
}
\newcommand{\df}[1]{
\mqty{#1}
}
\newcommand{\n}[1]{
\norm{#1}
}
\newcommand{\sys}[1]{
\left\{\smqty{#1}\right.
}


\newcommand{\eqdef}{\ensuremath{\overset{\text{def}}=}}


\def\Circlearrowright{\ensuremath{%
  \rotatebox[origin=c]{230}{$\circlearrowright$}}}

\newcommand\ct[1]{\text{\rmfamily\upshape #1}}
\newcommand\question[1]{ {\color{red} ...!? \small #1}}
\newcommand\caz[1]{\left\{\begin{array} #1 \end{array}\right.}
\newcommand\const{\text{\rmfamily\upshape const}}
\newcommand\toP{ \overset{\pro}{\to}}
\newcommand\toPP{ \overset{\text{PP}}{\to}}
\newcommand{\oeq}{\mathrel{\text{\textcircled{$=$}}}}





\usepackage{xcolor}
% \usepackage[normalem]{ulem}
\usepackage{lipsum}
\makeatletter
% \newcommand\colorwave[1][blue]{\bgroup \markoverwith{\lower3.5\p@\hbox{\sixly \textcolor{#1}{\char58}}}\ULon}
%\font\sixly=lasy6 % does not re-load if already loaded, so no memory problem.

\newmdtheoremenv[
linewidth= 1pt,linecolor= blue,%
leftmargin=20,rightmargin=20,innertopmargin=0pt, innerrightmargin=40,%
tikzsetting = { draw=lightgray, line width = 0.3pt,dashed,%
dash pattern = on 15pt off 3pt},%
splittopskip=\topskip,skipbelow=\baselineskip,%
skipabove=\baselineskip,ntheorem,roundcorner=0pt,
% backgroundcolor=pagebg,font=\color{orange}\sffamily, fontcolor=white
]{examplebox}{Exemple}[section]



\newcommand\R{\mathbb{R}}
\newcommand\Z{\mathbb{Z}}
\newcommand\N{\mathbb{N}}
\newcommand\E{\mathbb{E}}
\newcommand\F{\mathcal{F}}
\newcommand\cH{\mathcal{H}}
\newcommand\V{\mathbb{V}}
\newcommand\dmo{ ^{-1} }
\newcommand\kapa{\kappa}
\newcommand\im{Im}
\newcommand\hs{\mathcal{H}}





\usepackage{soul}

\makeatletter
\newcommand*{\whiten}[1]{\llap{\textcolor{white}{{\the\SOUL@token}}\hspace{#1pt}}}
\DeclareRobustCommand*\myul{%
    \def\SOUL@everyspace{\underline{\space}\kern\z@}%
    \def\SOUL@everytoken{%
     \setbox0=\hbox{\the\SOUL@token}%
     \ifdim\dp0>\z@
        \raisebox{\dp0}{\underline{\phantom{\the\SOUL@token}}}%
        \whiten{1}\whiten{0}%
        \whiten{-1}\whiten{-2}%
        \llap{\the\SOUL@token}%
     \else
        \underline{\the\SOUL@token}%
     \fi}%
\SOUL@}
\makeatother

\newcommand*{\demp}{\fontfamily{lmtt}\selectfont}

\DeclareTextFontCommand{\textdemp}{\demp}

\begin{document}

\ifcomment
Multiline
comment
\fi
\ifcomment
\myul{Typesetting test}
% \color[rgb]{1,1,1}
$∑_i^n≠ 60º±∞π∆¬≈√j∫h≤≥µ$

$\CR \R\pro\ind\pro\gS\pro
\mqty[a&b\\c&d]$
$\pro\mathbb{P}$
$\dd{x}$

  \[
    \alpha(x)=\left\{
                \begin{array}{ll}
                  x\\
                  \frac{1}{1+e^{-kx}}\\
                  \frac{e^x-e^{-x}}{e^x+e^{-x}}
                \end{array}
              \right.
  \]

  $\expval{x}$
  
  $\chi_\rho(ghg\dmo)=\Tr(\rho_{ghg\dmo})=\Tr(\rho_g\circ\rho_h\circ\rho\dmo_g)=\Tr(\rho_h)\overset{\mbox{\scalebox{0.5}{$\Tr(AB)=\Tr(BA)$}}}{=}\chi_\rho(h)$
  	$\mathop{\oplus}_{\substack{x\in X}}$

$\mat(\rho_g)=(a_{ij}(g))_{\scriptsize \substack{1\leq i\leq d \\ 1\leq j\leq d}}$ et $\mat(\rho'_g)=(a'_{ij}(g))_{\scriptsize \substack{1\leq i'\leq d' \\ 1\leq j'\leq d'}}$



\[\int_a^b{\mathbb{R}^2}g(u, v)\dd{P_{XY}}(u, v)=\iint g(u,v) f_{XY}(u, v)\dd \lambda(u) \dd \lambda(v)\]
$$\lim_{x\to\infty} f(x)$$	
$$\iiiint_V \mu(t,u,v,w) \,dt\,du\,dv\,dw$$
$$\sum_{n=1}^{\infty} 2^{-n} = 1$$	
\begin{definition}
	Si $X$ et $Y$ sont 2 v.a. ou definit la \textsc{Covariance} entre $X$ et $Y$ comme
	$\cov(X,Y)\overset{\text{def}}{=}\E\left[(X-\E(X))(Y-\E(Y))\right]=\E(XY)-\E(X)\E(Y)$.
\end{definition}
\fi
\pagebreak

% \tableofcontents

% insert your code here
%\input{./algebra/main.tex}
%\input{./geometrie-differentielle/main.tex}
%\input{./probabilite/main.tex}
%\input{./analyse-fonctionnelle/main.tex}
% \input{./Analyse-convexe-et-dualite-en-optimisation/main.tex}
%\input{./tikz/main.tex}
%\input{./Theorie-du-distributions/main.tex}
%\input{./optimisation/mine.tex}
 \input{./modelisation/main.tex}

% yves.aubry@univ-tln.fr : algebra

\end{document}

%% !TEX encoding = UTF-8 Unicode
% !TEX TS-program = xelatex

\documentclass[french]{report}

%\usepackage[utf8]{inputenc}
%\usepackage[T1]{fontenc}
\usepackage{babel}


\newif\ifcomment
%\commenttrue # Show comments

\usepackage{physics}
\usepackage{amssymb}


\usepackage{amsthm}
% \usepackage{thmtools}
\usepackage{mathtools}
\usepackage{amsfonts}

\usepackage{color}

\usepackage{tikz}

\usepackage{geometry}
\geometry{a5paper, margin=0.1in, right=1cm}

\usepackage{dsfont}

\usepackage{graphicx}
\graphicspath{ {images/} }

\usepackage{faktor}

\usepackage{IEEEtrantools}
\usepackage{enumerate}   
\usepackage[PostScript=dvips]{"/Users/aware/Documents/Courses/diagrams"}


\newtheorem{theorem}{Théorème}[section]
\renewcommand{\thetheorem}{\arabic{theorem}}
\newtheorem{lemme}{Lemme}[section]
\renewcommand{\thelemme}{\arabic{lemme}}
\newtheorem{proposition}{Proposition}[section]
\renewcommand{\theproposition}{\arabic{proposition}}
\newtheorem{notations}{Notations}[section]
\newtheorem{problem}{Problème}[section]
\newtheorem{corollary}{Corollaire}[theorem]
\renewcommand{\thecorollary}{\arabic{corollary}}
\newtheorem{property}{Propriété}[section]
\newtheorem{objective}{Objectif}[section]

\theoremstyle{definition}
\newtheorem{definition}{Définition}[section]
\renewcommand{\thedefinition}{\arabic{definition}}
\newtheorem{exercise}{Exercice}[chapter]
\renewcommand{\theexercise}{\arabic{exercise}}
\newtheorem{example}{Exemple}[chapter]
\renewcommand{\theexample}{\arabic{example}}
\newtheorem*{solution}{Solution}
\newtheorem*{application}{Application}
\newtheorem*{notation}{Notation}
\newtheorem*{vocabulary}{Vocabulaire}
\newtheorem*{properties}{Propriétés}



\theoremstyle{remark}
\newtheorem*{remark}{Remarque}
\newtheorem*{rappel}{Rappel}


\usepackage{etoolbox}
\AtBeginEnvironment{exercise}{\small}
\AtBeginEnvironment{example}{\small}

\usepackage{cases}
\usepackage[red]{mypack}

\usepackage[framemethod=TikZ]{mdframed}

\definecolor{bg}{rgb}{0.4,0.25,0.95}
\definecolor{pagebg}{rgb}{0,0,0.5}
\surroundwithmdframed[
   topline=false,
   rightline=false,
   bottomline=false,
   leftmargin=\parindent,
   skipabove=8pt,
   skipbelow=8pt,
   linecolor=blue,
   innerbottommargin=10pt,
   % backgroundcolor=bg,font=\color{orange}\sffamily, fontcolor=white
]{definition}

\usepackage{empheq}
\usepackage[most]{tcolorbox}

\newtcbox{\mymath}[1][]{%
    nobeforeafter, math upper, tcbox raise base,
    enhanced, colframe=blue!30!black,
    colback=red!10, boxrule=1pt,
    #1}

\usepackage{unixode}


\DeclareMathOperator{\ord}{ord}
\DeclareMathOperator{\orb}{orb}
\DeclareMathOperator{\stab}{stab}
\DeclareMathOperator{\Stab}{stab}
\DeclareMathOperator{\ppcm}{ppcm}
\DeclareMathOperator{\conj}{Conj}
\DeclareMathOperator{\End}{End}
\DeclareMathOperator{\rot}{rot}
\DeclareMathOperator{\trs}{trace}
\DeclareMathOperator{\Ind}{Ind}
\DeclareMathOperator{\mat}{Mat}
\DeclareMathOperator{\id}{Id}
\DeclareMathOperator{\vect}{vect}
\DeclareMathOperator{\img}{img}
\DeclareMathOperator{\cov}{Cov}
\DeclareMathOperator{\dist}{dist}
\DeclareMathOperator{\irr}{Irr}
\DeclareMathOperator{\image}{Im}
\DeclareMathOperator{\pd}{\partial}
\DeclareMathOperator{\epi}{epi}
\DeclareMathOperator{\Argmin}{Argmin}
\DeclareMathOperator{\dom}{dom}
\DeclareMathOperator{\proj}{proj}
\DeclareMathOperator{\ctg}{ctg}
\DeclareMathOperator{\supp}{supp}
\DeclareMathOperator{\argmin}{argmin}
\DeclareMathOperator{\mult}{mult}
\DeclareMathOperator{\ch}{ch}
\DeclareMathOperator{\sh}{sh}
\DeclareMathOperator{\rang}{rang}
\DeclareMathOperator{\diam}{diam}
\DeclareMathOperator{\Epigraphe}{Epigraphe}




\usepackage{xcolor}
\everymath{\color{blue}}
%\everymath{\color[rgb]{0,1,1}}
%\pagecolor[rgb]{0,0,0.5}


\newcommand*{\pdtest}[3][]{\ensuremath{\frac{\partial^{#1} #2}{\partial #3}}}

\newcommand*{\deffunc}[6][]{\ensuremath{
\begin{array}{rcl}
#2 : #3 &\rightarrow& #4\\
#5 &\mapsto& #6
\end{array}
}}

\newcommand{\eqcolon}{\mathrel{\resizebox{\widthof{$\mathord{=}$}}{\height}{ $\!\!=\!\!\resizebox{1.2\width}{0.8\height}{\raisebox{0.23ex}{$\mathop{:}$}}\!\!$ }}}
\newcommand{\coloneq}{\mathrel{\resizebox{\widthof{$\mathord{=}$}}{\height}{ $\!\!\resizebox{1.2\width}{0.8\height}{\raisebox{0.23ex}{$\mathop{:}$}}\!\!=\!\!$ }}}
\newcommand{\eqcolonl}{\ensuremath{\mathrel{=\!\!\mathop{:}}}}
\newcommand{\coloneql}{\ensuremath{\mathrel{\mathop{:} \!\! =}}}
\newcommand{\vc}[1]{% inline column vector
  \left(\begin{smallmatrix}#1\end{smallmatrix}\right)%
}
\newcommand{\vr}[1]{% inline row vector
  \begin{smallmatrix}(\,#1\,)\end{smallmatrix}%
}
\makeatletter
\newcommand*{\defeq}{\ =\mathrel{\rlap{%
                     \raisebox{0.3ex}{$\m@th\cdot$}}%
                     \raisebox{-0.3ex}{$\m@th\cdot$}}%
                     }
\makeatother

\newcommand{\mathcircle}[1]{% inline row vector
 \overset{\circ}{#1}
}
\newcommand{\ulim}{% low limit
 \underline{\lim}
}
\newcommand{\ssi}{% iff
\iff
}
\newcommand{\ps}[2]{
\expval{#1 | #2}
}
\newcommand{\df}[1]{
\mqty{#1}
}
\newcommand{\n}[1]{
\norm{#1}
}
\newcommand{\sys}[1]{
\left\{\smqty{#1}\right.
}


\newcommand{\eqdef}{\ensuremath{\overset{\text{def}}=}}


\def\Circlearrowright{\ensuremath{%
  \rotatebox[origin=c]{230}{$\circlearrowright$}}}

\newcommand\ct[1]{\text{\rmfamily\upshape #1}}
\newcommand\question[1]{ {\color{red} ...!? \small #1}}
\newcommand\caz[1]{\left\{\begin{array} #1 \end{array}\right.}
\newcommand\const{\text{\rmfamily\upshape const}}
\newcommand\toP{ \overset{\pro}{\to}}
\newcommand\toPP{ \overset{\text{PP}}{\to}}
\newcommand{\oeq}{\mathrel{\text{\textcircled{$=$}}}}





\usepackage{xcolor}
% \usepackage[normalem]{ulem}
\usepackage{lipsum}
\makeatletter
% \newcommand\colorwave[1][blue]{\bgroup \markoverwith{\lower3.5\p@\hbox{\sixly \textcolor{#1}{\char58}}}\ULon}
%\font\sixly=lasy6 % does not re-load if already loaded, so no memory problem.

\newmdtheoremenv[
linewidth= 1pt,linecolor= blue,%
leftmargin=20,rightmargin=20,innertopmargin=0pt, innerrightmargin=40,%
tikzsetting = { draw=lightgray, line width = 0.3pt,dashed,%
dash pattern = on 15pt off 3pt},%
splittopskip=\topskip,skipbelow=\baselineskip,%
skipabove=\baselineskip,ntheorem,roundcorner=0pt,
% backgroundcolor=pagebg,font=\color{orange}\sffamily, fontcolor=white
]{examplebox}{Exemple}[section]



\newcommand\R{\mathbb{R}}
\newcommand\Z{\mathbb{Z}}
\newcommand\N{\mathbb{N}}
\newcommand\E{\mathbb{E}}
\newcommand\F{\mathcal{F}}
\newcommand\cH{\mathcal{H}}
\newcommand\V{\mathbb{V}}
\newcommand\dmo{ ^{-1} }
\newcommand\kapa{\kappa}
\newcommand\im{Im}
\newcommand\hs{\mathcal{H}}





\usepackage{soul}

\makeatletter
\newcommand*{\whiten}[1]{\llap{\textcolor{white}{{\the\SOUL@token}}\hspace{#1pt}}}
\DeclareRobustCommand*\myul{%
    \def\SOUL@everyspace{\underline{\space}\kern\z@}%
    \def\SOUL@everytoken{%
     \setbox0=\hbox{\the\SOUL@token}%
     \ifdim\dp0>\z@
        \raisebox{\dp0}{\underline{\phantom{\the\SOUL@token}}}%
        \whiten{1}\whiten{0}%
        \whiten{-1}\whiten{-2}%
        \llap{\the\SOUL@token}%
     \else
        \underline{\the\SOUL@token}%
     \fi}%
\SOUL@}
\makeatother

\newcommand*{\demp}{\fontfamily{lmtt}\selectfont}

\DeclareTextFontCommand{\textdemp}{\demp}

\begin{document}

\ifcomment
Multiline
comment
\fi
\ifcomment
\myul{Typesetting test}
% \color[rgb]{1,1,1}
$∑_i^n≠ 60º±∞π∆¬≈√j∫h≤≥µ$

$\CR \R\pro\ind\pro\gS\pro
\mqty[a&b\\c&d]$
$\pro\mathbb{P}$
$\dd{x}$

  \[
    \alpha(x)=\left\{
                \begin{array}{ll}
                  x\\
                  \frac{1}{1+e^{-kx}}\\
                  \frac{e^x-e^{-x}}{e^x+e^{-x}}
                \end{array}
              \right.
  \]

  $\expval{x}$
  
  $\chi_\rho(ghg\dmo)=\Tr(\rho_{ghg\dmo})=\Tr(\rho_g\circ\rho_h\circ\rho\dmo_g)=\Tr(\rho_h)\overset{\mbox{\scalebox{0.5}{$\Tr(AB)=\Tr(BA)$}}}{=}\chi_\rho(h)$
  	$\mathop{\oplus}_{\substack{x\in X}}$

$\mat(\rho_g)=(a_{ij}(g))_{\scriptsize \substack{1\leq i\leq d \\ 1\leq j\leq d}}$ et $\mat(\rho'_g)=(a'_{ij}(g))_{\scriptsize \substack{1\leq i'\leq d' \\ 1\leq j'\leq d'}}$



\[\int_a^b{\mathbb{R}^2}g(u, v)\dd{P_{XY}}(u, v)=\iint g(u,v) f_{XY}(u, v)\dd \lambda(u) \dd \lambda(v)\]
$$\lim_{x\to\infty} f(x)$$	
$$\iiiint_V \mu(t,u,v,w) \,dt\,du\,dv\,dw$$
$$\sum_{n=1}^{\infty} 2^{-n} = 1$$	
\begin{definition}
	Si $X$ et $Y$ sont 2 v.a. ou definit la \textsc{Covariance} entre $X$ et $Y$ comme
	$\cov(X,Y)\overset{\text{def}}{=}\E\left[(X-\E(X))(Y-\E(Y))\right]=\E(XY)-\E(X)\E(Y)$.
\end{definition}
\fi
\pagebreak

% \tableofcontents

% insert your code here
%\input{./algebra/main.tex}
%\input{./geometrie-differentielle/main.tex}
%\input{./probabilite/main.tex}
%\input{./analyse-fonctionnelle/main.tex}
% \input{./Analyse-convexe-et-dualite-en-optimisation/main.tex}
%\input{./tikz/main.tex}
%\input{./Theorie-du-distributions/main.tex}
%\input{./optimisation/mine.tex}
 \input{./modelisation/main.tex}

% yves.aubry@univ-tln.fr : algebra

\end{document}

%\input{./optimisation/mine.tex}
 % !TEX encoding = UTF-8 Unicode
% !TEX TS-program = xelatex

\documentclass[french]{report}

%\usepackage[utf8]{inputenc}
%\usepackage[T1]{fontenc}
\usepackage{babel}


\newif\ifcomment
%\commenttrue # Show comments

\usepackage{physics}
\usepackage{amssymb}


\usepackage{amsthm}
% \usepackage{thmtools}
\usepackage{mathtools}
\usepackage{amsfonts}

\usepackage{color}

\usepackage{tikz}

\usepackage{geometry}
\geometry{a5paper, margin=0.1in, right=1cm}

\usepackage{dsfont}

\usepackage{graphicx}
\graphicspath{ {images/} }

\usepackage{faktor}

\usepackage{IEEEtrantools}
\usepackage{enumerate}   
\usepackage[PostScript=dvips]{"/Users/aware/Documents/Courses/diagrams"}


\newtheorem{theorem}{Théorème}[section]
\renewcommand{\thetheorem}{\arabic{theorem}}
\newtheorem{lemme}{Lemme}[section]
\renewcommand{\thelemme}{\arabic{lemme}}
\newtheorem{proposition}{Proposition}[section]
\renewcommand{\theproposition}{\arabic{proposition}}
\newtheorem{notations}{Notations}[section]
\newtheorem{problem}{Problème}[section]
\newtheorem{corollary}{Corollaire}[theorem]
\renewcommand{\thecorollary}{\arabic{corollary}}
\newtheorem{property}{Propriété}[section]
\newtheorem{objective}{Objectif}[section]

\theoremstyle{definition}
\newtheorem{definition}{Définition}[section]
\renewcommand{\thedefinition}{\arabic{definition}}
\newtheorem{exercise}{Exercice}[chapter]
\renewcommand{\theexercise}{\arabic{exercise}}
\newtheorem{example}{Exemple}[chapter]
\renewcommand{\theexample}{\arabic{example}}
\newtheorem*{solution}{Solution}
\newtheorem*{application}{Application}
\newtheorem*{notation}{Notation}
\newtheorem*{vocabulary}{Vocabulaire}
\newtheorem*{properties}{Propriétés}



\theoremstyle{remark}
\newtheorem*{remark}{Remarque}
\newtheorem*{rappel}{Rappel}


\usepackage{etoolbox}
\AtBeginEnvironment{exercise}{\small}
\AtBeginEnvironment{example}{\small}

\usepackage{cases}
\usepackage[red]{mypack}

\usepackage[framemethod=TikZ]{mdframed}

\definecolor{bg}{rgb}{0.4,0.25,0.95}
\definecolor{pagebg}{rgb}{0,0,0.5}
\surroundwithmdframed[
   topline=false,
   rightline=false,
   bottomline=false,
   leftmargin=\parindent,
   skipabove=8pt,
   skipbelow=8pt,
   linecolor=blue,
   innerbottommargin=10pt,
   % backgroundcolor=bg,font=\color{orange}\sffamily, fontcolor=white
]{definition}

\usepackage{empheq}
\usepackage[most]{tcolorbox}

\newtcbox{\mymath}[1][]{%
    nobeforeafter, math upper, tcbox raise base,
    enhanced, colframe=blue!30!black,
    colback=red!10, boxrule=1pt,
    #1}

\usepackage{unixode}


\DeclareMathOperator{\ord}{ord}
\DeclareMathOperator{\orb}{orb}
\DeclareMathOperator{\stab}{stab}
\DeclareMathOperator{\Stab}{stab}
\DeclareMathOperator{\ppcm}{ppcm}
\DeclareMathOperator{\conj}{Conj}
\DeclareMathOperator{\End}{End}
\DeclareMathOperator{\rot}{rot}
\DeclareMathOperator{\trs}{trace}
\DeclareMathOperator{\Ind}{Ind}
\DeclareMathOperator{\mat}{Mat}
\DeclareMathOperator{\id}{Id}
\DeclareMathOperator{\vect}{vect}
\DeclareMathOperator{\img}{img}
\DeclareMathOperator{\cov}{Cov}
\DeclareMathOperator{\dist}{dist}
\DeclareMathOperator{\irr}{Irr}
\DeclareMathOperator{\image}{Im}
\DeclareMathOperator{\pd}{\partial}
\DeclareMathOperator{\epi}{epi}
\DeclareMathOperator{\Argmin}{Argmin}
\DeclareMathOperator{\dom}{dom}
\DeclareMathOperator{\proj}{proj}
\DeclareMathOperator{\ctg}{ctg}
\DeclareMathOperator{\supp}{supp}
\DeclareMathOperator{\argmin}{argmin}
\DeclareMathOperator{\mult}{mult}
\DeclareMathOperator{\ch}{ch}
\DeclareMathOperator{\sh}{sh}
\DeclareMathOperator{\rang}{rang}
\DeclareMathOperator{\diam}{diam}
\DeclareMathOperator{\Epigraphe}{Epigraphe}




\usepackage{xcolor}
\everymath{\color{blue}}
%\everymath{\color[rgb]{0,1,1}}
%\pagecolor[rgb]{0,0,0.5}


\newcommand*{\pdtest}[3][]{\ensuremath{\frac{\partial^{#1} #2}{\partial #3}}}

\newcommand*{\deffunc}[6][]{\ensuremath{
\begin{array}{rcl}
#2 : #3 &\rightarrow& #4\\
#5 &\mapsto& #6
\end{array}
}}

\newcommand{\eqcolon}{\mathrel{\resizebox{\widthof{$\mathord{=}$}}{\height}{ $\!\!=\!\!\resizebox{1.2\width}{0.8\height}{\raisebox{0.23ex}{$\mathop{:}$}}\!\!$ }}}
\newcommand{\coloneq}{\mathrel{\resizebox{\widthof{$\mathord{=}$}}{\height}{ $\!\!\resizebox{1.2\width}{0.8\height}{\raisebox{0.23ex}{$\mathop{:}$}}\!\!=\!\!$ }}}
\newcommand{\eqcolonl}{\ensuremath{\mathrel{=\!\!\mathop{:}}}}
\newcommand{\coloneql}{\ensuremath{\mathrel{\mathop{:} \!\! =}}}
\newcommand{\vc}[1]{% inline column vector
  \left(\begin{smallmatrix}#1\end{smallmatrix}\right)%
}
\newcommand{\vr}[1]{% inline row vector
  \begin{smallmatrix}(\,#1\,)\end{smallmatrix}%
}
\makeatletter
\newcommand*{\defeq}{\ =\mathrel{\rlap{%
                     \raisebox{0.3ex}{$\m@th\cdot$}}%
                     \raisebox{-0.3ex}{$\m@th\cdot$}}%
                     }
\makeatother

\newcommand{\mathcircle}[1]{% inline row vector
 \overset{\circ}{#1}
}
\newcommand{\ulim}{% low limit
 \underline{\lim}
}
\newcommand{\ssi}{% iff
\iff
}
\newcommand{\ps}[2]{
\expval{#1 | #2}
}
\newcommand{\df}[1]{
\mqty{#1}
}
\newcommand{\n}[1]{
\norm{#1}
}
\newcommand{\sys}[1]{
\left\{\smqty{#1}\right.
}


\newcommand{\eqdef}{\ensuremath{\overset{\text{def}}=}}


\def\Circlearrowright{\ensuremath{%
  \rotatebox[origin=c]{230}{$\circlearrowright$}}}

\newcommand\ct[1]{\text{\rmfamily\upshape #1}}
\newcommand\question[1]{ {\color{red} ...!? \small #1}}
\newcommand\caz[1]{\left\{\begin{array} #1 \end{array}\right.}
\newcommand\const{\text{\rmfamily\upshape const}}
\newcommand\toP{ \overset{\pro}{\to}}
\newcommand\toPP{ \overset{\text{PP}}{\to}}
\newcommand{\oeq}{\mathrel{\text{\textcircled{$=$}}}}





\usepackage{xcolor}
% \usepackage[normalem]{ulem}
\usepackage{lipsum}
\makeatletter
% \newcommand\colorwave[1][blue]{\bgroup \markoverwith{\lower3.5\p@\hbox{\sixly \textcolor{#1}{\char58}}}\ULon}
%\font\sixly=lasy6 % does not re-load if already loaded, so no memory problem.

\newmdtheoremenv[
linewidth= 1pt,linecolor= blue,%
leftmargin=20,rightmargin=20,innertopmargin=0pt, innerrightmargin=40,%
tikzsetting = { draw=lightgray, line width = 0.3pt,dashed,%
dash pattern = on 15pt off 3pt},%
splittopskip=\topskip,skipbelow=\baselineskip,%
skipabove=\baselineskip,ntheorem,roundcorner=0pt,
% backgroundcolor=pagebg,font=\color{orange}\sffamily, fontcolor=white
]{examplebox}{Exemple}[section]



\newcommand\R{\mathbb{R}}
\newcommand\Z{\mathbb{Z}}
\newcommand\N{\mathbb{N}}
\newcommand\E{\mathbb{E}}
\newcommand\F{\mathcal{F}}
\newcommand\cH{\mathcal{H}}
\newcommand\V{\mathbb{V}}
\newcommand\dmo{ ^{-1} }
\newcommand\kapa{\kappa}
\newcommand\im{Im}
\newcommand\hs{\mathcal{H}}





\usepackage{soul}

\makeatletter
\newcommand*{\whiten}[1]{\llap{\textcolor{white}{{\the\SOUL@token}}\hspace{#1pt}}}
\DeclareRobustCommand*\myul{%
    \def\SOUL@everyspace{\underline{\space}\kern\z@}%
    \def\SOUL@everytoken{%
     \setbox0=\hbox{\the\SOUL@token}%
     \ifdim\dp0>\z@
        \raisebox{\dp0}{\underline{\phantom{\the\SOUL@token}}}%
        \whiten{1}\whiten{0}%
        \whiten{-1}\whiten{-2}%
        \llap{\the\SOUL@token}%
     \else
        \underline{\the\SOUL@token}%
     \fi}%
\SOUL@}
\makeatother

\newcommand*{\demp}{\fontfamily{lmtt}\selectfont}

\DeclareTextFontCommand{\textdemp}{\demp}

\begin{document}

\ifcomment
Multiline
comment
\fi
\ifcomment
\myul{Typesetting test}
% \color[rgb]{1,1,1}
$∑_i^n≠ 60º±∞π∆¬≈√j∫h≤≥µ$

$\CR \R\pro\ind\pro\gS\pro
\mqty[a&b\\c&d]$
$\pro\mathbb{P}$
$\dd{x}$

  \[
    \alpha(x)=\left\{
                \begin{array}{ll}
                  x\\
                  \frac{1}{1+e^{-kx}}\\
                  \frac{e^x-e^{-x}}{e^x+e^{-x}}
                \end{array}
              \right.
  \]

  $\expval{x}$
  
  $\chi_\rho(ghg\dmo)=\Tr(\rho_{ghg\dmo})=\Tr(\rho_g\circ\rho_h\circ\rho\dmo_g)=\Tr(\rho_h)\overset{\mbox{\scalebox{0.5}{$\Tr(AB)=\Tr(BA)$}}}{=}\chi_\rho(h)$
  	$\mathop{\oplus}_{\substack{x\in X}}$

$\mat(\rho_g)=(a_{ij}(g))_{\scriptsize \substack{1\leq i\leq d \\ 1\leq j\leq d}}$ et $\mat(\rho'_g)=(a'_{ij}(g))_{\scriptsize \substack{1\leq i'\leq d' \\ 1\leq j'\leq d'}}$



\[\int_a^b{\mathbb{R}^2}g(u, v)\dd{P_{XY}}(u, v)=\iint g(u,v) f_{XY}(u, v)\dd \lambda(u) \dd \lambda(v)\]
$$\lim_{x\to\infty} f(x)$$	
$$\iiiint_V \mu(t,u,v,w) \,dt\,du\,dv\,dw$$
$$\sum_{n=1}^{\infty} 2^{-n} = 1$$	
\begin{definition}
	Si $X$ et $Y$ sont 2 v.a. ou definit la \textsc{Covariance} entre $X$ et $Y$ comme
	$\cov(X,Y)\overset{\text{def}}{=}\E\left[(X-\E(X))(Y-\E(Y))\right]=\E(XY)-\E(X)\E(Y)$.
\end{definition}
\fi
\pagebreak

% \tableofcontents

% insert your code here
%\input{./algebra/main.tex}
%\input{./geometrie-differentielle/main.tex}
%\input{./probabilite/main.tex}
%\input{./analyse-fonctionnelle/main.tex}
% \input{./Analyse-convexe-et-dualite-en-optimisation/main.tex}
%\input{./tikz/main.tex}
%\input{./Theorie-du-distributions/main.tex}
%\input{./optimisation/mine.tex}
 \input{./modelisation/main.tex}

% yves.aubry@univ-tln.fr : algebra

\end{document}


% yves.aubry@univ-tln.fr : algebra

\end{document}

%% !TEX encoding = UTF-8 Unicode
% !TEX TS-program = xelatex

\documentclass[french]{report}

%\usepackage[utf8]{inputenc}
%\usepackage[T1]{fontenc}
\usepackage{babel}


\newif\ifcomment
%\commenttrue # Show comments

\usepackage{physics}
\usepackage{amssymb}


\usepackage{amsthm}
% \usepackage{thmtools}
\usepackage{mathtools}
\usepackage{amsfonts}

\usepackage{color}

\usepackage{tikz}

\usepackage{geometry}
\geometry{a5paper, margin=0.1in, right=1cm}

\usepackage{dsfont}

\usepackage{graphicx}
\graphicspath{ {images/} }

\usepackage{faktor}

\usepackage{IEEEtrantools}
\usepackage{enumerate}   
\usepackage[PostScript=dvips]{"/Users/aware/Documents/Courses/diagrams"}


\newtheorem{theorem}{Théorème}[section]
\renewcommand{\thetheorem}{\arabic{theorem}}
\newtheorem{lemme}{Lemme}[section]
\renewcommand{\thelemme}{\arabic{lemme}}
\newtheorem{proposition}{Proposition}[section]
\renewcommand{\theproposition}{\arabic{proposition}}
\newtheorem{notations}{Notations}[section]
\newtheorem{problem}{Problème}[section]
\newtheorem{corollary}{Corollaire}[theorem]
\renewcommand{\thecorollary}{\arabic{corollary}}
\newtheorem{property}{Propriété}[section]
\newtheorem{objective}{Objectif}[section]

\theoremstyle{definition}
\newtheorem{definition}{Définition}[section]
\renewcommand{\thedefinition}{\arabic{definition}}
\newtheorem{exercise}{Exercice}[chapter]
\renewcommand{\theexercise}{\arabic{exercise}}
\newtheorem{example}{Exemple}[chapter]
\renewcommand{\theexample}{\arabic{example}}
\newtheorem*{solution}{Solution}
\newtheorem*{application}{Application}
\newtheorem*{notation}{Notation}
\newtheorem*{vocabulary}{Vocabulaire}
\newtheorem*{properties}{Propriétés}



\theoremstyle{remark}
\newtheorem*{remark}{Remarque}
\newtheorem*{rappel}{Rappel}


\usepackage{etoolbox}
\AtBeginEnvironment{exercise}{\small}
\AtBeginEnvironment{example}{\small}

\usepackage{cases}
\usepackage[red]{mypack}

\usepackage[framemethod=TikZ]{mdframed}

\definecolor{bg}{rgb}{0.4,0.25,0.95}
\definecolor{pagebg}{rgb}{0,0,0.5}
\surroundwithmdframed[
   topline=false,
   rightline=false,
   bottomline=false,
   leftmargin=\parindent,
   skipabove=8pt,
   skipbelow=8pt,
   linecolor=blue,
   innerbottommargin=10pt,
   % backgroundcolor=bg,font=\color{orange}\sffamily, fontcolor=white
]{definition}

\usepackage{empheq}
\usepackage[most]{tcolorbox}

\newtcbox{\mymath}[1][]{%
    nobeforeafter, math upper, tcbox raise base,
    enhanced, colframe=blue!30!black,
    colback=red!10, boxrule=1pt,
    #1}

\usepackage{unixode}


\DeclareMathOperator{\ord}{ord}
\DeclareMathOperator{\orb}{orb}
\DeclareMathOperator{\stab}{stab}
\DeclareMathOperator{\Stab}{stab}
\DeclareMathOperator{\ppcm}{ppcm}
\DeclareMathOperator{\conj}{Conj}
\DeclareMathOperator{\End}{End}
\DeclareMathOperator{\rot}{rot}
\DeclareMathOperator{\trs}{trace}
\DeclareMathOperator{\Ind}{Ind}
\DeclareMathOperator{\mat}{Mat}
\DeclareMathOperator{\id}{Id}
\DeclareMathOperator{\vect}{vect}
\DeclareMathOperator{\img}{img}
\DeclareMathOperator{\cov}{Cov}
\DeclareMathOperator{\dist}{dist}
\DeclareMathOperator{\irr}{Irr}
\DeclareMathOperator{\image}{Im}
\DeclareMathOperator{\pd}{\partial}
\DeclareMathOperator{\epi}{epi}
\DeclareMathOperator{\Argmin}{Argmin}
\DeclareMathOperator{\dom}{dom}
\DeclareMathOperator{\proj}{proj}
\DeclareMathOperator{\ctg}{ctg}
\DeclareMathOperator{\supp}{supp}
\DeclareMathOperator{\argmin}{argmin}
\DeclareMathOperator{\mult}{mult}
\DeclareMathOperator{\ch}{ch}
\DeclareMathOperator{\sh}{sh}
\DeclareMathOperator{\rang}{rang}
\DeclareMathOperator{\diam}{diam}
\DeclareMathOperator{\Epigraphe}{Epigraphe}




\usepackage{xcolor}
\everymath{\color{blue}}
%\everymath{\color[rgb]{0,1,1}}
%\pagecolor[rgb]{0,0,0.5}


\newcommand*{\pdtest}[3][]{\ensuremath{\frac{\partial^{#1} #2}{\partial #3}}}

\newcommand*{\deffunc}[6][]{\ensuremath{
\begin{array}{rcl}
#2 : #3 &\rightarrow& #4\\
#5 &\mapsto& #6
\end{array}
}}

\newcommand{\eqcolon}{\mathrel{\resizebox{\widthof{$\mathord{=}$}}{\height}{ $\!\!=\!\!\resizebox{1.2\width}{0.8\height}{\raisebox{0.23ex}{$\mathop{:}$}}\!\!$ }}}
\newcommand{\coloneq}{\mathrel{\resizebox{\widthof{$\mathord{=}$}}{\height}{ $\!\!\resizebox{1.2\width}{0.8\height}{\raisebox{0.23ex}{$\mathop{:}$}}\!\!=\!\!$ }}}
\newcommand{\eqcolonl}{\ensuremath{\mathrel{=\!\!\mathop{:}}}}
\newcommand{\coloneql}{\ensuremath{\mathrel{\mathop{:} \!\! =}}}
\newcommand{\vc}[1]{% inline column vector
  \left(\begin{smallmatrix}#1\end{smallmatrix}\right)%
}
\newcommand{\vr}[1]{% inline row vector
  \begin{smallmatrix}(\,#1\,)\end{smallmatrix}%
}
\makeatletter
\newcommand*{\defeq}{\ =\mathrel{\rlap{%
                     \raisebox{0.3ex}{$\m@th\cdot$}}%
                     \raisebox{-0.3ex}{$\m@th\cdot$}}%
                     }
\makeatother

\newcommand{\mathcircle}[1]{% inline row vector
 \overset{\circ}{#1}
}
\newcommand{\ulim}{% low limit
 \underline{\lim}
}
\newcommand{\ssi}{% iff
\iff
}
\newcommand{\ps}[2]{
\expval{#1 | #2}
}
\newcommand{\df}[1]{
\mqty{#1}
}
\newcommand{\n}[1]{
\norm{#1}
}
\newcommand{\sys}[1]{
\left\{\smqty{#1}\right.
}


\newcommand{\eqdef}{\ensuremath{\overset{\text{def}}=}}


\def\Circlearrowright{\ensuremath{%
  \rotatebox[origin=c]{230}{$\circlearrowright$}}}

\newcommand\ct[1]{\text{\rmfamily\upshape #1}}
\newcommand\question[1]{ {\color{red} ...!? \small #1}}
\newcommand\caz[1]{\left\{\begin{array} #1 \end{array}\right.}
\newcommand\const{\text{\rmfamily\upshape const}}
\newcommand\toP{ \overset{\pro}{\to}}
\newcommand\toPP{ \overset{\text{PP}}{\to}}
\newcommand{\oeq}{\mathrel{\text{\textcircled{$=$}}}}





\usepackage{xcolor}
% \usepackage[normalem]{ulem}
\usepackage{lipsum}
\makeatletter
% \newcommand\colorwave[1][blue]{\bgroup \markoverwith{\lower3.5\p@\hbox{\sixly \textcolor{#1}{\char58}}}\ULon}
%\font\sixly=lasy6 % does not re-load if already loaded, so no memory problem.

\newmdtheoremenv[
linewidth= 1pt,linecolor= blue,%
leftmargin=20,rightmargin=20,innertopmargin=0pt, innerrightmargin=40,%
tikzsetting = { draw=lightgray, line width = 0.3pt,dashed,%
dash pattern = on 15pt off 3pt},%
splittopskip=\topskip,skipbelow=\baselineskip,%
skipabove=\baselineskip,ntheorem,roundcorner=0pt,
% backgroundcolor=pagebg,font=\color{orange}\sffamily, fontcolor=white
]{examplebox}{Exemple}[section]



\newcommand\R{\mathbb{R}}
\newcommand\Z{\mathbb{Z}}
\newcommand\N{\mathbb{N}}
\newcommand\E{\mathbb{E}}
\newcommand\F{\mathcal{F}}
\newcommand\cH{\mathcal{H}}
\newcommand\V{\mathbb{V}}
\newcommand\dmo{ ^{-1} }
\newcommand\kapa{\kappa}
\newcommand\im{Im}
\newcommand\hs{\mathcal{H}}





\usepackage{soul}

\makeatletter
\newcommand*{\whiten}[1]{\llap{\textcolor{white}{{\the\SOUL@token}}\hspace{#1pt}}}
\DeclareRobustCommand*\myul{%
    \def\SOUL@everyspace{\underline{\space}\kern\z@}%
    \def\SOUL@everytoken{%
     \setbox0=\hbox{\the\SOUL@token}%
     \ifdim\dp0>\z@
        \raisebox{\dp0}{\underline{\phantom{\the\SOUL@token}}}%
        \whiten{1}\whiten{0}%
        \whiten{-1}\whiten{-2}%
        \llap{\the\SOUL@token}%
     \else
        \underline{\the\SOUL@token}%
     \fi}%
\SOUL@}
\makeatother

\newcommand*{\demp}{\fontfamily{lmtt}\selectfont}

\DeclareTextFontCommand{\textdemp}{\demp}

\begin{document}

\ifcomment
Multiline
comment
\fi
\ifcomment
\myul{Typesetting test}
% \color[rgb]{1,1,1}
$∑_i^n≠ 60º±∞π∆¬≈√j∫h≤≥µ$

$\CR \R\pro\ind\pro\gS\pro
\mqty[a&b\\c&d]$
$\pro\mathbb{P}$
$\dd{x}$

  \[
    \alpha(x)=\left\{
                \begin{array}{ll}
                  x\\
                  \frac{1}{1+e^{-kx}}\\
                  \frac{e^x-e^{-x}}{e^x+e^{-x}}
                \end{array}
              \right.
  \]

  $\expval{x}$
  
  $\chi_\rho(ghg\dmo)=\Tr(\rho_{ghg\dmo})=\Tr(\rho_g\circ\rho_h\circ\rho\dmo_g)=\Tr(\rho_h)\overset{\mbox{\scalebox{0.5}{$\Tr(AB)=\Tr(BA)$}}}{=}\chi_\rho(h)$
  	$\mathop{\oplus}_{\substack{x\in X}}$

$\mat(\rho_g)=(a_{ij}(g))_{\scriptsize \substack{1\leq i\leq d \\ 1\leq j\leq d}}$ et $\mat(\rho'_g)=(a'_{ij}(g))_{\scriptsize \substack{1\leq i'\leq d' \\ 1\leq j'\leq d'}}$



\[\int_a^b{\mathbb{R}^2}g(u, v)\dd{P_{XY}}(u, v)=\iint g(u,v) f_{XY}(u, v)\dd \lambda(u) \dd \lambda(v)\]
$$\lim_{x\to\infty} f(x)$$	
$$\iiiint_V \mu(t,u,v,w) \,dt\,du\,dv\,dw$$
$$\sum_{n=1}^{\infty} 2^{-n} = 1$$	
\begin{definition}
	Si $X$ et $Y$ sont 2 v.a. ou definit la \textsc{Covariance} entre $X$ et $Y$ comme
	$\cov(X,Y)\overset{\text{def}}{=}\E\left[(X-\E(X))(Y-\E(Y))\right]=\E(XY)-\E(X)\E(Y)$.
\end{definition}
\fi
\pagebreak

% \tableofcontents

% insert your code here
%% !TEX encoding = UTF-8 Unicode
% !TEX TS-program = xelatex

\documentclass[french]{report}

%\usepackage[utf8]{inputenc}
%\usepackage[T1]{fontenc}
\usepackage{babel}


\newif\ifcomment
%\commenttrue # Show comments

\usepackage{physics}
\usepackage{amssymb}


\usepackage{amsthm}
% \usepackage{thmtools}
\usepackage{mathtools}
\usepackage{amsfonts}

\usepackage{color}

\usepackage{tikz}

\usepackage{geometry}
\geometry{a5paper, margin=0.1in, right=1cm}

\usepackage{dsfont}

\usepackage{graphicx}
\graphicspath{ {images/} }

\usepackage{faktor}

\usepackage{IEEEtrantools}
\usepackage{enumerate}   
\usepackage[PostScript=dvips]{"/Users/aware/Documents/Courses/diagrams"}


\newtheorem{theorem}{Théorème}[section]
\renewcommand{\thetheorem}{\arabic{theorem}}
\newtheorem{lemme}{Lemme}[section]
\renewcommand{\thelemme}{\arabic{lemme}}
\newtheorem{proposition}{Proposition}[section]
\renewcommand{\theproposition}{\arabic{proposition}}
\newtheorem{notations}{Notations}[section]
\newtheorem{problem}{Problème}[section]
\newtheorem{corollary}{Corollaire}[theorem]
\renewcommand{\thecorollary}{\arabic{corollary}}
\newtheorem{property}{Propriété}[section]
\newtheorem{objective}{Objectif}[section]

\theoremstyle{definition}
\newtheorem{definition}{Définition}[section]
\renewcommand{\thedefinition}{\arabic{definition}}
\newtheorem{exercise}{Exercice}[chapter]
\renewcommand{\theexercise}{\arabic{exercise}}
\newtheorem{example}{Exemple}[chapter]
\renewcommand{\theexample}{\arabic{example}}
\newtheorem*{solution}{Solution}
\newtheorem*{application}{Application}
\newtheorem*{notation}{Notation}
\newtheorem*{vocabulary}{Vocabulaire}
\newtheorem*{properties}{Propriétés}



\theoremstyle{remark}
\newtheorem*{remark}{Remarque}
\newtheorem*{rappel}{Rappel}


\usepackage{etoolbox}
\AtBeginEnvironment{exercise}{\small}
\AtBeginEnvironment{example}{\small}

\usepackage{cases}
\usepackage[red]{mypack}

\usepackage[framemethod=TikZ]{mdframed}

\definecolor{bg}{rgb}{0.4,0.25,0.95}
\definecolor{pagebg}{rgb}{0,0,0.5}
\surroundwithmdframed[
   topline=false,
   rightline=false,
   bottomline=false,
   leftmargin=\parindent,
   skipabove=8pt,
   skipbelow=8pt,
   linecolor=blue,
   innerbottommargin=10pt,
   % backgroundcolor=bg,font=\color{orange}\sffamily, fontcolor=white
]{definition}

\usepackage{empheq}
\usepackage[most]{tcolorbox}

\newtcbox{\mymath}[1][]{%
    nobeforeafter, math upper, tcbox raise base,
    enhanced, colframe=blue!30!black,
    colback=red!10, boxrule=1pt,
    #1}

\usepackage{unixode}


\DeclareMathOperator{\ord}{ord}
\DeclareMathOperator{\orb}{orb}
\DeclareMathOperator{\stab}{stab}
\DeclareMathOperator{\Stab}{stab}
\DeclareMathOperator{\ppcm}{ppcm}
\DeclareMathOperator{\conj}{Conj}
\DeclareMathOperator{\End}{End}
\DeclareMathOperator{\rot}{rot}
\DeclareMathOperator{\trs}{trace}
\DeclareMathOperator{\Ind}{Ind}
\DeclareMathOperator{\mat}{Mat}
\DeclareMathOperator{\id}{Id}
\DeclareMathOperator{\vect}{vect}
\DeclareMathOperator{\img}{img}
\DeclareMathOperator{\cov}{Cov}
\DeclareMathOperator{\dist}{dist}
\DeclareMathOperator{\irr}{Irr}
\DeclareMathOperator{\image}{Im}
\DeclareMathOperator{\pd}{\partial}
\DeclareMathOperator{\epi}{epi}
\DeclareMathOperator{\Argmin}{Argmin}
\DeclareMathOperator{\dom}{dom}
\DeclareMathOperator{\proj}{proj}
\DeclareMathOperator{\ctg}{ctg}
\DeclareMathOperator{\supp}{supp}
\DeclareMathOperator{\argmin}{argmin}
\DeclareMathOperator{\mult}{mult}
\DeclareMathOperator{\ch}{ch}
\DeclareMathOperator{\sh}{sh}
\DeclareMathOperator{\rang}{rang}
\DeclareMathOperator{\diam}{diam}
\DeclareMathOperator{\Epigraphe}{Epigraphe}




\usepackage{xcolor}
\everymath{\color{blue}}
%\everymath{\color[rgb]{0,1,1}}
%\pagecolor[rgb]{0,0,0.5}


\newcommand*{\pdtest}[3][]{\ensuremath{\frac{\partial^{#1} #2}{\partial #3}}}

\newcommand*{\deffunc}[6][]{\ensuremath{
\begin{array}{rcl}
#2 : #3 &\rightarrow& #4\\
#5 &\mapsto& #6
\end{array}
}}

\newcommand{\eqcolon}{\mathrel{\resizebox{\widthof{$\mathord{=}$}}{\height}{ $\!\!=\!\!\resizebox{1.2\width}{0.8\height}{\raisebox{0.23ex}{$\mathop{:}$}}\!\!$ }}}
\newcommand{\coloneq}{\mathrel{\resizebox{\widthof{$\mathord{=}$}}{\height}{ $\!\!\resizebox{1.2\width}{0.8\height}{\raisebox{0.23ex}{$\mathop{:}$}}\!\!=\!\!$ }}}
\newcommand{\eqcolonl}{\ensuremath{\mathrel{=\!\!\mathop{:}}}}
\newcommand{\coloneql}{\ensuremath{\mathrel{\mathop{:} \!\! =}}}
\newcommand{\vc}[1]{% inline column vector
  \left(\begin{smallmatrix}#1\end{smallmatrix}\right)%
}
\newcommand{\vr}[1]{% inline row vector
  \begin{smallmatrix}(\,#1\,)\end{smallmatrix}%
}
\makeatletter
\newcommand*{\defeq}{\ =\mathrel{\rlap{%
                     \raisebox{0.3ex}{$\m@th\cdot$}}%
                     \raisebox{-0.3ex}{$\m@th\cdot$}}%
                     }
\makeatother

\newcommand{\mathcircle}[1]{% inline row vector
 \overset{\circ}{#1}
}
\newcommand{\ulim}{% low limit
 \underline{\lim}
}
\newcommand{\ssi}{% iff
\iff
}
\newcommand{\ps}[2]{
\expval{#1 | #2}
}
\newcommand{\df}[1]{
\mqty{#1}
}
\newcommand{\n}[1]{
\norm{#1}
}
\newcommand{\sys}[1]{
\left\{\smqty{#1}\right.
}


\newcommand{\eqdef}{\ensuremath{\overset{\text{def}}=}}


\def\Circlearrowright{\ensuremath{%
  \rotatebox[origin=c]{230}{$\circlearrowright$}}}

\newcommand\ct[1]{\text{\rmfamily\upshape #1}}
\newcommand\question[1]{ {\color{red} ...!? \small #1}}
\newcommand\caz[1]{\left\{\begin{array} #1 \end{array}\right.}
\newcommand\const{\text{\rmfamily\upshape const}}
\newcommand\toP{ \overset{\pro}{\to}}
\newcommand\toPP{ \overset{\text{PP}}{\to}}
\newcommand{\oeq}{\mathrel{\text{\textcircled{$=$}}}}





\usepackage{xcolor}
% \usepackage[normalem]{ulem}
\usepackage{lipsum}
\makeatletter
% \newcommand\colorwave[1][blue]{\bgroup \markoverwith{\lower3.5\p@\hbox{\sixly \textcolor{#1}{\char58}}}\ULon}
%\font\sixly=lasy6 % does not re-load if already loaded, so no memory problem.

\newmdtheoremenv[
linewidth= 1pt,linecolor= blue,%
leftmargin=20,rightmargin=20,innertopmargin=0pt, innerrightmargin=40,%
tikzsetting = { draw=lightgray, line width = 0.3pt,dashed,%
dash pattern = on 15pt off 3pt},%
splittopskip=\topskip,skipbelow=\baselineskip,%
skipabove=\baselineskip,ntheorem,roundcorner=0pt,
% backgroundcolor=pagebg,font=\color{orange}\sffamily, fontcolor=white
]{examplebox}{Exemple}[section]



\newcommand\R{\mathbb{R}}
\newcommand\Z{\mathbb{Z}}
\newcommand\N{\mathbb{N}}
\newcommand\E{\mathbb{E}}
\newcommand\F{\mathcal{F}}
\newcommand\cH{\mathcal{H}}
\newcommand\V{\mathbb{V}}
\newcommand\dmo{ ^{-1} }
\newcommand\kapa{\kappa}
\newcommand\im{Im}
\newcommand\hs{\mathcal{H}}





\usepackage{soul}

\makeatletter
\newcommand*{\whiten}[1]{\llap{\textcolor{white}{{\the\SOUL@token}}\hspace{#1pt}}}
\DeclareRobustCommand*\myul{%
    \def\SOUL@everyspace{\underline{\space}\kern\z@}%
    \def\SOUL@everytoken{%
     \setbox0=\hbox{\the\SOUL@token}%
     \ifdim\dp0>\z@
        \raisebox{\dp0}{\underline{\phantom{\the\SOUL@token}}}%
        \whiten{1}\whiten{0}%
        \whiten{-1}\whiten{-2}%
        \llap{\the\SOUL@token}%
     \else
        \underline{\the\SOUL@token}%
     \fi}%
\SOUL@}
\makeatother

\newcommand*{\demp}{\fontfamily{lmtt}\selectfont}

\DeclareTextFontCommand{\textdemp}{\demp}

\begin{document}

\ifcomment
Multiline
comment
\fi
\ifcomment
\myul{Typesetting test}
% \color[rgb]{1,1,1}
$∑_i^n≠ 60º±∞π∆¬≈√j∫h≤≥µ$

$\CR \R\pro\ind\pro\gS\pro
\mqty[a&b\\c&d]$
$\pro\mathbb{P}$
$\dd{x}$

  \[
    \alpha(x)=\left\{
                \begin{array}{ll}
                  x\\
                  \frac{1}{1+e^{-kx}}\\
                  \frac{e^x-e^{-x}}{e^x+e^{-x}}
                \end{array}
              \right.
  \]

  $\expval{x}$
  
  $\chi_\rho(ghg\dmo)=\Tr(\rho_{ghg\dmo})=\Tr(\rho_g\circ\rho_h\circ\rho\dmo_g)=\Tr(\rho_h)\overset{\mbox{\scalebox{0.5}{$\Tr(AB)=\Tr(BA)$}}}{=}\chi_\rho(h)$
  	$\mathop{\oplus}_{\substack{x\in X}}$

$\mat(\rho_g)=(a_{ij}(g))_{\scriptsize \substack{1\leq i\leq d \\ 1\leq j\leq d}}$ et $\mat(\rho'_g)=(a'_{ij}(g))_{\scriptsize \substack{1\leq i'\leq d' \\ 1\leq j'\leq d'}}$



\[\int_a^b{\mathbb{R}^2}g(u, v)\dd{P_{XY}}(u, v)=\iint g(u,v) f_{XY}(u, v)\dd \lambda(u) \dd \lambda(v)\]
$$\lim_{x\to\infty} f(x)$$	
$$\iiiint_V \mu(t,u,v,w) \,dt\,du\,dv\,dw$$
$$\sum_{n=1}^{\infty} 2^{-n} = 1$$	
\begin{definition}
	Si $X$ et $Y$ sont 2 v.a. ou definit la \textsc{Covariance} entre $X$ et $Y$ comme
	$\cov(X,Y)\overset{\text{def}}{=}\E\left[(X-\E(X))(Y-\E(Y))\right]=\E(XY)-\E(X)\E(Y)$.
\end{definition}
\fi
\pagebreak

% \tableofcontents

% insert your code here
%\input{./algebra/main.tex}
%\input{./geometrie-differentielle/main.tex}
%\input{./probabilite/main.tex}
%\input{./analyse-fonctionnelle/main.tex}
% \input{./Analyse-convexe-et-dualite-en-optimisation/main.tex}
%\input{./tikz/main.tex}
%\input{./Theorie-du-distributions/main.tex}
%\input{./optimisation/mine.tex}
 \input{./modelisation/main.tex}

% yves.aubry@univ-tln.fr : algebra

\end{document}

%% !TEX encoding = UTF-8 Unicode
% !TEX TS-program = xelatex

\documentclass[french]{report}

%\usepackage[utf8]{inputenc}
%\usepackage[T1]{fontenc}
\usepackage{babel}


\newif\ifcomment
%\commenttrue # Show comments

\usepackage{physics}
\usepackage{amssymb}


\usepackage{amsthm}
% \usepackage{thmtools}
\usepackage{mathtools}
\usepackage{amsfonts}

\usepackage{color}

\usepackage{tikz}

\usepackage{geometry}
\geometry{a5paper, margin=0.1in, right=1cm}

\usepackage{dsfont}

\usepackage{graphicx}
\graphicspath{ {images/} }

\usepackage{faktor}

\usepackage{IEEEtrantools}
\usepackage{enumerate}   
\usepackage[PostScript=dvips]{"/Users/aware/Documents/Courses/diagrams"}


\newtheorem{theorem}{Théorème}[section]
\renewcommand{\thetheorem}{\arabic{theorem}}
\newtheorem{lemme}{Lemme}[section]
\renewcommand{\thelemme}{\arabic{lemme}}
\newtheorem{proposition}{Proposition}[section]
\renewcommand{\theproposition}{\arabic{proposition}}
\newtheorem{notations}{Notations}[section]
\newtheorem{problem}{Problème}[section]
\newtheorem{corollary}{Corollaire}[theorem]
\renewcommand{\thecorollary}{\arabic{corollary}}
\newtheorem{property}{Propriété}[section]
\newtheorem{objective}{Objectif}[section]

\theoremstyle{definition}
\newtheorem{definition}{Définition}[section]
\renewcommand{\thedefinition}{\arabic{definition}}
\newtheorem{exercise}{Exercice}[chapter]
\renewcommand{\theexercise}{\arabic{exercise}}
\newtheorem{example}{Exemple}[chapter]
\renewcommand{\theexample}{\arabic{example}}
\newtheorem*{solution}{Solution}
\newtheorem*{application}{Application}
\newtheorem*{notation}{Notation}
\newtheorem*{vocabulary}{Vocabulaire}
\newtheorem*{properties}{Propriétés}



\theoremstyle{remark}
\newtheorem*{remark}{Remarque}
\newtheorem*{rappel}{Rappel}


\usepackage{etoolbox}
\AtBeginEnvironment{exercise}{\small}
\AtBeginEnvironment{example}{\small}

\usepackage{cases}
\usepackage[red]{mypack}

\usepackage[framemethod=TikZ]{mdframed}

\definecolor{bg}{rgb}{0.4,0.25,0.95}
\definecolor{pagebg}{rgb}{0,0,0.5}
\surroundwithmdframed[
   topline=false,
   rightline=false,
   bottomline=false,
   leftmargin=\parindent,
   skipabove=8pt,
   skipbelow=8pt,
   linecolor=blue,
   innerbottommargin=10pt,
   % backgroundcolor=bg,font=\color{orange}\sffamily, fontcolor=white
]{definition}

\usepackage{empheq}
\usepackage[most]{tcolorbox}

\newtcbox{\mymath}[1][]{%
    nobeforeafter, math upper, tcbox raise base,
    enhanced, colframe=blue!30!black,
    colback=red!10, boxrule=1pt,
    #1}

\usepackage{unixode}


\DeclareMathOperator{\ord}{ord}
\DeclareMathOperator{\orb}{orb}
\DeclareMathOperator{\stab}{stab}
\DeclareMathOperator{\Stab}{stab}
\DeclareMathOperator{\ppcm}{ppcm}
\DeclareMathOperator{\conj}{Conj}
\DeclareMathOperator{\End}{End}
\DeclareMathOperator{\rot}{rot}
\DeclareMathOperator{\trs}{trace}
\DeclareMathOperator{\Ind}{Ind}
\DeclareMathOperator{\mat}{Mat}
\DeclareMathOperator{\id}{Id}
\DeclareMathOperator{\vect}{vect}
\DeclareMathOperator{\img}{img}
\DeclareMathOperator{\cov}{Cov}
\DeclareMathOperator{\dist}{dist}
\DeclareMathOperator{\irr}{Irr}
\DeclareMathOperator{\image}{Im}
\DeclareMathOperator{\pd}{\partial}
\DeclareMathOperator{\epi}{epi}
\DeclareMathOperator{\Argmin}{Argmin}
\DeclareMathOperator{\dom}{dom}
\DeclareMathOperator{\proj}{proj}
\DeclareMathOperator{\ctg}{ctg}
\DeclareMathOperator{\supp}{supp}
\DeclareMathOperator{\argmin}{argmin}
\DeclareMathOperator{\mult}{mult}
\DeclareMathOperator{\ch}{ch}
\DeclareMathOperator{\sh}{sh}
\DeclareMathOperator{\rang}{rang}
\DeclareMathOperator{\diam}{diam}
\DeclareMathOperator{\Epigraphe}{Epigraphe}




\usepackage{xcolor}
\everymath{\color{blue}}
%\everymath{\color[rgb]{0,1,1}}
%\pagecolor[rgb]{0,0,0.5}


\newcommand*{\pdtest}[3][]{\ensuremath{\frac{\partial^{#1} #2}{\partial #3}}}

\newcommand*{\deffunc}[6][]{\ensuremath{
\begin{array}{rcl}
#2 : #3 &\rightarrow& #4\\
#5 &\mapsto& #6
\end{array}
}}

\newcommand{\eqcolon}{\mathrel{\resizebox{\widthof{$\mathord{=}$}}{\height}{ $\!\!=\!\!\resizebox{1.2\width}{0.8\height}{\raisebox{0.23ex}{$\mathop{:}$}}\!\!$ }}}
\newcommand{\coloneq}{\mathrel{\resizebox{\widthof{$\mathord{=}$}}{\height}{ $\!\!\resizebox{1.2\width}{0.8\height}{\raisebox{0.23ex}{$\mathop{:}$}}\!\!=\!\!$ }}}
\newcommand{\eqcolonl}{\ensuremath{\mathrel{=\!\!\mathop{:}}}}
\newcommand{\coloneql}{\ensuremath{\mathrel{\mathop{:} \!\! =}}}
\newcommand{\vc}[1]{% inline column vector
  \left(\begin{smallmatrix}#1\end{smallmatrix}\right)%
}
\newcommand{\vr}[1]{% inline row vector
  \begin{smallmatrix}(\,#1\,)\end{smallmatrix}%
}
\makeatletter
\newcommand*{\defeq}{\ =\mathrel{\rlap{%
                     \raisebox{0.3ex}{$\m@th\cdot$}}%
                     \raisebox{-0.3ex}{$\m@th\cdot$}}%
                     }
\makeatother

\newcommand{\mathcircle}[1]{% inline row vector
 \overset{\circ}{#1}
}
\newcommand{\ulim}{% low limit
 \underline{\lim}
}
\newcommand{\ssi}{% iff
\iff
}
\newcommand{\ps}[2]{
\expval{#1 | #2}
}
\newcommand{\df}[1]{
\mqty{#1}
}
\newcommand{\n}[1]{
\norm{#1}
}
\newcommand{\sys}[1]{
\left\{\smqty{#1}\right.
}


\newcommand{\eqdef}{\ensuremath{\overset{\text{def}}=}}


\def\Circlearrowright{\ensuremath{%
  \rotatebox[origin=c]{230}{$\circlearrowright$}}}

\newcommand\ct[1]{\text{\rmfamily\upshape #1}}
\newcommand\question[1]{ {\color{red} ...!? \small #1}}
\newcommand\caz[1]{\left\{\begin{array} #1 \end{array}\right.}
\newcommand\const{\text{\rmfamily\upshape const}}
\newcommand\toP{ \overset{\pro}{\to}}
\newcommand\toPP{ \overset{\text{PP}}{\to}}
\newcommand{\oeq}{\mathrel{\text{\textcircled{$=$}}}}





\usepackage{xcolor}
% \usepackage[normalem]{ulem}
\usepackage{lipsum}
\makeatletter
% \newcommand\colorwave[1][blue]{\bgroup \markoverwith{\lower3.5\p@\hbox{\sixly \textcolor{#1}{\char58}}}\ULon}
%\font\sixly=lasy6 % does not re-load if already loaded, so no memory problem.

\newmdtheoremenv[
linewidth= 1pt,linecolor= blue,%
leftmargin=20,rightmargin=20,innertopmargin=0pt, innerrightmargin=40,%
tikzsetting = { draw=lightgray, line width = 0.3pt,dashed,%
dash pattern = on 15pt off 3pt},%
splittopskip=\topskip,skipbelow=\baselineskip,%
skipabove=\baselineskip,ntheorem,roundcorner=0pt,
% backgroundcolor=pagebg,font=\color{orange}\sffamily, fontcolor=white
]{examplebox}{Exemple}[section]



\newcommand\R{\mathbb{R}}
\newcommand\Z{\mathbb{Z}}
\newcommand\N{\mathbb{N}}
\newcommand\E{\mathbb{E}}
\newcommand\F{\mathcal{F}}
\newcommand\cH{\mathcal{H}}
\newcommand\V{\mathbb{V}}
\newcommand\dmo{ ^{-1} }
\newcommand\kapa{\kappa}
\newcommand\im{Im}
\newcommand\hs{\mathcal{H}}





\usepackage{soul}

\makeatletter
\newcommand*{\whiten}[1]{\llap{\textcolor{white}{{\the\SOUL@token}}\hspace{#1pt}}}
\DeclareRobustCommand*\myul{%
    \def\SOUL@everyspace{\underline{\space}\kern\z@}%
    \def\SOUL@everytoken{%
     \setbox0=\hbox{\the\SOUL@token}%
     \ifdim\dp0>\z@
        \raisebox{\dp0}{\underline{\phantom{\the\SOUL@token}}}%
        \whiten{1}\whiten{0}%
        \whiten{-1}\whiten{-2}%
        \llap{\the\SOUL@token}%
     \else
        \underline{\the\SOUL@token}%
     \fi}%
\SOUL@}
\makeatother

\newcommand*{\demp}{\fontfamily{lmtt}\selectfont}

\DeclareTextFontCommand{\textdemp}{\demp}

\begin{document}

\ifcomment
Multiline
comment
\fi
\ifcomment
\myul{Typesetting test}
% \color[rgb]{1,1,1}
$∑_i^n≠ 60º±∞π∆¬≈√j∫h≤≥µ$

$\CR \R\pro\ind\pro\gS\pro
\mqty[a&b\\c&d]$
$\pro\mathbb{P}$
$\dd{x}$

  \[
    \alpha(x)=\left\{
                \begin{array}{ll}
                  x\\
                  \frac{1}{1+e^{-kx}}\\
                  \frac{e^x-e^{-x}}{e^x+e^{-x}}
                \end{array}
              \right.
  \]

  $\expval{x}$
  
  $\chi_\rho(ghg\dmo)=\Tr(\rho_{ghg\dmo})=\Tr(\rho_g\circ\rho_h\circ\rho\dmo_g)=\Tr(\rho_h)\overset{\mbox{\scalebox{0.5}{$\Tr(AB)=\Tr(BA)$}}}{=}\chi_\rho(h)$
  	$\mathop{\oplus}_{\substack{x\in X}}$

$\mat(\rho_g)=(a_{ij}(g))_{\scriptsize \substack{1\leq i\leq d \\ 1\leq j\leq d}}$ et $\mat(\rho'_g)=(a'_{ij}(g))_{\scriptsize \substack{1\leq i'\leq d' \\ 1\leq j'\leq d'}}$



\[\int_a^b{\mathbb{R}^2}g(u, v)\dd{P_{XY}}(u, v)=\iint g(u,v) f_{XY}(u, v)\dd \lambda(u) \dd \lambda(v)\]
$$\lim_{x\to\infty} f(x)$$	
$$\iiiint_V \mu(t,u,v,w) \,dt\,du\,dv\,dw$$
$$\sum_{n=1}^{\infty} 2^{-n} = 1$$	
\begin{definition}
	Si $X$ et $Y$ sont 2 v.a. ou definit la \textsc{Covariance} entre $X$ et $Y$ comme
	$\cov(X,Y)\overset{\text{def}}{=}\E\left[(X-\E(X))(Y-\E(Y))\right]=\E(XY)-\E(X)\E(Y)$.
\end{definition}
\fi
\pagebreak

% \tableofcontents

% insert your code here
%\input{./algebra/main.tex}
%\input{./geometrie-differentielle/main.tex}
%\input{./probabilite/main.tex}
%\input{./analyse-fonctionnelle/main.tex}
% \input{./Analyse-convexe-et-dualite-en-optimisation/main.tex}
%\input{./tikz/main.tex}
%\input{./Theorie-du-distributions/main.tex}
%\input{./optimisation/mine.tex}
 \input{./modelisation/main.tex}

% yves.aubry@univ-tln.fr : algebra

\end{document}

%% !TEX encoding = UTF-8 Unicode
% !TEX TS-program = xelatex

\documentclass[french]{report}

%\usepackage[utf8]{inputenc}
%\usepackage[T1]{fontenc}
\usepackage{babel}


\newif\ifcomment
%\commenttrue # Show comments

\usepackage{physics}
\usepackage{amssymb}


\usepackage{amsthm}
% \usepackage{thmtools}
\usepackage{mathtools}
\usepackage{amsfonts}

\usepackage{color}

\usepackage{tikz}

\usepackage{geometry}
\geometry{a5paper, margin=0.1in, right=1cm}

\usepackage{dsfont}

\usepackage{graphicx}
\graphicspath{ {images/} }

\usepackage{faktor}

\usepackage{IEEEtrantools}
\usepackage{enumerate}   
\usepackage[PostScript=dvips]{"/Users/aware/Documents/Courses/diagrams"}


\newtheorem{theorem}{Théorème}[section]
\renewcommand{\thetheorem}{\arabic{theorem}}
\newtheorem{lemme}{Lemme}[section]
\renewcommand{\thelemme}{\arabic{lemme}}
\newtheorem{proposition}{Proposition}[section]
\renewcommand{\theproposition}{\arabic{proposition}}
\newtheorem{notations}{Notations}[section]
\newtheorem{problem}{Problème}[section]
\newtheorem{corollary}{Corollaire}[theorem]
\renewcommand{\thecorollary}{\arabic{corollary}}
\newtheorem{property}{Propriété}[section]
\newtheorem{objective}{Objectif}[section]

\theoremstyle{definition}
\newtheorem{definition}{Définition}[section]
\renewcommand{\thedefinition}{\arabic{definition}}
\newtheorem{exercise}{Exercice}[chapter]
\renewcommand{\theexercise}{\arabic{exercise}}
\newtheorem{example}{Exemple}[chapter]
\renewcommand{\theexample}{\arabic{example}}
\newtheorem*{solution}{Solution}
\newtheorem*{application}{Application}
\newtheorem*{notation}{Notation}
\newtheorem*{vocabulary}{Vocabulaire}
\newtheorem*{properties}{Propriétés}



\theoremstyle{remark}
\newtheorem*{remark}{Remarque}
\newtheorem*{rappel}{Rappel}


\usepackage{etoolbox}
\AtBeginEnvironment{exercise}{\small}
\AtBeginEnvironment{example}{\small}

\usepackage{cases}
\usepackage[red]{mypack}

\usepackage[framemethod=TikZ]{mdframed}

\definecolor{bg}{rgb}{0.4,0.25,0.95}
\definecolor{pagebg}{rgb}{0,0,0.5}
\surroundwithmdframed[
   topline=false,
   rightline=false,
   bottomline=false,
   leftmargin=\parindent,
   skipabove=8pt,
   skipbelow=8pt,
   linecolor=blue,
   innerbottommargin=10pt,
   % backgroundcolor=bg,font=\color{orange}\sffamily, fontcolor=white
]{definition}

\usepackage{empheq}
\usepackage[most]{tcolorbox}

\newtcbox{\mymath}[1][]{%
    nobeforeafter, math upper, tcbox raise base,
    enhanced, colframe=blue!30!black,
    colback=red!10, boxrule=1pt,
    #1}

\usepackage{unixode}


\DeclareMathOperator{\ord}{ord}
\DeclareMathOperator{\orb}{orb}
\DeclareMathOperator{\stab}{stab}
\DeclareMathOperator{\Stab}{stab}
\DeclareMathOperator{\ppcm}{ppcm}
\DeclareMathOperator{\conj}{Conj}
\DeclareMathOperator{\End}{End}
\DeclareMathOperator{\rot}{rot}
\DeclareMathOperator{\trs}{trace}
\DeclareMathOperator{\Ind}{Ind}
\DeclareMathOperator{\mat}{Mat}
\DeclareMathOperator{\id}{Id}
\DeclareMathOperator{\vect}{vect}
\DeclareMathOperator{\img}{img}
\DeclareMathOperator{\cov}{Cov}
\DeclareMathOperator{\dist}{dist}
\DeclareMathOperator{\irr}{Irr}
\DeclareMathOperator{\image}{Im}
\DeclareMathOperator{\pd}{\partial}
\DeclareMathOperator{\epi}{epi}
\DeclareMathOperator{\Argmin}{Argmin}
\DeclareMathOperator{\dom}{dom}
\DeclareMathOperator{\proj}{proj}
\DeclareMathOperator{\ctg}{ctg}
\DeclareMathOperator{\supp}{supp}
\DeclareMathOperator{\argmin}{argmin}
\DeclareMathOperator{\mult}{mult}
\DeclareMathOperator{\ch}{ch}
\DeclareMathOperator{\sh}{sh}
\DeclareMathOperator{\rang}{rang}
\DeclareMathOperator{\diam}{diam}
\DeclareMathOperator{\Epigraphe}{Epigraphe}




\usepackage{xcolor}
\everymath{\color{blue}}
%\everymath{\color[rgb]{0,1,1}}
%\pagecolor[rgb]{0,0,0.5}


\newcommand*{\pdtest}[3][]{\ensuremath{\frac{\partial^{#1} #2}{\partial #3}}}

\newcommand*{\deffunc}[6][]{\ensuremath{
\begin{array}{rcl}
#2 : #3 &\rightarrow& #4\\
#5 &\mapsto& #6
\end{array}
}}

\newcommand{\eqcolon}{\mathrel{\resizebox{\widthof{$\mathord{=}$}}{\height}{ $\!\!=\!\!\resizebox{1.2\width}{0.8\height}{\raisebox{0.23ex}{$\mathop{:}$}}\!\!$ }}}
\newcommand{\coloneq}{\mathrel{\resizebox{\widthof{$\mathord{=}$}}{\height}{ $\!\!\resizebox{1.2\width}{0.8\height}{\raisebox{0.23ex}{$\mathop{:}$}}\!\!=\!\!$ }}}
\newcommand{\eqcolonl}{\ensuremath{\mathrel{=\!\!\mathop{:}}}}
\newcommand{\coloneql}{\ensuremath{\mathrel{\mathop{:} \!\! =}}}
\newcommand{\vc}[1]{% inline column vector
  \left(\begin{smallmatrix}#1\end{smallmatrix}\right)%
}
\newcommand{\vr}[1]{% inline row vector
  \begin{smallmatrix}(\,#1\,)\end{smallmatrix}%
}
\makeatletter
\newcommand*{\defeq}{\ =\mathrel{\rlap{%
                     \raisebox{0.3ex}{$\m@th\cdot$}}%
                     \raisebox{-0.3ex}{$\m@th\cdot$}}%
                     }
\makeatother

\newcommand{\mathcircle}[1]{% inline row vector
 \overset{\circ}{#1}
}
\newcommand{\ulim}{% low limit
 \underline{\lim}
}
\newcommand{\ssi}{% iff
\iff
}
\newcommand{\ps}[2]{
\expval{#1 | #2}
}
\newcommand{\df}[1]{
\mqty{#1}
}
\newcommand{\n}[1]{
\norm{#1}
}
\newcommand{\sys}[1]{
\left\{\smqty{#1}\right.
}


\newcommand{\eqdef}{\ensuremath{\overset{\text{def}}=}}


\def\Circlearrowright{\ensuremath{%
  \rotatebox[origin=c]{230}{$\circlearrowright$}}}

\newcommand\ct[1]{\text{\rmfamily\upshape #1}}
\newcommand\question[1]{ {\color{red} ...!? \small #1}}
\newcommand\caz[1]{\left\{\begin{array} #1 \end{array}\right.}
\newcommand\const{\text{\rmfamily\upshape const}}
\newcommand\toP{ \overset{\pro}{\to}}
\newcommand\toPP{ \overset{\text{PP}}{\to}}
\newcommand{\oeq}{\mathrel{\text{\textcircled{$=$}}}}





\usepackage{xcolor}
% \usepackage[normalem]{ulem}
\usepackage{lipsum}
\makeatletter
% \newcommand\colorwave[1][blue]{\bgroup \markoverwith{\lower3.5\p@\hbox{\sixly \textcolor{#1}{\char58}}}\ULon}
%\font\sixly=lasy6 % does not re-load if already loaded, so no memory problem.

\newmdtheoremenv[
linewidth= 1pt,linecolor= blue,%
leftmargin=20,rightmargin=20,innertopmargin=0pt, innerrightmargin=40,%
tikzsetting = { draw=lightgray, line width = 0.3pt,dashed,%
dash pattern = on 15pt off 3pt},%
splittopskip=\topskip,skipbelow=\baselineskip,%
skipabove=\baselineskip,ntheorem,roundcorner=0pt,
% backgroundcolor=pagebg,font=\color{orange}\sffamily, fontcolor=white
]{examplebox}{Exemple}[section]



\newcommand\R{\mathbb{R}}
\newcommand\Z{\mathbb{Z}}
\newcommand\N{\mathbb{N}}
\newcommand\E{\mathbb{E}}
\newcommand\F{\mathcal{F}}
\newcommand\cH{\mathcal{H}}
\newcommand\V{\mathbb{V}}
\newcommand\dmo{ ^{-1} }
\newcommand\kapa{\kappa}
\newcommand\im{Im}
\newcommand\hs{\mathcal{H}}





\usepackage{soul}

\makeatletter
\newcommand*{\whiten}[1]{\llap{\textcolor{white}{{\the\SOUL@token}}\hspace{#1pt}}}
\DeclareRobustCommand*\myul{%
    \def\SOUL@everyspace{\underline{\space}\kern\z@}%
    \def\SOUL@everytoken{%
     \setbox0=\hbox{\the\SOUL@token}%
     \ifdim\dp0>\z@
        \raisebox{\dp0}{\underline{\phantom{\the\SOUL@token}}}%
        \whiten{1}\whiten{0}%
        \whiten{-1}\whiten{-2}%
        \llap{\the\SOUL@token}%
     \else
        \underline{\the\SOUL@token}%
     \fi}%
\SOUL@}
\makeatother

\newcommand*{\demp}{\fontfamily{lmtt}\selectfont}

\DeclareTextFontCommand{\textdemp}{\demp}

\begin{document}

\ifcomment
Multiline
comment
\fi
\ifcomment
\myul{Typesetting test}
% \color[rgb]{1,1,1}
$∑_i^n≠ 60º±∞π∆¬≈√j∫h≤≥µ$

$\CR \R\pro\ind\pro\gS\pro
\mqty[a&b\\c&d]$
$\pro\mathbb{P}$
$\dd{x}$

  \[
    \alpha(x)=\left\{
                \begin{array}{ll}
                  x\\
                  \frac{1}{1+e^{-kx}}\\
                  \frac{e^x-e^{-x}}{e^x+e^{-x}}
                \end{array}
              \right.
  \]

  $\expval{x}$
  
  $\chi_\rho(ghg\dmo)=\Tr(\rho_{ghg\dmo})=\Tr(\rho_g\circ\rho_h\circ\rho\dmo_g)=\Tr(\rho_h)\overset{\mbox{\scalebox{0.5}{$\Tr(AB)=\Tr(BA)$}}}{=}\chi_\rho(h)$
  	$\mathop{\oplus}_{\substack{x\in X}}$

$\mat(\rho_g)=(a_{ij}(g))_{\scriptsize \substack{1\leq i\leq d \\ 1\leq j\leq d}}$ et $\mat(\rho'_g)=(a'_{ij}(g))_{\scriptsize \substack{1\leq i'\leq d' \\ 1\leq j'\leq d'}}$



\[\int_a^b{\mathbb{R}^2}g(u, v)\dd{P_{XY}}(u, v)=\iint g(u,v) f_{XY}(u, v)\dd \lambda(u) \dd \lambda(v)\]
$$\lim_{x\to\infty} f(x)$$	
$$\iiiint_V \mu(t,u,v,w) \,dt\,du\,dv\,dw$$
$$\sum_{n=1}^{\infty} 2^{-n} = 1$$	
\begin{definition}
	Si $X$ et $Y$ sont 2 v.a. ou definit la \textsc{Covariance} entre $X$ et $Y$ comme
	$\cov(X,Y)\overset{\text{def}}{=}\E\left[(X-\E(X))(Y-\E(Y))\right]=\E(XY)-\E(X)\E(Y)$.
\end{definition}
\fi
\pagebreak

% \tableofcontents

% insert your code here
%\input{./algebra/main.tex}
%\input{./geometrie-differentielle/main.tex}
%\input{./probabilite/main.tex}
%\input{./analyse-fonctionnelle/main.tex}
% \input{./Analyse-convexe-et-dualite-en-optimisation/main.tex}
%\input{./tikz/main.tex}
%\input{./Theorie-du-distributions/main.tex}
%\input{./optimisation/mine.tex}
 \input{./modelisation/main.tex}

% yves.aubry@univ-tln.fr : algebra

\end{document}

%% !TEX encoding = UTF-8 Unicode
% !TEX TS-program = xelatex

\documentclass[french]{report}

%\usepackage[utf8]{inputenc}
%\usepackage[T1]{fontenc}
\usepackage{babel}


\newif\ifcomment
%\commenttrue # Show comments

\usepackage{physics}
\usepackage{amssymb}


\usepackage{amsthm}
% \usepackage{thmtools}
\usepackage{mathtools}
\usepackage{amsfonts}

\usepackage{color}

\usepackage{tikz}

\usepackage{geometry}
\geometry{a5paper, margin=0.1in, right=1cm}

\usepackage{dsfont}

\usepackage{graphicx}
\graphicspath{ {images/} }

\usepackage{faktor}

\usepackage{IEEEtrantools}
\usepackage{enumerate}   
\usepackage[PostScript=dvips]{"/Users/aware/Documents/Courses/diagrams"}


\newtheorem{theorem}{Théorème}[section]
\renewcommand{\thetheorem}{\arabic{theorem}}
\newtheorem{lemme}{Lemme}[section]
\renewcommand{\thelemme}{\arabic{lemme}}
\newtheorem{proposition}{Proposition}[section]
\renewcommand{\theproposition}{\arabic{proposition}}
\newtheorem{notations}{Notations}[section]
\newtheorem{problem}{Problème}[section]
\newtheorem{corollary}{Corollaire}[theorem]
\renewcommand{\thecorollary}{\arabic{corollary}}
\newtheorem{property}{Propriété}[section]
\newtheorem{objective}{Objectif}[section]

\theoremstyle{definition}
\newtheorem{definition}{Définition}[section]
\renewcommand{\thedefinition}{\arabic{definition}}
\newtheorem{exercise}{Exercice}[chapter]
\renewcommand{\theexercise}{\arabic{exercise}}
\newtheorem{example}{Exemple}[chapter]
\renewcommand{\theexample}{\arabic{example}}
\newtheorem*{solution}{Solution}
\newtheorem*{application}{Application}
\newtheorem*{notation}{Notation}
\newtheorem*{vocabulary}{Vocabulaire}
\newtheorem*{properties}{Propriétés}



\theoremstyle{remark}
\newtheorem*{remark}{Remarque}
\newtheorem*{rappel}{Rappel}


\usepackage{etoolbox}
\AtBeginEnvironment{exercise}{\small}
\AtBeginEnvironment{example}{\small}

\usepackage{cases}
\usepackage[red]{mypack}

\usepackage[framemethod=TikZ]{mdframed}

\definecolor{bg}{rgb}{0.4,0.25,0.95}
\definecolor{pagebg}{rgb}{0,0,0.5}
\surroundwithmdframed[
   topline=false,
   rightline=false,
   bottomline=false,
   leftmargin=\parindent,
   skipabove=8pt,
   skipbelow=8pt,
   linecolor=blue,
   innerbottommargin=10pt,
   % backgroundcolor=bg,font=\color{orange}\sffamily, fontcolor=white
]{definition}

\usepackage{empheq}
\usepackage[most]{tcolorbox}

\newtcbox{\mymath}[1][]{%
    nobeforeafter, math upper, tcbox raise base,
    enhanced, colframe=blue!30!black,
    colback=red!10, boxrule=1pt,
    #1}

\usepackage{unixode}


\DeclareMathOperator{\ord}{ord}
\DeclareMathOperator{\orb}{orb}
\DeclareMathOperator{\stab}{stab}
\DeclareMathOperator{\Stab}{stab}
\DeclareMathOperator{\ppcm}{ppcm}
\DeclareMathOperator{\conj}{Conj}
\DeclareMathOperator{\End}{End}
\DeclareMathOperator{\rot}{rot}
\DeclareMathOperator{\trs}{trace}
\DeclareMathOperator{\Ind}{Ind}
\DeclareMathOperator{\mat}{Mat}
\DeclareMathOperator{\id}{Id}
\DeclareMathOperator{\vect}{vect}
\DeclareMathOperator{\img}{img}
\DeclareMathOperator{\cov}{Cov}
\DeclareMathOperator{\dist}{dist}
\DeclareMathOperator{\irr}{Irr}
\DeclareMathOperator{\image}{Im}
\DeclareMathOperator{\pd}{\partial}
\DeclareMathOperator{\epi}{epi}
\DeclareMathOperator{\Argmin}{Argmin}
\DeclareMathOperator{\dom}{dom}
\DeclareMathOperator{\proj}{proj}
\DeclareMathOperator{\ctg}{ctg}
\DeclareMathOperator{\supp}{supp}
\DeclareMathOperator{\argmin}{argmin}
\DeclareMathOperator{\mult}{mult}
\DeclareMathOperator{\ch}{ch}
\DeclareMathOperator{\sh}{sh}
\DeclareMathOperator{\rang}{rang}
\DeclareMathOperator{\diam}{diam}
\DeclareMathOperator{\Epigraphe}{Epigraphe}




\usepackage{xcolor}
\everymath{\color{blue}}
%\everymath{\color[rgb]{0,1,1}}
%\pagecolor[rgb]{0,0,0.5}


\newcommand*{\pdtest}[3][]{\ensuremath{\frac{\partial^{#1} #2}{\partial #3}}}

\newcommand*{\deffunc}[6][]{\ensuremath{
\begin{array}{rcl}
#2 : #3 &\rightarrow& #4\\
#5 &\mapsto& #6
\end{array}
}}

\newcommand{\eqcolon}{\mathrel{\resizebox{\widthof{$\mathord{=}$}}{\height}{ $\!\!=\!\!\resizebox{1.2\width}{0.8\height}{\raisebox{0.23ex}{$\mathop{:}$}}\!\!$ }}}
\newcommand{\coloneq}{\mathrel{\resizebox{\widthof{$\mathord{=}$}}{\height}{ $\!\!\resizebox{1.2\width}{0.8\height}{\raisebox{0.23ex}{$\mathop{:}$}}\!\!=\!\!$ }}}
\newcommand{\eqcolonl}{\ensuremath{\mathrel{=\!\!\mathop{:}}}}
\newcommand{\coloneql}{\ensuremath{\mathrel{\mathop{:} \!\! =}}}
\newcommand{\vc}[1]{% inline column vector
  \left(\begin{smallmatrix}#1\end{smallmatrix}\right)%
}
\newcommand{\vr}[1]{% inline row vector
  \begin{smallmatrix}(\,#1\,)\end{smallmatrix}%
}
\makeatletter
\newcommand*{\defeq}{\ =\mathrel{\rlap{%
                     \raisebox{0.3ex}{$\m@th\cdot$}}%
                     \raisebox{-0.3ex}{$\m@th\cdot$}}%
                     }
\makeatother

\newcommand{\mathcircle}[1]{% inline row vector
 \overset{\circ}{#1}
}
\newcommand{\ulim}{% low limit
 \underline{\lim}
}
\newcommand{\ssi}{% iff
\iff
}
\newcommand{\ps}[2]{
\expval{#1 | #2}
}
\newcommand{\df}[1]{
\mqty{#1}
}
\newcommand{\n}[1]{
\norm{#1}
}
\newcommand{\sys}[1]{
\left\{\smqty{#1}\right.
}


\newcommand{\eqdef}{\ensuremath{\overset{\text{def}}=}}


\def\Circlearrowright{\ensuremath{%
  \rotatebox[origin=c]{230}{$\circlearrowright$}}}

\newcommand\ct[1]{\text{\rmfamily\upshape #1}}
\newcommand\question[1]{ {\color{red} ...!? \small #1}}
\newcommand\caz[1]{\left\{\begin{array} #1 \end{array}\right.}
\newcommand\const{\text{\rmfamily\upshape const}}
\newcommand\toP{ \overset{\pro}{\to}}
\newcommand\toPP{ \overset{\text{PP}}{\to}}
\newcommand{\oeq}{\mathrel{\text{\textcircled{$=$}}}}





\usepackage{xcolor}
% \usepackage[normalem]{ulem}
\usepackage{lipsum}
\makeatletter
% \newcommand\colorwave[1][blue]{\bgroup \markoverwith{\lower3.5\p@\hbox{\sixly \textcolor{#1}{\char58}}}\ULon}
%\font\sixly=lasy6 % does not re-load if already loaded, so no memory problem.

\newmdtheoremenv[
linewidth= 1pt,linecolor= blue,%
leftmargin=20,rightmargin=20,innertopmargin=0pt, innerrightmargin=40,%
tikzsetting = { draw=lightgray, line width = 0.3pt,dashed,%
dash pattern = on 15pt off 3pt},%
splittopskip=\topskip,skipbelow=\baselineskip,%
skipabove=\baselineskip,ntheorem,roundcorner=0pt,
% backgroundcolor=pagebg,font=\color{orange}\sffamily, fontcolor=white
]{examplebox}{Exemple}[section]



\newcommand\R{\mathbb{R}}
\newcommand\Z{\mathbb{Z}}
\newcommand\N{\mathbb{N}}
\newcommand\E{\mathbb{E}}
\newcommand\F{\mathcal{F}}
\newcommand\cH{\mathcal{H}}
\newcommand\V{\mathbb{V}}
\newcommand\dmo{ ^{-1} }
\newcommand\kapa{\kappa}
\newcommand\im{Im}
\newcommand\hs{\mathcal{H}}





\usepackage{soul}

\makeatletter
\newcommand*{\whiten}[1]{\llap{\textcolor{white}{{\the\SOUL@token}}\hspace{#1pt}}}
\DeclareRobustCommand*\myul{%
    \def\SOUL@everyspace{\underline{\space}\kern\z@}%
    \def\SOUL@everytoken{%
     \setbox0=\hbox{\the\SOUL@token}%
     \ifdim\dp0>\z@
        \raisebox{\dp0}{\underline{\phantom{\the\SOUL@token}}}%
        \whiten{1}\whiten{0}%
        \whiten{-1}\whiten{-2}%
        \llap{\the\SOUL@token}%
     \else
        \underline{\the\SOUL@token}%
     \fi}%
\SOUL@}
\makeatother

\newcommand*{\demp}{\fontfamily{lmtt}\selectfont}

\DeclareTextFontCommand{\textdemp}{\demp}

\begin{document}

\ifcomment
Multiline
comment
\fi
\ifcomment
\myul{Typesetting test}
% \color[rgb]{1,1,1}
$∑_i^n≠ 60º±∞π∆¬≈√j∫h≤≥µ$

$\CR \R\pro\ind\pro\gS\pro
\mqty[a&b\\c&d]$
$\pro\mathbb{P}$
$\dd{x}$

  \[
    \alpha(x)=\left\{
                \begin{array}{ll}
                  x\\
                  \frac{1}{1+e^{-kx}}\\
                  \frac{e^x-e^{-x}}{e^x+e^{-x}}
                \end{array}
              \right.
  \]

  $\expval{x}$
  
  $\chi_\rho(ghg\dmo)=\Tr(\rho_{ghg\dmo})=\Tr(\rho_g\circ\rho_h\circ\rho\dmo_g)=\Tr(\rho_h)\overset{\mbox{\scalebox{0.5}{$\Tr(AB)=\Tr(BA)$}}}{=}\chi_\rho(h)$
  	$\mathop{\oplus}_{\substack{x\in X}}$

$\mat(\rho_g)=(a_{ij}(g))_{\scriptsize \substack{1\leq i\leq d \\ 1\leq j\leq d}}$ et $\mat(\rho'_g)=(a'_{ij}(g))_{\scriptsize \substack{1\leq i'\leq d' \\ 1\leq j'\leq d'}}$



\[\int_a^b{\mathbb{R}^2}g(u, v)\dd{P_{XY}}(u, v)=\iint g(u,v) f_{XY}(u, v)\dd \lambda(u) \dd \lambda(v)\]
$$\lim_{x\to\infty} f(x)$$	
$$\iiiint_V \mu(t,u,v,w) \,dt\,du\,dv\,dw$$
$$\sum_{n=1}^{\infty} 2^{-n} = 1$$	
\begin{definition}
	Si $X$ et $Y$ sont 2 v.a. ou definit la \textsc{Covariance} entre $X$ et $Y$ comme
	$\cov(X,Y)\overset{\text{def}}{=}\E\left[(X-\E(X))(Y-\E(Y))\right]=\E(XY)-\E(X)\E(Y)$.
\end{definition}
\fi
\pagebreak

% \tableofcontents

% insert your code here
%\input{./algebra/main.tex}
%\input{./geometrie-differentielle/main.tex}
%\input{./probabilite/main.tex}
%\input{./analyse-fonctionnelle/main.tex}
% \input{./Analyse-convexe-et-dualite-en-optimisation/main.tex}
%\input{./tikz/main.tex}
%\input{./Theorie-du-distributions/main.tex}
%\input{./optimisation/mine.tex}
 \input{./modelisation/main.tex}

% yves.aubry@univ-tln.fr : algebra

\end{document}

% % !TEX encoding = UTF-8 Unicode
% !TEX TS-program = xelatex

\documentclass[french]{report}

%\usepackage[utf8]{inputenc}
%\usepackage[T1]{fontenc}
\usepackage{babel}


\newif\ifcomment
%\commenttrue # Show comments

\usepackage{physics}
\usepackage{amssymb}


\usepackage{amsthm}
% \usepackage{thmtools}
\usepackage{mathtools}
\usepackage{amsfonts}

\usepackage{color}

\usepackage{tikz}

\usepackage{geometry}
\geometry{a5paper, margin=0.1in, right=1cm}

\usepackage{dsfont}

\usepackage{graphicx}
\graphicspath{ {images/} }

\usepackage{faktor}

\usepackage{IEEEtrantools}
\usepackage{enumerate}   
\usepackage[PostScript=dvips]{"/Users/aware/Documents/Courses/diagrams"}


\newtheorem{theorem}{Théorème}[section]
\renewcommand{\thetheorem}{\arabic{theorem}}
\newtheorem{lemme}{Lemme}[section]
\renewcommand{\thelemme}{\arabic{lemme}}
\newtheorem{proposition}{Proposition}[section]
\renewcommand{\theproposition}{\arabic{proposition}}
\newtheorem{notations}{Notations}[section]
\newtheorem{problem}{Problème}[section]
\newtheorem{corollary}{Corollaire}[theorem]
\renewcommand{\thecorollary}{\arabic{corollary}}
\newtheorem{property}{Propriété}[section]
\newtheorem{objective}{Objectif}[section]

\theoremstyle{definition}
\newtheorem{definition}{Définition}[section]
\renewcommand{\thedefinition}{\arabic{definition}}
\newtheorem{exercise}{Exercice}[chapter]
\renewcommand{\theexercise}{\arabic{exercise}}
\newtheorem{example}{Exemple}[chapter]
\renewcommand{\theexample}{\arabic{example}}
\newtheorem*{solution}{Solution}
\newtheorem*{application}{Application}
\newtheorem*{notation}{Notation}
\newtheorem*{vocabulary}{Vocabulaire}
\newtheorem*{properties}{Propriétés}



\theoremstyle{remark}
\newtheorem*{remark}{Remarque}
\newtheorem*{rappel}{Rappel}


\usepackage{etoolbox}
\AtBeginEnvironment{exercise}{\small}
\AtBeginEnvironment{example}{\small}

\usepackage{cases}
\usepackage[red]{mypack}

\usepackage[framemethod=TikZ]{mdframed}

\definecolor{bg}{rgb}{0.4,0.25,0.95}
\definecolor{pagebg}{rgb}{0,0,0.5}
\surroundwithmdframed[
   topline=false,
   rightline=false,
   bottomline=false,
   leftmargin=\parindent,
   skipabove=8pt,
   skipbelow=8pt,
   linecolor=blue,
   innerbottommargin=10pt,
   % backgroundcolor=bg,font=\color{orange}\sffamily, fontcolor=white
]{definition}

\usepackage{empheq}
\usepackage[most]{tcolorbox}

\newtcbox{\mymath}[1][]{%
    nobeforeafter, math upper, tcbox raise base,
    enhanced, colframe=blue!30!black,
    colback=red!10, boxrule=1pt,
    #1}

\usepackage{unixode}


\DeclareMathOperator{\ord}{ord}
\DeclareMathOperator{\orb}{orb}
\DeclareMathOperator{\stab}{stab}
\DeclareMathOperator{\Stab}{stab}
\DeclareMathOperator{\ppcm}{ppcm}
\DeclareMathOperator{\conj}{Conj}
\DeclareMathOperator{\End}{End}
\DeclareMathOperator{\rot}{rot}
\DeclareMathOperator{\trs}{trace}
\DeclareMathOperator{\Ind}{Ind}
\DeclareMathOperator{\mat}{Mat}
\DeclareMathOperator{\id}{Id}
\DeclareMathOperator{\vect}{vect}
\DeclareMathOperator{\img}{img}
\DeclareMathOperator{\cov}{Cov}
\DeclareMathOperator{\dist}{dist}
\DeclareMathOperator{\irr}{Irr}
\DeclareMathOperator{\image}{Im}
\DeclareMathOperator{\pd}{\partial}
\DeclareMathOperator{\epi}{epi}
\DeclareMathOperator{\Argmin}{Argmin}
\DeclareMathOperator{\dom}{dom}
\DeclareMathOperator{\proj}{proj}
\DeclareMathOperator{\ctg}{ctg}
\DeclareMathOperator{\supp}{supp}
\DeclareMathOperator{\argmin}{argmin}
\DeclareMathOperator{\mult}{mult}
\DeclareMathOperator{\ch}{ch}
\DeclareMathOperator{\sh}{sh}
\DeclareMathOperator{\rang}{rang}
\DeclareMathOperator{\diam}{diam}
\DeclareMathOperator{\Epigraphe}{Epigraphe}




\usepackage{xcolor}
\everymath{\color{blue}}
%\everymath{\color[rgb]{0,1,1}}
%\pagecolor[rgb]{0,0,0.5}


\newcommand*{\pdtest}[3][]{\ensuremath{\frac{\partial^{#1} #2}{\partial #3}}}

\newcommand*{\deffunc}[6][]{\ensuremath{
\begin{array}{rcl}
#2 : #3 &\rightarrow& #4\\
#5 &\mapsto& #6
\end{array}
}}

\newcommand{\eqcolon}{\mathrel{\resizebox{\widthof{$\mathord{=}$}}{\height}{ $\!\!=\!\!\resizebox{1.2\width}{0.8\height}{\raisebox{0.23ex}{$\mathop{:}$}}\!\!$ }}}
\newcommand{\coloneq}{\mathrel{\resizebox{\widthof{$\mathord{=}$}}{\height}{ $\!\!\resizebox{1.2\width}{0.8\height}{\raisebox{0.23ex}{$\mathop{:}$}}\!\!=\!\!$ }}}
\newcommand{\eqcolonl}{\ensuremath{\mathrel{=\!\!\mathop{:}}}}
\newcommand{\coloneql}{\ensuremath{\mathrel{\mathop{:} \!\! =}}}
\newcommand{\vc}[1]{% inline column vector
  \left(\begin{smallmatrix}#1\end{smallmatrix}\right)%
}
\newcommand{\vr}[1]{% inline row vector
  \begin{smallmatrix}(\,#1\,)\end{smallmatrix}%
}
\makeatletter
\newcommand*{\defeq}{\ =\mathrel{\rlap{%
                     \raisebox{0.3ex}{$\m@th\cdot$}}%
                     \raisebox{-0.3ex}{$\m@th\cdot$}}%
                     }
\makeatother

\newcommand{\mathcircle}[1]{% inline row vector
 \overset{\circ}{#1}
}
\newcommand{\ulim}{% low limit
 \underline{\lim}
}
\newcommand{\ssi}{% iff
\iff
}
\newcommand{\ps}[2]{
\expval{#1 | #2}
}
\newcommand{\df}[1]{
\mqty{#1}
}
\newcommand{\n}[1]{
\norm{#1}
}
\newcommand{\sys}[1]{
\left\{\smqty{#1}\right.
}


\newcommand{\eqdef}{\ensuremath{\overset{\text{def}}=}}


\def\Circlearrowright{\ensuremath{%
  \rotatebox[origin=c]{230}{$\circlearrowright$}}}

\newcommand\ct[1]{\text{\rmfamily\upshape #1}}
\newcommand\question[1]{ {\color{red} ...!? \small #1}}
\newcommand\caz[1]{\left\{\begin{array} #1 \end{array}\right.}
\newcommand\const{\text{\rmfamily\upshape const}}
\newcommand\toP{ \overset{\pro}{\to}}
\newcommand\toPP{ \overset{\text{PP}}{\to}}
\newcommand{\oeq}{\mathrel{\text{\textcircled{$=$}}}}





\usepackage{xcolor}
% \usepackage[normalem]{ulem}
\usepackage{lipsum}
\makeatletter
% \newcommand\colorwave[1][blue]{\bgroup \markoverwith{\lower3.5\p@\hbox{\sixly \textcolor{#1}{\char58}}}\ULon}
%\font\sixly=lasy6 % does not re-load if already loaded, so no memory problem.

\newmdtheoremenv[
linewidth= 1pt,linecolor= blue,%
leftmargin=20,rightmargin=20,innertopmargin=0pt, innerrightmargin=40,%
tikzsetting = { draw=lightgray, line width = 0.3pt,dashed,%
dash pattern = on 15pt off 3pt},%
splittopskip=\topskip,skipbelow=\baselineskip,%
skipabove=\baselineskip,ntheorem,roundcorner=0pt,
% backgroundcolor=pagebg,font=\color{orange}\sffamily, fontcolor=white
]{examplebox}{Exemple}[section]



\newcommand\R{\mathbb{R}}
\newcommand\Z{\mathbb{Z}}
\newcommand\N{\mathbb{N}}
\newcommand\E{\mathbb{E}}
\newcommand\F{\mathcal{F}}
\newcommand\cH{\mathcal{H}}
\newcommand\V{\mathbb{V}}
\newcommand\dmo{ ^{-1} }
\newcommand\kapa{\kappa}
\newcommand\im{Im}
\newcommand\hs{\mathcal{H}}





\usepackage{soul}

\makeatletter
\newcommand*{\whiten}[1]{\llap{\textcolor{white}{{\the\SOUL@token}}\hspace{#1pt}}}
\DeclareRobustCommand*\myul{%
    \def\SOUL@everyspace{\underline{\space}\kern\z@}%
    \def\SOUL@everytoken{%
     \setbox0=\hbox{\the\SOUL@token}%
     \ifdim\dp0>\z@
        \raisebox{\dp0}{\underline{\phantom{\the\SOUL@token}}}%
        \whiten{1}\whiten{0}%
        \whiten{-1}\whiten{-2}%
        \llap{\the\SOUL@token}%
     \else
        \underline{\the\SOUL@token}%
     \fi}%
\SOUL@}
\makeatother

\newcommand*{\demp}{\fontfamily{lmtt}\selectfont}

\DeclareTextFontCommand{\textdemp}{\demp}

\begin{document}

\ifcomment
Multiline
comment
\fi
\ifcomment
\myul{Typesetting test}
% \color[rgb]{1,1,1}
$∑_i^n≠ 60º±∞π∆¬≈√j∫h≤≥µ$

$\CR \R\pro\ind\pro\gS\pro
\mqty[a&b\\c&d]$
$\pro\mathbb{P}$
$\dd{x}$

  \[
    \alpha(x)=\left\{
                \begin{array}{ll}
                  x\\
                  \frac{1}{1+e^{-kx}}\\
                  \frac{e^x-e^{-x}}{e^x+e^{-x}}
                \end{array}
              \right.
  \]

  $\expval{x}$
  
  $\chi_\rho(ghg\dmo)=\Tr(\rho_{ghg\dmo})=\Tr(\rho_g\circ\rho_h\circ\rho\dmo_g)=\Tr(\rho_h)\overset{\mbox{\scalebox{0.5}{$\Tr(AB)=\Tr(BA)$}}}{=}\chi_\rho(h)$
  	$\mathop{\oplus}_{\substack{x\in X}}$

$\mat(\rho_g)=(a_{ij}(g))_{\scriptsize \substack{1\leq i\leq d \\ 1\leq j\leq d}}$ et $\mat(\rho'_g)=(a'_{ij}(g))_{\scriptsize \substack{1\leq i'\leq d' \\ 1\leq j'\leq d'}}$



\[\int_a^b{\mathbb{R}^2}g(u, v)\dd{P_{XY}}(u, v)=\iint g(u,v) f_{XY}(u, v)\dd \lambda(u) \dd \lambda(v)\]
$$\lim_{x\to\infty} f(x)$$	
$$\iiiint_V \mu(t,u,v,w) \,dt\,du\,dv\,dw$$
$$\sum_{n=1}^{\infty} 2^{-n} = 1$$	
\begin{definition}
	Si $X$ et $Y$ sont 2 v.a. ou definit la \textsc{Covariance} entre $X$ et $Y$ comme
	$\cov(X,Y)\overset{\text{def}}{=}\E\left[(X-\E(X))(Y-\E(Y))\right]=\E(XY)-\E(X)\E(Y)$.
\end{definition}
\fi
\pagebreak

% \tableofcontents

% insert your code here
%\input{./algebra/main.tex}
%\input{./geometrie-differentielle/main.tex}
%\input{./probabilite/main.tex}
%\input{./analyse-fonctionnelle/main.tex}
% \input{./Analyse-convexe-et-dualite-en-optimisation/main.tex}
%\input{./tikz/main.tex}
%\input{./Theorie-du-distributions/main.tex}
%\input{./optimisation/mine.tex}
 \input{./modelisation/main.tex}

% yves.aubry@univ-tln.fr : algebra

\end{document}

%% !TEX encoding = UTF-8 Unicode
% !TEX TS-program = xelatex

\documentclass[french]{report}

%\usepackage[utf8]{inputenc}
%\usepackage[T1]{fontenc}
\usepackage{babel}


\newif\ifcomment
%\commenttrue # Show comments

\usepackage{physics}
\usepackage{amssymb}


\usepackage{amsthm}
% \usepackage{thmtools}
\usepackage{mathtools}
\usepackage{amsfonts}

\usepackage{color}

\usepackage{tikz}

\usepackage{geometry}
\geometry{a5paper, margin=0.1in, right=1cm}

\usepackage{dsfont}

\usepackage{graphicx}
\graphicspath{ {images/} }

\usepackage{faktor}

\usepackage{IEEEtrantools}
\usepackage{enumerate}   
\usepackage[PostScript=dvips]{"/Users/aware/Documents/Courses/diagrams"}


\newtheorem{theorem}{Théorème}[section]
\renewcommand{\thetheorem}{\arabic{theorem}}
\newtheorem{lemme}{Lemme}[section]
\renewcommand{\thelemme}{\arabic{lemme}}
\newtheorem{proposition}{Proposition}[section]
\renewcommand{\theproposition}{\arabic{proposition}}
\newtheorem{notations}{Notations}[section]
\newtheorem{problem}{Problème}[section]
\newtheorem{corollary}{Corollaire}[theorem]
\renewcommand{\thecorollary}{\arabic{corollary}}
\newtheorem{property}{Propriété}[section]
\newtheorem{objective}{Objectif}[section]

\theoremstyle{definition}
\newtheorem{definition}{Définition}[section]
\renewcommand{\thedefinition}{\arabic{definition}}
\newtheorem{exercise}{Exercice}[chapter]
\renewcommand{\theexercise}{\arabic{exercise}}
\newtheorem{example}{Exemple}[chapter]
\renewcommand{\theexample}{\arabic{example}}
\newtheorem*{solution}{Solution}
\newtheorem*{application}{Application}
\newtheorem*{notation}{Notation}
\newtheorem*{vocabulary}{Vocabulaire}
\newtheorem*{properties}{Propriétés}



\theoremstyle{remark}
\newtheorem*{remark}{Remarque}
\newtheorem*{rappel}{Rappel}


\usepackage{etoolbox}
\AtBeginEnvironment{exercise}{\small}
\AtBeginEnvironment{example}{\small}

\usepackage{cases}
\usepackage[red]{mypack}

\usepackage[framemethod=TikZ]{mdframed}

\definecolor{bg}{rgb}{0.4,0.25,0.95}
\definecolor{pagebg}{rgb}{0,0,0.5}
\surroundwithmdframed[
   topline=false,
   rightline=false,
   bottomline=false,
   leftmargin=\parindent,
   skipabove=8pt,
   skipbelow=8pt,
   linecolor=blue,
   innerbottommargin=10pt,
   % backgroundcolor=bg,font=\color{orange}\sffamily, fontcolor=white
]{definition}

\usepackage{empheq}
\usepackage[most]{tcolorbox}

\newtcbox{\mymath}[1][]{%
    nobeforeafter, math upper, tcbox raise base,
    enhanced, colframe=blue!30!black,
    colback=red!10, boxrule=1pt,
    #1}

\usepackage{unixode}


\DeclareMathOperator{\ord}{ord}
\DeclareMathOperator{\orb}{orb}
\DeclareMathOperator{\stab}{stab}
\DeclareMathOperator{\Stab}{stab}
\DeclareMathOperator{\ppcm}{ppcm}
\DeclareMathOperator{\conj}{Conj}
\DeclareMathOperator{\End}{End}
\DeclareMathOperator{\rot}{rot}
\DeclareMathOperator{\trs}{trace}
\DeclareMathOperator{\Ind}{Ind}
\DeclareMathOperator{\mat}{Mat}
\DeclareMathOperator{\id}{Id}
\DeclareMathOperator{\vect}{vect}
\DeclareMathOperator{\img}{img}
\DeclareMathOperator{\cov}{Cov}
\DeclareMathOperator{\dist}{dist}
\DeclareMathOperator{\irr}{Irr}
\DeclareMathOperator{\image}{Im}
\DeclareMathOperator{\pd}{\partial}
\DeclareMathOperator{\epi}{epi}
\DeclareMathOperator{\Argmin}{Argmin}
\DeclareMathOperator{\dom}{dom}
\DeclareMathOperator{\proj}{proj}
\DeclareMathOperator{\ctg}{ctg}
\DeclareMathOperator{\supp}{supp}
\DeclareMathOperator{\argmin}{argmin}
\DeclareMathOperator{\mult}{mult}
\DeclareMathOperator{\ch}{ch}
\DeclareMathOperator{\sh}{sh}
\DeclareMathOperator{\rang}{rang}
\DeclareMathOperator{\diam}{diam}
\DeclareMathOperator{\Epigraphe}{Epigraphe}




\usepackage{xcolor}
\everymath{\color{blue}}
%\everymath{\color[rgb]{0,1,1}}
%\pagecolor[rgb]{0,0,0.5}


\newcommand*{\pdtest}[3][]{\ensuremath{\frac{\partial^{#1} #2}{\partial #3}}}

\newcommand*{\deffunc}[6][]{\ensuremath{
\begin{array}{rcl}
#2 : #3 &\rightarrow& #4\\
#5 &\mapsto& #6
\end{array}
}}

\newcommand{\eqcolon}{\mathrel{\resizebox{\widthof{$\mathord{=}$}}{\height}{ $\!\!=\!\!\resizebox{1.2\width}{0.8\height}{\raisebox{0.23ex}{$\mathop{:}$}}\!\!$ }}}
\newcommand{\coloneq}{\mathrel{\resizebox{\widthof{$\mathord{=}$}}{\height}{ $\!\!\resizebox{1.2\width}{0.8\height}{\raisebox{0.23ex}{$\mathop{:}$}}\!\!=\!\!$ }}}
\newcommand{\eqcolonl}{\ensuremath{\mathrel{=\!\!\mathop{:}}}}
\newcommand{\coloneql}{\ensuremath{\mathrel{\mathop{:} \!\! =}}}
\newcommand{\vc}[1]{% inline column vector
  \left(\begin{smallmatrix}#1\end{smallmatrix}\right)%
}
\newcommand{\vr}[1]{% inline row vector
  \begin{smallmatrix}(\,#1\,)\end{smallmatrix}%
}
\makeatletter
\newcommand*{\defeq}{\ =\mathrel{\rlap{%
                     \raisebox{0.3ex}{$\m@th\cdot$}}%
                     \raisebox{-0.3ex}{$\m@th\cdot$}}%
                     }
\makeatother

\newcommand{\mathcircle}[1]{% inline row vector
 \overset{\circ}{#1}
}
\newcommand{\ulim}{% low limit
 \underline{\lim}
}
\newcommand{\ssi}{% iff
\iff
}
\newcommand{\ps}[2]{
\expval{#1 | #2}
}
\newcommand{\df}[1]{
\mqty{#1}
}
\newcommand{\n}[1]{
\norm{#1}
}
\newcommand{\sys}[1]{
\left\{\smqty{#1}\right.
}


\newcommand{\eqdef}{\ensuremath{\overset{\text{def}}=}}


\def\Circlearrowright{\ensuremath{%
  \rotatebox[origin=c]{230}{$\circlearrowright$}}}

\newcommand\ct[1]{\text{\rmfamily\upshape #1}}
\newcommand\question[1]{ {\color{red} ...!? \small #1}}
\newcommand\caz[1]{\left\{\begin{array} #1 \end{array}\right.}
\newcommand\const{\text{\rmfamily\upshape const}}
\newcommand\toP{ \overset{\pro}{\to}}
\newcommand\toPP{ \overset{\text{PP}}{\to}}
\newcommand{\oeq}{\mathrel{\text{\textcircled{$=$}}}}





\usepackage{xcolor}
% \usepackage[normalem]{ulem}
\usepackage{lipsum}
\makeatletter
% \newcommand\colorwave[1][blue]{\bgroup \markoverwith{\lower3.5\p@\hbox{\sixly \textcolor{#1}{\char58}}}\ULon}
%\font\sixly=lasy6 % does not re-load if already loaded, so no memory problem.

\newmdtheoremenv[
linewidth= 1pt,linecolor= blue,%
leftmargin=20,rightmargin=20,innertopmargin=0pt, innerrightmargin=40,%
tikzsetting = { draw=lightgray, line width = 0.3pt,dashed,%
dash pattern = on 15pt off 3pt},%
splittopskip=\topskip,skipbelow=\baselineskip,%
skipabove=\baselineskip,ntheorem,roundcorner=0pt,
% backgroundcolor=pagebg,font=\color{orange}\sffamily, fontcolor=white
]{examplebox}{Exemple}[section]



\newcommand\R{\mathbb{R}}
\newcommand\Z{\mathbb{Z}}
\newcommand\N{\mathbb{N}}
\newcommand\E{\mathbb{E}}
\newcommand\F{\mathcal{F}}
\newcommand\cH{\mathcal{H}}
\newcommand\V{\mathbb{V}}
\newcommand\dmo{ ^{-1} }
\newcommand\kapa{\kappa}
\newcommand\im{Im}
\newcommand\hs{\mathcal{H}}





\usepackage{soul}

\makeatletter
\newcommand*{\whiten}[1]{\llap{\textcolor{white}{{\the\SOUL@token}}\hspace{#1pt}}}
\DeclareRobustCommand*\myul{%
    \def\SOUL@everyspace{\underline{\space}\kern\z@}%
    \def\SOUL@everytoken{%
     \setbox0=\hbox{\the\SOUL@token}%
     \ifdim\dp0>\z@
        \raisebox{\dp0}{\underline{\phantom{\the\SOUL@token}}}%
        \whiten{1}\whiten{0}%
        \whiten{-1}\whiten{-2}%
        \llap{\the\SOUL@token}%
     \else
        \underline{\the\SOUL@token}%
     \fi}%
\SOUL@}
\makeatother

\newcommand*{\demp}{\fontfamily{lmtt}\selectfont}

\DeclareTextFontCommand{\textdemp}{\demp}

\begin{document}

\ifcomment
Multiline
comment
\fi
\ifcomment
\myul{Typesetting test}
% \color[rgb]{1,1,1}
$∑_i^n≠ 60º±∞π∆¬≈√j∫h≤≥µ$

$\CR \R\pro\ind\pro\gS\pro
\mqty[a&b\\c&d]$
$\pro\mathbb{P}$
$\dd{x}$

  \[
    \alpha(x)=\left\{
                \begin{array}{ll}
                  x\\
                  \frac{1}{1+e^{-kx}}\\
                  \frac{e^x-e^{-x}}{e^x+e^{-x}}
                \end{array}
              \right.
  \]

  $\expval{x}$
  
  $\chi_\rho(ghg\dmo)=\Tr(\rho_{ghg\dmo})=\Tr(\rho_g\circ\rho_h\circ\rho\dmo_g)=\Tr(\rho_h)\overset{\mbox{\scalebox{0.5}{$\Tr(AB)=\Tr(BA)$}}}{=}\chi_\rho(h)$
  	$\mathop{\oplus}_{\substack{x\in X}}$

$\mat(\rho_g)=(a_{ij}(g))_{\scriptsize \substack{1\leq i\leq d \\ 1\leq j\leq d}}$ et $\mat(\rho'_g)=(a'_{ij}(g))_{\scriptsize \substack{1\leq i'\leq d' \\ 1\leq j'\leq d'}}$



\[\int_a^b{\mathbb{R}^2}g(u, v)\dd{P_{XY}}(u, v)=\iint g(u,v) f_{XY}(u, v)\dd \lambda(u) \dd \lambda(v)\]
$$\lim_{x\to\infty} f(x)$$	
$$\iiiint_V \mu(t,u,v,w) \,dt\,du\,dv\,dw$$
$$\sum_{n=1}^{\infty} 2^{-n} = 1$$	
\begin{definition}
	Si $X$ et $Y$ sont 2 v.a. ou definit la \textsc{Covariance} entre $X$ et $Y$ comme
	$\cov(X,Y)\overset{\text{def}}{=}\E\left[(X-\E(X))(Y-\E(Y))\right]=\E(XY)-\E(X)\E(Y)$.
\end{definition}
\fi
\pagebreak

% \tableofcontents

% insert your code here
%\input{./algebra/main.tex}
%\input{./geometrie-differentielle/main.tex}
%\input{./probabilite/main.tex}
%\input{./analyse-fonctionnelle/main.tex}
% \input{./Analyse-convexe-et-dualite-en-optimisation/main.tex}
%\input{./tikz/main.tex}
%\input{./Theorie-du-distributions/main.tex}
%\input{./optimisation/mine.tex}
 \input{./modelisation/main.tex}

% yves.aubry@univ-tln.fr : algebra

\end{document}

%% !TEX encoding = UTF-8 Unicode
% !TEX TS-program = xelatex

\documentclass[french]{report}

%\usepackage[utf8]{inputenc}
%\usepackage[T1]{fontenc}
\usepackage{babel}


\newif\ifcomment
%\commenttrue # Show comments

\usepackage{physics}
\usepackage{amssymb}


\usepackage{amsthm}
% \usepackage{thmtools}
\usepackage{mathtools}
\usepackage{amsfonts}

\usepackage{color}

\usepackage{tikz}

\usepackage{geometry}
\geometry{a5paper, margin=0.1in, right=1cm}

\usepackage{dsfont}

\usepackage{graphicx}
\graphicspath{ {images/} }

\usepackage{faktor}

\usepackage{IEEEtrantools}
\usepackage{enumerate}   
\usepackage[PostScript=dvips]{"/Users/aware/Documents/Courses/diagrams"}


\newtheorem{theorem}{Théorème}[section]
\renewcommand{\thetheorem}{\arabic{theorem}}
\newtheorem{lemme}{Lemme}[section]
\renewcommand{\thelemme}{\arabic{lemme}}
\newtheorem{proposition}{Proposition}[section]
\renewcommand{\theproposition}{\arabic{proposition}}
\newtheorem{notations}{Notations}[section]
\newtheorem{problem}{Problème}[section]
\newtheorem{corollary}{Corollaire}[theorem]
\renewcommand{\thecorollary}{\arabic{corollary}}
\newtheorem{property}{Propriété}[section]
\newtheorem{objective}{Objectif}[section]

\theoremstyle{definition}
\newtheorem{definition}{Définition}[section]
\renewcommand{\thedefinition}{\arabic{definition}}
\newtheorem{exercise}{Exercice}[chapter]
\renewcommand{\theexercise}{\arabic{exercise}}
\newtheorem{example}{Exemple}[chapter]
\renewcommand{\theexample}{\arabic{example}}
\newtheorem*{solution}{Solution}
\newtheorem*{application}{Application}
\newtheorem*{notation}{Notation}
\newtheorem*{vocabulary}{Vocabulaire}
\newtheorem*{properties}{Propriétés}



\theoremstyle{remark}
\newtheorem*{remark}{Remarque}
\newtheorem*{rappel}{Rappel}


\usepackage{etoolbox}
\AtBeginEnvironment{exercise}{\small}
\AtBeginEnvironment{example}{\small}

\usepackage{cases}
\usepackage[red]{mypack}

\usepackage[framemethod=TikZ]{mdframed}

\definecolor{bg}{rgb}{0.4,0.25,0.95}
\definecolor{pagebg}{rgb}{0,0,0.5}
\surroundwithmdframed[
   topline=false,
   rightline=false,
   bottomline=false,
   leftmargin=\parindent,
   skipabove=8pt,
   skipbelow=8pt,
   linecolor=blue,
   innerbottommargin=10pt,
   % backgroundcolor=bg,font=\color{orange}\sffamily, fontcolor=white
]{definition}

\usepackage{empheq}
\usepackage[most]{tcolorbox}

\newtcbox{\mymath}[1][]{%
    nobeforeafter, math upper, tcbox raise base,
    enhanced, colframe=blue!30!black,
    colback=red!10, boxrule=1pt,
    #1}

\usepackage{unixode}


\DeclareMathOperator{\ord}{ord}
\DeclareMathOperator{\orb}{orb}
\DeclareMathOperator{\stab}{stab}
\DeclareMathOperator{\Stab}{stab}
\DeclareMathOperator{\ppcm}{ppcm}
\DeclareMathOperator{\conj}{Conj}
\DeclareMathOperator{\End}{End}
\DeclareMathOperator{\rot}{rot}
\DeclareMathOperator{\trs}{trace}
\DeclareMathOperator{\Ind}{Ind}
\DeclareMathOperator{\mat}{Mat}
\DeclareMathOperator{\id}{Id}
\DeclareMathOperator{\vect}{vect}
\DeclareMathOperator{\img}{img}
\DeclareMathOperator{\cov}{Cov}
\DeclareMathOperator{\dist}{dist}
\DeclareMathOperator{\irr}{Irr}
\DeclareMathOperator{\image}{Im}
\DeclareMathOperator{\pd}{\partial}
\DeclareMathOperator{\epi}{epi}
\DeclareMathOperator{\Argmin}{Argmin}
\DeclareMathOperator{\dom}{dom}
\DeclareMathOperator{\proj}{proj}
\DeclareMathOperator{\ctg}{ctg}
\DeclareMathOperator{\supp}{supp}
\DeclareMathOperator{\argmin}{argmin}
\DeclareMathOperator{\mult}{mult}
\DeclareMathOperator{\ch}{ch}
\DeclareMathOperator{\sh}{sh}
\DeclareMathOperator{\rang}{rang}
\DeclareMathOperator{\diam}{diam}
\DeclareMathOperator{\Epigraphe}{Epigraphe}




\usepackage{xcolor}
\everymath{\color{blue}}
%\everymath{\color[rgb]{0,1,1}}
%\pagecolor[rgb]{0,0,0.5}


\newcommand*{\pdtest}[3][]{\ensuremath{\frac{\partial^{#1} #2}{\partial #3}}}

\newcommand*{\deffunc}[6][]{\ensuremath{
\begin{array}{rcl}
#2 : #3 &\rightarrow& #4\\
#5 &\mapsto& #6
\end{array}
}}

\newcommand{\eqcolon}{\mathrel{\resizebox{\widthof{$\mathord{=}$}}{\height}{ $\!\!=\!\!\resizebox{1.2\width}{0.8\height}{\raisebox{0.23ex}{$\mathop{:}$}}\!\!$ }}}
\newcommand{\coloneq}{\mathrel{\resizebox{\widthof{$\mathord{=}$}}{\height}{ $\!\!\resizebox{1.2\width}{0.8\height}{\raisebox{0.23ex}{$\mathop{:}$}}\!\!=\!\!$ }}}
\newcommand{\eqcolonl}{\ensuremath{\mathrel{=\!\!\mathop{:}}}}
\newcommand{\coloneql}{\ensuremath{\mathrel{\mathop{:} \!\! =}}}
\newcommand{\vc}[1]{% inline column vector
  \left(\begin{smallmatrix}#1\end{smallmatrix}\right)%
}
\newcommand{\vr}[1]{% inline row vector
  \begin{smallmatrix}(\,#1\,)\end{smallmatrix}%
}
\makeatletter
\newcommand*{\defeq}{\ =\mathrel{\rlap{%
                     \raisebox{0.3ex}{$\m@th\cdot$}}%
                     \raisebox{-0.3ex}{$\m@th\cdot$}}%
                     }
\makeatother

\newcommand{\mathcircle}[1]{% inline row vector
 \overset{\circ}{#1}
}
\newcommand{\ulim}{% low limit
 \underline{\lim}
}
\newcommand{\ssi}{% iff
\iff
}
\newcommand{\ps}[2]{
\expval{#1 | #2}
}
\newcommand{\df}[1]{
\mqty{#1}
}
\newcommand{\n}[1]{
\norm{#1}
}
\newcommand{\sys}[1]{
\left\{\smqty{#1}\right.
}


\newcommand{\eqdef}{\ensuremath{\overset{\text{def}}=}}


\def\Circlearrowright{\ensuremath{%
  \rotatebox[origin=c]{230}{$\circlearrowright$}}}

\newcommand\ct[1]{\text{\rmfamily\upshape #1}}
\newcommand\question[1]{ {\color{red} ...!? \small #1}}
\newcommand\caz[1]{\left\{\begin{array} #1 \end{array}\right.}
\newcommand\const{\text{\rmfamily\upshape const}}
\newcommand\toP{ \overset{\pro}{\to}}
\newcommand\toPP{ \overset{\text{PP}}{\to}}
\newcommand{\oeq}{\mathrel{\text{\textcircled{$=$}}}}





\usepackage{xcolor}
% \usepackage[normalem]{ulem}
\usepackage{lipsum}
\makeatletter
% \newcommand\colorwave[1][blue]{\bgroup \markoverwith{\lower3.5\p@\hbox{\sixly \textcolor{#1}{\char58}}}\ULon}
%\font\sixly=lasy6 % does not re-load if already loaded, so no memory problem.

\newmdtheoremenv[
linewidth= 1pt,linecolor= blue,%
leftmargin=20,rightmargin=20,innertopmargin=0pt, innerrightmargin=40,%
tikzsetting = { draw=lightgray, line width = 0.3pt,dashed,%
dash pattern = on 15pt off 3pt},%
splittopskip=\topskip,skipbelow=\baselineskip,%
skipabove=\baselineskip,ntheorem,roundcorner=0pt,
% backgroundcolor=pagebg,font=\color{orange}\sffamily, fontcolor=white
]{examplebox}{Exemple}[section]



\newcommand\R{\mathbb{R}}
\newcommand\Z{\mathbb{Z}}
\newcommand\N{\mathbb{N}}
\newcommand\E{\mathbb{E}}
\newcommand\F{\mathcal{F}}
\newcommand\cH{\mathcal{H}}
\newcommand\V{\mathbb{V}}
\newcommand\dmo{ ^{-1} }
\newcommand\kapa{\kappa}
\newcommand\im{Im}
\newcommand\hs{\mathcal{H}}





\usepackage{soul}

\makeatletter
\newcommand*{\whiten}[1]{\llap{\textcolor{white}{{\the\SOUL@token}}\hspace{#1pt}}}
\DeclareRobustCommand*\myul{%
    \def\SOUL@everyspace{\underline{\space}\kern\z@}%
    \def\SOUL@everytoken{%
     \setbox0=\hbox{\the\SOUL@token}%
     \ifdim\dp0>\z@
        \raisebox{\dp0}{\underline{\phantom{\the\SOUL@token}}}%
        \whiten{1}\whiten{0}%
        \whiten{-1}\whiten{-2}%
        \llap{\the\SOUL@token}%
     \else
        \underline{\the\SOUL@token}%
     \fi}%
\SOUL@}
\makeatother

\newcommand*{\demp}{\fontfamily{lmtt}\selectfont}

\DeclareTextFontCommand{\textdemp}{\demp}

\begin{document}

\ifcomment
Multiline
comment
\fi
\ifcomment
\myul{Typesetting test}
% \color[rgb]{1,1,1}
$∑_i^n≠ 60º±∞π∆¬≈√j∫h≤≥µ$

$\CR \R\pro\ind\pro\gS\pro
\mqty[a&b\\c&d]$
$\pro\mathbb{P}$
$\dd{x}$

  \[
    \alpha(x)=\left\{
                \begin{array}{ll}
                  x\\
                  \frac{1}{1+e^{-kx}}\\
                  \frac{e^x-e^{-x}}{e^x+e^{-x}}
                \end{array}
              \right.
  \]

  $\expval{x}$
  
  $\chi_\rho(ghg\dmo)=\Tr(\rho_{ghg\dmo})=\Tr(\rho_g\circ\rho_h\circ\rho\dmo_g)=\Tr(\rho_h)\overset{\mbox{\scalebox{0.5}{$\Tr(AB)=\Tr(BA)$}}}{=}\chi_\rho(h)$
  	$\mathop{\oplus}_{\substack{x\in X}}$

$\mat(\rho_g)=(a_{ij}(g))_{\scriptsize \substack{1\leq i\leq d \\ 1\leq j\leq d}}$ et $\mat(\rho'_g)=(a'_{ij}(g))_{\scriptsize \substack{1\leq i'\leq d' \\ 1\leq j'\leq d'}}$



\[\int_a^b{\mathbb{R}^2}g(u, v)\dd{P_{XY}}(u, v)=\iint g(u,v) f_{XY}(u, v)\dd \lambda(u) \dd \lambda(v)\]
$$\lim_{x\to\infty} f(x)$$	
$$\iiiint_V \mu(t,u,v,w) \,dt\,du\,dv\,dw$$
$$\sum_{n=1}^{\infty} 2^{-n} = 1$$	
\begin{definition}
	Si $X$ et $Y$ sont 2 v.a. ou definit la \textsc{Covariance} entre $X$ et $Y$ comme
	$\cov(X,Y)\overset{\text{def}}{=}\E\left[(X-\E(X))(Y-\E(Y))\right]=\E(XY)-\E(X)\E(Y)$.
\end{definition}
\fi
\pagebreak

% \tableofcontents

% insert your code here
%\input{./algebra/main.tex}
%\input{./geometrie-differentielle/main.tex}
%\input{./probabilite/main.tex}
%\input{./analyse-fonctionnelle/main.tex}
% \input{./Analyse-convexe-et-dualite-en-optimisation/main.tex}
%\input{./tikz/main.tex}
%\input{./Theorie-du-distributions/main.tex}
%\input{./optimisation/mine.tex}
 \input{./modelisation/main.tex}

% yves.aubry@univ-tln.fr : algebra

\end{document}

%\input{./optimisation/mine.tex}
 % !TEX encoding = UTF-8 Unicode
% !TEX TS-program = xelatex

\documentclass[french]{report}

%\usepackage[utf8]{inputenc}
%\usepackage[T1]{fontenc}
\usepackage{babel}


\newif\ifcomment
%\commenttrue # Show comments

\usepackage{physics}
\usepackage{amssymb}


\usepackage{amsthm}
% \usepackage{thmtools}
\usepackage{mathtools}
\usepackage{amsfonts}

\usepackage{color}

\usepackage{tikz}

\usepackage{geometry}
\geometry{a5paper, margin=0.1in, right=1cm}

\usepackage{dsfont}

\usepackage{graphicx}
\graphicspath{ {images/} }

\usepackage{faktor}

\usepackage{IEEEtrantools}
\usepackage{enumerate}   
\usepackage[PostScript=dvips]{"/Users/aware/Documents/Courses/diagrams"}


\newtheorem{theorem}{Théorème}[section]
\renewcommand{\thetheorem}{\arabic{theorem}}
\newtheorem{lemme}{Lemme}[section]
\renewcommand{\thelemme}{\arabic{lemme}}
\newtheorem{proposition}{Proposition}[section]
\renewcommand{\theproposition}{\arabic{proposition}}
\newtheorem{notations}{Notations}[section]
\newtheorem{problem}{Problème}[section]
\newtheorem{corollary}{Corollaire}[theorem]
\renewcommand{\thecorollary}{\arabic{corollary}}
\newtheorem{property}{Propriété}[section]
\newtheorem{objective}{Objectif}[section]

\theoremstyle{definition}
\newtheorem{definition}{Définition}[section]
\renewcommand{\thedefinition}{\arabic{definition}}
\newtheorem{exercise}{Exercice}[chapter]
\renewcommand{\theexercise}{\arabic{exercise}}
\newtheorem{example}{Exemple}[chapter]
\renewcommand{\theexample}{\arabic{example}}
\newtheorem*{solution}{Solution}
\newtheorem*{application}{Application}
\newtheorem*{notation}{Notation}
\newtheorem*{vocabulary}{Vocabulaire}
\newtheorem*{properties}{Propriétés}



\theoremstyle{remark}
\newtheorem*{remark}{Remarque}
\newtheorem*{rappel}{Rappel}


\usepackage{etoolbox}
\AtBeginEnvironment{exercise}{\small}
\AtBeginEnvironment{example}{\small}

\usepackage{cases}
\usepackage[red]{mypack}

\usepackage[framemethod=TikZ]{mdframed}

\definecolor{bg}{rgb}{0.4,0.25,0.95}
\definecolor{pagebg}{rgb}{0,0,0.5}
\surroundwithmdframed[
   topline=false,
   rightline=false,
   bottomline=false,
   leftmargin=\parindent,
   skipabove=8pt,
   skipbelow=8pt,
   linecolor=blue,
   innerbottommargin=10pt,
   % backgroundcolor=bg,font=\color{orange}\sffamily, fontcolor=white
]{definition}

\usepackage{empheq}
\usepackage[most]{tcolorbox}

\newtcbox{\mymath}[1][]{%
    nobeforeafter, math upper, tcbox raise base,
    enhanced, colframe=blue!30!black,
    colback=red!10, boxrule=1pt,
    #1}

\usepackage{unixode}


\DeclareMathOperator{\ord}{ord}
\DeclareMathOperator{\orb}{orb}
\DeclareMathOperator{\stab}{stab}
\DeclareMathOperator{\Stab}{stab}
\DeclareMathOperator{\ppcm}{ppcm}
\DeclareMathOperator{\conj}{Conj}
\DeclareMathOperator{\End}{End}
\DeclareMathOperator{\rot}{rot}
\DeclareMathOperator{\trs}{trace}
\DeclareMathOperator{\Ind}{Ind}
\DeclareMathOperator{\mat}{Mat}
\DeclareMathOperator{\id}{Id}
\DeclareMathOperator{\vect}{vect}
\DeclareMathOperator{\img}{img}
\DeclareMathOperator{\cov}{Cov}
\DeclareMathOperator{\dist}{dist}
\DeclareMathOperator{\irr}{Irr}
\DeclareMathOperator{\image}{Im}
\DeclareMathOperator{\pd}{\partial}
\DeclareMathOperator{\epi}{epi}
\DeclareMathOperator{\Argmin}{Argmin}
\DeclareMathOperator{\dom}{dom}
\DeclareMathOperator{\proj}{proj}
\DeclareMathOperator{\ctg}{ctg}
\DeclareMathOperator{\supp}{supp}
\DeclareMathOperator{\argmin}{argmin}
\DeclareMathOperator{\mult}{mult}
\DeclareMathOperator{\ch}{ch}
\DeclareMathOperator{\sh}{sh}
\DeclareMathOperator{\rang}{rang}
\DeclareMathOperator{\diam}{diam}
\DeclareMathOperator{\Epigraphe}{Epigraphe}




\usepackage{xcolor}
\everymath{\color{blue}}
%\everymath{\color[rgb]{0,1,1}}
%\pagecolor[rgb]{0,0,0.5}


\newcommand*{\pdtest}[3][]{\ensuremath{\frac{\partial^{#1} #2}{\partial #3}}}

\newcommand*{\deffunc}[6][]{\ensuremath{
\begin{array}{rcl}
#2 : #3 &\rightarrow& #4\\
#5 &\mapsto& #6
\end{array}
}}

\newcommand{\eqcolon}{\mathrel{\resizebox{\widthof{$\mathord{=}$}}{\height}{ $\!\!=\!\!\resizebox{1.2\width}{0.8\height}{\raisebox{0.23ex}{$\mathop{:}$}}\!\!$ }}}
\newcommand{\coloneq}{\mathrel{\resizebox{\widthof{$\mathord{=}$}}{\height}{ $\!\!\resizebox{1.2\width}{0.8\height}{\raisebox{0.23ex}{$\mathop{:}$}}\!\!=\!\!$ }}}
\newcommand{\eqcolonl}{\ensuremath{\mathrel{=\!\!\mathop{:}}}}
\newcommand{\coloneql}{\ensuremath{\mathrel{\mathop{:} \!\! =}}}
\newcommand{\vc}[1]{% inline column vector
  \left(\begin{smallmatrix}#1\end{smallmatrix}\right)%
}
\newcommand{\vr}[1]{% inline row vector
  \begin{smallmatrix}(\,#1\,)\end{smallmatrix}%
}
\makeatletter
\newcommand*{\defeq}{\ =\mathrel{\rlap{%
                     \raisebox{0.3ex}{$\m@th\cdot$}}%
                     \raisebox{-0.3ex}{$\m@th\cdot$}}%
                     }
\makeatother

\newcommand{\mathcircle}[1]{% inline row vector
 \overset{\circ}{#1}
}
\newcommand{\ulim}{% low limit
 \underline{\lim}
}
\newcommand{\ssi}{% iff
\iff
}
\newcommand{\ps}[2]{
\expval{#1 | #2}
}
\newcommand{\df}[1]{
\mqty{#1}
}
\newcommand{\n}[1]{
\norm{#1}
}
\newcommand{\sys}[1]{
\left\{\smqty{#1}\right.
}


\newcommand{\eqdef}{\ensuremath{\overset{\text{def}}=}}


\def\Circlearrowright{\ensuremath{%
  \rotatebox[origin=c]{230}{$\circlearrowright$}}}

\newcommand\ct[1]{\text{\rmfamily\upshape #1}}
\newcommand\question[1]{ {\color{red} ...!? \small #1}}
\newcommand\caz[1]{\left\{\begin{array} #1 \end{array}\right.}
\newcommand\const{\text{\rmfamily\upshape const}}
\newcommand\toP{ \overset{\pro}{\to}}
\newcommand\toPP{ \overset{\text{PP}}{\to}}
\newcommand{\oeq}{\mathrel{\text{\textcircled{$=$}}}}





\usepackage{xcolor}
% \usepackage[normalem]{ulem}
\usepackage{lipsum}
\makeatletter
% \newcommand\colorwave[1][blue]{\bgroup \markoverwith{\lower3.5\p@\hbox{\sixly \textcolor{#1}{\char58}}}\ULon}
%\font\sixly=lasy6 % does not re-load if already loaded, so no memory problem.

\newmdtheoremenv[
linewidth= 1pt,linecolor= blue,%
leftmargin=20,rightmargin=20,innertopmargin=0pt, innerrightmargin=40,%
tikzsetting = { draw=lightgray, line width = 0.3pt,dashed,%
dash pattern = on 15pt off 3pt},%
splittopskip=\topskip,skipbelow=\baselineskip,%
skipabove=\baselineskip,ntheorem,roundcorner=0pt,
% backgroundcolor=pagebg,font=\color{orange}\sffamily, fontcolor=white
]{examplebox}{Exemple}[section]



\newcommand\R{\mathbb{R}}
\newcommand\Z{\mathbb{Z}}
\newcommand\N{\mathbb{N}}
\newcommand\E{\mathbb{E}}
\newcommand\F{\mathcal{F}}
\newcommand\cH{\mathcal{H}}
\newcommand\V{\mathbb{V}}
\newcommand\dmo{ ^{-1} }
\newcommand\kapa{\kappa}
\newcommand\im{Im}
\newcommand\hs{\mathcal{H}}





\usepackage{soul}

\makeatletter
\newcommand*{\whiten}[1]{\llap{\textcolor{white}{{\the\SOUL@token}}\hspace{#1pt}}}
\DeclareRobustCommand*\myul{%
    \def\SOUL@everyspace{\underline{\space}\kern\z@}%
    \def\SOUL@everytoken{%
     \setbox0=\hbox{\the\SOUL@token}%
     \ifdim\dp0>\z@
        \raisebox{\dp0}{\underline{\phantom{\the\SOUL@token}}}%
        \whiten{1}\whiten{0}%
        \whiten{-1}\whiten{-2}%
        \llap{\the\SOUL@token}%
     \else
        \underline{\the\SOUL@token}%
     \fi}%
\SOUL@}
\makeatother

\newcommand*{\demp}{\fontfamily{lmtt}\selectfont}

\DeclareTextFontCommand{\textdemp}{\demp}

\begin{document}

\ifcomment
Multiline
comment
\fi
\ifcomment
\myul{Typesetting test}
% \color[rgb]{1,1,1}
$∑_i^n≠ 60º±∞π∆¬≈√j∫h≤≥µ$

$\CR \R\pro\ind\pro\gS\pro
\mqty[a&b\\c&d]$
$\pro\mathbb{P}$
$\dd{x}$

  \[
    \alpha(x)=\left\{
                \begin{array}{ll}
                  x\\
                  \frac{1}{1+e^{-kx}}\\
                  \frac{e^x-e^{-x}}{e^x+e^{-x}}
                \end{array}
              \right.
  \]

  $\expval{x}$
  
  $\chi_\rho(ghg\dmo)=\Tr(\rho_{ghg\dmo})=\Tr(\rho_g\circ\rho_h\circ\rho\dmo_g)=\Tr(\rho_h)\overset{\mbox{\scalebox{0.5}{$\Tr(AB)=\Tr(BA)$}}}{=}\chi_\rho(h)$
  	$\mathop{\oplus}_{\substack{x\in X}}$

$\mat(\rho_g)=(a_{ij}(g))_{\scriptsize \substack{1\leq i\leq d \\ 1\leq j\leq d}}$ et $\mat(\rho'_g)=(a'_{ij}(g))_{\scriptsize \substack{1\leq i'\leq d' \\ 1\leq j'\leq d'}}$



\[\int_a^b{\mathbb{R}^2}g(u, v)\dd{P_{XY}}(u, v)=\iint g(u,v) f_{XY}(u, v)\dd \lambda(u) \dd \lambda(v)\]
$$\lim_{x\to\infty} f(x)$$	
$$\iiiint_V \mu(t,u,v,w) \,dt\,du\,dv\,dw$$
$$\sum_{n=1}^{\infty} 2^{-n} = 1$$	
\begin{definition}
	Si $X$ et $Y$ sont 2 v.a. ou definit la \textsc{Covariance} entre $X$ et $Y$ comme
	$\cov(X,Y)\overset{\text{def}}{=}\E\left[(X-\E(X))(Y-\E(Y))\right]=\E(XY)-\E(X)\E(Y)$.
\end{definition}
\fi
\pagebreak

% \tableofcontents

% insert your code here
%\input{./algebra/main.tex}
%\input{./geometrie-differentielle/main.tex}
%\input{./probabilite/main.tex}
%\input{./analyse-fonctionnelle/main.tex}
% \input{./Analyse-convexe-et-dualite-en-optimisation/main.tex}
%\input{./tikz/main.tex}
%\input{./Theorie-du-distributions/main.tex}
%\input{./optimisation/mine.tex}
 \input{./modelisation/main.tex}

% yves.aubry@univ-tln.fr : algebra

\end{document}


% yves.aubry@univ-tln.fr : algebra

\end{document}

%% !TEX encoding = UTF-8 Unicode
% !TEX TS-program = xelatex

\documentclass[french]{report}

%\usepackage[utf8]{inputenc}
%\usepackage[T1]{fontenc}
\usepackage{babel}


\newif\ifcomment
%\commenttrue # Show comments

\usepackage{physics}
\usepackage{amssymb}


\usepackage{amsthm}
% \usepackage{thmtools}
\usepackage{mathtools}
\usepackage{amsfonts}

\usepackage{color}

\usepackage{tikz}

\usepackage{geometry}
\geometry{a5paper, margin=0.1in, right=1cm}

\usepackage{dsfont}

\usepackage{graphicx}
\graphicspath{ {images/} }

\usepackage{faktor}

\usepackage{IEEEtrantools}
\usepackage{enumerate}   
\usepackage[PostScript=dvips]{"/Users/aware/Documents/Courses/diagrams"}


\newtheorem{theorem}{Théorème}[section]
\renewcommand{\thetheorem}{\arabic{theorem}}
\newtheorem{lemme}{Lemme}[section]
\renewcommand{\thelemme}{\arabic{lemme}}
\newtheorem{proposition}{Proposition}[section]
\renewcommand{\theproposition}{\arabic{proposition}}
\newtheorem{notations}{Notations}[section]
\newtheorem{problem}{Problème}[section]
\newtheorem{corollary}{Corollaire}[theorem]
\renewcommand{\thecorollary}{\arabic{corollary}}
\newtheorem{property}{Propriété}[section]
\newtheorem{objective}{Objectif}[section]

\theoremstyle{definition}
\newtheorem{definition}{Définition}[section]
\renewcommand{\thedefinition}{\arabic{definition}}
\newtheorem{exercise}{Exercice}[chapter]
\renewcommand{\theexercise}{\arabic{exercise}}
\newtheorem{example}{Exemple}[chapter]
\renewcommand{\theexample}{\arabic{example}}
\newtheorem*{solution}{Solution}
\newtheorem*{application}{Application}
\newtheorem*{notation}{Notation}
\newtheorem*{vocabulary}{Vocabulaire}
\newtheorem*{properties}{Propriétés}



\theoremstyle{remark}
\newtheorem*{remark}{Remarque}
\newtheorem*{rappel}{Rappel}


\usepackage{etoolbox}
\AtBeginEnvironment{exercise}{\small}
\AtBeginEnvironment{example}{\small}

\usepackage{cases}
\usepackage[red]{mypack}

\usepackage[framemethod=TikZ]{mdframed}

\definecolor{bg}{rgb}{0.4,0.25,0.95}
\definecolor{pagebg}{rgb}{0,0,0.5}
\surroundwithmdframed[
   topline=false,
   rightline=false,
   bottomline=false,
   leftmargin=\parindent,
   skipabove=8pt,
   skipbelow=8pt,
   linecolor=blue,
   innerbottommargin=10pt,
   % backgroundcolor=bg,font=\color{orange}\sffamily, fontcolor=white
]{definition}

\usepackage{empheq}
\usepackage[most]{tcolorbox}

\newtcbox{\mymath}[1][]{%
    nobeforeafter, math upper, tcbox raise base,
    enhanced, colframe=blue!30!black,
    colback=red!10, boxrule=1pt,
    #1}

\usepackage{unixode}


\DeclareMathOperator{\ord}{ord}
\DeclareMathOperator{\orb}{orb}
\DeclareMathOperator{\stab}{stab}
\DeclareMathOperator{\Stab}{stab}
\DeclareMathOperator{\ppcm}{ppcm}
\DeclareMathOperator{\conj}{Conj}
\DeclareMathOperator{\End}{End}
\DeclareMathOperator{\rot}{rot}
\DeclareMathOperator{\trs}{trace}
\DeclareMathOperator{\Ind}{Ind}
\DeclareMathOperator{\mat}{Mat}
\DeclareMathOperator{\id}{Id}
\DeclareMathOperator{\vect}{vect}
\DeclareMathOperator{\img}{img}
\DeclareMathOperator{\cov}{Cov}
\DeclareMathOperator{\dist}{dist}
\DeclareMathOperator{\irr}{Irr}
\DeclareMathOperator{\image}{Im}
\DeclareMathOperator{\pd}{\partial}
\DeclareMathOperator{\epi}{epi}
\DeclareMathOperator{\Argmin}{Argmin}
\DeclareMathOperator{\dom}{dom}
\DeclareMathOperator{\proj}{proj}
\DeclareMathOperator{\ctg}{ctg}
\DeclareMathOperator{\supp}{supp}
\DeclareMathOperator{\argmin}{argmin}
\DeclareMathOperator{\mult}{mult}
\DeclareMathOperator{\ch}{ch}
\DeclareMathOperator{\sh}{sh}
\DeclareMathOperator{\rang}{rang}
\DeclareMathOperator{\diam}{diam}
\DeclareMathOperator{\Epigraphe}{Epigraphe}




\usepackage{xcolor}
\everymath{\color{blue}}
%\everymath{\color[rgb]{0,1,1}}
%\pagecolor[rgb]{0,0,0.5}


\newcommand*{\pdtest}[3][]{\ensuremath{\frac{\partial^{#1} #2}{\partial #3}}}

\newcommand*{\deffunc}[6][]{\ensuremath{
\begin{array}{rcl}
#2 : #3 &\rightarrow& #4\\
#5 &\mapsto& #6
\end{array}
}}

\newcommand{\eqcolon}{\mathrel{\resizebox{\widthof{$\mathord{=}$}}{\height}{ $\!\!=\!\!\resizebox{1.2\width}{0.8\height}{\raisebox{0.23ex}{$\mathop{:}$}}\!\!$ }}}
\newcommand{\coloneq}{\mathrel{\resizebox{\widthof{$\mathord{=}$}}{\height}{ $\!\!\resizebox{1.2\width}{0.8\height}{\raisebox{0.23ex}{$\mathop{:}$}}\!\!=\!\!$ }}}
\newcommand{\eqcolonl}{\ensuremath{\mathrel{=\!\!\mathop{:}}}}
\newcommand{\coloneql}{\ensuremath{\mathrel{\mathop{:} \!\! =}}}
\newcommand{\vc}[1]{% inline column vector
  \left(\begin{smallmatrix}#1\end{smallmatrix}\right)%
}
\newcommand{\vr}[1]{% inline row vector
  \begin{smallmatrix}(\,#1\,)\end{smallmatrix}%
}
\makeatletter
\newcommand*{\defeq}{\ =\mathrel{\rlap{%
                     \raisebox{0.3ex}{$\m@th\cdot$}}%
                     \raisebox{-0.3ex}{$\m@th\cdot$}}%
                     }
\makeatother

\newcommand{\mathcircle}[1]{% inline row vector
 \overset{\circ}{#1}
}
\newcommand{\ulim}{% low limit
 \underline{\lim}
}
\newcommand{\ssi}{% iff
\iff
}
\newcommand{\ps}[2]{
\expval{#1 | #2}
}
\newcommand{\df}[1]{
\mqty{#1}
}
\newcommand{\n}[1]{
\norm{#1}
}
\newcommand{\sys}[1]{
\left\{\smqty{#1}\right.
}


\newcommand{\eqdef}{\ensuremath{\overset{\text{def}}=}}


\def\Circlearrowright{\ensuremath{%
  \rotatebox[origin=c]{230}{$\circlearrowright$}}}

\newcommand\ct[1]{\text{\rmfamily\upshape #1}}
\newcommand\question[1]{ {\color{red} ...!? \small #1}}
\newcommand\caz[1]{\left\{\begin{array} #1 \end{array}\right.}
\newcommand\const{\text{\rmfamily\upshape const}}
\newcommand\toP{ \overset{\pro}{\to}}
\newcommand\toPP{ \overset{\text{PP}}{\to}}
\newcommand{\oeq}{\mathrel{\text{\textcircled{$=$}}}}





\usepackage{xcolor}
% \usepackage[normalem]{ulem}
\usepackage{lipsum}
\makeatletter
% \newcommand\colorwave[1][blue]{\bgroup \markoverwith{\lower3.5\p@\hbox{\sixly \textcolor{#1}{\char58}}}\ULon}
%\font\sixly=lasy6 % does not re-load if already loaded, so no memory problem.

\newmdtheoremenv[
linewidth= 1pt,linecolor= blue,%
leftmargin=20,rightmargin=20,innertopmargin=0pt, innerrightmargin=40,%
tikzsetting = { draw=lightgray, line width = 0.3pt,dashed,%
dash pattern = on 15pt off 3pt},%
splittopskip=\topskip,skipbelow=\baselineskip,%
skipabove=\baselineskip,ntheorem,roundcorner=0pt,
% backgroundcolor=pagebg,font=\color{orange}\sffamily, fontcolor=white
]{examplebox}{Exemple}[section]



\newcommand\R{\mathbb{R}}
\newcommand\Z{\mathbb{Z}}
\newcommand\N{\mathbb{N}}
\newcommand\E{\mathbb{E}}
\newcommand\F{\mathcal{F}}
\newcommand\cH{\mathcal{H}}
\newcommand\V{\mathbb{V}}
\newcommand\dmo{ ^{-1} }
\newcommand\kapa{\kappa}
\newcommand\im{Im}
\newcommand\hs{\mathcal{H}}





\usepackage{soul}

\makeatletter
\newcommand*{\whiten}[1]{\llap{\textcolor{white}{{\the\SOUL@token}}\hspace{#1pt}}}
\DeclareRobustCommand*\myul{%
    \def\SOUL@everyspace{\underline{\space}\kern\z@}%
    \def\SOUL@everytoken{%
     \setbox0=\hbox{\the\SOUL@token}%
     \ifdim\dp0>\z@
        \raisebox{\dp0}{\underline{\phantom{\the\SOUL@token}}}%
        \whiten{1}\whiten{0}%
        \whiten{-1}\whiten{-2}%
        \llap{\the\SOUL@token}%
     \else
        \underline{\the\SOUL@token}%
     \fi}%
\SOUL@}
\makeatother

\newcommand*{\demp}{\fontfamily{lmtt}\selectfont}

\DeclareTextFontCommand{\textdemp}{\demp}

\begin{document}

\ifcomment
Multiline
comment
\fi
\ifcomment
\myul{Typesetting test}
% \color[rgb]{1,1,1}
$∑_i^n≠ 60º±∞π∆¬≈√j∫h≤≥µ$

$\CR \R\pro\ind\pro\gS\pro
\mqty[a&b\\c&d]$
$\pro\mathbb{P}$
$\dd{x}$

  \[
    \alpha(x)=\left\{
                \begin{array}{ll}
                  x\\
                  \frac{1}{1+e^{-kx}}\\
                  \frac{e^x-e^{-x}}{e^x+e^{-x}}
                \end{array}
              \right.
  \]

  $\expval{x}$
  
  $\chi_\rho(ghg\dmo)=\Tr(\rho_{ghg\dmo})=\Tr(\rho_g\circ\rho_h\circ\rho\dmo_g)=\Tr(\rho_h)\overset{\mbox{\scalebox{0.5}{$\Tr(AB)=\Tr(BA)$}}}{=}\chi_\rho(h)$
  	$\mathop{\oplus}_{\substack{x\in X}}$

$\mat(\rho_g)=(a_{ij}(g))_{\scriptsize \substack{1\leq i\leq d \\ 1\leq j\leq d}}$ et $\mat(\rho'_g)=(a'_{ij}(g))_{\scriptsize \substack{1\leq i'\leq d' \\ 1\leq j'\leq d'}}$



\[\int_a^b{\mathbb{R}^2}g(u, v)\dd{P_{XY}}(u, v)=\iint g(u,v) f_{XY}(u, v)\dd \lambda(u) \dd \lambda(v)\]
$$\lim_{x\to\infty} f(x)$$	
$$\iiiint_V \mu(t,u,v,w) \,dt\,du\,dv\,dw$$
$$\sum_{n=1}^{\infty} 2^{-n} = 1$$	
\begin{definition}
	Si $X$ et $Y$ sont 2 v.a. ou definit la \textsc{Covariance} entre $X$ et $Y$ comme
	$\cov(X,Y)\overset{\text{def}}{=}\E\left[(X-\E(X))(Y-\E(Y))\right]=\E(XY)-\E(X)\E(Y)$.
\end{definition}
\fi
\pagebreak

% \tableofcontents

% insert your code here
%% !TEX encoding = UTF-8 Unicode
% !TEX TS-program = xelatex

\documentclass[french]{report}

%\usepackage[utf8]{inputenc}
%\usepackage[T1]{fontenc}
\usepackage{babel}


\newif\ifcomment
%\commenttrue # Show comments

\usepackage{physics}
\usepackage{amssymb}


\usepackage{amsthm}
% \usepackage{thmtools}
\usepackage{mathtools}
\usepackage{amsfonts}

\usepackage{color}

\usepackage{tikz}

\usepackage{geometry}
\geometry{a5paper, margin=0.1in, right=1cm}

\usepackage{dsfont}

\usepackage{graphicx}
\graphicspath{ {images/} }

\usepackage{faktor}

\usepackage{IEEEtrantools}
\usepackage{enumerate}   
\usepackage[PostScript=dvips]{"/Users/aware/Documents/Courses/diagrams"}


\newtheorem{theorem}{Théorème}[section]
\renewcommand{\thetheorem}{\arabic{theorem}}
\newtheorem{lemme}{Lemme}[section]
\renewcommand{\thelemme}{\arabic{lemme}}
\newtheorem{proposition}{Proposition}[section]
\renewcommand{\theproposition}{\arabic{proposition}}
\newtheorem{notations}{Notations}[section]
\newtheorem{problem}{Problème}[section]
\newtheorem{corollary}{Corollaire}[theorem]
\renewcommand{\thecorollary}{\arabic{corollary}}
\newtheorem{property}{Propriété}[section]
\newtheorem{objective}{Objectif}[section]

\theoremstyle{definition}
\newtheorem{definition}{Définition}[section]
\renewcommand{\thedefinition}{\arabic{definition}}
\newtheorem{exercise}{Exercice}[chapter]
\renewcommand{\theexercise}{\arabic{exercise}}
\newtheorem{example}{Exemple}[chapter]
\renewcommand{\theexample}{\arabic{example}}
\newtheorem*{solution}{Solution}
\newtheorem*{application}{Application}
\newtheorem*{notation}{Notation}
\newtheorem*{vocabulary}{Vocabulaire}
\newtheorem*{properties}{Propriétés}



\theoremstyle{remark}
\newtheorem*{remark}{Remarque}
\newtheorem*{rappel}{Rappel}


\usepackage{etoolbox}
\AtBeginEnvironment{exercise}{\small}
\AtBeginEnvironment{example}{\small}

\usepackage{cases}
\usepackage[red]{mypack}

\usepackage[framemethod=TikZ]{mdframed}

\definecolor{bg}{rgb}{0.4,0.25,0.95}
\definecolor{pagebg}{rgb}{0,0,0.5}
\surroundwithmdframed[
   topline=false,
   rightline=false,
   bottomline=false,
   leftmargin=\parindent,
   skipabove=8pt,
   skipbelow=8pt,
   linecolor=blue,
   innerbottommargin=10pt,
   % backgroundcolor=bg,font=\color{orange}\sffamily, fontcolor=white
]{definition}

\usepackage{empheq}
\usepackage[most]{tcolorbox}

\newtcbox{\mymath}[1][]{%
    nobeforeafter, math upper, tcbox raise base,
    enhanced, colframe=blue!30!black,
    colback=red!10, boxrule=1pt,
    #1}

\usepackage{unixode}


\DeclareMathOperator{\ord}{ord}
\DeclareMathOperator{\orb}{orb}
\DeclareMathOperator{\stab}{stab}
\DeclareMathOperator{\Stab}{stab}
\DeclareMathOperator{\ppcm}{ppcm}
\DeclareMathOperator{\conj}{Conj}
\DeclareMathOperator{\End}{End}
\DeclareMathOperator{\rot}{rot}
\DeclareMathOperator{\trs}{trace}
\DeclareMathOperator{\Ind}{Ind}
\DeclareMathOperator{\mat}{Mat}
\DeclareMathOperator{\id}{Id}
\DeclareMathOperator{\vect}{vect}
\DeclareMathOperator{\img}{img}
\DeclareMathOperator{\cov}{Cov}
\DeclareMathOperator{\dist}{dist}
\DeclareMathOperator{\irr}{Irr}
\DeclareMathOperator{\image}{Im}
\DeclareMathOperator{\pd}{\partial}
\DeclareMathOperator{\epi}{epi}
\DeclareMathOperator{\Argmin}{Argmin}
\DeclareMathOperator{\dom}{dom}
\DeclareMathOperator{\proj}{proj}
\DeclareMathOperator{\ctg}{ctg}
\DeclareMathOperator{\supp}{supp}
\DeclareMathOperator{\argmin}{argmin}
\DeclareMathOperator{\mult}{mult}
\DeclareMathOperator{\ch}{ch}
\DeclareMathOperator{\sh}{sh}
\DeclareMathOperator{\rang}{rang}
\DeclareMathOperator{\diam}{diam}
\DeclareMathOperator{\Epigraphe}{Epigraphe}




\usepackage{xcolor}
\everymath{\color{blue}}
%\everymath{\color[rgb]{0,1,1}}
%\pagecolor[rgb]{0,0,0.5}


\newcommand*{\pdtest}[3][]{\ensuremath{\frac{\partial^{#1} #2}{\partial #3}}}

\newcommand*{\deffunc}[6][]{\ensuremath{
\begin{array}{rcl}
#2 : #3 &\rightarrow& #4\\
#5 &\mapsto& #6
\end{array}
}}

\newcommand{\eqcolon}{\mathrel{\resizebox{\widthof{$\mathord{=}$}}{\height}{ $\!\!=\!\!\resizebox{1.2\width}{0.8\height}{\raisebox{0.23ex}{$\mathop{:}$}}\!\!$ }}}
\newcommand{\coloneq}{\mathrel{\resizebox{\widthof{$\mathord{=}$}}{\height}{ $\!\!\resizebox{1.2\width}{0.8\height}{\raisebox{0.23ex}{$\mathop{:}$}}\!\!=\!\!$ }}}
\newcommand{\eqcolonl}{\ensuremath{\mathrel{=\!\!\mathop{:}}}}
\newcommand{\coloneql}{\ensuremath{\mathrel{\mathop{:} \!\! =}}}
\newcommand{\vc}[1]{% inline column vector
  \left(\begin{smallmatrix}#1\end{smallmatrix}\right)%
}
\newcommand{\vr}[1]{% inline row vector
  \begin{smallmatrix}(\,#1\,)\end{smallmatrix}%
}
\makeatletter
\newcommand*{\defeq}{\ =\mathrel{\rlap{%
                     \raisebox{0.3ex}{$\m@th\cdot$}}%
                     \raisebox{-0.3ex}{$\m@th\cdot$}}%
                     }
\makeatother

\newcommand{\mathcircle}[1]{% inline row vector
 \overset{\circ}{#1}
}
\newcommand{\ulim}{% low limit
 \underline{\lim}
}
\newcommand{\ssi}{% iff
\iff
}
\newcommand{\ps}[2]{
\expval{#1 | #2}
}
\newcommand{\df}[1]{
\mqty{#1}
}
\newcommand{\n}[1]{
\norm{#1}
}
\newcommand{\sys}[1]{
\left\{\smqty{#1}\right.
}


\newcommand{\eqdef}{\ensuremath{\overset{\text{def}}=}}


\def\Circlearrowright{\ensuremath{%
  \rotatebox[origin=c]{230}{$\circlearrowright$}}}

\newcommand\ct[1]{\text{\rmfamily\upshape #1}}
\newcommand\question[1]{ {\color{red} ...!? \small #1}}
\newcommand\caz[1]{\left\{\begin{array} #1 \end{array}\right.}
\newcommand\const{\text{\rmfamily\upshape const}}
\newcommand\toP{ \overset{\pro}{\to}}
\newcommand\toPP{ \overset{\text{PP}}{\to}}
\newcommand{\oeq}{\mathrel{\text{\textcircled{$=$}}}}





\usepackage{xcolor}
% \usepackage[normalem]{ulem}
\usepackage{lipsum}
\makeatletter
% \newcommand\colorwave[1][blue]{\bgroup \markoverwith{\lower3.5\p@\hbox{\sixly \textcolor{#1}{\char58}}}\ULon}
%\font\sixly=lasy6 % does not re-load if already loaded, so no memory problem.

\newmdtheoremenv[
linewidth= 1pt,linecolor= blue,%
leftmargin=20,rightmargin=20,innertopmargin=0pt, innerrightmargin=40,%
tikzsetting = { draw=lightgray, line width = 0.3pt,dashed,%
dash pattern = on 15pt off 3pt},%
splittopskip=\topskip,skipbelow=\baselineskip,%
skipabove=\baselineskip,ntheorem,roundcorner=0pt,
% backgroundcolor=pagebg,font=\color{orange}\sffamily, fontcolor=white
]{examplebox}{Exemple}[section]



\newcommand\R{\mathbb{R}}
\newcommand\Z{\mathbb{Z}}
\newcommand\N{\mathbb{N}}
\newcommand\E{\mathbb{E}}
\newcommand\F{\mathcal{F}}
\newcommand\cH{\mathcal{H}}
\newcommand\V{\mathbb{V}}
\newcommand\dmo{ ^{-1} }
\newcommand\kapa{\kappa}
\newcommand\im{Im}
\newcommand\hs{\mathcal{H}}





\usepackage{soul}

\makeatletter
\newcommand*{\whiten}[1]{\llap{\textcolor{white}{{\the\SOUL@token}}\hspace{#1pt}}}
\DeclareRobustCommand*\myul{%
    \def\SOUL@everyspace{\underline{\space}\kern\z@}%
    \def\SOUL@everytoken{%
     \setbox0=\hbox{\the\SOUL@token}%
     \ifdim\dp0>\z@
        \raisebox{\dp0}{\underline{\phantom{\the\SOUL@token}}}%
        \whiten{1}\whiten{0}%
        \whiten{-1}\whiten{-2}%
        \llap{\the\SOUL@token}%
     \else
        \underline{\the\SOUL@token}%
     \fi}%
\SOUL@}
\makeatother

\newcommand*{\demp}{\fontfamily{lmtt}\selectfont}

\DeclareTextFontCommand{\textdemp}{\demp}

\begin{document}

\ifcomment
Multiline
comment
\fi
\ifcomment
\myul{Typesetting test}
% \color[rgb]{1,1,1}
$∑_i^n≠ 60º±∞π∆¬≈√j∫h≤≥µ$

$\CR \R\pro\ind\pro\gS\pro
\mqty[a&b\\c&d]$
$\pro\mathbb{P}$
$\dd{x}$

  \[
    \alpha(x)=\left\{
                \begin{array}{ll}
                  x\\
                  \frac{1}{1+e^{-kx}}\\
                  \frac{e^x-e^{-x}}{e^x+e^{-x}}
                \end{array}
              \right.
  \]

  $\expval{x}$
  
  $\chi_\rho(ghg\dmo)=\Tr(\rho_{ghg\dmo})=\Tr(\rho_g\circ\rho_h\circ\rho\dmo_g)=\Tr(\rho_h)\overset{\mbox{\scalebox{0.5}{$\Tr(AB)=\Tr(BA)$}}}{=}\chi_\rho(h)$
  	$\mathop{\oplus}_{\substack{x\in X}}$

$\mat(\rho_g)=(a_{ij}(g))_{\scriptsize \substack{1\leq i\leq d \\ 1\leq j\leq d}}$ et $\mat(\rho'_g)=(a'_{ij}(g))_{\scriptsize \substack{1\leq i'\leq d' \\ 1\leq j'\leq d'}}$



\[\int_a^b{\mathbb{R}^2}g(u, v)\dd{P_{XY}}(u, v)=\iint g(u,v) f_{XY}(u, v)\dd \lambda(u) \dd \lambda(v)\]
$$\lim_{x\to\infty} f(x)$$	
$$\iiiint_V \mu(t,u,v,w) \,dt\,du\,dv\,dw$$
$$\sum_{n=1}^{\infty} 2^{-n} = 1$$	
\begin{definition}
	Si $X$ et $Y$ sont 2 v.a. ou definit la \textsc{Covariance} entre $X$ et $Y$ comme
	$\cov(X,Y)\overset{\text{def}}{=}\E\left[(X-\E(X))(Y-\E(Y))\right]=\E(XY)-\E(X)\E(Y)$.
\end{definition}
\fi
\pagebreak

% \tableofcontents

% insert your code here
%\input{./algebra/main.tex}
%\input{./geometrie-differentielle/main.tex}
%\input{./probabilite/main.tex}
%\input{./analyse-fonctionnelle/main.tex}
% \input{./Analyse-convexe-et-dualite-en-optimisation/main.tex}
%\input{./tikz/main.tex}
%\input{./Theorie-du-distributions/main.tex}
%\input{./optimisation/mine.tex}
 \input{./modelisation/main.tex}

% yves.aubry@univ-tln.fr : algebra

\end{document}

%% !TEX encoding = UTF-8 Unicode
% !TEX TS-program = xelatex

\documentclass[french]{report}

%\usepackage[utf8]{inputenc}
%\usepackage[T1]{fontenc}
\usepackage{babel}


\newif\ifcomment
%\commenttrue # Show comments

\usepackage{physics}
\usepackage{amssymb}


\usepackage{amsthm}
% \usepackage{thmtools}
\usepackage{mathtools}
\usepackage{amsfonts}

\usepackage{color}

\usepackage{tikz}

\usepackage{geometry}
\geometry{a5paper, margin=0.1in, right=1cm}

\usepackage{dsfont}

\usepackage{graphicx}
\graphicspath{ {images/} }

\usepackage{faktor}

\usepackage{IEEEtrantools}
\usepackage{enumerate}   
\usepackage[PostScript=dvips]{"/Users/aware/Documents/Courses/diagrams"}


\newtheorem{theorem}{Théorème}[section]
\renewcommand{\thetheorem}{\arabic{theorem}}
\newtheorem{lemme}{Lemme}[section]
\renewcommand{\thelemme}{\arabic{lemme}}
\newtheorem{proposition}{Proposition}[section]
\renewcommand{\theproposition}{\arabic{proposition}}
\newtheorem{notations}{Notations}[section]
\newtheorem{problem}{Problème}[section]
\newtheorem{corollary}{Corollaire}[theorem]
\renewcommand{\thecorollary}{\arabic{corollary}}
\newtheorem{property}{Propriété}[section]
\newtheorem{objective}{Objectif}[section]

\theoremstyle{definition}
\newtheorem{definition}{Définition}[section]
\renewcommand{\thedefinition}{\arabic{definition}}
\newtheorem{exercise}{Exercice}[chapter]
\renewcommand{\theexercise}{\arabic{exercise}}
\newtheorem{example}{Exemple}[chapter]
\renewcommand{\theexample}{\arabic{example}}
\newtheorem*{solution}{Solution}
\newtheorem*{application}{Application}
\newtheorem*{notation}{Notation}
\newtheorem*{vocabulary}{Vocabulaire}
\newtheorem*{properties}{Propriétés}



\theoremstyle{remark}
\newtheorem*{remark}{Remarque}
\newtheorem*{rappel}{Rappel}


\usepackage{etoolbox}
\AtBeginEnvironment{exercise}{\small}
\AtBeginEnvironment{example}{\small}

\usepackage{cases}
\usepackage[red]{mypack}

\usepackage[framemethod=TikZ]{mdframed}

\definecolor{bg}{rgb}{0.4,0.25,0.95}
\definecolor{pagebg}{rgb}{0,0,0.5}
\surroundwithmdframed[
   topline=false,
   rightline=false,
   bottomline=false,
   leftmargin=\parindent,
   skipabove=8pt,
   skipbelow=8pt,
   linecolor=blue,
   innerbottommargin=10pt,
   % backgroundcolor=bg,font=\color{orange}\sffamily, fontcolor=white
]{definition}

\usepackage{empheq}
\usepackage[most]{tcolorbox}

\newtcbox{\mymath}[1][]{%
    nobeforeafter, math upper, tcbox raise base,
    enhanced, colframe=blue!30!black,
    colback=red!10, boxrule=1pt,
    #1}

\usepackage{unixode}


\DeclareMathOperator{\ord}{ord}
\DeclareMathOperator{\orb}{orb}
\DeclareMathOperator{\stab}{stab}
\DeclareMathOperator{\Stab}{stab}
\DeclareMathOperator{\ppcm}{ppcm}
\DeclareMathOperator{\conj}{Conj}
\DeclareMathOperator{\End}{End}
\DeclareMathOperator{\rot}{rot}
\DeclareMathOperator{\trs}{trace}
\DeclareMathOperator{\Ind}{Ind}
\DeclareMathOperator{\mat}{Mat}
\DeclareMathOperator{\id}{Id}
\DeclareMathOperator{\vect}{vect}
\DeclareMathOperator{\img}{img}
\DeclareMathOperator{\cov}{Cov}
\DeclareMathOperator{\dist}{dist}
\DeclareMathOperator{\irr}{Irr}
\DeclareMathOperator{\image}{Im}
\DeclareMathOperator{\pd}{\partial}
\DeclareMathOperator{\epi}{epi}
\DeclareMathOperator{\Argmin}{Argmin}
\DeclareMathOperator{\dom}{dom}
\DeclareMathOperator{\proj}{proj}
\DeclareMathOperator{\ctg}{ctg}
\DeclareMathOperator{\supp}{supp}
\DeclareMathOperator{\argmin}{argmin}
\DeclareMathOperator{\mult}{mult}
\DeclareMathOperator{\ch}{ch}
\DeclareMathOperator{\sh}{sh}
\DeclareMathOperator{\rang}{rang}
\DeclareMathOperator{\diam}{diam}
\DeclareMathOperator{\Epigraphe}{Epigraphe}




\usepackage{xcolor}
\everymath{\color{blue}}
%\everymath{\color[rgb]{0,1,1}}
%\pagecolor[rgb]{0,0,0.5}


\newcommand*{\pdtest}[3][]{\ensuremath{\frac{\partial^{#1} #2}{\partial #3}}}

\newcommand*{\deffunc}[6][]{\ensuremath{
\begin{array}{rcl}
#2 : #3 &\rightarrow& #4\\
#5 &\mapsto& #6
\end{array}
}}

\newcommand{\eqcolon}{\mathrel{\resizebox{\widthof{$\mathord{=}$}}{\height}{ $\!\!=\!\!\resizebox{1.2\width}{0.8\height}{\raisebox{0.23ex}{$\mathop{:}$}}\!\!$ }}}
\newcommand{\coloneq}{\mathrel{\resizebox{\widthof{$\mathord{=}$}}{\height}{ $\!\!\resizebox{1.2\width}{0.8\height}{\raisebox{0.23ex}{$\mathop{:}$}}\!\!=\!\!$ }}}
\newcommand{\eqcolonl}{\ensuremath{\mathrel{=\!\!\mathop{:}}}}
\newcommand{\coloneql}{\ensuremath{\mathrel{\mathop{:} \!\! =}}}
\newcommand{\vc}[1]{% inline column vector
  \left(\begin{smallmatrix}#1\end{smallmatrix}\right)%
}
\newcommand{\vr}[1]{% inline row vector
  \begin{smallmatrix}(\,#1\,)\end{smallmatrix}%
}
\makeatletter
\newcommand*{\defeq}{\ =\mathrel{\rlap{%
                     \raisebox{0.3ex}{$\m@th\cdot$}}%
                     \raisebox{-0.3ex}{$\m@th\cdot$}}%
                     }
\makeatother

\newcommand{\mathcircle}[1]{% inline row vector
 \overset{\circ}{#1}
}
\newcommand{\ulim}{% low limit
 \underline{\lim}
}
\newcommand{\ssi}{% iff
\iff
}
\newcommand{\ps}[2]{
\expval{#1 | #2}
}
\newcommand{\df}[1]{
\mqty{#1}
}
\newcommand{\n}[1]{
\norm{#1}
}
\newcommand{\sys}[1]{
\left\{\smqty{#1}\right.
}


\newcommand{\eqdef}{\ensuremath{\overset{\text{def}}=}}


\def\Circlearrowright{\ensuremath{%
  \rotatebox[origin=c]{230}{$\circlearrowright$}}}

\newcommand\ct[1]{\text{\rmfamily\upshape #1}}
\newcommand\question[1]{ {\color{red} ...!? \small #1}}
\newcommand\caz[1]{\left\{\begin{array} #1 \end{array}\right.}
\newcommand\const{\text{\rmfamily\upshape const}}
\newcommand\toP{ \overset{\pro}{\to}}
\newcommand\toPP{ \overset{\text{PP}}{\to}}
\newcommand{\oeq}{\mathrel{\text{\textcircled{$=$}}}}





\usepackage{xcolor}
% \usepackage[normalem]{ulem}
\usepackage{lipsum}
\makeatletter
% \newcommand\colorwave[1][blue]{\bgroup \markoverwith{\lower3.5\p@\hbox{\sixly \textcolor{#1}{\char58}}}\ULon}
%\font\sixly=lasy6 % does not re-load if already loaded, so no memory problem.

\newmdtheoremenv[
linewidth= 1pt,linecolor= blue,%
leftmargin=20,rightmargin=20,innertopmargin=0pt, innerrightmargin=40,%
tikzsetting = { draw=lightgray, line width = 0.3pt,dashed,%
dash pattern = on 15pt off 3pt},%
splittopskip=\topskip,skipbelow=\baselineskip,%
skipabove=\baselineskip,ntheorem,roundcorner=0pt,
% backgroundcolor=pagebg,font=\color{orange}\sffamily, fontcolor=white
]{examplebox}{Exemple}[section]



\newcommand\R{\mathbb{R}}
\newcommand\Z{\mathbb{Z}}
\newcommand\N{\mathbb{N}}
\newcommand\E{\mathbb{E}}
\newcommand\F{\mathcal{F}}
\newcommand\cH{\mathcal{H}}
\newcommand\V{\mathbb{V}}
\newcommand\dmo{ ^{-1} }
\newcommand\kapa{\kappa}
\newcommand\im{Im}
\newcommand\hs{\mathcal{H}}





\usepackage{soul}

\makeatletter
\newcommand*{\whiten}[1]{\llap{\textcolor{white}{{\the\SOUL@token}}\hspace{#1pt}}}
\DeclareRobustCommand*\myul{%
    \def\SOUL@everyspace{\underline{\space}\kern\z@}%
    \def\SOUL@everytoken{%
     \setbox0=\hbox{\the\SOUL@token}%
     \ifdim\dp0>\z@
        \raisebox{\dp0}{\underline{\phantom{\the\SOUL@token}}}%
        \whiten{1}\whiten{0}%
        \whiten{-1}\whiten{-2}%
        \llap{\the\SOUL@token}%
     \else
        \underline{\the\SOUL@token}%
     \fi}%
\SOUL@}
\makeatother

\newcommand*{\demp}{\fontfamily{lmtt}\selectfont}

\DeclareTextFontCommand{\textdemp}{\demp}

\begin{document}

\ifcomment
Multiline
comment
\fi
\ifcomment
\myul{Typesetting test}
% \color[rgb]{1,1,1}
$∑_i^n≠ 60º±∞π∆¬≈√j∫h≤≥µ$

$\CR \R\pro\ind\pro\gS\pro
\mqty[a&b\\c&d]$
$\pro\mathbb{P}$
$\dd{x}$

  \[
    \alpha(x)=\left\{
                \begin{array}{ll}
                  x\\
                  \frac{1}{1+e^{-kx}}\\
                  \frac{e^x-e^{-x}}{e^x+e^{-x}}
                \end{array}
              \right.
  \]

  $\expval{x}$
  
  $\chi_\rho(ghg\dmo)=\Tr(\rho_{ghg\dmo})=\Tr(\rho_g\circ\rho_h\circ\rho\dmo_g)=\Tr(\rho_h)\overset{\mbox{\scalebox{0.5}{$\Tr(AB)=\Tr(BA)$}}}{=}\chi_\rho(h)$
  	$\mathop{\oplus}_{\substack{x\in X}}$

$\mat(\rho_g)=(a_{ij}(g))_{\scriptsize \substack{1\leq i\leq d \\ 1\leq j\leq d}}$ et $\mat(\rho'_g)=(a'_{ij}(g))_{\scriptsize \substack{1\leq i'\leq d' \\ 1\leq j'\leq d'}}$



\[\int_a^b{\mathbb{R}^2}g(u, v)\dd{P_{XY}}(u, v)=\iint g(u,v) f_{XY}(u, v)\dd \lambda(u) \dd \lambda(v)\]
$$\lim_{x\to\infty} f(x)$$	
$$\iiiint_V \mu(t,u,v,w) \,dt\,du\,dv\,dw$$
$$\sum_{n=1}^{\infty} 2^{-n} = 1$$	
\begin{definition}
	Si $X$ et $Y$ sont 2 v.a. ou definit la \textsc{Covariance} entre $X$ et $Y$ comme
	$\cov(X,Y)\overset{\text{def}}{=}\E\left[(X-\E(X))(Y-\E(Y))\right]=\E(XY)-\E(X)\E(Y)$.
\end{definition}
\fi
\pagebreak

% \tableofcontents

% insert your code here
%\input{./algebra/main.tex}
%\input{./geometrie-differentielle/main.tex}
%\input{./probabilite/main.tex}
%\input{./analyse-fonctionnelle/main.tex}
% \input{./Analyse-convexe-et-dualite-en-optimisation/main.tex}
%\input{./tikz/main.tex}
%\input{./Theorie-du-distributions/main.tex}
%\input{./optimisation/mine.tex}
 \input{./modelisation/main.tex}

% yves.aubry@univ-tln.fr : algebra

\end{document}

%% !TEX encoding = UTF-8 Unicode
% !TEX TS-program = xelatex

\documentclass[french]{report}

%\usepackage[utf8]{inputenc}
%\usepackage[T1]{fontenc}
\usepackage{babel}


\newif\ifcomment
%\commenttrue # Show comments

\usepackage{physics}
\usepackage{amssymb}


\usepackage{amsthm}
% \usepackage{thmtools}
\usepackage{mathtools}
\usepackage{amsfonts}

\usepackage{color}

\usepackage{tikz}

\usepackage{geometry}
\geometry{a5paper, margin=0.1in, right=1cm}

\usepackage{dsfont}

\usepackage{graphicx}
\graphicspath{ {images/} }

\usepackage{faktor}

\usepackage{IEEEtrantools}
\usepackage{enumerate}   
\usepackage[PostScript=dvips]{"/Users/aware/Documents/Courses/diagrams"}


\newtheorem{theorem}{Théorème}[section]
\renewcommand{\thetheorem}{\arabic{theorem}}
\newtheorem{lemme}{Lemme}[section]
\renewcommand{\thelemme}{\arabic{lemme}}
\newtheorem{proposition}{Proposition}[section]
\renewcommand{\theproposition}{\arabic{proposition}}
\newtheorem{notations}{Notations}[section]
\newtheorem{problem}{Problème}[section]
\newtheorem{corollary}{Corollaire}[theorem]
\renewcommand{\thecorollary}{\arabic{corollary}}
\newtheorem{property}{Propriété}[section]
\newtheorem{objective}{Objectif}[section]

\theoremstyle{definition}
\newtheorem{definition}{Définition}[section]
\renewcommand{\thedefinition}{\arabic{definition}}
\newtheorem{exercise}{Exercice}[chapter]
\renewcommand{\theexercise}{\arabic{exercise}}
\newtheorem{example}{Exemple}[chapter]
\renewcommand{\theexample}{\arabic{example}}
\newtheorem*{solution}{Solution}
\newtheorem*{application}{Application}
\newtheorem*{notation}{Notation}
\newtheorem*{vocabulary}{Vocabulaire}
\newtheorem*{properties}{Propriétés}



\theoremstyle{remark}
\newtheorem*{remark}{Remarque}
\newtheorem*{rappel}{Rappel}


\usepackage{etoolbox}
\AtBeginEnvironment{exercise}{\small}
\AtBeginEnvironment{example}{\small}

\usepackage{cases}
\usepackage[red]{mypack}

\usepackage[framemethod=TikZ]{mdframed}

\definecolor{bg}{rgb}{0.4,0.25,0.95}
\definecolor{pagebg}{rgb}{0,0,0.5}
\surroundwithmdframed[
   topline=false,
   rightline=false,
   bottomline=false,
   leftmargin=\parindent,
   skipabove=8pt,
   skipbelow=8pt,
   linecolor=blue,
   innerbottommargin=10pt,
   % backgroundcolor=bg,font=\color{orange}\sffamily, fontcolor=white
]{definition}

\usepackage{empheq}
\usepackage[most]{tcolorbox}

\newtcbox{\mymath}[1][]{%
    nobeforeafter, math upper, tcbox raise base,
    enhanced, colframe=blue!30!black,
    colback=red!10, boxrule=1pt,
    #1}

\usepackage{unixode}


\DeclareMathOperator{\ord}{ord}
\DeclareMathOperator{\orb}{orb}
\DeclareMathOperator{\stab}{stab}
\DeclareMathOperator{\Stab}{stab}
\DeclareMathOperator{\ppcm}{ppcm}
\DeclareMathOperator{\conj}{Conj}
\DeclareMathOperator{\End}{End}
\DeclareMathOperator{\rot}{rot}
\DeclareMathOperator{\trs}{trace}
\DeclareMathOperator{\Ind}{Ind}
\DeclareMathOperator{\mat}{Mat}
\DeclareMathOperator{\id}{Id}
\DeclareMathOperator{\vect}{vect}
\DeclareMathOperator{\img}{img}
\DeclareMathOperator{\cov}{Cov}
\DeclareMathOperator{\dist}{dist}
\DeclareMathOperator{\irr}{Irr}
\DeclareMathOperator{\image}{Im}
\DeclareMathOperator{\pd}{\partial}
\DeclareMathOperator{\epi}{epi}
\DeclareMathOperator{\Argmin}{Argmin}
\DeclareMathOperator{\dom}{dom}
\DeclareMathOperator{\proj}{proj}
\DeclareMathOperator{\ctg}{ctg}
\DeclareMathOperator{\supp}{supp}
\DeclareMathOperator{\argmin}{argmin}
\DeclareMathOperator{\mult}{mult}
\DeclareMathOperator{\ch}{ch}
\DeclareMathOperator{\sh}{sh}
\DeclareMathOperator{\rang}{rang}
\DeclareMathOperator{\diam}{diam}
\DeclareMathOperator{\Epigraphe}{Epigraphe}




\usepackage{xcolor}
\everymath{\color{blue}}
%\everymath{\color[rgb]{0,1,1}}
%\pagecolor[rgb]{0,0,0.5}


\newcommand*{\pdtest}[3][]{\ensuremath{\frac{\partial^{#1} #2}{\partial #3}}}

\newcommand*{\deffunc}[6][]{\ensuremath{
\begin{array}{rcl}
#2 : #3 &\rightarrow& #4\\
#5 &\mapsto& #6
\end{array}
}}

\newcommand{\eqcolon}{\mathrel{\resizebox{\widthof{$\mathord{=}$}}{\height}{ $\!\!=\!\!\resizebox{1.2\width}{0.8\height}{\raisebox{0.23ex}{$\mathop{:}$}}\!\!$ }}}
\newcommand{\coloneq}{\mathrel{\resizebox{\widthof{$\mathord{=}$}}{\height}{ $\!\!\resizebox{1.2\width}{0.8\height}{\raisebox{0.23ex}{$\mathop{:}$}}\!\!=\!\!$ }}}
\newcommand{\eqcolonl}{\ensuremath{\mathrel{=\!\!\mathop{:}}}}
\newcommand{\coloneql}{\ensuremath{\mathrel{\mathop{:} \!\! =}}}
\newcommand{\vc}[1]{% inline column vector
  \left(\begin{smallmatrix}#1\end{smallmatrix}\right)%
}
\newcommand{\vr}[1]{% inline row vector
  \begin{smallmatrix}(\,#1\,)\end{smallmatrix}%
}
\makeatletter
\newcommand*{\defeq}{\ =\mathrel{\rlap{%
                     \raisebox{0.3ex}{$\m@th\cdot$}}%
                     \raisebox{-0.3ex}{$\m@th\cdot$}}%
                     }
\makeatother

\newcommand{\mathcircle}[1]{% inline row vector
 \overset{\circ}{#1}
}
\newcommand{\ulim}{% low limit
 \underline{\lim}
}
\newcommand{\ssi}{% iff
\iff
}
\newcommand{\ps}[2]{
\expval{#1 | #2}
}
\newcommand{\df}[1]{
\mqty{#1}
}
\newcommand{\n}[1]{
\norm{#1}
}
\newcommand{\sys}[1]{
\left\{\smqty{#1}\right.
}


\newcommand{\eqdef}{\ensuremath{\overset{\text{def}}=}}


\def\Circlearrowright{\ensuremath{%
  \rotatebox[origin=c]{230}{$\circlearrowright$}}}

\newcommand\ct[1]{\text{\rmfamily\upshape #1}}
\newcommand\question[1]{ {\color{red} ...!? \small #1}}
\newcommand\caz[1]{\left\{\begin{array} #1 \end{array}\right.}
\newcommand\const{\text{\rmfamily\upshape const}}
\newcommand\toP{ \overset{\pro}{\to}}
\newcommand\toPP{ \overset{\text{PP}}{\to}}
\newcommand{\oeq}{\mathrel{\text{\textcircled{$=$}}}}





\usepackage{xcolor}
% \usepackage[normalem]{ulem}
\usepackage{lipsum}
\makeatletter
% \newcommand\colorwave[1][blue]{\bgroup \markoverwith{\lower3.5\p@\hbox{\sixly \textcolor{#1}{\char58}}}\ULon}
%\font\sixly=lasy6 % does not re-load if already loaded, so no memory problem.

\newmdtheoremenv[
linewidth= 1pt,linecolor= blue,%
leftmargin=20,rightmargin=20,innertopmargin=0pt, innerrightmargin=40,%
tikzsetting = { draw=lightgray, line width = 0.3pt,dashed,%
dash pattern = on 15pt off 3pt},%
splittopskip=\topskip,skipbelow=\baselineskip,%
skipabove=\baselineskip,ntheorem,roundcorner=0pt,
% backgroundcolor=pagebg,font=\color{orange}\sffamily, fontcolor=white
]{examplebox}{Exemple}[section]



\newcommand\R{\mathbb{R}}
\newcommand\Z{\mathbb{Z}}
\newcommand\N{\mathbb{N}}
\newcommand\E{\mathbb{E}}
\newcommand\F{\mathcal{F}}
\newcommand\cH{\mathcal{H}}
\newcommand\V{\mathbb{V}}
\newcommand\dmo{ ^{-1} }
\newcommand\kapa{\kappa}
\newcommand\im{Im}
\newcommand\hs{\mathcal{H}}





\usepackage{soul}

\makeatletter
\newcommand*{\whiten}[1]{\llap{\textcolor{white}{{\the\SOUL@token}}\hspace{#1pt}}}
\DeclareRobustCommand*\myul{%
    \def\SOUL@everyspace{\underline{\space}\kern\z@}%
    \def\SOUL@everytoken{%
     \setbox0=\hbox{\the\SOUL@token}%
     \ifdim\dp0>\z@
        \raisebox{\dp0}{\underline{\phantom{\the\SOUL@token}}}%
        \whiten{1}\whiten{0}%
        \whiten{-1}\whiten{-2}%
        \llap{\the\SOUL@token}%
     \else
        \underline{\the\SOUL@token}%
     \fi}%
\SOUL@}
\makeatother

\newcommand*{\demp}{\fontfamily{lmtt}\selectfont}

\DeclareTextFontCommand{\textdemp}{\demp}

\begin{document}

\ifcomment
Multiline
comment
\fi
\ifcomment
\myul{Typesetting test}
% \color[rgb]{1,1,1}
$∑_i^n≠ 60º±∞π∆¬≈√j∫h≤≥µ$

$\CR \R\pro\ind\pro\gS\pro
\mqty[a&b\\c&d]$
$\pro\mathbb{P}$
$\dd{x}$

  \[
    \alpha(x)=\left\{
                \begin{array}{ll}
                  x\\
                  \frac{1}{1+e^{-kx}}\\
                  \frac{e^x-e^{-x}}{e^x+e^{-x}}
                \end{array}
              \right.
  \]

  $\expval{x}$
  
  $\chi_\rho(ghg\dmo)=\Tr(\rho_{ghg\dmo})=\Tr(\rho_g\circ\rho_h\circ\rho\dmo_g)=\Tr(\rho_h)\overset{\mbox{\scalebox{0.5}{$\Tr(AB)=\Tr(BA)$}}}{=}\chi_\rho(h)$
  	$\mathop{\oplus}_{\substack{x\in X}}$

$\mat(\rho_g)=(a_{ij}(g))_{\scriptsize \substack{1\leq i\leq d \\ 1\leq j\leq d}}$ et $\mat(\rho'_g)=(a'_{ij}(g))_{\scriptsize \substack{1\leq i'\leq d' \\ 1\leq j'\leq d'}}$



\[\int_a^b{\mathbb{R}^2}g(u, v)\dd{P_{XY}}(u, v)=\iint g(u,v) f_{XY}(u, v)\dd \lambda(u) \dd \lambda(v)\]
$$\lim_{x\to\infty} f(x)$$	
$$\iiiint_V \mu(t,u,v,w) \,dt\,du\,dv\,dw$$
$$\sum_{n=1}^{\infty} 2^{-n} = 1$$	
\begin{definition}
	Si $X$ et $Y$ sont 2 v.a. ou definit la \textsc{Covariance} entre $X$ et $Y$ comme
	$\cov(X,Y)\overset{\text{def}}{=}\E\left[(X-\E(X))(Y-\E(Y))\right]=\E(XY)-\E(X)\E(Y)$.
\end{definition}
\fi
\pagebreak

% \tableofcontents

% insert your code here
%\input{./algebra/main.tex}
%\input{./geometrie-differentielle/main.tex}
%\input{./probabilite/main.tex}
%\input{./analyse-fonctionnelle/main.tex}
% \input{./Analyse-convexe-et-dualite-en-optimisation/main.tex}
%\input{./tikz/main.tex}
%\input{./Theorie-du-distributions/main.tex}
%\input{./optimisation/mine.tex}
 \input{./modelisation/main.tex}

% yves.aubry@univ-tln.fr : algebra

\end{document}

%% !TEX encoding = UTF-8 Unicode
% !TEX TS-program = xelatex

\documentclass[french]{report}

%\usepackage[utf8]{inputenc}
%\usepackage[T1]{fontenc}
\usepackage{babel}


\newif\ifcomment
%\commenttrue # Show comments

\usepackage{physics}
\usepackage{amssymb}


\usepackage{amsthm}
% \usepackage{thmtools}
\usepackage{mathtools}
\usepackage{amsfonts}

\usepackage{color}

\usepackage{tikz}

\usepackage{geometry}
\geometry{a5paper, margin=0.1in, right=1cm}

\usepackage{dsfont}

\usepackage{graphicx}
\graphicspath{ {images/} }

\usepackage{faktor}

\usepackage{IEEEtrantools}
\usepackage{enumerate}   
\usepackage[PostScript=dvips]{"/Users/aware/Documents/Courses/diagrams"}


\newtheorem{theorem}{Théorème}[section]
\renewcommand{\thetheorem}{\arabic{theorem}}
\newtheorem{lemme}{Lemme}[section]
\renewcommand{\thelemme}{\arabic{lemme}}
\newtheorem{proposition}{Proposition}[section]
\renewcommand{\theproposition}{\arabic{proposition}}
\newtheorem{notations}{Notations}[section]
\newtheorem{problem}{Problème}[section]
\newtheorem{corollary}{Corollaire}[theorem]
\renewcommand{\thecorollary}{\arabic{corollary}}
\newtheorem{property}{Propriété}[section]
\newtheorem{objective}{Objectif}[section]

\theoremstyle{definition}
\newtheorem{definition}{Définition}[section]
\renewcommand{\thedefinition}{\arabic{definition}}
\newtheorem{exercise}{Exercice}[chapter]
\renewcommand{\theexercise}{\arabic{exercise}}
\newtheorem{example}{Exemple}[chapter]
\renewcommand{\theexample}{\arabic{example}}
\newtheorem*{solution}{Solution}
\newtheorem*{application}{Application}
\newtheorem*{notation}{Notation}
\newtheorem*{vocabulary}{Vocabulaire}
\newtheorem*{properties}{Propriétés}



\theoremstyle{remark}
\newtheorem*{remark}{Remarque}
\newtheorem*{rappel}{Rappel}


\usepackage{etoolbox}
\AtBeginEnvironment{exercise}{\small}
\AtBeginEnvironment{example}{\small}

\usepackage{cases}
\usepackage[red]{mypack}

\usepackage[framemethod=TikZ]{mdframed}

\definecolor{bg}{rgb}{0.4,0.25,0.95}
\definecolor{pagebg}{rgb}{0,0,0.5}
\surroundwithmdframed[
   topline=false,
   rightline=false,
   bottomline=false,
   leftmargin=\parindent,
   skipabove=8pt,
   skipbelow=8pt,
   linecolor=blue,
   innerbottommargin=10pt,
   % backgroundcolor=bg,font=\color{orange}\sffamily, fontcolor=white
]{definition}

\usepackage{empheq}
\usepackage[most]{tcolorbox}

\newtcbox{\mymath}[1][]{%
    nobeforeafter, math upper, tcbox raise base,
    enhanced, colframe=blue!30!black,
    colback=red!10, boxrule=1pt,
    #1}

\usepackage{unixode}


\DeclareMathOperator{\ord}{ord}
\DeclareMathOperator{\orb}{orb}
\DeclareMathOperator{\stab}{stab}
\DeclareMathOperator{\Stab}{stab}
\DeclareMathOperator{\ppcm}{ppcm}
\DeclareMathOperator{\conj}{Conj}
\DeclareMathOperator{\End}{End}
\DeclareMathOperator{\rot}{rot}
\DeclareMathOperator{\trs}{trace}
\DeclareMathOperator{\Ind}{Ind}
\DeclareMathOperator{\mat}{Mat}
\DeclareMathOperator{\id}{Id}
\DeclareMathOperator{\vect}{vect}
\DeclareMathOperator{\img}{img}
\DeclareMathOperator{\cov}{Cov}
\DeclareMathOperator{\dist}{dist}
\DeclareMathOperator{\irr}{Irr}
\DeclareMathOperator{\image}{Im}
\DeclareMathOperator{\pd}{\partial}
\DeclareMathOperator{\epi}{epi}
\DeclareMathOperator{\Argmin}{Argmin}
\DeclareMathOperator{\dom}{dom}
\DeclareMathOperator{\proj}{proj}
\DeclareMathOperator{\ctg}{ctg}
\DeclareMathOperator{\supp}{supp}
\DeclareMathOperator{\argmin}{argmin}
\DeclareMathOperator{\mult}{mult}
\DeclareMathOperator{\ch}{ch}
\DeclareMathOperator{\sh}{sh}
\DeclareMathOperator{\rang}{rang}
\DeclareMathOperator{\diam}{diam}
\DeclareMathOperator{\Epigraphe}{Epigraphe}




\usepackage{xcolor}
\everymath{\color{blue}}
%\everymath{\color[rgb]{0,1,1}}
%\pagecolor[rgb]{0,0,0.5}


\newcommand*{\pdtest}[3][]{\ensuremath{\frac{\partial^{#1} #2}{\partial #3}}}

\newcommand*{\deffunc}[6][]{\ensuremath{
\begin{array}{rcl}
#2 : #3 &\rightarrow& #4\\
#5 &\mapsto& #6
\end{array}
}}

\newcommand{\eqcolon}{\mathrel{\resizebox{\widthof{$\mathord{=}$}}{\height}{ $\!\!=\!\!\resizebox{1.2\width}{0.8\height}{\raisebox{0.23ex}{$\mathop{:}$}}\!\!$ }}}
\newcommand{\coloneq}{\mathrel{\resizebox{\widthof{$\mathord{=}$}}{\height}{ $\!\!\resizebox{1.2\width}{0.8\height}{\raisebox{0.23ex}{$\mathop{:}$}}\!\!=\!\!$ }}}
\newcommand{\eqcolonl}{\ensuremath{\mathrel{=\!\!\mathop{:}}}}
\newcommand{\coloneql}{\ensuremath{\mathrel{\mathop{:} \!\! =}}}
\newcommand{\vc}[1]{% inline column vector
  \left(\begin{smallmatrix}#1\end{smallmatrix}\right)%
}
\newcommand{\vr}[1]{% inline row vector
  \begin{smallmatrix}(\,#1\,)\end{smallmatrix}%
}
\makeatletter
\newcommand*{\defeq}{\ =\mathrel{\rlap{%
                     \raisebox{0.3ex}{$\m@th\cdot$}}%
                     \raisebox{-0.3ex}{$\m@th\cdot$}}%
                     }
\makeatother

\newcommand{\mathcircle}[1]{% inline row vector
 \overset{\circ}{#1}
}
\newcommand{\ulim}{% low limit
 \underline{\lim}
}
\newcommand{\ssi}{% iff
\iff
}
\newcommand{\ps}[2]{
\expval{#1 | #2}
}
\newcommand{\df}[1]{
\mqty{#1}
}
\newcommand{\n}[1]{
\norm{#1}
}
\newcommand{\sys}[1]{
\left\{\smqty{#1}\right.
}


\newcommand{\eqdef}{\ensuremath{\overset{\text{def}}=}}


\def\Circlearrowright{\ensuremath{%
  \rotatebox[origin=c]{230}{$\circlearrowright$}}}

\newcommand\ct[1]{\text{\rmfamily\upshape #1}}
\newcommand\question[1]{ {\color{red} ...!? \small #1}}
\newcommand\caz[1]{\left\{\begin{array} #1 \end{array}\right.}
\newcommand\const{\text{\rmfamily\upshape const}}
\newcommand\toP{ \overset{\pro}{\to}}
\newcommand\toPP{ \overset{\text{PP}}{\to}}
\newcommand{\oeq}{\mathrel{\text{\textcircled{$=$}}}}





\usepackage{xcolor}
% \usepackage[normalem]{ulem}
\usepackage{lipsum}
\makeatletter
% \newcommand\colorwave[1][blue]{\bgroup \markoverwith{\lower3.5\p@\hbox{\sixly \textcolor{#1}{\char58}}}\ULon}
%\font\sixly=lasy6 % does not re-load if already loaded, so no memory problem.

\newmdtheoremenv[
linewidth= 1pt,linecolor= blue,%
leftmargin=20,rightmargin=20,innertopmargin=0pt, innerrightmargin=40,%
tikzsetting = { draw=lightgray, line width = 0.3pt,dashed,%
dash pattern = on 15pt off 3pt},%
splittopskip=\topskip,skipbelow=\baselineskip,%
skipabove=\baselineskip,ntheorem,roundcorner=0pt,
% backgroundcolor=pagebg,font=\color{orange}\sffamily, fontcolor=white
]{examplebox}{Exemple}[section]



\newcommand\R{\mathbb{R}}
\newcommand\Z{\mathbb{Z}}
\newcommand\N{\mathbb{N}}
\newcommand\E{\mathbb{E}}
\newcommand\F{\mathcal{F}}
\newcommand\cH{\mathcal{H}}
\newcommand\V{\mathbb{V}}
\newcommand\dmo{ ^{-1} }
\newcommand\kapa{\kappa}
\newcommand\im{Im}
\newcommand\hs{\mathcal{H}}





\usepackage{soul}

\makeatletter
\newcommand*{\whiten}[1]{\llap{\textcolor{white}{{\the\SOUL@token}}\hspace{#1pt}}}
\DeclareRobustCommand*\myul{%
    \def\SOUL@everyspace{\underline{\space}\kern\z@}%
    \def\SOUL@everytoken{%
     \setbox0=\hbox{\the\SOUL@token}%
     \ifdim\dp0>\z@
        \raisebox{\dp0}{\underline{\phantom{\the\SOUL@token}}}%
        \whiten{1}\whiten{0}%
        \whiten{-1}\whiten{-2}%
        \llap{\the\SOUL@token}%
     \else
        \underline{\the\SOUL@token}%
     \fi}%
\SOUL@}
\makeatother

\newcommand*{\demp}{\fontfamily{lmtt}\selectfont}

\DeclareTextFontCommand{\textdemp}{\demp}

\begin{document}

\ifcomment
Multiline
comment
\fi
\ifcomment
\myul{Typesetting test}
% \color[rgb]{1,1,1}
$∑_i^n≠ 60º±∞π∆¬≈√j∫h≤≥µ$

$\CR \R\pro\ind\pro\gS\pro
\mqty[a&b\\c&d]$
$\pro\mathbb{P}$
$\dd{x}$

  \[
    \alpha(x)=\left\{
                \begin{array}{ll}
                  x\\
                  \frac{1}{1+e^{-kx}}\\
                  \frac{e^x-e^{-x}}{e^x+e^{-x}}
                \end{array}
              \right.
  \]

  $\expval{x}$
  
  $\chi_\rho(ghg\dmo)=\Tr(\rho_{ghg\dmo})=\Tr(\rho_g\circ\rho_h\circ\rho\dmo_g)=\Tr(\rho_h)\overset{\mbox{\scalebox{0.5}{$\Tr(AB)=\Tr(BA)$}}}{=}\chi_\rho(h)$
  	$\mathop{\oplus}_{\substack{x\in X}}$

$\mat(\rho_g)=(a_{ij}(g))_{\scriptsize \substack{1\leq i\leq d \\ 1\leq j\leq d}}$ et $\mat(\rho'_g)=(a'_{ij}(g))_{\scriptsize \substack{1\leq i'\leq d' \\ 1\leq j'\leq d'}}$



\[\int_a^b{\mathbb{R}^2}g(u, v)\dd{P_{XY}}(u, v)=\iint g(u,v) f_{XY}(u, v)\dd \lambda(u) \dd \lambda(v)\]
$$\lim_{x\to\infty} f(x)$$	
$$\iiiint_V \mu(t,u,v,w) \,dt\,du\,dv\,dw$$
$$\sum_{n=1}^{\infty} 2^{-n} = 1$$	
\begin{definition}
	Si $X$ et $Y$ sont 2 v.a. ou definit la \textsc{Covariance} entre $X$ et $Y$ comme
	$\cov(X,Y)\overset{\text{def}}{=}\E\left[(X-\E(X))(Y-\E(Y))\right]=\E(XY)-\E(X)\E(Y)$.
\end{definition}
\fi
\pagebreak

% \tableofcontents

% insert your code here
%\input{./algebra/main.tex}
%\input{./geometrie-differentielle/main.tex}
%\input{./probabilite/main.tex}
%\input{./analyse-fonctionnelle/main.tex}
% \input{./Analyse-convexe-et-dualite-en-optimisation/main.tex}
%\input{./tikz/main.tex}
%\input{./Theorie-du-distributions/main.tex}
%\input{./optimisation/mine.tex}
 \input{./modelisation/main.tex}

% yves.aubry@univ-tln.fr : algebra

\end{document}

% % !TEX encoding = UTF-8 Unicode
% !TEX TS-program = xelatex

\documentclass[french]{report}

%\usepackage[utf8]{inputenc}
%\usepackage[T1]{fontenc}
\usepackage{babel}


\newif\ifcomment
%\commenttrue # Show comments

\usepackage{physics}
\usepackage{amssymb}


\usepackage{amsthm}
% \usepackage{thmtools}
\usepackage{mathtools}
\usepackage{amsfonts}

\usepackage{color}

\usepackage{tikz}

\usepackage{geometry}
\geometry{a5paper, margin=0.1in, right=1cm}

\usepackage{dsfont}

\usepackage{graphicx}
\graphicspath{ {images/} }

\usepackage{faktor}

\usepackage{IEEEtrantools}
\usepackage{enumerate}   
\usepackage[PostScript=dvips]{"/Users/aware/Documents/Courses/diagrams"}


\newtheorem{theorem}{Théorème}[section]
\renewcommand{\thetheorem}{\arabic{theorem}}
\newtheorem{lemme}{Lemme}[section]
\renewcommand{\thelemme}{\arabic{lemme}}
\newtheorem{proposition}{Proposition}[section]
\renewcommand{\theproposition}{\arabic{proposition}}
\newtheorem{notations}{Notations}[section]
\newtheorem{problem}{Problème}[section]
\newtheorem{corollary}{Corollaire}[theorem]
\renewcommand{\thecorollary}{\arabic{corollary}}
\newtheorem{property}{Propriété}[section]
\newtheorem{objective}{Objectif}[section]

\theoremstyle{definition}
\newtheorem{definition}{Définition}[section]
\renewcommand{\thedefinition}{\arabic{definition}}
\newtheorem{exercise}{Exercice}[chapter]
\renewcommand{\theexercise}{\arabic{exercise}}
\newtheorem{example}{Exemple}[chapter]
\renewcommand{\theexample}{\arabic{example}}
\newtheorem*{solution}{Solution}
\newtheorem*{application}{Application}
\newtheorem*{notation}{Notation}
\newtheorem*{vocabulary}{Vocabulaire}
\newtheorem*{properties}{Propriétés}



\theoremstyle{remark}
\newtheorem*{remark}{Remarque}
\newtheorem*{rappel}{Rappel}


\usepackage{etoolbox}
\AtBeginEnvironment{exercise}{\small}
\AtBeginEnvironment{example}{\small}

\usepackage{cases}
\usepackage[red]{mypack}

\usepackage[framemethod=TikZ]{mdframed}

\definecolor{bg}{rgb}{0.4,0.25,0.95}
\definecolor{pagebg}{rgb}{0,0,0.5}
\surroundwithmdframed[
   topline=false,
   rightline=false,
   bottomline=false,
   leftmargin=\parindent,
   skipabove=8pt,
   skipbelow=8pt,
   linecolor=blue,
   innerbottommargin=10pt,
   % backgroundcolor=bg,font=\color{orange}\sffamily, fontcolor=white
]{definition}

\usepackage{empheq}
\usepackage[most]{tcolorbox}

\newtcbox{\mymath}[1][]{%
    nobeforeafter, math upper, tcbox raise base,
    enhanced, colframe=blue!30!black,
    colback=red!10, boxrule=1pt,
    #1}

\usepackage{unixode}


\DeclareMathOperator{\ord}{ord}
\DeclareMathOperator{\orb}{orb}
\DeclareMathOperator{\stab}{stab}
\DeclareMathOperator{\Stab}{stab}
\DeclareMathOperator{\ppcm}{ppcm}
\DeclareMathOperator{\conj}{Conj}
\DeclareMathOperator{\End}{End}
\DeclareMathOperator{\rot}{rot}
\DeclareMathOperator{\trs}{trace}
\DeclareMathOperator{\Ind}{Ind}
\DeclareMathOperator{\mat}{Mat}
\DeclareMathOperator{\id}{Id}
\DeclareMathOperator{\vect}{vect}
\DeclareMathOperator{\img}{img}
\DeclareMathOperator{\cov}{Cov}
\DeclareMathOperator{\dist}{dist}
\DeclareMathOperator{\irr}{Irr}
\DeclareMathOperator{\image}{Im}
\DeclareMathOperator{\pd}{\partial}
\DeclareMathOperator{\epi}{epi}
\DeclareMathOperator{\Argmin}{Argmin}
\DeclareMathOperator{\dom}{dom}
\DeclareMathOperator{\proj}{proj}
\DeclareMathOperator{\ctg}{ctg}
\DeclareMathOperator{\supp}{supp}
\DeclareMathOperator{\argmin}{argmin}
\DeclareMathOperator{\mult}{mult}
\DeclareMathOperator{\ch}{ch}
\DeclareMathOperator{\sh}{sh}
\DeclareMathOperator{\rang}{rang}
\DeclareMathOperator{\diam}{diam}
\DeclareMathOperator{\Epigraphe}{Epigraphe}




\usepackage{xcolor}
\everymath{\color{blue}}
%\everymath{\color[rgb]{0,1,1}}
%\pagecolor[rgb]{0,0,0.5}


\newcommand*{\pdtest}[3][]{\ensuremath{\frac{\partial^{#1} #2}{\partial #3}}}

\newcommand*{\deffunc}[6][]{\ensuremath{
\begin{array}{rcl}
#2 : #3 &\rightarrow& #4\\
#5 &\mapsto& #6
\end{array}
}}

\newcommand{\eqcolon}{\mathrel{\resizebox{\widthof{$\mathord{=}$}}{\height}{ $\!\!=\!\!\resizebox{1.2\width}{0.8\height}{\raisebox{0.23ex}{$\mathop{:}$}}\!\!$ }}}
\newcommand{\coloneq}{\mathrel{\resizebox{\widthof{$\mathord{=}$}}{\height}{ $\!\!\resizebox{1.2\width}{0.8\height}{\raisebox{0.23ex}{$\mathop{:}$}}\!\!=\!\!$ }}}
\newcommand{\eqcolonl}{\ensuremath{\mathrel{=\!\!\mathop{:}}}}
\newcommand{\coloneql}{\ensuremath{\mathrel{\mathop{:} \!\! =}}}
\newcommand{\vc}[1]{% inline column vector
  \left(\begin{smallmatrix}#1\end{smallmatrix}\right)%
}
\newcommand{\vr}[1]{% inline row vector
  \begin{smallmatrix}(\,#1\,)\end{smallmatrix}%
}
\makeatletter
\newcommand*{\defeq}{\ =\mathrel{\rlap{%
                     \raisebox{0.3ex}{$\m@th\cdot$}}%
                     \raisebox{-0.3ex}{$\m@th\cdot$}}%
                     }
\makeatother

\newcommand{\mathcircle}[1]{% inline row vector
 \overset{\circ}{#1}
}
\newcommand{\ulim}{% low limit
 \underline{\lim}
}
\newcommand{\ssi}{% iff
\iff
}
\newcommand{\ps}[2]{
\expval{#1 | #2}
}
\newcommand{\df}[1]{
\mqty{#1}
}
\newcommand{\n}[1]{
\norm{#1}
}
\newcommand{\sys}[1]{
\left\{\smqty{#1}\right.
}


\newcommand{\eqdef}{\ensuremath{\overset{\text{def}}=}}


\def\Circlearrowright{\ensuremath{%
  \rotatebox[origin=c]{230}{$\circlearrowright$}}}

\newcommand\ct[1]{\text{\rmfamily\upshape #1}}
\newcommand\question[1]{ {\color{red} ...!? \small #1}}
\newcommand\caz[1]{\left\{\begin{array} #1 \end{array}\right.}
\newcommand\const{\text{\rmfamily\upshape const}}
\newcommand\toP{ \overset{\pro}{\to}}
\newcommand\toPP{ \overset{\text{PP}}{\to}}
\newcommand{\oeq}{\mathrel{\text{\textcircled{$=$}}}}





\usepackage{xcolor}
% \usepackage[normalem]{ulem}
\usepackage{lipsum}
\makeatletter
% \newcommand\colorwave[1][blue]{\bgroup \markoverwith{\lower3.5\p@\hbox{\sixly \textcolor{#1}{\char58}}}\ULon}
%\font\sixly=lasy6 % does not re-load if already loaded, so no memory problem.

\newmdtheoremenv[
linewidth= 1pt,linecolor= blue,%
leftmargin=20,rightmargin=20,innertopmargin=0pt, innerrightmargin=40,%
tikzsetting = { draw=lightgray, line width = 0.3pt,dashed,%
dash pattern = on 15pt off 3pt},%
splittopskip=\topskip,skipbelow=\baselineskip,%
skipabove=\baselineskip,ntheorem,roundcorner=0pt,
% backgroundcolor=pagebg,font=\color{orange}\sffamily, fontcolor=white
]{examplebox}{Exemple}[section]



\newcommand\R{\mathbb{R}}
\newcommand\Z{\mathbb{Z}}
\newcommand\N{\mathbb{N}}
\newcommand\E{\mathbb{E}}
\newcommand\F{\mathcal{F}}
\newcommand\cH{\mathcal{H}}
\newcommand\V{\mathbb{V}}
\newcommand\dmo{ ^{-1} }
\newcommand\kapa{\kappa}
\newcommand\im{Im}
\newcommand\hs{\mathcal{H}}





\usepackage{soul}

\makeatletter
\newcommand*{\whiten}[1]{\llap{\textcolor{white}{{\the\SOUL@token}}\hspace{#1pt}}}
\DeclareRobustCommand*\myul{%
    \def\SOUL@everyspace{\underline{\space}\kern\z@}%
    \def\SOUL@everytoken{%
     \setbox0=\hbox{\the\SOUL@token}%
     \ifdim\dp0>\z@
        \raisebox{\dp0}{\underline{\phantom{\the\SOUL@token}}}%
        \whiten{1}\whiten{0}%
        \whiten{-1}\whiten{-2}%
        \llap{\the\SOUL@token}%
     \else
        \underline{\the\SOUL@token}%
     \fi}%
\SOUL@}
\makeatother

\newcommand*{\demp}{\fontfamily{lmtt}\selectfont}

\DeclareTextFontCommand{\textdemp}{\demp}

\begin{document}

\ifcomment
Multiline
comment
\fi
\ifcomment
\myul{Typesetting test}
% \color[rgb]{1,1,1}
$∑_i^n≠ 60º±∞π∆¬≈√j∫h≤≥µ$

$\CR \R\pro\ind\pro\gS\pro
\mqty[a&b\\c&d]$
$\pro\mathbb{P}$
$\dd{x}$

  \[
    \alpha(x)=\left\{
                \begin{array}{ll}
                  x\\
                  \frac{1}{1+e^{-kx}}\\
                  \frac{e^x-e^{-x}}{e^x+e^{-x}}
                \end{array}
              \right.
  \]

  $\expval{x}$
  
  $\chi_\rho(ghg\dmo)=\Tr(\rho_{ghg\dmo})=\Tr(\rho_g\circ\rho_h\circ\rho\dmo_g)=\Tr(\rho_h)\overset{\mbox{\scalebox{0.5}{$\Tr(AB)=\Tr(BA)$}}}{=}\chi_\rho(h)$
  	$\mathop{\oplus}_{\substack{x\in X}}$

$\mat(\rho_g)=(a_{ij}(g))_{\scriptsize \substack{1\leq i\leq d \\ 1\leq j\leq d}}$ et $\mat(\rho'_g)=(a'_{ij}(g))_{\scriptsize \substack{1\leq i'\leq d' \\ 1\leq j'\leq d'}}$



\[\int_a^b{\mathbb{R}^2}g(u, v)\dd{P_{XY}}(u, v)=\iint g(u,v) f_{XY}(u, v)\dd \lambda(u) \dd \lambda(v)\]
$$\lim_{x\to\infty} f(x)$$	
$$\iiiint_V \mu(t,u,v,w) \,dt\,du\,dv\,dw$$
$$\sum_{n=1}^{\infty} 2^{-n} = 1$$	
\begin{definition}
	Si $X$ et $Y$ sont 2 v.a. ou definit la \textsc{Covariance} entre $X$ et $Y$ comme
	$\cov(X,Y)\overset{\text{def}}{=}\E\left[(X-\E(X))(Y-\E(Y))\right]=\E(XY)-\E(X)\E(Y)$.
\end{definition}
\fi
\pagebreak

% \tableofcontents

% insert your code here
%\input{./algebra/main.tex}
%\input{./geometrie-differentielle/main.tex}
%\input{./probabilite/main.tex}
%\input{./analyse-fonctionnelle/main.tex}
% \input{./Analyse-convexe-et-dualite-en-optimisation/main.tex}
%\input{./tikz/main.tex}
%\input{./Theorie-du-distributions/main.tex}
%\input{./optimisation/mine.tex}
 \input{./modelisation/main.tex}

% yves.aubry@univ-tln.fr : algebra

\end{document}

%% !TEX encoding = UTF-8 Unicode
% !TEX TS-program = xelatex

\documentclass[french]{report}

%\usepackage[utf8]{inputenc}
%\usepackage[T1]{fontenc}
\usepackage{babel}


\newif\ifcomment
%\commenttrue # Show comments

\usepackage{physics}
\usepackage{amssymb}


\usepackage{amsthm}
% \usepackage{thmtools}
\usepackage{mathtools}
\usepackage{amsfonts}

\usepackage{color}

\usepackage{tikz}

\usepackage{geometry}
\geometry{a5paper, margin=0.1in, right=1cm}

\usepackage{dsfont}

\usepackage{graphicx}
\graphicspath{ {images/} }

\usepackage{faktor}

\usepackage{IEEEtrantools}
\usepackage{enumerate}   
\usepackage[PostScript=dvips]{"/Users/aware/Documents/Courses/diagrams"}


\newtheorem{theorem}{Théorème}[section]
\renewcommand{\thetheorem}{\arabic{theorem}}
\newtheorem{lemme}{Lemme}[section]
\renewcommand{\thelemme}{\arabic{lemme}}
\newtheorem{proposition}{Proposition}[section]
\renewcommand{\theproposition}{\arabic{proposition}}
\newtheorem{notations}{Notations}[section]
\newtheorem{problem}{Problème}[section]
\newtheorem{corollary}{Corollaire}[theorem]
\renewcommand{\thecorollary}{\arabic{corollary}}
\newtheorem{property}{Propriété}[section]
\newtheorem{objective}{Objectif}[section]

\theoremstyle{definition}
\newtheorem{definition}{Définition}[section]
\renewcommand{\thedefinition}{\arabic{definition}}
\newtheorem{exercise}{Exercice}[chapter]
\renewcommand{\theexercise}{\arabic{exercise}}
\newtheorem{example}{Exemple}[chapter]
\renewcommand{\theexample}{\arabic{example}}
\newtheorem*{solution}{Solution}
\newtheorem*{application}{Application}
\newtheorem*{notation}{Notation}
\newtheorem*{vocabulary}{Vocabulaire}
\newtheorem*{properties}{Propriétés}



\theoremstyle{remark}
\newtheorem*{remark}{Remarque}
\newtheorem*{rappel}{Rappel}


\usepackage{etoolbox}
\AtBeginEnvironment{exercise}{\small}
\AtBeginEnvironment{example}{\small}

\usepackage{cases}
\usepackage[red]{mypack}

\usepackage[framemethod=TikZ]{mdframed}

\definecolor{bg}{rgb}{0.4,0.25,0.95}
\definecolor{pagebg}{rgb}{0,0,0.5}
\surroundwithmdframed[
   topline=false,
   rightline=false,
   bottomline=false,
   leftmargin=\parindent,
   skipabove=8pt,
   skipbelow=8pt,
   linecolor=blue,
   innerbottommargin=10pt,
   % backgroundcolor=bg,font=\color{orange}\sffamily, fontcolor=white
]{definition}

\usepackage{empheq}
\usepackage[most]{tcolorbox}

\newtcbox{\mymath}[1][]{%
    nobeforeafter, math upper, tcbox raise base,
    enhanced, colframe=blue!30!black,
    colback=red!10, boxrule=1pt,
    #1}

\usepackage{unixode}


\DeclareMathOperator{\ord}{ord}
\DeclareMathOperator{\orb}{orb}
\DeclareMathOperator{\stab}{stab}
\DeclareMathOperator{\Stab}{stab}
\DeclareMathOperator{\ppcm}{ppcm}
\DeclareMathOperator{\conj}{Conj}
\DeclareMathOperator{\End}{End}
\DeclareMathOperator{\rot}{rot}
\DeclareMathOperator{\trs}{trace}
\DeclareMathOperator{\Ind}{Ind}
\DeclareMathOperator{\mat}{Mat}
\DeclareMathOperator{\id}{Id}
\DeclareMathOperator{\vect}{vect}
\DeclareMathOperator{\img}{img}
\DeclareMathOperator{\cov}{Cov}
\DeclareMathOperator{\dist}{dist}
\DeclareMathOperator{\irr}{Irr}
\DeclareMathOperator{\image}{Im}
\DeclareMathOperator{\pd}{\partial}
\DeclareMathOperator{\epi}{epi}
\DeclareMathOperator{\Argmin}{Argmin}
\DeclareMathOperator{\dom}{dom}
\DeclareMathOperator{\proj}{proj}
\DeclareMathOperator{\ctg}{ctg}
\DeclareMathOperator{\supp}{supp}
\DeclareMathOperator{\argmin}{argmin}
\DeclareMathOperator{\mult}{mult}
\DeclareMathOperator{\ch}{ch}
\DeclareMathOperator{\sh}{sh}
\DeclareMathOperator{\rang}{rang}
\DeclareMathOperator{\diam}{diam}
\DeclareMathOperator{\Epigraphe}{Epigraphe}




\usepackage{xcolor}
\everymath{\color{blue}}
%\everymath{\color[rgb]{0,1,1}}
%\pagecolor[rgb]{0,0,0.5}


\newcommand*{\pdtest}[3][]{\ensuremath{\frac{\partial^{#1} #2}{\partial #3}}}

\newcommand*{\deffunc}[6][]{\ensuremath{
\begin{array}{rcl}
#2 : #3 &\rightarrow& #4\\
#5 &\mapsto& #6
\end{array}
}}

\newcommand{\eqcolon}{\mathrel{\resizebox{\widthof{$\mathord{=}$}}{\height}{ $\!\!=\!\!\resizebox{1.2\width}{0.8\height}{\raisebox{0.23ex}{$\mathop{:}$}}\!\!$ }}}
\newcommand{\coloneq}{\mathrel{\resizebox{\widthof{$\mathord{=}$}}{\height}{ $\!\!\resizebox{1.2\width}{0.8\height}{\raisebox{0.23ex}{$\mathop{:}$}}\!\!=\!\!$ }}}
\newcommand{\eqcolonl}{\ensuremath{\mathrel{=\!\!\mathop{:}}}}
\newcommand{\coloneql}{\ensuremath{\mathrel{\mathop{:} \!\! =}}}
\newcommand{\vc}[1]{% inline column vector
  \left(\begin{smallmatrix}#1\end{smallmatrix}\right)%
}
\newcommand{\vr}[1]{% inline row vector
  \begin{smallmatrix}(\,#1\,)\end{smallmatrix}%
}
\makeatletter
\newcommand*{\defeq}{\ =\mathrel{\rlap{%
                     \raisebox{0.3ex}{$\m@th\cdot$}}%
                     \raisebox{-0.3ex}{$\m@th\cdot$}}%
                     }
\makeatother

\newcommand{\mathcircle}[1]{% inline row vector
 \overset{\circ}{#1}
}
\newcommand{\ulim}{% low limit
 \underline{\lim}
}
\newcommand{\ssi}{% iff
\iff
}
\newcommand{\ps}[2]{
\expval{#1 | #2}
}
\newcommand{\df}[1]{
\mqty{#1}
}
\newcommand{\n}[1]{
\norm{#1}
}
\newcommand{\sys}[1]{
\left\{\smqty{#1}\right.
}


\newcommand{\eqdef}{\ensuremath{\overset{\text{def}}=}}


\def\Circlearrowright{\ensuremath{%
  \rotatebox[origin=c]{230}{$\circlearrowright$}}}

\newcommand\ct[1]{\text{\rmfamily\upshape #1}}
\newcommand\question[1]{ {\color{red} ...!? \small #1}}
\newcommand\caz[1]{\left\{\begin{array} #1 \end{array}\right.}
\newcommand\const{\text{\rmfamily\upshape const}}
\newcommand\toP{ \overset{\pro}{\to}}
\newcommand\toPP{ \overset{\text{PP}}{\to}}
\newcommand{\oeq}{\mathrel{\text{\textcircled{$=$}}}}





\usepackage{xcolor}
% \usepackage[normalem]{ulem}
\usepackage{lipsum}
\makeatletter
% \newcommand\colorwave[1][blue]{\bgroup \markoverwith{\lower3.5\p@\hbox{\sixly \textcolor{#1}{\char58}}}\ULon}
%\font\sixly=lasy6 % does not re-load if already loaded, so no memory problem.

\newmdtheoremenv[
linewidth= 1pt,linecolor= blue,%
leftmargin=20,rightmargin=20,innertopmargin=0pt, innerrightmargin=40,%
tikzsetting = { draw=lightgray, line width = 0.3pt,dashed,%
dash pattern = on 15pt off 3pt},%
splittopskip=\topskip,skipbelow=\baselineskip,%
skipabove=\baselineskip,ntheorem,roundcorner=0pt,
% backgroundcolor=pagebg,font=\color{orange}\sffamily, fontcolor=white
]{examplebox}{Exemple}[section]



\newcommand\R{\mathbb{R}}
\newcommand\Z{\mathbb{Z}}
\newcommand\N{\mathbb{N}}
\newcommand\E{\mathbb{E}}
\newcommand\F{\mathcal{F}}
\newcommand\cH{\mathcal{H}}
\newcommand\V{\mathbb{V}}
\newcommand\dmo{ ^{-1} }
\newcommand\kapa{\kappa}
\newcommand\im{Im}
\newcommand\hs{\mathcal{H}}





\usepackage{soul}

\makeatletter
\newcommand*{\whiten}[1]{\llap{\textcolor{white}{{\the\SOUL@token}}\hspace{#1pt}}}
\DeclareRobustCommand*\myul{%
    \def\SOUL@everyspace{\underline{\space}\kern\z@}%
    \def\SOUL@everytoken{%
     \setbox0=\hbox{\the\SOUL@token}%
     \ifdim\dp0>\z@
        \raisebox{\dp0}{\underline{\phantom{\the\SOUL@token}}}%
        \whiten{1}\whiten{0}%
        \whiten{-1}\whiten{-2}%
        \llap{\the\SOUL@token}%
     \else
        \underline{\the\SOUL@token}%
     \fi}%
\SOUL@}
\makeatother

\newcommand*{\demp}{\fontfamily{lmtt}\selectfont}

\DeclareTextFontCommand{\textdemp}{\demp}

\begin{document}

\ifcomment
Multiline
comment
\fi
\ifcomment
\myul{Typesetting test}
% \color[rgb]{1,1,1}
$∑_i^n≠ 60º±∞π∆¬≈√j∫h≤≥µ$

$\CR \R\pro\ind\pro\gS\pro
\mqty[a&b\\c&d]$
$\pro\mathbb{P}$
$\dd{x}$

  \[
    \alpha(x)=\left\{
                \begin{array}{ll}
                  x\\
                  \frac{1}{1+e^{-kx}}\\
                  \frac{e^x-e^{-x}}{e^x+e^{-x}}
                \end{array}
              \right.
  \]

  $\expval{x}$
  
  $\chi_\rho(ghg\dmo)=\Tr(\rho_{ghg\dmo})=\Tr(\rho_g\circ\rho_h\circ\rho\dmo_g)=\Tr(\rho_h)\overset{\mbox{\scalebox{0.5}{$\Tr(AB)=\Tr(BA)$}}}{=}\chi_\rho(h)$
  	$\mathop{\oplus}_{\substack{x\in X}}$

$\mat(\rho_g)=(a_{ij}(g))_{\scriptsize \substack{1\leq i\leq d \\ 1\leq j\leq d}}$ et $\mat(\rho'_g)=(a'_{ij}(g))_{\scriptsize \substack{1\leq i'\leq d' \\ 1\leq j'\leq d'}}$



\[\int_a^b{\mathbb{R}^2}g(u, v)\dd{P_{XY}}(u, v)=\iint g(u,v) f_{XY}(u, v)\dd \lambda(u) \dd \lambda(v)\]
$$\lim_{x\to\infty} f(x)$$	
$$\iiiint_V \mu(t,u,v,w) \,dt\,du\,dv\,dw$$
$$\sum_{n=1}^{\infty} 2^{-n} = 1$$	
\begin{definition}
	Si $X$ et $Y$ sont 2 v.a. ou definit la \textsc{Covariance} entre $X$ et $Y$ comme
	$\cov(X,Y)\overset{\text{def}}{=}\E\left[(X-\E(X))(Y-\E(Y))\right]=\E(XY)-\E(X)\E(Y)$.
\end{definition}
\fi
\pagebreak

% \tableofcontents

% insert your code here
%\input{./algebra/main.tex}
%\input{./geometrie-differentielle/main.tex}
%\input{./probabilite/main.tex}
%\input{./analyse-fonctionnelle/main.tex}
% \input{./Analyse-convexe-et-dualite-en-optimisation/main.tex}
%\input{./tikz/main.tex}
%\input{./Theorie-du-distributions/main.tex}
%\input{./optimisation/mine.tex}
 \input{./modelisation/main.tex}

% yves.aubry@univ-tln.fr : algebra

\end{document}

%% !TEX encoding = UTF-8 Unicode
% !TEX TS-program = xelatex

\documentclass[french]{report}

%\usepackage[utf8]{inputenc}
%\usepackage[T1]{fontenc}
\usepackage{babel}


\newif\ifcomment
%\commenttrue # Show comments

\usepackage{physics}
\usepackage{amssymb}


\usepackage{amsthm}
% \usepackage{thmtools}
\usepackage{mathtools}
\usepackage{amsfonts}

\usepackage{color}

\usepackage{tikz}

\usepackage{geometry}
\geometry{a5paper, margin=0.1in, right=1cm}

\usepackage{dsfont}

\usepackage{graphicx}
\graphicspath{ {images/} }

\usepackage{faktor}

\usepackage{IEEEtrantools}
\usepackage{enumerate}   
\usepackage[PostScript=dvips]{"/Users/aware/Documents/Courses/diagrams"}


\newtheorem{theorem}{Théorème}[section]
\renewcommand{\thetheorem}{\arabic{theorem}}
\newtheorem{lemme}{Lemme}[section]
\renewcommand{\thelemme}{\arabic{lemme}}
\newtheorem{proposition}{Proposition}[section]
\renewcommand{\theproposition}{\arabic{proposition}}
\newtheorem{notations}{Notations}[section]
\newtheorem{problem}{Problème}[section]
\newtheorem{corollary}{Corollaire}[theorem]
\renewcommand{\thecorollary}{\arabic{corollary}}
\newtheorem{property}{Propriété}[section]
\newtheorem{objective}{Objectif}[section]

\theoremstyle{definition}
\newtheorem{definition}{Définition}[section]
\renewcommand{\thedefinition}{\arabic{definition}}
\newtheorem{exercise}{Exercice}[chapter]
\renewcommand{\theexercise}{\arabic{exercise}}
\newtheorem{example}{Exemple}[chapter]
\renewcommand{\theexample}{\arabic{example}}
\newtheorem*{solution}{Solution}
\newtheorem*{application}{Application}
\newtheorem*{notation}{Notation}
\newtheorem*{vocabulary}{Vocabulaire}
\newtheorem*{properties}{Propriétés}



\theoremstyle{remark}
\newtheorem*{remark}{Remarque}
\newtheorem*{rappel}{Rappel}


\usepackage{etoolbox}
\AtBeginEnvironment{exercise}{\small}
\AtBeginEnvironment{example}{\small}

\usepackage{cases}
\usepackage[red]{mypack}

\usepackage[framemethod=TikZ]{mdframed}

\definecolor{bg}{rgb}{0.4,0.25,0.95}
\definecolor{pagebg}{rgb}{0,0,0.5}
\surroundwithmdframed[
   topline=false,
   rightline=false,
   bottomline=false,
   leftmargin=\parindent,
   skipabove=8pt,
   skipbelow=8pt,
   linecolor=blue,
   innerbottommargin=10pt,
   % backgroundcolor=bg,font=\color{orange}\sffamily, fontcolor=white
]{definition}

\usepackage{empheq}
\usepackage[most]{tcolorbox}

\newtcbox{\mymath}[1][]{%
    nobeforeafter, math upper, tcbox raise base,
    enhanced, colframe=blue!30!black,
    colback=red!10, boxrule=1pt,
    #1}

\usepackage{unixode}


\DeclareMathOperator{\ord}{ord}
\DeclareMathOperator{\orb}{orb}
\DeclareMathOperator{\stab}{stab}
\DeclareMathOperator{\Stab}{stab}
\DeclareMathOperator{\ppcm}{ppcm}
\DeclareMathOperator{\conj}{Conj}
\DeclareMathOperator{\End}{End}
\DeclareMathOperator{\rot}{rot}
\DeclareMathOperator{\trs}{trace}
\DeclareMathOperator{\Ind}{Ind}
\DeclareMathOperator{\mat}{Mat}
\DeclareMathOperator{\id}{Id}
\DeclareMathOperator{\vect}{vect}
\DeclareMathOperator{\img}{img}
\DeclareMathOperator{\cov}{Cov}
\DeclareMathOperator{\dist}{dist}
\DeclareMathOperator{\irr}{Irr}
\DeclareMathOperator{\image}{Im}
\DeclareMathOperator{\pd}{\partial}
\DeclareMathOperator{\epi}{epi}
\DeclareMathOperator{\Argmin}{Argmin}
\DeclareMathOperator{\dom}{dom}
\DeclareMathOperator{\proj}{proj}
\DeclareMathOperator{\ctg}{ctg}
\DeclareMathOperator{\supp}{supp}
\DeclareMathOperator{\argmin}{argmin}
\DeclareMathOperator{\mult}{mult}
\DeclareMathOperator{\ch}{ch}
\DeclareMathOperator{\sh}{sh}
\DeclareMathOperator{\rang}{rang}
\DeclareMathOperator{\diam}{diam}
\DeclareMathOperator{\Epigraphe}{Epigraphe}




\usepackage{xcolor}
\everymath{\color{blue}}
%\everymath{\color[rgb]{0,1,1}}
%\pagecolor[rgb]{0,0,0.5}


\newcommand*{\pdtest}[3][]{\ensuremath{\frac{\partial^{#1} #2}{\partial #3}}}

\newcommand*{\deffunc}[6][]{\ensuremath{
\begin{array}{rcl}
#2 : #3 &\rightarrow& #4\\
#5 &\mapsto& #6
\end{array}
}}

\newcommand{\eqcolon}{\mathrel{\resizebox{\widthof{$\mathord{=}$}}{\height}{ $\!\!=\!\!\resizebox{1.2\width}{0.8\height}{\raisebox{0.23ex}{$\mathop{:}$}}\!\!$ }}}
\newcommand{\coloneq}{\mathrel{\resizebox{\widthof{$\mathord{=}$}}{\height}{ $\!\!\resizebox{1.2\width}{0.8\height}{\raisebox{0.23ex}{$\mathop{:}$}}\!\!=\!\!$ }}}
\newcommand{\eqcolonl}{\ensuremath{\mathrel{=\!\!\mathop{:}}}}
\newcommand{\coloneql}{\ensuremath{\mathrel{\mathop{:} \!\! =}}}
\newcommand{\vc}[1]{% inline column vector
  \left(\begin{smallmatrix}#1\end{smallmatrix}\right)%
}
\newcommand{\vr}[1]{% inline row vector
  \begin{smallmatrix}(\,#1\,)\end{smallmatrix}%
}
\makeatletter
\newcommand*{\defeq}{\ =\mathrel{\rlap{%
                     \raisebox{0.3ex}{$\m@th\cdot$}}%
                     \raisebox{-0.3ex}{$\m@th\cdot$}}%
                     }
\makeatother

\newcommand{\mathcircle}[1]{% inline row vector
 \overset{\circ}{#1}
}
\newcommand{\ulim}{% low limit
 \underline{\lim}
}
\newcommand{\ssi}{% iff
\iff
}
\newcommand{\ps}[2]{
\expval{#1 | #2}
}
\newcommand{\df}[1]{
\mqty{#1}
}
\newcommand{\n}[1]{
\norm{#1}
}
\newcommand{\sys}[1]{
\left\{\smqty{#1}\right.
}


\newcommand{\eqdef}{\ensuremath{\overset{\text{def}}=}}


\def\Circlearrowright{\ensuremath{%
  \rotatebox[origin=c]{230}{$\circlearrowright$}}}

\newcommand\ct[1]{\text{\rmfamily\upshape #1}}
\newcommand\question[1]{ {\color{red} ...!? \small #1}}
\newcommand\caz[1]{\left\{\begin{array} #1 \end{array}\right.}
\newcommand\const{\text{\rmfamily\upshape const}}
\newcommand\toP{ \overset{\pro}{\to}}
\newcommand\toPP{ \overset{\text{PP}}{\to}}
\newcommand{\oeq}{\mathrel{\text{\textcircled{$=$}}}}





\usepackage{xcolor}
% \usepackage[normalem]{ulem}
\usepackage{lipsum}
\makeatletter
% \newcommand\colorwave[1][blue]{\bgroup \markoverwith{\lower3.5\p@\hbox{\sixly \textcolor{#1}{\char58}}}\ULon}
%\font\sixly=lasy6 % does not re-load if already loaded, so no memory problem.

\newmdtheoremenv[
linewidth= 1pt,linecolor= blue,%
leftmargin=20,rightmargin=20,innertopmargin=0pt, innerrightmargin=40,%
tikzsetting = { draw=lightgray, line width = 0.3pt,dashed,%
dash pattern = on 15pt off 3pt},%
splittopskip=\topskip,skipbelow=\baselineskip,%
skipabove=\baselineskip,ntheorem,roundcorner=0pt,
% backgroundcolor=pagebg,font=\color{orange}\sffamily, fontcolor=white
]{examplebox}{Exemple}[section]



\newcommand\R{\mathbb{R}}
\newcommand\Z{\mathbb{Z}}
\newcommand\N{\mathbb{N}}
\newcommand\E{\mathbb{E}}
\newcommand\F{\mathcal{F}}
\newcommand\cH{\mathcal{H}}
\newcommand\V{\mathbb{V}}
\newcommand\dmo{ ^{-1} }
\newcommand\kapa{\kappa}
\newcommand\im{Im}
\newcommand\hs{\mathcal{H}}





\usepackage{soul}

\makeatletter
\newcommand*{\whiten}[1]{\llap{\textcolor{white}{{\the\SOUL@token}}\hspace{#1pt}}}
\DeclareRobustCommand*\myul{%
    \def\SOUL@everyspace{\underline{\space}\kern\z@}%
    \def\SOUL@everytoken{%
     \setbox0=\hbox{\the\SOUL@token}%
     \ifdim\dp0>\z@
        \raisebox{\dp0}{\underline{\phantom{\the\SOUL@token}}}%
        \whiten{1}\whiten{0}%
        \whiten{-1}\whiten{-2}%
        \llap{\the\SOUL@token}%
     \else
        \underline{\the\SOUL@token}%
     \fi}%
\SOUL@}
\makeatother

\newcommand*{\demp}{\fontfamily{lmtt}\selectfont}

\DeclareTextFontCommand{\textdemp}{\demp}

\begin{document}

\ifcomment
Multiline
comment
\fi
\ifcomment
\myul{Typesetting test}
% \color[rgb]{1,1,1}
$∑_i^n≠ 60º±∞π∆¬≈√j∫h≤≥µ$

$\CR \R\pro\ind\pro\gS\pro
\mqty[a&b\\c&d]$
$\pro\mathbb{P}$
$\dd{x}$

  \[
    \alpha(x)=\left\{
                \begin{array}{ll}
                  x\\
                  \frac{1}{1+e^{-kx}}\\
                  \frac{e^x-e^{-x}}{e^x+e^{-x}}
                \end{array}
              \right.
  \]

  $\expval{x}$
  
  $\chi_\rho(ghg\dmo)=\Tr(\rho_{ghg\dmo})=\Tr(\rho_g\circ\rho_h\circ\rho\dmo_g)=\Tr(\rho_h)\overset{\mbox{\scalebox{0.5}{$\Tr(AB)=\Tr(BA)$}}}{=}\chi_\rho(h)$
  	$\mathop{\oplus}_{\substack{x\in X}}$

$\mat(\rho_g)=(a_{ij}(g))_{\scriptsize \substack{1\leq i\leq d \\ 1\leq j\leq d}}$ et $\mat(\rho'_g)=(a'_{ij}(g))_{\scriptsize \substack{1\leq i'\leq d' \\ 1\leq j'\leq d'}}$



\[\int_a^b{\mathbb{R}^2}g(u, v)\dd{P_{XY}}(u, v)=\iint g(u,v) f_{XY}(u, v)\dd \lambda(u) \dd \lambda(v)\]
$$\lim_{x\to\infty} f(x)$$	
$$\iiiint_V \mu(t,u,v,w) \,dt\,du\,dv\,dw$$
$$\sum_{n=1}^{\infty} 2^{-n} = 1$$	
\begin{definition}
	Si $X$ et $Y$ sont 2 v.a. ou definit la \textsc{Covariance} entre $X$ et $Y$ comme
	$\cov(X,Y)\overset{\text{def}}{=}\E\left[(X-\E(X))(Y-\E(Y))\right]=\E(XY)-\E(X)\E(Y)$.
\end{definition}
\fi
\pagebreak

% \tableofcontents

% insert your code here
%\input{./algebra/main.tex}
%\input{./geometrie-differentielle/main.tex}
%\input{./probabilite/main.tex}
%\input{./analyse-fonctionnelle/main.tex}
% \input{./Analyse-convexe-et-dualite-en-optimisation/main.tex}
%\input{./tikz/main.tex}
%\input{./Theorie-du-distributions/main.tex}
%\input{./optimisation/mine.tex}
 \input{./modelisation/main.tex}

% yves.aubry@univ-tln.fr : algebra

\end{document}

%\input{./optimisation/mine.tex}
 % !TEX encoding = UTF-8 Unicode
% !TEX TS-program = xelatex

\documentclass[french]{report}

%\usepackage[utf8]{inputenc}
%\usepackage[T1]{fontenc}
\usepackage{babel}


\newif\ifcomment
%\commenttrue # Show comments

\usepackage{physics}
\usepackage{amssymb}


\usepackage{amsthm}
% \usepackage{thmtools}
\usepackage{mathtools}
\usepackage{amsfonts}

\usepackage{color}

\usepackage{tikz}

\usepackage{geometry}
\geometry{a5paper, margin=0.1in, right=1cm}

\usepackage{dsfont}

\usepackage{graphicx}
\graphicspath{ {images/} }

\usepackage{faktor}

\usepackage{IEEEtrantools}
\usepackage{enumerate}   
\usepackage[PostScript=dvips]{"/Users/aware/Documents/Courses/diagrams"}


\newtheorem{theorem}{Théorème}[section]
\renewcommand{\thetheorem}{\arabic{theorem}}
\newtheorem{lemme}{Lemme}[section]
\renewcommand{\thelemme}{\arabic{lemme}}
\newtheorem{proposition}{Proposition}[section]
\renewcommand{\theproposition}{\arabic{proposition}}
\newtheorem{notations}{Notations}[section]
\newtheorem{problem}{Problème}[section]
\newtheorem{corollary}{Corollaire}[theorem]
\renewcommand{\thecorollary}{\arabic{corollary}}
\newtheorem{property}{Propriété}[section]
\newtheorem{objective}{Objectif}[section]

\theoremstyle{definition}
\newtheorem{definition}{Définition}[section]
\renewcommand{\thedefinition}{\arabic{definition}}
\newtheorem{exercise}{Exercice}[chapter]
\renewcommand{\theexercise}{\arabic{exercise}}
\newtheorem{example}{Exemple}[chapter]
\renewcommand{\theexample}{\arabic{example}}
\newtheorem*{solution}{Solution}
\newtheorem*{application}{Application}
\newtheorem*{notation}{Notation}
\newtheorem*{vocabulary}{Vocabulaire}
\newtheorem*{properties}{Propriétés}



\theoremstyle{remark}
\newtheorem*{remark}{Remarque}
\newtheorem*{rappel}{Rappel}


\usepackage{etoolbox}
\AtBeginEnvironment{exercise}{\small}
\AtBeginEnvironment{example}{\small}

\usepackage{cases}
\usepackage[red]{mypack}

\usepackage[framemethod=TikZ]{mdframed}

\definecolor{bg}{rgb}{0.4,0.25,0.95}
\definecolor{pagebg}{rgb}{0,0,0.5}
\surroundwithmdframed[
   topline=false,
   rightline=false,
   bottomline=false,
   leftmargin=\parindent,
   skipabove=8pt,
   skipbelow=8pt,
   linecolor=blue,
   innerbottommargin=10pt,
   % backgroundcolor=bg,font=\color{orange}\sffamily, fontcolor=white
]{definition}

\usepackage{empheq}
\usepackage[most]{tcolorbox}

\newtcbox{\mymath}[1][]{%
    nobeforeafter, math upper, tcbox raise base,
    enhanced, colframe=blue!30!black,
    colback=red!10, boxrule=1pt,
    #1}

\usepackage{unixode}


\DeclareMathOperator{\ord}{ord}
\DeclareMathOperator{\orb}{orb}
\DeclareMathOperator{\stab}{stab}
\DeclareMathOperator{\Stab}{stab}
\DeclareMathOperator{\ppcm}{ppcm}
\DeclareMathOperator{\conj}{Conj}
\DeclareMathOperator{\End}{End}
\DeclareMathOperator{\rot}{rot}
\DeclareMathOperator{\trs}{trace}
\DeclareMathOperator{\Ind}{Ind}
\DeclareMathOperator{\mat}{Mat}
\DeclareMathOperator{\id}{Id}
\DeclareMathOperator{\vect}{vect}
\DeclareMathOperator{\img}{img}
\DeclareMathOperator{\cov}{Cov}
\DeclareMathOperator{\dist}{dist}
\DeclareMathOperator{\irr}{Irr}
\DeclareMathOperator{\image}{Im}
\DeclareMathOperator{\pd}{\partial}
\DeclareMathOperator{\epi}{epi}
\DeclareMathOperator{\Argmin}{Argmin}
\DeclareMathOperator{\dom}{dom}
\DeclareMathOperator{\proj}{proj}
\DeclareMathOperator{\ctg}{ctg}
\DeclareMathOperator{\supp}{supp}
\DeclareMathOperator{\argmin}{argmin}
\DeclareMathOperator{\mult}{mult}
\DeclareMathOperator{\ch}{ch}
\DeclareMathOperator{\sh}{sh}
\DeclareMathOperator{\rang}{rang}
\DeclareMathOperator{\diam}{diam}
\DeclareMathOperator{\Epigraphe}{Epigraphe}




\usepackage{xcolor}
\everymath{\color{blue}}
%\everymath{\color[rgb]{0,1,1}}
%\pagecolor[rgb]{0,0,0.5}


\newcommand*{\pdtest}[3][]{\ensuremath{\frac{\partial^{#1} #2}{\partial #3}}}

\newcommand*{\deffunc}[6][]{\ensuremath{
\begin{array}{rcl}
#2 : #3 &\rightarrow& #4\\
#5 &\mapsto& #6
\end{array}
}}

\newcommand{\eqcolon}{\mathrel{\resizebox{\widthof{$\mathord{=}$}}{\height}{ $\!\!=\!\!\resizebox{1.2\width}{0.8\height}{\raisebox{0.23ex}{$\mathop{:}$}}\!\!$ }}}
\newcommand{\coloneq}{\mathrel{\resizebox{\widthof{$\mathord{=}$}}{\height}{ $\!\!\resizebox{1.2\width}{0.8\height}{\raisebox{0.23ex}{$\mathop{:}$}}\!\!=\!\!$ }}}
\newcommand{\eqcolonl}{\ensuremath{\mathrel{=\!\!\mathop{:}}}}
\newcommand{\coloneql}{\ensuremath{\mathrel{\mathop{:} \!\! =}}}
\newcommand{\vc}[1]{% inline column vector
  \left(\begin{smallmatrix}#1\end{smallmatrix}\right)%
}
\newcommand{\vr}[1]{% inline row vector
  \begin{smallmatrix}(\,#1\,)\end{smallmatrix}%
}
\makeatletter
\newcommand*{\defeq}{\ =\mathrel{\rlap{%
                     \raisebox{0.3ex}{$\m@th\cdot$}}%
                     \raisebox{-0.3ex}{$\m@th\cdot$}}%
                     }
\makeatother

\newcommand{\mathcircle}[1]{% inline row vector
 \overset{\circ}{#1}
}
\newcommand{\ulim}{% low limit
 \underline{\lim}
}
\newcommand{\ssi}{% iff
\iff
}
\newcommand{\ps}[2]{
\expval{#1 | #2}
}
\newcommand{\df}[1]{
\mqty{#1}
}
\newcommand{\n}[1]{
\norm{#1}
}
\newcommand{\sys}[1]{
\left\{\smqty{#1}\right.
}


\newcommand{\eqdef}{\ensuremath{\overset{\text{def}}=}}


\def\Circlearrowright{\ensuremath{%
  \rotatebox[origin=c]{230}{$\circlearrowright$}}}

\newcommand\ct[1]{\text{\rmfamily\upshape #1}}
\newcommand\question[1]{ {\color{red} ...!? \small #1}}
\newcommand\caz[1]{\left\{\begin{array} #1 \end{array}\right.}
\newcommand\const{\text{\rmfamily\upshape const}}
\newcommand\toP{ \overset{\pro}{\to}}
\newcommand\toPP{ \overset{\text{PP}}{\to}}
\newcommand{\oeq}{\mathrel{\text{\textcircled{$=$}}}}





\usepackage{xcolor}
% \usepackage[normalem]{ulem}
\usepackage{lipsum}
\makeatletter
% \newcommand\colorwave[1][blue]{\bgroup \markoverwith{\lower3.5\p@\hbox{\sixly \textcolor{#1}{\char58}}}\ULon}
%\font\sixly=lasy6 % does not re-load if already loaded, so no memory problem.

\newmdtheoremenv[
linewidth= 1pt,linecolor= blue,%
leftmargin=20,rightmargin=20,innertopmargin=0pt, innerrightmargin=40,%
tikzsetting = { draw=lightgray, line width = 0.3pt,dashed,%
dash pattern = on 15pt off 3pt},%
splittopskip=\topskip,skipbelow=\baselineskip,%
skipabove=\baselineskip,ntheorem,roundcorner=0pt,
% backgroundcolor=pagebg,font=\color{orange}\sffamily, fontcolor=white
]{examplebox}{Exemple}[section]



\newcommand\R{\mathbb{R}}
\newcommand\Z{\mathbb{Z}}
\newcommand\N{\mathbb{N}}
\newcommand\E{\mathbb{E}}
\newcommand\F{\mathcal{F}}
\newcommand\cH{\mathcal{H}}
\newcommand\V{\mathbb{V}}
\newcommand\dmo{ ^{-1} }
\newcommand\kapa{\kappa}
\newcommand\im{Im}
\newcommand\hs{\mathcal{H}}





\usepackage{soul}

\makeatletter
\newcommand*{\whiten}[1]{\llap{\textcolor{white}{{\the\SOUL@token}}\hspace{#1pt}}}
\DeclareRobustCommand*\myul{%
    \def\SOUL@everyspace{\underline{\space}\kern\z@}%
    \def\SOUL@everytoken{%
     \setbox0=\hbox{\the\SOUL@token}%
     \ifdim\dp0>\z@
        \raisebox{\dp0}{\underline{\phantom{\the\SOUL@token}}}%
        \whiten{1}\whiten{0}%
        \whiten{-1}\whiten{-2}%
        \llap{\the\SOUL@token}%
     \else
        \underline{\the\SOUL@token}%
     \fi}%
\SOUL@}
\makeatother

\newcommand*{\demp}{\fontfamily{lmtt}\selectfont}

\DeclareTextFontCommand{\textdemp}{\demp}

\begin{document}

\ifcomment
Multiline
comment
\fi
\ifcomment
\myul{Typesetting test}
% \color[rgb]{1,1,1}
$∑_i^n≠ 60º±∞π∆¬≈√j∫h≤≥µ$

$\CR \R\pro\ind\pro\gS\pro
\mqty[a&b\\c&d]$
$\pro\mathbb{P}$
$\dd{x}$

  \[
    \alpha(x)=\left\{
                \begin{array}{ll}
                  x\\
                  \frac{1}{1+e^{-kx}}\\
                  \frac{e^x-e^{-x}}{e^x+e^{-x}}
                \end{array}
              \right.
  \]

  $\expval{x}$
  
  $\chi_\rho(ghg\dmo)=\Tr(\rho_{ghg\dmo})=\Tr(\rho_g\circ\rho_h\circ\rho\dmo_g)=\Tr(\rho_h)\overset{\mbox{\scalebox{0.5}{$\Tr(AB)=\Tr(BA)$}}}{=}\chi_\rho(h)$
  	$\mathop{\oplus}_{\substack{x\in X}}$

$\mat(\rho_g)=(a_{ij}(g))_{\scriptsize \substack{1\leq i\leq d \\ 1\leq j\leq d}}$ et $\mat(\rho'_g)=(a'_{ij}(g))_{\scriptsize \substack{1\leq i'\leq d' \\ 1\leq j'\leq d'}}$



\[\int_a^b{\mathbb{R}^2}g(u, v)\dd{P_{XY}}(u, v)=\iint g(u,v) f_{XY}(u, v)\dd \lambda(u) \dd \lambda(v)\]
$$\lim_{x\to\infty} f(x)$$	
$$\iiiint_V \mu(t,u,v,w) \,dt\,du\,dv\,dw$$
$$\sum_{n=1}^{\infty} 2^{-n} = 1$$	
\begin{definition}
	Si $X$ et $Y$ sont 2 v.a. ou definit la \textsc{Covariance} entre $X$ et $Y$ comme
	$\cov(X,Y)\overset{\text{def}}{=}\E\left[(X-\E(X))(Y-\E(Y))\right]=\E(XY)-\E(X)\E(Y)$.
\end{definition}
\fi
\pagebreak

% \tableofcontents

% insert your code here
%\input{./algebra/main.tex}
%\input{./geometrie-differentielle/main.tex}
%\input{./probabilite/main.tex}
%\input{./analyse-fonctionnelle/main.tex}
% \input{./Analyse-convexe-et-dualite-en-optimisation/main.tex}
%\input{./tikz/main.tex}
%\input{./Theorie-du-distributions/main.tex}
%\input{./optimisation/mine.tex}
 \input{./modelisation/main.tex}

% yves.aubry@univ-tln.fr : algebra

\end{document}


% yves.aubry@univ-tln.fr : algebra

\end{document}

%% !TEX encoding = UTF-8 Unicode
% !TEX TS-program = xelatex

\documentclass[french]{report}

%\usepackage[utf8]{inputenc}
%\usepackage[T1]{fontenc}
\usepackage{babel}


\newif\ifcomment
%\commenttrue # Show comments

\usepackage{physics}
\usepackage{amssymb}


\usepackage{amsthm}
% \usepackage{thmtools}
\usepackage{mathtools}
\usepackage{amsfonts}

\usepackage{color}

\usepackage{tikz}

\usepackage{geometry}
\geometry{a5paper, margin=0.1in, right=1cm}

\usepackage{dsfont}

\usepackage{graphicx}
\graphicspath{ {images/} }

\usepackage{faktor}

\usepackage{IEEEtrantools}
\usepackage{enumerate}   
\usepackage[PostScript=dvips]{"/Users/aware/Documents/Courses/diagrams"}


\newtheorem{theorem}{Théorème}[section]
\renewcommand{\thetheorem}{\arabic{theorem}}
\newtheorem{lemme}{Lemme}[section]
\renewcommand{\thelemme}{\arabic{lemme}}
\newtheorem{proposition}{Proposition}[section]
\renewcommand{\theproposition}{\arabic{proposition}}
\newtheorem{notations}{Notations}[section]
\newtheorem{problem}{Problème}[section]
\newtheorem{corollary}{Corollaire}[theorem]
\renewcommand{\thecorollary}{\arabic{corollary}}
\newtheorem{property}{Propriété}[section]
\newtheorem{objective}{Objectif}[section]

\theoremstyle{definition}
\newtheorem{definition}{Définition}[section]
\renewcommand{\thedefinition}{\arabic{definition}}
\newtheorem{exercise}{Exercice}[chapter]
\renewcommand{\theexercise}{\arabic{exercise}}
\newtheorem{example}{Exemple}[chapter]
\renewcommand{\theexample}{\arabic{example}}
\newtheorem*{solution}{Solution}
\newtheorem*{application}{Application}
\newtheorem*{notation}{Notation}
\newtheorem*{vocabulary}{Vocabulaire}
\newtheorem*{properties}{Propriétés}



\theoremstyle{remark}
\newtheorem*{remark}{Remarque}
\newtheorem*{rappel}{Rappel}


\usepackage{etoolbox}
\AtBeginEnvironment{exercise}{\small}
\AtBeginEnvironment{example}{\small}

\usepackage{cases}
\usepackage[red]{mypack}

\usepackage[framemethod=TikZ]{mdframed}

\definecolor{bg}{rgb}{0.4,0.25,0.95}
\definecolor{pagebg}{rgb}{0,0,0.5}
\surroundwithmdframed[
   topline=false,
   rightline=false,
   bottomline=false,
   leftmargin=\parindent,
   skipabove=8pt,
   skipbelow=8pt,
   linecolor=blue,
   innerbottommargin=10pt,
   % backgroundcolor=bg,font=\color{orange}\sffamily, fontcolor=white
]{definition}

\usepackage{empheq}
\usepackage[most]{tcolorbox}

\newtcbox{\mymath}[1][]{%
    nobeforeafter, math upper, tcbox raise base,
    enhanced, colframe=blue!30!black,
    colback=red!10, boxrule=1pt,
    #1}

\usepackage{unixode}


\DeclareMathOperator{\ord}{ord}
\DeclareMathOperator{\orb}{orb}
\DeclareMathOperator{\stab}{stab}
\DeclareMathOperator{\Stab}{stab}
\DeclareMathOperator{\ppcm}{ppcm}
\DeclareMathOperator{\conj}{Conj}
\DeclareMathOperator{\End}{End}
\DeclareMathOperator{\rot}{rot}
\DeclareMathOperator{\trs}{trace}
\DeclareMathOperator{\Ind}{Ind}
\DeclareMathOperator{\mat}{Mat}
\DeclareMathOperator{\id}{Id}
\DeclareMathOperator{\vect}{vect}
\DeclareMathOperator{\img}{img}
\DeclareMathOperator{\cov}{Cov}
\DeclareMathOperator{\dist}{dist}
\DeclareMathOperator{\irr}{Irr}
\DeclareMathOperator{\image}{Im}
\DeclareMathOperator{\pd}{\partial}
\DeclareMathOperator{\epi}{epi}
\DeclareMathOperator{\Argmin}{Argmin}
\DeclareMathOperator{\dom}{dom}
\DeclareMathOperator{\proj}{proj}
\DeclareMathOperator{\ctg}{ctg}
\DeclareMathOperator{\supp}{supp}
\DeclareMathOperator{\argmin}{argmin}
\DeclareMathOperator{\mult}{mult}
\DeclareMathOperator{\ch}{ch}
\DeclareMathOperator{\sh}{sh}
\DeclareMathOperator{\rang}{rang}
\DeclareMathOperator{\diam}{diam}
\DeclareMathOperator{\Epigraphe}{Epigraphe}




\usepackage{xcolor}
\everymath{\color{blue}}
%\everymath{\color[rgb]{0,1,1}}
%\pagecolor[rgb]{0,0,0.5}


\newcommand*{\pdtest}[3][]{\ensuremath{\frac{\partial^{#1} #2}{\partial #3}}}

\newcommand*{\deffunc}[6][]{\ensuremath{
\begin{array}{rcl}
#2 : #3 &\rightarrow& #4\\
#5 &\mapsto& #6
\end{array}
}}

\newcommand{\eqcolon}{\mathrel{\resizebox{\widthof{$\mathord{=}$}}{\height}{ $\!\!=\!\!\resizebox{1.2\width}{0.8\height}{\raisebox{0.23ex}{$\mathop{:}$}}\!\!$ }}}
\newcommand{\coloneq}{\mathrel{\resizebox{\widthof{$\mathord{=}$}}{\height}{ $\!\!\resizebox{1.2\width}{0.8\height}{\raisebox{0.23ex}{$\mathop{:}$}}\!\!=\!\!$ }}}
\newcommand{\eqcolonl}{\ensuremath{\mathrel{=\!\!\mathop{:}}}}
\newcommand{\coloneql}{\ensuremath{\mathrel{\mathop{:} \!\! =}}}
\newcommand{\vc}[1]{% inline column vector
  \left(\begin{smallmatrix}#1\end{smallmatrix}\right)%
}
\newcommand{\vr}[1]{% inline row vector
  \begin{smallmatrix}(\,#1\,)\end{smallmatrix}%
}
\makeatletter
\newcommand*{\defeq}{\ =\mathrel{\rlap{%
                     \raisebox{0.3ex}{$\m@th\cdot$}}%
                     \raisebox{-0.3ex}{$\m@th\cdot$}}%
                     }
\makeatother

\newcommand{\mathcircle}[1]{% inline row vector
 \overset{\circ}{#1}
}
\newcommand{\ulim}{% low limit
 \underline{\lim}
}
\newcommand{\ssi}{% iff
\iff
}
\newcommand{\ps}[2]{
\expval{#1 | #2}
}
\newcommand{\df}[1]{
\mqty{#1}
}
\newcommand{\n}[1]{
\norm{#1}
}
\newcommand{\sys}[1]{
\left\{\smqty{#1}\right.
}


\newcommand{\eqdef}{\ensuremath{\overset{\text{def}}=}}


\def\Circlearrowright{\ensuremath{%
  \rotatebox[origin=c]{230}{$\circlearrowright$}}}

\newcommand\ct[1]{\text{\rmfamily\upshape #1}}
\newcommand\question[1]{ {\color{red} ...!? \small #1}}
\newcommand\caz[1]{\left\{\begin{array} #1 \end{array}\right.}
\newcommand\const{\text{\rmfamily\upshape const}}
\newcommand\toP{ \overset{\pro}{\to}}
\newcommand\toPP{ \overset{\text{PP}}{\to}}
\newcommand{\oeq}{\mathrel{\text{\textcircled{$=$}}}}





\usepackage{xcolor}
% \usepackage[normalem]{ulem}
\usepackage{lipsum}
\makeatletter
% \newcommand\colorwave[1][blue]{\bgroup \markoverwith{\lower3.5\p@\hbox{\sixly \textcolor{#1}{\char58}}}\ULon}
%\font\sixly=lasy6 % does not re-load if already loaded, so no memory problem.

\newmdtheoremenv[
linewidth= 1pt,linecolor= blue,%
leftmargin=20,rightmargin=20,innertopmargin=0pt, innerrightmargin=40,%
tikzsetting = { draw=lightgray, line width = 0.3pt,dashed,%
dash pattern = on 15pt off 3pt},%
splittopskip=\topskip,skipbelow=\baselineskip,%
skipabove=\baselineskip,ntheorem,roundcorner=0pt,
% backgroundcolor=pagebg,font=\color{orange}\sffamily, fontcolor=white
]{examplebox}{Exemple}[section]



\newcommand\R{\mathbb{R}}
\newcommand\Z{\mathbb{Z}}
\newcommand\N{\mathbb{N}}
\newcommand\E{\mathbb{E}}
\newcommand\F{\mathcal{F}}
\newcommand\cH{\mathcal{H}}
\newcommand\V{\mathbb{V}}
\newcommand\dmo{ ^{-1} }
\newcommand\kapa{\kappa}
\newcommand\im{Im}
\newcommand\hs{\mathcal{H}}





\usepackage{soul}

\makeatletter
\newcommand*{\whiten}[1]{\llap{\textcolor{white}{{\the\SOUL@token}}\hspace{#1pt}}}
\DeclareRobustCommand*\myul{%
    \def\SOUL@everyspace{\underline{\space}\kern\z@}%
    \def\SOUL@everytoken{%
     \setbox0=\hbox{\the\SOUL@token}%
     \ifdim\dp0>\z@
        \raisebox{\dp0}{\underline{\phantom{\the\SOUL@token}}}%
        \whiten{1}\whiten{0}%
        \whiten{-1}\whiten{-2}%
        \llap{\the\SOUL@token}%
     \else
        \underline{\the\SOUL@token}%
     \fi}%
\SOUL@}
\makeatother

\newcommand*{\demp}{\fontfamily{lmtt}\selectfont}

\DeclareTextFontCommand{\textdemp}{\demp}

\begin{document}

\ifcomment
Multiline
comment
\fi
\ifcomment
\myul{Typesetting test}
% \color[rgb]{1,1,1}
$∑_i^n≠ 60º±∞π∆¬≈√j∫h≤≥µ$

$\CR \R\pro\ind\pro\gS\pro
\mqty[a&b\\c&d]$
$\pro\mathbb{P}$
$\dd{x}$

  \[
    \alpha(x)=\left\{
                \begin{array}{ll}
                  x\\
                  \frac{1}{1+e^{-kx}}\\
                  \frac{e^x-e^{-x}}{e^x+e^{-x}}
                \end{array}
              \right.
  \]

  $\expval{x}$
  
  $\chi_\rho(ghg\dmo)=\Tr(\rho_{ghg\dmo})=\Tr(\rho_g\circ\rho_h\circ\rho\dmo_g)=\Tr(\rho_h)\overset{\mbox{\scalebox{0.5}{$\Tr(AB)=\Tr(BA)$}}}{=}\chi_\rho(h)$
  	$\mathop{\oplus}_{\substack{x\in X}}$

$\mat(\rho_g)=(a_{ij}(g))_{\scriptsize \substack{1\leq i\leq d \\ 1\leq j\leq d}}$ et $\mat(\rho'_g)=(a'_{ij}(g))_{\scriptsize \substack{1\leq i'\leq d' \\ 1\leq j'\leq d'}}$



\[\int_a^b{\mathbb{R}^2}g(u, v)\dd{P_{XY}}(u, v)=\iint g(u,v) f_{XY}(u, v)\dd \lambda(u) \dd \lambda(v)\]
$$\lim_{x\to\infty} f(x)$$	
$$\iiiint_V \mu(t,u,v,w) \,dt\,du\,dv\,dw$$
$$\sum_{n=1}^{\infty} 2^{-n} = 1$$	
\begin{definition}
	Si $X$ et $Y$ sont 2 v.a. ou definit la \textsc{Covariance} entre $X$ et $Y$ comme
	$\cov(X,Y)\overset{\text{def}}{=}\E\left[(X-\E(X))(Y-\E(Y))\right]=\E(XY)-\E(X)\E(Y)$.
\end{definition}
\fi
\pagebreak

% \tableofcontents

% insert your code here
%% !TEX encoding = UTF-8 Unicode
% !TEX TS-program = xelatex

\documentclass[french]{report}

%\usepackage[utf8]{inputenc}
%\usepackage[T1]{fontenc}
\usepackage{babel}


\newif\ifcomment
%\commenttrue # Show comments

\usepackage{physics}
\usepackage{amssymb}


\usepackage{amsthm}
% \usepackage{thmtools}
\usepackage{mathtools}
\usepackage{amsfonts}

\usepackage{color}

\usepackage{tikz}

\usepackage{geometry}
\geometry{a5paper, margin=0.1in, right=1cm}

\usepackage{dsfont}

\usepackage{graphicx}
\graphicspath{ {images/} }

\usepackage{faktor}

\usepackage{IEEEtrantools}
\usepackage{enumerate}   
\usepackage[PostScript=dvips]{"/Users/aware/Documents/Courses/diagrams"}


\newtheorem{theorem}{Théorème}[section]
\renewcommand{\thetheorem}{\arabic{theorem}}
\newtheorem{lemme}{Lemme}[section]
\renewcommand{\thelemme}{\arabic{lemme}}
\newtheorem{proposition}{Proposition}[section]
\renewcommand{\theproposition}{\arabic{proposition}}
\newtheorem{notations}{Notations}[section]
\newtheorem{problem}{Problème}[section]
\newtheorem{corollary}{Corollaire}[theorem]
\renewcommand{\thecorollary}{\arabic{corollary}}
\newtheorem{property}{Propriété}[section]
\newtheorem{objective}{Objectif}[section]

\theoremstyle{definition}
\newtheorem{definition}{Définition}[section]
\renewcommand{\thedefinition}{\arabic{definition}}
\newtheorem{exercise}{Exercice}[chapter]
\renewcommand{\theexercise}{\arabic{exercise}}
\newtheorem{example}{Exemple}[chapter]
\renewcommand{\theexample}{\arabic{example}}
\newtheorem*{solution}{Solution}
\newtheorem*{application}{Application}
\newtheorem*{notation}{Notation}
\newtheorem*{vocabulary}{Vocabulaire}
\newtheorem*{properties}{Propriétés}



\theoremstyle{remark}
\newtheorem*{remark}{Remarque}
\newtheorem*{rappel}{Rappel}


\usepackage{etoolbox}
\AtBeginEnvironment{exercise}{\small}
\AtBeginEnvironment{example}{\small}

\usepackage{cases}
\usepackage[red]{mypack}

\usepackage[framemethod=TikZ]{mdframed}

\definecolor{bg}{rgb}{0.4,0.25,0.95}
\definecolor{pagebg}{rgb}{0,0,0.5}
\surroundwithmdframed[
   topline=false,
   rightline=false,
   bottomline=false,
   leftmargin=\parindent,
   skipabove=8pt,
   skipbelow=8pt,
   linecolor=blue,
   innerbottommargin=10pt,
   % backgroundcolor=bg,font=\color{orange}\sffamily, fontcolor=white
]{definition}

\usepackage{empheq}
\usepackage[most]{tcolorbox}

\newtcbox{\mymath}[1][]{%
    nobeforeafter, math upper, tcbox raise base,
    enhanced, colframe=blue!30!black,
    colback=red!10, boxrule=1pt,
    #1}

\usepackage{unixode}


\DeclareMathOperator{\ord}{ord}
\DeclareMathOperator{\orb}{orb}
\DeclareMathOperator{\stab}{stab}
\DeclareMathOperator{\Stab}{stab}
\DeclareMathOperator{\ppcm}{ppcm}
\DeclareMathOperator{\conj}{Conj}
\DeclareMathOperator{\End}{End}
\DeclareMathOperator{\rot}{rot}
\DeclareMathOperator{\trs}{trace}
\DeclareMathOperator{\Ind}{Ind}
\DeclareMathOperator{\mat}{Mat}
\DeclareMathOperator{\id}{Id}
\DeclareMathOperator{\vect}{vect}
\DeclareMathOperator{\img}{img}
\DeclareMathOperator{\cov}{Cov}
\DeclareMathOperator{\dist}{dist}
\DeclareMathOperator{\irr}{Irr}
\DeclareMathOperator{\image}{Im}
\DeclareMathOperator{\pd}{\partial}
\DeclareMathOperator{\epi}{epi}
\DeclareMathOperator{\Argmin}{Argmin}
\DeclareMathOperator{\dom}{dom}
\DeclareMathOperator{\proj}{proj}
\DeclareMathOperator{\ctg}{ctg}
\DeclareMathOperator{\supp}{supp}
\DeclareMathOperator{\argmin}{argmin}
\DeclareMathOperator{\mult}{mult}
\DeclareMathOperator{\ch}{ch}
\DeclareMathOperator{\sh}{sh}
\DeclareMathOperator{\rang}{rang}
\DeclareMathOperator{\diam}{diam}
\DeclareMathOperator{\Epigraphe}{Epigraphe}




\usepackage{xcolor}
\everymath{\color{blue}}
%\everymath{\color[rgb]{0,1,1}}
%\pagecolor[rgb]{0,0,0.5}


\newcommand*{\pdtest}[3][]{\ensuremath{\frac{\partial^{#1} #2}{\partial #3}}}

\newcommand*{\deffunc}[6][]{\ensuremath{
\begin{array}{rcl}
#2 : #3 &\rightarrow& #4\\
#5 &\mapsto& #6
\end{array}
}}

\newcommand{\eqcolon}{\mathrel{\resizebox{\widthof{$\mathord{=}$}}{\height}{ $\!\!=\!\!\resizebox{1.2\width}{0.8\height}{\raisebox{0.23ex}{$\mathop{:}$}}\!\!$ }}}
\newcommand{\coloneq}{\mathrel{\resizebox{\widthof{$\mathord{=}$}}{\height}{ $\!\!\resizebox{1.2\width}{0.8\height}{\raisebox{0.23ex}{$\mathop{:}$}}\!\!=\!\!$ }}}
\newcommand{\eqcolonl}{\ensuremath{\mathrel{=\!\!\mathop{:}}}}
\newcommand{\coloneql}{\ensuremath{\mathrel{\mathop{:} \!\! =}}}
\newcommand{\vc}[1]{% inline column vector
  \left(\begin{smallmatrix}#1\end{smallmatrix}\right)%
}
\newcommand{\vr}[1]{% inline row vector
  \begin{smallmatrix}(\,#1\,)\end{smallmatrix}%
}
\makeatletter
\newcommand*{\defeq}{\ =\mathrel{\rlap{%
                     \raisebox{0.3ex}{$\m@th\cdot$}}%
                     \raisebox{-0.3ex}{$\m@th\cdot$}}%
                     }
\makeatother

\newcommand{\mathcircle}[1]{% inline row vector
 \overset{\circ}{#1}
}
\newcommand{\ulim}{% low limit
 \underline{\lim}
}
\newcommand{\ssi}{% iff
\iff
}
\newcommand{\ps}[2]{
\expval{#1 | #2}
}
\newcommand{\df}[1]{
\mqty{#1}
}
\newcommand{\n}[1]{
\norm{#1}
}
\newcommand{\sys}[1]{
\left\{\smqty{#1}\right.
}


\newcommand{\eqdef}{\ensuremath{\overset{\text{def}}=}}


\def\Circlearrowright{\ensuremath{%
  \rotatebox[origin=c]{230}{$\circlearrowright$}}}

\newcommand\ct[1]{\text{\rmfamily\upshape #1}}
\newcommand\question[1]{ {\color{red} ...!? \small #1}}
\newcommand\caz[1]{\left\{\begin{array} #1 \end{array}\right.}
\newcommand\const{\text{\rmfamily\upshape const}}
\newcommand\toP{ \overset{\pro}{\to}}
\newcommand\toPP{ \overset{\text{PP}}{\to}}
\newcommand{\oeq}{\mathrel{\text{\textcircled{$=$}}}}





\usepackage{xcolor}
% \usepackage[normalem]{ulem}
\usepackage{lipsum}
\makeatletter
% \newcommand\colorwave[1][blue]{\bgroup \markoverwith{\lower3.5\p@\hbox{\sixly \textcolor{#1}{\char58}}}\ULon}
%\font\sixly=lasy6 % does not re-load if already loaded, so no memory problem.

\newmdtheoremenv[
linewidth= 1pt,linecolor= blue,%
leftmargin=20,rightmargin=20,innertopmargin=0pt, innerrightmargin=40,%
tikzsetting = { draw=lightgray, line width = 0.3pt,dashed,%
dash pattern = on 15pt off 3pt},%
splittopskip=\topskip,skipbelow=\baselineskip,%
skipabove=\baselineskip,ntheorem,roundcorner=0pt,
% backgroundcolor=pagebg,font=\color{orange}\sffamily, fontcolor=white
]{examplebox}{Exemple}[section]



\newcommand\R{\mathbb{R}}
\newcommand\Z{\mathbb{Z}}
\newcommand\N{\mathbb{N}}
\newcommand\E{\mathbb{E}}
\newcommand\F{\mathcal{F}}
\newcommand\cH{\mathcal{H}}
\newcommand\V{\mathbb{V}}
\newcommand\dmo{ ^{-1} }
\newcommand\kapa{\kappa}
\newcommand\im{Im}
\newcommand\hs{\mathcal{H}}





\usepackage{soul}

\makeatletter
\newcommand*{\whiten}[1]{\llap{\textcolor{white}{{\the\SOUL@token}}\hspace{#1pt}}}
\DeclareRobustCommand*\myul{%
    \def\SOUL@everyspace{\underline{\space}\kern\z@}%
    \def\SOUL@everytoken{%
     \setbox0=\hbox{\the\SOUL@token}%
     \ifdim\dp0>\z@
        \raisebox{\dp0}{\underline{\phantom{\the\SOUL@token}}}%
        \whiten{1}\whiten{0}%
        \whiten{-1}\whiten{-2}%
        \llap{\the\SOUL@token}%
     \else
        \underline{\the\SOUL@token}%
     \fi}%
\SOUL@}
\makeatother

\newcommand*{\demp}{\fontfamily{lmtt}\selectfont}

\DeclareTextFontCommand{\textdemp}{\demp}

\begin{document}

\ifcomment
Multiline
comment
\fi
\ifcomment
\myul{Typesetting test}
% \color[rgb]{1,1,1}
$∑_i^n≠ 60º±∞π∆¬≈√j∫h≤≥µ$

$\CR \R\pro\ind\pro\gS\pro
\mqty[a&b\\c&d]$
$\pro\mathbb{P}$
$\dd{x}$

  \[
    \alpha(x)=\left\{
                \begin{array}{ll}
                  x\\
                  \frac{1}{1+e^{-kx}}\\
                  \frac{e^x-e^{-x}}{e^x+e^{-x}}
                \end{array}
              \right.
  \]

  $\expval{x}$
  
  $\chi_\rho(ghg\dmo)=\Tr(\rho_{ghg\dmo})=\Tr(\rho_g\circ\rho_h\circ\rho\dmo_g)=\Tr(\rho_h)\overset{\mbox{\scalebox{0.5}{$\Tr(AB)=\Tr(BA)$}}}{=}\chi_\rho(h)$
  	$\mathop{\oplus}_{\substack{x\in X}}$

$\mat(\rho_g)=(a_{ij}(g))_{\scriptsize \substack{1\leq i\leq d \\ 1\leq j\leq d}}$ et $\mat(\rho'_g)=(a'_{ij}(g))_{\scriptsize \substack{1\leq i'\leq d' \\ 1\leq j'\leq d'}}$



\[\int_a^b{\mathbb{R}^2}g(u, v)\dd{P_{XY}}(u, v)=\iint g(u,v) f_{XY}(u, v)\dd \lambda(u) \dd \lambda(v)\]
$$\lim_{x\to\infty} f(x)$$	
$$\iiiint_V \mu(t,u,v,w) \,dt\,du\,dv\,dw$$
$$\sum_{n=1}^{\infty} 2^{-n} = 1$$	
\begin{definition}
	Si $X$ et $Y$ sont 2 v.a. ou definit la \textsc{Covariance} entre $X$ et $Y$ comme
	$\cov(X,Y)\overset{\text{def}}{=}\E\left[(X-\E(X))(Y-\E(Y))\right]=\E(XY)-\E(X)\E(Y)$.
\end{definition}
\fi
\pagebreak

% \tableofcontents

% insert your code here
%\input{./algebra/main.tex}
%\input{./geometrie-differentielle/main.tex}
%\input{./probabilite/main.tex}
%\input{./analyse-fonctionnelle/main.tex}
% \input{./Analyse-convexe-et-dualite-en-optimisation/main.tex}
%\input{./tikz/main.tex}
%\input{./Theorie-du-distributions/main.tex}
%\input{./optimisation/mine.tex}
 \input{./modelisation/main.tex}

% yves.aubry@univ-tln.fr : algebra

\end{document}

%% !TEX encoding = UTF-8 Unicode
% !TEX TS-program = xelatex

\documentclass[french]{report}

%\usepackage[utf8]{inputenc}
%\usepackage[T1]{fontenc}
\usepackage{babel}


\newif\ifcomment
%\commenttrue # Show comments

\usepackage{physics}
\usepackage{amssymb}


\usepackage{amsthm}
% \usepackage{thmtools}
\usepackage{mathtools}
\usepackage{amsfonts}

\usepackage{color}

\usepackage{tikz}

\usepackage{geometry}
\geometry{a5paper, margin=0.1in, right=1cm}

\usepackage{dsfont}

\usepackage{graphicx}
\graphicspath{ {images/} }

\usepackage{faktor}

\usepackage{IEEEtrantools}
\usepackage{enumerate}   
\usepackage[PostScript=dvips]{"/Users/aware/Documents/Courses/diagrams"}


\newtheorem{theorem}{Théorème}[section]
\renewcommand{\thetheorem}{\arabic{theorem}}
\newtheorem{lemme}{Lemme}[section]
\renewcommand{\thelemme}{\arabic{lemme}}
\newtheorem{proposition}{Proposition}[section]
\renewcommand{\theproposition}{\arabic{proposition}}
\newtheorem{notations}{Notations}[section]
\newtheorem{problem}{Problème}[section]
\newtheorem{corollary}{Corollaire}[theorem]
\renewcommand{\thecorollary}{\arabic{corollary}}
\newtheorem{property}{Propriété}[section]
\newtheorem{objective}{Objectif}[section]

\theoremstyle{definition}
\newtheorem{definition}{Définition}[section]
\renewcommand{\thedefinition}{\arabic{definition}}
\newtheorem{exercise}{Exercice}[chapter]
\renewcommand{\theexercise}{\arabic{exercise}}
\newtheorem{example}{Exemple}[chapter]
\renewcommand{\theexample}{\arabic{example}}
\newtheorem*{solution}{Solution}
\newtheorem*{application}{Application}
\newtheorem*{notation}{Notation}
\newtheorem*{vocabulary}{Vocabulaire}
\newtheorem*{properties}{Propriétés}



\theoremstyle{remark}
\newtheorem*{remark}{Remarque}
\newtheorem*{rappel}{Rappel}


\usepackage{etoolbox}
\AtBeginEnvironment{exercise}{\small}
\AtBeginEnvironment{example}{\small}

\usepackage{cases}
\usepackage[red]{mypack}

\usepackage[framemethod=TikZ]{mdframed}

\definecolor{bg}{rgb}{0.4,0.25,0.95}
\definecolor{pagebg}{rgb}{0,0,0.5}
\surroundwithmdframed[
   topline=false,
   rightline=false,
   bottomline=false,
   leftmargin=\parindent,
   skipabove=8pt,
   skipbelow=8pt,
   linecolor=blue,
   innerbottommargin=10pt,
   % backgroundcolor=bg,font=\color{orange}\sffamily, fontcolor=white
]{definition}

\usepackage{empheq}
\usepackage[most]{tcolorbox}

\newtcbox{\mymath}[1][]{%
    nobeforeafter, math upper, tcbox raise base,
    enhanced, colframe=blue!30!black,
    colback=red!10, boxrule=1pt,
    #1}

\usepackage{unixode}


\DeclareMathOperator{\ord}{ord}
\DeclareMathOperator{\orb}{orb}
\DeclareMathOperator{\stab}{stab}
\DeclareMathOperator{\Stab}{stab}
\DeclareMathOperator{\ppcm}{ppcm}
\DeclareMathOperator{\conj}{Conj}
\DeclareMathOperator{\End}{End}
\DeclareMathOperator{\rot}{rot}
\DeclareMathOperator{\trs}{trace}
\DeclareMathOperator{\Ind}{Ind}
\DeclareMathOperator{\mat}{Mat}
\DeclareMathOperator{\id}{Id}
\DeclareMathOperator{\vect}{vect}
\DeclareMathOperator{\img}{img}
\DeclareMathOperator{\cov}{Cov}
\DeclareMathOperator{\dist}{dist}
\DeclareMathOperator{\irr}{Irr}
\DeclareMathOperator{\image}{Im}
\DeclareMathOperator{\pd}{\partial}
\DeclareMathOperator{\epi}{epi}
\DeclareMathOperator{\Argmin}{Argmin}
\DeclareMathOperator{\dom}{dom}
\DeclareMathOperator{\proj}{proj}
\DeclareMathOperator{\ctg}{ctg}
\DeclareMathOperator{\supp}{supp}
\DeclareMathOperator{\argmin}{argmin}
\DeclareMathOperator{\mult}{mult}
\DeclareMathOperator{\ch}{ch}
\DeclareMathOperator{\sh}{sh}
\DeclareMathOperator{\rang}{rang}
\DeclareMathOperator{\diam}{diam}
\DeclareMathOperator{\Epigraphe}{Epigraphe}




\usepackage{xcolor}
\everymath{\color{blue}}
%\everymath{\color[rgb]{0,1,1}}
%\pagecolor[rgb]{0,0,0.5}


\newcommand*{\pdtest}[3][]{\ensuremath{\frac{\partial^{#1} #2}{\partial #3}}}

\newcommand*{\deffunc}[6][]{\ensuremath{
\begin{array}{rcl}
#2 : #3 &\rightarrow& #4\\
#5 &\mapsto& #6
\end{array}
}}

\newcommand{\eqcolon}{\mathrel{\resizebox{\widthof{$\mathord{=}$}}{\height}{ $\!\!=\!\!\resizebox{1.2\width}{0.8\height}{\raisebox{0.23ex}{$\mathop{:}$}}\!\!$ }}}
\newcommand{\coloneq}{\mathrel{\resizebox{\widthof{$\mathord{=}$}}{\height}{ $\!\!\resizebox{1.2\width}{0.8\height}{\raisebox{0.23ex}{$\mathop{:}$}}\!\!=\!\!$ }}}
\newcommand{\eqcolonl}{\ensuremath{\mathrel{=\!\!\mathop{:}}}}
\newcommand{\coloneql}{\ensuremath{\mathrel{\mathop{:} \!\! =}}}
\newcommand{\vc}[1]{% inline column vector
  \left(\begin{smallmatrix}#1\end{smallmatrix}\right)%
}
\newcommand{\vr}[1]{% inline row vector
  \begin{smallmatrix}(\,#1\,)\end{smallmatrix}%
}
\makeatletter
\newcommand*{\defeq}{\ =\mathrel{\rlap{%
                     \raisebox{0.3ex}{$\m@th\cdot$}}%
                     \raisebox{-0.3ex}{$\m@th\cdot$}}%
                     }
\makeatother

\newcommand{\mathcircle}[1]{% inline row vector
 \overset{\circ}{#1}
}
\newcommand{\ulim}{% low limit
 \underline{\lim}
}
\newcommand{\ssi}{% iff
\iff
}
\newcommand{\ps}[2]{
\expval{#1 | #2}
}
\newcommand{\df}[1]{
\mqty{#1}
}
\newcommand{\n}[1]{
\norm{#1}
}
\newcommand{\sys}[1]{
\left\{\smqty{#1}\right.
}


\newcommand{\eqdef}{\ensuremath{\overset{\text{def}}=}}


\def\Circlearrowright{\ensuremath{%
  \rotatebox[origin=c]{230}{$\circlearrowright$}}}

\newcommand\ct[1]{\text{\rmfamily\upshape #1}}
\newcommand\question[1]{ {\color{red} ...!? \small #1}}
\newcommand\caz[1]{\left\{\begin{array} #1 \end{array}\right.}
\newcommand\const{\text{\rmfamily\upshape const}}
\newcommand\toP{ \overset{\pro}{\to}}
\newcommand\toPP{ \overset{\text{PP}}{\to}}
\newcommand{\oeq}{\mathrel{\text{\textcircled{$=$}}}}





\usepackage{xcolor}
% \usepackage[normalem]{ulem}
\usepackage{lipsum}
\makeatletter
% \newcommand\colorwave[1][blue]{\bgroup \markoverwith{\lower3.5\p@\hbox{\sixly \textcolor{#1}{\char58}}}\ULon}
%\font\sixly=lasy6 % does not re-load if already loaded, so no memory problem.

\newmdtheoremenv[
linewidth= 1pt,linecolor= blue,%
leftmargin=20,rightmargin=20,innertopmargin=0pt, innerrightmargin=40,%
tikzsetting = { draw=lightgray, line width = 0.3pt,dashed,%
dash pattern = on 15pt off 3pt},%
splittopskip=\topskip,skipbelow=\baselineskip,%
skipabove=\baselineskip,ntheorem,roundcorner=0pt,
% backgroundcolor=pagebg,font=\color{orange}\sffamily, fontcolor=white
]{examplebox}{Exemple}[section]



\newcommand\R{\mathbb{R}}
\newcommand\Z{\mathbb{Z}}
\newcommand\N{\mathbb{N}}
\newcommand\E{\mathbb{E}}
\newcommand\F{\mathcal{F}}
\newcommand\cH{\mathcal{H}}
\newcommand\V{\mathbb{V}}
\newcommand\dmo{ ^{-1} }
\newcommand\kapa{\kappa}
\newcommand\im{Im}
\newcommand\hs{\mathcal{H}}





\usepackage{soul}

\makeatletter
\newcommand*{\whiten}[1]{\llap{\textcolor{white}{{\the\SOUL@token}}\hspace{#1pt}}}
\DeclareRobustCommand*\myul{%
    \def\SOUL@everyspace{\underline{\space}\kern\z@}%
    \def\SOUL@everytoken{%
     \setbox0=\hbox{\the\SOUL@token}%
     \ifdim\dp0>\z@
        \raisebox{\dp0}{\underline{\phantom{\the\SOUL@token}}}%
        \whiten{1}\whiten{0}%
        \whiten{-1}\whiten{-2}%
        \llap{\the\SOUL@token}%
     \else
        \underline{\the\SOUL@token}%
     \fi}%
\SOUL@}
\makeatother

\newcommand*{\demp}{\fontfamily{lmtt}\selectfont}

\DeclareTextFontCommand{\textdemp}{\demp}

\begin{document}

\ifcomment
Multiline
comment
\fi
\ifcomment
\myul{Typesetting test}
% \color[rgb]{1,1,1}
$∑_i^n≠ 60º±∞π∆¬≈√j∫h≤≥µ$

$\CR \R\pro\ind\pro\gS\pro
\mqty[a&b\\c&d]$
$\pro\mathbb{P}$
$\dd{x}$

  \[
    \alpha(x)=\left\{
                \begin{array}{ll}
                  x\\
                  \frac{1}{1+e^{-kx}}\\
                  \frac{e^x-e^{-x}}{e^x+e^{-x}}
                \end{array}
              \right.
  \]

  $\expval{x}$
  
  $\chi_\rho(ghg\dmo)=\Tr(\rho_{ghg\dmo})=\Tr(\rho_g\circ\rho_h\circ\rho\dmo_g)=\Tr(\rho_h)\overset{\mbox{\scalebox{0.5}{$\Tr(AB)=\Tr(BA)$}}}{=}\chi_\rho(h)$
  	$\mathop{\oplus}_{\substack{x\in X}}$

$\mat(\rho_g)=(a_{ij}(g))_{\scriptsize \substack{1\leq i\leq d \\ 1\leq j\leq d}}$ et $\mat(\rho'_g)=(a'_{ij}(g))_{\scriptsize \substack{1\leq i'\leq d' \\ 1\leq j'\leq d'}}$



\[\int_a^b{\mathbb{R}^2}g(u, v)\dd{P_{XY}}(u, v)=\iint g(u,v) f_{XY}(u, v)\dd \lambda(u) \dd \lambda(v)\]
$$\lim_{x\to\infty} f(x)$$	
$$\iiiint_V \mu(t,u,v,w) \,dt\,du\,dv\,dw$$
$$\sum_{n=1}^{\infty} 2^{-n} = 1$$	
\begin{definition}
	Si $X$ et $Y$ sont 2 v.a. ou definit la \textsc{Covariance} entre $X$ et $Y$ comme
	$\cov(X,Y)\overset{\text{def}}{=}\E\left[(X-\E(X))(Y-\E(Y))\right]=\E(XY)-\E(X)\E(Y)$.
\end{definition}
\fi
\pagebreak

% \tableofcontents

% insert your code here
%\input{./algebra/main.tex}
%\input{./geometrie-differentielle/main.tex}
%\input{./probabilite/main.tex}
%\input{./analyse-fonctionnelle/main.tex}
% \input{./Analyse-convexe-et-dualite-en-optimisation/main.tex}
%\input{./tikz/main.tex}
%\input{./Theorie-du-distributions/main.tex}
%\input{./optimisation/mine.tex}
 \input{./modelisation/main.tex}

% yves.aubry@univ-tln.fr : algebra

\end{document}

%% !TEX encoding = UTF-8 Unicode
% !TEX TS-program = xelatex

\documentclass[french]{report}

%\usepackage[utf8]{inputenc}
%\usepackage[T1]{fontenc}
\usepackage{babel}


\newif\ifcomment
%\commenttrue # Show comments

\usepackage{physics}
\usepackage{amssymb}


\usepackage{amsthm}
% \usepackage{thmtools}
\usepackage{mathtools}
\usepackage{amsfonts}

\usepackage{color}

\usepackage{tikz}

\usepackage{geometry}
\geometry{a5paper, margin=0.1in, right=1cm}

\usepackage{dsfont}

\usepackage{graphicx}
\graphicspath{ {images/} }

\usepackage{faktor}

\usepackage{IEEEtrantools}
\usepackage{enumerate}   
\usepackage[PostScript=dvips]{"/Users/aware/Documents/Courses/diagrams"}


\newtheorem{theorem}{Théorème}[section]
\renewcommand{\thetheorem}{\arabic{theorem}}
\newtheorem{lemme}{Lemme}[section]
\renewcommand{\thelemme}{\arabic{lemme}}
\newtheorem{proposition}{Proposition}[section]
\renewcommand{\theproposition}{\arabic{proposition}}
\newtheorem{notations}{Notations}[section]
\newtheorem{problem}{Problème}[section]
\newtheorem{corollary}{Corollaire}[theorem]
\renewcommand{\thecorollary}{\arabic{corollary}}
\newtheorem{property}{Propriété}[section]
\newtheorem{objective}{Objectif}[section]

\theoremstyle{definition}
\newtheorem{definition}{Définition}[section]
\renewcommand{\thedefinition}{\arabic{definition}}
\newtheorem{exercise}{Exercice}[chapter]
\renewcommand{\theexercise}{\arabic{exercise}}
\newtheorem{example}{Exemple}[chapter]
\renewcommand{\theexample}{\arabic{example}}
\newtheorem*{solution}{Solution}
\newtheorem*{application}{Application}
\newtheorem*{notation}{Notation}
\newtheorem*{vocabulary}{Vocabulaire}
\newtheorem*{properties}{Propriétés}



\theoremstyle{remark}
\newtheorem*{remark}{Remarque}
\newtheorem*{rappel}{Rappel}


\usepackage{etoolbox}
\AtBeginEnvironment{exercise}{\small}
\AtBeginEnvironment{example}{\small}

\usepackage{cases}
\usepackage[red]{mypack}

\usepackage[framemethod=TikZ]{mdframed}

\definecolor{bg}{rgb}{0.4,0.25,0.95}
\definecolor{pagebg}{rgb}{0,0,0.5}
\surroundwithmdframed[
   topline=false,
   rightline=false,
   bottomline=false,
   leftmargin=\parindent,
   skipabove=8pt,
   skipbelow=8pt,
   linecolor=blue,
   innerbottommargin=10pt,
   % backgroundcolor=bg,font=\color{orange}\sffamily, fontcolor=white
]{definition}

\usepackage{empheq}
\usepackage[most]{tcolorbox}

\newtcbox{\mymath}[1][]{%
    nobeforeafter, math upper, tcbox raise base,
    enhanced, colframe=blue!30!black,
    colback=red!10, boxrule=1pt,
    #1}

\usepackage{unixode}


\DeclareMathOperator{\ord}{ord}
\DeclareMathOperator{\orb}{orb}
\DeclareMathOperator{\stab}{stab}
\DeclareMathOperator{\Stab}{stab}
\DeclareMathOperator{\ppcm}{ppcm}
\DeclareMathOperator{\conj}{Conj}
\DeclareMathOperator{\End}{End}
\DeclareMathOperator{\rot}{rot}
\DeclareMathOperator{\trs}{trace}
\DeclareMathOperator{\Ind}{Ind}
\DeclareMathOperator{\mat}{Mat}
\DeclareMathOperator{\id}{Id}
\DeclareMathOperator{\vect}{vect}
\DeclareMathOperator{\img}{img}
\DeclareMathOperator{\cov}{Cov}
\DeclareMathOperator{\dist}{dist}
\DeclareMathOperator{\irr}{Irr}
\DeclareMathOperator{\image}{Im}
\DeclareMathOperator{\pd}{\partial}
\DeclareMathOperator{\epi}{epi}
\DeclareMathOperator{\Argmin}{Argmin}
\DeclareMathOperator{\dom}{dom}
\DeclareMathOperator{\proj}{proj}
\DeclareMathOperator{\ctg}{ctg}
\DeclareMathOperator{\supp}{supp}
\DeclareMathOperator{\argmin}{argmin}
\DeclareMathOperator{\mult}{mult}
\DeclareMathOperator{\ch}{ch}
\DeclareMathOperator{\sh}{sh}
\DeclareMathOperator{\rang}{rang}
\DeclareMathOperator{\diam}{diam}
\DeclareMathOperator{\Epigraphe}{Epigraphe}




\usepackage{xcolor}
\everymath{\color{blue}}
%\everymath{\color[rgb]{0,1,1}}
%\pagecolor[rgb]{0,0,0.5}


\newcommand*{\pdtest}[3][]{\ensuremath{\frac{\partial^{#1} #2}{\partial #3}}}

\newcommand*{\deffunc}[6][]{\ensuremath{
\begin{array}{rcl}
#2 : #3 &\rightarrow& #4\\
#5 &\mapsto& #6
\end{array}
}}

\newcommand{\eqcolon}{\mathrel{\resizebox{\widthof{$\mathord{=}$}}{\height}{ $\!\!=\!\!\resizebox{1.2\width}{0.8\height}{\raisebox{0.23ex}{$\mathop{:}$}}\!\!$ }}}
\newcommand{\coloneq}{\mathrel{\resizebox{\widthof{$\mathord{=}$}}{\height}{ $\!\!\resizebox{1.2\width}{0.8\height}{\raisebox{0.23ex}{$\mathop{:}$}}\!\!=\!\!$ }}}
\newcommand{\eqcolonl}{\ensuremath{\mathrel{=\!\!\mathop{:}}}}
\newcommand{\coloneql}{\ensuremath{\mathrel{\mathop{:} \!\! =}}}
\newcommand{\vc}[1]{% inline column vector
  \left(\begin{smallmatrix}#1\end{smallmatrix}\right)%
}
\newcommand{\vr}[1]{% inline row vector
  \begin{smallmatrix}(\,#1\,)\end{smallmatrix}%
}
\makeatletter
\newcommand*{\defeq}{\ =\mathrel{\rlap{%
                     \raisebox{0.3ex}{$\m@th\cdot$}}%
                     \raisebox{-0.3ex}{$\m@th\cdot$}}%
                     }
\makeatother

\newcommand{\mathcircle}[1]{% inline row vector
 \overset{\circ}{#1}
}
\newcommand{\ulim}{% low limit
 \underline{\lim}
}
\newcommand{\ssi}{% iff
\iff
}
\newcommand{\ps}[2]{
\expval{#1 | #2}
}
\newcommand{\df}[1]{
\mqty{#1}
}
\newcommand{\n}[1]{
\norm{#1}
}
\newcommand{\sys}[1]{
\left\{\smqty{#1}\right.
}


\newcommand{\eqdef}{\ensuremath{\overset{\text{def}}=}}


\def\Circlearrowright{\ensuremath{%
  \rotatebox[origin=c]{230}{$\circlearrowright$}}}

\newcommand\ct[1]{\text{\rmfamily\upshape #1}}
\newcommand\question[1]{ {\color{red} ...!? \small #1}}
\newcommand\caz[1]{\left\{\begin{array} #1 \end{array}\right.}
\newcommand\const{\text{\rmfamily\upshape const}}
\newcommand\toP{ \overset{\pro}{\to}}
\newcommand\toPP{ \overset{\text{PP}}{\to}}
\newcommand{\oeq}{\mathrel{\text{\textcircled{$=$}}}}





\usepackage{xcolor}
% \usepackage[normalem]{ulem}
\usepackage{lipsum}
\makeatletter
% \newcommand\colorwave[1][blue]{\bgroup \markoverwith{\lower3.5\p@\hbox{\sixly \textcolor{#1}{\char58}}}\ULon}
%\font\sixly=lasy6 % does not re-load if already loaded, so no memory problem.

\newmdtheoremenv[
linewidth= 1pt,linecolor= blue,%
leftmargin=20,rightmargin=20,innertopmargin=0pt, innerrightmargin=40,%
tikzsetting = { draw=lightgray, line width = 0.3pt,dashed,%
dash pattern = on 15pt off 3pt},%
splittopskip=\topskip,skipbelow=\baselineskip,%
skipabove=\baselineskip,ntheorem,roundcorner=0pt,
% backgroundcolor=pagebg,font=\color{orange}\sffamily, fontcolor=white
]{examplebox}{Exemple}[section]



\newcommand\R{\mathbb{R}}
\newcommand\Z{\mathbb{Z}}
\newcommand\N{\mathbb{N}}
\newcommand\E{\mathbb{E}}
\newcommand\F{\mathcal{F}}
\newcommand\cH{\mathcal{H}}
\newcommand\V{\mathbb{V}}
\newcommand\dmo{ ^{-1} }
\newcommand\kapa{\kappa}
\newcommand\im{Im}
\newcommand\hs{\mathcal{H}}





\usepackage{soul}

\makeatletter
\newcommand*{\whiten}[1]{\llap{\textcolor{white}{{\the\SOUL@token}}\hspace{#1pt}}}
\DeclareRobustCommand*\myul{%
    \def\SOUL@everyspace{\underline{\space}\kern\z@}%
    \def\SOUL@everytoken{%
     \setbox0=\hbox{\the\SOUL@token}%
     \ifdim\dp0>\z@
        \raisebox{\dp0}{\underline{\phantom{\the\SOUL@token}}}%
        \whiten{1}\whiten{0}%
        \whiten{-1}\whiten{-2}%
        \llap{\the\SOUL@token}%
     \else
        \underline{\the\SOUL@token}%
     \fi}%
\SOUL@}
\makeatother

\newcommand*{\demp}{\fontfamily{lmtt}\selectfont}

\DeclareTextFontCommand{\textdemp}{\demp}

\begin{document}

\ifcomment
Multiline
comment
\fi
\ifcomment
\myul{Typesetting test}
% \color[rgb]{1,1,1}
$∑_i^n≠ 60º±∞π∆¬≈√j∫h≤≥µ$

$\CR \R\pro\ind\pro\gS\pro
\mqty[a&b\\c&d]$
$\pro\mathbb{P}$
$\dd{x}$

  \[
    \alpha(x)=\left\{
                \begin{array}{ll}
                  x\\
                  \frac{1}{1+e^{-kx}}\\
                  \frac{e^x-e^{-x}}{e^x+e^{-x}}
                \end{array}
              \right.
  \]

  $\expval{x}$
  
  $\chi_\rho(ghg\dmo)=\Tr(\rho_{ghg\dmo})=\Tr(\rho_g\circ\rho_h\circ\rho\dmo_g)=\Tr(\rho_h)\overset{\mbox{\scalebox{0.5}{$\Tr(AB)=\Tr(BA)$}}}{=}\chi_\rho(h)$
  	$\mathop{\oplus}_{\substack{x\in X}}$

$\mat(\rho_g)=(a_{ij}(g))_{\scriptsize \substack{1\leq i\leq d \\ 1\leq j\leq d}}$ et $\mat(\rho'_g)=(a'_{ij}(g))_{\scriptsize \substack{1\leq i'\leq d' \\ 1\leq j'\leq d'}}$



\[\int_a^b{\mathbb{R}^2}g(u, v)\dd{P_{XY}}(u, v)=\iint g(u,v) f_{XY}(u, v)\dd \lambda(u) \dd \lambda(v)\]
$$\lim_{x\to\infty} f(x)$$	
$$\iiiint_V \mu(t,u,v,w) \,dt\,du\,dv\,dw$$
$$\sum_{n=1}^{\infty} 2^{-n} = 1$$	
\begin{definition}
	Si $X$ et $Y$ sont 2 v.a. ou definit la \textsc{Covariance} entre $X$ et $Y$ comme
	$\cov(X,Y)\overset{\text{def}}{=}\E\left[(X-\E(X))(Y-\E(Y))\right]=\E(XY)-\E(X)\E(Y)$.
\end{definition}
\fi
\pagebreak

% \tableofcontents

% insert your code here
%\input{./algebra/main.tex}
%\input{./geometrie-differentielle/main.tex}
%\input{./probabilite/main.tex}
%\input{./analyse-fonctionnelle/main.tex}
% \input{./Analyse-convexe-et-dualite-en-optimisation/main.tex}
%\input{./tikz/main.tex}
%\input{./Theorie-du-distributions/main.tex}
%\input{./optimisation/mine.tex}
 \input{./modelisation/main.tex}

% yves.aubry@univ-tln.fr : algebra

\end{document}

%% !TEX encoding = UTF-8 Unicode
% !TEX TS-program = xelatex

\documentclass[french]{report}

%\usepackage[utf8]{inputenc}
%\usepackage[T1]{fontenc}
\usepackage{babel}


\newif\ifcomment
%\commenttrue # Show comments

\usepackage{physics}
\usepackage{amssymb}


\usepackage{amsthm}
% \usepackage{thmtools}
\usepackage{mathtools}
\usepackage{amsfonts}

\usepackage{color}

\usepackage{tikz}

\usepackage{geometry}
\geometry{a5paper, margin=0.1in, right=1cm}

\usepackage{dsfont}

\usepackage{graphicx}
\graphicspath{ {images/} }

\usepackage{faktor}

\usepackage{IEEEtrantools}
\usepackage{enumerate}   
\usepackage[PostScript=dvips]{"/Users/aware/Documents/Courses/diagrams"}


\newtheorem{theorem}{Théorème}[section]
\renewcommand{\thetheorem}{\arabic{theorem}}
\newtheorem{lemme}{Lemme}[section]
\renewcommand{\thelemme}{\arabic{lemme}}
\newtheorem{proposition}{Proposition}[section]
\renewcommand{\theproposition}{\arabic{proposition}}
\newtheorem{notations}{Notations}[section]
\newtheorem{problem}{Problème}[section]
\newtheorem{corollary}{Corollaire}[theorem]
\renewcommand{\thecorollary}{\arabic{corollary}}
\newtheorem{property}{Propriété}[section]
\newtheorem{objective}{Objectif}[section]

\theoremstyle{definition}
\newtheorem{definition}{Définition}[section]
\renewcommand{\thedefinition}{\arabic{definition}}
\newtheorem{exercise}{Exercice}[chapter]
\renewcommand{\theexercise}{\arabic{exercise}}
\newtheorem{example}{Exemple}[chapter]
\renewcommand{\theexample}{\arabic{example}}
\newtheorem*{solution}{Solution}
\newtheorem*{application}{Application}
\newtheorem*{notation}{Notation}
\newtheorem*{vocabulary}{Vocabulaire}
\newtheorem*{properties}{Propriétés}



\theoremstyle{remark}
\newtheorem*{remark}{Remarque}
\newtheorem*{rappel}{Rappel}


\usepackage{etoolbox}
\AtBeginEnvironment{exercise}{\small}
\AtBeginEnvironment{example}{\small}

\usepackage{cases}
\usepackage[red]{mypack}

\usepackage[framemethod=TikZ]{mdframed}

\definecolor{bg}{rgb}{0.4,0.25,0.95}
\definecolor{pagebg}{rgb}{0,0,0.5}
\surroundwithmdframed[
   topline=false,
   rightline=false,
   bottomline=false,
   leftmargin=\parindent,
   skipabove=8pt,
   skipbelow=8pt,
   linecolor=blue,
   innerbottommargin=10pt,
   % backgroundcolor=bg,font=\color{orange}\sffamily, fontcolor=white
]{definition}

\usepackage{empheq}
\usepackage[most]{tcolorbox}

\newtcbox{\mymath}[1][]{%
    nobeforeafter, math upper, tcbox raise base,
    enhanced, colframe=blue!30!black,
    colback=red!10, boxrule=1pt,
    #1}

\usepackage{unixode}


\DeclareMathOperator{\ord}{ord}
\DeclareMathOperator{\orb}{orb}
\DeclareMathOperator{\stab}{stab}
\DeclareMathOperator{\Stab}{stab}
\DeclareMathOperator{\ppcm}{ppcm}
\DeclareMathOperator{\conj}{Conj}
\DeclareMathOperator{\End}{End}
\DeclareMathOperator{\rot}{rot}
\DeclareMathOperator{\trs}{trace}
\DeclareMathOperator{\Ind}{Ind}
\DeclareMathOperator{\mat}{Mat}
\DeclareMathOperator{\id}{Id}
\DeclareMathOperator{\vect}{vect}
\DeclareMathOperator{\img}{img}
\DeclareMathOperator{\cov}{Cov}
\DeclareMathOperator{\dist}{dist}
\DeclareMathOperator{\irr}{Irr}
\DeclareMathOperator{\image}{Im}
\DeclareMathOperator{\pd}{\partial}
\DeclareMathOperator{\epi}{epi}
\DeclareMathOperator{\Argmin}{Argmin}
\DeclareMathOperator{\dom}{dom}
\DeclareMathOperator{\proj}{proj}
\DeclareMathOperator{\ctg}{ctg}
\DeclareMathOperator{\supp}{supp}
\DeclareMathOperator{\argmin}{argmin}
\DeclareMathOperator{\mult}{mult}
\DeclareMathOperator{\ch}{ch}
\DeclareMathOperator{\sh}{sh}
\DeclareMathOperator{\rang}{rang}
\DeclareMathOperator{\diam}{diam}
\DeclareMathOperator{\Epigraphe}{Epigraphe}




\usepackage{xcolor}
\everymath{\color{blue}}
%\everymath{\color[rgb]{0,1,1}}
%\pagecolor[rgb]{0,0,0.5}


\newcommand*{\pdtest}[3][]{\ensuremath{\frac{\partial^{#1} #2}{\partial #3}}}

\newcommand*{\deffunc}[6][]{\ensuremath{
\begin{array}{rcl}
#2 : #3 &\rightarrow& #4\\
#5 &\mapsto& #6
\end{array}
}}

\newcommand{\eqcolon}{\mathrel{\resizebox{\widthof{$\mathord{=}$}}{\height}{ $\!\!=\!\!\resizebox{1.2\width}{0.8\height}{\raisebox{0.23ex}{$\mathop{:}$}}\!\!$ }}}
\newcommand{\coloneq}{\mathrel{\resizebox{\widthof{$\mathord{=}$}}{\height}{ $\!\!\resizebox{1.2\width}{0.8\height}{\raisebox{0.23ex}{$\mathop{:}$}}\!\!=\!\!$ }}}
\newcommand{\eqcolonl}{\ensuremath{\mathrel{=\!\!\mathop{:}}}}
\newcommand{\coloneql}{\ensuremath{\mathrel{\mathop{:} \!\! =}}}
\newcommand{\vc}[1]{% inline column vector
  \left(\begin{smallmatrix}#1\end{smallmatrix}\right)%
}
\newcommand{\vr}[1]{% inline row vector
  \begin{smallmatrix}(\,#1\,)\end{smallmatrix}%
}
\makeatletter
\newcommand*{\defeq}{\ =\mathrel{\rlap{%
                     \raisebox{0.3ex}{$\m@th\cdot$}}%
                     \raisebox{-0.3ex}{$\m@th\cdot$}}%
                     }
\makeatother

\newcommand{\mathcircle}[1]{% inline row vector
 \overset{\circ}{#1}
}
\newcommand{\ulim}{% low limit
 \underline{\lim}
}
\newcommand{\ssi}{% iff
\iff
}
\newcommand{\ps}[2]{
\expval{#1 | #2}
}
\newcommand{\df}[1]{
\mqty{#1}
}
\newcommand{\n}[1]{
\norm{#1}
}
\newcommand{\sys}[1]{
\left\{\smqty{#1}\right.
}


\newcommand{\eqdef}{\ensuremath{\overset{\text{def}}=}}


\def\Circlearrowright{\ensuremath{%
  \rotatebox[origin=c]{230}{$\circlearrowright$}}}

\newcommand\ct[1]{\text{\rmfamily\upshape #1}}
\newcommand\question[1]{ {\color{red} ...!? \small #1}}
\newcommand\caz[1]{\left\{\begin{array} #1 \end{array}\right.}
\newcommand\const{\text{\rmfamily\upshape const}}
\newcommand\toP{ \overset{\pro}{\to}}
\newcommand\toPP{ \overset{\text{PP}}{\to}}
\newcommand{\oeq}{\mathrel{\text{\textcircled{$=$}}}}





\usepackage{xcolor}
% \usepackage[normalem]{ulem}
\usepackage{lipsum}
\makeatletter
% \newcommand\colorwave[1][blue]{\bgroup \markoverwith{\lower3.5\p@\hbox{\sixly \textcolor{#1}{\char58}}}\ULon}
%\font\sixly=lasy6 % does not re-load if already loaded, so no memory problem.

\newmdtheoremenv[
linewidth= 1pt,linecolor= blue,%
leftmargin=20,rightmargin=20,innertopmargin=0pt, innerrightmargin=40,%
tikzsetting = { draw=lightgray, line width = 0.3pt,dashed,%
dash pattern = on 15pt off 3pt},%
splittopskip=\topskip,skipbelow=\baselineskip,%
skipabove=\baselineskip,ntheorem,roundcorner=0pt,
% backgroundcolor=pagebg,font=\color{orange}\sffamily, fontcolor=white
]{examplebox}{Exemple}[section]



\newcommand\R{\mathbb{R}}
\newcommand\Z{\mathbb{Z}}
\newcommand\N{\mathbb{N}}
\newcommand\E{\mathbb{E}}
\newcommand\F{\mathcal{F}}
\newcommand\cH{\mathcal{H}}
\newcommand\V{\mathbb{V}}
\newcommand\dmo{ ^{-1} }
\newcommand\kapa{\kappa}
\newcommand\im{Im}
\newcommand\hs{\mathcal{H}}





\usepackage{soul}

\makeatletter
\newcommand*{\whiten}[1]{\llap{\textcolor{white}{{\the\SOUL@token}}\hspace{#1pt}}}
\DeclareRobustCommand*\myul{%
    \def\SOUL@everyspace{\underline{\space}\kern\z@}%
    \def\SOUL@everytoken{%
     \setbox0=\hbox{\the\SOUL@token}%
     \ifdim\dp0>\z@
        \raisebox{\dp0}{\underline{\phantom{\the\SOUL@token}}}%
        \whiten{1}\whiten{0}%
        \whiten{-1}\whiten{-2}%
        \llap{\the\SOUL@token}%
     \else
        \underline{\the\SOUL@token}%
     \fi}%
\SOUL@}
\makeatother

\newcommand*{\demp}{\fontfamily{lmtt}\selectfont}

\DeclareTextFontCommand{\textdemp}{\demp}

\begin{document}

\ifcomment
Multiline
comment
\fi
\ifcomment
\myul{Typesetting test}
% \color[rgb]{1,1,1}
$∑_i^n≠ 60º±∞π∆¬≈√j∫h≤≥µ$

$\CR \R\pro\ind\pro\gS\pro
\mqty[a&b\\c&d]$
$\pro\mathbb{P}$
$\dd{x}$

  \[
    \alpha(x)=\left\{
                \begin{array}{ll}
                  x\\
                  \frac{1}{1+e^{-kx}}\\
                  \frac{e^x-e^{-x}}{e^x+e^{-x}}
                \end{array}
              \right.
  \]

  $\expval{x}$
  
  $\chi_\rho(ghg\dmo)=\Tr(\rho_{ghg\dmo})=\Tr(\rho_g\circ\rho_h\circ\rho\dmo_g)=\Tr(\rho_h)\overset{\mbox{\scalebox{0.5}{$\Tr(AB)=\Tr(BA)$}}}{=}\chi_\rho(h)$
  	$\mathop{\oplus}_{\substack{x\in X}}$

$\mat(\rho_g)=(a_{ij}(g))_{\scriptsize \substack{1\leq i\leq d \\ 1\leq j\leq d}}$ et $\mat(\rho'_g)=(a'_{ij}(g))_{\scriptsize \substack{1\leq i'\leq d' \\ 1\leq j'\leq d'}}$



\[\int_a^b{\mathbb{R}^2}g(u, v)\dd{P_{XY}}(u, v)=\iint g(u,v) f_{XY}(u, v)\dd \lambda(u) \dd \lambda(v)\]
$$\lim_{x\to\infty} f(x)$$	
$$\iiiint_V \mu(t,u,v,w) \,dt\,du\,dv\,dw$$
$$\sum_{n=1}^{\infty} 2^{-n} = 1$$	
\begin{definition}
	Si $X$ et $Y$ sont 2 v.a. ou definit la \textsc{Covariance} entre $X$ et $Y$ comme
	$\cov(X,Y)\overset{\text{def}}{=}\E\left[(X-\E(X))(Y-\E(Y))\right]=\E(XY)-\E(X)\E(Y)$.
\end{definition}
\fi
\pagebreak

% \tableofcontents

% insert your code here
%\input{./algebra/main.tex}
%\input{./geometrie-differentielle/main.tex}
%\input{./probabilite/main.tex}
%\input{./analyse-fonctionnelle/main.tex}
% \input{./Analyse-convexe-et-dualite-en-optimisation/main.tex}
%\input{./tikz/main.tex}
%\input{./Theorie-du-distributions/main.tex}
%\input{./optimisation/mine.tex}
 \input{./modelisation/main.tex}

% yves.aubry@univ-tln.fr : algebra

\end{document}

% % !TEX encoding = UTF-8 Unicode
% !TEX TS-program = xelatex

\documentclass[french]{report}

%\usepackage[utf8]{inputenc}
%\usepackage[T1]{fontenc}
\usepackage{babel}


\newif\ifcomment
%\commenttrue # Show comments

\usepackage{physics}
\usepackage{amssymb}


\usepackage{amsthm}
% \usepackage{thmtools}
\usepackage{mathtools}
\usepackage{amsfonts}

\usepackage{color}

\usepackage{tikz}

\usepackage{geometry}
\geometry{a5paper, margin=0.1in, right=1cm}

\usepackage{dsfont}

\usepackage{graphicx}
\graphicspath{ {images/} }

\usepackage{faktor}

\usepackage{IEEEtrantools}
\usepackage{enumerate}   
\usepackage[PostScript=dvips]{"/Users/aware/Documents/Courses/diagrams"}


\newtheorem{theorem}{Théorème}[section]
\renewcommand{\thetheorem}{\arabic{theorem}}
\newtheorem{lemme}{Lemme}[section]
\renewcommand{\thelemme}{\arabic{lemme}}
\newtheorem{proposition}{Proposition}[section]
\renewcommand{\theproposition}{\arabic{proposition}}
\newtheorem{notations}{Notations}[section]
\newtheorem{problem}{Problème}[section]
\newtheorem{corollary}{Corollaire}[theorem]
\renewcommand{\thecorollary}{\arabic{corollary}}
\newtheorem{property}{Propriété}[section]
\newtheorem{objective}{Objectif}[section]

\theoremstyle{definition}
\newtheorem{definition}{Définition}[section]
\renewcommand{\thedefinition}{\arabic{definition}}
\newtheorem{exercise}{Exercice}[chapter]
\renewcommand{\theexercise}{\arabic{exercise}}
\newtheorem{example}{Exemple}[chapter]
\renewcommand{\theexample}{\arabic{example}}
\newtheorem*{solution}{Solution}
\newtheorem*{application}{Application}
\newtheorem*{notation}{Notation}
\newtheorem*{vocabulary}{Vocabulaire}
\newtheorem*{properties}{Propriétés}



\theoremstyle{remark}
\newtheorem*{remark}{Remarque}
\newtheorem*{rappel}{Rappel}


\usepackage{etoolbox}
\AtBeginEnvironment{exercise}{\small}
\AtBeginEnvironment{example}{\small}

\usepackage{cases}
\usepackage[red]{mypack}

\usepackage[framemethod=TikZ]{mdframed}

\definecolor{bg}{rgb}{0.4,0.25,0.95}
\definecolor{pagebg}{rgb}{0,0,0.5}
\surroundwithmdframed[
   topline=false,
   rightline=false,
   bottomline=false,
   leftmargin=\parindent,
   skipabove=8pt,
   skipbelow=8pt,
   linecolor=blue,
   innerbottommargin=10pt,
   % backgroundcolor=bg,font=\color{orange}\sffamily, fontcolor=white
]{definition}

\usepackage{empheq}
\usepackage[most]{tcolorbox}

\newtcbox{\mymath}[1][]{%
    nobeforeafter, math upper, tcbox raise base,
    enhanced, colframe=blue!30!black,
    colback=red!10, boxrule=1pt,
    #1}

\usepackage{unixode}


\DeclareMathOperator{\ord}{ord}
\DeclareMathOperator{\orb}{orb}
\DeclareMathOperator{\stab}{stab}
\DeclareMathOperator{\Stab}{stab}
\DeclareMathOperator{\ppcm}{ppcm}
\DeclareMathOperator{\conj}{Conj}
\DeclareMathOperator{\End}{End}
\DeclareMathOperator{\rot}{rot}
\DeclareMathOperator{\trs}{trace}
\DeclareMathOperator{\Ind}{Ind}
\DeclareMathOperator{\mat}{Mat}
\DeclareMathOperator{\id}{Id}
\DeclareMathOperator{\vect}{vect}
\DeclareMathOperator{\img}{img}
\DeclareMathOperator{\cov}{Cov}
\DeclareMathOperator{\dist}{dist}
\DeclareMathOperator{\irr}{Irr}
\DeclareMathOperator{\image}{Im}
\DeclareMathOperator{\pd}{\partial}
\DeclareMathOperator{\epi}{epi}
\DeclareMathOperator{\Argmin}{Argmin}
\DeclareMathOperator{\dom}{dom}
\DeclareMathOperator{\proj}{proj}
\DeclareMathOperator{\ctg}{ctg}
\DeclareMathOperator{\supp}{supp}
\DeclareMathOperator{\argmin}{argmin}
\DeclareMathOperator{\mult}{mult}
\DeclareMathOperator{\ch}{ch}
\DeclareMathOperator{\sh}{sh}
\DeclareMathOperator{\rang}{rang}
\DeclareMathOperator{\diam}{diam}
\DeclareMathOperator{\Epigraphe}{Epigraphe}




\usepackage{xcolor}
\everymath{\color{blue}}
%\everymath{\color[rgb]{0,1,1}}
%\pagecolor[rgb]{0,0,0.5}


\newcommand*{\pdtest}[3][]{\ensuremath{\frac{\partial^{#1} #2}{\partial #3}}}

\newcommand*{\deffunc}[6][]{\ensuremath{
\begin{array}{rcl}
#2 : #3 &\rightarrow& #4\\
#5 &\mapsto& #6
\end{array}
}}

\newcommand{\eqcolon}{\mathrel{\resizebox{\widthof{$\mathord{=}$}}{\height}{ $\!\!=\!\!\resizebox{1.2\width}{0.8\height}{\raisebox{0.23ex}{$\mathop{:}$}}\!\!$ }}}
\newcommand{\coloneq}{\mathrel{\resizebox{\widthof{$\mathord{=}$}}{\height}{ $\!\!\resizebox{1.2\width}{0.8\height}{\raisebox{0.23ex}{$\mathop{:}$}}\!\!=\!\!$ }}}
\newcommand{\eqcolonl}{\ensuremath{\mathrel{=\!\!\mathop{:}}}}
\newcommand{\coloneql}{\ensuremath{\mathrel{\mathop{:} \!\! =}}}
\newcommand{\vc}[1]{% inline column vector
  \left(\begin{smallmatrix}#1\end{smallmatrix}\right)%
}
\newcommand{\vr}[1]{% inline row vector
  \begin{smallmatrix}(\,#1\,)\end{smallmatrix}%
}
\makeatletter
\newcommand*{\defeq}{\ =\mathrel{\rlap{%
                     \raisebox{0.3ex}{$\m@th\cdot$}}%
                     \raisebox{-0.3ex}{$\m@th\cdot$}}%
                     }
\makeatother

\newcommand{\mathcircle}[1]{% inline row vector
 \overset{\circ}{#1}
}
\newcommand{\ulim}{% low limit
 \underline{\lim}
}
\newcommand{\ssi}{% iff
\iff
}
\newcommand{\ps}[2]{
\expval{#1 | #2}
}
\newcommand{\df}[1]{
\mqty{#1}
}
\newcommand{\n}[1]{
\norm{#1}
}
\newcommand{\sys}[1]{
\left\{\smqty{#1}\right.
}


\newcommand{\eqdef}{\ensuremath{\overset{\text{def}}=}}


\def\Circlearrowright{\ensuremath{%
  \rotatebox[origin=c]{230}{$\circlearrowright$}}}

\newcommand\ct[1]{\text{\rmfamily\upshape #1}}
\newcommand\question[1]{ {\color{red} ...!? \small #1}}
\newcommand\caz[1]{\left\{\begin{array} #1 \end{array}\right.}
\newcommand\const{\text{\rmfamily\upshape const}}
\newcommand\toP{ \overset{\pro}{\to}}
\newcommand\toPP{ \overset{\text{PP}}{\to}}
\newcommand{\oeq}{\mathrel{\text{\textcircled{$=$}}}}





\usepackage{xcolor}
% \usepackage[normalem]{ulem}
\usepackage{lipsum}
\makeatletter
% \newcommand\colorwave[1][blue]{\bgroup \markoverwith{\lower3.5\p@\hbox{\sixly \textcolor{#1}{\char58}}}\ULon}
%\font\sixly=lasy6 % does not re-load if already loaded, so no memory problem.

\newmdtheoremenv[
linewidth= 1pt,linecolor= blue,%
leftmargin=20,rightmargin=20,innertopmargin=0pt, innerrightmargin=40,%
tikzsetting = { draw=lightgray, line width = 0.3pt,dashed,%
dash pattern = on 15pt off 3pt},%
splittopskip=\topskip,skipbelow=\baselineskip,%
skipabove=\baselineskip,ntheorem,roundcorner=0pt,
% backgroundcolor=pagebg,font=\color{orange}\sffamily, fontcolor=white
]{examplebox}{Exemple}[section]



\newcommand\R{\mathbb{R}}
\newcommand\Z{\mathbb{Z}}
\newcommand\N{\mathbb{N}}
\newcommand\E{\mathbb{E}}
\newcommand\F{\mathcal{F}}
\newcommand\cH{\mathcal{H}}
\newcommand\V{\mathbb{V}}
\newcommand\dmo{ ^{-1} }
\newcommand\kapa{\kappa}
\newcommand\im{Im}
\newcommand\hs{\mathcal{H}}





\usepackage{soul}

\makeatletter
\newcommand*{\whiten}[1]{\llap{\textcolor{white}{{\the\SOUL@token}}\hspace{#1pt}}}
\DeclareRobustCommand*\myul{%
    \def\SOUL@everyspace{\underline{\space}\kern\z@}%
    \def\SOUL@everytoken{%
     \setbox0=\hbox{\the\SOUL@token}%
     \ifdim\dp0>\z@
        \raisebox{\dp0}{\underline{\phantom{\the\SOUL@token}}}%
        \whiten{1}\whiten{0}%
        \whiten{-1}\whiten{-2}%
        \llap{\the\SOUL@token}%
     \else
        \underline{\the\SOUL@token}%
     \fi}%
\SOUL@}
\makeatother

\newcommand*{\demp}{\fontfamily{lmtt}\selectfont}

\DeclareTextFontCommand{\textdemp}{\demp}

\begin{document}

\ifcomment
Multiline
comment
\fi
\ifcomment
\myul{Typesetting test}
% \color[rgb]{1,1,1}
$∑_i^n≠ 60º±∞π∆¬≈√j∫h≤≥µ$

$\CR \R\pro\ind\pro\gS\pro
\mqty[a&b\\c&d]$
$\pro\mathbb{P}$
$\dd{x}$

  \[
    \alpha(x)=\left\{
                \begin{array}{ll}
                  x\\
                  \frac{1}{1+e^{-kx}}\\
                  \frac{e^x-e^{-x}}{e^x+e^{-x}}
                \end{array}
              \right.
  \]

  $\expval{x}$
  
  $\chi_\rho(ghg\dmo)=\Tr(\rho_{ghg\dmo})=\Tr(\rho_g\circ\rho_h\circ\rho\dmo_g)=\Tr(\rho_h)\overset{\mbox{\scalebox{0.5}{$\Tr(AB)=\Tr(BA)$}}}{=}\chi_\rho(h)$
  	$\mathop{\oplus}_{\substack{x\in X}}$

$\mat(\rho_g)=(a_{ij}(g))_{\scriptsize \substack{1\leq i\leq d \\ 1\leq j\leq d}}$ et $\mat(\rho'_g)=(a'_{ij}(g))_{\scriptsize \substack{1\leq i'\leq d' \\ 1\leq j'\leq d'}}$



\[\int_a^b{\mathbb{R}^2}g(u, v)\dd{P_{XY}}(u, v)=\iint g(u,v) f_{XY}(u, v)\dd \lambda(u) \dd \lambda(v)\]
$$\lim_{x\to\infty} f(x)$$	
$$\iiiint_V \mu(t,u,v,w) \,dt\,du\,dv\,dw$$
$$\sum_{n=1}^{\infty} 2^{-n} = 1$$	
\begin{definition}
	Si $X$ et $Y$ sont 2 v.a. ou definit la \textsc{Covariance} entre $X$ et $Y$ comme
	$\cov(X,Y)\overset{\text{def}}{=}\E\left[(X-\E(X))(Y-\E(Y))\right]=\E(XY)-\E(X)\E(Y)$.
\end{definition}
\fi
\pagebreak

% \tableofcontents

% insert your code here
%\input{./algebra/main.tex}
%\input{./geometrie-differentielle/main.tex}
%\input{./probabilite/main.tex}
%\input{./analyse-fonctionnelle/main.tex}
% \input{./Analyse-convexe-et-dualite-en-optimisation/main.tex}
%\input{./tikz/main.tex}
%\input{./Theorie-du-distributions/main.tex}
%\input{./optimisation/mine.tex}
 \input{./modelisation/main.tex}

% yves.aubry@univ-tln.fr : algebra

\end{document}

%% !TEX encoding = UTF-8 Unicode
% !TEX TS-program = xelatex

\documentclass[french]{report}

%\usepackage[utf8]{inputenc}
%\usepackage[T1]{fontenc}
\usepackage{babel}


\newif\ifcomment
%\commenttrue # Show comments

\usepackage{physics}
\usepackage{amssymb}


\usepackage{amsthm}
% \usepackage{thmtools}
\usepackage{mathtools}
\usepackage{amsfonts}

\usepackage{color}

\usepackage{tikz}

\usepackage{geometry}
\geometry{a5paper, margin=0.1in, right=1cm}

\usepackage{dsfont}

\usepackage{graphicx}
\graphicspath{ {images/} }

\usepackage{faktor}

\usepackage{IEEEtrantools}
\usepackage{enumerate}   
\usepackage[PostScript=dvips]{"/Users/aware/Documents/Courses/diagrams"}


\newtheorem{theorem}{Théorème}[section]
\renewcommand{\thetheorem}{\arabic{theorem}}
\newtheorem{lemme}{Lemme}[section]
\renewcommand{\thelemme}{\arabic{lemme}}
\newtheorem{proposition}{Proposition}[section]
\renewcommand{\theproposition}{\arabic{proposition}}
\newtheorem{notations}{Notations}[section]
\newtheorem{problem}{Problème}[section]
\newtheorem{corollary}{Corollaire}[theorem]
\renewcommand{\thecorollary}{\arabic{corollary}}
\newtheorem{property}{Propriété}[section]
\newtheorem{objective}{Objectif}[section]

\theoremstyle{definition}
\newtheorem{definition}{Définition}[section]
\renewcommand{\thedefinition}{\arabic{definition}}
\newtheorem{exercise}{Exercice}[chapter]
\renewcommand{\theexercise}{\arabic{exercise}}
\newtheorem{example}{Exemple}[chapter]
\renewcommand{\theexample}{\arabic{example}}
\newtheorem*{solution}{Solution}
\newtheorem*{application}{Application}
\newtheorem*{notation}{Notation}
\newtheorem*{vocabulary}{Vocabulaire}
\newtheorem*{properties}{Propriétés}



\theoremstyle{remark}
\newtheorem*{remark}{Remarque}
\newtheorem*{rappel}{Rappel}


\usepackage{etoolbox}
\AtBeginEnvironment{exercise}{\small}
\AtBeginEnvironment{example}{\small}

\usepackage{cases}
\usepackage[red]{mypack}

\usepackage[framemethod=TikZ]{mdframed}

\definecolor{bg}{rgb}{0.4,0.25,0.95}
\definecolor{pagebg}{rgb}{0,0,0.5}
\surroundwithmdframed[
   topline=false,
   rightline=false,
   bottomline=false,
   leftmargin=\parindent,
   skipabove=8pt,
   skipbelow=8pt,
   linecolor=blue,
   innerbottommargin=10pt,
   % backgroundcolor=bg,font=\color{orange}\sffamily, fontcolor=white
]{definition}

\usepackage{empheq}
\usepackage[most]{tcolorbox}

\newtcbox{\mymath}[1][]{%
    nobeforeafter, math upper, tcbox raise base,
    enhanced, colframe=blue!30!black,
    colback=red!10, boxrule=1pt,
    #1}

\usepackage{unixode}


\DeclareMathOperator{\ord}{ord}
\DeclareMathOperator{\orb}{orb}
\DeclareMathOperator{\stab}{stab}
\DeclareMathOperator{\Stab}{stab}
\DeclareMathOperator{\ppcm}{ppcm}
\DeclareMathOperator{\conj}{Conj}
\DeclareMathOperator{\End}{End}
\DeclareMathOperator{\rot}{rot}
\DeclareMathOperator{\trs}{trace}
\DeclareMathOperator{\Ind}{Ind}
\DeclareMathOperator{\mat}{Mat}
\DeclareMathOperator{\id}{Id}
\DeclareMathOperator{\vect}{vect}
\DeclareMathOperator{\img}{img}
\DeclareMathOperator{\cov}{Cov}
\DeclareMathOperator{\dist}{dist}
\DeclareMathOperator{\irr}{Irr}
\DeclareMathOperator{\image}{Im}
\DeclareMathOperator{\pd}{\partial}
\DeclareMathOperator{\epi}{epi}
\DeclareMathOperator{\Argmin}{Argmin}
\DeclareMathOperator{\dom}{dom}
\DeclareMathOperator{\proj}{proj}
\DeclareMathOperator{\ctg}{ctg}
\DeclareMathOperator{\supp}{supp}
\DeclareMathOperator{\argmin}{argmin}
\DeclareMathOperator{\mult}{mult}
\DeclareMathOperator{\ch}{ch}
\DeclareMathOperator{\sh}{sh}
\DeclareMathOperator{\rang}{rang}
\DeclareMathOperator{\diam}{diam}
\DeclareMathOperator{\Epigraphe}{Epigraphe}




\usepackage{xcolor}
\everymath{\color{blue}}
%\everymath{\color[rgb]{0,1,1}}
%\pagecolor[rgb]{0,0,0.5}


\newcommand*{\pdtest}[3][]{\ensuremath{\frac{\partial^{#1} #2}{\partial #3}}}

\newcommand*{\deffunc}[6][]{\ensuremath{
\begin{array}{rcl}
#2 : #3 &\rightarrow& #4\\
#5 &\mapsto& #6
\end{array}
}}

\newcommand{\eqcolon}{\mathrel{\resizebox{\widthof{$\mathord{=}$}}{\height}{ $\!\!=\!\!\resizebox{1.2\width}{0.8\height}{\raisebox{0.23ex}{$\mathop{:}$}}\!\!$ }}}
\newcommand{\coloneq}{\mathrel{\resizebox{\widthof{$\mathord{=}$}}{\height}{ $\!\!\resizebox{1.2\width}{0.8\height}{\raisebox{0.23ex}{$\mathop{:}$}}\!\!=\!\!$ }}}
\newcommand{\eqcolonl}{\ensuremath{\mathrel{=\!\!\mathop{:}}}}
\newcommand{\coloneql}{\ensuremath{\mathrel{\mathop{:} \!\! =}}}
\newcommand{\vc}[1]{% inline column vector
  \left(\begin{smallmatrix}#1\end{smallmatrix}\right)%
}
\newcommand{\vr}[1]{% inline row vector
  \begin{smallmatrix}(\,#1\,)\end{smallmatrix}%
}
\makeatletter
\newcommand*{\defeq}{\ =\mathrel{\rlap{%
                     \raisebox{0.3ex}{$\m@th\cdot$}}%
                     \raisebox{-0.3ex}{$\m@th\cdot$}}%
                     }
\makeatother

\newcommand{\mathcircle}[1]{% inline row vector
 \overset{\circ}{#1}
}
\newcommand{\ulim}{% low limit
 \underline{\lim}
}
\newcommand{\ssi}{% iff
\iff
}
\newcommand{\ps}[2]{
\expval{#1 | #2}
}
\newcommand{\df}[1]{
\mqty{#1}
}
\newcommand{\n}[1]{
\norm{#1}
}
\newcommand{\sys}[1]{
\left\{\smqty{#1}\right.
}


\newcommand{\eqdef}{\ensuremath{\overset{\text{def}}=}}


\def\Circlearrowright{\ensuremath{%
  \rotatebox[origin=c]{230}{$\circlearrowright$}}}

\newcommand\ct[1]{\text{\rmfamily\upshape #1}}
\newcommand\question[1]{ {\color{red} ...!? \small #1}}
\newcommand\caz[1]{\left\{\begin{array} #1 \end{array}\right.}
\newcommand\const{\text{\rmfamily\upshape const}}
\newcommand\toP{ \overset{\pro}{\to}}
\newcommand\toPP{ \overset{\text{PP}}{\to}}
\newcommand{\oeq}{\mathrel{\text{\textcircled{$=$}}}}





\usepackage{xcolor}
% \usepackage[normalem]{ulem}
\usepackage{lipsum}
\makeatletter
% \newcommand\colorwave[1][blue]{\bgroup \markoverwith{\lower3.5\p@\hbox{\sixly \textcolor{#1}{\char58}}}\ULon}
%\font\sixly=lasy6 % does not re-load if already loaded, so no memory problem.

\newmdtheoremenv[
linewidth= 1pt,linecolor= blue,%
leftmargin=20,rightmargin=20,innertopmargin=0pt, innerrightmargin=40,%
tikzsetting = { draw=lightgray, line width = 0.3pt,dashed,%
dash pattern = on 15pt off 3pt},%
splittopskip=\topskip,skipbelow=\baselineskip,%
skipabove=\baselineskip,ntheorem,roundcorner=0pt,
% backgroundcolor=pagebg,font=\color{orange}\sffamily, fontcolor=white
]{examplebox}{Exemple}[section]



\newcommand\R{\mathbb{R}}
\newcommand\Z{\mathbb{Z}}
\newcommand\N{\mathbb{N}}
\newcommand\E{\mathbb{E}}
\newcommand\F{\mathcal{F}}
\newcommand\cH{\mathcal{H}}
\newcommand\V{\mathbb{V}}
\newcommand\dmo{ ^{-1} }
\newcommand\kapa{\kappa}
\newcommand\im{Im}
\newcommand\hs{\mathcal{H}}





\usepackage{soul}

\makeatletter
\newcommand*{\whiten}[1]{\llap{\textcolor{white}{{\the\SOUL@token}}\hspace{#1pt}}}
\DeclareRobustCommand*\myul{%
    \def\SOUL@everyspace{\underline{\space}\kern\z@}%
    \def\SOUL@everytoken{%
     \setbox0=\hbox{\the\SOUL@token}%
     \ifdim\dp0>\z@
        \raisebox{\dp0}{\underline{\phantom{\the\SOUL@token}}}%
        \whiten{1}\whiten{0}%
        \whiten{-1}\whiten{-2}%
        \llap{\the\SOUL@token}%
     \else
        \underline{\the\SOUL@token}%
     \fi}%
\SOUL@}
\makeatother

\newcommand*{\demp}{\fontfamily{lmtt}\selectfont}

\DeclareTextFontCommand{\textdemp}{\demp}

\begin{document}

\ifcomment
Multiline
comment
\fi
\ifcomment
\myul{Typesetting test}
% \color[rgb]{1,1,1}
$∑_i^n≠ 60º±∞π∆¬≈√j∫h≤≥µ$

$\CR \R\pro\ind\pro\gS\pro
\mqty[a&b\\c&d]$
$\pro\mathbb{P}$
$\dd{x}$

  \[
    \alpha(x)=\left\{
                \begin{array}{ll}
                  x\\
                  \frac{1}{1+e^{-kx}}\\
                  \frac{e^x-e^{-x}}{e^x+e^{-x}}
                \end{array}
              \right.
  \]

  $\expval{x}$
  
  $\chi_\rho(ghg\dmo)=\Tr(\rho_{ghg\dmo})=\Tr(\rho_g\circ\rho_h\circ\rho\dmo_g)=\Tr(\rho_h)\overset{\mbox{\scalebox{0.5}{$\Tr(AB)=\Tr(BA)$}}}{=}\chi_\rho(h)$
  	$\mathop{\oplus}_{\substack{x\in X}}$

$\mat(\rho_g)=(a_{ij}(g))_{\scriptsize \substack{1\leq i\leq d \\ 1\leq j\leq d}}$ et $\mat(\rho'_g)=(a'_{ij}(g))_{\scriptsize \substack{1\leq i'\leq d' \\ 1\leq j'\leq d'}}$



\[\int_a^b{\mathbb{R}^2}g(u, v)\dd{P_{XY}}(u, v)=\iint g(u,v) f_{XY}(u, v)\dd \lambda(u) \dd \lambda(v)\]
$$\lim_{x\to\infty} f(x)$$	
$$\iiiint_V \mu(t,u,v,w) \,dt\,du\,dv\,dw$$
$$\sum_{n=1}^{\infty} 2^{-n} = 1$$	
\begin{definition}
	Si $X$ et $Y$ sont 2 v.a. ou definit la \textsc{Covariance} entre $X$ et $Y$ comme
	$\cov(X,Y)\overset{\text{def}}{=}\E\left[(X-\E(X))(Y-\E(Y))\right]=\E(XY)-\E(X)\E(Y)$.
\end{definition}
\fi
\pagebreak

% \tableofcontents

% insert your code here
%\input{./algebra/main.tex}
%\input{./geometrie-differentielle/main.tex}
%\input{./probabilite/main.tex}
%\input{./analyse-fonctionnelle/main.tex}
% \input{./Analyse-convexe-et-dualite-en-optimisation/main.tex}
%\input{./tikz/main.tex}
%\input{./Theorie-du-distributions/main.tex}
%\input{./optimisation/mine.tex}
 \input{./modelisation/main.tex}

% yves.aubry@univ-tln.fr : algebra

\end{document}

%% !TEX encoding = UTF-8 Unicode
% !TEX TS-program = xelatex

\documentclass[french]{report}

%\usepackage[utf8]{inputenc}
%\usepackage[T1]{fontenc}
\usepackage{babel}


\newif\ifcomment
%\commenttrue # Show comments

\usepackage{physics}
\usepackage{amssymb}


\usepackage{amsthm}
% \usepackage{thmtools}
\usepackage{mathtools}
\usepackage{amsfonts}

\usepackage{color}

\usepackage{tikz}

\usepackage{geometry}
\geometry{a5paper, margin=0.1in, right=1cm}

\usepackage{dsfont}

\usepackage{graphicx}
\graphicspath{ {images/} }

\usepackage{faktor}

\usepackage{IEEEtrantools}
\usepackage{enumerate}   
\usepackage[PostScript=dvips]{"/Users/aware/Documents/Courses/diagrams"}


\newtheorem{theorem}{Théorème}[section]
\renewcommand{\thetheorem}{\arabic{theorem}}
\newtheorem{lemme}{Lemme}[section]
\renewcommand{\thelemme}{\arabic{lemme}}
\newtheorem{proposition}{Proposition}[section]
\renewcommand{\theproposition}{\arabic{proposition}}
\newtheorem{notations}{Notations}[section]
\newtheorem{problem}{Problème}[section]
\newtheorem{corollary}{Corollaire}[theorem]
\renewcommand{\thecorollary}{\arabic{corollary}}
\newtheorem{property}{Propriété}[section]
\newtheorem{objective}{Objectif}[section]

\theoremstyle{definition}
\newtheorem{definition}{Définition}[section]
\renewcommand{\thedefinition}{\arabic{definition}}
\newtheorem{exercise}{Exercice}[chapter]
\renewcommand{\theexercise}{\arabic{exercise}}
\newtheorem{example}{Exemple}[chapter]
\renewcommand{\theexample}{\arabic{example}}
\newtheorem*{solution}{Solution}
\newtheorem*{application}{Application}
\newtheorem*{notation}{Notation}
\newtheorem*{vocabulary}{Vocabulaire}
\newtheorem*{properties}{Propriétés}



\theoremstyle{remark}
\newtheorem*{remark}{Remarque}
\newtheorem*{rappel}{Rappel}


\usepackage{etoolbox}
\AtBeginEnvironment{exercise}{\small}
\AtBeginEnvironment{example}{\small}

\usepackage{cases}
\usepackage[red]{mypack}

\usepackage[framemethod=TikZ]{mdframed}

\definecolor{bg}{rgb}{0.4,0.25,0.95}
\definecolor{pagebg}{rgb}{0,0,0.5}
\surroundwithmdframed[
   topline=false,
   rightline=false,
   bottomline=false,
   leftmargin=\parindent,
   skipabove=8pt,
   skipbelow=8pt,
   linecolor=blue,
   innerbottommargin=10pt,
   % backgroundcolor=bg,font=\color{orange}\sffamily, fontcolor=white
]{definition}

\usepackage{empheq}
\usepackage[most]{tcolorbox}

\newtcbox{\mymath}[1][]{%
    nobeforeafter, math upper, tcbox raise base,
    enhanced, colframe=blue!30!black,
    colback=red!10, boxrule=1pt,
    #1}

\usepackage{unixode}


\DeclareMathOperator{\ord}{ord}
\DeclareMathOperator{\orb}{orb}
\DeclareMathOperator{\stab}{stab}
\DeclareMathOperator{\Stab}{stab}
\DeclareMathOperator{\ppcm}{ppcm}
\DeclareMathOperator{\conj}{Conj}
\DeclareMathOperator{\End}{End}
\DeclareMathOperator{\rot}{rot}
\DeclareMathOperator{\trs}{trace}
\DeclareMathOperator{\Ind}{Ind}
\DeclareMathOperator{\mat}{Mat}
\DeclareMathOperator{\id}{Id}
\DeclareMathOperator{\vect}{vect}
\DeclareMathOperator{\img}{img}
\DeclareMathOperator{\cov}{Cov}
\DeclareMathOperator{\dist}{dist}
\DeclareMathOperator{\irr}{Irr}
\DeclareMathOperator{\image}{Im}
\DeclareMathOperator{\pd}{\partial}
\DeclareMathOperator{\epi}{epi}
\DeclareMathOperator{\Argmin}{Argmin}
\DeclareMathOperator{\dom}{dom}
\DeclareMathOperator{\proj}{proj}
\DeclareMathOperator{\ctg}{ctg}
\DeclareMathOperator{\supp}{supp}
\DeclareMathOperator{\argmin}{argmin}
\DeclareMathOperator{\mult}{mult}
\DeclareMathOperator{\ch}{ch}
\DeclareMathOperator{\sh}{sh}
\DeclareMathOperator{\rang}{rang}
\DeclareMathOperator{\diam}{diam}
\DeclareMathOperator{\Epigraphe}{Epigraphe}




\usepackage{xcolor}
\everymath{\color{blue}}
%\everymath{\color[rgb]{0,1,1}}
%\pagecolor[rgb]{0,0,0.5}


\newcommand*{\pdtest}[3][]{\ensuremath{\frac{\partial^{#1} #2}{\partial #3}}}

\newcommand*{\deffunc}[6][]{\ensuremath{
\begin{array}{rcl}
#2 : #3 &\rightarrow& #4\\
#5 &\mapsto& #6
\end{array}
}}

\newcommand{\eqcolon}{\mathrel{\resizebox{\widthof{$\mathord{=}$}}{\height}{ $\!\!=\!\!\resizebox{1.2\width}{0.8\height}{\raisebox{0.23ex}{$\mathop{:}$}}\!\!$ }}}
\newcommand{\coloneq}{\mathrel{\resizebox{\widthof{$\mathord{=}$}}{\height}{ $\!\!\resizebox{1.2\width}{0.8\height}{\raisebox{0.23ex}{$\mathop{:}$}}\!\!=\!\!$ }}}
\newcommand{\eqcolonl}{\ensuremath{\mathrel{=\!\!\mathop{:}}}}
\newcommand{\coloneql}{\ensuremath{\mathrel{\mathop{:} \!\! =}}}
\newcommand{\vc}[1]{% inline column vector
  \left(\begin{smallmatrix}#1\end{smallmatrix}\right)%
}
\newcommand{\vr}[1]{% inline row vector
  \begin{smallmatrix}(\,#1\,)\end{smallmatrix}%
}
\makeatletter
\newcommand*{\defeq}{\ =\mathrel{\rlap{%
                     \raisebox{0.3ex}{$\m@th\cdot$}}%
                     \raisebox{-0.3ex}{$\m@th\cdot$}}%
                     }
\makeatother

\newcommand{\mathcircle}[1]{% inline row vector
 \overset{\circ}{#1}
}
\newcommand{\ulim}{% low limit
 \underline{\lim}
}
\newcommand{\ssi}{% iff
\iff
}
\newcommand{\ps}[2]{
\expval{#1 | #2}
}
\newcommand{\df}[1]{
\mqty{#1}
}
\newcommand{\n}[1]{
\norm{#1}
}
\newcommand{\sys}[1]{
\left\{\smqty{#1}\right.
}


\newcommand{\eqdef}{\ensuremath{\overset{\text{def}}=}}


\def\Circlearrowright{\ensuremath{%
  \rotatebox[origin=c]{230}{$\circlearrowright$}}}

\newcommand\ct[1]{\text{\rmfamily\upshape #1}}
\newcommand\question[1]{ {\color{red} ...!? \small #1}}
\newcommand\caz[1]{\left\{\begin{array} #1 \end{array}\right.}
\newcommand\const{\text{\rmfamily\upshape const}}
\newcommand\toP{ \overset{\pro}{\to}}
\newcommand\toPP{ \overset{\text{PP}}{\to}}
\newcommand{\oeq}{\mathrel{\text{\textcircled{$=$}}}}





\usepackage{xcolor}
% \usepackage[normalem]{ulem}
\usepackage{lipsum}
\makeatletter
% \newcommand\colorwave[1][blue]{\bgroup \markoverwith{\lower3.5\p@\hbox{\sixly \textcolor{#1}{\char58}}}\ULon}
%\font\sixly=lasy6 % does not re-load if already loaded, so no memory problem.

\newmdtheoremenv[
linewidth= 1pt,linecolor= blue,%
leftmargin=20,rightmargin=20,innertopmargin=0pt, innerrightmargin=40,%
tikzsetting = { draw=lightgray, line width = 0.3pt,dashed,%
dash pattern = on 15pt off 3pt},%
splittopskip=\topskip,skipbelow=\baselineskip,%
skipabove=\baselineskip,ntheorem,roundcorner=0pt,
% backgroundcolor=pagebg,font=\color{orange}\sffamily, fontcolor=white
]{examplebox}{Exemple}[section]



\newcommand\R{\mathbb{R}}
\newcommand\Z{\mathbb{Z}}
\newcommand\N{\mathbb{N}}
\newcommand\E{\mathbb{E}}
\newcommand\F{\mathcal{F}}
\newcommand\cH{\mathcal{H}}
\newcommand\V{\mathbb{V}}
\newcommand\dmo{ ^{-1} }
\newcommand\kapa{\kappa}
\newcommand\im{Im}
\newcommand\hs{\mathcal{H}}





\usepackage{soul}

\makeatletter
\newcommand*{\whiten}[1]{\llap{\textcolor{white}{{\the\SOUL@token}}\hspace{#1pt}}}
\DeclareRobustCommand*\myul{%
    \def\SOUL@everyspace{\underline{\space}\kern\z@}%
    \def\SOUL@everytoken{%
     \setbox0=\hbox{\the\SOUL@token}%
     \ifdim\dp0>\z@
        \raisebox{\dp0}{\underline{\phantom{\the\SOUL@token}}}%
        \whiten{1}\whiten{0}%
        \whiten{-1}\whiten{-2}%
        \llap{\the\SOUL@token}%
     \else
        \underline{\the\SOUL@token}%
     \fi}%
\SOUL@}
\makeatother

\newcommand*{\demp}{\fontfamily{lmtt}\selectfont}

\DeclareTextFontCommand{\textdemp}{\demp}

\begin{document}

\ifcomment
Multiline
comment
\fi
\ifcomment
\myul{Typesetting test}
% \color[rgb]{1,1,1}
$∑_i^n≠ 60º±∞π∆¬≈√j∫h≤≥µ$

$\CR \R\pro\ind\pro\gS\pro
\mqty[a&b\\c&d]$
$\pro\mathbb{P}$
$\dd{x}$

  \[
    \alpha(x)=\left\{
                \begin{array}{ll}
                  x\\
                  \frac{1}{1+e^{-kx}}\\
                  \frac{e^x-e^{-x}}{e^x+e^{-x}}
                \end{array}
              \right.
  \]

  $\expval{x}$
  
  $\chi_\rho(ghg\dmo)=\Tr(\rho_{ghg\dmo})=\Tr(\rho_g\circ\rho_h\circ\rho\dmo_g)=\Tr(\rho_h)\overset{\mbox{\scalebox{0.5}{$\Tr(AB)=\Tr(BA)$}}}{=}\chi_\rho(h)$
  	$\mathop{\oplus}_{\substack{x\in X}}$

$\mat(\rho_g)=(a_{ij}(g))_{\scriptsize \substack{1\leq i\leq d \\ 1\leq j\leq d}}$ et $\mat(\rho'_g)=(a'_{ij}(g))_{\scriptsize \substack{1\leq i'\leq d' \\ 1\leq j'\leq d'}}$



\[\int_a^b{\mathbb{R}^2}g(u, v)\dd{P_{XY}}(u, v)=\iint g(u,v) f_{XY}(u, v)\dd \lambda(u) \dd \lambda(v)\]
$$\lim_{x\to\infty} f(x)$$	
$$\iiiint_V \mu(t,u,v,w) \,dt\,du\,dv\,dw$$
$$\sum_{n=1}^{\infty} 2^{-n} = 1$$	
\begin{definition}
	Si $X$ et $Y$ sont 2 v.a. ou definit la \textsc{Covariance} entre $X$ et $Y$ comme
	$\cov(X,Y)\overset{\text{def}}{=}\E\left[(X-\E(X))(Y-\E(Y))\right]=\E(XY)-\E(X)\E(Y)$.
\end{definition}
\fi
\pagebreak

% \tableofcontents

% insert your code here
%\input{./algebra/main.tex}
%\input{./geometrie-differentielle/main.tex}
%\input{./probabilite/main.tex}
%\input{./analyse-fonctionnelle/main.tex}
% \input{./Analyse-convexe-et-dualite-en-optimisation/main.tex}
%\input{./tikz/main.tex}
%\input{./Theorie-du-distributions/main.tex}
%\input{./optimisation/mine.tex}
 \input{./modelisation/main.tex}

% yves.aubry@univ-tln.fr : algebra

\end{document}

%\input{./optimisation/mine.tex}
 % !TEX encoding = UTF-8 Unicode
% !TEX TS-program = xelatex

\documentclass[french]{report}

%\usepackage[utf8]{inputenc}
%\usepackage[T1]{fontenc}
\usepackage{babel}


\newif\ifcomment
%\commenttrue # Show comments

\usepackage{physics}
\usepackage{amssymb}


\usepackage{amsthm}
% \usepackage{thmtools}
\usepackage{mathtools}
\usepackage{amsfonts}

\usepackage{color}

\usepackage{tikz}

\usepackage{geometry}
\geometry{a5paper, margin=0.1in, right=1cm}

\usepackage{dsfont}

\usepackage{graphicx}
\graphicspath{ {images/} }

\usepackage{faktor}

\usepackage{IEEEtrantools}
\usepackage{enumerate}   
\usepackage[PostScript=dvips]{"/Users/aware/Documents/Courses/diagrams"}


\newtheorem{theorem}{Théorème}[section]
\renewcommand{\thetheorem}{\arabic{theorem}}
\newtheorem{lemme}{Lemme}[section]
\renewcommand{\thelemme}{\arabic{lemme}}
\newtheorem{proposition}{Proposition}[section]
\renewcommand{\theproposition}{\arabic{proposition}}
\newtheorem{notations}{Notations}[section]
\newtheorem{problem}{Problème}[section]
\newtheorem{corollary}{Corollaire}[theorem]
\renewcommand{\thecorollary}{\arabic{corollary}}
\newtheorem{property}{Propriété}[section]
\newtheorem{objective}{Objectif}[section]

\theoremstyle{definition}
\newtheorem{definition}{Définition}[section]
\renewcommand{\thedefinition}{\arabic{definition}}
\newtheorem{exercise}{Exercice}[chapter]
\renewcommand{\theexercise}{\arabic{exercise}}
\newtheorem{example}{Exemple}[chapter]
\renewcommand{\theexample}{\arabic{example}}
\newtheorem*{solution}{Solution}
\newtheorem*{application}{Application}
\newtheorem*{notation}{Notation}
\newtheorem*{vocabulary}{Vocabulaire}
\newtheorem*{properties}{Propriétés}



\theoremstyle{remark}
\newtheorem*{remark}{Remarque}
\newtheorem*{rappel}{Rappel}


\usepackage{etoolbox}
\AtBeginEnvironment{exercise}{\small}
\AtBeginEnvironment{example}{\small}

\usepackage{cases}
\usepackage[red]{mypack}

\usepackage[framemethod=TikZ]{mdframed}

\definecolor{bg}{rgb}{0.4,0.25,0.95}
\definecolor{pagebg}{rgb}{0,0,0.5}
\surroundwithmdframed[
   topline=false,
   rightline=false,
   bottomline=false,
   leftmargin=\parindent,
   skipabove=8pt,
   skipbelow=8pt,
   linecolor=blue,
   innerbottommargin=10pt,
   % backgroundcolor=bg,font=\color{orange}\sffamily, fontcolor=white
]{definition}

\usepackage{empheq}
\usepackage[most]{tcolorbox}

\newtcbox{\mymath}[1][]{%
    nobeforeafter, math upper, tcbox raise base,
    enhanced, colframe=blue!30!black,
    colback=red!10, boxrule=1pt,
    #1}

\usepackage{unixode}


\DeclareMathOperator{\ord}{ord}
\DeclareMathOperator{\orb}{orb}
\DeclareMathOperator{\stab}{stab}
\DeclareMathOperator{\Stab}{stab}
\DeclareMathOperator{\ppcm}{ppcm}
\DeclareMathOperator{\conj}{Conj}
\DeclareMathOperator{\End}{End}
\DeclareMathOperator{\rot}{rot}
\DeclareMathOperator{\trs}{trace}
\DeclareMathOperator{\Ind}{Ind}
\DeclareMathOperator{\mat}{Mat}
\DeclareMathOperator{\id}{Id}
\DeclareMathOperator{\vect}{vect}
\DeclareMathOperator{\img}{img}
\DeclareMathOperator{\cov}{Cov}
\DeclareMathOperator{\dist}{dist}
\DeclareMathOperator{\irr}{Irr}
\DeclareMathOperator{\image}{Im}
\DeclareMathOperator{\pd}{\partial}
\DeclareMathOperator{\epi}{epi}
\DeclareMathOperator{\Argmin}{Argmin}
\DeclareMathOperator{\dom}{dom}
\DeclareMathOperator{\proj}{proj}
\DeclareMathOperator{\ctg}{ctg}
\DeclareMathOperator{\supp}{supp}
\DeclareMathOperator{\argmin}{argmin}
\DeclareMathOperator{\mult}{mult}
\DeclareMathOperator{\ch}{ch}
\DeclareMathOperator{\sh}{sh}
\DeclareMathOperator{\rang}{rang}
\DeclareMathOperator{\diam}{diam}
\DeclareMathOperator{\Epigraphe}{Epigraphe}




\usepackage{xcolor}
\everymath{\color{blue}}
%\everymath{\color[rgb]{0,1,1}}
%\pagecolor[rgb]{0,0,0.5}


\newcommand*{\pdtest}[3][]{\ensuremath{\frac{\partial^{#1} #2}{\partial #3}}}

\newcommand*{\deffunc}[6][]{\ensuremath{
\begin{array}{rcl}
#2 : #3 &\rightarrow& #4\\
#5 &\mapsto& #6
\end{array}
}}

\newcommand{\eqcolon}{\mathrel{\resizebox{\widthof{$\mathord{=}$}}{\height}{ $\!\!=\!\!\resizebox{1.2\width}{0.8\height}{\raisebox{0.23ex}{$\mathop{:}$}}\!\!$ }}}
\newcommand{\coloneq}{\mathrel{\resizebox{\widthof{$\mathord{=}$}}{\height}{ $\!\!\resizebox{1.2\width}{0.8\height}{\raisebox{0.23ex}{$\mathop{:}$}}\!\!=\!\!$ }}}
\newcommand{\eqcolonl}{\ensuremath{\mathrel{=\!\!\mathop{:}}}}
\newcommand{\coloneql}{\ensuremath{\mathrel{\mathop{:} \!\! =}}}
\newcommand{\vc}[1]{% inline column vector
  \left(\begin{smallmatrix}#1\end{smallmatrix}\right)%
}
\newcommand{\vr}[1]{% inline row vector
  \begin{smallmatrix}(\,#1\,)\end{smallmatrix}%
}
\makeatletter
\newcommand*{\defeq}{\ =\mathrel{\rlap{%
                     \raisebox{0.3ex}{$\m@th\cdot$}}%
                     \raisebox{-0.3ex}{$\m@th\cdot$}}%
                     }
\makeatother

\newcommand{\mathcircle}[1]{% inline row vector
 \overset{\circ}{#1}
}
\newcommand{\ulim}{% low limit
 \underline{\lim}
}
\newcommand{\ssi}{% iff
\iff
}
\newcommand{\ps}[2]{
\expval{#1 | #2}
}
\newcommand{\df}[1]{
\mqty{#1}
}
\newcommand{\n}[1]{
\norm{#1}
}
\newcommand{\sys}[1]{
\left\{\smqty{#1}\right.
}


\newcommand{\eqdef}{\ensuremath{\overset{\text{def}}=}}


\def\Circlearrowright{\ensuremath{%
  \rotatebox[origin=c]{230}{$\circlearrowright$}}}

\newcommand\ct[1]{\text{\rmfamily\upshape #1}}
\newcommand\question[1]{ {\color{red} ...!? \small #1}}
\newcommand\caz[1]{\left\{\begin{array} #1 \end{array}\right.}
\newcommand\const{\text{\rmfamily\upshape const}}
\newcommand\toP{ \overset{\pro}{\to}}
\newcommand\toPP{ \overset{\text{PP}}{\to}}
\newcommand{\oeq}{\mathrel{\text{\textcircled{$=$}}}}





\usepackage{xcolor}
% \usepackage[normalem]{ulem}
\usepackage{lipsum}
\makeatletter
% \newcommand\colorwave[1][blue]{\bgroup \markoverwith{\lower3.5\p@\hbox{\sixly \textcolor{#1}{\char58}}}\ULon}
%\font\sixly=lasy6 % does not re-load if already loaded, so no memory problem.

\newmdtheoremenv[
linewidth= 1pt,linecolor= blue,%
leftmargin=20,rightmargin=20,innertopmargin=0pt, innerrightmargin=40,%
tikzsetting = { draw=lightgray, line width = 0.3pt,dashed,%
dash pattern = on 15pt off 3pt},%
splittopskip=\topskip,skipbelow=\baselineskip,%
skipabove=\baselineskip,ntheorem,roundcorner=0pt,
% backgroundcolor=pagebg,font=\color{orange}\sffamily, fontcolor=white
]{examplebox}{Exemple}[section]



\newcommand\R{\mathbb{R}}
\newcommand\Z{\mathbb{Z}}
\newcommand\N{\mathbb{N}}
\newcommand\E{\mathbb{E}}
\newcommand\F{\mathcal{F}}
\newcommand\cH{\mathcal{H}}
\newcommand\V{\mathbb{V}}
\newcommand\dmo{ ^{-1} }
\newcommand\kapa{\kappa}
\newcommand\im{Im}
\newcommand\hs{\mathcal{H}}





\usepackage{soul}

\makeatletter
\newcommand*{\whiten}[1]{\llap{\textcolor{white}{{\the\SOUL@token}}\hspace{#1pt}}}
\DeclareRobustCommand*\myul{%
    \def\SOUL@everyspace{\underline{\space}\kern\z@}%
    \def\SOUL@everytoken{%
     \setbox0=\hbox{\the\SOUL@token}%
     \ifdim\dp0>\z@
        \raisebox{\dp0}{\underline{\phantom{\the\SOUL@token}}}%
        \whiten{1}\whiten{0}%
        \whiten{-1}\whiten{-2}%
        \llap{\the\SOUL@token}%
     \else
        \underline{\the\SOUL@token}%
     \fi}%
\SOUL@}
\makeatother

\newcommand*{\demp}{\fontfamily{lmtt}\selectfont}

\DeclareTextFontCommand{\textdemp}{\demp}

\begin{document}

\ifcomment
Multiline
comment
\fi
\ifcomment
\myul{Typesetting test}
% \color[rgb]{1,1,1}
$∑_i^n≠ 60º±∞π∆¬≈√j∫h≤≥µ$

$\CR \R\pro\ind\pro\gS\pro
\mqty[a&b\\c&d]$
$\pro\mathbb{P}$
$\dd{x}$

  \[
    \alpha(x)=\left\{
                \begin{array}{ll}
                  x\\
                  \frac{1}{1+e^{-kx}}\\
                  \frac{e^x-e^{-x}}{e^x+e^{-x}}
                \end{array}
              \right.
  \]

  $\expval{x}$
  
  $\chi_\rho(ghg\dmo)=\Tr(\rho_{ghg\dmo})=\Tr(\rho_g\circ\rho_h\circ\rho\dmo_g)=\Tr(\rho_h)\overset{\mbox{\scalebox{0.5}{$\Tr(AB)=\Tr(BA)$}}}{=}\chi_\rho(h)$
  	$\mathop{\oplus}_{\substack{x\in X}}$

$\mat(\rho_g)=(a_{ij}(g))_{\scriptsize \substack{1\leq i\leq d \\ 1\leq j\leq d}}$ et $\mat(\rho'_g)=(a'_{ij}(g))_{\scriptsize \substack{1\leq i'\leq d' \\ 1\leq j'\leq d'}}$



\[\int_a^b{\mathbb{R}^2}g(u, v)\dd{P_{XY}}(u, v)=\iint g(u,v) f_{XY}(u, v)\dd \lambda(u) \dd \lambda(v)\]
$$\lim_{x\to\infty} f(x)$$	
$$\iiiint_V \mu(t,u,v,w) \,dt\,du\,dv\,dw$$
$$\sum_{n=1}^{\infty} 2^{-n} = 1$$	
\begin{definition}
	Si $X$ et $Y$ sont 2 v.a. ou definit la \textsc{Covariance} entre $X$ et $Y$ comme
	$\cov(X,Y)\overset{\text{def}}{=}\E\left[(X-\E(X))(Y-\E(Y))\right]=\E(XY)-\E(X)\E(Y)$.
\end{definition}
\fi
\pagebreak

% \tableofcontents

% insert your code here
%\input{./algebra/main.tex}
%\input{./geometrie-differentielle/main.tex}
%\input{./probabilite/main.tex}
%\input{./analyse-fonctionnelle/main.tex}
% \input{./Analyse-convexe-et-dualite-en-optimisation/main.tex}
%\input{./tikz/main.tex}
%\input{./Theorie-du-distributions/main.tex}
%\input{./optimisation/mine.tex}
 \input{./modelisation/main.tex}

% yves.aubry@univ-tln.fr : algebra

\end{document}


% yves.aubry@univ-tln.fr : algebra

\end{document}

% % !TEX encoding = UTF-8 Unicode
% !TEX TS-program = xelatex

\documentclass[french]{report}

%\usepackage[utf8]{inputenc}
%\usepackage[T1]{fontenc}
\usepackage{babel}


\newif\ifcomment
%\commenttrue # Show comments

\usepackage{physics}
\usepackage{amssymb}


\usepackage{amsthm}
% \usepackage{thmtools}
\usepackage{mathtools}
\usepackage{amsfonts}

\usepackage{color}

\usepackage{tikz}

\usepackage{geometry}
\geometry{a5paper, margin=0.1in, right=1cm}

\usepackage{dsfont}

\usepackage{graphicx}
\graphicspath{ {images/} }

\usepackage{faktor}

\usepackage{IEEEtrantools}
\usepackage{enumerate}   
\usepackage[PostScript=dvips]{"/Users/aware/Documents/Courses/diagrams"}


\newtheorem{theorem}{Théorème}[section]
\renewcommand{\thetheorem}{\arabic{theorem}}
\newtheorem{lemme}{Lemme}[section]
\renewcommand{\thelemme}{\arabic{lemme}}
\newtheorem{proposition}{Proposition}[section]
\renewcommand{\theproposition}{\arabic{proposition}}
\newtheorem{notations}{Notations}[section]
\newtheorem{problem}{Problème}[section]
\newtheorem{corollary}{Corollaire}[theorem]
\renewcommand{\thecorollary}{\arabic{corollary}}
\newtheorem{property}{Propriété}[section]
\newtheorem{objective}{Objectif}[section]

\theoremstyle{definition}
\newtheorem{definition}{Définition}[section]
\renewcommand{\thedefinition}{\arabic{definition}}
\newtheorem{exercise}{Exercice}[chapter]
\renewcommand{\theexercise}{\arabic{exercise}}
\newtheorem{example}{Exemple}[chapter]
\renewcommand{\theexample}{\arabic{example}}
\newtheorem*{solution}{Solution}
\newtheorem*{application}{Application}
\newtheorem*{notation}{Notation}
\newtheorem*{vocabulary}{Vocabulaire}
\newtheorem*{properties}{Propriétés}



\theoremstyle{remark}
\newtheorem*{remark}{Remarque}
\newtheorem*{rappel}{Rappel}


\usepackage{etoolbox}
\AtBeginEnvironment{exercise}{\small}
\AtBeginEnvironment{example}{\small}

\usepackage{cases}
\usepackage[red]{mypack}

\usepackage[framemethod=TikZ]{mdframed}

\definecolor{bg}{rgb}{0.4,0.25,0.95}
\definecolor{pagebg}{rgb}{0,0,0.5}
\surroundwithmdframed[
   topline=false,
   rightline=false,
   bottomline=false,
   leftmargin=\parindent,
   skipabove=8pt,
   skipbelow=8pt,
   linecolor=blue,
   innerbottommargin=10pt,
   % backgroundcolor=bg,font=\color{orange}\sffamily, fontcolor=white
]{definition}

\usepackage{empheq}
\usepackage[most]{tcolorbox}

\newtcbox{\mymath}[1][]{%
    nobeforeafter, math upper, tcbox raise base,
    enhanced, colframe=blue!30!black,
    colback=red!10, boxrule=1pt,
    #1}

\usepackage{unixode}


\DeclareMathOperator{\ord}{ord}
\DeclareMathOperator{\orb}{orb}
\DeclareMathOperator{\stab}{stab}
\DeclareMathOperator{\Stab}{stab}
\DeclareMathOperator{\ppcm}{ppcm}
\DeclareMathOperator{\conj}{Conj}
\DeclareMathOperator{\End}{End}
\DeclareMathOperator{\rot}{rot}
\DeclareMathOperator{\trs}{trace}
\DeclareMathOperator{\Ind}{Ind}
\DeclareMathOperator{\mat}{Mat}
\DeclareMathOperator{\id}{Id}
\DeclareMathOperator{\vect}{vect}
\DeclareMathOperator{\img}{img}
\DeclareMathOperator{\cov}{Cov}
\DeclareMathOperator{\dist}{dist}
\DeclareMathOperator{\irr}{Irr}
\DeclareMathOperator{\image}{Im}
\DeclareMathOperator{\pd}{\partial}
\DeclareMathOperator{\epi}{epi}
\DeclareMathOperator{\Argmin}{Argmin}
\DeclareMathOperator{\dom}{dom}
\DeclareMathOperator{\proj}{proj}
\DeclareMathOperator{\ctg}{ctg}
\DeclareMathOperator{\supp}{supp}
\DeclareMathOperator{\argmin}{argmin}
\DeclareMathOperator{\mult}{mult}
\DeclareMathOperator{\ch}{ch}
\DeclareMathOperator{\sh}{sh}
\DeclareMathOperator{\rang}{rang}
\DeclareMathOperator{\diam}{diam}
\DeclareMathOperator{\Epigraphe}{Epigraphe}




\usepackage{xcolor}
\everymath{\color{blue}}
%\everymath{\color[rgb]{0,1,1}}
%\pagecolor[rgb]{0,0,0.5}


\newcommand*{\pdtest}[3][]{\ensuremath{\frac{\partial^{#1} #2}{\partial #3}}}

\newcommand*{\deffunc}[6][]{\ensuremath{
\begin{array}{rcl}
#2 : #3 &\rightarrow& #4\\
#5 &\mapsto& #6
\end{array}
}}

\newcommand{\eqcolon}{\mathrel{\resizebox{\widthof{$\mathord{=}$}}{\height}{ $\!\!=\!\!\resizebox{1.2\width}{0.8\height}{\raisebox{0.23ex}{$\mathop{:}$}}\!\!$ }}}
\newcommand{\coloneq}{\mathrel{\resizebox{\widthof{$\mathord{=}$}}{\height}{ $\!\!\resizebox{1.2\width}{0.8\height}{\raisebox{0.23ex}{$\mathop{:}$}}\!\!=\!\!$ }}}
\newcommand{\eqcolonl}{\ensuremath{\mathrel{=\!\!\mathop{:}}}}
\newcommand{\coloneql}{\ensuremath{\mathrel{\mathop{:} \!\! =}}}
\newcommand{\vc}[1]{% inline column vector
  \left(\begin{smallmatrix}#1\end{smallmatrix}\right)%
}
\newcommand{\vr}[1]{% inline row vector
  \begin{smallmatrix}(\,#1\,)\end{smallmatrix}%
}
\makeatletter
\newcommand*{\defeq}{\ =\mathrel{\rlap{%
                     \raisebox{0.3ex}{$\m@th\cdot$}}%
                     \raisebox{-0.3ex}{$\m@th\cdot$}}%
                     }
\makeatother

\newcommand{\mathcircle}[1]{% inline row vector
 \overset{\circ}{#1}
}
\newcommand{\ulim}{% low limit
 \underline{\lim}
}
\newcommand{\ssi}{% iff
\iff
}
\newcommand{\ps}[2]{
\expval{#1 | #2}
}
\newcommand{\df}[1]{
\mqty{#1}
}
\newcommand{\n}[1]{
\norm{#1}
}
\newcommand{\sys}[1]{
\left\{\smqty{#1}\right.
}


\newcommand{\eqdef}{\ensuremath{\overset{\text{def}}=}}


\def\Circlearrowright{\ensuremath{%
  \rotatebox[origin=c]{230}{$\circlearrowright$}}}

\newcommand\ct[1]{\text{\rmfamily\upshape #1}}
\newcommand\question[1]{ {\color{red} ...!? \small #1}}
\newcommand\caz[1]{\left\{\begin{array} #1 \end{array}\right.}
\newcommand\const{\text{\rmfamily\upshape const}}
\newcommand\toP{ \overset{\pro}{\to}}
\newcommand\toPP{ \overset{\text{PP}}{\to}}
\newcommand{\oeq}{\mathrel{\text{\textcircled{$=$}}}}





\usepackage{xcolor}
% \usepackage[normalem]{ulem}
\usepackage{lipsum}
\makeatletter
% \newcommand\colorwave[1][blue]{\bgroup \markoverwith{\lower3.5\p@\hbox{\sixly \textcolor{#1}{\char58}}}\ULon}
%\font\sixly=lasy6 % does not re-load if already loaded, so no memory problem.

\newmdtheoremenv[
linewidth= 1pt,linecolor= blue,%
leftmargin=20,rightmargin=20,innertopmargin=0pt, innerrightmargin=40,%
tikzsetting = { draw=lightgray, line width = 0.3pt,dashed,%
dash pattern = on 15pt off 3pt},%
splittopskip=\topskip,skipbelow=\baselineskip,%
skipabove=\baselineskip,ntheorem,roundcorner=0pt,
% backgroundcolor=pagebg,font=\color{orange}\sffamily, fontcolor=white
]{examplebox}{Exemple}[section]



\newcommand\R{\mathbb{R}}
\newcommand\Z{\mathbb{Z}}
\newcommand\N{\mathbb{N}}
\newcommand\E{\mathbb{E}}
\newcommand\F{\mathcal{F}}
\newcommand\cH{\mathcal{H}}
\newcommand\V{\mathbb{V}}
\newcommand\dmo{ ^{-1} }
\newcommand\kapa{\kappa}
\newcommand\im{Im}
\newcommand\hs{\mathcal{H}}





\usepackage{soul}

\makeatletter
\newcommand*{\whiten}[1]{\llap{\textcolor{white}{{\the\SOUL@token}}\hspace{#1pt}}}
\DeclareRobustCommand*\myul{%
    \def\SOUL@everyspace{\underline{\space}\kern\z@}%
    \def\SOUL@everytoken{%
     \setbox0=\hbox{\the\SOUL@token}%
     \ifdim\dp0>\z@
        \raisebox{\dp0}{\underline{\phantom{\the\SOUL@token}}}%
        \whiten{1}\whiten{0}%
        \whiten{-1}\whiten{-2}%
        \llap{\the\SOUL@token}%
     \else
        \underline{\the\SOUL@token}%
     \fi}%
\SOUL@}
\makeatother

\newcommand*{\demp}{\fontfamily{lmtt}\selectfont}

\DeclareTextFontCommand{\textdemp}{\demp}

\begin{document}

\ifcomment
Multiline
comment
\fi
\ifcomment
\myul{Typesetting test}
% \color[rgb]{1,1,1}
$∑_i^n≠ 60º±∞π∆¬≈√j∫h≤≥µ$

$\CR \R\pro\ind\pro\gS\pro
\mqty[a&b\\c&d]$
$\pro\mathbb{P}$
$\dd{x}$

  \[
    \alpha(x)=\left\{
                \begin{array}{ll}
                  x\\
                  \frac{1}{1+e^{-kx}}\\
                  \frac{e^x-e^{-x}}{e^x+e^{-x}}
                \end{array}
              \right.
  \]

  $\expval{x}$
  
  $\chi_\rho(ghg\dmo)=\Tr(\rho_{ghg\dmo})=\Tr(\rho_g\circ\rho_h\circ\rho\dmo_g)=\Tr(\rho_h)\overset{\mbox{\scalebox{0.5}{$\Tr(AB)=\Tr(BA)$}}}{=}\chi_\rho(h)$
  	$\mathop{\oplus}_{\substack{x\in X}}$

$\mat(\rho_g)=(a_{ij}(g))_{\scriptsize \substack{1\leq i\leq d \\ 1\leq j\leq d}}$ et $\mat(\rho'_g)=(a'_{ij}(g))_{\scriptsize \substack{1\leq i'\leq d' \\ 1\leq j'\leq d'}}$



\[\int_a^b{\mathbb{R}^2}g(u, v)\dd{P_{XY}}(u, v)=\iint g(u,v) f_{XY}(u, v)\dd \lambda(u) \dd \lambda(v)\]
$$\lim_{x\to\infty} f(x)$$	
$$\iiiint_V \mu(t,u,v,w) \,dt\,du\,dv\,dw$$
$$\sum_{n=1}^{\infty} 2^{-n} = 1$$	
\begin{definition}
	Si $X$ et $Y$ sont 2 v.a. ou definit la \textsc{Covariance} entre $X$ et $Y$ comme
	$\cov(X,Y)\overset{\text{def}}{=}\E\left[(X-\E(X))(Y-\E(Y))\right]=\E(XY)-\E(X)\E(Y)$.
\end{definition}
\fi
\pagebreak

% \tableofcontents

% insert your code here
%% !TEX encoding = UTF-8 Unicode
% !TEX TS-program = xelatex

\documentclass[french]{report}

%\usepackage[utf8]{inputenc}
%\usepackage[T1]{fontenc}
\usepackage{babel}


\newif\ifcomment
%\commenttrue # Show comments

\usepackage{physics}
\usepackage{amssymb}


\usepackage{amsthm}
% \usepackage{thmtools}
\usepackage{mathtools}
\usepackage{amsfonts}

\usepackage{color}

\usepackage{tikz}

\usepackage{geometry}
\geometry{a5paper, margin=0.1in, right=1cm}

\usepackage{dsfont}

\usepackage{graphicx}
\graphicspath{ {images/} }

\usepackage{faktor}

\usepackage{IEEEtrantools}
\usepackage{enumerate}   
\usepackage[PostScript=dvips]{"/Users/aware/Documents/Courses/diagrams"}


\newtheorem{theorem}{Théorème}[section]
\renewcommand{\thetheorem}{\arabic{theorem}}
\newtheorem{lemme}{Lemme}[section]
\renewcommand{\thelemme}{\arabic{lemme}}
\newtheorem{proposition}{Proposition}[section]
\renewcommand{\theproposition}{\arabic{proposition}}
\newtheorem{notations}{Notations}[section]
\newtheorem{problem}{Problème}[section]
\newtheorem{corollary}{Corollaire}[theorem]
\renewcommand{\thecorollary}{\arabic{corollary}}
\newtheorem{property}{Propriété}[section]
\newtheorem{objective}{Objectif}[section]

\theoremstyle{definition}
\newtheorem{definition}{Définition}[section]
\renewcommand{\thedefinition}{\arabic{definition}}
\newtheorem{exercise}{Exercice}[chapter]
\renewcommand{\theexercise}{\arabic{exercise}}
\newtheorem{example}{Exemple}[chapter]
\renewcommand{\theexample}{\arabic{example}}
\newtheorem*{solution}{Solution}
\newtheorem*{application}{Application}
\newtheorem*{notation}{Notation}
\newtheorem*{vocabulary}{Vocabulaire}
\newtheorem*{properties}{Propriétés}



\theoremstyle{remark}
\newtheorem*{remark}{Remarque}
\newtheorem*{rappel}{Rappel}


\usepackage{etoolbox}
\AtBeginEnvironment{exercise}{\small}
\AtBeginEnvironment{example}{\small}

\usepackage{cases}
\usepackage[red]{mypack}

\usepackage[framemethod=TikZ]{mdframed}

\definecolor{bg}{rgb}{0.4,0.25,0.95}
\definecolor{pagebg}{rgb}{0,0,0.5}
\surroundwithmdframed[
   topline=false,
   rightline=false,
   bottomline=false,
   leftmargin=\parindent,
   skipabove=8pt,
   skipbelow=8pt,
   linecolor=blue,
   innerbottommargin=10pt,
   % backgroundcolor=bg,font=\color{orange}\sffamily, fontcolor=white
]{definition}

\usepackage{empheq}
\usepackage[most]{tcolorbox}

\newtcbox{\mymath}[1][]{%
    nobeforeafter, math upper, tcbox raise base,
    enhanced, colframe=blue!30!black,
    colback=red!10, boxrule=1pt,
    #1}

\usepackage{unixode}


\DeclareMathOperator{\ord}{ord}
\DeclareMathOperator{\orb}{orb}
\DeclareMathOperator{\stab}{stab}
\DeclareMathOperator{\Stab}{stab}
\DeclareMathOperator{\ppcm}{ppcm}
\DeclareMathOperator{\conj}{Conj}
\DeclareMathOperator{\End}{End}
\DeclareMathOperator{\rot}{rot}
\DeclareMathOperator{\trs}{trace}
\DeclareMathOperator{\Ind}{Ind}
\DeclareMathOperator{\mat}{Mat}
\DeclareMathOperator{\id}{Id}
\DeclareMathOperator{\vect}{vect}
\DeclareMathOperator{\img}{img}
\DeclareMathOperator{\cov}{Cov}
\DeclareMathOperator{\dist}{dist}
\DeclareMathOperator{\irr}{Irr}
\DeclareMathOperator{\image}{Im}
\DeclareMathOperator{\pd}{\partial}
\DeclareMathOperator{\epi}{epi}
\DeclareMathOperator{\Argmin}{Argmin}
\DeclareMathOperator{\dom}{dom}
\DeclareMathOperator{\proj}{proj}
\DeclareMathOperator{\ctg}{ctg}
\DeclareMathOperator{\supp}{supp}
\DeclareMathOperator{\argmin}{argmin}
\DeclareMathOperator{\mult}{mult}
\DeclareMathOperator{\ch}{ch}
\DeclareMathOperator{\sh}{sh}
\DeclareMathOperator{\rang}{rang}
\DeclareMathOperator{\diam}{diam}
\DeclareMathOperator{\Epigraphe}{Epigraphe}




\usepackage{xcolor}
\everymath{\color{blue}}
%\everymath{\color[rgb]{0,1,1}}
%\pagecolor[rgb]{0,0,0.5}


\newcommand*{\pdtest}[3][]{\ensuremath{\frac{\partial^{#1} #2}{\partial #3}}}

\newcommand*{\deffunc}[6][]{\ensuremath{
\begin{array}{rcl}
#2 : #3 &\rightarrow& #4\\
#5 &\mapsto& #6
\end{array}
}}

\newcommand{\eqcolon}{\mathrel{\resizebox{\widthof{$\mathord{=}$}}{\height}{ $\!\!=\!\!\resizebox{1.2\width}{0.8\height}{\raisebox{0.23ex}{$\mathop{:}$}}\!\!$ }}}
\newcommand{\coloneq}{\mathrel{\resizebox{\widthof{$\mathord{=}$}}{\height}{ $\!\!\resizebox{1.2\width}{0.8\height}{\raisebox{0.23ex}{$\mathop{:}$}}\!\!=\!\!$ }}}
\newcommand{\eqcolonl}{\ensuremath{\mathrel{=\!\!\mathop{:}}}}
\newcommand{\coloneql}{\ensuremath{\mathrel{\mathop{:} \!\! =}}}
\newcommand{\vc}[1]{% inline column vector
  \left(\begin{smallmatrix}#1\end{smallmatrix}\right)%
}
\newcommand{\vr}[1]{% inline row vector
  \begin{smallmatrix}(\,#1\,)\end{smallmatrix}%
}
\makeatletter
\newcommand*{\defeq}{\ =\mathrel{\rlap{%
                     \raisebox{0.3ex}{$\m@th\cdot$}}%
                     \raisebox{-0.3ex}{$\m@th\cdot$}}%
                     }
\makeatother

\newcommand{\mathcircle}[1]{% inline row vector
 \overset{\circ}{#1}
}
\newcommand{\ulim}{% low limit
 \underline{\lim}
}
\newcommand{\ssi}{% iff
\iff
}
\newcommand{\ps}[2]{
\expval{#1 | #2}
}
\newcommand{\df}[1]{
\mqty{#1}
}
\newcommand{\n}[1]{
\norm{#1}
}
\newcommand{\sys}[1]{
\left\{\smqty{#1}\right.
}


\newcommand{\eqdef}{\ensuremath{\overset{\text{def}}=}}


\def\Circlearrowright{\ensuremath{%
  \rotatebox[origin=c]{230}{$\circlearrowright$}}}

\newcommand\ct[1]{\text{\rmfamily\upshape #1}}
\newcommand\question[1]{ {\color{red} ...!? \small #1}}
\newcommand\caz[1]{\left\{\begin{array} #1 \end{array}\right.}
\newcommand\const{\text{\rmfamily\upshape const}}
\newcommand\toP{ \overset{\pro}{\to}}
\newcommand\toPP{ \overset{\text{PP}}{\to}}
\newcommand{\oeq}{\mathrel{\text{\textcircled{$=$}}}}





\usepackage{xcolor}
% \usepackage[normalem]{ulem}
\usepackage{lipsum}
\makeatletter
% \newcommand\colorwave[1][blue]{\bgroup \markoverwith{\lower3.5\p@\hbox{\sixly \textcolor{#1}{\char58}}}\ULon}
%\font\sixly=lasy6 % does not re-load if already loaded, so no memory problem.

\newmdtheoremenv[
linewidth= 1pt,linecolor= blue,%
leftmargin=20,rightmargin=20,innertopmargin=0pt, innerrightmargin=40,%
tikzsetting = { draw=lightgray, line width = 0.3pt,dashed,%
dash pattern = on 15pt off 3pt},%
splittopskip=\topskip,skipbelow=\baselineskip,%
skipabove=\baselineskip,ntheorem,roundcorner=0pt,
% backgroundcolor=pagebg,font=\color{orange}\sffamily, fontcolor=white
]{examplebox}{Exemple}[section]



\newcommand\R{\mathbb{R}}
\newcommand\Z{\mathbb{Z}}
\newcommand\N{\mathbb{N}}
\newcommand\E{\mathbb{E}}
\newcommand\F{\mathcal{F}}
\newcommand\cH{\mathcal{H}}
\newcommand\V{\mathbb{V}}
\newcommand\dmo{ ^{-1} }
\newcommand\kapa{\kappa}
\newcommand\im{Im}
\newcommand\hs{\mathcal{H}}





\usepackage{soul}

\makeatletter
\newcommand*{\whiten}[1]{\llap{\textcolor{white}{{\the\SOUL@token}}\hspace{#1pt}}}
\DeclareRobustCommand*\myul{%
    \def\SOUL@everyspace{\underline{\space}\kern\z@}%
    \def\SOUL@everytoken{%
     \setbox0=\hbox{\the\SOUL@token}%
     \ifdim\dp0>\z@
        \raisebox{\dp0}{\underline{\phantom{\the\SOUL@token}}}%
        \whiten{1}\whiten{0}%
        \whiten{-1}\whiten{-2}%
        \llap{\the\SOUL@token}%
     \else
        \underline{\the\SOUL@token}%
     \fi}%
\SOUL@}
\makeatother

\newcommand*{\demp}{\fontfamily{lmtt}\selectfont}

\DeclareTextFontCommand{\textdemp}{\demp}

\begin{document}

\ifcomment
Multiline
comment
\fi
\ifcomment
\myul{Typesetting test}
% \color[rgb]{1,1,1}
$∑_i^n≠ 60º±∞π∆¬≈√j∫h≤≥µ$

$\CR \R\pro\ind\pro\gS\pro
\mqty[a&b\\c&d]$
$\pro\mathbb{P}$
$\dd{x}$

  \[
    \alpha(x)=\left\{
                \begin{array}{ll}
                  x\\
                  \frac{1}{1+e^{-kx}}\\
                  \frac{e^x-e^{-x}}{e^x+e^{-x}}
                \end{array}
              \right.
  \]

  $\expval{x}$
  
  $\chi_\rho(ghg\dmo)=\Tr(\rho_{ghg\dmo})=\Tr(\rho_g\circ\rho_h\circ\rho\dmo_g)=\Tr(\rho_h)\overset{\mbox{\scalebox{0.5}{$\Tr(AB)=\Tr(BA)$}}}{=}\chi_\rho(h)$
  	$\mathop{\oplus}_{\substack{x\in X}}$

$\mat(\rho_g)=(a_{ij}(g))_{\scriptsize \substack{1\leq i\leq d \\ 1\leq j\leq d}}$ et $\mat(\rho'_g)=(a'_{ij}(g))_{\scriptsize \substack{1\leq i'\leq d' \\ 1\leq j'\leq d'}}$



\[\int_a^b{\mathbb{R}^2}g(u, v)\dd{P_{XY}}(u, v)=\iint g(u,v) f_{XY}(u, v)\dd \lambda(u) \dd \lambda(v)\]
$$\lim_{x\to\infty} f(x)$$	
$$\iiiint_V \mu(t,u,v,w) \,dt\,du\,dv\,dw$$
$$\sum_{n=1}^{\infty} 2^{-n} = 1$$	
\begin{definition}
	Si $X$ et $Y$ sont 2 v.a. ou definit la \textsc{Covariance} entre $X$ et $Y$ comme
	$\cov(X,Y)\overset{\text{def}}{=}\E\left[(X-\E(X))(Y-\E(Y))\right]=\E(XY)-\E(X)\E(Y)$.
\end{definition}
\fi
\pagebreak

% \tableofcontents

% insert your code here
%\input{./algebra/main.tex}
%\input{./geometrie-differentielle/main.tex}
%\input{./probabilite/main.tex}
%\input{./analyse-fonctionnelle/main.tex}
% \input{./Analyse-convexe-et-dualite-en-optimisation/main.tex}
%\input{./tikz/main.tex}
%\input{./Theorie-du-distributions/main.tex}
%\input{./optimisation/mine.tex}
 \input{./modelisation/main.tex}

% yves.aubry@univ-tln.fr : algebra

\end{document}

%% !TEX encoding = UTF-8 Unicode
% !TEX TS-program = xelatex

\documentclass[french]{report}

%\usepackage[utf8]{inputenc}
%\usepackage[T1]{fontenc}
\usepackage{babel}


\newif\ifcomment
%\commenttrue # Show comments

\usepackage{physics}
\usepackage{amssymb}


\usepackage{amsthm}
% \usepackage{thmtools}
\usepackage{mathtools}
\usepackage{amsfonts}

\usepackage{color}

\usepackage{tikz}

\usepackage{geometry}
\geometry{a5paper, margin=0.1in, right=1cm}

\usepackage{dsfont}

\usepackage{graphicx}
\graphicspath{ {images/} }

\usepackage{faktor}

\usepackage{IEEEtrantools}
\usepackage{enumerate}   
\usepackage[PostScript=dvips]{"/Users/aware/Documents/Courses/diagrams"}


\newtheorem{theorem}{Théorème}[section]
\renewcommand{\thetheorem}{\arabic{theorem}}
\newtheorem{lemme}{Lemme}[section]
\renewcommand{\thelemme}{\arabic{lemme}}
\newtheorem{proposition}{Proposition}[section]
\renewcommand{\theproposition}{\arabic{proposition}}
\newtheorem{notations}{Notations}[section]
\newtheorem{problem}{Problème}[section]
\newtheorem{corollary}{Corollaire}[theorem]
\renewcommand{\thecorollary}{\arabic{corollary}}
\newtheorem{property}{Propriété}[section]
\newtheorem{objective}{Objectif}[section]

\theoremstyle{definition}
\newtheorem{definition}{Définition}[section]
\renewcommand{\thedefinition}{\arabic{definition}}
\newtheorem{exercise}{Exercice}[chapter]
\renewcommand{\theexercise}{\arabic{exercise}}
\newtheorem{example}{Exemple}[chapter]
\renewcommand{\theexample}{\arabic{example}}
\newtheorem*{solution}{Solution}
\newtheorem*{application}{Application}
\newtheorem*{notation}{Notation}
\newtheorem*{vocabulary}{Vocabulaire}
\newtheorem*{properties}{Propriétés}



\theoremstyle{remark}
\newtheorem*{remark}{Remarque}
\newtheorem*{rappel}{Rappel}


\usepackage{etoolbox}
\AtBeginEnvironment{exercise}{\small}
\AtBeginEnvironment{example}{\small}

\usepackage{cases}
\usepackage[red]{mypack}

\usepackage[framemethod=TikZ]{mdframed}

\definecolor{bg}{rgb}{0.4,0.25,0.95}
\definecolor{pagebg}{rgb}{0,0,0.5}
\surroundwithmdframed[
   topline=false,
   rightline=false,
   bottomline=false,
   leftmargin=\parindent,
   skipabove=8pt,
   skipbelow=8pt,
   linecolor=blue,
   innerbottommargin=10pt,
   % backgroundcolor=bg,font=\color{orange}\sffamily, fontcolor=white
]{definition}

\usepackage{empheq}
\usepackage[most]{tcolorbox}

\newtcbox{\mymath}[1][]{%
    nobeforeafter, math upper, tcbox raise base,
    enhanced, colframe=blue!30!black,
    colback=red!10, boxrule=1pt,
    #1}

\usepackage{unixode}


\DeclareMathOperator{\ord}{ord}
\DeclareMathOperator{\orb}{orb}
\DeclareMathOperator{\stab}{stab}
\DeclareMathOperator{\Stab}{stab}
\DeclareMathOperator{\ppcm}{ppcm}
\DeclareMathOperator{\conj}{Conj}
\DeclareMathOperator{\End}{End}
\DeclareMathOperator{\rot}{rot}
\DeclareMathOperator{\trs}{trace}
\DeclareMathOperator{\Ind}{Ind}
\DeclareMathOperator{\mat}{Mat}
\DeclareMathOperator{\id}{Id}
\DeclareMathOperator{\vect}{vect}
\DeclareMathOperator{\img}{img}
\DeclareMathOperator{\cov}{Cov}
\DeclareMathOperator{\dist}{dist}
\DeclareMathOperator{\irr}{Irr}
\DeclareMathOperator{\image}{Im}
\DeclareMathOperator{\pd}{\partial}
\DeclareMathOperator{\epi}{epi}
\DeclareMathOperator{\Argmin}{Argmin}
\DeclareMathOperator{\dom}{dom}
\DeclareMathOperator{\proj}{proj}
\DeclareMathOperator{\ctg}{ctg}
\DeclareMathOperator{\supp}{supp}
\DeclareMathOperator{\argmin}{argmin}
\DeclareMathOperator{\mult}{mult}
\DeclareMathOperator{\ch}{ch}
\DeclareMathOperator{\sh}{sh}
\DeclareMathOperator{\rang}{rang}
\DeclareMathOperator{\diam}{diam}
\DeclareMathOperator{\Epigraphe}{Epigraphe}




\usepackage{xcolor}
\everymath{\color{blue}}
%\everymath{\color[rgb]{0,1,1}}
%\pagecolor[rgb]{0,0,0.5}


\newcommand*{\pdtest}[3][]{\ensuremath{\frac{\partial^{#1} #2}{\partial #3}}}

\newcommand*{\deffunc}[6][]{\ensuremath{
\begin{array}{rcl}
#2 : #3 &\rightarrow& #4\\
#5 &\mapsto& #6
\end{array}
}}

\newcommand{\eqcolon}{\mathrel{\resizebox{\widthof{$\mathord{=}$}}{\height}{ $\!\!=\!\!\resizebox{1.2\width}{0.8\height}{\raisebox{0.23ex}{$\mathop{:}$}}\!\!$ }}}
\newcommand{\coloneq}{\mathrel{\resizebox{\widthof{$\mathord{=}$}}{\height}{ $\!\!\resizebox{1.2\width}{0.8\height}{\raisebox{0.23ex}{$\mathop{:}$}}\!\!=\!\!$ }}}
\newcommand{\eqcolonl}{\ensuremath{\mathrel{=\!\!\mathop{:}}}}
\newcommand{\coloneql}{\ensuremath{\mathrel{\mathop{:} \!\! =}}}
\newcommand{\vc}[1]{% inline column vector
  \left(\begin{smallmatrix}#1\end{smallmatrix}\right)%
}
\newcommand{\vr}[1]{% inline row vector
  \begin{smallmatrix}(\,#1\,)\end{smallmatrix}%
}
\makeatletter
\newcommand*{\defeq}{\ =\mathrel{\rlap{%
                     \raisebox{0.3ex}{$\m@th\cdot$}}%
                     \raisebox{-0.3ex}{$\m@th\cdot$}}%
                     }
\makeatother

\newcommand{\mathcircle}[1]{% inline row vector
 \overset{\circ}{#1}
}
\newcommand{\ulim}{% low limit
 \underline{\lim}
}
\newcommand{\ssi}{% iff
\iff
}
\newcommand{\ps}[2]{
\expval{#1 | #2}
}
\newcommand{\df}[1]{
\mqty{#1}
}
\newcommand{\n}[1]{
\norm{#1}
}
\newcommand{\sys}[1]{
\left\{\smqty{#1}\right.
}


\newcommand{\eqdef}{\ensuremath{\overset{\text{def}}=}}


\def\Circlearrowright{\ensuremath{%
  \rotatebox[origin=c]{230}{$\circlearrowright$}}}

\newcommand\ct[1]{\text{\rmfamily\upshape #1}}
\newcommand\question[1]{ {\color{red} ...!? \small #1}}
\newcommand\caz[1]{\left\{\begin{array} #1 \end{array}\right.}
\newcommand\const{\text{\rmfamily\upshape const}}
\newcommand\toP{ \overset{\pro}{\to}}
\newcommand\toPP{ \overset{\text{PP}}{\to}}
\newcommand{\oeq}{\mathrel{\text{\textcircled{$=$}}}}





\usepackage{xcolor}
% \usepackage[normalem]{ulem}
\usepackage{lipsum}
\makeatletter
% \newcommand\colorwave[1][blue]{\bgroup \markoverwith{\lower3.5\p@\hbox{\sixly \textcolor{#1}{\char58}}}\ULon}
%\font\sixly=lasy6 % does not re-load if already loaded, so no memory problem.

\newmdtheoremenv[
linewidth= 1pt,linecolor= blue,%
leftmargin=20,rightmargin=20,innertopmargin=0pt, innerrightmargin=40,%
tikzsetting = { draw=lightgray, line width = 0.3pt,dashed,%
dash pattern = on 15pt off 3pt},%
splittopskip=\topskip,skipbelow=\baselineskip,%
skipabove=\baselineskip,ntheorem,roundcorner=0pt,
% backgroundcolor=pagebg,font=\color{orange}\sffamily, fontcolor=white
]{examplebox}{Exemple}[section]



\newcommand\R{\mathbb{R}}
\newcommand\Z{\mathbb{Z}}
\newcommand\N{\mathbb{N}}
\newcommand\E{\mathbb{E}}
\newcommand\F{\mathcal{F}}
\newcommand\cH{\mathcal{H}}
\newcommand\V{\mathbb{V}}
\newcommand\dmo{ ^{-1} }
\newcommand\kapa{\kappa}
\newcommand\im{Im}
\newcommand\hs{\mathcal{H}}





\usepackage{soul}

\makeatletter
\newcommand*{\whiten}[1]{\llap{\textcolor{white}{{\the\SOUL@token}}\hspace{#1pt}}}
\DeclareRobustCommand*\myul{%
    \def\SOUL@everyspace{\underline{\space}\kern\z@}%
    \def\SOUL@everytoken{%
     \setbox0=\hbox{\the\SOUL@token}%
     \ifdim\dp0>\z@
        \raisebox{\dp0}{\underline{\phantom{\the\SOUL@token}}}%
        \whiten{1}\whiten{0}%
        \whiten{-1}\whiten{-2}%
        \llap{\the\SOUL@token}%
     \else
        \underline{\the\SOUL@token}%
     \fi}%
\SOUL@}
\makeatother

\newcommand*{\demp}{\fontfamily{lmtt}\selectfont}

\DeclareTextFontCommand{\textdemp}{\demp}

\begin{document}

\ifcomment
Multiline
comment
\fi
\ifcomment
\myul{Typesetting test}
% \color[rgb]{1,1,1}
$∑_i^n≠ 60º±∞π∆¬≈√j∫h≤≥µ$

$\CR \R\pro\ind\pro\gS\pro
\mqty[a&b\\c&d]$
$\pro\mathbb{P}$
$\dd{x}$

  \[
    \alpha(x)=\left\{
                \begin{array}{ll}
                  x\\
                  \frac{1}{1+e^{-kx}}\\
                  \frac{e^x-e^{-x}}{e^x+e^{-x}}
                \end{array}
              \right.
  \]

  $\expval{x}$
  
  $\chi_\rho(ghg\dmo)=\Tr(\rho_{ghg\dmo})=\Tr(\rho_g\circ\rho_h\circ\rho\dmo_g)=\Tr(\rho_h)\overset{\mbox{\scalebox{0.5}{$\Tr(AB)=\Tr(BA)$}}}{=}\chi_\rho(h)$
  	$\mathop{\oplus}_{\substack{x\in X}}$

$\mat(\rho_g)=(a_{ij}(g))_{\scriptsize \substack{1\leq i\leq d \\ 1\leq j\leq d}}$ et $\mat(\rho'_g)=(a'_{ij}(g))_{\scriptsize \substack{1\leq i'\leq d' \\ 1\leq j'\leq d'}}$



\[\int_a^b{\mathbb{R}^2}g(u, v)\dd{P_{XY}}(u, v)=\iint g(u,v) f_{XY}(u, v)\dd \lambda(u) \dd \lambda(v)\]
$$\lim_{x\to\infty} f(x)$$	
$$\iiiint_V \mu(t,u,v,w) \,dt\,du\,dv\,dw$$
$$\sum_{n=1}^{\infty} 2^{-n} = 1$$	
\begin{definition}
	Si $X$ et $Y$ sont 2 v.a. ou definit la \textsc{Covariance} entre $X$ et $Y$ comme
	$\cov(X,Y)\overset{\text{def}}{=}\E\left[(X-\E(X))(Y-\E(Y))\right]=\E(XY)-\E(X)\E(Y)$.
\end{definition}
\fi
\pagebreak

% \tableofcontents

% insert your code here
%\input{./algebra/main.tex}
%\input{./geometrie-differentielle/main.tex}
%\input{./probabilite/main.tex}
%\input{./analyse-fonctionnelle/main.tex}
% \input{./Analyse-convexe-et-dualite-en-optimisation/main.tex}
%\input{./tikz/main.tex}
%\input{./Theorie-du-distributions/main.tex}
%\input{./optimisation/mine.tex}
 \input{./modelisation/main.tex}

% yves.aubry@univ-tln.fr : algebra

\end{document}

%% !TEX encoding = UTF-8 Unicode
% !TEX TS-program = xelatex

\documentclass[french]{report}

%\usepackage[utf8]{inputenc}
%\usepackage[T1]{fontenc}
\usepackage{babel}


\newif\ifcomment
%\commenttrue # Show comments

\usepackage{physics}
\usepackage{amssymb}


\usepackage{amsthm}
% \usepackage{thmtools}
\usepackage{mathtools}
\usepackage{amsfonts}

\usepackage{color}

\usepackage{tikz}

\usepackage{geometry}
\geometry{a5paper, margin=0.1in, right=1cm}

\usepackage{dsfont}

\usepackage{graphicx}
\graphicspath{ {images/} }

\usepackage{faktor}

\usepackage{IEEEtrantools}
\usepackage{enumerate}   
\usepackage[PostScript=dvips]{"/Users/aware/Documents/Courses/diagrams"}


\newtheorem{theorem}{Théorème}[section]
\renewcommand{\thetheorem}{\arabic{theorem}}
\newtheorem{lemme}{Lemme}[section]
\renewcommand{\thelemme}{\arabic{lemme}}
\newtheorem{proposition}{Proposition}[section]
\renewcommand{\theproposition}{\arabic{proposition}}
\newtheorem{notations}{Notations}[section]
\newtheorem{problem}{Problème}[section]
\newtheorem{corollary}{Corollaire}[theorem]
\renewcommand{\thecorollary}{\arabic{corollary}}
\newtheorem{property}{Propriété}[section]
\newtheorem{objective}{Objectif}[section]

\theoremstyle{definition}
\newtheorem{definition}{Définition}[section]
\renewcommand{\thedefinition}{\arabic{definition}}
\newtheorem{exercise}{Exercice}[chapter]
\renewcommand{\theexercise}{\arabic{exercise}}
\newtheorem{example}{Exemple}[chapter]
\renewcommand{\theexample}{\arabic{example}}
\newtheorem*{solution}{Solution}
\newtheorem*{application}{Application}
\newtheorem*{notation}{Notation}
\newtheorem*{vocabulary}{Vocabulaire}
\newtheorem*{properties}{Propriétés}



\theoremstyle{remark}
\newtheorem*{remark}{Remarque}
\newtheorem*{rappel}{Rappel}


\usepackage{etoolbox}
\AtBeginEnvironment{exercise}{\small}
\AtBeginEnvironment{example}{\small}

\usepackage{cases}
\usepackage[red]{mypack}

\usepackage[framemethod=TikZ]{mdframed}

\definecolor{bg}{rgb}{0.4,0.25,0.95}
\definecolor{pagebg}{rgb}{0,0,0.5}
\surroundwithmdframed[
   topline=false,
   rightline=false,
   bottomline=false,
   leftmargin=\parindent,
   skipabove=8pt,
   skipbelow=8pt,
   linecolor=blue,
   innerbottommargin=10pt,
   % backgroundcolor=bg,font=\color{orange}\sffamily, fontcolor=white
]{definition}

\usepackage{empheq}
\usepackage[most]{tcolorbox}

\newtcbox{\mymath}[1][]{%
    nobeforeafter, math upper, tcbox raise base,
    enhanced, colframe=blue!30!black,
    colback=red!10, boxrule=1pt,
    #1}

\usepackage{unixode}


\DeclareMathOperator{\ord}{ord}
\DeclareMathOperator{\orb}{orb}
\DeclareMathOperator{\stab}{stab}
\DeclareMathOperator{\Stab}{stab}
\DeclareMathOperator{\ppcm}{ppcm}
\DeclareMathOperator{\conj}{Conj}
\DeclareMathOperator{\End}{End}
\DeclareMathOperator{\rot}{rot}
\DeclareMathOperator{\trs}{trace}
\DeclareMathOperator{\Ind}{Ind}
\DeclareMathOperator{\mat}{Mat}
\DeclareMathOperator{\id}{Id}
\DeclareMathOperator{\vect}{vect}
\DeclareMathOperator{\img}{img}
\DeclareMathOperator{\cov}{Cov}
\DeclareMathOperator{\dist}{dist}
\DeclareMathOperator{\irr}{Irr}
\DeclareMathOperator{\image}{Im}
\DeclareMathOperator{\pd}{\partial}
\DeclareMathOperator{\epi}{epi}
\DeclareMathOperator{\Argmin}{Argmin}
\DeclareMathOperator{\dom}{dom}
\DeclareMathOperator{\proj}{proj}
\DeclareMathOperator{\ctg}{ctg}
\DeclareMathOperator{\supp}{supp}
\DeclareMathOperator{\argmin}{argmin}
\DeclareMathOperator{\mult}{mult}
\DeclareMathOperator{\ch}{ch}
\DeclareMathOperator{\sh}{sh}
\DeclareMathOperator{\rang}{rang}
\DeclareMathOperator{\diam}{diam}
\DeclareMathOperator{\Epigraphe}{Epigraphe}




\usepackage{xcolor}
\everymath{\color{blue}}
%\everymath{\color[rgb]{0,1,1}}
%\pagecolor[rgb]{0,0,0.5}


\newcommand*{\pdtest}[3][]{\ensuremath{\frac{\partial^{#1} #2}{\partial #3}}}

\newcommand*{\deffunc}[6][]{\ensuremath{
\begin{array}{rcl}
#2 : #3 &\rightarrow& #4\\
#5 &\mapsto& #6
\end{array}
}}

\newcommand{\eqcolon}{\mathrel{\resizebox{\widthof{$\mathord{=}$}}{\height}{ $\!\!=\!\!\resizebox{1.2\width}{0.8\height}{\raisebox{0.23ex}{$\mathop{:}$}}\!\!$ }}}
\newcommand{\coloneq}{\mathrel{\resizebox{\widthof{$\mathord{=}$}}{\height}{ $\!\!\resizebox{1.2\width}{0.8\height}{\raisebox{0.23ex}{$\mathop{:}$}}\!\!=\!\!$ }}}
\newcommand{\eqcolonl}{\ensuremath{\mathrel{=\!\!\mathop{:}}}}
\newcommand{\coloneql}{\ensuremath{\mathrel{\mathop{:} \!\! =}}}
\newcommand{\vc}[1]{% inline column vector
  \left(\begin{smallmatrix}#1\end{smallmatrix}\right)%
}
\newcommand{\vr}[1]{% inline row vector
  \begin{smallmatrix}(\,#1\,)\end{smallmatrix}%
}
\makeatletter
\newcommand*{\defeq}{\ =\mathrel{\rlap{%
                     \raisebox{0.3ex}{$\m@th\cdot$}}%
                     \raisebox{-0.3ex}{$\m@th\cdot$}}%
                     }
\makeatother

\newcommand{\mathcircle}[1]{% inline row vector
 \overset{\circ}{#1}
}
\newcommand{\ulim}{% low limit
 \underline{\lim}
}
\newcommand{\ssi}{% iff
\iff
}
\newcommand{\ps}[2]{
\expval{#1 | #2}
}
\newcommand{\df}[1]{
\mqty{#1}
}
\newcommand{\n}[1]{
\norm{#1}
}
\newcommand{\sys}[1]{
\left\{\smqty{#1}\right.
}


\newcommand{\eqdef}{\ensuremath{\overset{\text{def}}=}}


\def\Circlearrowright{\ensuremath{%
  \rotatebox[origin=c]{230}{$\circlearrowright$}}}

\newcommand\ct[1]{\text{\rmfamily\upshape #1}}
\newcommand\question[1]{ {\color{red} ...!? \small #1}}
\newcommand\caz[1]{\left\{\begin{array} #1 \end{array}\right.}
\newcommand\const{\text{\rmfamily\upshape const}}
\newcommand\toP{ \overset{\pro}{\to}}
\newcommand\toPP{ \overset{\text{PP}}{\to}}
\newcommand{\oeq}{\mathrel{\text{\textcircled{$=$}}}}





\usepackage{xcolor}
% \usepackage[normalem]{ulem}
\usepackage{lipsum}
\makeatletter
% \newcommand\colorwave[1][blue]{\bgroup \markoverwith{\lower3.5\p@\hbox{\sixly \textcolor{#1}{\char58}}}\ULon}
%\font\sixly=lasy6 % does not re-load if already loaded, so no memory problem.

\newmdtheoremenv[
linewidth= 1pt,linecolor= blue,%
leftmargin=20,rightmargin=20,innertopmargin=0pt, innerrightmargin=40,%
tikzsetting = { draw=lightgray, line width = 0.3pt,dashed,%
dash pattern = on 15pt off 3pt},%
splittopskip=\topskip,skipbelow=\baselineskip,%
skipabove=\baselineskip,ntheorem,roundcorner=0pt,
% backgroundcolor=pagebg,font=\color{orange}\sffamily, fontcolor=white
]{examplebox}{Exemple}[section]



\newcommand\R{\mathbb{R}}
\newcommand\Z{\mathbb{Z}}
\newcommand\N{\mathbb{N}}
\newcommand\E{\mathbb{E}}
\newcommand\F{\mathcal{F}}
\newcommand\cH{\mathcal{H}}
\newcommand\V{\mathbb{V}}
\newcommand\dmo{ ^{-1} }
\newcommand\kapa{\kappa}
\newcommand\im{Im}
\newcommand\hs{\mathcal{H}}





\usepackage{soul}

\makeatletter
\newcommand*{\whiten}[1]{\llap{\textcolor{white}{{\the\SOUL@token}}\hspace{#1pt}}}
\DeclareRobustCommand*\myul{%
    \def\SOUL@everyspace{\underline{\space}\kern\z@}%
    \def\SOUL@everytoken{%
     \setbox0=\hbox{\the\SOUL@token}%
     \ifdim\dp0>\z@
        \raisebox{\dp0}{\underline{\phantom{\the\SOUL@token}}}%
        \whiten{1}\whiten{0}%
        \whiten{-1}\whiten{-2}%
        \llap{\the\SOUL@token}%
     \else
        \underline{\the\SOUL@token}%
     \fi}%
\SOUL@}
\makeatother

\newcommand*{\demp}{\fontfamily{lmtt}\selectfont}

\DeclareTextFontCommand{\textdemp}{\demp}

\begin{document}

\ifcomment
Multiline
comment
\fi
\ifcomment
\myul{Typesetting test}
% \color[rgb]{1,1,1}
$∑_i^n≠ 60º±∞π∆¬≈√j∫h≤≥µ$

$\CR \R\pro\ind\pro\gS\pro
\mqty[a&b\\c&d]$
$\pro\mathbb{P}$
$\dd{x}$

  \[
    \alpha(x)=\left\{
                \begin{array}{ll}
                  x\\
                  \frac{1}{1+e^{-kx}}\\
                  \frac{e^x-e^{-x}}{e^x+e^{-x}}
                \end{array}
              \right.
  \]

  $\expval{x}$
  
  $\chi_\rho(ghg\dmo)=\Tr(\rho_{ghg\dmo})=\Tr(\rho_g\circ\rho_h\circ\rho\dmo_g)=\Tr(\rho_h)\overset{\mbox{\scalebox{0.5}{$\Tr(AB)=\Tr(BA)$}}}{=}\chi_\rho(h)$
  	$\mathop{\oplus}_{\substack{x\in X}}$

$\mat(\rho_g)=(a_{ij}(g))_{\scriptsize \substack{1\leq i\leq d \\ 1\leq j\leq d}}$ et $\mat(\rho'_g)=(a'_{ij}(g))_{\scriptsize \substack{1\leq i'\leq d' \\ 1\leq j'\leq d'}}$



\[\int_a^b{\mathbb{R}^2}g(u, v)\dd{P_{XY}}(u, v)=\iint g(u,v) f_{XY}(u, v)\dd \lambda(u) \dd \lambda(v)\]
$$\lim_{x\to\infty} f(x)$$	
$$\iiiint_V \mu(t,u,v,w) \,dt\,du\,dv\,dw$$
$$\sum_{n=1}^{\infty} 2^{-n} = 1$$	
\begin{definition}
	Si $X$ et $Y$ sont 2 v.a. ou definit la \textsc{Covariance} entre $X$ et $Y$ comme
	$\cov(X,Y)\overset{\text{def}}{=}\E\left[(X-\E(X))(Y-\E(Y))\right]=\E(XY)-\E(X)\E(Y)$.
\end{definition}
\fi
\pagebreak

% \tableofcontents

% insert your code here
%\input{./algebra/main.tex}
%\input{./geometrie-differentielle/main.tex}
%\input{./probabilite/main.tex}
%\input{./analyse-fonctionnelle/main.tex}
% \input{./Analyse-convexe-et-dualite-en-optimisation/main.tex}
%\input{./tikz/main.tex}
%\input{./Theorie-du-distributions/main.tex}
%\input{./optimisation/mine.tex}
 \input{./modelisation/main.tex}

% yves.aubry@univ-tln.fr : algebra

\end{document}

%% !TEX encoding = UTF-8 Unicode
% !TEX TS-program = xelatex

\documentclass[french]{report}

%\usepackage[utf8]{inputenc}
%\usepackage[T1]{fontenc}
\usepackage{babel}


\newif\ifcomment
%\commenttrue # Show comments

\usepackage{physics}
\usepackage{amssymb}


\usepackage{amsthm}
% \usepackage{thmtools}
\usepackage{mathtools}
\usepackage{amsfonts}

\usepackage{color}

\usepackage{tikz}

\usepackage{geometry}
\geometry{a5paper, margin=0.1in, right=1cm}

\usepackage{dsfont}

\usepackage{graphicx}
\graphicspath{ {images/} }

\usepackage{faktor}

\usepackage{IEEEtrantools}
\usepackage{enumerate}   
\usepackage[PostScript=dvips]{"/Users/aware/Documents/Courses/diagrams"}


\newtheorem{theorem}{Théorème}[section]
\renewcommand{\thetheorem}{\arabic{theorem}}
\newtheorem{lemme}{Lemme}[section]
\renewcommand{\thelemme}{\arabic{lemme}}
\newtheorem{proposition}{Proposition}[section]
\renewcommand{\theproposition}{\arabic{proposition}}
\newtheorem{notations}{Notations}[section]
\newtheorem{problem}{Problème}[section]
\newtheorem{corollary}{Corollaire}[theorem]
\renewcommand{\thecorollary}{\arabic{corollary}}
\newtheorem{property}{Propriété}[section]
\newtheorem{objective}{Objectif}[section]

\theoremstyle{definition}
\newtheorem{definition}{Définition}[section]
\renewcommand{\thedefinition}{\arabic{definition}}
\newtheorem{exercise}{Exercice}[chapter]
\renewcommand{\theexercise}{\arabic{exercise}}
\newtheorem{example}{Exemple}[chapter]
\renewcommand{\theexample}{\arabic{example}}
\newtheorem*{solution}{Solution}
\newtheorem*{application}{Application}
\newtheorem*{notation}{Notation}
\newtheorem*{vocabulary}{Vocabulaire}
\newtheorem*{properties}{Propriétés}



\theoremstyle{remark}
\newtheorem*{remark}{Remarque}
\newtheorem*{rappel}{Rappel}


\usepackage{etoolbox}
\AtBeginEnvironment{exercise}{\small}
\AtBeginEnvironment{example}{\small}

\usepackage{cases}
\usepackage[red]{mypack}

\usepackage[framemethod=TikZ]{mdframed}

\definecolor{bg}{rgb}{0.4,0.25,0.95}
\definecolor{pagebg}{rgb}{0,0,0.5}
\surroundwithmdframed[
   topline=false,
   rightline=false,
   bottomline=false,
   leftmargin=\parindent,
   skipabove=8pt,
   skipbelow=8pt,
   linecolor=blue,
   innerbottommargin=10pt,
   % backgroundcolor=bg,font=\color{orange}\sffamily, fontcolor=white
]{definition}

\usepackage{empheq}
\usepackage[most]{tcolorbox}

\newtcbox{\mymath}[1][]{%
    nobeforeafter, math upper, tcbox raise base,
    enhanced, colframe=blue!30!black,
    colback=red!10, boxrule=1pt,
    #1}

\usepackage{unixode}


\DeclareMathOperator{\ord}{ord}
\DeclareMathOperator{\orb}{orb}
\DeclareMathOperator{\stab}{stab}
\DeclareMathOperator{\Stab}{stab}
\DeclareMathOperator{\ppcm}{ppcm}
\DeclareMathOperator{\conj}{Conj}
\DeclareMathOperator{\End}{End}
\DeclareMathOperator{\rot}{rot}
\DeclareMathOperator{\trs}{trace}
\DeclareMathOperator{\Ind}{Ind}
\DeclareMathOperator{\mat}{Mat}
\DeclareMathOperator{\id}{Id}
\DeclareMathOperator{\vect}{vect}
\DeclareMathOperator{\img}{img}
\DeclareMathOperator{\cov}{Cov}
\DeclareMathOperator{\dist}{dist}
\DeclareMathOperator{\irr}{Irr}
\DeclareMathOperator{\image}{Im}
\DeclareMathOperator{\pd}{\partial}
\DeclareMathOperator{\epi}{epi}
\DeclareMathOperator{\Argmin}{Argmin}
\DeclareMathOperator{\dom}{dom}
\DeclareMathOperator{\proj}{proj}
\DeclareMathOperator{\ctg}{ctg}
\DeclareMathOperator{\supp}{supp}
\DeclareMathOperator{\argmin}{argmin}
\DeclareMathOperator{\mult}{mult}
\DeclareMathOperator{\ch}{ch}
\DeclareMathOperator{\sh}{sh}
\DeclareMathOperator{\rang}{rang}
\DeclareMathOperator{\diam}{diam}
\DeclareMathOperator{\Epigraphe}{Epigraphe}




\usepackage{xcolor}
\everymath{\color{blue}}
%\everymath{\color[rgb]{0,1,1}}
%\pagecolor[rgb]{0,0,0.5}


\newcommand*{\pdtest}[3][]{\ensuremath{\frac{\partial^{#1} #2}{\partial #3}}}

\newcommand*{\deffunc}[6][]{\ensuremath{
\begin{array}{rcl}
#2 : #3 &\rightarrow& #4\\
#5 &\mapsto& #6
\end{array}
}}

\newcommand{\eqcolon}{\mathrel{\resizebox{\widthof{$\mathord{=}$}}{\height}{ $\!\!=\!\!\resizebox{1.2\width}{0.8\height}{\raisebox{0.23ex}{$\mathop{:}$}}\!\!$ }}}
\newcommand{\coloneq}{\mathrel{\resizebox{\widthof{$\mathord{=}$}}{\height}{ $\!\!\resizebox{1.2\width}{0.8\height}{\raisebox{0.23ex}{$\mathop{:}$}}\!\!=\!\!$ }}}
\newcommand{\eqcolonl}{\ensuremath{\mathrel{=\!\!\mathop{:}}}}
\newcommand{\coloneql}{\ensuremath{\mathrel{\mathop{:} \!\! =}}}
\newcommand{\vc}[1]{% inline column vector
  \left(\begin{smallmatrix}#1\end{smallmatrix}\right)%
}
\newcommand{\vr}[1]{% inline row vector
  \begin{smallmatrix}(\,#1\,)\end{smallmatrix}%
}
\makeatletter
\newcommand*{\defeq}{\ =\mathrel{\rlap{%
                     \raisebox{0.3ex}{$\m@th\cdot$}}%
                     \raisebox{-0.3ex}{$\m@th\cdot$}}%
                     }
\makeatother

\newcommand{\mathcircle}[1]{% inline row vector
 \overset{\circ}{#1}
}
\newcommand{\ulim}{% low limit
 \underline{\lim}
}
\newcommand{\ssi}{% iff
\iff
}
\newcommand{\ps}[2]{
\expval{#1 | #2}
}
\newcommand{\df}[1]{
\mqty{#1}
}
\newcommand{\n}[1]{
\norm{#1}
}
\newcommand{\sys}[1]{
\left\{\smqty{#1}\right.
}


\newcommand{\eqdef}{\ensuremath{\overset{\text{def}}=}}


\def\Circlearrowright{\ensuremath{%
  \rotatebox[origin=c]{230}{$\circlearrowright$}}}

\newcommand\ct[1]{\text{\rmfamily\upshape #1}}
\newcommand\question[1]{ {\color{red} ...!? \small #1}}
\newcommand\caz[1]{\left\{\begin{array} #1 \end{array}\right.}
\newcommand\const{\text{\rmfamily\upshape const}}
\newcommand\toP{ \overset{\pro}{\to}}
\newcommand\toPP{ \overset{\text{PP}}{\to}}
\newcommand{\oeq}{\mathrel{\text{\textcircled{$=$}}}}





\usepackage{xcolor}
% \usepackage[normalem]{ulem}
\usepackage{lipsum}
\makeatletter
% \newcommand\colorwave[1][blue]{\bgroup \markoverwith{\lower3.5\p@\hbox{\sixly \textcolor{#1}{\char58}}}\ULon}
%\font\sixly=lasy6 % does not re-load if already loaded, so no memory problem.

\newmdtheoremenv[
linewidth= 1pt,linecolor= blue,%
leftmargin=20,rightmargin=20,innertopmargin=0pt, innerrightmargin=40,%
tikzsetting = { draw=lightgray, line width = 0.3pt,dashed,%
dash pattern = on 15pt off 3pt},%
splittopskip=\topskip,skipbelow=\baselineskip,%
skipabove=\baselineskip,ntheorem,roundcorner=0pt,
% backgroundcolor=pagebg,font=\color{orange}\sffamily, fontcolor=white
]{examplebox}{Exemple}[section]



\newcommand\R{\mathbb{R}}
\newcommand\Z{\mathbb{Z}}
\newcommand\N{\mathbb{N}}
\newcommand\E{\mathbb{E}}
\newcommand\F{\mathcal{F}}
\newcommand\cH{\mathcal{H}}
\newcommand\V{\mathbb{V}}
\newcommand\dmo{ ^{-1} }
\newcommand\kapa{\kappa}
\newcommand\im{Im}
\newcommand\hs{\mathcal{H}}





\usepackage{soul}

\makeatletter
\newcommand*{\whiten}[1]{\llap{\textcolor{white}{{\the\SOUL@token}}\hspace{#1pt}}}
\DeclareRobustCommand*\myul{%
    \def\SOUL@everyspace{\underline{\space}\kern\z@}%
    \def\SOUL@everytoken{%
     \setbox0=\hbox{\the\SOUL@token}%
     \ifdim\dp0>\z@
        \raisebox{\dp0}{\underline{\phantom{\the\SOUL@token}}}%
        \whiten{1}\whiten{0}%
        \whiten{-1}\whiten{-2}%
        \llap{\the\SOUL@token}%
     \else
        \underline{\the\SOUL@token}%
     \fi}%
\SOUL@}
\makeatother

\newcommand*{\demp}{\fontfamily{lmtt}\selectfont}

\DeclareTextFontCommand{\textdemp}{\demp}

\begin{document}

\ifcomment
Multiline
comment
\fi
\ifcomment
\myul{Typesetting test}
% \color[rgb]{1,1,1}
$∑_i^n≠ 60º±∞π∆¬≈√j∫h≤≥µ$

$\CR \R\pro\ind\pro\gS\pro
\mqty[a&b\\c&d]$
$\pro\mathbb{P}$
$\dd{x}$

  \[
    \alpha(x)=\left\{
                \begin{array}{ll}
                  x\\
                  \frac{1}{1+e^{-kx}}\\
                  \frac{e^x-e^{-x}}{e^x+e^{-x}}
                \end{array}
              \right.
  \]

  $\expval{x}$
  
  $\chi_\rho(ghg\dmo)=\Tr(\rho_{ghg\dmo})=\Tr(\rho_g\circ\rho_h\circ\rho\dmo_g)=\Tr(\rho_h)\overset{\mbox{\scalebox{0.5}{$\Tr(AB)=\Tr(BA)$}}}{=}\chi_\rho(h)$
  	$\mathop{\oplus}_{\substack{x\in X}}$

$\mat(\rho_g)=(a_{ij}(g))_{\scriptsize \substack{1\leq i\leq d \\ 1\leq j\leq d}}$ et $\mat(\rho'_g)=(a'_{ij}(g))_{\scriptsize \substack{1\leq i'\leq d' \\ 1\leq j'\leq d'}}$



\[\int_a^b{\mathbb{R}^2}g(u, v)\dd{P_{XY}}(u, v)=\iint g(u,v) f_{XY}(u, v)\dd \lambda(u) \dd \lambda(v)\]
$$\lim_{x\to\infty} f(x)$$	
$$\iiiint_V \mu(t,u,v,w) \,dt\,du\,dv\,dw$$
$$\sum_{n=1}^{\infty} 2^{-n} = 1$$	
\begin{definition}
	Si $X$ et $Y$ sont 2 v.a. ou definit la \textsc{Covariance} entre $X$ et $Y$ comme
	$\cov(X,Y)\overset{\text{def}}{=}\E\left[(X-\E(X))(Y-\E(Y))\right]=\E(XY)-\E(X)\E(Y)$.
\end{definition}
\fi
\pagebreak

% \tableofcontents

% insert your code here
%\input{./algebra/main.tex}
%\input{./geometrie-differentielle/main.tex}
%\input{./probabilite/main.tex}
%\input{./analyse-fonctionnelle/main.tex}
% \input{./Analyse-convexe-et-dualite-en-optimisation/main.tex}
%\input{./tikz/main.tex}
%\input{./Theorie-du-distributions/main.tex}
%\input{./optimisation/mine.tex}
 \input{./modelisation/main.tex}

% yves.aubry@univ-tln.fr : algebra

\end{document}

% % !TEX encoding = UTF-8 Unicode
% !TEX TS-program = xelatex

\documentclass[french]{report}

%\usepackage[utf8]{inputenc}
%\usepackage[T1]{fontenc}
\usepackage{babel}


\newif\ifcomment
%\commenttrue # Show comments

\usepackage{physics}
\usepackage{amssymb}


\usepackage{amsthm}
% \usepackage{thmtools}
\usepackage{mathtools}
\usepackage{amsfonts}

\usepackage{color}

\usepackage{tikz}

\usepackage{geometry}
\geometry{a5paper, margin=0.1in, right=1cm}

\usepackage{dsfont}

\usepackage{graphicx}
\graphicspath{ {images/} }

\usepackage{faktor}

\usepackage{IEEEtrantools}
\usepackage{enumerate}   
\usepackage[PostScript=dvips]{"/Users/aware/Documents/Courses/diagrams"}


\newtheorem{theorem}{Théorème}[section]
\renewcommand{\thetheorem}{\arabic{theorem}}
\newtheorem{lemme}{Lemme}[section]
\renewcommand{\thelemme}{\arabic{lemme}}
\newtheorem{proposition}{Proposition}[section]
\renewcommand{\theproposition}{\arabic{proposition}}
\newtheorem{notations}{Notations}[section]
\newtheorem{problem}{Problème}[section]
\newtheorem{corollary}{Corollaire}[theorem]
\renewcommand{\thecorollary}{\arabic{corollary}}
\newtheorem{property}{Propriété}[section]
\newtheorem{objective}{Objectif}[section]

\theoremstyle{definition}
\newtheorem{definition}{Définition}[section]
\renewcommand{\thedefinition}{\arabic{definition}}
\newtheorem{exercise}{Exercice}[chapter]
\renewcommand{\theexercise}{\arabic{exercise}}
\newtheorem{example}{Exemple}[chapter]
\renewcommand{\theexample}{\arabic{example}}
\newtheorem*{solution}{Solution}
\newtheorem*{application}{Application}
\newtheorem*{notation}{Notation}
\newtheorem*{vocabulary}{Vocabulaire}
\newtheorem*{properties}{Propriétés}



\theoremstyle{remark}
\newtheorem*{remark}{Remarque}
\newtheorem*{rappel}{Rappel}


\usepackage{etoolbox}
\AtBeginEnvironment{exercise}{\small}
\AtBeginEnvironment{example}{\small}

\usepackage{cases}
\usepackage[red]{mypack}

\usepackage[framemethod=TikZ]{mdframed}

\definecolor{bg}{rgb}{0.4,0.25,0.95}
\definecolor{pagebg}{rgb}{0,0,0.5}
\surroundwithmdframed[
   topline=false,
   rightline=false,
   bottomline=false,
   leftmargin=\parindent,
   skipabove=8pt,
   skipbelow=8pt,
   linecolor=blue,
   innerbottommargin=10pt,
   % backgroundcolor=bg,font=\color{orange}\sffamily, fontcolor=white
]{definition}

\usepackage{empheq}
\usepackage[most]{tcolorbox}

\newtcbox{\mymath}[1][]{%
    nobeforeafter, math upper, tcbox raise base,
    enhanced, colframe=blue!30!black,
    colback=red!10, boxrule=1pt,
    #1}

\usepackage{unixode}


\DeclareMathOperator{\ord}{ord}
\DeclareMathOperator{\orb}{orb}
\DeclareMathOperator{\stab}{stab}
\DeclareMathOperator{\Stab}{stab}
\DeclareMathOperator{\ppcm}{ppcm}
\DeclareMathOperator{\conj}{Conj}
\DeclareMathOperator{\End}{End}
\DeclareMathOperator{\rot}{rot}
\DeclareMathOperator{\trs}{trace}
\DeclareMathOperator{\Ind}{Ind}
\DeclareMathOperator{\mat}{Mat}
\DeclareMathOperator{\id}{Id}
\DeclareMathOperator{\vect}{vect}
\DeclareMathOperator{\img}{img}
\DeclareMathOperator{\cov}{Cov}
\DeclareMathOperator{\dist}{dist}
\DeclareMathOperator{\irr}{Irr}
\DeclareMathOperator{\image}{Im}
\DeclareMathOperator{\pd}{\partial}
\DeclareMathOperator{\epi}{epi}
\DeclareMathOperator{\Argmin}{Argmin}
\DeclareMathOperator{\dom}{dom}
\DeclareMathOperator{\proj}{proj}
\DeclareMathOperator{\ctg}{ctg}
\DeclareMathOperator{\supp}{supp}
\DeclareMathOperator{\argmin}{argmin}
\DeclareMathOperator{\mult}{mult}
\DeclareMathOperator{\ch}{ch}
\DeclareMathOperator{\sh}{sh}
\DeclareMathOperator{\rang}{rang}
\DeclareMathOperator{\diam}{diam}
\DeclareMathOperator{\Epigraphe}{Epigraphe}




\usepackage{xcolor}
\everymath{\color{blue}}
%\everymath{\color[rgb]{0,1,1}}
%\pagecolor[rgb]{0,0,0.5}


\newcommand*{\pdtest}[3][]{\ensuremath{\frac{\partial^{#1} #2}{\partial #3}}}

\newcommand*{\deffunc}[6][]{\ensuremath{
\begin{array}{rcl}
#2 : #3 &\rightarrow& #4\\
#5 &\mapsto& #6
\end{array}
}}

\newcommand{\eqcolon}{\mathrel{\resizebox{\widthof{$\mathord{=}$}}{\height}{ $\!\!=\!\!\resizebox{1.2\width}{0.8\height}{\raisebox{0.23ex}{$\mathop{:}$}}\!\!$ }}}
\newcommand{\coloneq}{\mathrel{\resizebox{\widthof{$\mathord{=}$}}{\height}{ $\!\!\resizebox{1.2\width}{0.8\height}{\raisebox{0.23ex}{$\mathop{:}$}}\!\!=\!\!$ }}}
\newcommand{\eqcolonl}{\ensuremath{\mathrel{=\!\!\mathop{:}}}}
\newcommand{\coloneql}{\ensuremath{\mathrel{\mathop{:} \!\! =}}}
\newcommand{\vc}[1]{% inline column vector
  \left(\begin{smallmatrix}#1\end{smallmatrix}\right)%
}
\newcommand{\vr}[1]{% inline row vector
  \begin{smallmatrix}(\,#1\,)\end{smallmatrix}%
}
\makeatletter
\newcommand*{\defeq}{\ =\mathrel{\rlap{%
                     \raisebox{0.3ex}{$\m@th\cdot$}}%
                     \raisebox{-0.3ex}{$\m@th\cdot$}}%
                     }
\makeatother

\newcommand{\mathcircle}[1]{% inline row vector
 \overset{\circ}{#1}
}
\newcommand{\ulim}{% low limit
 \underline{\lim}
}
\newcommand{\ssi}{% iff
\iff
}
\newcommand{\ps}[2]{
\expval{#1 | #2}
}
\newcommand{\df}[1]{
\mqty{#1}
}
\newcommand{\n}[1]{
\norm{#1}
}
\newcommand{\sys}[1]{
\left\{\smqty{#1}\right.
}


\newcommand{\eqdef}{\ensuremath{\overset{\text{def}}=}}


\def\Circlearrowright{\ensuremath{%
  \rotatebox[origin=c]{230}{$\circlearrowright$}}}

\newcommand\ct[1]{\text{\rmfamily\upshape #1}}
\newcommand\question[1]{ {\color{red} ...!? \small #1}}
\newcommand\caz[1]{\left\{\begin{array} #1 \end{array}\right.}
\newcommand\const{\text{\rmfamily\upshape const}}
\newcommand\toP{ \overset{\pro}{\to}}
\newcommand\toPP{ \overset{\text{PP}}{\to}}
\newcommand{\oeq}{\mathrel{\text{\textcircled{$=$}}}}





\usepackage{xcolor}
% \usepackage[normalem]{ulem}
\usepackage{lipsum}
\makeatletter
% \newcommand\colorwave[1][blue]{\bgroup \markoverwith{\lower3.5\p@\hbox{\sixly \textcolor{#1}{\char58}}}\ULon}
%\font\sixly=lasy6 % does not re-load if already loaded, so no memory problem.

\newmdtheoremenv[
linewidth= 1pt,linecolor= blue,%
leftmargin=20,rightmargin=20,innertopmargin=0pt, innerrightmargin=40,%
tikzsetting = { draw=lightgray, line width = 0.3pt,dashed,%
dash pattern = on 15pt off 3pt},%
splittopskip=\topskip,skipbelow=\baselineskip,%
skipabove=\baselineskip,ntheorem,roundcorner=0pt,
% backgroundcolor=pagebg,font=\color{orange}\sffamily, fontcolor=white
]{examplebox}{Exemple}[section]



\newcommand\R{\mathbb{R}}
\newcommand\Z{\mathbb{Z}}
\newcommand\N{\mathbb{N}}
\newcommand\E{\mathbb{E}}
\newcommand\F{\mathcal{F}}
\newcommand\cH{\mathcal{H}}
\newcommand\V{\mathbb{V}}
\newcommand\dmo{ ^{-1} }
\newcommand\kapa{\kappa}
\newcommand\im{Im}
\newcommand\hs{\mathcal{H}}





\usepackage{soul}

\makeatletter
\newcommand*{\whiten}[1]{\llap{\textcolor{white}{{\the\SOUL@token}}\hspace{#1pt}}}
\DeclareRobustCommand*\myul{%
    \def\SOUL@everyspace{\underline{\space}\kern\z@}%
    \def\SOUL@everytoken{%
     \setbox0=\hbox{\the\SOUL@token}%
     \ifdim\dp0>\z@
        \raisebox{\dp0}{\underline{\phantom{\the\SOUL@token}}}%
        \whiten{1}\whiten{0}%
        \whiten{-1}\whiten{-2}%
        \llap{\the\SOUL@token}%
     \else
        \underline{\the\SOUL@token}%
     \fi}%
\SOUL@}
\makeatother

\newcommand*{\demp}{\fontfamily{lmtt}\selectfont}

\DeclareTextFontCommand{\textdemp}{\demp}

\begin{document}

\ifcomment
Multiline
comment
\fi
\ifcomment
\myul{Typesetting test}
% \color[rgb]{1,1,1}
$∑_i^n≠ 60º±∞π∆¬≈√j∫h≤≥µ$

$\CR \R\pro\ind\pro\gS\pro
\mqty[a&b\\c&d]$
$\pro\mathbb{P}$
$\dd{x}$

  \[
    \alpha(x)=\left\{
                \begin{array}{ll}
                  x\\
                  \frac{1}{1+e^{-kx}}\\
                  \frac{e^x-e^{-x}}{e^x+e^{-x}}
                \end{array}
              \right.
  \]

  $\expval{x}$
  
  $\chi_\rho(ghg\dmo)=\Tr(\rho_{ghg\dmo})=\Tr(\rho_g\circ\rho_h\circ\rho\dmo_g)=\Tr(\rho_h)\overset{\mbox{\scalebox{0.5}{$\Tr(AB)=\Tr(BA)$}}}{=}\chi_\rho(h)$
  	$\mathop{\oplus}_{\substack{x\in X}}$

$\mat(\rho_g)=(a_{ij}(g))_{\scriptsize \substack{1\leq i\leq d \\ 1\leq j\leq d}}$ et $\mat(\rho'_g)=(a'_{ij}(g))_{\scriptsize \substack{1\leq i'\leq d' \\ 1\leq j'\leq d'}}$



\[\int_a^b{\mathbb{R}^2}g(u, v)\dd{P_{XY}}(u, v)=\iint g(u,v) f_{XY}(u, v)\dd \lambda(u) \dd \lambda(v)\]
$$\lim_{x\to\infty} f(x)$$	
$$\iiiint_V \mu(t,u,v,w) \,dt\,du\,dv\,dw$$
$$\sum_{n=1}^{\infty} 2^{-n} = 1$$	
\begin{definition}
	Si $X$ et $Y$ sont 2 v.a. ou definit la \textsc{Covariance} entre $X$ et $Y$ comme
	$\cov(X,Y)\overset{\text{def}}{=}\E\left[(X-\E(X))(Y-\E(Y))\right]=\E(XY)-\E(X)\E(Y)$.
\end{definition}
\fi
\pagebreak

% \tableofcontents

% insert your code here
%\input{./algebra/main.tex}
%\input{./geometrie-differentielle/main.tex}
%\input{./probabilite/main.tex}
%\input{./analyse-fonctionnelle/main.tex}
% \input{./Analyse-convexe-et-dualite-en-optimisation/main.tex}
%\input{./tikz/main.tex}
%\input{./Theorie-du-distributions/main.tex}
%\input{./optimisation/mine.tex}
 \input{./modelisation/main.tex}

% yves.aubry@univ-tln.fr : algebra

\end{document}

%% !TEX encoding = UTF-8 Unicode
% !TEX TS-program = xelatex

\documentclass[french]{report}

%\usepackage[utf8]{inputenc}
%\usepackage[T1]{fontenc}
\usepackage{babel}


\newif\ifcomment
%\commenttrue # Show comments

\usepackage{physics}
\usepackage{amssymb}


\usepackage{amsthm}
% \usepackage{thmtools}
\usepackage{mathtools}
\usepackage{amsfonts}

\usepackage{color}

\usepackage{tikz}

\usepackage{geometry}
\geometry{a5paper, margin=0.1in, right=1cm}

\usepackage{dsfont}

\usepackage{graphicx}
\graphicspath{ {images/} }

\usepackage{faktor}

\usepackage{IEEEtrantools}
\usepackage{enumerate}   
\usepackage[PostScript=dvips]{"/Users/aware/Documents/Courses/diagrams"}


\newtheorem{theorem}{Théorème}[section]
\renewcommand{\thetheorem}{\arabic{theorem}}
\newtheorem{lemme}{Lemme}[section]
\renewcommand{\thelemme}{\arabic{lemme}}
\newtheorem{proposition}{Proposition}[section]
\renewcommand{\theproposition}{\arabic{proposition}}
\newtheorem{notations}{Notations}[section]
\newtheorem{problem}{Problème}[section]
\newtheorem{corollary}{Corollaire}[theorem]
\renewcommand{\thecorollary}{\arabic{corollary}}
\newtheorem{property}{Propriété}[section]
\newtheorem{objective}{Objectif}[section]

\theoremstyle{definition}
\newtheorem{definition}{Définition}[section]
\renewcommand{\thedefinition}{\arabic{definition}}
\newtheorem{exercise}{Exercice}[chapter]
\renewcommand{\theexercise}{\arabic{exercise}}
\newtheorem{example}{Exemple}[chapter]
\renewcommand{\theexample}{\arabic{example}}
\newtheorem*{solution}{Solution}
\newtheorem*{application}{Application}
\newtheorem*{notation}{Notation}
\newtheorem*{vocabulary}{Vocabulaire}
\newtheorem*{properties}{Propriétés}



\theoremstyle{remark}
\newtheorem*{remark}{Remarque}
\newtheorem*{rappel}{Rappel}


\usepackage{etoolbox}
\AtBeginEnvironment{exercise}{\small}
\AtBeginEnvironment{example}{\small}

\usepackage{cases}
\usepackage[red]{mypack}

\usepackage[framemethod=TikZ]{mdframed}

\definecolor{bg}{rgb}{0.4,0.25,0.95}
\definecolor{pagebg}{rgb}{0,0,0.5}
\surroundwithmdframed[
   topline=false,
   rightline=false,
   bottomline=false,
   leftmargin=\parindent,
   skipabove=8pt,
   skipbelow=8pt,
   linecolor=blue,
   innerbottommargin=10pt,
   % backgroundcolor=bg,font=\color{orange}\sffamily, fontcolor=white
]{definition}

\usepackage{empheq}
\usepackage[most]{tcolorbox}

\newtcbox{\mymath}[1][]{%
    nobeforeafter, math upper, tcbox raise base,
    enhanced, colframe=blue!30!black,
    colback=red!10, boxrule=1pt,
    #1}

\usepackage{unixode}


\DeclareMathOperator{\ord}{ord}
\DeclareMathOperator{\orb}{orb}
\DeclareMathOperator{\stab}{stab}
\DeclareMathOperator{\Stab}{stab}
\DeclareMathOperator{\ppcm}{ppcm}
\DeclareMathOperator{\conj}{Conj}
\DeclareMathOperator{\End}{End}
\DeclareMathOperator{\rot}{rot}
\DeclareMathOperator{\trs}{trace}
\DeclareMathOperator{\Ind}{Ind}
\DeclareMathOperator{\mat}{Mat}
\DeclareMathOperator{\id}{Id}
\DeclareMathOperator{\vect}{vect}
\DeclareMathOperator{\img}{img}
\DeclareMathOperator{\cov}{Cov}
\DeclareMathOperator{\dist}{dist}
\DeclareMathOperator{\irr}{Irr}
\DeclareMathOperator{\image}{Im}
\DeclareMathOperator{\pd}{\partial}
\DeclareMathOperator{\epi}{epi}
\DeclareMathOperator{\Argmin}{Argmin}
\DeclareMathOperator{\dom}{dom}
\DeclareMathOperator{\proj}{proj}
\DeclareMathOperator{\ctg}{ctg}
\DeclareMathOperator{\supp}{supp}
\DeclareMathOperator{\argmin}{argmin}
\DeclareMathOperator{\mult}{mult}
\DeclareMathOperator{\ch}{ch}
\DeclareMathOperator{\sh}{sh}
\DeclareMathOperator{\rang}{rang}
\DeclareMathOperator{\diam}{diam}
\DeclareMathOperator{\Epigraphe}{Epigraphe}




\usepackage{xcolor}
\everymath{\color{blue}}
%\everymath{\color[rgb]{0,1,1}}
%\pagecolor[rgb]{0,0,0.5}


\newcommand*{\pdtest}[3][]{\ensuremath{\frac{\partial^{#1} #2}{\partial #3}}}

\newcommand*{\deffunc}[6][]{\ensuremath{
\begin{array}{rcl}
#2 : #3 &\rightarrow& #4\\
#5 &\mapsto& #6
\end{array}
}}

\newcommand{\eqcolon}{\mathrel{\resizebox{\widthof{$\mathord{=}$}}{\height}{ $\!\!=\!\!\resizebox{1.2\width}{0.8\height}{\raisebox{0.23ex}{$\mathop{:}$}}\!\!$ }}}
\newcommand{\coloneq}{\mathrel{\resizebox{\widthof{$\mathord{=}$}}{\height}{ $\!\!\resizebox{1.2\width}{0.8\height}{\raisebox{0.23ex}{$\mathop{:}$}}\!\!=\!\!$ }}}
\newcommand{\eqcolonl}{\ensuremath{\mathrel{=\!\!\mathop{:}}}}
\newcommand{\coloneql}{\ensuremath{\mathrel{\mathop{:} \!\! =}}}
\newcommand{\vc}[1]{% inline column vector
  \left(\begin{smallmatrix}#1\end{smallmatrix}\right)%
}
\newcommand{\vr}[1]{% inline row vector
  \begin{smallmatrix}(\,#1\,)\end{smallmatrix}%
}
\makeatletter
\newcommand*{\defeq}{\ =\mathrel{\rlap{%
                     \raisebox{0.3ex}{$\m@th\cdot$}}%
                     \raisebox{-0.3ex}{$\m@th\cdot$}}%
                     }
\makeatother

\newcommand{\mathcircle}[1]{% inline row vector
 \overset{\circ}{#1}
}
\newcommand{\ulim}{% low limit
 \underline{\lim}
}
\newcommand{\ssi}{% iff
\iff
}
\newcommand{\ps}[2]{
\expval{#1 | #2}
}
\newcommand{\df}[1]{
\mqty{#1}
}
\newcommand{\n}[1]{
\norm{#1}
}
\newcommand{\sys}[1]{
\left\{\smqty{#1}\right.
}


\newcommand{\eqdef}{\ensuremath{\overset{\text{def}}=}}


\def\Circlearrowright{\ensuremath{%
  \rotatebox[origin=c]{230}{$\circlearrowright$}}}

\newcommand\ct[1]{\text{\rmfamily\upshape #1}}
\newcommand\question[1]{ {\color{red} ...!? \small #1}}
\newcommand\caz[1]{\left\{\begin{array} #1 \end{array}\right.}
\newcommand\const{\text{\rmfamily\upshape const}}
\newcommand\toP{ \overset{\pro}{\to}}
\newcommand\toPP{ \overset{\text{PP}}{\to}}
\newcommand{\oeq}{\mathrel{\text{\textcircled{$=$}}}}





\usepackage{xcolor}
% \usepackage[normalem]{ulem}
\usepackage{lipsum}
\makeatletter
% \newcommand\colorwave[1][blue]{\bgroup \markoverwith{\lower3.5\p@\hbox{\sixly \textcolor{#1}{\char58}}}\ULon}
%\font\sixly=lasy6 % does not re-load if already loaded, so no memory problem.

\newmdtheoremenv[
linewidth= 1pt,linecolor= blue,%
leftmargin=20,rightmargin=20,innertopmargin=0pt, innerrightmargin=40,%
tikzsetting = { draw=lightgray, line width = 0.3pt,dashed,%
dash pattern = on 15pt off 3pt},%
splittopskip=\topskip,skipbelow=\baselineskip,%
skipabove=\baselineskip,ntheorem,roundcorner=0pt,
% backgroundcolor=pagebg,font=\color{orange}\sffamily, fontcolor=white
]{examplebox}{Exemple}[section]



\newcommand\R{\mathbb{R}}
\newcommand\Z{\mathbb{Z}}
\newcommand\N{\mathbb{N}}
\newcommand\E{\mathbb{E}}
\newcommand\F{\mathcal{F}}
\newcommand\cH{\mathcal{H}}
\newcommand\V{\mathbb{V}}
\newcommand\dmo{ ^{-1} }
\newcommand\kapa{\kappa}
\newcommand\im{Im}
\newcommand\hs{\mathcal{H}}





\usepackage{soul}

\makeatletter
\newcommand*{\whiten}[1]{\llap{\textcolor{white}{{\the\SOUL@token}}\hspace{#1pt}}}
\DeclareRobustCommand*\myul{%
    \def\SOUL@everyspace{\underline{\space}\kern\z@}%
    \def\SOUL@everytoken{%
     \setbox0=\hbox{\the\SOUL@token}%
     \ifdim\dp0>\z@
        \raisebox{\dp0}{\underline{\phantom{\the\SOUL@token}}}%
        \whiten{1}\whiten{0}%
        \whiten{-1}\whiten{-2}%
        \llap{\the\SOUL@token}%
     \else
        \underline{\the\SOUL@token}%
     \fi}%
\SOUL@}
\makeatother

\newcommand*{\demp}{\fontfamily{lmtt}\selectfont}

\DeclareTextFontCommand{\textdemp}{\demp}

\begin{document}

\ifcomment
Multiline
comment
\fi
\ifcomment
\myul{Typesetting test}
% \color[rgb]{1,1,1}
$∑_i^n≠ 60º±∞π∆¬≈√j∫h≤≥µ$

$\CR \R\pro\ind\pro\gS\pro
\mqty[a&b\\c&d]$
$\pro\mathbb{P}$
$\dd{x}$

  \[
    \alpha(x)=\left\{
                \begin{array}{ll}
                  x\\
                  \frac{1}{1+e^{-kx}}\\
                  \frac{e^x-e^{-x}}{e^x+e^{-x}}
                \end{array}
              \right.
  \]

  $\expval{x}$
  
  $\chi_\rho(ghg\dmo)=\Tr(\rho_{ghg\dmo})=\Tr(\rho_g\circ\rho_h\circ\rho\dmo_g)=\Tr(\rho_h)\overset{\mbox{\scalebox{0.5}{$\Tr(AB)=\Tr(BA)$}}}{=}\chi_\rho(h)$
  	$\mathop{\oplus}_{\substack{x\in X}}$

$\mat(\rho_g)=(a_{ij}(g))_{\scriptsize \substack{1\leq i\leq d \\ 1\leq j\leq d}}$ et $\mat(\rho'_g)=(a'_{ij}(g))_{\scriptsize \substack{1\leq i'\leq d' \\ 1\leq j'\leq d'}}$



\[\int_a^b{\mathbb{R}^2}g(u, v)\dd{P_{XY}}(u, v)=\iint g(u,v) f_{XY}(u, v)\dd \lambda(u) \dd \lambda(v)\]
$$\lim_{x\to\infty} f(x)$$	
$$\iiiint_V \mu(t,u,v,w) \,dt\,du\,dv\,dw$$
$$\sum_{n=1}^{\infty} 2^{-n} = 1$$	
\begin{definition}
	Si $X$ et $Y$ sont 2 v.a. ou definit la \textsc{Covariance} entre $X$ et $Y$ comme
	$\cov(X,Y)\overset{\text{def}}{=}\E\left[(X-\E(X))(Y-\E(Y))\right]=\E(XY)-\E(X)\E(Y)$.
\end{definition}
\fi
\pagebreak

% \tableofcontents

% insert your code here
%\input{./algebra/main.tex}
%\input{./geometrie-differentielle/main.tex}
%\input{./probabilite/main.tex}
%\input{./analyse-fonctionnelle/main.tex}
% \input{./Analyse-convexe-et-dualite-en-optimisation/main.tex}
%\input{./tikz/main.tex}
%\input{./Theorie-du-distributions/main.tex}
%\input{./optimisation/mine.tex}
 \input{./modelisation/main.tex}

% yves.aubry@univ-tln.fr : algebra

\end{document}

%% !TEX encoding = UTF-8 Unicode
% !TEX TS-program = xelatex

\documentclass[french]{report}

%\usepackage[utf8]{inputenc}
%\usepackage[T1]{fontenc}
\usepackage{babel}


\newif\ifcomment
%\commenttrue # Show comments

\usepackage{physics}
\usepackage{amssymb}


\usepackage{amsthm}
% \usepackage{thmtools}
\usepackage{mathtools}
\usepackage{amsfonts}

\usepackage{color}

\usepackage{tikz}

\usepackage{geometry}
\geometry{a5paper, margin=0.1in, right=1cm}

\usepackage{dsfont}

\usepackage{graphicx}
\graphicspath{ {images/} }

\usepackage{faktor}

\usepackage{IEEEtrantools}
\usepackage{enumerate}   
\usepackage[PostScript=dvips]{"/Users/aware/Documents/Courses/diagrams"}


\newtheorem{theorem}{Théorème}[section]
\renewcommand{\thetheorem}{\arabic{theorem}}
\newtheorem{lemme}{Lemme}[section]
\renewcommand{\thelemme}{\arabic{lemme}}
\newtheorem{proposition}{Proposition}[section]
\renewcommand{\theproposition}{\arabic{proposition}}
\newtheorem{notations}{Notations}[section]
\newtheorem{problem}{Problème}[section]
\newtheorem{corollary}{Corollaire}[theorem]
\renewcommand{\thecorollary}{\arabic{corollary}}
\newtheorem{property}{Propriété}[section]
\newtheorem{objective}{Objectif}[section]

\theoremstyle{definition}
\newtheorem{definition}{Définition}[section]
\renewcommand{\thedefinition}{\arabic{definition}}
\newtheorem{exercise}{Exercice}[chapter]
\renewcommand{\theexercise}{\arabic{exercise}}
\newtheorem{example}{Exemple}[chapter]
\renewcommand{\theexample}{\arabic{example}}
\newtheorem*{solution}{Solution}
\newtheorem*{application}{Application}
\newtheorem*{notation}{Notation}
\newtheorem*{vocabulary}{Vocabulaire}
\newtheorem*{properties}{Propriétés}



\theoremstyle{remark}
\newtheorem*{remark}{Remarque}
\newtheorem*{rappel}{Rappel}


\usepackage{etoolbox}
\AtBeginEnvironment{exercise}{\small}
\AtBeginEnvironment{example}{\small}

\usepackage{cases}
\usepackage[red]{mypack}

\usepackage[framemethod=TikZ]{mdframed}

\definecolor{bg}{rgb}{0.4,0.25,0.95}
\definecolor{pagebg}{rgb}{0,0,0.5}
\surroundwithmdframed[
   topline=false,
   rightline=false,
   bottomline=false,
   leftmargin=\parindent,
   skipabove=8pt,
   skipbelow=8pt,
   linecolor=blue,
   innerbottommargin=10pt,
   % backgroundcolor=bg,font=\color{orange}\sffamily, fontcolor=white
]{definition}

\usepackage{empheq}
\usepackage[most]{tcolorbox}

\newtcbox{\mymath}[1][]{%
    nobeforeafter, math upper, tcbox raise base,
    enhanced, colframe=blue!30!black,
    colback=red!10, boxrule=1pt,
    #1}

\usepackage{unixode}


\DeclareMathOperator{\ord}{ord}
\DeclareMathOperator{\orb}{orb}
\DeclareMathOperator{\stab}{stab}
\DeclareMathOperator{\Stab}{stab}
\DeclareMathOperator{\ppcm}{ppcm}
\DeclareMathOperator{\conj}{Conj}
\DeclareMathOperator{\End}{End}
\DeclareMathOperator{\rot}{rot}
\DeclareMathOperator{\trs}{trace}
\DeclareMathOperator{\Ind}{Ind}
\DeclareMathOperator{\mat}{Mat}
\DeclareMathOperator{\id}{Id}
\DeclareMathOperator{\vect}{vect}
\DeclareMathOperator{\img}{img}
\DeclareMathOperator{\cov}{Cov}
\DeclareMathOperator{\dist}{dist}
\DeclareMathOperator{\irr}{Irr}
\DeclareMathOperator{\image}{Im}
\DeclareMathOperator{\pd}{\partial}
\DeclareMathOperator{\epi}{epi}
\DeclareMathOperator{\Argmin}{Argmin}
\DeclareMathOperator{\dom}{dom}
\DeclareMathOperator{\proj}{proj}
\DeclareMathOperator{\ctg}{ctg}
\DeclareMathOperator{\supp}{supp}
\DeclareMathOperator{\argmin}{argmin}
\DeclareMathOperator{\mult}{mult}
\DeclareMathOperator{\ch}{ch}
\DeclareMathOperator{\sh}{sh}
\DeclareMathOperator{\rang}{rang}
\DeclareMathOperator{\diam}{diam}
\DeclareMathOperator{\Epigraphe}{Epigraphe}




\usepackage{xcolor}
\everymath{\color{blue}}
%\everymath{\color[rgb]{0,1,1}}
%\pagecolor[rgb]{0,0,0.5}


\newcommand*{\pdtest}[3][]{\ensuremath{\frac{\partial^{#1} #2}{\partial #3}}}

\newcommand*{\deffunc}[6][]{\ensuremath{
\begin{array}{rcl}
#2 : #3 &\rightarrow& #4\\
#5 &\mapsto& #6
\end{array}
}}

\newcommand{\eqcolon}{\mathrel{\resizebox{\widthof{$\mathord{=}$}}{\height}{ $\!\!=\!\!\resizebox{1.2\width}{0.8\height}{\raisebox{0.23ex}{$\mathop{:}$}}\!\!$ }}}
\newcommand{\coloneq}{\mathrel{\resizebox{\widthof{$\mathord{=}$}}{\height}{ $\!\!\resizebox{1.2\width}{0.8\height}{\raisebox{0.23ex}{$\mathop{:}$}}\!\!=\!\!$ }}}
\newcommand{\eqcolonl}{\ensuremath{\mathrel{=\!\!\mathop{:}}}}
\newcommand{\coloneql}{\ensuremath{\mathrel{\mathop{:} \!\! =}}}
\newcommand{\vc}[1]{% inline column vector
  \left(\begin{smallmatrix}#1\end{smallmatrix}\right)%
}
\newcommand{\vr}[1]{% inline row vector
  \begin{smallmatrix}(\,#1\,)\end{smallmatrix}%
}
\makeatletter
\newcommand*{\defeq}{\ =\mathrel{\rlap{%
                     \raisebox{0.3ex}{$\m@th\cdot$}}%
                     \raisebox{-0.3ex}{$\m@th\cdot$}}%
                     }
\makeatother

\newcommand{\mathcircle}[1]{% inline row vector
 \overset{\circ}{#1}
}
\newcommand{\ulim}{% low limit
 \underline{\lim}
}
\newcommand{\ssi}{% iff
\iff
}
\newcommand{\ps}[2]{
\expval{#1 | #2}
}
\newcommand{\df}[1]{
\mqty{#1}
}
\newcommand{\n}[1]{
\norm{#1}
}
\newcommand{\sys}[1]{
\left\{\smqty{#1}\right.
}


\newcommand{\eqdef}{\ensuremath{\overset{\text{def}}=}}


\def\Circlearrowright{\ensuremath{%
  \rotatebox[origin=c]{230}{$\circlearrowright$}}}

\newcommand\ct[1]{\text{\rmfamily\upshape #1}}
\newcommand\question[1]{ {\color{red} ...!? \small #1}}
\newcommand\caz[1]{\left\{\begin{array} #1 \end{array}\right.}
\newcommand\const{\text{\rmfamily\upshape const}}
\newcommand\toP{ \overset{\pro}{\to}}
\newcommand\toPP{ \overset{\text{PP}}{\to}}
\newcommand{\oeq}{\mathrel{\text{\textcircled{$=$}}}}





\usepackage{xcolor}
% \usepackage[normalem]{ulem}
\usepackage{lipsum}
\makeatletter
% \newcommand\colorwave[1][blue]{\bgroup \markoverwith{\lower3.5\p@\hbox{\sixly \textcolor{#1}{\char58}}}\ULon}
%\font\sixly=lasy6 % does not re-load if already loaded, so no memory problem.

\newmdtheoremenv[
linewidth= 1pt,linecolor= blue,%
leftmargin=20,rightmargin=20,innertopmargin=0pt, innerrightmargin=40,%
tikzsetting = { draw=lightgray, line width = 0.3pt,dashed,%
dash pattern = on 15pt off 3pt},%
splittopskip=\topskip,skipbelow=\baselineskip,%
skipabove=\baselineskip,ntheorem,roundcorner=0pt,
% backgroundcolor=pagebg,font=\color{orange}\sffamily, fontcolor=white
]{examplebox}{Exemple}[section]



\newcommand\R{\mathbb{R}}
\newcommand\Z{\mathbb{Z}}
\newcommand\N{\mathbb{N}}
\newcommand\E{\mathbb{E}}
\newcommand\F{\mathcal{F}}
\newcommand\cH{\mathcal{H}}
\newcommand\V{\mathbb{V}}
\newcommand\dmo{ ^{-1} }
\newcommand\kapa{\kappa}
\newcommand\im{Im}
\newcommand\hs{\mathcal{H}}





\usepackage{soul}

\makeatletter
\newcommand*{\whiten}[1]{\llap{\textcolor{white}{{\the\SOUL@token}}\hspace{#1pt}}}
\DeclareRobustCommand*\myul{%
    \def\SOUL@everyspace{\underline{\space}\kern\z@}%
    \def\SOUL@everytoken{%
     \setbox0=\hbox{\the\SOUL@token}%
     \ifdim\dp0>\z@
        \raisebox{\dp0}{\underline{\phantom{\the\SOUL@token}}}%
        \whiten{1}\whiten{0}%
        \whiten{-1}\whiten{-2}%
        \llap{\the\SOUL@token}%
     \else
        \underline{\the\SOUL@token}%
     \fi}%
\SOUL@}
\makeatother

\newcommand*{\demp}{\fontfamily{lmtt}\selectfont}

\DeclareTextFontCommand{\textdemp}{\demp}

\begin{document}

\ifcomment
Multiline
comment
\fi
\ifcomment
\myul{Typesetting test}
% \color[rgb]{1,1,1}
$∑_i^n≠ 60º±∞π∆¬≈√j∫h≤≥µ$

$\CR \R\pro\ind\pro\gS\pro
\mqty[a&b\\c&d]$
$\pro\mathbb{P}$
$\dd{x}$

  \[
    \alpha(x)=\left\{
                \begin{array}{ll}
                  x\\
                  \frac{1}{1+e^{-kx}}\\
                  \frac{e^x-e^{-x}}{e^x+e^{-x}}
                \end{array}
              \right.
  \]

  $\expval{x}$
  
  $\chi_\rho(ghg\dmo)=\Tr(\rho_{ghg\dmo})=\Tr(\rho_g\circ\rho_h\circ\rho\dmo_g)=\Tr(\rho_h)\overset{\mbox{\scalebox{0.5}{$\Tr(AB)=\Tr(BA)$}}}{=}\chi_\rho(h)$
  	$\mathop{\oplus}_{\substack{x\in X}}$

$\mat(\rho_g)=(a_{ij}(g))_{\scriptsize \substack{1\leq i\leq d \\ 1\leq j\leq d}}$ et $\mat(\rho'_g)=(a'_{ij}(g))_{\scriptsize \substack{1\leq i'\leq d' \\ 1\leq j'\leq d'}}$



\[\int_a^b{\mathbb{R}^2}g(u, v)\dd{P_{XY}}(u, v)=\iint g(u,v) f_{XY}(u, v)\dd \lambda(u) \dd \lambda(v)\]
$$\lim_{x\to\infty} f(x)$$	
$$\iiiint_V \mu(t,u,v,w) \,dt\,du\,dv\,dw$$
$$\sum_{n=1}^{\infty} 2^{-n} = 1$$	
\begin{definition}
	Si $X$ et $Y$ sont 2 v.a. ou definit la \textsc{Covariance} entre $X$ et $Y$ comme
	$\cov(X,Y)\overset{\text{def}}{=}\E\left[(X-\E(X))(Y-\E(Y))\right]=\E(XY)-\E(X)\E(Y)$.
\end{definition}
\fi
\pagebreak

% \tableofcontents

% insert your code here
%\input{./algebra/main.tex}
%\input{./geometrie-differentielle/main.tex}
%\input{./probabilite/main.tex}
%\input{./analyse-fonctionnelle/main.tex}
% \input{./Analyse-convexe-et-dualite-en-optimisation/main.tex}
%\input{./tikz/main.tex}
%\input{./Theorie-du-distributions/main.tex}
%\input{./optimisation/mine.tex}
 \input{./modelisation/main.tex}

% yves.aubry@univ-tln.fr : algebra

\end{document}

%\input{./optimisation/mine.tex}
 % !TEX encoding = UTF-8 Unicode
% !TEX TS-program = xelatex

\documentclass[french]{report}

%\usepackage[utf8]{inputenc}
%\usepackage[T1]{fontenc}
\usepackage{babel}


\newif\ifcomment
%\commenttrue # Show comments

\usepackage{physics}
\usepackage{amssymb}


\usepackage{amsthm}
% \usepackage{thmtools}
\usepackage{mathtools}
\usepackage{amsfonts}

\usepackage{color}

\usepackage{tikz}

\usepackage{geometry}
\geometry{a5paper, margin=0.1in, right=1cm}

\usepackage{dsfont}

\usepackage{graphicx}
\graphicspath{ {images/} }

\usepackage{faktor}

\usepackage{IEEEtrantools}
\usepackage{enumerate}   
\usepackage[PostScript=dvips]{"/Users/aware/Documents/Courses/diagrams"}


\newtheorem{theorem}{Théorème}[section]
\renewcommand{\thetheorem}{\arabic{theorem}}
\newtheorem{lemme}{Lemme}[section]
\renewcommand{\thelemme}{\arabic{lemme}}
\newtheorem{proposition}{Proposition}[section]
\renewcommand{\theproposition}{\arabic{proposition}}
\newtheorem{notations}{Notations}[section]
\newtheorem{problem}{Problème}[section]
\newtheorem{corollary}{Corollaire}[theorem]
\renewcommand{\thecorollary}{\arabic{corollary}}
\newtheorem{property}{Propriété}[section]
\newtheorem{objective}{Objectif}[section]

\theoremstyle{definition}
\newtheorem{definition}{Définition}[section]
\renewcommand{\thedefinition}{\arabic{definition}}
\newtheorem{exercise}{Exercice}[chapter]
\renewcommand{\theexercise}{\arabic{exercise}}
\newtheorem{example}{Exemple}[chapter]
\renewcommand{\theexample}{\arabic{example}}
\newtheorem*{solution}{Solution}
\newtheorem*{application}{Application}
\newtheorem*{notation}{Notation}
\newtheorem*{vocabulary}{Vocabulaire}
\newtheorem*{properties}{Propriétés}



\theoremstyle{remark}
\newtheorem*{remark}{Remarque}
\newtheorem*{rappel}{Rappel}


\usepackage{etoolbox}
\AtBeginEnvironment{exercise}{\small}
\AtBeginEnvironment{example}{\small}

\usepackage{cases}
\usepackage[red]{mypack}

\usepackage[framemethod=TikZ]{mdframed}

\definecolor{bg}{rgb}{0.4,0.25,0.95}
\definecolor{pagebg}{rgb}{0,0,0.5}
\surroundwithmdframed[
   topline=false,
   rightline=false,
   bottomline=false,
   leftmargin=\parindent,
   skipabove=8pt,
   skipbelow=8pt,
   linecolor=blue,
   innerbottommargin=10pt,
   % backgroundcolor=bg,font=\color{orange}\sffamily, fontcolor=white
]{definition}

\usepackage{empheq}
\usepackage[most]{tcolorbox}

\newtcbox{\mymath}[1][]{%
    nobeforeafter, math upper, tcbox raise base,
    enhanced, colframe=blue!30!black,
    colback=red!10, boxrule=1pt,
    #1}

\usepackage{unixode}


\DeclareMathOperator{\ord}{ord}
\DeclareMathOperator{\orb}{orb}
\DeclareMathOperator{\stab}{stab}
\DeclareMathOperator{\Stab}{stab}
\DeclareMathOperator{\ppcm}{ppcm}
\DeclareMathOperator{\conj}{Conj}
\DeclareMathOperator{\End}{End}
\DeclareMathOperator{\rot}{rot}
\DeclareMathOperator{\trs}{trace}
\DeclareMathOperator{\Ind}{Ind}
\DeclareMathOperator{\mat}{Mat}
\DeclareMathOperator{\id}{Id}
\DeclareMathOperator{\vect}{vect}
\DeclareMathOperator{\img}{img}
\DeclareMathOperator{\cov}{Cov}
\DeclareMathOperator{\dist}{dist}
\DeclareMathOperator{\irr}{Irr}
\DeclareMathOperator{\image}{Im}
\DeclareMathOperator{\pd}{\partial}
\DeclareMathOperator{\epi}{epi}
\DeclareMathOperator{\Argmin}{Argmin}
\DeclareMathOperator{\dom}{dom}
\DeclareMathOperator{\proj}{proj}
\DeclareMathOperator{\ctg}{ctg}
\DeclareMathOperator{\supp}{supp}
\DeclareMathOperator{\argmin}{argmin}
\DeclareMathOperator{\mult}{mult}
\DeclareMathOperator{\ch}{ch}
\DeclareMathOperator{\sh}{sh}
\DeclareMathOperator{\rang}{rang}
\DeclareMathOperator{\diam}{diam}
\DeclareMathOperator{\Epigraphe}{Epigraphe}




\usepackage{xcolor}
\everymath{\color{blue}}
%\everymath{\color[rgb]{0,1,1}}
%\pagecolor[rgb]{0,0,0.5}


\newcommand*{\pdtest}[3][]{\ensuremath{\frac{\partial^{#1} #2}{\partial #3}}}

\newcommand*{\deffunc}[6][]{\ensuremath{
\begin{array}{rcl}
#2 : #3 &\rightarrow& #4\\
#5 &\mapsto& #6
\end{array}
}}

\newcommand{\eqcolon}{\mathrel{\resizebox{\widthof{$\mathord{=}$}}{\height}{ $\!\!=\!\!\resizebox{1.2\width}{0.8\height}{\raisebox{0.23ex}{$\mathop{:}$}}\!\!$ }}}
\newcommand{\coloneq}{\mathrel{\resizebox{\widthof{$\mathord{=}$}}{\height}{ $\!\!\resizebox{1.2\width}{0.8\height}{\raisebox{0.23ex}{$\mathop{:}$}}\!\!=\!\!$ }}}
\newcommand{\eqcolonl}{\ensuremath{\mathrel{=\!\!\mathop{:}}}}
\newcommand{\coloneql}{\ensuremath{\mathrel{\mathop{:} \!\! =}}}
\newcommand{\vc}[1]{% inline column vector
  \left(\begin{smallmatrix}#1\end{smallmatrix}\right)%
}
\newcommand{\vr}[1]{% inline row vector
  \begin{smallmatrix}(\,#1\,)\end{smallmatrix}%
}
\makeatletter
\newcommand*{\defeq}{\ =\mathrel{\rlap{%
                     \raisebox{0.3ex}{$\m@th\cdot$}}%
                     \raisebox{-0.3ex}{$\m@th\cdot$}}%
                     }
\makeatother

\newcommand{\mathcircle}[1]{% inline row vector
 \overset{\circ}{#1}
}
\newcommand{\ulim}{% low limit
 \underline{\lim}
}
\newcommand{\ssi}{% iff
\iff
}
\newcommand{\ps}[2]{
\expval{#1 | #2}
}
\newcommand{\df}[1]{
\mqty{#1}
}
\newcommand{\n}[1]{
\norm{#1}
}
\newcommand{\sys}[1]{
\left\{\smqty{#1}\right.
}


\newcommand{\eqdef}{\ensuremath{\overset{\text{def}}=}}


\def\Circlearrowright{\ensuremath{%
  \rotatebox[origin=c]{230}{$\circlearrowright$}}}

\newcommand\ct[1]{\text{\rmfamily\upshape #1}}
\newcommand\question[1]{ {\color{red} ...!? \small #1}}
\newcommand\caz[1]{\left\{\begin{array} #1 \end{array}\right.}
\newcommand\const{\text{\rmfamily\upshape const}}
\newcommand\toP{ \overset{\pro}{\to}}
\newcommand\toPP{ \overset{\text{PP}}{\to}}
\newcommand{\oeq}{\mathrel{\text{\textcircled{$=$}}}}





\usepackage{xcolor}
% \usepackage[normalem]{ulem}
\usepackage{lipsum}
\makeatletter
% \newcommand\colorwave[1][blue]{\bgroup \markoverwith{\lower3.5\p@\hbox{\sixly \textcolor{#1}{\char58}}}\ULon}
%\font\sixly=lasy6 % does not re-load if already loaded, so no memory problem.

\newmdtheoremenv[
linewidth= 1pt,linecolor= blue,%
leftmargin=20,rightmargin=20,innertopmargin=0pt, innerrightmargin=40,%
tikzsetting = { draw=lightgray, line width = 0.3pt,dashed,%
dash pattern = on 15pt off 3pt},%
splittopskip=\topskip,skipbelow=\baselineskip,%
skipabove=\baselineskip,ntheorem,roundcorner=0pt,
% backgroundcolor=pagebg,font=\color{orange}\sffamily, fontcolor=white
]{examplebox}{Exemple}[section]



\newcommand\R{\mathbb{R}}
\newcommand\Z{\mathbb{Z}}
\newcommand\N{\mathbb{N}}
\newcommand\E{\mathbb{E}}
\newcommand\F{\mathcal{F}}
\newcommand\cH{\mathcal{H}}
\newcommand\V{\mathbb{V}}
\newcommand\dmo{ ^{-1} }
\newcommand\kapa{\kappa}
\newcommand\im{Im}
\newcommand\hs{\mathcal{H}}





\usepackage{soul}

\makeatletter
\newcommand*{\whiten}[1]{\llap{\textcolor{white}{{\the\SOUL@token}}\hspace{#1pt}}}
\DeclareRobustCommand*\myul{%
    \def\SOUL@everyspace{\underline{\space}\kern\z@}%
    \def\SOUL@everytoken{%
     \setbox0=\hbox{\the\SOUL@token}%
     \ifdim\dp0>\z@
        \raisebox{\dp0}{\underline{\phantom{\the\SOUL@token}}}%
        \whiten{1}\whiten{0}%
        \whiten{-1}\whiten{-2}%
        \llap{\the\SOUL@token}%
     \else
        \underline{\the\SOUL@token}%
     \fi}%
\SOUL@}
\makeatother

\newcommand*{\demp}{\fontfamily{lmtt}\selectfont}

\DeclareTextFontCommand{\textdemp}{\demp}

\begin{document}

\ifcomment
Multiline
comment
\fi
\ifcomment
\myul{Typesetting test}
% \color[rgb]{1,1,1}
$∑_i^n≠ 60º±∞π∆¬≈√j∫h≤≥µ$

$\CR \R\pro\ind\pro\gS\pro
\mqty[a&b\\c&d]$
$\pro\mathbb{P}$
$\dd{x}$

  \[
    \alpha(x)=\left\{
                \begin{array}{ll}
                  x\\
                  \frac{1}{1+e^{-kx}}\\
                  \frac{e^x-e^{-x}}{e^x+e^{-x}}
                \end{array}
              \right.
  \]

  $\expval{x}$
  
  $\chi_\rho(ghg\dmo)=\Tr(\rho_{ghg\dmo})=\Tr(\rho_g\circ\rho_h\circ\rho\dmo_g)=\Tr(\rho_h)\overset{\mbox{\scalebox{0.5}{$\Tr(AB)=\Tr(BA)$}}}{=}\chi_\rho(h)$
  	$\mathop{\oplus}_{\substack{x\in X}}$

$\mat(\rho_g)=(a_{ij}(g))_{\scriptsize \substack{1\leq i\leq d \\ 1\leq j\leq d}}$ et $\mat(\rho'_g)=(a'_{ij}(g))_{\scriptsize \substack{1\leq i'\leq d' \\ 1\leq j'\leq d'}}$



\[\int_a^b{\mathbb{R}^2}g(u, v)\dd{P_{XY}}(u, v)=\iint g(u,v) f_{XY}(u, v)\dd \lambda(u) \dd \lambda(v)\]
$$\lim_{x\to\infty} f(x)$$	
$$\iiiint_V \mu(t,u,v,w) \,dt\,du\,dv\,dw$$
$$\sum_{n=1}^{\infty} 2^{-n} = 1$$	
\begin{definition}
	Si $X$ et $Y$ sont 2 v.a. ou definit la \textsc{Covariance} entre $X$ et $Y$ comme
	$\cov(X,Y)\overset{\text{def}}{=}\E\left[(X-\E(X))(Y-\E(Y))\right]=\E(XY)-\E(X)\E(Y)$.
\end{definition}
\fi
\pagebreak

% \tableofcontents

% insert your code here
%\input{./algebra/main.tex}
%\input{./geometrie-differentielle/main.tex}
%\input{./probabilite/main.tex}
%\input{./analyse-fonctionnelle/main.tex}
% \input{./Analyse-convexe-et-dualite-en-optimisation/main.tex}
%\input{./tikz/main.tex}
%\input{./Theorie-du-distributions/main.tex}
%\input{./optimisation/mine.tex}
 \input{./modelisation/main.tex}

% yves.aubry@univ-tln.fr : algebra

\end{document}


% yves.aubry@univ-tln.fr : algebra

\end{document}

%% !TEX encoding = UTF-8 Unicode
% !TEX TS-program = xelatex

\documentclass[french]{report}

%\usepackage[utf8]{inputenc}
%\usepackage[T1]{fontenc}
\usepackage{babel}


\newif\ifcomment
%\commenttrue # Show comments

\usepackage{physics}
\usepackage{amssymb}


\usepackage{amsthm}
% \usepackage{thmtools}
\usepackage{mathtools}
\usepackage{amsfonts}

\usepackage{color}

\usepackage{tikz}

\usepackage{geometry}
\geometry{a5paper, margin=0.1in, right=1cm}

\usepackage{dsfont}

\usepackage{graphicx}
\graphicspath{ {images/} }

\usepackage{faktor}

\usepackage{IEEEtrantools}
\usepackage{enumerate}   
\usepackage[PostScript=dvips]{"/Users/aware/Documents/Courses/diagrams"}


\newtheorem{theorem}{Théorème}[section]
\renewcommand{\thetheorem}{\arabic{theorem}}
\newtheorem{lemme}{Lemme}[section]
\renewcommand{\thelemme}{\arabic{lemme}}
\newtheorem{proposition}{Proposition}[section]
\renewcommand{\theproposition}{\arabic{proposition}}
\newtheorem{notations}{Notations}[section]
\newtheorem{problem}{Problème}[section]
\newtheorem{corollary}{Corollaire}[theorem]
\renewcommand{\thecorollary}{\arabic{corollary}}
\newtheorem{property}{Propriété}[section]
\newtheorem{objective}{Objectif}[section]

\theoremstyle{definition}
\newtheorem{definition}{Définition}[section]
\renewcommand{\thedefinition}{\arabic{definition}}
\newtheorem{exercise}{Exercice}[chapter]
\renewcommand{\theexercise}{\arabic{exercise}}
\newtheorem{example}{Exemple}[chapter]
\renewcommand{\theexample}{\arabic{example}}
\newtheorem*{solution}{Solution}
\newtheorem*{application}{Application}
\newtheorem*{notation}{Notation}
\newtheorem*{vocabulary}{Vocabulaire}
\newtheorem*{properties}{Propriétés}



\theoremstyle{remark}
\newtheorem*{remark}{Remarque}
\newtheorem*{rappel}{Rappel}


\usepackage{etoolbox}
\AtBeginEnvironment{exercise}{\small}
\AtBeginEnvironment{example}{\small}

\usepackage{cases}
\usepackage[red]{mypack}

\usepackage[framemethod=TikZ]{mdframed}

\definecolor{bg}{rgb}{0.4,0.25,0.95}
\definecolor{pagebg}{rgb}{0,0,0.5}
\surroundwithmdframed[
   topline=false,
   rightline=false,
   bottomline=false,
   leftmargin=\parindent,
   skipabove=8pt,
   skipbelow=8pt,
   linecolor=blue,
   innerbottommargin=10pt,
   % backgroundcolor=bg,font=\color{orange}\sffamily, fontcolor=white
]{definition}

\usepackage{empheq}
\usepackage[most]{tcolorbox}

\newtcbox{\mymath}[1][]{%
    nobeforeafter, math upper, tcbox raise base,
    enhanced, colframe=blue!30!black,
    colback=red!10, boxrule=1pt,
    #1}

\usepackage{unixode}


\DeclareMathOperator{\ord}{ord}
\DeclareMathOperator{\orb}{orb}
\DeclareMathOperator{\stab}{stab}
\DeclareMathOperator{\Stab}{stab}
\DeclareMathOperator{\ppcm}{ppcm}
\DeclareMathOperator{\conj}{Conj}
\DeclareMathOperator{\End}{End}
\DeclareMathOperator{\rot}{rot}
\DeclareMathOperator{\trs}{trace}
\DeclareMathOperator{\Ind}{Ind}
\DeclareMathOperator{\mat}{Mat}
\DeclareMathOperator{\id}{Id}
\DeclareMathOperator{\vect}{vect}
\DeclareMathOperator{\img}{img}
\DeclareMathOperator{\cov}{Cov}
\DeclareMathOperator{\dist}{dist}
\DeclareMathOperator{\irr}{Irr}
\DeclareMathOperator{\image}{Im}
\DeclareMathOperator{\pd}{\partial}
\DeclareMathOperator{\epi}{epi}
\DeclareMathOperator{\Argmin}{Argmin}
\DeclareMathOperator{\dom}{dom}
\DeclareMathOperator{\proj}{proj}
\DeclareMathOperator{\ctg}{ctg}
\DeclareMathOperator{\supp}{supp}
\DeclareMathOperator{\argmin}{argmin}
\DeclareMathOperator{\mult}{mult}
\DeclareMathOperator{\ch}{ch}
\DeclareMathOperator{\sh}{sh}
\DeclareMathOperator{\rang}{rang}
\DeclareMathOperator{\diam}{diam}
\DeclareMathOperator{\Epigraphe}{Epigraphe}




\usepackage{xcolor}
\everymath{\color{blue}}
%\everymath{\color[rgb]{0,1,1}}
%\pagecolor[rgb]{0,0,0.5}


\newcommand*{\pdtest}[3][]{\ensuremath{\frac{\partial^{#1} #2}{\partial #3}}}

\newcommand*{\deffunc}[6][]{\ensuremath{
\begin{array}{rcl}
#2 : #3 &\rightarrow& #4\\
#5 &\mapsto& #6
\end{array}
}}

\newcommand{\eqcolon}{\mathrel{\resizebox{\widthof{$\mathord{=}$}}{\height}{ $\!\!=\!\!\resizebox{1.2\width}{0.8\height}{\raisebox{0.23ex}{$\mathop{:}$}}\!\!$ }}}
\newcommand{\coloneq}{\mathrel{\resizebox{\widthof{$\mathord{=}$}}{\height}{ $\!\!\resizebox{1.2\width}{0.8\height}{\raisebox{0.23ex}{$\mathop{:}$}}\!\!=\!\!$ }}}
\newcommand{\eqcolonl}{\ensuremath{\mathrel{=\!\!\mathop{:}}}}
\newcommand{\coloneql}{\ensuremath{\mathrel{\mathop{:} \!\! =}}}
\newcommand{\vc}[1]{% inline column vector
  \left(\begin{smallmatrix}#1\end{smallmatrix}\right)%
}
\newcommand{\vr}[1]{% inline row vector
  \begin{smallmatrix}(\,#1\,)\end{smallmatrix}%
}
\makeatletter
\newcommand*{\defeq}{\ =\mathrel{\rlap{%
                     \raisebox{0.3ex}{$\m@th\cdot$}}%
                     \raisebox{-0.3ex}{$\m@th\cdot$}}%
                     }
\makeatother

\newcommand{\mathcircle}[1]{% inline row vector
 \overset{\circ}{#1}
}
\newcommand{\ulim}{% low limit
 \underline{\lim}
}
\newcommand{\ssi}{% iff
\iff
}
\newcommand{\ps}[2]{
\expval{#1 | #2}
}
\newcommand{\df}[1]{
\mqty{#1}
}
\newcommand{\n}[1]{
\norm{#1}
}
\newcommand{\sys}[1]{
\left\{\smqty{#1}\right.
}


\newcommand{\eqdef}{\ensuremath{\overset{\text{def}}=}}


\def\Circlearrowright{\ensuremath{%
  \rotatebox[origin=c]{230}{$\circlearrowright$}}}

\newcommand\ct[1]{\text{\rmfamily\upshape #1}}
\newcommand\question[1]{ {\color{red} ...!? \small #1}}
\newcommand\caz[1]{\left\{\begin{array} #1 \end{array}\right.}
\newcommand\const{\text{\rmfamily\upshape const}}
\newcommand\toP{ \overset{\pro}{\to}}
\newcommand\toPP{ \overset{\text{PP}}{\to}}
\newcommand{\oeq}{\mathrel{\text{\textcircled{$=$}}}}





\usepackage{xcolor}
% \usepackage[normalem]{ulem}
\usepackage{lipsum}
\makeatletter
% \newcommand\colorwave[1][blue]{\bgroup \markoverwith{\lower3.5\p@\hbox{\sixly \textcolor{#1}{\char58}}}\ULon}
%\font\sixly=lasy6 % does not re-load if already loaded, so no memory problem.

\newmdtheoremenv[
linewidth= 1pt,linecolor= blue,%
leftmargin=20,rightmargin=20,innertopmargin=0pt, innerrightmargin=40,%
tikzsetting = { draw=lightgray, line width = 0.3pt,dashed,%
dash pattern = on 15pt off 3pt},%
splittopskip=\topskip,skipbelow=\baselineskip,%
skipabove=\baselineskip,ntheorem,roundcorner=0pt,
% backgroundcolor=pagebg,font=\color{orange}\sffamily, fontcolor=white
]{examplebox}{Exemple}[section]



\newcommand\R{\mathbb{R}}
\newcommand\Z{\mathbb{Z}}
\newcommand\N{\mathbb{N}}
\newcommand\E{\mathbb{E}}
\newcommand\F{\mathcal{F}}
\newcommand\cH{\mathcal{H}}
\newcommand\V{\mathbb{V}}
\newcommand\dmo{ ^{-1} }
\newcommand\kapa{\kappa}
\newcommand\im{Im}
\newcommand\hs{\mathcal{H}}





\usepackage{soul}

\makeatletter
\newcommand*{\whiten}[1]{\llap{\textcolor{white}{{\the\SOUL@token}}\hspace{#1pt}}}
\DeclareRobustCommand*\myul{%
    \def\SOUL@everyspace{\underline{\space}\kern\z@}%
    \def\SOUL@everytoken{%
     \setbox0=\hbox{\the\SOUL@token}%
     \ifdim\dp0>\z@
        \raisebox{\dp0}{\underline{\phantom{\the\SOUL@token}}}%
        \whiten{1}\whiten{0}%
        \whiten{-1}\whiten{-2}%
        \llap{\the\SOUL@token}%
     \else
        \underline{\the\SOUL@token}%
     \fi}%
\SOUL@}
\makeatother

\newcommand*{\demp}{\fontfamily{lmtt}\selectfont}

\DeclareTextFontCommand{\textdemp}{\demp}

\begin{document}

\ifcomment
Multiline
comment
\fi
\ifcomment
\myul{Typesetting test}
% \color[rgb]{1,1,1}
$∑_i^n≠ 60º±∞π∆¬≈√j∫h≤≥µ$

$\CR \R\pro\ind\pro\gS\pro
\mqty[a&b\\c&d]$
$\pro\mathbb{P}$
$\dd{x}$

  \[
    \alpha(x)=\left\{
                \begin{array}{ll}
                  x\\
                  \frac{1}{1+e^{-kx}}\\
                  \frac{e^x-e^{-x}}{e^x+e^{-x}}
                \end{array}
              \right.
  \]

  $\expval{x}$
  
  $\chi_\rho(ghg\dmo)=\Tr(\rho_{ghg\dmo})=\Tr(\rho_g\circ\rho_h\circ\rho\dmo_g)=\Tr(\rho_h)\overset{\mbox{\scalebox{0.5}{$\Tr(AB)=\Tr(BA)$}}}{=}\chi_\rho(h)$
  	$\mathop{\oplus}_{\substack{x\in X}}$

$\mat(\rho_g)=(a_{ij}(g))_{\scriptsize \substack{1\leq i\leq d \\ 1\leq j\leq d}}$ et $\mat(\rho'_g)=(a'_{ij}(g))_{\scriptsize \substack{1\leq i'\leq d' \\ 1\leq j'\leq d'}}$



\[\int_a^b{\mathbb{R}^2}g(u, v)\dd{P_{XY}}(u, v)=\iint g(u,v) f_{XY}(u, v)\dd \lambda(u) \dd \lambda(v)\]
$$\lim_{x\to\infty} f(x)$$	
$$\iiiint_V \mu(t,u,v,w) \,dt\,du\,dv\,dw$$
$$\sum_{n=1}^{\infty} 2^{-n} = 1$$	
\begin{definition}
	Si $X$ et $Y$ sont 2 v.a. ou definit la \textsc{Covariance} entre $X$ et $Y$ comme
	$\cov(X,Y)\overset{\text{def}}{=}\E\left[(X-\E(X))(Y-\E(Y))\right]=\E(XY)-\E(X)\E(Y)$.
\end{definition}
\fi
\pagebreak

% \tableofcontents

% insert your code here
%% !TEX encoding = UTF-8 Unicode
% !TEX TS-program = xelatex

\documentclass[french]{report}

%\usepackage[utf8]{inputenc}
%\usepackage[T1]{fontenc}
\usepackage{babel}


\newif\ifcomment
%\commenttrue # Show comments

\usepackage{physics}
\usepackage{amssymb}


\usepackage{amsthm}
% \usepackage{thmtools}
\usepackage{mathtools}
\usepackage{amsfonts}

\usepackage{color}

\usepackage{tikz}

\usepackage{geometry}
\geometry{a5paper, margin=0.1in, right=1cm}

\usepackage{dsfont}

\usepackage{graphicx}
\graphicspath{ {images/} }

\usepackage{faktor}

\usepackage{IEEEtrantools}
\usepackage{enumerate}   
\usepackage[PostScript=dvips]{"/Users/aware/Documents/Courses/diagrams"}


\newtheorem{theorem}{Théorème}[section]
\renewcommand{\thetheorem}{\arabic{theorem}}
\newtheorem{lemme}{Lemme}[section]
\renewcommand{\thelemme}{\arabic{lemme}}
\newtheorem{proposition}{Proposition}[section]
\renewcommand{\theproposition}{\arabic{proposition}}
\newtheorem{notations}{Notations}[section]
\newtheorem{problem}{Problème}[section]
\newtheorem{corollary}{Corollaire}[theorem]
\renewcommand{\thecorollary}{\arabic{corollary}}
\newtheorem{property}{Propriété}[section]
\newtheorem{objective}{Objectif}[section]

\theoremstyle{definition}
\newtheorem{definition}{Définition}[section]
\renewcommand{\thedefinition}{\arabic{definition}}
\newtheorem{exercise}{Exercice}[chapter]
\renewcommand{\theexercise}{\arabic{exercise}}
\newtheorem{example}{Exemple}[chapter]
\renewcommand{\theexample}{\arabic{example}}
\newtheorem*{solution}{Solution}
\newtheorem*{application}{Application}
\newtheorem*{notation}{Notation}
\newtheorem*{vocabulary}{Vocabulaire}
\newtheorem*{properties}{Propriétés}



\theoremstyle{remark}
\newtheorem*{remark}{Remarque}
\newtheorem*{rappel}{Rappel}


\usepackage{etoolbox}
\AtBeginEnvironment{exercise}{\small}
\AtBeginEnvironment{example}{\small}

\usepackage{cases}
\usepackage[red]{mypack}

\usepackage[framemethod=TikZ]{mdframed}

\definecolor{bg}{rgb}{0.4,0.25,0.95}
\definecolor{pagebg}{rgb}{0,0,0.5}
\surroundwithmdframed[
   topline=false,
   rightline=false,
   bottomline=false,
   leftmargin=\parindent,
   skipabove=8pt,
   skipbelow=8pt,
   linecolor=blue,
   innerbottommargin=10pt,
   % backgroundcolor=bg,font=\color{orange}\sffamily, fontcolor=white
]{definition}

\usepackage{empheq}
\usepackage[most]{tcolorbox}

\newtcbox{\mymath}[1][]{%
    nobeforeafter, math upper, tcbox raise base,
    enhanced, colframe=blue!30!black,
    colback=red!10, boxrule=1pt,
    #1}

\usepackage{unixode}


\DeclareMathOperator{\ord}{ord}
\DeclareMathOperator{\orb}{orb}
\DeclareMathOperator{\stab}{stab}
\DeclareMathOperator{\Stab}{stab}
\DeclareMathOperator{\ppcm}{ppcm}
\DeclareMathOperator{\conj}{Conj}
\DeclareMathOperator{\End}{End}
\DeclareMathOperator{\rot}{rot}
\DeclareMathOperator{\trs}{trace}
\DeclareMathOperator{\Ind}{Ind}
\DeclareMathOperator{\mat}{Mat}
\DeclareMathOperator{\id}{Id}
\DeclareMathOperator{\vect}{vect}
\DeclareMathOperator{\img}{img}
\DeclareMathOperator{\cov}{Cov}
\DeclareMathOperator{\dist}{dist}
\DeclareMathOperator{\irr}{Irr}
\DeclareMathOperator{\image}{Im}
\DeclareMathOperator{\pd}{\partial}
\DeclareMathOperator{\epi}{epi}
\DeclareMathOperator{\Argmin}{Argmin}
\DeclareMathOperator{\dom}{dom}
\DeclareMathOperator{\proj}{proj}
\DeclareMathOperator{\ctg}{ctg}
\DeclareMathOperator{\supp}{supp}
\DeclareMathOperator{\argmin}{argmin}
\DeclareMathOperator{\mult}{mult}
\DeclareMathOperator{\ch}{ch}
\DeclareMathOperator{\sh}{sh}
\DeclareMathOperator{\rang}{rang}
\DeclareMathOperator{\diam}{diam}
\DeclareMathOperator{\Epigraphe}{Epigraphe}




\usepackage{xcolor}
\everymath{\color{blue}}
%\everymath{\color[rgb]{0,1,1}}
%\pagecolor[rgb]{0,0,0.5}


\newcommand*{\pdtest}[3][]{\ensuremath{\frac{\partial^{#1} #2}{\partial #3}}}

\newcommand*{\deffunc}[6][]{\ensuremath{
\begin{array}{rcl}
#2 : #3 &\rightarrow& #4\\
#5 &\mapsto& #6
\end{array}
}}

\newcommand{\eqcolon}{\mathrel{\resizebox{\widthof{$\mathord{=}$}}{\height}{ $\!\!=\!\!\resizebox{1.2\width}{0.8\height}{\raisebox{0.23ex}{$\mathop{:}$}}\!\!$ }}}
\newcommand{\coloneq}{\mathrel{\resizebox{\widthof{$\mathord{=}$}}{\height}{ $\!\!\resizebox{1.2\width}{0.8\height}{\raisebox{0.23ex}{$\mathop{:}$}}\!\!=\!\!$ }}}
\newcommand{\eqcolonl}{\ensuremath{\mathrel{=\!\!\mathop{:}}}}
\newcommand{\coloneql}{\ensuremath{\mathrel{\mathop{:} \!\! =}}}
\newcommand{\vc}[1]{% inline column vector
  \left(\begin{smallmatrix}#1\end{smallmatrix}\right)%
}
\newcommand{\vr}[1]{% inline row vector
  \begin{smallmatrix}(\,#1\,)\end{smallmatrix}%
}
\makeatletter
\newcommand*{\defeq}{\ =\mathrel{\rlap{%
                     \raisebox{0.3ex}{$\m@th\cdot$}}%
                     \raisebox{-0.3ex}{$\m@th\cdot$}}%
                     }
\makeatother

\newcommand{\mathcircle}[1]{% inline row vector
 \overset{\circ}{#1}
}
\newcommand{\ulim}{% low limit
 \underline{\lim}
}
\newcommand{\ssi}{% iff
\iff
}
\newcommand{\ps}[2]{
\expval{#1 | #2}
}
\newcommand{\df}[1]{
\mqty{#1}
}
\newcommand{\n}[1]{
\norm{#1}
}
\newcommand{\sys}[1]{
\left\{\smqty{#1}\right.
}


\newcommand{\eqdef}{\ensuremath{\overset{\text{def}}=}}


\def\Circlearrowright{\ensuremath{%
  \rotatebox[origin=c]{230}{$\circlearrowright$}}}

\newcommand\ct[1]{\text{\rmfamily\upshape #1}}
\newcommand\question[1]{ {\color{red} ...!? \small #1}}
\newcommand\caz[1]{\left\{\begin{array} #1 \end{array}\right.}
\newcommand\const{\text{\rmfamily\upshape const}}
\newcommand\toP{ \overset{\pro}{\to}}
\newcommand\toPP{ \overset{\text{PP}}{\to}}
\newcommand{\oeq}{\mathrel{\text{\textcircled{$=$}}}}





\usepackage{xcolor}
% \usepackage[normalem]{ulem}
\usepackage{lipsum}
\makeatletter
% \newcommand\colorwave[1][blue]{\bgroup \markoverwith{\lower3.5\p@\hbox{\sixly \textcolor{#1}{\char58}}}\ULon}
%\font\sixly=lasy6 % does not re-load if already loaded, so no memory problem.

\newmdtheoremenv[
linewidth= 1pt,linecolor= blue,%
leftmargin=20,rightmargin=20,innertopmargin=0pt, innerrightmargin=40,%
tikzsetting = { draw=lightgray, line width = 0.3pt,dashed,%
dash pattern = on 15pt off 3pt},%
splittopskip=\topskip,skipbelow=\baselineskip,%
skipabove=\baselineskip,ntheorem,roundcorner=0pt,
% backgroundcolor=pagebg,font=\color{orange}\sffamily, fontcolor=white
]{examplebox}{Exemple}[section]



\newcommand\R{\mathbb{R}}
\newcommand\Z{\mathbb{Z}}
\newcommand\N{\mathbb{N}}
\newcommand\E{\mathbb{E}}
\newcommand\F{\mathcal{F}}
\newcommand\cH{\mathcal{H}}
\newcommand\V{\mathbb{V}}
\newcommand\dmo{ ^{-1} }
\newcommand\kapa{\kappa}
\newcommand\im{Im}
\newcommand\hs{\mathcal{H}}





\usepackage{soul}

\makeatletter
\newcommand*{\whiten}[1]{\llap{\textcolor{white}{{\the\SOUL@token}}\hspace{#1pt}}}
\DeclareRobustCommand*\myul{%
    \def\SOUL@everyspace{\underline{\space}\kern\z@}%
    \def\SOUL@everytoken{%
     \setbox0=\hbox{\the\SOUL@token}%
     \ifdim\dp0>\z@
        \raisebox{\dp0}{\underline{\phantom{\the\SOUL@token}}}%
        \whiten{1}\whiten{0}%
        \whiten{-1}\whiten{-2}%
        \llap{\the\SOUL@token}%
     \else
        \underline{\the\SOUL@token}%
     \fi}%
\SOUL@}
\makeatother

\newcommand*{\demp}{\fontfamily{lmtt}\selectfont}

\DeclareTextFontCommand{\textdemp}{\demp}

\begin{document}

\ifcomment
Multiline
comment
\fi
\ifcomment
\myul{Typesetting test}
% \color[rgb]{1,1,1}
$∑_i^n≠ 60º±∞π∆¬≈√j∫h≤≥µ$

$\CR \R\pro\ind\pro\gS\pro
\mqty[a&b\\c&d]$
$\pro\mathbb{P}$
$\dd{x}$

  \[
    \alpha(x)=\left\{
                \begin{array}{ll}
                  x\\
                  \frac{1}{1+e^{-kx}}\\
                  \frac{e^x-e^{-x}}{e^x+e^{-x}}
                \end{array}
              \right.
  \]

  $\expval{x}$
  
  $\chi_\rho(ghg\dmo)=\Tr(\rho_{ghg\dmo})=\Tr(\rho_g\circ\rho_h\circ\rho\dmo_g)=\Tr(\rho_h)\overset{\mbox{\scalebox{0.5}{$\Tr(AB)=\Tr(BA)$}}}{=}\chi_\rho(h)$
  	$\mathop{\oplus}_{\substack{x\in X}}$

$\mat(\rho_g)=(a_{ij}(g))_{\scriptsize \substack{1\leq i\leq d \\ 1\leq j\leq d}}$ et $\mat(\rho'_g)=(a'_{ij}(g))_{\scriptsize \substack{1\leq i'\leq d' \\ 1\leq j'\leq d'}}$



\[\int_a^b{\mathbb{R}^2}g(u, v)\dd{P_{XY}}(u, v)=\iint g(u,v) f_{XY}(u, v)\dd \lambda(u) \dd \lambda(v)\]
$$\lim_{x\to\infty} f(x)$$	
$$\iiiint_V \mu(t,u,v,w) \,dt\,du\,dv\,dw$$
$$\sum_{n=1}^{\infty} 2^{-n} = 1$$	
\begin{definition}
	Si $X$ et $Y$ sont 2 v.a. ou definit la \textsc{Covariance} entre $X$ et $Y$ comme
	$\cov(X,Y)\overset{\text{def}}{=}\E\left[(X-\E(X))(Y-\E(Y))\right]=\E(XY)-\E(X)\E(Y)$.
\end{definition}
\fi
\pagebreak

% \tableofcontents

% insert your code here
%\input{./algebra/main.tex}
%\input{./geometrie-differentielle/main.tex}
%\input{./probabilite/main.tex}
%\input{./analyse-fonctionnelle/main.tex}
% \input{./Analyse-convexe-et-dualite-en-optimisation/main.tex}
%\input{./tikz/main.tex}
%\input{./Theorie-du-distributions/main.tex}
%\input{./optimisation/mine.tex}
 \input{./modelisation/main.tex}

% yves.aubry@univ-tln.fr : algebra

\end{document}

%% !TEX encoding = UTF-8 Unicode
% !TEX TS-program = xelatex

\documentclass[french]{report}

%\usepackage[utf8]{inputenc}
%\usepackage[T1]{fontenc}
\usepackage{babel}


\newif\ifcomment
%\commenttrue # Show comments

\usepackage{physics}
\usepackage{amssymb}


\usepackage{amsthm}
% \usepackage{thmtools}
\usepackage{mathtools}
\usepackage{amsfonts}

\usepackage{color}

\usepackage{tikz}

\usepackage{geometry}
\geometry{a5paper, margin=0.1in, right=1cm}

\usepackage{dsfont}

\usepackage{graphicx}
\graphicspath{ {images/} }

\usepackage{faktor}

\usepackage{IEEEtrantools}
\usepackage{enumerate}   
\usepackage[PostScript=dvips]{"/Users/aware/Documents/Courses/diagrams"}


\newtheorem{theorem}{Théorème}[section]
\renewcommand{\thetheorem}{\arabic{theorem}}
\newtheorem{lemme}{Lemme}[section]
\renewcommand{\thelemme}{\arabic{lemme}}
\newtheorem{proposition}{Proposition}[section]
\renewcommand{\theproposition}{\arabic{proposition}}
\newtheorem{notations}{Notations}[section]
\newtheorem{problem}{Problème}[section]
\newtheorem{corollary}{Corollaire}[theorem]
\renewcommand{\thecorollary}{\arabic{corollary}}
\newtheorem{property}{Propriété}[section]
\newtheorem{objective}{Objectif}[section]

\theoremstyle{definition}
\newtheorem{definition}{Définition}[section]
\renewcommand{\thedefinition}{\arabic{definition}}
\newtheorem{exercise}{Exercice}[chapter]
\renewcommand{\theexercise}{\arabic{exercise}}
\newtheorem{example}{Exemple}[chapter]
\renewcommand{\theexample}{\arabic{example}}
\newtheorem*{solution}{Solution}
\newtheorem*{application}{Application}
\newtheorem*{notation}{Notation}
\newtheorem*{vocabulary}{Vocabulaire}
\newtheorem*{properties}{Propriétés}



\theoremstyle{remark}
\newtheorem*{remark}{Remarque}
\newtheorem*{rappel}{Rappel}


\usepackage{etoolbox}
\AtBeginEnvironment{exercise}{\small}
\AtBeginEnvironment{example}{\small}

\usepackage{cases}
\usepackage[red]{mypack}

\usepackage[framemethod=TikZ]{mdframed}

\definecolor{bg}{rgb}{0.4,0.25,0.95}
\definecolor{pagebg}{rgb}{0,0,0.5}
\surroundwithmdframed[
   topline=false,
   rightline=false,
   bottomline=false,
   leftmargin=\parindent,
   skipabove=8pt,
   skipbelow=8pt,
   linecolor=blue,
   innerbottommargin=10pt,
   % backgroundcolor=bg,font=\color{orange}\sffamily, fontcolor=white
]{definition}

\usepackage{empheq}
\usepackage[most]{tcolorbox}

\newtcbox{\mymath}[1][]{%
    nobeforeafter, math upper, tcbox raise base,
    enhanced, colframe=blue!30!black,
    colback=red!10, boxrule=1pt,
    #1}

\usepackage{unixode}


\DeclareMathOperator{\ord}{ord}
\DeclareMathOperator{\orb}{orb}
\DeclareMathOperator{\stab}{stab}
\DeclareMathOperator{\Stab}{stab}
\DeclareMathOperator{\ppcm}{ppcm}
\DeclareMathOperator{\conj}{Conj}
\DeclareMathOperator{\End}{End}
\DeclareMathOperator{\rot}{rot}
\DeclareMathOperator{\trs}{trace}
\DeclareMathOperator{\Ind}{Ind}
\DeclareMathOperator{\mat}{Mat}
\DeclareMathOperator{\id}{Id}
\DeclareMathOperator{\vect}{vect}
\DeclareMathOperator{\img}{img}
\DeclareMathOperator{\cov}{Cov}
\DeclareMathOperator{\dist}{dist}
\DeclareMathOperator{\irr}{Irr}
\DeclareMathOperator{\image}{Im}
\DeclareMathOperator{\pd}{\partial}
\DeclareMathOperator{\epi}{epi}
\DeclareMathOperator{\Argmin}{Argmin}
\DeclareMathOperator{\dom}{dom}
\DeclareMathOperator{\proj}{proj}
\DeclareMathOperator{\ctg}{ctg}
\DeclareMathOperator{\supp}{supp}
\DeclareMathOperator{\argmin}{argmin}
\DeclareMathOperator{\mult}{mult}
\DeclareMathOperator{\ch}{ch}
\DeclareMathOperator{\sh}{sh}
\DeclareMathOperator{\rang}{rang}
\DeclareMathOperator{\diam}{diam}
\DeclareMathOperator{\Epigraphe}{Epigraphe}




\usepackage{xcolor}
\everymath{\color{blue}}
%\everymath{\color[rgb]{0,1,1}}
%\pagecolor[rgb]{0,0,0.5}


\newcommand*{\pdtest}[3][]{\ensuremath{\frac{\partial^{#1} #2}{\partial #3}}}

\newcommand*{\deffunc}[6][]{\ensuremath{
\begin{array}{rcl}
#2 : #3 &\rightarrow& #4\\
#5 &\mapsto& #6
\end{array}
}}

\newcommand{\eqcolon}{\mathrel{\resizebox{\widthof{$\mathord{=}$}}{\height}{ $\!\!=\!\!\resizebox{1.2\width}{0.8\height}{\raisebox{0.23ex}{$\mathop{:}$}}\!\!$ }}}
\newcommand{\coloneq}{\mathrel{\resizebox{\widthof{$\mathord{=}$}}{\height}{ $\!\!\resizebox{1.2\width}{0.8\height}{\raisebox{0.23ex}{$\mathop{:}$}}\!\!=\!\!$ }}}
\newcommand{\eqcolonl}{\ensuremath{\mathrel{=\!\!\mathop{:}}}}
\newcommand{\coloneql}{\ensuremath{\mathrel{\mathop{:} \!\! =}}}
\newcommand{\vc}[1]{% inline column vector
  \left(\begin{smallmatrix}#1\end{smallmatrix}\right)%
}
\newcommand{\vr}[1]{% inline row vector
  \begin{smallmatrix}(\,#1\,)\end{smallmatrix}%
}
\makeatletter
\newcommand*{\defeq}{\ =\mathrel{\rlap{%
                     \raisebox{0.3ex}{$\m@th\cdot$}}%
                     \raisebox{-0.3ex}{$\m@th\cdot$}}%
                     }
\makeatother

\newcommand{\mathcircle}[1]{% inline row vector
 \overset{\circ}{#1}
}
\newcommand{\ulim}{% low limit
 \underline{\lim}
}
\newcommand{\ssi}{% iff
\iff
}
\newcommand{\ps}[2]{
\expval{#1 | #2}
}
\newcommand{\df}[1]{
\mqty{#1}
}
\newcommand{\n}[1]{
\norm{#1}
}
\newcommand{\sys}[1]{
\left\{\smqty{#1}\right.
}


\newcommand{\eqdef}{\ensuremath{\overset{\text{def}}=}}


\def\Circlearrowright{\ensuremath{%
  \rotatebox[origin=c]{230}{$\circlearrowright$}}}

\newcommand\ct[1]{\text{\rmfamily\upshape #1}}
\newcommand\question[1]{ {\color{red} ...!? \small #1}}
\newcommand\caz[1]{\left\{\begin{array} #1 \end{array}\right.}
\newcommand\const{\text{\rmfamily\upshape const}}
\newcommand\toP{ \overset{\pro}{\to}}
\newcommand\toPP{ \overset{\text{PP}}{\to}}
\newcommand{\oeq}{\mathrel{\text{\textcircled{$=$}}}}





\usepackage{xcolor}
% \usepackage[normalem]{ulem}
\usepackage{lipsum}
\makeatletter
% \newcommand\colorwave[1][blue]{\bgroup \markoverwith{\lower3.5\p@\hbox{\sixly \textcolor{#1}{\char58}}}\ULon}
%\font\sixly=lasy6 % does not re-load if already loaded, so no memory problem.

\newmdtheoremenv[
linewidth= 1pt,linecolor= blue,%
leftmargin=20,rightmargin=20,innertopmargin=0pt, innerrightmargin=40,%
tikzsetting = { draw=lightgray, line width = 0.3pt,dashed,%
dash pattern = on 15pt off 3pt},%
splittopskip=\topskip,skipbelow=\baselineskip,%
skipabove=\baselineskip,ntheorem,roundcorner=0pt,
% backgroundcolor=pagebg,font=\color{orange}\sffamily, fontcolor=white
]{examplebox}{Exemple}[section]



\newcommand\R{\mathbb{R}}
\newcommand\Z{\mathbb{Z}}
\newcommand\N{\mathbb{N}}
\newcommand\E{\mathbb{E}}
\newcommand\F{\mathcal{F}}
\newcommand\cH{\mathcal{H}}
\newcommand\V{\mathbb{V}}
\newcommand\dmo{ ^{-1} }
\newcommand\kapa{\kappa}
\newcommand\im{Im}
\newcommand\hs{\mathcal{H}}





\usepackage{soul}

\makeatletter
\newcommand*{\whiten}[1]{\llap{\textcolor{white}{{\the\SOUL@token}}\hspace{#1pt}}}
\DeclareRobustCommand*\myul{%
    \def\SOUL@everyspace{\underline{\space}\kern\z@}%
    \def\SOUL@everytoken{%
     \setbox0=\hbox{\the\SOUL@token}%
     \ifdim\dp0>\z@
        \raisebox{\dp0}{\underline{\phantom{\the\SOUL@token}}}%
        \whiten{1}\whiten{0}%
        \whiten{-1}\whiten{-2}%
        \llap{\the\SOUL@token}%
     \else
        \underline{\the\SOUL@token}%
     \fi}%
\SOUL@}
\makeatother

\newcommand*{\demp}{\fontfamily{lmtt}\selectfont}

\DeclareTextFontCommand{\textdemp}{\demp}

\begin{document}

\ifcomment
Multiline
comment
\fi
\ifcomment
\myul{Typesetting test}
% \color[rgb]{1,1,1}
$∑_i^n≠ 60º±∞π∆¬≈√j∫h≤≥µ$

$\CR \R\pro\ind\pro\gS\pro
\mqty[a&b\\c&d]$
$\pro\mathbb{P}$
$\dd{x}$

  \[
    \alpha(x)=\left\{
                \begin{array}{ll}
                  x\\
                  \frac{1}{1+e^{-kx}}\\
                  \frac{e^x-e^{-x}}{e^x+e^{-x}}
                \end{array}
              \right.
  \]

  $\expval{x}$
  
  $\chi_\rho(ghg\dmo)=\Tr(\rho_{ghg\dmo})=\Tr(\rho_g\circ\rho_h\circ\rho\dmo_g)=\Tr(\rho_h)\overset{\mbox{\scalebox{0.5}{$\Tr(AB)=\Tr(BA)$}}}{=}\chi_\rho(h)$
  	$\mathop{\oplus}_{\substack{x\in X}}$

$\mat(\rho_g)=(a_{ij}(g))_{\scriptsize \substack{1\leq i\leq d \\ 1\leq j\leq d}}$ et $\mat(\rho'_g)=(a'_{ij}(g))_{\scriptsize \substack{1\leq i'\leq d' \\ 1\leq j'\leq d'}}$



\[\int_a^b{\mathbb{R}^2}g(u, v)\dd{P_{XY}}(u, v)=\iint g(u,v) f_{XY}(u, v)\dd \lambda(u) \dd \lambda(v)\]
$$\lim_{x\to\infty} f(x)$$	
$$\iiiint_V \mu(t,u,v,w) \,dt\,du\,dv\,dw$$
$$\sum_{n=1}^{\infty} 2^{-n} = 1$$	
\begin{definition}
	Si $X$ et $Y$ sont 2 v.a. ou definit la \textsc{Covariance} entre $X$ et $Y$ comme
	$\cov(X,Y)\overset{\text{def}}{=}\E\left[(X-\E(X))(Y-\E(Y))\right]=\E(XY)-\E(X)\E(Y)$.
\end{definition}
\fi
\pagebreak

% \tableofcontents

% insert your code here
%\input{./algebra/main.tex}
%\input{./geometrie-differentielle/main.tex}
%\input{./probabilite/main.tex}
%\input{./analyse-fonctionnelle/main.tex}
% \input{./Analyse-convexe-et-dualite-en-optimisation/main.tex}
%\input{./tikz/main.tex}
%\input{./Theorie-du-distributions/main.tex}
%\input{./optimisation/mine.tex}
 \input{./modelisation/main.tex}

% yves.aubry@univ-tln.fr : algebra

\end{document}

%% !TEX encoding = UTF-8 Unicode
% !TEX TS-program = xelatex

\documentclass[french]{report}

%\usepackage[utf8]{inputenc}
%\usepackage[T1]{fontenc}
\usepackage{babel}


\newif\ifcomment
%\commenttrue # Show comments

\usepackage{physics}
\usepackage{amssymb}


\usepackage{amsthm}
% \usepackage{thmtools}
\usepackage{mathtools}
\usepackage{amsfonts}

\usepackage{color}

\usepackage{tikz}

\usepackage{geometry}
\geometry{a5paper, margin=0.1in, right=1cm}

\usepackage{dsfont}

\usepackage{graphicx}
\graphicspath{ {images/} }

\usepackage{faktor}

\usepackage{IEEEtrantools}
\usepackage{enumerate}   
\usepackage[PostScript=dvips]{"/Users/aware/Documents/Courses/diagrams"}


\newtheorem{theorem}{Théorème}[section]
\renewcommand{\thetheorem}{\arabic{theorem}}
\newtheorem{lemme}{Lemme}[section]
\renewcommand{\thelemme}{\arabic{lemme}}
\newtheorem{proposition}{Proposition}[section]
\renewcommand{\theproposition}{\arabic{proposition}}
\newtheorem{notations}{Notations}[section]
\newtheorem{problem}{Problème}[section]
\newtheorem{corollary}{Corollaire}[theorem]
\renewcommand{\thecorollary}{\arabic{corollary}}
\newtheorem{property}{Propriété}[section]
\newtheorem{objective}{Objectif}[section]

\theoremstyle{definition}
\newtheorem{definition}{Définition}[section]
\renewcommand{\thedefinition}{\arabic{definition}}
\newtheorem{exercise}{Exercice}[chapter]
\renewcommand{\theexercise}{\arabic{exercise}}
\newtheorem{example}{Exemple}[chapter]
\renewcommand{\theexample}{\arabic{example}}
\newtheorem*{solution}{Solution}
\newtheorem*{application}{Application}
\newtheorem*{notation}{Notation}
\newtheorem*{vocabulary}{Vocabulaire}
\newtheorem*{properties}{Propriétés}



\theoremstyle{remark}
\newtheorem*{remark}{Remarque}
\newtheorem*{rappel}{Rappel}


\usepackage{etoolbox}
\AtBeginEnvironment{exercise}{\small}
\AtBeginEnvironment{example}{\small}

\usepackage{cases}
\usepackage[red]{mypack}

\usepackage[framemethod=TikZ]{mdframed}

\definecolor{bg}{rgb}{0.4,0.25,0.95}
\definecolor{pagebg}{rgb}{0,0,0.5}
\surroundwithmdframed[
   topline=false,
   rightline=false,
   bottomline=false,
   leftmargin=\parindent,
   skipabove=8pt,
   skipbelow=8pt,
   linecolor=blue,
   innerbottommargin=10pt,
   % backgroundcolor=bg,font=\color{orange}\sffamily, fontcolor=white
]{definition}

\usepackage{empheq}
\usepackage[most]{tcolorbox}

\newtcbox{\mymath}[1][]{%
    nobeforeafter, math upper, tcbox raise base,
    enhanced, colframe=blue!30!black,
    colback=red!10, boxrule=1pt,
    #1}

\usepackage{unixode}


\DeclareMathOperator{\ord}{ord}
\DeclareMathOperator{\orb}{orb}
\DeclareMathOperator{\stab}{stab}
\DeclareMathOperator{\Stab}{stab}
\DeclareMathOperator{\ppcm}{ppcm}
\DeclareMathOperator{\conj}{Conj}
\DeclareMathOperator{\End}{End}
\DeclareMathOperator{\rot}{rot}
\DeclareMathOperator{\trs}{trace}
\DeclareMathOperator{\Ind}{Ind}
\DeclareMathOperator{\mat}{Mat}
\DeclareMathOperator{\id}{Id}
\DeclareMathOperator{\vect}{vect}
\DeclareMathOperator{\img}{img}
\DeclareMathOperator{\cov}{Cov}
\DeclareMathOperator{\dist}{dist}
\DeclareMathOperator{\irr}{Irr}
\DeclareMathOperator{\image}{Im}
\DeclareMathOperator{\pd}{\partial}
\DeclareMathOperator{\epi}{epi}
\DeclareMathOperator{\Argmin}{Argmin}
\DeclareMathOperator{\dom}{dom}
\DeclareMathOperator{\proj}{proj}
\DeclareMathOperator{\ctg}{ctg}
\DeclareMathOperator{\supp}{supp}
\DeclareMathOperator{\argmin}{argmin}
\DeclareMathOperator{\mult}{mult}
\DeclareMathOperator{\ch}{ch}
\DeclareMathOperator{\sh}{sh}
\DeclareMathOperator{\rang}{rang}
\DeclareMathOperator{\diam}{diam}
\DeclareMathOperator{\Epigraphe}{Epigraphe}




\usepackage{xcolor}
\everymath{\color{blue}}
%\everymath{\color[rgb]{0,1,1}}
%\pagecolor[rgb]{0,0,0.5}


\newcommand*{\pdtest}[3][]{\ensuremath{\frac{\partial^{#1} #2}{\partial #3}}}

\newcommand*{\deffunc}[6][]{\ensuremath{
\begin{array}{rcl}
#2 : #3 &\rightarrow& #4\\
#5 &\mapsto& #6
\end{array}
}}

\newcommand{\eqcolon}{\mathrel{\resizebox{\widthof{$\mathord{=}$}}{\height}{ $\!\!=\!\!\resizebox{1.2\width}{0.8\height}{\raisebox{0.23ex}{$\mathop{:}$}}\!\!$ }}}
\newcommand{\coloneq}{\mathrel{\resizebox{\widthof{$\mathord{=}$}}{\height}{ $\!\!\resizebox{1.2\width}{0.8\height}{\raisebox{0.23ex}{$\mathop{:}$}}\!\!=\!\!$ }}}
\newcommand{\eqcolonl}{\ensuremath{\mathrel{=\!\!\mathop{:}}}}
\newcommand{\coloneql}{\ensuremath{\mathrel{\mathop{:} \!\! =}}}
\newcommand{\vc}[1]{% inline column vector
  \left(\begin{smallmatrix}#1\end{smallmatrix}\right)%
}
\newcommand{\vr}[1]{% inline row vector
  \begin{smallmatrix}(\,#1\,)\end{smallmatrix}%
}
\makeatletter
\newcommand*{\defeq}{\ =\mathrel{\rlap{%
                     \raisebox{0.3ex}{$\m@th\cdot$}}%
                     \raisebox{-0.3ex}{$\m@th\cdot$}}%
                     }
\makeatother

\newcommand{\mathcircle}[1]{% inline row vector
 \overset{\circ}{#1}
}
\newcommand{\ulim}{% low limit
 \underline{\lim}
}
\newcommand{\ssi}{% iff
\iff
}
\newcommand{\ps}[2]{
\expval{#1 | #2}
}
\newcommand{\df}[1]{
\mqty{#1}
}
\newcommand{\n}[1]{
\norm{#1}
}
\newcommand{\sys}[1]{
\left\{\smqty{#1}\right.
}


\newcommand{\eqdef}{\ensuremath{\overset{\text{def}}=}}


\def\Circlearrowright{\ensuremath{%
  \rotatebox[origin=c]{230}{$\circlearrowright$}}}

\newcommand\ct[1]{\text{\rmfamily\upshape #1}}
\newcommand\question[1]{ {\color{red} ...!? \small #1}}
\newcommand\caz[1]{\left\{\begin{array} #1 \end{array}\right.}
\newcommand\const{\text{\rmfamily\upshape const}}
\newcommand\toP{ \overset{\pro}{\to}}
\newcommand\toPP{ \overset{\text{PP}}{\to}}
\newcommand{\oeq}{\mathrel{\text{\textcircled{$=$}}}}





\usepackage{xcolor}
% \usepackage[normalem]{ulem}
\usepackage{lipsum}
\makeatletter
% \newcommand\colorwave[1][blue]{\bgroup \markoverwith{\lower3.5\p@\hbox{\sixly \textcolor{#1}{\char58}}}\ULon}
%\font\sixly=lasy6 % does not re-load if already loaded, so no memory problem.

\newmdtheoremenv[
linewidth= 1pt,linecolor= blue,%
leftmargin=20,rightmargin=20,innertopmargin=0pt, innerrightmargin=40,%
tikzsetting = { draw=lightgray, line width = 0.3pt,dashed,%
dash pattern = on 15pt off 3pt},%
splittopskip=\topskip,skipbelow=\baselineskip,%
skipabove=\baselineskip,ntheorem,roundcorner=0pt,
% backgroundcolor=pagebg,font=\color{orange}\sffamily, fontcolor=white
]{examplebox}{Exemple}[section]



\newcommand\R{\mathbb{R}}
\newcommand\Z{\mathbb{Z}}
\newcommand\N{\mathbb{N}}
\newcommand\E{\mathbb{E}}
\newcommand\F{\mathcal{F}}
\newcommand\cH{\mathcal{H}}
\newcommand\V{\mathbb{V}}
\newcommand\dmo{ ^{-1} }
\newcommand\kapa{\kappa}
\newcommand\im{Im}
\newcommand\hs{\mathcal{H}}





\usepackage{soul}

\makeatletter
\newcommand*{\whiten}[1]{\llap{\textcolor{white}{{\the\SOUL@token}}\hspace{#1pt}}}
\DeclareRobustCommand*\myul{%
    \def\SOUL@everyspace{\underline{\space}\kern\z@}%
    \def\SOUL@everytoken{%
     \setbox0=\hbox{\the\SOUL@token}%
     \ifdim\dp0>\z@
        \raisebox{\dp0}{\underline{\phantom{\the\SOUL@token}}}%
        \whiten{1}\whiten{0}%
        \whiten{-1}\whiten{-2}%
        \llap{\the\SOUL@token}%
     \else
        \underline{\the\SOUL@token}%
     \fi}%
\SOUL@}
\makeatother

\newcommand*{\demp}{\fontfamily{lmtt}\selectfont}

\DeclareTextFontCommand{\textdemp}{\demp}

\begin{document}

\ifcomment
Multiline
comment
\fi
\ifcomment
\myul{Typesetting test}
% \color[rgb]{1,1,1}
$∑_i^n≠ 60º±∞π∆¬≈√j∫h≤≥µ$

$\CR \R\pro\ind\pro\gS\pro
\mqty[a&b\\c&d]$
$\pro\mathbb{P}$
$\dd{x}$

  \[
    \alpha(x)=\left\{
                \begin{array}{ll}
                  x\\
                  \frac{1}{1+e^{-kx}}\\
                  \frac{e^x-e^{-x}}{e^x+e^{-x}}
                \end{array}
              \right.
  \]

  $\expval{x}$
  
  $\chi_\rho(ghg\dmo)=\Tr(\rho_{ghg\dmo})=\Tr(\rho_g\circ\rho_h\circ\rho\dmo_g)=\Tr(\rho_h)\overset{\mbox{\scalebox{0.5}{$\Tr(AB)=\Tr(BA)$}}}{=}\chi_\rho(h)$
  	$\mathop{\oplus}_{\substack{x\in X}}$

$\mat(\rho_g)=(a_{ij}(g))_{\scriptsize \substack{1\leq i\leq d \\ 1\leq j\leq d}}$ et $\mat(\rho'_g)=(a'_{ij}(g))_{\scriptsize \substack{1\leq i'\leq d' \\ 1\leq j'\leq d'}}$



\[\int_a^b{\mathbb{R}^2}g(u, v)\dd{P_{XY}}(u, v)=\iint g(u,v) f_{XY}(u, v)\dd \lambda(u) \dd \lambda(v)\]
$$\lim_{x\to\infty} f(x)$$	
$$\iiiint_V \mu(t,u,v,w) \,dt\,du\,dv\,dw$$
$$\sum_{n=1}^{\infty} 2^{-n} = 1$$	
\begin{definition}
	Si $X$ et $Y$ sont 2 v.a. ou definit la \textsc{Covariance} entre $X$ et $Y$ comme
	$\cov(X,Y)\overset{\text{def}}{=}\E\left[(X-\E(X))(Y-\E(Y))\right]=\E(XY)-\E(X)\E(Y)$.
\end{definition}
\fi
\pagebreak

% \tableofcontents

% insert your code here
%\input{./algebra/main.tex}
%\input{./geometrie-differentielle/main.tex}
%\input{./probabilite/main.tex}
%\input{./analyse-fonctionnelle/main.tex}
% \input{./Analyse-convexe-et-dualite-en-optimisation/main.tex}
%\input{./tikz/main.tex}
%\input{./Theorie-du-distributions/main.tex}
%\input{./optimisation/mine.tex}
 \input{./modelisation/main.tex}

% yves.aubry@univ-tln.fr : algebra

\end{document}

%% !TEX encoding = UTF-8 Unicode
% !TEX TS-program = xelatex

\documentclass[french]{report}

%\usepackage[utf8]{inputenc}
%\usepackage[T1]{fontenc}
\usepackage{babel}


\newif\ifcomment
%\commenttrue # Show comments

\usepackage{physics}
\usepackage{amssymb}


\usepackage{amsthm}
% \usepackage{thmtools}
\usepackage{mathtools}
\usepackage{amsfonts}

\usepackage{color}

\usepackage{tikz}

\usepackage{geometry}
\geometry{a5paper, margin=0.1in, right=1cm}

\usepackage{dsfont}

\usepackage{graphicx}
\graphicspath{ {images/} }

\usepackage{faktor}

\usepackage{IEEEtrantools}
\usepackage{enumerate}   
\usepackage[PostScript=dvips]{"/Users/aware/Documents/Courses/diagrams"}


\newtheorem{theorem}{Théorème}[section]
\renewcommand{\thetheorem}{\arabic{theorem}}
\newtheorem{lemme}{Lemme}[section]
\renewcommand{\thelemme}{\arabic{lemme}}
\newtheorem{proposition}{Proposition}[section]
\renewcommand{\theproposition}{\arabic{proposition}}
\newtheorem{notations}{Notations}[section]
\newtheorem{problem}{Problème}[section]
\newtheorem{corollary}{Corollaire}[theorem]
\renewcommand{\thecorollary}{\arabic{corollary}}
\newtheorem{property}{Propriété}[section]
\newtheorem{objective}{Objectif}[section]

\theoremstyle{definition}
\newtheorem{definition}{Définition}[section]
\renewcommand{\thedefinition}{\arabic{definition}}
\newtheorem{exercise}{Exercice}[chapter]
\renewcommand{\theexercise}{\arabic{exercise}}
\newtheorem{example}{Exemple}[chapter]
\renewcommand{\theexample}{\arabic{example}}
\newtheorem*{solution}{Solution}
\newtheorem*{application}{Application}
\newtheorem*{notation}{Notation}
\newtheorem*{vocabulary}{Vocabulaire}
\newtheorem*{properties}{Propriétés}



\theoremstyle{remark}
\newtheorem*{remark}{Remarque}
\newtheorem*{rappel}{Rappel}


\usepackage{etoolbox}
\AtBeginEnvironment{exercise}{\small}
\AtBeginEnvironment{example}{\small}

\usepackage{cases}
\usepackage[red]{mypack}

\usepackage[framemethod=TikZ]{mdframed}

\definecolor{bg}{rgb}{0.4,0.25,0.95}
\definecolor{pagebg}{rgb}{0,0,0.5}
\surroundwithmdframed[
   topline=false,
   rightline=false,
   bottomline=false,
   leftmargin=\parindent,
   skipabove=8pt,
   skipbelow=8pt,
   linecolor=blue,
   innerbottommargin=10pt,
   % backgroundcolor=bg,font=\color{orange}\sffamily, fontcolor=white
]{definition}

\usepackage{empheq}
\usepackage[most]{tcolorbox}

\newtcbox{\mymath}[1][]{%
    nobeforeafter, math upper, tcbox raise base,
    enhanced, colframe=blue!30!black,
    colback=red!10, boxrule=1pt,
    #1}

\usepackage{unixode}


\DeclareMathOperator{\ord}{ord}
\DeclareMathOperator{\orb}{orb}
\DeclareMathOperator{\stab}{stab}
\DeclareMathOperator{\Stab}{stab}
\DeclareMathOperator{\ppcm}{ppcm}
\DeclareMathOperator{\conj}{Conj}
\DeclareMathOperator{\End}{End}
\DeclareMathOperator{\rot}{rot}
\DeclareMathOperator{\trs}{trace}
\DeclareMathOperator{\Ind}{Ind}
\DeclareMathOperator{\mat}{Mat}
\DeclareMathOperator{\id}{Id}
\DeclareMathOperator{\vect}{vect}
\DeclareMathOperator{\img}{img}
\DeclareMathOperator{\cov}{Cov}
\DeclareMathOperator{\dist}{dist}
\DeclareMathOperator{\irr}{Irr}
\DeclareMathOperator{\image}{Im}
\DeclareMathOperator{\pd}{\partial}
\DeclareMathOperator{\epi}{epi}
\DeclareMathOperator{\Argmin}{Argmin}
\DeclareMathOperator{\dom}{dom}
\DeclareMathOperator{\proj}{proj}
\DeclareMathOperator{\ctg}{ctg}
\DeclareMathOperator{\supp}{supp}
\DeclareMathOperator{\argmin}{argmin}
\DeclareMathOperator{\mult}{mult}
\DeclareMathOperator{\ch}{ch}
\DeclareMathOperator{\sh}{sh}
\DeclareMathOperator{\rang}{rang}
\DeclareMathOperator{\diam}{diam}
\DeclareMathOperator{\Epigraphe}{Epigraphe}




\usepackage{xcolor}
\everymath{\color{blue}}
%\everymath{\color[rgb]{0,1,1}}
%\pagecolor[rgb]{0,0,0.5}


\newcommand*{\pdtest}[3][]{\ensuremath{\frac{\partial^{#1} #2}{\partial #3}}}

\newcommand*{\deffunc}[6][]{\ensuremath{
\begin{array}{rcl}
#2 : #3 &\rightarrow& #4\\
#5 &\mapsto& #6
\end{array}
}}

\newcommand{\eqcolon}{\mathrel{\resizebox{\widthof{$\mathord{=}$}}{\height}{ $\!\!=\!\!\resizebox{1.2\width}{0.8\height}{\raisebox{0.23ex}{$\mathop{:}$}}\!\!$ }}}
\newcommand{\coloneq}{\mathrel{\resizebox{\widthof{$\mathord{=}$}}{\height}{ $\!\!\resizebox{1.2\width}{0.8\height}{\raisebox{0.23ex}{$\mathop{:}$}}\!\!=\!\!$ }}}
\newcommand{\eqcolonl}{\ensuremath{\mathrel{=\!\!\mathop{:}}}}
\newcommand{\coloneql}{\ensuremath{\mathrel{\mathop{:} \!\! =}}}
\newcommand{\vc}[1]{% inline column vector
  \left(\begin{smallmatrix}#1\end{smallmatrix}\right)%
}
\newcommand{\vr}[1]{% inline row vector
  \begin{smallmatrix}(\,#1\,)\end{smallmatrix}%
}
\makeatletter
\newcommand*{\defeq}{\ =\mathrel{\rlap{%
                     \raisebox{0.3ex}{$\m@th\cdot$}}%
                     \raisebox{-0.3ex}{$\m@th\cdot$}}%
                     }
\makeatother

\newcommand{\mathcircle}[1]{% inline row vector
 \overset{\circ}{#1}
}
\newcommand{\ulim}{% low limit
 \underline{\lim}
}
\newcommand{\ssi}{% iff
\iff
}
\newcommand{\ps}[2]{
\expval{#1 | #2}
}
\newcommand{\df}[1]{
\mqty{#1}
}
\newcommand{\n}[1]{
\norm{#1}
}
\newcommand{\sys}[1]{
\left\{\smqty{#1}\right.
}


\newcommand{\eqdef}{\ensuremath{\overset{\text{def}}=}}


\def\Circlearrowright{\ensuremath{%
  \rotatebox[origin=c]{230}{$\circlearrowright$}}}

\newcommand\ct[1]{\text{\rmfamily\upshape #1}}
\newcommand\question[1]{ {\color{red} ...!? \small #1}}
\newcommand\caz[1]{\left\{\begin{array} #1 \end{array}\right.}
\newcommand\const{\text{\rmfamily\upshape const}}
\newcommand\toP{ \overset{\pro}{\to}}
\newcommand\toPP{ \overset{\text{PP}}{\to}}
\newcommand{\oeq}{\mathrel{\text{\textcircled{$=$}}}}





\usepackage{xcolor}
% \usepackage[normalem]{ulem}
\usepackage{lipsum}
\makeatletter
% \newcommand\colorwave[1][blue]{\bgroup \markoverwith{\lower3.5\p@\hbox{\sixly \textcolor{#1}{\char58}}}\ULon}
%\font\sixly=lasy6 % does not re-load if already loaded, so no memory problem.

\newmdtheoremenv[
linewidth= 1pt,linecolor= blue,%
leftmargin=20,rightmargin=20,innertopmargin=0pt, innerrightmargin=40,%
tikzsetting = { draw=lightgray, line width = 0.3pt,dashed,%
dash pattern = on 15pt off 3pt},%
splittopskip=\topskip,skipbelow=\baselineskip,%
skipabove=\baselineskip,ntheorem,roundcorner=0pt,
% backgroundcolor=pagebg,font=\color{orange}\sffamily, fontcolor=white
]{examplebox}{Exemple}[section]



\newcommand\R{\mathbb{R}}
\newcommand\Z{\mathbb{Z}}
\newcommand\N{\mathbb{N}}
\newcommand\E{\mathbb{E}}
\newcommand\F{\mathcal{F}}
\newcommand\cH{\mathcal{H}}
\newcommand\V{\mathbb{V}}
\newcommand\dmo{ ^{-1} }
\newcommand\kapa{\kappa}
\newcommand\im{Im}
\newcommand\hs{\mathcal{H}}





\usepackage{soul}

\makeatletter
\newcommand*{\whiten}[1]{\llap{\textcolor{white}{{\the\SOUL@token}}\hspace{#1pt}}}
\DeclareRobustCommand*\myul{%
    \def\SOUL@everyspace{\underline{\space}\kern\z@}%
    \def\SOUL@everytoken{%
     \setbox0=\hbox{\the\SOUL@token}%
     \ifdim\dp0>\z@
        \raisebox{\dp0}{\underline{\phantom{\the\SOUL@token}}}%
        \whiten{1}\whiten{0}%
        \whiten{-1}\whiten{-2}%
        \llap{\the\SOUL@token}%
     \else
        \underline{\the\SOUL@token}%
     \fi}%
\SOUL@}
\makeatother

\newcommand*{\demp}{\fontfamily{lmtt}\selectfont}

\DeclareTextFontCommand{\textdemp}{\demp}

\begin{document}

\ifcomment
Multiline
comment
\fi
\ifcomment
\myul{Typesetting test}
% \color[rgb]{1,1,1}
$∑_i^n≠ 60º±∞π∆¬≈√j∫h≤≥µ$

$\CR \R\pro\ind\pro\gS\pro
\mqty[a&b\\c&d]$
$\pro\mathbb{P}$
$\dd{x}$

  \[
    \alpha(x)=\left\{
                \begin{array}{ll}
                  x\\
                  \frac{1}{1+e^{-kx}}\\
                  \frac{e^x-e^{-x}}{e^x+e^{-x}}
                \end{array}
              \right.
  \]

  $\expval{x}$
  
  $\chi_\rho(ghg\dmo)=\Tr(\rho_{ghg\dmo})=\Tr(\rho_g\circ\rho_h\circ\rho\dmo_g)=\Tr(\rho_h)\overset{\mbox{\scalebox{0.5}{$\Tr(AB)=\Tr(BA)$}}}{=}\chi_\rho(h)$
  	$\mathop{\oplus}_{\substack{x\in X}}$

$\mat(\rho_g)=(a_{ij}(g))_{\scriptsize \substack{1\leq i\leq d \\ 1\leq j\leq d}}$ et $\mat(\rho'_g)=(a'_{ij}(g))_{\scriptsize \substack{1\leq i'\leq d' \\ 1\leq j'\leq d'}}$



\[\int_a^b{\mathbb{R}^2}g(u, v)\dd{P_{XY}}(u, v)=\iint g(u,v) f_{XY}(u, v)\dd \lambda(u) \dd \lambda(v)\]
$$\lim_{x\to\infty} f(x)$$	
$$\iiiint_V \mu(t,u,v,w) \,dt\,du\,dv\,dw$$
$$\sum_{n=1}^{\infty} 2^{-n} = 1$$	
\begin{definition}
	Si $X$ et $Y$ sont 2 v.a. ou definit la \textsc{Covariance} entre $X$ et $Y$ comme
	$\cov(X,Y)\overset{\text{def}}{=}\E\left[(X-\E(X))(Y-\E(Y))\right]=\E(XY)-\E(X)\E(Y)$.
\end{definition}
\fi
\pagebreak

% \tableofcontents

% insert your code here
%\input{./algebra/main.tex}
%\input{./geometrie-differentielle/main.tex}
%\input{./probabilite/main.tex}
%\input{./analyse-fonctionnelle/main.tex}
% \input{./Analyse-convexe-et-dualite-en-optimisation/main.tex}
%\input{./tikz/main.tex}
%\input{./Theorie-du-distributions/main.tex}
%\input{./optimisation/mine.tex}
 \input{./modelisation/main.tex}

% yves.aubry@univ-tln.fr : algebra

\end{document}

% % !TEX encoding = UTF-8 Unicode
% !TEX TS-program = xelatex

\documentclass[french]{report}

%\usepackage[utf8]{inputenc}
%\usepackage[T1]{fontenc}
\usepackage{babel}


\newif\ifcomment
%\commenttrue # Show comments

\usepackage{physics}
\usepackage{amssymb}


\usepackage{amsthm}
% \usepackage{thmtools}
\usepackage{mathtools}
\usepackage{amsfonts}

\usepackage{color}

\usepackage{tikz}

\usepackage{geometry}
\geometry{a5paper, margin=0.1in, right=1cm}

\usepackage{dsfont}

\usepackage{graphicx}
\graphicspath{ {images/} }

\usepackage{faktor}

\usepackage{IEEEtrantools}
\usepackage{enumerate}   
\usepackage[PostScript=dvips]{"/Users/aware/Documents/Courses/diagrams"}


\newtheorem{theorem}{Théorème}[section]
\renewcommand{\thetheorem}{\arabic{theorem}}
\newtheorem{lemme}{Lemme}[section]
\renewcommand{\thelemme}{\arabic{lemme}}
\newtheorem{proposition}{Proposition}[section]
\renewcommand{\theproposition}{\arabic{proposition}}
\newtheorem{notations}{Notations}[section]
\newtheorem{problem}{Problème}[section]
\newtheorem{corollary}{Corollaire}[theorem]
\renewcommand{\thecorollary}{\arabic{corollary}}
\newtheorem{property}{Propriété}[section]
\newtheorem{objective}{Objectif}[section]

\theoremstyle{definition}
\newtheorem{definition}{Définition}[section]
\renewcommand{\thedefinition}{\arabic{definition}}
\newtheorem{exercise}{Exercice}[chapter]
\renewcommand{\theexercise}{\arabic{exercise}}
\newtheorem{example}{Exemple}[chapter]
\renewcommand{\theexample}{\arabic{example}}
\newtheorem*{solution}{Solution}
\newtheorem*{application}{Application}
\newtheorem*{notation}{Notation}
\newtheorem*{vocabulary}{Vocabulaire}
\newtheorem*{properties}{Propriétés}



\theoremstyle{remark}
\newtheorem*{remark}{Remarque}
\newtheorem*{rappel}{Rappel}


\usepackage{etoolbox}
\AtBeginEnvironment{exercise}{\small}
\AtBeginEnvironment{example}{\small}

\usepackage{cases}
\usepackage[red]{mypack}

\usepackage[framemethod=TikZ]{mdframed}

\definecolor{bg}{rgb}{0.4,0.25,0.95}
\definecolor{pagebg}{rgb}{0,0,0.5}
\surroundwithmdframed[
   topline=false,
   rightline=false,
   bottomline=false,
   leftmargin=\parindent,
   skipabove=8pt,
   skipbelow=8pt,
   linecolor=blue,
   innerbottommargin=10pt,
   % backgroundcolor=bg,font=\color{orange}\sffamily, fontcolor=white
]{definition}

\usepackage{empheq}
\usepackage[most]{tcolorbox}

\newtcbox{\mymath}[1][]{%
    nobeforeafter, math upper, tcbox raise base,
    enhanced, colframe=blue!30!black,
    colback=red!10, boxrule=1pt,
    #1}

\usepackage{unixode}


\DeclareMathOperator{\ord}{ord}
\DeclareMathOperator{\orb}{orb}
\DeclareMathOperator{\stab}{stab}
\DeclareMathOperator{\Stab}{stab}
\DeclareMathOperator{\ppcm}{ppcm}
\DeclareMathOperator{\conj}{Conj}
\DeclareMathOperator{\End}{End}
\DeclareMathOperator{\rot}{rot}
\DeclareMathOperator{\trs}{trace}
\DeclareMathOperator{\Ind}{Ind}
\DeclareMathOperator{\mat}{Mat}
\DeclareMathOperator{\id}{Id}
\DeclareMathOperator{\vect}{vect}
\DeclareMathOperator{\img}{img}
\DeclareMathOperator{\cov}{Cov}
\DeclareMathOperator{\dist}{dist}
\DeclareMathOperator{\irr}{Irr}
\DeclareMathOperator{\image}{Im}
\DeclareMathOperator{\pd}{\partial}
\DeclareMathOperator{\epi}{epi}
\DeclareMathOperator{\Argmin}{Argmin}
\DeclareMathOperator{\dom}{dom}
\DeclareMathOperator{\proj}{proj}
\DeclareMathOperator{\ctg}{ctg}
\DeclareMathOperator{\supp}{supp}
\DeclareMathOperator{\argmin}{argmin}
\DeclareMathOperator{\mult}{mult}
\DeclareMathOperator{\ch}{ch}
\DeclareMathOperator{\sh}{sh}
\DeclareMathOperator{\rang}{rang}
\DeclareMathOperator{\diam}{diam}
\DeclareMathOperator{\Epigraphe}{Epigraphe}




\usepackage{xcolor}
\everymath{\color{blue}}
%\everymath{\color[rgb]{0,1,1}}
%\pagecolor[rgb]{0,0,0.5}


\newcommand*{\pdtest}[3][]{\ensuremath{\frac{\partial^{#1} #2}{\partial #3}}}

\newcommand*{\deffunc}[6][]{\ensuremath{
\begin{array}{rcl}
#2 : #3 &\rightarrow& #4\\
#5 &\mapsto& #6
\end{array}
}}

\newcommand{\eqcolon}{\mathrel{\resizebox{\widthof{$\mathord{=}$}}{\height}{ $\!\!=\!\!\resizebox{1.2\width}{0.8\height}{\raisebox{0.23ex}{$\mathop{:}$}}\!\!$ }}}
\newcommand{\coloneq}{\mathrel{\resizebox{\widthof{$\mathord{=}$}}{\height}{ $\!\!\resizebox{1.2\width}{0.8\height}{\raisebox{0.23ex}{$\mathop{:}$}}\!\!=\!\!$ }}}
\newcommand{\eqcolonl}{\ensuremath{\mathrel{=\!\!\mathop{:}}}}
\newcommand{\coloneql}{\ensuremath{\mathrel{\mathop{:} \!\! =}}}
\newcommand{\vc}[1]{% inline column vector
  \left(\begin{smallmatrix}#1\end{smallmatrix}\right)%
}
\newcommand{\vr}[1]{% inline row vector
  \begin{smallmatrix}(\,#1\,)\end{smallmatrix}%
}
\makeatletter
\newcommand*{\defeq}{\ =\mathrel{\rlap{%
                     \raisebox{0.3ex}{$\m@th\cdot$}}%
                     \raisebox{-0.3ex}{$\m@th\cdot$}}%
                     }
\makeatother

\newcommand{\mathcircle}[1]{% inline row vector
 \overset{\circ}{#1}
}
\newcommand{\ulim}{% low limit
 \underline{\lim}
}
\newcommand{\ssi}{% iff
\iff
}
\newcommand{\ps}[2]{
\expval{#1 | #2}
}
\newcommand{\df}[1]{
\mqty{#1}
}
\newcommand{\n}[1]{
\norm{#1}
}
\newcommand{\sys}[1]{
\left\{\smqty{#1}\right.
}


\newcommand{\eqdef}{\ensuremath{\overset{\text{def}}=}}


\def\Circlearrowright{\ensuremath{%
  \rotatebox[origin=c]{230}{$\circlearrowright$}}}

\newcommand\ct[1]{\text{\rmfamily\upshape #1}}
\newcommand\question[1]{ {\color{red} ...!? \small #1}}
\newcommand\caz[1]{\left\{\begin{array} #1 \end{array}\right.}
\newcommand\const{\text{\rmfamily\upshape const}}
\newcommand\toP{ \overset{\pro}{\to}}
\newcommand\toPP{ \overset{\text{PP}}{\to}}
\newcommand{\oeq}{\mathrel{\text{\textcircled{$=$}}}}





\usepackage{xcolor}
% \usepackage[normalem]{ulem}
\usepackage{lipsum}
\makeatletter
% \newcommand\colorwave[1][blue]{\bgroup \markoverwith{\lower3.5\p@\hbox{\sixly \textcolor{#1}{\char58}}}\ULon}
%\font\sixly=lasy6 % does not re-load if already loaded, so no memory problem.

\newmdtheoremenv[
linewidth= 1pt,linecolor= blue,%
leftmargin=20,rightmargin=20,innertopmargin=0pt, innerrightmargin=40,%
tikzsetting = { draw=lightgray, line width = 0.3pt,dashed,%
dash pattern = on 15pt off 3pt},%
splittopskip=\topskip,skipbelow=\baselineskip,%
skipabove=\baselineskip,ntheorem,roundcorner=0pt,
% backgroundcolor=pagebg,font=\color{orange}\sffamily, fontcolor=white
]{examplebox}{Exemple}[section]



\newcommand\R{\mathbb{R}}
\newcommand\Z{\mathbb{Z}}
\newcommand\N{\mathbb{N}}
\newcommand\E{\mathbb{E}}
\newcommand\F{\mathcal{F}}
\newcommand\cH{\mathcal{H}}
\newcommand\V{\mathbb{V}}
\newcommand\dmo{ ^{-1} }
\newcommand\kapa{\kappa}
\newcommand\im{Im}
\newcommand\hs{\mathcal{H}}





\usepackage{soul}

\makeatletter
\newcommand*{\whiten}[1]{\llap{\textcolor{white}{{\the\SOUL@token}}\hspace{#1pt}}}
\DeclareRobustCommand*\myul{%
    \def\SOUL@everyspace{\underline{\space}\kern\z@}%
    \def\SOUL@everytoken{%
     \setbox0=\hbox{\the\SOUL@token}%
     \ifdim\dp0>\z@
        \raisebox{\dp0}{\underline{\phantom{\the\SOUL@token}}}%
        \whiten{1}\whiten{0}%
        \whiten{-1}\whiten{-2}%
        \llap{\the\SOUL@token}%
     \else
        \underline{\the\SOUL@token}%
     \fi}%
\SOUL@}
\makeatother

\newcommand*{\demp}{\fontfamily{lmtt}\selectfont}

\DeclareTextFontCommand{\textdemp}{\demp}

\begin{document}

\ifcomment
Multiline
comment
\fi
\ifcomment
\myul{Typesetting test}
% \color[rgb]{1,1,1}
$∑_i^n≠ 60º±∞π∆¬≈√j∫h≤≥µ$

$\CR \R\pro\ind\pro\gS\pro
\mqty[a&b\\c&d]$
$\pro\mathbb{P}$
$\dd{x}$

  \[
    \alpha(x)=\left\{
                \begin{array}{ll}
                  x\\
                  \frac{1}{1+e^{-kx}}\\
                  \frac{e^x-e^{-x}}{e^x+e^{-x}}
                \end{array}
              \right.
  \]

  $\expval{x}$
  
  $\chi_\rho(ghg\dmo)=\Tr(\rho_{ghg\dmo})=\Tr(\rho_g\circ\rho_h\circ\rho\dmo_g)=\Tr(\rho_h)\overset{\mbox{\scalebox{0.5}{$\Tr(AB)=\Tr(BA)$}}}{=}\chi_\rho(h)$
  	$\mathop{\oplus}_{\substack{x\in X}}$

$\mat(\rho_g)=(a_{ij}(g))_{\scriptsize \substack{1\leq i\leq d \\ 1\leq j\leq d}}$ et $\mat(\rho'_g)=(a'_{ij}(g))_{\scriptsize \substack{1\leq i'\leq d' \\ 1\leq j'\leq d'}}$



\[\int_a^b{\mathbb{R}^2}g(u, v)\dd{P_{XY}}(u, v)=\iint g(u,v) f_{XY}(u, v)\dd \lambda(u) \dd \lambda(v)\]
$$\lim_{x\to\infty} f(x)$$	
$$\iiiint_V \mu(t,u,v,w) \,dt\,du\,dv\,dw$$
$$\sum_{n=1}^{\infty} 2^{-n} = 1$$	
\begin{definition}
	Si $X$ et $Y$ sont 2 v.a. ou definit la \textsc{Covariance} entre $X$ et $Y$ comme
	$\cov(X,Y)\overset{\text{def}}{=}\E\left[(X-\E(X))(Y-\E(Y))\right]=\E(XY)-\E(X)\E(Y)$.
\end{definition}
\fi
\pagebreak

% \tableofcontents

% insert your code here
%\input{./algebra/main.tex}
%\input{./geometrie-differentielle/main.tex}
%\input{./probabilite/main.tex}
%\input{./analyse-fonctionnelle/main.tex}
% \input{./Analyse-convexe-et-dualite-en-optimisation/main.tex}
%\input{./tikz/main.tex}
%\input{./Theorie-du-distributions/main.tex}
%\input{./optimisation/mine.tex}
 \input{./modelisation/main.tex}

% yves.aubry@univ-tln.fr : algebra

\end{document}

%% !TEX encoding = UTF-8 Unicode
% !TEX TS-program = xelatex

\documentclass[french]{report}

%\usepackage[utf8]{inputenc}
%\usepackage[T1]{fontenc}
\usepackage{babel}


\newif\ifcomment
%\commenttrue # Show comments

\usepackage{physics}
\usepackage{amssymb}


\usepackage{amsthm}
% \usepackage{thmtools}
\usepackage{mathtools}
\usepackage{amsfonts}

\usepackage{color}

\usepackage{tikz}

\usepackage{geometry}
\geometry{a5paper, margin=0.1in, right=1cm}

\usepackage{dsfont}

\usepackage{graphicx}
\graphicspath{ {images/} }

\usepackage{faktor}

\usepackage{IEEEtrantools}
\usepackage{enumerate}   
\usepackage[PostScript=dvips]{"/Users/aware/Documents/Courses/diagrams"}


\newtheorem{theorem}{Théorème}[section]
\renewcommand{\thetheorem}{\arabic{theorem}}
\newtheorem{lemme}{Lemme}[section]
\renewcommand{\thelemme}{\arabic{lemme}}
\newtheorem{proposition}{Proposition}[section]
\renewcommand{\theproposition}{\arabic{proposition}}
\newtheorem{notations}{Notations}[section]
\newtheorem{problem}{Problème}[section]
\newtheorem{corollary}{Corollaire}[theorem]
\renewcommand{\thecorollary}{\arabic{corollary}}
\newtheorem{property}{Propriété}[section]
\newtheorem{objective}{Objectif}[section]

\theoremstyle{definition}
\newtheorem{definition}{Définition}[section]
\renewcommand{\thedefinition}{\arabic{definition}}
\newtheorem{exercise}{Exercice}[chapter]
\renewcommand{\theexercise}{\arabic{exercise}}
\newtheorem{example}{Exemple}[chapter]
\renewcommand{\theexample}{\arabic{example}}
\newtheorem*{solution}{Solution}
\newtheorem*{application}{Application}
\newtheorem*{notation}{Notation}
\newtheorem*{vocabulary}{Vocabulaire}
\newtheorem*{properties}{Propriétés}



\theoremstyle{remark}
\newtheorem*{remark}{Remarque}
\newtheorem*{rappel}{Rappel}


\usepackage{etoolbox}
\AtBeginEnvironment{exercise}{\small}
\AtBeginEnvironment{example}{\small}

\usepackage{cases}
\usepackage[red]{mypack}

\usepackage[framemethod=TikZ]{mdframed}

\definecolor{bg}{rgb}{0.4,0.25,0.95}
\definecolor{pagebg}{rgb}{0,0,0.5}
\surroundwithmdframed[
   topline=false,
   rightline=false,
   bottomline=false,
   leftmargin=\parindent,
   skipabove=8pt,
   skipbelow=8pt,
   linecolor=blue,
   innerbottommargin=10pt,
   % backgroundcolor=bg,font=\color{orange}\sffamily, fontcolor=white
]{definition}

\usepackage{empheq}
\usepackage[most]{tcolorbox}

\newtcbox{\mymath}[1][]{%
    nobeforeafter, math upper, tcbox raise base,
    enhanced, colframe=blue!30!black,
    colback=red!10, boxrule=1pt,
    #1}

\usepackage{unixode}


\DeclareMathOperator{\ord}{ord}
\DeclareMathOperator{\orb}{orb}
\DeclareMathOperator{\stab}{stab}
\DeclareMathOperator{\Stab}{stab}
\DeclareMathOperator{\ppcm}{ppcm}
\DeclareMathOperator{\conj}{Conj}
\DeclareMathOperator{\End}{End}
\DeclareMathOperator{\rot}{rot}
\DeclareMathOperator{\trs}{trace}
\DeclareMathOperator{\Ind}{Ind}
\DeclareMathOperator{\mat}{Mat}
\DeclareMathOperator{\id}{Id}
\DeclareMathOperator{\vect}{vect}
\DeclareMathOperator{\img}{img}
\DeclareMathOperator{\cov}{Cov}
\DeclareMathOperator{\dist}{dist}
\DeclareMathOperator{\irr}{Irr}
\DeclareMathOperator{\image}{Im}
\DeclareMathOperator{\pd}{\partial}
\DeclareMathOperator{\epi}{epi}
\DeclareMathOperator{\Argmin}{Argmin}
\DeclareMathOperator{\dom}{dom}
\DeclareMathOperator{\proj}{proj}
\DeclareMathOperator{\ctg}{ctg}
\DeclareMathOperator{\supp}{supp}
\DeclareMathOperator{\argmin}{argmin}
\DeclareMathOperator{\mult}{mult}
\DeclareMathOperator{\ch}{ch}
\DeclareMathOperator{\sh}{sh}
\DeclareMathOperator{\rang}{rang}
\DeclareMathOperator{\diam}{diam}
\DeclareMathOperator{\Epigraphe}{Epigraphe}




\usepackage{xcolor}
\everymath{\color{blue}}
%\everymath{\color[rgb]{0,1,1}}
%\pagecolor[rgb]{0,0,0.5}


\newcommand*{\pdtest}[3][]{\ensuremath{\frac{\partial^{#1} #2}{\partial #3}}}

\newcommand*{\deffunc}[6][]{\ensuremath{
\begin{array}{rcl}
#2 : #3 &\rightarrow& #4\\
#5 &\mapsto& #6
\end{array}
}}

\newcommand{\eqcolon}{\mathrel{\resizebox{\widthof{$\mathord{=}$}}{\height}{ $\!\!=\!\!\resizebox{1.2\width}{0.8\height}{\raisebox{0.23ex}{$\mathop{:}$}}\!\!$ }}}
\newcommand{\coloneq}{\mathrel{\resizebox{\widthof{$\mathord{=}$}}{\height}{ $\!\!\resizebox{1.2\width}{0.8\height}{\raisebox{0.23ex}{$\mathop{:}$}}\!\!=\!\!$ }}}
\newcommand{\eqcolonl}{\ensuremath{\mathrel{=\!\!\mathop{:}}}}
\newcommand{\coloneql}{\ensuremath{\mathrel{\mathop{:} \!\! =}}}
\newcommand{\vc}[1]{% inline column vector
  \left(\begin{smallmatrix}#1\end{smallmatrix}\right)%
}
\newcommand{\vr}[1]{% inline row vector
  \begin{smallmatrix}(\,#1\,)\end{smallmatrix}%
}
\makeatletter
\newcommand*{\defeq}{\ =\mathrel{\rlap{%
                     \raisebox{0.3ex}{$\m@th\cdot$}}%
                     \raisebox{-0.3ex}{$\m@th\cdot$}}%
                     }
\makeatother

\newcommand{\mathcircle}[1]{% inline row vector
 \overset{\circ}{#1}
}
\newcommand{\ulim}{% low limit
 \underline{\lim}
}
\newcommand{\ssi}{% iff
\iff
}
\newcommand{\ps}[2]{
\expval{#1 | #2}
}
\newcommand{\df}[1]{
\mqty{#1}
}
\newcommand{\n}[1]{
\norm{#1}
}
\newcommand{\sys}[1]{
\left\{\smqty{#1}\right.
}


\newcommand{\eqdef}{\ensuremath{\overset{\text{def}}=}}


\def\Circlearrowright{\ensuremath{%
  \rotatebox[origin=c]{230}{$\circlearrowright$}}}

\newcommand\ct[1]{\text{\rmfamily\upshape #1}}
\newcommand\question[1]{ {\color{red} ...!? \small #1}}
\newcommand\caz[1]{\left\{\begin{array} #1 \end{array}\right.}
\newcommand\const{\text{\rmfamily\upshape const}}
\newcommand\toP{ \overset{\pro}{\to}}
\newcommand\toPP{ \overset{\text{PP}}{\to}}
\newcommand{\oeq}{\mathrel{\text{\textcircled{$=$}}}}





\usepackage{xcolor}
% \usepackage[normalem]{ulem}
\usepackage{lipsum}
\makeatletter
% \newcommand\colorwave[1][blue]{\bgroup \markoverwith{\lower3.5\p@\hbox{\sixly \textcolor{#1}{\char58}}}\ULon}
%\font\sixly=lasy6 % does not re-load if already loaded, so no memory problem.

\newmdtheoremenv[
linewidth= 1pt,linecolor= blue,%
leftmargin=20,rightmargin=20,innertopmargin=0pt, innerrightmargin=40,%
tikzsetting = { draw=lightgray, line width = 0.3pt,dashed,%
dash pattern = on 15pt off 3pt},%
splittopskip=\topskip,skipbelow=\baselineskip,%
skipabove=\baselineskip,ntheorem,roundcorner=0pt,
% backgroundcolor=pagebg,font=\color{orange}\sffamily, fontcolor=white
]{examplebox}{Exemple}[section]



\newcommand\R{\mathbb{R}}
\newcommand\Z{\mathbb{Z}}
\newcommand\N{\mathbb{N}}
\newcommand\E{\mathbb{E}}
\newcommand\F{\mathcal{F}}
\newcommand\cH{\mathcal{H}}
\newcommand\V{\mathbb{V}}
\newcommand\dmo{ ^{-1} }
\newcommand\kapa{\kappa}
\newcommand\im{Im}
\newcommand\hs{\mathcal{H}}





\usepackage{soul}

\makeatletter
\newcommand*{\whiten}[1]{\llap{\textcolor{white}{{\the\SOUL@token}}\hspace{#1pt}}}
\DeclareRobustCommand*\myul{%
    \def\SOUL@everyspace{\underline{\space}\kern\z@}%
    \def\SOUL@everytoken{%
     \setbox0=\hbox{\the\SOUL@token}%
     \ifdim\dp0>\z@
        \raisebox{\dp0}{\underline{\phantom{\the\SOUL@token}}}%
        \whiten{1}\whiten{0}%
        \whiten{-1}\whiten{-2}%
        \llap{\the\SOUL@token}%
     \else
        \underline{\the\SOUL@token}%
     \fi}%
\SOUL@}
\makeatother

\newcommand*{\demp}{\fontfamily{lmtt}\selectfont}

\DeclareTextFontCommand{\textdemp}{\demp}

\begin{document}

\ifcomment
Multiline
comment
\fi
\ifcomment
\myul{Typesetting test}
% \color[rgb]{1,1,1}
$∑_i^n≠ 60º±∞π∆¬≈√j∫h≤≥µ$

$\CR \R\pro\ind\pro\gS\pro
\mqty[a&b\\c&d]$
$\pro\mathbb{P}$
$\dd{x}$

  \[
    \alpha(x)=\left\{
                \begin{array}{ll}
                  x\\
                  \frac{1}{1+e^{-kx}}\\
                  \frac{e^x-e^{-x}}{e^x+e^{-x}}
                \end{array}
              \right.
  \]

  $\expval{x}$
  
  $\chi_\rho(ghg\dmo)=\Tr(\rho_{ghg\dmo})=\Tr(\rho_g\circ\rho_h\circ\rho\dmo_g)=\Tr(\rho_h)\overset{\mbox{\scalebox{0.5}{$\Tr(AB)=\Tr(BA)$}}}{=}\chi_\rho(h)$
  	$\mathop{\oplus}_{\substack{x\in X}}$

$\mat(\rho_g)=(a_{ij}(g))_{\scriptsize \substack{1\leq i\leq d \\ 1\leq j\leq d}}$ et $\mat(\rho'_g)=(a'_{ij}(g))_{\scriptsize \substack{1\leq i'\leq d' \\ 1\leq j'\leq d'}}$



\[\int_a^b{\mathbb{R}^2}g(u, v)\dd{P_{XY}}(u, v)=\iint g(u,v) f_{XY}(u, v)\dd \lambda(u) \dd \lambda(v)\]
$$\lim_{x\to\infty} f(x)$$	
$$\iiiint_V \mu(t,u,v,w) \,dt\,du\,dv\,dw$$
$$\sum_{n=1}^{\infty} 2^{-n} = 1$$	
\begin{definition}
	Si $X$ et $Y$ sont 2 v.a. ou definit la \textsc{Covariance} entre $X$ et $Y$ comme
	$\cov(X,Y)\overset{\text{def}}{=}\E\left[(X-\E(X))(Y-\E(Y))\right]=\E(XY)-\E(X)\E(Y)$.
\end{definition}
\fi
\pagebreak

% \tableofcontents

% insert your code here
%\input{./algebra/main.tex}
%\input{./geometrie-differentielle/main.tex}
%\input{./probabilite/main.tex}
%\input{./analyse-fonctionnelle/main.tex}
% \input{./Analyse-convexe-et-dualite-en-optimisation/main.tex}
%\input{./tikz/main.tex}
%\input{./Theorie-du-distributions/main.tex}
%\input{./optimisation/mine.tex}
 \input{./modelisation/main.tex}

% yves.aubry@univ-tln.fr : algebra

\end{document}

%% !TEX encoding = UTF-8 Unicode
% !TEX TS-program = xelatex

\documentclass[french]{report}

%\usepackage[utf8]{inputenc}
%\usepackage[T1]{fontenc}
\usepackage{babel}


\newif\ifcomment
%\commenttrue # Show comments

\usepackage{physics}
\usepackage{amssymb}


\usepackage{amsthm}
% \usepackage{thmtools}
\usepackage{mathtools}
\usepackage{amsfonts}

\usepackage{color}

\usepackage{tikz}

\usepackage{geometry}
\geometry{a5paper, margin=0.1in, right=1cm}

\usepackage{dsfont}

\usepackage{graphicx}
\graphicspath{ {images/} }

\usepackage{faktor}

\usepackage{IEEEtrantools}
\usepackage{enumerate}   
\usepackage[PostScript=dvips]{"/Users/aware/Documents/Courses/diagrams"}


\newtheorem{theorem}{Théorème}[section]
\renewcommand{\thetheorem}{\arabic{theorem}}
\newtheorem{lemme}{Lemme}[section]
\renewcommand{\thelemme}{\arabic{lemme}}
\newtheorem{proposition}{Proposition}[section]
\renewcommand{\theproposition}{\arabic{proposition}}
\newtheorem{notations}{Notations}[section]
\newtheorem{problem}{Problème}[section]
\newtheorem{corollary}{Corollaire}[theorem]
\renewcommand{\thecorollary}{\arabic{corollary}}
\newtheorem{property}{Propriété}[section]
\newtheorem{objective}{Objectif}[section]

\theoremstyle{definition}
\newtheorem{definition}{Définition}[section]
\renewcommand{\thedefinition}{\arabic{definition}}
\newtheorem{exercise}{Exercice}[chapter]
\renewcommand{\theexercise}{\arabic{exercise}}
\newtheorem{example}{Exemple}[chapter]
\renewcommand{\theexample}{\arabic{example}}
\newtheorem*{solution}{Solution}
\newtheorem*{application}{Application}
\newtheorem*{notation}{Notation}
\newtheorem*{vocabulary}{Vocabulaire}
\newtheorem*{properties}{Propriétés}



\theoremstyle{remark}
\newtheorem*{remark}{Remarque}
\newtheorem*{rappel}{Rappel}


\usepackage{etoolbox}
\AtBeginEnvironment{exercise}{\small}
\AtBeginEnvironment{example}{\small}

\usepackage{cases}
\usepackage[red]{mypack}

\usepackage[framemethod=TikZ]{mdframed}

\definecolor{bg}{rgb}{0.4,0.25,0.95}
\definecolor{pagebg}{rgb}{0,0,0.5}
\surroundwithmdframed[
   topline=false,
   rightline=false,
   bottomline=false,
   leftmargin=\parindent,
   skipabove=8pt,
   skipbelow=8pt,
   linecolor=blue,
   innerbottommargin=10pt,
   % backgroundcolor=bg,font=\color{orange}\sffamily, fontcolor=white
]{definition}

\usepackage{empheq}
\usepackage[most]{tcolorbox}

\newtcbox{\mymath}[1][]{%
    nobeforeafter, math upper, tcbox raise base,
    enhanced, colframe=blue!30!black,
    colback=red!10, boxrule=1pt,
    #1}

\usepackage{unixode}


\DeclareMathOperator{\ord}{ord}
\DeclareMathOperator{\orb}{orb}
\DeclareMathOperator{\stab}{stab}
\DeclareMathOperator{\Stab}{stab}
\DeclareMathOperator{\ppcm}{ppcm}
\DeclareMathOperator{\conj}{Conj}
\DeclareMathOperator{\End}{End}
\DeclareMathOperator{\rot}{rot}
\DeclareMathOperator{\trs}{trace}
\DeclareMathOperator{\Ind}{Ind}
\DeclareMathOperator{\mat}{Mat}
\DeclareMathOperator{\id}{Id}
\DeclareMathOperator{\vect}{vect}
\DeclareMathOperator{\img}{img}
\DeclareMathOperator{\cov}{Cov}
\DeclareMathOperator{\dist}{dist}
\DeclareMathOperator{\irr}{Irr}
\DeclareMathOperator{\image}{Im}
\DeclareMathOperator{\pd}{\partial}
\DeclareMathOperator{\epi}{epi}
\DeclareMathOperator{\Argmin}{Argmin}
\DeclareMathOperator{\dom}{dom}
\DeclareMathOperator{\proj}{proj}
\DeclareMathOperator{\ctg}{ctg}
\DeclareMathOperator{\supp}{supp}
\DeclareMathOperator{\argmin}{argmin}
\DeclareMathOperator{\mult}{mult}
\DeclareMathOperator{\ch}{ch}
\DeclareMathOperator{\sh}{sh}
\DeclareMathOperator{\rang}{rang}
\DeclareMathOperator{\diam}{diam}
\DeclareMathOperator{\Epigraphe}{Epigraphe}




\usepackage{xcolor}
\everymath{\color{blue}}
%\everymath{\color[rgb]{0,1,1}}
%\pagecolor[rgb]{0,0,0.5}


\newcommand*{\pdtest}[3][]{\ensuremath{\frac{\partial^{#1} #2}{\partial #3}}}

\newcommand*{\deffunc}[6][]{\ensuremath{
\begin{array}{rcl}
#2 : #3 &\rightarrow& #4\\
#5 &\mapsto& #6
\end{array}
}}

\newcommand{\eqcolon}{\mathrel{\resizebox{\widthof{$\mathord{=}$}}{\height}{ $\!\!=\!\!\resizebox{1.2\width}{0.8\height}{\raisebox{0.23ex}{$\mathop{:}$}}\!\!$ }}}
\newcommand{\coloneq}{\mathrel{\resizebox{\widthof{$\mathord{=}$}}{\height}{ $\!\!\resizebox{1.2\width}{0.8\height}{\raisebox{0.23ex}{$\mathop{:}$}}\!\!=\!\!$ }}}
\newcommand{\eqcolonl}{\ensuremath{\mathrel{=\!\!\mathop{:}}}}
\newcommand{\coloneql}{\ensuremath{\mathrel{\mathop{:} \!\! =}}}
\newcommand{\vc}[1]{% inline column vector
  \left(\begin{smallmatrix}#1\end{smallmatrix}\right)%
}
\newcommand{\vr}[1]{% inline row vector
  \begin{smallmatrix}(\,#1\,)\end{smallmatrix}%
}
\makeatletter
\newcommand*{\defeq}{\ =\mathrel{\rlap{%
                     \raisebox{0.3ex}{$\m@th\cdot$}}%
                     \raisebox{-0.3ex}{$\m@th\cdot$}}%
                     }
\makeatother

\newcommand{\mathcircle}[1]{% inline row vector
 \overset{\circ}{#1}
}
\newcommand{\ulim}{% low limit
 \underline{\lim}
}
\newcommand{\ssi}{% iff
\iff
}
\newcommand{\ps}[2]{
\expval{#1 | #2}
}
\newcommand{\df}[1]{
\mqty{#1}
}
\newcommand{\n}[1]{
\norm{#1}
}
\newcommand{\sys}[1]{
\left\{\smqty{#1}\right.
}


\newcommand{\eqdef}{\ensuremath{\overset{\text{def}}=}}


\def\Circlearrowright{\ensuremath{%
  \rotatebox[origin=c]{230}{$\circlearrowright$}}}

\newcommand\ct[1]{\text{\rmfamily\upshape #1}}
\newcommand\question[1]{ {\color{red} ...!? \small #1}}
\newcommand\caz[1]{\left\{\begin{array} #1 \end{array}\right.}
\newcommand\const{\text{\rmfamily\upshape const}}
\newcommand\toP{ \overset{\pro}{\to}}
\newcommand\toPP{ \overset{\text{PP}}{\to}}
\newcommand{\oeq}{\mathrel{\text{\textcircled{$=$}}}}





\usepackage{xcolor}
% \usepackage[normalem]{ulem}
\usepackage{lipsum}
\makeatletter
% \newcommand\colorwave[1][blue]{\bgroup \markoverwith{\lower3.5\p@\hbox{\sixly \textcolor{#1}{\char58}}}\ULon}
%\font\sixly=lasy6 % does not re-load if already loaded, so no memory problem.

\newmdtheoremenv[
linewidth= 1pt,linecolor= blue,%
leftmargin=20,rightmargin=20,innertopmargin=0pt, innerrightmargin=40,%
tikzsetting = { draw=lightgray, line width = 0.3pt,dashed,%
dash pattern = on 15pt off 3pt},%
splittopskip=\topskip,skipbelow=\baselineskip,%
skipabove=\baselineskip,ntheorem,roundcorner=0pt,
% backgroundcolor=pagebg,font=\color{orange}\sffamily, fontcolor=white
]{examplebox}{Exemple}[section]



\newcommand\R{\mathbb{R}}
\newcommand\Z{\mathbb{Z}}
\newcommand\N{\mathbb{N}}
\newcommand\E{\mathbb{E}}
\newcommand\F{\mathcal{F}}
\newcommand\cH{\mathcal{H}}
\newcommand\V{\mathbb{V}}
\newcommand\dmo{ ^{-1} }
\newcommand\kapa{\kappa}
\newcommand\im{Im}
\newcommand\hs{\mathcal{H}}





\usepackage{soul}

\makeatletter
\newcommand*{\whiten}[1]{\llap{\textcolor{white}{{\the\SOUL@token}}\hspace{#1pt}}}
\DeclareRobustCommand*\myul{%
    \def\SOUL@everyspace{\underline{\space}\kern\z@}%
    \def\SOUL@everytoken{%
     \setbox0=\hbox{\the\SOUL@token}%
     \ifdim\dp0>\z@
        \raisebox{\dp0}{\underline{\phantom{\the\SOUL@token}}}%
        \whiten{1}\whiten{0}%
        \whiten{-1}\whiten{-2}%
        \llap{\the\SOUL@token}%
     \else
        \underline{\the\SOUL@token}%
     \fi}%
\SOUL@}
\makeatother

\newcommand*{\demp}{\fontfamily{lmtt}\selectfont}

\DeclareTextFontCommand{\textdemp}{\demp}

\begin{document}

\ifcomment
Multiline
comment
\fi
\ifcomment
\myul{Typesetting test}
% \color[rgb]{1,1,1}
$∑_i^n≠ 60º±∞π∆¬≈√j∫h≤≥µ$

$\CR \R\pro\ind\pro\gS\pro
\mqty[a&b\\c&d]$
$\pro\mathbb{P}$
$\dd{x}$

  \[
    \alpha(x)=\left\{
                \begin{array}{ll}
                  x\\
                  \frac{1}{1+e^{-kx}}\\
                  \frac{e^x-e^{-x}}{e^x+e^{-x}}
                \end{array}
              \right.
  \]

  $\expval{x}$
  
  $\chi_\rho(ghg\dmo)=\Tr(\rho_{ghg\dmo})=\Tr(\rho_g\circ\rho_h\circ\rho\dmo_g)=\Tr(\rho_h)\overset{\mbox{\scalebox{0.5}{$\Tr(AB)=\Tr(BA)$}}}{=}\chi_\rho(h)$
  	$\mathop{\oplus}_{\substack{x\in X}}$

$\mat(\rho_g)=(a_{ij}(g))_{\scriptsize \substack{1\leq i\leq d \\ 1\leq j\leq d}}$ et $\mat(\rho'_g)=(a'_{ij}(g))_{\scriptsize \substack{1\leq i'\leq d' \\ 1\leq j'\leq d'}}$



\[\int_a^b{\mathbb{R}^2}g(u, v)\dd{P_{XY}}(u, v)=\iint g(u,v) f_{XY}(u, v)\dd \lambda(u) \dd \lambda(v)\]
$$\lim_{x\to\infty} f(x)$$	
$$\iiiint_V \mu(t,u,v,w) \,dt\,du\,dv\,dw$$
$$\sum_{n=1}^{\infty} 2^{-n} = 1$$	
\begin{definition}
	Si $X$ et $Y$ sont 2 v.a. ou definit la \textsc{Covariance} entre $X$ et $Y$ comme
	$\cov(X,Y)\overset{\text{def}}{=}\E\left[(X-\E(X))(Y-\E(Y))\right]=\E(XY)-\E(X)\E(Y)$.
\end{definition}
\fi
\pagebreak

% \tableofcontents

% insert your code here
%\input{./algebra/main.tex}
%\input{./geometrie-differentielle/main.tex}
%\input{./probabilite/main.tex}
%\input{./analyse-fonctionnelle/main.tex}
% \input{./Analyse-convexe-et-dualite-en-optimisation/main.tex}
%\input{./tikz/main.tex}
%\input{./Theorie-du-distributions/main.tex}
%\input{./optimisation/mine.tex}
 \input{./modelisation/main.tex}

% yves.aubry@univ-tln.fr : algebra

\end{document}

%\input{./optimisation/mine.tex}
 % !TEX encoding = UTF-8 Unicode
% !TEX TS-program = xelatex

\documentclass[french]{report}

%\usepackage[utf8]{inputenc}
%\usepackage[T1]{fontenc}
\usepackage{babel}


\newif\ifcomment
%\commenttrue # Show comments

\usepackage{physics}
\usepackage{amssymb}


\usepackage{amsthm}
% \usepackage{thmtools}
\usepackage{mathtools}
\usepackage{amsfonts}

\usepackage{color}

\usepackage{tikz}

\usepackage{geometry}
\geometry{a5paper, margin=0.1in, right=1cm}

\usepackage{dsfont}

\usepackage{graphicx}
\graphicspath{ {images/} }

\usepackage{faktor}

\usepackage{IEEEtrantools}
\usepackage{enumerate}   
\usepackage[PostScript=dvips]{"/Users/aware/Documents/Courses/diagrams"}


\newtheorem{theorem}{Théorème}[section]
\renewcommand{\thetheorem}{\arabic{theorem}}
\newtheorem{lemme}{Lemme}[section]
\renewcommand{\thelemme}{\arabic{lemme}}
\newtheorem{proposition}{Proposition}[section]
\renewcommand{\theproposition}{\arabic{proposition}}
\newtheorem{notations}{Notations}[section]
\newtheorem{problem}{Problème}[section]
\newtheorem{corollary}{Corollaire}[theorem]
\renewcommand{\thecorollary}{\arabic{corollary}}
\newtheorem{property}{Propriété}[section]
\newtheorem{objective}{Objectif}[section]

\theoremstyle{definition}
\newtheorem{definition}{Définition}[section]
\renewcommand{\thedefinition}{\arabic{definition}}
\newtheorem{exercise}{Exercice}[chapter]
\renewcommand{\theexercise}{\arabic{exercise}}
\newtheorem{example}{Exemple}[chapter]
\renewcommand{\theexample}{\arabic{example}}
\newtheorem*{solution}{Solution}
\newtheorem*{application}{Application}
\newtheorem*{notation}{Notation}
\newtheorem*{vocabulary}{Vocabulaire}
\newtheorem*{properties}{Propriétés}



\theoremstyle{remark}
\newtheorem*{remark}{Remarque}
\newtheorem*{rappel}{Rappel}


\usepackage{etoolbox}
\AtBeginEnvironment{exercise}{\small}
\AtBeginEnvironment{example}{\small}

\usepackage{cases}
\usepackage[red]{mypack}

\usepackage[framemethod=TikZ]{mdframed}

\definecolor{bg}{rgb}{0.4,0.25,0.95}
\definecolor{pagebg}{rgb}{0,0,0.5}
\surroundwithmdframed[
   topline=false,
   rightline=false,
   bottomline=false,
   leftmargin=\parindent,
   skipabove=8pt,
   skipbelow=8pt,
   linecolor=blue,
   innerbottommargin=10pt,
   % backgroundcolor=bg,font=\color{orange}\sffamily, fontcolor=white
]{definition}

\usepackage{empheq}
\usepackage[most]{tcolorbox}

\newtcbox{\mymath}[1][]{%
    nobeforeafter, math upper, tcbox raise base,
    enhanced, colframe=blue!30!black,
    colback=red!10, boxrule=1pt,
    #1}

\usepackage{unixode}


\DeclareMathOperator{\ord}{ord}
\DeclareMathOperator{\orb}{orb}
\DeclareMathOperator{\stab}{stab}
\DeclareMathOperator{\Stab}{stab}
\DeclareMathOperator{\ppcm}{ppcm}
\DeclareMathOperator{\conj}{Conj}
\DeclareMathOperator{\End}{End}
\DeclareMathOperator{\rot}{rot}
\DeclareMathOperator{\trs}{trace}
\DeclareMathOperator{\Ind}{Ind}
\DeclareMathOperator{\mat}{Mat}
\DeclareMathOperator{\id}{Id}
\DeclareMathOperator{\vect}{vect}
\DeclareMathOperator{\img}{img}
\DeclareMathOperator{\cov}{Cov}
\DeclareMathOperator{\dist}{dist}
\DeclareMathOperator{\irr}{Irr}
\DeclareMathOperator{\image}{Im}
\DeclareMathOperator{\pd}{\partial}
\DeclareMathOperator{\epi}{epi}
\DeclareMathOperator{\Argmin}{Argmin}
\DeclareMathOperator{\dom}{dom}
\DeclareMathOperator{\proj}{proj}
\DeclareMathOperator{\ctg}{ctg}
\DeclareMathOperator{\supp}{supp}
\DeclareMathOperator{\argmin}{argmin}
\DeclareMathOperator{\mult}{mult}
\DeclareMathOperator{\ch}{ch}
\DeclareMathOperator{\sh}{sh}
\DeclareMathOperator{\rang}{rang}
\DeclareMathOperator{\diam}{diam}
\DeclareMathOperator{\Epigraphe}{Epigraphe}




\usepackage{xcolor}
\everymath{\color{blue}}
%\everymath{\color[rgb]{0,1,1}}
%\pagecolor[rgb]{0,0,0.5}


\newcommand*{\pdtest}[3][]{\ensuremath{\frac{\partial^{#1} #2}{\partial #3}}}

\newcommand*{\deffunc}[6][]{\ensuremath{
\begin{array}{rcl}
#2 : #3 &\rightarrow& #4\\
#5 &\mapsto& #6
\end{array}
}}

\newcommand{\eqcolon}{\mathrel{\resizebox{\widthof{$\mathord{=}$}}{\height}{ $\!\!=\!\!\resizebox{1.2\width}{0.8\height}{\raisebox{0.23ex}{$\mathop{:}$}}\!\!$ }}}
\newcommand{\coloneq}{\mathrel{\resizebox{\widthof{$\mathord{=}$}}{\height}{ $\!\!\resizebox{1.2\width}{0.8\height}{\raisebox{0.23ex}{$\mathop{:}$}}\!\!=\!\!$ }}}
\newcommand{\eqcolonl}{\ensuremath{\mathrel{=\!\!\mathop{:}}}}
\newcommand{\coloneql}{\ensuremath{\mathrel{\mathop{:} \!\! =}}}
\newcommand{\vc}[1]{% inline column vector
  \left(\begin{smallmatrix}#1\end{smallmatrix}\right)%
}
\newcommand{\vr}[1]{% inline row vector
  \begin{smallmatrix}(\,#1\,)\end{smallmatrix}%
}
\makeatletter
\newcommand*{\defeq}{\ =\mathrel{\rlap{%
                     \raisebox{0.3ex}{$\m@th\cdot$}}%
                     \raisebox{-0.3ex}{$\m@th\cdot$}}%
                     }
\makeatother

\newcommand{\mathcircle}[1]{% inline row vector
 \overset{\circ}{#1}
}
\newcommand{\ulim}{% low limit
 \underline{\lim}
}
\newcommand{\ssi}{% iff
\iff
}
\newcommand{\ps}[2]{
\expval{#1 | #2}
}
\newcommand{\df}[1]{
\mqty{#1}
}
\newcommand{\n}[1]{
\norm{#1}
}
\newcommand{\sys}[1]{
\left\{\smqty{#1}\right.
}


\newcommand{\eqdef}{\ensuremath{\overset{\text{def}}=}}


\def\Circlearrowright{\ensuremath{%
  \rotatebox[origin=c]{230}{$\circlearrowright$}}}

\newcommand\ct[1]{\text{\rmfamily\upshape #1}}
\newcommand\question[1]{ {\color{red} ...!? \small #1}}
\newcommand\caz[1]{\left\{\begin{array} #1 \end{array}\right.}
\newcommand\const{\text{\rmfamily\upshape const}}
\newcommand\toP{ \overset{\pro}{\to}}
\newcommand\toPP{ \overset{\text{PP}}{\to}}
\newcommand{\oeq}{\mathrel{\text{\textcircled{$=$}}}}





\usepackage{xcolor}
% \usepackage[normalem]{ulem}
\usepackage{lipsum}
\makeatletter
% \newcommand\colorwave[1][blue]{\bgroup \markoverwith{\lower3.5\p@\hbox{\sixly \textcolor{#1}{\char58}}}\ULon}
%\font\sixly=lasy6 % does not re-load if already loaded, so no memory problem.

\newmdtheoremenv[
linewidth= 1pt,linecolor= blue,%
leftmargin=20,rightmargin=20,innertopmargin=0pt, innerrightmargin=40,%
tikzsetting = { draw=lightgray, line width = 0.3pt,dashed,%
dash pattern = on 15pt off 3pt},%
splittopskip=\topskip,skipbelow=\baselineskip,%
skipabove=\baselineskip,ntheorem,roundcorner=0pt,
% backgroundcolor=pagebg,font=\color{orange}\sffamily, fontcolor=white
]{examplebox}{Exemple}[section]



\newcommand\R{\mathbb{R}}
\newcommand\Z{\mathbb{Z}}
\newcommand\N{\mathbb{N}}
\newcommand\E{\mathbb{E}}
\newcommand\F{\mathcal{F}}
\newcommand\cH{\mathcal{H}}
\newcommand\V{\mathbb{V}}
\newcommand\dmo{ ^{-1} }
\newcommand\kapa{\kappa}
\newcommand\im{Im}
\newcommand\hs{\mathcal{H}}





\usepackage{soul}

\makeatletter
\newcommand*{\whiten}[1]{\llap{\textcolor{white}{{\the\SOUL@token}}\hspace{#1pt}}}
\DeclareRobustCommand*\myul{%
    \def\SOUL@everyspace{\underline{\space}\kern\z@}%
    \def\SOUL@everytoken{%
     \setbox0=\hbox{\the\SOUL@token}%
     \ifdim\dp0>\z@
        \raisebox{\dp0}{\underline{\phantom{\the\SOUL@token}}}%
        \whiten{1}\whiten{0}%
        \whiten{-1}\whiten{-2}%
        \llap{\the\SOUL@token}%
     \else
        \underline{\the\SOUL@token}%
     \fi}%
\SOUL@}
\makeatother

\newcommand*{\demp}{\fontfamily{lmtt}\selectfont}

\DeclareTextFontCommand{\textdemp}{\demp}

\begin{document}

\ifcomment
Multiline
comment
\fi
\ifcomment
\myul{Typesetting test}
% \color[rgb]{1,1,1}
$∑_i^n≠ 60º±∞π∆¬≈√j∫h≤≥µ$

$\CR \R\pro\ind\pro\gS\pro
\mqty[a&b\\c&d]$
$\pro\mathbb{P}$
$\dd{x}$

  \[
    \alpha(x)=\left\{
                \begin{array}{ll}
                  x\\
                  \frac{1}{1+e^{-kx}}\\
                  \frac{e^x-e^{-x}}{e^x+e^{-x}}
                \end{array}
              \right.
  \]

  $\expval{x}$
  
  $\chi_\rho(ghg\dmo)=\Tr(\rho_{ghg\dmo})=\Tr(\rho_g\circ\rho_h\circ\rho\dmo_g)=\Tr(\rho_h)\overset{\mbox{\scalebox{0.5}{$\Tr(AB)=\Tr(BA)$}}}{=}\chi_\rho(h)$
  	$\mathop{\oplus}_{\substack{x\in X}}$

$\mat(\rho_g)=(a_{ij}(g))_{\scriptsize \substack{1\leq i\leq d \\ 1\leq j\leq d}}$ et $\mat(\rho'_g)=(a'_{ij}(g))_{\scriptsize \substack{1\leq i'\leq d' \\ 1\leq j'\leq d'}}$



\[\int_a^b{\mathbb{R}^2}g(u, v)\dd{P_{XY}}(u, v)=\iint g(u,v) f_{XY}(u, v)\dd \lambda(u) \dd \lambda(v)\]
$$\lim_{x\to\infty} f(x)$$	
$$\iiiint_V \mu(t,u,v,w) \,dt\,du\,dv\,dw$$
$$\sum_{n=1}^{\infty} 2^{-n} = 1$$	
\begin{definition}
	Si $X$ et $Y$ sont 2 v.a. ou definit la \textsc{Covariance} entre $X$ et $Y$ comme
	$\cov(X,Y)\overset{\text{def}}{=}\E\left[(X-\E(X))(Y-\E(Y))\right]=\E(XY)-\E(X)\E(Y)$.
\end{definition}
\fi
\pagebreak

% \tableofcontents

% insert your code here
%\input{./algebra/main.tex}
%\input{./geometrie-differentielle/main.tex}
%\input{./probabilite/main.tex}
%\input{./analyse-fonctionnelle/main.tex}
% \input{./Analyse-convexe-et-dualite-en-optimisation/main.tex}
%\input{./tikz/main.tex}
%\input{./Theorie-du-distributions/main.tex}
%\input{./optimisation/mine.tex}
 \input{./modelisation/main.tex}

% yves.aubry@univ-tln.fr : algebra

\end{document}


% yves.aubry@univ-tln.fr : algebra

\end{document}

%% !TEX encoding = UTF-8 Unicode
% !TEX TS-program = xelatex

\documentclass[french]{report}

%\usepackage[utf8]{inputenc}
%\usepackage[T1]{fontenc}
\usepackage{babel}


\newif\ifcomment
%\commenttrue # Show comments

\usepackage{physics}
\usepackage{amssymb}


\usepackage{amsthm}
% \usepackage{thmtools}
\usepackage{mathtools}
\usepackage{amsfonts}

\usepackage{color}

\usepackage{tikz}

\usepackage{geometry}
\geometry{a5paper, margin=0.1in, right=1cm}

\usepackage{dsfont}

\usepackage{graphicx}
\graphicspath{ {images/} }

\usepackage{faktor}

\usepackage{IEEEtrantools}
\usepackage{enumerate}   
\usepackage[PostScript=dvips]{"/Users/aware/Documents/Courses/diagrams"}


\newtheorem{theorem}{Théorème}[section]
\renewcommand{\thetheorem}{\arabic{theorem}}
\newtheorem{lemme}{Lemme}[section]
\renewcommand{\thelemme}{\arabic{lemme}}
\newtheorem{proposition}{Proposition}[section]
\renewcommand{\theproposition}{\arabic{proposition}}
\newtheorem{notations}{Notations}[section]
\newtheorem{problem}{Problème}[section]
\newtheorem{corollary}{Corollaire}[theorem]
\renewcommand{\thecorollary}{\arabic{corollary}}
\newtheorem{property}{Propriété}[section]
\newtheorem{objective}{Objectif}[section]

\theoremstyle{definition}
\newtheorem{definition}{Définition}[section]
\renewcommand{\thedefinition}{\arabic{definition}}
\newtheorem{exercise}{Exercice}[chapter]
\renewcommand{\theexercise}{\arabic{exercise}}
\newtheorem{example}{Exemple}[chapter]
\renewcommand{\theexample}{\arabic{example}}
\newtheorem*{solution}{Solution}
\newtheorem*{application}{Application}
\newtheorem*{notation}{Notation}
\newtheorem*{vocabulary}{Vocabulaire}
\newtheorem*{properties}{Propriétés}



\theoremstyle{remark}
\newtheorem*{remark}{Remarque}
\newtheorem*{rappel}{Rappel}


\usepackage{etoolbox}
\AtBeginEnvironment{exercise}{\small}
\AtBeginEnvironment{example}{\small}

\usepackage{cases}
\usepackage[red]{mypack}

\usepackage[framemethod=TikZ]{mdframed}

\definecolor{bg}{rgb}{0.4,0.25,0.95}
\definecolor{pagebg}{rgb}{0,0,0.5}
\surroundwithmdframed[
   topline=false,
   rightline=false,
   bottomline=false,
   leftmargin=\parindent,
   skipabove=8pt,
   skipbelow=8pt,
   linecolor=blue,
   innerbottommargin=10pt,
   % backgroundcolor=bg,font=\color{orange}\sffamily, fontcolor=white
]{definition}

\usepackage{empheq}
\usepackage[most]{tcolorbox}

\newtcbox{\mymath}[1][]{%
    nobeforeafter, math upper, tcbox raise base,
    enhanced, colframe=blue!30!black,
    colback=red!10, boxrule=1pt,
    #1}

\usepackage{unixode}


\DeclareMathOperator{\ord}{ord}
\DeclareMathOperator{\orb}{orb}
\DeclareMathOperator{\stab}{stab}
\DeclareMathOperator{\Stab}{stab}
\DeclareMathOperator{\ppcm}{ppcm}
\DeclareMathOperator{\conj}{Conj}
\DeclareMathOperator{\End}{End}
\DeclareMathOperator{\rot}{rot}
\DeclareMathOperator{\trs}{trace}
\DeclareMathOperator{\Ind}{Ind}
\DeclareMathOperator{\mat}{Mat}
\DeclareMathOperator{\id}{Id}
\DeclareMathOperator{\vect}{vect}
\DeclareMathOperator{\img}{img}
\DeclareMathOperator{\cov}{Cov}
\DeclareMathOperator{\dist}{dist}
\DeclareMathOperator{\irr}{Irr}
\DeclareMathOperator{\image}{Im}
\DeclareMathOperator{\pd}{\partial}
\DeclareMathOperator{\epi}{epi}
\DeclareMathOperator{\Argmin}{Argmin}
\DeclareMathOperator{\dom}{dom}
\DeclareMathOperator{\proj}{proj}
\DeclareMathOperator{\ctg}{ctg}
\DeclareMathOperator{\supp}{supp}
\DeclareMathOperator{\argmin}{argmin}
\DeclareMathOperator{\mult}{mult}
\DeclareMathOperator{\ch}{ch}
\DeclareMathOperator{\sh}{sh}
\DeclareMathOperator{\rang}{rang}
\DeclareMathOperator{\diam}{diam}
\DeclareMathOperator{\Epigraphe}{Epigraphe}




\usepackage{xcolor}
\everymath{\color{blue}}
%\everymath{\color[rgb]{0,1,1}}
%\pagecolor[rgb]{0,0,0.5}


\newcommand*{\pdtest}[3][]{\ensuremath{\frac{\partial^{#1} #2}{\partial #3}}}

\newcommand*{\deffunc}[6][]{\ensuremath{
\begin{array}{rcl}
#2 : #3 &\rightarrow& #4\\
#5 &\mapsto& #6
\end{array}
}}

\newcommand{\eqcolon}{\mathrel{\resizebox{\widthof{$\mathord{=}$}}{\height}{ $\!\!=\!\!\resizebox{1.2\width}{0.8\height}{\raisebox{0.23ex}{$\mathop{:}$}}\!\!$ }}}
\newcommand{\coloneq}{\mathrel{\resizebox{\widthof{$\mathord{=}$}}{\height}{ $\!\!\resizebox{1.2\width}{0.8\height}{\raisebox{0.23ex}{$\mathop{:}$}}\!\!=\!\!$ }}}
\newcommand{\eqcolonl}{\ensuremath{\mathrel{=\!\!\mathop{:}}}}
\newcommand{\coloneql}{\ensuremath{\mathrel{\mathop{:} \!\! =}}}
\newcommand{\vc}[1]{% inline column vector
  \left(\begin{smallmatrix}#1\end{smallmatrix}\right)%
}
\newcommand{\vr}[1]{% inline row vector
  \begin{smallmatrix}(\,#1\,)\end{smallmatrix}%
}
\makeatletter
\newcommand*{\defeq}{\ =\mathrel{\rlap{%
                     \raisebox{0.3ex}{$\m@th\cdot$}}%
                     \raisebox{-0.3ex}{$\m@th\cdot$}}%
                     }
\makeatother

\newcommand{\mathcircle}[1]{% inline row vector
 \overset{\circ}{#1}
}
\newcommand{\ulim}{% low limit
 \underline{\lim}
}
\newcommand{\ssi}{% iff
\iff
}
\newcommand{\ps}[2]{
\expval{#1 | #2}
}
\newcommand{\df}[1]{
\mqty{#1}
}
\newcommand{\n}[1]{
\norm{#1}
}
\newcommand{\sys}[1]{
\left\{\smqty{#1}\right.
}


\newcommand{\eqdef}{\ensuremath{\overset{\text{def}}=}}


\def\Circlearrowright{\ensuremath{%
  \rotatebox[origin=c]{230}{$\circlearrowright$}}}

\newcommand\ct[1]{\text{\rmfamily\upshape #1}}
\newcommand\question[1]{ {\color{red} ...!? \small #1}}
\newcommand\caz[1]{\left\{\begin{array} #1 \end{array}\right.}
\newcommand\const{\text{\rmfamily\upshape const}}
\newcommand\toP{ \overset{\pro}{\to}}
\newcommand\toPP{ \overset{\text{PP}}{\to}}
\newcommand{\oeq}{\mathrel{\text{\textcircled{$=$}}}}





\usepackage{xcolor}
% \usepackage[normalem]{ulem}
\usepackage{lipsum}
\makeatletter
% \newcommand\colorwave[1][blue]{\bgroup \markoverwith{\lower3.5\p@\hbox{\sixly \textcolor{#1}{\char58}}}\ULon}
%\font\sixly=lasy6 % does not re-load if already loaded, so no memory problem.

\newmdtheoremenv[
linewidth= 1pt,linecolor= blue,%
leftmargin=20,rightmargin=20,innertopmargin=0pt, innerrightmargin=40,%
tikzsetting = { draw=lightgray, line width = 0.3pt,dashed,%
dash pattern = on 15pt off 3pt},%
splittopskip=\topskip,skipbelow=\baselineskip,%
skipabove=\baselineskip,ntheorem,roundcorner=0pt,
% backgroundcolor=pagebg,font=\color{orange}\sffamily, fontcolor=white
]{examplebox}{Exemple}[section]



\newcommand\R{\mathbb{R}}
\newcommand\Z{\mathbb{Z}}
\newcommand\N{\mathbb{N}}
\newcommand\E{\mathbb{E}}
\newcommand\F{\mathcal{F}}
\newcommand\cH{\mathcal{H}}
\newcommand\V{\mathbb{V}}
\newcommand\dmo{ ^{-1} }
\newcommand\kapa{\kappa}
\newcommand\im{Im}
\newcommand\hs{\mathcal{H}}





\usepackage{soul}

\makeatletter
\newcommand*{\whiten}[1]{\llap{\textcolor{white}{{\the\SOUL@token}}\hspace{#1pt}}}
\DeclareRobustCommand*\myul{%
    \def\SOUL@everyspace{\underline{\space}\kern\z@}%
    \def\SOUL@everytoken{%
     \setbox0=\hbox{\the\SOUL@token}%
     \ifdim\dp0>\z@
        \raisebox{\dp0}{\underline{\phantom{\the\SOUL@token}}}%
        \whiten{1}\whiten{0}%
        \whiten{-1}\whiten{-2}%
        \llap{\the\SOUL@token}%
     \else
        \underline{\the\SOUL@token}%
     \fi}%
\SOUL@}
\makeatother

\newcommand*{\demp}{\fontfamily{lmtt}\selectfont}

\DeclareTextFontCommand{\textdemp}{\demp}

\begin{document}

\ifcomment
Multiline
comment
\fi
\ifcomment
\myul{Typesetting test}
% \color[rgb]{1,1,1}
$∑_i^n≠ 60º±∞π∆¬≈√j∫h≤≥µ$

$\CR \R\pro\ind\pro\gS\pro
\mqty[a&b\\c&d]$
$\pro\mathbb{P}$
$\dd{x}$

  \[
    \alpha(x)=\left\{
                \begin{array}{ll}
                  x\\
                  \frac{1}{1+e^{-kx}}\\
                  \frac{e^x-e^{-x}}{e^x+e^{-x}}
                \end{array}
              \right.
  \]

  $\expval{x}$
  
  $\chi_\rho(ghg\dmo)=\Tr(\rho_{ghg\dmo})=\Tr(\rho_g\circ\rho_h\circ\rho\dmo_g)=\Tr(\rho_h)\overset{\mbox{\scalebox{0.5}{$\Tr(AB)=\Tr(BA)$}}}{=}\chi_\rho(h)$
  	$\mathop{\oplus}_{\substack{x\in X}}$

$\mat(\rho_g)=(a_{ij}(g))_{\scriptsize \substack{1\leq i\leq d \\ 1\leq j\leq d}}$ et $\mat(\rho'_g)=(a'_{ij}(g))_{\scriptsize \substack{1\leq i'\leq d' \\ 1\leq j'\leq d'}}$



\[\int_a^b{\mathbb{R}^2}g(u, v)\dd{P_{XY}}(u, v)=\iint g(u,v) f_{XY}(u, v)\dd \lambda(u) \dd \lambda(v)\]
$$\lim_{x\to\infty} f(x)$$	
$$\iiiint_V \mu(t,u,v,w) \,dt\,du\,dv\,dw$$
$$\sum_{n=1}^{\infty} 2^{-n} = 1$$	
\begin{definition}
	Si $X$ et $Y$ sont 2 v.a. ou definit la \textsc{Covariance} entre $X$ et $Y$ comme
	$\cov(X,Y)\overset{\text{def}}{=}\E\left[(X-\E(X))(Y-\E(Y))\right]=\E(XY)-\E(X)\E(Y)$.
\end{definition}
\fi
\pagebreak

% \tableofcontents

% insert your code here
%% !TEX encoding = UTF-8 Unicode
% !TEX TS-program = xelatex

\documentclass[french]{report}

%\usepackage[utf8]{inputenc}
%\usepackage[T1]{fontenc}
\usepackage{babel}


\newif\ifcomment
%\commenttrue # Show comments

\usepackage{physics}
\usepackage{amssymb}


\usepackage{amsthm}
% \usepackage{thmtools}
\usepackage{mathtools}
\usepackage{amsfonts}

\usepackage{color}

\usepackage{tikz}

\usepackage{geometry}
\geometry{a5paper, margin=0.1in, right=1cm}

\usepackage{dsfont}

\usepackage{graphicx}
\graphicspath{ {images/} }

\usepackage{faktor}

\usepackage{IEEEtrantools}
\usepackage{enumerate}   
\usepackage[PostScript=dvips]{"/Users/aware/Documents/Courses/diagrams"}


\newtheorem{theorem}{Théorème}[section]
\renewcommand{\thetheorem}{\arabic{theorem}}
\newtheorem{lemme}{Lemme}[section]
\renewcommand{\thelemme}{\arabic{lemme}}
\newtheorem{proposition}{Proposition}[section]
\renewcommand{\theproposition}{\arabic{proposition}}
\newtheorem{notations}{Notations}[section]
\newtheorem{problem}{Problème}[section]
\newtheorem{corollary}{Corollaire}[theorem]
\renewcommand{\thecorollary}{\arabic{corollary}}
\newtheorem{property}{Propriété}[section]
\newtheorem{objective}{Objectif}[section]

\theoremstyle{definition}
\newtheorem{definition}{Définition}[section]
\renewcommand{\thedefinition}{\arabic{definition}}
\newtheorem{exercise}{Exercice}[chapter]
\renewcommand{\theexercise}{\arabic{exercise}}
\newtheorem{example}{Exemple}[chapter]
\renewcommand{\theexample}{\arabic{example}}
\newtheorem*{solution}{Solution}
\newtheorem*{application}{Application}
\newtheorem*{notation}{Notation}
\newtheorem*{vocabulary}{Vocabulaire}
\newtheorem*{properties}{Propriétés}



\theoremstyle{remark}
\newtheorem*{remark}{Remarque}
\newtheorem*{rappel}{Rappel}


\usepackage{etoolbox}
\AtBeginEnvironment{exercise}{\small}
\AtBeginEnvironment{example}{\small}

\usepackage{cases}
\usepackage[red]{mypack}

\usepackage[framemethod=TikZ]{mdframed}

\definecolor{bg}{rgb}{0.4,0.25,0.95}
\definecolor{pagebg}{rgb}{0,0,0.5}
\surroundwithmdframed[
   topline=false,
   rightline=false,
   bottomline=false,
   leftmargin=\parindent,
   skipabove=8pt,
   skipbelow=8pt,
   linecolor=blue,
   innerbottommargin=10pt,
   % backgroundcolor=bg,font=\color{orange}\sffamily, fontcolor=white
]{definition}

\usepackage{empheq}
\usepackage[most]{tcolorbox}

\newtcbox{\mymath}[1][]{%
    nobeforeafter, math upper, tcbox raise base,
    enhanced, colframe=blue!30!black,
    colback=red!10, boxrule=1pt,
    #1}

\usepackage{unixode}


\DeclareMathOperator{\ord}{ord}
\DeclareMathOperator{\orb}{orb}
\DeclareMathOperator{\stab}{stab}
\DeclareMathOperator{\Stab}{stab}
\DeclareMathOperator{\ppcm}{ppcm}
\DeclareMathOperator{\conj}{Conj}
\DeclareMathOperator{\End}{End}
\DeclareMathOperator{\rot}{rot}
\DeclareMathOperator{\trs}{trace}
\DeclareMathOperator{\Ind}{Ind}
\DeclareMathOperator{\mat}{Mat}
\DeclareMathOperator{\id}{Id}
\DeclareMathOperator{\vect}{vect}
\DeclareMathOperator{\img}{img}
\DeclareMathOperator{\cov}{Cov}
\DeclareMathOperator{\dist}{dist}
\DeclareMathOperator{\irr}{Irr}
\DeclareMathOperator{\image}{Im}
\DeclareMathOperator{\pd}{\partial}
\DeclareMathOperator{\epi}{epi}
\DeclareMathOperator{\Argmin}{Argmin}
\DeclareMathOperator{\dom}{dom}
\DeclareMathOperator{\proj}{proj}
\DeclareMathOperator{\ctg}{ctg}
\DeclareMathOperator{\supp}{supp}
\DeclareMathOperator{\argmin}{argmin}
\DeclareMathOperator{\mult}{mult}
\DeclareMathOperator{\ch}{ch}
\DeclareMathOperator{\sh}{sh}
\DeclareMathOperator{\rang}{rang}
\DeclareMathOperator{\diam}{diam}
\DeclareMathOperator{\Epigraphe}{Epigraphe}




\usepackage{xcolor}
\everymath{\color{blue}}
%\everymath{\color[rgb]{0,1,1}}
%\pagecolor[rgb]{0,0,0.5}


\newcommand*{\pdtest}[3][]{\ensuremath{\frac{\partial^{#1} #2}{\partial #3}}}

\newcommand*{\deffunc}[6][]{\ensuremath{
\begin{array}{rcl}
#2 : #3 &\rightarrow& #4\\
#5 &\mapsto& #6
\end{array}
}}

\newcommand{\eqcolon}{\mathrel{\resizebox{\widthof{$\mathord{=}$}}{\height}{ $\!\!=\!\!\resizebox{1.2\width}{0.8\height}{\raisebox{0.23ex}{$\mathop{:}$}}\!\!$ }}}
\newcommand{\coloneq}{\mathrel{\resizebox{\widthof{$\mathord{=}$}}{\height}{ $\!\!\resizebox{1.2\width}{0.8\height}{\raisebox{0.23ex}{$\mathop{:}$}}\!\!=\!\!$ }}}
\newcommand{\eqcolonl}{\ensuremath{\mathrel{=\!\!\mathop{:}}}}
\newcommand{\coloneql}{\ensuremath{\mathrel{\mathop{:} \!\! =}}}
\newcommand{\vc}[1]{% inline column vector
  \left(\begin{smallmatrix}#1\end{smallmatrix}\right)%
}
\newcommand{\vr}[1]{% inline row vector
  \begin{smallmatrix}(\,#1\,)\end{smallmatrix}%
}
\makeatletter
\newcommand*{\defeq}{\ =\mathrel{\rlap{%
                     \raisebox{0.3ex}{$\m@th\cdot$}}%
                     \raisebox{-0.3ex}{$\m@th\cdot$}}%
                     }
\makeatother

\newcommand{\mathcircle}[1]{% inline row vector
 \overset{\circ}{#1}
}
\newcommand{\ulim}{% low limit
 \underline{\lim}
}
\newcommand{\ssi}{% iff
\iff
}
\newcommand{\ps}[2]{
\expval{#1 | #2}
}
\newcommand{\df}[1]{
\mqty{#1}
}
\newcommand{\n}[1]{
\norm{#1}
}
\newcommand{\sys}[1]{
\left\{\smqty{#1}\right.
}


\newcommand{\eqdef}{\ensuremath{\overset{\text{def}}=}}


\def\Circlearrowright{\ensuremath{%
  \rotatebox[origin=c]{230}{$\circlearrowright$}}}

\newcommand\ct[1]{\text{\rmfamily\upshape #1}}
\newcommand\question[1]{ {\color{red} ...!? \small #1}}
\newcommand\caz[1]{\left\{\begin{array} #1 \end{array}\right.}
\newcommand\const{\text{\rmfamily\upshape const}}
\newcommand\toP{ \overset{\pro}{\to}}
\newcommand\toPP{ \overset{\text{PP}}{\to}}
\newcommand{\oeq}{\mathrel{\text{\textcircled{$=$}}}}





\usepackage{xcolor}
% \usepackage[normalem]{ulem}
\usepackage{lipsum}
\makeatletter
% \newcommand\colorwave[1][blue]{\bgroup \markoverwith{\lower3.5\p@\hbox{\sixly \textcolor{#1}{\char58}}}\ULon}
%\font\sixly=lasy6 % does not re-load if already loaded, so no memory problem.

\newmdtheoremenv[
linewidth= 1pt,linecolor= blue,%
leftmargin=20,rightmargin=20,innertopmargin=0pt, innerrightmargin=40,%
tikzsetting = { draw=lightgray, line width = 0.3pt,dashed,%
dash pattern = on 15pt off 3pt},%
splittopskip=\topskip,skipbelow=\baselineskip,%
skipabove=\baselineskip,ntheorem,roundcorner=0pt,
% backgroundcolor=pagebg,font=\color{orange}\sffamily, fontcolor=white
]{examplebox}{Exemple}[section]



\newcommand\R{\mathbb{R}}
\newcommand\Z{\mathbb{Z}}
\newcommand\N{\mathbb{N}}
\newcommand\E{\mathbb{E}}
\newcommand\F{\mathcal{F}}
\newcommand\cH{\mathcal{H}}
\newcommand\V{\mathbb{V}}
\newcommand\dmo{ ^{-1} }
\newcommand\kapa{\kappa}
\newcommand\im{Im}
\newcommand\hs{\mathcal{H}}





\usepackage{soul}

\makeatletter
\newcommand*{\whiten}[1]{\llap{\textcolor{white}{{\the\SOUL@token}}\hspace{#1pt}}}
\DeclareRobustCommand*\myul{%
    \def\SOUL@everyspace{\underline{\space}\kern\z@}%
    \def\SOUL@everytoken{%
     \setbox0=\hbox{\the\SOUL@token}%
     \ifdim\dp0>\z@
        \raisebox{\dp0}{\underline{\phantom{\the\SOUL@token}}}%
        \whiten{1}\whiten{0}%
        \whiten{-1}\whiten{-2}%
        \llap{\the\SOUL@token}%
     \else
        \underline{\the\SOUL@token}%
     \fi}%
\SOUL@}
\makeatother

\newcommand*{\demp}{\fontfamily{lmtt}\selectfont}

\DeclareTextFontCommand{\textdemp}{\demp}

\begin{document}

\ifcomment
Multiline
comment
\fi
\ifcomment
\myul{Typesetting test}
% \color[rgb]{1,1,1}
$∑_i^n≠ 60º±∞π∆¬≈√j∫h≤≥µ$

$\CR \R\pro\ind\pro\gS\pro
\mqty[a&b\\c&d]$
$\pro\mathbb{P}$
$\dd{x}$

  \[
    \alpha(x)=\left\{
                \begin{array}{ll}
                  x\\
                  \frac{1}{1+e^{-kx}}\\
                  \frac{e^x-e^{-x}}{e^x+e^{-x}}
                \end{array}
              \right.
  \]

  $\expval{x}$
  
  $\chi_\rho(ghg\dmo)=\Tr(\rho_{ghg\dmo})=\Tr(\rho_g\circ\rho_h\circ\rho\dmo_g)=\Tr(\rho_h)\overset{\mbox{\scalebox{0.5}{$\Tr(AB)=\Tr(BA)$}}}{=}\chi_\rho(h)$
  	$\mathop{\oplus}_{\substack{x\in X}}$

$\mat(\rho_g)=(a_{ij}(g))_{\scriptsize \substack{1\leq i\leq d \\ 1\leq j\leq d}}$ et $\mat(\rho'_g)=(a'_{ij}(g))_{\scriptsize \substack{1\leq i'\leq d' \\ 1\leq j'\leq d'}}$



\[\int_a^b{\mathbb{R}^2}g(u, v)\dd{P_{XY}}(u, v)=\iint g(u,v) f_{XY}(u, v)\dd \lambda(u) \dd \lambda(v)\]
$$\lim_{x\to\infty} f(x)$$	
$$\iiiint_V \mu(t,u,v,w) \,dt\,du\,dv\,dw$$
$$\sum_{n=1}^{\infty} 2^{-n} = 1$$	
\begin{definition}
	Si $X$ et $Y$ sont 2 v.a. ou definit la \textsc{Covariance} entre $X$ et $Y$ comme
	$\cov(X,Y)\overset{\text{def}}{=}\E\left[(X-\E(X))(Y-\E(Y))\right]=\E(XY)-\E(X)\E(Y)$.
\end{definition}
\fi
\pagebreak

% \tableofcontents

% insert your code here
%\input{./algebra/main.tex}
%\input{./geometrie-differentielle/main.tex}
%\input{./probabilite/main.tex}
%\input{./analyse-fonctionnelle/main.tex}
% \input{./Analyse-convexe-et-dualite-en-optimisation/main.tex}
%\input{./tikz/main.tex}
%\input{./Theorie-du-distributions/main.tex}
%\input{./optimisation/mine.tex}
 \input{./modelisation/main.tex}

% yves.aubry@univ-tln.fr : algebra

\end{document}

%% !TEX encoding = UTF-8 Unicode
% !TEX TS-program = xelatex

\documentclass[french]{report}

%\usepackage[utf8]{inputenc}
%\usepackage[T1]{fontenc}
\usepackage{babel}


\newif\ifcomment
%\commenttrue # Show comments

\usepackage{physics}
\usepackage{amssymb}


\usepackage{amsthm}
% \usepackage{thmtools}
\usepackage{mathtools}
\usepackage{amsfonts}

\usepackage{color}

\usepackage{tikz}

\usepackage{geometry}
\geometry{a5paper, margin=0.1in, right=1cm}

\usepackage{dsfont}

\usepackage{graphicx}
\graphicspath{ {images/} }

\usepackage{faktor}

\usepackage{IEEEtrantools}
\usepackage{enumerate}   
\usepackage[PostScript=dvips]{"/Users/aware/Documents/Courses/diagrams"}


\newtheorem{theorem}{Théorème}[section]
\renewcommand{\thetheorem}{\arabic{theorem}}
\newtheorem{lemme}{Lemme}[section]
\renewcommand{\thelemme}{\arabic{lemme}}
\newtheorem{proposition}{Proposition}[section]
\renewcommand{\theproposition}{\arabic{proposition}}
\newtheorem{notations}{Notations}[section]
\newtheorem{problem}{Problème}[section]
\newtheorem{corollary}{Corollaire}[theorem]
\renewcommand{\thecorollary}{\arabic{corollary}}
\newtheorem{property}{Propriété}[section]
\newtheorem{objective}{Objectif}[section]

\theoremstyle{definition}
\newtheorem{definition}{Définition}[section]
\renewcommand{\thedefinition}{\arabic{definition}}
\newtheorem{exercise}{Exercice}[chapter]
\renewcommand{\theexercise}{\arabic{exercise}}
\newtheorem{example}{Exemple}[chapter]
\renewcommand{\theexample}{\arabic{example}}
\newtheorem*{solution}{Solution}
\newtheorem*{application}{Application}
\newtheorem*{notation}{Notation}
\newtheorem*{vocabulary}{Vocabulaire}
\newtheorem*{properties}{Propriétés}



\theoremstyle{remark}
\newtheorem*{remark}{Remarque}
\newtheorem*{rappel}{Rappel}


\usepackage{etoolbox}
\AtBeginEnvironment{exercise}{\small}
\AtBeginEnvironment{example}{\small}

\usepackage{cases}
\usepackage[red]{mypack}

\usepackage[framemethod=TikZ]{mdframed}

\definecolor{bg}{rgb}{0.4,0.25,0.95}
\definecolor{pagebg}{rgb}{0,0,0.5}
\surroundwithmdframed[
   topline=false,
   rightline=false,
   bottomline=false,
   leftmargin=\parindent,
   skipabove=8pt,
   skipbelow=8pt,
   linecolor=blue,
   innerbottommargin=10pt,
   % backgroundcolor=bg,font=\color{orange}\sffamily, fontcolor=white
]{definition}

\usepackage{empheq}
\usepackage[most]{tcolorbox}

\newtcbox{\mymath}[1][]{%
    nobeforeafter, math upper, tcbox raise base,
    enhanced, colframe=blue!30!black,
    colback=red!10, boxrule=1pt,
    #1}

\usepackage{unixode}


\DeclareMathOperator{\ord}{ord}
\DeclareMathOperator{\orb}{orb}
\DeclareMathOperator{\stab}{stab}
\DeclareMathOperator{\Stab}{stab}
\DeclareMathOperator{\ppcm}{ppcm}
\DeclareMathOperator{\conj}{Conj}
\DeclareMathOperator{\End}{End}
\DeclareMathOperator{\rot}{rot}
\DeclareMathOperator{\trs}{trace}
\DeclareMathOperator{\Ind}{Ind}
\DeclareMathOperator{\mat}{Mat}
\DeclareMathOperator{\id}{Id}
\DeclareMathOperator{\vect}{vect}
\DeclareMathOperator{\img}{img}
\DeclareMathOperator{\cov}{Cov}
\DeclareMathOperator{\dist}{dist}
\DeclareMathOperator{\irr}{Irr}
\DeclareMathOperator{\image}{Im}
\DeclareMathOperator{\pd}{\partial}
\DeclareMathOperator{\epi}{epi}
\DeclareMathOperator{\Argmin}{Argmin}
\DeclareMathOperator{\dom}{dom}
\DeclareMathOperator{\proj}{proj}
\DeclareMathOperator{\ctg}{ctg}
\DeclareMathOperator{\supp}{supp}
\DeclareMathOperator{\argmin}{argmin}
\DeclareMathOperator{\mult}{mult}
\DeclareMathOperator{\ch}{ch}
\DeclareMathOperator{\sh}{sh}
\DeclareMathOperator{\rang}{rang}
\DeclareMathOperator{\diam}{diam}
\DeclareMathOperator{\Epigraphe}{Epigraphe}




\usepackage{xcolor}
\everymath{\color{blue}}
%\everymath{\color[rgb]{0,1,1}}
%\pagecolor[rgb]{0,0,0.5}


\newcommand*{\pdtest}[3][]{\ensuremath{\frac{\partial^{#1} #2}{\partial #3}}}

\newcommand*{\deffunc}[6][]{\ensuremath{
\begin{array}{rcl}
#2 : #3 &\rightarrow& #4\\
#5 &\mapsto& #6
\end{array}
}}

\newcommand{\eqcolon}{\mathrel{\resizebox{\widthof{$\mathord{=}$}}{\height}{ $\!\!=\!\!\resizebox{1.2\width}{0.8\height}{\raisebox{0.23ex}{$\mathop{:}$}}\!\!$ }}}
\newcommand{\coloneq}{\mathrel{\resizebox{\widthof{$\mathord{=}$}}{\height}{ $\!\!\resizebox{1.2\width}{0.8\height}{\raisebox{0.23ex}{$\mathop{:}$}}\!\!=\!\!$ }}}
\newcommand{\eqcolonl}{\ensuremath{\mathrel{=\!\!\mathop{:}}}}
\newcommand{\coloneql}{\ensuremath{\mathrel{\mathop{:} \!\! =}}}
\newcommand{\vc}[1]{% inline column vector
  \left(\begin{smallmatrix}#1\end{smallmatrix}\right)%
}
\newcommand{\vr}[1]{% inline row vector
  \begin{smallmatrix}(\,#1\,)\end{smallmatrix}%
}
\makeatletter
\newcommand*{\defeq}{\ =\mathrel{\rlap{%
                     \raisebox{0.3ex}{$\m@th\cdot$}}%
                     \raisebox{-0.3ex}{$\m@th\cdot$}}%
                     }
\makeatother

\newcommand{\mathcircle}[1]{% inline row vector
 \overset{\circ}{#1}
}
\newcommand{\ulim}{% low limit
 \underline{\lim}
}
\newcommand{\ssi}{% iff
\iff
}
\newcommand{\ps}[2]{
\expval{#1 | #2}
}
\newcommand{\df}[1]{
\mqty{#1}
}
\newcommand{\n}[1]{
\norm{#1}
}
\newcommand{\sys}[1]{
\left\{\smqty{#1}\right.
}


\newcommand{\eqdef}{\ensuremath{\overset{\text{def}}=}}


\def\Circlearrowright{\ensuremath{%
  \rotatebox[origin=c]{230}{$\circlearrowright$}}}

\newcommand\ct[1]{\text{\rmfamily\upshape #1}}
\newcommand\question[1]{ {\color{red} ...!? \small #1}}
\newcommand\caz[1]{\left\{\begin{array} #1 \end{array}\right.}
\newcommand\const{\text{\rmfamily\upshape const}}
\newcommand\toP{ \overset{\pro}{\to}}
\newcommand\toPP{ \overset{\text{PP}}{\to}}
\newcommand{\oeq}{\mathrel{\text{\textcircled{$=$}}}}





\usepackage{xcolor}
% \usepackage[normalem]{ulem}
\usepackage{lipsum}
\makeatletter
% \newcommand\colorwave[1][blue]{\bgroup \markoverwith{\lower3.5\p@\hbox{\sixly \textcolor{#1}{\char58}}}\ULon}
%\font\sixly=lasy6 % does not re-load if already loaded, so no memory problem.

\newmdtheoremenv[
linewidth= 1pt,linecolor= blue,%
leftmargin=20,rightmargin=20,innertopmargin=0pt, innerrightmargin=40,%
tikzsetting = { draw=lightgray, line width = 0.3pt,dashed,%
dash pattern = on 15pt off 3pt},%
splittopskip=\topskip,skipbelow=\baselineskip,%
skipabove=\baselineskip,ntheorem,roundcorner=0pt,
% backgroundcolor=pagebg,font=\color{orange}\sffamily, fontcolor=white
]{examplebox}{Exemple}[section]



\newcommand\R{\mathbb{R}}
\newcommand\Z{\mathbb{Z}}
\newcommand\N{\mathbb{N}}
\newcommand\E{\mathbb{E}}
\newcommand\F{\mathcal{F}}
\newcommand\cH{\mathcal{H}}
\newcommand\V{\mathbb{V}}
\newcommand\dmo{ ^{-1} }
\newcommand\kapa{\kappa}
\newcommand\im{Im}
\newcommand\hs{\mathcal{H}}





\usepackage{soul}

\makeatletter
\newcommand*{\whiten}[1]{\llap{\textcolor{white}{{\the\SOUL@token}}\hspace{#1pt}}}
\DeclareRobustCommand*\myul{%
    \def\SOUL@everyspace{\underline{\space}\kern\z@}%
    \def\SOUL@everytoken{%
     \setbox0=\hbox{\the\SOUL@token}%
     \ifdim\dp0>\z@
        \raisebox{\dp0}{\underline{\phantom{\the\SOUL@token}}}%
        \whiten{1}\whiten{0}%
        \whiten{-1}\whiten{-2}%
        \llap{\the\SOUL@token}%
     \else
        \underline{\the\SOUL@token}%
     \fi}%
\SOUL@}
\makeatother

\newcommand*{\demp}{\fontfamily{lmtt}\selectfont}

\DeclareTextFontCommand{\textdemp}{\demp}

\begin{document}

\ifcomment
Multiline
comment
\fi
\ifcomment
\myul{Typesetting test}
% \color[rgb]{1,1,1}
$∑_i^n≠ 60º±∞π∆¬≈√j∫h≤≥µ$

$\CR \R\pro\ind\pro\gS\pro
\mqty[a&b\\c&d]$
$\pro\mathbb{P}$
$\dd{x}$

  \[
    \alpha(x)=\left\{
                \begin{array}{ll}
                  x\\
                  \frac{1}{1+e^{-kx}}\\
                  \frac{e^x-e^{-x}}{e^x+e^{-x}}
                \end{array}
              \right.
  \]

  $\expval{x}$
  
  $\chi_\rho(ghg\dmo)=\Tr(\rho_{ghg\dmo})=\Tr(\rho_g\circ\rho_h\circ\rho\dmo_g)=\Tr(\rho_h)\overset{\mbox{\scalebox{0.5}{$\Tr(AB)=\Tr(BA)$}}}{=}\chi_\rho(h)$
  	$\mathop{\oplus}_{\substack{x\in X}}$

$\mat(\rho_g)=(a_{ij}(g))_{\scriptsize \substack{1\leq i\leq d \\ 1\leq j\leq d}}$ et $\mat(\rho'_g)=(a'_{ij}(g))_{\scriptsize \substack{1\leq i'\leq d' \\ 1\leq j'\leq d'}}$



\[\int_a^b{\mathbb{R}^2}g(u, v)\dd{P_{XY}}(u, v)=\iint g(u,v) f_{XY}(u, v)\dd \lambda(u) \dd \lambda(v)\]
$$\lim_{x\to\infty} f(x)$$	
$$\iiiint_V \mu(t,u,v,w) \,dt\,du\,dv\,dw$$
$$\sum_{n=1}^{\infty} 2^{-n} = 1$$	
\begin{definition}
	Si $X$ et $Y$ sont 2 v.a. ou definit la \textsc{Covariance} entre $X$ et $Y$ comme
	$\cov(X,Y)\overset{\text{def}}{=}\E\left[(X-\E(X))(Y-\E(Y))\right]=\E(XY)-\E(X)\E(Y)$.
\end{definition}
\fi
\pagebreak

% \tableofcontents

% insert your code here
%\input{./algebra/main.tex}
%\input{./geometrie-differentielle/main.tex}
%\input{./probabilite/main.tex}
%\input{./analyse-fonctionnelle/main.tex}
% \input{./Analyse-convexe-et-dualite-en-optimisation/main.tex}
%\input{./tikz/main.tex}
%\input{./Theorie-du-distributions/main.tex}
%\input{./optimisation/mine.tex}
 \input{./modelisation/main.tex}

% yves.aubry@univ-tln.fr : algebra

\end{document}

%% !TEX encoding = UTF-8 Unicode
% !TEX TS-program = xelatex

\documentclass[french]{report}

%\usepackage[utf8]{inputenc}
%\usepackage[T1]{fontenc}
\usepackage{babel}


\newif\ifcomment
%\commenttrue # Show comments

\usepackage{physics}
\usepackage{amssymb}


\usepackage{amsthm}
% \usepackage{thmtools}
\usepackage{mathtools}
\usepackage{amsfonts}

\usepackage{color}

\usepackage{tikz}

\usepackage{geometry}
\geometry{a5paper, margin=0.1in, right=1cm}

\usepackage{dsfont}

\usepackage{graphicx}
\graphicspath{ {images/} }

\usepackage{faktor}

\usepackage{IEEEtrantools}
\usepackage{enumerate}   
\usepackage[PostScript=dvips]{"/Users/aware/Documents/Courses/diagrams"}


\newtheorem{theorem}{Théorème}[section]
\renewcommand{\thetheorem}{\arabic{theorem}}
\newtheorem{lemme}{Lemme}[section]
\renewcommand{\thelemme}{\arabic{lemme}}
\newtheorem{proposition}{Proposition}[section]
\renewcommand{\theproposition}{\arabic{proposition}}
\newtheorem{notations}{Notations}[section]
\newtheorem{problem}{Problème}[section]
\newtheorem{corollary}{Corollaire}[theorem]
\renewcommand{\thecorollary}{\arabic{corollary}}
\newtheorem{property}{Propriété}[section]
\newtheorem{objective}{Objectif}[section]

\theoremstyle{definition}
\newtheorem{definition}{Définition}[section]
\renewcommand{\thedefinition}{\arabic{definition}}
\newtheorem{exercise}{Exercice}[chapter]
\renewcommand{\theexercise}{\arabic{exercise}}
\newtheorem{example}{Exemple}[chapter]
\renewcommand{\theexample}{\arabic{example}}
\newtheorem*{solution}{Solution}
\newtheorem*{application}{Application}
\newtheorem*{notation}{Notation}
\newtheorem*{vocabulary}{Vocabulaire}
\newtheorem*{properties}{Propriétés}



\theoremstyle{remark}
\newtheorem*{remark}{Remarque}
\newtheorem*{rappel}{Rappel}


\usepackage{etoolbox}
\AtBeginEnvironment{exercise}{\small}
\AtBeginEnvironment{example}{\small}

\usepackage{cases}
\usepackage[red]{mypack}

\usepackage[framemethod=TikZ]{mdframed}

\definecolor{bg}{rgb}{0.4,0.25,0.95}
\definecolor{pagebg}{rgb}{0,0,0.5}
\surroundwithmdframed[
   topline=false,
   rightline=false,
   bottomline=false,
   leftmargin=\parindent,
   skipabove=8pt,
   skipbelow=8pt,
   linecolor=blue,
   innerbottommargin=10pt,
   % backgroundcolor=bg,font=\color{orange}\sffamily, fontcolor=white
]{definition}

\usepackage{empheq}
\usepackage[most]{tcolorbox}

\newtcbox{\mymath}[1][]{%
    nobeforeafter, math upper, tcbox raise base,
    enhanced, colframe=blue!30!black,
    colback=red!10, boxrule=1pt,
    #1}

\usepackage{unixode}


\DeclareMathOperator{\ord}{ord}
\DeclareMathOperator{\orb}{orb}
\DeclareMathOperator{\stab}{stab}
\DeclareMathOperator{\Stab}{stab}
\DeclareMathOperator{\ppcm}{ppcm}
\DeclareMathOperator{\conj}{Conj}
\DeclareMathOperator{\End}{End}
\DeclareMathOperator{\rot}{rot}
\DeclareMathOperator{\trs}{trace}
\DeclareMathOperator{\Ind}{Ind}
\DeclareMathOperator{\mat}{Mat}
\DeclareMathOperator{\id}{Id}
\DeclareMathOperator{\vect}{vect}
\DeclareMathOperator{\img}{img}
\DeclareMathOperator{\cov}{Cov}
\DeclareMathOperator{\dist}{dist}
\DeclareMathOperator{\irr}{Irr}
\DeclareMathOperator{\image}{Im}
\DeclareMathOperator{\pd}{\partial}
\DeclareMathOperator{\epi}{epi}
\DeclareMathOperator{\Argmin}{Argmin}
\DeclareMathOperator{\dom}{dom}
\DeclareMathOperator{\proj}{proj}
\DeclareMathOperator{\ctg}{ctg}
\DeclareMathOperator{\supp}{supp}
\DeclareMathOperator{\argmin}{argmin}
\DeclareMathOperator{\mult}{mult}
\DeclareMathOperator{\ch}{ch}
\DeclareMathOperator{\sh}{sh}
\DeclareMathOperator{\rang}{rang}
\DeclareMathOperator{\diam}{diam}
\DeclareMathOperator{\Epigraphe}{Epigraphe}




\usepackage{xcolor}
\everymath{\color{blue}}
%\everymath{\color[rgb]{0,1,1}}
%\pagecolor[rgb]{0,0,0.5}


\newcommand*{\pdtest}[3][]{\ensuremath{\frac{\partial^{#1} #2}{\partial #3}}}

\newcommand*{\deffunc}[6][]{\ensuremath{
\begin{array}{rcl}
#2 : #3 &\rightarrow& #4\\
#5 &\mapsto& #6
\end{array}
}}

\newcommand{\eqcolon}{\mathrel{\resizebox{\widthof{$\mathord{=}$}}{\height}{ $\!\!=\!\!\resizebox{1.2\width}{0.8\height}{\raisebox{0.23ex}{$\mathop{:}$}}\!\!$ }}}
\newcommand{\coloneq}{\mathrel{\resizebox{\widthof{$\mathord{=}$}}{\height}{ $\!\!\resizebox{1.2\width}{0.8\height}{\raisebox{0.23ex}{$\mathop{:}$}}\!\!=\!\!$ }}}
\newcommand{\eqcolonl}{\ensuremath{\mathrel{=\!\!\mathop{:}}}}
\newcommand{\coloneql}{\ensuremath{\mathrel{\mathop{:} \!\! =}}}
\newcommand{\vc}[1]{% inline column vector
  \left(\begin{smallmatrix}#1\end{smallmatrix}\right)%
}
\newcommand{\vr}[1]{% inline row vector
  \begin{smallmatrix}(\,#1\,)\end{smallmatrix}%
}
\makeatletter
\newcommand*{\defeq}{\ =\mathrel{\rlap{%
                     \raisebox{0.3ex}{$\m@th\cdot$}}%
                     \raisebox{-0.3ex}{$\m@th\cdot$}}%
                     }
\makeatother

\newcommand{\mathcircle}[1]{% inline row vector
 \overset{\circ}{#1}
}
\newcommand{\ulim}{% low limit
 \underline{\lim}
}
\newcommand{\ssi}{% iff
\iff
}
\newcommand{\ps}[2]{
\expval{#1 | #2}
}
\newcommand{\df}[1]{
\mqty{#1}
}
\newcommand{\n}[1]{
\norm{#1}
}
\newcommand{\sys}[1]{
\left\{\smqty{#1}\right.
}


\newcommand{\eqdef}{\ensuremath{\overset{\text{def}}=}}


\def\Circlearrowright{\ensuremath{%
  \rotatebox[origin=c]{230}{$\circlearrowright$}}}

\newcommand\ct[1]{\text{\rmfamily\upshape #1}}
\newcommand\question[1]{ {\color{red} ...!? \small #1}}
\newcommand\caz[1]{\left\{\begin{array} #1 \end{array}\right.}
\newcommand\const{\text{\rmfamily\upshape const}}
\newcommand\toP{ \overset{\pro}{\to}}
\newcommand\toPP{ \overset{\text{PP}}{\to}}
\newcommand{\oeq}{\mathrel{\text{\textcircled{$=$}}}}





\usepackage{xcolor}
% \usepackage[normalem]{ulem}
\usepackage{lipsum}
\makeatletter
% \newcommand\colorwave[1][blue]{\bgroup \markoverwith{\lower3.5\p@\hbox{\sixly \textcolor{#1}{\char58}}}\ULon}
%\font\sixly=lasy6 % does not re-load if already loaded, so no memory problem.

\newmdtheoremenv[
linewidth= 1pt,linecolor= blue,%
leftmargin=20,rightmargin=20,innertopmargin=0pt, innerrightmargin=40,%
tikzsetting = { draw=lightgray, line width = 0.3pt,dashed,%
dash pattern = on 15pt off 3pt},%
splittopskip=\topskip,skipbelow=\baselineskip,%
skipabove=\baselineskip,ntheorem,roundcorner=0pt,
% backgroundcolor=pagebg,font=\color{orange}\sffamily, fontcolor=white
]{examplebox}{Exemple}[section]



\newcommand\R{\mathbb{R}}
\newcommand\Z{\mathbb{Z}}
\newcommand\N{\mathbb{N}}
\newcommand\E{\mathbb{E}}
\newcommand\F{\mathcal{F}}
\newcommand\cH{\mathcal{H}}
\newcommand\V{\mathbb{V}}
\newcommand\dmo{ ^{-1} }
\newcommand\kapa{\kappa}
\newcommand\im{Im}
\newcommand\hs{\mathcal{H}}





\usepackage{soul}

\makeatletter
\newcommand*{\whiten}[1]{\llap{\textcolor{white}{{\the\SOUL@token}}\hspace{#1pt}}}
\DeclareRobustCommand*\myul{%
    \def\SOUL@everyspace{\underline{\space}\kern\z@}%
    \def\SOUL@everytoken{%
     \setbox0=\hbox{\the\SOUL@token}%
     \ifdim\dp0>\z@
        \raisebox{\dp0}{\underline{\phantom{\the\SOUL@token}}}%
        \whiten{1}\whiten{0}%
        \whiten{-1}\whiten{-2}%
        \llap{\the\SOUL@token}%
     \else
        \underline{\the\SOUL@token}%
     \fi}%
\SOUL@}
\makeatother

\newcommand*{\demp}{\fontfamily{lmtt}\selectfont}

\DeclareTextFontCommand{\textdemp}{\demp}

\begin{document}

\ifcomment
Multiline
comment
\fi
\ifcomment
\myul{Typesetting test}
% \color[rgb]{1,1,1}
$∑_i^n≠ 60º±∞π∆¬≈√j∫h≤≥µ$

$\CR \R\pro\ind\pro\gS\pro
\mqty[a&b\\c&d]$
$\pro\mathbb{P}$
$\dd{x}$

  \[
    \alpha(x)=\left\{
                \begin{array}{ll}
                  x\\
                  \frac{1}{1+e^{-kx}}\\
                  \frac{e^x-e^{-x}}{e^x+e^{-x}}
                \end{array}
              \right.
  \]

  $\expval{x}$
  
  $\chi_\rho(ghg\dmo)=\Tr(\rho_{ghg\dmo})=\Tr(\rho_g\circ\rho_h\circ\rho\dmo_g)=\Tr(\rho_h)\overset{\mbox{\scalebox{0.5}{$\Tr(AB)=\Tr(BA)$}}}{=}\chi_\rho(h)$
  	$\mathop{\oplus}_{\substack{x\in X}}$

$\mat(\rho_g)=(a_{ij}(g))_{\scriptsize \substack{1\leq i\leq d \\ 1\leq j\leq d}}$ et $\mat(\rho'_g)=(a'_{ij}(g))_{\scriptsize \substack{1\leq i'\leq d' \\ 1\leq j'\leq d'}}$



\[\int_a^b{\mathbb{R}^2}g(u, v)\dd{P_{XY}}(u, v)=\iint g(u,v) f_{XY}(u, v)\dd \lambda(u) \dd \lambda(v)\]
$$\lim_{x\to\infty} f(x)$$	
$$\iiiint_V \mu(t,u,v,w) \,dt\,du\,dv\,dw$$
$$\sum_{n=1}^{\infty} 2^{-n} = 1$$	
\begin{definition}
	Si $X$ et $Y$ sont 2 v.a. ou definit la \textsc{Covariance} entre $X$ et $Y$ comme
	$\cov(X,Y)\overset{\text{def}}{=}\E\left[(X-\E(X))(Y-\E(Y))\right]=\E(XY)-\E(X)\E(Y)$.
\end{definition}
\fi
\pagebreak

% \tableofcontents

% insert your code here
%\input{./algebra/main.tex}
%\input{./geometrie-differentielle/main.tex}
%\input{./probabilite/main.tex}
%\input{./analyse-fonctionnelle/main.tex}
% \input{./Analyse-convexe-et-dualite-en-optimisation/main.tex}
%\input{./tikz/main.tex}
%\input{./Theorie-du-distributions/main.tex}
%\input{./optimisation/mine.tex}
 \input{./modelisation/main.tex}

% yves.aubry@univ-tln.fr : algebra

\end{document}

%% !TEX encoding = UTF-8 Unicode
% !TEX TS-program = xelatex

\documentclass[french]{report}

%\usepackage[utf8]{inputenc}
%\usepackage[T1]{fontenc}
\usepackage{babel}


\newif\ifcomment
%\commenttrue # Show comments

\usepackage{physics}
\usepackage{amssymb}


\usepackage{amsthm}
% \usepackage{thmtools}
\usepackage{mathtools}
\usepackage{amsfonts}

\usepackage{color}

\usepackage{tikz}

\usepackage{geometry}
\geometry{a5paper, margin=0.1in, right=1cm}

\usepackage{dsfont}

\usepackage{graphicx}
\graphicspath{ {images/} }

\usepackage{faktor}

\usepackage{IEEEtrantools}
\usepackage{enumerate}   
\usepackage[PostScript=dvips]{"/Users/aware/Documents/Courses/diagrams"}


\newtheorem{theorem}{Théorème}[section]
\renewcommand{\thetheorem}{\arabic{theorem}}
\newtheorem{lemme}{Lemme}[section]
\renewcommand{\thelemme}{\arabic{lemme}}
\newtheorem{proposition}{Proposition}[section]
\renewcommand{\theproposition}{\arabic{proposition}}
\newtheorem{notations}{Notations}[section]
\newtheorem{problem}{Problème}[section]
\newtheorem{corollary}{Corollaire}[theorem]
\renewcommand{\thecorollary}{\arabic{corollary}}
\newtheorem{property}{Propriété}[section]
\newtheorem{objective}{Objectif}[section]

\theoremstyle{definition}
\newtheorem{definition}{Définition}[section]
\renewcommand{\thedefinition}{\arabic{definition}}
\newtheorem{exercise}{Exercice}[chapter]
\renewcommand{\theexercise}{\arabic{exercise}}
\newtheorem{example}{Exemple}[chapter]
\renewcommand{\theexample}{\arabic{example}}
\newtheorem*{solution}{Solution}
\newtheorem*{application}{Application}
\newtheorem*{notation}{Notation}
\newtheorem*{vocabulary}{Vocabulaire}
\newtheorem*{properties}{Propriétés}



\theoremstyle{remark}
\newtheorem*{remark}{Remarque}
\newtheorem*{rappel}{Rappel}


\usepackage{etoolbox}
\AtBeginEnvironment{exercise}{\small}
\AtBeginEnvironment{example}{\small}

\usepackage{cases}
\usepackage[red]{mypack}

\usepackage[framemethod=TikZ]{mdframed}

\definecolor{bg}{rgb}{0.4,0.25,0.95}
\definecolor{pagebg}{rgb}{0,0,0.5}
\surroundwithmdframed[
   topline=false,
   rightline=false,
   bottomline=false,
   leftmargin=\parindent,
   skipabove=8pt,
   skipbelow=8pt,
   linecolor=blue,
   innerbottommargin=10pt,
   % backgroundcolor=bg,font=\color{orange}\sffamily, fontcolor=white
]{definition}

\usepackage{empheq}
\usepackage[most]{tcolorbox}

\newtcbox{\mymath}[1][]{%
    nobeforeafter, math upper, tcbox raise base,
    enhanced, colframe=blue!30!black,
    colback=red!10, boxrule=1pt,
    #1}

\usepackage{unixode}


\DeclareMathOperator{\ord}{ord}
\DeclareMathOperator{\orb}{orb}
\DeclareMathOperator{\stab}{stab}
\DeclareMathOperator{\Stab}{stab}
\DeclareMathOperator{\ppcm}{ppcm}
\DeclareMathOperator{\conj}{Conj}
\DeclareMathOperator{\End}{End}
\DeclareMathOperator{\rot}{rot}
\DeclareMathOperator{\trs}{trace}
\DeclareMathOperator{\Ind}{Ind}
\DeclareMathOperator{\mat}{Mat}
\DeclareMathOperator{\id}{Id}
\DeclareMathOperator{\vect}{vect}
\DeclareMathOperator{\img}{img}
\DeclareMathOperator{\cov}{Cov}
\DeclareMathOperator{\dist}{dist}
\DeclareMathOperator{\irr}{Irr}
\DeclareMathOperator{\image}{Im}
\DeclareMathOperator{\pd}{\partial}
\DeclareMathOperator{\epi}{epi}
\DeclareMathOperator{\Argmin}{Argmin}
\DeclareMathOperator{\dom}{dom}
\DeclareMathOperator{\proj}{proj}
\DeclareMathOperator{\ctg}{ctg}
\DeclareMathOperator{\supp}{supp}
\DeclareMathOperator{\argmin}{argmin}
\DeclareMathOperator{\mult}{mult}
\DeclareMathOperator{\ch}{ch}
\DeclareMathOperator{\sh}{sh}
\DeclareMathOperator{\rang}{rang}
\DeclareMathOperator{\diam}{diam}
\DeclareMathOperator{\Epigraphe}{Epigraphe}




\usepackage{xcolor}
\everymath{\color{blue}}
%\everymath{\color[rgb]{0,1,1}}
%\pagecolor[rgb]{0,0,0.5}


\newcommand*{\pdtest}[3][]{\ensuremath{\frac{\partial^{#1} #2}{\partial #3}}}

\newcommand*{\deffunc}[6][]{\ensuremath{
\begin{array}{rcl}
#2 : #3 &\rightarrow& #4\\
#5 &\mapsto& #6
\end{array}
}}

\newcommand{\eqcolon}{\mathrel{\resizebox{\widthof{$\mathord{=}$}}{\height}{ $\!\!=\!\!\resizebox{1.2\width}{0.8\height}{\raisebox{0.23ex}{$\mathop{:}$}}\!\!$ }}}
\newcommand{\coloneq}{\mathrel{\resizebox{\widthof{$\mathord{=}$}}{\height}{ $\!\!\resizebox{1.2\width}{0.8\height}{\raisebox{0.23ex}{$\mathop{:}$}}\!\!=\!\!$ }}}
\newcommand{\eqcolonl}{\ensuremath{\mathrel{=\!\!\mathop{:}}}}
\newcommand{\coloneql}{\ensuremath{\mathrel{\mathop{:} \!\! =}}}
\newcommand{\vc}[1]{% inline column vector
  \left(\begin{smallmatrix}#1\end{smallmatrix}\right)%
}
\newcommand{\vr}[1]{% inline row vector
  \begin{smallmatrix}(\,#1\,)\end{smallmatrix}%
}
\makeatletter
\newcommand*{\defeq}{\ =\mathrel{\rlap{%
                     \raisebox{0.3ex}{$\m@th\cdot$}}%
                     \raisebox{-0.3ex}{$\m@th\cdot$}}%
                     }
\makeatother

\newcommand{\mathcircle}[1]{% inline row vector
 \overset{\circ}{#1}
}
\newcommand{\ulim}{% low limit
 \underline{\lim}
}
\newcommand{\ssi}{% iff
\iff
}
\newcommand{\ps}[2]{
\expval{#1 | #2}
}
\newcommand{\df}[1]{
\mqty{#1}
}
\newcommand{\n}[1]{
\norm{#1}
}
\newcommand{\sys}[1]{
\left\{\smqty{#1}\right.
}


\newcommand{\eqdef}{\ensuremath{\overset{\text{def}}=}}


\def\Circlearrowright{\ensuremath{%
  \rotatebox[origin=c]{230}{$\circlearrowright$}}}

\newcommand\ct[1]{\text{\rmfamily\upshape #1}}
\newcommand\question[1]{ {\color{red} ...!? \small #1}}
\newcommand\caz[1]{\left\{\begin{array} #1 \end{array}\right.}
\newcommand\const{\text{\rmfamily\upshape const}}
\newcommand\toP{ \overset{\pro}{\to}}
\newcommand\toPP{ \overset{\text{PP}}{\to}}
\newcommand{\oeq}{\mathrel{\text{\textcircled{$=$}}}}





\usepackage{xcolor}
% \usepackage[normalem]{ulem}
\usepackage{lipsum}
\makeatletter
% \newcommand\colorwave[1][blue]{\bgroup \markoverwith{\lower3.5\p@\hbox{\sixly \textcolor{#1}{\char58}}}\ULon}
%\font\sixly=lasy6 % does not re-load if already loaded, so no memory problem.

\newmdtheoremenv[
linewidth= 1pt,linecolor= blue,%
leftmargin=20,rightmargin=20,innertopmargin=0pt, innerrightmargin=40,%
tikzsetting = { draw=lightgray, line width = 0.3pt,dashed,%
dash pattern = on 15pt off 3pt},%
splittopskip=\topskip,skipbelow=\baselineskip,%
skipabove=\baselineskip,ntheorem,roundcorner=0pt,
% backgroundcolor=pagebg,font=\color{orange}\sffamily, fontcolor=white
]{examplebox}{Exemple}[section]



\newcommand\R{\mathbb{R}}
\newcommand\Z{\mathbb{Z}}
\newcommand\N{\mathbb{N}}
\newcommand\E{\mathbb{E}}
\newcommand\F{\mathcal{F}}
\newcommand\cH{\mathcal{H}}
\newcommand\V{\mathbb{V}}
\newcommand\dmo{ ^{-1} }
\newcommand\kapa{\kappa}
\newcommand\im{Im}
\newcommand\hs{\mathcal{H}}





\usepackage{soul}

\makeatletter
\newcommand*{\whiten}[1]{\llap{\textcolor{white}{{\the\SOUL@token}}\hspace{#1pt}}}
\DeclareRobustCommand*\myul{%
    \def\SOUL@everyspace{\underline{\space}\kern\z@}%
    \def\SOUL@everytoken{%
     \setbox0=\hbox{\the\SOUL@token}%
     \ifdim\dp0>\z@
        \raisebox{\dp0}{\underline{\phantom{\the\SOUL@token}}}%
        \whiten{1}\whiten{0}%
        \whiten{-1}\whiten{-2}%
        \llap{\the\SOUL@token}%
     \else
        \underline{\the\SOUL@token}%
     \fi}%
\SOUL@}
\makeatother

\newcommand*{\demp}{\fontfamily{lmtt}\selectfont}

\DeclareTextFontCommand{\textdemp}{\demp}

\begin{document}

\ifcomment
Multiline
comment
\fi
\ifcomment
\myul{Typesetting test}
% \color[rgb]{1,1,1}
$∑_i^n≠ 60º±∞π∆¬≈√j∫h≤≥µ$

$\CR \R\pro\ind\pro\gS\pro
\mqty[a&b\\c&d]$
$\pro\mathbb{P}$
$\dd{x}$

  \[
    \alpha(x)=\left\{
                \begin{array}{ll}
                  x\\
                  \frac{1}{1+e^{-kx}}\\
                  \frac{e^x-e^{-x}}{e^x+e^{-x}}
                \end{array}
              \right.
  \]

  $\expval{x}$
  
  $\chi_\rho(ghg\dmo)=\Tr(\rho_{ghg\dmo})=\Tr(\rho_g\circ\rho_h\circ\rho\dmo_g)=\Tr(\rho_h)\overset{\mbox{\scalebox{0.5}{$\Tr(AB)=\Tr(BA)$}}}{=}\chi_\rho(h)$
  	$\mathop{\oplus}_{\substack{x\in X}}$

$\mat(\rho_g)=(a_{ij}(g))_{\scriptsize \substack{1\leq i\leq d \\ 1\leq j\leq d}}$ et $\mat(\rho'_g)=(a'_{ij}(g))_{\scriptsize \substack{1\leq i'\leq d' \\ 1\leq j'\leq d'}}$



\[\int_a^b{\mathbb{R}^2}g(u, v)\dd{P_{XY}}(u, v)=\iint g(u,v) f_{XY}(u, v)\dd \lambda(u) \dd \lambda(v)\]
$$\lim_{x\to\infty} f(x)$$	
$$\iiiint_V \mu(t,u,v,w) \,dt\,du\,dv\,dw$$
$$\sum_{n=1}^{\infty} 2^{-n} = 1$$	
\begin{definition}
	Si $X$ et $Y$ sont 2 v.a. ou definit la \textsc{Covariance} entre $X$ et $Y$ comme
	$\cov(X,Y)\overset{\text{def}}{=}\E\left[(X-\E(X))(Y-\E(Y))\right]=\E(XY)-\E(X)\E(Y)$.
\end{definition}
\fi
\pagebreak

% \tableofcontents

% insert your code here
%\input{./algebra/main.tex}
%\input{./geometrie-differentielle/main.tex}
%\input{./probabilite/main.tex}
%\input{./analyse-fonctionnelle/main.tex}
% \input{./Analyse-convexe-et-dualite-en-optimisation/main.tex}
%\input{./tikz/main.tex}
%\input{./Theorie-du-distributions/main.tex}
%\input{./optimisation/mine.tex}
 \input{./modelisation/main.tex}

% yves.aubry@univ-tln.fr : algebra

\end{document}

% % !TEX encoding = UTF-8 Unicode
% !TEX TS-program = xelatex

\documentclass[french]{report}

%\usepackage[utf8]{inputenc}
%\usepackage[T1]{fontenc}
\usepackage{babel}


\newif\ifcomment
%\commenttrue # Show comments

\usepackage{physics}
\usepackage{amssymb}


\usepackage{amsthm}
% \usepackage{thmtools}
\usepackage{mathtools}
\usepackage{amsfonts}

\usepackage{color}

\usepackage{tikz}

\usepackage{geometry}
\geometry{a5paper, margin=0.1in, right=1cm}

\usepackage{dsfont}

\usepackage{graphicx}
\graphicspath{ {images/} }

\usepackage{faktor}

\usepackage{IEEEtrantools}
\usepackage{enumerate}   
\usepackage[PostScript=dvips]{"/Users/aware/Documents/Courses/diagrams"}


\newtheorem{theorem}{Théorème}[section]
\renewcommand{\thetheorem}{\arabic{theorem}}
\newtheorem{lemme}{Lemme}[section]
\renewcommand{\thelemme}{\arabic{lemme}}
\newtheorem{proposition}{Proposition}[section]
\renewcommand{\theproposition}{\arabic{proposition}}
\newtheorem{notations}{Notations}[section]
\newtheorem{problem}{Problème}[section]
\newtheorem{corollary}{Corollaire}[theorem]
\renewcommand{\thecorollary}{\arabic{corollary}}
\newtheorem{property}{Propriété}[section]
\newtheorem{objective}{Objectif}[section]

\theoremstyle{definition}
\newtheorem{definition}{Définition}[section]
\renewcommand{\thedefinition}{\arabic{definition}}
\newtheorem{exercise}{Exercice}[chapter]
\renewcommand{\theexercise}{\arabic{exercise}}
\newtheorem{example}{Exemple}[chapter]
\renewcommand{\theexample}{\arabic{example}}
\newtheorem*{solution}{Solution}
\newtheorem*{application}{Application}
\newtheorem*{notation}{Notation}
\newtheorem*{vocabulary}{Vocabulaire}
\newtheorem*{properties}{Propriétés}



\theoremstyle{remark}
\newtheorem*{remark}{Remarque}
\newtheorem*{rappel}{Rappel}


\usepackage{etoolbox}
\AtBeginEnvironment{exercise}{\small}
\AtBeginEnvironment{example}{\small}

\usepackage{cases}
\usepackage[red]{mypack}

\usepackage[framemethod=TikZ]{mdframed}

\definecolor{bg}{rgb}{0.4,0.25,0.95}
\definecolor{pagebg}{rgb}{0,0,0.5}
\surroundwithmdframed[
   topline=false,
   rightline=false,
   bottomline=false,
   leftmargin=\parindent,
   skipabove=8pt,
   skipbelow=8pt,
   linecolor=blue,
   innerbottommargin=10pt,
   % backgroundcolor=bg,font=\color{orange}\sffamily, fontcolor=white
]{definition}

\usepackage{empheq}
\usepackage[most]{tcolorbox}

\newtcbox{\mymath}[1][]{%
    nobeforeafter, math upper, tcbox raise base,
    enhanced, colframe=blue!30!black,
    colback=red!10, boxrule=1pt,
    #1}

\usepackage{unixode}


\DeclareMathOperator{\ord}{ord}
\DeclareMathOperator{\orb}{orb}
\DeclareMathOperator{\stab}{stab}
\DeclareMathOperator{\Stab}{stab}
\DeclareMathOperator{\ppcm}{ppcm}
\DeclareMathOperator{\conj}{Conj}
\DeclareMathOperator{\End}{End}
\DeclareMathOperator{\rot}{rot}
\DeclareMathOperator{\trs}{trace}
\DeclareMathOperator{\Ind}{Ind}
\DeclareMathOperator{\mat}{Mat}
\DeclareMathOperator{\id}{Id}
\DeclareMathOperator{\vect}{vect}
\DeclareMathOperator{\img}{img}
\DeclareMathOperator{\cov}{Cov}
\DeclareMathOperator{\dist}{dist}
\DeclareMathOperator{\irr}{Irr}
\DeclareMathOperator{\image}{Im}
\DeclareMathOperator{\pd}{\partial}
\DeclareMathOperator{\epi}{epi}
\DeclareMathOperator{\Argmin}{Argmin}
\DeclareMathOperator{\dom}{dom}
\DeclareMathOperator{\proj}{proj}
\DeclareMathOperator{\ctg}{ctg}
\DeclareMathOperator{\supp}{supp}
\DeclareMathOperator{\argmin}{argmin}
\DeclareMathOperator{\mult}{mult}
\DeclareMathOperator{\ch}{ch}
\DeclareMathOperator{\sh}{sh}
\DeclareMathOperator{\rang}{rang}
\DeclareMathOperator{\diam}{diam}
\DeclareMathOperator{\Epigraphe}{Epigraphe}




\usepackage{xcolor}
\everymath{\color{blue}}
%\everymath{\color[rgb]{0,1,1}}
%\pagecolor[rgb]{0,0,0.5}


\newcommand*{\pdtest}[3][]{\ensuremath{\frac{\partial^{#1} #2}{\partial #3}}}

\newcommand*{\deffunc}[6][]{\ensuremath{
\begin{array}{rcl}
#2 : #3 &\rightarrow& #4\\
#5 &\mapsto& #6
\end{array}
}}

\newcommand{\eqcolon}{\mathrel{\resizebox{\widthof{$\mathord{=}$}}{\height}{ $\!\!=\!\!\resizebox{1.2\width}{0.8\height}{\raisebox{0.23ex}{$\mathop{:}$}}\!\!$ }}}
\newcommand{\coloneq}{\mathrel{\resizebox{\widthof{$\mathord{=}$}}{\height}{ $\!\!\resizebox{1.2\width}{0.8\height}{\raisebox{0.23ex}{$\mathop{:}$}}\!\!=\!\!$ }}}
\newcommand{\eqcolonl}{\ensuremath{\mathrel{=\!\!\mathop{:}}}}
\newcommand{\coloneql}{\ensuremath{\mathrel{\mathop{:} \!\! =}}}
\newcommand{\vc}[1]{% inline column vector
  \left(\begin{smallmatrix}#1\end{smallmatrix}\right)%
}
\newcommand{\vr}[1]{% inline row vector
  \begin{smallmatrix}(\,#1\,)\end{smallmatrix}%
}
\makeatletter
\newcommand*{\defeq}{\ =\mathrel{\rlap{%
                     \raisebox{0.3ex}{$\m@th\cdot$}}%
                     \raisebox{-0.3ex}{$\m@th\cdot$}}%
                     }
\makeatother

\newcommand{\mathcircle}[1]{% inline row vector
 \overset{\circ}{#1}
}
\newcommand{\ulim}{% low limit
 \underline{\lim}
}
\newcommand{\ssi}{% iff
\iff
}
\newcommand{\ps}[2]{
\expval{#1 | #2}
}
\newcommand{\df}[1]{
\mqty{#1}
}
\newcommand{\n}[1]{
\norm{#1}
}
\newcommand{\sys}[1]{
\left\{\smqty{#1}\right.
}


\newcommand{\eqdef}{\ensuremath{\overset{\text{def}}=}}


\def\Circlearrowright{\ensuremath{%
  \rotatebox[origin=c]{230}{$\circlearrowright$}}}

\newcommand\ct[1]{\text{\rmfamily\upshape #1}}
\newcommand\question[1]{ {\color{red} ...!? \small #1}}
\newcommand\caz[1]{\left\{\begin{array} #1 \end{array}\right.}
\newcommand\const{\text{\rmfamily\upshape const}}
\newcommand\toP{ \overset{\pro}{\to}}
\newcommand\toPP{ \overset{\text{PP}}{\to}}
\newcommand{\oeq}{\mathrel{\text{\textcircled{$=$}}}}





\usepackage{xcolor}
% \usepackage[normalem]{ulem}
\usepackage{lipsum}
\makeatletter
% \newcommand\colorwave[1][blue]{\bgroup \markoverwith{\lower3.5\p@\hbox{\sixly \textcolor{#1}{\char58}}}\ULon}
%\font\sixly=lasy6 % does not re-load if already loaded, so no memory problem.

\newmdtheoremenv[
linewidth= 1pt,linecolor= blue,%
leftmargin=20,rightmargin=20,innertopmargin=0pt, innerrightmargin=40,%
tikzsetting = { draw=lightgray, line width = 0.3pt,dashed,%
dash pattern = on 15pt off 3pt},%
splittopskip=\topskip,skipbelow=\baselineskip,%
skipabove=\baselineskip,ntheorem,roundcorner=0pt,
% backgroundcolor=pagebg,font=\color{orange}\sffamily, fontcolor=white
]{examplebox}{Exemple}[section]



\newcommand\R{\mathbb{R}}
\newcommand\Z{\mathbb{Z}}
\newcommand\N{\mathbb{N}}
\newcommand\E{\mathbb{E}}
\newcommand\F{\mathcal{F}}
\newcommand\cH{\mathcal{H}}
\newcommand\V{\mathbb{V}}
\newcommand\dmo{ ^{-1} }
\newcommand\kapa{\kappa}
\newcommand\im{Im}
\newcommand\hs{\mathcal{H}}





\usepackage{soul}

\makeatletter
\newcommand*{\whiten}[1]{\llap{\textcolor{white}{{\the\SOUL@token}}\hspace{#1pt}}}
\DeclareRobustCommand*\myul{%
    \def\SOUL@everyspace{\underline{\space}\kern\z@}%
    \def\SOUL@everytoken{%
     \setbox0=\hbox{\the\SOUL@token}%
     \ifdim\dp0>\z@
        \raisebox{\dp0}{\underline{\phantom{\the\SOUL@token}}}%
        \whiten{1}\whiten{0}%
        \whiten{-1}\whiten{-2}%
        \llap{\the\SOUL@token}%
     \else
        \underline{\the\SOUL@token}%
     \fi}%
\SOUL@}
\makeatother

\newcommand*{\demp}{\fontfamily{lmtt}\selectfont}

\DeclareTextFontCommand{\textdemp}{\demp}

\begin{document}

\ifcomment
Multiline
comment
\fi
\ifcomment
\myul{Typesetting test}
% \color[rgb]{1,1,1}
$∑_i^n≠ 60º±∞π∆¬≈√j∫h≤≥µ$

$\CR \R\pro\ind\pro\gS\pro
\mqty[a&b\\c&d]$
$\pro\mathbb{P}$
$\dd{x}$

  \[
    \alpha(x)=\left\{
                \begin{array}{ll}
                  x\\
                  \frac{1}{1+e^{-kx}}\\
                  \frac{e^x-e^{-x}}{e^x+e^{-x}}
                \end{array}
              \right.
  \]

  $\expval{x}$
  
  $\chi_\rho(ghg\dmo)=\Tr(\rho_{ghg\dmo})=\Tr(\rho_g\circ\rho_h\circ\rho\dmo_g)=\Tr(\rho_h)\overset{\mbox{\scalebox{0.5}{$\Tr(AB)=\Tr(BA)$}}}{=}\chi_\rho(h)$
  	$\mathop{\oplus}_{\substack{x\in X}}$

$\mat(\rho_g)=(a_{ij}(g))_{\scriptsize \substack{1\leq i\leq d \\ 1\leq j\leq d}}$ et $\mat(\rho'_g)=(a'_{ij}(g))_{\scriptsize \substack{1\leq i'\leq d' \\ 1\leq j'\leq d'}}$



\[\int_a^b{\mathbb{R}^2}g(u, v)\dd{P_{XY}}(u, v)=\iint g(u,v) f_{XY}(u, v)\dd \lambda(u) \dd \lambda(v)\]
$$\lim_{x\to\infty} f(x)$$	
$$\iiiint_V \mu(t,u,v,w) \,dt\,du\,dv\,dw$$
$$\sum_{n=1}^{\infty} 2^{-n} = 1$$	
\begin{definition}
	Si $X$ et $Y$ sont 2 v.a. ou definit la \textsc{Covariance} entre $X$ et $Y$ comme
	$\cov(X,Y)\overset{\text{def}}{=}\E\left[(X-\E(X))(Y-\E(Y))\right]=\E(XY)-\E(X)\E(Y)$.
\end{definition}
\fi
\pagebreak

% \tableofcontents

% insert your code here
%\input{./algebra/main.tex}
%\input{./geometrie-differentielle/main.tex}
%\input{./probabilite/main.tex}
%\input{./analyse-fonctionnelle/main.tex}
% \input{./Analyse-convexe-et-dualite-en-optimisation/main.tex}
%\input{./tikz/main.tex}
%\input{./Theorie-du-distributions/main.tex}
%\input{./optimisation/mine.tex}
 \input{./modelisation/main.tex}

% yves.aubry@univ-tln.fr : algebra

\end{document}

%% !TEX encoding = UTF-8 Unicode
% !TEX TS-program = xelatex

\documentclass[french]{report}

%\usepackage[utf8]{inputenc}
%\usepackage[T1]{fontenc}
\usepackage{babel}


\newif\ifcomment
%\commenttrue # Show comments

\usepackage{physics}
\usepackage{amssymb}


\usepackage{amsthm}
% \usepackage{thmtools}
\usepackage{mathtools}
\usepackage{amsfonts}

\usepackage{color}

\usepackage{tikz}

\usepackage{geometry}
\geometry{a5paper, margin=0.1in, right=1cm}

\usepackage{dsfont}

\usepackage{graphicx}
\graphicspath{ {images/} }

\usepackage{faktor}

\usepackage{IEEEtrantools}
\usepackage{enumerate}   
\usepackage[PostScript=dvips]{"/Users/aware/Documents/Courses/diagrams"}


\newtheorem{theorem}{Théorème}[section]
\renewcommand{\thetheorem}{\arabic{theorem}}
\newtheorem{lemme}{Lemme}[section]
\renewcommand{\thelemme}{\arabic{lemme}}
\newtheorem{proposition}{Proposition}[section]
\renewcommand{\theproposition}{\arabic{proposition}}
\newtheorem{notations}{Notations}[section]
\newtheorem{problem}{Problème}[section]
\newtheorem{corollary}{Corollaire}[theorem]
\renewcommand{\thecorollary}{\arabic{corollary}}
\newtheorem{property}{Propriété}[section]
\newtheorem{objective}{Objectif}[section]

\theoremstyle{definition}
\newtheorem{definition}{Définition}[section]
\renewcommand{\thedefinition}{\arabic{definition}}
\newtheorem{exercise}{Exercice}[chapter]
\renewcommand{\theexercise}{\arabic{exercise}}
\newtheorem{example}{Exemple}[chapter]
\renewcommand{\theexample}{\arabic{example}}
\newtheorem*{solution}{Solution}
\newtheorem*{application}{Application}
\newtheorem*{notation}{Notation}
\newtheorem*{vocabulary}{Vocabulaire}
\newtheorem*{properties}{Propriétés}



\theoremstyle{remark}
\newtheorem*{remark}{Remarque}
\newtheorem*{rappel}{Rappel}


\usepackage{etoolbox}
\AtBeginEnvironment{exercise}{\small}
\AtBeginEnvironment{example}{\small}

\usepackage{cases}
\usepackage[red]{mypack}

\usepackage[framemethod=TikZ]{mdframed}

\definecolor{bg}{rgb}{0.4,0.25,0.95}
\definecolor{pagebg}{rgb}{0,0,0.5}
\surroundwithmdframed[
   topline=false,
   rightline=false,
   bottomline=false,
   leftmargin=\parindent,
   skipabove=8pt,
   skipbelow=8pt,
   linecolor=blue,
   innerbottommargin=10pt,
   % backgroundcolor=bg,font=\color{orange}\sffamily, fontcolor=white
]{definition}

\usepackage{empheq}
\usepackage[most]{tcolorbox}

\newtcbox{\mymath}[1][]{%
    nobeforeafter, math upper, tcbox raise base,
    enhanced, colframe=blue!30!black,
    colback=red!10, boxrule=1pt,
    #1}

\usepackage{unixode}


\DeclareMathOperator{\ord}{ord}
\DeclareMathOperator{\orb}{orb}
\DeclareMathOperator{\stab}{stab}
\DeclareMathOperator{\Stab}{stab}
\DeclareMathOperator{\ppcm}{ppcm}
\DeclareMathOperator{\conj}{Conj}
\DeclareMathOperator{\End}{End}
\DeclareMathOperator{\rot}{rot}
\DeclareMathOperator{\trs}{trace}
\DeclareMathOperator{\Ind}{Ind}
\DeclareMathOperator{\mat}{Mat}
\DeclareMathOperator{\id}{Id}
\DeclareMathOperator{\vect}{vect}
\DeclareMathOperator{\img}{img}
\DeclareMathOperator{\cov}{Cov}
\DeclareMathOperator{\dist}{dist}
\DeclareMathOperator{\irr}{Irr}
\DeclareMathOperator{\image}{Im}
\DeclareMathOperator{\pd}{\partial}
\DeclareMathOperator{\epi}{epi}
\DeclareMathOperator{\Argmin}{Argmin}
\DeclareMathOperator{\dom}{dom}
\DeclareMathOperator{\proj}{proj}
\DeclareMathOperator{\ctg}{ctg}
\DeclareMathOperator{\supp}{supp}
\DeclareMathOperator{\argmin}{argmin}
\DeclareMathOperator{\mult}{mult}
\DeclareMathOperator{\ch}{ch}
\DeclareMathOperator{\sh}{sh}
\DeclareMathOperator{\rang}{rang}
\DeclareMathOperator{\diam}{diam}
\DeclareMathOperator{\Epigraphe}{Epigraphe}




\usepackage{xcolor}
\everymath{\color{blue}}
%\everymath{\color[rgb]{0,1,1}}
%\pagecolor[rgb]{0,0,0.5}


\newcommand*{\pdtest}[3][]{\ensuremath{\frac{\partial^{#1} #2}{\partial #3}}}

\newcommand*{\deffunc}[6][]{\ensuremath{
\begin{array}{rcl}
#2 : #3 &\rightarrow& #4\\
#5 &\mapsto& #6
\end{array}
}}

\newcommand{\eqcolon}{\mathrel{\resizebox{\widthof{$\mathord{=}$}}{\height}{ $\!\!=\!\!\resizebox{1.2\width}{0.8\height}{\raisebox{0.23ex}{$\mathop{:}$}}\!\!$ }}}
\newcommand{\coloneq}{\mathrel{\resizebox{\widthof{$\mathord{=}$}}{\height}{ $\!\!\resizebox{1.2\width}{0.8\height}{\raisebox{0.23ex}{$\mathop{:}$}}\!\!=\!\!$ }}}
\newcommand{\eqcolonl}{\ensuremath{\mathrel{=\!\!\mathop{:}}}}
\newcommand{\coloneql}{\ensuremath{\mathrel{\mathop{:} \!\! =}}}
\newcommand{\vc}[1]{% inline column vector
  \left(\begin{smallmatrix}#1\end{smallmatrix}\right)%
}
\newcommand{\vr}[1]{% inline row vector
  \begin{smallmatrix}(\,#1\,)\end{smallmatrix}%
}
\makeatletter
\newcommand*{\defeq}{\ =\mathrel{\rlap{%
                     \raisebox{0.3ex}{$\m@th\cdot$}}%
                     \raisebox{-0.3ex}{$\m@th\cdot$}}%
                     }
\makeatother

\newcommand{\mathcircle}[1]{% inline row vector
 \overset{\circ}{#1}
}
\newcommand{\ulim}{% low limit
 \underline{\lim}
}
\newcommand{\ssi}{% iff
\iff
}
\newcommand{\ps}[2]{
\expval{#1 | #2}
}
\newcommand{\df}[1]{
\mqty{#1}
}
\newcommand{\n}[1]{
\norm{#1}
}
\newcommand{\sys}[1]{
\left\{\smqty{#1}\right.
}


\newcommand{\eqdef}{\ensuremath{\overset{\text{def}}=}}


\def\Circlearrowright{\ensuremath{%
  \rotatebox[origin=c]{230}{$\circlearrowright$}}}

\newcommand\ct[1]{\text{\rmfamily\upshape #1}}
\newcommand\question[1]{ {\color{red} ...!? \small #1}}
\newcommand\caz[1]{\left\{\begin{array} #1 \end{array}\right.}
\newcommand\const{\text{\rmfamily\upshape const}}
\newcommand\toP{ \overset{\pro}{\to}}
\newcommand\toPP{ \overset{\text{PP}}{\to}}
\newcommand{\oeq}{\mathrel{\text{\textcircled{$=$}}}}





\usepackage{xcolor}
% \usepackage[normalem]{ulem}
\usepackage{lipsum}
\makeatletter
% \newcommand\colorwave[1][blue]{\bgroup \markoverwith{\lower3.5\p@\hbox{\sixly \textcolor{#1}{\char58}}}\ULon}
%\font\sixly=lasy6 % does not re-load if already loaded, so no memory problem.

\newmdtheoremenv[
linewidth= 1pt,linecolor= blue,%
leftmargin=20,rightmargin=20,innertopmargin=0pt, innerrightmargin=40,%
tikzsetting = { draw=lightgray, line width = 0.3pt,dashed,%
dash pattern = on 15pt off 3pt},%
splittopskip=\topskip,skipbelow=\baselineskip,%
skipabove=\baselineskip,ntheorem,roundcorner=0pt,
% backgroundcolor=pagebg,font=\color{orange}\sffamily, fontcolor=white
]{examplebox}{Exemple}[section]



\newcommand\R{\mathbb{R}}
\newcommand\Z{\mathbb{Z}}
\newcommand\N{\mathbb{N}}
\newcommand\E{\mathbb{E}}
\newcommand\F{\mathcal{F}}
\newcommand\cH{\mathcal{H}}
\newcommand\V{\mathbb{V}}
\newcommand\dmo{ ^{-1} }
\newcommand\kapa{\kappa}
\newcommand\im{Im}
\newcommand\hs{\mathcal{H}}





\usepackage{soul}

\makeatletter
\newcommand*{\whiten}[1]{\llap{\textcolor{white}{{\the\SOUL@token}}\hspace{#1pt}}}
\DeclareRobustCommand*\myul{%
    \def\SOUL@everyspace{\underline{\space}\kern\z@}%
    \def\SOUL@everytoken{%
     \setbox0=\hbox{\the\SOUL@token}%
     \ifdim\dp0>\z@
        \raisebox{\dp0}{\underline{\phantom{\the\SOUL@token}}}%
        \whiten{1}\whiten{0}%
        \whiten{-1}\whiten{-2}%
        \llap{\the\SOUL@token}%
     \else
        \underline{\the\SOUL@token}%
     \fi}%
\SOUL@}
\makeatother

\newcommand*{\demp}{\fontfamily{lmtt}\selectfont}

\DeclareTextFontCommand{\textdemp}{\demp}

\begin{document}

\ifcomment
Multiline
comment
\fi
\ifcomment
\myul{Typesetting test}
% \color[rgb]{1,1,1}
$∑_i^n≠ 60º±∞π∆¬≈√j∫h≤≥µ$

$\CR \R\pro\ind\pro\gS\pro
\mqty[a&b\\c&d]$
$\pro\mathbb{P}$
$\dd{x}$

  \[
    \alpha(x)=\left\{
                \begin{array}{ll}
                  x\\
                  \frac{1}{1+e^{-kx}}\\
                  \frac{e^x-e^{-x}}{e^x+e^{-x}}
                \end{array}
              \right.
  \]

  $\expval{x}$
  
  $\chi_\rho(ghg\dmo)=\Tr(\rho_{ghg\dmo})=\Tr(\rho_g\circ\rho_h\circ\rho\dmo_g)=\Tr(\rho_h)\overset{\mbox{\scalebox{0.5}{$\Tr(AB)=\Tr(BA)$}}}{=}\chi_\rho(h)$
  	$\mathop{\oplus}_{\substack{x\in X}}$

$\mat(\rho_g)=(a_{ij}(g))_{\scriptsize \substack{1\leq i\leq d \\ 1\leq j\leq d}}$ et $\mat(\rho'_g)=(a'_{ij}(g))_{\scriptsize \substack{1\leq i'\leq d' \\ 1\leq j'\leq d'}}$



\[\int_a^b{\mathbb{R}^2}g(u, v)\dd{P_{XY}}(u, v)=\iint g(u,v) f_{XY}(u, v)\dd \lambda(u) \dd \lambda(v)\]
$$\lim_{x\to\infty} f(x)$$	
$$\iiiint_V \mu(t,u,v,w) \,dt\,du\,dv\,dw$$
$$\sum_{n=1}^{\infty} 2^{-n} = 1$$	
\begin{definition}
	Si $X$ et $Y$ sont 2 v.a. ou definit la \textsc{Covariance} entre $X$ et $Y$ comme
	$\cov(X,Y)\overset{\text{def}}{=}\E\left[(X-\E(X))(Y-\E(Y))\right]=\E(XY)-\E(X)\E(Y)$.
\end{definition}
\fi
\pagebreak

% \tableofcontents

% insert your code here
%\input{./algebra/main.tex}
%\input{./geometrie-differentielle/main.tex}
%\input{./probabilite/main.tex}
%\input{./analyse-fonctionnelle/main.tex}
% \input{./Analyse-convexe-et-dualite-en-optimisation/main.tex}
%\input{./tikz/main.tex}
%\input{./Theorie-du-distributions/main.tex}
%\input{./optimisation/mine.tex}
 \input{./modelisation/main.tex}

% yves.aubry@univ-tln.fr : algebra

\end{document}

%% !TEX encoding = UTF-8 Unicode
% !TEX TS-program = xelatex

\documentclass[french]{report}

%\usepackage[utf8]{inputenc}
%\usepackage[T1]{fontenc}
\usepackage{babel}


\newif\ifcomment
%\commenttrue # Show comments

\usepackage{physics}
\usepackage{amssymb}


\usepackage{amsthm}
% \usepackage{thmtools}
\usepackage{mathtools}
\usepackage{amsfonts}

\usepackage{color}

\usepackage{tikz}

\usepackage{geometry}
\geometry{a5paper, margin=0.1in, right=1cm}

\usepackage{dsfont}

\usepackage{graphicx}
\graphicspath{ {images/} }

\usepackage{faktor}

\usepackage{IEEEtrantools}
\usepackage{enumerate}   
\usepackage[PostScript=dvips]{"/Users/aware/Documents/Courses/diagrams"}


\newtheorem{theorem}{Théorème}[section]
\renewcommand{\thetheorem}{\arabic{theorem}}
\newtheorem{lemme}{Lemme}[section]
\renewcommand{\thelemme}{\arabic{lemme}}
\newtheorem{proposition}{Proposition}[section]
\renewcommand{\theproposition}{\arabic{proposition}}
\newtheorem{notations}{Notations}[section]
\newtheorem{problem}{Problème}[section]
\newtheorem{corollary}{Corollaire}[theorem]
\renewcommand{\thecorollary}{\arabic{corollary}}
\newtheorem{property}{Propriété}[section]
\newtheorem{objective}{Objectif}[section]

\theoremstyle{definition}
\newtheorem{definition}{Définition}[section]
\renewcommand{\thedefinition}{\arabic{definition}}
\newtheorem{exercise}{Exercice}[chapter]
\renewcommand{\theexercise}{\arabic{exercise}}
\newtheorem{example}{Exemple}[chapter]
\renewcommand{\theexample}{\arabic{example}}
\newtheorem*{solution}{Solution}
\newtheorem*{application}{Application}
\newtheorem*{notation}{Notation}
\newtheorem*{vocabulary}{Vocabulaire}
\newtheorem*{properties}{Propriétés}



\theoremstyle{remark}
\newtheorem*{remark}{Remarque}
\newtheorem*{rappel}{Rappel}


\usepackage{etoolbox}
\AtBeginEnvironment{exercise}{\small}
\AtBeginEnvironment{example}{\small}

\usepackage{cases}
\usepackage[red]{mypack}

\usepackage[framemethod=TikZ]{mdframed}

\definecolor{bg}{rgb}{0.4,0.25,0.95}
\definecolor{pagebg}{rgb}{0,0,0.5}
\surroundwithmdframed[
   topline=false,
   rightline=false,
   bottomline=false,
   leftmargin=\parindent,
   skipabove=8pt,
   skipbelow=8pt,
   linecolor=blue,
   innerbottommargin=10pt,
   % backgroundcolor=bg,font=\color{orange}\sffamily, fontcolor=white
]{definition}

\usepackage{empheq}
\usepackage[most]{tcolorbox}

\newtcbox{\mymath}[1][]{%
    nobeforeafter, math upper, tcbox raise base,
    enhanced, colframe=blue!30!black,
    colback=red!10, boxrule=1pt,
    #1}

\usepackage{unixode}


\DeclareMathOperator{\ord}{ord}
\DeclareMathOperator{\orb}{orb}
\DeclareMathOperator{\stab}{stab}
\DeclareMathOperator{\Stab}{stab}
\DeclareMathOperator{\ppcm}{ppcm}
\DeclareMathOperator{\conj}{Conj}
\DeclareMathOperator{\End}{End}
\DeclareMathOperator{\rot}{rot}
\DeclareMathOperator{\trs}{trace}
\DeclareMathOperator{\Ind}{Ind}
\DeclareMathOperator{\mat}{Mat}
\DeclareMathOperator{\id}{Id}
\DeclareMathOperator{\vect}{vect}
\DeclareMathOperator{\img}{img}
\DeclareMathOperator{\cov}{Cov}
\DeclareMathOperator{\dist}{dist}
\DeclareMathOperator{\irr}{Irr}
\DeclareMathOperator{\image}{Im}
\DeclareMathOperator{\pd}{\partial}
\DeclareMathOperator{\epi}{epi}
\DeclareMathOperator{\Argmin}{Argmin}
\DeclareMathOperator{\dom}{dom}
\DeclareMathOperator{\proj}{proj}
\DeclareMathOperator{\ctg}{ctg}
\DeclareMathOperator{\supp}{supp}
\DeclareMathOperator{\argmin}{argmin}
\DeclareMathOperator{\mult}{mult}
\DeclareMathOperator{\ch}{ch}
\DeclareMathOperator{\sh}{sh}
\DeclareMathOperator{\rang}{rang}
\DeclareMathOperator{\diam}{diam}
\DeclareMathOperator{\Epigraphe}{Epigraphe}




\usepackage{xcolor}
\everymath{\color{blue}}
%\everymath{\color[rgb]{0,1,1}}
%\pagecolor[rgb]{0,0,0.5}


\newcommand*{\pdtest}[3][]{\ensuremath{\frac{\partial^{#1} #2}{\partial #3}}}

\newcommand*{\deffunc}[6][]{\ensuremath{
\begin{array}{rcl}
#2 : #3 &\rightarrow& #4\\
#5 &\mapsto& #6
\end{array}
}}

\newcommand{\eqcolon}{\mathrel{\resizebox{\widthof{$\mathord{=}$}}{\height}{ $\!\!=\!\!\resizebox{1.2\width}{0.8\height}{\raisebox{0.23ex}{$\mathop{:}$}}\!\!$ }}}
\newcommand{\coloneq}{\mathrel{\resizebox{\widthof{$\mathord{=}$}}{\height}{ $\!\!\resizebox{1.2\width}{0.8\height}{\raisebox{0.23ex}{$\mathop{:}$}}\!\!=\!\!$ }}}
\newcommand{\eqcolonl}{\ensuremath{\mathrel{=\!\!\mathop{:}}}}
\newcommand{\coloneql}{\ensuremath{\mathrel{\mathop{:} \!\! =}}}
\newcommand{\vc}[1]{% inline column vector
  \left(\begin{smallmatrix}#1\end{smallmatrix}\right)%
}
\newcommand{\vr}[1]{% inline row vector
  \begin{smallmatrix}(\,#1\,)\end{smallmatrix}%
}
\makeatletter
\newcommand*{\defeq}{\ =\mathrel{\rlap{%
                     \raisebox{0.3ex}{$\m@th\cdot$}}%
                     \raisebox{-0.3ex}{$\m@th\cdot$}}%
                     }
\makeatother

\newcommand{\mathcircle}[1]{% inline row vector
 \overset{\circ}{#1}
}
\newcommand{\ulim}{% low limit
 \underline{\lim}
}
\newcommand{\ssi}{% iff
\iff
}
\newcommand{\ps}[2]{
\expval{#1 | #2}
}
\newcommand{\df}[1]{
\mqty{#1}
}
\newcommand{\n}[1]{
\norm{#1}
}
\newcommand{\sys}[1]{
\left\{\smqty{#1}\right.
}


\newcommand{\eqdef}{\ensuremath{\overset{\text{def}}=}}


\def\Circlearrowright{\ensuremath{%
  \rotatebox[origin=c]{230}{$\circlearrowright$}}}

\newcommand\ct[1]{\text{\rmfamily\upshape #1}}
\newcommand\question[1]{ {\color{red} ...!? \small #1}}
\newcommand\caz[1]{\left\{\begin{array} #1 \end{array}\right.}
\newcommand\const{\text{\rmfamily\upshape const}}
\newcommand\toP{ \overset{\pro}{\to}}
\newcommand\toPP{ \overset{\text{PP}}{\to}}
\newcommand{\oeq}{\mathrel{\text{\textcircled{$=$}}}}





\usepackage{xcolor}
% \usepackage[normalem]{ulem}
\usepackage{lipsum}
\makeatletter
% \newcommand\colorwave[1][blue]{\bgroup \markoverwith{\lower3.5\p@\hbox{\sixly \textcolor{#1}{\char58}}}\ULon}
%\font\sixly=lasy6 % does not re-load if already loaded, so no memory problem.

\newmdtheoremenv[
linewidth= 1pt,linecolor= blue,%
leftmargin=20,rightmargin=20,innertopmargin=0pt, innerrightmargin=40,%
tikzsetting = { draw=lightgray, line width = 0.3pt,dashed,%
dash pattern = on 15pt off 3pt},%
splittopskip=\topskip,skipbelow=\baselineskip,%
skipabove=\baselineskip,ntheorem,roundcorner=0pt,
% backgroundcolor=pagebg,font=\color{orange}\sffamily, fontcolor=white
]{examplebox}{Exemple}[section]



\newcommand\R{\mathbb{R}}
\newcommand\Z{\mathbb{Z}}
\newcommand\N{\mathbb{N}}
\newcommand\E{\mathbb{E}}
\newcommand\F{\mathcal{F}}
\newcommand\cH{\mathcal{H}}
\newcommand\V{\mathbb{V}}
\newcommand\dmo{ ^{-1} }
\newcommand\kapa{\kappa}
\newcommand\im{Im}
\newcommand\hs{\mathcal{H}}





\usepackage{soul}

\makeatletter
\newcommand*{\whiten}[1]{\llap{\textcolor{white}{{\the\SOUL@token}}\hspace{#1pt}}}
\DeclareRobustCommand*\myul{%
    \def\SOUL@everyspace{\underline{\space}\kern\z@}%
    \def\SOUL@everytoken{%
     \setbox0=\hbox{\the\SOUL@token}%
     \ifdim\dp0>\z@
        \raisebox{\dp0}{\underline{\phantom{\the\SOUL@token}}}%
        \whiten{1}\whiten{0}%
        \whiten{-1}\whiten{-2}%
        \llap{\the\SOUL@token}%
     \else
        \underline{\the\SOUL@token}%
     \fi}%
\SOUL@}
\makeatother

\newcommand*{\demp}{\fontfamily{lmtt}\selectfont}

\DeclareTextFontCommand{\textdemp}{\demp}

\begin{document}

\ifcomment
Multiline
comment
\fi
\ifcomment
\myul{Typesetting test}
% \color[rgb]{1,1,1}
$∑_i^n≠ 60º±∞π∆¬≈√j∫h≤≥µ$

$\CR \R\pro\ind\pro\gS\pro
\mqty[a&b\\c&d]$
$\pro\mathbb{P}$
$\dd{x}$

  \[
    \alpha(x)=\left\{
                \begin{array}{ll}
                  x\\
                  \frac{1}{1+e^{-kx}}\\
                  \frac{e^x-e^{-x}}{e^x+e^{-x}}
                \end{array}
              \right.
  \]

  $\expval{x}$
  
  $\chi_\rho(ghg\dmo)=\Tr(\rho_{ghg\dmo})=\Tr(\rho_g\circ\rho_h\circ\rho\dmo_g)=\Tr(\rho_h)\overset{\mbox{\scalebox{0.5}{$\Tr(AB)=\Tr(BA)$}}}{=}\chi_\rho(h)$
  	$\mathop{\oplus}_{\substack{x\in X}}$

$\mat(\rho_g)=(a_{ij}(g))_{\scriptsize \substack{1\leq i\leq d \\ 1\leq j\leq d}}$ et $\mat(\rho'_g)=(a'_{ij}(g))_{\scriptsize \substack{1\leq i'\leq d' \\ 1\leq j'\leq d'}}$



\[\int_a^b{\mathbb{R}^2}g(u, v)\dd{P_{XY}}(u, v)=\iint g(u,v) f_{XY}(u, v)\dd \lambda(u) \dd \lambda(v)\]
$$\lim_{x\to\infty} f(x)$$	
$$\iiiint_V \mu(t,u,v,w) \,dt\,du\,dv\,dw$$
$$\sum_{n=1}^{\infty} 2^{-n} = 1$$	
\begin{definition}
	Si $X$ et $Y$ sont 2 v.a. ou definit la \textsc{Covariance} entre $X$ et $Y$ comme
	$\cov(X,Y)\overset{\text{def}}{=}\E\left[(X-\E(X))(Y-\E(Y))\right]=\E(XY)-\E(X)\E(Y)$.
\end{definition}
\fi
\pagebreak

% \tableofcontents

% insert your code here
%\input{./algebra/main.tex}
%\input{./geometrie-differentielle/main.tex}
%\input{./probabilite/main.tex}
%\input{./analyse-fonctionnelle/main.tex}
% \input{./Analyse-convexe-et-dualite-en-optimisation/main.tex}
%\input{./tikz/main.tex}
%\input{./Theorie-du-distributions/main.tex}
%\input{./optimisation/mine.tex}
 \input{./modelisation/main.tex}

% yves.aubry@univ-tln.fr : algebra

\end{document}

%\input{./optimisation/mine.tex}
 % !TEX encoding = UTF-8 Unicode
% !TEX TS-program = xelatex

\documentclass[french]{report}

%\usepackage[utf8]{inputenc}
%\usepackage[T1]{fontenc}
\usepackage{babel}


\newif\ifcomment
%\commenttrue # Show comments

\usepackage{physics}
\usepackage{amssymb}


\usepackage{amsthm}
% \usepackage{thmtools}
\usepackage{mathtools}
\usepackage{amsfonts}

\usepackage{color}

\usepackage{tikz}

\usepackage{geometry}
\geometry{a5paper, margin=0.1in, right=1cm}

\usepackage{dsfont}

\usepackage{graphicx}
\graphicspath{ {images/} }

\usepackage{faktor}

\usepackage{IEEEtrantools}
\usepackage{enumerate}   
\usepackage[PostScript=dvips]{"/Users/aware/Documents/Courses/diagrams"}


\newtheorem{theorem}{Théorème}[section]
\renewcommand{\thetheorem}{\arabic{theorem}}
\newtheorem{lemme}{Lemme}[section]
\renewcommand{\thelemme}{\arabic{lemme}}
\newtheorem{proposition}{Proposition}[section]
\renewcommand{\theproposition}{\arabic{proposition}}
\newtheorem{notations}{Notations}[section]
\newtheorem{problem}{Problème}[section]
\newtheorem{corollary}{Corollaire}[theorem]
\renewcommand{\thecorollary}{\arabic{corollary}}
\newtheorem{property}{Propriété}[section]
\newtheorem{objective}{Objectif}[section]

\theoremstyle{definition}
\newtheorem{definition}{Définition}[section]
\renewcommand{\thedefinition}{\arabic{definition}}
\newtheorem{exercise}{Exercice}[chapter]
\renewcommand{\theexercise}{\arabic{exercise}}
\newtheorem{example}{Exemple}[chapter]
\renewcommand{\theexample}{\arabic{example}}
\newtheorem*{solution}{Solution}
\newtheorem*{application}{Application}
\newtheorem*{notation}{Notation}
\newtheorem*{vocabulary}{Vocabulaire}
\newtheorem*{properties}{Propriétés}



\theoremstyle{remark}
\newtheorem*{remark}{Remarque}
\newtheorem*{rappel}{Rappel}


\usepackage{etoolbox}
\AtBeginEnvironment{exercise}{\small}
\AtBeginEnvironment{example}{\small}

\usepackage{cases}
\usepackage[red]{mypack}

\usepackage[framemethod=TikZ]{mdframed}

\definecolor{bg}{rgb}{0.4,0.25,0.95}
\definecolor{pagebg}{rgb}{0,0,0.5}
\surroundwithmdframed[
   topline=false,
   rightline=false,
   bottomline=false,
   leftmargin=\parindent,
   skipabove=8pt,
   skipbelow=8pt,
   linecolor=blue,
   innerbottommargin=10pt,
   % backgroundcolor=bg,font=\color{orange}\sffamily, fontcolor=white
]{definition}

\usepackage{empheq}
\usepackage[most]{tcolorbox}

\newtcbox{\mymath}[1][]{%
    nobeforeafter, math upper, tcbox raise base,
    enhanced, colframe=blue!30!black,
    colback=red!10, boxrule=1pt,
    #1}

\usepackage{unixode}


\DeclareMathOperator{\ord}{ord}
\DeclareMathOperator{\orb}{orb}
\DeclareMathOperator{\stab}{stab}
\DeclareMathOperator{\Stab}{stab}
\DeclareMathOperator{\ppcm}{ppcm}
\DeclareMathOperator{\conj}{Conj}
\DeclareMathOperator{\End}{End}
\DeclareMathOperator{\rot}{rot}
\DeclareMathOperator{\trs}{trace}
\DeclareMathOperator{\Ind}{Ind}
\DeclareMathOperator{\mat}{Mat}
\DeclareMathOperator{\id}{Id}
\DeclareMathOperator{\vect}{vect}
\DeclareMathOperator{\img}{img}
\DeclareMathOperator{\cov}{Cov}
\DeclareMathOperator{\dist}{dist}
\DeclareMathOperator{\irr}{Irr}
\DeclareMathOperator{\image}{Im}
\DeclareMathOperator{\pd}{\partial}
\DeclareMathOperator{\epi}{epi}
\DeclareMathOperator{\Argmin}{Argmin}
\DeclareMathOperator{\dom}{dom}
\DeclareMathOperator{\proj}{proj}
\DeclareMathOperator{\ctg}{ctg}
\DeclareMathOperator{\supp}{supp}
\DeclareMathOperator{\argmin}{argmin}
\DeclareMathOperator{\mult}{mult}
\DeclareMathOperator{\ch}{ch}
\DeclareMathOperator{\sh}{sh}
\DeclareMathOperator{\rang}{rang}
\DeclareMathOperator{\diam}{diam}
\DeclareMathOperator{\Epigraphe}{Epigraphe}




\usepackage{xcolor}
\everymath{\color{blue}}
%\everymath{\color[rgb]{0,1,1}}
%\pagecolor[rgb]{0,0,0.5}


\newcommand*{\pdtest}[3][]{\ensuremath{\frac{\partial^{#1} #2}{\partial #3}}}

\newcommand*{\deffunc}[6][]{\ensuremath{
\begin{array}{rcl}
#2 : #3 &\rightarrow& #4\\
#5 &\mapsto& #6
\end{array}
}}

\newcommand{\eqcolon}{\mathrel{\resizebox{\widthof{$\mathord{=}$}}{\height}{ $\!\!=\!\!\resizebox{1.2\width}{0.8\height}{\raisebox{0.23ex}{$\mathop{:}$}}\!\!$ }}}
\newcommand{\coloneq}{\mathrel{\resizebox{\widthof{$\mathord{=}$}}{\height}{ $\!\!\resizebox{1.2\width}{0.8\height}{\raisebox{0.23ex}{$\mathop{:}$}}\!\!=\!\!$ }}}
\newcommand{\eqcolonl}{\ensuremath{\mathrel{=\!\!\mathop{:}}}}
\newcommand{\coloneql}{\ensuremath{\mathrel{\mathop{:} \!\! =}}}
\newcommand{\vc}[1]{% inline column vector
  \left(\begin{smallmatrix}#1\end{smallmatrix}\right)%
}
\newcommand{\vr}[1]{% inline row vector
  \begin{smallmatrix}(\,#1\,)\end{smallmatrix}%
}
\makeatletter
\newcommand*{\defeq}{\ =\mathrel{\rlap{%
                     \raisebox{0.3ex}{$\m@th\cdot$}}%
                     \raisebox{-0.3ex}{$\m@th\cdot$}}%
                     }
\makeatother

\newcommand{\mathcircle}[1]{% inline row vector
 \overset{\circ}{#1}
}
\newcommand{\ulim}{% low limit
 \underline{\lim}
}
\newcommand{\ssi}{% iff
\iff
}
\newcommand{\ps}[2]{
\expval{#1 | #2}
}
\newcommand{\df}[1]{
\mqty{#1}
}
\newcommand{\n}[1]{
\norm{#1}
}
\newcommand{\sys}[1]{
\left\{\smqty{#1}\right.
}


\newcommand{\eqdef}{\ensuremath{\overset{\text{def}}=}}


\def\Circlearrowright{\ensuremath{%
  \rotatebox[origin=c]{230}{$\circlearrowright$}}}

\newcommand\ct[1]{\text{\rmfamily\upshape #1}}
\newcommand\question[1]{ {\color{red} ...!? \small #1}}
\newcommand\caz[1]{\left\{\begin{array} #1 \end{array}\right.}
\newcommand\const{\text{\rmfamily\upshape const}}
\newcommand\toP{ \overset{\pro}{\to}}
\newcommand\toPP{ \overset{\text{PP}}{\to}}
\newcommand{\oeq}{\mathrel{\text{\textcircled{$=$}}}}





\usepackage{xcolor}
% \usepackage[normalem]{ulem}
\usepackage{lipsum}
\makeatletter
% \newcommand\colorwave[1][blue]{\bgroup \markoverwith{\lower3.5\p@\hbox{\sixly \textcolor{#1}{\char58}}}\ULon}
%\font\sixly=lasy6 % does not re-load if already loaded, so no memory problem.

\newmdtheoremenv[
linewidth= 1pt,linecolor= blue,%
leftmargin=20,rightmargin=20,innertopmargin=0pt, innerrightmargin=40,%
tikzsetting = { draw=lightgray, line width = 0.3pt,dashed,%
dash pattern = on 15pt off 3pt},%
splittopskip=\topskip,skipbelow=\baselineskip,%
skipabove=\baselineskip,ntheorem,roundcorner=0pt,
% backgroundcolor=pagebg,font=\color{orange}\sffamily, fontcolor=white
]{examplebox}{Exemple}[section]



\newcommand\R{\mathbb{R}}
\newcommand\Z{\mathbb{Z}}
\newcommand\N{\mathbb{N}}
\newcommand\E{\mathbb{E}}
\newcommand\F{\mathcal{F}}
\newcommand\cH{\mathcal{H}}
\newcommand\V{\mathbb{V}}
\newcommand\dmo{ ^{-1} }
\newcommand\kapa{\kappa}
\newcommand\im{Im}
\newcommand\hs{\mathcal{H}}





\usepackage{soul}

\makeatletter
\newcommand*{\whiten}[1]{\llap{\textcolor{white}{{\the\SOUL@token}}\hspace{#1pt}}}
\DeclareRobustCommand*\myul{%
    \def\SOUL@everyspace{\underline{\space}\kern\z@}%
    \def\SOUL@everytoken{%
     \setbox0=\hbox{\the\SOUL@token}%
     \ifdim\dp0>\z@
        \raisebox{\dp0}{\underline{\phantom{\the\SOUL@token}}}%
        \whiten{1}\whiten{0}%
        \whiten{-1}\whiten{-2}%
        \llap{\the\SOUL@token}%
     \else
        \underline{\the\SOUL@token}%
     \fi}%
\SOUL@}
\makeatother

\newcommand*{\demp}{\fontfamily{lmtt}\selectfont}

\DeclareTextFontCommand{\textdemp}{\demp}

\begin{document}

\ifcomment
Multiline
comment
\fi
\ifcomment
\myul{Typesetting test}
% \color[rgb]{1,1,1}
$∑_i^n≠ 60º±∞π∆¬≈√j∫h≤≥µ$

$\CR \R\pro\ind\pro\gS\pro
\mqty[a&b\\c&d]$
$\pro\mathbb{P}$
$\dd{x}$

  \[
    \alpha(x)=\left\{
                \begin{array}{ll}
                  x\\
                  \frac{1}{1+e^{-kx}}\\
                  \frac{e^x-e^{-x}}{e^x+e^{-x}}
                \end{array}
              \right.
  \]

  $\expval{x}$
  
  $\chi_\rho(ghg\dmo)=\Tr(\rho_{ghg\dmo})=\Tr(\rho_g\circ\rho_h\circ\rho\dmo_g)=\Tr(\rho_h)\overset{\mbox{\scalebox{0.5}{$\Tr(AB)=\Tr(BA)$}}}{=}\chi_\rho(h)$
  	$\mathop{\oplus}_{\substack{x\in X}}$

$\mat(\rho_g)=(a_{ij}(g))_{\scriptsize \substack{1\leq i\leq d \\ 1\leq j\leq d}}$ et $\mat(\rho'_g)=(a'_{ij}(g))_{\scriptsize \substack{1\leq i'\leq d' \\ 1\leq j'\leq d'}}$



\[\int_a^b{\mathbb{R}^2}g(u, v)\dd{P_{XY}}(u, v)=\iint g(u,v) f_{XY}(u, v)\dd \lambda(u) \dd \lambda(v)\]
$$\lim_{x\to\infty} f(x)$$	
$$\iiiint_V \mu(t,u,v,w) \,dt\,du\,dv\,dw$$
$$\sum_{n=1}^{\infty} 2^{-n} = 1$$	
\begin{definition}
	Si $X$ et $Y$ sont 2 v.a. ou definit la \textsc{Covariance} entre $X$ et $Y$ comme
	$\cov(X,Y)\overset{\text{def}}{=}\E\left[(X-\E(X))(Y-\E(Y))\right]=\E(XY)-\E(X)\E(Y)$.
\end{definition}
\fi
\pagebreak

% \tableofcontents

% insert your code here
%\input{./algebra/main.tex}
%\input{./geometrie-differentielle/main.tex}
%\input{./probabilite/main.tex}
%\input{./analyse-fonctionnelle/main.tex}
% \input{./Analyse-convexe-et-dualite-en-optimisation/main.tex}
%\input{./tikz/main.tex}
%\input{./Theorie-du-distributions/main.tex}
%\input{./optimisation/mine.tex}
 \input{./modelisation/main.tex}

% yves.aubry@univ-tln.fr : algebra

\end{document}


% yves.aubry@univ-tln.fr : algebra

\end{document}

%\input{./optimisation/mine.tex}
 % !TEX encoding = UTF-8 Unicode
% !TEX TS-program = xelatex

\documentclass[french]{report}

%\usepackage[utf8]{inputenc}
%\usepackage[T1]{fontenc}
\usepackage{babel}


\newif\ifcomment
%\commenttrue # Show comments

\usepackage{physics}
\usepackage{amssymb}


\usepackage{amsthm}
% \usepackage{thmtools}
\usepackage{mathtools}
\usepackage{amsfonts}

\usepackage{color}

\usepackage{tikz}

\usepackage{geometry}
\geometry{a5paper, margin=0.1in, right=1cm}

\usepackage{dsfont}

\usepackage{graphicx}
\graphicspath{ {images/} }

\usepackage{faktor}

\usepackage{IEEEtrantools}
\usepackage{enumerate}   
\usepackage[PostScript=dvips]{"/Users/aware/Documents/Courses/diagrams"}


\newtheorem{theorem}{Théorème}[section]
\renewcommand{\thetheorem}{\arabic{theorem}}
\newtheorem{lemme}{Lemme}[section]
\renewcommand{\thelemme}{\arabic{lemme}}
\newtheorem{proposition}{Proposition}[section]
\renewcommand{\theproposition}{\arabic{proposition}}
\newtheorem{notations}{Notations}[section]
\newtheorem{problem}{Problème}[section]
\newtheorem{corollary}{Corollaire}[theorem]
\renewcommand{\thecorollary}{\arabic{corollary}}
\newtheorem{property}{Propriété}[section]
\newtheorem{objective}{Objectif}[section]

\theoremstyle{definition}
\newtheorem{definition}{Définition}[section]
\renewcommand{\thedefinition}{\arabic{definition}}
\newtheorem{exercise}{Exercice}[chapter]
\renewcommand{\theexercise}{\arabic{exercise}}
\newtheorem{example}{Exemple}[chapter]
\renewcommand{\theexample}{\arabic{example}}
\newtheorem*{solution}{Solution}
\newtheorem*{application}{Application}
\newtheorem*{notation}{Notation}
\newtheorem*{vocabulary}{Vocabulaire}
\newtheorem*{properties}{Propriétés}



\theoremstyle{remark}
\newtheorem*{remark}{Remarque}
\newtheorem*{rappel}{Rappel}


\usepackage{etoolbox}
\AtBeginEnvironment{exercise}{\small}
\AtBeginEnvironment{example}{\small}

\usepackage{cases}
\usepackage[red]{mypack}

\usepackage[framemethod=TikZ]{mdframed}

\definecolor{bg}{rgb}{0.4,0.25,0.95}
\definecolor{pagebg}{rgb}{0,0,0.5}
\surroundwithmdframed[
   topline=false,
   rightline=false,
   bottomline=false,
   leftmargin=\parindent,
   skipabove=8pt,
   skipbelow=8pt,
   linecolor=blue,
   innerbottommargin=10pt,
   % backgroundcolor=bg,font=\color{orange}\sffamily, fontcolor=white
]{definition}

\usepackage{empheq}
\usepackage[most]{tcolorbox}

\newtcbox{\mymath}[1][]{%
    nobeforeafter, math upper, tcbox raise base,
    enhanced, colframe=blue!30!black,
    colback=red!10, boxrule=1pt,
    #1}

\usepackage{unixode}


\DeclareMathOperator{\ord}{ord}
\DeclareMathOperator{\orb}{orb}
\DeclareMathOperator{\stab}{stab}
\DeclareMathOperator{\Stab}{stab}
\DeclareMathOperator{\ppcm}{ppcm}
\DeclareMathOperator{\conj}{Conj}
\DeclareMathOperator{\End}{End}
\DeclareMathOperator{\rot}{rot}
\DeclareMathOperator{\trs}{trace}
\DeclareMathOperator{\Ind}{Ind}
\DeclareMathOperator{\mat}{Mat}
\DeclareMathOperator{\id}{Id}
\DeclareMathOperator{\vect}{vect}
\DeclareMathOperator{\img}{img}
\DeclareMathOperator{\cov}{Cov}
\DeclareMathOperator{\dist}{dist}
\DeclareMathOperator{\irr}{Irr}
\DeclareMathOperator{\image}{Im}
\DeclareMathOperator{\pd}{\partial}
\DeclareMathOperator{\epi}{epi}
\DeclareMathOperator{\Argmin}{Argmin}
\DeclareMathOperator{\dom}{dom}
\DeclareMathOperator{\proj}{proj}
\DeclareMathOperator{\ctg}{ctg}
\DeclareMathOperator{\supp}{supp}
\DeclareMathOperator{\argmin}{argmin}
\DeclareMathOperator{\mult}{mult}
\DeclareMathOperator{\ch}{ch}
\DeclareMathOperator{\sh}{sh}
\DeclareMathOperator{\rang}{rang}
\DeclareMathOperator{\diam}{diam}
\DeclareMathOperator{\Epigraphe}{Epigraphe}




\usepackage{xcolor}
\everymath{\color{blue}}
%\everymath{\color[rgb]{0,1,1}}
%\pagecolor[rgb]{0,0,0.5}


\newcommand*{\pdtest}[3][]{\ensuremath{\frac{\partial^{#1} #2}{\partial #3}}}

\newcommand*{\deffunc}[6][]{\ensuremath{
\begin{array}{rcl}
#2 : #3 &\rightarrow& #4\\
#5 &\mapsto& #6
\end{array}
}}

\newcommand{\eqcolon}{\mathrel{\resizebox{\widthof{$\mathord{=}$}}{\height}{ $\!\!=\!\!\resizebox{1.2\width}{0.8\height}{\raisebox{0.23ex}{$\mathop{:}$}}\!\!$ }}}
\newcommand{\coloneq}{\mathrel{\resizebox{\widthof{$\mathord{=}$}}{\height}{ $\!\!\resizebox{1.2\width}{0.8\height}{\raisebox{0.23ex}{$\mathop{:}$}}\!\!=\!\!$ }}}
\newcommand{\eqcolonl}{\ensuremath{\mathrel{=\!\!\mathop{:}}}}
\newcommand{\coloneql}{\ensuremath{\mathrel{\mathop{:} \!\! =}}}
\newcommand{\vc}[1]{% inline column vector
  \left(\begin{smallmatrix}#1\end{smallmatrix}\right)%
}
\newcommand{\vr}[1]{% inline row vector
  \begin{smallmatrix}(\,#1\,)\end{smallmatrix}%
}
\makeatletter
\newcommand*{\defeq}{\ =\mathrel{\rlap{%
                     \raisebox{0.3ex}{$\m@th\cdot$}}%
                     \raisebox{-0.3ex}{$\m@th\cdot$}}%
                     }
\makeatother

\newcommand{\mathcircle}[1]{% inline row vector
 \overset{\circ}{#1}
}
\newcommand{\ulim}{% low limit
 \underline{\lim}
}
\newcommand{\ssi}{% iff
\iff
}
\newcommand{\ps}[2]{
\expval{#1 | #2}
}
\newcommand{\df}[1]{
\mqty{#1}
}
\newcommand{\n}[1]{
\norm{#1}
}
\newcommand{\sys}[1]{
\left\{\smqty{#1}\right.
}


\newcommand{\eqdef}{\ensuremath{\overset{\text{def}}=}}


\def\Circlearrowright{\ensuremath{%
  \rotatebox[origin=c]{230}{$\circlearrowright$}}}

\newcommand\ct[1]{\text{\rmfamily\upshape #1}}
\newcommand\question[1]{ {\color{red} ...!? \small #1}}
\newcommand\caz[1]{\left\{\begin{array} #1 \end{array}\right.}
\newcommand\const{\text{\rmfamily\upshape const}}
\newcommand\toP{ \overset{\pro}{\to}}
\newcommand\toPP{ \overset{\text{PP}}{\to}}
\newcommand{\oeq}{\mathrel{\text{\textcircled{$=$}}}}





\usepackage{xcolor}
% \usepackage[normalem]{ulem}
\usepackage{lipsum}
\makeatletter
% \newcommand\colorwave[1][blue]{\bgroup \markoverwith{\lower3.5\p@\hbox{\sixly \textcolor{#1}{\char58}}}\ULon}
%\font\sixly=lasy6 % does not re-load if already loaded, so no memory problem.

\newmdtheoremenv[
linewidth= 1pt,linecolor= blue,%
leftmargin=20,rightmargin=20,innertopmargin=0pt, innerrightmargin=40,%
tikzsetting = { draw=lightgray, line width = 0.3pt,dashed,%
dash pattern = on 15pt off 3pt},%
splittopskip=\topskip,skipbelow=\baselineskip,%
skipabove=\baselineskip,ntheorem,roundcorner=0pt,
% backgroundcolor=pagebg,font=\color{orange}\sffamily, fontcolor=white
]{examplebox}{Exemple}[section]



\newcommand\R{\mathbb{R}}
\newcommand\Z{\mathbb{Z}}
\newcommand\N{\mathbb{N}}
\newcommand\E{\mathbb{E}}
\newcommand\F{\mathcal{F}}
\newcommand\cH{\mathcal{H}}
\newcommand\V{\mathbb{V}}
\newcommand\dmo{ ^{-1} }
\newcommand\kapa{\kappa}
\newcommand\im{Im}
\newcommand\hs{\mathcal{H}}





\usepackage{soul}

\makeatletter
\newcommand*{\whiten}[1]{\llap{\textcolor{white}{{\the\SOUL@token}}\hspace{#1pt}}}
\DeclareRobustCommand*\myul{%
    \def\SOUL@everyspace{\underline{\space}\kern\z@}%
    \def\SOUL@everytoken{%
     \setbox0=\hbox{\the\SOUL@token}%
     \ifdim\dp0>\z@
        \raisebox{\dp0}{\underline{\phantom{\the\SOUL@token}}}%
        \whiten{1}\whiten{0}%
        \whiten{-1}\whiten{-2}%
        \llap{\the\SOUL@token}%
     \else
        \underline{\the\SOUL@token}%
     \fi}%
\SOUL@}
\makeatother

\newcommand*{\demp}{\fontfamily{lmtt}\selectfont}

\DeclareTextFontCommand{\textdemp}{\demp}

\begin{document}

\ifcomment
Multiline
comment
\fi
\ifcomment
\myul{Typesetting test}
% \color[rgb]{1,1,1}
$∑_i^n≠ 60º±∞π∆¬≈√j∫h≤≥µ$

$\CR \R\pro\ind\pro\gS\pro
\mqty[a&b\\c&d]$
$\pro\mathbb{P}$
$\dd{x}$

  \[
    \alpha(x)=\left\{
                \begin{array}{ll}
                  x\\
                  \frac{1}{1+e^{-kx}}\\
                  \frac{e^x-e^{-x}}{e^x+e^{-x}}
                \end{array}
              \right.
  \]

  $\expval{x}$
  
  $\chi_\rho(ghg\dmo)=\Tr(\rho_{ghg\dmo})=\Tr(\rho_g\circ\rho_h\circ\rho\dmo_g)=\Tr(\rho_h)\overset{\mbox{\scalebox{0.5}{$\Tr(AB)=\Tr(BA)$}}}{=}\chi_\rho(h)$
  	$\mathop{\oplus}_{\substack{x\in X}}$

$\mat(\rho_g)=(a_{ij}(g))_{\scriptsize \substack{1\leq i\leq d \\ 1\leq j\leq d}}$ et $\mat(\rho'_g)=(a'_{ij}(g))_{\scriptsize \substack{1\leq i'\leq d' \\ 1\leq j'\leq d'}}$



\[\int_a^b{\mathbb{R}^2}g(u, v)\dd{P_{XY}}(u, v)=\iint g(u,v) f_{XY}(u, v)\dd \lambda(u) \dd \lambda(v)\]
$$\lim_{x\to\infty} f(x)$$	
$$\iiiint_V \mu(t,u,v,w) \,dt\,du\,dv\,dw$$
$$\sum_{n=1}^{\infty} 2^{-n} = 1$$	
\begin{definition}
	Si $X$ et $Y$ sont 2 v.a. ou definit la \textsc{Covariance} entre $X$ et $Y$ comme
	$\cov(X,Y)\overset{\text{def}}{=}\E\left[(X-\E(X))(Y-\E(Y))\right]=\E(XY)-\E(X)\E(Y)$.
\end{definition}
\fi
\pagebreak

% \tableofcontents

% insert your code here
%% !TEX encoding = UTF-8 Unicode
% !TEX TS-program = xelatex

\documentclass[french]{report}

%\usepackage[utf8]{inputenc}
%\usepackage[T1]{fontenc}
\usepackage{babel}


\newif\ifcomment
%\commenttrue # Show comments

\usepackage{physics}
\usepackage{amssymb}


\usepackage{amsthm}
% \usepackage{thmtools}
\usepackage{mathtools}
\usepackage{amsfonts}

\usepackage{color}

\usepackage{tikz}

\usepackage{geometry}
\geometry{a5paper, margin=0.1in, right=1cm}

\usepackage{dsfont}

\usepackage{graphicx}
\graphicspath{ {images/} }

\usepackage{faktor}

\usepackage{IEEEtrantools}
\usepackage{enumerate}   
\usepackage[PostScript=dvips]{"/Users/aware/Documents/Courses/diagrams"}


\newtheorem{theorem}{Théorème}[section]
\renewcommand{\thetheorem}{\arabic{theorem}}
\newtheorem{lemme}{Lemme}[section]
\renewcommand{\thelemme}{\arabic{lemme}}
\newtheorem{proposition}{Proposition}[section]
\renewcommand{\theproposition}{\arabic{proposition}}
\newtheorem{notations}{Notations}[section]
\newtheorem{problem}{Problème}[section]
\newtheorem{corollary}{Corollaire}[theorem]
\renewcommand{\thecorollary}{\arabic{corollary}}
\newtheorem{property}{Propriété}[section]
\newtheorem{objective}{Objectif}[section]

\theoremstyle{definition}
\newtheorem{definition}{Définition}[section]
\renewcommand{\thedefinition}{\arabic{definition}}
\newtheorem{exercise}{Exercice}[chapter]
\renewcommand{\theexercise}{\arabic{exercise}}
\newtheorem{example}{Exemple}[chapter]
\renewcommand{\theexample}{\arabic{example}}
\newtheorem*{solution}{Solution}
\newtheorem*{application}{Application}
\newtheorem*{notation}{Notation}
\newtheorem*{vocabulary}{Vocabulaire}
\newtheorem*{properties}{Propriétés}



\theoremstyle{remark}
\newtheorem*{remark}{Remarque}
\newtheorem*{rappel}{Rappel}


\usepackage{etoolbox}
\AtBeginEnvironment{exercise}{\small}
\AtBeginEnvironment{example}{\small}

\usepackage{cases}
\usepackage[red]{mypack}

\usepackage[framemethod=TikZ]{mdframed}

\definecolor{bg}{rgb}{0.4,0.25,0.95}
\definecolor{pagebg}{rgb}{0,0,0.5}
\surroundwithmdframed[
   topline=false,
   rightline=false,
   bottomline=false,
   leftmargin=\parindent,
   skipabove=8pt,
   skipbelow=8pt,
   linecolor=blue,
   innerbottommargin=10pt,
   % backgroundcolor=bg,font=\color{orange}\sffamily, fontcolor=white
]{definition}

\usepackage{empheq}
\usepackage[most]{tcolorbox}

\newtcbox{\mymath}[1][]{%
    nobeforeafter, math upper, tcbox raise base,
    enhanced, colframe=blue!30!black,
    colback=red!10, boxrule=1pt,
    #1}

\usepackage{unixode}


\DeclareMathOperator{\ord}{ord}
\DeclareMathOperator{\orb}{orb}
\DeclareMathOperator{\stab}{stab}
\DeclareMathOperator{\Stab}{stab}
\DeclareMathOperator{\ppcm}{ppcm}
\DeclareMathOperator{\conj}{Conj}
\DeclareMathOperator{\End}{End}
\DeclareMathOperator{\rot}{rot}
\DeclareMathOperator{\trs}{trace}
\DeclareMathOperator{\Ind}{Ind}
\DeclareMathOperator{\mat}{Mat}
\DeclareMathOperator{\id}{Id}
\DeclareMathOperator{\vect}{vect}
\DeclareMathOperator{\img}{img}
\DeclareMathOperator{\cov}{Cov}
\DeclareMathOperator{\dist}{dist}
\DeclareMathOperator{\irr}{Irr}
\DeclareMathOperator{\image}{Im}
\DeclareMathOperator{\pd}{\partial}
\DeclareMathOperator{\epi}{epi}
\DeclareMathOperator{\Argmin}{Argmin}
\DeclareMathOperator{\dom}{dom}
\DeclareMathOperator{\proj}{proj}
\DeclareMathOperator{\ctg}{ctg}
\DeclareMathOperator{\supp}{supp}
\DeclareMathOperator{\argmin}{argmin}
\DeclareMathOperator{\mult}{mult}
\DeclareMathOperator{\ch}{ch}
\DeclareMathOperator{\sh}{sh}
\DeclareMathOperator{\rang}{rang}
\DeclareMathOperator{\diam}{diam}
\DeclareMathOperator{\Epigraphe}{Epigraphe}




\usepackage{xcolor}
\everymath{\color{blue}}
%\everymath{\color[rgb]{0,1,1}}
%\pagecolor[rgb]{0,0,0.5}


\newcommand*{\pdtest}[3][]{\ensuremath{\frac{\partial^{#1} #2}{\partial #3}}}

\newcommand*{\deffunc}[6][]{\ensuremath{
\begin{array}{rcl}
#2 : #3 &\rightarrow& #4\\
#5 &\mapsto& #6
\end{array}
}}

\newcommand{\eqcolon}{\mathrel{\resizebox{\widthof{$\mathord{=}$}}{\height}{ $\!\!=\!\!\resizebox{1.2\width}{0.8\height}{\raisebox{0.23ex}{$\mathop{:}$}}\!\!$ }}}
\newcommand{\coloneq}{\mathrel{\resizebox{\widthof{$\mathord{=}$}}{\height}{ $\!\!\resizebox{1.2\width}{0.8\height}{\raisebox{0.23ex}{$\mathop{:}$}}\!\!=\!\!$ }}}
\newcommand{\eqcolonl}{\ensuremath{\mathrel{=\!\!\mathop{:}}}}
\newcommand{\coloneql}{\ensuremath{\mathrel{\mathop{:} \!\! =}}}
\newcommand{\vc}[1]{% inline column vector
  \left(\begin{smallmatrix}#1\end{smallmatrix}\right)%
}
\newcommand{\vr}[1]{% inline row vector
  \begin{smallmatrix}(\,#1\,)\end{smallmatrix}%
}
\makeatletter
\newcommand*{\defeq}{\ =\mathrel{\rlap{%
                     \raisebox{0.3ex}{$\m@th\cdot$}}%
                     \raisebox{-0.3ex}{$\m@th\cdot$}}%
                     }
\makeatother

\newcommand{\mathcircle}[1]{% inline row vector
 \overset{\circ}{#1}
}
\newcommand{\ulim}{% low limit
 \underline{\lim}
}
\newcommand{\ssi}{% iff
\iff
}
\newcommand{\ps}[2]{
\expval{#1 | #2}
}
\newcommand{\df}[1]{
\mqty{#1}
}
\newcommand{\n}[1]{
\norm{#1}
}
\newcommand{\sys}[1]{
\left\{\smqty{#1}\right.
}


\newcommand{\eqdef}{\ensuremath{\overset{\text{def}}=}}


\def\Circlearrowright{\ensuremath{%
  \rotatebox[origin=c]{230}{$\circlearrowright$}}}

\newcommand\ct[1]{\text{\rmfamily\upshape #1}}
\newcommand\question[1]{ {\color{red} ...!? \small #1}}
\newcommand\caz[1]{\left\{\begin{array} #1 \end{array}\right.}
\newcommand\const{\text{\rmfamily\upshape const}}
\newcommand\toP{ \overset{\pro}{\to}}
\newcommand\toPP{ \overset{\text{PP}}{\to}}
\newcommand{\oeq}{\mathrel{\text{\textcircled{$=$}}}}





\usepackage{xcolor}
% \usepackage[normalem]{ulem}
\usepackage{lipsum}
\makeatletter
% \newcommand\colorwave[1][blue]{\bgroup \markoverwith{\lower3.5\p@\hbox{\sixly \textcolor{#1}{\char58}}}\ULon}
%\font\sixly=lasy6 % does not re-load if already loaded, so no memory problem.

\newmdtheoremenv[
linewidth= 1pt,linecolor= blue,%
leftmargin=20,rightmargin=20,innertopmargin=0pt, innerrightmargin=40,%
tikzsetting = { draw=lightgray, line width = 0.3pt,dashed,%
dash pattern = on 15pt off 3pt},%
splittopskip=\topskip,skipbelow=\baselineskip,%
skipabove=\baselineskip,ntheorem,roundcorner=0pt,
% backgroundcolor=pagebg,font=\color{orange}\sffamily, fontcolor=white
]{examplebox}{Exemple}[section]



\newcommand\R{\mathbb{R}}
\newcommand\Z{\mathbb{Z}}
\newcommand\N{\mathbb{N}}
\newcommand\E{\mathbb{E}}
\newcommand\F{\mathcal{F}}
\newcommand\cH{\mathcal{H}}
\newcommand\V{\mathbb{V}}
\newcommand\dmo{ ^{-1} }
\newcommand\kapa{\kappa}
\newcommand\im{Im}
\newcommand\hs{\mathcal{H}}





\usepackage{soul}

\makeatletter
\newcommand*{\whiten}[1]{\llap{\textcolor{white}{{\the\SOUL@token}}\hspace{#1pt}}}
\DeclareRobustCommand*\myul{%
    \def\SOUL@everyspace{\underline{\space}\kern\z@}%
    \def\SOUL@everytoken{%
     \setbox0=\hbox{\the\SOUL@token}%
     \ifdim\dp0>\z@
        \raisebox{\dp0}{\underline{\phantom{\the\SOUL@token}}}%
        \whiten{1}\whiten{0}%
        \whiten{-1}\whiten{-2}%
        \llap{\the\SOUL@token}%
     \else
        \underline{\the\SOUL@token}%
     \fi}%
\SOUL@}
\makeatother

\newcommand*{\demp}{\fontfamily{lmtt}\selectfont}

\DeclareTextFontCommand{\textdemp}{\demp}

\begin{document}

\ifcomment
Multiline
comment
\fi
\ifcomment
\myul{Typesetting test}
% \color[rgb]{1,1,1}
$∑_i^n≠ 60º±∞π∆¬≈√j∫h≤≥µ$

$\CR \R\pro\ind\pro\gS\pro
\mqty[a&b\\c&d]$
$\pro\mathbb{P}$
$\dd{x}$

  \[
    \alpha(x)=\left\{
                \begin{array}{ll}
                  x\\
                  \frac{1}{1+e^{-kx}}\\
                  \frac{e^x-e^{-x}}{e^x+e^{-x}}
                \end{array}
              \right.
  \]

  $\expval{x}$
  
  $\chi_\rho(ghg\dmo)=\Tr(\rho_{ghg\dmo})=\Tr(\rho_g\circ\rho_h\circ\rho\dmo_g)=\Tr(\rho_h)\overset{\mbox{\scalebox{0.5}{$\Tr(AB)=\Tr(BA)$}}}{=}\chi_\rho(h)$
  	$\mathop{\oplus}_{\substack{x\in X}}$

$\mat(\rho_g)=(a_{ij}(g))_{\scriptsize \substack{1\leq i\leq d \\ 1\leq j\leq d}}$ et $\mat(\rho'_g)=(a'_{ij}(g))_{\scriptsize \substack{1\leq i'\leq d' \\ 1\leq j'\leq d'}}$



\[\int_a^b{\mathbb{R}^2}g(u, v)\dd{P_{XY}}(u, v)=\iint g(u,v) f_{XY}(u, v)\dd \lambda(u) \dd \lambda(v)\]
$$\lim_{x\to\infty} f(x)$$	
$$\iiiint_V \mu(t,u,v,w) \,dt\,du\,dv\,dw$$
$$\sum_{n=1}^{\infty} 2^{-n} = 1$$	
\begin{definition}
	Si $X$ et $Y$ sont 2 v.a. ou definit la \textsc{Covariance} entre $X$ et $Y$ comme
	$\cov(X,Y)\overset{\text{def}}{=}\E\left[(X-\E(X))(Y-\E(Y))\right]=\E(XY)-\E(X)\E(Y)$.
\end{definition}
\fi
\pagebreak

% \tableofcontents

% insert your code here
%\input{./algebra/main.tex}
%\input{./geometrie-differentielle/main.tex}
%\input{./probabilite/main.tex}
%\input{./analyse-fonctionnelle/main.tex}
% \input{./Analyse-convexe-et-dualite-en-optimisation/main.tex}
%\input{./tikz/main.tex}
%\input{./Theorie-du-distributions/main.tex}
%\input{./optimisation/mine.tex}
 \input{./modelisation/main.tex}

% yves.aubry@univ-tln.fr : algebra

\end{document}

%% !TEX encoding = UTF-8 Unicode
% !TEX TS-program = xelatex

\documentclass[french]{report}

%\usepackage[utf8]{inputenc}
%\usepackage[T1]{fontenc}
\usepackage{babel}


\newif\ifcomment
%\commenttrue # Show comments

\usepackage{physics}
\usepackage{amssymb}


\usepackage{amsthm}
% \usepackage{thmtools}
\usepackage{mathtools}
\usepackage{amsfonts}

\usepackage{color}

\usepackage{tikz}

\usepackage{geometry}
\geometry{a5paper, margin=0.1in, right=1cm}

\usepackage{dsfont}

\usepackage{graphicx}
\graphicspath{ {images/} }

\usepackage{faktor}

\usepackage{IEEEtrantools}
\usepackage{enumerate}   
\usepackage[PostScript=dvips]{"/Users/aware/Documents/Courses/diagrams"}


\newtheorem{theorem}{Théorème}[section]
\renewcommand{\thetheorem}{\arabic{theorem}}
\newtheorem{lemme}{Lemme}[section]
\renewcommand{\thelemme}{\arabic{lemme}}
\newtheorem{proposition}{Proposition}[section]
\renewcommand{\theproposition}{\arabic{proposition}}
\newtheorem{notations}{Notations}[section]
\newtheorem{problem}{Problème}[section]
\newtheorem{corollary}{Corollaire}[theorem]
\renewcommand{\thecorollary}{\arabic{corollary}}
\newtheorem{property}{Propriété}[section]
\newtheorem{objective}{Objectif}[section]

\theoremstyle{definition}
\newtheorem{definition}{Définition}[section]
\renewcommand{\thedefinition}{\arabic{definition}}
\newtheorem{exercise}{Exercice}[chapter]
\renewcommand{\theexercise}{\arabic{exercise}}
\newtheorem{example}{Exemple}[chapter]
\renewcommand{\theexample}{\arabic{example}}
\newtheorem*{solution}{Solution}
\newtheorem*{application}{Application}
\newtheorem*{notation}{Notation}
\newtheorem*{vocabulary}{Vocabulaire}
\newtheorem*{properties}{Propriétés}



\theoremstyle{remark}
\newtheorem*{remark}{Remarque}
\newtheorem*{rappel}{Rappel}


\usepackage{etoolbox}
\AtBeginEnvironment{exercise}{\small}
\AtBeginEnvironment{example}{\small}

\usepackage{cases}
\usepackage[red]{mypack}

\usepackage[framemethod=TikZ]{mdframed}

\definecolor{bg}{rgb}{0.4,0.25,0.95}
\definecolor{pagebg}{rgb}{0,0,0.5}
\surroundwithmdframed[
   topline=false,
   rightline=false,
   bottomline=false,
   leftmargin=\parindent,
   skipabove=8pt,
   skipbelow=8pt,
   linecolor=blue,
   innerbottommargin=10pt,
   % backgroundcolor=bg,font=\color{orange}\sffamily, fontcolor=white
]{definition}

\usepackage{empheq}
\usepackage[most]{tcolorbox}

\newtcbox{\mymath}[1][]{%
    nobeforeafter, math upper, tcbox raise base,
    enhanced, colframe=blue!30!black,
    colback=red!10, boxrule=1pt,
    #1}

\usepackage{unixode}


\DeclareMathOperator{\ord}{ord}
\DeclareMathOperator{\orb}{orb}
\DeclareMathOperator{\stab}{stab}
\DeclareMathOperator{\Stab}{stab}
\DeclareMathOperator{\ppcm}{ppcm}
\DeclareMathOperator{\conj}{Conj}
\DeclareMathOperator{\End}{End}
\DeclareMathOperator{\rot}{rot}
\DeclareMathOperator{\trs}{trace}
\DeclareMathOperator{\Ind}{Ind}
\DeclareMathOperator{\mat}{Mat}
\DeclareMathOperator{\id}{Id}
\DeclareMathOperator{\vect}{vect}
\DeclareMathOperator{\img}{img}
\DeclareMathOperator{\cov}{Cov}
\DeclareMathOperator{\dist}{dist}
\DeclareMathOperator{\irr}{Irr}
\DeclareMathOperator{\image}{Im}
\DeclareMathOperator{\pd}{\partial}
\DeclareMathOperator{\epi}{epi}
\DeclareMathOperator{\Argmin}{Argmin}
\DeclareMathOperator{\dom}{dom}
\DeclareMathOperator{\proj}{proj}
\DeclareMathOperator{\ctg}{ctg}
\DeclareMathOperator{\supp}{supp}
\DeclareMathOperator{\argmin}{argmin}
\DeclareMathOperator{\mult}{mult}
\DeclareMathOperator{\ch}{ch}
\DeclareMathOperator{\sh}{sh}
\DeclareMathOperator{\rang}{rang}
\DeclareMathOperator{\diam}{diam}
\DeclareMathOperator{\Epigraphe}{Epigraphe}




\usepackage{xcolor}
\everymath{\color{blue}}
%\everymath{\color[rgb]{0,1,1}}
%\pagecolor[rgb]{0,0,0.5}


\newcommand*{\pdtest}[3][]{\ensuremath{\frac{\partial^{#1} #2}{\partial #3}}}

\newcommand*{\deffunc}[6][]{\ensuremath{
\begin{array}{rcl}
#2 : #3 &\rightarrow& #4\\
#5 &\mapsto& #6
\end{array}
}}

\newcommand{\eqcolon}{\mathrel{\resizebox{\widthof{$\mathord{=}$}}{\height}{ $\!\!=\!\!\resizebox{1.2\width}{0.8\height}{\raisebox{0.23ex}{$\mathop{:}$}}\!\!$ }}}
\newcommand{\coloneq}{\mathrel{\resizebox{\widthof{$\mathord{=}$}}{\height}{ $\!\!\resizebox{1.2\width}{0.8\height}{\raisebox{0.23ex}{$\mathop{:}$}}\!\!=\!\!$ }}}
\newcommand{\eqcolonl}{\ensuremath{\mathrel{=\!\!\mathop{:}}}}
\newcommand{\coloneql}{\ensuremath{\mathrel{\mathop{:} \!\! =}}}
\newcommand{\vc}[1]{% inline column vector
  \left(\begin{smallmatrix}#1\end{smallmatrix}\right)%
}
\newcommand{\vr}[1]{% inline row vector
  \begin{smallmatrix}(\,#1\,)\end{smallmatrix}%
}
\makeatletter
\newcommand*{\defeq}{\ =\mathrel{\rlap{%
                     \raisebox{0.3ex}{$\m@th\cdot$}}%
                     \raisebox{-0.3ex}{$\m@th\cdot$}}%
                     }
\makeatother

\newcommand{\mathcircle}[1]{% inline row vector
 \overset{\circ}{#1}
}
\newcommand{\ulim}{% low limit
 \underline{\lim}
}
\newcommand{\ssi}{% iff
\iff
}
\newcommand{\ps}[2]{
\expval{#1 | #2}
}
\newcommand{\df}[1]{
\mqty{#1}
}
\newcommand{\n}[1]{
\norm{#1}
}
\newcommand{\sys}[1]{
\left\{\smqty{#1}\right.
}


\newcommand{\eqdef}{\ensuremath{\overset{\text{def}}=}}


\def\Circlearrowright{\ensuremath{%
  \rotatebox[origin=c]{230}{$\circlearrowright$}}}

\newcommand\ct[1]{\text{\rmfamily\upshape #1}}
\newcommand\question[1]{ {\color{red} ...!? \small #1}}
\newcommand\caz[1]{\left\{\begin{array} #1 \end{array}\right.}
\newcommand\const{\text{\rmfamily\upshape const}}
\newcommand\toP{ \overset{\pro}{\to}}
\newcommand\toPP{ \overset{\text{PP}}{\to}}
\newcommand{\oeq}{\mathrel{\text{\textcircled{$=$}}}}





\usepackage{xcolor}
% \usepackage[normalem]{ulem}
\usepackage{lipsum}
\makeatletter
% \newcommand\colorwave[1][blue]{\bgroup \markoverwith{\lower3.5\p@\hbox{\sixly \textcolor{#1}{\char58}}}\ULon}
%\font\sixly=lasy6 % does not re-load if already loaded, so no memory problem.

\newmdtheoremenv[
linewidth= 1pt,linecolor= blue,%
leftmargin=20,rightmargin=20,innertopmargin=0pt, innerrightmargin=40,%
tikzsetting = { draw=lightgray, line width = 0.3pt,dashed,%
dash pattern = on 15pt off 3pt},%
splittopskip=\topskip,skipbelow=\baselineskip,%
skipabove=\baselineskip,ntheorem,roundcorner=0pt,
% backgroundcolor=pagebg,font=\color{orange}\sffamily, fontcolor=white
]{examplebox}{Exemple}[section]



\newcommand\R{\mathbb{R}}
\newcommand\Z{\mathbb{Z}}
\newcommand\N{\mathbb{N}}
\newcommand\E{\mathbb{E}}
\newcommand\F{\mathcal{F}}
\newcommand\cH{\mathcal{H}}
\newcommand\V{\mathbb{V}}
\newcommand\dmo{ ^{-1} }
\newcommand\kapa{\kappa}
\newcommand\im{Im}
\newcommand\hs{\mathcal{H}}





\usepackage{soul}

\makeatletter
\newcommand*{\whiten}[1]{\llap{\textcolor{white}{{\the\SOUL@token}}\hspace{#1pt}}}
\DeclareRobustCommand*\myul{%
    \def\SOUL@everyspace{\underline{\space}\kern\z@}%
    \def\SOUL@everytoken{%
     \setbox0=\hbox{\the\SOUL@token}%
     \ifdim\dp0>\z@
        \raisebox{\dp0}{\underline{\phantom{\the\SOUL@token}}}%
        \whiten{1}\whiten{0}%
        \whiten{-1}\whiten{-2}%
        \llap{\the\SOUL@token}%
     \else
        \underline{\the\SOUL@token}%
     \fi}%
\SOUL@}
\makeatother

\newcommand*{\demp}{\fontfamily{lmtt}\selectfont}

\DeclareTextFontCommand{\textdemp}{\demp}

\begin{document}

\ifcomment
Multiline
comment
\fi
\ifcomment
\myul{Typesetting test}
% \color[rgb]{1,1,1}
$∑_i^n≠ 60º±∞π∆¬≈√j∫h≤≥µ$

$\CR \R\pro\ind\pro\gS\pro
\mqty[a&b\\c&d]$
$\pro\mathbb{P}$
$\dd{x}$

  \[
    \alpha(x)=\left\{
                \begin{array}{ll}
                  x\\
                  \frac{1}{1+e^{-kx}}\\
                  \frac{e^x-e^{-x}}{e^x+e^{-x}}
                \end{array}
              \right.
  \]

  $\expval{x}$
  
  $\chi_\rho(ghg\dmo)=\Tr(\rho_{ghg\dmo})=\Tr(\rho_g\circ\rho_h\circ\rho\dmo_g)=\Tr(\rho_h)\overset{\mbox{\scalebox{0.5}{$\Tr(AB)=\Tr(BA)$}}}{=}\chi_\rho(h)$
  	$\mathop{\oplus}_{\substack{x\in X}}$

$\mat(\rho_g)=(a_{ij}(g))_{\scriptsize \substack{1\leq i\leq d \\ 1\leq j\leq d}}$ et $\mat(\rho'_g)=(a'_{ij}(g))_{\scriptsize \substack{1\leq i'\leq d' \\ 1\leq j'\leq d'}}$



\[\int_a^b{\mathbb{R}^2}g(u, v)\dd{P_{XY}}(u, v)=\iint g(u,v) f_{XY}(u, v)\dd \lambda(u) \dd \lambda(v)\]
$$\lim_{x\to\infty} f(x)$$	
$$\iiiint_V \mu(t,u,v,w) \,dt\,du\,dv\,dw$$
$$\sum_{n=1}^{\infty} 2^{-n} = 1$$	
\begin{definition}
	Si $X$ et $Y$ sont 2 v.a. ou definit la \textsc{Covariance} entre $X$ et $Y$ comme
	$\cov(X,Y)\overset{\text{def}}{=}\E\left[(X-\E(X))(Y-\E(Y))\right]=\E(XY)-\E(X)\E(Y)$.
\end{definition}
\fi
\pagebreak

% \tableofcontents

% insert your code here
%\input{./algebra/main.tex}
%\input{./geometrie-differentielle/main.tex}
%\input{./probabilite/main.tex}
%\input{./analyse-fonctionnelle/main.tex}
% \input{./Analyse-convexe-et-dualite-en-optimisation/main.tex}
%\input{./tikz/main.tex}
%\input{./Theorie-du-distributions/main.tex}
%\input{./optimisation/mine.tex}
 \input{./modelisation/main.tex}

% yves.aubry@univ-tln.fr : algebra

\end{document}

%% !TEX encoding = UTF-8 Unicode
% !TEX TS-program = xelatex

\documentclass[french]{report}

%\usepackage[utf8]{inputenc}
%\usepackage[T1]{fontenc}
\usepackage{babel}


\newif\ifcomment
%\commenttrue # Show comments

\usepackage{physics}
\usepackage{amssymb}


\usepackage{amsthm}
% \usepackage{thmtools}
\usepackage{mathtools}
\usepackage{amsfonts}

\usepackage{color}

\usepackage{tikz}

\usepackage{geometry}
\geometry{a5paper, margin=0.1in, right=1cm}

\usepackage{dsfont}

\usepackage{graphicx}
\graphicspath{ {images/} }

\usepackage{faktor}

\usepackage{IEEEtrantools}
\usepackage{enumerate}   
\usepackage[PostScript=dvips]{"/Users/aware/Documents/Courses/diagrams"}


\newtheorem{theorem}{Théorème}[section]
\renewcommand{\thetheorem}{\arabic{theorem}}
\newtheorem{lemme}{Lemme}[section]
\renewcommand{\thelemme}{\arabic{lemme}}
\newtheorem{proposition}{Proposition}[section]
\renewcommand{\theproposition}{\arabic{proposition}}
\newtheorem{notations}{Notations}[section]
\newtheorem{problem}{Problème}[section]
\newtheorem{corollary}{Corollaire}[theorem]
\renewcommand{\thecorollary}{\arabic{corollary}}
\newtheorem{property}{Propriété}[section]
\newtheorem{objective}{Objectif}[section]

\theoremstyle{definition}
\newtheorem{definition}{Définition}[section]
\renewcommand{\thedefinition}{\arabic{definition}}
\newtheorem{exercise}{Exercice}[chapter]
\renewcommand{\theexercise}{\arabic{exercise}}
\newtheorem{example}{Exemple}[chapter]
\renewcommand{\theexample}{\arabic{example}}
\newtheorem*{solution}{Solution}
\newtheorem*{application}{Application}
\newtheorem*{notation}{Notation}
\newtheorem*{vocabulary}{Vocabulaire}
\newtheorem*{properties}{Propriétés}



\theoremstyle{remark}
\newtheorem*{remark}{Remarque}
\newtheorem*{rappel}{Rappel}


\usepackage{etoolbox}
\AtBeginEnvironment{exercise}{\small}
\AtBeginEnvironment{example}{\small}

\usepackage{cases}
\usepackage[red]{mypack}

\usepackage[framemethod=TikZ]{mdframed}

\definecolor{bg}{rgb}{0.4,0.25,0.95}
\definecolor{pagebg}{rgb}{0,0,0.5}
\surroundwithmdframed[
   topline=false,
   rightline=false,
   bottomline=false,
   leftmargin=\parindent,
   skipabove=8pt,
   skipbelow=8pt,
   linecolor=blue,
   innerbottommargin=10pt,
   % backgroundcolor=bg,font=\color{orange}\sffamily, fontcolor=white
]{definition}

\usepackage{empheq}
\usepackage[most]{tcolorbox}

\newtcbox{\mymath}[1][]{%
    nobeforeafter, math upper, tcbox raise base,
    enhanced, colframe=blue!30!black,
    colback=red!10, boxrule=1pt,
    #1}

\usepackage{unixode}


\DeclareMathOperator{\ord}{ord}
\DeclareMathOperator{\orb}{orb}
\DeclareMathOperator{\stab}{stab}
\DeclareMathOperator{\Stab}{stab}
\DeclareMathOperator{\ppcm}{ppcm}
\DeclareMathOperator{\conj}{Conj}
\DeclareMathOperator{\End}{End}
\DeclareMathOperator{\rot}{rot}
\DeclareMathOperator{\trs}{trace}
\DeclareMathOperator{\Ind}{Ind}
\DeclareMathOperator{\mat}{Mat}
\DeclareMathOperator{\id}{Id}
\DeclareMathOperator{\vect}{vect}
\DeclareMathOperator{\img}{img}
\DeclareMathOperator{\cov}{Cov}
\DeclareMathOperator{\dist}{dist}
\DeclareMathOperator{\irr}{Irr}
\DeclareMathOperator{\image}{Im}
\DeclareMathOperator{\pd}{\partial}
\DeclareMathOperator{\epi}{epi}
\DeclareMathOperator{\Argmin}{Argmin}
\DeclareMathOperator{\dom}{dom}
\DeclareMathOperator{\proj}{proj}
\DeclareMathOperator{\ctg}{ctg}
\DeclareMathOperator{\supp}{supp}
\DeclareMathOperator{\argmin}{argmin}
\DeclareMathOperator{\mult}{mult}
\DeclareMathOperator{\ch}{ch}
\DeclareMathOperator{\sh}{sh}
\DeclareMathOperator{\rang}{rang}
\DeclareMathOperator{\diam}{diam}
\DeclareMathOperator{\Epigraphe}{Epigraphe}




\usepackage{xcolor}
\everymath{\color{blue}}
%\everymath{\color[rgb]{0,1,1}}
%\pagecolor[rgb]{0,0,0.5}


\newcommand*{\pdtest}[3][]{\ensuremath{\frac{\partial^{#1} #2}{\partial #3}}}

\newcommand*{\deffunc}[6][]{\ensuremath{
\begin{array}{rcl}
#2 : #3 &\rightarrow& #4\\
#5 &\mapsto& #6
\end{array}
}}

\newcommand{\eqcolon}{\mathrel{\resizebox{\widthof{$\mathord{=}$}}{\height}{ $\!\!=\!\!\resizebox{1.2\width}{0.8\height}{\raisebox{0.23ex}{$\mathop{:}$}}\!\!$ }}}
\newcommand{\coloneq}{\mathrel{\resizebox{\widthof{$\mathord{=}$}}{\height}{ $\!\!\resizebox{1.2\width}{0.8\height}{\raisebox{0.23ex}{$\mathop{:}$}}\!\!=\!\!$ }}}
\newcommand{\eqcolonl}{\ensuremath{\mathrel{=\!\!\mathop{:}}}}
\newcommand{\coloneql}{\ensuremath{\mathrel{\mathop{:} \!\! =}}}
\newcommand{\vc}[1]{% inline column vector
  \left(\begin{smallmatrix}#1\end{smallmatrix}\right)%
}
\newcommand{\vr}[1]{% inline row vector
  \begin{smallmatrix}(\,#1\,)\end{smallmatrix}%
}
\makeatletter
\newcommand*{\defeq}{\ =\mathrel{\rlap{%
                     \raisebox{0.3ex}{$\m@th\cdot$}}%
                     \raisebox{-0.3ex}{$\m@th\cdot$}}%
                     }
\makeatother

\newcommand{\mathcircle}[1]{% inline row vector
 \overset{\circ}{#1}
}
\newcommand{\ulim}{% low limit
 \underline{\lim}
}
\newcommand{\ssi}{% iff
\iff
}
\newcommand{\ps}[2]{
\expval{#1 | #2}
}
\newcommand{\df}[1]{
\mqty{#1}
}
\newcommand{\n}[1]{
\norm{#1}
}
\newcommand{\sys}[1]{
\left\{\smqty{#1}\right.
}


\newcommand{\eqdef}{\ensuremath{\overset{\text{def}}=}}


\def\Circlearrowright{\ensuremath{%
  \rotatebox[origin=c]{230}{$\circlearrowright$}}}

\newcommand\ct[1]{\text{\rmfamily\upshape #1}}
\newcommand\question[1]{ {\color{red} ...!? \small #1}}
\newcommand\caz[1]{\left\{\begin{array} #1 \end{array}\right.}
\newcommand\const{\text{\rmfamily\upshape const}}
\newcommand\toP{ \overset{\pro}{\to}}
\newcommand\toPP{ \overset{\text{PP}}{\to}}
\newcommand{\oeq}{\mathrel{\text{\textcircled{$=$}}}}





\usepackage{xcolor}
% \usepackage[normalem]{ulem}
\usepackage{lipsum}
\makeatletter
% \newcommand\colorwave[1][blue]{\bgroup \markoverwith{\lower3.5\p@\hbox{\sixly \textcolor{#1}{\char58}}}\ULon}
%\font\sixly=lasy6 % does not re-load if already loaded, so no memory problem.

\newmdtheoremenv[
linewidth= 1pt,linecolor= blue,%
leftmargin=20,rightmargin=20,innertopmargin=0pt, innerrightmargin=40,%
tikzsetting = { draw=lightgray, line width = 0.3pt,dashed,%
dash pattern = on 15pt off 3pt},%
splittopskip=\topskip,skipbelow=\baselineskip,%
skipabove=\baselineskip,ntheorem,roundcorner=0pt,
% backgroundcolor=pagebg,font=\color{orange}\sffamily, fontcolor=white
]{examplebox}{Exemple}[section]



\newcommand\R{\mathbb{R}}
\newcommand\Z{\mathbb{Z}}
\newcommand\N{\mathbb{N}}
\newcommand\E{\mathbb{E}}
\newcommand\F{\mathcal{F}}
\newcommand\cH{\mathcal{H}}
\newcommand\V{\mathbb{V}}
\newcommand\dmo{ ^{-1} }
\newcommand\kapa{\kappa}
\newcommand\im{Im}
\newcommand\hs{\mathcal{H}}





\usepackage{soul}

\makeatletter
\newcommand*{\whiten}[1]{\llap{\textcolor{white}{{\the\SOUL@token}}\hspace{#1pt}}}
\DeclareRobustCommand*\myul{%
    \def\SOUL@everyspace{\underline{\space}\kern\z@}%
    \def\SOUL@everytoken{%
     \setbox0=\hbox{\the\SOUL@token}%
     \ifdim\dp0>\z@
        \raisebox{\dp0}{\underline{\phantom{\the\SOUL@token}}}%
        \whiten{1}\whiten{0}%
        \whiten{-1}\whiten{-2}%
        \llap{\the\SOUL@token}%
     \else
        \underline{\the\SOUL@token}%
     \fi}%
\SOUL@}
\makeatother

\newcommand*{\demp}{\fontfamily{lmtt}\selectfont}

\DeclareTextFontCommand{\textdemp}{\demp}

\begin{document}

\ifcomment
Multiline
comment
\fi
\ifcomment
\myul{Typesetting test}
% \color[rgb]{1,1,1}
$∑_i^n≠ 60º±∞π∆¬≈√j∫h≤≥µ$

$\CR \R\pro\ind\pro\gS\pro
\mqty[a&b\\c&d]$
$\pro\mathbb{P}$
$\dd{x}$

  \[
    \alpha(x)=\left\{
                \begin{array}{ll}
                  x\\
                  \frac{1}{1+e^{-kx}}\\
                  \frac{e^x-e^{-x}}{e^x+e^{-x}}
                \end{array}
              \right.
  \]

  $\expval{x}$
  
  $\chi_\rho(ghg\dmo)=\Tr(\rho_{ghg\dmo})=\Tr(\rho_g\circ\rho_h\circ\rho\dmo_g)=\Tr(\rho_h)\overset{\mbox{\scalebox{0.5}{$\Tr(AB)=\Tr(BA)$}}}{=}\chi_\rho(h)$
  	$\mathop{\oplus}_{\substack{x\in X}}$

$\mat(\rho_g)=(a_{ij}(g))_{\scriptsize \substack{1\leq i\leq d \\ 1\leq j\leq d}}$ et $\mat(\rho'_g)=(a'_{ij}(g))_{\scriptsize \substack{1\leq i'\leq d' \\ 1\leq j'\leq d'}}$



\[\int_a^b{\mathbb{R}^2}g(u, v)\dd{P_{XY}}(u, v)=\iint g(u,v) f_{XY}(u, v)\dd \lambda(u) \dd \lambda(v)\]
$$\lim_{x\to\infty} f(x)$$	
$$\iiiint_V \mu(t,u,v,w) \,dt\,du\,dv\,dw$$
$$\sum_{n=1}^{\infty} 2^{-n} = 1$$	
\begin{definition}
	Si $X$ et $Y$ sont 2 v.a. ou definit la \textsc{Covariance} entre $X$ et $Y$ comme
	$\cov(X,Y)\overset{\text{def}}{=}\E\left[(X-\E(X))(Y-\E(Y))\right]=\E(XY)-\E(X)\E(Y)$.
\end{definition}
\fi
\pagebreak

% \tableofcontents

% insert your code here
%\input{./algebra/main.tex}
%\input{./geometrie-differentielle/main.tex}
%\input{./probabilite/main.tex}
%\input{./analyse-fonctionnelle/main.tex}
% \input{./Analyse-convexe-et-dualite-en-optimisation/main.tex}
%\input{./tikz/main.tex}
%\input{./Theorie-du-distributions/main.tex}
%\input{./optimisation/mine.tex}
 \input{./modelisation/main.tex}

% yves.aubry@univ-tln.fr : algebra

\end{document}

%% !TEX encoding = UTF-8 Unicode
% !TEX TS-program = xelatex

\documentclass[french]{report}

%\usepackage[utf8]{inputenc}
%\usepackage[T1]{fontenc}
\usepackage{babel}


\newif\ifcomment
%\commenttrue # Show comments

\usepackage{physics}
\usepackage{amssymb}


\usepackage{amsthm}
% \usepackage{thmtools}
\usepackage{mathtools}
\usepackage{amsfonts}

\usepackage{color}

\usepackage{tikz}

\usepackage{geometry}
\geometry{a5paper, margin=0.1in, right=1cm}

\usepackage{dsfont}

\usepackage{graphicx}
\graphicspath{ {images/} }

\usepackage{faktor}

\usepackage{IEEEtrantools}
\usepackage{enumerate}   
\usepackage[PostScript=dvips]{"/Users/aware/Documents/Courses/diagrams"}


\newtheorem{theorem}{Théorème}[section]
\renewcommand{\thetheorem}{\arabic{theorem}}
\newtheorem{lemme}{Lemme}[section]
\renewcommand{\thelemme}{\arabic{lemme}}
\newtheorem{proposition}{Proposition}[section]
\renewcommand{\theproposition}{\arabic{proposition}}
\newtheorem{notations}{Notations}[section]
\newtheorem{problem}{Problème}[section]
\newtheorem{corollary}{Corollaire}[theorem]
\renewcommand{\thecorollary}{\arabic{corollary}}
\newtheorem{property}{Propriété}[section]
\newtheorem{objective}{Objectif}[section]

\theoremstyle{definition}
\newtheorem{definition}{Définition}[section]
\renewcommand{\thedefinition}{\arabic{definition}}
\newtheorem{exercise}{Exercice}[chapter]
\renewcommand{\theexercise}{\arabic{exercise}}
\newtheorem{example}{Exemple}[chapter]
\renewcommand{\theexample}{\arabic{example}}
\newtheorem*{solution}{Solution}
\newtheorem*{application}{Application}
\newtheorem*{notation}{Notation}
\newtheorem*{vocabulary}{Vocabulaire}
\newtheorem*{properties}{Propriétés}



\theoremstyle{remark}
\newtheorem*{remark}{Remarque}
\newtheorem*{rappel}{Rappel}


\usepackage{etoolbox}
\AtBeginEnvironment{exercise}{\small}
\AtBeginEnvironment{example}{\small}

\usepackage{cases}
\usepackage[red]{mypack}

\usepackage[framemethod=TikZ]{mdframed}

\definecolor{bg}{rgb}{0.4,0.25,0.95}
\definecolor{pagebg}{rgb}{0,0,0.5}
\surroundwithmdframed[
   topline=false,
   rightline=false,
   bottomline=false,
   leftmargin=\parindent,
   skipabove=8pt,
   skipbelow=8pt,
   linecolor=blue,
   innerbottommargin=10pt,
   % backgroundcolor=bg,font=\color{orange}\sffamily, fontcolor=white
]{definition}

\usepackage{empheq}
\usepackage[most]{tcolorbox}

\newtcbox{\mymath}[1][]{%
    nobeforeafter, math upper, tcbox raise base,
    enhanced, colframe=blue!30!black,
    colback=red!10, boxrule=1pt,
    #1}

\usepackage{unixode}


\DeclareMathOperator{\ord}{ord}
\DeclareMathOperator{\orb}{orb}
\DeclareMathOperator{\stab}{stab}
\DeclareMathOperator{\Stab}{stab}
\DeclareMathOperator{\ppcm}{ppcm}
\DeclareMathOperator{\conj}{Conj}
\DeclareMathOperator{\End}{End}
\DeclareMathOperator{\rot}{rot}
\DeclareMathOperator{\trs}{trace}
\DeclareMathOperator{\Ind}{Ind}
\DeclareMathOperator{\mat}{Mat}
\DeclareMathOperator{\id}{Id}
\DeclareMathOperator{\vect}{vect}
\DeclareMathOperator{\img}{img}
\DeclareMathOperator{\cov}{Cov}
\DeclareMathOperator{\dist}{dist}
\DeclareMathOperator{\irr}{Irr}
\DeclareMathOperator{\image}{Im}
\DeclareMathOperator{\pd}{\partial}
\DeclareMathOperator{\epi}{epi}
\DeclareMathOperator{\Argmin}{Argmin}
\DeclareMathOperator{\dom}{dom}
\DeclareMathOperator{\proj}{proj}
\DeclareMathOperator{\ctg}{ctg}
\DeclareMathOperator{\supp}{supp}
\DeclareMathOperator{\argmin}{argmin}
\DeclareMathOperator{\mult}{mult}
\DeclareMathOperator{\ch}{ch}
\DeclareMathOperator{\sh}{sh}
\DeclareMathOperator{\rang}{rang}
\DeclareMathOperator{\diam}{diam}
\DeclareMathOperator{\Epigraphe}{Epigraphe}




\usepackage{xcolor}
\everymath{\color{blue}}
%\everymath{\color[rgb]{0,1,1}}
%\pagecolor[rgb]{0,0,0.5}


\newcommand*{\pdtest}[3][]{\ensuremath{\frac{\partial^{#1} #2}{\partial #3}}}

\newcommand*{\deffunc}[6][]{\ensuremath{
\begin{array}{rcl}
#2 : #3 &\rightarrow& #4\\
#5 &\mapsto& #6
\end{array}
}}

\newcommand{\eqcolon}{\mathrel{\resizebox{\widthof{$\mathord{=}$}}{\height}{ $\!\!=\!\!\resizebox{1.2\width}{0.8\height}{\raisebox{0.23ex}{$\mathop{:}$}}\!\!$ }}}
\newcommand{\coloneq}{\mathrel{\resizebox{\widthof{$\mathord{=}$}}{\height}{ $\!\!\resizebox{1.2\width}{0.8\height}{\raisebox{0.23ex}{$\mathop{:}$}}\!\!=\!\!$ }}}
\newcommand{\eqcolonl}{\ensuremath{\mathrel{=\!\!\mathop{:}}}}
\newcommand{\coloneql}{\ensuremath{\mathrel{\mathop{:} \!\! =}}}
\newcommand{\vc}[1]{% inline column vector
  \left(\begin{smallmatrix}#1\end{smallmatrix}\right)%
}
\newcommand{\vr}[1]{% inline row vector
  \begin{smallmatrix}(\,#1\,)\end{smallmatrix}%
}
\makeatletter
\newcommand*{\defeq}{\ =\mathrel{\rlap{%
                     \raisebox{0.3ex}{$\m@th\cdot$}}%
                     \raisebox{-0.3ex}{$\m@th\cdot$}}%
                     }
\makeatother

\newcommand{\mathcircle}[1]{% inline row vector
 \overset{\circ}{#1}
}
\newcommand{\ulim}{% low limit
 \underline{\lim}
}
\newcommand{\ssi}{% iff
\iff
}
\newcommand{\ps}[2]{
\expval{#1 | #2}
}
\newcommand{\df}[1]{
\mqty{#1}
}
\newcommand{\n}[1]{
\norm{#1}
}
\newcommand{\sys}[1]{
\left\{\smqty{#1}\right.
}


\newcommand{\eqdef}{\ensuremath{\overset{\text{def}}=}}


\def\Circlearrowright{\ensuremath{%
  \rotatebox[origin=c]{230}{$\circlearrowright$}}}

\newcommand\ct[1]{\text{\rmfamily\upshape #1}}
\newcommand\question[1]{ {\color{red} ...!? \small #1}}
\newcommand\caz[1]{\left\{\begin{array} #1 \end{array}\right.}
\newcommand\const{\text{\rmfamily\upshape const}}
\newcommand\toP{ \overset{\pro}{\to}}
\newcommand\toPP{ \overset{\text{PP}}{\to}}
\newcommand{\oeq}{\mathrel{\text{\textcircled{$=$}}}}





\usepackage{xcolor}
% \usepackage[normalem]{ulem}
\usepackage{lipsum}
\makeatletter
% \newcommand\colorwave[1][blue]{\bgroup \markoverwith{\lower3.5\p@\hbox{\sixly \textcolor{#1}{\char58}}}\ULon}
%\font\sixly=lasy6 % does not re-load if already loaded, so no memory problem.

\newmdtheoremenv[
linewidth= 1pt,linecolor= blue,%
leftmargin=20,rightmargin=20,innertopmargin=0pt, innerrightmargin=40,%
tikzsetting = { draw=lightgray, line width = 0.3pt,dashed,%
dash pattern = on 15pt off 3pt},%
splittopskip=\topskip,skipbelow=\baselineskip,%
skipabove=\baselineskip,ntheorem,roundcorner=0pt,
% backgroundcolor=pagebg,font=\color{orange}\sffamily, fontcolor=white
]{examplebox}{Exemple}[section]



\newcommand\R{\mathbb{R}}
\newcommand\Z{\mathbb{Z}}
\newcommand\N{\mathbb{N}}
\newcommand\E{\mathbb{E}}
\newcommand\F{\mathcal{F}}
\newcommand\cH{\mathcal{H}}
\newcommand\V{\mathbb{V}}
\newcommand\dmo{ ^{-1} }
\newcommand\kapa{\kappa}
\newcommand\im{Im}
\newcommand\hs{\mathcal{H}}





\usepackage{soul}

\makeatletter
\newcommand*{\whiten}[1]{\llap{\textcolor{white}{{\the\SOUL@token}}\hspace{#1pt}}}
\DeclareRobustCommand*\myul{%
    \def\SOUL@everyspace{\underline{\space}\kern\z@}%
    \def\SOUL@everytoken{%
     \setbox0=\hbox{\the\SOUL@token}%
     \ifdim\dp0>\z@
        \raisebox{\dp0}{\underline{\phantom{\the\SOUL@token}}}%
        \whiten{1}\whiten{0}%
        \whiten{-1}\whiten{-2}%
        \llap{\the\SOUL@token}%
     \else
        \underline{\the\SOUL@token}%
     \fi}%
\SOUL@}
\makeatother

\newcommand*{\demp}{\fontfamily{lmtt}\selectfont}

\DeclareTextFontCommand{\textdemp}{\demp}

\begin{document}

\ifcomment
Multiline
comment
\fi
\ifcomment
\myul{Typesetting test}
% \color[rgb]{1,1,1}
$∑_i^n≠ 60º±∞π∆¬≈√j∫h≤≥µ$

$\CR \R\pro\ind\pro\gS\pro
\mqty[a&b\\c&d]$
$\pro\mathbb{P}$
$\dd{x}$

  \[
    \alpha(x)=\left\{
                \begin{array}{ll}
                  x\\
                  \frac{1}{1+e^{-kx}}\\
                  \frac{e^x-e^{-x}}{e^x+e^{-x}}
                \end{array}
              \right.
  \]

  $\expval{x}$
  
  $\chi_\rho(ghg\dmo)=\Tr(\rho_{ghg\dmo})=\Tr(\rho_g\circ\rho_h\circ\rho\dmo_g)=\Tr(\rho_h)\overset{\mbox{\scalebox{0.5}{$\Tr(AB)=\Tr(BA)$}}}{=}\chi_\rho(h)$
  	$\mathop{\oplus}_{\substack{x\in X}}$

$\mat(\rho_g)=(a_{ij}(g))_{\scriptsize \substack{1\leq i\leq d \\ 1\leq j\leq d}}$ et $\mat(\rho'_g)=(a'_{ij}(g))_{\scriptsize \substack{1\leq i'\leq d' \\ 1\leq j'\leq d'}}$



\[\int_a^b{\mathbb{R}^2}g(u, v)\dd{P_{XY}}(u, v)=\iint g(u,v) f_{XY}(u, v)\dd \lambda(u) \dd \lambda(v)\]
$$\lim_{x\to\infty} f(x)$$	
$$\iiiint_V \mu(t,u,v,w) \,dt\,du\,dv\,dw$$
$$\sum_{n=1}^{\infty} 2^{-n} = 1$$	
\begin{definition}
	Si $X$ et $Y$ sont 2 v.a. ou definit la \textsc{Covariance} entre $X$ et $Y$ comme
	$\cov(X,Y)\overset{\text{def}}{=}\E\left[(X-\E(X))(Y-\E(Y))\right]=\E(XY)-\E(X)\E(Y)$.
\end{definition}
\fi
\pagebreak

% \tableofcontents

% insert your code here
%\input{./algebra/main.tex}
%\input{./geometrie-differentielle/main.tex}
%\input{./probabilite/main.tex}
%\input{./analyse-fonctionnelle/main.tex}
% \input{./Analyse-convexe-et-dualite-en-optimisation/main.tex}
%\input{./tikz/main.tex}
%\input{./Theorie-du-distributions/main.tex}
%\input{./optimisation/mine.tex}
 \input{./modelisation/main.tex}

% yves.aubry@univ-tln.fr : algebra

\end{document}

% % !TEX encoding = UTF-8 Unicode
% !TEX TS-program = xelatex

\documentclass[french]{report}

%\usepackage[utf8]{inputenc}
%\usepackage[T1]{fontenc}
\usepackage{babel}


\newif\ifcomment
%\commenttrue # Show comments

\usepackage{physics}
\usepackage{amssymb}


\usepackage{amsthm}
% \usepackage{thmtools}
\usepackage{mathtools}
\usepackage{amsfonts}

\usepackage{color}

\usepackage{tikz}

\usepackage{geometry}
\geometry{a5paper, margin=0.1in, right=1cm}

\usepackage{dsfont}

\usepackage{graphicx}
\graphicspath{ {images/} }

\usepackage{faktor}

\usepackage{IEEEtrantools}
\usepackage{enumerate}   
\usepackage[PostScript=dvips]{"/Users/aware/Documents/Courses/diagrams"}


\newtheorem{theorem}{Théorème}[section]
\renewcommand{\thetheorem}{\arabic{theorem}}
\newtheorem{lemme}{Lemme}[section]
\renewcommand{\thelemme}{\arabic{lemme}}
\newtheorem{proposition}{Proposition}[section]
\renewcommand{\theproposition}{\arabic{proposition}}
\newtheorem{notations}{Notations}[section]
\newtheorem{problem}{Problème}[section]
\newtheorem{corollary}{Corollaire}[theorem]
\renewcommand{\thecorollary}{\arabic{corollary}}
\newtheorem{property}{Propriété}[section]
\newtheorem{objective}{Objectif}[section]

\theoremstyle{definition}
\newtheorem{definition}{Définition}[section]
\renewcommand{\thedefinition}{\arabic{definition}}
\newtheorem{exercise}{Exercice}[chapter]
\renewcommand{\theexercise}{\arabic{exercise}}
\newtheorem{example}{Exemple}[chapter]
\renewcommand{\theexample}{\arabic{example}}
\newtheorem*{solution}{Solution}
\newtheorem*{application}{Application}
\newtheorem*{notation}{Notation}
\newtheorem*{vocabulary}{Vocabulaire}
\newtheorem*{properties}{Propriétés}



\theoremstyle{remark}
\newtheorem*{remark}{Remarque}
\newtheorem*{rappel}{Rappel}


\usepackage{etoolbox}
\AtBeginEnvironment{exercise}{\small}
\AtBeginEnvironment{example}{\small}

\usepackage{cases}
\usepackage[red]{mypack}

\usepackage[framemethod=TikZ]{mdframed}

\definecolor{bg}{rgb}{0.4,0.25,0.95}
\definecolor{pagebg}{rgb}{0,0,0.5}
\surroundwithmdframed[
   topline=false,
   rightline=false,
   bottomline=false,
   leftmargin=\parindent,
   skipabove=8pt,
   skipbelow=8pt,
   linecolor=blue,
   innerbottommargin=10pt,
   % backgroundcolor=bg,font=\color{orange}\sffamily, fontcolor=white
]{definition}

\usepackage{empheq}
\usepackage[most]{tcolorbox}

\newtcbox{\mymath}[1][]{%
    nobeforeafter, math upper, tcbox raise base,
    enhanced, colframe=blue!30!black,
    colback=red!10, boxrule=1pt,
    #1}

\usepackage{unixode}


\DeclareMathOperator{\ord}{ord}
\DeclareMathOperator{\orb}{orb}
\DeclareMathOperator{\stab}{stab}
\DeclareMathOperator{\Stab}{stab}
\DeclareMathOperator{\ppcm}{ppcm}
\DeclareMathOperator{\conj}{Conj}
\DeclareMathOperator{\End}{End}
\DeclareMathOperator{\rot}{rot}
\DeclareMathOperator{\trs}{trace}
\DeclareMathOperator{\Ind}{Ind}
\DeclareMathOperator{\mat}{Mat}
\DeclareMathOperator{\id}{Id}
\DeclareMathOperator{\vect}{vect}
\DeclareMathOperator{\img}{img}
\DeclareMathOperator{\cov}{Cov}
\DeclareMathOperator{\dist}{dist}
\DeclareMathOperator{\irr}{Irr}
\DeclareMathOperator{\image}{Im}
\DeclareMathOperator{\pd}{\partial}
\DeclareMathOperator{\epi}{epi}
\DeclareMathOperator{\Argmin}{Argmin}
\DeclareMathOperator{\dom}{dom}
\DeclareMathOperator{\proj}{proj}
\DeclareMathOperator{\ctg}{ctg}
\DeclareMathOperator{\supp}{supp}
\DeclareMathOperator{\argmin}{argmin}
\DeclareMathOperator{\mult}{mult}
\DeclareMathOperator{\ch}{ch}
\DeclareMathOperator{\sh}{sh}
\DeclareMathOperator{\rang}{rang}
\DeclareMathOperator{\diam}{diam}
\DeclareMathOperator{\Epigraphe}{Epigraphe}




\usepackage{xcolor}
\everymath{\color{blue}}
%\everymath{\color[rgb]{0,1,1}}
%\pagecolor[rgb]{0,0,0.5}


\newcommand*{\pdtest}[3][]{\ensuremath{\frac{\partial^{#1} #2}{\partial #3}}}

\newcommand*{\deffunc}[6][]{\ensuremath{
\begin{array}{rcl}
#2 : #3 &\rightarrow& #4\\
#5 &\mapsto& #6
\end{array}
}}

\newcommand{\eqcolon}{\mathrel{\resizebox{\widthof{$\mathord{=}$}}{\height}{ $\!\!=\!\!\resizebox{1.2\width}{0.8\height}{\raisebox{0.23ex}{$\mathop{:}$}}\!\!$ }}}
\newcommand{\coloneq}{\mathrel{\resizebox{\widthof{$\mathord{=}$}}{\height}{ $\!\!\resizebox{1.2\width}{0.8\height}{\raisebox{0.23ex}{$\mathop{:}$}}\!\!=\!\!$ }}}
\newcommand{\eqcolonl}{\ensuremath{\mathrel{=\!\!\mathop{:}}}}
\newcommand{\coloneql}{\ensuremath{\mathrel{\mathop{:} \!\! =}}}
\newcommand{\vc}[1]{% inline column vector
  \left(\begin{smallmatrix}#1\end{smallmatrix}\right)%
}
\newcommand{\vr}[1]{% inline row vector
  \begin{smallmatrix}(\,#1\,)\end{smallmatrix}%
}
\makeatletter
\newcommand*{\defeq}{\ =\mathrel{\rlap{%
                     \raisebox{0.3ex}{$\m@th\cdot$}}%
                     \raisebox{-0.3ex}{$\m@th\cdot$}}%
                     }
\makeatother

\newcommand{\mathcircle}[1]{% inline row vector
 \overset{\circ}{#1}
}
\newcommand{\ulim}{% low limit
 \underline{\lim}
}
\newcommand{\ssi}{% iff
\iff
}
\newcommand{\ps}[2]{
\expval{#1 | #2}
}
\newcommand{\df}[1]{
\mqty{#1}
}
\newcommand{\n}[1]{
\norm{#1}
}
\newcommand{\sys}[1]{
\left\{\smqty{#1}\right.
}


\newcommand{\eqdef}{\ensuremath{\overset{\text{def}}=}}


\def\Circlearrowright{\ensuremath{%
  \rotatebox[origin=c]{230}{$\circlearrowright$}}}

\newcommand\ct[1]{\text{\rmfamily\upshape #1}}
\newcommand\question[1]{ {\color{red} ...!? \small #1}}
\newcommand\caz[1]{\left\{\begin{array} #1 \end{array}\right.}
\newcommand\const{\text{\rmfamily\upshape const}}
\newcommand\toP{ \overset{\pro}{\to}}
\newcommand\toPP{ \overset{\text{PP}}{\to}}
\newcommand{\oeq}{\mathrel{\text{\textcircled{$=$}}}}





\usepackage{xcolor}
% \usepackage[normalem]{ulem}
\usepackage{lipsum}
\makeatletter
% \newcommand\colorwave[1][blue]{\bgroup \markoverwith{\lower3.5\p@\hbox{\sixly \textcolor{#1}{\char58}}}\ULon}
%\font\sixly=lasy6 % does not re-load if already loaded, so no memory problem.

\newmdtheoremenv[
linewidth= 1pt,linecolor= blue,%
leftmargin=20,rightmargin=20,innertopmargin=0pt, innerrightmargin=40,%
tikzsetting = { draw=lightgray, line width = 0.3pt,dashed,%
dash pattern = on 15pt off 3pt},%
splittopskip=\topskip,skipbelow=\baselineskip,%
skipabove=\baselineskip,ntheorem,roundcorner=0pt,
% backgroundcolor=pagebg,font=\color{orange}\sffamily, fontcolor=white
]{examplebox}{Exemple}[section]



\newcommand\R{\mathbb{R}}
\newcommand\Z{\mathbb{Z}}
\newcommand\N{\mathbb{N}}
\newcommand\E{\mathbb{E}}
\newcommand\F{\mathcal{F}}
\newcommand\cH{\mathcal{H}}
\newcommand\V{\mathbb{V}}
\newcommand\dmo{ ^{-1} }
\newcommand\kapa{\kappa}
\newcommand\im{Im}
\newcommand\hs{\mathcal{H}}





\usepackage{soul}

\makeatletter
\newcommand*{\whiten}[1]{\llap{\textcolor{white}{{\the\SOUL@token}}\hspace{#1pt}}}
\DeclareRobustCommand*\myul{%
    \def\SOUL@everyspace{\underline{\space}\kern\z@}%
    \def\SOUL@everytoken{%
     \setbox0=\hbox{\the\SOUL@token}%
     \ifdim\dp0>\z@
        \raisebox{\dp0}{\underline{\phantom{\the\SOUL@token}}}%
        \whiten{1}\whiten{0}%
        \whiten{-1}\whiten{-2}%
        \llap{\the\SOUL@token}%
     \else
        \underline{\the\SOUL@token}%
     \fi}%
\SOUL@}
\makeatother

\newcommand*{\demp}{\fontfamily{lmtt}\selectfont}

\DeclareTextFontCommand{\textdemp}{\demp}

\begin{document}

\ifcomment
Multiline
comment
\fi
\ifcomment
\myul{Typesetting test}
% \color[rgb]{1,1,1}
$∑_i^n≠ 60º±∞π∆¬≈√j∫h≤≥µ$

$\CR \R\pro\ind\pro\gS\pro
\mqty[a&b\\c&d]$
$\pro\mathbb{P}$
$\dd{x}$

  \[
    \alpha(x)=\left\{
                \begin{array}{ll}
                  x\\
                  \frac{1}{1+e^{-kx}}\\
                  \frac{e^x-e^{-x}}{e^x+e^{-x}}
                \end{array}
              \right.
  \]

  $\expval{x}$
  
  $\chi_\rho(ghg\dmo)=\Tr(\rho_{ghg\dmo})=\Tr(\rho_g\circ\rho_h\circ\rho\dmo_g)=\Tr(\rho_h)\overset{\mbox{\scalebox{0.5}{$\Tr(AB)=\Tr(BA)$}}}{=}\chi_\rho(h)$
  	$\mathop{\oplus}_{\substack{x\in X}}$

$\mat(\rho_g)=(a_{ij}(g))_{\scriptsize \substack{1\leq i\leq d \\ 1\leq j\leq d}}$ et $\mat(\rho'_g)=(a'_{ij}(g))_{\scriptsize \substack{1\leq i'\leq d' \\ 1\leq j'\leq d'}}$



\[\int_a^b{\mathbb{R}^2}g(u, v)\dd{P_{XY}}(u, v)=\iint g(u,v) f_{XY}(u, v)\dd \lambda(u) \dd \lambda(v)\]
$$\lim_{x\to\infty} f(x)$$	
$$\iiiint_V \mu(t,u,v,w) \,dt\,du\,dv\,dw$$
$$\sum_{n=1}^{\infty} 2^{-n} = 1$$	
\begin{definition}
	Si $X$ et $Y$ sont 2 v.a. ou definit la \textsc{Covariance} entre $X$ et $Y$ comme
	$\cov(X,Y)\overset{\text{def}}{=}\E\left[(X-\E(X))(Y-\E(Y))\right]=\E(XY)-\E(X)\E(Y)$.
\end{definition}
\fi
\pagebreak

% \tableofcontents

% insert your code here
%\input{./algebra/main.tex}
%\input{./geometrie-differentielle/main.tex}
%\input{./probabilite/main.tex}
%\input{./analyse-fonctionnelle/main.tex}
% \input{./Analyse-convexe-et-dualite-en-optimisation/main.tex}
%\input{./tikz/main.tex}
%\input{./Theorie-du-distributions/main.tex}
%\input{./optimisation/mine.tex}
 \input{./modelisation/main.tex}

% yves.aubry@univ-tln.fr : algebra

\end{document}

%% !TEX encoding = UTF-8 Unicode
% !TEX TS-program = xelatex

\documentclass[french]{report}

%\usepackage[utf8]{inputenc}
%\usepackage[T1]{fontenc}
\usepackage{babel}


\newif\ifcomment
%\commenttrue # Show comments

\usepackage{physics}
\usepackage{amssymb}


\usepackage{amsthm}
% \usepackage{thmtools}
\usepackage{mathtools}
\usepackage{amsfonts}

\usepackage{color}

\usepackage{tikz}

\usepackage{geometry}
\geometry{a5paper, margin=0.1in, right=1cm}

\usepackage{dsfont}

\usepackage{graphicx}
\graphicspath{ {images/} }

\usepackage{faktor}

\usepackage{IEEEtrantools}
\usepackage{enumerate}   
\usepackage[PostScript=dvips]{"/Users/aware/Documents/Courses/diagrams"}


\newtheorem{theorem}{Théorème}[section]
\renewcommand{\thetheorem}{\arabic{theorem}}
\newtheorem{lemme}{Lemme}[section]
\renewcommand{\thelemme}{\arabic{lemme}}
\newtheorem{proposition}{Proposition}[section]
\renewcommand{\theproposition}{\arabic{proposition}}
\newtheorem{notations}{Notations}[section]
\newtheorem{problem}{Problème}[section]
\newtheorem{corollary}{Corollaire}[theorem]
\renewcommand{\thecorollary}{\arabic{corollary}}
\newtheorem{property}{Propriété}[section]
\newtheorem{objective}{Objectif}[section]

\theoremstyle{definition}
\newtheorem{definition}{Définition}[section]
\renewcommand{\thedefinition}{\arabic{definition}}
\newtheorem{exercise}{Exercice}[chapter]
\renewcommand{\theexercise}{\arabic{exercise}}
\newtheorem{example}{Exemple}[chapter]
\renewcommand{\theexample}{\arabic{example}}
\newtheorem*{solution}{Solution}
\newtheorem*{application}{Application}
\newtheorem*{notation}{Notation}
\newtheorem*{vocabulary}{Vocabulaire}
\newtheorem*{properties}{Propriétés}



\theoremstyle{remark}
\newtheorem*{remark}{Remarque}
\newtheorem*{rappel}{Rappel}


\usepackage{etoolbox}
\AtBeginEnvironment{exercise}{\small}
\AtBeginEnvironment{example}{\small}

\usepackage{cases}
\usepackage[red]{mypack}

\usepackage[framemethod=TikZ]{mdframed}

\definecolor{bg}{rgb}{0.4,0.25,0.95}
\definecolor{pagebg}{rgb}{0,0,0.5}
\surroundwithmdframed[
   topline=false,
   rightline=false,
   bottomline=false,
   leftmargin=\parindent,
   skipabove=8pt,
   skipbelow=8pt,
   linecolor=blue,
   innerbottommargin=10pt,
   % backgroundcolor=bg,font=\color{orange}\sffamily, fontcolor=white
]{definition}

\usepackage{empheq}
\usepackage[most]{tcolorbox}

\newtcbox{\mymath}[1][]{%
    nobeforeafter, math upper, tcbox raise base,
    enhanced, colframe=blue!30!black,
    colback=red!10, boxrule=1pt,
    #1}

\usepackage{unixode}


\DeclareMathOperator{\ord}{ord}
\DeclareMathOperator{\orb}{orb}
\DeclareMathOperator{\stab}{stab}
\DeclareMathOperator{\Stab}{stab}
\DeclareMathOperator{\ppcm}{ppcm}
\DeclareMathOperator{\conj}{Conj}
\DeclareMathOperator{\End}{End}
\DeclareMathOperator{\rot}{rot}
\DeclareMathOperator{\trs}{trace}
\DeclareMathOperator{\Ind}{Ind}
\DeclareMathOperator{\mat}{Mat}
\DeclareMathOperator{\id}{Id}
\DeclareMathOperator{\vect}{vect}
\DeclareMathOperator{\img}{img}
\DeclareMathOperator{\cov}{Cov}
\DeclareMathOperator{\dist}{dist}
\DeclareMathOperator{\irr}{Irr}
\DeclareMathOperator{\image}{Im}
\DeclareMathOperator{\pd}{\partial}
\DeclareMathOperator{\epi}{epi}
\DeclareMathOperator{\Argmin}{Argmin}
\DeclareMathOperator{\dom}{dom}
\DeclareMathOperator{\proj}{proj}
\DeclareMathOperator{\ctg}{ctg}
\DeclareMathOperator{\supp}{supp}
\DeclareMathOperator{\argmin}{argmin}
\DeclareMathOperator{\mult}{mult}
\DeclareMathOperator{\ch}{ch}
\DeclareMathOperator{\sh}{sh}
\DeclareMathOperator{\rang}{rang}
\DeclareMathOperator{\diam}{diam}
\DeclareMathOperator{\Epigraphe}{Epigraphe}




\usepackage{xcolor}
\everymath{\color{blue}}
%\everymath{\color[rgb]{0,1,1}}
%\pagecolor[rgb]{0,0,0.5}


\newcommand*{\pdtest}[3][]{\ensuremath{\frac{\partial^{#1} #2}{\partial #3}}}

\newcommand*{\deffunc}[6][]{\ensuremath{
\begin{array}{rcl}
#2 : #3 &\rightarrow& #4\\
#5 &\mapsto& #6
\end{array}
}}

\newcommand{\eqcolon}{\mathrel{\resizebox{\widthof{$\mathord{=}$}}{\height}{ $\!\!=\!\!\resizebox{1.2\width}{0.8\height}{\raisebox{0.23ex}{$\mathop{:}$}}\!\!$ }}}
\newcommand{\coloneq}{\mathrel{\resizebox{\widthof{$\mathord{=}$}}{\height}{ $\!\!\resizebox{1.2\width}{0.8\height}{\raisebox{0.23ex}{$\mathop{:}$}}\!\!=\!\!$ }}}
\newcommand{\eqcolonl}{\ensuremath{\mathrel{=\!\!\mathop{:}}}}
\newcommand{\coloneql}{\ensuremath{\mathrel{\mathop{:} \!\! =}}}
\newcommand{\vc}[1]{% inline column vector
  \left(\begin{smallmatrix}#1\end{smallmatrix}\right)%
}
\newcommand{\vr}[1]{% inline row vector
  \begin{smallmatrix}(\,#1\,)\end{smallmatrix}%
}
\makeatletter
\newcommand*{\defeq}{\ =\mathrel{\rlap{%
                     \raisebox{0.3ex}{$\m@th\cdot$}}%
                     \raisebox{-0.3ex}{$\m@th\cdot$}}%
                     }
\makeatother

\newcommand{\mathcircle}[1]{% inline row vector
 \overset{\circ}{#1}
}
\newcommand{\ulim}{% low limit
 \underline{\lim}
}
\newcommand{\ssi}{% iff
\iff
}
\newcommand{\ps}[2]{
\expval{#1 | #2}
}
\newcommand{\df}[1]{
\mqty{#1}
}
\newcommand{\n}[1]{
\norm{#1}
}
\newcommand{\sys}[1]{
\left\{\smqty{#1}\right.
}


\newcommand{\eqdef}{\ensuremath{\overset{\text{def}}=}}


\def\Circlearrowright{\ensuremath{%
  \rotatebox[origin=c]{230}{$\circlearrowright$}}}

\newcommand\ct[1]{\text{\rmfamily\upshape #1}}
\newcommand\question[1]{ {\color{red} ...!? \small #1}}
\newcommand\caz[1]{\left\{\begin{array} #1 \end{array}\right.}
\newcommand\const{\text{\rmfamily\upshape const}}
\newcommand\toP{ \overset{\pro}{\to}}
\newcommand\toPP{ \overset{\text{PP}}{\to}}
\newcommand{\oeq}{\mathrel{\text{\textcircled{$=$}}}}





\usepackage{xcolor}
% \usepackage[normalem]{ulem}
\usepackage{lipsum}
\makeatletter
% \newcommand\colorwave[1][blue]{\bgroup \markoverwith{\lower3.5\p@\hbox{\sixly \textcolor{#1}{\char58}}}\ULon}
%\font\sixly=lasy6 % does not re-load if already loaded, so no memory problem.

\newmdtheoremenv[
linewidth= 1pt,linecolor= blue,%
leftmargin=20,rightmargin=20,innertopmargin=0pt, innerrightmargin=40,%
tikzsetting = { draw=lightgray, line width = 0.3pt,dashed,%
dash pattern = on 15pt off 3pt},%
splittopskip=\topskip,skipbelow=\baselineskip,%
skipabove=\baselineskip,ntheorem,roundcorner=0pt,
% backgroundcolor=pagebg,font=\color{orange}\sffamily, fontcolor=white
]{examplebox}{Exemple}[section]



\newcommand\R{\mathbb{R}}
\newcommand\Z{\mathbb{Z}}
\newcommand\N{\mathbb{N}}
\newcommand\E{\mathbb{E}}
\newcommand\F{\mathcal{F}}
\newcommand\cH{\mathcal{H}}
\newcommand\V{\mathbb{V}}
\newcommand\dmo{ ^{-1} }
\newcommand\kapa{\kappa}
\newcommand\im{Im}
\newcommand\hs{\mathcal{H}}





\usepackage{soul}

\makeatletter
\newcommand*{\whiten}[1]{\llap{\textcolor{white}{{\the\SOUL@token}}\hspace{#1pt}}}
\DeclareRobustCommand*\myul{%
    \def\SOUL@everyspace{\underline{\space}\kern\z@}%
    \def\SOUL@everytoken{%
     \setbox0=\hbox{\the\SOUL@token}%
     \ifdim\dp0>\z@
        \raisebox{\dp0}{\underline{\phantom{\the\SOUL@token}}}%
        \whiten{1}\whiten{0}%
        \whiten{-1}\whiten{-2}%
        \llap{\the\SOUL@token}%
     \else
        \underline{\the\SOUL@token}%
     \fi}%
\SOUL@}
\makeatother

\newcommand*{\demp}{\fontfamily{lmtt}\selectfont}

\DeclareTextFontCommand{\textdemp}{\demp}

\begin{document}

\ifcomment
Multiline
comment
\fi
\ifcomment
\myul{Typesetting test}
% \color[rgb]{1,1,1}
$∑_i^n≠ 60º±∞π∆¬≈√j∫h≤≥µ$

$\CR \R\pro\ind\pro\gS\pro
\mqty[a&b\\c&d]$
$\pro\mathbb{P}$
$\dd{x}$

  \[
    \alpha(x)=\left\{
                \begin{array}{ll}
                  x\\
                  \frac{1}{1+e^{-kx}}\\
                  \frac{e^x-e^{-x}}{e^x+e^{-x}}
                \end{array}
              \right.
  \]

  $\expval{x}$
  
  $\chi_\rho(ghg\dmo)=\Tr(\rho_{ghg\dmo})=\Tr(\rho_g\circ\rho_h\circ\rho\dmo_g)=\Tr(\rho_h)\overset{\mbox{\scalebox{0.5}{$\Tr(AB)=\Tr(BA)$}}}{=}\chi_\rho(h)$
  	$\mathop{\oplus}_{\substack{x\in X}}$

$\mat(\rho_g)=(a_{ij}(g))_{\scriptsize \substack{1\leq i\leq d \\ 1\leq j\leq d}}$ et $\mat(\rho'_g)=(a'_{ij}(g))_{\scriptsize \substack{1\leq i'\leq d' \\ 1\leq j'\leq d'}}$



\[\int_a^b{\mathbb{R}^2}g(u, v)\dd{P_{XY}}(u, v)=\iint g(u,v) f_{XY}(u, v)\dd \lambda(u) \dd \lambda(v)\]
$$\lim_{x\to\infty} f(x)$$	
$$\iiiint_V \mu(t,u,v,w) \,dt\,du\,dv\,dw$$
$$\sum_{n=1}^{\infty} 2^{-n} = 1$$	
\begin{definition}
	Si $X$ et $Y$ sont 2 v.a. ou definit la \textsc{Covariance} entre $X$ et $Y$ comme
	$\cov(X,Y)\overset{\text{def}}{=}\E\left[(X-\E(X))(Y-\E(Y))\right]=\E(XY)-\E(X)\E(Y)$.
\end{definition}
\fi
\pagebreak

% \tableofcontents

% insert your code here
%\input{./algebra/main.tex}
%\input{./geometrie-differentielle/main.tex}
%\input{./probabilite/main.tex}
%\input{./analyse-fonctionnelle/main.tex}
% \input{./Analyse-convexe-et-dualite-en-optimisation/main.tex}
%\input{./tikz/main.tex}
%\input{./Theorie-du-distributions/main.tex}
%\input{./optimisation/mine.tex}
 \input{./modelisation/main.tex}

% yves.aubry@univ-tln.fr : algebra

\end{document}

%% !TEX encoding = UTF-8 Unicode
% !TEX TS-program = xelatex

\documentclass[french]{report}

%\usepackage[utf8]{inputenc}
%\usepackage[T1]{fontenc}
\usepackage{babel}


\newif\ifcomment
%\commenttrue # Show comments

\usepackage{physics}
\usepackage{amssymb}


\usepackage{amsthm}
% \usepackage{thmtools}
\usepackage{mathtools}
\usepackage{amsfonts}

\usepackage{color}

\usepackage{tikz}

\usepackage{geometry}
\geometry{a5paper, margin=0.1in, right=1cm}

\usepackage{dsfont}

\usepackage{graphicx}
\graphicspath{ {images/} }

\usepackage{faktor}

\usepackage{IEEEtrantools}
\usepackage{enumerate}   
\usepackage[PostScript=dvips]{"/Users/aware/Documents/Courses/diagrams"}


\newtheorem{theorem}{Théorème}[section]
\renewcommand{\thetheorem}{\arabic{theorem}}
\newtheorem{lemme}{Lemme}[section]
\renewcommand{\thelemme}{\arabic{lemme}}
\newtheorem{proposition}{Proposition}[section]
\renewcommand{\theproposition}{\arabic{proposition}}
\newtheorem{notations}{Notations}[section]
\newtheorem{problem}{Problème}[section]
\newtheorem{corollary}{Corollaire}[theorem]
\renewcommand{\thecorollary}{\arabic{corollary}}
\newtheorem{property}{Propriété}[section]
\newtheorem{objective}{Objectif}[section]

\theoremstyle{definition}
\newtheorem{definition}{Définition}[section]
\renewcommand{\thedefinition}{\arabic{definition}}
\newtheorem{exercise}{Exercice}[chapter]
\renewcommand{\theexercise}{\arabic{exercise}}
\newtheorem{example}{Exemple}[chapter]
\renewcommand{\theexample}{\arabic{example}}
\newtheorem*{solution}{Solution}
\newtheorem*{application}{Application}
\newtheorem*{notation}{Notation}
\newtheorem*{vocabulary}{Vocabulaire}
\newtheorem*{properties}{Propriétés}



\theoremstyle{remark}
\newtheorem*{remark}{Remarque}
\newtheorem*{rappel}{Rappel}


\usepackage{etoolbox}
\AtBeginEnvironment{exercise}{\small}
\AtBeginEnvironment{example}{\small}

\usepackage{cases}
\usepackage[red]{mypack}

\usepackage[framemethod=TikZ]{mdframed}

\definecolor{bg}{rgb}{0.4,0.25,0.95}
\definecolor{pagebg}{rgb}{0,0,0.5}
\surroundwithmdframed[
   topline=false,
   rightline=false,
   bottomline=false,
   leftmargin=\parindent,
   skipabove=8pt,
   skipbelow=8pt,
   linecolor=blue,
   innerbottommargin=10pt,
   % backgroundcolor=bg,font=\color{orange}\sffamily, fontcolor=white
]{definition}

\usepackage{empheq}
\usepackage[most]{tcolorbox}

\newtcbox{\mymath}[1][]{%
    nobeforeafter, math upper, tcbox raise base,
    enhanced, colframe=blue!30!black,
    colback=red!10, boxrule=1pt,
    #1}

\usepackage{unixode}


\DeclareMathOperator{\ord}{ord}
\DeclareMathOperator{\orb}{orb}
\DeclareMathOperator{\stab}{stab}
\DeclareMathOperator{\Stab}{stab}
\DeclareMathOperator{\ppcm}{ppcm}
\DeclareMathOperator{\conj}{Conj}
\DeclareMathOperator{\End}{End}
\DeclareMathOperator{\rot}{rot}
\DeclareMathOperator{\trs}{trace}
\DeclareMathOperator{\Ind}{Ind}
\DeclareMathOperator{\mat}{Mat}
\DeclareMathOperator{\id}{Id}
\DeclareMathOperator{\vect}{vect}
\DeclareMathOperator{\img}{img}
\DeclareMathOperator{\cov}{Cov}
\DeclareMathOperator{\dist}{dist}
\DeclareMathOperator{\irr}{Irr}
\DeclareMathOperator{\image}{Im}
\DeclareMathOperator{\pd}{\partial}
\DeclareMathOperator{\epi}{epi}
\DeclareMathOperator{\Argmin}{Argmin}
\DeclareMathOperator{\dom}{dom}
\DeclareMathOperator{\proj}{proj}
\DeclareMathOperator{\ctg}{ctg}
\DeclareMathOperator{\supp}{supp}
\DeclareMathOperator{\argmin}{argmin}
\DeclareMathOperator{\mult}{mult}
\DeclareMathOperator{\ch}{ch}
\DeclareMathOperator{\sh}{sh}
\DeclareMathOperator{\rang}{rang}
\DeclareMathOperator{\diam}{diam}
\DeclareMathOperator{\Epigraphe}{Epigraphe}




\usepackage{xcolor}
\everymath{\color{blue}}
%\everymath{\color[rgb]{0,1,1}}
%\pagecolor[rgb]{0,0,0.5}


\newcommand*{\pdtest}[3][]{\ensuremath{\frac{\partial^{#1} #2}{\partial #3}}}

\newcommand*{\deffunc}[6][]{\ensuremath{
\begin{array}{rcl}
#2 : #3 &\rightarrow& #4\\
#5 &\mapsto& #6
\end{array}
}}

\newcommand{\eqcolon}{\mathrel{\resizebox{\widthof{$\mathord{=}$}}{\height}{ $\!\!=\!\!\resizebox{1.2\width}{0.8\height}{\raisebox{0.23ex}{$\mathop{:}$}}\!\!$ }}}
\newcommand{\coloneq}{\mathrel{\resizebox{\widthof{$\mathord{=}$}}{\height}{ $\!\!\resizebox{1.2\width}{0.8\height}{\raisebox{0.23ex}{$\mathop{:}$}}\!\!=\!\!$ }}}
\newcommand{\eqcolonl}{\ensuremath{\mathrel{=\!\!\mathop{:}}}}
\newcommand{\coloneql}{\ensuremath{\mathrel{\mathop{:} \!\! =}}}
\newcommand{\vc}[1]{% inline column vector
  \left(\begin{smallmatrix}#1\end{smallmatrix}\right)%
}
\newcommand{\vr}[1]{% inline row vector
  \begin{smallmatrix}(\,#1\,)\end{smallmatrix}%
}
\makeatletter
\newcommand*{\defeq}{\ =\mathrel{\rlap{%
                     \raisebox{0.3ex}{$\m@th\cdot$}}%
                     \raisebox{-0.3ex}{$\m@th\cdot$}}%
                     }
\makeatother

\newcommand{\mathcircle}[1]{% inline row vector
 \overset{\circ}{#1}
}
\newcommand{\ulim}{% low limit
 \underline{\lim}
}
\newcommand{\ssi}{% iff
\iff
}
\newcommand{\ps}[2]{
\expval{#1 | #2}
}
\newcommand{\df}[1]{
\mqty{#1}
}
\newcommand{\n}[1]{
\norm{#1}
}
\newcommand{\sys}[1]{
\left\{\smqty{#1}\right.
}


\newcommand{\eqdef}{\ensuremath{\overset{\text{def}}=}}


\def\Circlearrowright{\ensuremath{%
  \rotatebox[origin=c]{230}{$\circlearrowright$}}}

\newcommand\ct[1]{\text{\rmfamily\upshape #1}}
\newcommand\question[1]{ {\color{red} ...!? \small #1}}
\newcommand\caz[1]{\left\{\begin{array} #1 \end{array}\right.}
\newcommand\const{\text{\rmfamily\upshape const}}
\newcommand\toP{ \overset{\pro}{\to}}
\newcommand\toPP{ \overset{\text{PP}}{\to}}
\newcommand{\oeq}{\mathrel{\text{\textcircled{$=$}}}}





\usepackage{xcolor}
% \usepackage[normalem]{ulem}
\usepackage{lipsum}
\makeatletter
% \newcommand\colorwave[1][blue]{\bgroup \markoverwith{\lower3.5\p@\hbox{\sixly \textcolor{#1}{\char58}}}\ULon}
%\font\sixly=lasy6 % does not re-load if already loaded, so no memory problem.

\newmdtheoremenv[
linewidth= 1pt,linecolor= blue,%
leftmargin=20,rightmargin=20,innertopmargin=0pt, innerrightmargin=40,%
tikzsetting = { draw=lightgray, line width = 0.3pt,dashed,%
dash pattern = on 15pt off 3pt},%
splittopskip=\topskip,skipbelow=\baselineskip,%
skipabove=\baselineskip,ntheorem,roundcorner=0pt,
% backgroundcolor=pagebg,font=\color{orange}\sffamily, fontcolor=white
]{examplebox}{Exemple}[section]



\newcommand\R{\mathbb{R}}
\newcommand\Z{\mathbb{Z}}
\newcommand\N{\mathbb{N}}
\newcommand\E{\mathbb{E}}
\newcommand\F{\mathcal{F}}
\newcommand\cH{\mathcal{H}}
\newcommand\V{\mathbb{V}}
\newcommand\dmo{ ^{-1} }
\newcommand\kapa{\kappa}
\newcommand\im{Im}
\newcommand\hs{\mathcal{H}}





\usepackage{soul}

\makeatletter
\newcommand*{\whiten}[1]{\llap{\textcolor{white}{{\the\SOUL@token}}\hspace{#1pt}}}
\DeclareRobustCommand*\myul{%
    \def\SOUL@everyspace{\underline{\space}\kern\z@}%
    \def\SOUL@everytoken{%
     \setbox0=\hbox{\the\SOUL@token}%
     \ifdim\dp0>\z@
        \raisebox{\dp0}{\underline{\phantom{\the\SOUL@token}}}%
        \whiten{1}\whiten{0}%
        \whiten{-1}\whiten{-2}%
        \llap{\the\SOUL@token}%
     \else
        \underline{\the\SOUL@token}%
     \fi}%
\SOUL@}
\makeatother

\newcommand*{\demp}{\fontfamily{lmtt}\selectfont}

\DeclareTextFontCommand{\textdemp}{\demp}

\begin{document}

\ifcomment
Multiline
comment
\fi
\ifcomment
\myul{Typesetting test}
% \color[rgb]{1,1,1}
$∑_i^n≠ 60º±∞π∆¬≈√j∫h≤≥µ$

$\CR \R\pro\ind\pro\gS\pro
\mqty[a&b\\c&d]$
$\pro\mathbb{P}$
$\dd{x}$

  \[
    \alpha(x)=\left\{
                \begin{array}{ll}
                  x\\
                  \frac{1}{1+e^{-kx}}\\
                  \frac{e^x-e^{-x}}{e^x+e^{-x}}
                \end{array}
              \right.
  \]

  $\expval{x}$
  
  $\chi_\rho(ghg\dmo)=\Tr(\rho_{ghg\dmo})=\Tr(\rho_g\circ\rho_h\circ\rho\dmo_g)=\Tr(\rho_h)\overset{\mbox{\scalebox{0.5}{$\Tr(AB)=\Tr(BA)$}}}{=}\chi_\rho(h)$
  	$\mathop{\oplus}_{\substack{x\in X}}$

$\mat(\rho_g)=(a_{ij}(g))_{\scriptsize \substack{1\leq i\leq d \\ 1\leq j\leq d}}$ et $\mat(\rho'_g)=(a'_{ij}(g))_{\scriptsize \substack{1\leq i'\leq d' \\ 1\leq j'\leq d'}}$



\[\int_a^b{\mathbb{R}^2}g(u, v)\dd{P_{XY}}(u, v)=\iint g(u,v) f_{XY}(u, v)\dd \lambda(u) \dd \lambda(v)\]
$$\lim_{x\to\infty} f(x)$$	
$$\iiiint_V \mu(t,u,v,w) \,dt\,du\,dv\,dw$$
$$\sum_{n=1}^{\infty} 2^{-n} = 1$$	
\begin{definition}
	Si $X$ et $Y$ sont 2 v.a. ou definit la \textsc{Covariance} entre $X$ et $Y$ comme
	$\cov(X,Y)\overset{\text{def}}{=}\E\left[(X-\E(X))(Y-\E(Y))\right]=\E(XY)-\E(X)\E(Y)$.
\end{definition}
\fi
\pagebreak

% \tableofcontents

% insert your code here
%\input{./algebra/main.tex}
%\input{./geometrie-differentielle/main.tex}
%\input{./probabilite/main.tex}
%\input{./analyse-fonctionnelle/main.tex}
% \input{./Analyse-convexe-et-dualite-en-optimisation/main.tex}
%\input{./tikz/main.tex}
%\input{./Theorie-du-distributions/main.tex}
%\input{./optimisation/mine.tex}
 \input{./modelisation/main.tex}

% yves.aubry@univ-tln.fr : algebra

\end{document}

%\input{./optimisation/mine.tex}
 % !TEX encoding = UTF-8 Unicode
% !TEX TS-program = xelatex

\documentclass[french]{report}

%\usepackage[utf8]{inputenc}
%\usepackage[T1]{fontenc}
\usepackage{babel}


\newif\ifcomment
%\commenttrue # Show comments

\usepackage{physics}
\usepackage{amssymb}


\usepackage{amsthm}
% \usepackage{thmtools}
\usepackage{mathtools}
\usepackage{amsfonts}

\usepackage{color}

\usepackage{tikz}

\usepackage{geometry}
\geometry{a5paper, margin=0.1in, right=1cm}

\usepackage{dsfont}

\usepackage{graphicx}
\graphicspath{ {images/} }

\usepackage{faktor}

\usepackage{IEEEtrantools}
\usepackage{enumerate}   
\usepackage[PostScript=dvips]{"/Users/aware/Documents/Courses/diagrams"}


\newtheorem{theorem}{Théorème}[section]
\renewcommand{\thetheorem}{\arabic{theorem}}
\newtheorem{lemme}{Lemme}[section]
\renewcommand{\thelemme}{\arabic{lemme}}
\newtheorem{proposition}{Proposition}[section]
\renewcommand{\theproposition}{\arabic{proposition}}
\newtheorem{notations}{Notations}[section]
\newtheorem{problem}{Problème}[section]
\newtheorem{corollary}{Corollaire}[theorem]
\renewcommand{\thecorollary}{\arabic{corollary}}
\newtheorem{property}{Propriété}[section]
\newtheorem{objective}{Objectif}[section]

\theoremstyle{definition}
\newtheorem{definition}{Définition}[section]
\renewcommand{\thedefinition}{\arabic{definition}}
\newtheorem{exercise}{Exercice}[chapter]
\renewcommand{\theexercise}{\arabic{exercise}}
\newtheorem{example}{Exemple}[chapter]
\renewcommand{\theexample}{\arabic{example}}
\newtheorem*{solution}{Solution}
\newtheorem*{application}{Application}
\newtheorem*{notation}{Notation}
\newtheorem*{vocabulary}{Vocabulaire}
\newtheorem*{properties}{Propriétés}



\theoremstyle{remark}
\newtheorem*{remark}{Remarque}
\newtheorem*{rappel}{Rappel}


\usepackage{etoolbox}
\AtBeginEnvironment{exercise}{\small}
\AtBeginEnvironment{example}{\small}

\usepackage{cases}
\usepackage[red]{mypack}

\usepackage[framemethod=TikZ]{mdframed}

\definecolor{bg}{rgb}{0.4,0.25,0.95}
\definecolor{pagebg}{rgb}{0,0,0.5}
\surroundwithmdframed[
   topline=false,
   rightline=false,
   bottomline=false,
   leftmargin=\parindent,
   skipabove=8pt,
   skipbelow=8pt,
   linecolor=blue,
   innerbottommargin=10pt,
   % backgroundcolor=bg,font=\color{orange}\sffamily, fontcolor=white
]{definition}

\usepackage{empheq}
\usepackage[most]{tcolorbox}

\newtcbox{\mymath}[1][]{%
    nobeforeafter, math upper, tcbox raise base,
    enhanced, colframe=blue!30!black,
    colback=red!10, boxrule=1pt,
    #1}

\usepackage{unixode}


\DeclareMathOperator{\ord}{ord}
\DeclareMathOperator{\orb}{orb}
\DeclareMathOperator{\stab}{stab}
\DeclareMathOperator{\Stab}{stab}
\DeclareMathOperator{\ppcm}{ppcm}
\DeclareMathOperator{\conj}{Conj}
\DeclareMathOperator{\End}{End}
\DeclareMathOperator{\rot}{rot}
\DeclareMathOperator{\trs}{trace}
\DeclareMathOperator{\Ind}{Ind}
\DeclareMathOperator{\mat}{Mat}
\DeclareMathOperator{\id}{Id}
\DeclareMathOperator{\vect}{vect}
\DeclareMathOperator{\img}{img}
\DeclareMathOperator{\cov}{Cov}
\DeclareMathOperator{\dist}{dist}
\DeclareMathOperator{\irr}{Irr}
\DeclareMathOperator{\image}{Im}
\DeclareMathOperator{\pd}{\partial}
\DeclareMathOperator{\epi}{epi}
\DeclareMathOperator{\Argmin}{Argmin}
\DeclareMathOperator{\dom}{dom}
\DeclareMathOperator{\proj}{proj}
\DeclareMathOperator{\ctg}{ctg}
\DeclareMathOperator{\supp}{supp}
\DeclareMathOperator{\argmin}{argmin}
\DeclareMathOperator{\mult}{mult}
\DeclareMathOperator{\ch}{ch}
\DeclareMathOperator{\sh}{sh}
\DeclareMathOperator{\rang}{rang}
\DeclareMathOperator{\diam}{diam}
\DeclareMathOperator{\Epigraphe}{Epigraphe}




\usepackage{xcolor}
\everymath{\color{blue}}
%\everymath{\color[rgb]{0,1,1}}
%\pagecolor[rgb]{0,0,0.5}


\newcommand*{\pdtest}[3][]{\ensuremath{\frac{\partial^{#1} #2}{\partial #3}}}

\newcommand*{\deffunc}[6][]{\ensuremath{
\begin{array}{rcl}
#2 : #3 &\rightarrow& #4\\
#5 &\mapsto& #6
\end{array}
}}

\newcommand{\eqcolon}{\mathrel{\resizebox{\widthof{$\mathord{=}$}}{\height}{ $\!\!=\!\!\resizebox{1.2\width}{0.8\height}{\raisebox{0.23ex}{$\mathop{:}$}}\!\!$ }}}
\newcommand{\coloneq}{\mathrel{\resizebox{\widthof{$\mathord{=}$}}{\height}{ $\!\!\resizebox{1.2\width}{0.8\height}{\raisebox{0.23ex}{$\mathop{:}$}}\!\!=\!\!$ }}}
\newcommand{\eqcolonl}{\ensuremath{\mathrel{=\!\!\mathop{:}}}}
\newcommand{\coloneql}{\ensuremath{\mathrel{\mathop{:} \!\! =}}}
\newcommand{\vc}[1]{% inline column vector
  \left(\begin{smallmatrix}#1\end{smallmatrix}\right)%
}
\newcommand{\vr}[1]{% inline row vector
  \begin{smallmatrix}(\,#1\,)\end{smallmatrix}%
}
\makeatletter
\newcommand*{\defeq}{\ =\mathrel{\rlap{%
                     \raisebox{0.3ex}{$\m@th\cdot$}}%
                     \raisebox{-0.3ex}{$\m@th\cdot$}}%
                     }
\makeatother

\newcommand{\mathcircle}[1]{% inline row vector
 \overset{\circ}{#1}
}
\newcommand{\ulim}{% low limit
 \underline{\lim}
}
\newcommand{\ssi}{% iff
\iff
}
\newcommand{\ps}[2]{
\expval{#1 | #2}
}
\newcommand{\df}[1]{
\mqty{#1}
}
\newcommand{\n}[1]{
\norm{#1}
}
\newcommand{\sys}[1]{
\left\{\smqty{#1}\right.
}


\newcommand{\eqdef}{\ensuremath{\overset{\text{def}}=}}


\def\Circlearrowright{\ensuremath{%
  \rotatebox[origin=c]{230}{$\circlearrowright$}}}

\newcommand\ct[1]{\text{\rmfamily\upshape #1}}
\newcommand\question[1]{ {\color{red} ...!? \small #1}}
\newcommand\caz[1]{\left\{\begin{array} #1 \end{array}\right.}
\newcommand\const{\text{\rmfamily\upshape const}}
\newcommand\toP{ \overset{\pro}{\to}}
\newcommand\toPP{ \overset{\text{PP}}{\to}}
\newcommand{\oeq}{\mathrel{\text{\textcircled{$=$}}}}





\usepackage{xcolor}
% \usepackage[normalem]{ulem}
\usepackage{lipsum}
\makeatletter
% \newcommand\colorwave[1][blue]{\bgroup \markoverwith{\lower3.5\p@\hbox{\sixly \textcolor{#1}{\char58}}}\ULon}
%\font\sixly=lasy6 % does not re-load if already loaded, so no memory problem.

\newmdtheoremenv[
linewidth= 1pt,linecolor= blue,%
leftmargin=20,rightmargin=20,innertopmargin=0pt, innerrightmargin=40,%
tikzsetting = { draw=lightgray, line width = 0.3pt,dashed,%
dash pattern = on 15pt off 3pt},%
splittopskip=\topskip,skipbelow=\baselineskip,%
skipabove=\baselineskip,ntheorem,roundcorner=0pt,
% backgroundcolor=pagebg,font=\color{orange}\sffamily, fontcolor=white
]{examplebox}{Exemple}[section]



\newcommand\R{\mathbb{R}}
\newcommand\Z{\mathbb{Z}}
\newcommand\N{\mathbb{N}}
\newcommand\E{\mathbb{E}}
\newcommand\F{\mathcal{F}}
\newcommand\cH{\mathcal{H}}
\newcommand\V{\mathbb{V}}
\newcommand\dmo{ ^{-1} }
\newcommand\kapa{\kappa}
\newcommand\im{Im}
\newcommand\hs{\mathcal{H}}





\usepackage{soul}

\makeatletter
\newcommand*{\whiten}[1]{\llap{\textcolor{white}{{\the\SOUL@token}}\hspace{#1pt}}}
\DeclareRobustCommand*\myul{%
    \def\SOUL@everyspace{\underline{\space}\kern\z@}%
    \def\SOUL@everytoken{%
     \setbox0=\hbox{\the\SOUL@token}%
     \ifdim\dp0>\z@
        \raisebox{\dp0}{\underline{\phantom{\the\SOUL@token}}}%
        \whiten{1}\whiten{0}%
        \whiten{-1}\whiten{-2}%
        \llap{\the\SOUL@token}%
     \else
        \underline{\the\SOUL@token}%
     \fi}%
\SOUL@}
\makeatother

\newcommand*{\demp}{\fontfamily{lmtt}\selectfont}

\DeclareTextFontCommand{\textdemp}{\demp}

\begin{document}

\ifcomment
Multiline
comment
\fi
\ifcomment
\myul{Typesetting test}
% \color[rgb]{1,1,1}
$∑_i^n≠ 60º±∞π∆¬≈√j∫h≤≥µ$

$\CR \R\pro\ind\pro\gS\pro
\mqty[a&b\\c&d]$
$\pro\mathbb{P}$
$\dd{x}$

  \[
    \alpha(x)=\left\{
                \begin{array}{ll}
                  x\\
                  \frac{1}{1+e^{-kx}}\\
                  \frac{e^x-e^{-x}}{e^x+e^{-x}}
                \end{array}
              \right.
  \]

  $\expval{x}$
  
  $\chi_\rho(ghg\dmo)=\Tr(\rho_{ghg\dmo})=\Tr(\rho_g\circ\rho_h\circ\rho\dmo_g)=\Tr(\rho_h)\overset{\mbox{\scalebox{0.5}{$\Tr(AB)=\Tr(BA)$}}}{=}\chi_\rho(h)$
  	$\mathop{\oplus}_{\substack{x\in X}}$

$\mat(\rho_g)=(a_{ij}(g))_{\scriptsize \substack{1\leq i\leq d \\ 1\leq j\leq d}}$ et $\mat(\rho'_g)=(a'_{ij}(g))_{\scriptsize \substack{1\leq i'\leq d' \\ 1\leq j'\leq d'}}$



\[\int_a^b{\mathbb{R}^2}g(u, v)\dd{P_{XY}}(u, v)=\iint g(u,v) f_{XY}(u, v)\dd \lambda(u) \dd \lambda(v)\]
$$\lim_{x\to\infty} f(x)$$	
$$\iiiint_V \mu(t,u,v,w) \,dt\,du\,dv\,dw$$
$$\sum_{n=1}^{\infty} 2^{-n} = 1$$	
\begin{definition}
	Si $X$ et $Y$ sont 2 v.a. ou definit la \textsc{Covariance} entre $X$ et $Y$ comme
	$\cov(X,Y)\overset{\text{def}}{=}\E\left[(X-\E(X))(Y-\E(Y))\right]=\E(XY)-\E(X)\E(Y)$.
\end{definition}
\fi
\pagebreak

% \tableofcontents

% insert your code here
%\input{./algebra/main.tex}
%\input{./geometrie-differentielle/main.tex}
%\input{./probabilite/main.tex}
%\input{./analyse-fonctionnelle/main.tex}
% \input{./Analyse-convexe-et-dualite-en-optimisation/main.tex}
%\input{./tikz/main.tex}
%\input{./Theorie-du-distributions/main.tex}
%\input{./optimisation/mine.tex}
 \input{./modelisation/main.tex}

% yves.aubry@univ-tln.fr : algebra

\end{document}


% yves.aubry@univ-tln.fr : algebra

\end{document}


% yves.aubry@univ-tln.fr : algebra

\end{document}

%% !TEX encoding = UTF-8 Unicode
% !TEX TS-program = xelatex

\documentclass[french]{report}

%\usepackage[utf8]{inputenc}
%\usepackage[T1]{fontenc}
\usepackage{babel}


\newif\ifcomment
%\commenttrue # Show comments

\usepackage{physics}
\usepackage{amssymb}


\usepackage{amsthm}
% \usepackage{thmtools}
\usepackage{mathtools}
\usepackage{amsfonts}

\usepackage{color}

\usepackage{tikz}

\usepackage{geometry}
\geometry{a5paper, margin=0.1in, right=1cm}

\usepackage{dsfont}

\usepackage{graphicx}
\graphicspath{ {images/} }

\usepackage{faktor}

\usepackage{IEEEtrantools}
\usepackage{enumerate}   
\usepackage[PostScript=dvips]{"/Users/aware/Documents/Courses/diagrams"}


\newtheorem{theorem}{Théorème}[section]
\renewcommand{\thetheorem}{\arabic{theorem}}
\newtheorem{lemme}{Lemme}[section]
\renewcommand{\thelemme}{\arabic{lemme}}
\newtheorem{proposition}{Proposition}[section]
\renewcommand{\theproposition}{\arabic{proposition}}
\newtheorem{notations}{Notations}[section]
\newtheorem{problem}{Problème}[section]
\newtheorem{corollary}{Corollaire}[theorem]
\renewcommand{\thecorollary}{\arabic{corollary}}
\newtheorem{property}{Propriété}[section]
\newtheorem{objective}{Objectif}[section]

\theoremstyle{definition}
\newtheorem{definition}{Définition}[section]
\renewcommand{\thedefinition}{\arabic{definition}}
\newtheorem{exercise}{Exercice}[chapter]
\renewcommand{\theexercise}{\arabic{exercise}}
\newtheorem{example}{Exemple}[chapter]
\renewcommand{\theexample}{\arabic{example}}
\newtheorem*{solution}{Solution}
\newtheorem*{application}{Application}
\newtheorem*{notation}{Notation}
\newtheorem*{vocabulary}{Vocabulaire}
\newtheorem*{properties}{Propriétés}



\theoremstyle{remark}
\newtheorem*{remark}{Remarque}
\newtheorem*{rappel}{Rappel}


\usepackage{etoolbox}
\AtBeginEnvironment{exercise}{\small}
\AtBeginEnvironment{example}{\small}

\usepackage{cases}
\usepackage[red]{mypack}

\usepackage[framemethod=TikZ]{mdframed}

\definecolor{bg}{rgb}{0.4,0.25,0.95}
\definecolor{pagebg}{rgb}{0,0,0.5}
\surroundwithmdframed[
   topline=false,
   rightline=false,
   bottomline=false,
   leftmargin=\parindent,
   skipabove=8pt,
   skipbelow=8pt,
   linecolor=blue,
   innerbottommargin=10pt,
   % backgroundcolor=bg,font=\color{orange}\sffamily, fontcolor=white
]{definition}

\usepackage{empheq}
\usepackage[most]{tcolorbox}

\newtcbox{\mymath}[1][]{%
    nobeforeafter, math upper, tcbox raise base,
    enhanced, colframe=blue!30!black,
    colback=red!10, boxrule=1pt,
    #1}

\usepackage{unixode}


\DeclareMathOperator{\ord}{ord}
\DeclareMathOperator{\orb}{orb}
\DeclareMathOperator{\stab}{stab}
\DeclareMathOperator{\Stab}{stab}
\DeclareMathOperator{\ppcm}{ppcm}
\DeclareMathOperator{\conj}{Conj}
\DeclareMathOperator{\End}{End}
\DeclareMathOperator{\rot}{rot}
\DeclareMathOperator{\trs}{trace}
\DeclareMathOperator{\Ind}{Ind}
\DeclareMathOperator{\mat}{Mat}
\DeclareMathOperator{\id}{Id}
\DeclareMathOperator{\vect}{vect}
\DeclareMathOperator{\img}{img}
\DeclareMathOperator{\cov}{Cov}
\DeclareMathOperator{\dist}{dist}
\DeclareMathOperator{\irr}{Irr}
\DeclareMathOperator{\image}{Im}
\DeclareMathOperator{\pd}{\partial}
\DeclareMathOperator{\epi}{epi}
\DeclareMathOperator{\Argmin}{Argmin}
\DeclareMathOperator{\dom}{dom}
\DeclareMathOperator{\proj}{proj}
\DeclareMathOperator{\ctg}{ctg}
\DeclareMathOperator{\supp}{supp}
\DeclareMathOperator{\argmin}{argmin}
\DeclareMathOperator{\mult}{mult}
\DeclareMathOperator{\ch}{ch}
\DeclareMathOperator{\sh}{sh}
\DeclareMathOperator{\rang}{rang}
\DeclareMathOperator{\diam}{diam}
\DeclareMathOperator{\Epigraphe}{Epigraphe}




\usepackage{xcolor}
\everymath{\color{blue}}
%\everymath{\color[rgb]{0,1,1}}
%\pagecolor[rgb]{0,0,0.5}


\newcommand*{\pdtest}[3][]{\ensuremath{\frac{\partial^{#1} #2}{\partial #3}}}

\newcommand*{\deffunc}[6][]{\ensuremath{
\begin{array}{rcl}
#2 : #3 &\rightarrow& #4\\
#5 &\mapsto& #6
\end{array}
}}

\newcommand{\eqcolon}{\mathrel{\resizebox{\widthof{$\mathord{=}$}}{\height}{ $\!\!=\!\!\resizebox{1.2\width}{0.8\height}{\raisebox{0.23ex}{$\mathop{:}$}}\!\!$ }}}
\newcommand{\coloneq}{\mathrel{\resizebox{\widthof{$\mathord{=}$}}{\height}{ $\!\!\resizebox{1.2\width}{0.8\height}{\raisebox{0.23ex}{$\mathop{:}$}}\!\!=\!\!$ }}}
\newcommand{\eqcolonl}{\ensuremath{\mathrel{=\!\!\mathop{:}}}}
\newcommand{\coloneql}{\ensuremath{\mathrel{\mathop{:} \!\! =}}}
\newcommand{\vc}[1]{% inline column vector
  \left(\begin{smallmatrix}#1\end{smallmatrix}\right)%
}
\newcommand{\vr}[1]{% inline row vector
  \begin{smallmatrix}(\,#1\,)\end{smallmatrix}%
}
\makeatletter
\newcommand*{\defeq}{\ =\mathrel{\rlap{%
                     \raisebox{0.3ex}{$\m@th\cdot$}}%
                     \raisebox{-0.3ex}{$\m@th\cdot$}}%
                     }
\makeatother

\newcommand{\mathcircle}[1]{% inline row vector
 \overset{\circ}{#1}
}
\newcommand{\ulim}{% low limit
 \underline{\lim}
}
\newcommand{\ssi}{% iff
\iff
}
\newcommand{\ps}[2]{
\expval{#1 | #2}
}
\newcommand{\df}[1]{
\mqty{#1}
}
\newcommand{\n}[1]{
\norm{#1}
}
\newcommand{\sys}[1]{
\left\{\smqty{#1}\right.
}


\newcommand{\eqdef}{\ensuremath{\overset{\text{def}}=}}


\def\Circlearrowright{\ensuremath{%
  \rotatebox[origin=c]{230}{$\circlearrowright$}}}

\newcommand\ct[1]{\text{\rmfamily\upshape #1}}
\newcommand\question[1]{ {\color{red} ...!? \small #1}}
\newcommand\caz[1]{\left\{\begin{array} #1 \end{array}\right.}
\newcommand\const{\text{\rmfamily\upshape const}}
\newcommand\toP{ \overset{\pro}{\to}}
\newcommand\toPP{ \overset{\text{PP}}{\to}}
\newcommand{\oeq}{\mathrel{\text{\textcircled{$=$}}}}





\usepackage{xcolor}
% \usepackage[normalem]{ulem}
\usepackage{lipsum}
\makeatletter
% \newcommand\colorwave[1][blue]{\bgroup \markoverwith{\lower3.5\p@\hbox{\sixly \textcolor{#1}{\char58}}}\ULon}
%\font\sixly=lasy6 % does not re-load if already loaded, so no memory problem.

\newmdtheoremenv[
linewidth= 1pt,linecolor= blue,%
leftmargin=20,rightmargin=20,innertopmargin=0pt, innerrightmargin=40,%
tikzsetting = { draw=lightgray, line width = 0.3pt,dashed,%
dash pattern = on 15pt off 3pt},%
splittopskip=\topskip,skipbelow=\baselineskip,%
skipabove=\baselineskip,ntheorem,roundcorner=0pt,
% backgroundcolor=pagebg,font=\color{orange}\sffamily, fontcolor=white
]{examplebox}{Exemple}[section]



\newcommand\R{\mathbb{R}}
\newcommand\Z{\mathbb{Z}}
\newcommand\N{\mathbb{N}}
\newcommand\E{\mathbb{E}}
\newcommand\F{\mathcal{F}}
\newcommand\cH{\mathcal{H}}
\newcommand\V{\mathbb{V}}
\newcommand\dmo{ ^{-1} }
\newcommand\kapa{\kappa}
\newcommand\im{Im}
\newcommand\hs{\mathcal{H}}





\usepackage{soul}

\makeatletter
\newcommand*{\whiten}[1]{\llap{\textcolor{white}{{\the\SOUL@token}}\hspace{#1pt}}}
\DeclareRobustCommand*\myul{%
    \def\SOUL@everyspace{\underline{\space}\kern\z@}%
    \def\SOUL@everytoken{%
     \setbox0=\hbox{\the\SOUL@token}%
     \ifdim\dp0>\z@
        \raisebox{\dp0}{\underline{\phantom{\the\SOUL@token}}}%
        \whiten{1}\whiten{0}%
        \whiten{-1}\whiten{-2}%
        \llap{\the\SOUL@token}%
     \else
        \underline{\the\SOUL@token}%
     \fi}%
\SOUL@}
\makeatother

\newcommand*{\demp}{\fontfamily{lmtt}\selectfont}

\DeclareTextFontCommand{\textdemp}{\demp}

\begin{document}

\ifcomment
Multiline
comment
\fi
\ifcomment
\myul{Typesetting test}
% \color[rgb]{1,1,1}
$∑_i^n≠ 60º±∞π∆¬≈√j∫h≤≥µ$

$\CR \R\pro\ind\pro\gS\pro
\mqty[a&b\\c&d]$
$\pro\mathbb{P}$
$\dd{x}$

  \[
    \alpha(x)=\left\{
                \begin{array}{ll}
                  x\\
                  \frac{1}{1+e^{-kx}}\\
                  \frac{e^x-e^{-x}}{e^x+e^{-x}}
                \end{array}
              \right.
  \]

  $\expval{x}$
  
  $\chi_\rho(ghg\dmo)=\Tr(\rho_{ghg\dmo})=\Tr(\rho_g\circ\rho_h\circ\rho\dmo_g)=\Tr(\rho_h)\overset{\mbox{\scalebox{0.5}{$\Tr(AB)=\Tr(BA)$}}}{=}\chi_\rho(h)$
  	$\mathop{\oplus}_{\substack{x\in X}}$

$\mat(\rho_g)=(a_{ij}(g))_{\scriptsize \substack{1\leq i\leq d \\ 1\leq j\leq d}}$ et $\mat(\rho'_g)=(a'_{ij}(g))_{\scriptsize \substack{1\leq i'\leq d' \\ 1\leq j'\leq d'}}$



\[\int_a^b{\mathbb{R}^2}g(u, v)\dd{P_{XY}}(u, v)=\iint g(u,v) f_{XY}(u, v)\dd \lambda(u) \dd \lambda(v)\]
$$\lim_{x\to\infty} f(x)$$	
$$\iiiint_V \mu(t,u,v,w) \,dt\,du\,dv\,dw$$
$$\sum_{n=1}^{\infty} 2^{-n} = 1$$	
\begin{definition}
	Si $X$ et $Y$ sont 2 v.a. ou definit la \textsc{Covariance} entre $X$ et $Y$ comme
	$\cov(X,Y)\overset{\text{def}}{=}\E\left[(X-\E(X))(Y-\E(Y))\right]=\E(XY)-\E(X)\E(Y)$.
\end{definition}
\fi
\pagebreak

% \tableofcontents

% insert your code here
%% !TEX encoding = UTF-8 Unicode
% !TEX TS-program = xelatex

\documentclass[french]{report}

%\usepackage[utf8]{inputenc}
%\usepackage[T1]{fontenc}
\usepackage{babel}


\newif\ifcomment
%\commenttrue # Show comments

\usepackage{physics}
\usepackage{amssymb}


\usepackage{amsthm}
% \usepackage{thmtools}
\usepackage{mathtools}
\usepackage{amsfonts}

\usepackage{color}

\usepackage{tikz}

\usepackage{geometry}
\geometry{a5paper, margin=0.1in, right=1cm}

\usepackage{dsfont}

\usepackage{graphicx}
\graphicspath{ {images/} }

\usepackage{faktor}

\usepackage{IEEEtrantools}
\usepackage{enumerate}   
\usepackage[PostScript=dvips]{"/Users/aware/Documents/Courses/diagrams"}


\newtheorem{theorem}{Théorème}[section]
\renewcommand{\thetheorem}{\arabic{theorem}}
\newtheorem{lemme}{Lemme}[section]
\renewcommand{\thelemme}{\arabic{lemme}}
\newtheorem{proposition}{Proposition}[section]
\renewcommand{\theproposition}{\arabic{proposition}}
\newtheorem{notations}{Notations}[section]
\newtheorem{problem}{Problème}[section]
\newtheorem{corollary}{Corollaire}[theorem]
\renewcommand{\thecorollary}{\arabic{corollary}}
\newtheorem{property}{Propriété}[section]
\newtheorem{objective}{Objectif}[section]

\theoremstyle{definition}
\newtheorem{definition}{Définition}[section]
\renewcommand{\thedefinition}{\arabic{definition}}
\newtheorem{exercise}{Exercice}[chapter]
\renewcommand{\theexercise}{\arabic{exercise}}
\newtheorem{example}{Exemple}[chapter]
\renewcommand{\theexample}{\arabic{example}}
\newtheorem*{solution}{Solution}
\newtheorem*{application}{Application}
\newtheorem*{notation}{Notation}
\newtheorem*{vocabulary}{Vocabulaire}
\newtheorem*{properties}{Propriétés}



\theoremstyle{remark}
\newtheorem*{remark}{Remarque}
\newtheorem*{rappel}{Rappel}


\usepackage{etoolbox}
\AtBeginEnvironment{exercise}{\small}
\AtBeginEnvironment{example}{\small}

\usepackage{cases}
\usepackage[red]{mypack}

\usepackage[framemethod=TikZ]{mdframed}

\definecolor{bg}{rgb}{0.4,0.25,0.95}
\definecolor{pagebg}{rgb}{0,0,0.5}
\surroundwithmdframed[
   topline=false,
   rightline=false,
   bottomline=false,
   leftmargin=\parindent,
   skipabove=8pt,
   skipbelow=8pt,
   linecolor=blue,
   innerbottommargin=10pt,
   % backgroundcolor=bg,font=\color{orange}\sffamily, fontcolor=white
]{definition}

\usepackage{empheq}
\usepackage[most]{tcolorbox}

\newtcbox{\mymath}[1][]{%
    nobeforeafter, math upper, tcbox raise base,
    enhanced, colframe=blue!30!black,
    colback=red!10, boxrule=1pt,
    #1}

\usepackage{unixode}


\DeclareMathOperator{\ord}{ord}
\DeclareMathOperator{\orb}{orb}
\DeclareMathOperator{\stab}{stab}
\DeclareMathOperator{\Stab}{stab}
\DeclareMathOperator{\ppcm}{ppcm}
\DeclareMathOperator{\conj}{Conj}
\DeclareMathOperator{\End}{End}
\DeclareMathOperator{\rot}{rot}
\DeclareMathOperator{\trs}{trace}
\DeclareMathOperator{\Ind}{Ind}
\DeclareMathOperator{\mat}{Mat}
\DeclareMathOperator{\id}{Id}
\DeclareMathOperator{\vect}{vect}
\DeclareMathOperator{\img}{img}
\DeclareMathOperator{\cov}{Cov}
\DeclareMathOperator{\dist}{dist}
\DeclareMathOperator{\irr}{Irr}
\DeclareMathOperator{\image}{Im}
\DeclareMathOperator{\pd}{\partial}
\DeclareMathOperator{\epi}{epi}
\DeclareMathOperator{\Argmin}{Argmin}
\DeclareMathOperator{\dom}{dom}
\DeclareMathOperator{\proj}{proj}
\DeclareMathOperator{\ctg}{ctg}
\DeclareMathOperator{\supp}{supp}
\DeclareMathOperator{\argmin}{argmin}
\DeclareMathOperator{\mult}{mult}
\DeclareMathOperator{\ch}{ch}
\DeclareMathOperator{\sh}{sh}
\DeclareMathOperator{\rang}{rang}
\DeclareMathOperator{\diam}{diam}
\DeclareMathOperator{\Epigraphe}{Epigraphe}




\usepackage{xcolor}
\everymath{\color{blue}}
%\everymath{\color[rgb]{0,1,1}}
%\pagecolor[rgb]{0,0,0.5}


\newcommand*{\pdtest}[3][]{\ensuremath{\frac{\partial^{#1} #2}{\partial #3}}}

\newcommand*{\deffunc}[6][]{\ensuremath{
\begin{array}{rcl}
#2 : #3 &\rightarrow& #4\\
#5 &\mapsto& #6
\end{array}
}}

\newcommand{\eqcolon}{\mathrel{\resizebox{\widthof{$\mathord{=}$}}{\height}{ $\!\!=\!\!\resizebox{1.2\width}{0.8\height}{\raisebox{0.23ex}{$\mathop{:}$}}\!\!$ }}}
\newcommand{\coloneq}{\mathrel{\resizebox{\widthof{$\mathord{=}$}}{\height}{ $\!\!\resizebox{1.2\width}{0.8\height}{\raisebox{0.23ex}{$\mathop{:}$}}\!\!=\!\!$ }}}
\newcommand{\eqcolonl}{\ensuremath{\mathrel{=\!\!\mathop{:}}}}
\newcommand{\coloneql}{\ensuremath{\mathrel{\mathop{:} \!\! =}}}
\newcommand{\vc}[1]{% inline column vector
  \left(\begin{smallmatrix}#1\end{smallmatrix}\right)%
}
\newcommand{\vr}[1]{% inline row vector
  \begin{smallmatrix}(\,#1\,)\end{smallmatrix}%
}
\makeatletter
\newcommand*{\defeq}{\ =\mathrel{\rlap{%
                     \raisebox{0.3ex}{$\m@th\cdot$}}%
                     \raisebox{-0.3ex}{$\m@th\cdot$}}%
                     }
\makeatother

\newcommand{\mathcircle}[1]{% inline row vector
 \overset{\circ}{#1}
}
\newcommand{\ulim}{% low limit
 \underline{\lim}
}
\newcommand{\ssi}{% iff
\iff
}
\newcommand{\ps}[2]{
\expval{#1 | #2}
}
\newcommand{\df}[1]{
\mqty{#1}
}
\newcommand{\n}[1]{
\norm{#1}
}
\newcommand{\sys}[1]{
\left\{\smqty{#1}\right.
}


\newcommand{\eqdef}{\ensuremath{\overset{\text{def}}=}}


\def\Circlearrowright{\ensuremath{%
  \rotatebox[origin=c]{230}{$\circlearrowright$}}}

\newcommand\ct[1]{\text{\rmfamily\upshape #1}}
\newcommand\question[1]{ {\color{red} ...!? \small #1}}
\newcommand\caz[1]{\left\{\begin{array} #1 \end{array}\right.}
\newcommand\const{\text{\rmfamily\upshape const}}
\newcommand\toP{ \overset{\pro}{\to}}
\newcommand\toPP{ \overset{\text{PP}}{\to}}
\newcommand{\oeq}{\mathrel{\text{\textcircled{$=$}}}}





\usepackage{xcolor}
% \usepackage[normalem]{ulem}
\usepackage{lipsum}
\makeatletter
% \newcommand\colorwave[1][blue]{\bgroup \markoverwith{\lower3.5\p@\hbox{\sixly \textcolor{#1}{\char58}}}\ULon}
%\font\sixly=lasy6 % does not re-load if already loaded, so no memory problem.

\newmdtheoremenv[
linewidth= 1pt,linecolor= blue,%
leftmargin=20,rightmargin=20,innertopmargin=0pt, innerrightmargin=40,%
tikzsetting = { draw=lightgray, line width = 0.3pt,dashed,%
dash pattern = on 15pt off 3pt},%
splittopskip=\topskip,skipbelow=\baselineskip,%
skipabove=\baselineskip,ntheorem,roundcorner=0pt,
% backgroundcolor=pagebg,font=\color{orange}\sffamily, fontcolor=white
]{examplebox}{Exemple}[section]



\newcommand\R{\mathbb{R}}
\newcommand\Z{\mathbb{Z}}
\newcommand\N{\mathbb{N}}
\newcommand\E{\mathbb{E}}
\newcommand\F{\mathcal{F}}
\newcommand\cH{\mathcal{H}}
\newcommand\V{\mathbb{V}}
\newcommand\dmo{ ^{-1} }
\newcommand\kapa{\kappa}
\newcommand\im{Im}
\newcommand\hs{\mathcal{H}}





\usepackage{soul}

\makeatletter
\newcommand*{\whiten}[1]{\llap{\textcolor{white}{{\the\SOUL@token}}\hspace{#1pt}}}
\DeclareRobustCommand*\myul{%
    \def\SOUL@everyspace{\underline{\space}\kern\z@}%
    \def\SOUL@everytoken{%
     \setbox0=\hbox{\the\SOUL@token}%
     \ifdim\dp0>\z@
        \raisebox{\dp0}{\underline{\phantom{\the\SOUL@token}}}%
        \whiten{1}\whiten{0}%
        \whiten{-1}\whiten{-2}%
        \llap{\the\SOUL@token}%
     \else
        \underline{\the\SOUL@token}%
     \fi}%
\SOUL@}
\makeatother

\newcommand*{\demp}{\fontfamily{lmtt}\selectfont}

\DeclareTextFontCommand{\textdemp}{\demp}

\begin{document}

\ifcomment
Multiline
comment
\fi
\ifcomment
\myul{Typesetting test}
% \color[rgb]{1,1,1}
$∑_i^n≠ 60º±∞π∆¬≈√j∫h≤≥µ$

$\CR \R\pro\ind\pro\gS\pro
\mqty[a&b\\c&d]$
$\pro\mathbb{P}$
$\dd{x}$

  \[
    \alpha(x)=\left\{
                \begin{array}{ll}
                  x\\
                  \frac{1}{1+e^{-kx}}\\
                  \frac{e^x-e^{-x}}{e^x+e^{-x}}
                \end{array}
              \right.
  \]

  $\expval{x}$
  
  $\chi_\rho(ghg\dmo)=\Tr(\rho_{ghg\dmo})=\Tr(\rho_g\circ\rho_h\circ\rho\dmo_g)=\Tr(\rho_h)\overset{\mbox{\scalebox{0.5}{$\Tr(AB)=\Tr(BA)$}}}{=}\chi_\rho(h)$
  	$\mathop{\oplus}_{\substack{x\in X}}$

$\mat(\rho_g)=(a_{ij}(g))_{\scriptsize \substack{1\leq i\leq d \\ 1\leq j\leq d}}$ et $\mat(\rho'_g)=(a'_{ij}(g))_{\scriptsize \substack{1\leq i'\leq d' \\ 1\leq j'\leq d'}}$



\[\int_a^b{\mathbb{R}^2}g(u, v)\dd{P_{XY}}(u, v)=\iint g(u,v) f_{XY}(u, v)\dd \lambda(u) \dd \lambda(v)\]
$$\lim_{x\to\infty} f(x)$$	
$$\iiiint_V \mu(t,u,v,w) \,dt\,du\,dv\,dw$$
$$\sum_{n=1}^{\infty} 2^{-n} = 1$$	
\begin{definition}
	Si $X$ et $Y$ sont 2 v.a. ou definit la \textsc{Covariance} entre $X$ et $Y$ comme
	$\cov(X,Y)\overset{\text{def}}{=}\E\left[(X-\E(X))(Y-\E(Y))\right]=\E(XY)-\E(X)\E(Y)$.
\end{definition}
\fi
\pagebreak

% \tableofcontents

% insert your code here
%% !TEX encoding = UTF-8 Unicode
% !TEX TS-program = xelatex

\documentclass[french]{report}

%\usepackage[utf8]{inputenc}
%\usepackage[T1]{fontenc}
\usepackage{babel}


\newif\ifcomment
%\commenttrue # Show comments

\usepackage{physics}
\usepackage{amssymb}


\usepackage{amsthm}
% \usepackage{thmtools}
\usepackage{mathtools}
\usepackage{amsfonts}

\usepackage{color}

\usepackage{tikz}

\usepackage{geometry}
\geometry{a5paper, margin=0.1in, right=1cm}

\usepackage{dsfont}

\usepackage{graphicx}
\graphicspath{ {images/} }

\usepackage{faktor}

\usepackage{IEEEtrantools}
\usepackage{enumerate}   
\usepackage[PostScript=dvips]{"/Users/aware/Documents/Courses/diagrams"}


\newtheorem{theorem}{Théorème}[section]
\renewcommand{\thetheorem}{\arabic{theorem}}
\newtheorem{lemme}{Lemme}[section]
\renewcommand{\thelemme}{\arabic{lemme}}
\newtheorem{proposition}{Proposition}[section]
\renewcommand{\theproposition}{\arabic{proposition}}
\newtheorem{notations}{Notations}[section]
\newtheorem{problem}{Problème}[section]
\newtheorem{corollary}{Corollaire}[theorem]
\renewcommand{\thecorollary}{\arabic{corollary}}
\newtheorem{property}{Propriété}[section]
\newtheorem{objective}{Objectif}[section]

\theoremstyle{definition}
\newtheorem{definition}{Définition}[section]
\renewcommand{\thedefinition}{\arabic{definition}}
\newtheorem{exercise}{Exercice}[chapter]
\renewcommand{\theexercise}{\arabic{exercise}}
\newtheorem{example}{Exemple}[chapter]
\renewcommand{\theexample}{\arabic{example}}
\newtheorem*{solution}{Solution}
\newtheorem*{application}{Application}
\newtheorem*{notation}{Notation}
\newtheorem*{vocabulary}{Vocabulaire}
\newtheorem*{properties}{Propriétés}



\theoremstyle{remark}
\newtheorem*{remark}{Remarque}
\newtheorem*{rappel}{Rappel}


\usepackage{etoolbox}
\AtBeginEnvironment{exercise}{\small}
\AtBeginEnvironment{example}{\small}

\usepackage{cases}
\usepackage[red]{mypack}

\usepackage[framemethod=TikZ]{mdframed}

\definecolor{bg}{rgb}{0.4,0.25,0.95}
\definecolor{pagebg}{rgb}{0,0,0.5}
\surroundwithmdframed[
   topline=false,
   rightline=false,
   bottomline=false,
   leftmargin=\parindent,
   skipabove=8pt,
   skipbelow=8pt,
   linecolor=blue,
   innerbottommargin=10pt,
   % backgroundcolor=bg,font=\color{orange}\sffamily, fontcolor=white
]{definition}

\usepackage{empheq}
\usepackage[most]{tcolorbox}

\newtcbox{\mymath}[1][]{%
    nobeforeafter, math upper, tcbox raise base,
    enhanced, colframe=blue!30!black,
    colback=red!10, boxrule=1pt,
    #1}

\usepackage{unixode}


\DeclareMathOperator{\ord}{ord}
\DeclareMathOperator{\orb}{orb}
\DeclareMathOperator{\stab}{stab}
\DeclareMathOperator{\Stab}{stab}
\DeclareMathOperator{\ppcm}{ppcm}
\DeclareMathOperator{\conj}{Conj}
\DeclareMathOperator{\End}{End}
\DeclareMathOperator{\rot}{rot}
\DeclareMathOperator{\trs}{trace}
\DeclareMathOperator{\Ind}{Ind}
\DeclareMathOperator{\mat}{Mat}
\DeclareMathOperator{\id}{Id}
\DeclareMathOperator{\vect}{vect}
\DeclareMathOperator{\img}{img}
\DeclareMathOperator{\cov}{Cov}
\DeclareMathOperator{\dist}{dist}
\DeclareMathOperator{\irr}{Irr}
\DeclareMathOperator{\image}{Im}
\DeclareMathOperator{\pd}{\partial}
\DeclareMathOperator{\epi}{epi}
\DeclareMathOperator{\Argmin}{Argmin}
\DeclareMathOperator{\dom}{dom}
\DeclareMathOperator{\proj}{proj}
\DeclareMathOperator{\ctg}{ctg}
\DeclareMathOperator{\supp}{supp}
\DeclareMathOperator{\argmin}{argmin}
\DeclareMathOperator{\mult}{mult}
\DeclareMathOperator{\ch}{ch}
\DeclareMathOperator{\sh}{sh}
\DeclareMathOperator{\rang}{rang}
\DeclareMathOperator{\diam}{diam}
\DeclareMathOperator{\Epigraphe}{Epigraphe}




\usepackage{xcolor}
\everymath{\color{blue}}
%\everymath{\color[rgb]{0,1,1}}
%\pagecolor[rgb]{0,0,0.5}


\newcommand*{\pdtest}[3][]{\ensuremath{\frac{\partial^{#1} #2}{\partial #3}}}

\newcommand*{\deffunc}[6][]{\ensuremath{
\begin{array}{rcl}
#2 : #3 &\rightarrow& #4\\
#5 &\mapsto& #6
\end{array}
}}

\newcommand{\eqcolon}{\mathrel{\resizebox{\widthof{$\mathord{=}$}}{\height}{ $\!\!=\!\!\resizebox{1.2\width}{0.8\height}{\raisebox{0.23ex}{$\mathop{:}$}}\!\!$ }}}
\newcommand{\coloneq}{\mathrel{\resizebox{\widthof{$\mathord{=}$}}{\height}{ $\!\!\resizebox{1.2\width}{0.8\height}{\raisebox{0.23ex}{$\mathop{:}$}}\!\!=\!\!$ }}}
\newcommand{\eqcolonl}{\ensuremath{\mathrel{=\!\!\mathop{:}}}}
\newcommand{\coloneql}{\ensuremath{\mathrel{\mathop{:} \!\! =}}}
\newcommand{\vc}[1]{% inline column vector
  \left(\begin{smallmatrix}#1\end{smallmatrix}\right)%
}
\newcommand{\vr}[1]{% inline row vector
  \begin{smallmatrix}(\,#1\,)\end{smallmatrix}%
}
\makeatletter
\newcommand*{\defeq}{\ =\mathrel{\rlap{%
                     \raisebox{0.3ex}{$\m@th\cdot$}}%
                     \raisebox{-0.3ex}{$\m@th\cdot$}}%
                     }
\makeatother

\newcommand{\mathcircle}[1]{% inline row vector
 \overset{\circ}{#1}
}
\newcommand{\ulim}{% low limit
 \underline{\lim}
}
\newcommand{\ssi}{% iff
\iff
}
\newcommand{\ps}[2]{
\expval{#1 | #2}
}
\newcommand{\df}[1]{
\mqty{#1}
}
\newcommand{\n}[1]{
\norm{#1}
}
\newcommand{\sys}[1]{
\left\{\smqty{#1}\right.
}


\newcommand{\eqdef}{\ensuremath{\overset{\text{def}}=}}


\def\Circlearrowright{\ensuremath{%
  \rotatebox[origin=c]{230}{$\circlearrowright$}}}

\newcommand\ct[1]{\text{\rmfamily\upshape #1}}
\newcommand\question[1]{ {\color{red} ...!? \small #1}}
\newcommand\caz[1]{\left\{\begin{array} #1 \end{array}\right.}
\newcommand\const{\text{\rmfamily\upshape const}}
\newcommand\toP{ \overset{\pro}{\to}}
\newcommand\toPP{ \overset{\text{PP}}{\to}}
\newcommand{\oeq}{\mathrel{\text{\textcircled{$=$}}}}





\usepackage{xcolor}
% \usepackage[normalem]{ulem}
\usepackage{lipsum}
\makeatletter
% \newcommand\colorwave[1][blue]{\bgroup \markoverwith{\lower3.5\p@\hbox{\sixly \textcolor{#1}{\char58}}}\ULon}
%\font\sixly=lasy6 % does not re-load if already loaded, so no memory problem.

\newmdtheoremenv[
linewidth= 1pt,linecolor= blue,%
leftmargin=20,rightmargin=20,innertopmargin=0pt, innerrightmargin=40,%
tikzsetting = { draw=lightgray, line width = 0.3pt,dashed,%
dash pattern = on 15pt off 3pt},%
splittopskip=\topskip,skipbelow=\baselineskip,%
skipabove=\baselineskip,ntheorem,roundcorner=0pt,
% backgroundcolor=pagebg,font=\color{orange}\sffamily, fontcolor=white
]{examplebox}{Exemple}[section]



\newcommand\R{\mathbb{R}}
\newcommand\Z{\mathbb{Z}}
\newcommand\N{\mathbb{N}}
\newcommand\E{\mathbb{E}}
\newcommand\F{\mathcal{F}}
\newcommand\cH{\mathcal{H}}
\newcommand\V{\mathbb{V}}
\newcommand\dmo{ ^{-1} }
\newcommand\kapa{\kappa}
\newcommand\im{Im}
\newcommand\hs{\mathcal{H}}





\usepackage{soul}

\makeatletter
\newcommand*{\whiten}[1]{\llap{\textcolor{white}{{\the\SOUL@token}}\hspace{#1pt}}}
\DeclareRobustCommand*\myul{%
    \def\SOUL@everyspace{\underline{\space}\kern\z@}%
    \def\SOUL@everytoken{%
     \setbox0=\hbox{\the\SOUL@token}%
     \ifdim\dp0>\z@
        \raisebox{\dp0}{\underline{\phantom{\the\SOUL@token}}}%
        \whiten{1}\whiten{0}%
        \whiten{-1}\whiten{-2}%
        \llap{\the\SOUL@token}%
     \else
        \underline{\the\SOUL@token}%
     \fi}%
\SOUL@}
\makeatother

\newcommand*{\demp}{\fontfamily{lmtt}\selectfont}

\DeclareTextFontCommand{\textdemp}{\demp}

\begin{document}

\ifcomment
Multiline
comment
\fi
\ifcomment
\myul{Typesetting test}
% \color[rgb]{1,1,1}
$∑_i^n≠ 60º±∞π∆¬≈√j∫h≤≥µ$

$\CR \R\pro\ind\pro\gS\pro
\mqty[a&b\\c&d]$
$\pro\mathbb{P}$
$\dd{x}$

  \[
    \alpha(x)=\left\{
                \begin{array}{ll}
                  x\\
                  \frac{1}{1+e^{-kx}}\\
                  \frac{e^x-e^{-x}}{e^x+e^{-x}}
                \end{array}
              \right.
  \]

  $\expval{x}$
  
  $\chi_\rho(ghg\dmo)=\Tr(\rho_{ghg\dmo})=\Tr(\rho_g\circ\rho_h\circ\rho\dmo_g)=\Tr(\rho_h)\overset{\mbox{\scalebox{0.5}{$\Tr(AB)=\Tr(BA)$}}}{=}\chi_\rho(h)$
  	$\mathop{\oplus}_{\substack{x\in X}}$

$\mat(\rho_g)=(a_{ij}(g))_{\scriptsize \substack{1\leq i\leq d \\ 1\leq j\leq d}}$ et $\mat(\rho'_g)=(a'_{ij}(g))_{\scriptsize \substack{1\leq i'\leq d' \\ 1\leq j'\leq d'}}$



\[\int_a^b{\mathbb{R}^2}g(u, v)\dd{P_{XY}}(u, v)=\iint g(u,v) f_{XY}(u, v)\dd \lambda(u) \dd \lambda(v)\]
$$\lim_{x\to\infty} f(x)$$	
$$\iiiint_V \mu(t,u,v,w) \,dt\,du\,dv\,dw$$
$$\sum_{n=1}^{\infty} 2^{-n} = 1$$	
\begin{definition}
	Si $X$ et $Y$ sont 2 v.a. ou definit la \textsc{Covariance} entre $X$ et $Y$ comme
	$\cov(X,Y)\overset{\text{def}}{=}\E\left[(X-\E(X))(Y-\E(Y))\right]=\E(XY)-\E(X)\E(Y)$.
\end{definition}
\fi
\pagebreak

% \tableofcontents

% insert your code here
%\input{./algebra/main.tex}
%\input{./geometrie-differentielle/main.tex}
%\input{./probabilite/main.tex}
%\input{./analyse-fonctionnelle/main.tex}
% \input{./Analyse-convexe-et-dualite-en-optimisation/main.tex}
%\input{./tikz/main.tex}
%\input{./Theorie-du-distributions/main.tex}
%\input{./optimisation/mine.tex}
 \input{./modelisation/main.tex}

% yves.aubry@univ-tln.fr : algebra

\end{document}

%% !TEX encoding = UTF-8 Unicode
% !TEX TS-program = xelatex

\documentclass[french]{report}

%\usepackage[utf8]{inputenc}
%\usepackage[T1]{fontenc}
\usepackage{babel}


\newif\ifcomment
%\commenttrue # Show comments

\usepackage{physics}
\usepackage{amssymb}


\usepackage{amsthm}
% \usepackage{thmtools}
\usepackage{mathtools}
\usepackage{amsfonts}

\usepackage{color}

\usepackage{tikz}

\usepackage{geometry}
\geometry{a5paper, margin=0.1in, right=1cm}

\usepackage{dsfont}

\usepackage{graphicx}
\graphicspath{ {images/} }

\usepackage{faktor}

\usepackage{IEEEtrantools}
\usepackage{enumerate}   
\usepackage[PostScript=dvips]{"/Users/aware/Documents/Courses/diagrams"}


\newtheorem{theorem}{Théorème}[section]
\renewcommand{\thetheorem}{\arabic{theorem}}
\newtheorem{lemme}{Lemme}[section]
\renewcommand{\thelemme}{\arabic{lemme}}
\newtheorem{proposition}{Proposition}[section]
\renewcommand{\theproposition}{\arabic{proposition}}
\newtheorem{notations}{Notations}[section]
\newtheorem{problem}{Problème}[section]
\newtheorem{corollary}{Corollaire}[theorem]
\renewcommand{\thecorollary}{\arabic{corollary}}
\newtheorem{property}{Propriété}[section]
\newtheorem{objective}{Objectif}[section]

\theoremstyle{definition}
\newtheorem{definition}{Définition}[section]
\renewcommand{\thedefinition}{\arabic{definition}}
\newtheorem{exercise}{Exercice}[chapter]
\renewcommand{\theexercise}{\arabic{exercise}}
\newtheorem{example}{Exemple}[chapter]
\renewcommand{\theexample}{\arabic{example}}
\newtheorem*{solution}{Solution}
\newtheorem*{application}{Application}
\newtheorem*{notation}{Notation}
\newtheorem*{vocabulary}{Vocabulaire}
\newtheorem*{properties}{Propriétés}



\theoremstyle{remark}
\newtheorem*{remark}{Remarque}
\newtheorem*{rappel}{Rappel}


\usepackage{etoolbox}
\AtBeginEnvironment{exercise}{\small}
\AtBeginEnvironment{example}{\small}

\usepackage{cases}
\usepackage[red]{mypack}

\usepackage[framemethod=TikZ]{mdframed}

\definecolor{bg}{rgb}{0.4,0.25,0.95}
\definecolor{pagebg}{rgb}{0,0,0.5}
\surroundwithmdframed[
   topline=false,
   rightline=false,
   bottomline=false,
   leftmargin=\parindent,
   skipabove=8pt,
   skipbelow=8pt,
   linecolor=blue,
   innerbottommargin=10pt,
   % backgroundcolor=bg,font=\color{orange}\sffamily, fontcolor=white
]{definition}

\usepackage{empheq}
\usepackage[most]{tcolorbox}

\newtcbox{\mymath}[1][]{%
    nobeforeafter, math upper, tcbox raise base,
    enhanced, colframe=blue!30!black,
    colback=red!10, boxrule=1pt,
    #1}

\usepackage{unixode}


\DeclareMathOperator{\ord}{ord}
\DeclareMathOperator{\orb}{orb}
\DeclareMathOperator{\stab}{stab}
\DeclareMathOperator{\Stab}{stab}
\DeclareMathOperator{\ppcm}{ppcm}
\DeclareMathOperator{\conj}{Conj}
\DeclareMathOperator{\End}{End}
\DeclareMathOperator{\rot}{rot}
\DeclareMathOperator{\trs}{trace}
\DeclareMathOperator{\Ind}{Ind}
\DeclareMathOperator{\mat}{Mat}
\DeclareMathOperator{\id}{Id}
\DeclareMathOperator{\vect}{vect}
\DeclareMathOperator{\img}{img}
\DeclareMathOperator{\cov}{Cov}
\DeclareMathOperator{\dist}{dist}
\DeclareMathOperator{\irr}{Irr}
\DeclareMathOperator{\image}{Im}
\DeclareMathOperator{\pd}{\partial}
\DeclareMathOperator{\epi}{epi}
\DeclareMathOperator{\Argmin}{Argmin}
\DeclareMathOperator{\dom}{dom}
\DeclareMathOperator{\proj}{proj}
\DeclareMathOperator{\ctg}{ctg}
\DeclareMathOperator{\supp}{supp}
\DeclareMathOperator{\argmin}{argmin}
\DeclareMathOperator{\mult}{mult}
\DeclareMathOperator{\ch}{ch}
\DeclareMathOperator{\sh}{sh}
\DeclareMathOperator{\rang}{rang}
\DeclareMathOperator{\diam}{diam}
\DeclareMathOperator{\Epigraphe}{Epigraphe}




\usepackage{xcolor}
\everymath{\color{blue}}
%\everymath{\color[rgb]{0,1,1}}
%\pagecolor[rgb]{0,0,0.5}


\newcommand*{\pdtest}[3][]{\ensuremath{\frac{\partial^{#1} #2}{\partial #3}}}

\newcommand*{\deffunc}[6][]{\ensuremath{
\begin{array}{rcl}
#2 : #3 &\rightarrow& #4\\
#5 &\mapsto& #6
\end{array}
}}

\newcommand{\eqcolon}{\mathrel{\resizebox{\widthof{$\mathord{=}$}}{\height}{ $\!\!=\!\!\resizebox{1.2\width}{0.8\height}{\raisebox{0.23ex}{$\mathop{:}$}}\!\!$ }}}
\newcommand{\coloneq}{\mathrel{\resizebox{\widthof{$\mathord{=}$}}{\height}{ $\!\!\resizebox{1.2\width}{0.8\height}{\raisebox{0.23ex}{$\mathop{:}$}}\!\!=\!\!$ }}}
\newcommand{\eqcolonl}{\ensuremath{\mathrel{=\!\!\mathop{:}}}}
\newcommand{\coloneql}{\ensuremath{\mathrel{\mathop{:} \!\! =}}}
\newcommand{\vc}[1]{% inline column vector
  \left(\begin{smallmatrix}#1\end{smallmatrix}\right)%
}
\newcommand{\vr}[1]{% inline row vector
  \begin{smallmatrix}(\,#1\,)\end{smallmatrix}%
}
\makeatletter
\newcommand*{\defeq}{\ =\mathrel{\rlap{%
                     \raisebox{0.3ex}{$\m@th\cdot$}}%
                     \raisebox{-0.3ex}{$\m@th\cdot$}}%
                     }
\makeatother

\newcommand{\mathcircle}[1]{% inline row vector
 \overset{\circ}{#1}
}
\newcommand{\ulim}{% low limit
 \underline{\lim}
}
\newcommand{\ssi}{% iff
\iff
}
\newcommand{\ps}[2]{
\expval{#1 | #2}
}
\newcommand{\df}[1]{
\mqty{#1}
}
\newcommand{\n}[1]{
\norm{#1}
}
\newcommand{\sys}[1]{
\left\{\smqty{#1}\right.
}


\newcommand{\eqdef}{\ensuremath{\overset{\text{def}}=}}


\def\Circlearrowright{\ensuremath{%
  \rotatebox[origin=c]{230}{$\circlearrowright$}}}

\newcommand\ct[1]{\text{\rmfamily\upshape #1}}
\newcommand\question[1]{ {\color{red} ...!? \small #1}}
\newcommand\caz[1]{\left\{\begin{array} #1 \end{array}\right.}
\newcommand\const{\text{\rmfamily\upshape const}}
\newcommand\toP{ \overset{\pro}{\to}}
\newcommand\toPP{ \overset{\text{PP}}{\to}}
\newcommand{\oeq}{\mathrel{\text{\textcircled{$=$}}}}





\usepackage{xcolor}
% \usepackage[normalem]{ulem}
\usepackage{lipsum}
\makeatletter
% \newcommand\colorwave[1][blue]{\bgroup \markoverwith{\lower3.5\p@\hbox{\sixly \textcolor{#1}{\char58}}}\ULon}
%\font\sixly=lasy6 % does not re-load if already loaded, so no memory problem.

\newmdtheoremenv[
linewidth= 1pt,linecolor= blue,%
leftmargin=20,rightmargin=20,innertopmargin=0pt, innerrightmargin=40,%
tikzsetting = { draw=lightgray, line width = 0.3pt,dashed,%
dash pattern = on 15pt off 3pt},%
splittopskip=\topskip,skipbelow=\baselineskip,%
skipabove=\baselineskip,ntheorem,roundcorner=0pt,
% backgroundcolor=pagebg,font=\color{orange}\sffamily, fontcolor=white
]{examplebox}{Exemple}[section]



\newcommand\R{\mathbb{R}}
\newcommand\Z{\mathbb{Z}}
\newcommand\N{\mathbb{N}}
\newcommand\E{\mathbb{E}}
\newcommand\F{\mathcal{F}}
\newcommand\cH{\mathcal{H}}
\newcommand\V{\mathbb{V}}
\newcommand\dmo{ ^{-1} }
\newcommand\kapa{\kappa}
\newcommand\im{Im}
\newcommand\hs{\mathcal{H}}





\usepackage{soul}

\makeatletter
\newcommand*{\whiten}[1]{\llap{\textcolor{white}{{\the\SOUL@token}}\hspace{#1pt}}}
\DeclareRobustCommand*\myul{%
    \def\SOUL@everyspace{\underline{\space}\kern\z@}%
    \def\SOUL@everytoken{%
     \setbox0=\hbox{\the\SOUL@token}%
     \ifdim\dp0>\z@
        \raisebox{\dp0}{\underline{\phantom{\the\SOUL@token}}}%
        \whiten{1}\whiten{0}%
        \whiten{-1}\whiten{-2}%
        \llap{\the\SOUL@token}%
     \else
        \underline{\the\SOUL@token}%
     \fi}%
\SOUL@}
\makeatother

\newcommand*{\demp}{\fontfamily{lmtt}\selectfont}

\DeclareTextFontCommand{\textdemp}{\demp}

\begin{document}

\ifcomment
Multiline
comment
\fi
\ifcomment
\myul{Typesetting test}
% \color[rgb]{1,1,1}
$∑_i^n≠ 60º±∞π∆¬≈√j∫h≤≥µ$

$\CR \R\pro\ind\pro\gS\pro
\mqty[a&b\\c&d]$
$\pro\mathbb{P}$
$\dd{x}$

  \[
    \alpha(x)=\left\{
                \begin{array}{ll}
                  x\\
                  \frac{1}{1+e^{-kx}}\\
                  \frac{e^x-e^{-x}}{e^x+e^{-x}}
                \end{array}
              \right.
  \]

  $\expval{x}$
  
  $\chi_\rho(ghg\dmo)=\Tr(\rho_{ghg\dmo})=\Tr(\rho_g\circ\rho_h\circ\rho\dmo_g)=\Tr(\rho_h)\overset{\mbox{\scalebox{0.5}{$\Tr(AB)=\Tr(BA)$}}}{=}\chi_\rho(h)$
  	$\mathop{\oplus}_{\substack{x\in X}}$

$\mat(\rho_g)=(a_{ij}(g))_{\scriptsize \substack{1\leq i\leq d \\ 1\leq j\leq d}}$ et $\mat(\rho'_g)=(a'_{ij}(g))_{\scriptsize \substack{1\leq i'\leq d' \\ 1\leq j'\leq d'}}$



\[\int_a^b{\mathbb{R}^2}g(u, v)\dd{P_{XY}}(u, v)=\iint g(u,v) f_{XY}(u, v)\dd \lambda(u) \dd \lambda(v)\]
$$\lim_{x\to\infty} f(x)$$	
$$\iiiint_V \mu(t,u,v,w) \,dt\,du\,dv\,dw$$
$$\sum_{n=1}^{\infty} 2^{-n} = 1$$	
\begin{definition}
	Si $X$ et $Y$ sont 2 v.a. ou definit la \textsc{Covariance} entre $X$ et $Y$ comme
	$\cov(X,Y)\overset{\text{def}}{=}\E\left[(X-\E(X))(Y-\E(Y))\right]=\E(XY)-\E(X)\E(Y)$.
\end{definition}
\fi
\pagebreak

% \tableofcontents

% insert your code here
%\input{./algebra/main.tex}
%\input{./geometrie-differentielle/main.tex}
%\input{./probabilite/main.tex}
%\input{./analyse-fonctionnelle/main.tex}
% \input{./Analyse-convexe-et-dualite-en-optimisation/main.tex}
%\input{./tikz/main.tex}
%\input{./Theorie-du-distributions/main.tex}
%\input{./optimisation/mine.tex}
 \input{./modelisation/main.tex}

% yves.aubry@univ-tln.fr : algebra

\end{document}

%% !TEX encoding = UTF-8 Unicode
% !TEX TS-program = xelatex

\documentclass[french]{report}

%\usepackage[utf8]{inputenc}
%\usepackage[T1]{fontenc}
\usepackage{babel}


\newif\ifcomment
%\commenttrue # Show comments

\usepackage{physics}
\usepackage{amssymb}


\usepackage{amsthm}
% \usepackage{thmtools}
\usepackage{mathtools}
\usepackage{amsfonts}

\usepackage{color}

\usepackage{tikz}

\usepackage{geometry}
\geometry{a5paper, margin=0.1in, right=1cm}

\usepackage{dsfont}

\usepackage{graphicx}
\graphicspath{ {images/} }

\usepackage{faktor}

\usepackage{IEEEtrantools}
\usepackage{enumerate}   
\usepackage[PostScript=dvips]{"/Users/aware/Documents/Courses/diagrams"}


\newtheorem{theorem}{Théorème}[section]
\renewcommand{\thetheorem}{\arabic{theorem}}
\newtheorem{lemme}{Lemme}[section]
\renewcommand{\thelemme}{\arabic{lemme}}
\newtheorem{proposition}{Proposition}[section]
\renewcommand{\theproposition}{\arabic{proposition}}
\newtheorem{notations}{Notations}[section]
\newtheorem{problem}{Problème}[section]
\newtheorem{corollary}{Corollaire}[theorem]
\renewcommand{\thecorollary}{\arabic{corollary}}
\newtheorem{property}{Propriété}[section]
\newtheorem{objective}{Objectif}[section]

\theoremstyle{definition}
\newtheorem{definition}{Définition}[section]
\renewcommand{\thedefinition}{\arabic{definition}}
\newtheorem{exercise}{Exercice}[chapter]
\renewcommand{\theexercise}{\arabic{exercise}}
\newtheorem{example}{Exemple}[chapter]
\renewcommand{\theexample}{\arabic{example}}
\newtheorem*{solution}{Solution}
\newtheorem*{application}{Application}
\newtheorem*{notation}{Notation}
\newtheorem*{vocabulary}{Vocabulaire}
\newtheorem*{properties}{Propriétés}



\theoremstyle{remark}
\newtheorem*{remark}{Remarque}
\newtheorem*{rappel}{Rappel}


\usepackage{etoolbox}
\AtBeginEnvironment{exercise}{\small}
\AtBeginEnvironment{example}{\small}

\usepackage{cases}
\usepackage[red]{mypack}

\usepackage[framemethod=TikZ]{mdframed}

\definecolor{bg}{rgb}{0.4,0.25,0.95}
\definecolor{pagebg}{rgb}{0,0,0.5}
\surroundwithmdframed[
   topline=false,
   rightline=false,
   bottomline=false,
   leftmargin=\parindent,
   skipabove=8pt,
   skipbelow=8pt,
   linecolor=blue,
   innerbottommargin=10pt,
   % backgroundcolor=bg,font=\color{orange}\sffamily, fontcolor=white
]{definition}

\usepackage{empheq}
\usepackage[most]{tcolorbox}

\newtcbox{\mymath}[1][]{%
    nobeforeafter, math upper, tcbox raise base,
    enhanced, colframe=blue!30!black,
    colback=red!10, boxrule=1pt,
    #1}

\usepackage{unixode}


\DeclareMathOperator{\ord}{ord}
\DeclareMathOperator{\orb}{orb}
\DeclareMathOperator{\stab}{stab}
\DeclareMathOperator{\Stab}{stab}
\DeclareMathOperator{\ppcm}{ppcm}
\DeclareMathOperator{\conj}{Conj}
\DeclareMathOperator{\End}{End}
\DeclareMathOperator{\rot}{rot}
\DeclareMathOperator{\trs}{trace}
\DeclareMathOperator{\Ind}{Ind}
\DeclareMathOperator{\mat}{Mat}
\DeclareMathOperator{\id}{Id}
\DeclareMathOperator{\vect}{vect}
\DeclareMathOperator{\img}{img}
\DeclareMathOperator{\cov}{Cov}
\DeclareMathOperator{\dist}{dist}
\DeclareMathOperator{\irr}{Irr}
\DeclareMathOperator{\image}{Im}
\DeclareMathOperator{\pd}{\partial}
\DeclareMathOperator{\epi}{epi}
\DeclareMathOperator{\Argmin}{Argmin}
\DeclareMathOperator{\dom}{dom}
\DeclareMathOperator{\proj}{proj}
\DeclareMathOperator{\ctg}{ctg}
\DeclareMathOperator{\supp}{supp}
\DeclareMathOperator{\argmin}{argmin}
\DeclareMathOperator{\mult}{mult}
\DeclareMathOperator{\ch}{ch}
\DeclareMathOperator{\sh}{sh}
\DeclareMathOperator{\rang}{rang}
\DeclareMathOperator{\diam}{diam}
\DeclareMathOperator{\Epigraphe}{Epigraphe}




\usepackage{xcolor}
\everymath{\color{blue}}
%\everymath{\color[rgb]{0,1,1}}
%\pagecolor[rgb]{0,0,0.5}


\newcommand*{\pdtest}[3][]{\ensuremath{\frac{\partial^{#1} #2}{\partial #3}}}

\newcommand*{\deffunc}[6][]{\ensuremath{
\begin{array}{rcl}
#2 : #3 &\rightarrow& #4\\
#5 &\mapsto& #6
\end{array}
}}

\newcommand{\eqcolon}{\mathrel{\resizebox{\widthof{$\mathord{=}$}}{\height}{ $\!\!=\!\!\resizebox{1.2\width}{0.8\height}{\raisebox{0.23ex}{$\mathop{:}$}}\!\!$ }}}
\newcommand{\coloneq}{\mathrel{\resizebox{\widthof{$\mathord{=}$}}{\height}{ $\!\!\resizebox{1.2\width}{0.8\height}{\raisebox{0.23ex}{$\mathop{:}$}}\!\!=\!\!$ }}}
\newcommand{\eqcolonl}{\ensuremath{\mathrel{=\!\!\mathop{:}}}}
\newcommand{\coloneql}{\ensuremath{\mathrel{\mathop{:} \!\! =}}}
\newcommand{\vc}[1]{% inline column vector
  \left(\begin{smallmatrix}#1\end{smallmatrix}\right)%
}
\newcommand{\vr}[1]{% inline row vector
  \begin{smallmatrix}(\,#1\,)\end{smallmatrix}%
}
\makeatletter
\newcommand*{\defeq}{\ =\mathrel{\rlap{%
                     \raisebox{0.3ex}{$\m@th\cdot$}}%
                     \raisebox{-0.3ex}{$\m@th\cdot$}}%
                     }
\makeatother

\newcommand{\mathcircle}[1]{% inline row vector
 \overset{\circ}{#1}
}
\newcommand{\ulim}{% low limit
 \underline{\lim}
}
\newcommand{\ssi}{% iff
\iff
}
\newcommand{\ps}[2]{
\expval{#1 | #2}
}
\newcommand{\df}[1]{
\mqty{#1}
}
\newcommand{\n}[1]{
\norm{#1}
}
\newcommand{\sys}[1]{
\left\{\smqty{#1}\right.
}


\newcommand{\eqdef}{\ensuremath{\overset{\text{def}}=}}


\def\Circlearrowright{\ensuremath{%
  \rotatebox[origin=c]{230}{$\circlearrowright$}}}

\newcommand\ct[1]{\text{\rmfamily\upshape #1}}
\newcommand\question[1]{ {\color{red} ...!? \small #1}}
\newcommand\caz[1]{\left\{\begin{array} #1 \end{array}\right.}
\newcommand\const{\text{\rmfamily\upshape const}}
\newcommand\toP{ \overset{\pro}{\to}}
\newcommand\toPP{ \overset{\text{PP}}{\to}}
\newcommand{\oeq}{\mathrel{\text{\textcircled{$=$}}}}





\usepackage{xcolor}
% \usepackage[normalem]{ulem}
\usepackage{lipsum}
\makeatletter
% \newcommand\colorwave[1][blue]{\bgroup \markoverwith{\lower3.5\p@\hbox{\sixly \textcolor{#1}{\char58}}}\ULon}
%\font\sixly=lasy6 % does not re-load if already loaded, so no memory problem.

\newmdtheoremenv[
linewidth= 1pt,linecolor= blue,%
leftmargin=20,rightmargin=20,innertopmargin=0pt, innerrightmargin=40,%
tikzsetting = { draw=lightgray, line width = 0.3pt,dashed,%
dash pattern = on 15pt off 3pt},%
splittopskip=\topskip,skipbelow=\baselineskip,%
skipabove=\baselineskip,ntheorem,roundcorner=0pt,
% backgroundcolor=pagebg,font=\color{orange}\sffamily, fontcolor=white
]{examplebox}{Exemple}[section]



\newcommand\R{\mathbb{R}}
\newcommand\Z{\mathbb{Z}}
\newcommand\N{\mathbb{N}}
\newcommand\E{\mathbb{E}}
\newcommand\F{\mathcal{F}}
\newcommand\cH{\mathcal{H}}
\newcommand\V{\mathbb{V}}
\newcommand\dmo{ ^{-1} }
\newcommand\kapa{\kappa}
\newcommand\im{Im}
\newcommand\hs{\mathcal{H}}





\usepackage{soul}

\makeatletter
\newcommand*{\whiten}[1]{\llap{\textcolor{white}{{\the\SOUL@token}}\hspace{#1pt}}}
\DeclareRobustCommand*\myul{%
    \def\SOUL@everyspace{\underline{\space}\kern\z@}%
    \def\SOUL@everytoken{%
     \setbox0=\hbox{\the\SOUL@token}%
     \ifdim\dp0>\z@
        \raisebox{\dp0}{\underline{\phantom{\the\SOUL@token}}}%
        \whiten{1}\whiten{0}%
        \whiten{-1}\whiten{-2}%
        \llap{\the\SOUL@token}%
     \else
        \underline{\the\SOUL@token}%
     \fi}%
\SOUL@}
\makeatother

\newcommand*{\demp}{\fontfamily{lmtt}\selectfont}

\DeclareTextFontCommand{\textdemp}{\demp}

\begin{document}

\ifcomment
Multiline
comment
\fi
\ifcomment
\myul{Typesetting test}
% \color[rgb]{1,1,1}
$∑_i^n≠ 60º±∞π∆¬≈√j∫h≤≥µ$

$\CR \R\pro\ind\pro\gS\pro
\mqty[a&b\\c&d]$
$\pro\mathbb{P}$
$\dd{x}$

  \[
    \alpha(x)=\left\{
                \begin{array}{ll}
                  x\\
                  \frac{1}{1+e^{-kx}}\\
                  \frac{e^x-e^{-x}}{e^x+e^{-x}}
                \end{array}
              \right.
  \]

  $\expval{x}$
  
  $\chi_\rho(ghg\dmo)=\Tr(\rho_{ghg\dmo})=\Tr(\rho_g\circ\rho_h\circ\rho\dmo_g)=\Tr(\rho_h)\overset{\mbox{\scalebox{0.5}{$\Tr(AB)=\Tr(BA)$}}}{=}\chi_\rho(h)$
  	$\mathop{\oplus}_{\substack{x\in X}}$

$\mat(\rho_g)=(a_{ij}(g))_{\scriptsize \substack{1\leq i\leq d \\ 1\leq j\leq d}}$ et $\mat(\rho'_g)=(a'_{ij}(g))_{\scriptsize \substack{1\leq i'\leq d' \\ 1\leq j'\leq d'}}$



\[\int_a^b{\mathbb{R}^2}g(u, v)\dd{P_{XY}}(u, v)=\iint g(u,v) f_{XY}(u, v)\dd \lambda(u) \dd \lambda(v)\]
$$\lim_{x\to\infty} f(x)$$	
$$\iiiint_V \mu(t,u,v,w) \,dt\,du\,dv\,dw$$
$$\sum_{n=1}^{\infty} 2^{-n} = 1$$	
\begin{definition}
	Si $X$ et $Y$ sont 2 v.a. ou definit la \textsc{Covariance} entre $X$ et $Y$ comme
	$\cov(X,Y)\overset{\text{def}}{=}\E\left[(X-\E(X))(Y-\E(Y))\right]=\E(XY)-\E(X)\E(Y)$.
\end{definition}
\fi
\pagebreak

% \tableofcontents

% insert your code here
%\input{./algebra/main.tex}
%\input{./geometrie-differentielle/main.tex}
%\input{./probabilite/main.tex}
%\input{./analyse-fonctionnelle/main.tex}
% \input{./Analyse-convexe-et-dualite-en-optimisation/main.tex}
%\input{./tikz/main.tex}
%\input{./Theorie-du-distributions/main.tex}
%\input{./optimisation/mine.tex}
 \input{./modelisation/main.tex}

% yves.aubry@univ-tln.fr : algebra

\end{document}

%% !TEX encoding = UTF-8 Unicode
% !TEX TS-program = xelatex

\documentclass[french]{report}

%\usepackage[utf8]{inputenc}
%\usepackage[T1]{fontenc}
\usepackage{babel}


\newif\ifcomment
%\commenttrue # Show comments

\usepackage{physics}
\usepackage{amssymb}


\usepackage{amsthm}
% \usepackage{thmtools}
\usepackage{mathtools}
\usepackage{amsfonts}

\usepackage{color}

\usepackage{tikz}

\usepackage{geometry}
\geometry{a5paper, margin=0.1in, right=1cm}

\usepackage{dsfont}

\usepackage{graphicx}
\graphicspath{ {images/} }

\usepackage{faktor}

\usepackage{IEEEtrantools}
\usepackage{enumerate}   
\usepackage[PostScript=dvips]{"/Users/aware/Documents/Courses/diagrams"}


\newtheorem{theorem}{Théorème}[section]
\renewcommand{\thetheorem}{\arabic{theorem}}
\newtheorem{lemme}{Lemme}[section]
\renewcommand{\thelemme}{\arabic{lemme}}
\newtheorem{proposition}{Proposition}[section]
\renewcommand{\theproposition}{\arabic{proposition}}
\newtheorem{notations}{Notations}[section]
\newtheorem{problem}{Problème}[section]
\newtheorem{corollary}{Corollaire}[theorem]
\renewcommand{\thecorollary}{\arabic{corollary}}
\newtheorem{property}{Propriété}[section]
\newtheorem{objective}{Objectif}[section]

\theoremstyle{definition}
\newtheorem{definition}{Définition}[section]
\renewcommand{\thedefinition}{\arabic{definition}}
\newtheorem{exercise}{Exercice}[chapter]
\renewcommand{\theexercise}{\arabic{exercise}}
\newtheorem{example}{Exemple}[chapter]
\renewcommand{\theexample}{\arabic{example}}
\newtheorem*{solution}{Solution}
\newtheorem*{application}{Application}
\newtheorem*{notation}{Notation}
\newtheorem*{vocabulary}{Vocabulaire}
\newtheorem*{properties}{Propriétés}



\theoremstyle{remark}
\newtheorem*{remark}{Remarque}
\newtheorem*{rappel}{Rappel}


\usepackage{etoolbox}
\AtBeginEnvironment{exercise}{\small}
\AtBeginEnvironment{example}{\small}

\usepackage{cases}
\usepackage[red]{mypack}

\usepackage[framemethod=TikZ]{mdframed}

\definecolor{bg}{rgb}{0.4,0.25,0.95}
\definecolor{pagebg}{rgb}{0,0,0.5}
\surroundwithmdframed[
   topline=false,
   rightline=false,
   bottomline=false,
   leftmargin=\parindent,
   skipabove=8pt,
   skipbelow=8pt,
   linecolor=blue,
   innerbottommargin=10pt,
   % backgroundcolor=bg,font=\color{orange}\sffamily, fontcolor=white
]{definition}

\usepackage{empheq}
\usepackage[most]{tcolorbox}

\newtcbox{\mymath}[1][]{%
    nobeforeafter, math upper, tcbox raise base,
    enhanced, colframe=blue!30!black,
    colback=red!10, boxrule=1pt,
    #1}

\usepackage{unixode}


\DeclareMathOperator{\ord}{ord}
\DeclareMathOperator{\orb}{orb}
\DeclareMathOperator{\stab}{stab}
\DeclareMathOperator{\Stab}{stab}
\DeclareMathOperator{\ppcm}{ppcm}
\DeclareMathOperator{\conj}{Conj}
\DeclareMathOperator{\End}{End}
\DeclareMathOperator{\rot}{rot}
\DeclareMathOperator{\trs}{trace}
\DeclareMathOperator{\Ind}{Ind}
\DeclareMathOperator{\mat}{Mat}
\DeclareMathOperator{\id}{Id}
\DeclareMathOperator{\vect}{vect}
\DeclareMathOperator{\img}{img}
\DeclareMathOperator{\cov}{Cov}
\DeclareMathOperator{\dist}{dist}
\DeclareMathOperator{\irr}{Irr}
\DeclareMathOperator{\image}{Im}
\DeclareMathOperator{\pd}{\partial}
\DeclareMathOperator{\epi}{epi}
\DeclareMathOperator{\Argmin}{Argmin}
\DeclareMathOperator{\dom}{dom}
\DeclareMathOperator{\proj}{proj}
\DeclareMathOperator{\ctg}{ctg}
\DeclareMathOperator{\supp}{supp}
\DeclareMathOperator{\argmin}{argmin}
\DeclareMathOperator{\mult}{mult}
\DeclareMathOperator{\ch}{ch}
\DeclareMathOperator{\sh}{sh}
\DeclareMathOperator{\rang}{rang}
\DeclareMathOperator{\diam}{diam}
\DeclareMathOperator{\Epigraphe}{Epigraphe}




\usepackage{xcolor}
\everymath{\color{blue}}
%\everymath{\color[rgb]{0,1,1}}
%\pagecolor[rgb]{0,0,0.5}


\newcommand*{\pdtest}[3][]{\ensuremath{\frac{\partial^{#1} #2}{\partial #3}}}

\newcommand*{\deffunc}[6][]{\ensuremath{
\begin{array}{rcl}
#2 : #3 &\rightarrow& #4\\
#5 &\mapsto& #6
\end{array}
}}

\newcommand{\eqcolon}{\mathrel{\resizebox{\widthof{$\mathord{=}$}}{\height}{ $\!\!=\!\!\resizebox{1.2\width}{0.8\height}{\raisebox{0.23ex}{$\mathop{:}$}}\!\!$ }}}
\newcommand{\coloneq}{\mathrel{\resizebox{\widthof{$\mathord{=}$}}{\height}{ $\!\!\resizebox{1.2\width}{0.8\height}{\raisebox{0.23ex}{$\mathop{:}$}}\!\!=\!\!$ }}}
\newcommand{\eqcolonl}{\ensuremath{\mathrel{=\!\!\mathop{:}}}}
\newcommand{\coloneql}{\ensuremath{\mathrel{\mathop{:} \!\! =}}}
\newcommand{\vc}[1]{% inline column vector
  \left(\begin{smallmatrix}#1\end{smallmatrix}\right)%
}
\newcommand{\vr}[1]{% inline row vector
  \begin{smallmatrix}(\,#1\,)\end{smallmatrix}%
}
\makeatletter
\newcommand*{\defeq}{\ =\mathrel{\rlap{%
                     \raisebox{0.3ex}{$\m@th\cdot$}}%
                     \raisebox{-0.3ex}{$\m@th\cdot$}}%
                     }
\makeatother

\newcommand{\mathcircle}[1]{% inline row vector
 \overset{\circ}{#1}
}
\newcommand{\ulim}{% low limit
 \underline{\lim}
}
\newcommand{\ssi}{% iff
\iff
}
\newcommand{\ps}[2]{
\expval{#1 | #2}
}
\newcommand{\df}[1]{
\mqty{#1}
}
\newcommand{\n}[1]{
\norm{#1}
}
\newcommand{\sys}[1]{
\left\{\smqty{#1}\right.
}


\newcommand{\eqdef}{\ensuremath{\overset{\text{def}}=}}


\def\Circlearrowright{\ensuremath{%
  \rotatebox[origin=c]{230}{$\circlearrowright$}}}

\newcommand\ct[1]{\text{\rmfamily\upshape #1}}
\newcommand\question[1]{ {\color{red} ...!? \small #1}}
\newcommand\caz[1]{\left\{\begin{array} #1 \end{array}\right.}
\newcommand\const{\text{\rmfamily\upshape const}}
\newcommand\toP{ \overset{\pro}{\to}}
\newcommand\toPP{ \overset{\text{PP}}{\to}}
\newcommand{\oeq}{\mathrel{\text{\textcircled{$=$}}}}





\usepackage{xcolor}
% \usepackage[normalem]{ulem}
\usepackage{lipsum}
\makeatletter
% \newcommand\colorwave[1][blue]{\bgroup \markoverwith{\lower3.5\p@\hbox{\sixly \textcolor{#1}{\char58}}}\ULon}
%\font\sixly=lasy6 % does not re-load if already loaded, so no memory problem.

\newmdtheoremenv[
linewidth= 1pt,linecolor= blue,%
leftmargin=20,rightmargin=20,innertopmargin=0pt, innerrightmargin=40,%
tikzsetting = { draw=lightgray, line width = 0.3pt,dashed,%
dash pattern = on 15pt off 3pt},%
splittopskip=\topskip,skipbelow=\baselineskip,%
skipabove=\baselineskip,ntheorem,roundcorner=0pt,
% backgroundcolor=pagebg,font=\color{orange}\sffamily, fontcolor=white
]{examplebox}{Exemple}[section]



\newcommand\R{\mathbb{R}}
\newcommand\Z{\mathbb{Z}}
\newcommand\N{\mathbb{N}}
\newcommand\E{\mathbb{E}}
\newcommand\F{\mathcal{F}}
\newcommand\cH{\mathcal{H}}
\newcommand\V{\mathbb{V}}
\newcommand\dmo{ ^{-1} }
\newcommand\kapa{\kappa}
\newcommand\im{Im}
\newcommand\hs{\mathcal{H}}





\usepackage{soul}

\makeatletter
\newcommand*{\whiten}[1]{\llap{\textcolor{white}{{\the\SOUL@token}}\hspace{#1pt}}}
\DeclareRobustCommand*\myul{%
    \def\SOUL@everyspace{\underline{\space}\kern\z@}%
    \def\SOUL@everytoken{%
     \setbox0=\hbox{\the\SOUL@token}%
     \ifdim\dp0>\z@
        \raisebox{\dp0}{\underline{\phantom{\the\SOUL@token}}}%
        \whiten{1}\whiten{0}%
        \whiten{-1}\whiten{-2}%
        \llap{\the\SOUL@token}%
     \else
        \underline{\the\SOUL@token}%
     \fi}%
\SOUL@}
\makeatother

\newcommand*{\demp}{\fontfamily{lmtt}\selectfont}

\DeclareTextFontCommand{\textdemp}{\demp}

\begin{document}

\ifcomment
Multiline
comment
\fi
\ifcomment
\myul{Typesetting test}
% \color[rgb]{1,1,1}
$∑_i^n≠ 60º±∞π∆¬≈√j∫h≤≥µ$

$\CR \R\pro\ind\pro\gS\pro
\mqty[a&b\\c&d]$
$\pro\mathbb{P}$
$\dd{x}$

  \[
    \alpha(x)=\left\{
                \begin{array}{ll}
                  x\\
                  \frac{1}{1+e^{-kx}}\\
                  \frac{e^x-e^{-x}}{e^x+e^{-x}}
                \end{array}
              \right.
  \]

  $\expval{x}$
  
  $\chi_\rho(ghg\dmo)=\Tr(\rho_{ghg\dmo})=\Tr(\rho_g\circ\rho_h\circ\rho\dmo_g)=\Tr(\rho_h)\overset{\mbox{\scalebox{0.5}{$\Tr(AB)=\Tr(BA)$}}}{=}\chi_\rho(h)$
  	$\mathop{\oplus}_{\substack{x\in X}}$

$\mat(\rho_g)=(a_{ij}(g))_{\scriptsize \substack{1\leq i\leq d \\ 1\leq j\leq d}}$ et $\mat(\rho'_g)=(a'_{ij}(g))_{\scriptsize \substack{1\leq i'\leq d' \\ 1\leq j'\leq d'}}$



\[\int_a^b{\mathbb{R}^2}g(u, v)\dd{P_{XY}}(u, v)=\iint g(u,v) f_{XY}(u, v)\dd \lambda(u) \dd \lambda(v)\]
$$\lim_{x\to\infty} f(x)$$	
$$\iiiint_V \mu(t,u,v,w) \,dt\,du\,dv\,dw$$
$$\sum_{n=1}^{\infty} 2^{-n} = 1$$	
\begin{definition}
	Si $X$ et $Y$ sont 2 v.a. ou definit la \textsc{Covariance} entre $X$ et $Y$ comme
	$\cov(X,Y)\overset{\text{def}}{=}\E\left[(X-\E(X))(Y-\E(Y))\right]=\E(XY)-\E(X)\E(Y)$.
\end{definition}
\fi
\pagebreak

% \tableofcontents

% insert your code here
%\input{./algebra/main.tex}
%\input{./geometrie-differentielle/main.tex}
%\input{./probabilite/main.tex}
%\input{./analyse-fonctionnelle/main.tex}
% \input{./Analyse-convexe-et-dualite-en-optimisation/main.tex}
%\input{./tikz/main.tex}
%\input{./Theorie-du-distributions/main.tex}
%\input{./optimisation/mine.tex}
 \input{./modelisation/main.tex}

% yves.aubry@univ-tln.fr : algebra

\end{document}

% % !TEX encoding = UTF-8 Unicode
% !TEX TS-program = xelatex

\documentclass[french]{report}

%\usepackage[utf8]{inputenc}
%\usepackage[T1]{fontenc}
\usepackage{babel}


\newif\ifcomment
%\commenttrue # Show comments

\usepackage{physics}
\usepackage{amssymb}


\usepackage{amsthm}
% \usepackage{thmtools}
\usepackage{mathtools}
\usepackage{amsfonts}

\usepackage{color}

\usepackage{tikz}

\usepackage{geometry}
\geometry{a5paper, margin=0.1in, right=1cm}

\usepackage{dsfont}

\usepackage{graphicx}
\graphicspath{ {images/} }

\usepackage{faktor}

\usepackage{IEEEtrantools}
\usepackage{enumerate}   
\usepackage[PostScript=dvips]{"/Users/aware/Documents/Courses/diagrams"}


\newtheorem{theorem}{Théorème}[section]
\renewcommand{\thetheorem}{\arabic{theorem}}
\newtheorem{lemme}{Lemme}[section]
\renewcommand{\thelemme}{\arabic{lemme}}
\newtheorem{proposition}{Proposition}[section]
\renewcommand{\theproposition}{\arabic{proposition}}
\newtheorem{notations}{Notations}[section]
\newtheorem{problem}{Problème}[section]
\newtheorem{corollary}{Corollaire}[theorem]
\renewcommand{\thecorollary}{\arabic{corollary}}
\newtheorem{property}{Propriété}[section]
\newtheorem{objective}{Objectif}[section]

\theoremstyle{definition}
\newtheorem{definition}{Définition}[section]
\renewcommand{\thedefinition}{\arabic{definition}}
\newtheorem{exercise}{Exercice}[chapter]
\renewcommand{\theexercise}{\arabic{exercise}}
\newtheorem{example}{Exemple}[chapter]
\renewcommand{\theexample}{\arabic{example}}
\newtheorem*{solution}{Solution}
\newtheorem*{application}{Application}
\newtheorem*{notation}{Notation}
\newtheorem*{vocabulary}{Vocabulaire}
\newtheorem*{properties}{Propriétés}



\theoremstyle{remark}
\newtheorem*{remark}{Remarque}
\newtheorem*{rappel}{Rappel}


\usepackage{etoolbox}
\AtBeginEnvironment{exercise}{\small}
\AtBeginEnvironment{example}{\small}

\usepackage{cases}
\usepackage[red]{mypack}

\usepackage[framemethod=TikZ]{mdframed}

\definecolor{bg}{rgb}{0.4,0.25,0.95}
\definecolor{pagebg}{rgb}{0,0,0.5}
\surroundwithmdframed[
   topline=false,
   rightline=false,
   bottomline=false,
   leftmargin=\parindent,
   skipabove=8pt,
   skipbelow=8pt,
   linecolor=blue,
   innerbottommargin=10pt,
   % backgroundcolor=bg,font=\color{orange}\sffamily, fontcolor=white
]{definition}

\usepackage{empheq}
\usepackage[most]{tcolorbox}

\newtcbox{\mymath}[1][]{%
    nobeforeafter, math upper, tcbox raise base,
    enhanced, colframe=blue!30!black,
    colback=red!10, boxrule=1pt,
    #1}

\usepackage{unixode}


\DeclareMathOperator{\ord}{ord}
\DeclareMathOperator{\orb}{orb}
\DeclareMathOperator{\stab}{stab}
\DeclareMathOperator{\Stab}{stab}
\DeclareMathOperator{\ppcm}{ppcm}
\DeclareMathOperator{\conj}{Conj}
\DeclareMathOperator{\End}{End}
\DeclareMathOperator{\rot}{rot}
\DeclareMathOperator{\trs}{trace}
\DeclareMathOperator{\Ind}{Ind}
\DeclareMathOperator{\mat}{Mat}
\DeclareMathOperator{\id}{Id}
\DeclareMathOperator{\vect}{vect}
\DeclareMathOperator{\img}{img}
\DeclareMathOperator{\cov}{Cov}
\DeclareMathOperator{\dist}{dist}
\DeclareMathOperator{\irr}{Irr}
\DeclareMathOperator{\image}{Im}
\DeclareMathOperator{\pd}{\partial}
\DeclareMathOperator{\epi}{epi}
\DeclareMathOperator{\Argmin}{Argmin}
\DeclareMathOperator{\dom}{dom}
\DeclareMathOperator{\proj}{proj}
\DeclareMathOperator{\ctg}{ctg}
\DeclareMathOperator{\supp}{supp}
\DeclareMathOperator{\argmin}{argmin}
\DeclareMathOperator{\mult}{mult}
\DeclareMathOperator{\ch}{ch}
\DeclareMathOperator{\sh}{sh}
\DeclareMathOperator{\rang}{rang}
\DeclareMathOperator{\diam}{diam}
\DeclareMathOperator{\Epigraphe}{Epigraphe}




\usepackage{xcolor}
\everymath{\color{blue}}
%\everymath{\color[rgb]{0,1,1}}
%\pagecolor[rgb]{0,0,0.5}


\newcommand*{\pdtest}[3][]{\ensuremath{\frac{\partial^{#1} #2}{\partial #3}}}

\newcommand*{\deffunc}[6][]{\ensuremath{
\begin{array}{rcl}
#2 : #3 &\rightarrow& #4\\
#5 &\mapsto& #6
\end{array}
}}

\newcommand{\eqcolon}{\mathrel{\resizebox{\widthof{$\mathord{=}$}}{\height}{ $\!\!=\!\!\resizebox{1.2\width}{0.8\height}{\raisebox{0.23ex}{$\mathop{:}$}}\!\!$ }}}
\newcommand{\coloneq}{\mathrel{\resizebox{\widthof{$\mathord{=}$}}{\height}{ $\!\!\resizebox{1.2\width}{0.8\height}{\raisebox{0.23ex}{$\mathop{:}$}}\!\!=\!\!$ }}}
\newcommand{\eqcolonl}{\ensuremath{\mathrel{=\!\!\mathop{:}}}}
\newcommand{\coloneql}{\ensuremath{\mathrel{\mathop{:} \!\! =}}}
\newcommand{\vc}[1]{% inline column vector
  \left(\begin{smallmatrix}#1\end{smallmatrix}\right)%
}
\newcommand{\vr}[1]{% inline row vector
  \begin{smallmatrix}(\,#1\,)\end{smallmatrix}%
}
\makeatletter
\newcommand*{\defeq}{\ =\mathrel{\rlap{%
                     \raisebox{0.3ex}{$\m@th\cdot$}}%
                     \raisebox{-0.3ex}{$\m@th\cdot$}}%
                     }
\makeatother

\newcommand{\mathcircle}[1]{% inline row vector
 \overset{\circ}{#1}
}
\newcommand{\ulim}{% low limit
 \underline{\lim}
}
\newcommand{\ssi}{% iff
\iff
}
\newcommand{\ps}[2]{
\expval{#1 | #2}
}
\newcommand{\df}[1]{
\mqty{#1}
}
\newcommand{\n}[1]{
\norm{#1}
}
\newcommand{\sys}[1]{
\left\{\smqty{#1}\right.
}


\newcommand{\eqdef}{\ensuremath{\overset{\text{def}}=}}


\def\Circlearrowright{\ensuremath{%
  \rotatebox[origin=c]{230}{$\circlearrowright$}}}

\newcommand\ct[1]{\text{\rmfamily\upshape #1}}
\newcommand\question[1]{ {\color{red} ...!? \small #1}}
\newcommand\caz[1]{\left\{\begin{array} #1 \end{array}\right.}
\newcommand\const{\text{\rmfamily\upshape const}}
\newcommand\toP{ \overset{\pro}{\to}}
\newcommand\toPP{ \overset{\text{PP}}{\to}}
\newcommand{\oeq}{\mathrel{\text{\textcircled{$=$}}}}





\usepackage{xcolor}
% \usepackage[normalem]{ulem}
\usepackage{lipsum}
\makeatletter
% \newcommand\colorwave[1][blue]{\bgroup \markoverwith{\lower3.5\p@\hbox{\sixly \textcolor{#1}{\char58}}}\ULon}
%\font\sixly=lasy6 % does not re-load if already loaded, so no memory problem.

\newmdtheoremenv[
linewidth= 1pt,linecolor= blue,%
leftmargin=20,rightmargin=20,innertopmargin=0pt, innerrightmargin=40,%
tikzsetting = { draw=lightgray, line width = 0.3pt,dashed,%
dash pattern = on 15pt off 3pt},%
splittopskip=\topskip,skipbelow=\baselineskip,%
skipabove=\baselineskip,ntheorem,roundcorner=0pt,
% backgroundcolor=pagebg,font=\color{orange}\sffamily, fontcolor=white
]{examplebox}{Exemple}[section]



\newcommand\R{\mathbb{R}}
\newcommand\Z{\mathbb{Z}}
\newcommand\N{\mathbb{N}}
\newcommand\E{\mathbb{E}}
\newcommand\F{\mathcal{F}}
\newcommand\cH{\mathcal{H}}
\newcommand\V{\mathbb{V}}
\newcommand\dmo{ ^{-1} }
\newcommand\kapa{\kappa}
\newcommand\im{Im}
\newcommand\hs{\mathcal{H}}





\usepackage{soul}

\makeatletter
\newcommand*{\whiten}[1]{\llap{\textcolor{white}{{\the\SOUL@token}}\hspace{#1pt}}}
\DeclareRobustCommand*\myul{%
    \def\SOUL@everyspace{\underline{\space}\kern\z@}%
    \def\SOUL@everytoken{%
     \setbox0=\hbox{\the\SOUL@token}%
     \ifdim\dp0>\z@
        \raisebox{\dp0}{\underline{\phantom{\the\SOUL@token}}}%
        \whiten{1}\whiten{0}%
        \whiten{-1}\whiten{-2}%
        \llap{\the\SOUL@token}%
     \else
        \underline{\the\SOUL@token}%
     \fi}%
\SOUL@}
\makeatother

\newcommand*{\demp}{\fontfamily{lmtt}\selectfont}

\DeclareTextFontCommand{\textdemp}{\demp}

\begin{document}

\ifcomment
Multiline
comment
\fi
\ifcomment
\myul{Typesetting test}
% \color[rgb]{1,1,1}
$∑_i^n≠ 60º±∞π∆¬≈√j∫h≤≥µ$

$\CR \R\pro\ind\pro\gS\pro
\mqty[a&b\\c&d]$
$\pro\mathbb{P}$
$\dd{x}$

  \[
    \alpha(x)=\left\{
                \begin{array}{ll}
                  x\\
                  \frac{1}{1+e^{-kx}}\\
                  \frac{e^x-e^{-x}}{e^x+e^{-x}}
                \end{array}
              \right.
  \]

  $\expval{x}$
  
  $\chi_\rho(ghg\dmo)=\Tr(\rho_{ghg\dmo})=\Tr(\rho_g\circ\rho_h\circ\rho\dmo_g)=\Tr(\rho_h)\overset{\mbox{\scalebox{0.5}{$\Tr(AB)=\Tr(BA)$}}}{=}\chi_\rho(h)$
  	$\mathop{\oplus}_{\substack{x\in X}}$

$\mat(\rho_g)=(a_{ij}(g))_{\scriptsize \substack{1\leq i\leq d \\ 1\leq j\leq d}}$ et $\mat(\rho'_g)=(a'_{ij}(g))_{\scriptsize \substack{1\leq i'\leq d' \\ 1\leq j'\leq d'}}$



\[\int_a^b{\mathbb{R}^2}g(u, v)\dd{P_{XY}}(u, v)=\iint g(u,v) f_{XY}(u, v)\dd \lambda(u) \dd \lambda(v)\]
$$\lim_{x\to\infty} f(x)$$	
$$\iiiint_V \mu(t,u,v,w) \,dt\,du\,dv\,dw$$
$$\sum_{n=1}^{\infty} 2^{-n} = 1$$	
\begin{definition}
	Si $X$ et $Y$ sont 2 v.a. ou definit la \textsc{Covariance} entre $X$ et $Y$ comme
	$\cov(X,Y)\overset{\text{def}}{=}\E\left[(X-\E(X))(Y-\E(Y))\right]=\E(XY)-\E(X)\E(Y)$.
\end{definition}
\fi
\pagebreak

% \tableofcontents

% insert your code here
%\input{./algebra/main.tex}
%\input{./geometrie-differentielle/main.tex}
%\input{./probabilite/main.tex}
%\input{./analyse-fonctionnelle/main.tex}
% \input{./Analyse-convexe-et-dualite-en-optimisation/main.tex}
%\input{./tikz/main.tex}
%\input{./Theorie-du-distributions/main.tex}
%\input{./optimisation/mine.tex}
 \input{./modelisation/main.tex}

% yves.aubry@univ-tln.fr : algebra

\end{document}

%% !TEX encoding = UTF-8 Unicode
% !TEX TS-program = xelatex

\documentclass[french]{report}

%\usepackage[utf8]{inputenc}
%\usepackage[T1]{fontenc}
\usepackage{babel}


\newif\ifcomment
%\commenttrue # Show comments

\usepackage{physics}
\usepackage{amssymb}


\usepackage{amsthm}
% \usepackage{thmtools}
\usepackage{mathtools}
\usepackage{amsfonts}

\usepackage{color}

\usepackage{tikz}

\usepackage{geometry}
\geometry{a5paper, margin=0.1in, right=1cm}

\usepackage{dsfont}

\usepackage{graphicx}
\graphicspath{ {images/} }

\usepackage{faktor}

\usepackage{IEEEtrantools}
\usepackage{enumerate}   
\usepackage[PostScript=dvips]{"/Users/aware/Documents/Courses/diagrams"}


\newtheorem{theorem}{Théorème}[section]
\renewcommand{\thetheorem}{\arabic{theorem}}
\newtheorem{lemme}{Lemme}[section]
\renewcommand{\thelemme}{\arabic{lemme}}
\newtheorem{proposition}{Proposition}[section]
\renewcommand{\theproposition}{\arabic{proposition}}
\newtheorem{notations}{Notations}[section]
\newtheorem{problem}{Problème}[section]
\newtheorem{corollary}{Corollaire}[theorem]
\renewcommand{\thecorollary}{\arabic{corollary}}
\newtheorem{property}{Propriété}[section]
\newtheorem{objective}{Objectif}[section]

\theoremstyle{definition}
\newtheorem{definition}{Définition}[section]
\renewcommand{\thedefinition}{\arabic{definition}}
\newtheorem{exercise}{Exercice}[chapter]
\renewcommand{\theexercise}{\arabic{exercise}}
\newtheorem{example}{Exemple}[chapter]
\renewcommand{\theexample}{\arabic{example}}
\newtheorem*{solution}{Solution}
\newtheorem*{application}{Application}
\newtheorem*{notation}{Notation}
\newtheorem*{vocabulary}{Vocabulaire}
\newtheorem*{properties}{Propriétés}



\theoremstyle{remark}
\newtheorem*{remark}{Remarque}
\newtheorem*{rappel}{Rappel}


\usepackage{etoolbox}
\AtBeginEnvironment{exercise}{\small}
\AtBeginEnvironment{example}{\small}

\usepackage{cases}
\usepackage[red]{mypack}

\usepackage[framemethod=TikZ]{mdframed}

\definecolor{bg}{rgb}{0.4,0.25,0.95}
\definecolor{pagebg}{rgb}{0,0,0.5}
\surroundwithmdframed[
   topline=false,
   rightline=false,
   bottomline=false,
   leftmargin=\parindent,
   skipabove=8pt,
   skipbelow=8pt,
   linecolor=blue,
   innerbottommargin=10pt,
   % backgroundcolor=bg,font=\color{orange}\sffamily, fontcolor=white
]{definition}

\usepackage{empheq}
\usepackage[most]{tcolorbox}

\newtcbox{\mymath}[1][]{%
    nobeforeafter, math upper, tcbox raise base,
    enhanced, colframe=blue!30!black,
    colback=red!10, boxrule=1pt,
    #1}

\usepackage{unixode}


\DeclareMathOperator{\ord}{ord}
\DeclareMathOperator{\orb}{orb}
\DeclareMathOperator{\stab}{stab}
\DeclareMathOperator{\Stab}{stab}
\DeclareMathOperator{\ppcm}{ppcm}
\DeclareMathOperator{\conj}{Conj}
\DeclareMathOperator{\End}{End}
\DeclareMathOperator{\rot}{rot}
\DeclareMathOperator{\trs}{trace}
\DeclareMathOperator{\Ind}{Ind}
\DeclareMathOperator{\mat}{Mat}
\DeclareMathOperator{\id}{Id}
\DeclareMathOperator{\vect}{vect}
\DeclareMathOperator{\img}{img}
\DeclareMathOperator{\cov}{Cov}
\DeclareMathOperator{\dist}{dist}
\DeclareMathOperator{\irr}{Irr}
\DeclareMathOperator{\image}{Im}
\DeclareMathOperator{\pd}{\partial}
\DeclareMathOperator{\epi}{epi}
\DeclareMathOperator{\Argmin}{Argmin}
\DeclareMathOperator{\dom}{dom}
\DeclareMathOperator{\proj}{proj}
\DeclareMathOperator{\ctg}{ctg}
\DeclareMathOperator{\supp}{supp}
\DeclareMathOperator{\argmin}{argmin}
\DeclareMathOperator{\mult}{mult}
\DeclareMathOperator{\ch}{ch}
\DeclareMathOperator{\sh}{sh}
\DeclareMathOperator{\rang}{rang}
\DeclareMathOperator{\diam}{diam}
\DeclareMathOperator{\Epigraphe}{Epigraphe}




\usepackage{xcolor}
\everymath{\color{blue}}
%\everymath{\color[rgb]{0,1,1}}
%\pagecolor[rgb]{0,0,0.5}


\newcommand*{\pdtest}[3][]{\ensuremath{\frac{\partial^{#1} #2}{\partial #3}}}

\newcommand*{\deffunc}[6][]{\ensuremath{
\begin{array}{rcl}
#2 : #3 &\rightarrow& #4\\
#5 &\mapsto& #6
\end{array}
}}

\newcommand{\eqcolon}{\mathrel{\resizebox{\widthof{$\mathord{=}$}}{\height}{ $\!\!=\!\!\resizebox{1.2\width}{0.8\height}{\raisebox{0.23ex}{$\mathop{:}$}}\!\!$ }}}
\newcommand{\coloneq}{\mathrel{\resizebox{\widthof{$\mathord{=}$}}{\height}{ $\!\!\resizebox{1.2\width}{0.8\height}{\raisebox{0.23ex}{$\mathop{:}$}}\!\!=\!\!$ }}}
\newcommand{\eqcolonl}{\ensuremath{\mathrel{=\!\!\mathop{:}}}}
\newcommand{\coloneql}{\ensuremath{\mathrel{\mathop{:} \!\! =}}}
\newcommand{\vc}[1]{% inline column vector
  \left(\begin{smallmatrix}#1\end{smallmatrix}\right)%
}
\newcommand{\vr}[1]{% inline row vector
  \begin{smallmatrix}(\,#1\,)\end{smallmatrix}%
}
\makeatletter
\newcommand*{\defeq}{\ =\mathrel{\rlap{%
                     \raisebox{0.3ex}{$\m@th\cdot$}}%
                     \raisebox{-0.3ex}{$\m@th\cdot$}}%
                     }
\makeatother

\newcommand{\mathcircle}[1]{% inline row vector
 \overset{\circ}{#1}
}
\newcommand{\ulim}{% low limit
 \underline{\lim}
}
\newcommand{\ssi}{% iff
\iff
}
\newcommand{\ps}[2]{
\expval{#1 | #2}
}
\newcommand{\df}[1]{
\mqty{#1}
}
\newcommand{\n}[1]{
\norm{#1}
}
\newcommand{\sys}[1]{
\left\{\smqty{#1}\right.
}


\newcommand{\eqdef}{\ensuremath{\overset{\text{def}}=}}


\def\Circlearrowright{\ensuremath{%
  \rotatebox[origin=c]{230}{$\circlearrowright$}}}

\newcommand\ct[1]{\text{\rmfamily\upshape #1}}
\newcommand\question[1]{ {\color{red} ...!? \small #1}}
\newcommand\caz[1]{\left\{\begin{array} #1 \end{array}\right.}
\newcommand\const{\text{\rmfamily\upshape const}}
\newcommand\toP{ \overset{\pro}{\to}}
\newcommand\toPP{ \overset{\text{PP}}{\to}}
\newcommand{\oeq}{\mathrel{\text{\textcircled{$=$}}}}





\usepackage{xcolor}
% \usepackage[normalem]{ulem}
\usepackage{lipsum}
\makeatletter
% \newcommand\colorwave[1][blue]{\bgroup \markoverwith{\lower3.5\p@\hbox{\sixly \textcolor{#1}{\char58}}}\ULon}
%\font\sixly=lasy6 % does not re-load if already loaded, so no memory problem.

\newmdtheoremenv[
linewidth= 1pt,linecolor= blue,%
leftmargin=20,rightmargin=20,innertopmargin=0pt, innerrightmargin=40,%
tikzsetting = { draw=lightgray, line width = 0.3pt,dashed,%
dash pattern = on 15pt off 3pt},%
splittopskip=\topskip,skipbelow=\baselineskip,%
skipabove=\baselineskip,ntheorem,roundcorner=0pt,
% backgroundcolor=pagebg,font=\color{orange}\sffamily, fontcolor=white
]{examplebox}{Exemple}[section]



\newcommand\R{\mathbb{R}}
\newcommand\Z{\mathbb{Z}}
\newcommand\N{\mathbb{N}}
\newcommand\E{\mathbb{E}}
\newcommand\F{\mathcal{F}}
\newcommand\cH{\mathcal{H}}
\newcommand\V{\mathbb{V}}
\newcommand\dmo{ ^{-1} }
\newcommand\kapa{\kappa}
\newcommand\im{Im}
\newcommand\hs{\mathcal{H}}





\usepackage{soul}

\makeatletter
\newcommand*{\whiten}[1]{\llap{\textcolor{white}{{\the\SOUL@token}}\hspace{#1pt}}}
\DeclareRobustCommand*\myul{%
    \def\SOUL@everyspace{\underline{\space}\kern\z@}%
    \def\SOUL@everytoken{%
     \setbox0=\hbox{\the\SOUL@token}%
     \ifdim\dp0>\z@
        \raisebox{\dp0}{\underline{\phantom{\the\SOUL@token}}}%
        \whiten{1}\whiten{0}%
        \whiten{-1}\whiten{-2}%
        \llap{\the\SOUL@token}%
     \else
        \underline{\the\SOUL@token}%
     \fi}%
\SOUL@}
\makeatother

\newcommand*{\demp}{\fontfamily{lmtt}\selectfont}

\DeclareTextFontCommand{\textdemp}{\demp}

\begin{document}

\ifcomment
Multiline
comment
\fi
\ifcomment
\myul{Typesetting test}
% \color[rgb]{1,1,1}
$∑_i^n≠ 60º±∞π∆¬≈√j∫h≤≥µ$

$\CR \R\pro\ind\pro\gS\pro
\mqty[a&b\\c&d]$
$\pro\mathbb{P}$
$\dd{x}$

  \[
    \alpha(x)=\left\{
                \begin{array}{ll}
                  x\\
                  \frac{1}{1+e^{-kx}}\\
                  \frac{e^x-e^{-x}}{e^x+e^{-x}}
                \end{array}
              \right.
  \]

  $\expval{x}$
  
  $\chi_\rho(ghg\dmo)=\Tr(\rho_{ghg\dmo})=\Tr(\rho_g\circ\rho_h\circ\rho\dmo_g)=\Tr(\rho_h)\overset{\mbox{\scalebox{0.5}{$\Tr(AB)=\Tr(BA)$}}}{=}\chi_\rho(h)$
  	$\mathop{\oplus}_{\substack{x\in X}}$

$\mat(\rho_g)=(a_{ij}(g))_{\scriptsize \substack{1\leq i\leq d \\ 1\leq j\leq d}}$ et $\mat(\rho'_g)=(a'_{ij}(g))_{\scriptsize \substack{1\leq i'\leq d' \\ 1\leq j'\leq d'}}$



\[\int_a^b{\mathbb{R}^2}g(u, v)\dd{P_{XY}}(u, v)=\iint g(u,v) f_{XY}(u, v)\dd \lambda(u) \dd \lambda(v)\]
$$\lim_{x\to\infty} f(x)$$	
$$\iiiint_V \mu(t,u,v,w) \,dt\,du\,dv\,dw$$
$$\sum_{n=1}^{\infty} 2^{-n} = 1$$	
\begin{definition}
	Si $X$ et $Y$ sont 2 v.a. ou definit la \textsc{Covariance} entre $X$ et $Y$ comme
	$\cov(X,Y)\overset{\text{def}}{=}\E\left[(X-\E(X))(Y-\E(Y))\right]=\E(XY)-\E(X)\E(Y)$.
\end{definition}
\fi
\pagebreak

% \tableofcontents

% insert your code here
%\input{./algebra/main.tex}
%\input{./geometrie-differentielle/main.tex}
%\input{./probabilite/main.tex}
%\input{./analyse-fonctionnelle/main.tex}
% \input{./Analyse-convexe-et-dualite-en-optimisation/main.tex}
%\input{./tikz/main.tex}
%\input{./Theorie-du-distributions/main.tex}
%\input{./optimisation/mine.tex}
 \input{./modelisation/main.tex}

% yves.aubry@univ-tln.fr : algebra

\end{document}

%% !TEX encoding = UTF-8 Unicode
% !TEX TS-program = xelatex

\documentclass[french]{report}

%\usepackage[utf8]{inputenc}
%\usepackage[T1]{fontenc}
\usepackage{babel}


\newif\ifcomment
%\commenttrue # Show comments

\usepackage{physics}
\usepackage{amssymb}


\usepackage{amsthm}
% \usepackage{thmtools}
\usepackage{mathtools}
\usepackage{amsfonts}

\usepackage{color}

\usepackage{tikz}

\usepackage{geometry}
\geometry{a5paper, margin=0.1in, right=1cm}

\usepackage{dsfont}

\usepackage{graphicx}
\graphicspath{ {images/} }

\usepackage{faktor}

\usepackage{IEEEtrantools}
\usepackage{enumerate}   
\usepackage[PostScript=dvips]{"/Users/aware/Documents/Courses/diagrams"}


\newtheorem{theorem}{Théorème}[section]
\renewcommand{\thetheorem}{\arabic{theorem}}
\newtheorem{lemme}{Lemme}[section]
\renewcommand{\thelemme}{\arabic{lemme}}
\newtheorem{proposition}{Proposition}[section]
\renewcommand{\theproposition}{\arabic{proposition}}
\newtheorem{notations}{Notations}[section]
\newtheorem{problem}{Problème}[section]
\newtheorem{corollary}{Corollaire}[theorem]
\renewcommand{\thecorollary}{\arabic{corollary}}
\newtheorem{property}{Propriété}[section]
\newtheorem{objective}{Objectif}[section]

\theoremstyle{definition}
\newtheorem{definition}{Définition}[section]
\renewcommand{\thedefinition}{\arabic{definition}}
\newtheorem{exercise}{Exercice}[chapter]
\renewcommand{\theexercise}{\arabic{exercise}}
\newtheorem{example}{Exemple}[chapter]
\renewcommand{\theexample}{\arabic{example}}
\newtheorem*{solution}{Solution}
\newtheorem*{application}{Application}
\newtheorem*{notation}{Notation}
\newtheorem*{vocabulary}{Vocabulaire}
\newtheorem*{properties}{Propriétés}



\theoremstyle{remark}
\newtheorem*{remark}{Remarque}
\newtheorem*{rappel}{Rappel}


\usepackage{etoolbox}
\AtBeginEnvironment{exercise}{\small}
\AtBeginEnvironment{example}{\small}

\usepackage{cases}
\usepackage[red]{mypack}

\usepackage[framemethod=TikZ]{mdframed}

\definecolor{bg}{rgb}{0.4,0.25,0.95}
\definecolor{pagebg}{rgb}{0,0,0.5}
\surroundwithmdframed[
   topline=false,
   rightline=false,
   bottomline=false,
   leftmargin=\parindent,
   skipabove=8pt,
   skipbelow=8pt,
   linecolor=blue,
   innerbottommargin=10pt,
   % backgroundcolor=bg,font=\color{orange}\sffamily, fontcolor=white
]{definition}

\usepackage{empheq}
\usepackage[most]{tcolorbox}

\newtcbox{\mymath}[1][]{%
    nobeforeafter, math upper, tcbox raise base,
    enhanced, colframe=blue!30!black,
    colback=red!10, boxrule=1pt,
    #1}

\usepackage{unixode}


\DeclareMathOperator{\ord}{ord}
\DeclareMathOperator{\orb}{orb}
\DeclareMathOperator{\stab}{stab}
\DeclareMathOperator{\Stab}{stab}
\DeclareMathOperator{\ppcm}{ppcm}
\DeclareMathOperator{\conj}{Conj}
\DeclareMathOperator{\End}{End}
\DeclareMathOperator{\rot}{rot}
\DeclareMathOperator{\trs}{trace}
\DeclareMathOperator{\Ind}{Ind}
\DeclareMathOperator{\mat}{Mat}
\DeclareMathOperator{\id}{Id}
\DeclareMathOperator{\vect}{vect}
\DeclareMathOperator{\img}{img}
\DeclareMathOperator{\cov}{Cov}
\DeclareMathOperator{\dist}{dist}
\DeclareMathOperator{\irr}{Irr}
\DeclareMathOperator{\image}{Im}
\DeclareMathOperator{\pd}{\partial}
\DeclareMathOperator{\epi}{epi}
\DeclareMathOperator{\Argmin}{Argmin}
\DeclareMathOperator{\dom}{dom}
\DeclareMathOperator{\proj}{proj}
\DeclareMathOperator{\ctg}{ctg}
\DeclareMathOperator{\supp}{supp}
\DeclareMathOperator{\argmin}{argmin}
\DeclareMathOperator{\mult}{mult}
\DeclareMathOperator{\ch}{ch}
\DeclareMathOperator{\sh}{sh}
\DeclareMathOperator{\rang}{rang}
\DeclareMathOperator{\diam}{diam}
\DeclareMathOperator{\Epigraphe}{Epigraphe}




\usepackage{xcolor}
\everymath{\color{blue}}
%\everymath{\color[rgb]{0,1,1}}
%\pagecolor[rgb]{0,0,0.5}


\newcommand*{\pdtest}[3][]{\ensuremath{\frac{\partial^{#1} #2}{\partial #3}}}

\newcommand*{\deffunc}[6][]{\ensuremath{
\begin{array}{rcl}
#2 : #3 &\rightarrow& #4\\
#5 &\mapsto& #6
\end{array}
}}

\newcommand{\eqcolon}{\mathrel{\resizebox{\widthof{$\mathord{=}$}}{\height}{ $\!\!=\!\!\resizebox{1.2\width}{0.8\height}{\raisebox{0.23ex}{$\mathop{:}$}}\!\!$ }}}
\newcommand{\coloneq}{\mathrel{\resizebox{\widthof{$\mathord{=}$}}{\height}{ $\!\!\resizebox{1.2\width}{0.8\height}{\raisebox{0.23ex}{$\mathop{:}$}}\!\!=\!\!$ }}}
\newcommand{\eqcolonl}{\ensuremath{\mathrel{=\!\!\mathop{:}}}}
\newcommand{\coloneql}{\ensuremath{\mathrel{\mathop{:} \!\! =}}}
\newcommand{\vc}[1]{% inline column vector
  \left(\begin{smallmatrix}#1\end{smallmatrix}\right)%
}
\newcommand{\vr}[1]{% inline row vector
  \begin{smallmatrix}(\,#1\,)\end{smallmatrix}%
}
\makeatletter
\newcommand*{\defeq}{\ =\mathrel{\rlap{%
                     \raisebox{0.3ex}{$\m@th\cdot$}}%
                     \raisebox{-0.3ex}{$\m@th\cdot$}}%
                     }
\makeatother

\newcommand{\mathcircle}[1]{% inline row vector
 \overset{\circ}{#1}
}
\newcommand{\ulim}{% low limit
 \underline{\lim}
}
\newcommand{\ssi}{% iff
\iff
}
\newcommand{\ps}[2]{
\expval{#1 | #2}
}
\newcommand{\df}[1]{
\mqty{#1}
}
\newcommand{\n}[1]{
\norm{#1}
}
\newcommand{\sys}[1]{
\left\{\smqty{#1}\right.
}


\newcommand{\eqdef}{\ensuremath{\overset{\text{def}}=}}


\def\Circlearrowright{\ensuremath{%
  \rotatebox[origin=c]{230}{$\circlearrowright$}}}

\newcommand\ct[1]{\text{\rmfamily\upshape #1}}
\newcommand\question[1]{ {\color{red} ...!? \small #1}}
\newcommand\caz[1]{\left\{\begin{array} #1 \end{array}\right.}
\newcommand\const{\text{\rmfamily\upshape const}}
\newcommand\toP{ \overset{\pro}{\to}}
\newcommand\toPP{ \overset{\text{PP}}{\to}}
\newcommand{\oeq}{\mathrel{\text{\textcircled{$=$}}}}





\usepackage{xcolor}
% \usepackage[normalem]{ulem}
\usepackage{lipsum}
\makeatletter
% \newcommand\colorwave[1][blue]{\bgroup \markoverwith{\lower3.5\p@\hbox{\sixly \textcolor{#1}{\char58}}}\ULon}
%\font\sixly=lasy6 % does not re-load if already loaded, so no memory problem.

\newmdtheoremenv[
linewidth= 1pt,linecolor= blue,%
leftmargin=20,rightmargin=20,innertopmargin=0pt, innerrightmargin=40,%
tikzsetting = { draw=lightgray, line width = 0.3pt,dashed,%
dash pattern = on 15pt off 3pt},%
splittopskip=\topskip,skipbelow=\baselineskip,%
skipabove=\baselineskip,ntheorem,roundcorner=0pt,
% backgroundcolor=pagebg,font=\color{orange}\sffamily, fontcolor=white
]{examplebox}{Exemple}[section]



\newcommand\R{\mathbb{R}}
\newcommand\Z{\mathbb{Z}}
\newcommand\N{\mathbb{N}}
\newcommand\E{\mathbb{E}}
\newcommand\F{\mathcal{F}}
\newcommand\cH{\mathcal{H}}
\newcommand\V{\mathbb{V}}
\newcommand\dmo{ ^{-1} }
\newcommand\kapa{\kappa}
\newcommand\im{Im}
\newcommand\hs{\mathcal{H}}





\usepackage{soul}

\makeatletter
\newcommand*{\whiten}[1]{\llap{\textcolor{white}{{\the\SOUL@token}}\hspace{#1pt}}}
\DeclareRobustCommand*\myul{%
    \def\SOUL@everyspace{\underline{\space}\kern\z@}%
    \def\SOUL@everytoken{%
     \setbox0=\hbox{\the\SOUL@token}%
     \ifdim\dp0>\z@
        \raisebox{\dp0}{\underline{\phantom{\the\SOUL@token}}}%
        \whiten{1}\whiten{0}%
        \whiten{-1}\whiten{-2}%
        \llap{\the\SOUL@token}%
     \else
        \underline{\the\SOUL@token}%
     \fi}%
\SOUL@}
\makeatother

\newcommand*{\demp}{\fontfamily{lmtt}\selectfont}

\DeclareTextFontCommand{\textdemp}{\demp}

\begin{document}

\ifcomment
Multiline
comment
\fi
\ifcomment
\myul{Typesetting test}
% \color[rgb]{1,1,1}
$∑_i^n≠ 60º±∞π∆¬≈√j∫h≤≥µ$

$\CR \R\pro\ind\pro\gS\pro
\mqty[a&b\\c&d]$
$\pro\mathbb{P}$
$\dd{x}$

  \[
    \alpha(x)=\left\{
                \begin{array}{ll}
                  x\\
                  \frac{1}{1+e^{-kx}}\\
                  \frac{e^x-e^{-x}}{e^x+e^{-x}}
                \end{array}
              \right.
  \]

  $\expval{x}$
  
  $\chi_\rho(ghg\dmo)=\Tr(\rho_{ghg\dmo})=\Tr(\rho_g\circ\rho_h\circ\rho\dmo_g)=\Tr(\rho_h)\overset{\mbox{\scalebox{0.5}{$\Tr(AB)=\Tr(BA)$}}}{=}\chi_\rho(h)$
  	$\mathop{\oplus}_{\substack{x\in X}}$

$\mat(\rho_g)=(a_{ij}(g))_{\scriptsize \substack{1\leq i\leq d \\ 1\leq j\leq d}}$ et $\mat(\rho'_g)=(a'_{ij}(g))_{\scriptsize \substack{1\leq i'\leq d' \\ 1\leq j'\leq d'}}$



\[\int_a^b{\mathbb{R}^2}g(u, v)\dd{P_{XY}}(u, v)=\iint g(u,v) f_{XY}(u, v)\dd \lambda(u) \dd \lambda(v)\]
$$\lim_{x\to\infty} f(x)$$	
$$\iiiint_V \mu(t,u,v,w) \,dt\,du\,dv\,dw$$
$$\sum_{n=1}^{\infty} 2^{-n} = 1$$	
\begin{definition}
	Si $X$ et $Y$ sont 2 v.a. ou definit la \textsc{Covariance} entre $X$ et $Y$ comme
	$\cov(X,Y)\overset{\text{def}}{=}\E\left[(X-\E(X))(Y-\E(Y))\right]=\E(XY)-\E(X)\E(Y)$.
\end{definition}
\fi
\pagebreak

% \tableofcontents

% insert your code here
%\input{./algebra/main.tex}
%\input{./geometrie-differentielle/main.tex}
%\input{./probabilite/main.tex}
%\input{./analyse-fonctionnelle/main.tex}
% \input{./Analyse-convexe-et-dualite-en-optimisation/main.tex}
%\input{./tikz/main.tex}
%\input{./Theorie-du-distributions/main.tex}
%\input{./optimisation/mine.tex}
 \input{./modelisation/main.tex}

% yves.aubry@univ-tln.fr : algebra

\end{document}

%\input{./optimisation/mine.tex}
 % !TEX encoding = UTF-8 Unicode
% !TEX TS-program = xelatex

\documentclass[french]{report}

%\usepackage[utf8]{inputenc}
%\usepackage[T1]{fontenc}
\usepackage{babel}


\newif\ifcomment
%\commenttrue # Show comments

\usepackage{physics}
\usepackage{amssymb}


\usepackage{amsthm}
% \usepackage{thmtools}
\usepackage{mathtools}
\usepackage{amsfonts}

\usepackage{color}

\usepackage{tikz}

\usepackage{geometry}
\geometry{a5paper, margin=0.1in, right=1cm}

\usepackage{dsfont}

\usepackage{graphicx}
\graphicspath{ {images/} }

\usepackage{faktor}

\usepackage{IEEEtrantools}
\usepackage{enumerate}   
\usepackage[PostScript=dvips]{"/Users/aware/Documents/Courses/diagrams"}


\newtheorem{theorem}{Théorème}[section]
\renewcommand{\thetheorem}{\arabic{theorem}}
\newtheorem{lemme}{Lemme}[section]
\renewcommand{\thelemme}{\arabic{lemme}}
\newtheorem{proposition}{Proposition}[section]
\renewcommand{\theproposition}{\arabic{proposition}}
\newtheorem{notations}{Notations}[section]
\newtheorem{problem}{Problème}[section]
\newtheorem{corollary}{Corollaire}[theorem]
\renewcommand{\thecorollary}{\arabic{corollary}}
\newtheorem{property}{Propriété}[section]
\newtheorem{objective}{Objectif}[section]

\theoremstyle{definition}
\newtheorem{definition}{Définition}[section]
\renewcommand{\thedefinition}{\arabic{definition}}
\newtheorem{exercise}{Exercice}[chapter]
\renewcommand{\theexercise}{\arabic{exercise}}
\newtheorem{example}{Exemple}[chapter]
\renewcommand{\theexample}{\arabic{example}}
\newtheorem*{solution}{Solution}
\newtheorem*{application}{Application}
\newtheorem*{notation}{Notation}
\newtheorem*{vocabulary}{Vocabulaire}
\newtheorem*{properties}{Propriétés}



\theoremstyle{remark}
\newtheorem*{remark}{Remarque}
\newtheorem*{rappel}{Rappel}


\usepackage{etoolbox}
\AtBeginEnvironment{exercise}{\small}
\AtBeginEnvironment{example}{\small}

\usepackage{cases}
\usepackage[red]{mypack}

\usepackage[framemethod=TikZ]{mdframed}

\definecolor{bg}{rgb}{0.4,0.25,0.95}
\definecolor{pagebg}{rgb}{0,0,0.5}
\surroundwithmdframed[
   topline=false,
   rightline=false,
   bottomline=false,
   leftmargin=\parindent,
   skipabove=8pt,
   skipbelow=8pt,
   linecolor=blue,
   innerbottommargin=10pt,
   % backgroundcolor=bg,font=\color{orange}\sffamily, fontcolor=white
]{definition}

\usepackage{empheq}
\usepackage[most]{tcolorbox}

\newtcbox{\mymath}[1][]{%
    nobeforeafter, math upper, tcbox raise base,
    enhanced, colframe=blue!30!black,
    colback=red!10, boxrule=1pt,
    #1}

\usepackage{unixode}


\DeclareMathOperator{\ord}{ord}
\DeclareMathOperator{\orb}{orb}
\DeclareMathOperator{\stab}{stab}
\DeclareMathOperator{\Stab}{stab}
\DeclareMathOperator{\ppcm}{ppcm}
\DeclareMathOperator{\conj}{Conj}
\DeclareMathOperator{\End}{End}
\DeclareMathOperator{\rot}{rot}
\DeclareMathOperator{\trs}{trace}
\DeclareMathOperator{\Ind}{Ind}
\DeclareMathOperator{\mat}{Mat}
\DeclareMathOperator{\id}{Id}
\DeclareMathOperator{\vect}{vect}
\DeclareMathOperator{\img}{img}
\DeclareMathOperator{\cov}{Cov}
\DeclareMathOperator{\dist}{dist}
\DeclareMathOperator{\irr}{Irr}
\DeclareMathOperator{\image}{Im}
\DeclareMathOperator{\pd}{\partial}
\DeclareMathOperator{\epi}{epi}
\DeclareMathOperator{\Argmin}{Argmin}
\DeclareMathOperator{\dom}{dom}
\DeclareMathOperator{\proj}{proj}
\DeclareMathOperator{\ctg}{ctg}
\DeclareMathOperator{\supp}{supp}
\DeclareMathOperator{\argmin}{argmin}
\DeclareMathOperator{\mult}{mult}
\DeclareMathOperator{\ch}{ch}
\DeclareMathOperator{\sh}{sh}
\DeclareMathOperator{\rang}{rang}
\DeclareMathOperator{\diam}{diam}
\DeclareMathOperator{\Epigraphe}{Epigraphe}




\usepackage{xcolor}
\everymath{\color{blue}}
%\everymath{\color[rgb]{0,1,1}}
%\pagecolor[rgb]{0,0,0.5}


\newcommand*{\pdtest}[3][]{\ensuremath{\frac{\partial^{#1} #2}{\partial #3}}}

\newcommand*{\deffunc}[6][]{\ensuremath{
\begin{array}{rcl}
#2 : #3 &\rightarrow& #4\\
#5 &\mapsto& #6
\end{array}
}}

\newcommand{\eqcolon}{\mathrel{\resizebox{\widthof{$\mathord{=}$}}{\height}{ $\!\!=\!\!\resizebox{1.2\width}{0.8\height}{\raisebox{0.23ex}{$\mathop{:}$}}\!\!$ }}}
\newcommand{\coloneq}{\mathrel{\resizebox{\widthof{$\mathord{=}$}}{\height}{ $\!\!\resizebox{1.2\width}{0.8\height}{\raisebox{0.23ex}{$\mathop{:}$}}\!\!=\!\!$ }}}
\newcommand{\eqcolonl}{\ensuremath{\mathrel{=\!\!\mathop{:}}}}
\newcommand{\coloneql}{\ensuremath{\mathrel{\mathop{:} \!\! =}}}
\newcommand{\vc}[1]{% inline column vector
  \left(\begin{smallmatrix}#1\end{smallmatrix}\right)%
}
\newcommand{\vr}[1]{% inline row vector
  \begin{smallmatrix}(\,#1\,)\end{smallmatrix}%
}
\makeatletter
\newcommand*{\defeq}{\ =\mathrel{\rlap{%
                     \raisebox{0.3ex}{$\m@th\cdot$}}%
                     \raisebox{-0.3ex}{$\m@th\cdot$}}%
                     }
\makeatother

\newcommand{\mathcircle}[1]{% inline row vector
 \overset{\circ}{#1}
}
\newcommand{\ulim}{% low limit
 \underline{\lim}
}
\newcommand{\ssi}{% iff
\iff
}
\newcommand{\ps}[2]{
\expval{#1 | #2}
}
\newcommand{\df}[1]{
\mqty{#1}
}
\newcommand{\n}[1]{
\norm{#1}
}
\newcommand{\sys}[1]{
\left\{\smqty{#1}\right.
}


\newcommand{\eqdef}{\ensuremath{\overset{\text{def}}=}}


\def\Circlearrowright{\ensuremath{%
  \rotatebox[origin=c]{230}{$\circlearrowright$}}}

\newcommand\ct[1]{\text{\rmfamily\upshape #1}}
\newcommand\question[1]{ {\color{red} ...!? \small #1}}
\newcommand\caz[1]{\left\{\begin{array} #1 \end{array}\right.}
\newcommand\const{\text{\rmfamily\upshape const}}
\newcommand\toP{ \overset{\pro}{\to}}
\newcommand\toPP{ \overset{\text{PP}}{\to}}
\newcommand{\oeq}{\mathrel{\text{\textcircled{$=$}}}}





\usepackage{xcolor}
% \usepackage[normalem]{ulem}
\usepackage{lipsum}
\makeatletter
% \newcommand\colorwave[1][blue]{\bgroup \markoverwith{\lower3.5\p@\hbox{\sixly \textcolor{#1}{\char58}}}\ULon}
%\font\sixly=lasy6 % does not re-load if already loaded, so no memory problem.

\newmdtheoremenv[
linewidth= 1pt,linecolor= blue,%
leftmargin=20,rightmargin=20,innertopmargin=0pt, innerrightmargin=40,%
tikzsetting = { draw=lightgray, line width = 0.3pt,dashed,%
dash pattern = on 15pt off 3pt},%
splittopskip=\topskip,skipbelow=\baselineskip,%
skipabove=\baselineskip,ntheorem,roundcorner=0pt,
% backgroundcolor=pagebg,font=\color{orange}\sffamily, fontcolor=white
]{examplebox}{Exemple}[section]



\newcommand\R{\mathbb{R}}
\newcommand\Z{\mathbb{Z}}
\newcommand\N{\mathbb{N}}
\newcommand\E{\mathbb{E}}
\newcommand\F{\mathcal{F}}
\newcommand\cH{\mathcal{H}}
\newcommand\V{\mathbb{V}}
\newcommand\dmo{ ^{-1} }
\newcommand\kapa{\kappa}
\newcommand\im{Im}
\newcommand\hs{\mathcal{H}}





\usepackage{soul}

\makeatletter
\newcommand*{\whiten}[1]{\llap{\textcolor{white}{{\the\SOUL@token}}\hspace{#1pt}}}
\DeclareRobustCommand*\myul{%
    \def\SOUL@everyspace{\underline{\space}\kern\z@}%
    \def\SOUL@everytoken{%
     \setbox0=\hbox{\the\SOUL@token}%
     \ifdim\dp0>\z@
        \raisebox{\dp0}{\underline{\phantom{\the\SOUL@token}}}%
        \whiten{1}\whiten{0}%
        \whiten{-1}\whiten{-2}%
        \llap{\the\SOUL@token}%
     \else
        \underline{\the\SOUL@token}%
     \fi}%
\SOUL@}
\makeatother

\newcommand*{\demp}{\fontfamily{lmtt}\selectfont}

\DeclareTextFontCommand{\textdemp}{\demp}

\begin{document}

\ifcomment
Multiline
comment
\fi
\ifcomment
\myul{Typesetting test}
% \color[rgb]{1,1,1}
$∑_i^n≠ 60º±∞π∆¬≈√j∫h≤≥µ$

$\CR \R\pro\ind\pro\gS\pro
\mqty[a&b\\c&d]$
$\pro\mathbb{P}$
$\dd{x}$

  \[
    \alpha(x)=\left\{
                \begin{array}{ll}
                  x\\
                  \frac{1}{1+e^{-kx}}\\
                  \frac{e^x-e^{-x}}{e^x+e^{-x}}
                \end{array}
              \right.
  \]

  $\expval{x}$
  
  $\chi_\rho(ghg\dmo)=\Tr(\rho_{ghg\dmo})=\Tr(\rho_g\circ\rho_h\circ\rho\dmo_g)=\Tr(\rho_h)\overset{\mbox{\scalebox{0.5}{$\Tr(AB)=\Tr(BA)$}}}{=}\chi_\rho(h)$
  	$\mathop{\oplus}_{\substack{x\in X}}$

$\mat(\rho_g)=(a_{ij}(g))_{\scriptsize \substack{1\leq i\leq d \\ 1\leq j\leq d}}$ et $\mat(\rho'_g)=(a'_{ij}(g))_{\scriptsize \substack{1\leq i'\leq d' \\ 1\leq j'\leq d'}}$



\[\int_a^b{\mathbb{R}^2}g(u, v)\dd{P_{XY}}(u, v)=\iint g(u,v) f_{XY}(u, v)\dd \lambda(u) \dd \lambda(v)\]
$$\lim_{x\to\infty} f(x)$$	
$$\iiiint_V \mu(t,u,v,w) \,dt\,du\,dv\,dw$$
$$\sum_{n=1}^{\infty} 2^{-n} = 1$$	
\begin{definition}
	Si $X$ et $Y$ sont 2 v.a. ou definit la \textsc{Covariance} entre $X$ et $Y$ comme
	$\cov(X,Y)\overset{\text{def}}{=}\E\left[(X-\E(X))(Y-\E(Y))\right]=\E(XY)-\E(X)\E(Y)$.
\end{definition}
\fi
\pagebreak

% \tableofcontents

% insert your code here
%\input{./algebra/main.tex}
%\input{./geometrie-differentielle/main.tex}
%\input{./probabilite/main.tex}
%\input{./analyse-fonctionnelle/main.tex}
% \input{./Analyse-convexe-et-dualite-en-optimisation/main.tex}
%\input{./tikz/main.tex}
%\input{./Theorie-du-distributions/main.tex}
%\input{./optimisation/mine.tex}
 \input{./modelisation/main.tex}

% yves.aubry@univ-tln.fr : algebra

\end{document}


% yves.aubry@univ-tln.fr : algebra

\end{document}

%% !TEX encoding = UTF-8 Unicode
% !TEX TS-program = xelatex

\documentclass[french]{report}

%\usepackage[utf8]{inputenc}
%\usepackage[T1]{fontenc}
\usepackage{babel}


\newif\ifcomment
%\commenttrue # Show comments

\usepackage{physics}
\usepackage{amssymb}


\usepackage{amsthm}
% \usepackage{thmtools}
\usepackage{mathtools}
\usepackage{amsfonts}

\usepackage{color}

\usepackage{tikz}

\usepackage{geometry}
\geometry{a5paper, margin=0.1in, right=1cm}

\usepackage{dsfont}

\usepackage{graphicx}
\graphicspath{ {images/} }

\usepackage{faktor}

\usepackage{IEEEtrantools}
\usepackage{enumerate}   
\usepackage[PostScript=dvips]{"/Users/aware/Documents/Courses/diagrams"}


\newtheorem{theorem}{Théorème}[section]
\renewcommand{\thetheorem}{\arabic{theorem}}
\newtheorem{lemme}{Lemme}[section]
\renewcommand{\thelemme}{\arabic{lemme}}
\newtheorem{proposition}{Proposition}[section]
\renewcommand{\theproposition}{\arabic{proposition}}
\newtheorem{notations}{Notations}[section]
\newtheorem{problem}{Problème}[section]
\newtheorem{corollary}{Corollaire}[theorem]
\renewcommand{\thecorollary}{\arabic{corollary}}
\newtheorem{property}{Propriété}[section]
\newtheorem{objective}{Objectif}[section]

\theoremstyle{definition}
\newtheorem{definition}{Définition}[section]
\renewcommand{\thedefinition}{\arabic{definition}}
\newtheorem{exercise}{Exercice}[chapter]
\renewcommand{\theexercise}{\arabic{exercise}}
\newtheorem{example}{Exemple}[chapter]
\renewcommand{\theexample}{\arabic{example}}
\newtheorem*{solution}{Solution}
\newtheorem*{application}{Application}
\newtheorem*{notation}{Notation}
\newtheorem*{vocabulary}{Vocabulaire}
\newtheorem*{properties}{Propriétés}



\theoremstyle{remark}
\newtheorem*{remark}{Remarque}
\newtheorem*{rappel}{Rappel}


\usepackage{etoolbox}
\AtBeginEnvironment{exercise}{\small}
\AtBeginEnvironment{example}{\small}

\usepackage{cases}
\usepackage[red]{mypack}

\usepackage[framemethod=TikZ]{mdframed}

\definecolor{bg}{rgb}{0.4,0.25,0.95}
\definecolor{pagebg}{rgb}{0,0,0.5}
\surroundwithmdframed[
   topline=false,
   rightline=false,
   bottomline=false,
   leftmargin=\parindent,
   skipabove=8pt,
   skipbelow=8pt,
   linecolor=blue,
   innerbottommargin=10pt,
   % backgroundcolor=bg,font=\color{orange}\sffamily, fontcolor=white
]{definition}

\usepackage{empheq}
\usepackage[most]{tcolorbox}

\newtcbox{\mymath}[1][]{%
    nobeforeafter, math upper, tcbox raise base,
    enhanced, colframe=blue!30!black,
    colback=red!10, boxrule=1pt,
    #1}

\usepackage{unixode}


\DeclareMathOperator{\ord}{ord}
\DeclareMathOperator{\orb}{orb}
\DeclareMathOperator{\stab}{stab}
\DeclareMathOperator{\Stab}{stab}
\DeclareMathOperator{\ppcm}{ppcm}
\DeclareMathOperator{\conj}{Conj}
\DeclareMathOperator{\End}{End}
\DeclareMathOperator{\rot}{rot}
\DeclareMathOperator{\trs}{trace}
\DeclareMathOperator{\Ind}{Ind}
\DeclareMathOperator{\mat}{Mat}
\DeclareMathOperator{\id}{Id}
\DeclareMathOperator{\vect}{vect}
\DeclareMathOperator{\img}{img}
\DeclareMathOperator{\cov}{Cov}
\DeclareMathOperator{\dist}{dist}
\DeclareMathOperator{\irr}{Irr}
\DeclareMathOperator{\image}{Im}
\DeclareMathOperator{\pd}{\partial}
\DeclareMathOperator{\epi}{epi}
\DeclareMathOperator{\Argmin}{Argmin}
\DeclareMathOperator{\dom}{dom}
\DeclareMathOperator{\proj}{proj}
\DeclareMathOperator{\ctg}{ctg}
\DeclareMathOperator{\supp}{supp}
\DeclareMathOperator{\argmin}{argmin}
\DeclareMathOperator{\mult}{mult}
\DeclareMathOperator{\ch}{ch}
\DeclareMathOperator{\sh}{sh}
\DeclareMathOperator{\rang}{rang}
\DeclareMathOperator{\diam}{diam}
\DeclareMathOperator{\Epigraphe}{Epigraphe}




\usepackage{xcolor}
\everymath{\color{blue}}
%\everymath{\color[rgb]{0,1,1}}
%\pagecolor[rgb]{0,0,0.5}


\newcommand*{\pdtest}[3][]{\ensuremath{\frac{\partial^{#1} #2}{\partial #3}}}

\newcommand*{\deffunc}[6][]{\ensuremath{
\begin{array}{rcl}
#2 : #3 &\rightarrow& #4\\
#5 &\mapsto& #6
\end{array}
}}

\newcommand{\eqcolon}{\mathrel{\resizebox{\widthof{$\mathord{=}$}}{\height}{ $\!\!=\!\!\resizebox{1.2\width}{0.8\height}{\raisebox{0.23ex}{$\mathop{:}$}}\!\!$ }}}
\newcommand{\coloneq}{\mathrel{\resizebox{\widthof{$\mathord{=}$}}{\height}{ $\!\!\resizebox{1.2\width}{0.8\height}{\raisebox{0.23ex}{$\mathop{:}$}}\!\!=\!\!$ }}}
\newcommand{\eqcolonl}{\ensuremath{\mathrel{=\!\!\mathop{:}}}}
\newcommand{\coloneql}{\ensuremath{\mathrel{\mathop{:} \!\! =}}}
\newcommand{\vc}[1]{% inline column vector
  \left(\begin{smallmatrix}#1\end{smallmatrix}\right)%
}
\newcommand{\vr}[1]{% inline row vector
  \begin{smallmatrix}(\,#1\,)\end{smallmatrix}%
}
\makeatletter
\newcommand*{\defeq}{\ =\mathrel{\rlap{%
                     \raisebox{0.3ex}{$\m@th\cdot$}}%
                     \raisebox{-0.3ex}{$\m@th\cdot$}}%
                     }
\makeatother

\newcommand{\mathcircle}[1]{% inline row vector
 \overset{\circ}{#1}
}
\newcommand{\ulim}{% low limit
 \underline{\lim}
}
\newcommand{\ssi}{% iff
\iff
}
\newcommand{\ps}[2]{
\expval{#1 | #2}
}
\newcommand{\df}[1]{
\mqty{#1}
}
\newcommand{\n}[1]{
\norm{#1}
}
\newcommand{\sys}[1]{
\left\{\smqty{#1}\right.
}


\newcommand{\eqdef}{\ensuremath{\overset{\text{def}}=}}


\def\Circlearrowright{\ensuremath{%
  \rotatebox[origin=c]{230}{$\circlearrowright$}}}

\newcommand\ct[1]{\text{\rmfamily\upshape #1}}
\newcommand\question[1]{ {\color{red} ...!? \small #1}}
\newcommand\caz[1]{\left\{\begin{array} #1 \end{array}\right.}
\newcommand\const{\text{\rmfamily\upshape const}}
\newcommand\toP{ \overset{\pro}{\to}}
\newcommand\toPP{ \overset{\text{PP}}{\to}}
\newcommand{\oeq}{\mathrel{\text{\textcircled{$=$}}}}





\usepackage{xcolor}
% \usepackage[normalem]{ulem}
\usepackage{lipsum}
\makeatletter
% \newcommand\colorwave[1][blue]{\bgroup \markoverwith{\lower3.5\p@\hbox{\sixly \textcolor{#1}{\char58}}}\ULon}
%\font\sixly=lasy6 % does not re-load if already loaded, so no memory problem.

\newmdtheoremenv[
linewidth= 1pt,linecolor= blue,%
leftmargin=20,rightmargin=20,innertopmargin=0pt, innerrightmargin=40,%
tikzsetting = { draw=lightgray, line width = 0.3pt,dashed,%
dash pattern = on 15pt off 3pt},%
splittopskip=\topskip,skipbelow=\baselineskip,%
skipabove=\baselineskip,ntheorem,roundcorner=0pt,
% backgroundcolor=pagebg,font=\color{orange}\sffamily, fontcolor=white
]{examplebox}{Exemple}[section]



\newcommand\R{\mathbb{R}}
\newcommand\Z{\mathbb{Z}}
\newcommand\N{\mathbb{N}}
\newcommand\E{\mathbb{E}}
\newcommand\F{\mathcal{F}}
\newcommand\cH{\mathcal{H}}
\newcommand\V{\mathbb{V}}
\newcommand\dmo{ ^{-1} }
\newcommand\kapa{\kappa}
\newcommand\im{Im}
\newcommand\hs{\mathcal{H}}





\usepackage{soul}

\makeatletter
\newcommand*{\whiten}[1]{\llap{\textcolor{white}{{\the\SOUL@token}}\hspace{#1pt}}}
\DeclareRobustCommand*\myul{%
    \def\SOUL@everyspace{\underline{\space}\kern\z@}%
    \def\SOUL@everytoken{%
     \setbox0=\hbox{\the\SOUL@token}%
     \ifdim\dp0>\z@
        \raisebox{\dp0}{\underline{\phantom{\the\SOUL@token}}}%
        \whiten{1}\whiten{0}%
        \whiten{-1}\whiten{-2}%
        \llap{\the\SOUL@token}%
     \else
        \underline{\the\SOUL@token}%
     \fi}%
\SOUL@}
\makeatother

\newcommand*{\demp}{\fontfamily{lmtt}\selectfont}

\DeclareTextFontCommand{\textdemp}{\demp}

\begin{document}

\ifcomment
Multiline
comment
\fi
\ifcomment
\myul{Typesetting test}
% \color[rgb]{1,1,1}
$∑_i^n≠ 60º±∞π∆¬≈√j∫h≤≥µ$

$\CR \R\pro\ind\pro\gS\pro
\mqty[a&b\\c&d]$
$\pro\mathbb{P}$
$\dd{x}$

  \[
    \alpha(x)=\left\{
                \begin{array}{ll}
                  x\\
                  \frac{1}{1+e^{-kx}}\\
                  \frac{e^x-e^{-x}}{e^x+e^{-x}}
                \end{array}
              \right.
  \]

  $\expval{x}$
  
  $\chi_\rho(ghg\dmo)=\Tr(\rho_{ghg\dmo})=\Tr(\rho_g\circ\rho_h\circ\rho\dmo_g)=\Tr(\rho_h)\overset{\mbox{\scalebox{0.5}{$\Tr(AB)=\Tr(BA)$}}}{=}\chi_\rho(h)$
  	$\mathop{\oplus}_{\substack{x\in X}}$

$\mat(\rho_g)=(a_{ij}(g))_{\scriptsize \substack{1\leq i\leq d \\ 1\leq j\leq d}}$ et $\mat(\rho'_g)=(a'_{ij}(g))_{\scriptsize \substack{1\leq i'\leq d' \\ 1\leq j'\leq d'}}$



\[\int_a^b{\mathbb{R}^2}g(u, v)\dd{P_{XY}}(u, v)=\iint g(u,v) f_{XY}(u, v)\dd \lambda(u) \dd \lambda(v)\]
$$\lim_{x\to\infty} f(x)$$	
$$\iiiint_V \mu(t,u,v,w) \,dt\,du\,dv\,dw$$
$$\sum_{n=1}^{\infty} 2^{-n} = 1$$	
\begin{definition}
	Si $X$ et $Y$ sont 2 v.a. ou definit la \textsc{Covariance} entre $X$ et $Y$ comme
	$\cov(X,Y)\overset{\text{def}}{=}\E\left[(X-\E(X))(Y-\E(Y))\right]=\E(XY)-\E(X)\E(Y)$.
\end{definition}
\fi
\pagebreak

% \tableofcontents

% insert your code here
%% !TEX encoding = UTF-8 Unicode
% !TEX TS-program = xelatex

\documentclass[french]{report}

%\usepackage[utf8]{inputenc}
%\usepackage[T1]{fontenc}
\usepackage{babel}


\newif\ifcomment
%\commenttrue # Show comments

\usepackage{physics}
\usepackage{amssymb}


\usepackage{amsthm}
% \usepackage{thmtools}
\usepackage{mathtools}
\usepackage{amsfonts}

\usepackage{color}

\usepackage{tikz}

\usepackage{geometry}
\geometry{a5paper, margin=0.1in, right=1cm}

\usepackage{dsfont}

\usepackage{graphicx}
\graphicspath{ {images/} }

\usepackage{faktor}

\usepackage{IEEEtrantools}
\usepackage{enumerate}   
\usepackage[PostScript=dvips]{"/Users/aware/Documents/Courses/diagrams"}


\newtheorem{theorem}{Théorème}[section]
\renewcommand{\thetheorem}{\arabic{theorem}}
\newtheorem{lemme}{Lemme}[section]
\renewcommand{\thelemme}{\arabic{lemme}}
\newtheorem{proposition}{Proposition}[section]
\renewcommand{\theproposition}{\arabic{proposition}}
\newtheorem{notations}{Notations}[section]
\newtheorem{problem}{Problème}[section]
\newtheorem{corollary}{Corollaire}[theorem]
\renewcommand{\thecorollary}{\arabic{corollary}}
\newtheorem{property}{Propriété}[section]
\newtheorem{objective}{Objectif}[section]

\theoremstyle{definition}
\newtheorem{definition}{Définition}[section]
\renewcommand{\thedefinition}{\arabic{definition}}
\newtheorem{exercise}{Exercice}[chapter]
\renewcommand{\theexercise}{\arabic{exercise}}
\newtheorem{example}{Exemple}[chapter]
\renewcommand{\theexample}{\arabic{example}}
\newtheorem*{solution}{Solution}
\newtheorem*{application}{Application}
\newtheorem*{notation}{Notation}
\newtheorem*{vocabulary}{Vocabulaire}
\newtheorem*{properties}{Propriétés}



\theoremstyle{remark}
\newtheorem*{remark}{Remarque}
\newtheorem*{rappel}{Rappel}


\usepackage{etoolbox}
\AtBeginEnvironment{exercise}{\small}
\AtBeginEnvironment{example}{\small}

\usepackage{cases}
\usepackage[red]{mypack}

\usepackage[framemethod=TikZ]{mdframed}

\definecolor{bg}{rgb}{0.4,0.25,0.95}
\definecolor{pagebg}{rgb}{0,0,0.5}
\surroundwithmdframed[
   topline=false,
   rightline=false,
   bottomline=false,
   leftmargin=\parindent,
   skipabove=8pt,
   skipbelow=8pt,
   linecolor=blue,
   innerbottommargin=10pt,
   % backgroundcolor=bg,font=\color{orange}\sffamily, fontcolor=white
]{definition}

\usepackage{empheq}
\usepackage[most]{tcolorbox}

\newtcbox{\mymath}[1][]{%
    nobeforeafter, math upper, tcbox raise base,
    enhanced, colframe=blue!30!black,
    colback=red!10, boxrule=1pt,
    #1}

\usepackage{unixode}


\DeclareMathOperator{\ord}{ord}
\DeclareMathOperator{\orb}{orb}
\DeclareMathOperator{\stab}{stab}
\DeclareMathOperator{\Stab}{stab}
\DeclareMathOperator{\ppcm}{ppcm}
\DeclareMathOperator{\conj}{Conj}
\DeclareMathOperator{\End}{End}
\DeclareMathOperator{\rot}{rot}
\DeclareMathOperator{\trs}{trace}
\DeclareMathOperator{\Ind}{Ind}
\DeclareMathOperator{\mat}{Mat}
\DeclareMathOperator{\id}{Id}
\DeclareMathOperator{\vect}{vect}
\DeclareMathOperator{\img}{img}
\DeclareMathOperator{\cov}{Cov}
\DeclareMathOperator{\dist}{dist}
\DeclareMathOperator{\irr}{Irr}
\DeclareMathOperator{\image}{Im}
\DeclareMathOperator{\pd}{\partial}
\DeclareMathOperator{\epi}{epi}
\DeclareMathOperator{\Argmin}{Argmin}
\DeclareMathOperator{\dom}{dom}
\DeclareMathOperator{\proj}{proj}
\DeclareMathOperator{\ctg}{ctg}
\DeclareMathOperator{\supp}{supp}
\DeclareMathOperator{\argmin}{argmin}
\DeclareMathOperator{\mult}{mult}
\DeclareMathOperator{\ch}{ch}
\DeclareMathOperator{\sh}{sh}
\DeclareMathOperator{\rang}{rang}
\DeclareMathOperator{\diam}{diam}
\DeclareMathOperator{\Epigraphe}{Epigraphe}




\usepackage{xcolor}
\everymath{\color{blue}}
%\everymath{\color[rgb]{0,1,1}}
%\pagecolor[rgb]{0,0,0.5}


\newcommand*{\pdtest}[3][]{\ensuremath{\frac{\partial^{#1} #2}{\partial #3}}}

\newcommand*{\deffunc}[6][]{\ensuremath{
\begin{array}{rcl}
#2 : #3 &\rightarrow& #4\\
#5 &\mapsto& #6
\end{array}
}}

\newcommand{\eqcolon}{\mathrel{\resizebox{\widthof{$\mathord{=}$}}{\height}{ $\!\!=\!\!\resizebox{1.2\width}{0.8\height}{\raisebox{0.23ex}{$\mathop{:}$}}\!\!$ }}}
\newcommand{\coloneq}{\mathrel{\resizebox{\widthof{$\mathord{=}$}}{\height}{ $\!\!\resizebox{1.2\width}{0.8\height}{\raisebox{0.23ex}{$\mathop{:}$}}\!\!=\!\!$ }}}
\newcommand{\eqcolonl}{\ensuremath{\mathrel{=\!\!\mathop{:}}}}
\newcommand{\coloneql}{\ensuremath{\mathrel{\mathop{:} \!\! =}}}
\newcommand{\vc}[1]{% inline column vector
  \left(\begin{smallmatrix}#1\end{smallmatrix}\right)%
}
\newcommand{\vr}[1]{% inline row vector
  \begin{smallmatrix}(\,#1\,)\end{smallmatrix}%
}
\makeatletter
\newcommand*{\defeq}{\ =\mathrel{\rlap{%
                     \raisebox{0.3ex}{$\m@th\cdot$}}%
                     \raisebox{-0.3ex}{$\m@th\cdot$}}%
                     }
\makeatother

\newcommand{\mathcircle}[1]{% inline row vector
 \overset{\circ}{#1}
}
\newcommand{\ulim}{% low limit
 \underline{\lim}
}
\newcommand{\ssi}{% iff
\iff
}
\newcommand{\ps}[2]{
\expval{#1 | #2}
}
\newcommand{\df}[1]{
\mqty{#1}
}
\newcommand{\n}[1]{
\norm{#1}
}
\newcommand{\sys}[1]{
\left\{\smqty{#1}\right.
}


\newcommand{\eqdef}{\ensuremath{\overset{\text{def}}=}}


\def\Circlearrowright{\ensuremath{%
  \rotatebox[origin=c]{230}{$\circlearrowright$}}}

\newcommand\ct[1]{\text{\rmfamily\upshape #1}}
\newcommand\question[1]{ {\color{red} ...!? \small #1}}
\newcommand\caz[1]{\left\{\begin{array} #1 \end{array}\right.}
\newcommand\const{\text{\rmfamily\upshape const}}
\newcommand\toP{ \overset{\pro}{\to}}
\newcommand\toPP{ \overset{\text{PP}}{\to}}
\newcommand{\oeq}{\mathrel{\text{\textcircled{$=$}}}}





\usepackage{xcolor}
% \usepackage[normalem]{ulem}
\usepackage{lipsum}
\makeatletter
% \newcommand\colorwave[1][blue]{\bgroup \markoverwith{\lower3.5\p@\hbox{\sixly \textcolor{#1}{\char58}}}\ULon}
%\font\sixly=lasy6 % does not re-load if already loaded, so no memory problem.

\newmdtheoremenv[
linewidth= 1pt,linecolor= blue,%
leftmargin=20,rightmargin=20,innertopmargin=0pt, innerrightmargin=40,%
tikzsetting = { draw=lightgray, line width = 0.3pt,dashed,%
dash pattern = on 15pt off 3pt},%
splittopskip=\topskip,skipbelow=\baselineskip,%
skipabove=\baselineskip,ntheorem,roundcorner=0pt,
% backgroundcolor=pagebg,font=\color{orange}\sffamily, fontcolor=white
]{examplebox}{Exemple}[section]



\newcommand\R{\mathbb{R}}
\newcommand\Z{\mathbb{Z}}
\newcommand\N{\mathbb{N}}
\newcommand\E{\mathbb{E}}
\newcommand\F{\mathcal{F}}
\newcommand\cH{\mathcal{H}}
\newcommand\V{\mathbb{V}}
\newcommand\dmo{ ^{-1} }
\newcommand\kapa{\kappa}
\newcommand\im{Im}
\newcommand\hs{\mathcal{H}}





\usepackage{soul}

\makeatletter
\newcommand*{\whiten}[1]{\llap{\textcolor{white}{{\the\SOUL@token}}\hspace{#1pt}}}
\DeclareRobustCommand*\myul{%
    \def\SOUL@everyspace{\underline{\space}\kern\z@}%
    \def\SOUL@everytoken{%
     \setbox0=\hbox{\the\SOUL@token}%
     \ifdim\dp0>\z@
        \raisebox{\dp0}{\underline{\phantom{\the\SOUL@token}}}%
        \whiten{1}\whiten{0}%
        \whiten{-1}\whiten{-2}%
        \llap{\the\SOUL@token}%
     \else
        \underline{\the\SOUL@token}%
     \fi}%
\SOUL@}
\makeatother

\newcommand*{\demp}{\fontfamily{lmtt}\selectfont}

\DeclareTextFontCommand{\textdemp}{\demp}

\begin{document}

\ifcomment
Multiline
comment
\fi
\ifcomment
\myul{Typesetting test}
% \color[rgb]{1,1,1}
$∑_i^n≠ 60º±∞π∆¬≈√j∫h≤≥µ$

$\CR \R\pro\ind\pro\gS\pro
\mqty[a&b\\c&d]$
$\pro\mathbb{P}$
$\dd{x}$

  \[
    \alpha(x)=\left\{
                \begin{array}{ll}
                  x\\
                  \frac{1}{1+e^{-kx}}\\
                  \frac{e^x-e^{-x}}{e^x+e^{-x}}
                \end{array}
              \right.
  \]

  $\expval{x}$
  
  $\chi_\rho(ghg\dmo)=\Tr(\rho_{ghg\dmo})=\Tr(\rho_g\circ\rho_h\circ\rho\dmo_g)=\Tr(\rho_h)\overset{\mbox{\scalebox{0.5}{$\Tr(AB)=\Tr(BA)$}}}{=}\chi_\rho(h)$
  	$\mathop{\oplus}_{\substack{x\in X}}$

$\mat(\rho_g)=(a_{ij}(g))_{\scriptsize \substack{1\leq i\leq d \\ 1\leq j\leq d}}$ et $\mat(\rho'_g)=(a'_{ij}(g))_{\scriptsize \substack{1\leq i'\leq d' \\ 1\leq j'\leq d'}}$



\[\int_a^b{\mathbb{R}^2}g(u, v)\dd{P_{XY}}(u, v)=\iint g(u,v) f_{XY}(u, v)\dd \lambda(u) \dd \lambda(v)\]
$$\lim_{x\to\infty} f(x)$$	
$$\iiiint_V \mu(t,u,v,w) \,dt\,du\,dv\,dw$$
$$\sum_{n=1}^{\infty} 2^{-n} = 1$$	
\begin{definition}
	Si $X$ et $Y$ sont 2 v.a. ou definit la \textsc{Covariance} entre $X$ et $Y$ comme
	$\cov(X,Y)\overset{\text{def}}{=}\E\left[(X-\E(X))(Y-\E(Y))\right]=\E(XY)-\E(X)\E(Y)$.
\end{definition}
\fi
\pagebreak

% \tableofcontents

% insert your code here
%\input{./algebra/main.tex}
%\input{./geometrie-differentielle/main.tex}
%\input{./probabilite/main.tex}
%\input{./analyse-fonctionnelle/main.tex}
% \input{./Analyse-convexe-et-dualite-en-optimisation/main.tex}
%\input{./tikz/main.tex}
%\input{./Theorie-du-distributions/main.tex}
%\input{./optimisation/mine.tex}
 \input{./modelisation/main.tex}

% yves.aubry@univ-tln.fr : algebra

\end{document}

%% !TEX encoding = UTF-8 Unicode
% !TEX TS-program = xelatex

\documentclass[french]{report}

%\usepackage[utf8]{inputenc}
%\usepackage[T1]{fontenc}
\usepackage{babel}


\newif\ifcomment
%\commenttrue # Show comments

\usepackage{physics}
\usepackage{amssymb}


\usepackage{amsthm}
% \usepackage{thmtools}
\usepackage{mathtools}
\usepackage{amsfonts}

\usepackage{color}

\usepackage{tikz}

\usepackage{geometry}
\geometry{a5paper, margin=0.1in, right=1cm}

\usepackage{dsfont}

\usepackage{graphicx}
\graphicspath{ {images/} }

\usepackage{faktor}

\usepackage{IEEEtrantools}
\usepackage{enumerate}   
\usepackage[PostScript=dvips]{"/Users/aware/Documents/Courses/diagrams"}


\newtheorem{theorem}{Théorème}[section]
\renewcommand{\thetheorem}{\arabic{theorem}}
\newtheorem{lemme}{Lemme}[section]
\renewcommand{\thelemme}{\arabic{lemme}}
\newtheorem{proposition}{Proposition}[section]
\renewcommand{\theproposition}{\arabic{proposition}}
\newtheorem{notations}{Notations}[section]
\newtheorem{problem}{Problème}[section]
\newtheorem{corollary}{Corollaire}[theorem]
\renewcommand{\thecorollary}{\arabic{corollary}}
\newtheorem{property}{Propriété}[section]
\newtheorem{objective}{Objectif}[section]

\theoremstyle{definition}
\newtheorem{definition}{Définition}[section]
\renewcommand{\thedefinition}{\arabic{definition}}
\newtheorem{exercise}{Exercice}[chapter]
\renewcommand{\theexercise}{\arabic{exercise}}
\newtheorem{example}{Exemple}[chapter]
\renewcommand{\theexample}{\arabic{example}}
\newtheorem*{solution}{Solution}
\newtheorem*{application}{Application}
\newtheorem*{notation}{Notation}
\newtheorem*{vocabulary}{Vocabulaire}
\newtheorem*{properties}{Propriétés}



\theoremstyle{remark}
\newtheorem*{remark}{Remarque}
\newtheorem*{rappel}{Rappel}


\usepackage{etoolbox}
\AtBeginEnvironment{exercise}{\small}
\AtBeginEnvironment{example}{\small}

\usepackage{cases}
\usepackage[red]{mypack}

\usepackage[framemethod=TikZ]{mdframed}

\definecolor{bg}{rgb}{0.4,0.25,0.95}
\definecolor{pagebg}{rgb}{0,0,0.5}
\surroundwithmdframed[
   topline=false,
   rightline=false,
   bottomline=false,
   leftmargin=\parindent,
   skipabove=8pt,
   skipbelow=8pt,
   linecolor=blue,
   innerbottommargin=10pt,
   % backgroundcolor=bg,font=\color{orange}\sffamily, fontcolor=white
]{definition}

\usepackage{empheq}
\usepackage[most]{tcolorbox}

\newtcbox{\mymath}[1][]{%
    nobeforeafter, math upper, tcbox raise base,
    enhanced, colframe=blue!30!black,
    colback=red!10, boxrule=1pt,
    #1}

\usepackage{unixode}


\DeclareMathOperator{\ord}{ord}
\DeclareMathOperator{\orb}{orb}
\DeclareMathOperator{\stab}{stab}
\DeclareMathOperator{\Stab}{stab}
\DeclareMathOperator{\ppcm}{ppcm}
\DeclareMathOperator{\conj}{Conj}
\DeclareMathOperator{\End}{End}
\DeclareMathOperator{\rot}{rot}
\DeclareMathOperator{\trs}{trace}
\DeclareMathOperator{\Ind}{Ind}
\DeclareMathOperator{\mat}{Mat}
\DeclareMathOperator{\id}{Id}
\DeclareMathOperator{\vect}{vect}
\DeclareMathOperator{\img}{img}
\DeclareMathOperator{\cov}{Cov}
\DeclareMathOperator{\dist}{dist}
\DeclareMathOperator{\irr}{Irr}
\DeclareMathOperator{\image}{Im}
\DeclareMathOperator{\pd}{\partial}
\DeclareMathOperator{\epi}{epi}
\DeclareMathOperator{\Argmin}{Argmin}
\DeclareMathOperator{\dom}{dom}
\DeclareMathOperator{\proj}{proj}
\DeclareMathOperator{\ctg}{ctg}
\DeclareMathOperator{\supp}{supp}
\DeclareMathOperator{\argmin}{argmin}
\DeclareMathOperator{\mult}{mult}
\DeclareMathOperator{\ch}{ch}
\DeclareMathOperator{\sh}{sh}
\DeclareMathOperator{\rang}{rang}
\DeclareMathOperator{\diam}{diam}
\DeclareMathOperator{\Epigraphe}{Epigraphe}




\usepackage{xcolor}
\everymath{\color{blue}}
%\everymath{\color[rgb]{0,1,1}}
%\pagecolor[rgb]{0,0,0.5}


\newcommand*{\pdtest}[3][]{\ensuremath{\frac{\partial^{#1} #2}{\partial #3}}}

\newcommand*{\deffunc}[6][]{\ensuremath{
\begin{array}{rcl}
#2 : #3 &\rightarrow& #4\\
#5 &\mapsto& #6
\end{array}
}}

\newcommand{\eqcolon}{\mathrel{\resizebox{\widthof{$\mathord{=}$}}{\height}{ $\!\!=\!\!\resizebox{1.2\width}{0.8\height}{\raisebox{0.23ex}{$\mathop{:}$}}\!\!$ }}}
\newcommand{\coloneq}{\mathrel{\resizebox{\widthof{$\mathord{=}$}}{\height}{ $\!\!\resizebox{1.2\width}{0.8\height}{\raisebox{0.23ex}{$\mathop{:}$}}\!\!=\!\!$ }}}
\newcommand{\eqcolonl}{\ensuremath{\mathrel{=\!\!\mathop{:}}}}
\newcommand{\coloneql}{\ensuremath{\mathrel{\mathop{:} \!\! =}}}
\newcommand{\vc}[1]{% inline column vector
  \left(\begin{smallmatrix}#1\end{smallmatrix}\right)%
}
\newcommand{\vr}[1]{% inline row vector
  \begin{smallmatrix}(\,#1\,)\end{smallmatrix}%
}
\makeatletter
\newcommand*{\defeq}{\ =\mathrel{\rlap{%
                     \raisebox{0.3ex}{$\m@th\cdot$}}%
                     \raisebox{-0.3ex}{$\m@th\cdot$}}%
                     }
\makeatother

\newcommand{\mathcircle}[1]{% inline row vector
 \overset{\circ}{#1}
}
\newcommand{\ulim}{% low limit
 \underline{\lim}
}
\newcommand{\ssi}{% iff
\iff
}
\newcommand{\ps}[2]{
\expval{#1 | #2}
}
\newcommand{\df}[1]{
\mqty{#1}
}
\newcommand{\n}[1]{
\norm{#1}
}
\newcommand{\sys}[1]{
\left\{\smqty{#1}\right.
}


\newcommand{\eqdef}{\ensuremath{\overset{\text{def}}=}}


\def\Circlearrowright{\ensuremath{%
  \rotatebox[origin=c]{230}{$\circlearrowright$}}}

\newcommand\ct[1]{\text{\rmfamily\upshape #1}}
\newcommand\question[1]{ {\color{red} ...!? \small #1}}
\newcommand\caz[1]{\left\{\begin{array} #1 \end{array}\right.}
\newcommand\const{\text{\rmfamily\upshape const}}
\newcommand\toP{ \overset{\pro}{\to}}
\newcommand\toPP{ \overset{\text{PP}}{\to}}
\newcommand{\oeq}{\mathrel{\text{\textcircled{$=$}}}}





\usepackage{xcolor}
% \usepackage[normalem]{ulem}
\usepackage{lipsum}
\makeatletter
% \newcommand\colorwave[1][blue]{\bgroup \markoverwith{\lower3.5\p@\hbox{\sixly \textcolor{#1}{\char58}}}\ULon}
%\font\sixly=lasy6 % does not re-load if already loaded, so no memory problem.

\newmdtheoremenv[
linewidth= 1pt,linecolor= blue,%
leftmargin=20,rightmargin=20,innertopmargin=0pt, innerrightmargin=40,%
tikzsetting = { draw=lightgray, line width = 0.3pt,dashed,%
dash pattern = on 15pt off 3pt},%
splittopskip=\topskip,skipbelow=\baselineskip,%
skipabove=\baselineskip,ntheorem,roundcorner=0pt,
% backgroundcolor=pagebg,font=\color{orange}\sffamily, fontcolor=white
]{examplebox}{Exemple}[section]



\newcommand\R{\mathbb{R}}
\newcommand\Z{\mathbb{Z}}
\newcommand\N{\mathbb{N}}
\newcommand\E{\mathbb{E}}
\newcommand\F{\mathcal{F}}
\newcommand\cH{\mathcal{H}}
\newcommand\V{\mathbb{V}}
\newcommand\dmo{ ^{-1} }
\newcommand\kapa{\kappa}
\newcommand\im{Im}
\newcommand\hs{\mathcal{H}}





\usepackage{soul}

\makeatletter
\newcommand*{\whiten}[1]{\llap{\textcolor{white}{{\the\SOUL@token}}\hspace{#1pt}}}
\DeclareRobustCommand*\myul{%
    \def\SOUL@everyspace{\underline{\space}\kern\z@}%
    \def\SOUL@everytoken{%
     \setbox0=\hbox{\the\SOUL@token}%
     \ifdim\dp0>\z@
        \raisebox{\dp0}{\underline{\phantom{\the\SOUL@token}}}%
        \whiten{1}\whiten{0}%
        \whiten{-1}\whiten{-2}%
        \llap{\the\SOUL@token}%
     \else
        \underline{\the\SOUL@token}%
     \fi}%
\SOUL@}
\makeatother

\newcommand*{\demp}{\fontfamily{lmtt}\selectfont}

\DeclareTextFontCommand{\textdemp}{\demp}

\begin{document}

\ifcomment
Multiline
comment
\fi
\ifcomment
\myul{Typesetting test}
% \color[rgb]{1,1,1}
$∑_i^n≠ 60º±∞π∆¬≈√j∫h≤≥µ$

$\CR \R\pro\ind\pro\gS\pro
\mqty[a&b\\c&d]$
$\pro\mathbb{P}$
$\dd{x}$

  \[
    \alpha(x)=\left\{
                \begin{array}{ll}
                  x\\
                  \frac{1}{1+e^{-kx}}\\
                  \frac{e^x-e^{-x}}{e^x+e^{-x}}
                \end{array}
              \right.
  \]

  $\expval{x}$
  
  $\chi_\rho(ghg\dmo)=\Tr(\rho_{ghg\dmo})=\Tr(\rho_g\circ\rho_h\circ\rho\dmo_g)=\Tr(\rho_h)\overset{\mbox{\scalebox{0.5}{$\Tr(AB)=\Tr(BA)$}}}{=}\chi_\rho(h)$
  	$\mathop{\oplus}_{\substack{x\in X}}$

$\mat(\rho_g)=(a_{ij}(g))_{\scriptsize \substack{1\leq i\leq d \\ 1\leq j\leq d}}$ et $\mat(\rho'_g)=(a'_{ij}(g))_{\scriptsize \substack{1\leq i'\leq d' \\ 1\leq j'\leq d'}}$



\[\int_a^b{\mathbb{R}^2}g(u, v)\dd{P_{XY}}(u, v)=\iint g(u,v) f_{XY}(u, v)\dd \lambda(u) \dd \lambda(v)\]
$$\lim_{x\to\infty} f(x)$$	
$$\iiiint_V \mu(t,u,v,w) \,dt\,du\,dv\,dw$$
$$\sum_{n=1}^{\infty} 2^{-n} = 1$$	
\begin{definition}
	Si $X$ et $Y$ sont 2 v.a. ou definit la \textsc{Covariance} entre $X$ et $Y$ comme
	$\cov(X,Y)\overset{\text{def}}{=}\E\left[(X-\E(X))(Y-\E(Y))\right]=\E(XY)-\E(X)\E(Y)$.
\end{definition}
\fi
\pagebreak

% \tableofcontents

% insert your code here
%\input{./algebra/main.tex}
%\input{./geometrie-differentielle/main.tex}
%\input{./probabilite/main.tex}
%\input{./analyse-fonctionnelle/main.tex}
% \input{./Analyse-convexe-et-dualite-en-optimisation/main.tex}
%\input{./tikz/main.tex}
%\input{./Theorie-du-distributions/main.tex}
%\input{./optimisation/mine.tex}
 \input{./modelisation/main.tex}

% yves.aubry@univ-tln.fr : algebra

\end{document}

%% !TEX encoding = UTF-8 Unicode
% !TEX TS-program = xelatex

\documentclass[french]{report}

%\usepackage[utf8]{inputenc}
%\usepackage[T1]{fontenc}
\usepackage{babel}


\newif\ifcomment
%\commenttrue # Show comments

\usepackage{physics}
\usepackage{amssymb}


\usepackage{amsthm}
% \usepackage{thmtools}
\usepackage{mathtools}
\usepackage{amsfonts}

\usepackage{color}

\usepackage{tikz}

\usepackage{geometry}
\geometry{a5paper, margin=0.1in, right=1cm}

\usepackage{dsfont}

\usepackage{graphicx}
\graphicspath{ {images/} }

\usepackage{faktor}

\usepackage{IEEEtrantools}
\usepackage{enumerate}   
\usepackage[PostScript=dvips]{"/Users/aware/Documents/Courses/diagrams"}


\newtheorem{theorem}{Théorème}[section]
\renewcommand{\thetheorem}{\arabic{theorem}}
\newtheorem{lemme}{Lemme}[section]
\renewcommand{\thelemme}{\arabic{lemme}}
\newtheorem{proposition}{Proposition}[section]
\renewcommand{\theproposition}{\arabic{proposition}}
\newtheorem{notations}{Notations}[section]
\newtheorem{problem}{Problème}[section]
\newtheorem{corollary}{Corollaire}[theorem]
\renewcommand{\thecorollary}{\arabic{corollary}}
\newtheorem{property}{Propriété}[section]
\newtheorem{objective}{Objectif}[section]

\theoremstyle{definition}
\newtheorem{definition}{Définition}[section]
\renewcommand{\thedefinition}{\arabic{definition}}
\newtheorem{exercise}{Exercice}[chapter]
\renewcommand{\theexercise}{\arabic{exercise}}
\newtheorem{example}{Exemple}[chapter]
\renewcommand{\theexample}{\arabic{example}}
\newtheorem*{solution}{Solution}
\newtheorem*{application}{Application}
\newtheorem*{notation}{Notation}
\newtheorem*{vocabulary}{Vocabulaire}
\newtheorem*{properties}{Propriétés}



\theoremstyle{remark}
\newtheorem*{remark}{Remarque}
\newtheorem*{rappel}{Rappel}


\usepackage{etoolbox}
\AtBeginEnvironment{exercise}{\small}
\AtBeginEnvironment{example}{\small}

\usepackage{cases}
\usepackage[red]{mypack}

\usepackage[framemethod=TikZ]{mdframed}

\definecolor{bg}{rgb}{0.4,0.25,0.95}
\definecolor{pagebg}{rgb}{0,0,0.5}
\surroundwithmdframed[
   topline=false,
   rightline=false,
   bottomline=false,
   leftmargin=\parindent,
   skipabove=8pt,
   skipbelow=8pt,
   linecolor=blue,
   innerbottommargin=10pt,
   % backgroundcolor=bg,font=\color{orange}\sffamily, fontcolor=white
]{definition}

\usepackage{empheq}
\usepackage[most]{tcolorbox}

\newtcbox{\mymath}[1][]{%
    nobeforeafter, math upper, tcbox raise base,
    enhanced, colframe=blue!30!black,
    colback=red!10, boxrule=1pt,
    #1}

\usepackage{unixode}


\DeclareMathOperator{\ord}{ord}
\DeclareMathOperator{\orb}{orb}
\DeclareMathOperator{\stab}{stab}
\DeclareMathOperator{\Stab}{stab}
\DeclareMathOperator{\ppcm}{ppcm}
\DeclareMathOperator{\conj}{Conj}
\DeclareMathOperator{\End}{End}
\DeclareMathOperator{\rot}{rot}
\DeclareMathOperator{\trs}{trace}
\DeclareMathOperator{\Ind}{Ind}
\DeclareMathOperator{\mat}{Mat}
\DeclareMathOperator{\id}{Id}
\DeclareMathOperator{\vect}{vect}
\DeclareMathOperator{\img}{img}
\DeclareMathOperator{\cov}{Cov}
\DeclareMathOperator{\dist}{dist}
\DeclareMathOperator{\irr}{Irr}
\DeclareMathOperator{\image}{Im}
\DeclareMathOperator{\pd}{\partial}
\DeclareMathOperator{\epi}{epi}
\DeclareMathOperator{\Argmin}{Argmin}
\DeclareMathOperator{\dom}{dom}
\DeclareMathOperator{\proj}{proj}
\DeclareMathOperator{\ctg}{ctg}
\DeclareMathOperator{\supp}{supp}
\DeclareMathOperator{\argmin}{argmin}
\DeclareMathOperator{\mult}{mult}
\DeclareMathOperator{\ch}{ch}
\DeclareMathOperator{\sh}{sh}
\DeclareMathOperator{\rang}{rang}
\DeclareMathOperator{\diam}{diam}
\DeclareMathOperator{\Epigraphe}{Epigraphe}




\usepackage{xcolor}
\everymath{\color{blue}}
%\everymath{\color[rgb]{0,1,1}}
%\pagecolor[rgb]{0,0,0.5}


\newcommand*{\pdtest}[3][]{\ensuremath{\frac{\partial^{#1} #2}{\partial #3}}}

\newcommand*{\deffunc}[6][]{\ensuremath{
\begin{array}{rcl}
#2 : #3 &\rightarrow& #4\\
#5 &\mapsto& #6
\end{array}
}}

\newcommand{\eqcolon}{\mathrel{\resizebox{\widthof{$\mathord{=}$}}{\height}{ $\!\!=\!\!\resizebox{1.2\width}{0.8\height}{\raisebox{0.23ex}{$\mathop{:}$}}\!\!$ }}}
\newcommand{\coloneq}{\mathrel{\resizebox{\widthof{$\mathord{=}$}}{\height}{ $\!\!\resizebox{1.2\width}{0.8\height}{\raisebox{0.23ex}{$\mathop{:}$}}\!\!=\!\!$ }}}
\newcommand{\eqcolonl}{\ensuremath{\mathrel{=\!\!\mathop{:}}}}
\newcommand{\coloneql}{\ensuremath{\mathrel{\mathop{:} \!\! =}}}
\newcommand{\vc}[1]{% inline column vector
  \left(\begin{smallmatrix}#1\end{smallmatrix}\right)%
}
\newcommand{\vr}[1]{% inline row vector
  \begin{smallmatrix}(\,#1\,)\end{smallmatrix}%
}
\makeatletter
\newcommand*{\defeq}{\ =\mathrel{\rlap{%
                     \raisebox{0.3ex}{$\m@th\cdot$}}%
                     \raisebox{-0.3ex}{$\m@th\cdot$}}%
                     }
\makeatother

\newcommand{\mathcircle}[1]{% inline row vector
 \overset{\circ}{#1}
}
\newcommand{\ulim}{% low limit
 \underline{\lim}
}
\newcommand{\ssi}{% iff
\iff
}
\newcommand{\ps}[2]{
\expval{#1 | #2}
}
\newcommand{\df}[1]{
\mqty{#1}
}
\newcommand{\n}[1]{
\norm{#1}
}
\newcommand{\sys}[1]{
\left\{\smqty{#1}\right.
}


\newcommand{\eqdef}{\ensuremath{\overset{\text{def}}=}}


\def\Circlearrowright{\ensuremath{%
  \rotatebox[origin=c]{230}{$\circlearrowright$}}}

\newcommand\ct[1]{\text{\rmfamily\upshape #1}}
\newcommand\question[1]{ {\color{red} ...!? \small #1}}
\newcommand\caz[1]{\left\{\begin{array} #1 \end{array}\right.}
\newcommand\const{\text{\rmfamily\upshape const}}
\newcommand\toP{ \overset{\pro}{\to}}
\newcommand\toPP{ \overset{\text{PP}}{\to}}
\newcommand{\oeq}{\mathrel{\text{\textcircled{$=$}}}}





\usepackage{xcolor}
% \usepackage[normalem]{ulem}
\usepackage{lipsum}
\makeatletter
% \newcommand\colorwave[1][blue]{\bgroup \markoverwith{\lower3.5\p@\hbox{\sixly \textcolor{#1}{\char58}}}\ULon}
%\font\sixly=lasy6 % does not re-load if already loaded, so no memory problem.

\newmdtheoremenv[
linewidth= 1pt,linecolor= blue,%
leftmargin=20,rightmargin=20,innertopmargin=0pt, innerrightmargin=40,%
tikzsetting = { draw=lightgray, line width = 0.3pt,dashed,%
dash pattern = on 15pt off 3pt},%
splittopskip=\topskip,skipbelow=\baselineskip,%
skipabove=\baselineskip,ntheorem,roundcorner=0pt,
% backgroundcolor=pagebg,font=\color{orange}\sffamily, fontcolor=white
]{examplebox}{Exemple}[section]



\newcommand\R{\mathbb{R}}
\newcommand\Z{\mathbb{Z}}
\newcommand\N{\mathbb{N}}
\newcommand\E{\mathbb{E}}
\newcommand\F{\mathcal{F}}
\newcommand\cH{\mathcal{H}}
\newcommand\V{\mathbb{V}}
\newcommand\dmo{ ^{-1} }
\newcommand\kapa{\kappa}
\newcommand\im{Im}
\newcommand\hs{\mathcal{H}}





\usepackage{soul}

\makeatletter
\newcommand*{\whiten}[1]{\llap{\textcolor{white}{{\the\SOUL@token}}\hspace{#1pt}}}
\DeclareRobustCommand*\myul{%
    \def\SOUL@everyspace{\underline{\space}\kern\z@}%
    \def\SOUL@everytoken{%
     \setbox0=\hbox{\the\SOUL@token}%
     \ifdim\dp0>\z@
        \raisebox{\dp0}{\underline{\phantom{\the\SOUL@token}}}%
        \whiten{1}\whiten{0}%
        \whiten{-1}\whiten{-2}%
        \llap{\the\SOUL@token}%
     \else
        \underline{\the\SOUL@token}%
     \fi}%
\SOUL@}
\makeatother

\newcommand*{\demp}{\fontfamily{lmtt}\selectfont}

\DeclareTextFontCommand{\textdemp}{\demp}

\begin{document}

\ifcomment
Multiline
comment
\fi
\ifcomment
\myul{Typesetting test}
% \color[rgb]{1,1,1}
$∑_i^n≠ 60º±∞π∆¬≈√j∫h≤≥µ$

$\CR \R\pro\ind\pro\gS\pro
\mqty[a&b\\c&d]$
$\pro\mathbb{P}$
$\dd{x}$

  \[
    \alpha(x)=\left\{
                \begin{array}{ll}
                  x\\
                  \frac{1}{1+e^{-kx}}\\
                  \frac{e^x-e^{-x}}{e^x+e^{-x}}
                \end{array}
              \right.
  \]

  $\expval{x}$
  
  $\chi_\rho(ghg\dmo)=\Tr(\rho_{ghg\dmo})=\Tr(\rho_g\circ\rho_h\circ\rho\dmo_g)=\Tr(\rho_h)\overset{\mbox{\scalebox{0.5}{$\Tr(AB)=\Tr(BA)$}}}{=}\chi_\rho(h)$
  	$\mathop{\oplus}_{\substack{x\in X}}$

$\mat(\rho_g)=(a_{ij}(g))_{\scriptsize \substack{1\leq i\leq d \\ 1\leq j\leq d}}$ et $\mat(\rho'_g)=(a'_{ij}(g))_{\scriptsize \substack{1\leq i'\leq d' \\ 1\leq j'\leq d'}}$



\[\int_a^b{\mathbb{R}^2}g(u, v)\dd{P_{XY}}(u, v)=\iint g(u,v) f_{XY}(u, v)\dd \lambda(u) \dd \lambda(v)\]
$$\lim_{x\to\infty} f(x)$$	
$$\iiiint_V \mu(t,u,v,w) \,dt\,du\,dv\,dw$$
$$\sum_{n=1}^{\infty} 2^{-n} = 1$$	
\begin{definition}
	Si $X$ et $Y$ sont 2 v.a. ou definit la \textsc{Covariance} entre $X$ et $Y$ comme
	$\cov(X,Y)\overset{\text{def}}{=}\E\left[(X-\E(X))(Y-\E(Y))\right]=\E(XY)-\E(X)\E(Y)$.
\end{definition}
\fi
\pagebreak

% \tableofcontents

% insert your code here
%\input{./algebra/main.tex}
%\input{./geometrie-differentielle/main.tex}
%\input{./probabilite/main.tex}
%\input{./analyse-fonctionnelle/main.tex}
% \input{./Analyse-convexe-et-dualite-en-optimisation/main.tex}
%\input{./tikz/main.tex}
%\input{./Theorie-du-distributions/main.tex}
%\input{./optimisation/mine.tex}
 \input{./modelisation/main.tex}

% yves.aubry@univ-tln.fr : algebra

\end{document}

%% !TEX encoding = UTF-8 Unicode
% !TEX TS-program = xelatex

\documentclass[french]{report}

%\usepackage[utf8]{inputenc}
%\usepackage[T1]{fontenc}
\usepackage{babel}


\newif\ifcomment
%\commenttrue # Show comments

\usepackage{physics}
\usepackage{amssymb}


\usepackage{amsthm}
% \usepackage{thmtools}
\usepackage{mathtools}
\usepackage{amsfonts}

\usepackage{color}

\usepackage{tikz}

\usepackage{geometry}
\geometry{a5paper, margin=0.1in, right=1cm}

\usepackage{dsfont}

\usepackage{graphicx}
\graphicspath{ {images/} }

\usepackage{faktor}

\usepackage{IEEEtrantools}
\usepackage{enumerate}   
\usepackage[PostScript=dvips]{"/Users/aware/Documents/Courses/diagrams"}


\newtheorem{theorem}{Théorème}[section]
\renewcommand{\thetheorem}{\arabic{theorem}}
\newtheorem{lemme}{Lemme}[section]
\renewcommand{\thelemme}{\arabic{lemme}}
\newtheorem{proposition}{Proposition}[section]
\renewcommand{\theproposition}{\arabic{proposition}}
\newtheorem{notations}{Notations}[section]
\newtheorem{problem}{Problème}[section]
\newtheorem{corollary}{Corollaire}[theorem]
\renewcommand{\thecorollary}{\arabic{corollary}}
\newtheorem{property}{Propriété}[section]
\newtheorem{objective}{Objectif}[section]

\theoremstyle{definition}
\newtheorem{definition}{Définition}[section]
\renewcommand{\thedefinition}{\arabic{definition}}
\newtheorem{exercise}{Exercice}[chapter]
\renewcommand{\theexercise}{\arabic{exercise}}
\newtheorem{example}{Exemple}[chapter]
\renewcommand{\theexample}{\arabic{example}}
\newtheorem*{solution}{Solution}
\newtheorem*{application}{Application}
\newtheorem*{notation}{Notation}
\newtheorem*{vocabulary}{Vocabulaire}
\newtheorem*{properties}{Propriétés}



\theoremstyle{remark}
\newtheorem*{remark}{Remarque}
\newtheorem*{rappel}{Rappel}


\usepackage{etoolbox}
\AtBeginEnvironment{exercise}{\small}
\AtBeginEnvironment{example}{\small}

\usepackage{cases}
\usepackage[red]{mypack}

\usepackage[framemethod=TikZ]{mdframed}

\definecolor{bg}{rgb}{0.4,0.25,0.95}
\definecolor{pagebg}{rgb}{0,0,0.5}
\surroundwithmdframed[
   topline=false,
   rightline=false,
   bottomline=false,
   leftmargin=\parindent,
   skipabove=8pt,
   skipbelow=8pt,
   linecolor=blue,
   innerbottommargin=10pt,
   % backgroundcolor=bg,font=\color{orange}\sffamily, fontcolor=white
]{definition}

\usepackage{empheq}
\usepackage[most]{tcolorbox}

\newtcbox{\mymath}[1][]{%
    nobeforeafter, math upper, tcbox raise base,
    enhanced, colframe=blue!30!black,
    colback=red!10, boxrule=1pt,
    #1}

\usepackage{unixode}


\DeclareMathOperator{\ord}{ord}
\DeclareMathOperator{\orb}{orb}
\DeclareMathOperator{\stab}{stab}
\DeclareMathOperator{\Stab}{stab}
\DeclareMathOperator{\ppcm}{ppcm}
\DeclareMathOperator{\conj}{Conj}
\DeclareMathOperator{\End}{End}
\DeclareMathOperator{\rot}{rot}
\DeclareMathOperator{\trs}{trace}
\DeclareMathOperator{\Ind}{Ind}
\DeclareMathOperator{\mat}{Mat}
\DeclareMathOperator{\id}{Id}
\DeclareMathOperator{\vect}{vect}
\DeclareMathOperator{\img}{img}
\DeclareMathOperator{\cov}{Cov}
\DeclareMathOperator{\dist}{dist}
\DeclareMathOperator{\irr}{Irr}
\DeclareMathOperator{\image}{Im}
\DeclareMathOperator{\pd}{\partial}
\DeclareMathOperator{\epi}{epi}
\DeclareMathOperator{\Argmin}{Argmin}
\DeclareMathOperator{\dom}{dom}
\DeclareMathOperator{\proj}{proj}
\DeclareMathOperator{\ctg}{ctg}
\DeclareMathOperator{\supp}{supp}
\DeclareMathOperator{\argmin}{argmin}
\DeclareMathOperator{\mult}{mult}
\DeclareMathOperator{\ch}{ch}
\DeclareMathOperator{\sh}{sh}
\DeclareMathOperator{\rang}{rang}
\DeclareMathOperator{\diam}{diam}
\DeclareMathOperator{\Epigraphe}{Epigraphe}




\usepackage{xcolor}
\everymath{\color{blue}}
%\everymath{\color[rgb]{0,1,1}}
%\pagecolor[rgb]{0,0,0.5}


\newcommand*{\pdtest}[3][]{\ensuremath{\frac{\partial^{#1} #2}{\partial #3}}}

\newcommand*{\deffunc}[6][]{\ensuremath{
\begin{array}{rcl}
#2 : #3 &\rightarrow& #4\\
#5 &\mapsto& #6
\end{array}
}}

\newcommand{\eqcolon}{\mathrel{\resizebox{\widthof{$\mathord{=}$}}{\height}{ $\!\!=\!\!\resizebox{1.2\width}{0.8\height}{\raisebox{0.23ex}{$\mathop{:}$}}\!\!$ }}}
\newcommand{\coloneq}{\mathrel{\resizebox{\widthof{$\mathord{=}$}}{\height}{ $\!\!\resizebox{1.2\width}{0.8\height}{\raisebox{0.23ex}{$\mathop{:}$}}\!\!=\!\!$ }}}
\newcommand{\eqcolonl}{\ensuremath{\mathrel{=\!\!\mathop{:}}}}
\newcommand{\coloneql}{\ensuremath{\mathrel{\mathop{:} \!\! =}}}
\newcommand{\vc}[1]{% inline column vector
  \left(\begin{smallmatrix}#1\end{smallmatrix}\right)%
}
\newcommand{\vr}[1]{% inline row vector
  \begin{smallmatrix}(\,#1\,)\end{smallmatrix}%
}
\makeatletter
\newcommand*{\defeq}{\ =\mathrel{\rlap{%
                     \raisebox{0.3ex}{$\m@th\cdot$}}%
                     \raisebox{-0.3ex}{$\m@th\cdot$}}%
                     }
\makeatother

\newcommand{\mathcircle}[1]{% inline row vector
 \overset{\circ}{#1}
}
\newcommand{\ulim}{% low limit
 \underline{\lim}
}
\newcommand{\ssi}{% iff
\iff
}
\newcommand{\ps}[2]{
\expval{#1 | #2}
}
\newcommand{\df}[1]{
\mqty{#1}
}
\newcommand{\n}[1]{
\norm{#1}
}
\newcommand{\sys}[1]{
\left\{\smqty{#1}\right.
}


\newcommand{\eqdef}{\ensuremath{\overset{\text{def}}=}}


\def\Circlearrowright{\ensuremath{%
  \rotatebox[origin=c]{230}{$\circlearrowright$}}}

\newcommand\ct[1]{\text{\rmfamily\upshape #1}}
\newcommand\question[1]{ {\color{red} ...!? \small #1}}
\newcommand\caz[1]{\left\{\begin{array} #1 \end{array}\right.}
\newcommand\const{\text{\rmfamily\upshape const}}
\newcommand\toP{ \overset{\pro}{\to}}
\newcommand\toPP{ \overset{\text{PP}}{\to}}
\newcommand{\oeq}{\mathrel{\text{\textcircled{$=$}}}}





\usepackage{xcolor}
% \usepackage[normalem]{ulem}
\usepackage{lipsum}
\makeatletter
% \newcommand\colorwave[1][blue]{\bgroup \markoverwith{\lower3.5\p@\hbox{\sixly \textcolor{#1}{\char58}}}\ULon}
%\font\sixly=lasy6 % does not re-load if already loaded, so no memory problem.

\newmdtheoremenv[
linewidth= 1pt,linecolor= blue,%
leftmargin=20,rightmargin=20,innertopmargin=0pt, innerrightmargin=40,%
tikzsetting = { draw=lightgray, line width = 0.3pt,dashed,%
dash pattern = on 15pt off 3pt},%
splittopskip=\topskip,skipbelow=\baselineskip,%
skipabove=\baselineskip,ntheorem,roundcorner=0pt,
% backgroundcolor=pagebg,font=\color{orange}\sffamily, fontcolor=white
]{examplebox}{Exemple}[section]



\newcommand\R{\mathbb{R}}
\newcommand\Z{\mathbb{Z}}
\newcommand\N{\mathbb{N}}
\newcommand\E{\mathbb{E}}
\newcommand\F{\mathcal{F}}
\newcommand\cH{\mathcal{H}}
\newcommand\V{\mathbb{V}}
\newcommand\dmo{ ^{-1} }
\newcommand\kapa{\kappa}
\newcommand\im{Im}
\newcommand\hs{\mathcal{H}}





\usepackage{soul}

\makeatletter
\newcommand*{\whiten}[1]{\llap{\textcolor{white}{{\the\SOUL@token}}\hspace{#1pt}}}
\DeclareRobustCommand*\myul{%
    \def\SOUL@everyspace{\underline{\space}\kern\z@}%
    \def\SOUL@everytoken{%
     \setbox0=\hbox{\the\SOUL@token}%
     \ifdim\dp0>\z@
        \raisebox{\dp0}{\underline{\phantom{\the\SOUL@token}}}%
        \whiten{1}\whiten{0}%
        \whiten{-1}\whiten{-2}%
        \llap{\the\SOUL@token}%
     \else
        \underline{\the\SOUL@token}%
     \fi}%
\SOUL@}
\makeatother

\newcommand*{\demp}{\fontfamily{lmtt}\selectfont}

\DeclareTextFontCommand{\textdemp}{\demp}

\begin{document}

\ifcomment
Multiline
comment
\fi
\ifcomment
\myul{Typesetting test}
% \color[rgb]{1,1,1}
$∑_i^n≠ 60º±∞π∆¬≈√j∫h≤≥µ$

$\CR \R\pro\ind\pro\gS\pro
\mqty[a&b\\c&d]$
$\pro\mathbb{P}$
$\dd{x}$

  \[
    \alpha(x)=\left\{
                \begin{array}{ll}
                  x\\
                  \frac{1}{1+e^{-kx}}\\
                  \frac{e^x-e^{-x}}{e^x+e^{-x}}
                \end{array}
              \right.
  \]

  $\expval{x}$
  
  $\chi_\rho(ghg\dmo)=\Tr(\rho_{ghg\dmo})=\Tr(\rho_g\circ\rho_h\circ\rho\dmo_g)=\Tr(\rho_h)\overset{\mbox{\scalebox{0.5}{$\Tr(AB)=\Tr(BA)$}}}{=}\chi_\rho(h)$
  	$\mathop{\oplus}_{\substack{x\in X}}$

$\mat(\rho_g)=(a_{ij}(g))_{\scriptsize \substack{1\leq i\leq d \\ 1\leq j\leq d}}$ et $\mat(\rho'_g)=(a'_{ij}(g))_{\scriptsize \substack{1\leq i'\leq d' \\ 1\leq j'\leq d'}}$



\[\int_a^b{\mathbb{R}^2}g(u, v)\dd{P_{XY}}(u, v)=\iint g(u,v) f_{XY}(u, v)\dd \lambda(u) \dd \lambda(v)\]
$$\lim_{x\to\infty} f(x)$$	
$$\iiiint_V \mu(t,u,v,w) \,dt\,du\,dv\,dw$$
$$\sum_{n=1}^{\infty} 2^{-n} = 1$$	
\begin{definition}
	Si $X$ et $Y$ sont 2 v.a. ou definit la \textsc{Covariance} entre $X$ et $Y$ comme
	$\cov(X,Y)\overset{\text{def}}{=}\E\left[(X-\E(X))(Y-\E(Y))\right]=\E(XY)-\E(X)\E(Y)$.
\end{definition}
\fi
\pagebreak

% \tableofcontents

% insert your code here
%\input{./algebra/main.tex}
%\input{./geometrie-differentielle/main.tex}
%\input{./probabilite/main.tex}
%\input{./analyse-fonctionnelle/main.tex}
% \input{./Analyse-convexe-et-dualite-en-optimisation/main.tex}
%\input{./tikz/main.tex}
%\input{./Theorie-du-distributions/main.tex}
%\input{./optimisation/mine.tex}
 \input{./modelisation/main.tex}

% yves.aubry@univ-tln.fr : algebra

\end{document}

% % !TEX encoding = UTF-8 Unicode
% !TEX TS-program = xelatex

\documentclass[french]{report}

%\usepackage[utf8]{inputenc}
%\usepackage[T1]{fontenc}
\usepackage{babel}


\newif\ifcomment
%\commenttrue # Show comments

\usepackage{physics}
\usepackage{amssymb}


\usepackage{amsthm}
% \usepackage{thmtools}
\usepackage{mathtools}
\usepackage{amsfonts}

\usepackage{color}

\usepackage{tikz}

\usepackage{geometry}
\geometry{a5paper, margin=0.1in, right=1cm}

\usepackage{dsfont}

\usepackage{graphicx}
\graphicspath{ {images/} }

\usepackage{faktor}

\usepackage{IEEEtrantools}
\usepackage{enumerate}   
\usepackage[PostScript=dvips]{"/Users/aware/Documents/Courses/diagrams"}


\newtheorem{theorem}{Théorème}[section]
\renewcommand{\thetheorem}{\arabic{theorem}}
\newtheorem{lemme}{Lemme}[section]
\renewcommand{\thelemme}{\arabic{lemme}}
\newtheorem{proposition}{Proposition}[section]
\renewcommand{\theproposition}{\arabic{proposition}}
\newtheorem{notations}{Notations}[section]
\newtheorem{problem}{Problème}[section]
\newtheorem{corollary}{Corollaire}[theorem]
\renewcommand{\thecorollary}{\arabic{corollary}}
\newtheorem{property}{Propriété}[section]
\newtheorem{objective}{Objectif}[section]

\theoremstyle{definition}
\newtheorem{definition}{Définition}[section]
\renewcommand{\thedefinition}{\arabic{definition}}
\newtheorem{exercise}{Exercice}[chapter]
\renewcommand{\theexercise}{\arabic{exercise}}
\newtheorem{example}{Exemple}[chapter]
\renewcommand{\theexample}{\arabic{example}}
\newtheorem*{solution}{Solution}
\newtheorem*{application}{Application}
\newtheorem*{notation}{Notation}
\newtheorem*{vocabulary}{Vocabulaire}
\newtheorem*{properties}{Propriétés}



\theoremstyle{remark}
\newtheorem*{remark}{Remarque}
\newtheorem*{rappel}{Rappel}


\usepackage{etoolbox}
\AtBeginEnvironment{exercise}{\small}
\AtBeginEnvironment{example}{\small}

\usepackage{cases}
\usepackage[red]{mypack}

\usepackage[framemethod=TikZ]{mdframed}

\definecolor{bg}{rgb}{0.4,0.25,0.95}
\definecolor{pagebg}{rgb}{0,0,0.5}
\surroundwithmdframed[
   topline=false,
   rightline=false,
   bottomline=false,
   leftmargin=\parindent,
   skipabove=8pt,
   skipbelow=8pt,
   linecolor=blue,
   innerbottommargin=10pt,
   % backgroundcolor=bg,font=\color{orange}\sffamily, fontcolor=white
]{definition}

\usepackage{empheq}
\usepackage[most]{tcolorbox}

\newtcbox{\mymath}[1][]{%
    nobeforeafter, math upper, tcbox raise base,
    enhanced, colframe=blue!30!black,
    colback=red!10, boxrule=1pt,
    #1}

\usepackage{unixode}


\DeclareMathOperator{\ord}{ord}
\DeclareMathOperator{\orb}{orb}
\DeclareMathOperator{\stab}{stab}
\DeclareMathOperator{\Stab}{stab}
\DeclareMathOperator{\ppcm}{ppcm}
\DeclareMathOperator{\conj}{Conj}
\DeclareMathOperator{\End}{End}
\DeclareMathOperator{\rot}{rot}
\DeclareMathOperator{\trs}{trace}
\DeclareMathOperator{\Ind}{Ind}
\DeclareMathOperator{\mat}{Mat}
\DeclareMathOperator{\id}{Id}
\DeclareMathOperator{\vect}{vect}
\DeclareMathOperator{\img}{img}
\DeclareMathOperator{\cov}{Cov}
\DeclareMathOperator{\dist}{dist}
\DeclareMathOperator{\irr}{Irr}
\DeclareMathOperator{\image}{Im}
\DeclareMathOperator{\pd}{\partial}
\DeclareMathOperator{\epi}{epi}
\DeclareMathOperator{\Argmin}{Argmin}
\DeclareMathOperator{\dom}{dom}
\DeclareMathOperator{\proj}{proj}
\DeclareMathOperator{\ctg}{ctg}
\DeclareMathOperator{\supp}{supp}
\DeclareMathOperator{\argmin}{argmin}
\DeclareMathOperator{\mult}{mult}
\DeclareMathOperator{\ch}{ch}
\DeclareMathOperator{\sh}{sh}
\DeclareMathOperator{\rang}{rang}
\DeclareMathOperator{\diam}{diam}
\DeclareMathOperator{\Epigraphe}{Epigraphe}




\usepackage{xcolor}
\everymath{\color{blue}}
%\everymath{\color[rgb]{0,1,1}}
%\pagecolor[rgb]{0,0,0.5}


\newcommand*{\pdtest}[3][]{\ensuremath{\frac{\partial^{#1} #2}{\partial #3}}}

\newcommand*{\deffunc}[6][]{\ensuremath{
\begin{array}{rcl}
#2 : #3 &\rightarrow& #4\\
#5 &\mapsto& #6
\end{array}
}}

\newcommand{\eqcolon}{\mathrel{\resizebox{\widthof{$\mathord{=}$}}{\height}{ $\!\!=\!\!\resizebox{1.2\width}{0.8\height}{\raisebox{0.23ex}{$\mathop{:}$}}\!\!$ }}}
\newcommand{\coloneq}{\mathrel{\resizebox{\widthof{$\mathord{=}$}}{\height}{ $\!\!\resizebox{1.2\width}{0.8\height}{\raisebox{0.23ex}{$\mathop{:}$}}\!\!=\!\!$ }}}
\newcommand{\eqcolonl}{\ensuremath{\mathrel{=\!\!\mathop{:}}}}
\newcommand{\coloneql}{\ensuremath{\mathrel{\mathop{:} \!\! =}}}
\newcommand{\vc}[1]{% inline column vector
  \left(\begin{smallmatrix}#1\end{smallmatrix}\right)%
}
\newcommand{\vr}[1]{% inline row vector
  \begin{smallmatrix}(\,#1\,)\end{smallmatrix}%
}
\makeatletter
\newcommand*{\defeq}{\ =\mathrel{\rlap{%
                     \raisebox{0.3ex}{$\m@th\cdot$}}%
                     \raisebox{-0.3ex}{$\m@th\cdot$}}%
                     }
\makeatother

\newcommand{\mathcircle}[1]{% inline row vector
 \overset{\circ}{#1}
}
\newcommand{\ulim}{% low limit
 \underline{\lim}
}
\newcommand{\ssi}{% iff
\iff
}
\newcommand{\ps}[2]{
\expval{#1 | #2}
}
\newcommand{\df}[1]{
\mqty{#1}
}
\newcommand{\n}[1]{
\norm{#1}
}
\newcommand{\sys}[1]{
\left\{\smqty{#1}\right.
}


\newcommand{\eqdef}{\ensuremath{\overset{\text{def}}=}}


\def\Circlearrowright{\ensuremath{%
  \rotatebox[origin=c]{230}{$\circlearrowright$}}}

\newcommand\ct[1]{\text{\rmfamily\upshape #1}}
\newcommand\question[1]{ {\color{red} ...!? \small #1}}
\newcommand\caz[1]{\left\{\begin{array} #1 \end{array}\right.}
\newcommand\const{\text{\rmfamily\upshape const}}
\newcommand\toP{ \overset{\pro}{\to}}
\newcommand\toPP{ \overset{\text{PP}}{\to}}
\newcommand{\oeq}{\mathrel{\text{\textcircled{$=$}}}}





\usepackage{xcolor}
% \usepackage[normalem]{ulem}
\usepackage{lipsum}
\makeatletter
% \newcommand\colorwave[1][blue]{\bgroup \markoverwith{\lower3.5\p@\hbox{\sixly \textcolor{#1}{\char58}}}\ULon}
%\font\sixly=lasy6 % does not re-load if already loaded, so no memory problem.

\newmdtheoremenv[
linewidth= 1pt,linecolor= blue,%
leftmargin=20,rightmargin=20,innertopmargin=0pt, innerrightmargin=40,%
tikzsetting = { draw=lightgray, line width = 0.3pt,dashed,%
dash pattern = on 15pt off 3pt},%
splittopskip=\topskip,skipbelow=\baselineskip,%
skipabove=\baselineskip,ntheorem,roundcorner=0pt,
% backgroundcolor=pagebg,font=\color{orange}\sffamily, fontcolor=white
]{examplebox}{Exemple}[section]



\newcommand\R{\mathbb{R}}
\newcommand\Z{\mathbb{Z}}
\newcommand\N{\mathbb{N}}
\newcommand\E{\mathbb{E}}
\newcommand\F{\mathcal{F}}
\newcommand\cH{\mathcal{H}}
\newcommand\V{\mathbb{V}}
\newcommand\dmo{ ^{-1} }
\newcommand\kapa{\kappa}
\newcommand\im{Im}
\newcommand\hs{\mathcal{H}}





\usepackage{soul}

\makeatletter
\newcommand*{\whiten}[1]{\llap{\textcolor{white}{{\the\SOUL@token}}\hspace{#1pt}}}
\DeclareRobustCommand*\myul{%
    \def\SOUL@everyspace{\underline{\space}\kern\z@}%
    \def\SOUL@everytoken{%
     \setbox0=\hbox{\the\SOUL@token}%
     \ifdim\dp0>\z@
        \raisebox{\dp0}{\underline{\phantom{\the\SOUL@token}}}%
        \whiten{1}\whiten{0}%
        \whiten{-1}\whiten{-2}%
        \llap{\the\SOUL@token}%
     \else
        \underline{\the\SOUL@token}%
     \fi}%
\SOUL@}
\makeatother

\newcommand*{\demp}{\fontfamily{lmtt}\selectfont}

\DeclareTextFontCommand{\textdemp}{\demp}

\begin{document}

\ifcomment
Multiline
comment
\fi
\ifcomment
\myul{Typesetting test}
% \color[rgb]{1,1,1}
$∑_i^n≠ 60º±∞π∆¬≈√j∫h≤≥µ$

$\CR \R\pro\ind\pro\gS\pro
\mqty[a&b\\c&d]$
$\pro\mathbb{P}$
$\dd{x}$

  \[
    \alpha(x)=\left\{
                \begin{array}{ll}
                  x\\
                  \frac{1}{1+e^{-kx}}\\
                  \frac{e^x-e^{-x}}{e^x+e^{-x}}
                \end{array}
              \right.
  \]

  $\expval{x}$
  
  $\chi_\rho(ghg\dmo)=\Tr(\rho_{ghg\dmo})=\Tr(\rho_g\circ\rho_h\circ\rho\dmo_g)=\Tr(\rho_h)\overset{\mbox{\scalebox{0.5}{$\Tr(AB)=\Tr(BA)$}}}{=}\chi_\rho(h)$
  	$\mathop{\oplus}_{\substack{x\in X}}$

$\mat(\rho_g)=(a_{ij}(g))_{\scriptsize \substack{1\leq i\leq d \\ 1\leq j\leq d}}$ et $\mat(\rho'_g)=(a'_{ij}(g))_{\scriptsize \substack{1\leq i'\leq d' \\ 1\leq j'\leq d'}}$



\[\int_a^b{\mathbb{R}^2}g(u, v)\dd{P_{XY}}(u, v)=\iint g(u,v) f_{XY}(u, v)\dd \lambda(u) \dd \lambda(v)\]
$$\lim_{x\to\infty} f(x)$$	
$$\iiiint_V \mu(t,u,v,w) \,dt\,du\,dv\,dw$$
$$\sum_{n=1}^{\infty} 2^{-n} = 1$$	
\begin{definition}
	Si $X$ et $Y$ sont 2 v.a. ou definit la \textsc{Covariance} entre $X$ et $Y$ comme
	$\cov(X,Y)\overset{\text{def}}{=}\E\left[(X-\E(X))(Y-\E(Y))\right]=\E(XY)-\E(X)\E(Y)$.
\end{definition}
\fi
\pagebreak

% \tableofcontents

% insert your code here
%\input{./algebra/main.tex}
%\input{./geometrie-differentielle/main.tex}
%\input{./probabilite/main.tex}
%\input{./analyse-fonctionnelle/main.tex}
% \input{./Analyse-convexe-et-dualite-en-optimisation/main.tex}
%\input{./tikz/main.tex}
%\input{./Theorie-du-distributions/main.tex}
%\input{./optimisation/mine.tex}
 \input{./modelisation/main.tex}

% yves.aubry@univ-tln.fr : algebra

\end{document}

%% !TEX encoding = UTF-8 Unicode
% !TEX TS-program = xelatex

\documentclass[french]{report}

%\usepackage[utf8]{inputenc}
%\usepackage[T1]{fontenc}
\usepackage{babel}


\newif\ifcomment
%\commenttrue # Show comments

\usepackage{physics}
\usepackage{amssymb}


\usepackage{amsthm}
% \usepackage{thmtools}
\usepackage{mathtools}
\usepackage{amsfonts}

\usepackage{color}

\usepackage{tikz}

\usepackage{geometry}
\geometry{a5paper, margin=0.1in, right=1cm}

\usepackage{dsfont}

\usepackage{graphicx}
\graphicspath{ {images/} }

\usepackage{faktor}

\usepackage{IEEEtrantools}
\usepackage{enumerate}   
\usepackage[PostScript=dvips]{"/Users/aware/Documents/Courses/diagrams"}


\newtheorem{theorem}{Théorème}[section]
\renewcommand{\thetheorem}{\arabic{theorem}}
\newtheorem{lemme}{Lemme}[section]
\renewcommand{\thelemme}{\arabic{lemme}}
\newtheorem{proposition}{Proposition}[section]
\renewcommand{\theproposition}{\arabic{proposition}}
\newtheorem{notations}{Notations}[section]
\newtheorem{problem}{Problème}[section]
\newtheorem{corollary}{Corollaire}[theorem]
\renewcommand{\thecorollary}{\arabic{corollary}}
\newtheorem{property}{Propriété}[section]
\newtheorem{objective}{Objectif}[section]

\theoremstyle{definition}
\newtheorem{definition}{Définition}[section]
\renewcommand{\thedefinition}{\arabic{definition}}
\newtheorem{exercise}{Exercice}[chapter]
\renewcommand{\theexercise}{\arabic{exercise}}
\newtheorem{example}{Exemple}[chapter]
\renewcommand{\theexample}{\arabic{example}}
\newtheorem*{solution}{Solution}
\newtheorem*{application}{Application}
\newtheorem*{notation}{Notation}
\newtheorem*{vocabulary}{Vocabulaire}
\newtheorem*{properties}{Propriétés}



\theoremstyle{remark}
\newtheorem*{remark}{Remarque}
\newtheorem*{rappel}{Rappel}


\usepackage{etoolbox}
\AtBeginEnvironment{exercise}{\small}
\AtBeginEnvironment{example}{\small}

\usepackage{cases}
\usepackage[red]{mypack}

\usepackage[framemethod=TikZ]{mdframed}

\definecolor{bg}{rgb}{0.4,0.25,0.95}
\definecolor{pagebg}{rgb}{0,0,0.5}
\surroundwithmdframed[
   topline=false,
   rightline=false,
   bottomline=false,
   leftmargin=\parindent,
   skipabove=8pt,
   skipbelow=8pt,
   linecolor=blue,
   innerbottommargin=10pt,
   % backgroundcolor=bg,font=\color{orange}\sffamily, fontcolor=white
]{definition}

\usepackage{empheq}
\usepackage[most]{tcolorbox}

\newtcbox{\mymath}[1][]{%
    nobeforeafter, math upper, tcbox raise base,
    enhanced, colframe=blue!30!black,
    colback=red!10, boxrule=1pt,
    #1}

\usepackage{unixode}


\DeclareMathOperator{\ord}{ord}
\DeclareMathOperator{\orb}{orb}
\DeclareMathOperator{\stab}{stab}
\DeclareMathOperator{\Stab}{stab}
\DeclareMathOperator{\ppcm}{ppcm}
\DeclareMathOperator{\conj}{Conj}
\DeclareMathOperator{\End}{End}
\DeclareMathOperator{\rot}{rot}
\DeclareMathOperator{\trs}{trace}
\DeclareMathOperator{\Ind}{Ind}
\DeclareMathOperator{\mat}{Mat}
\DeclareMathOperator{\id}{Id}
\DeclareMathOperator{\vect}{vect}
\DeclareMathOperator{\img}{img}
\DeclareMathOperator{\cov}{Cov}
\DeclareMathOperator{\dist}{dist}
\DeclareMathOperator{\irr}{Irr}
\DeclareMathOperator{\image}{Im}
\DeclareMathOperator{\pd}{\partial}
\DeclareMathOperator{\epi}{epi}
\DeclareMathOperator{\Argmin}{Argmin}
\DeclareMathOperator{\dom}{dom}
\DeclareMathOperator{\proj}{proj}
\DeclareMathOperator{\ctg}{ctg}
\DeclareMathOperator{\supp}{supp}
\DeclareMathOperator{\argmin}{argmin}
\DeclareMathOperator{\mult}{mult}
\DeclareMathOperator{\ch}{ch}
\DeclareMathOperator{\sh}{sh}
\DeclareMathOperator{\rang}{rang}
\DeclareMathOperator{\diam}{diam}
\DeclareMathOperator{\Epigraphe}{Epigraphe}




\usepackage{xcolor}
\everymath{\color{blue}}
%\everymath{\color[rgb]{0,1,1}}
%\pagecolor[rgb]{0,0,0.5}


\newcommand*{\pdtest}[3][]{\ensuremath{\frac{\partial^{#1} #2}{\partial #3}}}

\newcommand*{\deffunc}[6][]{\ensuremath{
\begin{array}{rcl}
#2 : #3 &\rightarrow& #4\\
#5 &\mapsto& #6
\end{array}
}}

\newcommand{\eqcolon}{\mathrel{\resizebox{\widthof{$\mathord{=}$}}{\height}{ $\!\!=\!\!\resizebox{1.2\width}{0.8\height}{\raisebox{0.23ex}{$\mathop{:}$}}\!\!$ }}}
\newcommand{\coloneq}{\mathrel{\resizebox{\widthof{$\mathord{=}$}}{\height}{ $\!\!\resizebox{1.2\width}{0.8\height}{\raisebox{0.23ex}{$\mathop{:}$}}\!\!=\!\!$ }}}
\newcommand{\eqcolonl}{\ensuremath{\mathrel{=\!\!\mathop{:}}}}
\newcommand{\coloneql}{\ensuremath{\mathrel{\mathop{:} \!\! =}}}
\newcommand{\vc}[1]{% inline column vector
  \left(\begin{smallmatrix}#1\end{smallmatrix}\right)%
}
\newcommand{\vr}[1]{% inline row vector
  \begin{smallmatrix}(\,#1\,)\end{smallmatrix}%
}
\makeatletter
\newcommand*{\defeq}{\ =\mathrel{\rlap{%
                     \raisebox{0.3ex}{$\m@th\cdot$}}%
                     \raisebox{-0.3ex}{$\m@th\cdot$}}%
                     }
\makeatother

\newcommand{\mathcircle}[1]{% inline row vector
 \overset{\circ}{#1}
}
\newcommand{\ulim}{% low limit
 \underline{\lim}
}
\newcommand{\ssi}{% iff
\iff
}
\newcommand{\ps}[2]{
\expval{#1 | #2}
}
\newcommand{\df}[1]{
\mqty{#1}
}
\newcommand{\n}[1]{
\norm{#1}
}
\newcommand{\sys}[1]{
\left\{\smqty{#1}\right.
}


\newcommand{\eqdef}{\ensuremath{\overset{\text{def}}=}}


\def\Circlearrowright{\ensuremath{%
  \rotatebox[origin=c]{230}{$\circlearrowright$}}}

\newcommand\ct[1]{\text{\rmfamily\upshape #1}}
\newcommand\question[1]{ {\color{red} ...!? \small #1}}
\newcommand\caz[1]{\left\{\begin{array} #1 \end{array}\right.}
\newcommand\const{\text{\rmfamily\upshape const}}
\newcommand\toP{ \overset{\pro}{\to}}
\newcommand\toPP{ \overset{\text{PP}}{\to}}
\newcommand{\oeq}{\mathrel{\text{\textcircled{$=$}}}}





\usepackage{xcolor}
% \usepackage[normalem]{ulem}
\usepackage{lipsum}
\makeatletter
% \newcommand\colorwave[1][blue]{\bgroup \markoverwith{\lower3.5\p@\hbox{\sixly \textcolor{#1}{\char58}}}\ULon}
%\font\sixly=lasy6 % does not re-load if already loaded, so no memory problem.

\newmdtheoremenv[
linewidth= 1pt,linecolor= blue,%
leftmargin=20,rightmargin=20,innertopmargin=0pt, innerrightmargin=40,%
tikzsetting = { draw=lightgray, line width = 0.3pt,dashed,%
dash pattern = on 15pt off 3pt},%
splittopskip=\topskip,skipbelow=\baselineskip,%
skipabove=\baselineskip,ntheorem,roundcorner=0pt,
% backgroundcolor=pagebg,font=\color{orange}\sffamily, fontcolor=white
]{examplebox}{Exemple}[section]



\newcommand\R{\mathbb{R}}
\newcommand\Z{\mathbb{Z}}
\newcommand\N{\mathbb{N}}
\newcommand\E{\mathbb{E}}
\newcommand\F{\mathcal{F}}
\newcommand\cH{\mathcal{H}}
\newcommand\V{\mathbb{V}}
\newcommand\dmo{ ^{-1} }
\newcommand\kapa{\kappa}
\newcommand\im{Im}
\newcommand\hs{\mathcal{H}}





\usepackage{soul}

\makeatletter
\newcommand*{\whiten}[1]{\llap{\textcolor{white}{{\the\SOUL@token}}\hspace{#1pt}}}
\DeclareRobustCommand*\myul{%
    \def\SOUL@everyspace{\underline{\space}\kern\z@}%
    \def\SOUL@everytoken{%
     \setbox0=\hbox{\the\SOUL@token}%
     \ifdim\dp0>\z@
        \raisebox{\dp0}{\underline{\phantom{\the\SOUL@token}}}%
        \whiten{1}\whiten{0}%
        \whiten{-1}\whiten{-2}%
        \llap{\the\SOUL@token}%
     \else
        \underline{\the\SOUL@token}%
     \fi}%
\SOUL@}
\makeatother

\newcommand*{\demp}{\fontfamily{lmtt}\selectfont}

\DeclareTextFontCommand{\textdemp}{\demp}

\begin{document}

\ifcomment
Multiline
comment
\fi
\ifcomment
\myul{Typesetting test}
% \color[rgb]{1,1,1}
$∑_i^n≠ 60º±∞π∆¬≈√j∫h≤≥µ$

$\CR \R\pro\ind\pro\gS\pro
\mqty[a&b\\c&d]$
$\pro\mathbb{P}$
$\dd{x}$

  \[
    \alpha(x)=\left\{
                \begin{array}{ll}
                  x\\
                  \frac{1}{1+e^{-kx}}\\
                  \frac{e^x-e^{-x}}{e^x+e^{-x}}
                \end{array}
              \right.
  \]

  $\expval{x}$
  
  $\chi_\rho(ghg\dmo)=\Tr(\rho_{ghg\dmo})=\Tr(\rho_g\circ\rho_h\circ\rho\dmo_g)=\Tr(\rho_h)\overset{\mbox{\scalebox{0.5}{$\Tr(AB)=\Tr(BA)$}}}{=}\chi_\rho(h)$
  	$\mathop{\oplus}_{\substack{x\in X}}$

$\mat(\rho_g)=(a_{ij}(g))_{\scriptsize \substack{1\leq i\leq d \\ 1\leq j\leq d}}$ et $\mat(\rho'_g)=(a'_{ij}(g))_{\scriptsize \substack{1\leq i'\leq d' \\ 1\leq j'\leq d'}}$



\[\int_a^b{\mathbb{R}^2}g(u, v)\dd{P_{XY}}(u, v)=\iint g(u,v) f_{XY}(u, v)\dd \lambda(u) \dd \lambda(v)\]
$$\lim_{x\to\infty} f(x)$$	
$$\iiiint_V \mu(t,u,v,w) \,dt\,du\,dv\,dw$$
$$\sum_{n=1}^{\infty} 2^{-n} = 1$$	
\begin{definition}
	Si $X$ et $Y$ sont 2 v.a. ou definit la \textsc{Covariance} entre $X$ et $Y$ comme
	$\cov(X,Y)\overset{\text{def}}{=}\E\left[(X-\E(X))(Y-\E(Y))\right]=\E(XY)-\E(X)\E(Y)$.
\end{definition}
\fi
\pagebreak

% \tableofcontents

% insert your code here
%\input{./algebra/main.tex}
%\input{./geometrie-differentielle/main.tex}
%\input{./probabilite/main.tex}
%\input{./analyse-fonctionnelle/main.tex}
% \input{./Analyse-convexe-et-dualite-en-optimisation/main.tex}
%\input{./tikz/main.tex}
%\input{./Theorie-du-distributions/main.tex}
%\input{./optimisation/mine.tex}
 \input{./modelisation/main.tex}

% yves.aubry@univ-tln.fr : algebra

\end{document}

%% !TEX encoding = UTF-8 Unicode
% !TEX TS-program = xelatex

\documentclass[french]{report}

%\usepackage[utf8]{inputenc}
%\usepackage[T1]{fontenc}
\usepackage{babel}


\newif\ifcomment
%\commenttrue # Show comments

\usepackage{physics}
\usepackage{amssymb}


\usepackage{amsthm}
% \usepackage{thmtools}
\usepackage{mathtools}
\usepackage{amsfonts}

\usepackage{color}

\usepackage{tikz}

\usepackage{geometry}
\geometry{a5paper, margin=0.1in, right=1cm}

\usepackage{dsfont}

\usepackage{graphicx}
\graphicspath{ {images/} }

\usepackage{faktor}

\usepackage{IEEEtrantools}
\usepackage{enumerate}   
\usepackage[PostScript=dvips]{"/Users/aware/Documents/Courses/diagrams"}


\newtheorem{theorem}{Théorème}[section]
\renewcommand{\thetheorem}{\arabic{theorem}}
\newtheorem{lemme}{Lemme}[section]
\renewcommand{\thelemme}{\arabic{lemme}}
\newtheorem{proposition}{Proposition}[section]
\renewcommand{\theproposition}{\arabic{proposition}}
\newtheorem{notations}{Notations}[section]
\newtheorem{problem}{Problème}[section]
\newtheorem{corollary}{Corollaire}[theorem]
\renewcommand{\thecorollary}{\arabic{corollary}}
\newtheorem{property}{Propriété}[section]
\newtheorem{objective}{Objectif}[section]

\theoremstyle{definition}
\newtheorem{definition}{Définition}[section]
\renewcommand{\thedefinition}{\arabic{definition}}
\newtheorem{exercise}{Exercice}[chapter]
\renewcommand{\theexercise}{\arabic{exercise}}
\newtheorem{example}{Exemple}[chapter]
\renewcommand{\theexample}{\arabic{example}}
\newtheorem*{solution}{Solution}
\newtheorem*{application}{Application}
\newtheorem*{notation}{Notation}
\newtheorem*{vocabulary}{Vocabulaire}
\newtheorem*{properties}{Propriétés}



\theoremstyle{remark}
\newtheorem*{remark}{Remarque}
\newtheorem*{rappel}{Rappel}


\usepackage{etoolbox}
\AtBeginEnvironment{exercise}{\small}
\AtBeginEnvironment{example}{\small}

\usepackage{cases}
\usepackage[red]{mypack}

\usepackage[framemethod=TikZ]{mdframed}

\definecolor{bg}{rgb}{0.4,0.25,0.95}
\definecolor{pagebg}{rgb}{0,0,0.5}
\surroundwithmdframed[
   topline=false,
   rightline=false,
   bottomline=false,
   leftmargin=\parindent,
   skipabove=8pt,
   skipbelow=8pt,
   linecolor=blue,
   innerbottommargin=10pt,
   % backgroundcolor=bg,font=\color{orange}\sffamily, fontcolor=white
]{definition}

\usepackage{empheq}
\usepackage[most]{tcolorbox}

\newtcbox{\mymath}[1][]{%
    nobeforeafter, math upper, tcbox raise base,
    enhanced, colframe=blue!30!black,
    colback=red!10, boxrule=1pt,
    #1}

\usepackage{unixode}


\DeclareMathOperator{\ord}{ord}
\DeclareMathOperator{\orb}{orb}
\DeclareMathOperator{\stab}{stab}
\DeclareMathOperator{\Stab}{stab}
\DeclareMathOperator{\ppcm}{ppcm}
\DeclareMathOperator{\conj}{Conj}
\DeclareMathOperator{\End}{End}
\DeclareMathOperator{\rot}{rot}
\DeclareMathOperator{\trs}{trace}
\DeclareMathOperator{\Ind}{Ind}
\DeclareMathOperator{\mat}{Mat}
\DeclareMathOperator{\id}{Id}
\DeclareMathOperator{\vect}{vect}
\DeclareMathOperator{\img}{img}
\DeclareMathOperator{\cov}{Cov}
\DeclareMathOperator{\dist}{dist}
\DeclareMathOperator{\irr}{Irr}
\DeclareMathOperator{\image}{Im}
\DeclareMathOperator{\pd}{\partial}
\DeclareMathOperator{\epi}{epi}
\DeclareMathOperator{\Argmin}{Argmin}
\DeclareMathOperator{\dom}{dom}
\DeclareMathOperator{\proj}{proj}
\DeclareMathOperator{\ctg}{ctg}
\DeclareMathOperator{\supp}{supp}
\DeclareMathOperator{\argmin}{argmin}
\DeclareMathOperator{\mult}{mult}
\DeclareMathOperator{\ch}{ch}
\DeclareMathOperator{\sh}{sh}
\DeclareMathOperator{\rang}{rang}
\DeclareMathOperator{\diam}{diam}
\DeclareMathOperator{\Epigraphe}{Epigraphe}




\usepackage{xcolor}
\everymath{\color{blue}}
%\everymath{\color[rgb]{0,1,1}}
%\pagecolor[rgb]{0,0,0.5}


\newcommand*{\pdtest}[3][]{\ensuremath{\frac{\partial^{#1} #2}{\partial #3}}}

\newcommand*{\deffunc}[6][]{\ensuremath{
\begin{array}{rcl}
#2 : #3 &\rightarrow& #4\\
#5 &\mapsto& #6
\end{array}
}}

\newcommand{\eqcolon}{\mathrel{\resizebox{\widthof{$\mathord{=}$}}{\height}{ $\!\!=\!\!\resizebox{1.2\width}{0.8\height}{\raisebox{0.23ex}{$\mathop{:}$}}\!\!$ }}}
\newcommand{\coloneq}{\mathrel{\resizebox{\widthof{$\mathord{=}$}}{\height}{ $\!\!\resizebox{1.2\width}{0.8\height}{\raisebox{0.23ex}{$\mathop{:}$}}\!\!=\!\!$ }}}
\newcommand{\eqcolonl}{\ensuremath{\mathrel{=\!\!\mathop{:}}}}
\newcommand{\coloneql}{\ensuremath{\mathrel{\mathop{:} \!\! =}}}
\newcommand{\vc}[1]{% inline column vector
  \left(\begin{smallmatrix}#1\end{smallmatrix}\right)%
}
\newcommand{\vr}[1]{% inline row vector
  \begin{smallmatrix}(\,#1\,)\end{smallmatrix}%
}
\makeatletter
\newcommand*{\defeq}{\ =\mathrel{\rlap{%
                     \raisebox{0.3ex}{$\m@th\cdot$}}%
                     \raisebox{-0.3ex}{$\m@th\cdot$}}%
                     }
\makeatother

\newcommand{\mathcircle}[1]{% inline row vector
 \overset{\circ}{#1}
}
\newcommand{\ulim}{% low limit
 \underline{\lim}
}
\newcommand{\ssi}{% iff
\iff
}
\newcommand{\ps}[2]{
\expval{#1 | #2}
}
\newcommand{\df}[1]{
\mqty{#1}
}
\newcommand{\n}[1]{
\norm{#1}
}
\newcommand{\sys}[1]{
\left\{\smqty{#1}\right.
}


\newcommand{\eqdef}{\ensuremath{\overset{\text{def}}=}}


\def\Circlearrowright{\ensuremath{%
  \rotatebox[origin=c]{230}{$\circlearrowright$}}}

\newcommand\ct[1]{\text{\rmfamily\upshape #1}}
\newcommand\question[1]{ {\color{red} ...!? \small #1}}
\newcommand\caz[1]{\left\{\begin{array} #1 \end{array}\right.}
\newcommand\const{\text{\rmfamily\upshape const}}
\newcommand\toP{ \overset{\pro}{\to}}
\newcommand\toPP{ \overset{\text{PP}}{\to}}
\newcommand{\oeq}{\mathrel{\text{\textcircled{$=$}}}}





\usepackage{xcolor}
% \usepackage[normalem]{ulem}
\usepackage{lipsum}
\makeatletter
% \newcommand\colorwave[1][blue]{\bgroup \markoverwith{\lower3.5\p@\hbox{\sixly \textcolor{#1}{\char58}}}\ULon}
%\font\sixly=lasy6 % does not re-load if already loaded, so no memory problem.

\newmdtheoremenv[
linewidth= 1pt,linecolor= blue,%
leftmargin=20,rightmargin=20,innertopmargin=0pt, innerrightmargin=40,%
tikzsetting = { draw=lightgray, line width = 0.3pt,dashed,%
dash pattern = on 15pt off 3pt},%
splittopskip=\topskip,skipbelow=\baselineskip,%
skipabove=\baselineskip,ntheorem,roundcorner=0pt,
% backgroundcolor=pagebg,font=\color{orange}\sffamily, fontcolor=white
]{examplebox}{Exemple}[section]



\newcommand\R{\mathbb{R}}
\newcommand\Z{\mathbb{Z}}
\newcommand\N{\mathbb{N}}
\newcommand\E{\mathbb{E}}
\newcommand\F{\mathcal{F}}
\newcommand\cH{\mathcal{H}}
\newcommand\V{\mathbb{V}}
\newcommand\dmo{ ^{-1} }
\newcommand\kapa{\kappa}
\newcommand\im{Im}
\newcommand\hs{\mathcal{H}}





\usepackage{soul}

\makeatletter
\newcommand*{\whiten}[1]{\llap{\textcolor{white}{{\the\SOUL@token}}\hspace{#1pt}}}
\DeclareRobustCommand*\myul{%
    \def\SOUL@everyspace{\underline{\space}\kern\z@}%
    \def\SOUL@everytoken{%
     \setbox0=\hbox{\the\SOUL@token}%
     \ifdim\dp0>\z@
        \raisebox{\dp0}{\underline{\phantom{\the\SOUL@token}}}%
        \whiten{1}\whiten{0}%
        \whiten{-1}\whiten{-2}%
        \llap{\the\SOUL@token}%
     \else
        \underline{\the\SOUL@token}%
     \fi}%
\SOUL@}
\makeatother

\newcommand*{\demp}{\fontfamily{lmtt}\selectfont}

\DeclareTextFontCommand{\textdemp}{\demp}

\begin{document}

\ifcomment
Multiline
comment
\fi
\ifcomment
\myul{Typesetting test}
% \color[rgb]{1,1,1}
$∑_i^n≠ 60º±∞π∆¬≈√j∫h≤≥µ$

$\CR \R\pro\ind\pro\gS\pro
\mqty[a&b\\c&d]$
$\pro\mathbb{P}$
$\dd{x}$

  \[
    \alpha(x)=\left\{
                \begin{array}{ll}
                  x\\
                  \frac{1}{1+e^{-kx}}\\
                  \frac{e^x-e^{-x}}{e^x+e^{-x}}
                \end{array}
              \right.
  \]

  $\expval{x}$
  
  $\chi_\rho(ghg\dmo)=\Tr(\rho_{ghg\dmo})=\Tr(\rho_g\circ\rho_h\circ\rho\dmo_g)=\Tr(\rho_h)\overset{\mbox{\scalebox{0.5}{$\Tr(AB)=\Tr(BA)$}}}{=}\chi_\rho(h)$
  	$\mathop{\oplus}_{\substack{x\in X}}$

$\mat(\rho_g)=(a_{ij}(g))_{\scriptsize \substack{1\leq i\leq d \\ 1\leq j\leq d}}$ et $\mat(\rho'_g)=(a'_{ij}(g))_{\scriptsize \substack{1\leq i'\leq d' \\ 1\leq j'\leq d'}}$



\[\int_a^b{\mathbb{R}^2}g(u, v)\dd{P_{XY}}(u, v)=\iint g(u,v) f_{XY}(u, v)\dd \lambda(u) \dd \lambda(v)\]
$$\lim_{x\to\infty} f(x)$$	
$$\iiiint_V \mu(t,u,v,w) \,dt\,du\,dv\,dw$$
$$\sum_{n=1}^{\infty} 2^{-n} = 1$$	
\begin{definition}
	Si $X$ et $Y$ sont 2 v.a. ou definit la \textsc{Covariance} entre $X$ et $Y$ comme
	$\cov(X,Y)\overset{\text{def}}{=}\E\left[(X-\E(X))(Y-\E(Y))\right]=\E(XY)-\E(X)\E(Y)$.
\end{definition}
\fi
\pagebreak

% \tableofcontents

% insert your code here
%\input{./algebra/main.tex}
%\input{./geometrie-differentielle/main.tex}
%\input{./probabilite/main.tex}
%\input{./analyse-fonctionnelle/main.tex}
% \input{./Analyse-convexe-et-dualite-en-optimisation/main.tex}
%\input{./tikz/main.tex}
%\input{./Theorie-du-distributions/main.tex}
%\input{./optimisation/mine.tex}
 \input{./modelisation/main.tex}

% yves.aubry@univ-tln.fr : algebra

\end{document}

%\input{./optimisation/mine.tex}
 % !TEX encoding = UTF-8 Unicode
% !TEX TS-program = xelatex

\documentclass[french]{report}

%\usepackage[utf8]{inputenc}
%\usepackage[T1]{fontenc}
\usepackage{babel}


\newif\ifcomment
%\commenttrue # Show comments

\usepackage{physics}
\usepackage{amssymb}


\usepackage{amsthm}
% \usepackage{thmtools}
\usepackage{mathtools}
\usepackage{amsfonts}

\usepackage{color}

\usepackage{tikz}

\usepackage{geometry}
\geometry{a5paper, margin=0.1in, right=1cm}

\usepackage{dsfont}

\usepackage{graphicx}
\graphicspath{ {images/} }

\usepackage{faktor}

\usepackage{IEEEtrantools}
\usepackage{enumerate}   
\usepackage[PostScript=dvips]{"/Users/aware/Documents/Courses/diagrams"}


\newtheorem{theorem}{Théorème}[section]
\renewcommand{\thetheorem}{\arabic{theorem}}
\newtheorem{lemme}{Lemme}[section]
\renewcommand{\thelemme}{\arabic{lemme}}
\newtheorem{proposition}{Proposition}[section]
\renewcommand{\theproposition}{\arabic{proposition}}
\newtheorem{notations}{Notations}[section]
\newtheorem{problem}{Problème}[section]
\newtheorem{corollary}{Corollaire}[theorem]
\renewcommand{\thecorollary}{\arabic{corollary}}
\newtheorem{property}{Propriété}[section]
\newtheorem{objective}{Objectif}[section]

\theoremstyle{definition}
\newtheorem{definition}{Définition}[section]
\renewcommand{\thedefinition}{\arabic{definition}}
\newtheorem{exercise}{Exercice}[chapter]
\renewcommand{\theexercise}{\arabic{exercise}}
\newtheorem{example}{Exemple}[chapter]
\renewcommand{\theexample}{\arabic{example}}
\newtheorem*{solution}{Solution}
\newtheorem*{application}{Application}
\newtheorem*{notation}{Notation}
\newtheorem*{vocabulary}{Vocabulaire}
\newtheorem*{properties}{Propriétés}



\theoremstyle{remark}
\newtheorem*{remark}{Remarque}
\newtheorem*{rappel}{Rappel}


\usepackage{etoolbox}
\AtBeginEnvironment{exercise}{\small}
\AtBeginEnvironment{example}{\small}

\usepackage{cases}
\usepackage[red]{mypack}

\usepackage[framemethod=TikZ]{mdframed}

\definecolor{bg}{rgb}{0.4,0.25,0.95}
\definecolor{pagebg}{rgb}{0,0,0.5}
\surroundwithmdframed[
   topline=false,
   rightline=false,
   bottomline=false,
   leftmargin=\parindent,
   skipabove=8pt,
   skipbelow=8pt,
   linecolor=blue,
   innerbottommargin=10pt,
   % backgroundcolor=bg,font=\color{orange}\sffamily, fontcolor=white
]{definition}

\usepackage{empheq}
\usepackage[most]{tcolorbox}

\newtcbox{\mymath}[1][]{%
    nobeforeafter, math upper, tcbox raise base,
    enhanced, colframe=blue!30!black,
    colback=red!10, boxrule=1pt,
    #1}

\usepackage{unixode}


\DeclareMathOperator{\ord}{ord}
\DeclareMathOperator{\orb}{orb}
\DeclareMathOperator{\stab}{stab}
\DeclareMathOperator{\Stab}{stab}
\DeclareMathOperator{\ppcm}{ppcm}
\DeclareMathOperator{\conj}{Conj}
\DeclareMathOperator{\End}{End}
\DeclareMathOperator{\rot}{rot}
\DeclareMathOperator{\trs}{trace}
\DeclareMathOperator{\Ind}{Ind}
\DeclareMathOperator{\mat}{Mat}
\DeclareMathOperator{\id}{Id}
\DeclareMathOperator{\vect}{vect}
\DeclareMathOperator{\img}{img}
\DeclareMathOperator{\cov}{Cov}
\DeclareMathOperator{\dist}{dist}
\DeclareMathOperator{\irr}{Irr}
\DeclareMathOperator{\image}{Im}
\DeclareMathOperator{\pd}{\partial}
\DeclareMathOperator{\epi}{epi}
\DeclareMathOperator{\Argmin}{Argmin}
\DeclareMathOperator{\dom}{dom}
\DeclareMathOperator{\proj}{proj}
\DeclareMathOperator{\ctg}{ctg}
\DeclareMathOperator{\supp}{supp}
\DeclareMathOperator{\argmin}{argmin}
\DeclareMathOperator{\mult}{mult}
\DeclareMathOperator{\ch}{ch}
\DeclareMathOperator{\sh}{sh}
\DeclareMathOperator{\rang}{rang}
\DeclareMathOperator{\diam}{diam}
\DeclareMathOperator{\Epigraphe}{Epigraphe}




\usepackage{xcolor}
\everymath{\color{blue}}
%\everymath{\color[rgb]{0,1,1}}
%\pagecolor[rgb]{0,0,0.5}


\newcommand*{\pdtest}[3][]{\ensuremath{\frac{\partial^{#1} #2}{\partial #3}}}

\newcommand*{\deffunc}[6][]{\ensuremath{
\begin{array}{rcl}
#2 : #3 &\rightarrow& #4\\
#5 &\mapsto& #6
\end{array}
}}

\newcommand{\eqcolon}{\mathrel{\resizebox{\widthof{$\mathord{=}$}}{\height}{ $\!\!=\!\!\resizebox{1.2\width}{0.8\height}{\raisebox{0.23ex}{$\mathop{:}$}}\!\!$ }}}
\newcommand{\coloneq}{\mathrel{\resizebox{\widthof{$\mathord{=}$}}{\height}{ $\!\!\resizebox{1.2\width}{0.8\height}{\raisebox{0.23ex}{$\mathop{:}$}}\!\!=\!\!$ }}}
\newcommand{\eqcolonl}{\ensuremath{\mathrel{=\!\!\mathop{:}}}}
\newcommand{\coloneql}{\ensuremath{\mathrel{\mathop{:} \!\! =}}}
\newcommand{\vc}[1]{% inline column vector
  \left(\begin{smallmatrix}#1\end{smallmatrix}\right)%
}
\newcommand{\vr}[1]{% inline row vector
  \begin{smallmatrix}(\,#1\,)\end{smallmatrix}%
}
\makeatletter
\newcommand*{\defeq}{\ =\mathrel{\rlap{%
                     \raisebox{0.3ex}{$\m@th\cdot$}}%
                     \raisebox{-0.3ex}{$\m@th\cdot$}}%
                     }
\makeatother

\newcommand{\mathcircle}[1]{% inline row vector
 \overset{\circ}{#1}
}
\newcommand{\ulim}{% low limit
 \underline{\lim}
}
\newcommand{\ssi}{% iff
\iff
}
\newcommand{\ps}[2]{
\expval{#1 | #2}
}
\newcommand{\df}[1]{
\mqty{#1}
}
\newcommand{\n}[1]{
\norm{#1}
}
\newcommand{\sys}[1]{
\left\{\smqty{#1}\right.
}


\newcommand{\eqdef}{\ensuremath{\overset{\text{def}}=}}


\def\Circlearrowright{\ensuremath{%
  \rotatebox[origin=c]{230}{$\circlearrowright$}}}

\newcommand\ct[1]{\text{\rmfamily\upshape #1}}
\newcommand\question[1]{ {\color{red} ...!? \small #1}}
\newcommand\caz[1]{\left\{\begin{array} #1 \end{array}\right.}
\newcommand\const{\text{\rmfamily\upshape const}}
\newcommand\toP{ \overset{\pro}{\to}}
\newcommand\toPP{ \overset{\text{PP}}{\to}}
\newcommand{\oeq}{\mathrel{\text{\textcircled{$=$}}}}





\usepackage{xcolor}
% \usepackage[normalem]{ulem}
\usepackage{lipsum}
\makeatletter
% \newcommand\colorwave[1][blue]{\bgroup \markoverwith{\lower3.5\p@\hbox{\sixly \textcolor{#1}{\char58}}}\ULon}
%\font\sixly=lasy6 % does not re-load if already loaded, so no memory problem.

\newmdtheoremenv[
linewidth= 1pt,linecolor= blue,%
leftmargin=20,rightmargin=20,innertopmargin=0pt, innerrightmargin=40,%
tikzsetting = { draw=lightgray, line width = 0.3pt,dashed,%
dash pattern = on 15pt off 3pt},%
splittopskip=\topskip,skipbelow=\baselineskip,%
skipabove=\baselineskip,ntheorem,roundcorner=0pt,
% backgroundcolor=pagebg,font=\color{orange}\sffamily, fontcolor=white
]{examplebox}{Exemple}[section]



\newcommand\R{\mathbb{R}}
\newcommand\Z{\mathbb{Z}}
\newcommand\N{\mathbb{N}}
\newcommand\E{\mathbb{E}}
\newcommand\F{\mathcal{F}}
\newcommand\cH{\mathcal{H}}
\newcommand\V{\mathbb{V}}
\newcommand\dmo{ ^{-1} }
\newcommand\kapa{\kappa}
\newcommand\im{Im}
\newcommand\hs{\mathcal{H}}





\usepackage{soul}

\makeatletter
\newcommand*{\whiten}[1]{\llap{\textcolor{white}{{\the\SOUL@token}}\hspace{#1pt}}}
\DeclareRobustCommand*\myul{%
    \def\SOUL@everyspace{\underline{\space}\kern\z@}%
    \def\SOUL@everytoken{%
     \setbox0=\hbox{\the\SOUL@token}%
     \ifdim\dp0>\z@
        \raisebox{\dp0}{\underline{\phantom{\the\SOUL@token}}}%
        \whiten{1}\whiten{0}%
        \whiten{-1}\whiten{-2}%
        \llap{\the\SOUL@token}%
     \else
        \underline{\the\SOUL@token}%
     \fi}%
\SOUL@}
\makeatother

\newcommand*{\demp}{\fontfamily{lmtt}\selectfont}

\DeclareTextFontCommand{\textdemp}{\demp}

\begin{document}

\ifcomment
Multiline
comment
\fi
\ifcomment
\myul{Typesetting test}
% \color[rgb]{1,1,1}
$∑_i^n≠ 60º±∞π∆¬≈√j∫h≤≥µ$

$\CR \R\pro\ind\pro\gS\pro
\mqty[a&b\\c&d]$
$\pro\mathbb{P}$
$\dd{x}$

  \[
    \alpha(x)=\left\{
                \begin{array}{ll}
                  x\\
                  \frac{1}{1+e^{-kx}}\\
                  \frac{e^x-e^{-x}}{e^x+e^{-x}}
                \end{array}
              \right.
  \]

  $\expval{x}$
  
  $\chi_\rho(ghg\dmo)=\Tr(\rho_{ghg\dmo})=\Tr(\rho_g\circ\rho_h\circ\rho\dmo_g)=\Tr(\rho_h)\overset{\mbox{\scalebox{0.5}{$\Tr(AB)=\Tr(BA)$}}}{=}\chi_\rho(h)$
  	$\mathop{\oplus}_{\substack{x\in X}}$

$\mat(\rho_g)=(a_{ij}(g))_{\scriptsize \substack{1\leq i\leq d \\ 1\leq j\leq d}}$ et $\mat(\rho'_g)=(a'_{ij}(g))_{\scriptsize \substack{1\leq i'\leq d' \\ 1\leq j'\leq d'}}$



\[\int_a^b{\mathbb{R}^2}g(u, v)\dd{P_{XY}}(u, v)=\iint g(u,v) f_{XY}(u, v)\dd \lambda(u) \dd \lambda(v)\]
$$\lim_{x\to\infty} f(x)$$	
$$\iiiint_V \mu(t,u,v,w) \,dt\,du\,dv\,dw$$
$$\sum_{n=1}^{\infty} 2^{-n} = 1$$	
\begin{definition}
	Si $X$ et $Y$ sont 2 v.a. ou definit la \textsc{Covariance} entre $X$ et $Y$ comme
	$\cov(X,Y)\overset{\text{def}}{=}\E\left[(X-\E(X))(Y-\E(Y))\right]=\E(XY)-\E(X)\E(Y)$.
\end{definition}
\fi
\pagebreak

% \tableofcontents

% insert your code here
%\input{./algebra/main.tex}
%\input{./geometrie-differentielle/main.tex}
%\input{./probabilite/main.tex}
%\input{./analyse-fonctionnelle/main.tex}
% \input{./Analyse-convexe-et-dualite-en-optimisation/main.tex}
%\input{./tikz/main.tex}
%\input{./Theorie-du-distributions/main.tex}
%\input{./optimisation/mine.tex}
 \input{./modelisation/main.tex}

% yves.aubry@univ-tln.fr : algebra

\end{document}


% yves.aubry@univ-tln.fr : algebra

\end{document}

%% !TEX encoding = UTF-8 Unicode
% !TEX TS-program = xelatex

\documentclass[french]{report}

%\usepackage[utf8]{inputenc}
%\usepackage[T1]{fontenc}
\usepackage{babel}


\newif\ifcomment
%\commenttrue # Show comments

\usepackage{physics}
\usepackage{amssymb}


\usepackage{amsthm}
% \usepackage{thmtools}
\usepackage{mathtools}
\usepackage{amsfonts}

\usepackage{color}

\usepackage{tikz}

\usepackage{geometry}
\geometry{a5paper, margin=0.1in, right=1cm}

\usepackage{dsfont}

\usepackage{graphicx}
\graphicspath{ {images/} }

\usepackage{faktor}

\usepackage{IEEEtrantools}
\usepackage{enumerate}   
\usepackage[PostScript=dvips]{"/Users/aware/Documents/Courses/diagrams"}


\newtheorem{theorem}{Théorème}[section]
\renewcommand{\thetheorem}{\arabic{theorem}}
\newtheorem{lemme}{Lemme}[section]
\renewcommand{\thelemme}{\arabic{lemme}}
\newtheorem{proposition}{Proposition}[section]
\renewcommand{\theproposition}{\arabic{proposition}}
\newtheorem{notations}{Notations}[section]
\newtheorem{problem}{Problème}[section]
\newtheorem{corollary}{Corollaire}[theorem]
\renewcommand{\thecorollary}{\arabic{corollary}}
\newtheorem{property}{Propriété}[section]
\newtheorem{objective}{Objectif}[section]

\theoremstyle{definition}
\newtheorem{definition}{Définition}[section]
\renewcommand{\thedefinition}{\arabic{definition}}
\newtheorem{exercise}{Exercice}[chapter]
\renewcommand{\theexercise}{\arabic{exercise}}
\newtheorem{example}{Exemple}[chapter]
\renewcommand{\theexample}{\arabic{example}}
\newtheorem*{solution}{Solution}
\newtheorem*{application}{Application}
\newtheorem*{notation}{Notation}
\newtheorem*{vocabulary}{Vocabulaire}
\newtheorem*{properties}{Propriétés}



\theoremstyle{remark}
\newtheorem*{remark}{Remarque}
\newtheorem*{rappel}{Rappel}


\usepackage{etoolbox}
\AtBeginEnvironment{exercise}{\small}
\AtBeginEnvironment{example}{\small}

\usepackage{cases}
\usepackage[red]{mypack}

\usepackage[framemethod=TikZ]{mdframed}

\definecolor{bg}{rgb}{0.4,0.25,0.95}
\definecolor{pagebg}{rgb}{0,0,0.5}
\surroundwithmdframed[
   topline=false,
   rightline=false,
   bottomline=false,
   leftmargin=\parindent,
   skipabove=8pt,
   skipbelow=8pt,
   linecolor=blue,
   innerbottommargin=10pt,
   % backgroundcolor=bg,font=\color{orange}\sffamily, fontcolor=white
]{definition}

\usepackage{empheq}
\usepackage[most]{tcolorbox}

\newtcbox{\mymath}[1][]{%
    nobeforeafter, math upper, tcbox raise base,
    enhanced, colframe=blue!30!black,
    colback=red!10, boxrule=1pt,
    #1}

\usepackage{unixode}


\DeclareMathOperator{\ord}{ord}
\DeclareMathOperator{\orb}{orb}
\DeclareMathOperator{\stab}{stab}
\DeclareMathOperator{\Stab}{stab}
\DeclareMathOperator{\ppcm}{ppcm}
\DeclareMathOperator{\conj}{Conj}
\DeclareMathOperator{\End}{End}
\DeclareMathOperator{\rot}{rot}
\DeclareMathOperator{\trs}{trace}
\DeclareMathOperator{\Ind}{Ind}
\DeclareMathOperator{\mat}{Mat}
\DeclareMathOperator{\id}{Id}
\DeclareMathOperator{\vect}{vect}
\DeclareMathOperator{\img}{img}
\DeclareMathOperator{\cov}{Cov}
\DeclareMathOperator{\dist}{dist}
\DeclareMathOperator{\irr}{Irr}
\DeclareMathOperator{\image}{Im}
\DeclareMathOperator{\pd}{\partial}
\DeclareMathOperator{\epi}{epi}
\DeclareMathOperator{\Argmin}{Argmin}
\DeclareMathOperator{\dom}{dom}
\DeclareMathOperator{\proj}{proj}
\DeclareMathOperator{\ctg}{ctg}
\DeclareMathOperator{\supp}{supp}
\DeclareMathOperator{\argmin}{argmin}
\DeclareMathOperator{\mult}{mult}
\DeclareMathOperator{\ch}{ch}
\DeclareMathOperator{\sh}{sh}
\DeclareMathOperator{\rang}{rang}
\DeclareMathOperator{\diam}{diam}
\DeclareMathOperator{\Epigraphe}{Epigraphe}




\usepackage{xcolor}
\everymath{\color{blue}}
%\everymath{\color[rgb]{0,1,1}}
%\pagecolor[rgb]{0,0,0.5}


\newcommand*{\pdtest}[3][]{\ensuremath{\frac{\partial^{#1} #2}{\partial #3}}}

\newcommand*{\deffunc}[6][]{\ensuremath{
\begin{array}{rcl}
#2 : #3 &\rightarrow& #4\\
#5 &\mapsto& #6
\end{array}
}}

\newcommand{\eqcolon}{\mathrel{\resizebox{\widthof{$\mathord{=}$}}{\height}{ $\!\!=\!\!\resizebox{1.2\width}{0.8\height}{\raisebox{0.23ex}{$\mathop{:}$}}\!\!$ }}}
\newcommand{\coloneq}{\mathrel{\resizebox{\widthof{$\mathord{=}$}}{\height}{ $\!\!\resizebox{1.2\width}{0.8\height}{\raisebox{0.23ex}{$\mathop{:}$}}\!\!=\!\!$ }}}
\newcommand{\eqcolonl}{\ensuremath{\mathrel{=\!\!\mathop{:}}}}
\newcommand{\coloneql}{\ensuremath{\mathrel{\mathop{:} \!\! =}}}
\newcommand{\vc}[1]{% inline column vector
  \left(\begin{smallmatrix}#1\end{smallmatrix}\right)%
}
\newcommand{\vr}[1]{% inline row vector
  \begin{smallmatrix}(\,#1\,)\end{smallmatrix}%
}
\makeatletter
\newcommand*{\defeq}{\ =\mathrel{\rlap{%
                     \raisebox{0.3ex}{$\m@th\cdot$}}%
                     \raisebox{-0.3ex}{$\m@th\cdot$}}%
                     }
\makeatother

\newcommand{\mathcircle}[1]{% inline row vector
 \overset{\circ}{#1}
}
\newcommand{\ulim}{% low limit
 \underline{\lim}
}
\newcommand{\ssi}{% iff
\iff
}
\newcommand{\ps}[2]{
\expval{#1 | #2}
}
\newcommand{\df}[1]{
\mqty{#1}
}
\newcommand{\n}[1]{
\norm{#1}
}
\newcommand{\sys}[1]{
\left\{\smqty{#1}\right.
}


\newcommand{\eqdef}{\ensuremath{\overset{\text{def}}=}}


\def\Circlearrowright{\ensuremath{%
  \rotatebox[origin=c]{230}{$\circlearrowright$}}}

\newcommand\ct[1]{\text{\rmfamily\upshape #1}}
\newcommand\question[1]{ {\color{red} ...!? \small #1}}
\newcommand\caz[1]{\left\{\begin{array} #1 \end{array}\right.}
\newcommand\const{\text{\rmfamily\upshape const}}
\newcommand\toP{ \overset{\pro}{\to}}
\newcommand\toPP{ \overset{\text{PP}}{\to}}
\newcommand{\oeq}{\mathrel{\text{\textcircled{$=$}}}}





\usepackage{xcolor}
% \usepackage[normalem]{ulem}
\usepackage{lipsum}
\makeatletter
% \newcommand\colorwave[1][blue]{\bgroup \markoverwith{\lower3.5\p@\hbox{\sixly \textcolor{#1}{\char58}}}\ULon}
%\font\sixly=lasy6 % does not re-load if already loaded, so no memory problem.

\newmdtheoremenv[
linewidth= 1pt,linecolor= blue,%
leftmargin=20,rightmargin=20,innertopmargin=0pt, innerrightmargin=40,%
tikzsetting = { draw=lightgray, line width = 0.3pt,dashed,%
dash pattern = on 15pt off 3pt},%
splittopskip=\topskip,skipbelow=\baselineskip,%
skipabove=\baselineskip,ntheorem,roundcorner=0pt,
% backgroundcolor=pagebg,font=\color{orange}\sffamily, fontcolor=white
]{examplebox}{Exemple}[section]



\newcommand\R{\mathbb{R}}
\newcommand\Z{\mathbb{Z}}
\newcommand\N{\mathbb{N}}
\newcommand\E{\mathbb{E}}
\newcommand\F{\mathcal{F}}
\newcommand\cH{\mathcal{H}}
\newcommand\V{\mathbb{V}}
\newcommand\dmo{ ^{-1} }
\newcommand\kapa{\kappa}
\newcommand\im{Im}
\newcommand\hs{\mathcal{H}}





\usepackage{soul}

\makeatletter
\newcommand*{\whiten}[1]{\llap{\textcolor{white}{{\the\SOUL@token}}\hspace{#1pt}}}
\DeclareRobustCommand*\myul{%
    \def\SOUL@everyspace{\underline{\space}\kern\z@}%
    \def\SOUL@everytoken{%
     \setbox0=\hbox{\the\SOUL@token}%
     \ifdim\dp0>\z@
        \raisebox{\dp0}{\underline{\phantom{\the\SOUL@token}}}%
        \whiten{1}\whiten{0}%
        \whiten{-1}\whiten{-2}%
        \llap{\the\SOUL@token}%
     \else
        \underline{\the\SOUL@token}%
     \fi}%
\SOUL@}
\makeatother

\newcommand*{\demp}{\fontfamily{lmtt}\selectfont}

\DeclareTextFontCommand{\textdemp}{\demp}

\begin{document}

\ifcomment
Multiline
comment
\fi
\ifcomment
\myul{Typesetting test}
% \color[rgb]{1,1,1}
$∑_i^n≠ 60º±∞π∆¬≈√j∫h≤≥µ$

$\CR \R\pro\ind\pro\gS\pro
\mqty[a&b\\c&d]$
$\pro\mathbb{P}$
$\dd{x}$

  \[
    \alpha(x)=\left\{
                \begin{array}{ll}
                  x\\
                  \frac{1}{1+e^{-kx}}\\
                  \frac{e^x-e^{-x}}{e^x+e^{-x}}
                \end{array}
              \right.
  \]

  $\expval{x}$
  
  $\chi_\rho(ghg\dmo)=\Tr(\rho_{ghg\dmo})=\Tr(\rho_g\circ\rho_h\circ\rho\dmo_g)=\Tr(\rho_h)\overset{\mbox{\scalebox{0.5}{$\Tr(AB)=\Tr(BA)$}}}{=}\chi_\rho(h)$
  	$\mathop{\oplus}_{\substack{x\in X}}$

$\mat(\rho_g)=(a_{ij}(g))_{\scriptsize \substack{1\leq i\leq d \\ 1\leq j\leq d}}$ et $\mat(\rho'_g)=(a'_{ij}(g))_{\scriptsize \substack{1\leq i'\leq d' \\ 1\leq j'\leq d'}}$



\[\int_a^b{\mathbb{R}^2}g(u, v)\dd{P_{XY}}(u, v)=\iint g(u,v) f_{XY}(u, v)\dd \lambda(u) \dd \lambda(v)\]
$$\lim_{x\to\infty} f(x)$$	
$$\iiiint_V \mu(t,u,v,w) \,dt\,du\,dv\,dw$$
$$\sum_{n=1}^{\infty} 2^{-n} = 1$$	
\begin{definition}
	Si $X$ et $Y$ sont 2 v.a. ou definit la \textsc{Covariance} entre $X$ et $Y$ comme
	$\cov(X,Y)\overset{\text{def}}{=}\E\left[(X-\E(X))(Y-\E(Y))\right]=\E(XY)-\E(X)\E(Y)$.
\end{definition}
\fi
\pagebreak

% \tableofcontents

% insert your code here
%% !TEX encoding = UTF-8 Unicode
% !TEX TS-program = xelatex

\documentclass[french]{report}

%\usepackage[utf8]{inputenc}
%\usepackage[T1]{fontenc}
\usepackage{babel}


\newif\ifcomment
%\commenttrue # Show comments

\usepackage{physics}
\usepackage{amssymb}


\usepackage{amsthm}
% \usepackage{thmtools}
\usepackage{mathtools}
\usepackage{amsfonts}

\usepackage{color}

\usepackage{tikz}

\usepackage{geometry}
\geometry{a5paper, margin=0.1in, right=1cm}

\usepackage{dsfont}

\usepackage{graphicx}
\graphicspath{ {images/} }

\usepackage{faktor}

\usepackage{IEEEtrantools}
\usepackage{enumerate}   
\usepackage[PostScript=dvips]{"/Users/aware/Documents/Courses/diagrams"}


\newtheorem{theorem}{Théorème}[section]
\renewcommand{\thetheorem}{\arabic{theorem}}
\newtheorem{lemme}{Lemme}[section]
\renewcommand{\thelemme}{\arabic{lemme}}
\newtheorem{proposition}{Proposition}[section]
\renewcommand{\theproposition}{\arabic{proposition}}
\newtheorem{notations}{Notations}[section]
\newtheorem{problem}{Problème}[section]
\newtheorem{corollary}{Corollaire}[theorem]
\renewcommand{\thecorollary}{\arabic{corollary}}
\newtheorem{property}{Propriété}[section]
\newtheorem{objective}{Objectif}[section]

\theoremstyle{definition}
\newtheorem{definition}{Définition}[section]
\renewcommand{\thedefinition}{\arabic{definition}}
\newtheorem{exercise}{Exercice}[chapter]
\renewcommand{\theexercise}{\arabic{exercise}}
\newtheorem{example}{Exemple}[chapter]
\renewcommand{\theexample}{\arabic{example}}
\newtheorem*{solution}{Solution}
\newtheorem*{application}{Application}
\newtheorem*{notation}{Notation}
\newtheorem*{vocabulary}{Vocabulaire}
\newtheorem*{properties}{Propriétés}



\theoremstyle{remark}
\newtheorem*{remark}{Remarque}
\newtheorem*{rappel}{Rappel}


\usepackage{etoolbox}
\AtBeginEnvironment{exercise}{\small}
\AtBeginEnvironment{example}{\small}

\usepackage{cases}
\usepackage[red]{mypack}

\usepackage[framemethod=TikZ]{mdframed}

\definecolor{bg}{rgb}{0.4,0.25,0.95}
\definecolor{pagebg}{rgb}{0,0,0.5}
\surroundwithmdframed[
   topline=false,
   rightline=false,
   bottomline=false,
   leftmargin=\parindent,
   skipabove=8pt,
   skipbelow=8pt,
   linecolor=blue,
   innerbottommargin=10pt,
   % backgroundcolor=bg,font=\color{orange}\sffamily, fontcolor=white
]{definition}

\usepackage{empheq}
\usepackage[most]{tcolorbox}

\newtcbox{\mymath}[1][]{%
    nobeforeafter, math upper, tcbox raise base,
    enhanced, colframe=blue!30!black,
    colback=red!10, boxrule=1pt,
    #1}

\usepackage{unixode}


\DeclareMathOperator{\ord}{ord}
\DeclareMathOperator{\orb}{orb}
\DeclareMathOperator{\stab}{stab}
\DeclareMathOperator{\Stab}{stab}
\DeclareMathOperator{\ppcm}{ppcm}
\DeclareMathOperator{\conj}{Conj}
\DeclareMathOperator{\End}{End}
\DeclareMathOperator{\rot}{rot}
\DeclareMathOperator{\trs}{trace}
\DeclareMathOperator{\Ind}{Ind}
\DeclareMathOperator{\mat}{Mat}
\DeclareMathOperator{\id}{Id}
\DeclareMathOperator{\vect}{vect}
\DeclareMathOperator{\img}{img}
\DeclareMathOperator{\cov}{Cov}
\DeclareMathOperator{\dist}{dist}
\DeclareMathOperator{\irr}{Irr}
\DeclareMathOperator{\image}{Im}
\DeclareMathOperator{\pd}{\partial}
\DeclareMathOperator{\epi}{epi}
\DeclareMathOperator{\Argmin}{Argmin}
\DeclareMathOperator{\dom}{dom}
\DeclareMathOperator{\proj}{proj}
\DeclareMathOperator{\ctg}{ctg}
\DeclareMathOperator{\supp}{supp}
\DeclareMathOperator{\argmin}{argmin}
\DeclareMathOperator{\mult}{mult}
\DeclareMathOperator{\ch}{ch}
\DeclareMathOperator{\sh}{sh}
\DeclareMathOperator{\rang}{rang}
\DeclareMathOperator{\diam}{diam}
\DeclareMathOperator{\Epigraphe}{Epigraphe}




\usepackage{xcolor}
\everymath{\color{blue}}
%\everymath{\color[rgb]{0,1,1}}
%\pagecolor[rgb]{0,0,0.5}


\newcommand*{\pdtest}[3][]{\ensuremath{\frac{\partial^{#1} #2}{\partial #3}}}

\newcommand*{\deffunc}[6][]{\ensuremath{
\begin{array}{rcl}
#2 : #3 &\rightarrow& #4\\
#5 &\mapsto& #6
\end{array}
}}

\newcommand{\eqcolon}{\mathrel{\resizebox{\widthof{$\mathord{=}$}}{\height}{ $\!\!=\!\!\resizebox{1.2\width}{0.8\height}{\raisebox{0.23ex}{$\mathop{:}$}}\!\!$ }}}
\newcommand{\coloneq}{\mathrel{\resizebox{\widthof{$\mathord{=}$}}{\height}{ $\!\!\resizebox{1.2\width}{0.8\height}{\raisebox{0.23ex}{$\mathop{:}$}}\!\!=\!\!$ }}}
\newcommand{\eqcolonl}{\ensuremath{\mathrel{=\!\!\mathop{:}}}}
\newcommand{\coloneql}{\ensuremath{\mathrel{\mathop{:} \!\! =}}}
\newcommand{\vc}[1]{% inline column vector
  \left(\begin{smallmatrix}#1\end{smallmatrix}\right)%
}
\newcommand{\vr}[1]{% inline row vector
  \begin{smallmatrix}(\,#1\,)\end{smallmatrix}%
}
\makeatletter
\newcommand*{\defeq}{\ =\mathrel{\rlap{%
                     \raisebox{0.3ex}{$\m@th\cdot$}}%
                     \raisebox{-0.3ex}{$\m@th\cdot$}}%
                     }
\makeatother

\newcommand{\mathcircle}[1]{% inline row vector
 \overset{\circ}{#1}
}
\newcommand{\ulim}{% low limit
 \underline{\lim}
}
\newcommand{\ssi}{% iff
\iff
}
\newcommand{\ps}[2]{
\expval{#1 | #2}
}
\newcommand{\df}[1]{
\mqty{#1}
}
\newcommand{\n}[1]{
\norm{#1}
}
\newcommand{\sys}[1]{
\left\{\smqty{#1}\right.
}


\newcommand{\eqdef}{\ensuremath{\overset{\text{def}}=}}


\def\Circlearrowright{\ensuremath{%
  \rotatebox[origin=c]{230}{$\circlearrowright$}}}

\newcommand\ct[1]{\text{\rmfamily\upshape #1}}
\newcommand\question[1]{ {\color{red} ...!? \small #1}}
\newcommand\caz[1]{\left\{\begin{array} #1 \end{array}\right.}
\newcommand\const{\text{\rmfamily\upshape const}}
\newcommand\toP{ \overset{\pro}{\to}}
\newcommand\toPP{ \overset{\text{PP}}{\to}}
\newcommand{\oeq}{\mathrel{\text{\textcircled{$=$}}}}





\usepackage{xcolor}
% \usepackage[normalem]{ulem}
\usepackage{lipsum}
\makeatletter
% \newcommand\colorwave[1][blue]{\bgroup \markoverwith{\lower3.5\p@\hbox{\sixly \textcolor{#1}{\char58}}}\ULon}
%\font\sixly=lasy6 % does not re-load if already loaded, so no memory problem.

\newmdtheoremenv[
linewidth= 1pt,linecolor= blue,%
leftmargin=20,rightmargin=20,innertopmargin=0pt, innerrightmargin=40,%
tikzsetting = { draw=lightgray, line width = 0.3pt,dashed,%
dash pattern = on 15pt off 3pt},%
splittopskip=\topskip,skipbelow=\baselineskip,%
skipabove=\baselineskip,ntheorem,roundcorner=0pt,
% backgroundcolor=pagebg,font=\color{orange}\sffamily, fontcolor=white
]{examplebox}{Exemple}[section]



\newcommand\R{\mathbb{R}}
\newcommand\Z{\mathbb{Z}}
\newcommand\N{\mathbb{N}}
\newcommand\E{\mathbb{E}}
\newcommand\F{\mathcal{F}}
\newcommand\cH{\mathcal{H}}
\newcommand\V{\mathbb{V}}
\newcommand\dmo{ ^{-1} }
\newcommand\kapa{\kappa}
\newcommand\im{Im}
\newcommand\hs{\mathcal{H}}





\usepackage{soul}

\makeatletter
\newcommand*{\whiten}[1]{\llap{\textcolor{white}{{\the\SOUL@token}}\hspace{#1pt}}}
\DeclareRobustCommand*\myul{%
    \def\SOUL@everyspace{\underline{\space}\kern\z@}%
    \def\SOUL@everytoken{%
     \setbox0=\hbox{\the\SOUL@token}%
     \ifdim\dp0>\z@
        \raisebox{\dp0}{\underline{\phantom{\the\SOUL@token}}}%
        \whiten{1}\whiten{0}%
        \whiten{-1}\whiten{-2}%
        \llap{\the\SOUL@token}%
     \else
        \underline{\the\SOUL@token}%
     \fi}%
\SOUL@}
\makeatother

\newcommand*{\demp}{\fontfamily{lmtt}\selectfont}

\DeclareTextFontCommand{\textdemp}{\demp}

\begin{document}

\ifcomment
Multiline
comment
\fi
\ifcomment
\myul{Typesetting test}
% \color[rgb]{1,1,1}
$∑_i^n≠ 60º±∞π∆¬≈√j∫h≤≥µ$

$\CR \R\pro\ind\pro\gS\pro
\mqty[a&b\\c&d]$
$\pro\mathbb{P}$
$\dd{x}$

  \[
    \alpha(x)=\left\{
                \begin{array}{ll}
                  x\\
                  \frac{1}{1+e^{-kx}}\\
                  \frac{e^x-e^{-x}}{e^x+e^{-x}}
                \end{array}
              \right.
  \]

  $\expval{x}$
  
  $\chi_\rho(ghg\dmo)=\Tr(\rho_{ghg\dmo})=\Tr(\rho_g\circ\rho_h\circ\rho\dmo_g)=\Tr(\rho_h)\overset{\mbox{\scalebox{0.5}{$\Tr(AB)=\Tr(BA)$}}}{=}\chi_\rho(h)$
  	$\mathop{\oplus}_{\substack{x\in X}}$

$\mat(\rho_g)=(a_{ij}(g))_{\scriptsize \substack{1\leq i\leq d \\ 1\leq j\leq d}}$ et $\mat(\rho'_g)=(a'_{ij}(g))_{\scriptsize \substack{1\leq i'\leq d' \\ 1\leq j'\leq d'}}$



\[\int_a^b{\mathbb{R}^2}g(u, v)\dd{P_{XY}}(u, v)=\iint g(u,v) f_{XY}(u, v)\dd \lambda(u) \dd \lambda(v)\]
$$\lim_{x\to\infty} f(x)$$	
$$\iiiint_V \mu(t,u,v,w) \,dt\,du\,dv\,dw$$
$$\sum_{n=1}^{\infty} 2^{-n} = 1$$	
\begin{definition}
	Si $X$ et $Y$ sont 2 v.a. ou definit la \textsc{Covariance} entre $X$ et $Y$ comme
	$\cov(X,Y)\overset{\text{def}}{=}\E\left[(X-\E(X))(Y-\E(Y))\right]=\E(XY)-\E(X)\E(Y)$.
\end{definition}
\fi
\pagebreak

% \tableofcontents

% insert your code here
%\input{./algebra/main.tex}
%\input{./geometrie-differentielle/main.tex}
%\input{./probabilite/main.tex}
%\input{./analyse-fonctionnelle/main.tex}
% \input{./Analyse-convexe-et-dualite-en-optimisation/main.tex}
%\input{./tikz/main.tex}
%\input{./Theorie-du-distributions/main.tex}
%\input{./optimisation/mine.tex}
 \input{./modelisation/main.tex}

% yves.aubry@univ-tln.fr : algebra

\end{document}

%% !TEX encoding = UTF-8 Unicode
% !TEX TS-program = xelatex

\documentclass[french]{report}

%\usepackage[utf8]{inputenc}
%\usepackage[T1]{fontenc}
\usepackage{babel}


\newif\ifcomment
%\commenttrue # Show comments

\usepackage{physics}
\usepackage{amssymb}


\usepackage{amsthm}
% \usepackage{thmtools}
\usepackage{mathtools}
\usepackage{amsfonts}

\usepackage{color}

\usepackage{tikz}

\usepackage{geometry}
\geometry{a5paper, margin=0.1in, right=1cm}

\usepackage{dsfont}

\usepackage{graphicx}
\graphicspath{ {images/} }

\usepackage{faktor}

\usepackage{IEEEtrantools}
\usepackage{enumerate}   
\usepackage[PostScript=dvips]{"/Users/aware/Documents/Courses/diagrams"}


\newtheorem{theorem}{Théorème}[section]
\renewcommand{\thetheorem}{\arabic{theorem}}
\newtheorem{lemme}{Lemme}[section]
\renewcommand{\thelemme}{\arabic{lemme}}
\newtheorem{proposition}{Proposition}[section]
\renewcommand{\theproposition}{\arabic{proposition}}
\newtheorem{notations}{Notations}[section]
\newtheorem{problem}{Problème}[section]
\newtheorem{corollary}{Corollaire}[theorem]
\renewcommand{\thecorollary}{\arabic{corollary}}
\newtheorem{property}{Propriété}[section]
\newtheorem{objective}{Objectif}[section]

\theoremstyle{definition}
\newtheorem{definition}{Définition}[section]
\renewcommand{\thedefinition}{\arabic{definition}}
\newtheorem{exercise}{Exercice}[chapter]
\renewcommand{\theexercise}{\arabic{exercise}}
\newtheorem{example}{Exemple}[chapter]
\renewcommand{\theexample}{\arabic{example}}
\newtheorem*{solution}{Solution}
\newtheorem*{application}{Application}
\newtheorem*{notation}{Notation}
\newtheorem*{vocabulary}{Vocabulaire}
\newtheorem*{properties}{Propriétés}



\theoremstyle{remark}
\newtheorem*{remark}{Remarque}
\newtheorem*{rappel}{Rappel}


\usepackage{etoolbox}
\AtBeginEnvironment{exercise}{\small}
\AtBeginEnvironment{example}{\small}

\usepackage{cases}
\usepackage[red]{mypack}

\usepackage[framemethod=TikZ]{mdframed}

\definecolor{bg}{rgb}{0.4,0.25,0.95}
\definecolor{pagebg}{rgb}{0,0,0.5}
\surroundwithmdframed[
   topline=false,
   rightline=false,
   bottomline=false,
   leftmargin=\parindent,
   skipabove=8pt,
   skipbelow=8pt,
   linecolor=blue,
   innerbottommargin=10pt,
   % backgroundcolor=bg,font=\color{orange}\sffamily, fontcolor=white
]{definition}

\usepackage{empheq}
\usepackage[most]{tcolorbox}

\newtcbox{\mymath}[1][]{%
    nobeforeafter, math upper, tcbox raise base,
    enhanced, colframe=blue!30!black,
    colback=red!10, boxrule=1pt,
    #1}

\usepackage{unixode}


\DeclareMathOperator{\ord}{ord}
\DeclareMathOperator{\orb}{orb}
\DeclareMathOperator{\stab}{stab}
\DeclareMathOperator{\Stab}{stab}
\DeclareMathOperator{\ppcm}{ppcm}
\DeclareMathOperator{\conj}{Conj}
\DeclareMathOperator{\End}{End}
\DeclareMathOperator{\rot}{rot}
\DeclareMathOperator{\trs}{trace}
\DeclareMathOperator{\Ind}{Ind}
\DeclareMathOperator{\mat}{Mat}
\DeclareMathOperator{\id}{Id}
\DeclareMathOperator{\vect}{vect}
\DeclareMathOperator{\img}{img}
\DeclareMathOperator{\cov}{Cov}
\DeclareMathOperator{\dist}{dist}
\DeclareMathOperator{\irr}{Irr}
\DeclareMathOperator{\image}{Im}
\DeclareMathOperator{\pd}{\partial}
\DeclareMathOperator{\epi}{epi}
\DeclareMathOperator{\Argmin}{Argmin}
\DeclareMathOperator{\dom}{dom}
\DeclareMathOperator{\proj}{proj}
\DeclareMathOperator{\ctg}{ctg}
\DeclareMathOperator{\supp}{supp}
\DeclareMathOperator{\argmin}{argmin}
\DeclareMathOperator{\mult}{mult}
\DeclareMathOperator{\ch}{ch}
\DeclareMathOperator{\sh}{sh}
\DeclareMathOperator{\rang}{rang}
\DeclareMathOperator{\diam}{diam}
\DeclareMathOperator{\Epigraphe}{Epigraphe}




\usepackage{xcolor}
\everymath{\color{blue}}
%\everymath{\color[rgb]{0,1,1}}
%\pagecolor[rgb]{0,0,0.5}


\newcommand*{\pdtest}[3][]{\ensuremath{\frac{\partial^{#1} #2}{\partial #3}}}

\newcommand*{\deffunc}[6][]{\ensuremath{
\begin{array}{rcl}
#2 : #3 &\rightarrow& #4\\
#5 &\mapsto& #6
\end{array}
}}

\newcommand{\eqcolon}{\mathrel{\resizebox{\widthof{$\mathord{=}$}}{\height}{ $\!\!=\!\!\resizebox{1.2\width}{0.8\height}{\raisebox{0.23ex}{$\mathop{:}$}}\!\!$ }}}
\newcommand{\coloneq}{\mathrel{\resizebox{\widthof{$\mathord{=}$}}{\height}{ $\!\!\resizebox{1.2\width}{0.8\height}{\raisebox{0.23ex}{$\mathop{:}$}}\!\!=\!\!$ }}}
\newcommand{\eqcolonl}{\ensuremath{\mathrel{=\!\!\mathop{:}}}}
\newcommand{\coloneql}{\ensuremath{\mathrel{\mathop{:} \!\! =}}}
\newcommand{\vc}[1]{% inline column vector
  \left(\begin{smallmatrix}#1\end{smallmatrix}\right)%
}
\newcommand{\vr}[1]{% inline row vector
  \begin{smallmatrix}(\,#1\,)\end{smallmatrix}%
}
\makeatletter
\newcommand*{\defeq}{\ =\mathrel{\rlap{%
                     \raisebox{0.3ex}{$\m@th\cdot$}}%
                     \raisebox{-0.3ex}{$\m@th\cdot$}}%
                     }
\makeatother

\newcommand{\mathcircle}[1]{% inline row vector
 \overset{\circ}{#1}
}
\newcommand{\ulim}{% low limit
 \underline{\lim}
}
\newcommand{\ssi}{% iff
\iff
}
\newcommand{\ps}[2]{
\expval{#1 | #2}
}
\newcommand{\df}[1]{
\mqty{#1}
}
\newcommand{\n}[1]{
\norm{#1}
}
\newcommand{\sys}[1]{
\left\{\smqty{#1}\right.
}


\newcommand{\eqdef}{\ensuremath{\overset{\text{def}}=}}


\def\Circlearrowright{\ensuremath{%
  \rotatebox[origin=c]{230}{$\circlearrowright$}}}

\newcommand\ct[1]{\text{\rmfamily\upshape #1}}
\newcommand\question[1]{ {\color{red} ...!? \small #1}}
\newcommand\caz[1]{\left\{\begin{array} #1 \end{array}\right.}
\newcommand\const{\text{\rmfamily\upshape const}}
\newcommand\toP{ \overset{\pro}{\to}}
\newcommand\toPP{ \overset{\text{PP}}{\to}}
\newcommand{\oeq}{\mathrel{\text{\textcircled{$=$}}}}





\usepackage{xcolor}
% \usepackage[normalem]{ulem}
\usepackage{lipsum}
\makeatletter
% \newcommand\colorwave[1][blue]{\bgroup \markoverwith{\lower3.5\p@\hbox{\sixly \textcolor{#1}{\char58}}}\ULon}
%\font\sixly=lasy6 % does not re-load if already loaded, so no memory problem.

\newmdtheoremenv[
linewidth= 1pt,linecolor= blue,%
leftmargin=20,rightmargin=20,innertopmargin=0pt, innerrightmargin=40,%
tikzsetting = { draw=lightgray, line width = 0.3pt,dashed,%
dash pattern = on 15pt off 3pt},%
splittopskip=\topskip,skipbelow=\baselineskip,%
skipabove=\baselineskip,ntheorem,roundcorner=0pt,
% backgroundcolor=pagebg,font=\color{orange}\sffamily, fontcolor=white
]{examplebox}{Exemple}[section]



\newcommand\R{\mathbb{R}}
\newcommand\Z{\mathbb{Z}}
\newcommand\N{\mathbb{N}}
\newcommand\E{\mathbb{E}}
\newcommand\F{\mathcal{F}}
\newcommand\cH{\mathcal{H}}
\newcommand\V{\mathbb{V}}
\newcommand\dmo{ ^{-1} }
\newcommand\kapa{\kappa}
\newcommand\im{Im}
\newcommand\hs{\mathcal{H}}





\usepackage{soul}

\makeatletter
\newcommand*{\whiten}[1]{\llap{\textcolor{white}{{\the\SOUL@token}}\hspace{#1pt}}}
\DeclareRobustCommand*\myul{%
    \def\SOUL@everyspace{\underline{\space}\kern\z@}%
    \def\SOUL@everytoken{%
     \setbox0=\hbox{\the\SOUL@token}%
     \ifdim\dp0>\z@
        \raisebox{\dp0}{\underline{\phantom{\the\SOUL@token}}}%
        \whiten{1}\whiten{0}%
        \whiten{-1}\whiten{-2}%
        \llap{\the\SOUL@token}%
     \else
        \underline{\the\SOUL@token}%
     \fi}%
\SOUL@}
\makeatother

\newcommand*{\demp}{\fontfamily{lmtt}\selectfont}

\DeclareTextFontCommand{\textdemp}{\demp}

\begin{document}

\ifcomment
Multiline
comment
\fi
\ifcomment
\myul{Typesetting test}
% \color[rgb]{1,1,1}
$∑_i^n≠ 60º±∞π∆¬≈√j∫h≤≥µ$

$\CR \R\pro\ind\pro\gS\pro
\mqty[a&b\\c&d]$
$\pro\mathbb{P}$
$\dd{x}$

  \[
    \alpha(x)=\left\{
                \begin{array}{ll}
                  x\\
                  \frac{1}{1+e^{-kx}}\\
                  \frac{e^x-e^{-x}}{e^x+e^{-x}}
                \end{array}
              \right.
  \]

  $\expval{x}$
  
  $\chi_\rho(ghg\dmo)=\Tr(\rho_{ghg\dmo})=\Tr(\rho_g\circ\rho_h\circ\rho\dmo_g)=\Tr(\rho_h)\overset{\mbox{\scalebox{0.5}{$\Tr(AB)=\Tr(BA)$}}}{=}\chi_\rho(h)$
  	$\mathop{\oplus}_{\substack{x\in X}}$

$\mat(\rho_g)=(a_{ij}(g))_{\scriptsize \substack{1\leq i\leq d \\ 1\leq j\leq d}}$ et $\mat(\rho'_g)=(a'_{ij}(g))_{\scriptsize \substack{1\leq i'\leq d' \\ 1\leq j'\leq d'}}$



\[\int_a^b{\mathbb{R}^2}g(u, v)\dd{P_{XY}}(u, v)=\iint g(u,v) f_{XY}(u, v)\dd \lambda(u) \dd \lambda(v)\]
$$\lim_{x\to\infty} f(x)$$	
$$\iiiint_V \mu(t,u,v,w) \,dt\,du\,dv\,dw$$
$$\sum_{n=1}^{\infty} 2^{-n} = 1$$	
\begin{definition}
	Si $X$ et $Y$ sont 2 v.a. ou definit la \textsc{Covariance} entre $X$ et $Y$ comme
	$\cov(X,Y)\overset{\text{def}}{=}\E\left[(X-\E(X))(Y-\E(Y))\right]=\E(XY)-\E(X)\E(Y)$.
\end{definition}
\fi
\pagebreak

% \tableofcontents

% insert your code here
%\input{./algebra/main.tex}
%\input{./geometrie-differentielle/main.tex}
%\input{./probabilite/main.tex}
%\input{./analyse-fonctionnelle/main.tex}
% \input{./Analyse-convexe-et-dualite-en-optimisation/main.tex}
%\input{./tikz/main.tex}
%\input{./Theorie-du-distributions/main.tex}
%\input{./optimisation/mine.tex}
 \input{./modelisation/main.tex}

% yves.aubry@univ-tln.fr : algebra

\end{document}

%% !TEX encoding = UTF-8 Unicode
% !TEX TS-program = xelatex

\documentclass[french]{report}

%\usepackage[utf8]{inputenc}
%\usepackage[T1]{fontenc}
\usepackage{babel}


\newif\ifcomment
%\commenttrue # Show comments

\usepackage{physics}
\usepackage{amssymb}


\usepackage{amsthm}
% \usepackage{thmtools}
\usepackage{mathtools}
\usepackage{amsfonts}

\usepackage{color}

\usepackage{tikz}

\usepackage{geometry}
\geometry{a5paper, margin=0.1in, right=1cm}

\usepackage{dsfont}

\usepackage{graphicx}
\graphicspath{ {images/} }

\usepackage{faktor}

\usepackage{IEEEtrantools}
\usepackage{enumerate}   
\usepackage[PostScript=dvips]{"/Users/aware/Documents/Courses/diagrams"}


\newtheorem{theorem}{Théorème}[section]
\renewcommand{\thetheorem}{\arabic{theorem}}
\newtheorem{lemme}{Lemme}[section]
\renewcommand{\thelemme}{\arabic{lemme}}
\newtheorem{proposition}{Proposition}[section]
\renewcommand{\theproposition}{\arabic{proposition}}
\newtheorem{notations}{Notations}[section]
\newtheorem{problem}{Problème}[section]
\newtheorem{corollary}{Corollaire}[theorem]
\renewcommand{\thecorollary}{\arabic{corollary}}
\newtheorem{property}{Propriété}[section]
\newtheorem{objective}{Objectif}[section]

\theoremstyle{definition}
\newtheorem{definition}{Définition}[section]
\renewcommand{\thedefinition}{\arabic{definition}}
\newtheorem{exercise}{Exercice}[chapter]
\renewcommand{\theexercise}{\arabic{exercise}}
\newtheorem{example}{Exemple}[chapter]
\renewcommand{\theexample}{\arabic{example}}
\newtheorem*{solution}{Solution}
\newtheorem*{application}{Application}
\newtheorem*{notation}{Notation}
\newtheorem*{vocabulary}{Vocabulaire}
\newtheorem*{properties}{Propriétés}



\theoremstyle{remark}
\newtheorem*{remark}{Remarque}
\newtheorem*{rappel}{Rappel}


\usepackage{etoolbox}
\AtBeginEnvironment{exercise}{\small}
\AtBeginEnvironment{example}{\small}

\usepackage{cases}
\usepackage[red]{mypack}

\usepackage[framemethod=TikZ]{mdframed}

\definecolor{bg}{rgb}{0.4,0.25,0.95}
\definecolor{pagebg}{rgb}{0,0,0.5}
\surroundwithmdframed[
   topline=false,
   rightline=false,
   bottomline=false,
   leftmargin=\parindent,
   skipabove=8pt,
   skipbelow=8pt,
   linecolor=blue,
   innerbottommargin=10pt,
   % backgroundcolor=bg,font=\color{orange}\sffamily, fontcolor=white
]{definition}

\usepackage{empheq}
\usepackage[most]{tcolorbox}

\newtcbox{\mymath}[1][]{%
    nobeforeafter, math upper, tcbox raise base,
    enhanced, colframe=blue!30!black,
    colback=red!10, boxrule=1pt,
    #1}

\usepackage{unixode}


\DeclareMathOperator{\ord}{ord}
\DeclareMathOperator{\orb}{orb}
\DeclareMathOperator{\stab}{stab}
\DeclareMathOperator{\Stab}{stab}
\DeclareMathOperator{\ppcm}{ppcm}
\DeclareMathOperator{\conj}{Conj}
\DeclareMathOperator{\End}{End}
\DeclareMathOperator{\rot}{rot}
\DeclareMathOperator{\trs}{trace}
\DeclareMathOperator{\Ind}{Ind}
\DeclareMathOperator{\mat}{Mat}
\DeclareMathOperator{\id}{Id}
\DeclareMathOperator{\vect}{vect}
\DeclareMathOperator{\img}{img}
\DeclareMathOperator{\cov}{Cov}
\DeclareMathOperator{\dist}{dist}
\DeclareMathOperator{\irr}{Irr}
\DeclareMathOperator{\image}{Im}
\DeclareMathOperator{\pd}{\partial}
\DeclareMathOperator{\epi}{epi}
\DeclareMathOperator{\Argmin}{Argmin}
\DeclareMathOperator{\dom}{dom}
\DeclareMathOperator{\proj}{proj}
\DeclareMathOperator{\ctg}{ctg}
\DeclareMathOperator{\supp}{supp}
\DeclareMathOperator{\argmin}{argmin}
\DeclareMathOperator{\mult}{mult}
\DeclareMathOperator{\ch}{ch}
\DeclareMathOperator{\sh}{sh}
\DeclareMathOperator{\rang}{rang}
\DeclareMathOperator{\diam}{diam}
\DeclareMathOperator{\Epigraphe}{Epigraphe}




\usepackage{xcolor}
\everymath{\color{blue}}
%\everymath{\color[rgb]{0,1,1}}
%\pagecolor[rgb]{0,0,0.5}


\newcommand*{\pdtest}[3][]{\ensuremath{\frac{\partial^{#1} #2}{\partial #3}}}

\newcommand*{\deffunc}[6][]{\ensuremath{
\begin{array}{rcl}
#2 : #3 &\rightarrow& #4\\
#5 &\mapsto& #6
\end{array}
}}

\newcommand{\eqcolon}{\mathrel{\resizebox{\widthof{$\mathord{=}$}}{\height}{ $\!\!=\!\!\resizebox{1.2\width}{0.8\height}{\raisebox{0.23ex}{$\mathop{:}$}}\!\!$ }}}
\newcommand{\coloneq}{\mathrel{\resizebox{\widthof{$\mathord{=}$}}{\height}{ $\!\!\resizebox{1.2\width}{0.8\height}{\raisebox{0.23ex}{$\mathop{:}$}}\!\!=\!\!$ }}}
\newcommand{\eqcolonl}{\ensuremath{\mathrel{=\!\!\mathop{:}}}}
\newcommand{\coloneql}{\ensuremath{\mathrel{\mathop{:} \!\! =}}}
\newcommand{\vc}[1]{% inline column vector
  \left(\begin{smallmatrix}#1\end{smallmatrix}\right)%
}
\newcommand{\vr}[1]{% inline row vector
  \begin{smallmatrix}(\,#1\,)\end{smallmatrix}%
}
\makeatletter
\newcommand*{\defeq}{\ =\mathrel{\rlap{%
                     \raisebox{0.3ex}{$\m@th\cdot$}}%
                     \raisebox{-0.3ex}{$\m@th\cdot$}}%
                     }
\makeatother

\newcommand{\mathcircle}[1]{% inline row vector
 \overset{\circ}{#1}
}
\newcommand{\ulim}{% low limit
 \underline{\lim}
}
\newcommand{\ssi}{% iff
\iff
}
\newcommand{\ps}[2]{
\expval{#1 | #2}
}
\newcommand{\df}[1]{
\mqty{#1}
}
\newcommand{\n}[1]{
\norm{#1}
}
\newcommand{\sys}[1]{
\left\{\smqty{#1}\right.
}


\newcommand{\eqdef}{\ensuremath{\overset{\text{def}}=}}


\def\Circlearrowright{\ensuremath{%
  \rotatebox[origin=c]{230}{$\circlearrowright$}}}

\newcommand\ct[1]{\text{\rmfamily\upshape #1}}
\newcommand\question[1]{ {\color{red} ...!? \small #1}}
\newcommand\caz[1]{\left\{\begin{array} #1 \end{array}\right.}
\newcommand\const{\text{\rmfamily\upshape const}}
\newcommand\toP{ \overset{\pro}{\to}}
\newcommand\toPP{ \overset{\text{PP}}{\to}}
\newcommand{\oeq}{\mathrel{\text{\textcircled{$=$}}}}





\usepackage{xcolor}
% \usepackage[normalem]{ulem}
\usepackage{lipsum}
\makeatletter
% \newcommand\colorwave[1][blue]{\bgroup \markoverwith{\lower3.5\p@\hbox{\sixly \textcolor{#1}{\char58}}}\ULon}
%\font\sixly=lasy6 % does not re-load if already loaded, so no memory problem.

\newmdtheoremenv[
linewidth= 1pt,linecolor= blue,%
leftmargin=20,rightmargin=20,innertopmargin=0pt, innerrightmargin=40,%
tikzsetting = { draw=lightgray, line width = 0.3pt,dashed,%
dash pattern = on 15pt off 3pt},%
splittopskip=\topskip,skipbelow=\baselineskip,%
skipabove=\baselineskip,ntheorem,roundcorner=0pt,
% backgroundcolor=pagebg,font=\color{orange}\sffamily, fontcolor=white
]{examplebox}{Exemple}[section]



\newcommand\R{\mathbb{R}}
\newcommand\Z{\mathbb{Z}}
\newcommand\N{\mathbb{N}}
\newcommand\E{\mathbb{E}}
\newcommand\F{\mathcal{F}}
\newcommand\cH{\mathcal{H}}
\newcommand\V{\mathbb{V}}
\newcommand\dmo{ ^{-1} }
\newcommand\kapa{\kappa}
\newcommand\im{Im}
\newcommand\hs{\mathcal{H}}





\usepackage{soul}

\makeatletter
\newcommand*{\whiten}[1]{\llap{\textcolor{white}{{\the\SOUL@token}}\hspace{#1pt}}}
\DeclareRobustCommand*\myul{%
    \def\SOUL@everyspace{\underline{\space}\kern\z@}%
    \def\SOUL@everytoken{%
     \setbox0=\hbox{\the\SOUL@token}%
     \ifdim\dp0>\z@
        \raisebox{\dp0}{\underline{\phantom{\the\SOUL@token}}}%
        \whiten{1}\whiten{0}%
        \whiten{-1}\whiten{-2}%
        \llap{\the\SOUL@token}%
     \else
        \underline{\the\SOUL@token}%
     \fi}%
\SOUL@}
\makeatother

\newcommand*{\demp}{\fontfamily{lmtt}\selectfont}

\DeclareTextFontCommand{\textdemp}{\demp}

\begin{document}

\ifcomment
Multiline
comment
\fi
\ifcomment
\myul{Typesetting test}
% \color[rgb]{1,1,1}
$∑_i^n≠ 60º±∞π∆¬≈√j∫h≤≥µ$

$\CR \R\pro\ind\pro\gS\pro
\mqty[a&b\\c&d]$
$\pro\mathbb{P}$
$\dd{x}$

  \[
    \alpha(x)=\left\{
                \begin{array}{ll}
                  x\\
                  \frac{1}{1+e^{-kx}}\\
                  \frac{e^x-e^{-x}}{e^x+e^{-x}}
                \end{array}
              \right.
  \]

  $\expval{x}$
  
  $\chi_\rho(ghg\dmo)=\Tr(\rho_{ghg\dmo})=\Tr(\rho_g\circ\rho_h\circ\rho\dmo_g)=\Tr(\rho_h)\overset{\mbox{\scalebox{0.5}{$\Tr(AB)=\Tr(BA)$}}}{=}\chi_\rho(h)$
  	$\mathop{\oplus}_{\substack{x\in X}}$

$\mat(\rho_g)=(a_{ij}(g))_{\scriptsize \substack{1\leq i\leq d \\ 1\leq j\leq d}}$ et $\mat(\rho'_g)=(a'_{ij}(g))_{\scriptsize \substack{1\leq i'\leq d' \\ 1\leq j'\leq d'}}$



\[\int_a^b{\mathbb{R}^2}g(u, v)\dd{P_{XY}}(u, v)=\iint g(u,v) f_{XY}(u, v)\dd \lambda(u) \dd \lambda(v)\]
$$\lim_{x\to\infty} f(x)$$	
$$\iiiint_V \mu(t,u,v,w) \,dt\,du\,dv\,dw$$
$$\sum_{n=1}^{\infty} 2^{-n} = 1$$	
\begin{definition}
	Si $X$ et $Y$ sont 2 v.a. ou definit la \textsc{Covariance} entre $X$ et $Y$ comme
	$\cov(X,Y)\overset{\text{def}}{=}\E\left[(X-\E(X))(Y-\E(Y))\right]=\E(XY)-\E(X)\E(Y)$.
\end{definition}
\fi
\pagebreak

% \tableofcontents

% insert your code here
%\input{./algebra/main.tex}
%\input{./geometrie-differentielle/main.tex}
%\input{./probabilite/main.tex}
%\input{./analyse-fonctionnelle/main.tex}
% \input{./Analyse-convexe-et-dualite-en-optimisation/main.tex}
%\input{./tikz/main.tex}
%\input{./Theorie-du-distributions/main.tex}
%\input{./optimisation/mine.tex}
 \input{./modelisation/main.tex}

% yves.aubry@univ-tln.fr : algebra

\end{document}

%% !TEX encoding = UTF-8 Unicode
% !TEX TS-program = xelatex

\documentclass[french]{report}

%\usepackage[utf8]{inputenc}
%\usepackage[T1]{fontenc}
\usepackage{babel}


\newif\ifcomment
%\commenttrue # Show comments

\usepackage{physics}
\usepackage{amssymb}


\usepackage{amsthm}
% \usepackage{thmtools}
\usepackage{mathtools}
\usepackage{amsfonts}

\usepackage{color}

\usepackage{tikz}

\usepackage{geometry}
\geometry{a5paper, margin=0.1in, right=1cm}

\usepackage{dsfont}

\usepackage{graphicx}
\graphicspath{ {images/} }

\usepackage{faktor}

\usepackage{IEEEtrantools}
\usepackage{enumerate}   
\usepackage[PostScript=dvips]{"/Users/aware/Documents/Courses/diagrams"}


\newtheorem{theorem}{Théorème}[section]
\renewcommand{\thetheorem}{\arabic{theorem}}
\newtheorem{lemme}{Lemme}[section]
\renewcommand{\thelemme}{\arabic{lemme}}
\newtheorem{proposition}{Proposition}[section]
\renewcommand{\theproposition}{\arabic{proposition}}
\newtheorem{notations}{Notations}[section]
\newtheorem{problem}{Problème}[section]
\newtheorem{corollary}{Corollaire}[theorem]
\renewcommand{\thecorollary}{\arabic{corollary}}
\newtheorem{property}{Propriété}[section]
\newtheorem{objective}{Objectif}[section]

\theoremstyle{definition}
\newtheorem{definition}{Définition}[section]
\renewcommand{\thedefinition}{\arabic{definition}}
\newtheorem{exercise}{Exercice}[chapter]
\renewcommand{\theexercise}{\arabic{exercise}}
\newtheorem{example}{Exemple}[chapter]
\renewcommand{\theexample}{\arabic{example}}
\newtheorem*{solution}{Solution}
\newtheorem*{application}{Application}
\newtheorem*{notation}{Notation}
\newtheorem*{vocabulary}{Vocabulaire}
\newtheorem*{properties}{Propriétés}



\theoremstyle{remark}
\newtheorem*{remark}{Remarque}
\newtheorem*{rappel}{Rappel}


\usepackage{etoolbox}
\AtBeginEnvironment{exercise}{\small}
\AtBeginEnvironment{example}{\small}

\usepackage{cases}
\usepackage[red]{mypack}

\usepackage[framemethod=TikZ]{mdframed}

\definecolor{bg}{rgb}{0.4,0.25,0.95}
\definecolor{pagebg}{rgb}{0,0,0.5}
\surroundwithmdframed[
   topline=false,
   rightline=false,
   bottomline=false,
   leftmargin=\parindent,
   skipabove=8pt,
   skipbelow=8pt,
   linecolor=blue,
   innerbottommargin=10pt,
   % backgroundcolor=bg,font=\color{orange}\sffamily, fontcolor=white
]{definition}

\usepackage{empheq}
\usepackage[most]{tcolorbox}

\newtcbox{\mymath}[1][]{%
    nobeforeafter, math upper, tcbox raise base,
    enhanced, colframe=blue!30!black,
    colback=red!10, boxrule=1pt,
    #1}

\usepackage{unixode}


\DeclareMathOperator{\ord}{ord}
\DeclareMathOperator{\orb}{orb}
\DeclareMathOperator{\stab}{stab}
\DeclareMathOperator{\Stab}{stab}
\DeclareMathOperator{\ppcm}{ppcm}
\DeclareMathOperator{\conj}{Conj}
\DeclareMathOperator{\End}{End}
\DeclareMathOperator{\rot}{rot}
\DeclareMathOperator{\trs}{trace}
\DeclareMathOperator{\Ind}{Ind}
\DeclareMathOperator{\mat}{Mat}
\DeclareMathOperator{\id}{Id}
\DeclareMathOperator{\vect}{vect}
\DeclareMathOperator{\img}{img}
\DeclareMathOperator{\cov}{Cov}
\DeclareMathOperator{\dist}{dist}
\DeclareMathOperator{\irr}{Irr}
\DeclareMathOperator{\image}{Im}
\DeclareMathOperator{\pd}{\partial}
\DeclareMathOperator{\epi}{epi}
\DeclareMathOperator{\Argmin}{Argmin}
\DeclareMathOperator{\dom}{dom}
\DeclareMathOperator{\proj}{proj}
\DeclareMathOperator{\ctg}{ctg}
\DeclareMathOperator{\supp}{supp}
\DeclareMathOperator{\argmin}{argmin}
\DeclareMathOperator{\mult}{mult}
\DeclareMathOperator{\ch}{ch}
\DeclareMathOperator{\sh}{sh}
\DeclareMathOperator{\rang}{rang}
\DeclareMathOperator{\diam}{diam}
\DeclareMathOperator{\Epigraphe}{Epigraphe}




\usepackage{xcolor}
\everymath{\color{blue}}
%\everymath{\color[rgb]{0,1,1}}
%\pagecolor[rgb]{0,0,0.5}


\newcommand*{\pdtest}[3][]{\ensuremath{\frac{\partial^{#1} #2}{\partial #3}}}

\newcommand*{\deffunc}[6][]{\ensuremath{
\begin{array}{rcl}
#2 : #3 &\rightarrow& #4\\
#5 &\mapsto& #6
\end{array}
}}

\newcommand{\eqcolon}{\mathrel{\resizebox{\widthof{$\mathord{=}$}}{\height}{ $\!\!=\!\!\resizebox{1.2\width}{0.8\height}{\raisebox{0.23ex}{$\mathop{:}$}}\!\!$ }}}
\newcommand{\coloneq}{\mathrel{\resizebox{\widthof{$\mathord{=}$}}{\height}{ $\!\!\resizebox{1.2\width}{0.8\height}{\raisebox{0.23ex}{$\mathop{:}$}}\!\!=\!\!$ }}}
\newcommand{\eqcolonl}{\ensuremath{\mathrel{=\!\!\mathop{:}}}}
\newcommand{\coloneql}{\ensuremath{\mathrel{\mathop{:} \!\! =}}}
\newcommand{\vc}[1]{% inline column vector
  \left(\begin{smallmatrix}#1\end{smallmatrix}\right)%
}
\newcommand{\vr}[1]{% inline row vector
  \begin{smallmatrix}(\,#1\,)\end{smallmatrix}%
}
\makeatletter
\newcommand*{\defeq}{\ =\mathrel{\rlap{%
                     \raisebox{0.3ex}{$\m@th\cdot$}}%
                     \raisebox{-0.3ex}{$\m@th\cdot$}}%
                     }
\makeatother

\newcommand{\mathcircle}[1]{% inline row vector
 \overset{\circ}{#1}
}
\newcommand{\ulim}{% low limit
 \underline{\lim}
}
\newcommand{\ssi}{% iff
\iff
}
\newcommand{\ps}[2]{
\expval{#1 | #2}
}
\newcommand{\df}[1]{
\mqty{#1}
}
\newcommand{\n}[1]{
\norm{#1}
}
\newcommand{\sys}[1]{
\left\{\smqty{#1}\right.
}


\newcommand{\eqdef}{\ensuremath{\overset{\text{def}}=}}


\def\Circlearrowright{\ensuremath{%
  \rotatebox[origin=c]{230}{$\circlearrowright$}}}

\newcommand\ct[1]{\text{\rmfamily\upshape #1}}
\newcommand\question[1]{ {\color{red} ...!? \small #1}}
\newcommand\caz[1]{\left\{\begin{array} #1 \end{array}\right.}
\newcommand\const{\text{\rmfamily\upshape const}}
\newcommand\toP{ \overset{\pro}{\to}}
\newcommand\toPP{ \overset{\text{PP}}{\to}}
\newcommand{\oeq}{\mathrel{\text{\textcircled{$=$}}}}





\usepackage{xcolor}
% \usepackage[normalem]{ulem}
\usepackage{lipsum}
\makeatletter
% \newcommand\colorwave[1][blue]{\bgroup \markoverwith{\lower3.5\p@\hbox{\sixly \textcolor{#1}{\char58}}}\ULon}
%\font\sixly=lasy6 % does not re-load if already loaded, so no memory problem.

\newmdtheoremenv[
linewidth= 1pt,linecolor= blue,%
leftmargin=20,rightmargin=20,innertopmargin=0pt, innerrightmargin=40,%
tikzsetting = { draw=lightgray, line width = 0.3pt,dashed,%
dash pattern = on 15pt off 3pt},%
splittopskip=\topskip,skipbelow=\baselineskip,%
skipabove=\baselineskip,ntheorem,roundcorner=0pt,
% backgroundcolor=pagebg,font=\color{orange}\sffamily, fontcolor=white
]{examplebox}{Exemple}[section]



\newcommand\R{\mathbb{R}}
\newcommand\Z{\mathbb{Z}}
\newcommand\N{\mathbb{N}}
\newcommand\E{\mathbb{E}}
\newcommand\F{\mathcal{F}}
\newcommand\cH{\mathcal{H}}
\newcommand\V{\mathbb{V}}
\newcommand\dmo{ ^{-1} }
\newcommand\kapa{\kappa}
\newcommand\im{Im}
\newcommand\hs{\mathcal{H}}





\usepackage{soul}

\makeatletter
\newcommand*{\whiten}[1]{\llap{\textcolor{white}{{\the\SOUL@token}}\hspace{#1pt}}}
\DeclareRobustCommand*\myul{%
    \def\SOUL@everyspace{\underline{\space}\kern\z@}%
    \def\SOUL@everytoken{%
     \setbox0=\hbox{\the\SOUL@token}%
     \ifdim\dp0>\z@
        \raisebox{\dp0}{\underline{\phantom{\the\SOUL@token}}}%
        \whiten{1}\whiten{0}%
        \whiten{-1}\whiten{-2}%
        \llap{\the\SOUL@token}%
     \else
        \underline{\the\SOUL@token}%
     \fi}%
\SOUL@}
\makeatother

\newcommand*{\demp}{\fontfamily{lmtt}\selectfont}

\DeclareTextFontCommand{\textdemp}{\demp}

\begin{document}

\ifcomment
Multiline
comment
\fi
\ifcomment
\myul{Typesetting test}
% \color[rgb]{1,1,1}
$∑_i^n≠ 60º±∞π∆¬≈√j∫h≤≥µ$

$\CR \R\pro\ind\pro\gS\pro
\mqty[a&b\\c&d]$
$\pro\mathbb{P}$
$\dd{x}$

  \[
    \alpha(x)=\left\{
                \begin{array}{ll}
                  x\\
                  \frac{1}{1+e^{-kx}}\\
                  \frac{e^x-e^{-x}}{e^x+e^{-x}}
                \end{array}
              \right.
  \]

  $\expval{x}$
  
  $\chi_\rho(ghg\dmo)=\Tr(\rho_{ghg\dmo})=\Tr(\rho_g\circ\rho_h\circ\rho\dmo_g)=\Tr(\rho_h)\overset{\mbox{\scalebox{0.5}{$\Tr(AB)=\Tr(BA)$}}}{=}\chi_\rho(h)$
  	$\mathop{\oplus}_{\substack{x\in X}}$

$\mat(\rho_g)=(a_{ij}(g))_{\scriptsize \substack{1\leq i\leq d \\ 1\leq j\leq d}}$ et $\mat(\rho'_g)=(a'_{ij}(g))_{\scriptsize \substack{1\leq i'\leq d' \\ 1\leq j'\leq d'}}$



\[\int_a^b{\mathbb{R}^2}g(u, v)\dd{P_{XY}}(u, v)=\iint g(u,v) f_{XY}(u, v)\dd \lambda(u) \dd \lambda(v)\]
$$\lim_{x\to\infty} f(x)$$	
$$\iiiint_V \mu(t,u,v,w) \,dt\,du\,dv\,dw$$
$$\sum_{n=1}^{\infty} 2^{-n} = 1$$	
\begin{definition}
	Si $X$ et $Y$ sont 2 v.a. ou definit la \textsc{Covariance} entre $X$ et $Y$ comme
	$\cov(X,Y)\overset{\text{def}}{=}\E\left[(X-\E(X))(Y-\E(Y))\right]=\E(XY)-\E(X)\E(Y)$.
\end{definition}
\fi
\pagebreak

% \tableofcontents

% insert your code here
%\input{./algebra/main.tex}
%\input{./geometrie-differentielle/main.tex}
%\input{./probabilite/main.tex}
%\input{./analyse-fonctionnelle/main.tex}
% \input{./Analyse-convexe-et-dualite-en-optimisation/main.tex}
%\input{./tikz/main.tex}
%\input{./Theorie-du-distributions/main.tex}
%\input{./optimisation/mine.tex}
 \input{./modelisation/main.tex}

% yves.aubry@univ-tln.fr : algebra

\end{document}

% % !TEX encoding = UTF-8 Unicode
% !TEX TS-program = xelatex

\documentclass[french]{report}

%\usepackage[utf8]{inputenc}
%\usepackage[T1]{fontenc}
\usepackage{babel}


\newif\ifcomment
%\commenttrue # Show comments

\usepackage{physics}
\usepackage{amssymb}


\usepackage{amsthm}
% \usepackage{thmtools}
\usepackage{mathtools}
\usepackage{amsfonts}

\usepackage{color}

\usepackage{tikz}

\usepackage{geometry}
\geometry{a5paper, margin=0.1in, right=1cm}

\usepackage{dsfont}

\usepackage{graphicx}
\graphicspath{ {images/} }

\usepackage{faktor}

\usepackage{IEEEtrantools}
\usepackage{enumerate}   
\usepackage[PostScript=dvips]{"/Users/aware/Documents/Courses/diagrams"}


\newtheorem{theorem}{Théorème}[section]
\renewcommand{\thetheorem}{\arabic{theorem}}
\newtheorem{lemme}{Lemme}[section]
\renewcommand{\thelemme}{\arabic{lemme}}
\newtheorem{proposition}{Proposition}[section]
\renewcommand{\theproposition}{\arabic{proposition}}
\newtheorem{notations}{Notations}[section]
\newtheorem{problem}{Problème}[section]
\newtheorem{corollary}{Corollaire}[theorem]
\renewcommand{\thecorollary}{\arabic{corollary}}
\newtheorem{property}{Propriété}[section]
\newtheorem{objective}{Objectif}[section]

\theoremstyle{definition}
\newtheorem{definition}{Définition}[section]
\renewcommand{\thedefinition}{\arabic{definition}}
\newtheorem{exercise}{Exercice}[chapter]
\renewcommand{\theexercise}{\arabic{exercise}}
\newtheorem{example}{Exemple}[chapter]
\renewcommand{\theexample}{\arabic{example}}
\newtheorem*{solution}{Solution}
\newtheorem*{application}{Application}
\newtheorem*{notation}{Notation}
\newtheorem*{vocabulary}{Vocabulaire}
\newtheorem*{properties}{Propriétés}



\theoremstyle{remark}
\newtheorem*{remark}{Remarque}
\newtheorem*{rappel}{Rappel}


\usepackage{etoolbox}
\AtBeginEnvironment{exercise}{\small}
\AtBeginEnvironment{example}{\small}

\usepackage{cases}
\usepackage[red]{mypack}

\usepackage[framemethod=TikZ]{mdframed}

\definecolor{bg}{rgb}{0.4,0.25,0.95}
\definecolor{pagebg}{rgb}{0,0,0.5}
\surroundwithmdframed[
   topline=false,
   rightline=false,
   bottomline=false,
   leftmargin=\parindent,
   skipabove=8pt,
   skipbelow=8pt,
   linecolor=blue,
   innerbottommargin=10pt,
   % backgroundcolor=bg,font=\color{orange}\sffamily, fontcolor=white
]{definition}

\usepackage{empheq}
\usepackage[most]{tcolorbox}

\newtcbox{\mymath}[1][]{%
    nobeforeafter, math upper, tcbox raise base,
    enhanced, colframe=blue!30!black,
    colback=red!10, boxrule=1pt,
    #1}

\usepackage{unixode}


\DeclareMathOperator{\ord}{ord}
\DeclareMathOperator{\orb}{orb}
\DeclareMathOperator{\stab}{stab}
\DeclareMathOperator{\Stab}{stab}
\DeclareMathOperator{\ppcm}{ppcm}
\DeclareMathOperator{\conj}{Conj}
\DeclareMathOperator{\End}{End}
\DeclareMathOperator{\rot}{rot}
\DeclareMathOperator{\trs}{trace}
\DeclareMathOperator{\Ind}{Ind}
\DeclareMathOperator{\mat}{Mat}
\DeclareMathOperator{\id}{Id}
\DeclareMathOperator{\vect}{vect}
\DeclareMathOperator{\img}{img}
\DeclareMathOperator{\cov}{Cov}
\DeclareMathOperator{\dist}{dist}
\DeclareMathOperator{\irr}{Irr}
\DeclareMathOperator{\image}{Im}
\DeclareMathOperator{\pd}{\partial}
\DeclareMathOperator{\epi}{epi}
\DeclareMathOperator{\Argmin}{Argmin}
\DeclareMathOperator{\dom}{dom}
\DeclareMathOperator{\proj}{proj}
\DeclareMathOperator{\ctg}{ctg}
\DeclareMathOperator{\supp}{supp}
\DeclareMathOperator{\argmin}{argmin}
\DeclareMathOperator{\mult}{mult}
\DeclareMathOperator{\ch}{ch}
\DeclareMathOperator{\sh}{sh}
\DeclareMathOperator{\rang}{rang}
\DeclareMathOperator{\diam}{diam}
\DeclareMathOperator{\Epigraphe}{Epigraphe}




\usepackage{xcolor}
\everymath{\color{blue}}
%\everymath{\color[rgb]{0,1,1}}
%\pagecolor[rgb]{0,0,0.5}


\newcommand*{\pdtest}[3][]{\ensuremath{\frac{\partial^{#1} #2}{\partial #3}}}

\newcommand*{\deffunc}[6][]{\ensuremath{
\begin{array}{rcl}
#2 : #3 &\rightarrow& #4\\
#5 &\mapsto& #6
\end{array}
}}

\newcommand{\eqcolon}{\mathrel{\resizebox{\widthof{$\mathord{=}$}}{\height}{ $\!\!=\!\!\resizebox{1.2\width}{0.8\height}{\raisebox{0.23ex}{$\mathop{:}$}}\!\!$ }}}
\newcommand{\coloneq}{\mathrel{\resizebox{\widthof{$\mathord{=}$}}{\height}{ $\!\!\resizebox{1.2\width}{0.8\height}{\raisebox{0.23ex}{$\mathop{:}$}}\!\!=\!\!$ }}}
\newcommand{\eqcolonl}{\ensuremath{\mathrel{=\!\!\mathop{:}}}}
\newcommand{\coloneql}{\ensuremath{\mathrel{\mathop{:} \!\! =}}}
\newcommand{\vc}[1]{% inline column vector
  \left(\begin{smallmatrix}#1\end{smallmatrix}\right)%
}
\newcommand{\vr}[1]{% inline row vector
  \begin{smallmatrix}(\,#1\,)\end{smallmatrix}%
}
\makeatletter
\newcommand*{\defeq}{\ =\mathrel{\rlap{%
                     \raisebox{0.3ex}{$\m@th\cdot$}}%
                     \raisebox{-0.3ex}{$\m@th\cdot$}}%
                     }
\makeatother

\newcommand{\mathcircle}[1]{% inline row vector
 \overset{\circ}{#1}
}
\newcommand{\ulim}{% low limit
 \underline{\lim}
}
\newcommand{\ssi}{% iff
\iff
}
\newcommand{\ps}[2]{
\expval{#1 | #2}
}
\newcommand{\df}[1]{
\mqty{#1}
}
\newcommand{\n}[1]{
\norm{#1}
}
\newcommand{\sys}[1]{
\left\{\smqty{#1}\right.
}


\newcommand{\eqdef}{\ensuremath{\overset{\text{def}}=}}


\def\Circlearrowright{\ensuremath{%
  \rotatebox[origin=c]{230}{$\circlearrowright$}}}

\newcommand\ct[1]{\text{\rmfamily\upshape #1}}
\newcommand\question[1]{ {\color{red} ...!? \small #1}}
\newcommand\caz[1]{\left\{\begin{array} #1 \end{array}\right.}
\newcommand\const{\text{\rmfamily\upshape const}}
\newcommand\toP{ \overset{\pro}{\to}}
\newcommand\toPP{ \overset{\text{PP}}{\to}}
\newcommand{\oeq}{\mathrel{\text{\textcircled{$=$}}}}





\usepackage{xcolor}
% \usepackage[normalem]{ulem}
\usepackage{lipsum}
\makeatletter
% \newcommand\colorwave[1][blue]{\bgroup \markoverwith{\lower3.5\p@\hbox{\sixly \textcolor{#1}{\char58}}}\ULon}
%\font\sixly=lasy6 % does not re-load if already loaded, so no memory problem.

\newmdtheoremenv[
linewidth= 1pt,linecolor= blue,%
leftmargin=20,rightmargin=20,innertopmargin=0pt, innerrightmargin=40,%
tikzsetting = { draw=lightgray, line width = 0.3pt,dashed,%
dash pattern = on 15pt off 3pt},%
splittopskip=\topskip,skipbelow=\baselineskip,%
skipabove=\baselineskip,ntheorem,roundcorner=0pt,
% backgroundcolor=pagebg,font=\color{orange}\sffamily, fontcolor=white
]{examplebox}{Exemple}[section]



\newcommand\R{\mathbb{R}}
\newcommand\Z{\mathbb{Z}}
\newcommand\N{\mathbb{N}}
\newcommand\E{\mathbb{E}}
\newcommand\F{\mathcal{F}}
\newcommand\cH{\mathcal{H}}
\newcommand\V{\mathbb{V}}
\newcommand\dmo{ ^{-1} }
\newcommand\kapa{\kappa}
\newcommand\im{Im}
\newcommand\hs{\mathcal{H}}





\usepackage{soul}

\makeatletter
\newcommand*{\whiten}[1]{\llap{\textcolor{white}{{\the\SOUL@token}}\hspace{#1pt}}}
\DeclareRobustCommand*\myul{%
    \def\SOUL@everyspace{\underline{\space}\kern\z@}%
    \def\SOUL@everytoken{%
     \setbox0=\hbox{\the\SOUL@token}%
     \ifdim\dp0>\z@
        \raisebox{\dp0}{\underline{\phantom{\the\SOUL@token}}}%
        \whiten{1}\whiten{0}%
        \whiten{-1}\whiten{-2}%
        \llap{\the\SOUL@token}%
     \else
        \underline{\the\SOUL@token}%
     \fi}%
\SOUL@}
\makeatother

\newcommand*{\demp}{\fontfamily{lmtt}\selectfont}

\DeclareTextFontCommand{\textdemp}{\demp}

\begin{document}

\ifcomment
Multiline
comment
\fi
\ifcomment
\myul{Typesetting test}
% \color[rgb]{1,1,1}
$∑_i^n≠ 60º±∞π∆¬≈√j∫h≤≥µ$

$\CR \R\pro\ind\pro\gS\pro
\mqty[a&b\\c&d]$
$\pro\mathbb{P}$
$\dd{x}$

  \[
    \alpha(x)=\left\{
                \begin{array}{ll}
                  x\\
                  \frac{1}{1+e^{-kx}}\\
                  \frac{e^x-e^{-x}}{e^x+e^{-x}}
                \end{array}
              \right.
  \]

  $\expval{x}$
  
  $\chi_\rho(ghg\dmo)=\Tr(\rho_{ghg\dmo})=\Tr(\rho_g\circ\rho_h\circ\rho\dmo_g)=\Tr(\rho_h)\overset{\mbox{\scalebox{0.5}{$\Tr(AB)=\Tr(BA)$}}}{=}\chi_\rho(h)$
  	$\mathop{\oplus}_{\substack{x\in X}}$

$\mat(\rho_g)=(a_{ij}(g))_{\scriptsize \substack{1\leq i\leq d \\ 1\leq j\leq d}}$ et $\mat(\rho'_g)=(a'_{ij}(g))_{\scriptsize \substack{1\leq i'\leq d' \\ 1\leq j'\leq d'}}$



\[\int_a^b{\mathbb{R}^2}g(u, v)\dd{P_{XY}}(u, v)=\iint g(u,v) f_{XY}(u, v)\dd \lambda(u) \dd \lambda(v)\]
$$\lim_{x\to\infty} f(x)$$	
$$\iiiint_V \mu(t,u,v,w) \,dt\,du\,dv\,dw$$
$$\sum_{n=1}^{\infty} 2^{-n} = 1$$	
\begin{definition}
	Si $X$ et $Y$ sont 2 v.a. ou definit la \textsc{Covariance} entre $X$ et $Y$ comme
	$\cov(X,Y)\overset{\text{def}}{=}\E\left[(X-\E(X))(Y-\E(Y))\right]=\E(XY)-\E(X)\E(Y)$.
\end{definition}
\fi
\pagebreak

% \tableofcontents

% insert your code here
%\input{./algebra/main.tex}
%\input{./geometrie-differentielle/main.tex}
%\input{./probabilite/main.tex}
%\input{./analyse-fonctionnelle/main.tex}
% \input{./Analyse-convexe-et-dualite-en-optimisation/main.tex}
%\input{./tikz/main.tex}
%\input{./Theorie-du-distributions/main.tex}
%\input{./optimisation/mine.tex}
 \input{./modelisation/main.tex}

% yves.aubry@univ-tln.fr : algebra

\end{document}

%% !TEX encoding = UTF-8 Unicode
% !TEX TS-program = xelatex

\documentclass[french]{report}

%\usepackage[utf8]{inputenc}
%\usepackage[T1]{fontenc}
\usepackage{babel}


\newif\ifcomment
%\commenttrue # Show comments

\usepackage{physics}
\usepackage{amssymb}


\usepackage{amsthm}
% \usepackage{thmtools}
\usepackage{mathtools}
\usepackage{amsfonts}

\usepackage{color}

\usepackage{tikz}

\usepackage{geometry}
\geometry{a5paper, margin=0.1in, right=1cm}

\usepackage{dsfont}

\usepackage{graphicx}
\graphicspath{ {images/} }

\usepackage{faktor}

\usepackage{IEEEtrantools}
\usepackage{enumerate}   
\usepackage[PostScript=dvips]{"/Users/aware/Documents/Courses/diagrams"}


\newtheorem{theorem}{Théorème}[section]
\renewcommand{\thetheorem}{\arabic{theorem}}
\newtheorem{lemme}{Lemme}[section]
\renewcommand{\thelemme}{\arabic{lemme}}
\newtheorem{proposition}{Proposition}[section]
\renewcommand{\theproposition}{\arabic{proposition}}
\newtheorem{notations}{Notations}[section]
\newtheorem{problem}{Problème}[section]
\newtheorem{corollary}{Corollaire}[theorem]
\renewcommand{\thecorollary}{\arabic{corollary}}
\newtheorem{property}{Propriété}[section]
\newtheorem{objective}{Objectif}[section]

\theoremstyle{definition}
\newtheorem{definition}{Définition}[section]
\renewcommand{\thedefinition}{\arabic{definition}}
\newtheorem{exercise}{Exercice}[chapter]
\renewcommand{\theexercise}{\arabic{exercise}}
\newtheorem{example}{Exemple}[chapter]
\renewcommand{\theexample}{\arabic{example}}
\newtheorem*{solution}{Solution}
\newtheorem*{application}{Application}
\newtheorem*{notation}{Notation}
\newtheorem*{vocabulary}{Vocabulaire}
\newtheorem*{properties}{Propriétés}



\theoremstyle{remark}
\newtheorem*{remark}{Remarque}
\newtheorem*{rappel}{Rappel}


\usepackage{etoolbox}
\AtBeginEnvironment{exercise}{\small}
\AtBeginEnvironment{example}{\small}

\usepackage{cases}
\usepackage[red]{mypack}

\usepackage[framemethod=TikZ]{mdframed}

\definecolor{bg}{rgb}{0.4,0.25,0.95}
\definecolor{pagebg}{rgb}{0,0,0.5}
\surroundwithmdframed[
   topline=false,
   rightline=false,
   bottomline=false,
   leftmargin=\parindent,
   skipabove=8pt,
   skipbelow=8pt,
   linecolor=blue,
   innerbottommargin=10pt,
   % backgroundcolor=bg,font=\color{orange}\sffamily, fontcolor=white
]{definition}

\usepackage{empheq}
\usepackage[most]{tcolorbox}

\newtcbox{\mymath}[1][]{%
    nobeforeafter, math upper, tcbox raise base,
    enhanced, colframe=blue!30!black,
    colback=red!10, boxrule=1pt,
    #1}

\usepackage{unixode}


\DeclareMathOperator{\ord}{ord}
\DeclareMathOperator{\orb}{orb}
\DeclareMathOperator{\stab}{stab}
\DeclareMathOperator{\Stab}{stab}
\DeclareMathOperator{\ppcm}{ppcm}
\DeclareMathOperator{\conj}{Conj}
\DeclareMathOperator{\End}{End}
\DeclareMathOperator{\rot}{rot}
\DeclareMathOperator{\trs}{trace}
\DeclareMathOperator{\Ind}{Ind}
\DeclareMathOperator{\mat}{Mat}
\DeclareMathOperator{\id}{Id}
\DeclareMathOperator{\vect}{vect}
\DeclareMathOperator{\img}{img}
\DeclareMathOperator{\cov}{Cov}
\DeclareMathOperator{\dist}{dist}
\DeclareMathOperator{\irr}{Irr}
\DeclareMathOperator{\image}{Im}
\DeclareMathOperator{\pd}{\partial}
\DeclareMathOperator{\epi}{epi}
\DeclareMathOperator{\Argmin}{Argmin}
\DeclareMathOperator{\dom}{dom}
\DeclareMathOperator{\proj}{proj}
\DeclareMathOperator{\ctg}{ctg}
\DeclareMathOperator{\supp}{supp}
\DeclareMathOperator{\argmin}{argmin}
\DeclareMathOperator{\mult}{mult}
\DeclareMathOperator{\ch}{ch}
\DeclareMathOperator{\sh}{sh}
\DeclareMathOperator{\rang}{rang}
\DeclareMathOperator{\diam}{diam}
\DeclareMathOperator{\Epigraphe}{Epigraphe}




\usepackage{xcolor}
\everymath{\color{blue}}
%\everymath{\color[rgb]{0,1,1}}
%\pagecolor[rgb]{0,0,0.5}


\newcommand*{\pdtest}[3][]{\ensuremath{\frac{\partial^{#1} #2}{\partial #3}}}

\newcommand*{\deffunc}[6][]{\ensuremath{
\begin{array}{rcl}
#2 : #3 &\rightarrow& #4\\
#5 &\mapsto& #6
\end{array}
}}

\newcommand{\eqcolon}{\mathrel{\resizebox{\widthof{$\mathord{=}$}}{\height}{ $\!\!=\!\!\resizebox{1.2\width}{0.8\height}{\raisebox{0.23ex}{$\mathop{:}$}}\!\!$ }}}
\newcommand{\coloneq}{\mathrel{\resizebox{\widthof{$\mathord{=}$}}{\height}{ $\!\!\resizebox{1.2\width}{0.8\height}{\raisebox{0.23ex}{$\mathop{:}$}}\!\!=\!\!$ }}}
\newcommand{\eqcolonl}{\ensuremath{\mathrel{=\!\!\mathop{:}}}}
\newcommand{\coloneql}{\ensuremath{\mathrel{\mathop{:} \!\! =}}}
\newcommand{\vc}[1]{% inline column vector
  \left(\begin{smallmatrix}#1\end{smallmatrix}\right)%
}
\newcommand{\vr}[1]{% inline row vector
  \begin{smallmatrix}(\,#1\,)\end{smallmatrix}%
}
\makeatletter
\newcommand*{\defeq}{\ =\mathrel{\rlap{%
                     \raisebox{0.3ex}{$\m@th\cdot$}}%
                     \raisebox{-0.3ex}{$\m@th\cdot$}}%
                     }
\makeatother

\newcommand{\mathcircle}[1]{% inline row vector
 \overset{\circ}{#1}
}
\newcommand{\ulim}{% low limit
 \underline{\lim}
}
\newcommand{\ssi}{% iff
\iff
}
\newcommand{\ps}[2]{
\expval{#1 | #2}
}
\newcommand{\df}[1]{
\mqty{#1}
}
\newcommand{\n}[1]{
\norm{#1}
}
\newcommand{\sys}[1]{
\left\{\smqty{#1}\right.
}


\newcommand{\eqdef}{\ensuremath{\overset{\text{def}}=}}


\def\Circlearrowright{\ensuremath{%
  \rotatebox[origin=c]{230}{$\circlearrowright$}}}

\newcommand\ct[1]{\text{\rmfamily\upshape #1}}
\newcommand\question[1]{ {\color{red} ...!? \small #1}}
\newcommand\caz[1]{\left\{\begin{array} #1 \end{array}\right.}
\newcommand\const{\text{\rmfamily\upshape const}}
\newcommand\toP{ \overset{\pro}{\to}}
\newcommand\toPP{ \overset{\text{PP}}{\to}}
\newcommand{\oeq}{\mathrel{\text{\textcircled{$=$}}}}





\usepackage{xcolor}
% \usepackage[normalem]{ulem}
\usepackage{lipsum}
\makeatletter
% \newcommand\colorwave[1][blue]{\bgroup \markoverwith{\lower3.5\p@\hbox{\sixly \textcolor{#1}{\char58}}}\ULon}
%\font\sixly=lasy6 % does not re-load if already loaded, so no memory problem.

\newmdtheoremenv[
linewidth= 1pt,linecolor= blue,%
leftmargin=20,rightmargin=20,innertopmargin=0pt, innerrightmargin=40,%
tikzsetting = { draw=lightgray, line width = 0.3pt,dashed,%
dash pattern = on 15pt off 3pt},%
splittopskip=\topskip,skipbelow=\baselineskip,%
skipabove=\baselineskip,ntheorem,roundcorner=0pt,
% backgroundcolor=pagebg,font=\color{orange}\sffamily, fontcolor=white
]{examplebox}{Exemple}[section]



\newcommand\R{\mathbb{R}}
\newcommand\Z{\mathbb{Z}}
\newcommand\N{\mathbb{N}}
\newcommand\E{\mathbb{E}}
\newcommand\F{\mathcal{F}}
\newcommand\cH{\mathcal{H}}
\newcommand\V{\mathbb{V}}
\newcommand\dmo{ ^{-1} }
\newcommand\kapa{\kappa}
\newcommand\im{Im}
\newcommand\hs{\mathcal{H}}





\usepackage{soul}

\makeatletter
\newcommand*{\whiten}[1]{\llap{\textcolor{white}{{\the\SOUL@token}}\hspace{#1pt}}}
\DeclareRobustCommand*\myul{%
    \def\SOUL@everyspace{\underline{\space}\kern\z@}%
    \def\SOUL@everytoken{%
     \setbox0=\hbox{\the\SOUL@token}%
     \ifdim\dp0>\z@
        \raisebox{\dp0}{\underline{\phantom{\the\SOUL@token}}}%
        \whiten{1}\whiten{0}%
        \whiten{-1}\whiten{-2}%
        \llap{\the\SOUL@token}%
     \else
        \underline{\the\SOUL@token}%
     \fi}%
\SOUL@}
\makeatother

\newcommand*{\demp}{\fontfamily{lmtt}\selectfont}

\DeclareTextFontCommand{\textdemp}{\demp}

\begin{document}

\ifcomment
Multiline
comment
\fi
\ifcomment
\myul{Typesetting test}
% \color[rgb]{1,1,1}
$∑_i^n≠ 60º±∞π∆¬≈√j∫h≤≥µ$

$\CR \R\pro\ind\pro\gS\pro
\mqty[a&b\\c&d]$
$\pro\mathbb{P}$
$\dd{x}$

  \[
    \alpha(x)=\left\{
                \begin{array}{ll}
                  x\\
                  \frac{1}{1+e^{-kx}}\\
                  \frac{e^x-e^{-x}}{e^x+e^{-x}}
                \end{array}
              \right.
  \]

  $\expval{x}$
  
  $\chi_\rho(ghg\dmo)=\Tr(\rho_{ghg\dmo})=\Tr(\rho_g\circ\rho_h\circ\rho\dmo_g)=\Tr(\rho_h)\overset{\mbox{\scalebox{0.5}{$\Tr(AB)=\Tr(BA)$}}}{=}\chi_\rho(h)$
  	$\mathop{\oplus}_{\substack{x\in X}}$

$\mat(\rho_g)=(a_{ij}(g))_{\scriptsize \substack{1\leq i\leq d \\ 1\leq j\leq d}}$ et $\mat(\rho'_g)=(a'_{ij}(g))_{\scriptsize \substack{1\leq i'\leq d' \\ 1\leq j'\leq d'}}$



\[\int_a^b{\mathbb{R}^2}g(u, v)\dd{P_{XY}}(u, v)=\iint g(u,v) f_{XY}(u, v)\dd \lambda(u) \dd \lambda(v)\]
$$\lim_{x\to\infty} f(x)$$	
$$\iiiint_V \mu(t,u,v,w) \,dt\,du\,dv\,dw$$
$$\sum_{n=1}^{\infty} 2^{-n} = 1$$	
\begin{definition}
	Si $X$ et $Y$ sont 2 v.a. ou definit la \textsc{Covariance} entre $X$ et $Y$ comme
	$\cov(X,Y)\overset{\text{def}}{=}\E\left[(X-\E(X))(Y-\E(Y))\right]=\E(XY)-\E(X)\E(Y)$.
\end{definition}
\fi
\pagebreak

% \tableofcontents

% insert your code here
%\input{./algebra/main.tex}
%\input{./geometrie-differentielle/main.tex}
%\input{./probabilite/main.tex}
%\input{./analyse-fonctionnelle/main.tex}
% \input{./Analyse-convexe-et-dualite-en-optimisation/main.tex}
%\input{./tikz/main.tex}
%\input{./Theorie-du-distributions/main.tex}
%\input{./optimisation/mine.tex}
 \input{./modelisation/main.tex}

% yves.aubry@univ-tln.fr : algebra

\end{document}

%% !TEX encoding = UTF-8 Unicode
% !TEX TS-program = xelatex

\documentclass[french]{report}

%\usepackage[utf8]{inputenc}
%\usepackage[T1]{fontenc}
\usepackage{babel}


\newif\ifcomment
%\commenttrue # Show comments

\usepackage{physics}
\usepackage{amssymb}


\usepackage{amsthm}
% \usepackage{thmtools}
\usepackage{mathtools}
\usepackage{amsfonts}

\usepackage{color}

\usepackage{tikz}

\usepackage{geometry}
\geometry{a5paper, margin=0.1in, right=1cm}

\usepackage{dsfont}

\usepackage{graphicx}
\graphicspath{ {images/} }

\usepackage{faktor}

\usepackage{IEEEtrantools}
\usepackage{enumerate}   
\usepackage[PostScript=dvips]{"/Users/aware/Documents/Courses/diagrams"}


\newtheorem{theorem}{Théorème}[section]
\renewcommand{\thetheorem}{\arabic{theorem}}
\newtheorem{lemme}{Lemme}[section]
\renewcommand{\thelemme}{\arabic{lemme}}
\newtheorem{proposition}{Proposition}[section]
\renewcommand{\theproposition}{\arabic{proposition}}
\newtheorem{notations}{Notations}[section]
\newtheorem{problem}{Problème}[section]
\newtheorem{corollary}{Corollaire}[theorem]
\renewcommand{\thecorollary}{\arabic{corollary}}
\newtheorem{property}{Propriété}[section]
\newtheorem{objective}{Objectif}[section]

\theoremstyle{definition}
\newtheorem{definition}{Définition}[section]
\renewcommand{\thedefinition}{\arabic{definition}}
\newtheorem{exercise}{Exercice}[chapter]
\renewcommand{\theexercise}{\arabic{exercise}}
\newtheorem{example}{Exemple}[chapter]
\renewcommand{\theexample}{\arabic{example}}
\newtheorem*{solution}{Solution}
\newtheorem*{application}{Application}
\newtheorem*{notation}{Notation}
\newtheorem*{vocabulary}{Vocabulaire}
\newtheorem*{properties}{Propriétés}



\theoremstyle{remark}
\newtheorem*{remark}{Remarque}
\newtheorem*{rappel}{Rappel}


\usepackage{etoolbox}
\AtBeginEnvironment{exercise}{\small}
\AtBeginEnvironment{example}{\small}

\usepackage{cases}
\usepackage[red]{mypack}

\usepackage[framemethod=TikZ]{mdframed}

\definecolor{bg}{rgb}{0.4,0.25,0.95}
\definecolor{pagebg}{rgb}{0,0,0.5}
\surroundwithmdframed[
   topline=false,
   rightline=false,
   bottomline=false,
   leftmargin=\parindent,
   skipabove=8pt,
   skipbelow=8pt,
   linecolor=blue,
   innerbottommargin=10pt,
   % backgroundcolor=bg,font=\color{orange}\sffamily, fontcolor=white
]{definition}

\usepackage{empheq}
\usepackage[most]{tcolorbox}

\newtcbox{\mymath}[1][]{%
    nobeforeafter, math upper, tcbox raise base,
    enhanced, colframe=blue!30!black,
    colback=red!10, boxrule=1pt,
    #1}

\usepackage{unixode}


\DeclareMathOperator{\ord}{ord}
\DeclareMathOperator{\orb}{orb}
\DeclareMathOperator{\stab}{stab}
\DeclareMathOperator{\Stab}{stab}
\DeclareMathOperator{\ppcm}{ppcm}
\DeclareMathOperator{\conj}{Conj}
\DeclareMathOperator{\End}{End}
\DeclareMathOperator{\rot}{rot}
\DeclareMathOperator{\trs}{trace}
\DeclareMathOperator{\Ind}{Ind}
\DeclareMathOperator{\mat}{Mat}
\DeclareMathOperator{\id}{Id}
\DeclareMathOperator{\vect}{vect}
\DeclareMathOperator{\img}{img}
\DeclareMathOperator{\cov}{Cov}
\DeclareMathOperator{\dist}{dist}
\DeclareMathOperator{\irr}{Irr}
\DeclareMathOperator{\image}{Im}
\DeclareMathOperator{\pd}{\partial}
\DeclareMathOperator{\epi}{epi}
\DeclareMathOperator{\Argmin}{Argmin}
\DeclareMathOperator{\dom}{dom}
\DeclareMathOperator{\proj}{proj}
\DeclareMathOperator{\ctg}{ctg}
\DeclareMathOperator{\supp}{supp}
\DeclareMathOperator{\argmin}{argmin}
\DeclareMathOperator{\mult}{mult}
\DeclareMathOperator{\ch}{ch}
\DeclareMathOperator{\sh}{sh}
\DeclareMathOperator{\rang}{rang}
\DeclareMathOperator{\diam}{diam}
\DeclareMathOperator{\Epigraphe}{Epigraphe}




\usepackage{xcolor}
\everymath{\color{blue}}
%\everymath{\color[rgb]{0,1,1}}
%\pagecolor[rgb]{0,0,0.5}


\newcommand*{\pdtest}[3][]{\ensuremath{\frac{\partial^{#1} #2}{\partial #3}}}

\newcommand*{\deffunc}[6][]{\ensuremath{
\begin{array}{rcl}
#2 : #3 &\rightarrow& #4\\
#5 &\mapsto& #6
\end{array}
}}

\newcommand{\eqcolon}{\mathrel{\resizebox{\widthof{$\mathord{=}$}}{\height}{ $\!\!=\!\!\resizebox{1.2\width}{0.8\height}{\raisebox{0.23ex}{$\mathop{:}$}}\!\!$ }}}
\newcommand{\coloneq}{\mathrel{\resizebox{\widthof{$\mathord{=}$}}{\height}{ $\!\!\resizebox{1.2\width}{0.8\height}{\raisebox{0.23ex}{$\mathop{:}$}}\!\!=\!\!$ }}}
\newcommand{\eqcolonl}{\ensuremath{\mathrel{=\!\!\mathop{:}}}}
\newcommand{\coloneql}{\ensuremath{\mathrel{\mathop{:} \!\! =}}}
\newcommand{\vc}[1]{% inline column vector
  \left(\begin{smallmatrix}#1\end{smallmatrix}\right)%
}
\newcommand{\vr}[1]{% inline row vector
  \begin{smallmatrix}(\,#1\,)\end{smallmatrix}%
}
\makeatletter
\newcommand*{\defeq}{\ =\mathrel{\rlap{%
                     \raisebox{0.3ex}{$\m@th\cdot$}}%
                     \raisebox{-0.3ex}{$\m@th\cdot$}}%
                     }
\makeatother

\newcommand{\mathcircle}[1]{% inline row vector
 \overset{\circ}{#1}
}
\newcommand{\ulim}{% low limit
 \underline{\lim}
}
\newcommand{\ssi}{% iff
\iff
}
\newcommand{\ps}[2]{
\expval{#1 | #2}
}
\newcommand{\df}[1]{
\mqty{#1}
}
\newcommand{\n}[1]{
\norm{#1}
}
\newcommand{\sys}[1]{
\left\{\smqty{#1}\right.
}


\newcommand{\eqdef}{\ensuremath{\overset{\text{def}}=}}


\def\Circlearrowright{\ensuremath{%
  \rotatebox[origin=c]{230}{$\circlearrowright$}}}

\newcommand\ct[1]{\text{\rmfamily\upshape #1}}
\newcommand\question[1]{ {\color{red} ...!? \small #1}}
\newcommand\caz[1]{\left\{\begin{array} #1 \end{array}\right.}
\newcommand\const{\text{\rmfamily\upshape const}}
\newcommand\toP{ \overset{\pro}{\to}}
\newcommand\toPP{ \overset{\text{PP}}{\to}}
\newcommand{\oeq}{\mathrel{\text{\textcircled{$=$}}}}





\usepackage{xcolor}
% \usepackage[normalem]{ulem}
\usepackage{lipsum}
\makeatletter
% \newcommand\colorwave[1][blue]{\bgroup \markoverwith{\lower3.5\p@\hbox{\sixly \textcolor{#1}{\char58}}}\ULon}
%\font\sixly=lasy6 % does not re-load if already loaded, so no memory problem.

\newmdtheoremenv[
linewidth= 1pt,linecolor= blue,%
leftmargin=20,rightmargin=20,innertopmargin=0pt, innerrightmargin=40,%
tikzsetting = { draw=lightgray, line width = 0.3pt,dashed,%
dash pattern = on 15pt off 3pt},%
splittopskip=\topskip,skipbelow=\baselineskip,%
skipabove=\baselineskip,ntheorem,roundcorner=0pt,
% backgroundcolor=pagebg,font=\color{orange}\sffamily, fontcolor=white
]{examplebox}{Exemple}[section]



\newcommand\R{\mathbb{R}}
\newcommand\Z{\mathbb{Z}}
\newcommand\N{\mathbb{N}}
\newcommand\E{\mathbb{E}}
\newcommand\F{\mathcal{F}}
\newcommand\cH{\mathcal{H}}
\newcommand\V{\mathbb{V}}
\newcommand\dmo{ ^{-1} }
\newcommand\kapa{\kappa}
\newcommand\im{Im}
\newcommand\hs{\mathcal{H}}





\usepackage{soul}

\makeatletter
\newcommand*{\whiten}[1]{\llap{\textcolor{white}{{\the\SOUL@token}}\hspace{#1pt}}}
\DeclareRobustCommand*\myul{%
    \def\SOUL@everyspace{\underline{\space}\kern\z@}%
    \def\SOUL@everytoken{%
     \setbox0=\hbox{\the\SOUL@token}%
     \ifdim\dp0>\z@
        \raisebox{\dp0}{\underline{\phantom{\the\SOUL@token}}}%
        \whiten{1}\whiten{0}%
        \whiten{-1}\whiten{-2}%
        \llap{\the\SOUL@token}%
     \else
        \underline{\the\SOUL@token}%
     \fi}%
\SOUL@}
\makeatother

\newcommand*{\demp}{\fontfamily{lmtt}\selectfont}

\DeclareTextFontCommand{\textdemp}{\demp}

\begin{document}

\ifcomment
Multiline
comment
\fi
\ifcomment
\myul{Typesetting test}
% \color[rgb]{1,1,1}
$∑_i^n≠ 60º±∞π∆¬≈√j∫h≤≥µ$

$\CR \R\pro\ind\pro\gS\pro
\mqty[a&b\\c&d]$
$\pro\mathbb{P}$
$\dd{x}$

  \[
    \alpha(x)=\left\{
                \begin{array}{ll}
                  x\\
                  \frac{1}{1+e^{-kx}}\\
                  \frac{e^x-e^{-x}}{e^x+e^{-x}}
                \end{array}
              \right.
  \]

  $\expval{x}$
  
  $\chi_\rho(ghg\dmo)=\Tr(\rho_{ghg\dmo})=\Tr(\rho_g\circ\rho_h\circ\rho\dmo_g)=\Tr(\rho_h)\overset{\mbox{\scalebox{0.5}{$\Tr(AB)=\Tr(BA)$}}}{=}\chi_\rho(h)$
  	$\mathop{\oplus}_{\substack{x\in X}}$

$\mat(\rho_g)=(a_{ij}(g))_{\scriptsize \substack{1\leq i\leq d \\ 1\leq j\leq d}}$ et $\mat(\rho'_g)=(a'_{ij}(g))_{\scriptsize \substack{1\leq i'\leq d' \\ 1\leq j'\leq d'}}$



\[\int_a^b{\mathbb{R}^2}g(u, v)\dd{P_{XY}}(u, v)=\iint g(u,v) f_{XY}(u, v)\dd \lambda(u) \dd \lambda(v)\]
$$\lim_{x\to\infty} f(x)$$	
$$\iiiint_V \mu(t,u,v,w) \,dt\,du\,dv\,dw$$
$$\sum_{n=1}^{\infty} 2^{-n} = 1$$	
\begin{definition}
	Si $X$ et $Y$ sont 2 v.a. ou definit la \textsc{Covariance} entre $X$ et $Y$ comme
	$\cov(X,Y)\overset{\text{def}}{=}\E\left[(X-\E(X))(Y-\E(Y))\right]=\E(XY)-\E(X)\E(Y)$.
\end{definition}
\fi
\pagebreak

% \tableofcontents

% insert your code here
%\input{./algebra/main.tex}
%\input{./geometrie-differentielle/main.tex}
%\input{./probabilite/main.tex}
%\input{./analyse-fonctionnelle/main.tex}
% \input{./Analyse-convexe-et-dualite-en-optimisation/main.tex}
%\input{./tikz/main.tex}
%\input{./Theorie-du-distributions/main.tex}
%\input{./optimisation/mine.tex}
 \input{./modelisation/main.tex}

% yves.aubry@univ-tln.fr : algebra

\end{document}

%\input{./optimisation/mine.tex}
 % !TEX encoding = UTF-8 Unicode
% !TEX TS-program = xelatex

\documentclass[french]{report}

%\usepackage[utf8]{inputenc}
%\usepackage[T1]{fontenc}
\usepackage{babel}


\newif\ifcomment
%\commenttrue # Show comments

\usepackage{physics}
\usepackage{amssymb}


\usepackage{amsthm}
% \usepackage{thmtools}
\usepackage{mathtools}
\usepackage{amsfonts}

\usepackage{color}

\usepackage{tikz}

\usepackage{geometry}
\geometry{a5paper, margin=0.1in, right=1cm}

\usepackage{dsfont}

\usepackage{graphicx}
\graphicspath{ {images/} }

\usepackage{faktor}

\usepackage{IEEEtrantools}
\usepackage{enumerate}   
\usepackage[PostScript=dvips]{"/Users/aware/Documents/Courses/diagrams"}


\newtheorem{theorem}{Théorème}[section]
\renewcommand{\thetheorem}{\arabic{theorem}}
\newtheorem{lemme}{Lemme}[section]
\renewcommand{\thelemme}{\arabic{lemme}}
\newtheorem{proposition}{Proposition}[section]
\renewcommand{\theproposition}{\arabic{proposition}}
\newtheorem{notations}{Notations}[section]
\newtheorem{problem}{Problème}[section]
\newtheorem{corollary}{Corollaire}[theorem]
\renewcommand{\thecorollary}{\arabic{corollary}}
\newtheorem{property}{Propriété}[section]
\newtheorem{objective}{Objectif}[section]

\theoremstyle{definition}
\newtheorem{definition}{Définition}[section]
\renewcommand{\thedefinition}{\arabic{definition}}
\newtheorem{exercise}{Exercice}[chapter]
\renewcommand{\theexercise}{\arabic{exercise}}
\newtheorem{example}{Exemple}[chapter]
\renewcommand{\theexample}{\arabic{example}}
\newtheorem*{solution}{Solution}
\newtheorem*{application}{Application}
\newtheorem*{notation}{Notation}
\newtheorem*{vocabulary}{Vocabulaire}
\newtheorem*{properties}{Propriétés}



\theoremstyle{remark}
\newtheorem*{remark}{Remarque}
\newtheorem*{rappel}{Rappel}


\usepackage{etoolbox}
\AtBeginEnvironment{exercise}{\small}
\AtBeginEnvironment{example}{\small}

\usepackage{cases}
\usepackage[red]{mypack}

\usepackage[framemethod=TikZ]{mdframed}

\definecolor{bg}{rgb}{0.4,0.25,0.95}
\definecolor{pagebg}{rgb}{0,0,0.5}
\surroundwithmdframed[
   topline=false,
   rightline=false,
   bottomline=false,
   leftmargin=\parindent,
   skipabove=8pt,
   skipbelow=8pt,
   linecolor=blue,
   innerbottommargin=10pt,
   % backgroundcolor=bg,font=\color{orange}\sffamily, fontcolor=white
]{definition}

\usepackage{empheq}
\usepackage[most]{tcolorbox}

\newtcbox{\mymath}[1][]{%
    nobeforeafter, math upper, tcbox raise base,
    enhanced, colframe=blue!30!black,
    colback=red!10, boxrule=1pt,
    #1}

\usepackage{unixode}


\DeclareMathOperator{\ord}{ord}
\DeclareMathOperator{\orb}{orb}
\DeclareMathOperator{\stab}{stab}
\DeclareMathOperator{\Stab}{stab}
\DeclareMathOperator{\ppcm}{ppcm}
\DeclareMathOperator{\conj}{Conj}
\DeclareMathOperator{\End}{End}
\DeclareMathOperator{\rot}{rot}
\DeclareMathOperator{\trs}{trace}
\DeclareMathOperator{\Ind}{Ind}
\DeclareMathOperator{\mat}{Mat}
\DeclareMathOperator{\id}{Id}
\DeclareMathOperator{\vect}{vect}
\DeclareMathOperator{\img}{img}
\DeclareMathOperator{\cov}{Cov}
\DeclareMathOperator{\dist}{dist}
\DeclareMathOperator{\irr}{Irr}
\DeclareMathOperator{\image}{Im}
\DeclareMathOperator{\pd}{\partial}
\DeclareMathOperator{\epi}{epi}
\DeclareMathOperator{\Argmin}{Argmin}
\DeclareMathOperator{\dom}{dom}
\DeclareMathOperator{\proj}{proj}
\DeclareMathOperator{\ctg}{ctg}
\DeclareMathOperator{\supp}{supp}
\DeclareMathOperator{\argmin}{argmin}
\DeclareMathOperator{\mult}{mult}
\DeclareMathOperator{\ch}{ch}
\DeclareMathOperator{\sh}{sh}
\DeclareMathOperator{\rang}{rang}
\DeclareMathOperator{\diam}{diam}
\DeclareMathOperator{\Epigraphe}{Epigraphe}




\usepackage{xcolor}
\everymath{\color{blue}}
%\everymath{\color[rgb]{0,1,1}}
%\pagecolor[rgb]{0,0,0.5}


\newcommand*{\pdtest}[3][]{\ensuremath{\frac{\partial^{#1} #2}{\partial #3}}}

\newcommand*{\deffunc}[6][]{\ensuremath{
\begin{array}{rcl}
#2 : #3 &\rightarrow& #4\\
#5 &\mapsto& #6
\end{array}
}}

\newcommand{\eqcolon}{\mathrel{\resizebox{\widthof{$\mathord{=}$}}{\height}{ $\!\!=\!\!\resizebox{1.2\width}{0.8\height}{\raisebox{0.23ex}{$\mathop{:}$}}\!\!$ }}}
\newcommand{\coloneq}{\mathrel{\resizebox{\widthof{$\mathord{=}$}}{\height}{ $\!\!\resizebox{1.2\width}{0.8\height}{\raisebox{0.23ex}{$\mathop{:}$}}\!\!=\!\!$ }}}
\newcommand{\eqcolonl}{\ensuremath{\mathrel{=\!\!\mathop{:}}}}
\newcommand{\coloneql}{\ensuremath{\mathrel{\mathop{:} \!\! =}}}
\newcommand{\vc}[1]{% inline column vector
  \left(\begin{smallmatrix}#1\end{smallmatrix}\right)%
}
\newcommand{\vr}[1]{% inline row vector
  \begin{smallmatrix}(\,#1\,)\end{smallmatrix}%
}
\makeatletter
\newcommand*{\defeq}{\ =\mathrel{\rlap{%
                     \raisebox{0.3ex}{$\m@th\cdot$}}%
                     \raisebox{-0.3ex}{$\m@th\cdot$}}%
                     }
\makeatother

\newcommand{\mathcircle}[1]{% inline row vector
 \overset{\circ}{#1}
}
\newcommand{\ulim}{% low limit
 \underline{\lim}
}
\newcommand{\ssi}{% iff
\iff
}
\newcommand{\ps}[2]{
\expval{#1 | #2}
}
\newcommand{\df}[1]{
\mqty{#1}
}
\newcommand{\n}[1]{
\norm{#1}
}
\newcommand{\sys}[1]{
\left\{\smqty{#1}\right.
}


\newcommand{\eqdef}{\ensuremath{\overset{\text{def}}=}}


\def\Circlearrowright{\ensuremath{%
  \rotatebox[origin=c]{230}{$\circlearrowright$}}}

\newcommand\ct[1]{\text{\rmfamily\upshape #1}}
\newcommand\question[1]{ {\color{red} ...!? \small #1}}
\newcommand\caz[1]{\left\{\begin{array} #1 \end{array}\right.}
\newcommand\const{\text{\rmfamily\upshape const}}
\newcommand\toP{ \overset{\pro}{\to}}
\newcommand\toPP{ \overset{\text{PP}}{\to}}
\newcommand{\oeq}{\mathrel{\text{\textcircled{$=$}}}}





\usepackage{xcolor}
% \usepackage[normalem]{ulem}
\usepackage{lipsum}
\makeatletter
% \newcommand\colorwave[1][blue]{\bgroup \markoverwith{\lower3.5\p@\hbox{\sixly \textcolor{#1}{\char58}}}\ULon}
%\font\sixly=lasy6 % does not re-load if already loaded, so no memory problem.

\newmdtheoremenv[
linewidth= 1pt,linecolor= blue,%
leftmargin=20,rightmargin=20,innertopmargin=0pt, innerrightmargin=40,%
tikzsetting = { draw=lightgray, line width = 0.3pt,dashed,%
dash pattern = on 15pt off 3pt},%
splittopskip=\topskip,skipbelow=\baselineskip,%
skipabove=\baselineskip,ntheorem,roundcorner=0pt,
% backgroundcolor=pagebg,font=\color{orange}\sffamily, fontcolor=white
]{examplebox}{Exemple}[section]



\newcommand\R{\mathbb{R}}
\newcommand\Z{\mathbb{Z}}
\newcommand\N{\mathbb{N}}
\newcommand\E{\mathbb{E}}
\newcommand\F{\mathcal{F}}
\newcommand\cH{\mathcal{H}}
\newcommand\V{\mathbb{V}}
\newcommand\dmo{ ^{-1} }
\newcommand\kapa{\kappa}
\newcommand\im{Im}
\newcommand\hs{\mathcal{H}}





\usepackage{soul}

\makeatletter
\newcommand*{\whiten}[1]{\llap{\textcolor{white}{{\the\SOUL@token}}\hspace{#1pt}}}
\DeclareRobustCommand*\myul{%
    \def\SOUL@everyspace{\underline{\space}\kern\z@}%
    \def\SOUL@everytoken{%
     \setbox0=\hbox{\the\SOUL@token}%
     \ifdim\dp0>\z@
        \raisebox{\dp0}{\underline{\phantom{\the\SOUL@token}}}%
        \whiten{1}\whiten{0}%
        \whiten{-1}\whiten{-2}%
        \llap{\the\SOUL@token}%
     \else
        \underline{\the\SOUL@token}%
     \fi}%
\SOUL@}
\makeatother

\newcommand*{\demp}{\fontfamily{lmtt}\selectfont}

\DeclareTextFontCommand{\textdemp}{\demp}

\begin{document}

\ifcomment
Multiline
comment
\fi
\ifcomment
\myul{Typesetting test}
% \color[rgb]{1,1,1}
$∑_i^n≠ 60º±∞π∆¬≈√j∫h≤≥µ$

$\CR \R\pro\ind\pro\gS\pro
\mqty[a&b\\c&d]$
$\pro\mathbb{P}$
$\dd{x}$

  \[
    \alpha(x)=\left\{
                \begin{array}{ll}
                  x\\
                  \frac{1}{1+e^{-kx}}\\
                  \frac{e^x-e^{-x}}{e^x+e^{-x}}
                \end{array}
              \right.
  \]

  $\expval{x}$
  
  $\chi_\rho(ghg\dmo)=\Tr(\rho_{ghg\dmo})=\Tr(\rho_g\circ\rho_h\circ\rho\dmo_g)=\Tr(\rho_h)\overset{\mbox{\scalebox{0.5}{$\Tr(AB)=\Tr(BA)$}}}{=}\chi_\rho(h)$
  	$\mathop{\oplus}_{\substack{x\in X}}$

$\mat(\rho_g)=(a_{ij}(g))_{\scriptsize \substack{1\leq i\leq d \\ 1\leq j\leq d}}$ et $\mat(\rho'_g)=(a'_{ij}(g))_{\scriptsize \substack{1\leq i'\leq d' \\ 1\leq j'\leq d'}}$



\[\int_a^b{\mathbb{R}^2}g(u, v)\dd{P_{XY}}(u, v)=\iint g(u,v) f_{XY}(u, v)\dd \lambda(u) \dd \lambda(v)\]
$$\lim_{x\to\infty} f(x)$$	
$$\iiiint_V \mu(t,u,v,w) \,dt\,du\,dv\,dw$$
$$\sum_{n=1}^{\infty} 2^{-n} = 1$$	
\begin{definition}
	Si $X$ et $Y$ sont 2 v.a. ou definit la \textsc{Covariance} entre $X$ et $Y$ comme
	$\cov(X,Y)\overset{\text{def}}{=}\E\left[(X-\E(X))(Y-\E(Y))\right]=\E(XY)-\E(X)\E(Y)$.
\end{definition}
\fi
\pagebreak

% \tableofcontents

% insert your code here
%\input{./algebra/main.tex}
%\input{./geometrie-differentielle/main.tex}
%\input{./probabilite/main.tex}
%\input{./analyse-fonctionnelle/main.tex}
% \input{./Analyse-convexe-et-dualite-en-optimisation/main.tex}
%\input{./tikz/main.tex}
%\input{./Theorie-du-distributions/main.tex}
%\input{./optimisation/mine.tex}
 \input{./modelisation/main.tex}

% yves.aubry@univ-tln.fr : algebra

\end{document}


% yves.aubry@univ-tln.fr : algebra

\end{document}

%% !TEX encoding = UTF-8 Unicode
% !TEX TS-program = xelatex

\documentclass[french]{report}

%\usepackage[utf8]{inputenc}
%\usepackage[T1]{fontenc}
\usepackage{babel}


\newif\ifcomment
%\commenttrue # Show comments

\usepackage{physics}
\usepackage{amssymb}


\usepackage{amsthm}
% \usepackage{thmtools}
\usepackage{mathtools}
\usepackage{amsfonts}

\usepackage{color}

\usepackage{tikz}

\usepackage{geometry}
\geometry{a5paper, margin=0.1in, right=1cm}

\usepackage{dsfont}

\usepackage{graphicx}
\graphicspath{ {images/} }

\usepackage{faktor}

\usepackage{IEEEtrantools}
\usepackage{enumerate}   
\usepackage[PostScript=dvips]{"/Users/aware/Documents/Courses/diagrams"}


\newtheorem{theorem}{Théorème}[section]
\renewcommand{\thetheorem}{\arabic{theorem}}
\newtheorem{lemme}{Lemme}[section]
\renewcommand{\thelemme}{\arabic{lemme}}
\newtheorem{proposition}{Proposition}[section]
\renewcommand{\theproposition}{\arabic{proposition}}
\newtheorem{notations}{Notations}[section]
\newtheorem{problem}{Problème}[section]
\newtheorem{corollary}{Corollaire}[theorem]
\renewcommand{\thecorollary}{\arabic{corollary}}
\newtheorem{property}{Propriété}[section]
\newtheorem{objective}{Objectif}[section]

\theoremstyle{definition}
\newtheorem{definition}{Définition}[section]
\renewcommand{\thedefinition}{\arabic{definition}}
\newtheorem{exercise}{Exercice}[chapter]
\renewcommand{\theexercise}{\arabic{exercise}}
\newtheorem{example}{Exemple}[chapter]
\renewcommand{\theexample}{\arabic{example}}
\newtheorem*{solution}{Solution}
\newtheorem*{application}{Application}
\newtheorem*{notation}{Notation}
\newtheorem*{vocabulary}{Vocabulaire}
\newtheorem*{properties}{Propriétés}



\theoremstyle{remark}
\newtheorem*{remark}{Remarque}
\newtheorem*{rappel}{Rappel}


\usepackage{etoolbox}
\AtBeginEnvironment{exercise}{\small}
\AtBeginEnvironment{example}{\small}

\usepackage{cases}
\usepackage[red]{mypack}

\usepackage[framemethod=TikZ]{mdframed}

\definecolor{bg}{rgb}{0.4,0.25,0.95}
\definecolor{pagebg}{rgb}{0,0,0.5}
\surroundwithmdframed[
   topline=false,
   rightline=false,
   bottomline=false,
   leftmargin=\parindent,
   skipabove=8pt,
   skipbelow=8pt,
   linecolor=blue,
   innerbottommargin=10pt,
   % backgroundcolor=bg,font=\color{orange}\sffamily, fontcolor=white
]{definition}

\usepackage{empheq}
\usepackage[most]{tcolorbox}

\newtcbox{\mymath}[1][]{%
    nobeforeafter, math upper, tcbox raise base,
    enhanced, colframe=blue!30!black,
    colback=red!10, boxrule=1pt,
    #1}

\usepackage{unixode}


\DeclareMathOperator{\ord}{ord}
\DeclareMathOperator{\orb}{orb}
\DeclareMathOperator{\stab}{stab}
\DeclareMathOperator{\Stab}{stab}
\DeclareMathOperator{\ppcm}{ppcm}
\DeclareMathOperator{\conj}{Conj}
\DeclareMathOperator{\End}{End}
\DeclareMathOperator{\rot}{rot}
\DeclareMathOperator{\trs}{trace}
\DeclareMathOperator{\Ind}{Ind}
\DeclareMathOperator{\mat}{Mat}
\DeclareMathOperator{\id}{Id}
\DeclareMathOperator{\vect}{vect}
\DeclareMathOperator{\img}{img}
\DeclareMathOperator{\cov}{Cov}
\DeclareMathOperator{\dist}{dist}
\DeclareMathOperator{\irr}{Irr}
\DeclareMathOperator{\image}{Im}
\DeclareMathOperator{\pd}{\partial}
\DeclareMathOperator{\epi}{epi}
\DeclareMathOperator{\Argmin}{Argmin}
\DeclareMathOperator{\dom}{dom}
\DeclareMathOperator{\proj}{proj}
\DeclareMathOperator{\ctg}{ctg}
\DeclareMathOperator{\supp}{supp}
\DeclareMathOperator{\argmin}{argmin}
\DeclareMathOperator{\mult}{mult}
\DeclareMathOperator{\ch}{ch}
\DeclareMathOperator{\sh}{sh}
\DeclareMathOperator{\rang}{rang}
\DeclareMathOperator{\diam}{diam}
\DeclareMathOperator{\Epigraphe}{Epigraphe}




\usepackage{xcolor}
\everymath{\color{blue}}
%\everymath{\color[rgb]{0,1,1}}
%\pagecolor[rgb]{0,0,0.5}


\newcommand*{\pdtest}[3][]{\ensuremath{\frac{\partial^{#1} #2}{\partial #3}}}

\newcommand*{\deffunc}[6][]{\ensuremath{
\begin{array}{rcl}
#2 : #3 &\rightarrow& #4\\
#5 &\mapsto& #6
\end{array}
}}

\newcommand{\eqcolon}{\mathrel{\resizebox{\widthof{$\mathord{=}$}}{\height}{ $\!\!=\!\!\resizebox{1.2\width}{0.8\height}{\raisebox{0.23ex}{$\mathop{:}$}}\!\!$ }}}
\newcommand{\coloneq}{\mathrel{\resizebox{\widthof{$\mathord{=}$}}{\height}{ $\!\!\resizebox{1.2\width}{0.8\height}{\raisebox{0.23ex}{$\mathop{:}$}}\!\!=\!\!$ }}}
\newcommand{\eqcolonl}{\ensuremath{\mathrel{=\!\!\mathop{:}}}}
\newcommand{\coloneql}{\ensuremath{\mathrel{\mathop{:} \!\! =}}}
\newcommand{\vc}[1]{% inline column vector
  \left(\begin{smallmatrix}#1\end{smallmatrix}\right)%
}
\newcommand{\vr}[1]{% inline row vector
  \begin{smallmatrix}(\,#1\,)\end{smallmatrix}%
}
\makeatletter
\newcommand*{\defeq}{\ =\mathrel{\rlap{%
                     \raisebox{0.3ex}{$\m@th\cdot$}}%
                     \raisebox{-0.3ex}{$\m@th\cdot$}}%
                     }
\makeatother

\newcommand{\mathcircle}[1]{% inline row vector
 \overset{\circ}{#1}
}
\newcommand{\ulim}{% low limit
 \underline{\lim}
}
\newcommand{\ssi}{% iff
\iff
}
\newcommand{\ps}[2]{
\expval{#1 | #2}
}
\newcommand{\df}[1]{
\mqty{#1}
}
\newcommand{\n}[1]{
\norm{#1}
}
\newcommand{\sys}[1]{
\left\{\smqty{#1}\right.
}


\newcommand{\eqdef}{\ensuremath{\overset{\text{def}}=}}


\def\Circlearrowright{\ensuremath{%
  \rotatebox[origin=c]{230}{$\circlearrowright$}}}

\newcommand\ct[1]{\text{\rmfamily\upshape #1}}
\newcommand\question[1]{ {\color{red} ...!? \small #1}}
\newcommand\caz[1]{\left\{\begin{array} #1 \end{array}\right.}
\newcommand\const{\text{\rmfamily\upshape const}}
\newcommand\toP{ \overset{\pro}{\to}}
\newcommand\toPP{ \overset{\text{PP}}{\to}}
\newcommand{\oeq}{\mathrel{\text{\textcircled{$=$}}}}





\usepackage{xcolor}
% \usepackage[normalem]{ulem}
\usepackage{lipsum}
\makeatletter
% \newcommand\colorwave[1][blue]{\bgroup \markoverwith{\lower3.5\p@\hbox{\sixly \textcolor{#1}{\char58}}}\ULon}
%\font\sixly=lasy6 % does not re-load if already loaded, so no memory problem.

\newmdtheoremenv[
linewidth= 1pt,linecolor= blue,%
leftmargin=20,rightmargin=20,innertopmargin=0pt, innerrightmargin=40,%
tikzsetting = { draw=lightgray, line width = 0.3pt,dashed,%
dash pattern = on 15pt off 3pt},%
splittopskip=\topskip,skipbelow=\baselineskip,%
skipabove=\baselineskip,ntheorem,roundcorner=0pt,
% backgroundcolor=pagebg,font=\color{orange}\sffamily, fontcolor=white
]{examplebox}{Exemple}[section]



\newcommand\R{\mathbb{R}}
\newcommand\Z{\mathbb{Z}}
\newcommand\N{\mathbb{N}}
\newcommand\E{\mathbb{E}}
\newcommand\F{\mathcal{F}}
\newcommand\cH{\mathcal{H}}
\newcommand\V{\mathbb{V}}
\newcommand\dmo{ ^{-1} }
\newcommand\kapa{\kappa}
\newcommand\im{Im}
\newcommand\hs{\mathcal{H}}





\usepackage{soul}

\makeatletter
\newcommand*{\whiten}[1]{\llap{\textcolor{white}{{\the\SOUL@token}}\hspace{#1pt}}}
\DeclareRobustCommand*\myul{%
    \def\SOUL@everyspace{\underline{\space}\kern\z@}%
    \def\SOUL@everytoken{%
     \setbox0=\hbox{\the\SOUL@token}%
     \ifdim\dp0>\z@
        \raisebox{\dp0}{\underline{\phantom{\the\SOUL@token}}}%
        \whiten{1}\whiten{0}%
        \whiten{-1}\whiten{-2}%
        \llap{\the\SOUL@token}%
     \else
        \underline{\the\SOUL@token}%
     \fi}%
\SOUL@}
\makeatother

\newcommand*{\demp}{\fontfamily{lmtt}\selectfont}

\DeclareTextFontCommand{\textdemp}{\demp}

\begin{document}

\ifcomment
Multiline
comment
\fi
\ifcomment
\myul{Typesetting test}
% \color[rgb]{1,1,1}
$∑_i^n≠ 60º±∞π∆¬≈√j∫h≤≥µ$

$\CR \R\pro\ind\pro\gS\pro
\mqty[a&b\\c&d]$
$\pro\mathbb{P}$
$\dd{x}$

  \[
    \alpha(x)=\left\{
                \begin{array}{ll}
                  x\\
                  \frac{1}{1+e^{-kx}}\\
                  \frac{e^x-e^{-x}}{e^x+e^{-x}}
                \end{array}
              \right.
  \]

  $\expval{x}$
  
  $\chi_\rho(ghg\dmo)=\Tr(\rho_{ghg\dmo})=\Tr(\rho_g\circ\rho_h\circ\rho\dmo_g)=\Tr(\rho_h)\overset{\mbox{\scalebox{0.5}{$\Tr(AB)=\Tr(BA)$}}}{=}\chi_\rho(h)$
  	$\mathop{\oplus}_{\substack{x\in X}}$

$\mat(\rho_g)=(a_{ij}(g))_{\scriptsize \substack{1\leq i\leq d \\ 1\leq j\leq d}}$ et $\mat(\rho'_g)=(a'_{ij}(g))_{\scriptsize \substack{1\leq i'\leq d' \\ 1\leq j'\leq d'}}$



\[\int_a^b{\mathbb{R}^2}g(u, v)\dd{P_{XY}}(u, v)=\iint g(u,v) f_{XY}(u, v)\dd \lambda(u) \dd \lambda(v)\]
$$\lim_{x\to\infty} f(x)$$	
$$\iiiint_V \mu(t,u,v,w) \,dt\,du\,dv\,dw$$
$$\sum_{n=1}^{\infty} 2^{-n} = 1$$	
\begin{definition}
	Si $X$ et $Y$ sont 2 v.a. ou definit la \textsc{Covariance} entre $X$ et $Y$ comme
	$\cov(X,Y)\overset{\text{def}}{=}\E\left[(X-\E(X))(Y-\E(Y))\right]=\E(XY)-\E(X)\E(Y)$.
\end{definition}
\fi
\pagebreak

% \tableofcontents

% insert your code here
%% !TEX encoding = UTF-8 Unicode
% !TEX TS-program = xelatex

\documentclass[french]{report}

%\usepackage[utf8]{inputenc}
%\usepackage[T1]{fontenc}
\usepackage{babel}


\newif\ifcomment
%\commenttrue # Show comments

\usepackage{physics}
\usepackage{amssymb}


\usepackage{amsthm}
% \usepackage{thmtools}
\usepackage{mathtools}
\usepackage{amsfonts}

\usepackage{color}

\usepackage{tikz}

\usepackage{geometry}
\geometry{a5paper, margin=0.1in, right=1cm}

\usepackage{dsfont}

\usepackage{graphicx}
\graphicspath{ {images/} }

\usepackage{faktor}

\usepackage{IEEEtrantools}
\usepackage{enumerate}   
\usepackage[PostScript=dvips]{"/Users/aware/Documents/Courses/diagrams"}


\newtheorem{theorem}{Théorème}[section]
\renewcommand{\thetheorem}{\arabic{theorem}}
\newtheorem{lemme}{Lemme}[section]
\renewcommand{\thelemme}{\arabic{lemme}}
\newtheorem{proposition}{Proposition}[section]
\renewcommand{\theproposition}{\arabic{proposition}}
\newtheorem{notations}{Notations}[section]
\newtheorem{problem}{Problème}[section]
\newtheorem{corollary}{Corollaire}[theorem]
\renewcommand{\thecorollary}{\arabic{corollary}}
\newtheorem{property}{Propriété}[section]
\newtheorem{objective}{Objectif}[section]

\theoremstyle{definition}
\newtheorem{definition}{Définition}[section]
\renewcommand{\thedefinition}{\arabic{definition}}
\newtheorem{exercise}{Exercice}[chapter]
\renewcommand{\theexercise}{\arabic{exercise}}
\newtheorem{example}{Exemple}[chapter]
\renewcommand{\theexample}{\arabic{example}}
\newtheorem*{solution}{Solution}
\newtheorem*{application}{Application}
\newtheorem*{notation}{Notation}
\newtheorem*{vocabulary}{Vocabulaire}
\newtheorem*{properties}{Propriétés}



\theoremstyle{remark}
\newtheorem*{remark}{Remarque}
\newtheorem*{rappel}{Rappel}


\usepackage{etoolbox}
\AtBeginEnvironment{exercise}{\small}
\AtBeginEnvironment{example}{\small}

\usepackage{cases}
\usepackage[red]{mypack}

\usepackage[framemethod=TikZ]{mdframed}

\definecolor{bg}{rgb}{0.4,0.25,0.95}
\definecolor{pagebg}{rgb}{0,0,0.5}
\surroundwithmdframed[
   topline=false,
   rightline=false,
   bottomline=false,
   leftmargin=\parindent,
   skipabove=8pt,
   skipbelow=8pt,
   linecolor=blue,
   innerbottommargin=10pt,
   % backgroundcolor=bg,font=\color{orange}\sffamily, fontcolor=white
]{definition}

\usepackage{empheq}
\usepackage[most]{tcolorbox}

\newtcbox{\mymath}[1][]{%
    nobeforeafter, math upper, tcbox raise base,
    enhanced, colframe=blue!30!black,
    colback=red!10, boxrule=1pt,
    #1}

\usepackage{unixode}


\DeclareMathOperator{\ord}{ord}
\DeclareMathOperator{\orb}{orb}
\DeclareMathOperator{\stab}{stab}
\DeclareMathOperator{\Stab}{stab}
\DeclareMathOperator{\ppcm}{ppcm}
\DeclareMathOperator{\conj}{Conj}
\DeclareMathOperator{\End}{End}
\DeclareMathOperator{\rot}{rot}
\DeclareMathOperator{\trs}{trace}
\DeclareMathOperator{\Ind}{Ind}
\DeclareMathOperator{\mat}{Mat}
\DeclareMathOperator{\id}{Id}
\DeclareMathOperator{\vect}{vect}
\DeclareMathOperator{\img}{img}
\DeclareMathOperator{\cov}{Cov}
\DeclareMathOperator{\dist}{dist}
\DeclareMathOperator{\irr}{Irr}
\DeclareMathOperator{\image}{Im}
\DeclareMathOperator{\pd}{\partial}
\DeclareMathOperator{\epi}{epi}
\DeclareMathOperator{\Argmin}{Argmin}
\DeclareMathOperator{\dom}{dom}
\DeclareMathOperator{\proj}{proj}
\DeclareMathOperator{\ctg}{ctg}
\DeclareMathOperator{\supp}{supp}
\DeclareMathOperator{\argmin}{argmin}
\DeclareMathOperator{\mult}{mult}
\DeclareMathOperator{\ch}{ch}
\DeclareMathOperator{\sh}{sh}
\DeclareMathOperator{\rang}{rang}
\DeclareMathOperator{\diam}{diam}
\DeclareMathOperator{\Epigraphe}{Epigraphe}




\usepackage{xcolor}
\everymath{\color{blue}}
%\everymath{\color[rgb]{0,1,1}}
%\pagecolor[rgb]{0,0,0.5}


\newcommand*{\pdtest}[3][]{\ensuremath{\frac{\partial^{#1} #2}{\partial #3}}}

\newcommand*{\deffunc}[6][]{\ensuremath{
\begin{array}{rcl}
#2 : #3 &\rightarrow& #4\\
#5 &\mapsto& #6
\end{array}
}}

\newcommand{\eqcolon}{\mathrel{\resizebox{\widthof{$\mathord{=}$}}{\height}{ $\!\!=\!\!\resizebox{1.2\width}{0.8\height}{\raisebox{0.23ex}{$\mathop{:}$}}\!\!$ }}}
\newcommand{\coloneq}{\mathrel{\resizebox{\widthof{$\mathord{=}$}}{\height}{ $\!\!\resizebox{1.2\width}{0.8\height}{\raisebox{0.23ex}{$\mathop{:}$}}\!\!=\!\!$ }}}
\newcommand{\eqcolonl}{\ensuremath{\mathrel{=\!\!\mathop{:}}}}
\newcommand{\coloneql}{\ensuremath{\mathrel{\mathop{:} \!\! =}}}
\newcommand{\vc}[1]{% inline column vector
  \left(\begin{smallmatrix}#1\end{smallmatrix}\right)%
}
\newcommand{\vr}[1]{% inline row vector
  \begin{smallmatrix}(\,#1\,)\end{smallmatrix}%
}
\makeatletter
\newcommand*{\defeq}{\ =\mathrel{\rlap{%
                     \raisebox{0.3ex}{$\m@th\cdot$}}%
                     \raisebox{-0.3ex}{$\m@th\cdot$}}%
                     }
\makeatother

\newcommand{\mathcircle}[1]{% inline row vector
 \overset{\circ}{#1}
}
\newcommand{\ulim}{% low limit
 \underline{\lim}
}
\newcommand{\ssi}{% iff
\iff
}
\newcommand{\ps}[2]{
\expval{#1 | #2}
}
\newcommand{\df}[1]{
\mqty{#1}
}
\newcommand{\n}[1]{
\norm{#1}
}
\newcommand{\sys}[1]{
\left\{\smqty{#1}\right.
}


\newcommand{\eqdef}{\ensuremath{\overset{\text{def}}=}}


\def\Circlearrowright{\ensuremath{%
  \rotatebox[origin=c]{230}{$\circlearrowright$}}}

\newcommand\ct[1]{\text{\rmfamily\upshape #1}}
\newcommand\question[1]{ {\color{red} ...!? \small #1}}
\newcommand\caz[1]{\left\{\begin{array} #1 \end{array}\right.}
\newcommand\const{\text{\rmfamily\upshape const}}
\newcommand\toP{ \overset{\pro}{\to}}
\newcommand\toPP{ \overset{\text{PP}}{\to}}
\newcommand{\oeq}{\mathrel{\text{\textcircled{$=$}}}}





\usepackage{xcolor}
% \usepackage[normalem]{ulem}
\usepackage{lipsum}
\makeatletter
% \newcommand\colorwave[1][blue]{\bgroup \markoverwith{\lower3.5\p@\hbox{\sixly \textcolor{#1}{\char58}}}\ULon}
%\font\sixly=lasy6 % does not re-load if already loaded, so no memory problem.

\newmdtheoremenv[
linewidth= 1pt,linecolor= blue,%
leftmargin=20,rightmargin=20,innertopmargin=0pt, innerrightmargin=40,%
tikzsetting = { draw=lightgray, line width = 0.3pt,dashed,%
dash pattern = on 15pt off 3pt},%
splittopskip=\topskip,skipbelow=\baselineskip,%
skipabove=\baselineskip,ntheorem,roundcorner=0pt,
% backgroundcolor=pagebg,font=\color{orange}\sffamily, fontcolor=white
]{examplebox}{Exemple}[section]



\newcommand\R{\mathbb{R}}
\newcommand\Z{\mathbb{Z}}
\newcommand\N{\mathbb{N}}
\newcommand\E{\mathbb{E}}
\newcommand\F{\mathcal{F}}
\newcommand\cH{\mathcal{H}}
\newcommand\V{\mathbb{V}}
\newcommand\dmo{ ^{-1} }
\newcommand\kapa{\kappa}
\newcommand\im{Im}
\newcommand\hs{\mathcal{H}}





\usepackage{soul}

\makeatletter
\newcommand*{\whiten}[1]{\llap{\textcolor{white}{{\the\SOUL@token}}\hspace{#1pt}}}
\DeclareRobustCommand*\myul{%
    \def\SOUL@everyspace{\underline{\space}\kern\z@}%
    \def\SOUL@everytoken{%
     \setbox0=\hbox{\the\SOUL@token}%
     \ifdim\dp0>\z@
        \raisebox{\dp0}{\underline{\phantom{\the\SOUL@token}}}%
        \whiten{1}\whiten{0}%
        \whiten{-1}\whiten{-2}%
        \llap{\the\SOUL@token}%
     \else
        \underline{\the\SOUL@token}%
     \fi}%
\SOUL@}
\makeatother

\newcommand*{\demp}{\fontfamily{lmtt}\selectfont}

\DeclareTextFontCommand{\textdemp}{\demp}

\begin{document}

\ifcomment
Multiline
comment
\fi
\ifcomment
\myul{Typesetting test}
% \color[rgb]{1,1,1}
$∑_i^n≠ 60º±∞π∆¬≈√j∫h≤≥µ$

$\CR \R\pro\ind\pro\gS\pro
\mqty[a&b\\c&d]$
$\pro\mathbb{P}$
$\dd{x}$

  \[
    \alpha(x)=\left\{
                \begin{array}{ll}
                  x\\
                  \frac{1}{1+e^{-kx}}\\
                  \frac{e^x-e^{-x}}{e^x+e^{-x}}
                \end{array}
              \right.
  \]

  $\expval{x}$
  
  $\chi_\rho(ghg\dmo)=\Tr(\rho_{ghg\dmo})=\Tr(\rho_g\circ\rho_h\circ\rho\dmo_g)=\Tr(\rho_h)\overset{\mbox{\scalebox{0.5}{$\Tr(AB)=\Tr(BA)$}}}{=}\chi_\rho(h)$
  	$\mathop{\oplus}_{\substack{x\in X}}$

$\mat(\rho_g)=(a_{ij}(g))_{\scriptsize \substack{1\leq i\leq d \\ 1\leq j\leq d}}$ et $\mat(\rho'_g)=(a'_{ij}(g))_{\scriptsize \substack{1\leq i'\leq d' \\ 1\leq j'\leq d'}}$



\[\int_a^b{\mathbb{R}^2}g(u, v)\dd{P_{XY}}(u, v)=\iint g(u,v) f_{XY}(u, v)\dd \lambda(u) \dd \lambda(v)\]
$$\lim_{x\to\infty} f(x)$$	
$$\iiiint_V \mu(t,u,v,w) \,dt\,du\,dv\,dw$$
$$\sum_{n=1}^{\infty} 2^{-n} = 1$$	
\begin{definition}
	Si $X$ et $Y$ sont 2 v.a. ou definit la \textsc{Covariance} entre $X$ et $Y$ comme
	$\cov(X,Y)\overset{\text{def}}{=}\E\left[(X-\E(X))(Y-\E(Y))\right]=\E(XY)-\E(X)\E(Y)$.
\end{definition}
\fi
\pagebreak

% \tableofcontents

% insert your code here
%\input{./algebra/main.tex}
%\input{./geometrie-differentielle/main.tex}
%\input{./probabilite/main.tex}
%\input{./analyse-fonctionnelle/main.tex}
% \input{./Analyse-convexe-et-dualite-en-optimisation/main.tex}
%\input{./tikz/main.tex}
%\input{./Theorie-du-distributions/main.tex}
%\input{./optimisation/mine.tex}
 \input{./modelisation/main.tex}

% yves.aubry@univ-tln.fr : algebra

\end{document}

%% !TEX encoding = UTF-8 Unicode
% !TEX TS-program = xelatex

\documentclass[french]{report}

%\usepackage[utf8]{inputenc}
%\usepackage[T1]{fontenc}
\usepackage{babel}


\newif\ifcomment
%\commenttrue # Show comments

\usepackage{physics}
\usepackage{amssymb}


\usepackage{amsthm}
% \usepackage{thmtools}
\usepackage{mathtools}
\usepackage{amsfonts}

\usepackage{color}

\usepackage{tikz}

\usepackage{geometry}
\geometry{a5paper, margin=0.1in, right=1cm}

\usepackage{dsfont}

\usepackage{graphicx}
\graphicspath{ {images/} }

\usepackage{faktor}

\usepackage{IEEEtrantools}
\usepackage{enumerate}   
\usepackage[PostScript=dvips]{"/Users/aware/Documents/Courses/diagrams"}


\newtheorem{theorem}{Théorème}[section]
\renewcommand{\thetheorem}{\arabic{theorem}}
\newtheorem{lemme}{Lemme}[section]
\renewcommand{\thelemme}{\arabic{lemme}}
\newtheorem{proposition}{Proposition}[section]
\renewcommand{\theproposition}{\arabic{proposition}}
\newtheorem{notations}{Notations}[section]
\newtheorem{problem}{Problème}[section]
\newtheorem{corollary}{Corollaire}[theorem]
\renewcommand{\thecorollary}{\arabic{corollary}}
\newtheorem{property}{Propriété}[section]
\newtheorem{objective}{Objectif}[section]

\theoremstyle{definition}
\newtheorem{definition}{Définition}[section]
\renewcommand{\thedefinition}{\arabic{definition}}
\newtheorem{exercise}{Exercice}[chapter]
\renewcommand{\theexercise}{\arabic{exercise}}
\newtheorem{example}{Exemple}[chapter]
\renewcommand{\theexample}{\arabic{example}}
\newtheorem*{solution}{Solution}
\newtheorem*{application}{Application}
\newtheorem*{notation}{Notation}
\newtheorem*{vocabulary}{Vocabulaire}
\newtheorem*{properties}{Propriétés}



\theoremstyle{remark}
\newtheorem*{remark}{Remarque}
\newtheorem*{rappel}{Rappel}


\usepackage{etoolbox}
\AtBeginEnvironment{exercise}{\small}
\AtBeginEnvironment{example}{\small}

\usepackage{cases}
\usepackage[red]{mypack}

\usepackage[framemethod=TikZ]{mdframed}

\definecolor{bg}{rgb}{0.4,0.25,0.95}
\definecolor{pagebg}{rgb}{0,0,0.5}
\surroundwithmdframed[
   topline=false,
   rightline=false,
   bottomline=false,
   leftmargin=\parindent,
   skipabove=8pt,
   skipbelow=8pt,
   linecolor=blue,
   innerbottommargin=10pt,
   % backgroundcolor=bg,font=\color{orange}\sffamily, fontcolor=white
]{definition}

\usepackage{empheq}
\usepackage[most]{tcolorbox}

\newtcbox{\mymath}[1][]{%
    nobeforeafter, math upper, tcbox raise base,
    enhanced, colframe=blue!30!black,
    colback=red!10, boxrule=1pt,
    #1}

\usepackage{unixode}


\DeclareMathOperator{\ord}{ord}
\DeclareMathOperator{\orb}{orb}
\DeclareMathOperator{\stab}{stab}
\DeclareMathOperator{\Stab}{stab}
\DeclareMathOperator{\ppcm}{ppcm}
\DeclareMathOperator{\conj}{Conj}
\DeclareMathOperator{\End}{End}
\DeclareMathOperator{\rot}{rot}
\DeclareMathOperator{\trs}{trace}
\DeclareMathOperator{\Ind}{Ind}
\DeclareMathOperator{\mat}{Mat}
\DeclareMathOperator{\id}{Id}
\DeclareMathOperator{\vect}{vect}
\DeclareMathOperator{\img}{img}
\DeclareMathOperator{\cov}{Cov}
\DeclareMathOperator{\dist}{dist}
\DeclareMathOperator{\irr}{Irr}
\DeclareMathOperator{\image}{Im}
\DeclareMathOperator{\pd}{\partial}
\DeclareMathOperator{\epi}{epi}
\DeclareMathOperator{\Argmin}{Argmin}
\DeclareMathOperator{\dom}{dom}
\DeclareMathOperator{\proj}{proj}
\DeclareMathOperator{\ctg}{ctg}
\DeclareMathOperator{\supp}{supp}
\DeclareMathOperator{\argmin}{argmin}
\DeclareMathOperator{\mult}{mult}
\DeclareMathOperator{\ch}{ch}
\DeclareMathOperator{\sh}{sh}
\DeclareMathOperator{\rang}{rang}
\DeclareMathOperator{\diam}{diam}
\DeclareMathOperator{\Epigraphe}{Epigraphe}




\usepackage{xcolor}
\everymath{\color{blue}}
%\everymath{\color[rgb]{0,1,1}}
%\pagecolor[rgb]{0,0,0.5}


\newcommand*{\pdtest}[3][]{\ensuremath{\frac{\partial^{#1} #2}{\partial #3}}}

\newcommand*{\deffunc}[6][]{\ensuremath{
\begin{array}{rcl}
#2 : #3 &\rightarrow& #4\\
#5 &\mapsto& #6
\end{array}
}}

\newcommand{\eqcolon}{\mathrel{\resizebox{\widthof{$\mathord{=}$}}{\height}{ $\!\!=\!\!\resizebox{1.2\width}{0.8\height}{\raisebox{0.23ex}{$\mathop{:}$}}\!\!$ }}}
\newcommand{\coloneq}{\mathrel{\resizebox{\widthof{$\mathord{=}$}}{\height}{ $\!\!\resizebox{1.2\width}{0.8\height}{\raisebox{0.23ex}{$\mathop{:}$}}\!\!=\!\!$ }}}
\newcommand{\eqcolonl}{\ensuremath{\mathrel{=\!\!\mathop{:}}}}
\newcommand{\coloneql}{\ensuremath{\mathrel{\mathop{:} \!\! =}}}
\newcommand{\vc}[1]{% inline column vector
  \left(\begin{smallmatrix}#1\end{smallmatrix}\right)%
}
\newcommand{\vr}[1]{% inline row vector
  \begin{smallmatrix}(\,#1\,)\end{smallmatrix}%
}
\makeatletter
\newcommand*{\defeq}{\ =\mathrel{\rlap{%
                     \raisebox{0.3ex}{$\m@th\cdot$}}%
                     \raisebox{-0.3ex}{$\m@th\cdot$}}%
                     }
\makeatother

\newcommand{\mathcircle}[1]{% inline row vector
 \overset{\circ}{#1}
}
\newcommand{\ulim}{% low limit
 \underline{\lim}
}
\newcommand{\ssi}{% iff
\iff
}
\newcommand{\ps}[2]{
\expval{#1 | #2}
}
\newcommand{\df}[1]{
\mqty{#1}
}
\newcommand{\n}[1]{
\norm{#1}
}
\newcommand{\sys}[1]{
\left\{\smqty{#1}\right.
}


\newcommand{\eqdef}{\ensuremath{\overset{\text{def}}=}}


\def\Circlearrowright{\ensuremath{%
  \rotatebox[origin=c]{230}{$\circlearrowright$}}}

\newcommand\ct[1]{\text{\rmfamily\upshape #1}}
\newcommand\question[1]{ {\color{red} ...!? \small #1}}
\newcommand\caz[1]{\left\{\begin{array} #1 \end{array}\right.}
\newcommand\const{\text{\rmfamily\upshape const}}
\newcommand\toP{ \overset{\pro}{\to}}
\newcommand\toPP{ \overset{\text{PP}}{\to}}
\newcommand{\oeq}{\mathrel{\text{\textcircled{$=$}}}}





\usepackage{xcolor}
% \usepackage[normalem]{ulem}
\usepackage{lipsum}
\makeatletter
% \newcommand\colorwave[1][blue]{\bgroup \markoverwith{\lower3.5\p@\hbox{\sixly \textcolor{#1}{\char58}}}\ULon}
%\font\sixly=lasy6 % does not re-load if already loaded, so no memory problem.

\newmdtheoremenv[
linewidth= 1pt,linecolor= blue,%
leftmargin=20,rightmargin=20,innertopmargin=0pt, innerrightmargin=40,%
tikzsetting = { draw=lightgray, line width = 0.3pt,dashed,%
dash pattern = on 15pt off 3pt},%
splittopskip=\topskip,skipbelow=\baselineskip,%
skipabove=\baselineskip,ntheorem,roundcorner=0pt,
% backgroundcolor=pagebg,font=\color{orange}\sffamily, fontcolor=white
]{examplebox}{Exemple}[section]



\newcommand\R{\mathbb{R}}
\newcommand\Z{\mathbb{Z}}
\newcommand\N{\mathbb{N}}
\newcommand\E{\mathbb{E}}
\newcommand\F{\mathcal{F}}
\newcommand\cH{\mathcal{H}}
\newcommand\V{\mathbb{V}}
\newcommand\dmo{ ^{-1} }
\newcommand\kapa{\kappa}
\newcommand\im{Im}
\newcommand\hs{\mathcal{H}}





\usepackage{soul}

\makeatletter
\newcommand*{\whiten}[1]{\llap{\textcolor{white}{{\the\SOUL@token}}\hspace{#1pt}}}
\DeclareRobustCommand*\myul{%
    \def\SOUL@everyspace{\underline{\space}\kern\z@}%
    \def\SOUL@everytoken{%
     \setbox0=\hbox{\the\SOUL@token}%
     \ifdim\dp0>\z@
        \raisebox{\dp0}{\underline{\phantom{\the\SOUL@token}}}%
        \whiten{1}\whiten{0}%
        \whiten{-1}\whiten{-2}%
        \llap{\the\SOUL@token}%
     \else
        \underline{\the\SOUL@token}%
     \fi}%
\SOUL@}
\makeatother

\newcommand*{\demp}{\fontfamily{lmtt}\selectfont}

\DeclareTextFontCommand{\textdemp}{\demp}

\begin{document}

\ifcomment
Multiline
comment
\fi
\ifcomment
\myul{Typesetting test}
% \color[rgb]{1,1,1}
$∑_i^n≠ 60º±∞π∆¬≈√j∫h≤≥µ$

$\CR \R\pro\ind\pro\gS\pro
\mqty[a&b\\c&d]$
$\pro\mathbb{P}$
$\dd{x}$

  \[
    \alpha(x)=\left\{
                \begin{array}{ll}
                  x\\
                  \frac{1}{1+e^{-kx}}\\
                  \frac{e^x-e^{-x}}{e^x+e^{-x}}
                \end{array}
              \right.
  \]

  $\expval{x}$
  
  $\chi_\rho(ghg\dmo)=\Tr(\rho_{ghg\dmo})=\Tr(\rho_g\circ\rho_h\circ\rho\dmo_g)=\Tr(\rho_h)\overset{\mbox{\scalebox{0.5}{$\Tr(AB)=\Tr(BA)$}}}{=}\chi_\rho(h)$
  	$\mathop{\oplus}_{\substack{x\in X}}$

$\mat(\rho_g)=(a_{ij}(g))_{\scriptsize \substack{1\leq i\leq d \\ 1\leq j\leq d}}$ et $\mat(\rho'_g)=(a'_{ij}(g))_{\scriptsize \substack{1\leq i'\leq d' \\ 1\leq j'\leq d'}}$



\[\int_a^b{\mathbb{R}^2}g(u, v)\dd{P_{XY}}(u, v)=\iint g(u,v) f_{XY}(u, v)\dd \lambda(u) \dd \lambda(v)\]
$$\lim_{x\to\infty} f(x)$$	
$$\iiiint_V \mu(t,u,v,w) \,dt\,du\,dv\,dw$$
$$\sum_{n=1}^{\infty} 2^{-n} = 1$$	
\begin{definition}
	Si $X$ et $Y$ sont 2 v.a. ou definit la \textsc{Covariance} entre $X$ et $Y$ comme
	$\cov(X,Y)\overset{\text{def}}{=}\E\left[(X-\E(X))(Y-\E(Y))\right]=\E(XY)-\E(X)\E(Y)$.
\end{definition}
\fi
\pagebreak

% \tableofcontents

% insert your code here
%\input{./algebra/main.tex}
%\input{./geometrie-differentielle/main.tex}
%\input{./probabilite/main.tex}
%\input{./analyse-fonctionnelle/main.tex}
% \input{./Analyse-convexe-et-dualite-en-optimisation/main.tex}
%\input{./tikz/main.tex}
%\input{./Theorie-du-distributions/main.tex}
%\input{./optimisation/mine.tex}
 \input{./modelisation/main.tex}

% yves.aubry@univ-tln.fr : algebra

\end{document}

%% !TEX encoding = UTF-8 Unicode
% !TEX TS-program = xelatex

\documentclass[french]{report}

%\usepackage[utf8]{inputenc}
%\usepackage[T1]{fontenc}
\usepackage{babel}


\newif\ifcomment
%\commenttrue # Show comments

\usepackage{physics}
\usepackage{amssymb}


\usepackage{amsthm}
% \usepackage{thmtools}
\usepackage{mathtools}
\usepackage{amsfonts}

\usepackage{color}

\usepackage{tikz}

\usepackage{geometry}
\geometry{a5paper, margin=0.1in, right=1cm}

\usepackage{dsfont}

\usepackage{graphicx}
\graphicspath{ {images/} }

\usepackage{faktor}

\usepackage{IEEEtrantools}
\usepackage{enumerate}   
\usepackage[PostScript=dvips]{"/Users/aware/Documents/Courses/diagrams"}


\newtheorem{theorem}{Théorème}[section]
\renewcommand{\thetheorem}{\arabic{theorem}}
\newtheorem{lemme}{Lemme}[section]
\renewcommand{\thelemme}{\arabic{lemme}}
\newtheorem{proposition}{Proposition}[section]
\renewcommand{\theproposition}{\arabic{proposition}}
\newtheorem{notations}{Notations}[section]
\newtheorem{problem}{Problème}[section]
\newtheorem{corollary}{Corollaire}[theorem]
\renewcommand{\thecorollary}{\arabic{corollary}}
\newtheorem{property}{Propriété}[section]
\newtheorem{objective}{Objectif}[section]

\theoremstyle{definition}
\newtheorem{definition}{Définition}[section]
\renewcommand{\thedefinition}{\arabic{definition}}
\newtheorem{exercise}{Exercice}[chapter]
\renewcommand{\theexercise}{\arabic{exercise}}
\newtheorem{example}{Exemple}[chapter]
\renewcommand{\theexample}{\arabic{example}}
\newtheorem*{solution}{Solution}
\newtheorem*{application}{Application}
\newtheorem*{notation}{Notation}
\newtheorem*{vocabulary}{Vocabulaire}
\newtheorem*{properties}{Propriétés}



\theoremstyle{remark}
\newtheorem*{remark}{Remarque}
\newtheorem*{rappel}{Rappel}


\usepackage{etoolbox}
\AtBeginEnvironment{exercise}{\small}
\AtBeginEnvironment{example}{\small}

\usepackage{cases}
\usepackage[red]{mypack}

\usepackage[framemethod=TikZ]{mdframed}

\definecolor{bg}{rgb}{0.4,0.25,0.95}
\definecolor{pagebg}{rgb}{0,0,0.5}
\surroundwithmdframed[
   topline=false,
   rightline=false,
   bottomline=false,
   leftmargin=\parindent,
   skipabove=8pt,
   skipbelow=8pt,
   linecolor=blue,
   innerbottommargin=10pt,
   % backgroundcolor=bg,font=\color{orange}\sffamily, fontcolor=white
]{definition}

\usepackage{empheq}
\usepackage[most]{tcolorbox}

\newtcbox{\mymath}[1][]{%
    nobeforeafter, math upper, tcbox raise base,
    enhanced, colframe=blue!30!black,
    colback=red!10, boxrule=1pt,
    #1}

\usepackage{unixode}


\DeclareMathOperator{\ord}{ord}
\DeclareMathOperator{\orb}{orb}
\DeclareMathOperator{\stab}{stab}
\DeclareMathOperator{\Stab}{stab}
\DeclareMathOperator{\ppcm}{ppcm}
\DeclareMathOperator{\conj}{Conj}
\DeclareMathOperator{\End}{End}
\DeclareMathOperator{\rot}{rot}
\DeclareMathOperator{\trs}{trace}
\DeclareMathOperator{\Ind}{Ind}
\DeclareMathOperator{\mat}{Mat}
\DeclareMathOperator{\id}{Id}
\DeclareMathOperator{\vect}{vect}
\DeclareMathOperator{\img}{img}
\DeclareMathOperator{\cov}{Cov}
\DeclareMathOperator{\dist}{dist}
\DeclareMathOperator{\irr}{Irr}
\DeclareMathOperator{\image}{Im}
\DeclareMathOperator{\pd}{\partial}
\DeclareMathOperator{\epi}{epi}
\DeclareMathOperator{\Argmin}{Argmin}
\DeclareMathOperator{\dom}{dom}
\DeclareMathOperator{\proj}{proj}
\DeclareMathOperator{\ctg}{ctg}
\DeclareMathOperator{\supp}{supp}
\DeclareMathOperator{\argmin}{argmin}
\DeclareMathOperator{\mult}{mult}
\DeclareMathOperator{\ch}{ch}
\DeclareMathOperator{\sh}{sh}
\DeclareMathOperator{\rang}{rang}
\DeclareMathOperator{\diam}{diam}
\DeclareMathOperator{\Epigraphe}{Epigraphe}




\usepackage{xcolor}
\everymath{\color{blue}}
%\everymath{\color[rgb]{0,1,1}}
%\pagecolor[rgb]{0,0,0.5}


\newcommand*{\pdtest}[3][]{\ensuremath{\frac{\partial^{#1} #2}{\partial #3}}}

\newcommand*{\deffunc}[6][]{\ensuremath{
\begin{array}{rcl}
#2 : #3 &\rightarrow& #4\\
#5 &\mapsto& #6
\end{array}
}}

\newcommand{\eqcolon}{\mathrel{\resizebox{\widthof{$\mathord{=}$}}{\height}{ $\!\!=\!\!\resizebox{1.2\width}{0.8\height}{\raisebox{0.23ex}{$\mathop{:}$}}\!\!$ }}}
\newcommand{\coloneq}{\mathrel{\resizebox{\widthof{$\mathord{=}$}}{\height}{ $\!\!\resizebox{1.2\width}{0.8\height}{\raisebox{0.23ex}{$\mathop{:}$}}\!\!=\!\!$ }}}
\newcommand{\eqcolonl}{\ensuremath{\mathrel{=\!\!\mathop{:}}}}
\newcommand{\coloneql}{\ensuremath{\mathrel{\mathop{:} \!\! =}}}
\newcommand{\vc}[1]{% inline column vector
  \left(\begin{smallmatrix}#1\end{smallmatrix}\right)%
}
\newcommand{\vr}[1]{% inline row vector
  \begin{smallmatrix}(\,#1\,)\end{smallmatrix}%
}
\makeatletter
\newcommand*{\defeq}{\ =\mathrel{\rlap{%
                     \raisebox{0.3ex}{$\m@th\cdot$}}%
                     \raisebox{-0.3ex}{$\m@th\cdot$}}%
                     }
\makeatother

\newcommand{\mathcircle}[1]{% inline row vector
 \overset{\circ}{#1}
}
\newcommand{\ulim}{% low limit
 \underline{\lim}
}
\newcommand{\ssi}{% iff
\iff
}
\newcommand{\ps}[2]{
\expval{#1 | #2}
}
\newcommand{\df}[1]{
\mqty{#1}
}
\newcommand{\n}[1]{
\norm{#1}
}
\newcommand{\sys}[1]{
\left\{\smqty{#1}\right.
}


\newcommand{\eqdef}{\ensuremath{\overset{\text{def}}=}}


\def\Circlearrowright{\ensuremath{%
  \rotatebox[origin=c]{230}{$\circlearrowright$}}}

\newcommand\ct[1]{\text{\rmfamily\upshape #1}}
\newcommand\question[1]{ {\color{red} ...!? \small #1}}
\newcommand\caz[1]{\left\{\begin{array} #1 \end{array}\right.}
\newcommand\const{\text{\rmfamily\upshape const}}
\newcommand\toP{ \overset{\pro}{\to}}
\newcommand\toPP{ \overset{\text{PP}}{\to}}
\newcommand{\oeq}{\mathrel{\text{\textcircled{$=$}}}}





\usepackage{xcolor}
% \usepackage[normalem]{ulem}
\usepackage{lipsum}
\makeatletter
% \newcommand\colorwave[1][blue]{\bgroup \markoverwith{\lower3.5\p@\hbox{\sixly \textcolor{#1}{\char58}}}\ULon}
%\font\sixly=lasy6 % does not re-load if already loaded, so no memory problem.

\newmdtheoremenv[
linewidth= 1pt,linecolor= blue,%
leftmargin=20,rightmargin=20,innertopmargin=0pt, innerrightmargin=40,%
tikzsetting = { draw=lightgray, line width = 0.3pt,dashed,%
dash pattern = on 15pt off 3pt},%
splittopskip=\topskip,skipbelow=\baselineskip,%
skipabove=\baselineskip,ntheorem,roundcorner=0pt,
% backgroundcolor=pagebg,font=\color{orange}\sffamily, fontcolor=white
]{examplebox}{Exemple}[section]



\newcommand\R{\mathbb{R}}
\newcommand\Z{\mathbb{Z}}
\newcommand\N{\mathbb{N}}
\newcommand\E{\mathbb{E}}
\newcommand\F{\mathcal{F}}
\newcommand\cH{\mathcal{H}}
\newcommand\V{\mathbb{V}}
\newcommand\dmo{ ^{-1} }
\newcommand\kapa{\kappa}
\newcommand\im{Im}
\newcommand\hs{\mathcal{H}}





\usepackage{soul}

\makeatletter
\newcommand*{\whiten}[1]{\llap{\textcolor{white}{{\the\SOUL@token}}\hspace{#1pt}}}
\DeclareRobustCommand*\myul{%
    \def\SOUL@everyspace{\underline{\space}\kern\z@}%
    \def\SOUL@everytoken{%
     \setbox0=\hbox{\the\SOUL@token}%
     \ifdim\dp0>\z@
        \raisebox{\dp0}{\underline{\phantom{\the\SOUL@token}}}%
        \whiten{1}\whiten{0}%
        \whiten{-1}\whiten{-2}%
        \llap{\the\SOUL@token}%
     \else
        \underline{\the\SOUL@token}%
     \fi}%
\SOUL@}
\makeatother

\newcommand*{\demp}{\fontfamily{lmtt}\selectfont}

\DeclareTextFontCommand{\textdemp}{\demp}

\begin{document}

\ifcomment
Multiline
comment
\fi
\ifcomment
\myul{Typesetting test}
% \color[rgb]{1,1,1}
$∑_i^n≠ 60º±∞π∆¬≈√j∫h≤≥µ$

$\CR \R\pro\ind\pro\gS\pro
\mqty[a&b\\c&d]$
$\pro\mathbb{P}$
$\dd{x}$

  \[
    \alpha(x)=\left\{
                \begin{array}{ll}
                  x\\
                  \frac{1}{1+e^{-kx}}\\
                  \frac{e^x-e^{-x}}{e^x+e^{-x}}
                \end{array}
              \right.
  \]

  $\expval{x}$
  
  $\chi_\rho(ghg\dmo)=\Tr(\rho_{ghg\dmo})=\Tr(\rho_g\circ\rho_h\circ\rho\dmo_g)=\Tr(\rho_h)\overset{\mbox{\scalebox{0.5}{$\Tr(AB)=\Tr(BA)$}}}{=}\chi_\rho(h)$
  	$\mathop{\oplus}_{\substack{x\in X}}$

$\mat(\rho_g)=(a_{ij}(g))_{\scriptsize \substack{1\leq i\leq d \\ 1\leq j\leq d}}$ et $\mat(\rho'_g)=(a'_{ij}(g))_{\scriptsize \substack{1\leq i'\leq d' \\ 1\leq j'\leq d'}}$



\[\int_a^b{\mathbb{R}^2}g(u, v)\dd{P_{XY}}(u, v)=\iint g(u,v) f_{XY}(u, v)\dd \lambda(u) \dd \lambda(v)\]
$$\lim_{x\to\infty} f(x)$$	
$$\iiiint_V \mu(t,u,v,w) \,dt\,du\,dv\,dw$$
$$\sum_{n=1}^{\infty} 2^{-n} = 1$$	
\begin{definition}
	Si $X$ et $Y$ sont 2 v.a. ou definit la \textsc{Covariance} entre $X$ et $Y$ comme
	$\cov(X,Y)\overset{\text{def}}{=}\E\left[(X-\E(X))(Y-\E(Y))\right]=\E(XY)-\E(X)\E(Y)$.
\end{definition}
\fi
\pagebreak

% \tableofcontents

% insert your code here
%\input{./algebra/main.tex}
%\input{./geometrie-differentielle/main.tex}
%\input{./probabilite/main.tex}
%\input{./analyse-fonctionnelle/main.tex}
% \input{./Analyse-convexe-et-dualite-en-optimisation/main.tex}
%\input{./tikz/main.tex}
%\input{./Theorie-du-distributions/main.tex}
%\input{./optimisation/mine.tex}
 \input{./modelisation/main.tex}

% yves.aubry@univ-tln.fr : algebra

\end{document}

%% !TEX encoding = UTF-8 Unicode
% !TEX TS-program = xelatex

\documentclass[french]{report}

%\usepackage[utf8]{inputenc}
%\usepackage[T1]{fontenc}
\usepackage{babel}


\newif\ifcomment
%\commenttrue # Show comments

\usepackage{physics}
\usepackage{amssymb}


\usepackage{amsthm}
% \usepackage{thmtools}
\usepackage{mathtools}
\usepackage{amsfonts}

\usepackage{color}

\usepackage{tikz}

\usepackage{geometry}
\geometry{a5paper, margin=0.1in, right=1cm}

\usepackage{dsfont}

\usepackage{graphicx}
\graphicspath{ {images/} }

\usepackage{faktor}

\usepackage{IEEEtrantools}
\usepackage{enumerate}   
\usepackage[PostScript=dvips]{"/Users/aware/Documents/Courses/diagrams"}


\newtheorem{theorem}{Théorème}[section]
\renewcommand{\thetheorem}{\arabic{theorem}}
\newtheorem{lemme}{Lemme}[section]
\renewcommand{\thelemme}{\arabic{lemme}}
\newtheorem{proposition}{Proposition}[section]
\renewcommand{\theproposition}{\arabic{proposition}}
\newtheorem{notations}{Notations}[section]
\newtheorem{problem}{Problème}[section]
\newtheorem{corollary}{Corollaire}[theorem]
\renewcommand{\thecorollary}{\arabic{corollary}}
\newtheorem{property}{Propriété}[section]
\newtheorem{objective}{Objectif}[section]

\theoremstyle{definition}
\newtheorem{definition}{Définition}[section]
\renewcommand{\thedefinition}{\arabic{definition}}
\newtheorem{exercise}{Exercice}[chapter]
\renewcommand{\theexercise}{\arabic{exercise}}
\newtheorem{example}{Exemple}[chapter]
\renewcommand{\theexample}{\arabic{example}}
\newtheorem*{solution}{Solution}
\newtheorem*{application}{Application}
\newtheorem*{notation}{Notation}
\newtheorem*{vocabulary}{Vocabulaire}
\newtheorem*{properties}{Propriétés}



\theoremstyle{remark}
\newtheorem*{remark}{Remarque}
\newtheorem*{rappel}{Rappel}


\usepackage{etoolbox}
\AtBeginEnvironment{exercise}{\small}
\AtBeginEnvironment{example}{\small}

\usepackage{cases}
\usepackage[red]{mypack}

\usepackage[framemethod=TikZ]{mdframed}

\definecolor{bg}{rgb}{0.4,0.25,0.95}
\definecolor{pagebg}{rgb}{0,0,0.5}
\surroundwithmdframed[
   topline=false,
   rightline=false,
   bottomline=false,
   leftmargin=\parindent,
   skipabove=8pt,
   skipbelow=8pt,
   linecolor=blue,
   innerbottommargin=10pt,
   % backgroundcolor=bg,font=\color{orange}\sffamily, fontcolor=white
]{definition}

\usepackage{empheq}
\usepackage[most]{tcolorbox}

\newtcbox{\mymath}[1][]{%
    nobeforeafter, math upper, tcbox raise base,
    enhanced, colframe=blue!30!black,
    colback=red!10, boxrule=1pt,
    #1}

\usepackage{unixode}


\DeclareMathOperator{\ord}{ord}
\DeclareMathOperator{\orb}{orb}
\DeclareMathOperator{\stab}{stab}
\DeclareMathOperator{\Stab}{stab}
\DeclareMathOperator{\ppcm}{ppcm}
\DeclareMathOperator{\conj}{Conj}
\DeclareMathOperator{\End}{End}
\DeclareMathOperator{\rot}{rot}
\DeclareMathOperator{\trs}{trace}
\DeclareMathOperator{\Ind}{Ind}
\DeclareMathOperator{\mat}{Mat}
\DeclareMathOperator{\id}{Id}
\DeclareMathOperator{\vect}{vect}
\DeclareMathOperator{\img}{img}
\DeclareMathOperator{\cov}{Cov}
\DeclareMathOperator{\dist}{dist}
\DeclareMathOperator{\irr}{Irr}
\DeclareMathOperator{\image}{Im}
\DeclareMathOperator{\pd}{\partial}
\DeclareMathOperator{\epi}{epi}
\DeclareMathOperator{\Argmin}{Argmin}
\DeclareMathOperator{\dom}{dom}
\DeclareMathOperator{\proj}{proj}
\DeclareMathOperator{\ctg}{ctg}
\DeclareMathOperator{\supp}{supp}
\DeclareMathOperator{\argmin}{argmin}
\DeclareMathOperator{\mult}{mult}
\DeclareMathOperator{\ch}{ch}
\DeclareMathOperator{\sh}{sh}
\DeclareMathOperator{\rang}{rang}
\DeclareMathOperator{\diam}{diam}
\DeclareMathOperator{\Epigraphe}{Epigraphe}




\usepackage{xcolor}
\everymath{\color{blue}}
%\everymath{\color[rgb]{0,1,1}}
%\pagecolor[rgb]{0,0,0.5}


\newcommand*{\pdtest}[3][]{\ensuremath{\frac{\partial^{#1} #2}{\partial #3}}}

\newcommand*{\deffunc}[6][]{\ensuremath{
\begin{array}{rcl}
#2 : #3 &\rightarrow& #4\\
#5 &\mapsto& #6
\end{array}
}}

\newcommand{\eqcolon}{\mathrel{\resizebox{\widthof{$\mathord{=}$}}{\height}{ $\!\!=\!\!\resizebox{1.2\width}{0.8\height}{\raisebox{0.23ex}{$\mathop{:}$}}\!\!$ }}}
\newcommand{\coloneq}{\mathrel{\resizebox{\widthof{$\mathord{=}$}}{\height}{ $\!\!\resizebox{1.2\width}{0.8\height}{\raisebox{0.23ex}{$\mathop{:}$}}\!\!=\!\!$ }}}
\newcommand{\eqcolonl}{\ensuremath{\mathrel{=\!\!\mathop{:}}}}
\newcommand{\coloneql}{\ensuremath{\mathrel{\mathop{:} \!\! =}}}
\newcommand{\vc}[1]{% inline column vector
  \left(\begin{smallmatrix}#1\end{smallmatrix}\right)%
}
\newcommand{\vr}[1]{% inline row vector
  \begin{smallmatrix}(\,#1\,)\end{smallmatrix}%
}
\makeatletter
\newcommand*{\defeq}{\ =\mathrel{\rlap{%
                     \raisebox{0.3ex}{$\m@th\cdot$}}%
                     \raisebox{-0.3ex}{$\m@th\cdot$}}%
                     }
\makeatother

\newcommand{\mathcircle}[1]{% inline row vector
 \overset{\circ}{#1}
}
\newcommand{\ulim}{% low limit
 \underline{\lim}
}
\newcommand{\ssi}{% iff
\iff
}
\newcommand{\ps}[2]{
\expval{#1 | #2}
}
\newcommand{\df}[1]{
\mqty{#1}
}
\newcommand{\n}[1]{
\norm{#1}
}
\newcommand{\sys}[1]{
\left\{\smqty{#1}\right.
}


\newcommand{\eqdef}{\ensuremath{\overset{\text{def}}=}}


\def\Circlearrowright{\ensuremath{%
  \rotatebox[origin=c]{230}{$\circlearrowright$}}}

\newcommand\ct[1]{\text{\rmfamily\upshape #1}}
\newcommand\question[1]{ {\color{red} ...!? \small #1}}
\newcommand\caz[1]{\left\{\begin{array} #1 \end{array}\right.}
\newcommand\const{\text{\rmfamily\upshape const}}
\newcommand\toP{ \overset{\pro}{\to}}
\newcommand\toPP{ \overset{\text{PP}}{\to}}
\newcommand{\oeq}{\mathrel{\text{\textcircled{$=$}}}}





\usepackage{xcolor}
% \usepackage[normalem]{ulem}
\usepackage{lipsum}
\makeatletter
% \newcommand\colorwave[1][blue]{\bgroup \markoverwith{\lower3.5\p@\hbox{\sixly \textcolor{#1}{\char58}}}\ULon}
%\font\sixly=lasy6 % does not re-load if already loaded, so no memory problem.

\newmdtheoremenv[
linewidth= 1pt,linecolor= blue,%
leftmargin=20,rightmargin=20,innertopmargin=0pt, innerrightmargin=40,%
tikzsetting = { draw=lightgray, line width = 0.3pt,dashed,%
dash pattern = on 15pt off 3pt},%
splittopskip=\topskip,skipbelow=\baselineskip,%
skipabove=\baselineskip,ntheorem,roundcorner=0pt,
% backgroundcolor=pagebg,font=\color{orange}\sffamily, fontcolor=white
]{examplebox}{Exemple}[section]



\newcommand\R{\mathbb{R}}
\newcommand\Z{\mathbb{Z}}
\newcommand\N{\mathbb{N}}
\newcommand\E{\mathbb{E}}
\newcommand\F{\mathcal{F}}
\newcommand\cH{\mathcal{H}}
\newcommand\V{\mathbb{V}}
\newcommand\dmo{ ^{-1} }
\newcommand\kapa{\kappa}
\newcommand\im{Im}
\newcommand\hs{\mathcal{H}}





\usepackage{soul}

\makeatletter
\newcommand*{\whiten}[1]{\llap{\textcolor{white}{{\the\SOUL@token}}\hspace{#1pt}}}
\DeclareRobustCommand*\myul{%
    \def\SOUL@everyspace{\underline{\space}\kern\z@}%
    \def\SOUL@everytoken{%
     \setbox0=\hbox{\the\SOUL@token}%
     \ifdim\dp0>\z@
        \raisebox{\dp0}{\underline{\phantom{\the\SOUL@token}}}%
        \whiten{1}\whiten{0}%
        \whiten{-1}\whiten{-2}%
        \llap{\the\SOUL@token}%
     \else
        \underline{\the\SOUL@token}%
     \fi}%
\SOUL@}
\makeatother

\newcommand*{\demp}{\fontfamily{lmtt}\selectfont}

\DeclareTextFontCommand{\textdemp}{\demp}

\begin{document}

\ifcomment
Multiline
comment
\fi
\ifcomment
\myul{Typesetting test}
% \color[rgb]{1,1,1}
$∑_i^n≠ 60º±∞π∆¬≈√j∫h≤≥µ$

$\CR \R\pro\ind\pro\gS\pro
\mqty[a&b\\c&d]$
$\pro\mathbb{P}$
$\dd{x}$

  \[
    \alpha(x)=\left\{
                \begin{array}{ll}
                  x\\
                  \frac{1}{1+e^{-kx}}\\
                  \frac{e^x-e^{-x}}{e^x+e^{-x}}
                \end{array}
              \right.
  \]

  $\expval{x}$
  
  $\chi_\rho(ghg\dmo)=\Tr(\rho_{ghg\dmo})=\Tr(\rho_g\circ\rho_h\circ\rho\dmo_g)=\Tr(\rho_h)\overset{\mbox{\scalebox{0.5}{$\Tr(AB)=\Tr(BA)$}}}{=}\chi_\rho(h)$
  	$\mathop{\oplus}_{\substack{x\in X}}$

$\mat(\rho_g)=(a_{ij}(g))_{\scriptsize \substack{1\leq i\leq d \\ 1\leq j\leq d}}$ et $\mat(\rho'_g)=(a'_{ij}(g))_{\scriptsize \substack{1\leq i'\leq d' \\ 1\leq j'\leq d'}}$



\[\int_a^b{\mathbb{R}^2}g(u, v)\dd{P_{XY}}(u, v)=\iint g(u,v) f_{XY}(u, v)\dd \lambda(u) \dd \lambda(v)\]
$$\lim_{x\to\infty} f(x)$$	
$$\iiiint_V \mu(t,u,v,w) \,dt\,du\,dv\,dw$$
$$\sum_{n=1}^{\infty} 2^{-n} = 1$$	
\begin{definition}
	Si $X$ et $Y$ sont 2 v.a. ou definit la \textsc{Covariance} entre $X$ et $Y$ comme
	$\cov(X,Y)\overset{\text{def}}{=}\E\left[(X-\E(X))(Y-\E(Y))\right]=\E(XY)-\E(X)\E(Y)$.
\end{definition}
\fi
\pagebreak

% \tableofcontents

% insert your code here
%\input{./algebra/main.tex}
%\input{./geometrie-differentielle/main.tex}
%\input{./probabilite/main.tex}
%\input{./analyse-fonctionnelle/main.tex}
% \input{./Analyse-convexe-et-dualite-en-optimisation/main.tex}
%\input{./tikz/main.tex}
%\input{./Theorie-du-distributions/main.tex}
%\input{./optimisation/mine.tex}
 \input{./modelisation/main.tex}

% yves.aubry@univ-tln.fr : algebra

\end{document}

% % !TEX encoding = UTF-8 Unicode
% !TEX TS-program = xelatex

\documentclass[french]{report}

%\usepackage[utf8]{inputenc}
%\usepackage[T1]{fontenc}
\usepackage{babel}


\newif\ifcomment
%\commenttrue # Show comments

\usepackage{physics}
\usepackage{amssymb}


\usepackage{amsthm}
% \usepackage{thmtools}
\usepackage{mathtools}
\usepackage{amsfonts}

\usepackage{color}

\usepackage{tikz}

\usepackage{geometry}
\geometry{a5paper, margin=0.1in, right=1cm}

\usepackage{dsfont}

\usepackage{graphicx}
\graphicspath{ {images/} }

\usepackage{faktor}

\usepackage{IEEEtrantools}
\usepackage{enumerate}   
\usepackage[PostScript=dvips]{"/Users/aware/Documents/Courses/diagrams"}


\newtheorem{theorem}{Théorème}[section]
\renewcommand{\thetheorem}{\arabic{theorem}}
\newtheorem{lemme}{Lemme}[section]
\renewcommand{\thelemme}{\arabic{lemme}}
\newtheorem{proposition}{Proposition}[section]
\renewcommand{\theproposition}{\arabic{proposition}}
\newtheorem{notations}{Notations}[section]
\newtheorem{problem}{Problème}[section]
\newtheorem{corollary}{Corollaire}[theorem]
\renewcommand{\thecorollary}{\arabic{corollary}}
\newtheorem{property}{Propriété}[section]
\newtheorem{objective}{Objectif}[section]

\theoremstyle{definition}
\newtheorem{definition}{Définition}[section]
\renewcommand{\thedefinition}{\arabic{definition}}
\newtheorem{exercise}{Exercice}[chapter]
\renewcommand{\theexercise}{\arabic{exercise}}
\newtheorem{example}{Exemple}[chapter]
\renewcommand{\theexample}{\arabic{example}}
\newtheorem*{solution}{Solution}
\newtheorem*{application}{Application}
\newtheorem*{notation}{Notation}
\newtheorem*{vocabulary}{Vocabulaire}
\newtheorem*{properties}{Propriétés}



\theoremstyle{remark}
\newtheorem*{remark}{Remarque}
\newtheorem*{rappel}{Rappel}


\usepackage{etoolbox}
\AtBeginEnvironment{exercise}{\small}
\AtBeginEnvironment{example}{\small}

\usepackage{cases}
\usepackage[red]{mypack}

\usepackage[framemethod=TikZ]{mdframed}

\definecolor{bg}{rgb}{0.4,0.25,0.95}
\definecolor{pagebg}{rgb}{0,0,0.5}
\surroundwithmdframed[
   topline=false,
   rightline=false,
   bottomline=false,
   leftmargin=\parindent,
   skipabove=8pt,
   skipbelow=8pt,
   linecolor=blue,
   innerbottommargin=10pt,
   % backgroundcolor=bg,font=\color{orange}\sffamily, fontcolor=white
]{definition}

\usepackage{empheq}
\usepackage[most]{tcolorbox}

\newtcbox{\mymath}[1][]{%
    nobeforeafter, math upper, tcbox raise base,
    enhanced, colframe=blue!30!black,
    colback=red!10, boxrule=1pt,
    #1}

\usepackage{unixode}


\DeclareMathOperator{\ord}{ord}
\DeclareMathOperator{\orb}{orb}
\DeclareMathOperator{\stab}{stab}
\DeclareMathOperator{\Stab}{stab}
\DeclareMathOperator{\ppcm}{ppcm}
\DeclareMathOperator{\conj}{Conj}
\DeclareMathOperator{\End}{End}
\DeclareMathOperator{\rot}{rot}
\DeclareMathOperator{\trs}{trace}
\DeclareMathOperator{\Ind}{Ind}
\DeclareMathOperator{\mat}{Mat}
\DeclareMathOperator{\id}{Id}
\DeclareMathOperator{\vect}{vect}
\DeclareMathOperator{\img}{img}
\DeclareMathOperator{\cov}{Cov}
\DeclareMathOperator{\dist}{dist}
\DeclareMathOperator{\irr}{Irr}
\DeclareMathOperator{\image}{Im}
\DeclareMathOperator{\pd}{\partial}
\DeclareMathOperator{\epi}{epi}
\DeclareMathOperator{\Argmin}{Argmin}
\DeclareMathOperator{\dom}{dom}
\DeclareMathOperator{\proj}{proj}
\DeclareMathOperator{\ctg}{ctg}
\DeclareMathOperator{\supp}{supp}
\DeclareMathOperator{\argmin}{argmin}
\DeclareMathOperator{\mult}{mult}
\DeclareMathOperator{\ch}{ch}
\DeclareMathOperator{\sh}{sh}
\DeclareMathOperator{\rang}{rang}
\DeclareMathOperator{\diam}{diam}
\DeclareMathOperator{\Epigraphe}{Epigraphe}




\usepackage{xcolor}
\everymath{\color{blue}}
%\everymath{\color[rgb]{0,1,1}}
%\pagecolor[rgb]{0,0,0.5}


\newcommand*{\pdtest}[3][]{\ensuremath{\frac{\partial^{#1} #2}{\partial #3}}}

\newcommand*{\deffunc}[6][]{\ensuremath{
\begin{array}{rcl}
#2 : #3 &\rightarrow& #4\\
#5 &\mapsto& #6
\end{array}
}}

\newcommand{\eqcolon}{\mathrel{\resizebox{\widthof{$\mathord{=}$}}{\height}{ $\!\!=\!\!\resizebox{1.2\width}{0.8\height}{\raisebox{0.23ex}{$\mathop{:}$}}\!\!$ }}}
\newcommand{\coloneq}{\mathrel{\resizebox{\widthof{$\mathord{=}$}}{\height}{ $\!\!\resizebox{1.2\width}{0.8\height}{\raisebox{0.23ex}{$\mathop{:}$}}\!\!=\!\!$ }}}
\newcommand{\eqcolonl}{\ensuremath{\mathrel{=\!\!\mathop{:}}}}
\newcommand{\coloneql}{\ensuremath{\mathrel{\mathop{:} \!\! =}}}
\newcommand{\vc}[1]{% inline column vector
  \left(\begin{smallmatrix}#1\end{smallmatrix}\right)%
}
\newcommand{\vr}[1]{% inline row vector
  \begin{smallmatrix}(\,#1\,)\end{smallmatrix}%
}
\makeatletter
\newcommand*{\defeq}{\ =\mathrel{\rlap{%
                     \raisebox{0.3ex}{$\m@th\cdot$}}%
                     \raisebox{-0.3ex}{$\m@th\cdot$}}%
                     }
\makeatother

\newcommand{\mathcircle}[1]{% inline row vector
 \overset{\circ}{#1}
}
\newcommand{\ulim}{% low limit
 \underline{\lim}
}
\newcommand{\ssi}{% iff
\iff
}
\newcommand{\ps}[2]{
\expval{#1 | #2}
}
\newcommand{\df}[1]{
\mqty{#1}
}
\newcommand{\n}[1]{
\norm{#1}
}
\newcommand{\sys}[1]{
\left\{\smqty{#1}\right.
}


\newcommand{\eqdef}{\ensuremath{\overset{\text{def}}=}}


\def\Circlearrowright{\ensuremath{%
  \rotatebox[origin=c]{230}{$\circlearrowright$}}}

\newcommand\ct[1]{\text{\rmfamily\upshape #1}}
\newcommand\question[1]{ {\color{red} ...!? \small #1}}
\newcommand\caz[1]{\left\{\begin{array} #1 \end{array}\right.}
\newcommand\const{\text{\rmfamily\upshape const}}
\newcommand\toP{ \overset{\pro}{\to}}
\newcommand\toPP{ \overset{\text{PP}}{\to}}
\newcommand{\oeq}{\mathrel{\text{\textcircled{$=$}}}}





\usepackage{xcolor}
% \usepackage[normalem]{ulem}
\usepackage{lipsum}
\makeatletter
% \newcommand\colorwave[1][blue]{\bgroup \markoverwith{\lower3.5\p@\hbox{\sixly \textcolor{#1}{\char58}}}\ULon}
%\font\sixly=lasy6 % does not re-load if already loaded, so no memory problem.

\newmdtheoremenv[
linewidth= 1pt,linecolor= blue,%
leftmargin=20,rightmargin=20,innertopmargin=0pt, innerrightmargin=40,%
tikzsetting = { draw=lightgray, line width = 0.3pt,dashed,%
dash pattern = on 15pt off 3pt},%
splittopskip=\topskip,skipbelow=\baselineskip,%
skipabove=\baselineskip,ntheorem,roundcorner=0pt,
% backgroundcolor=pagebg,font=\color{orange}\sffamily, fontcolor=white
]{examplebox}{Exemple}[section]



\newcommand\R{\mathbb{R}}
\newcommand\Z{\mathbb{Z}}
\newcommand\N{\mathbb{N}}
\newcommand\E{\mathbb{E}}
\newcommand\F{\mathcal{F}}
\newcommand\cH{\mathcal{H}}
\newcommand\V{\mathbb{V}}
\newcommand\dmo{ ^{-1} }
\newcommand\kapa{\kappa}
\newcommand\im{Im}
\newcommand\hs{\mathcal{H}}





\usepackage{soul}

\makeatletter
\newcommand*{\whiten}[1]{\llap{\textcolor{white}{{\the\SOUL@token}}\hspace{#1pt}}}
\DeclareRobustCommand*\myul{%
    \def\SOUL@everyspace{\underline{\space}\kern\z@}%
    \def\SOUL@everytoken{%
     \setbox0=\hbox{\the\SOUL@token}%
     \ifdim\dp0>\z@
        \raisebox{\dp0}{\underline{\phantom{\the\SOUL@token}}}%
        \whiten{1}\whiten{0}%
        \whiten{-1}\whiten{-2}%
        \llap{\the\SOUL@token}%
     \else
        \underline{\the\SOUL@token}%
     \fi}%
\SOUL@}
\makeatother

\newcommand*{\demp}{\fontfamily{lmtt}\selectfont}

\DeclareTextFontCommand{\textdemp}{\demp}

\begin{document}

\ifcomment
Multiline
comment
\fi
\ifcomment
\myul{Typesetting test}
% \color[rgb]{1,1,1}
$∑_i^n≠ 60º±∞π∆¬≈√j∫h≤≥µ$

$\CR \R\pro\ind\pro\gS\pro
\mqty[a&b\\c&d]$
$\pro\mathbb{P}$
$\dd{x}$

  \[
    \alpha(x)=\left\{
                \begin{array}{ll}
                  x\\
                  \frac{1}{1+e^{-kx}}\\
                  \frac{e^x-e^{-x}}{e^x+e^{-x}}
                \end{array}
              \right.
  \]

  $\expval{x}$
  
  $\chi_\rho(ghg\dmo)=\Tr(\rho_{ghg\dmo})=\Tr(\rho_g\circ\rho_h\circ\rho\dmo_g)=\Tr(\rho_h)\overset{\mbox{\scalebox{0.5}{$\Tr(AB)=\Tr(BA)$}}}{=}\chi_\rho(h)$
  	$\mathop{\oplus}_{\substack{x\in X}}$

$\mat(\rho_g)=(a_{ij}(g))_{\scriptsize \substack{1\leq i\leq d \\ 1\leq j\leq d}}$ et $\mat(\rho'_g)=(a'_{ij}(g))_{\scriptsize \substack{1\leq i'\leq d' \\ 1\leq j'\leq d'}}$



\[\int_a^b{\mathbb{R}^2}g(u, v)\dd{P_{XY}}(u, v)=\iint g(u,v) f_{XY}(u, v)\dd \lambda(u) \dd \lambda(v)\]
$$\lim_{x\to\infty} f(x)$$	
$$\iiiint_V \mu(t,u,v,w) \,dt\,du\,dv\,dw$$
$$\sum_{n=1}^{\infty} 2^{-n} = 1$$	
\begin{definition}
	Si $X$ et $Y$ sont 2 v.a. ou definit la \textsc{Covariance} entre $X$ et $Y$ comme
	$\cov(X,Y)\overset{\text{def}}{=}\E\left[(X-\E(X))(Y-\E(Y))\right]=\E(XY)-\E(X)\E(Y)$.
\end{definition}
\fi
\pagebreak

% \tableofcontents

% insert your code here
%\input{./algebra/main.tex}
%\input{./geometrie-differentielle/main.tex}
%\input{./probabilite/main.tex}
%\input{./analyse-fonctionnelle/main.tex}
% \input{./Analyse-convexe-et-dualite-en-optimisation/main.tex}
%\input{./tikz/main.tex}
%\input{./Theorie-du-distributions/main.tex}
%\input{./optimisation/mine.tex}
 \input{./modelisation/main.tex}

% yves.aubry@univ-tln.fr : algebra

\end{document}

%% !TEX encoding = UTF-8 Unicode
% !TEX TS-program = xelatex

\documentclass[french]{report}

%\usepackage[utf8]{inputenc}
%\usepackage[T1]{fontenc}
\usepackage{babel}


\newif\ifcomment
%\commenttrue # Show comments

\usepackage{physics}
\usepackage{amssymb}


\usepackage{amsthm}
% \usepackage{thmtools}
\usepackage{mathtools}
\usepackage{amsfonts}

\usepackage{color}

\usepackage{tikz}

\usepackage{geometry}
\geometry{a5paper, margin=0.1in, right=1cm}

\usepackage{dsfont}

\usepackage{graphicx}
\graphicspath{ {images/} }

\usepackage{faktor}

\usepackage{IEEEtrantools}
\usepackage{enumerate}   
\usepackage[PostScript=dvips]{"/Users/aware/Documents/Courses/diagrams"}


\newtheorem{theorem}{Théorème}[section]
\renewcommand{\thetheorem}{\arabic{theorem}}
\newtheorem{lemme}{Lemme}[section]
\renewcommand{\thelemme}{\arabic{lemme}}
\newtheorem{proposition}{Proposition}[section]
\renewcommand{\theproposition}{\arabic{proposition}}
\newtheorem{notations}{Notations}[section]
\newtheorem{problem}{Problème}[section]
\newtheorem{corollary}{Corollaire}[theorem]
\renewcommand{\thecorollary}{\arabic{corollary}}
\newtheorem{property}{Propriété}[section]
\newtheorem{objective}{Objectif}[section]

\theoremstyle{definition}
\newtheorem{definition}{Définition}[section]
\renewcommand{\thedefinition}{\arabic{definition}}
\newtheorem{exercise}{Exercice}[chapter]
\renewcommand{\theexercise}{\arabic{exercise}}
\newtheorem{example}{Exemple}[chapter]
\renewcommand{\theexample}{\arabic{example}}
\newtheorem*{solution}{Solution}
\newtheorem*{application}{Application}
\newtheorem*{notation}{Notation}
\newtheorem*{vocabulary}{Vocabulaire}
\newtheorem*{properties}{Propriétés}



\theoremstyle{remark}
\newtheorem*{remark}{Remarque}
\newtheorem*{rappel}{Rappel}


\usepackage{etoolbox}
\AtBeginEnvironment{exercise}{\small}
\AtBeginEnvironment{example}{\small}

\usepackage{cases}
\usepackage[red]{mypack}

\usepackage[framemethod=TikZ]{mdframed}

\definecolor{bg}{rgb}{0.4,0.25,0.95}
\definecolor{pagebg}{rgb}{0,0,0.5}
\surroundwithmdframed[
   topline=false,
   rightline=false,
   bottomline=false,
   leftmargin=\parindent,
   skipabove=8pt,
   skipbelow=8pt,
   linecolor=blue,
   innerbottommargin=10pt,
   % backgroundcolor=bg,font=\color{orange}\sffamily, fontcolor=white
]{definition}

\usepackage{empheq}
\usepackage[most]{tcolorbox}

\newtcbox{\mymath}[1][]{%
    nobeforeafter, math upper, tcbox raise base,
    enhanced, colframe=blue!30!black,
    colback=red!10, boxrule=1pt,
    #1}

\usepackage{unixode}


\DeclareMathOperator{\ord}{ord}
\DeclareMathOperator{\orb}{orb}
\DeclareMathOperator{\stab}{stab}
\DeclareMathOperator{\Stab}{stab}
\DeclareMathOperator{\ppcm}{ppcm}
\DeclareMathOperator{\conj}{Conj}
\DeclareMathOperator{\End}{End}
\DeclareMathOperator{\rot}{rot}
\DeclareMathOperator{\trs}{trace}
\DeclareMathOperator{\Ind}{Ind}
\DeclareMathOperator{\mat}{Mat}
\DeclareMathOperator{\id}{Id}
\DeclareMathOperator{\vect}{vect}
\DeclareMathOperator{\img}{img}
\DeclareMathOperator{\cov}{Cov}
\DeclareMathOperator{\dist}{dist}
\DeclareMathOperator{\irr}{Irr}
\DeclareMathOperator{\image}{Im}
\DeclareMathOperator{\pd}{\partial}
\DeclareMathOperator{\epi}{epi}
\DeclareMathOperator{\Argmin}{Argmin}
\DeclareMathOperator{\dom}{dom}
\DeclareMathOperator{\proj}{proj}
\DeclareMathOperator{\ctg}{ctg}
\DeclareMathOperator{\supp}{supp}
\DeclareMathOperator{\argmin}{argmin}
\DeclareMathOperator{\mult}{mult}
\DeclareMathOperator{\ch}{ch}
\DeclareMathOperator{\sh}{sh}
\DeclareMathOperator{\rang}{rang}
\DeclareMathOperator{\diam}{diam}
\DeclareMathOperator{\Epigraphe}{Epigraphe}




\usepackage{xcolor}
\everymath{\color{blue}}
%\everymath{\color[rgb]{0,1,1}}
%\pagecolor[rgb]{0,0,0.5}


\newcommand*{\pdtest}[3][]{\ensuremath{\frac{\partial^{#1} #2}{\partial #3}}}

\newcommand*{\deffunc}[6][]{\ensuremath{
\begin{array}{rcl}
#2 : #3 &\rightarrow& #4\\
#5 &\mapsto& #6
\end{array}
}}

\newcommand{\eqcolon}{\mathrel{\resizebox{\widthof{$\mathord{=}$}}{\height}{ $\!\!=\!\!\resizebox{1.2\width}{0.8\height}{\raisebox{0.23ex}{$\mathop{:}$}}\!\!$ }}}
\newcommand{\coloneq}{\mathrel{\resizebox{\widthof{$\mathord{=}$}}{\height}{ $\!\!\resizebox{1.2\width}{0.8\height}{\raisebox{0.23ex}{$\mathop{:}$}}\!\!=\!\!$ }}}
\newcommand{\eqcolonl}{\ensuremath{\mathrel{=\!\!\mathop{:}}}}
\newcommand{\coloneql}{\ensuremath{\mathrel{\mathop{:} \!\! =}}}
\newcommand{\vc}[1]{% inline column vector
  \left(\begin{smallmatrix}#1\end{smallmatrix}\right)%
}
\newcommand{\vr}[1]{% inline row vector
  \begin{smallmatrix}(\,#1\,)\end{smallmatrix}%
}
\makeatletter
\newcommand*{\defeq}{\ =\mathrel{\rlap{%
                     \raisebox{0.3ex}{$\m@th\cdot$}}%
                     \raisebox{-0.3ex}{$\m@th\cdot$}}%
                     }
\makeatother

\newcommand{\mathcircle}[1]{% inline row vector
 \overset{\circ}{#1}
}
\newcommand{\ulim}{% low limit
 \underline{\lim}
}
\newcommand{\ssi}{% iff
\iff
}
\newcommand{\ps}[2]{
\expval{#1 | #2}
}
\newcommand{\df}[1]{
\mqty{#1}
}
\newcommand{\n}[1]{
\norm{#1}
}
\newcommand{\sys}[1]{
\left\{\smqty{#1}\right.
}


\newcommand{\eqdef}{\ensuremath{\overset{\text{def}}=}}


\def\Circlearrowright{\ensuremath{%
  \rotatebox[origin=c]{230}{$\circlearrowright$}}}

\newcommand\ct[1]{\text{\rmfamily\upshape #1}}
\newcommand\question[1]{ {\color{red} ...!? \small #1}}
\newcommand\caz[1]{\left\{\begin{array} #1 \end{array}\right.}
\newcommand\const{\text{\rmfamily\upshape const}}
\newcommand\toP{ \overset{\pro}{\to}}
\newcommand\toPP{ \overset{\text{PP}}{\to}}
\newcommand{\oeq}{\mathrel{\text{\textcircled{$=$}}}}





\usepackage{xcolor}
% \usepackage[normalem]{ulem}
\usepackage{lipsum}
\makeatletter
% \newcommand\colorwave[1][blue]{\bgroup \markoverwith{\lower3.5\p@\hbox{\sixly \textcolor{#1}{\char58}}}\ULon}
%\font\sixly=lasy6 % does not re-load if already loaded, so no memory problem.

\newmdtheoremenv[
linewidth= 1pt,linecolor= blue,%
leftmargin=20,rightmargin=20,innertopmargin=0pt, innerrightmargin=40,%
tikzsetting = { draw=lightgray, line width = 0.3pt,dashed,%
dash pattern = on 15pt off 3pt},%
splittopskip=\topskip,skipbelow=\baselineskip,%
skipabove=\baselineskip,ntheorem,roundcorner=0pt,
% backgroundcolor=pagebg,font=\color{orange}\sffamily, fontcolor=white
]{examplebox}{Exemple}[section]



\newcommand\R{\mathbb{R}}
\newcommand\Z{\mathbb{Z}}
\newcommand\N{\mathbb{N}}
\newcommand\E{\mathbb{E}}
\newcommand\F{\mathcal{F}}
\newcommand\cH{\mathcal{H}}
\newcommand\V{\mathbb{V}}
\newcommand\dmo{ ^{-1} }
\newcommand\kapa{\kappa}
\newcommand\im{Im}
\newcommand\hs{\mathcal{H}}





\usepackage{soul}

\makeatletter
\newcommand*{\whiten}[1]{\llap{\textcolor{white}{{\the\SOUL@token}}\hspace{#1pt}}}
\DeclareRobustCommand*\myul{%
    \def\SOUL@everyspace{\underline{\space}\kern\z@}%
    \def\SOUL@everytoken{%
     \setbox0=\hbox{\the\SOUL@token}%
     \ifdim\dp0>\z@
        \raisebox{\dp0}{\underline{\phantom{\the\SOUL@token}}}%
        \whiten{1}\whiten{0}%
        \whiten{-1}\whiten{-2}%
        \llap{\the\SOUL@token}%
     \else
        \underline{\the\SOUL@token}%
     \fi}%
\SOUL@}
\makeatother

\newcommand*{\demp}{\fontfamily{lmtt}\selectfont}

\DeclareTextFontCommand{\textdemp}{\demp}

\begin{document}

\ifcomment
Multiline
comment
\fi
\ifcomment
\myul{Typesetting test}
% \color[rgb]{1,1,1}
$∑_i^n≠ 60º±∞π∆¬≈√j∫h≤≥µ$

$\CR \R\pro\ind\pro\gS\pro
\mqty[a&b\\c&d]$
$\pro\mathbb{P}$
$\dd{x}$

  \[
    \alpha(x)=\left\{
                \begin{array}{ll}
                  x\\
                  \frac{1}{1+e^{-kx}}\\
                  \frac{e^x-e^{-x}}{e^x+e^{-x}}
                \end{array}
              \right.
  \]

  $\expval{x}$
  
  $\chi_\rho(ghg\dmo)=\Tr(\rho_{ghg\dmo})=\Tr(\rho_g\circ\rho_h\circ\rho\dmo_g)=\Tr(\rho_h)\overset{\mbox{\scalebox{0.5}{$\Tr(AB)=\Tr(BA)$}}}{=}\chi_\rho(h)$
  	$\mathop{\oplus}_{\substack{x\in X}}$

$\mat(\rho_g)=(a_{ij}(g))_{\scriptsize \substack{1\leq i\leq d \\ 1\leq j\leq d}}$ et $\mat(\rho'_g)=(a'_{ij}(g))_{\scriptsize \substack{1\leq i'\leq d' \\ 1\leq j'\leq d'}}$



\[\int_a^b{\mathbb{R}^2}g(u, v)\dd{P_{XY}}(u, v)=\iint g(u,v) f_{XY}(u, v)\dd \lambda(u) \dd \lambda(v)\]
$$\lim_{x\to\infty} f(x)$$	
$$\iiiint_V \mu(t,u,v,w) \,dt\,du\,dv\,dw$$
$$\sum_{n=1}^{\infty} 2^{-n} = 1$$	
\begin{definition}
	Si $X$ et $Y$ sont 2 v.a. ou definit la \textsc{Covariance} entre $X$ et $Y$ comme
	$\cov(X,Y)\overset{\text{def}}{=}\E\left[(X-\E(X))(Y-\E(Y))\right]=\E(XY)-\E(X)\E(Y)$.
\end{definition}
\fi
\pagebreak

% \tableofcontents

% insert your code here
%\input{./algebra/main.tex}
%\input{./geometrie-differentielle/main.tex}
%\input{./probabilite/main.tex}
%\input{./analyse-fonctionnelle/main.tex}
% \input{./Analyse-convexe-et-dualite-en-optimisation/main.tex}
%\input{./tikz/main.tex}
%\input{./Theorie-du-distributions/main.tex}
%\input{./optimisation/mine.tex}
 \input{./modelisation/main.tex}

% yves.aubry@univ-tln.fr : algebra

\end{document}

%% !TEX encoding = UTF-8 Unicode
% !TEX TS-program = xelatex

\documentclass[french]{report}

%\usepackage[utf8]{inputenc}
%\usepackage[T1]{fontenc}
\usepackage{babel}


\newif\ifcomment
%\commenttrue # Show comments

\usepackage{physics}
\usepackage{amssymb}


\usepackage{amsthm}
% \usepackage{thmtools}
\usepackage{mathtools}
\usepackage{amsfonts}

\usepackage{color}

\usepackage{tikz}

\usepackage{geometry}
\geometry{a5paper, margin=0.1in, right=1cm}

\usepackage{dsfont}

\usepackage{graphicx}
\graphicspath{ {images/} }

\usepackage{faktor}

\usepackage{IEEEtrantools}
\usepackage{enumerate}   
\usepackage[PostScript=dvips]{"/Users/aware/Documents/Courses/diagrams"}


\newtheorem{theorem}{Théorème}[section]
\renewcommand{\thetheorem}{\arabic{theorem}}
\newtheorem{lemme}{Lemme}[section]
\renewcommand{\thelemme}{\arabic{lemme}}
\newtheorem{proposition}{Proposition}[section]
\renewcommand{\theproposition}{\arabic{proposition}}
\newtheorem{notations}{Notations}[section]
\newtheorem{problem}{Problème}[section]
\newtheorem{corollary}{Corollaire}[theorem]
\renewcommand{\thecorollary}{\arabic{corollary}}
\newtheorem{property}{Propriété}[section]
\newtheorem{objective}{Objectif}[section]

\theoremstyle{definition}
\newtheorem{definition}{Définition}[section]
\renewcommand{\thedefinition}{\arabic{definition}}
\newtheorem{exercise}{Exercice}[chapter]
\renewcommand{\theexercise}{\arabic{exercise}}
\newtheorem{example}{Exemple}[chapter]
\renewcommand{\theexample}{\arabic{example}}
\newtheorem*{solution}{Solution}
\newtheorem*{application}{Application}
\newtheorem*{notation}{Notation}
\newtheorem*{vocabulary}{Vocabulaire}
\newtheorem*{properties}{Propriétés}



\theoremstyle{remark}
\newtheorem*{remark}{Remarque}
\newtheorem*{rappel}{Rappel}


\usepackage{etoolbox}
\AtBeginEnvironment{exercise}{\small}
\AtBeginEnvironment{example}{\small}

\usepackage{cases}
\usepackage[red]{mypack}

\usepackage[framemethod=TikZ]{mdframed}

\definecolor{bg}{rgb}{0.4,0.25,0.95}
\definecolor{pagebg}{rgb}{0,0,0.5}
\surroundwithmdframed[
   topline=false,
   rightline=false,
   bottomline=false,
   leftmargin=\parindent,
   skipabove=8pt,
   skipbelow=8pt,
   linecolor=blue,
   innerbottommargin=10pt,
   % backgroundcolor=bg,font=\color{orange}\sffamily, fontcolor=white
]{definition}

\usepackage{empheq}
\usepackage[most]{tcolorbox}

\newtcbox{\mymath}[1][]{%
    nobeforeafter, math upper, tcbox raise base,
    enhanced, colframe=blue!30!black,
    colback=red!10, boxrule=1pt,
    #1}

\usepackage{unixode}


\DeclareMathOperator{\ord}{ord}
\DeclareMathOperator{\orb}{orb}
\DeclareMathOperator{\stab}{stab}
\DeclareMathOperator{\Stab}{stab}
\DeclareMathOperator{\ppcm}{ppcm}
\DeclareMathOperator{\conj}{Conj}
\DeclareMathOperator{\End}{End}
\DeclareMathOperator{\rot}{rot}
\DeclareMathOperator{\trs}{trace}
\DeclareMathOperator{\Ind}{Ind}
\DeclareMathOperator{\mat}{Mat}
\DeclareMathOperator{\id}{Id}
\DeclareMathOperator{\vect}{vect}
\DeclareMathOperator{\img}{img}
\DeclareMathOperator{\cov}{Cov}
\DeclareMathOperator{\dist}{dist}
\DeclareMathOperator{\irr}{Irr}
\DeclareMathOperator{\image}{Im}
\DeclareMathOperator{\pd}{\partial}
\DeclareMathOperator{\epi}{epi}
\DeclareMathOperator{\Argmin}{Argmin}
\DeclareMathOperator{\dom}{dom}
\DeclareMathOperator{\proj}{proj}
\DeclareMathOperator{\ctg}{ctg}
\DeclareMathOperator{\supp}{supp}
\DeclareMathOperator{\argmin}{argmin}
\DeclareMathOperator{\mult}{mult}
\DeclareMathOperator{\ch}{ch}
\DeclareMathOperator{\sh}{sh}
\DeclareMathOperator{\rang}{rang}
\DeclareMathOperator{\diam}{diam}
\DeclareMathOperator{\Epigraphe}{Epigraphe}




\usepackage{xcolor}
\everymath{\color{blue}}
%\everymath{\color[rgb]{0,1,1}}
%\pagecolor[rgb]{0,0,0.5}


\newcommand*{\pdtest}[3][]{\ensuremath{\frac{\partial^{#1} #2}{\partial #3}}}

\newcommand*{\deffunc}[6][]{\ensuremath{
\begin{array}{rcl}
#2 : #3 &\rightarrow& #4\\
#5 &\mapsto& #6
\end{array}
}}

\newcommand{\eqcolon}{\mathrel{\resizebox{\widthof{$\mathord{=}$}}{\height}{ $\!\!=\!\!\resizebox{1.2\width}{0.8\height}{\raisebox{0.23ex}{$\mathop{:}$}}\!\!$ }}}
\newcommand{\coloneq}{\mathrel{\resizebox{\widthof{$\mathord{=}$}}{\height}{ $\!\!\resizebox{1.2\width}{0.8\height}{\raisebox{0.23ex}{$\mathop{:}$}}\!\!=\!\!$ }}}
\newcommand{\eqcolonl}{\ensuremath{\mathrel{=\!\!\mathop{:}}}}
\newcommand{\coloneql}{\ensuremath{\mathrel{\mathop{:} \!\! =}}}
\newcommand{\vc}[1]{% inline column vector
  \left(\begin{smallmatrix}#1\end{smallmatrix}\right)%
}
\newcommand{\vr}[1]{% inline row vector
  \begin{smallmatrix}(\,#1\,)\end{smallmatrix}%
}
\makeatletter
\newcommand*{\defeq}{\ =\mathrel{\rlap{%
                     \raisebox{0.3ex}{$\m@th\cdot$}}%
                     \raisebox{-0.3ex}{$\m@th\cdot$}}%
                     }
\makeatother

\newcommand{\mathcircle}[1]{% inline row vector
 \overset{\circ}{#1}
}
\newcommand{\ulim}{% low limit
 \underline{\lim}
}
\newcommand{\ssi}{% iff
\iff
}
\newcommand{\ps}[2]{
\expval{#1 | #2}
}
\newcommand{\df}[1]{
\mqty{#1}
}
\newcommand{\n}[1]{
\norm{#1}
}
\newcommand{\sys}[1]{
\left\{\smqty{#1}\right.
}


\newcommand{\eqdef}{\ensuremath{\overset{\text{def}}=}}


\def\Circlearrowright{\ensuremath{%
  \rotatebox[origin=c]{230}{$\circlearrowright$}}}

\newcommand\ct[1]{\text{\rmfamily\upshape #1}}
\newcommand\question[1]{ {\color{red} ...!? \small #1}}
\newcommand\caz[1]{\left\{\begin{array} #1 \end{array}\right.}
\newcommand\const{\text{\rmfamily\upshape const}}
\newcommand\toP{ \overset{\pro}{\to}}
\newcommand\toPP{ \overset{\text{PP}}{\to}}
\newcommand{\oeq}{\mathrel{\text{\textcircled{$=$}}}}





\usepackage{xcolor}
% \usepackage[normalem]{ulem}
\usepackage{lipsum}
\makeatletter
% \newcommand\colorwave[1][blue]{\bgroup \markoverwith{\lower3.5\p@\hbox{\sixly \textcolor{#1}{\char58}}}\ULon}
%\font\sixly=lasy6 % does not re-load if already loaded, so no memory problem.

\newmdtheoremenv[
linewidth= 1pt,linecolor= blue,%
leftmargin=20,rightmargin=20,innertopmargin=0pt, innerrightmargin=40,%
tikzsetting = { draw=lightgray, line width = 0.3pt,dashed,%
dash pattern = on 15pt off 3pt},%
splittopskip=\topskip,skipbelow=\baselineskip,%
skipabove=\baselineskip,ntheorem,roundcorner=0pt,
% backgroundcolor=pagebg,font=\color{orange}\sffamily, fontcolor=white
]{examplebox}{Exemple}[section]



\newcommand\R{\mathbb{R}}
\newcommand\Z{\mathbb{Z}}
\newcommand\N{\mathbb{N}}
\newcommand\E{\mathbb{E}}
\newcommand\F{\mathcal{F}}
\newcommand\cH{\mathcal{H}}
\newcommand\V{\mathbb{V}}
\newcommand\dmo{ ^{-1} }
\newcommand\kapa{\kappa}
\newcommand\im{Im}
\newcommand\hs{\mathcal{H}}





\usepackage{soul}

\makeatletter
\newcommand*{\whiten}[1]{\llap{\textcolor{white}{{\the\SOUL@token}}\hspace{#1pt}}}
\DeclareRobustCommand*\myul{%
    \def\SOUL@everyspace{\underline{\space}\kern\z@}%
    \def\SOUL@everytoken{%
     \setbox0=\hbox{\the\SOUL@token}%
     \ifdim\dp0>\z@
        \raisebox{\dp0}{\underline{\phantom{\the\SOUL@token}}}%
        \whiten{1}\whiten{0}%
        \whiten{-1}\whiten{-2}%
        \llap{\the\SOUL@token}%
     \else
        \underline{\the\SOUL@token}%
     \fi}%
\SOUL@}
\makeatother

\newcommand*{\demp}{\fontfamily{lmtt}\selectfont}

\DeclareTextFontCommand{\textdemp}{\demp}

\begin{document}

\ifcomment
Multiline
comment
\fi
\ifcomment
\myul{Typesetting test}
% \color[rgb]{1,1,1}
$∑_i^n≠ 60º±∞π∆¬≈√j∫h≤≥µ$

$\CR \R\pro\ind\pro\gS\pro
\mqty[a&b\\c&d]$
$\pro\mathbb{P}$
$\dd{x}$

  \[
    \alpha(x)=\left\{
                \begin{array}{ll}
                  x\\
                  \frac{1}{1+e^{-kx}}\\
                  \frac{e^x-e^{-x}}{e^x+e^{-x}}
                \end{array}
              \right.
  \]

  $\expval{x}$
  
  $\chi_\rho(ghg\dmo)=\Tr(\rho_{ghg\dmo})=\Tr(\rho_g\circ\rho_h\circ\rho\dmo_g)=\Tr(\rho_h)\overset{\mbox{\scalebox{0.5}{$\Tr(AB)=\Tr(BA)$}}}{=}\chi_\rho(h)$
  	$\mathop{\oplus}_{\substack{x\in X}}$

$\mat(\rho_g)=(a_{ij}(g))_{\scriptsize \substack{1\leq i\leq d \\ 1\leq j\leq d}}$ et $\mat(\rho'_g)=(a'_{ij}(g))_{\scriptsize \substack{1\leq i'\leq d' \\ 1\leq j'\leq d'}}$



\[\int_a^b{\mathbb{R}^2}g(u, v)\dd{P_{XY}}(u, v)=\iint g(u,v) f_{XY}(u, v)\dd \lambda(u) \dd \lambda(v)\]
$$\lim_{x\to\infty} f(x)$$	
$$\iiiint_V \mu(t,u,v,w) \,dt\,du\,dv\,dw$$
$$\sum_{n=1}^{\infty} 2^{-n} = 1$$	
\begin{definition}
	Si $X$ et $Y$ sont 2 v.a. ou definit la \textsc{Covariance} entre $X$ et $Y$ comme
	$\cov(X,Y)\overset{\text{def}}{=}\E\left[(X-\E(X))(Y-\E(Y))\right]=\E(XY)-\E(X)\E(Y)$.
\end{definition}
\fi
\pagebreak

% \tableofcontents

% insert your code here
%\input{./algebra/main.tex}
%\input{./geometrie-differentielle/main.tex}
%\input{./probabilite/main.tex}
%\input{./analyse-fonctionnelle/main.tex}
% \input{./Analyse-convexe-et-dualite-en-optimisation/main.tex}
%\input{./tikz/main.tex}
%\input{./Theorie-du-distributions/main.tex}
%\input{./optimisation/mine.tex}
 \input{./modelisation/main.tex}

% yves.aubry@univ-tln.fr : algebra

\end{document}

%\input{./optimisation/mine.tex}
 % !TEX encoding = UTF-8 Unicode
% !TEX TS-program = xelatex

\documentclass[french]{report}

%\usepackage[utf8]{inputenc}
%\usepackage[T1]{fontenc}
\usepackage{babel}


\newif\ifcomment
%\commenttrue # Show comments

\usepackage{physics}
\usepackage{amssymb}


\usepackage{amsthm}
% \usepackage{thmtools}
\usepackage{mathtools}
\usepackage{amsfonts}

\usepackage{color}

\usepackage{tikz}

\usepackage{geometry}
\geometry{a5paper, margin=0.1in, right=1cm}

\usepackage{dsfont}

\usepackage{graphicx}
\graphicspath{ {images/} }

\usepackage{faktor}

\usepackage{IEEEtrantools}
\usepackage{enumerate}   
\usepackage[PostScript=dvips]{"/Users/aware/Documents/Courses/diagrams"}


\newtheorem{theorem}{Théorème}[section]
\renewcommand{\thetheorem}{\arabic{theorem}}
\newtheorem{lemme}{Lemme}[section]
\renewcommand{\thelemme}{\arabic{lemme}}
\newtheorem{proposition}{Proposition}[section]
\renewcommand{\theproposition}{\arabic{proposition}}
\newtheorem{notations}{Notations}[section]
\newtheorem{problem}{Problème}[section]
\newtheorem{corollary}{Corollaire}[theorem]
\renewcommand{\thecorollary}{\arabic{corollary}}
\newtheorem{property}{Propriété}[section]
\newtheorem{objective}{Objectif}[section]

\theoremstyle{definition}
\newtheorem{definition}{Définition}[section]
\renewcommand{\thedefinition}{\arabic{definition}}
\newtheorem{exercise}{Exercice}[chapter]
\renewcommand{\theexercise}{\arabic{exercise}}
\newtheorem{example}{Exemple}[chapter]
\renewcommand{\theexample}{\arabic{example}}
\newtheorem*{solution}{Solution}
\newtheorem*{application}{Application}
\newtheorem*{notation}{Notation}
\newtheorem*{vocabulary}{Vocabulaire}
\newtheorem*{properties}{Propriétés}



\theoremstyle{remark}
\newtheorem*{remark}{Remarque}
\newtheorem*{rappel}{Rappel}


\usepackage{etoolbox}
\AtBeginEnvironment{exercise}{\small}
\AtBeginEnvironment{example}{\small}

\usepackage{cases}
\usepackage[red]{mypack}

\usepackage[framemethod=TikZ]{mdframed}

\definecolor{bg}{rgb}{0.4,0.25,0.95}
\definecolor{pagebg}{rgb}{0,0,0.5}
\surroundwithmdframed[
   topline=false,
   rightline=false,
   bottomline=false,
   leftmargin=\parindent,
   skipabove=8pt,
   skipbelow=8pt,
   linecolor=blue,
   innerbottommargin=10pt,
   % backgroundcolor=bg,font=\color{orange}\sffamily, fontcolor=white
]{definition}

\usepackage{empheq}
\usepackage[most]{tcolorbox}

\newtcbox{\mymath}[1][]{%
    nobeforeafter, math upper, tcbox raise base,
    enhanced, colframe=blue!30!black,
    colback=red!10, boxrule=1pt,
    #1}

\usepackage{unixode}


\DeclareMathOperator{\ord}{ord}
\DeclareMathOperator{\orb}{orb}
\DeclareMathOperator{\stab}{stab}
\DeclareMathOperator{\Stab}{stab}
\DeclareMathOperator{\ppcm}{ppcm}
\DeclareMathOperator{\conj}{Conj}
\DeclareMathOperator{\End}{End}
\DeclareMathOperator{\rot}{rot}
\DeclareMathOperator{\trs}{trace}
\DeclareMathOperator{\Ind}{Ind}
\DeclareMathOperator{\mat}{Mat}
\DeclareMathOperator{\id}{Id}
\DeclareMathOperator{\vect}{vect}
\DeclareMathOperator{\img}{img}
\DeclareMathOperator{\cov}{Cov}
\DeclareMathOperator{\dist}{dist}
\DeclareMathOperator{\irr}{Irr}
\DeclareMathOperator{\image}{Im}
\DeclareMathOperator{\pd}{\partial}
\DeclareMathOperator{\epi}{epi}
\DeclareMathOperator{\Argmin}{Argmin}
\DeclareMathOperator{\dom}{dom}
\DeclareMathOperator{\proj}{proj}
\DeclareMathOperator{\ctg}{ctg}
\DeclareMathOperator{\supp}{supp}
\DeclareMathOperator{\argmin}{argmin}
\DeclareMathOperator{\mult}{mult}
\DeclareMathOperator{\ch}{ch}
\DeclareMathOperator{\sh}{sh}
\DeclareMathOperator{\rang}{rang}
\DeclareMathOperator{\diam}{diam}
\DeclareMathOperator{\Epigraphe}{Epigraphe}




\usepackage{xcolor}
\everymath{\color{blue}}
%\everymath{\color[rgb]{0,1,1}}
%\pagecolor[rgb]{0,0,0.5}


\newcommand*{\pdtest}[3][]{\ensuremath{\frac{\partial^{#1} #2}{\partial #3}}}

\newcommand*{\deffunc}[6][]{\ensuremath{
\begin{array}{rcl}
#2 : #3 &\rightarrow& #4\\
#5 &\mapsto& #6
\end{array}
}}

\newcommand{\eqcolon}{\mathrel{\resizebox{\widthof{$\mathord{=}$}}{\height}{ $\!\!=\!\!\resizebox{1.2\width}{0.8\height}{\raisebox{0.23ex}{$\mathop{:}$}}\!\!$ }}}
\newcommand{\coloneq}{\mathrel{\resizebox{\widthof{$\mathord{=}$}}{\height}{ $\!\!\resizebox{1.2\width}{0.8\height}{\raisebox{0.23ex}{$\mathop{:}$}}\!\!=\!\!$ }}}
\newcommand{\eqcolonl}{\ensuremath{\mathrel{=\!\!\mathop{:}}}}
\newcommand{\coloneql}{\ensuremath{\mathrel{\mathop{:} \!\! =}}}
\newcommand{\vc}[1]{% inline column vector
  \left(\begin{smallmatrix}#1\end{smallmatrix}\right)%
}
\newcommand{\vr}[1]{% inline row vector
  \begin{smallmatrix}(\,#1\,)\end{smallmatrix}%
}
\makeatletter
\newcommand*{\defeq}{\ =\mathrel{\rlap{%
                     \raisebox{0.3ex}{$\m@th\cdot$}}%
                     \raisebox{-0.3ex}{$\m@th\cdot$}}%
                     }
\makeatother

\newcommand{\mathcircle}[1]{% inline row vector
 \overset{\circ}{#1}
}
\newcommand{\ulim}{% low limit
 \underline{\lim}
}
\newcommand{\ssi}{% iff
\iff
}
\newcommand{\ps}[2]{
\expval{#1 | #2}
}
\newcommand{\df}[1]{
\mqty{#1}
}
\newcommand{\n}[1]{
\norm{#1}
}
\newcommand{\sys}[1]{
\left\{\smqty{#1}\right.
}


\newcommand{\eqdef}{\ensuremath{\overset{\text{def}}=}}


\def\Circlearrowright{\ensuremath{%
  \rotatebox[origin=c]{230}{$\circlearrowright$}}}

\newcommand\ct[1]{\text{\rmfamily\upshape #1}}
\newcommand\question[1]{ {\color{red} ...!? \small #1}}
\newcommand\caz[1]{\left\{\begin{array} #1 \end{array}\right.}
\newcommand\const{\text{\rmfamily\upshape const}}
\newcommand\toP{ \overset{\pro}{\to}}
\newcommand\toPP{ \overset{\text{PP}}{\to}}
\newcommand{\oeq}{\mathrel{\text{\textcircled{$=$}}}}





\usepackage{xcolor}
% \usepackage[normalem]{ulem}
\usepackage{lipsum}
\makeatletter
% \newcommand\colorwave[1][blue]{\bgroup \markoverwith{\lower3.5\p@\hbox{\sixly \textcolor{#1}{\char58}}}\ULon}
%\font\sixly=lasy6 % does not re-load if already loaded, so no memory problem.

\newmdtheoremenv[
linewidth= 1pt,linecolor= blue,%
leftmargin=20,rightmargin=20,innertopmargin=0pt, innerrightmargin=40,%
tikzsetting = { draw=lightgray, line width = 0.3pt,dashed,%
dash pattern = on 15pt off 3pt},%
splittopskip=\topskip,skipbelow=\baselineskip,%
skipabove=\baselineskip,ntheorem,roundcorner=0pt,
% backgroundcolor=pagebg,font=\color{orange}\sffamily, fontcolor=white
]{examplebox}{Exemple}[section]



\newcommand\R{\mathbb{R}}
\newcommand\Z{\mathbb{Z}}
\newcommand\N{\mathbb{N}}
\newcommand\E{\mathbb{E}}
\newcommand\F{\mathcal{F}}
\newcommand\cH{\mathcal{H}}
\newcommand\V{\mathbb{V}}
\newcommand\dmo{ ^{-1} }
\newcommand\kapa{\kappa}
\newcommand\im{Im}
\newcommand\hs{\mathcal{H}}





\usepackage{soul}

\makeatletter
\newcommand*{\whiten}[1]{\llap{\textcolor{white}{{\the\SOUL@token}}\hspace{#1pt}}}
\DeclareRobustCommand*\myul{%
    \def\SOUL@everyspace{\underline{\space}\kern\z@}%
    \def\SOUL@everytoken{%
     \setbox0=\hbox{\the\SOUL@token}%
     \ifdim\dp0>\z@
        \raisebox{\dp0}{\underline{\phantom{\the\SOUL@token}}}%
        \whiten{1}\whiten{0}%
        \whiten{-1}\whiten{-2}%
        \llap{\the\SOUL@token}%
     \else
        \underline{\the\SOUL@token}%
     \fi}%
\SOUL@}
\makeatother

\newcommand*{\demp}{\fontfamily{lmtt}\selectfont}

\DeclareTextFontCommand{\textdemp}{\demp}

\begin{document}

\ifcomment
Multiline
comment
\fi
\ifcomment
\myul{Typesetting test}
% \color[rgb]{1,1,1}
$∑_i^n≠ 60º±∞π∆¬≈√j∫h≤≥µ$

$\CR \R\pro\ind\pro\gS\pro
\mqty[a&b\\c&d]$
$\pro\mathbb{P}$
$\dd{x}$

  \[
    \alpha(x)=\left\{
                \begin{array}{ll}
                  x\\
                  \frac{1}{1+e^{-kx}}\\
                  \frac{e^x-e^{-x}}{e^x+e^{-x}}
                \end{array}
              \right.
  \]

  $\expval{x}$
  
  $\chi_\rho(ghg\dmo)=\Tr(\rho_{ghg\dmo})=\Tr(\rho_g\circ\rho_h\circ\rho\dmo_g)=\Tr(\rho_h)\overset{\mbox{\scalebox{0.5}{$\Tr(AB)=\Tr(BA)$}}}{=}\chi_\rho(h)$
  	$\mathop{\oplus}_{\substack{x\in X}}$

$\mat(\rho_g)=(a_{ij}(g))_{\scriptsize \substack{1\leq i\leq d \\ 1\leq j\leq d}}$ et $\mat(\rho'_g)=(a'_{ij}(g))_{\scriptsize \substack{1\leq i'\leq d' \\ 1\leq j'\leq d'}}$



\[\int_a^b{\mathbb{R}^2}g(u, v)\dd{P_{XY}}(u, v)=\iint g(u,v) f_{XY}(u, v)\dd \lambda(u) \dd \lambda(v)\]
$$\lim_{x\to\infty} f(x)$$	
$$\iiiint_V \mu(t,u,v,w) \,dt\,du\,dv\,dw$$
$$\sum_{n=1}^{\infty} 2^{-n} = 1$$	
\begin{definition}
	Si $X$ et $Y$ sont 2 v.a. ou definit la \textsc{Covariance} entre $X$ et $Y$ comme
	$\cov(X,Y)\overset{\text{def}}{=}\E\left[(X-\E(X))(Y-\E(Y))\right]=\E(XY)-\E(X)\E(Y)$.
\end{definition}
\fi
\pagebreak

% \tableofcontents

% insert your code here
%\input{./algebra/main.tex}
%\input{./geometrie-differentielle/main.tex}
%\input{./probabilite/main.tex}
%\input{./analyse-fonctionnelle/main.tex}
% \input{./Analyse-convexe-et-dualite-en-optimisation/main.tex}
%\input{./tikz/main.tex}
%\input{./Theorie-du-distributions/main.tex}
%\input{./optimisation/mine.tex}
 \input{./modelisation/main.tex}

% yves.aubry@univ-tln.fr : algebra

\end{document}


% yves.aubry@univ-tln.fr : algebra

\end{document}

% % !TEX encoding = UTF-8 Unicode
% !TEX TS-program = xelatex

\documentclass[french]{report}

%\usepackage[utf8]{inputenc}
%\usepackage[T1]{fontenc}
\usepackage{babel}


\newif\ifcomment
%\commenttrue # Show comments

\usepackage{physics}
\usepackage{amssymb}


\usepackage{amsthm}
% \usepackage{thmtools}
\usepackage{mathtools}
\usepackage{amsfonts}

\usepackage{color}

\usepackage{tikz}

\usepackage{geometry}
\geometry{a5paper, margin=0.1in, right=1cm}

\usepackage{dsfont}

\usepackage{graphicx}
\graphicspath{ {images/} }

\usepackage{faktor}

\usepackage{IEEEtrantools}
\usepackage{enumerate}   
\usepackage[PostScript=dvips]{"/Users/aware/Documents/Courses/diagrams"}


\newtheorem{theorem}{Théorème}[section]
\renewcommand{\thetheorem}{\arabic{theorem}}
\newtheorem{lemme}{Lemme}[section]
\renewcommand{\thelemme}{\arabic{lemme}}
\newtheorem{proposition}{Proposition}[section]
\renewcommand{\theproposition}{\arabic{proposition}}
\newtheorem{notations}{Notations}[section]
\newtheorem{problem}{Problème}[section]
\newtheorem{corollary}{Corollaire}[theorem]
\renewcommand{\thecorollary}{\arabic{corollary}}
\newtheorem{property}{Propriété}[section]
\newtheorem{objective}{Objectif}[section]

\theoremstyle{definition}
\newtheorem{definition}{Définition}[section]
\renewcommand{\thedefinition}{\arabic{definition}}
\newtheorem{exercise}{Exercice}[chapter]
\renewcommand{\theexercise}{\arabic{exercise}}
\newtheorem{example}{Exemple}[chapter]
\renewcommand{\theexample}{\arabic{example}}
\newtheorem*{solution}{Solution}
\newtheorem*{application}{Application}
\newtheorem*{notation}{Notation}
\newtheorem*{vocabulary}{Vocabulaire}
\newtheorem*{properties}{Propriétés}



\theoremstyle{remark}
\newtheorem*{remark}{Remarque}
\newtheorem*{rappel}{Rappel}


\usepackage{etoolbox}
\AtBeginEnvironment{exercise}{\small}
\AtBeginEnvironment{example}{\small}

\usepackage{cases}
\usepackage[red]{mypack}

\usepackage[framemethod=TikZ]{mdframed}

\definecolor{bg}{rgb}{0.4,0.25,0.95}
\definecolor{pagebg}{rgb}{0,0,0.5}
\surroundwithmdframed[
   topline=false,
   rightline=false,
   bottomline=false,
   leftmargin=\parindent,
   skipabove=8pt,
   skipbelow=8pt,
   linecolor=blue,
   innerbottommargin=10pt,
   % backgroundcolor=bg,font=\color{orange}\sffamily, fontcolor=white
]{definition}

\usepackage{empheq}
\usepackage[most]{tcolorbox}

\newtcbox{\mymath}[1][]{%
    nobeforeafter, math upper, tcbox raise base,
    enhanced, colframe=blue!30!black,
    colback=red!10, boxrule=1pt,
    #1}

\usepackage{unixode}


\DeclareMathOperator{\ord}{ord}
\DeclareMathOperator{\orb}{orb}
\DeclareMathOperator{\stab}{stab}
\DeclareMathOperator{\Stab}{stab}
\DeclareMathOperator{\ppcm}{ppcm}
\DeclareMathOperator{\conj}{Conj}
\DeclareMathOperator{\End}{End}
\DeclareMathOperator{\rot}{rot}
\DeclareMathOperator{\trs}{trace}
\DeclareMathOperator{\Ind}{Ind}
\DeclareMathOperator{\mat}{Mat}
\DeclareMathOperator{\id}{Id}
\DeclareMathOperator{\vect}{vect}
\DeclareMathOperator{\img}{img}
\DeclareMathOperator{\cov}{Cov}
\DeclareMathOperator{\dist}{dist}
\DeclareMathOperator{\irr}{Irr}
\DeclareMathOperator{\image}{Im}
\DeclareMathOperator{\pd}{\partial}
\DeclareMathOperator{\epi}{epi}
\DeclareMathOperator{\Argmin}{Argmin}
\DeclareMathOperator{\dom}{dom}
\DeclareMathOperator{\proj}{proj}
\DeclareMathOperator{\ctg}{ctg}
\DeclareMathOperator{\supp}{supp}
\DeclareMathOperator{\argmin}{argmin}
\DeclareMathOperator{\mult}{mult}
\DeclareMathOperator{\ch}{ch}
\DeclareMathOperator{\sh}{sh}
\DeclareMathOperator{\rang}{rang}
\DeclareMathOperator{\diam}{diam}
\DeclareMathOperator{\Epigraphe}{Epigraphe}




\usepackage{xcolor}
\everymath{\color{blue}}
%\everymath{\color[rgb]{0,1,1}}
%\pagecolor[rgb]{0,0,0.5}


\newcommand*{\pdtest}[3][]{\ensuremath{\frac{\partial^{#1} #2}{\partial #3}}}

\newcommand*{\deffunc}[6][]{\ensuremath{
\begin{array}{rcl}
#2 : #3 &\rightarrow& #4\\
#5 &\mapsto& #6
\end{array}
}}

\newcommand{\eqcolon}{\mathrel{\resizebox{\widthof{$\mathord{=}$}}{\height}{ $\!\!=\!\!\resizebox{1.2\width}{0.8\height}{\raisebox{0.23ex}{$\mathop{:}$}}\!\!$ }}}
\newcommand{\coloneq}{\mathrel{\resizebox{\widthof{$\mathord{=}$}}{\height}{ $\!\!\resizebox{1.2\width}{0.8\height}{\raisebox{0.23ex}{$\mathop{:}$}}\!\!=\!\!$ }}}
\newcommand{\eqcolonl}{\ensuremath{\mathrel{=\!\!\mathop{:}}}}
\newcommand{\coloneql}{\ensuremath{\mathrel{\mathop{:} \!\! =}}}
\newcommand{\vc}[1]{% inline column vector
  \left(\begin{smallmatrix}#1\end{smallmatrix}\right)%
}
\newcommand{\vr}[1]{% inline row vector
  \begin{smallmatrix}(\,#1\,)\end{smallmatrix}%
}
\makeatletter
\newcommand*{\defeq}{\ =\mathrel{\rlap{%
                     \raisebox{0.3ex}{$\m@th\cdot$}}%
                     \raisebox{-0.3ex}{$\m@th\cdot$}}%
                     }
\makeatother

\newcommand{\mathcircle}[1]{% inline row vector
 \overset{\circ}{#1}
}
\newcommand{\ulim}{% low limit
 \underline{\lim}
}
\newcommand{\ssi}{% iff
\iff
}
\newcommand{\ps}[2]{
\expval{#1 | #2}
}
\newcommand{\df}[1]{
\mqty{#1}
}
\newcommand{\n}[1]{
\norm{#1}
}
\newcommand{\sys}[1]{
\left\{\smqty{#1}\right.
}


\newcommand{\eqdef}{\ensuremath{\overset{\text{def}}=}}


\def\Circlearrowright{\ensuremath{%
  \rotatebox[origin=c]{230}{$\circlearrowright$}}}

\newcommand\ct[1]{\text{\rmfamily\upshape #1}}
\newcommand\question[1]{ {\color{red} ...!? \small #1}}
\newcommand\caz[1]{\left\{\begin{array} #1 \end{array}\right.}
\newcommand\const{\text{\rmfamily\upshape const}}
\newcommand\toP{ \overset{\pro}{\to}}
\newcommand\toPP{ \overset{\text{PP}}{\to}}
\newcommand{\oeq}{\mathrel{\text{\textcircled{$=$}}}}





\usepackage{xcolor}
% \usepackage[normalem]{ulem}
\usepackage{lipsum}
\makeatletter
% \newcommand\colorwave[1][blue]{\bgroup \markoverwith{\lower3.5\p@\hbox{\sixly \textcolor{#1}{\char58}}}\ULon}
%\font\sixly=lasy6 % does not re-load if already loaded, so no memory problem.

\newmdtheoremenv[
linewidth= 1pt,linecolor= blue,%
leftmargin=20,rightmargin=20,innertopmargin=0pt, innerrightmargin=40,%
tikzsetting = { draw=lightgray, line width = 0.3pt,dashed,%
dash pattern = on 15pt off 3pt},%
splittopskip=\topskip,skipbelow=\baselineskip,%
skipabove=\baselineskip,ntheorem,roundcorner=0pt,
% backgroundcolor=pagebg,font=\color{orange}\sffamily, fontcolor=white
]{examplebox}{Exemple}[section]



\newcommand\R{\mathbb{R}}
\newcommand\Z{\mathbb{Z}}
\newcommand\N{\mathbb{N}}
\newcommand\E{\mathbb{E}}
\newcommand\F{\mathcal{F}}
\newcommand\cH{\mathcal{H}}
\newcommand\V{\mathbb{V}}
\newcommand\dmo{ ^{-1} }
\newcommand\kapa{\kappa}
\newcommand\im{Im}
\newcommand\hs{\mathcal{H}}





\usepackage{soul}

\makeatletter
\newcommand*{\whiten}[1]{\llap{\textcolor{white}{{\the\SOUL@token}}\hspace{#1pt}}}
\DeclareRobustCommand*\myul{%
    \def\SOUL@everyspace{\underline{\space}\kern\z@}%
    \def\SOUL@everytoken{%
     \setbox0=\hbox{\the\SOUL@token}%
     \ifdim\dp0>\z@
        \raisebox{\dp0}{\underline{\phantom{\the\SOUL@token}}}%
        \whiten{1}\whiten{0}%
        \whiten{-1}\whiten{-2}%
        \llap{\the\SOUL@token}%
     \else
        \underline{\the\SOUL@token}%
     \fi}%
\SOUL@}
\makeatother

\newcommand*{\demp}{\fontfamily{lmtt}\selectfont}

\DeclareTextFontCommand{\textdemp}{\demp}

\begin{document}

\ifcomment
Multiline
comment
\fi
\ifcomment
\myul{Typesetting test}
% \color[rgb]{1,1,1}
$∑_i^n≠ 60º±∞π∆¬≈√j∫h≤≥µ$

$\CR \R\pro\ind\pro\gS\pro
\mqty[a&b\\c&d]$
$\pro\mathbb{P}$
$\dd{x}$

  \[
    \alpha(x)=\left\{
                \begin{array}{ll}
                  x\\
                  \frac{1}{1+e^{-kx}}\\
                  \frac{e^x-e^{-x}}{e^x+e^{-x}}
                \end{array}
              \right.
  \]

  $\expval{x}$
  
  $\chi_\rho(ghg\dmo)=\Tr(\rho_{ghg\dmo})=\Tr(\rho_g\circ\rho_h\circ\rho\dmo_g)=\Tr(\rho_h)\overset{\mbox{\scalebox{0.5}{$\Tr(AB)=\Tr(BA)$}}}{=}\chi_\rho(h)$
  	$\mathop{\oplus}_{\substack{x\in X}}$

$\mat(\rho_g)=(a_{ij}(g))_{\scriptsize \substack{1\leq i\leq d \\ 1\leq j\leq d}}$ et $\mat(\rho'_g)=(a'_{ij}(g))_{\scriptsize \substack{1\leq i'\leq d' \\ 1\leq j'\leq d'}}$



\[\int_a^b{\mathbb{R}^2}g(u, v)\dd{P_{XY}}(u, v)=\iint g(u,v) f_{XY}(u, v)\dd \lambda(u) \dd \lambda(v)\]
$$\lim_{x\to\infty} f(x)$$	
$$\iiiint_V \mu(t,u,v,w) \,dt\,du\,dv\,dw$$
$$\sum_{n=1}^{\infty} 2^{-n} = 1$$	
\begin{definition}
	Si $X$ et $Y$ sont 2 v.a. ou definit la \textsc{Covariance} entre $X$ et $Y$ comme
	$\cov(X,Y)\overset{\text{def}}{=}\E\left[(X-\E(X))(Y-\E(Y))\right]=\E(XY)-\E(X)\E(Y)$.
\end{definition}
\fi
\pagebreak

% \tableofcontents

% insert your code here
%% !TEX encoding = UTF-8 Unicode
% !TEX TS-program = xelatex

\documentclass[french]{report}

%\usepackage[utf8]{inputenc}
%\usepackage[T1]{fontenc}
\usepackage{babel}


\newif\ifcomment
%\commenttrue # Show comments

\usepackage{physics}
\usepackage{amssymb}


\usepackage{amsthm}
% \usepackage{thmtools}
\usepackage{mathtools}
\usepackage{amsfonts}

\usepackage{color}

\usepackage{tikz}

\usepackage{geometry}
\geometry{a5paper, margin=0.1in, right=1cm}

\usepackage{dsfont}

\usepackage{graphicx}
\graphicspath{ {images/} }

\usepackage{faktor}

\usepackage{IEEEtrantools}
\usepackage{enumerate}   
\usepackage[PostScript=dvips]{"/Users/aware/Documents/Courses/diagrams"}


\newtheorem{theorem}{Théorème}[section]
\renewcommand{\thetheorem}{\arabic{theorem}}
\newtheorem{lemme}{Lemme}[section]
\renewcommand{\thelemme}{\arabic{lemme}}
\newtheorem{proposition}{Proposition}[section]
\renewcommand{\theproposition}{\arabic{proposition}}
\newtheorem{notations}{Notations}[section]
\newtheorem{problem}{Problème}[section]
\newtheorem{corollary}{Corollaire}[theorem]
\renewcommand{\thecorollary}{\arabic{corollary}}
\newtheorem{property}{Propriété}[section]
\newtheorem{objective}{Objectif}[section]

\theoremstyle{definition}
\newtheorem{definition}{Définition}[section]
\renewcommand{\thedefinition}{\arabic{definition}}
\newtheorem{exercise}{Exercice}[chapter]
\renewcommand{\theexercise}{\arabic{exercise}}
\newtheorem{example}{Exemple}[chapter]
\renewcommand{\theexample}{\arabic{example}}
\newtheorem*{solution}{Solution}
\newtheorem*{application}{Application}
\newtheorem*{notation}{Notation}
\newtheorem*{vocabulary}{Vocabulaire}
\newtheorem*{properties}{Propriétés}



\theoremstyle{remark}
\newtheorem*{remark}{Remarque}
\newtheorem*{rappel}{Rappel}


\usepackage{etoolbox}
\AtBeginEnvironment{exercise}{\small}
\AtBeginEnvironment{example}{\small}

\usepackage{cases}
\usepackage[red]{mypack}

\usepackage[framemethod=TikZ]{mdframed}

\definecolor{bg}{rgb}{0.4,0.25,0.95}
\definecolor{pagebg}{rgb}{0,0,0.5}
\surroundwithmdframed[
   topline=false,
   rightline=false,
   bottomline=false,
   leftmargin=\parindent,
   skipabove=8pt,
   skipbelow=8pt,
   linecolor=blue,
   innerbottommargin=10pt,
   % backgroundcolor=bg,font=\color{orange}\sffamily, fontcolor=white
]{definition}

\usepackage{empheq}
\usepackage[most]{tcolorbox}

\newtcbox{\mymath}[1][]{%
    nobeforeafter, math upper, tcbox raise base,
    enhanced, colframe=blue!30!black,
    colback=red!10, boxrule=1pt,
    #1}

\usepackage{unixode}


\DeclareMathOperator{\ord}{ord}
\DeclareMathOperator{\orb}{orb}
\DeclareMathOperator{\stab}{stab}
\DeclareMathOperator{\Stab}{stab}
\DeclareMathOperator{\ppcm}{ppcm}
\DeclareMathOperator{\conj}{Conj}
\DeclareMathOperator{\End}{End}
\DeclareMathOperator{\rot}{rot}
\DeclareMathOperator{\trs}{trace}
\DeclareMathOperator{\Ind}{Ind}
\DeclareMathOperator{\mat}{Mat}
\DeclareMathOperator{\id}{Id}
\DeclareMathOperator{\vect}{vect}
\DeclareMathOperator{\img}{img}
\DeclareMathOperator{\cov}{Cov}
\DeclareMathOperator{\dist}{dist}
\DeclareMathOperator{\irr}{Irr}
\DeclareMathOperator{\image}{Im}
\DeclareMathOperator{\pd}{\partial}
\DeclareMathOperator{\epi}{epi}
\DeclareMathOperator{\Argmin}{Argmin}
\DeclareMathOperator{\dom}{dom}
\DeclareMathOperator{\proj}{proj}
\DeclareMathOperator{\ctg}{ctg}
\DeclareMathOperator{\supp}{supp}
\DeclareMathOperator{\argmin}{argmin}
\DeclareMathOperator{\mult}{mult}
\DeclareMathOperator{\ch}{ch}
\DeclareMathOperator{\sh}{sh}
\DeclareMathOperator{\rang}{rang}
\DeclareMathOperator{\diam}{diam}
\DeclareMathOperator{\Epigraphe}{Epigraphe}




\usepackage{xcolor}
\everymath{\color{blue}}
%\everymath{\color[rgb]{0,1,1}}
%\pagecolor[rgb]{0,0,0.5}


\newcommand*{\pdtest}[3][]{\ensuremath{\frac{\partial^{#1} #2}{\partial #3}}}

\newcommand*{\deffunc}[6][]{\ensuremath{
\begin{array}{rcl}
#2 : #3 &\rightarrow& #4\\
#5 &\mapsto& #6
\end{array}
}}

\newcommand{\eqcolon}{\mathrel{\resizebox{\widthof{$\mathord{=}$}}{\height}{ $\!\!=\!\!\resizebox{1.2\width}{0.8\height}{\raisebox{0.23ex}{$\mathop{:}$}}\!\!$ }}}
\newcommand{\coloneq}{\mathrel{\resizebox{\widthof{$\mathord{=}$}}{\height}{ $\!\!\resizebox{1.2\width}{0.8\height}{\raisebox{0.23ex}{$\mathop{:}$}}\!\!=\!\!$ }}}
\newcommand{\eqcolonl}{\ensuremath{\mathrel{=\!\!\mathop{:}}}}
\newcommand{\coloneql}{\ensuremath{\mathrel{\mathop{:} \!\! =}}}
\newcommand{\vc}[1]{% inline column vector
  \left(\begin{smallmatrix}#1\end{smallmatrix}\right)%
}
\newcommand{\vr}[1]{% inline row vector
  \begin{smallmatrix}(\,#1\,)\end{smallmatrix}%
}
\makeatletter
\newcommand*{\defeq}{\ =\mathrel{\rlap{%
                     \raisebox{0.3ex}{$\m@th\cdot$}}%
                     \raisebox{-0.3ex}{$\m@th\cdot$}}%
                     }
\makeatother

\newcommand{\mathcircle}[1]{% inline row vector
 \overset{\circ}{#1}
}
\newcommand{\ulim}{% low limit
 \underline{\lim}
}
\newcommand{\ssi}{% iff
\iff
}
\newcommand{\ps}[2]{
\expval{#1 | #2}
}
\newcommand{\df}[1]{
\mqty{#1}
}
\newcommand{\n}[1]{
\norm{#1}
}
\newcommand{\sys}[1]{
\left\{\smqty{#1}\right.
}


\newcommand{\eqdef}{\ensuremath{\overset{\text{def}}=}}


\def\Circlearrowright{\ensuremath{%
  \rotatebox[origin=c]{230}{$\circlearrowright$}}}

\newcommand\ct[1]{\text{\rmfamily\upshape #1}}
\newcommand\question[1]{ {\color{red} ...!? \small #1}}
\newcommand\caz[1]{\left\{\begin{array} #1 \end{array}\right.}
\newcommand\const{\text{\rmfamily\upshape const}}
\newcommand\toP{ \overset{\pro}{\to}}
\newcommand\toPP{ \overset{\text{PP}}{\to}}
\newcommand{\oeq}{\mathrel{\text{\textcircled{$=$}}}}





\usepackage{xcolor}
% \usepackage[normalem]{ulem}
\usepackage{lipsum}
\makeatletter
% \newcommand\colorwave[1][blue]{\bgroup \markoverwith{\lower3.5\p@\hbox{\sixly \textcolor{#1}{\char58}}}\ULon}
%\font\sixly=lasy6 % does not re-load if already loaded, so no memory problem.

\newmdtheoremenv[
linewidth= 1pt,linecolor= blue,%
leftmargin=20,rightmargin=20,innertopmargin=0pt, innerrightmargin=40,%
tikzsetting = { draw=lightgray, line width = 0.3pt,dashed,%
dash pattern = on 15pt off 3pt},%
splittopskip=\topskip,skipbelow=\baselineskip,%
skipabove=\baselineskip,ntheorem,roundcorner=0pt,
% backgroundcolor=pagebg,font=\color{orange}\sffamily, fontcolor=white
]{examplebox}{Exemple}[section]



\newcommand\R{\mathbb{R}}
\newcommand\Z{\mathbb{Z}}
\newcommand\N{\mathbb{N}}
\newcommand\E{\mathbb{E}}
\newcommand\F{\mathcal{F}}
\newcommand\cH{\mathcal{H}}
\newcommand\V{\mathbb{V}}
\newcommand\dmo{ ^{-1} }
\newcommand\kapa{\kappa}
\newcommand\im{Im}
\newcommand\hs{\mathcal{H}}





\usepackage{soul}

\makeatletter
\newcommand*{\whiten}[1]{\llap{\textcolor{white}{{\the\SOUL@token}}\hspace{#1pt}}}
\DeclareRobustCommand*\myul{%
    \def\SOUL@everyspace{\underline{\space}\kern\z@}%
    \def\SOUL@everytoken{%
     \setbox0=\hbox{\the\SOUL@token}%
     \ifdim\dp0>\z@
        \raisebox{\dp0}{\underline{\phantom{\the\SOUL@token}}}%
        \whiten{1}\whiten{0}%
        \whiten{-1}\whiten{-2}%
        \llap{\the\SOUL@token}%
     \else
        \underline{\the\SOUL@token}%
     \fi}%
\SOUL@}
\makeatother

\newcommand*{\demp}{\fontfamily{lmtt}\selectfont}

\DeclareTextFontCommand{\textdemp}{\demp}

\begin{document}

\ifcomment
Multiline
comment
\fi
\ifcomment
\myul{Typesetting test}
% \color[rgb]{1,1,1}
$∑_i^n≠ 60º±∞π∆¬≈√j∫h≤≥µ$

$\CR \R\pro\ind\pro\gS\pro
\mqty[a&b\\c&d]$
$\pro\mathbb{P}$
$\dd{x}$

  \[
    \alpha(x)=\left\{
                \begin{array}{ll}
                  x\\
                  \frac{1}{1+e^{-kx}}\\
                  \frac{e^x-e^{-x}}{e^x+e^{-x}}
                \end{array}
              \right.
  \]

  $\expval{x}$
  
  $\chi_\rho(ghg\dmo)=\Tr(\rho_{ghg\dmo})=\Tr(\rho_g\circ\rho_h\circ\rho\dmo_g)=\Tr(\rho_h)\overset{\mbox{\scalebox{0.5}{$\Tr(AB)=\Tr(BA)$}}}{=}\chi_\rho(h)$
  	$\mathop{\oplus}_{\substack{x\in X}}$

$\mat(\rho_g)=(a_{ij}(g))_{\scriptsize \substack{1\leq i\leq d \\ 1\leq j\leq d}}$ et $\mat(\rho'_g)=(a'_{ij}(g))_{\scriptsize \substack{1\leq i'\leq d' \\ 1\leq j'\leq d'}}$



\[\int_a^b{\mathbb{R}^2}g(u, v)\dd{P_{XY}}(u, v)=\iint g(u,v) f_{XY}(u, v)\dd \lambda(u) \dd \lambda(v)\]
$$\lim_{x\to\infty} f(x)$$	
$$\iiiint_V \mu(t,u,v,w) \,dt\,du\,dv\,dw$$
$$\sum_{n=1}^{\infty} 2^{-n} = 1$$	
\begin{definition}
	Si $X$ et $Y$ sont 2 v.a. ou definit la \textsc{Covariance} entre $X$ et $Y$ comme
	$\cov(X,Y)\overset{\text{def}}{=}\E\left[(X-\E(X))(Y-\E(Y))\right]=\E(XY)-\E(X)\E(Y)$.
\end{definition}
\fi
\pagebreak

% \tableofcontents

% insert your code here
%\input{./algebra/main.tex}
%\input{./geometrie-differentielle/main.tex}
%\input{./probabilite/main.tex}
%\input{./analyse-fonctionnelle/main.tex}
% \input{./Analyse-convexe-et-dualite-en-optimisation/main.tex}
%\input{./tikz/main.tex}
%\input{./Theorie-du-distributions/main.tex}
%\input{./optimisation/mine.tex}
 \input{./modelisation/main.tex}

% yves.aubry@univ-tln.fr : algebra

\end{document}

%% !TEX encoding = UTF-8 Unicode
% !TEX TS-program = xelatex

\documentclass[french]{report}

%\usepackage[utf8]{inputenc}
%\usepackage[T1]{fontenc}
\usepackage{babel}


\newif\ifcomment
%\commenttrue # Show comments

\usepackage{physics}
\usepackage{amssymb}


\usepackage{amsthm}
% \usepackage{thmtools}
\usepackage{mathtools}
\usepackage{amsfonts}

\usepackage{color}

\usepackage{tikz}

\usepackage{geometry}
\geometry{a5paper, margin=0.1in, right=1cm}

\usepackage{dsfont}

\usepackage{graphicx}
\graphicspath{ {images/} }

\usepackage{faktor}

\usepackage{IEEEtrantools}
\usepackage{enumerate}   
\usepackage[PostScript=dvips]{"/Users/aware/Documents/Courses/diagrams"}


\newtheorem{theorem}{Théorème}[section]
\renewcommand{\thetheorem}{\arabic{theorem}}
\newtheorem{lemme}{Lemme}[section]
\renewcommand{\thelemme}{\arabic{lemme}}
\newtheorem{proposition}{Proposition}[section]
\renewcommand{\theproposition}{\arabic{proposition}}
\newtheorem{notations}{Notations}[section]
\newtheorem{problem}{Problème}[section]
\newtheorem{corollary}{Corollaire}[theorem]
\renewcommand{\thecorollary}{\arabic{corollary}}
\newtheorem{property}{Propriété}[section]
\newtheorem{objective}{Objectif}[section]

\theoremstyle{definition}
\newtheorem{definition}{Définition}[section]
\renewcommand{\thedefinition}{\arabic{definition}}
\newtheorem{exercise}{Exercice}[chapter]
\renewcommand{\theexercise}{\arabic{exercise}}
\newtheorem{example}{Exemple}[chapter]
\renewcommand{\theexample}{\arabic{example}}
\newtheorem*{solution}{Solution}
\newtheorem*{application}{Application}
\newtheorem*{notation}{Notation}
\newtheorem*{vocabulary}{Vocabulaire}
\newtheorem*{properties}{Propriétés}



\theoremstyle{remark}
\newtheorem*{remark}{Remarque}
\newtheorem*{rappel}{Rappel}


\usepackage{etoolbox}
\AtBeginEnvironment{exercise}{\small}
\AtBeginEnvironment{example}{\small}

\usepackage{cases}
\usepackage[red]{mypack}

\usepackage[framemethod=TikZ]{mdframed}

\definecolor{bg}{rgb}{0.4,0.25,0.95}
\definecolor{pagebg}{rgb}{0,0,0.5}
\surroundwithmdframed[
   topline=false,
   rightline=false,
   bottomline=false,
   leftmargin=\parindent,
   skipabove=8pt,
   skipbelow=8pt,
   linecolor=blue,
   innerbottommargin=10pt,
   % backgroundcolor=bg,font=\color{orange}\sffamily, fontcolor=white
]{definition}

\usepackage{empheq}
\usepackage[most]{tcolorbox}

\newtcbox{\mymath}[1][]{%
    nobeforeafter, math upper, tcbox raise base,
    enhanced, colframe=blue!30!black,
    colback=red!10, boxrule=1pt,
    #1}

\usepackage{unixode}


\DeclareMathOperator{\ord}{ord}
\DeclareMathOperator{\orb}{orb}
\DeclareMathOperator{\stab}{stab}
\DeclareMathOperator{\Stab}{stab}
\DeclareMathOperator{\ppcm}{ppcm}
\DeclareMathOperator{\conj}{Conj}
\DeclareMathOperator{\End}{End}
\DeclareMathOperator{\rot}{rot}
\DeclareMathOperator{\trs}{trace}
\DeclareMathOperator{\Ind}{Ind}
\DeclareMathOperator{\mat}{Mat}
\DeclareMathOperator{\id}{Id}
\DeclareMathOperator{\vect}{vect}
\DeclareMathOperator{\img}{img}
\DeclareMathOperator{\cov}{Cov}
\DeclareMathOperator{\dist}{dist}
\DeclareMathOperator{\irr}{Irr}
\DeclareMathOperator{\image}{Im}
\DeclareMathOperator{\pd}{\partial}
\DeclareMathOperator{\epi}{epi}
\DeclareMathOperator{\Argmin}{Argmin}
\DeclareMathOperator{\dom}{dom}
\DeclareMathOperator{\proj}{proj}
\DeclareMathOperator{\ctg}{ctg}
\DeclareMathOperator{\supp}{supp}
\DeclareMathOperator{\argmin}{argmin}
\DeclareMathOperator{\mult}{mult}
\DeclareMathOperator{\ch}{ch}
\DeclareMathOperator{\sh}{sh}
\DeclareMathOperator{\rang}{rang}
\DeclareMathOperator{\diam}{diam}
\DeclareMathOperator{\Epigraphe}{Epigraphe}




\usepackage{xcolor}
\everymath{\color{blue}}
%\everymath{\color[rgb]{0,1,1}}
%\pagecolor[rgb]{0,0,0.5}


\newcommand*{\pdtest}[3][]{\ensuremath{\frac{\partial^{#1} #2}{\partial #3}}}

\newcommand*{\deffunc}[6][]{\ensuremath{
\begin{array}{rcl}
#2 : #3 &\rightarrow& #4\\
#5 &\mapsto& #6
\end{array}
}}

\newcommand{\eqcolon}{\mathrel{\resizebox{\widthof{$\mathord{=}$}}{\height}{ $\!\!=\!\!\resizebox{1.2\width}{0.8\height}{\raisebox{0.23ex}{$\mathop{:}$}}\!\!$ }}}
\newcommand{\coloneq}{\mathrel{\resizebox{\widthof{$\mathord{=}$}}{\height}{ $\!\!\resizebox{1.2\width}{0.8\height}{\raisebox{0.23ex}{$\mathop{:}$}}\!\!=\!\!$ }}}
\newcommand{\eqcolonl}{\ensuremath{\mathrel{=\!\!\mathop{:}}}}
\newcommand{\coloneql}{\ensuremath{\mathrel{\mathop{:} \!\! =}}}
\newcommand{\vc}[1]{% inline column vector
  \left(\begin{smallmatrix}#1\end{smallmatrix}\right)%
}
\newcommand{\vr}[1]{% inline row vector
  \begin{smallmatrix}(\,#1\,)\end{smallmatrix}%
}
\makeatletter
\newcommand*{\defeq}{\ =\mathrel{\rlap{%
                     \raisebox{0.3ex}{$\m@th\cdot$}}%
                     \raisebox{-0.3ex}{$\m@th\cdot$}}%
                     }
\makeatother

\newcommand{\mathcircle}[1]{% inline row vector
 \overset{\circ}{#1}
}
\newcommand{\ulim}{% low limit
 \underline{\lim}
}
\newcommand{\ssi}{% iff
\iff
}
\newcommand{\ps}[2]{
\expval{#1 | #2}
}
\newcommand{\df}[1]{
\mqty{#1}
}
\newcommand{\n}[1]{
\norm{#1}
}
\newcommand{\sys}[1]{
\left\{\smqty{#1}\right.
}


\newcommand{\eqdef}{\ensuremath{\overset{\text{def}}=}}


\def\Circlearrowright{\ensuremath{%
  \rotatebox[origin=c]{230}{$\circlearrowright$}}}

\newcommand\ct[1]{\text{\rmfamily\upshape #1}}
\newcommand\question[1]{ {\color{red} ...!? \small #1}}
\newcommand\caz[1]{\left\{\begin{array} #1 \end{array}\right.}
\newcommand\const{\text{\rmfamily\upshape const}}
\newcommand\toP{ \overset{\pro}{\to}}
\newcommand\toPP{ \overset{\text{PP}}{\to}}
\newcommand{\oeq}{\mathrel{\text{\textcircled{$=$}}}}





\usepackage{xcolor}
% \usepackage[normalem]{ulem}
\usepackage{lipsum}
\makeatletter
% \newcommand\colorwave[1][blue]{\bgroup \markoverwith{\lower3.5\p@\hbox{\sixly \textcolor{#1}{\char58}}}\ULon}
%\font\sixly=lasy6 % does not re-load if already loaded, so no memory problem.

\newmdtheoremenv[
linewidth= 1pt,linecolor= blue,%
leftmargin=20,rightmargin=20,innertopmargin=0pt, innerrightmargin=40,%
tikzsetting = { draw=lightgray, line width = 0.3pt,dashed,%
dash pattern = on 15pt off 3pt},%
splittopskip=\topskip,skipbelow=\baselineskip,%
skipabove=\baselineskip,ntheorem,roundcorner=0pt,
% backgroundcolor=pagebg,font=\color{orange}\sffamily, fontcolor=white
]{examplebox}{Exemple}[section]



\newcommand\R{\mathbb{R}}
\newcommand\Z{\mathbb{Z}}
\newcommand\N{\mathbb{N}}
\newcommand\E{\mathbb{E}}
\newcommand\F{\mathcal{F}}
\newcommand\cH{\mathcal{H}}
\newcommand\V{\mathbb{V}}
\newcommand\dmo{ ^{-1} }
\newcommand\kapa{\kappa}
\newcommand\im{Im}
\newcommand\hs{\mathcal{H}}





\usepackage{soul}

\makeatletter
\newcommand*{\whiten}[1]{\llap{\textcolor{white}{{\the\SOUL@token}}\hspace{#1pt}}}
\DeclareRobustCommand*\myul{%
    \def\SOUL@everyspace{\underline{\space}\kern\z@}%
    \def\SOUL@everytoken{%
     \setbox0=\hbox{\the\SOUL@token}%
     \ifdim\dp0>\z@
        \raisebox{\dp0}{\underline{\phantom{\the\SOUL@token}}}%
        \whiten{1}\whiten{0}%
        \whiten{-1}\whiten{-2}%
        \llap{\the\SOUL@token}%
     \else
        \underline{\the\SOUL@token}%
     \fi}%
\SOUL@}
\makeatother

\newcommand*{\demp}{\fontfamily{lmtt}\selectfont}

\DeclareTextFontCommand{\textdemp}{\demp}

\begin{document}

\ifcomment
Multiline
comment
\fi
\ifcomment
\myul{Typesetting test}
% \color[rgb]{1,1,1}
$∑_i^n≠ 60º±∞π∆¬≈√j∫h≤≥µ$

$\CR \R\pro\ind\pro\gS\pro
\mqty[a&b\\c&d]$
$\pro\mathbb{P}$
$\dd{x}$

  \[
    \alpha(x)=\left\{
                \begin{array}{ll}
                  x\\
                  \frac{1}{1+e^{-kx}}\\
                  \frac{e^x-e^{-x}}{e^x+e^{-x}}
                \end{array}
              \right.
  \]

  $\expval{x}$
  
  $\chi_\rho(ghg\dmo)=\Tr(\rho_{ghg\dmo})=\Tr(\rho_g\circ\rho_h\circ\rho\dmo_g)=\Tr(\rho_h)\overset{\mbox{\scalebox{0.5}{$\Tr(AB)=\Tr(BA)$}}}{=}\chi_\rho(h)$
  	$\mathop{\oplus}_{\substack{x\in X}}$

$\mat(\rho_g)=(a_{ij}(g))_{\scriptsize \substack{1\leq i\leq d \\ 1\leq j\leq d}}$ et $\mat(\rho'_g)=(a'_{ij}(g))_{\scriptsize \substack{1\leq i'\leq d' \\ 1\leq j'\leq d'}}$



\[\int_a^b{\mathbb{R}^2}g(u, v)\dd{P_{XY}}(u, v)=\iint g(u,v) f_{XY}(u, v)\dd \lambda(u) \dd \lambda(v)\]
$$\lim_{x\to\infty} f(x)$$	
$$\iiiint_V \mu(t,u,v,w) \,dt\,du\,dv\,dw$$
$$\sum_{n=1}^{\infty} 2^{-n} = 1$$	
\begin{definition}
	Si $X$ et $Y$ sont 2 v.a. ou definit la \textsc{Covariance} entre $X$ et $Y$ comme
	$\cov(X,Y)\overset{\text{def}}{=}\E\left[(X-\E(X))(Y-\E(Y))\right]=\E(XY)-\E(X)\E(Y)$.
\end{definition}
\fi
\pagebreak

% \tableofcontents

% insert your code here
%\input{./algebra/main.tex}
%\input{./geometrie-differentielle/main.tex}
%\input{./probabilite/main.tex}
%\input{./analyse-fonctionnelle/main.tex}
% \input{./Analyse-convexe-et-dualite-en-optimisation/main.tex}
%\input{./tikz/main.tex}
%\input{./Theorie-du-distributions/main.tex}
%\input{./optimisation/mine.tex}
 \input{./modelisation/main.tex}

% yves.aubry@univ-tln.fr : algebra

\end{document}

%% !TEX encoding = UTF-8 Unicode
% !TEX TS-program = xelatex

\documentclass[french]{report}

%\usepackage[utf8]{inputenc}
%\usepackage[T1]{fontenc}
\usepackage{babel}


\newif\ifcomment
%\commenttrue # Show comments

\usepackage{physics}
\usepackage{amssymb}


\usepackage{amsthm}
% \usepackage{thmtools}
\usepackage{mathtools}
\usepackage{amsfonts}

\usepackage{color}

\usepackage{tikz}

\usepackage{geometry}
\geometry{a5paper, margin=0.1in, right=1cm}

\usepackage{dsfont}

\usepackage{graphicx}
\graphicspath{ {images/} }

\usepackage{faktor}

\usepackage{IEEEtrantools}
\usepackage{enumerate}   
\usepackage[PostScript=dvips]{"/Users/aware/Documents/Courses/diagrams"}


\newtheorem{theorem}{Théorème}[section]
\renewcommand{\thetheorem}{\arabic{theorem}}
\newtheorem{lemme}{Lemme}[section]
\renewcommand{\thelemme}{\arabic{lemme}}
\newtheorem{proposition}{Proposition}[section]
\renewcommand{\theproposition}{\arabic{proposition}}
\newtheorem{notations}{Notations}[section]
\newtheorem{problem}{Problème}[section]
\newtheorem{corollary}{Corollaire}[theorem]
\renewcommand{\thecorollary}{\arabic{corollary}}
\newtheorem{property}{Propriété}[section]
\newtheorem{objective}{Objectif}[section]

\theoremstyle{definition}
\newtheorem{definition}{Définition}[section]
\renewcommand{\thedefinition}{\arabic{definition}}
\newtheorem{exercise}{Exercice}[chapter]
\renewcommand{\theexercise}{\arabic{exercise}}
\newtheorem{example}{Exemple}[chapter]
\renewcommand{\theexample}{\arabic{example}}
\newtheorem*{solution}{Solution}
\newtheorem*{application}{Application}
\newtheorem*{notation}{Notation}
\newtheorem*{vocabulary}{Vocabulaire}
\newtheorem*{properties}{Propriétés}



\theoremstyle{remark}
\newtheorem*{remark}{Remarque}
\newtheorem*{rappel}{Rappel}


\usepackage{etoolbox}
\AtBeginEnvironment{exercise}{\small}
\AtBeginEnvironment{example}{\small}

\usepackage{cases}
\usepackage[red]{mypack}

\usepackage[framemethod=TikZ]{mdframed}

\definecolor{bg}{rgb}{0.4,0.25,0.95}
\definecolor{pagebg}{rgb}{0,0,0.5}
\surroundwithmdframed[
   topline=false,
   rightline=false,
   bottomline=false,
   leftmargin=\parindent,
   skipabove=8pt,
   skipbelow=8pt,
   linecolor=blue,
   innerbottommargin=10pt,
   % backgroundcolor=bg,font=\color{orange}\sffamily, fontcolor=white
]{definition}

\usepackage{empheq}
\usepackage[most]{tcolorbox}

\newtcbox{\mymath}[1][]{%
    nobeforeafter, math upper, tcbox raise base,
    enhanced, colframe=blue!30!black,
    colback=red!10, boxrule=1pt,
    #1}

\usepackage{unixode}


\DeclareMathOperator{\ord}{ord}
\DeclareMathOperator{\orb}{orb}
\DeclareMathOperator{\stab}{stab}
\DeclareMathOperator{\Stab}{stab}
\DeclareMathOperator{\ppcm}{ppcm}
\DeclareMathOperator{\conj}{Conj}
\DeclareMathOperator{\End}{End}
\DeclareMathOperator{\rot}{rot}
\DeclareMathOperator{\trs}{trace}
\DeclareMathOperator{\Ind}{Ind}
\DeclareMathOperator{\mat}{Mat}
\DeclareMathOperator{\id}{Id}
\DeclareMathOperator{\vect}{vect}
\DeclareMathOperator{\img}{img}
\DeclareMathOperator{\cov}{Cov}
\DeclareMathOperator{\dist}{dist}
\DeclareMathOperator{\irr}{Irr}
\DeclareMathOperator{\image}{Im}
\DeclareMathOperator{\pd}{\partial}
\DeclareMathOperator{\epi}{epi}
\DeclareMathOperator{\Argmin}{Argmin}
\DeclareMathOperator{\dom}{dom}
\DeclareMathOperator{\proj}{proj}
\DeclareMathOperator{\ctg}{ctg}
\DeclareMathOperator{\supp}{supp}
\DeclareMathOperator{\argmin}{argmin}
\DeclareMathOperator{\mult}{mult}
\DeclareMathOperator{\ch}{ch}
\DeclareMathOperator{\sh}{sh}
\DeclareMathOperator{\rang}{rang}
\DeclareMathOperator{\diam}{diam}
\DeclareMathOperator{\Epigraphe}{Epigraphe}




\usepackage{xcolor}
\everymath{\color{blue}}
%\everymath{\color[rgb]{0,1,1}}
%\pagecolor[rgb]{0,0,0.5}


\newcommand*{\pdtest}[3][]{\ensuremath{\frac{\partial^{#1} #2}{\partial #3}}}

\newcommand*{\deffunc}[6][]{\ensuremath{
\begin{array}{rcl}
#2 : #3 &\rightarrow& #4\\
#5 &\mapsto& #6
\end{array}
}}

\newcommand{\eqcolon}{\mathrel{\resizebox{\widthof{$\mathord{=}$}}{\height}{ $\!\!=\!\!\resizebox{1.2\width}{0.8\height}{\raisebox{0.23ex}{$\mathop{:}$}}\!\!$ }}}
\newcommand{\coloneq}{\mathrel{\resizebox{\widthof{$\mathord{=}$}}{\height}{ $\!\!\resizebox{1.2\width}{0.8\height}{\raisebox{0.23ex}{$\mathop{:}$}}\!\!=\!\!$ }}}
\newcommand{\eqcolonl}{\ensuremath{\mathrel{=\!\!\mathop{:}}}}
\newcommand{\coloneql}{\ensuremath{\mathrel{\mathop{:} \!\! =}}}
\newcommand{\vc}[1]{% inline column vector
  \left(\begin{smallmatrix}#1\end{smallmatrix}\right)%
}
\newcommand{\vr}[1]{% inline row vector
  \begin{smallmatrix}(\,#1\,)\end{smallmatrix}%
}
\makeatletter
\newcommand*{\defeq}{\ =\mathrel{\rlap{%
                     \raisebox{0.3ex}{$\m@th\cdot$}}%
                     \raisebox{-0.3ex}{$\m@th\cdot$}}%
                     }
\makeatother

\newcommand{\mathcircle}[1]{% inline row vector
 \overset{\circ}{#1}
}
\newcommand{\ulim}{% low limit
 \underline{\lim}
}
\newcommand{\ssi}{% iff
\iff
}
\newcommand{\ps}[2]{
\expval{#1 | #2}
}
\newcommand{\df}[1]{
\mqty{#1}
}
\newcommand{\n}[1]{
\norm{#1}
}
\newcommand{\sys}[1]{
\left\{\smqty{#1}\right.
}


\newcommand{\eqdef}{\ensuremath{\overset{\text{def}}=}}


\def\Circlearrowright{\ensuremath{%
  \rotatebox[origin=c]{230}{$\circlearrowright$}}}

\newcommand\ct[1]{\text{\rmfamily\upshape #1}}
\newcommand\question[1]{ {\color{red} ...!? \small #1}}
\newcommand\caz[1]{\left\{\begin{array} #1 \end{array}\right.}
\newcommand\const{\text{\rmfamily\upshape const}}
\newcommand\toP{ \overset{\pro}{\to}}
\newcommand\toPP{ \overset{\text{PP}}{\to}}
\newcommand{\oeq}{\mathrel{\text{\textcircled{$=$}}}}





\usepackage{xcolor}
% \usepackage[normalem]{ulem}
\usepackage{lipsum}
\makeatletter
% \newcommand\colorwave[1][blue]{\bgroup \markoverwith{\lower3.5\p@\hbox{\sixly \textcolor{#1}{\char58}}}\ULon}
%\font\sixly=lasy6 % does not re-load if already loaded, so no memory problem.

\newmdtheoremenv[
linewidth= 1pt,linecolor= blue,%
leftmargin=20,rightmargin=20,innertopmargin=0pt, innerrightmargin=40,%
tikzsetting = { draw=lightgray, line width = 0.3pt,dashed,%
dash pattern = on 15pt off 3pt},%
splittopskip=\topskip,skipbelow=\baselineskip,%
skipabove=\baselineskip,ntheorem,roundcorner=0pt,
% backgroundcolor=pagebg,font=\color{orange}\sffamily, fontcolor=white
]{examplebox}{Exemple}[section]



\newcommand\R{\mathbb{R}}
\newcommand\Z{\mathbb{Z}}
\newcommand\N{\mathbb{N}}
\newcommand\E{\mathbb{E}}
\newcommand\F{\mathcal{F}}
\newcommand\cH{\mathcal{H}}
\newcommand\V{\mathbb{V}}
\newcommand\dmo{ ^{-1} }
\newcommand\kapa{\kappa}
\newcommand\im{Im}
\newcommand\hs{\mathcal{H}}





\usepackage{soul}

\makeatletter
\newcommand*{\whiten}[1]{\llap{\textcolor{white}{{\the\SOUL@token}}\hspace{#1pt}}}
\DeclareRobustCommand*\myul{%
    \def\SOUL@everyspace{\underline{\space}\kern\z@}%
    \def\SOUL@everytoken{%
     \setbox0=\hbox{\the\SOUL@token}%
     \ifdim\dp0>\z@
        \raisebox{\dp0}{\underline{\phantom{\the\SOUL@token}}}%
        \whiten{1}\whiten{0}%
        \whiten{-1}\whiten{-2}%
        \llap{\the\SOUL@token}%
     \else
        \underline{\the\SOUL@token}%
     \fi}%
\SOUL@}
\makeatother

\newcommand*{\demp}{\fontfamily{lmtt}\selectfont}

\DeclareTextFontCommand{\textdemp}{\demp}

\begin{document}

\ifcomment
Multiline
comment
\fi
\ifcomment
\myul{Typesetting test}
% \color[rgb]{1,1,1}
$∑_i^n≠ 60º±∞π∆¬≈√j∫h≤≥µ$

$\CR \R\pro\ind\pro\gS\pro
\mqty[a&b\\c&d]$
$\pro\mathbb{P}$
$\dd{x}$

  \[
    \alpha(x)=\left\{
                \begin{array}{ll}
                  x\\
                  \frac{1}{1+e^{-kx}}\\
                  \frac{e^x-e^{-x}}{e^x+e^{-x}}
                \end{array}
              \right.
  \]

  $\expval{x}$
  
  $\chi_\rho(ghg\dmo)=\Tr(\rho_{ghg\dmo})=\Tr(\rho_g\circ\rho_h\circ\rho\dmo_g)=\Tr(\rho_h)\overset{\mbox{\scalebox{0.5}{$\Tr(AB)=\Tr(BA)$}}}{=}\chi_\rho(h)$
  	$\mathop{\oplus}_{\substack{x\in X}}$

$\mat(\rho_g)=(a_{ij}(g))_{\scriptsize \substack{1\leq i\leq d \\ 1\leq j\leq d}}$ et $\mat(\rho'_g)=(a'_{ij}(g))_{\scriptsize \substack{1\leq i'\leq d' \\ 1\leq j'\leq d'}}$



\[\int_a^b{\mathbb{R}^2}g(u, v)\dd{P_{XY}}(u, v)=\iint g(u,v) f_{XY}(u, v)\dd \lambda(u) \dd \lambda(v)\]
$$\lim_{x\to\infty} f(x)$$	
$$\iiiint_V \mu(t,u,v,w) \,dt\,du\,dv\,dw$$
$$\sum_{n=1}^{\infty} 2^{-n} = 1$$	
\begin{definition}
	Si $X$ et $Y$ sont 2 v.a. ou definit la \textsc{Covariance} entre $X$ et $Y$ comme
	$\cov(X,Y)\overset{\text{def}}{=}\E\left[(X-\E(X))(Y-\E(Y))\right]=\E(XY)-\E(X)\E(Y)$.
\end{definition}
\fi
\pagebreak

% \tableofcontents

% insert your code here
%\input{./algebra/main.tex}
%\input{./geometrie-differentielle/main.tex}
%\input{./probabilite/main.tex}
%\input{./analyse-fonctionnelle/main.tex}
% \input{./Analyse-convexe-et-dualite-en-optimisation/main.tex}
%\input{./tikz/main.tex}
%\input{./Theorie-du-distributions/main.tex}
%\input{./optimisation/mine.tex}
 \input{./modelisation/main.tex}

% yves.aubry@univ-tln.fr : algebra

\end{document}

%% !TEX encoding = UTF-8 Unicode
% !TEX TS-program = xelatex

\documentclass[french]{report}

%\usepackage[utf8]{inputenc}
%\usepackage[T1]{fontenc}
\usepackage{babel}


\newif\ifcomment
%\commenttrue # Show comments

\usepackage{physics}
\usepackage{amssymb}


\usepackage{amsthm}
% \usepackage{thmtools}
\usepackage{mathtools}
\usepackage{amsfonts}

\usepackage{color}

\usepackage{tikz}

\usepackage{geometry}
\geometry{a5paper, margin=0.1in, right=1cm}

\usepackage{dsfont}

\usepackage{graphicx}
\graphicspath{ {images/} }

\usepackage{faktor}

\usepackage{IEEEtrantools}
\usepackage{enumerate}   
\usepackage[PostScript=dvips]{"/Users/aware/Documents/Courses/diagrams"}


\newtheorem{theorem}{Théorème}[section]
\renewcommand{\thetheorem}{\arabic{theorem}}
\newtheorem{lemme}{Lemme}[section]
\renewcommand{\thelemme}{\arabic{lemme}}
\newtheorem{proposition}{Proposition}[section]
\renewcommand{\theproposition}{\arabic{proposition}}
\newtheorem{notations}{Notations}[section]
\newtheorem{problem}{Problème}[section]
\newtheorem{corollary}{Corollaire}[theorem]
\renewcommand{\thecorollary}{\arabic{corollary}}
\newtheorem{property}{Propriété}[section]
\newtheorem{objective}{Objectif}[section]

\theoremstyle{definition}
\newtheorem{definition}{Définition}[section]
\renewcommand{\thedefinition}{\arabic{definition}}
\newtheorem{exercise}{Exercice}[chapter]
\renewcommand{\theexercise}{\arabic{exercise}}
\newtheorem{example}{Exemple}[chapter]
\renewcommand{\theexample}{\arabic{example}}
\newtheorem*{solution}{Solution}
\newtheorem*{application}{Application}
\newtheorem*{notation}{Notation}
\newtheorem*{vocabulary}{Vocabulaire}
\newtheorem*{properties}{Propriétés}



\theoremstyle{remark}
\newtheorem*{remark}{Remarque}
\newtheorem*{rappel}{Rappel}


\usepackage{etoolbox}
\AtBeginEnvironment{exercise}{\small}
\AtBeginEnvironment{example}{\small}

\usepackage{cases}
\usepackage[red]{mypack}

\usepackage[framemethod=TikZ]{mdframed}

\definecolor{bg}{rgb}{0.4,0.25,0.95}
\definecolor{pagebg}{rgb}{0,0,0.5}
\surroundwithmdframed[
   topline=false,
   rightline=false,
   bottomline=false,
   leftmargin=\parindent,
   skipabove=8pt,
   skipbelow=8pt,
   linecolor=blue,
   innerbottommargin=10pt,
   % backgroundcolor=bg,font=\color{orange}\sffamily, fontcolor=white
]{definition}

\usepackage{empheq}
\usepackage[most]{tcolorbox}

\newtcbox{\mymath}[1][]{%
    nobeforeafter, math upper, tcbox raise base,
    enhanced, colframe=blue!30!black,
    colback=red!10, boxrule=1pt,
    #1}

\usepackage{unixode}


\DeclareMathOperator{\ord}{ord}
\DeclareMathOperator{\orb}{orb}
\DeclareMathOperator{\stab}{stab}
\DeclareMathOperator{\Stab}{stab}
\DeclareMathOperator{\ppcm}{ppcm}
\DeclareMathOperator{\conj}{Conj}
\DeclareMathOperator{\End}{End}
\DeclareMathOperator{\rot}{rot}
\DeclareMathOperator{\trs}{trace}
\DeclareMathOperator{\Ind}{Ind}
\DeclareMathOperator{\mat}{Mat}
\DeclareMathOperator{\id}{Id}
\DeclareMathOperator{\vect}{vect}
\DeclareMathOperator{\img}{img}
\DeclareMathOperator{\cov}{Cov}
\DeclareMathOperator{\dist}{dist}
\DeclareMathOperator{\irr}{Irr}
\DeclareMathOperator{\image}{Im}
\DeclareMathOperator{\pd}{\partial}
\DeclareMathOperator{\epi}{epi}
\DeclareMathOperator{\Argmin}{Argmin}
\DeclareMathOperator{\dom}{dom}
\DeclareMathOperator{\proj}{proj}
\DeclareMathOperator{\ctg}{ctg}
\DeclareMathOperator{\supp}{supp}
\DeclareMathOperator{\argmin}{argmin}
\DeclareMathOperator{\mult}{mult}
\DeclareMathOperator{\ch}{ch}
\DeclareMathOperator{\sh}{sh}
\DeclareMathOperator{\rang}{rang}
\DeclareMathOperator{\diam}{diam}
\DeclareMathOperator{\Epigraphe}{Epigraphe}




\usepackage{xcolor}
\everymath{\color{blue}}
%\everymath{\color[rgb]{0,1,1}}
%\pagecolor[rgb]{0,0,0.5}


\newcommand*{\pdtest}[3][]{\ensuremath{\frac{\partial^{#1} #2}{\partial #3}}}

\newcommand*{\deffunc}[6][]{\ensuremath{
\begin{array}{rcl}
#2 : #3 &\rightarrow& #4\\
#5 &\mapsto& #6
\end{array}
}}

\newcommand{\eqcolon}{\mathrel{\resizebox{\widthof{$\mathord{=}$}}{\height}{ $\!\!=\!\!\resizebox{1.2\width}{0.8\height}{\raisebox{0.23ex}{$\mathop{:}$}}\!\!$ }}}
\newcommand{\coloneq}{\mathrel{\resizebox{\widthof{$\mathord{=}$}}{\height}{ $\!\!\resizebox{1.2\width}{0.8\height}{\raisebox{0.23ex}{$\mathop{:}$}}\!\!=\!\!$ }}}
\newcommand{\eqcolonl}{\ensuremath{\mathrel{=\!\!\mathop{:}}}}
\newcommand{\coloneql}{\ensuremath{\mathrel{\mathop{:} \!\! =}}}
\newcommand{\vc}[1]{% inline column vector
  \left(\begin{smallmatrix}#1\end{smallmatrix}\right)%
}
\newcommand{\vr}[1]{% inline row vector
  \begin{smallmatrix}(\,#1\,)\end{smallmatrix}%
}
\makeatletter
\newcommand*{\defeq}{\ =\mathrel{\rlap{%
                     \raisebox{0.3ex}{$\m@th\cdot$}}%
                     \raisebox{-0.3ex}{$\m@th\cdot$}}%
                     }
\makeatother

\newcommand{\mathcircle}[1]{% inline row vector
 \overset{\circ}{#1}
}
\newcommand{\ulim}{% low limit
 \underline{\lim}
}
\newcommand{\ssi}{% iff
\iff
}
\newcommand{\ps}[2]{
\expval{#1 | #2}
}
\newcommand{\df}[1]{
\mqty{#1}
}
\newcommand{\n}[1]{
\norm{#1}
}
\newcommand{\sys}[1]{
\left\{\smqty{#1}\right.
}


\newcommand{\eqdef}{\ensuremath{\overset{\text{def}}=}}


\def\Circlearrowright{\ensuremath{%
  \rotatebox[origin=c]{230}{$\circlearrowright$}}}

\newcommand\ct[1]{\text{\rmfamily\upshape #1}}
\newcommand\question[1]{ {\color{red} ...!? \small #1}}
\newcommand\caz[1]{\left\{\begin{array} #1 \end{array}\right.}
\newcommand\const{\text{\rmfamily\upshape const}}
\newcommand\toP{ \overset{\pro}{\to}}
\newcommand\toPP{ \overset{\text{PP}}{\to}}
\newcommand{\oeq}{\mathrel{\text{\textcircled{$=$}}}}





\usepackage{xcolor}
% \usepackage[normalem]{ulem}
\usepackage{lipsum}
\makeatletter
% \newcommand\colorwave[1][blue]{\bgroup \markoverwith{\lower3.5\p@\hbox{\sixly \textcolor{#1}{\char58}}}\ULon}
%\font\sixly=lasy6 % does not re-load if already loaded, so no memory problem.

\newmdtheoremenv[
linewidth= 1pt,linecolor= blue,%
leftmargin=20,rightmargin=20,innertopmargin=0pt, innerrightmargin=40,%
tikzsetting = { draw=lightgray, line width = 0.3pt,dashed,%
dash pattern = on 15pt off 3pt},%
splittopskip=\topskip,skipbelow=\baselineskip,%
skipabove=\baselineskip,ntheorem,roundcorner=0pt,
% backgroundcolor=pagebg,font=\color{orange}\sffamily, fontcolor=white
]{examplebox}{Exemple}[section]



\newcommand\R{\mathbb{R}}
\newcommand\Z{\mathbb{Z}}
\newcommand\N{\mathbb{N}}
\newcommand\E{\mathbb{E}}
\newcommand\F{\mathcal{F}}
\newcommand\cH{\mathcal{H}}
\newcommand\V{\mathbb{V}}
\newcommand\dmo{ ^{-1} }
\newcommand\kapa{\kappa}
\newcommand\im{Im}
\newcommand\hs{\mathcal{H}}





\usepackage{soul}

\makeatletter
\newcommand*{\whiten}[1]{\llap{\textcolor{white}{{\the\SOUL@token}}\hspace{#1pt}}}
\DeclareRobustCommand*\myul{%
    \def\SOUL@everyspace{\underline{\space}\kern\z@}%
    \def\SOUL@everytoken{%
     \setbox0=\hbox{\the\SOUL@token}%
     \ifdim\dp0>\z@
        \raisebox{\dp0}{\underline{\phantom{\the\SOUL@token}}}%
        \whiten{1}\whiten{0}%
        \whiten{-1}\whiten{-2}%
        \llap{\the\SOUL@token}%
     \else
        \underline{\the\SOUL@token}%
     \fi}%
\SOUL@}
\makeatother

\newcommand*{\demp}{\fontfamily{lmtt}\selectfont}

\DeclareTextFontCommand{\textdemp}{\demp}

\begin{document}

\ifcomment
Multiline
comment
\fi
\ifcomment
\myul{Typesetting test}
% \color[rgb]{1,1,1}
$∑_i^n≠ 60º±∞π∆¬≈√j∫h≤≥µ$

$\CR \R\pro\ind\pro\gS\pro
\mqty[a&b\\c&d]$
$\pro\mathbb{P}$
$\dd{x}$

  \[
    \alpha(x)=\left\{
                \begin{array}{ll}
                  x\\
                  \frac{1}{1+e^{-kx}}\\
                  \frac{e^x-e^{-x}}{e^x+e^{-x}}
                \end{array}
              \right.
  \]

  $\expval{x}$
  
  $\chi_\rho(ghg\dmo)=\Tr(\rho_{ghg\dmo})=\Tr(\rho_g\circ\rho_h\circ\rho\dmo_g)=\Tr(\rho_h)\overset{\mbox{\scalebox{0.5}{$\Tr(AB)=\Tr(BA)$}}}{=}\chi_\rho(h)$
  	$\mathop{\oplus}_{\substack{x\in X}}$

$\mat(\rho_g)=(a_{ij}(g))_{\scriptsize \substack{1\leq i\leq d \\ 1\leq j\leq d}}$ et $\mat(\rho'_g)=(a'_{ij}(g))_{\scriptsize \substack{1\leq i'\leq d' \\ 1\leq j'\leq d'}}$



\[\int_a^b{\mathbb{R}^2}g(u, v)\dd{P_{XY}}(u, v)=\iint g(u,v) f_{XY}(u, v)\dd \lambda(u) \dd \lambda(v)\]
$$\lim_{x\to\infty} f(x)$$	
$$\iiiint_V \mu(t,u,v,w) \,dt\,du\,dv\,dw$$
$$\sum_{n=1}^{\infty} 2^{-n} = 1$$	
\begin{definition}
	Si $X$ et $Y$ sont 2 v.a. ou definit la \textsc{Covariance} entre $X$ et $Y$ comme
	$\cov(X,Y)\overset{\text{def}}{=}\E\left[(X-\E(X))(Y-\E(Y))\right]=\E(XY)-\E(X)\E(Y)$.
\end{definition}
\fi
\pagebreak

% \tableofcontents

% insert your code here
%\input{./algebra/main.tex}
%\input{./geometrie-differentielle/main.tex}
%\input{./probabilite/main.tex}
%\input{./analyse-fonctionnelle/main.tex}
% \input{./Analyse-convexe-et-dualite-en-optimisation/main.tex}
%\input{./tikz/main.tex}
%\input{./Theorie-du-distributions/main.tex}
%\input{./optimisation/mine.tex}
 \input{./modelisation/main.tex}

% yves.aubry@univ-tln.fr : algebra

\end{document}

% % !TEX encoding = UTF-8 Unicode
% !TEX TS-program = xelatex

\documentclass[french]{report}

%\usepackage[utf8]{inputenc}
%\usepackage[T1]{fontenc}
\usepackage{babel}


\newif\ifcomment
%\commenttrue # Show comments

\usepackage{physics}
\usepackage{amssymb}


\usepackage{amsthm}
% \usepackage{thmtools}
\usepackage{mathtools}
\usepackage{amsfonts}

\usepackage{color}

\usepackage{tikz}

\usepackage{geometry}
\geometry{a5paper, margin=0.1in, right=1cm}

\usepackage{dsfont}

\usepackage{graphicx}
\graphicspath{ {images/} }

\usepackage{faktor}

\usepackage{IEEEtrantools}
\usepackage{enumerate}   
\usepackage[PostScript=dvips]{"/Users/aware/Documents/Courses/diagrams"}


\newtheorem{theorem}{Théorème}[section]
\renewcommand{\thetheorem}{\arabic{theorem}}
\newtheorem{lemme}{Lemme}[section]
\renewcommand{\thelemme}{\arabic{lemme}}
\newtheorem{proposition}{Proposition}[section]
\renewcommand{\theproposition}{\arabic{proposition}}
\newtheorem{notations}{Notations}[section]
\newtheorem{problem}{Problème}[section]
\newtheorem{corollary}{Corollaire}[theorem]
\renewcommand{\thecorollary}{\arabic{corollary}}
\newtheorem{property}{Propriété}[section]
\newtheorem{objective}{Objectif}[section]

\theoremstyle{definition}
\newtheorem{definition}{Définition}[section]
\renewcommand{\thedefinition}{\arabic{definition}}
\newtheorem{exercise}{Exercice}[chapter]
\renewcommand{\theexercise}{\arabic{exercise}}
\newtheorem{example}{Exemple}[chapter]
\renewcommand{\theexample}{\arabic{example}}
\newtheorem*{solution}{Solution}
\newtheorem*{application}{Application}
\newtheorem*{notation}{Notation}
\newtheorem*{vocabulary}{Vocabulaire}
\newtheorem*{properties}{Propriétés}



\theoremstyle{remark}
\newtheorem*{remark}{Remarque}
\newtheorem*{rappel}{Rappel}


\usepackage{etoolbox}
\AtBeginEnvironment{exercise}{\small}
\AtBeginEnvironment{example}{\small}

\usepackage{cases}
\usepackage[red]{mypack}

\usepackage[framemethod=TikZ]{mdframed}

\definecolor{bg}{rgb}{0.4,0.25,0.95}
\definecolor{pagebg}{rgb}{0,0,0.5}
\surroundwithmdframed[
   topline=false,
   rightline=false,
   bottomline=false,
   leftmargin=\parindent,
   skipabove=8pt,
   skipbelow=8pt,
   linecolor=blue,
   innerbottommargin=10pt,
   % backgroundcolor=bg,font=\color{orange}\sffamily, fontcolor=white
]{definition}

\usepackage{empheq}
\usepackage[most]{tcolorbox}

\newtcbox{\mymath}[1][]{%
    nobeforeafter, math upper, tcbox raise base,
    enhanced, colframe=blue!30!black,
    colback=red!10, boxrule=1pt,
    #1}

\usepackage{unixode}


\DeclareMathOperator{\ord}{ord}
\DeclareMathOperator{\orb}{orb}
\DeclareMathOperator{\stab}{stab}
\DeclareMathOperator{\Stab}{stab}
\DeclareMathOperator{\ppcm}{ppcm}
\DeclareMathOperator{\conj}{Conj}
\DeclareMathOperator{\End}{End}
\DeclareMathOperator{\rot}{rot}
\DeclareMathOperator{\trs}{trace}
\DeclareMathOperator{\Ind}{Ind}
\DeclareMathOperator{\mat}{Mat}
\DeclareMathOperator{\id}{Id}
\DeclareMathOperator{\vect}{vect}
\DeclareMathOperator{\img}{img}
\DeclareMathOperator{\cov}{Cov}
\DeclareMathOperator{\dist}{dist}
\DeclareMathOperator{\irr}{Irr}
\DeclareMathOperator{\image}{Im}
\DeclareMathOperator{\pd}{\partial}
\DeclareMathOperator{\epi}{epi}
\DeclareMathOperator{\Argmin}{Argmin}
\DeclareMathOperator{\dom}{dom}
\DeclareMathOperator{\proj}{proj}
\DeclareMathOperator{\ctg}{ctg}
\DeclareMathOperator{\supp}{supp}
\DeclareMathOperator{\argmin}{argmin}
\DeclareMathOperator{\mult}{mult}
\DeclareMathOperator{\ch}{ch}
\DeclareMathOperator{\sh}{sh}
\DeclareMathOperator{\rang}{rang}
\DeclareMathOperator{\diam}{diam}
\DeclareMathOperator{\Epigraphe}{Epigraphe}




\usepackage{xcolor}
\everymath{\color{blue}}
%\everymath{\color[rgb]{0,1,1}}
%\pagecolor[rgb]{0,0,0.5}


\newcommand*{\pdtest}[3][]{\ensuremath{\frac{\partial^{#1} #2}{\partial #3}}}

\newcommand*{\deffunc}[6][]{\ensuremath{
\begin{array}{rcl}
#2 : #3 &\rightarrow& #4\\
#5 &\mapsto& #6
\end{array}
}}

\newcommand{\eqcolon}{\mathrel{\resizebox{\widthof{$\mathord{=}$}}{\height}{ $\!\!=\!\!\resizebox{1.2\width}{0.8\height}{\raisebox{0.23ex}{$\mathop{:}$}}\!\!$ }}}
\newcommand{\coloneq}{\mathrel{\resizebox{\widthof{$\mathord{=}$}}{\height}{ $\!\!\resizebox{1.2\width}{0.8\height}{\raisebox{0.23ex}{$\mathop{:}$}}\!\!=\!\!$ }}}
\newcommand{\eqcolonl}{\ensuremath{\mathrel{=\!\!\mathop{:}}}}
\newcommand{\coloneql}{\ensuremath{\mathrel{\mathop{:} \!\! =}}}
\newcommand{\vc}[1]{% inline column vector
  \left(\begin{smallmatrix}#1\end{smallmatrix}\right)%
}
\newcommand{\vr}[1]{% inline row vector
  \begin{smallmatrix}(\,#1\,)\end{smallmatrix}%
}
\makeatletter
\newcommand*{\defeq}{\ =\mathrel{\rlap{%
                     \raisebox{0.3ex}{$\m@th\cdot$}}%
                     \raisebox{-0.3ex}{$\m@th\cdot$}}%
                     }
\makeatother

\newcommand{\mathcircle}[1]{% inline row vector
 \overset{\circ}{#1}
}
\newcommand{\ulim}{% low limit
 \underline{\lim}
}
\newcommand{\ssi}{% iff
\iff
}
\newcommand{\ps}[2]{
\expval{#1 | #2}
}
\newcommand{\df}[1]{
\mqty{#1}
}
\newcommand{\n}[1]{
\norm{#1}
}
\newcommand{\sys}[1]{
\left\{\smqty{#1}\right.
}


\newcommand{\eqdef}{\ensuremath{\overset{\text{def}}=}}


\def\Circlearrowright{\ensuremath{%
  \rotatebox[origin=c]{230}{$\circlearrowright$}}}

\newcommand\ct[1]{\text{\rmfamily\upshape #1}}
\newcommand\question[1]{ {\color{red} ...!? \small #1}}
\newcommand\caz[1]{\left\{\begin{array} #1 \end{array}\right.}
\newcommand\const{\text{\rmfamily\upshape const}}
\newcommand\toP{ \overset{\pro}{\to}}
\newcommand\toPP{ \overset{\text{PP}}{\to}}
\newcommand{\oeq}{\mathrel{\text{\textcircled{$=$}}}}





\usepackage{xcolor}
% \usepackage[normalem]{ulem}
\usepackage{lipsum}
\makeatletter
% \newcommand\colorwave[1][blue]{\bgroup \markoverwith{\lower3.5\p@\hbox{\sixly \textcolor{#1}{\char58}}}\ULon}
%\font\sixly=lasy6 % does not re-load if already loaded, so no memory problem.

\newmdtheoremenv[
linewidth= 1pt,linecolor= blue,%
leftmargin=20,rightmargin=20,innertopmargin=0pt, innerrightmargin=40,%
tikzsetting = { draw=lightgray, line width = 0.3pt,dashed,%
dash pattern = on 15pt off 3pt},%
splittopskip=\topskip,skipbelow=\baselineskip,%
skipabove=\baselineskip,ntheorem,roundcorner=0pt,
% backgroundcolor=pagebg,font=\color{orange}\sffamily, fontcolor=white
]{examplebox}{Exemple}[section]



\newcommand\R{\mathbb{R}}
\newcommand\Z{\mathbb{Z}}
\newcommand\N{\mathbb{N}}
\newcommand\E{\mathbb{E}}
\newcommand\F{\mathcal{F}}
\newcommand\cH{\mathcal{H}}
\newcommand\V{\mathbb{V}}
\newcommand\dmo{ ^{-1} }
\newcommand\kapa{\kappa}
\newcommand\im{Im}
\newcommand\hs{\mathcal{H}}





\usepackage{soul}

\makeatletter
\newcommand*{\whiten}[1]{\llap{\textcolor{white}{{\the\SOUL@token}}\hspace{#1pt}}}
\DeclareRobustCommand*\myul{%
    \def\SOUL@everyspace{\underline{\space}\kern\z@}%
    \def\SOUL@everytoken{%
     \setbox0=\hbox{\the\SOUL@token}%
     \ifdim\dp0>\z@
        \raisebox{\dp0}{\underline{\phantom{\the\SOUL@token}}}%
        \whiten{1}\whiten{0}%
        \whiten{-1}\whiten{-2}%
        \llap{\the\SOUL@token}%
     \else
        \underline{\the\SOUL@token}%
     \fi}%
\SOUL@}
\makeatother

\newcommand*{\demp}{\fontfamily{lmtt}\selectfont}

\DeclareTextFontCommand{\textdemp}{\demp}

\begin{document}

\ifcomment
Multiline
comment
\fi
\ifcomment
\myul{Typesetting test}
% \color[rgb]{1,1,1}
$∑_i^n≠ 60º±∞π∆¬≈√j∫h≤≥µ$

$\CR \R\pro\ind\pro\gS\pro
\mqty[a&b\\c&d]$
$\pro\mathbb{P}$
$\dd{x}$

  \[
    \alpha(x)=\left\{
                \begin{array}{ll}
                  x\\
                  \frac{1}{1+e^{-kx}}\\
                  \frac{e^x-e^{-x}}{e^x+e^{-x}}
                \end{array}
              \right.
  \]

  $\expval{x}$
  
  $\chi_\rho(ghg\dmo)=\Tr(\rho_{ghg\dmo})=\Tr(\rho_g\circ\rho_h\circ\rho\dmo_g)=\Tr(\rho_h)\overset{\mbox{\scalebox{0.5}{$\Tr(AB)=\Tr(BA)$}}}{=}\chi_\rho(h)$
  	$\mathop{\oplus}_{\substack{x\in X}}$

$\mat(\rho_g)=(a_{ij}(g))_{\scriptsize \substack{1\leq i\leq d \\ 1\leq j\leq d}}$ et $\mat(\rho'_g)=(a'_{ij}(g))_{\scriptsize \substack{1\leq i'\leq d' \\ 1\leq j'\leq d'}}$



\[\int_a^b{\mathbb{R}^2}g(u, v)\dd{P_{XY}}(u, v)=\iint g(u,v) f_{XY}(u, v)\dd \lambda(u) \dd \lambda(v)\]
$$\lim_{x\to\infty} f(x)$$	
$$\iiiint_V \mu(t,u,v,w) \,dt\,du\,dv\,dw$$
$$\sum_{n=1}^{\infty} 2^{-n} = 1$$	
\begin{definition}
	Si $X$ et $Y$ sont 2 v.a. ou definit la \textsc{Covariance} entre $X$ et $Y$ comme
	$\cov(X,Y)\overset{\text{def}}{=}\E\left[(X-\E(X))(Y-\E(Y))\right]=\E(XY)-\E(X)\E(Y)$.
\end{definition}
\fi
\pagebreak

% \tableofcontents

% insert your code here
%\input{./algebra/main.tex}
%\input{./geometrie-differentielle/main.tex}
%\input{./probabilite/main.tex}
%\input{./analyse-fonctionnelle/main.tex}
% \input{./Analyse-convexe-et-dualite-en-optimisation/main.tex}
%\input{./tikz/main.tex}
%\input{./Theorie-du-distributions/main.tex}
%\input{./optimisation/mine.tex}
 \input{./modelisation/main.tex}

% yves.aubry@univ-tln.fr : algebra

\end{document}

%% !TEX encoding = UTF-8 Unicode
% !TEX TS-program = xelatex

\documentclass[french]{report}

%\usepackage[utf8]{inputenc}
%\usepackage[T1]{fontenc}
\usepackage{babel}


\newif\ifcomment
%\commenttrue # Show comments

\usepackage{physics}
\usepackage{amssymb}


\usepackage{amsthm}
% \usepackage{thmtools}
\usepackage{mathtools}
\usepackage{amsfonts}

\usepackage{color}

\usepackage{tikz}

\usepackage{geometry}
\geometry{a5paper, margin=0.1in, right=1cm}

\usepackage{dsfont}

\usepackage{graphicx}
\graphicspath{ {images/} }

\usepackage{faktor}

\usepackage{IEEEtrantools}
\usepackage{enumerate}   
\usepackage[PostScript=dvips]{"/Users/aware/Documents/Courses/diagrams"}


\newtheorem{theorem}{Théorème}[section]
\renewcommand{\thetheorem}{\arabic{theorem}}
\newtheorem{lemme}{Lemme}[section]
\renewcommand{\thelemme}{\arabic{lemme}}
\newtheorem{proposition}{Proposition}[section]
\renewcommand{\theproposition}{\arabic{proposition}}
\newtheorem{notations}{Notations}[section]
\newtheorem{problem}{Problème}[section]
\newtheorem{corollary}{Corollaire}[theorem]
\renewcommand{\thecorollary}{\arabic{corollary}}
\newtheorem{property}{Propriété}[section]
\newtheorem{objective}{Objectif}[section]

\theoremstyle{definition}
\newtheorem{definition}{Définition}[section]
\renewcommand{\thedefinition}{\arabic{definition}}
\newtheorem{exercise}{Exercice}[chapter]
\renewcommand{\theexercise}{\arabic{exercise}}
\newtheorem{example}{Exemple}[chapter]
\renewcommand{\theexample}{\arabic{example}}
\newtheorem*{solution}{Solution}
\newtheorem*{application}{Application}
\newtheorem*{notation}{Notation}
\newtheorem*{vocabulary}{Vocabulaire}
\newtheorem*{properties}{Propriétés}



\theoremstyle{remark}
\newtheorem*{remark}{Remarque}
\newtheorem*{rappel}{Rappel}


\usepackage{etoolbox}
\AtBeginEnvironment{exercise}{\small}
\AtBeginEnvironment{example}{\small}

\usepackage{cases}
\usepackage[red]{mypack}

\usepackage[framemethod=TikZ]{mdframed}

\definecolor{bg}{rgb}{0.4,0.25,0.95}
\definecolor{pagebg}{rgb}{0,0,0.5}
\surroundwithmdframed[
   topline=false,
   rightline=false,
   bottomline=false,
   leftmargin=\parindent,
   skipabove=8pt,
   skipbelow=8pt,
   linecolor=blue,
   innerbottommargin=10pt,
   % backgroundcolor=bg,font=\color{orange}\sffamily, fontcolor=white
]{definition}

\usepackage{empheq}
\usepackage[most]{tcolorbox}

\newtcbox{\mymath}[1][]{%
    nobeforeafter, math upper, tcbox raise base,
    enhanced, colframe=blue!30!black,
    colback=red!10, boxrule=1pt,
    #1}

\usepackage{unixode}


\DeclareMathOperator{\ord}{ord}
\DeclareMathOperator{\orb}{orb}
\DeclareMathOperator{\stab}{stab}
\DeclareMathOperator{\Stab}{stab}
\DeclareMathOperator{\ppcm}{ppcm}
\DeclareMathOperator{\conj}{Conj}
\DeclareMathOperator{\End}{End}
\DeclareMathOperator{\rot}{rot}
\DeclareMathOperator{\trs}{trace}
\DeclareMathOperator{\Ind}{Ind}
\DeclareMathOperator{\mat}{Mat}
\DeclareMathOperator{\id}{Id}
\DeclareMathOperator{\vect}{vect}
\DeclareMathOperator{\img}{img}
\DeclareMathOperator{\cov}{Cov}
\DeclareMathOperator{\dist}{dist}
\DeclareMathOperator{\irr}{Irr}
\DeclareMathOperator{\image}{Im}
\DeclareMathOperator{\pd}{\partial}
\DeclareMathOperator{\epi}{epi}
\DeclareMathOperator{\Argmin}{Argmin}
\DeclareMathOperator{\dom}{dom}
\DeclareMathOperator{\proj}{proj}
\DeclareMathOperator{\ctg}{ctg}
\DeclareMathOperator{\supp}{supp}
\DeclareMathOperator{\argmin}{argmin}
\DeclareMathOperator{\mult}{mult}
\DeclareMathOperator{\ch}{ch}
\DeclareMathOperator{\sh}{sh}
\DeclareMathOperator{\rang}{rang}
\DeclareMathOperator{\diam}{diam}
\DeclareMathOperator{\Epigraphe}{Epigraphe}




\usepackage{xcolor}
\everymath{\color{blue}}
%\everymath{\color[rgb]{0,1,1}}
%\pagecolor[rgb]{0,0,0.5}


\newcommand*{\pdtest}[3][]{\ensuremath{\frac{\partial^{#1} #2}{\partial #3}}}

\newcommand*{\deffunc}[6][]{\ensuremath{
\begin{array}{rcl}
#2 : #3 &\rightarrow& #4\\
#5 &\mapsto& #6
\end{array}
}}

\newcommand{\eqcolon}{\mathrel{\resizebox{\widthof{$\mathord{=}$}}{\height}{ $\!\!=\!\!\resizebox{1.2\width}{0.8\height}{\raisebox{0.23ex}{$\mathop{:}$}}\!\!$ }}}
\newcommand{\coloneq}{\mathrel{\resizebox{\widthof{$\mathord{=}$}}{\height}{ $\!\!\resizebox{1.2\width}{0.8\height}{\raisebox{0.23ex}{$\mathop{:}$}}\!\!=\!\!$ }}}
\newcommand{\eqcolonl}{\ensuremath{\mathrel{=\!\!\mathop{:}}}}
\newcommand{\coloneql}{\ensuremath{\mathrel{\mathop{:} \!\! =}}}
\newcommand{\vc}[1]{% inline column vector
  \left(\begin{smallmatrix}#1\end{smallmatrix}\right)%
}
\newcommand{\vr}[1]{% inline row vector
  \begin{smallmatrix}(\,#1\,)\end{smallmatrix}%
}
\makeatletter
\newcommand*{\defeq}{\ =\mathrel{\rlap{%
                     \raisebox{0.3ex}{$\m@th\cdot$}}%
                     \raisebox{-0.3ex}{$\m@th\cdot$}}%
                     }
\makeatother

\newcommand{\mathcircle}[1]{% inline row vector
 \overset{\circ}{#1}
}
\newcommand{\ulim}{% low limit
 \underline{\lim}
}
\newcommand{\ssi}{% iff
\iff
}
\newcommand{\ps}[2]{
\expval{#1 | #2}
}
\newcommand{\df}[1]{
\mqty{#1}
}
\newcommand{\n}[1]{
\norm{#1}
}
\newcommand{\sys}[1]{
\left\{\smqty{#1}\right.
}


\newcommand{\eqdef}{\ensuremath{\overset{\text{def}}=}}


\def\Circlearrowright{\ensuremath{%
  \rotatebox[origin=c]{230}{$\circlearrowright$}}}

\newcommand\ct[1]{\text{\rmfamily\upshape #1}}
\newcommand\question[1]{ {\color{red} ...!? \small #1}}
\newcommand\caz[1]{\left\{\begin{array} #1 \end{array}\right.}
\newcommand\const{\text{\rmfamily\upshape const}}
\newcommand\toP{ \overset{\pro}{\to}}
\newcommand\toPP{ \overset{\text{PP}}{\to}}
\newcommand{\oeq}{\mathrel{\text{\textcircled{$=$}}}}





\usepackage{xcolor}
% \usepackage[normalem]{ulem}
\usepackage{lipsum}
\makeatletter
% \newcommand\colorwave[1][blue]{\bgroup \markoverwith{\lower3.5\p@\hbox{\sixly \textcolor{#1}{\char58}}}\ULon}
%\font\sixly=lasy6 % does not re-load if already loaded, so no memory problem.

\newmdtheoremenv[
linewidth= 1pt,linecolor= blue,%
leftmargin=20,rightmargin=20,innertopmargin=0pt, innerrightmargin=40,%
tikzsetting = { draw=lightgray, line width = 0.3pt,dashed,%
dash pattern = on 15pt off 3pt},%
splittopskip=\topskip,skipbelow=\baselineskip,%
skipabove=\baselineskip,ntheorem,roundcorner=0pt,
% backgroundcolor=pagebg,font=\color{orange}\sffamily, fontcolor=white
]{examplebox}{Exemple}[section]



\newcommand\R{\mathbb{R}}
\newcommand\Z{\mathbb{Z}}
\newcommand\N{\mathbb{N}}
\newcommand\E{\mathbb{E}}
\newcommand\F{\mathcal{F}}
\newcommand\cH{\mathcal{H}}
\newcommand\V{\mathbb{V}}
\newcommand\dmo{ ^{-1} }
\newcommand\kapa{\kappa}
\newcommand\im{Im}
\newcommand\hs{\mathcal{H}}





\usepackage{soul}

\makeatletter
\newcommand*{\whiten}[1]{\llap{\textcolor{white}{{\the\SOUL@token}}\hspace{#1pt}}}
\DeclareRobustCommand*\myul{%
    \def\SOUL@everyspace{\underline{\space}\kern\z@}%
    \def\SOUL@everytoken{%
     \setbox0=\hbox{\the\SOUL@token}%
     \ifdim\dp0>\z@
        \raisebox{\dp0}{\underline{\phantom{\the\SOUL@token}}}%
        \whiten{1}\whiten{0}%
        \whiten{-1}\whiten{-2}%
        \llap{\the\SOUL@token}%
     \else
        \underline{\the\SOUL@token}%
     \fi}%
\SOUL@}
\makeatother

\newcommand*{\demp}{\fontfamily{lmtt}\selectfont}

\DeclareTextFontCommand{\textdemp}{\demp}

\begin{document}

\ifcomment
Multiline
comment
\fi
\ifcomment
\myul{Typesetting test}
% \color[rgb]{1,1,1}
$∑_i^n≠ 60º±∞π∆¬≈√j∫h≤≥µ$

$\CR \R\pro\ind\pro\gS\pro
\mqty[a&b\\c&d]$
$\pro\mathbb{P}$
$\dd{x}$

  \[
    \alpha(x)=\left\{
                \begin{array}{ll}
                  x\\
                  \frac{1}{1+e^{-kx}}\\
                  \frac{e^x-e^{-x}}{e^x+e^{-x}}
                \end{array}
              \right.
  \]

  $\expval{x}$
  
  $\chi_\rho(ghg\dmo)=\Tr(\rho_{ghg\dmo})=\Tr(\rho_g\circ\rho_h\circ\rho\dmo_g)=\Tr(\rho_h)\overset{\mbox{\scalebox{0.5}{$\Tr(AB)=\Tr(BA)$}}}{=}\chi_\rho(h)$
  	$\mathop{\oplus}_{\substack{x\in X}}$

$\mat(\rho_g)=(a_{ij}(g))_{\scriptsize \substack{1\leq i\leq d \\ 1\leq j\leq d}}$ et $\mat(\rho'_g)=(a'_{ij}(g))_{\scriptsize \substack{1\leq i'\leq d' \\ 1\leq j'\leq d'}}$



\[\int_a^b{\mathbb{R}^2}g(u, v)\dd{P_{XY}}(u, v)=\iint g(u,v) f_{XY}(u, v)\dd \lambda(u) \dd \lambda(v)\]
$$\lim_{x\to\infty} f(x)$$	
$$\iiiint_V \mu(t,u,v,w) \,dt\,du\,dv\,dw$$
$$\sum_{n=1}^{\infty} 2^{-n} = 1$$	
\begin{definition}
	Si $X$ et $Y$ sont 2 v.a. ou definit la \textsc{Covariance} entre $X$ et $Y$ comme
	$\cov(X,Y)\overset{\text{def}}{=}\E\left[(X-\E(X))(Y-\E(Y))\right]=\E(XY)-\E(X)\E(Y)$.
\end{definition}
\fi
\pagebreak

% \tableofcontents

% insert your code here
%\input{./algebra/main.tex}
%\input{./geometrie-differentielle/main.tex}
%\input{./probabilite/main.tex}
%\input{./analyse-fonctionnelle/main.tex}
% \input{./Analyse-convexe-et-dualite-en-optimisation/main.tex}
%\input{./tikz/main.tex}
%\input{./Theorie-du-distributions/main.tex}
%\input{./optimisation/mine.tex}
 \input{./modelisation/main.tex}

% yves.aubry@univ-tln.fr : algebra

\end{document}

%% !TEX encoding = UTF-8 Unicode
% !TEX TS-program = xelatex

\documentclass[french]{report}

%\usepackage[utf8]{inputenc}
%\usepackage[T1]{fontenc}
\usepackage{babel}


\newif\ifcomment
%\commenttrue # Show comments

\usepackage{physics}
\usepackage{amssymb}


\usepackage{amsthm}
% \usepackage{thmtools}
\usepackage{mathtools}
\usepackage{amsfonts}

\usepackage{color}

\usepackage{tikz}

\usepackage{geometry}
\geometry{a5paper, margin=0.1in, right=1cm}

\usepackage{dsfont}

\usepackage{graphicx}
\graphicspath{ {images/} }

\usepackage{faktor}

\usepackage{IEEEtrantools}
\usepackage{enumerate}   
\usepackage[PostScript=dvips]{"/Users/aware/Documents/Courses/diagrams"}


\newtheorem{theorem}{Théorème}[section]
\renewcommand{\thetheorem}{\arabic{theorem}}
\newtheorem{lemme}{Lemme}[section]
\renewcommand{\thelemme}{\arabic{lemme}}
\newtheorem{proposition}{Proposition}[section]
\renewcommand{\theproposition}{\arabic{proposition}}
\newtheorem{notations}{Notations}[section]
\newtheorem{problem}{Problème}[section]
\newtheorem{corollary}{Corollaire}[theorem]
\renewcommand{\thecorollary}{\arabic{corollary}}
\newtheorem{property}{Propriété}[section]
\newtheorem{objective}{Objectif}[section]

\theoremstyle{definition}
\newtheorem{definition}{Définition}[section]
\renewcommand{\thedefinition}{\arabic{definition}}
\newtheorem{exercise}{Exercice}[chapter]
\renewcommand{\theexercise}{\arabic{exercise}}
\newtheorem{example}{Exemple}[chapter]
\renewcommand{\theexample}{\arabic{example}}
\newtheorem*{solution}{Solution}
\newtheorem*{application}{Application}
\newtheorem*{notation}{Notation}
\newtheorem*{vocabulary}{Vocabulaire}
\newtheorem*{properties}{Propriétés}



\theoremstyle{remark}
\newtheorem*{remark}{Remarque}
\newtheorem*{rappel}{Rappel}


\usepackage{etoolbox}
\AtBeginEnvironment{exercise}{\small}
\AtBeginEnvironment{example}{\small}

\usepackage{cases}
\usepackage[red]{mypack}

\usepackage[framemethod=TikZ]{mdframed}

\definecolor{bg}{rgb}{0.4,0.25,0.95}
\definecolor{pagebg}{rgb}{0,0,0.5}
\surroundwithmdframed[
   topline=false,
   rightline=false,
   bottomline=false,
   leftmargin=\parindent,
   skipabove=8pt,
   skipbelow=8pt,
   linecolor=blue,
   innerbottommargin=10pt,
   % backgroundcolor=bg,font=\color{orange}\sffamily, fontcolor=white
]{definition}

\usepackage{empheq}
\usepackage[most]{tcolorbox}

\newtcbox{\mymath}[1][]{%
    nobeforeafter, math upper, tcbox raise base,
    enhanced, colframe=blue!30!black,
    colback=red!10, boxrule=1pt,
    #1}

\usepackage{unixode}


\DeclareMathOperator{\ord}{ord}
\DeclareMathOperator{\orb}{orb}
\DeclareMathOperator{\stab}{stab}
\DeclareMathOperator{\Stab}{stab}
\DeclareMathOperator{\ppcm}{ppcm}
\DeclareMathOperator{\conj}{Conj}
\DeclareMathOperator{\End}{End}
\DeclareMathOperator{\rot}{rot}
\DeclareMathOperator{\trs}{trace}
\DeclareMathOperator{\Ind}{Ind}
\DeclareMathOperator{\mat}{Mat}
\DeclareMathOperator{\id}{Id}
\DeclareMathOperator{\vect}{vect}
\DeclareMathOperator{\img}{img}
\DeclareMathOperator{\cov}{Cov}
\DeclareMathOperator{\dist}{dist}
\DeclareMathOperator{\irr}{Irr}
\DeclareMathOperator{\image}{Im}
\DeclareMathOperator{\pd}{\partial}
\DeclareMathOperator{\epi}{epi}
\DeclareMathOperator{\Argmin}{Argmin}
\DeclareMathOperator{\dom}{dom}
\DeclareMathOperator{\proj}{proj}
\DeclareMathOperator{\ctg}{ctg}
\DeclareMathOperator{\supp}{supp}
\DeclareMathOperator{\argmin}{argmin}
\DeclareMathOperator{\mult}{mult}
\DeclareMathOperator{\ch}{ch}
\DeclareMathOperator{\sh}{sh}
\DeclareMathOperator{\rang}{rang}
\DeclareMathOperator{\diam}{diam}
\DeclareMathOperator{\Epigraphe}{Epigraphe}




\usepackage{xcolor}
\everymath{\color{blue}}
%\everymath{\color[rgb]{0,1,1}}
%\pagecolor[rgb]{0,0,0.5}


\newcommand*{\pdtest}[3][]{\ensuremath{\frac{\partial^{#1} #2}{\partial #3}}}

\newcommand*{\deffunc}[6][]{\ensuremath{
\begin{array}{rcl}
#2 : #3 &\rightarrow& #4\\
#5 &\mapsto& #6
\end{array}
}}

\newcommand{\eqcolon}{\mathrel{\resizebox{\widthof{$\mathord{=}$}}{\height}{ $\!\!=\!\!\resizebox{1.2\width}{0.8\height}{\raisebox{0.23ex}{$\mathop{:}$}}\!\!$ }}}
\newcommand{\coloneq}{\mathrel{\resizebox{\widthof{$\mathord{=}$}}{\height}{ $\!\!\resizebox{1.2\width}{0.8\height}{\raisebox{0.23ex}{$\mathop{:}$}}\!\!=\!\!$ }}}
\newcommand{\eqcolonl}{\ensuremath{\mathrel{=\!\!\mathop{:}}}}
\newcommand{\coloneql}{\ensuremath{\mathrel{\mathop{:} \!\! =}}}
\newcommand{\vc}[1]{% inline column vector
  \left(\begin{smallmatrix}#1\end{smallmatrix}\right)%
}
\newcommand{\vr}[1]{% inline row vector
  \begin{smallmatrix}(\,#1\,)\end{smallmatrix}%
}
\makeatletter
\newcommand*{\defeq}{\ =\mathrel{\rlap{%
                     \raisebox{0.3ex}{$\m@th\cdot$}}%
                     \raisebox{-0.3ex}{$\m@th\cdot$}}%
                     }
\makeatother

\newcommand{\mathcircle}[1]{% inline row vector
 \overset{\circ}{#1}
}
\newcommand{\ulim}{% low limit
 \underline{\lim}
}
\newcommand{\ssi}{% iff
\iff
}
\newcommand{\ps}[2]{
\expval{#1 | #2}
}
\newcommand{\df}[1]{
\mqty{#1}
}
\newcommand{\n}[1]{
\norm{#1}
}
\newcommand{\sys}[1]{
\left\{\smqty{#1}\right.
}


\newcommand{\eqdef}{\ensuremath{\overset{\text{def}}=}}


\def\Circlearrowright{\ensuremath{%
  \rotatebox[origin=c]{230}{$\circlearrowright$}}}

\newcommand\ct[1]{\text{\rmfamily\upshape #1}}
\newcommand\question[1]{ {\color{red} ...!? \small #1}}
\newcommand\caz[1]{\left\{\begin{array} #1 \end{array}\right.}
\newcommand\const{\text{\rmfamily\upshape const}}
\newcommand\toP{ \overset{\pro}{\to}}
\newcommand\toPP{ \overset{\text{PP}}{\to}}
\newcommand{\oeq}{\mathrel{\text{\textcircled{$=$}}}}





\usepackage{xcolor}
% \usepackage[normalem]{ulem}
\usepackage{lipsum}
\makeatletter
% \newcommand\colorwave[1][blue]{\bgroup \markoverwith{\lower3.5\p@\hbox{\sixly \textcolor{#1}{\char58}}}\ULon}
%\font\sixly=lasy6 % does not re-load if already loaded, so no memory problem.

\newmdtheoremenv[
linewidth= 1pt,linecolor= blue,%
leftmargin=20,rightmargin=20,innertopmargin=0pt, innerrightmargin=40,%
tikzsetting = { draw=lightgray, line width = 0.3pt,dashed,%
dash pattern = on 15pt off 3pt},%
splittopskip=\topskip,skipbelow=\baselineskip,%
skipabove=\baselineskip,ntheorem,roundcorner=0pt,
% backgroundcolor=pagebg,font=\color{orange}\sffamily, fontcolor=white
]{examplebox}{Exemple}[section]



\newcommand\R{\mathbb{R}}
\newcommand\Z{\mathbb{Z}}
\newcommand\N{\mathbb{N}}
\newcommand\E{\mathbb{E}}
\newcommand\F{\mathcal{F}}
\newcommand\cH{\mathcal{H}}
\newcommand\V{\mathbb{V}}
\newcommand\dmo{ ^{-1} }
\newcommand\kapa{\kappa}
\newcommand\im{Im}
\newcommand\hs{\mathcal{H}}





\usepackage{soul}

\makeatletter
\newcommand*{\whiten}[1]{\llap{\textcolor{white}{{\the\SOUL@token}}\hspace{#1pt}}}
\DeclareRobustCommand*\myul{%
    \def\SOUL@everyspace{\underline{\space}\kern\z@}%
    \def\SOUL@everytoken{%
     \setbox0=\hbox{\the\SOUL@token}%
     \ifdim\dp0>\z@
        \raisebox{\dp0}{\underline{\phantom{\the\SOUL@token}}}%
        \whiten{1}\whiten{0}%
        \whiten{-1}\whiten{-2}%
        \llap{\the\SOUL@token}%
     \else
        \underline{\the\SOUL@token}%
     \fi}%
\SOUL@}
\makeatother

\newcommand*{\demp}{\fontfamily{lmtt}\selectfont}

\DeclareTextFontCommand{\textdemp}{\demp}

\begin{document}

\ifcomment
Multiline
comment
\fi
\ifcomment
\myul{Typesetting test}
% \color[rgb]{1,1,1}
$∑_i^n≠ 60º±∞π∆¬≈√j∫h≤≥µ$

$\CR \R\pro\ind\pro\gS\pro
\mqty[a&b\\c&d]$
$\pro\mathbb{P}$
$\dd{x}$

  \[
    \alpha(x)=\left\{
                \begin{array}{ll}
                  x\\
                  \frac{1}{1+e^{-kx}}\\
                  \frac{e^x-e^{-x}}{e^x+e^{-x}}
                \end{array}
              \right.
  \]

  $\expval{x}$
  
  $\chi_\rho(ghg\dmo)=\Tr(\rho_{ghg\dmo})=\Tr(\rho_g\circ\rho_h\circ\rho\dmo_g)=\Tr(\rho_h)\overset{\mbox{\scalebox{0.5}{$\Tr(AB)=\Tr(BA)$}}}{=}\chi_\rho(h)$
  	$\mathop{\oplus}_{\substack{x\in X}}$

$\mat(\rho_g)=(a_{ij}(g))_{\scriptsize \substack{1\leq i\leq d \\ 1\leq j\leq d}}$ et $\mat(\rho'_g)=(a'_{ij}(g))_{\scriptsize \substack{1\leq i'\leq d' \\ 1\leq j'\leq d'}}$



\[\int_a^b{\mathbb{R}^2}g(u, v)\dd{P_{XY}}(u, v)=\iint g(u,v) f_{XY}(u, v)\dd \lambda(u) \dd \lambda(v)\]
$$\lim_{x\to\infty} f(x)$$	
$$\iiiint_V \mu(t,u,v,w) \,dt\,du\,dv\,dw$$
$$\sum_{n=1}^{\infty} 2^{-n} = 1$$	
\begin{definition}
	Si $X$ et $Y$ sont 2 v.a. ou definit la \textsc{Covariance} entre $X$ et $Y$ comme
	$\cov(X,Y)\overset{\text{def}}{=}\E\left[(X-\E(X))(Y-\E(Y))\right]=\E(XY)-\E(X)\E(Y)$.
\end{definition}
\fi
\pagebreak

% \tableofcontents

% insert your code here
%\input{./algebra/main.tex}
%\input{./geometrie-differentielle/main.tex}
%\input{./probabilite/main.tex}
%\input{./analyse-fonctionnelle/main.tex}
% \input{./Analyse-convexe-et-dualite-en-optimisation/main.tex}
%\input{./tikz/main.tex}
%\input{./Theorie-du-distributions/main.tex}
%\input{./optimisation/mine.tex}
 \input{./modelisation/main.tex}

% yves.aubry@univ-tln.fr : algebra

\end{document}

%\input{./optimisation/mine.tex}
 % !TEX encoding = UTF-8 Unicode
% !TEX TS-program = xelatex

\documentclass[french]{report}

%\usepackage[utf8]{inputenc}
%\usepackage[T1]{fontenc}
\usepackage{babel}


\newif\ifcomment
%\commenttrue # Show comments

\usepackage{physics}
\usepackage{amssymb}


\usepackage{amsthm}
% \usepackage{thmtools}
\usepackage{mathtools}
\usepackage{amsfonts}

\usepackage{color}

\usepackage{tikz}

\usepackage{geometry}
\geometry{a5paper, margin=0.1in, right=1cm}

\usepackage{dsfont}

\usepackage{graphicx}
\graphicspath{ {images/} }

\usepackage{faktor}

\usepackage{IEEEtrantools}
\usepackage{enumerate}   
\usepackage[PostScript=dvips]{"/Users/aware/Documents/Courses/diagrams"}


\newtheorem{theorem}{Théorème}[section]
\renewcommand{\thetheorem}{\arabic{theorem}}
\newtheorem{lemme}{Lemme}[section]
\renewcommand{\thelemme}{\arabic{lemme}}
\newtheorem{proposition}{Proposition}[section]
\renewcommand{\theproposition}{\arabic{proposition}}
\newtheorem{notations}{Notations}[section]
\newtheorem{problem}{Problème}[section]
\newtheorem{corollary}{Corollaire}[theorem]
\renewcommand{\thecorollary}{\arabic{corollary}}
\newtheorem{property}{Propriété}[section]
\newtheorem{objective}{Objectif}[section]

\theoremstyle{definition}
\newtheorem{definition}{Définition}[section]
\renewcommand{\thedefinition}{\arabic{definition}}
\newtheorem{exercise}{Exercice}[chapter]
\renewcommand{\theexercise}{\arabic{exercise}}
\newtheorem{example}{Exemple}[chapter]
\renewcommand{\theexample}{\arabic{example}}
\newtheorem*{solution}{Solution}
\newtheorem*{application}{Application}
\newtheorem*{notation}{Notation}
\newtheorem*{vocabulary}{Vocabulaire}
\newtheorem*{properties}{Propriétés}



\theoremstyle{remark}
\newtheorem*{remark}{Remarque}
\newtheorem*{rappel}{Rappel}


\usepackage{etoolbox}
\AtBeginEnvironment{exercise}{\small}
\AtBeginEnvironment{example}{\small}

\usepackage{cases}
\usepackage[red]{mypack}

\usepackage[framemethod=TikZ]{mdframed}

\definecolor{bg}{rgb}{0.4,0.25,0.95}
\definecolor{pagebg}{rgb}{0,0,0.5}
\surroundwithmdframed[
   topline=false,
   rightline=false,
   bottomline=false,
   leftmargin=\parindent,
   skipabove=8pt,
   skipbelow=8pt,
   linecolor=blue,
   innerbottommargin=10pt,
   % backgroundcolor=bg,font=\color{orange}\sffamily, fontcolor=white
]{definition}

\usepackage{empheq}
\usepackage[most]{tcolorbox}

\newtcbox{\mymath}[1][]{%
    nobeforeafter, math upper, tcbox raise base,
    enhanced, colframe=blue!30!black,
    colback=red!10, boxrule=1pt,
    #1}

\usepackage{unixode}


\DeclareMathOperator{\ord}{ord}
\DeclareMathOperator{\orb}{orb}
\DeclareMathOperator{\stab}{stab}
\DeclareMathOperator{\Stab}{stab}
\DeclareMathOperator{\ppcm}{ppcm}
\DeclareMathOperator{\conj}{Conj}
\DeclareMathOperator{\End}{End}
\DeclareMathOperator{\rot}{rot}
\DeclareMathOperator{\trs}{trace}
\DeclareMathOperator{\Ind}{Ind}
\DeclareMathOperator{\mat}{Mat}
\DeclareMathOperator{\id}{Id}
\DeclareMathOperator{\vect}{vect}
\DeclareMathOperator{\img}{img}
\DeclareMathOperator{\cov}{Cov}
\DeclareMathOperator{\dist}{dist}
\DeclareMathOperator{\irr}{Irr}
\DeclareMathOperator{\image}{Im}
\DeclareMathOperator{\pd}{\partial}
\DeclareMathOperator{\epi}{epi}
\DeclareMathOperator{\Argmin}{Argmin}
\DeclareMathOperator{\dom}{dom}
\DeclareMathOperator{\proj}{proj}
\DeclareMathOperator{\ctg}{ctg}
\DeclareMathOperator{\supp}{supp}
\DeclareMathOperator{\argmin}{argmin}
\DeclareMathOperator{\mult}{mult}
\DeclareMathOperator{\ch}{ch}
\DeclareMathOperator{\sh}{sh}
\DeclareMathOperator{\rang}{rang}
\DeclareMathOperator{\diam}{diam}
\DeclareMathOperator{\Epigraphe}{Epigraphe}




\usepackage{xcolor}
\everymath{\color{blue}}
%\everymath{\color[rgb]{0,1,1}}
%\pagecolor[rgb]{0,0,0.5}


\newcommand*{\pdtest}[3][]{\ensuremath{\frac{\partial^{#1} #2}{\partial #3}}}

\newcommand*{\deffunc}[6][]{\ensuremath{
\begin{array}{rcl}
#2 : #3 &\rightarrow& #4\\
#5 &\mapsto& #6
\end{array}
}}

\newcommand{\eqcolon}{\mathrel{\resizebox{\widthof{$\mathord{=}$}}{\height}{ $\!\!=\!\!\resizebox{1.2\width}{0.8\height}{\raisebox{0.23ex}{$\mathop{:}$}}\!\!$ }}}
\newcommand{\coloneq}{\mathrel{\resizebox{\widthof{$\mathord{=}$}}{\height}{ $\!\!\resizebox{1.2\width}{0.8\height}{\raisebox{0.23ex}{$\mathop{:}$}}\!\!=\!\!$ }}}
\newcommand{\eqcolonl}{\ensuremath{\mathrel{=\!\!\mathop{:}}}}
\newcommand{\coloneql}{\ensuremath{\mathrel{\mathop{:} \!\! =}}}
\newcommand{\vc}[1]{% inline column vector
  \left(\begin{smallmatrix}#1\end{smallmatrix}\right)%
}
\newcommand{\vr}[1]{% inline row vector
  \begin{smallmatrix}(\,#1\,)\end{smallmatrix}%
}
\makeatletter
\newcommand*{\defeq}{\ =\mathrel{\rlap{%
                     \raisebox{0.3ex}{$\m@th\cdot$}}%
                     \raisebox{-0.3ex}{$\m@th\cdot$}}%
                     }
\makeatother

\newcommand{\mathcircle}[1]{% inline row vector
 \overset{\circ}{#1}
}
\newcommand{\ulim}{% low limit
 \underline{\lim}
}
\newcommand{\ssi}{% iff
\iff
}
\newcommand{\ps}[2]{
\expval{#1 | #2}
}
\newcommand{\df}[1]{
\mqty{#1}
}
\newcommand{\n}[1]{
\norm{#1}
}
\newcommand{\sys}[1]{
\left\{\smqty{#1}\right.
}


\newcommand{\eqdef}{\ensuremath{\overset{\text{def}}=}}


\def\Circlearrowright{\ensuremath{%
  \rotatebox[origin=c]{230}{$\circlearrowright$}}}

\newcommand\ct[1]{\text{\rmfamily\upshape #1}}
\newcommand\question[1]{ {\color{red} ...!? \small #1}}
\newcommand\caz[1]{\left\{\begin{array} #1 \end{array}\right.}
\newcommand\const{\text{\rmfamily\upshape const}}
\newcommand\toP{ \overset{\pro}{\to}}
\newcommand\toPP{ \overset{\text{PP}}{\to}}
\newcommand{\oeq}{\mathrel{\text{\textcircled{$=$}}}}





\usepackage{xcolor}
% \usepackage[normalem]{ulem}
\usepackage{lipsum}
\makeatletter
% \newcommand\colorwave[1][blue]{\bgroup \markoverwith{\lower3.5\p@\hbox{\sixly \textcolor{#1}{\char58}}}\ULon}
%\font\sixly=lasy6 % does not re-load if already loaded, so no memory problem.

\newmdtheoremenv[
linewidth= 1pt,linecolor= blue,%
leftmargin=20,rightmargin=20,innertopmargin=0pt, innerrightmargin=40,%
tikzsetting = { draw=lightgray, line width = 0.3pt,dashed,%
dash pattern = on 15pt off 3pt},%
splittopskip=\topskip,skipbelow=\baselineskip,%
skipabove=\baselineskip,ntheorem,roundcorner=0pt,
% backgroundcolor=pagebg,font=\color{orange}\sffamily, fontcolor=white
]{examplebox}{Exemple}[section]



\newcommand\R{\mathbb{R}}
\newcommand\Z{\mathbb{Z}}
\newcommand\N{\mathbb{N}}
\newcommand\E{\mathbb{E}}
\newcommand\F{\mathcal{F}}
\newcommand\cH{\mathcal{H}}
\newcommand\V{\mathbb{V}}
\newcommand\dmo{ ^{-1} }
\newcommand\kapa{\kappa}
\newcommand\im{Im}
\newcommand\hs{\mathcal{H}}





\usepackage{soul}

\makeatletter
\newcommand*{\whiten}[1]{\llap{\textcolor{white}{{\the\SOUL@token}}\hspace{#1pt}}}
\DeclareRobustCommand*\myul{%
    \def\SOUL@everyspace{\underline{\space}\kern\z@}%
    \def\SOUL@everytoken{%
     \setbox0=\hbox{\the\SOUL@token}%
     \ifdim\dp0>\z@
        \raisebox{\dp0}{\underline{\phantom{\the\SOUL@token}}}%
        \whiten{1}\whiten{0}%
        \whiten{-1}\whiten{-2}%
        \llap{\the\SOUL@token}%
     \else
        \underline{\the\SOUL@token}%
     \fi}%
\SOUL@}
\makeatother

\newcommand*{\demp}{\fontfamily{lmtt}\selectfont}

\DeclareTextFontCommand{\textdemp}{\demp}

\begin{document}

\ifcomment
Multiline
comment
\fi
\ifcomment
\myul{Typesetting test}
% \color[rgb]{1,1,1}
$∑_i^n≠ 60º±∞π∆¬≈√j∫h≤≥µ$

$\CR \R\pro\ind\pro\gS\pro
\mqty[a&b\\c&d]$
$\pro\mathbb{P}$
$\dd{x}$

  \[
    \alpha(x)=\left\{
                \begin{array}{ll}
                  x\\
                  \frac{1}{1+e^{-kx}}\\
                  \frac{e^x-e^{-x}}{e^x+e^{-x}}
                \end{array}
              \right.
  \]

  $\expval{x}$
  
  $\chi_\rho(ghg\dmo)=\Tr(\rho_{ghg\dmo})=\Tr(\rho_g\circ\rho_h\circ\rho\dmo_g)=\Tr(\rho_h)\overset{\mbox{\scalebox{0.5}{$\Tr(AB)=\Tr(BA)$}}}{=}\chi_\rho(h)$
  	$\mathop{\oplus}_{\substack{x\in X}}$

$\mat(\rho_g)=(a_{ij}(g))_{\scriptsize \substack{1\leq i\leq d \\ 1\leq j\leq d}}$ et $\mat(\rho'_g)=(a'_{ij}(g))_{\scriptsize \substack{1\leq i'\leq d' \\ 1\leq j'\leq d'}}$



\[\int_a^b{\mathbb{R}^2}g(u, v)\dd{P_{XY}}(u, v)=\iint g(u,v) f_{XY}(u, v)\dd \lambda(u) \dd \lambda(v)\]
$$\lim_{x\to\infty} f(x)$$	
$$\iiiint_V \mu(t,u,v,w) \,dt\,du\,dv\,dw$$
$$\sum_{n=1}^{\infty} 2^{-n} = 1$$	
\begin{definition}
	Si $X$ et $Y$ sont 2 v.a. ou definit la \textsc{Covariance} entre $X$ et $Y$ comme
	$\cov(X,Y)\overset{\text{def}}{=}\E\left[(X-\E(X))(Y-\E(Y))\right]=\E(XY)-\E(X)\E(Y)$.
\end{definition}
\fi
\pagebreak

% \tableofcontents

% insert your code here
%\input{./algebra/main.tex}
%\input{./geometrie-differentielle/main.tex}
%\input{./probabilite/main.tex}
%\input{./analyse-fonctionnelle/main.tex}
% \input{./Analyse-convexe-et-dualite-en-optimisation/main.tex}
%\input{./tikz/main.tex}
%\input{./Theorie-du-distributions/main.tex}
%\input{./optimisation/mine.tex}
 \input{./modelisation/main.tex}

% yves.aubry@univ-tln.fr : algebra

\end{document}


% yves.aubry@univ-tln.fr : algebra

\end{document}

%% !TEX encoding = UTF-8 Unicode
% !TEX TS-program = xelatex

\documentclass[french]{report}

%\usepackage[utf8]{inputenc}
%\usepackage[T1]{fontenc}
\usepackage{babel}


\newif\ifcomment
%\commenttrue # Show comments

\usepackage{physics}
\usepackage{amssymb}


\usepackage{amsthm}
% \usepackage{thmtools}
\usepackage{mathtools}
\usepackage{amsfonts}

\usepackage{color}

\usepackage{tikz}

\usepackage{geometry}
\geometry{a5paper, margin=0.1in, right=1cm}

\usepackage{dsfont}

\usepackage{graphicx}
\graphicspath{ {images/} }

\usepackage{faktor}

\usepackage{IEEEtrantools}
\usepackage{enumerate}   
\usepackage[PostScript=dvips]{"/Users/aware/Documents/Courses/diagrams"}


\newtheorem{theorem}{Théorème}[section]
\renewcommand{\thetheorem}{\arabic{theorem}}
\newtheorem{lemme}{Lemme}[section]
\renewcommand{\thelemme}{\arabic{lemme}}
\newtheorem{proposition}{Proposition}[section]
\renewcommand{\theproposition}{\arabic{proposition}}
\newtheorem{notations}{Notations}[section]
\newtheorem{problem}{Problème}[section]
\newtheorem{corollary}{Corollaire}[theorem]
\renewcommand{\thecorollary}{\arabic{corollary}}
\newtheorem{property}{Propriété}[section]
\newtheorem{objective}{Objectif}[section]

\theoremstyle{definition}
\newtheorem{definition}{Définition}[section]
\renewcommand{\thedefinition}{\arabic{definition}}
\newtheorem{exercise}{Exercice}[chapter]
\renewcommand{\theexercise}{\arabic{exercise}}
\newtheorem{example}{Exemple}[chapter]
\renewcommand{\theexample}{\arabic{example}}
\newtheorem*{solution}{Solution}
\newtheorem*{application}{Application}
\newtheorem*{notation}{Notation}
\newtheorem*{vocabulary}{Vocabulaire}
\newtheorem*{properties}{Propriétés}



\theoremstyle{remark}
\newtheorem*{remark}{Remarque}
\newtheorem*{rappel}{Rappel}


\usepackage{etoolbox}
\AtBeginEnvironment{exercise}{\small}
\AtBeginEnvironment{example}{\small}

\usepackage{cases}
\usepackage[red]{mypack}

\usepackage[framemethod=TikZ]{mdframed}

\definecolor{bg}{rgb}{0.4,0.25,0.95}
\definecolor{pagebg}{rgb}{0,0,0.5}
\surroundwithmdframed[
   topline=false,
   rightline=false,
   bottomline=false,
   leftmargin=\parindent,
   skipabove=8pt,
   skipbelow=8pt,
   linecolor=blue,
   innerbottommargin=10pt,
   % backgroundcolor=bg,font=\color{orange}\sffamily, fontcolor=white
]{definition}

\usepackage{empheq}
\usepackage[most]{tcolorbox}

\newtcbox{\mymath}[1][]{%
    nobeforeafter, math upper, tcbox raise base,
    enhanced, colframe=blue!30!black,
    colback=red!10, boxrule=1pt,
    #1}

\usepackage{unixode}


\DeclareMathOperator{\ord}{ord}
\DeclareMathOperator{\orb}{orb}
\DeclareMathOperator{\stab}{stab}
\DeclareMathOperator{\Stab}{stab}
\DeclareMathOperator{\ppcm}{ppcm}
\DeclareMathOperator{\conj}{Conj}
\DeclareMathOperator{\End}{End}
\DeclareMathOperator{\rot}{rot}
\DeclareMathOperator{\trs}{trace}
\DeclareMathOperator{\Ind}{Ind}
\DeclareMathOperator{\mat}{Mat}
\DeclareMathOperator{\id}{Id}
\DeclareMathOperator{\vect}{vect}
\DeclareMathOperator{\img}{img}
\DeclareMathOperator{\cov}{Cov}
\DeclareMathOperator{\dist}{dist}
\DeclareMathOperator{\irr}{Irr}
\DeclareMathOperator{\image}{Im}
\DeclareMathOperator{\pd}{\partial}
\DeclareMathOperator{\epi}{epi}
\DeclareMathOperator{\Argmin}{Argmin}
\DeclareMathOperator{\dom}{dom}
\DeclareMathOperator{\proj}{proj}
\DeclareMathOperator{\ctg}{ctg}
\DeclareMathOperator{\supp}{supp}
\DeclareMathOperator{\argmin}{argmin}
\DeclareMathOperator{\mult}{mult}
\DeclareMathOperator{\ch}{ch}
\DeclareMathOperator{\sh}{sh}
\DeclareMathOperator{\rang}{rang}
\DeclareMathOperator{\diam}{diam}
\DeclareMathOperator{\Epigraphe}{Epigraphe}




\usepackage{xcolor}
\everymath{\color{blue}}
%\everymath{\color[rgb]{0,1,1}}
%\pagecolor[rgb]{0,0,0.5}


\newcommand*{\pdtest}[3][]{\ensuremath{\frac{\partial^{#1} #2}{\partial #3}}}

\newcommand*{\deffunc}[6][]{\ensuremath{
\begin{array}{rcl}
#2 : #3 &\rightarrow& #4\\
#5 &\mapsto& #6
\end{array}
}}

\newcommand{\eqcolon}{\mathrel{\resizebox{\widthof{$\mathord{=}$}}{\height}{ $\!\!=\!\!\resizebox{1.2\width}{0.8\height}{\raisebox{0.23ex}{$\mathop{:}$}}\!\!$ }}}
\newcommand{\coloneq}{\mathrel{\resizebox{\widthof{$\mathord{=}$}}{\height}{ $\!\!\resizebox{1.2\width}{0.8\height}{\raisebox{0.23ex}{$\mathop{:}$}}\!\!=\!\!$ }}}
\newcommand{\eqcolonl}{\ensuremath{\mathrel{=\!\!\mathop{:}}}}
\newcommand{\coloneql}{\ensuremath{\mathrel{\mathop{:} \!\! =}}}
\newcommand{\vc}[1]{% inline column vector
  \left(\begin{smallmatrix}#1\end{smallmatrix}\right)%
}
\newcommand{\vr}[1]{% inline row vector
  \begin{smallmatrix}(\,#1\,)\end{smallmatrix}%
}
\makeatletter
\newcommand*{\defeq}{\ =\mathrel{\rlap{%
                     \raisebox{0.3ex}{$\m@th\cdot$}}%
                     \raisebox{-0.3ex}{$\m@th\cdot$}}%
                     }
\makeatother

\newcommand{\mathcircle}[1]{% inline row vector
 \overset{\circ}{#1}
}
\newcommand{\ulim}{% low limit
 \underline{\lim}
}
\newcommand{\ssi}{% iff
\iff
}
\newcommand{\ps}[2]{
\expval{#1 | #2}
}
\newcommand{\df}[1]{
\mqty{#1}
}
\newcommand{\n}[1]{
\norm{#1}
}
\newcommand{\sys}[1]{
\left\{\smqty{#1}\right.
}


\newcommand{\eqdef}{\ensuremath{\overset{\text{def}}=}}


\def\Circlearrowright{\ensuremath{%
  \rotatebox[origin=c]{230}{$\circlearrowright$}}}

\newcommand\ct[1]{\text{\rmfamily\upshape #1}}
\newcommand\question[1]{ {\color{red} ...!? \small #1}}
\newcommand\caz[1]{\left\{\begin{array} #1 \end{array}\right.}
\newcommand\const{\text{\rmfamily\upshape const}}
\newcommand\toP{ \overset{\pro}{\to}}
\newcommand\toPP{ \overset{\text{PP}}{\to}}
\newcommand{\oeq}{\mathrel{\text{\textcircled{$=$}}}}





\usepackage{xcolor}
% \usepackage[normalem]{ulem}
\usepackage{lipsum}
\makeatletter
% \newcommand\colorwave[1][blue]{\bgroup \markoverwith{\lower3.5\p@\hbox{\sixly \textcolor{#1}{\char58}}}\ULon}
%\font\sixly=lasy6 % does not re-load if already loaded, so no memory problem.

\newmdtheoremenv[
linewidth= 1pt,linecolor= blue,%
leftmargin=20,rightmargin=20,innertopmargin=0pt, innerrightmargin=40,%
tikzsetting = { draw=lightgray, line width = 0.3pt,dashed,%
dash pattern = on 15pt off 3pt},%
splittopskip=\topskip,skipbelow=\baselineskip,%
skipabove=\baselineskip,ntheorem,roundcorner=0pt,
% backgroundcolor=pagebg,font=\color{orange}\sffamily, fontcolor=white
]{examplebox}{Exemple}[section]



\newcommand\R{\mathbb{R}}
\newcommand\Z{\mathbb{Z}}
\newcommand\N{\mathbb{N}}
\newcommand\E{\mathbb{E}}
\newcommand\F{\mathcal{F}}
\newcommand\cH{\mathcal{H}}
\newcommand\V{\mathbb{V}}
\newcommand\dmo{ ^{-1} }
\newcommand\kapa{\kappa}
\newcommand\im{Im}
\newcommand\hs{\mathcal{H}}





\usepackage{soul}

\makeatletter
\newcommand*{\whiten}[1]{\llap{\textcolor{white}{{\the\SOUL@token}}\hspace{#1pt}}}
\DeclareRobustCommand*\myul{%
    \def\SOUL@everyspace{\underline{\space}\kern\z@}%
    \def\SOUL@everytoken{%
     \setbox0=\hbox{\the\SOUL@token}%
     \ifdim\dp0>\z@
        \raisebox{\dp0}{\underline{\phantom{\the\SOUL@token}}}%
        \whiten{1}\whiten{0}%
        \whiten{-1}\whiten{-2}%
        \llap{\the\SOUL@token}%
     \else
        \underline{\the\SOUL@token}%
     \fi}%
\SOUL@}
\makeatother

\newcommand*{\demp}{\fontfamily{lmtt}\selectfont}

\DeclareTextFontCommand{\textdemp}{\demp}

\begin{document}

\ifcomment
Multiline
comment
\fi
\ifcomment
\myul{Typesetting test}
% \color[rgb]{1,1,1}
$∑_i^n≠ 60º±∞π∆¬≈√j∫h≤≥µ$

$\CR \R\pro\ind\pro\gS\pro
\mqty[a&b\\c&d]$
$\pro\mathbb{P}$
$\dd{x}$

  \[
    \alpha(x)=\left\{
                \begin{array}{ll}
                  x\\
                  \frac{1}{1+e^{-kx}}\\
                  \frac{e^x-e^{-x}}{e^x+e^{-x}}
                \end{array}
              \right.
  \]

  $\expval{x}$
  
  $\chi_\rho(ghg\dmo)=\Tr(\rho_{ghg\dmo})=\Tr(\rho_g\circ\rho_h\circ\rho\dmo_g)=\Tr(\rho_h)\overset{\mbox{\scalebox{0.5}{$\Tr(AB)=\Tr(BA)$}}}{=}\chi_\rho(h)$
  	$\mathop{\oplus}_{\substack{x\in X}}$

$\mat(\rho_g)=(a_{ij}(g))_{\scriptsize \substack{1\leq i\leq d \\ 1\leq j\leq d}}$ et $\mat(\rho'_g)=(a'_{ij}(g))_{\scriptsize \substack{1\leq i'\leq d' \\ 1\leq j'\leq d'}}$



\[\int_a^b{\mathbb{R}^2}g(u, v)\dd{P_{XY}}(u, v)=\iint g(u,v) f_{XY}(u, v)\dd \lambda(u) \dd \lambda(v)\]
$$\lim_{x\to\infty} f(x)$$	
$$\iiiint_V \mu(t,u,v,w) \,dt\,du\,dv\,dw$$
$$\sum_{n=1}^{\infty} 2^{-n} = 1$$	
\begin{definition}
	Si $X$ et $Y$ sont 2 v.a. ou definit la \textsc{Covariance} entre $X$ et $Y$ comme
	$\cov(X,Y)\overset{\text{def}}{=}\E\left[(X-\E(X))(Y-\E(Y))\right]=\E(XY)-\E(X)\E(Y)$.
\end{definition}
\fi
\pagebreak

% \tableofcontents

% insert your code here
%% !TEX encoding = UTF-8 Unicode
% !TEX TS-program = xelatex

\documentclass[french]{report}

%\usepackage[utf8]{inputenc}
%\usepackage[T1]{fontenc}
\usepackage{babel}


\newif\ifcomment
%\commenttrue # Show comments

\usepackage{physics}
\usepackage{amssymb}


\usepackage{amsthm}
% \usepackage{thmtools}
\usepackage{mathtools}
\usepackage{amsfonts}

\usepackage{color}

\usepackage{tikz}

\usepackage{geometry}
\geometry{a5paper, margin=0.1in, right=1cm}

\usepackage{dsfont}

\usepackage{graphicx}
\graphicspath{ {images/} }

\usepackage{faktor}

\usepackage{IEEEtrantools}
\usepackage{enumerate}   
\usepackage[PostScript=dvips]{"/Users/aware/Documents/Courses/diagrams"}


\newtheorem{theorem}{Théorème}[section]
\renewcommand{\thetheorem}{\arabic{theorem}}
\newtheorem{lemme}{Lemme}[section]
\renewcommand{\thelemme}{\arabic{lemme}}
\newtheorem{proposition}{Proposition}[section]
\renewcommand{\theproposition}{\arabic{proposition}}
\newtheorem{notations}{Notations}[section]
\newtheorem{problem}{Problème}[section]
\newtheorem{corollary}{Corollaire}[theorem]
\renewcommand{\thecorollary}{\arabic{corollary}}
\newtheorem{property}{Propriété}[section]
\newtheorem{objective}{Objectif}[section]

\theoremstyle{definition}
\newtheorem{definition}{Définition}[section]
\renewcommand{\thedefinition}{\arabic{definition}}
\newtheorem{exercise}{Exercice}[chapter]
\renewcommand{\theexercise}{\arabic{exercise}}
\newtheorem{example}{Exemple}[chapter]
\renewcommand{\theexample}{\arabic{example}}
\newtheorem*{solution}{Solution}
\newtheorem*{application}{Application}
\newtheorem*{notation}{Notation}
\newtheorem*{vocabulary}{Vocabulaire}
\newtheorem*{properties}{Propriétés}



\theoremstyle{remark}
\newtheorem*{remark}{Remarque}
\newtheorem*{rappel}{Rappel}


\usepackage{etoolbox}
\AtBeginEnvironment{exercise}{\small}
\AtBeginEnvironment{example}{\small}

\usepackage{cases}
\usepackage[red]{mypack}

\usepackage[framemethod=TikZ]{mdframed}

\definecolor{bg}{rgb}{0.4,0.25,0.95}
\definecolor{pagebg}{rgb}{0,0,0.5}
\surroundwithmdframed[
   topline=false,
   rightline=false,
   bottomline=false,
   leftmargin=\parindent,
   skipabove=8pt,
   skipbelow=8pt,
   linecolor=blue,
   innerbottommargin=10pt,
   % backgroundcolor=bg,font=\color{orange}\sffamily, fontcolor=white
]{definition}

\usepackage{empheq}
\usepackage[most]{tcolorbox}

\newtcbox{\mymath}[1][]{%
    nobeforeafter, math upper, tcbox raise base,
    enhanced, colframe=blue!30!black,
    colback=red!10, boxrule=1pt,
    #1}

\usepackage{unixode}


\DeclareMathOperator{\ord}{ord}
\DeclareMathOperator{\orb}{orb}
\DeclareMathOperator{\stab}{stab}
\DeclareMathOperator{\Stab}{stab}
\DeclareMathOperator{\ppcm}{ppcm}
\DeclareMathOperator{\conj}{Conj}
\DeclareMathOperator{\End}{End}
\DeclareMathOperator{\rot}{rot}
\DeclareMathOperator{\trs}{trace}
\DeclareMathOperator{\Ind}{Ind}
\DeclareMathOperator{\mat}{Mat}
\DeclareMathOperator{\id}{Id}
\DeclareMathOperator{\vect}{vect}
\DeclareMathOperator{\img}{img}
\DeclareMathOperator{\cov}{Cov}
\DeclareMathOperator{\dist}{dist}
\DeclareMathOperator{\irr}{Irr}
\DeclareMathOperator{\image}{Im}
\DeclareMathOperator{\pd}{\partial}
\DeclareMathOperator{\epi}{epi}
\DeclareMathOperator{\Argmin}{Argmin}
\DeclareMathOperator{\dom}{dom}
\DeclareMathOperator{\proj}{proj}
\DeclareMathOperator{\ctg}{ctg}
\DeclareMathOperator{\supp}{supp}
\DeclareMathOperator{\argmin}{argmin}
\DeclareMathOperator{\mult}{mult}
\DeclareMathOperator{\ch}{ch}
\DeclareMathOperator{\sh}{sh}
\DeclareMathOperator{\rang}{rang}
\DeclareMathOperator{\diam}{diam}
\DeclareMathOperator{\Epigraphe}{Epigraphe}




\usepackage{xcolor}
\everymath{\color{blue}}
%\everymath{\color[rgb]{0,1,1}}
%\pagecolor[rgb]{0,0,0.5}


\newcommand*{\pdtest}[3][]{\ensuremath{\frac{\partial^{#1} #2}{\partial #3}}}

\newcommand*{\deffunc}[6][]{\ensuremath{
\begin{array}{rcl}
#2 : #3 &\rightarrow& #4\\
#5 &\mapsto& #6
\end{array}
}}

\newcommand{\eqcolon}{\mathrel{\resizebox{\widthof{$\mathord{=}$}}{\height}{ $\!\!=\!\!\resizebox{1.2\width}{0.8\height}{\raisebox{0.23ex}{$\mathop{:}$}}\!\!$ }}}
\newcommand{\coloneq}{\mathrel{\resizebox{\widthof{$\mathord{=}$}}{\height}{ $\!\!\resizebox{1.2\width}{0.8\height}{\raisebox{0.23ex}{$\mathop{:}$}}\!\!=\!\!$ }}}
\newcommand{\eqcolonl}{\ensuremath{\mathrel{=\!\!\mathop{:}}}}
\newcommand{\coloneql}{\ensuremath{\mathrel{\mathop{:} \!\! =}}}
\newcommand{\vc}[1]{% inline column vector
  \left(\begin{smallmatrix}#1\end{smallmatrix}\right)%
}
\newcommand{\vr}[1]{% inline row vector
  \begin{smallmatrix}(\,#1\,)\end{smallmatrix}%
}
\makeatletter
\newcommand*{\defeq}{\ =\mathrel{\rlap{%
                     \raisebox{0.3ex}{$\m@th\cdot$}}%
                     \raisebox{-0.3ex}{$\m@th\cdot$}}%
                     }
\makeatother

\newcommand{\mathcircle}[1]{% inline row vector
 \overset{\circ}{#1}
}
\newcommand{\ulim}{% low limit
 \underline{\lim}
}
\newcommand{\ssi}{% iff
\iff
}
\newcommand{\ps}[2]{
\expval{#1 | #2}
}
\newcommand{\df}[1]{
\mqty{#1}
}
\newcommand{\n}[1]{
\norm{#1}
}
\newcommand{\sys}[1]{
\left\{\smqty{#1}\right.
}


\newcommand{\eqdef}{\ensuremath{\overset{\text{def}}=}}


\def\Circlearrowright{\ensuremath{%
  \rotatebox[origin=c]{230}{$\circlearrowright$}}}

\newcommand\ct[1]{\text{\rmfamily\upshape #1}}
\newcommand\question[1]{ {\color{red} ...!? \small #1}}
\newcommand\caz[1]{\left\{\begin{array} #1 \end{array}\right.}
\newcommand\const{\text{\rmfamily\upshape const}}
\newcommand\toP{ \overset{\pro}{\to}}
\newcommand\toPP{ \overset{\text{PP}}{\to}}
\newcommand{\oeq}{\mathrel{\text{\textcircled{$=$}}}}





\usepackage{xcolor}
% \usepackage[normalem]{ulem}
\usepackage{lipsum}
\makeatletter
% \newcommand\colorwave[1][blue]{\bgroup \markoverwith{\lower3.5\p@\hbox{\sixly \textcolor{#1}{\char58}}}\ULon}
%\font\sixly=lasy6 % does not re-load if already loaded, so no memory problem.

\newmdtheoremenv[
linewidth= 1pt,linecolor= blue,%
leftmargin=20,rightmargin=20,innertopmargin=0pt, innerrightmargin=40,%
tikzsetting = { draw=lightgray, line width = 0.3pt,dashed,%
dash pattern = on 15pt off 3pt},%
splittopskip=\topskip,skipbelow=\baselineskip,%
skipabove=\baselineskip,ntheorem,roundcorner=0pt,
% backgroundcolor=pagebg,font=\color{orange}\sffamily, fontcolor=white
]{examplebox}{Exemple}[section]



\newcommand\R{\mathbb{R}}
\newcommand\Z{\mathbb{Z}}
\newcommand\N{\mathbb{N}}
\newcommand\E{\mathbb{E}}
\newcommand\F{\mathcal{F}}
\newcommand\cH{\mathcal{H}}
\newcommand\V{\mathbb{V}}
\newcommand\dmo{ ^{-1} }
\newcommand\kapa{\kappa}
\newcommand\im{Im}
\newcommand\hs{\mathcal{H}}





\usepackage{soul}

\makeatletter
\newcommand*{\whiten}[1]{\llap{\textcolor{white}{{\the\SOUL@token}}\hspace{#1pt}}}
\DeclareRobustCommand*\myul{%
    \def\SOUL@everyspace{\underline{\space}\kern\z@}%
    \def\SOUL@everytoken{%
     \setbox0=\hbox{\the\SOUL@token}%
     \ifdim\dp0>\z@
        \raisebox{\dp0}{\underline{\phantom{\the\SOUL@token}}}%
        \whiten{1}\whiten{0}%
        \whiten{-1}\whiten{-2}%
        \llap{\the\SOUL@token}%
     \else
        \underline{\the\SOUL@token}%
     \fi}%
\SOUL@}
\makeatother

\newcommand*{\demp}{\fontfamily{lmtt}\selectfont}

\DeclareTextFontCommand{\textdemp}{\demp}

\begin{document}

\ifcomment
Multiline
comment
\fi
\ifcomment
\myul{Typesetting test}
% \color[rgb]{1,1,1}
$∑_i^n≠ 60º±∞π∆¬≈√j∫h≤≥µ$

$\CR \R\pro\ind\pro\gS\pro
\mqty[a&b\\c&d]$
$\pro\mathbb{P}$
$\dd{x}$

  \[
    \alpha(x)=\left\{
                \begin{array}{ll}
                  x\\
                  \frac{1}{1+e^{-kx}}\\
                  \frac{e^x-e^{-x}}{e^x+e^{-x}}
                \end{array}
              \right.
  \]

  $\expval{x}$
  
  $\chi_\rho(ghg\dmo)=\Tr(\rho_{ghg\dmo})=\Tr(\rho_g\circ\rho_h\circ\rho\dmo_g)=\Tr(\rho_h)\overset{\mbox{\scalebox{0.5}{$\Tr(AB)=\Tr(BA)$}}}{=}\chi_\rho(h)$
  	$\mathop{\oplus}_{\substack{x\in X}}$

$\mat(\rho_g)=(a_{ij}(g))_{\scriptsize \substack{1\leq i\leq d \\ 1\leq j\leq d}}$ et $\mat(\rho'_g)=(a'_{ij}(g))_{\scriptsize \substack{1\leq i'\leq d' \\ 1\leq j'\leq d'}}$



\[\int_a^b{\mathbb{R}^2}g(u, v)\dd{P_{XY}}(u, v)=\iint g(u,v) f_{XY}(u, v)\dd \lambda(u) \dd \lambda(v)\]
$$\lim_{x\to\infty} f(x)$$	
$$\iiiint_V \mu(t,u,v,w) \,dt\,du\,dv\,dw$$
$$\sum_{n=1}^{\infty} 2^{-n} = 1$$	
\begin{definition}
	Si $X$ et $Y$ sont 2 v.a. ou definit la \textsc{Covariance} entre $X$ et $Y$ comme
	$\cov(X,Y)\overset{\text{def}}{=}\E\left[(X-\E(X))(Y-\E(Y))\right]=\E(XY)-\E(X)\E(Y)$.
\end{definition}
\fi
\pagebreak

% \tableofcontents

% insert your code here
%\input{./algebra/main.tex}
%\input{./geometrie-differentielle/main.tex}
%\input{./probabilite/main.tex}
%\input{./analyse-fonctionnelle/main.tex}
% \input{./Analyse-convexe-et-dualite-en-optimisation/main.tex}
%\input{./tikz/main.tex}
%\input{./Theorie-du-distributions/main.tex}
%\input{./optimisation/mine.tex}
 \input{./modelisation/main.tex}

% yves.aubry@univ-tln.fr : algebra

\end{document}

%% !TEX encoding = UTF-8 Unicode
% !TEX TS-program = xelatex

\documentclass[french]{report}

%\usepackage[utf8]{inputenc}
%\usepackage[T1]{fontenc}
\usepackage{babel}


\newif\ifcomment
%\commenttrue # Show comments

\usepackage{physics}
\usepackage{amssymb}


\usepackage{amsthm}
% \usepackage{thmtools}
\usepackage{mathtools}
\usepackage{amsfonts}

\usepackage{color}

\usepackage{tikz}

\usepackage{geometry}
\geometry{a5paper, margin=0.1in, right=1cm}

\usepackage{dsfont}

\usepackage{graphicx}
\graphicspath{ {images/} }

\usepackage{faktor}

\usepackage{IEEEtrantools}
\usepackage{enumerate}   
\usepackage[PostScript=dvips]{"/Users/aware/Documents/Courses/diagrams"}


\newtheorem{theorem}{Théorème}[section]
\renewcommand{\thetheorem}{\arabic{theorem}}
\newtheorem{lemme}{Lemme}[section]
\renewcommand{\thelemme}{\arabic{lemme}}
\newtheorem{proposition}{Proposition}[section]
\renewcommand{\theproposition}{\arabic{proposition}}
\newtheorem{notations}{Notations}[section]
\newtheorem{problem}{Problème}[section]
\newtheorem{corollary}{Corollaire}[theorem]
\renewcommand{\thecorollary}{\arabic{corollary}}
\newtheorem{property}{Propriété}[section]
\newtheorem{objective}{Objectif}[section]

\theoremstyle{definition}
\newtheorem{definition}{Définition}[section]
\renewcommand{\thedefinition}{\arabic{definition}}
\newtheorem{exercise}{Exercice}[chapter]
\renewcommand{\theexercise}{\arabic{exercise}}
\newtheorem{example}{Exemple}[chapter]
\renewcommand{\theexample}{\arabic{example}}
\newtheorem*{solution}{Solution}
\newtheorem*{application}{Application}
\newtheorem*{notation}{Notation}
\newtheorem*{vocabulary}{Vocabulaire}
\newtheorem*{properties}{Propriétés}



\theoremstyle{remark}
\newtheorem*{remark}{Remarque}
\newtheorem*{rappel}{Rappel}


\usepackage{etoolbox}
\AtBeginEnvironment{exercise}{\small}
\AtBeginEnvironment{example}{\small}

\usepackage{cases}
\usepackage[red]{mypack}

\usepackage[framemethod=TikZ]{mdframed}

\definecolor{bg}{rgb}{0.4,0.25,0.95}
\definecolor{pagebg}{rgb}{0,0,0.5}
\surroundwithmdframed[
   topline=false,
   rightline=false,
   bottomline=false,
   leftmargin=\parindent,
   skipabove=8pt,
   skipbelow=8pt,
   linecolor=blue,
   innerbottommargin=10pt,
   % backgroundcolor=bg,font=\color{orange}\sffamily, fontcolor=white
]{definition}

\usepackage{empheq}
\usepackage[most]{tcolorbox}

\newtcbox{\mymath}[1][]{%
    nobeforeafter, math upper, tcbox raise base,
    enhanced, colframe=blue!30!black,
    colback=red!10, boxrule=1pt,
    #1}

\usepackage{unixode}


\DeclareMathOperator{\ord}{ord}
\DeclareMathOperator{\orb}{orb}
\DeclareMathOperator{\stab}{stab}
\DeclareMathOperator{\Stab}{stab}
\DeclareMathOperator{\ppcm}{ppcm}
\DeclareMathOperator{\conj}{Conj}
\DeclareMathOperator{\End}{End}
\DeclareMathOperator{\rot}{rot}
\DeclareMathOperator{\trs}{trace}
\DeclareMathOperator{\Ind}{Ind}
\DeclareMathOperator{\mat}{Mat}
\DeclareMathOperator{\id}{Id}
\DeclareMathOperator{\vect}{vect}
\DeclareMathOperator{\img}{img}
\DeclareMathOperator{\cov}{Cov}
\DeclareMathOperator{\dist}{dist}
\DeclareMathOperator{\irr}{Irr}
\DeclareMathOperator{\image}{Im}
\DeclareMathOperator{\pd}{\partial}
\DeclareMathOperator{\epi}{epi}
\DeclareMathOperator{\Argmin}{Argmin}
\DeclareMathOperator{\dom}{dom}
\DeclareMathOperator{\proj}{proj}
\DeclareMathOperator{\ctg}{ctg}
\DeclareMathOperator{\supp}{supp}
\DeclareMathOperator{\argmin}{argmin}
\DeclareMathOperator{\mult}{mult}
\DeclareMathOperator{\ch}{ch}
\DeclareMathOperator{\sh}{sh}
\DeclareMathOperator{\rang}{rang}
\DeclareMathOperator{\diam}{diam}
\DeclareMathOperator{\Epigraphe}{Epigraphe}




\usepackage{xcolor}
\everymath{\color{blue}}
%\everymath{\color[rgb]{0,1,1}}
%\pagecolor[rgb]{0,0,0.5}


\newcommand*{\pdtest}[3][]{\ensuremath{\frac{\partial^{#1} #2}{\partial #3}}}

\newcommand*{\deffunc}[6][]{\ensuremath{
\begin{array}{rcl}
#2 : #3 &\rightarrow& #4\\
#5 &\mapsto& #6
\end{array}
}}

\newcommand{\eqcolon}{\mathrel{\resizebox{\widthof{$\mathord{=}$}}{\height}{ $\!\!=\!\!\resizebox{1.2\width}{0.8\height}{\raisebox{0.23ex}{$\mathop{:}$}}\!\!$ }}}
\newcommand{\coloneq}{\mathrel{\resizebox{\widthof{$\mathord{=}$}}{\height}{ $\!\!\resizebox{1.2\width}{0.8\height}{\raisebox{0.23ex}{$\mathop{:}$}}\!\!=\!\!$ }}}
\newcommand{\eqcolonl}{\ensuremath{\mathrel{=\!\!\mathop{:}}}}
\newcommand{\coloneql}{\ensuremath{\mathrel{\mathop{:} \!\! =}}}
\newcommand{\vc}[1]{% inline column vector
  \left(\begin{smallmatrix}#1\end{smallmatrix}\right)%
}
\newcommand{\vr}[1]{% inline row vector
  \begin{smallmatrix}(\,#1\,)\end{smallmatrix}%
}
\makeatletter
\newcommand*{\defeq}{\ =\mathrel{\rlap{%
                     \raisebox{0.3ex}{$\m@th\cdot$}}%
                     \raisebox{-0.3ex}{$\m@th\cdot$}}%
                     }
\makeatother

\newcommand{\mathcircle}[1]{% inline row vector
 \overset{\circ}{#1}
}
\newcommand{\ulim}{% low limit
 \underline{\lim}
}
\newcommand{\ssi}{% iff
\iff
}
\newcommand{\ps}[2]{
\expval{#1 | #2}
}
\newcommand{\df}[1]{
\mqty{#1}
}
\newcommand{\n}[1]{
\norm{#1}
}
\newcommand{\sys}[1]{
\left\{\smqty{#1}\right.
}


\newcommand{\eqdef}{\ensuremath{\overset{\text{def}}=}}


\def\Circlearrowright{\ensuremath{%
  \rotatebox[origin=c]{230}{$\circlearrowright$}}}

\newcommand\ct[1]{\text{\rmfamily\upshape #1}}
\newcommand\question[1]{ {\color{red} ...!? \small #1}}
\newcommand\caz[1]{\left\{\begin{array} #1 \end{array}\right.}
\newcommand\const{\text{\rmfamily\upshape const}}
\newcommand\toP{ \overset{\pro}{\to}}
\newcommand\toPP{ \overset{\text{PP}}{\to}}
\newcommand{\oeq}{\mathrel{\text{\textcircled{$=$}}}}





\usepackage{xcolor}
% \usepackage[normalem]{ulem}
\usepackage{lipsum}
\makeatletter
% \newcommand\colorwave[1][blue]{\bgroup \markoverwith{\lower3.5\p@\hbox{\sixly \textcolor{#1}{\char58}}}\ULon}
%\font\sixly=lasy6 % does not re-load if already loaded, so no memory problem.

\newmdtheoremenv[
linewidth= 1pt,linecolor= blue,%
leftmargin=20,rightmargin=20,innertopmargin=0pt, innerrightmargin=40,%
tikzsetting = { draw=lightgray, line width = 0.3pt,dashed,%
dash pattern = on 15pt off 3pt},%
splittopskip=\topskip,skipbelow=\baselineskip,%
skipabove=\baselineskip,ntheorem,roundcorner=0pt,
% backgroundcolor=pagebg,font=\color{orange}\sffamily, fontcolor=white
]{examplebox}{Exemple}[section]



\newcommand\R{\mathbb{R}}
\newcommand\Z{\mathbb{Z}}
\newcommand\N{\mathbb{N}}
\newcommand\E{\mathbb{E}}
\newcommand\F{\mathcal{F}}
\newcommand\cH{\mathcal{H}}
\newcommand\V{\mathbb{V}}
\newcommand\dmo{ ^{-1} }
\newcommand\kapa{\kappa}
\newcommand\im{Im}
\newcommand\hs{\mathcal{H}}





\usepackage{soul}

\makeatletter
\newcommand*{\whiten}[1]{\llap{\textcolor{white}{{\the\SOUL@token}}\hspace{#1pt}}}
\DeclareRobustCommand*\myul{%
    \def\SOUL@everyspace{\underline{\space}\kern\z@}%
    \def\SOUL@everytoken{%
     \setbox0=\hbox{\the\SOUL@token}%
     \ifdim\dp0>\z@
        \raisebox{\dp0}{\underline{\phantom{\the\SOUL@token}}}%
        \whiten{1}\whiten{0}%
        \whiten{-1}\whiten{-2}%
        \llap{\the\SOUL@token}%
     \else
        \underline{\the\SOUL@token}%
     \fi}%
\SOUL@}
\makeatother

\newcommand*{\demp}{\fontfamily{lmtt}\selectfont}

\DeclareTextFontCommand{\textdemp}{\demp}

\begin{document}

\ifcomment
Multiline
comment
\fi
\ifcomment
\myul{Typesetting test}
% \color[rgb]{1,1,1}
$∑_i^n≠ 60º±∞π∆¬≈√j∫h≤≥µ$

$\CR \R\pro\ind\pro\gS\pro
\mqty[a&b\\c&d]$
$\pro\mathbb{P}$
$\dd{x}$

  \[
    \alpha(x)=\left\{
                \begin{array}{ll}
                  x\\
                  \frac{1}{1+e^{-kx}}\\
                  \frac{e^x-e^{-x}}{e^x+e^{-x}}
                \end{array}
              \right.
  \]

  $\expval{x}$
  
  $\chi_\rho(ghg\dmo)=\Tr(\rho_{ghg\dmo})=\Tr(\rho_g\circ\rho_h\circ\rho\dmo_g)=\Tr(\rho_h)\overset{\mbox{\scalebox{0.5}{$\Tr(AB)=\Tr(BA)$}}}{=}\chi_\rho(h)$
  	$\mathop{\oplus}_{\substack{x\in X}}$

$\mat(\rho_g)=(a_{ij}(g))_{\scriptsize \substack{1\leq i\leq d \\ 1\leq j\leq d}}$ et $\mat(\rho'_g)=(a'_{ij}(g))_{\scriptsize \substack{1\leq i'\leq d' \\ 1\leq j'\leq d'}}$



\[\int_a^b{\mathbb{R}^2}g(u, v)\dd{P_{XY}}(u, v)=\iint g(u,v) f_{XY}(u, v)\dd \lambda(u) \dd \lambda(v)\]
$$\lim_{x\to\infty} f(x)$$	
$$\iiiint_V \mu(t,u,v,w) \,dt\,du\,dv\,dw$$
$$\sum_{n=1}^{\infty} 2^{-n} = 1$$	
\begin{definition}
	Si $X$ et $Y$ sont 2 v.a. ou definit la \textsc{Covariance} entre $X$ et $Y$ comme
	$\cov(X,Y)\overset{\text{def}}{=}\E\left[(X-\E(X))(Y-\E(Y))\right]=\E(XY)-\E(X)\E(Y)$.
\end{definition}
\fi
\pagebreak

% \tableofcontents

% insert your code here
%\input{./algebra/main.tex}
%\input{./geometrie-differentielle/main.tex}
%\input{./probabilite/main.tex}
%\input{./analyse-fonctionnelle/main.tex}
% \input{./Analyse-convexe-et-dualite-en-optimisation/main.tex}
%\input{./tikz/main.tex}
%\input{./Theorie-du-distributions/main.tex}
%\input{./optimisation/mine.tex}
 \input{./modelisation/main.tex}

% yves.aubry@univ-tln.fr : algebra

\end{document}

%% !TEX encoding = UTF-8 Unicode
% !TEX TS-program = xelatex

\documentclass[french]{report}

%\usepackage[utf8]{inputenc}
%\usepackage[T1]{fontenc}
\usepackage{babel}


\newif\ifcomment
%\commenttrue # Show comments

\usepackage{physics}
\usepackage{amssymb}


\usepackage{amsthm}
% \usepackage{thmtools}
\usepackage{mathtools}
\usepackage{amsfonts}

\usepackage{color}

\usepackage{tikz}

\usepackage{geometry}
\geometry{a5paper, margin=0.1in, right=1cm}

\usepackage{dsfont}

\usepackage{graphicx}
\graphicspath{ {images/} }

\usepackage{faktor}

\usepackage{IEEEtrantools}
\usepackage{enumerate}   
\usepackage[PostScript=dvips]{"/Users/aware/Documents/Courses/diagrams"}


\newtheorem{theorem}{Théorème}[section]
\renewcommand{\thetheorem}{\arabic{theorem}}
\newtheorem{lemme}{Lemme}[section]
\renewcommand{\thelemme}{\arabic{lemme}}
\newtheorem{proposition}{Proposition}[section]
\renewcommand{\theproposition}{\arabic{proposition}}
\newtheorem{notations}{Notations}[section]
\newtheorem{problem}{Problème}[section]
\newtheorem{corollary}{Corollaire}[theorem]
\renewcommand{\thecorollary}{\arabic{corollary}}
\newtheorem{property}{Propriété}[section]
\newtheorem{objective}{Objectif}[section]

\theoremstyle{definition}
\newtheorem{definition}{Définition}[section]
\renewcommand{\thedefinition}{\arabic{definition}}
\newtheorem{exercise}{Exercice}[chapter]
\renewcommand{\theexercise}{\arabic{exercise}}
\newtheorem{example}{Exemple}[chapter]
\renewcommand{\theexample}{\arabic{example}}
\newtheorem*{solution}{Solution}
\newtheorem*{application}{Application}
\newtheorem*{notation}{Notation}
\newtheorem*{vocabulary}{Vocabulaire}
\newtheorem*{properties}{Propriétés}



\theoremstyle{remark}
\newtheorem*{remark}{Remarque}
\newtheorem*{rappel}{Rappel}


\usepackage{etoolbox}
\AtBeginEnvironment{exercise}{\small}
\AtBeginEnvironment{example}{\small}

\usepackage{cases}
\usepackage[red]{mypack}

\usepackage[framemethod=TikZ]{mdframed}

\definecolor{bg}{rgb}{0.4,0.25,0.95}
\definecolor{pagebg}{rgb}{0,0,0.5}
\surroundwithmdframed[
   topline=false,
   rightline=false,
   bottomline=false,
   leftmargin=\parindent,
   skipabove=8pt,
   skipbelow=8pt,
   linecolor=blue,
   innerbottommargin=10pt,
   % backgroundcolor=bg,font=\color{orange}\sffamily, fontcolor=white
]{definition}

\usepackage{empheq}
\usepackage[most]{tcolorbox}

\newtcbox{\mymath}[1][]{%
    nobeforeafter, math upper, tcbox raise base,
    enhanced, colframe=blue!30!black,
    colback=red!10, boxrule=1pt,
    #1}

\usepackage{unixode}


\DeclareMathOperator{\ord}{ord}
\DeclareMathOperator{\orb}{orb}
\DeclareMathOperator{\stab}{stab}
\DeclareMathOperator{\Stab}{stab}
\DeclareMathOperator{\ppcm}{ppcm}
\DeclareMathOperator{\conj}{Conj}
\DeclareMathOperator{\End}{End}
\DeclareMathOperator{\rot}{rot}
\DeclareMathOperator{\trs}{trace}
\DeclareMathOperator{\Ind}{Ind}
\DeclareMathOperator{\mat}{Mat}
\DeclareMathOperator{\id}{Id}
\DeclareMathOperator{\vect}{vect}
\DeclareMathOperator{\img}{img}
\DeclareMathOperator{\cov}{Cov}
\DeclareMathOperator{\dist}{dist}
\DeclareMathOperator{\irr}{Irr}
\DeclareMathOperator{\image}{Im}
\DeclareMathOperator{\pd}{\partial}
\DeclareMathOperator{\epi}{epi}
\DeclareMathOperator{\Argmin}{Argmin}
\DeclareMathOperator{\dom}{dom}
\DeclareMathOperator{\proj}{proj}
\DeclareMathOperator{\ctg}{ctg}
\DeclareMathOperator{\supp}{supp}
\DeclareMathOperator{\argmin}{argmin}
\DeclareMathOperator{\mult}{mult}
\DeclareMathOperator{\ch}{ch}
\DeclareMathOperator{\sh}{sh}
\DeclareMathOperator{\rang}{rang}
\DeclareMathOperator{\diam}{diam}
\DeclareMathOperator{\Epigraphe}{Epigraphe}




\usepackage{xcolor}
\everymath{\color{blue}}
%\everymath{\color[rgb]{0,1,1}}
%\pagecolor[rgb]{0,0,0.5}


\newcommand*{\pdtest}[3][]{\ensuremath{\frac{\partial^{#1} #2}{\partial #3}}}

\newcommand*{\deffunc}[6][]{\ensuremath{
\begin{array}{rcl}
#2 : #3 &\rightarrow& #4\\
#5 &\mapsto& #6
\end{array}
}}

\newcommand{\eqcolon}{\mathrel{\resizebox{\widthof{$\mathord{=}$}}{\height}{ $\!\!=\!\!\resizebox{1.2\width}{0.8\height}{\raisebox{0.23ex}{$\mathop{:}$}}\!\!$ }}}
\newcommand{\coloneq}{\mathrel{\resizebox{\widthof{$\mathord{=}$}}{\height}{ $\!\!\resizebox{1.2\width}{0.8\height}{\raisebox{0.23ex}{$\mathop{:}$}}\!\!=\!\!$ }}}
\newcommand{\eqcolonl}{\ensuremath{\mathrel{=\!\!\mathop{:}}}}
\newcommand{\coloneql}{\ensuremath{\mathrel{\mathop{:} \!\! =}}}
\newcommand{\vc}[1]{% inline column vector
  \left(\begin{smallmatrix}#1\end{smallmatrix}\right)%
}
\newcommand{\vr}[1]{% inline row vector
  \begin{smallmatrix}(\,#1\,)\end{smallmatrix}%
}
\makeatletter
\newcommand*{\defeq}{\ =\mathrel{\rlap{%
                     \raisebox{0.3ex}{$\m@th\cdot$}}%
                     \raisebox{-0.3ex}{$\m@th\cdot$}}%
                     }
\makeatother

\newcommand{\mathcircle}[1]{% inline row vector
 \overset{\circ}{#1}
}
\newcommand{\ulim}{% low limit
 \underline{\lim}
}
\newcommand{\ssi}{% iff
\iff
}
\newcommand{\ps}[2]{
\expval{#1 | #2}
}
\newcommand{\df}[1]{
\mqty{#1}
}
\newcommand{\n}[1]{
\norm{#1}
}
\newcommand{\sys}[1]{
\left\{\smqty{#1}\right.
}


\newcommand{\eqdef}{\ensuremath{\overset{\text{def}}=}}


\def\Circlearrowright{\ensuremath{%
  \rotatebox[origin=c]{230}{$\circlearrowright$}}}

\newcommand\ct[1]{\text{\rmfamily\upshape #1}}
\newcommand\question[1]{ {\color{red} ...!? \small #1}}
\newcommand\caz[1]{\left\{\begin{array} #1 \end{array}\right.}
\newcommand\const{\text{\rmfamily\upshape const}}
\newcommand\toP{ \overset{\pro}{\to}}
\newcommand\toPP{ \overset{\text{PP}}{\to}}
\newcommand{\oeq}{\mathrel{\text{\textcircled{$=$}}}}





\usepackage{xcolor}
% \usepackage[normalem]{ulem}
\usepackage{lipsum}
\makeatletter
% \newcommand\colorwave[1][blue]{\bgroup \markoverwith{\lower3.5\p@\hbox{\sixly \textcolor{#1}{\char58}}}\ULon}
%\font\sixly=lasy6 % does not re-load if already loaded, so no memory problem.

\newmdtheoremenv[
linewidth= 1pt,linecolor= blue,%
leftmargin=20,rightmargin=20,innertopmargin=0pt, innerrightmargin=40,%
tikzsetting = { draw=lightgray, line width = 0.3pt,dashed,%
dash pattern = on 15pt off 3pt},%
splittopskip=\topskip,skipbelow=\baselineskip,%
skipabove=\baselineskip,ntheorem,roundcorner=0pt,
% backgroundcolor=pagebg,font=\color{orange}\sffamily, fontcolor=white
]{examplebox}{Exemple}[section]



\newcommand\R{\mathbb{R}}
\newcommand\Z{\mathbb{Z}}
\newcommand\N{\mathbb{N}}
\newcommand\E{\mathbb{E}}
\newcommand\F{\mathcal{F}}
\newcommand\cH{\mathcal{H}}
\newcommand\V{\mathbb{V}}
\newcommand\dmo{ ^{-1} }
\newcommand\kapa{\kappa}
\newcommand\im{Im}
\newcommand\hs{\mathcal{H}}





\usepackage{soul}

\makeatletter
\newcommand*{\whiten}[1]{\llap{\textcolor{white}{{\the\SOUL@token}}\hspace{#1pt}}}
\DeclareRobustCommand*\myul{%
    \def\SOUL@everyspace{\underline{\space}\kern\z@}%
    \def\SOUL@everytoken{%
     \setbox0=\hbox{\the\SOUL@token}%
     \ifdim\dp0>\z@
        \raisebox{\dp0}{\underline{\phantom{\the\SOUL@token}}}%
        \whiten{1}\whiten{0}%
        \whiten{-1}\whiten{-2}%
        \llap{\the\SOUL@token}%
     \else
        \underline{\the\SOUL@token}%
     \fi}%
\SOUL@}
\makeatother

\newcommand*{\demp}{\fontfamily{lmtt}\selectfont}

\DeclareTextFontCommand{\textdemp}{\demp}

\begin{document}

\ifcomment
Multiline
comment
\fi
\ifcomment
\myul{Typesetting test}
% \color[rgb]{1,1,1}
$∑_i^n≠ 60º±∞π∆¬≈√j∫h≤≥µ$

$\CR \R\pro\ind\pro\gS\pro
\mqty[a&b\\c&d]$
$\pro\mathbb{P}$
$\dd{x}$

  \[
    \alpha(x)=\left\{
                \begin{array}{ll}
                  x\\
                  \frac{1}{1+e^{-kx}}\\
                  \frac{e^x-e^{-x}}{e^x+e^{-x}}
                \end{array}
              \right.
  \]

  $\expval{x}$
  
  $\chi_\rho(ghg\dmo)=\Tr(\rho_{ghg\dmo})=\Tr(\rho_g\circ\rho_h\circ\rho\dmo_g)=\Tr(\rho_h)\overset{\mbox{\scalebox{0.5}{$\Tr(AB)=\Tr(BA)$}}}{=}\chi_\rho(h)$
  	$\mathop{\oplus}_{\substack{x\in X}}$

$\mat(\rho_g)=(a_{ij}(g))_{\scriptsize \substack{1\leq i\leq d \\ 1\leq j\leq d}}$ et $\mat(\rho'_g)=(a'_{ij}(g))_{\scriptsize \substack{1\leq i'\leq d' \\ 1\leq j'\leq d'}}$



\[\int_a^b{\mathbb{R}^2}g(u, v)\dd{P_{XY}}(u, v)=\iint g(u,v) f_{XY}(u, v)\dd \lambda(u) \dd \lambda(v)\]
$$\lim_{x\to\infty} f(x)$$	
$$\iiiint_V \mu(t,u,v,w) \,dt\,du\,dv\,dw$$
$$\sum_{n=1}^{\infty} 2^{-n} = 1$$	
\begin{definition}
	Si $X$ et $Y$ sont 2 v.a. ou definit la \textsc{Covariance} entre $X$ et $Y$ comme
	$\cov(X,Y)\overset{\text{def}}{=}\E\left[(X-\E(X))(Y-\E(Y))\right]=\E(XY)-\E(X)\E(Y)$.
\end{definition}
\fi
\pagebreak

% \tableofcontents

% insert your code here
%\input{./algebra/main.tex}
%\input{./geometrie-differentielle/main.tex}
%\input{./probabilite/main.tex}
%\input{./analyse-fonctionnelle/main.tex}
% \input{./Analyse-convexe-et-dualite-en-optimisation/main.tex}
%\input{./tikz/main.tex}
%\input{./Theorie-du-distributions/main.tex}
%\input{./optimisation/mine.tex}
 \input{./modelisation/main.tex}

% yves.aubry@univ-tln.fr : algebra

\end{document}

%% !TEX encoding = UTF-8 Unicode
% !TEX TS-program = xelatex

\documentclass[french]{report}

%\usepackage[utf8]{inputenc}
%\usepackage[T1]{fontenc}
\usepackage{babel}


\newif\ifcomment
%\commenttrue # Show comments

\usepackage{physics}
\usepackage{amssymb}


\usepackage{amsthm}
% \usepackage{thmtools}
\usepackage{mathtools}
\usepackage{amsfonts}

\usepackage{color}

\usepackage{tikz}

\usepackage{geometry}
\geometry{a5paper, margin=0.1in, right=1cm}

\usepackage{dsfont}

\usepackage{graphicx}
\graphicspath{ {images/} }

\usepackage{faktor}

\usepackage{IEEEtrantools}
\usepackage{enumerate}   
\usepackage[PostScript=dvips]{"/Users/aware/Documents/Courses/diagrams"}


\newtheorem{theorem}{Théorème}[section]
\renewcommand{\thetheorem}{\arabic{theorem}}
\newtheorem{lemme}{Lemme}[section]
\renewcommand{\thelemme}{\arabic{lemme}}
\newtheorem{proposition}{Proposition}[section]
\renewcommand{\theproposition}{\arabic{proposition}}
\newtheorem{notations}{Notations}[section]
\newtheorem{problem}{Problème}[section]
\newtheorem{corollary}{Corollaire}[theorem]
\renewcommand{\thecorollary}{\arabic{corollary}}
\newtheorem{property}{Propriété}[section]
\newtheorem{objective}{Objectif}[section]

\theoremstyle{definition}
\newtheorem{definition}{Définition}[section]
\renewcommand{\thedefinition}{\arabic{definition}}
\newtheorem{exercise}{Exercice}[chapter]
\renewcommand{\theexercise}{\arabic{exercise}}
\newtheorem{example}{Exemple}[chapter]
\renewcommand{\theexample}{\arabic{example}}
\newtheorem*{solution}{Solution}
\newtheorem*{application}{Application}
\newtheorem*{notation}{Notation}
\newtheorem*{vocabulary}{Vocabulaire}
\newtheorem*{properties}{Propriétés}



\theoremstyle{remark}
\newtheorem*{remark}{Remarque}
\newtheorem*{rappel}{Rappel}


\usepackage{etoolbox}
\AtBeginEnvironment{exercise}{\small}
\AtBeginEnvironment{example}{\small}

\usepackage{cases}
\usepackage[red]{mypack}

\usepackage[framemethod=TikZ]{mdframed}

\definecolor{bg}{rgb}{0.4,0.25,0.95}
\definecolor{pagebg}{rgb}{0,0,0.5}
\surroundwithmdframed[
   topline=false,
   rightline=false,
   bottomline=false,
   leftmargin=\parindent,
   skipabove=8pt,
   skipbelow=8pt,
   linecolor=blue,
   innerbottommargin=10pt,
   % backgroundcolor=bg,font=\color{orange}\sffamily, fontcolor=white
]{definition}

\usepackage{empheq}
\usepackage[most]{tcolorbox}

\newtcbox{\mymath}[1][]{%
    nobeforeafter, math upper, tcbox raise base,
    enhanced, colframe=blue!30!black,
    colback=red!10, boxrule=1pt,
    #1}

\usepackage{unixode}


\DeclareMathOperator{\ord}{ord}
\DeclareMathOperator{\orb}{orb}
\DeclareMathOperator{\stab}{stab}
\DeclareMathOperator{\Stab}{stab}
\DeclareMathOperator{\ppcm}{ppcm}
\DeclareMathOperator{\conj}{Conj}
\DeclareMathOperator{\End}{End}
\DeclareMathOperator{\rot}{rot}
\DeclareMathOperator{\trs}{trace}
\DeclareMathOperator{\Ind}{Ind}
\DeclareMathOperator{\mat}{Mat}
\DeclareMathOperator{\id}{Id}
\DeclareMathOperator{\vect}{vect}
\DeclareMathOperator{\img}{img}
\DeclareMathOperator{\cov}{Cov}
\DeclareMathOperator{\dist}{dist}
\DeclareMathOperator{\irr}{Irr}
\DeclareMathOperator{\image}{Im}
\DeclareMathOperator{\pd}{\partial}
\DeclareMathOperator{\epi}{epi}
\DeclareMathOperator{\Argmin}{Argmin}
\DeclareMathOperator{\dom}{dom}
\DeclareMathOperator{\proj}{proj}
\DeclareMathOperator{\ctg}{ctg}
\DeclareMathOperator{\supp}{supp}
\DeclareMathOperator{\argmin}{argmin}
\DeclareMathOperator{\mult}{mult}
\DeclareMathOperator{\ch}{ch}
\DeclareMathOperator{\sh}{sh}
\DeclareMathOperator{\rang}{rang}
\DeclareMathOperator{\diam}{diam}
\DeclareMathOperator{\Epigraphe}{Epigraphe}




\usepackage{xcolor}
\everymath{\color{blue}}
%\everymath{\color[rgb]{0,1,1}}
%\pagecolor[rgb]{0,0,0.5}


\newcommand*{\pdtest}[3][]{\ensuremath{\frac{\partial^{#1} #2}{\partial #3}}}

\newcommand*{\deffunc}[6][]{\ensuremath{
\begin{array}{rcl}
#2 : #3 &\rightarrow& #4\\
#5 &\mapsto& #6
\end{array}
}}

\newcommand{\eqcolon}{\mathrel{\resizebox{\widthof{$\mathord{=}$}}{\height}{ $\!\!=\!\!\resizebox{1.2\width}{0.8\height}{\raisebox{0.23ex}{$\mathop{:}$}}\!\!$ }}}
\newcommand{\coloneq}{\mathrel{\resizebox{\widthof{$\mathord{=}$}}{\height}{ $\!\!\resizebox{1.2\width}{0.8\height}{\raisebox{0.23ex}{$\mathop{:}$}}\!\!=\!\!$ }}}
\newcommand{\eqcolonl}{\ensuremath{\mathrel{=\!\!\mathop{:}}}}
\newcommand{\coloneql}{\ensuremath{\mathrel{\mathop{:} \!\! =}}}
\newcommand{\vc}[1]{% inline column vector
  \left(\begin{smallmatrix}#1\end{smallmatrix}\right)%
}
\newcommand{\vr}[1]{% inline row vector
  \begin{smallmatrix}(\,#1\,)\end{smallmatrix}%
}
\makeatletter
\newcommand*{\defeq}{\ =\mathrel{\rlap{%
                     \raisebox{0.3ex}{$\m@th\cdot$}}%
                     \raisebox{-0.3ex}{$\m@th\cdot$}}%
                     }
\makeatother

\newcommand{\mathcircle}[1]{% inline row vector
 \overset{\circ}{#1}
}
\newcommand{\ulim}{% low limit
 \underline{\lim}
}
\newcommand{\ssi}{% iff
\iff
}
\newcommand{\ps}[2]{
\expval{#1 | #2}
}
\newcommand{\df}[1]{
\mqty{#1}
}
\newcommand{\n}[1]{
\norm{#1}
}
\newcommand{\sys}[1]{
\left\{\smqty{#1}\right.
}


\newcommand{\eqdef}{\ensuremath{\overset{\text{def}}=}}


\def\Circlearrowright{\ensuremath{%
  \rotatebox[origin=c]{230}{$\circlearrowright$}}}

\newcommand\ct[1]{\text{\rmfamily\upshape #1}}
\newcommand\question[1]{ {\color{red} ...!? \small #1}}
\newcommand\caz[1]{\left\{\begin{array} #1 \end{array}\right.}
\newcommand\const{\text{\rmfamily\upshape const}}
\newcommand\toP{ \overset{\pro}{\to}}
\newcommand\toPP{ \overset{\text{PP}}{\to}}
\newcommand{\oeq}{\mathrel{\text{\textcircled{$=$}}}}





\usepackage{xcolor}
% \usepackage[normalem]{ulem}
\usepackage{lipsum}
\makeatletter
% \newcommand\colorwave[1][blue]{\bgroup \markoverwith{\lower3.5\p@\hbox{\sixly \textcolor{#1}{\char58}}}\ULon}
%\font\sixly=lasy6 % does not re-load if already loaded, so no memory problem.

\newmdtheoremenv[
linewidth= 1pt,linecolor= blue,%
leftmargin=20,rightmargin=20,innertopmargin=0pt, innerrightmargin=40,%
tikzsetting = { draw=lightgray, line width = 0.3pt,dashed,%
dash pattern = on 15pt off 3pt},%
splittopskip=\topskip,skipbelow=\baselineskip,%
skipabove=\baselineskip,ntheorem,roundcorner=0pt,
% backgroundcolor=pagebg,font=\color{orange}\sffamily, fontcolor=white
]{examplebox}{Exemple}[section]



\newcommand\R{\mathbb{R}}
\newcommand\Z{\mathbb{Z}}
\newcommand\N{\mathbb{N}}
\newcommand\E{\mathbb{E}}
\newcommand\F{\mathcal{F}}
\newcommand\cH{\mathcal{H}}
\newcommand\V{\mathbb{V}}
\newcommand\dmo{ ^{-1} }
\newcommand\kapa{\kappa}
\newcommand\im{Im}
\newcommand\hs{\mathcal{H}}





\usepackage{soul}

\makeatletter
\newcommand*{\whiten}[1]{\llap{\textcolor{white}{{\the\SOUL@token}}\hspace{#1pt}}}
\DeclareRobustCommand*\myul{%
    \def\SOUL@everyspace{\underline{\space}\kern\z@}%
    \def\SOUL@everytoken{%
     \setbox0=\hbox{\the\SOUL@token}%
     \ifdim\dp0>\z@
        \raisebox{\dp0}{\underline{\phantom{\the\SOUL@token}}}%
        \whiten{1}\whiten{0}%
        \whiten{-1}\whiten{-2}%
        \llap{\the\SOUL@token}%
     \else
        \underline{\the\SOUL@token}%
     \fi}%
\SOUL@}
\makeatother

\newcommand*{\demp}{\fontfamily{lmtt}\selectfont}

\DeclareTextFontCommand{\textdemp}{\demp}

\begin{document}

\ifcomment
Multiline
comment
\fi
\ifcomment
\myul{Typesetting test}
% \color[rgb]{1,1,1}
$∑_i^n≠ 60º±∞π∆¬≈√j∫h≤≥µ$

$\CR \R\pro\ind\pro\gS\pro
\mqty[a&b\\c&d]$
$\pro\mathbb{P}$
$\dd{x}$

  \[
    \alpha(x)=\left\{
                \begin{array}{ll}
                  x\\
                  \frac{1}{1+e^{-kx}}\\
                  \frac{e^x-e^{-x}}{e^x+e^{-x}}
                \end{array}
              \right.
  \]

  $\expval{x}$
  
  $\chi_\rho(ghg\dmo)=\Tr(\rho_{ghg\dmo})=\Tr(\rho_g\circ\rho_h\circ\rho\dmo_g)=\Tr(\rho_h)\overset{\mbox{\scalebox{0.5}{$\Tr(AB)=\Tr(BA)$}}}{=}\chi_\rho(h)$
  	$\mathop{\oplus}_{\substack{x\in X}}$

$\mat(\rho_g)=(a_{ij}(g))_{\scriptsize \substack{1\leq i\leq d \\ 1\leq j\leq d}}$ et $\mat(\rho'_g)=(a'_{ij}(g))_{\scriptsize \substack{1\leq i'\leq d' \\ 1\leq j'\leq d'}}$



\[\int_a^b{\mathbb{R}^2}g(u, v)\dd{P_{XY}}(u, v)=\iint g(u,v) f_{XY}(u, v)\dd \lambda(u) \dd \lambda(v)\]
$$\lim_{x\to\infty} f(x)$$	
$$\iiiint_V \mu(t,u,v,w) \,dt\,du\,dv\,dw$$
$$\sum_{n=1}^{\infty} 2^{-n} = 1$$	
\begin{definition}
	Si $X$ et $Y$ sont 2 v.a. ou definit la \textsc{Covariance} entre $X$ et $Y$ comme
	$\cov(X,Y)\overset{\text{def}}{=}\E\left[(X-\E(X))(Y-\E(Y))\right]=\E(XY)-\E(X)\E(Y)$.
\end{definition}
\fi
\pagebreak

% \tableofcontents

% insert your code here
%\input{./algebra/main.tex}
%\input{./geometrie-differentielle/main.tex}
%\input{./probabilite/main.tex}
%\input{./analyse-fonctionnelle/main.tex}
% \input{./Analyse-convexe-et-dualite-en-optimisation/main.tex}
%\input{./tikz/main.tex}
%\input{./Theorie-du-distributions/main.tex}
%\input{./optimisation/mine.tex}
 \input{./modelisation/main.tex}

% yves.aubry@univ-tln.fr : algebra

\end{document}

% % !TEX encoding = UTF-8 Unicode
% !TEX TS-program = xelatex

\documentclass[french]{report}

%\usepackage[utf8]{inputenc}
%\usepackage[T1]{fontenc}
\usepackage{babel}


\newif\ifcomment
%\commenttrue # Show comments

\usepackage{physics}
\usepackage{amssymb}


\usepackage{amsthm}
% \usepackage{thmtools}
\usepackage{mathtools}
\usepackage{amsfonts}

\usepackage{color}

\usepackage{tikz}

\usepackage{geometry}
\geometry{a5paper, margin=0.1in, right=1cm}

\usepackage{dsfont}

\usepackage{graphicx}
\graphicspath{ {images/} }

\usepackage{faktor}

\usepackage{IEEEtrantools}
\usepackage{enumerate}   
\usepackage[PostScript=dvips]{"/Users/aware/Documents/Courses/diagrams"}


\newtheorem{theorem}{Théorème}[section]
\renewcommand{\thetheorem}{\arabic{theorem}}
\newtheorem{lemme}{Lemme}[section]
\renewcommand{\thelemme}{\arabic{lemme}}
\newtheorem{proposition}{Proposition}[section]
\renewcommand{\theproposition}{\arabic{proposition}}
\newtheorem{notations}{Notations}[section]
\newtheorem{problem}{Problème}[section]
\newtheorem{corollary}{Corollaire}[theorem]
\renewcommand{\thecorollary}{\arabic{corollary}}
\newtheorem{property}{Propriété}[section]
\newtheorem{objective}{Objectif}[section]

\theoremstyle{definition}
\newtheorem{definition}{Définition}[section]
\renewcommand{\thedefinition}{\arabic{definition}}
\newtheorem{exercise}{Exercice}[chapter]
\renewcommand{\theexercise}{\arabic{exercise}}
\newtheorem{example}{Exemple}[chapter]
\renewcommand{\theexample}{\arabic{example}}
\newtheorem*{solution}{Solution}
\newtheorem*{application}{Application}
\newtheorem*{notation}{Notation}
\newtheorem*{vocabulary}{Vocabulaire}
\newtheorem*{properties}{Propriétés}



\theoremstyle{remark}
\newtheorem*{remark}{Remarque}
\newtheorem*{rappel}{Rappel}


\usepackage{etoolbox}
\AtBeginEnvironment{exercise}{\small}
\AtBeginEnvironment{example}{\small}

\usepackage{cases}
\usepackage[red]{mypack}

\usepackage[framemethod=TikZ]{mdframed}

\definecolor{bg}{rgb}{0.4,0.25,0.95}
\definecolor{pagebg}{rgb}{0,0,0.5}
\surroundwithmdframed[
   topline=false,
   rightline=false,
   bottomline=false,
   leftmargin=\parindent,
   skipabove=8pt,
   skipbelow=8pt,
   linecolor=blue,
   innerbottommargin=10pt,
   % backgroundcolor=bg,font=\color{orange}\sffamily, fontcolor=white
]{definition}

\usepackage{empheq}
\usepackage[most]{tcolorbox}

\newtcbox{\mymath}[1][]{%
    nobeforeafter, math upper, tcbox raise base,
    enhanced, colframe=blue!30!black,
    colback=red!10, boxrule=1pt,
    #1}

\usepackage{unixode}


\DeclareMathOperator{\ord}{ord}
\DeclareMathOperator{\orb}{orb}
\DeclareMathOperator{\stab}{stab}
\DeclareMathOperator{\Stab}{stab}
\DeclareMathOperator{\ppcm}{ppcm}
\DeclareMathOperator{\conj}{Conj}
\DeclareMathOperator{\End}{End}
\DeclareMathOperator{\rot}{rot}
\DeclareMathOperator{\trs}{trace}
\DeclareMathOperator{\Ind}{Ind}
\DeclareMathOperator{\mat}{Mat}
\DeclareMathOperator{\id}{Id}
\DeclareMathOperator{\vect}{vect}
\DeclareMathOperator{\img}{img}
\DeclareMathOperator{\cov}{Cov}
\DeclareMathOperator{\dist}{dist}
\DeclareMathOperator{\irr}{Irr}
\DeclareMathOperator{\image}{Im}
\DeclareMathOperator{\pd}{\partial}
\DeclareMathOperator{\epi}{epi}
\DeclareMathOperator{\Argmin}{Argmin}
\DeclareMathOperator{\dom}{dom}
\DeclareMathOperator{\proj}{proj}
\DeclareMathOperator{\ctg}{ctg}
\DeclareMathOperator{\supp}{supp}
\DeclareMathOperator{\argmin}{argmin}
\DeclareMathOperator{\mult}{mult}
\DeclareMathOperator{\ch}{ch}
\DeclareMathOperator{\sh}{sh}
\DeclareMathOperator{\rang}{rang}
\DeclareMathOperator{\diam}{diam}
\DeclareMathOperator{\Epigraphe}{Epigraphe}




\usepackage{xcolor}
\everymath{\color{blue}}
%\everymath{\color[rgb]{0,1,1}}
%\pagecolor[rgb]{0,0,0.5}


\newcommand*{\pdtest}[3][]{\ensuremath{\frac{\partial^{#1} #2}{\partial #3}}}

\newcommand*{\deffunc}[6][]{\ensuremath{
\begin{array}{rcl}
#2 : #3 &\rightarrow& #4\\
#5 &\mapsto& #6
\end{array}
}}

\newcommand{\eqcolon}{\mathrel{\resizebox{\widthof{$\mathord{=}$}}{\height}{ $\!\!=\!\!\resizebox{1.2\width}{0.8\height}{\raisebox{0.23ex}{$\mathop{:}$}}\!\!$ }}}
\newcommand{\coloneq}{\mathrel{\resizebox{\widthof{$\mathord{=}$}}{\height}{ $\!\!\resizebox{1.2\width}{0.8\height}{\raisebox{0.23ex}{$\mathop{:}$}}\!\!=\!\!$ }}}
\newcommand{\eqcolonl}{\ensuremath{\mathrel{=\!\!\mathop{:}}}}
\newcommand{\coloneql}{\ensuremath{\mathrel{\mathop{:} \!\! =}}}
\newcommand{\vc}[1]{% inline column vector
  \left(\begin{smallmatrix}#1\end{smallmatrix}\right)%
}
\newcommand{\vr}[1]{% inline row vector
  \begin{smallmatrix}(\,#1\,)\end{smallmatrix}%
}
\makeatletter
\newcommand*{\defeq}{\ =\mathrel{\rlap{%
                     \raisebox{0.3ex}{$\m@th\cdot$}}%
                     \raisebox{-0.3ex}{$\m@th\cdot$}}%
                     }
\makeatother

\newcommand{\mathcircle}[1]{% inline row vector
 \overset{\circ}{#1}
}
\newcommand{\ulim}{% low limit
 \underline{\lim}
}
\newcommand{\ssi}{% iff
\iff
}
\newcommand{\ps}[2]{
\expval{#1 | #2}
}
\newcommand{\df}[1]{
\mqty{#1}
}
\newcommand{\n}[1]{
\norm{#1}
}
\newcommand{\sys}[1]{
\left\{\smqty{#1}\right.
}


\newcommand{\eqdef}{\ensuremath{\overset{\text{def}}=}}


\def\Circlearrowright{\ensuremath{%
  \rotatebox[origin=c]{230}{$\circlearrowright$}}}

\newcommand\ct[1]{\text{\rmfamily\upshape #1}}
\newcommand\question[1]{ {\color{red} ...!? \small #1}}
\newcommand\caz[1]{\left\{\begin{array} #1 \end{array}\right.}
\newcommand\const{\text{\rmfamily\upshape const}}
\newcommand\toP{ \overset{\pro}{\to}}
\newcommand\toPP{ \overset{\text{PP}}{\to}}
\newcommand{\oeq}{\mathrel{\text{\textcircled{$=$}}}}





\usepackage{xcolor}
% \usepackage[normalem]{ulem}
\usepackage{lipsum}
\makeatletter
% \newcommand\colorwave[1][blue]{\bgroup \markoverwith{\lower3.5\p@\hbox{\sixly \textcolor{#1}{\char58}}}\ULon}
%\font\sixly=lasy6 % does not re-load if already loaded, so no memory problem.

\newmdtheoremenv[
linewidth= 1pt,linecolor= blue,%
leftmargin=20,rightmargin=20,innertopmargin=0pt, innerrightmargin=40,%
tikzsetting = { draw=lightgray, line width = 0.3pt,dashed,%
dash pattern = on 15pt off 3pt},%
splittopskip=\topskip,skipbelow=\baselineskip,%
skipabove=\baselineskip,ntheorem,roundcorner=0pt,
% backgroundcolor=pagebg,font=\color{orange}\sffamily, fontcolor=white
]{examplebox}{Exemple}[section]



\newcommand\R{\mathbb{R}}
\newcommand\Z{\mathbb{Z}}
\newcommand\N{\mathbb{N}}
\newcommand\E{\mathbb{E}}
\newcommand\F{\mathcal{F}}
\newcommand\cH{\mathcal{H}}
\newcommand\V{\mathbb{V}}
\newcommand\dmo{ ^{-1} }
\newcommand\kapa{\kappa}
\newcommand\im{Im}
\newcommand\hs{\mathcal{H}}





\usepackage{soul}

\makeatletter
\newcommand*{\whiten}[1]{\llap{\textcolor{white}{{\the\SOUL@token}}\hspace{#1pt}}}
\DeclareRobustCommand*\myul{%
    \def\SOUL@everyspace{\underline{\space}\kern\z@}%
    \def\SOUL@everytoken{%
     \setbox0=\hbox{\the\SOUL@token}%
     \ifdim\dp0>\z@
        \raisebox{\dp0}{\underline{\phantom{\the\SOUL@token}}}%
        \whiten{1}\whiten{0}%
        \whiten{-1}\whiten{-2}%
        \llap{\the\SOUL@token}%
     \else
        \underline{\the\SOUL@token}%
     \fi}%
\SOUL@}
\makeatother

\newcommand*{\demp}{\fontfamily{lmtt}\selectfont}

\DeclareTextFontCommand{\textdemp}{\demp}

\begin{document}

\ifcomment
Multiline
comment
\fi
\ifcomment
\myul{Typesetting test}
% \color[rgb]{1,1,1}
$∑_i^n≠ 60º±∞π∆¬≈√j∫h≤≥µ$

$\CR \R\pro\ind\pro\gS\pro
\mqty[a&b\\c&d]$
$\pro\mathbb{P}$
$\dd{x}$

  \[
    \alpha(x)=\left\{
                \begin{array}{ll}
                  x\\
                  \frac{1}{1+e^{-kx}}\\
                  \frac{e^x-e^{-x}}{e^x+e^{-x}}
                \end{array}
              \right.
  \]

  $\expval{x}$
  
  $\chi_\rho(ghg\dmo)=\Tr(\rho_{ghg\dmo})=\Tr(\rho_g\circ\rho_h\circ\rho\dmo_g)=\Tr(\rho_h)\overset{\mbox{\scalebox{0.5}{$\Tr(AB)=\Tr(BA)$}}}{=}\chi_\rho(h)$
  	$\mathop{\oplus}_{\substack{x\in X}}$

$\mat(\rho_g)=(a_{ij}(g))_{\scriptsize \substack{1\leq i\leq d \\ 1\leq j\leq d}}$ et $\mat(\rho'_g)=(a'_{ij}(g))_{\scriptsize \substack{1\leq i'\leq d' \\ 1\leq j'\leq d'}}$



\[\int_a^b{\mathbb{R}^2}g(u, v)\dd{P_{XY}}(u, v)=\iint g(u,v) f_{XY}(u, v)\dd \lambda(u) \dd \lambda(v)\]
$$\lim_{x\to\infty} f(x)$$	
$$\iiiint_V \mu(t,u,v,w) \,dt\,du\,dv\,dw$$
$$\sum_{n=1}^{\infty} 2^{-n} = 1$$	
\begin{definition}
	Si $X$ et $Y$ sont 2 v.a. ou definit la \textsc{Covariance} entre $X$ et $Y$ comme
	$\cov(X,Y)\overset{\text{def}}{=}\E\left[(X-\E(X))(Y-\E(Y))\right]=\E(XY)-\E(X)\E(Y)$.
\end{definition}
\fi
\pagebreak

% \tableofcontents

% insert your code here
%\input{./algebra/main.tex}
%\input{./geometrie-differentielle/main.tex}
%\input{./probabilite/main.tex}
%\input{./analyse-fonctionnelle/main.tex}
% \input{./Analyse-convexe-et-dualite-en-optimisation/main.tex}
%\input{./tikz/main.tex}
%\input{./Theorie-du-distributions/main.tex}
%\input{./optimisation/mine.tex}
 \input{./modelisation/main.tex}

% yves.aubry@univ-tln.fr : algebra

\end{document}

%% !TEX encoding = UTF-8 Unicode
% !TEX TS-program = xelatex

\documentclass[french]{report}

%\usepackage[utf8]{inputenc}
%\usepackage[T1]{fontenc}
\usepackage{babel}


\newif\ifcomment
%\commenttrue # Show comments

\usepackage{physics}
\usepackage{amssymb}


\usepackage{amsthm}
% \usepackage{thmtools}
\usepackage{mathtools}
\usepackage{amsfonts}

\usepackage{color}

\usepackage{tikz}

\usepackage{geometry}
\geometry{a5paper, margin=0.1in, right=1cm}

\usepackage{dsfont}

\usepackage{graphicx}
\graphicspath{ {images/} }

\usepackage{faktor}

\usepackage{IEEEtrantools}
\usepackage{enumerate}   
\usepackage[PostScript=dvips]{"/Users/aware/Documents/Courses/diagrams"}


\newtheorem{theorem}{Théorème}[section]
\renewcommand{\thetheorem}{\arabic{theorem}}
\newtheorem{lemme}{Lemme}[section]
\renewcommand{\thelemme}{\arabic{lemme}}
\newtheorem{proposition}{Proposition}[section]
\renewcommand{\theproposition}{\arabic{proposition}}
\newtheorem{notations}{Notations}[section]
\newtheorem{problem}{Problème}[section]
\newtheorem{corollary}{Corollaire}[theorem]
\renewcommand{\thecorollary}{\arabic{corollary}}
\newtheorem{property}{Propriété}[section]
\newtheorem{objective}{Objectif}[section]

\theoremstyle{definition}
\newtheorem{definition}{Définition}[section]
\renewcommand{\thedefinition}{\arabic{definition}}
\newtheorem{exercise}{Exercice}[chapter]
\renewcommand{\theexercise}{\arabic{exercise}}
\newtheorem{example}{Exemple}[chapter]
\renewcommand{\theexample}{\arabic{example}}
\newtheorem*{solution}{Solution}
\newtheorem*{application}{Application}
\newtheorem*{notation}{Notation}
\newtheorem*{vocabulary}{Vocabulaire}
\newtheorem*{properties}{Propriétés}



\theoremstyle{remark}
\newtheorem*{remark}{Remarque}
\newtheorem*{rappel}{Rappel}


\usepackage{etoolbox}
\AtBeginEnvironment{exercise}{\small}
\AtBeginEnvironment{example}{\small}

\usepackage{cases}
\usepackage[red]{mypack}

\usepackage[framemethod=TikZ]{mdframed}

\definecolor{bg}{rgb}{0.4,0.25,0.95}
\definecolor{pagebg}{rgb}{0,0,0.5}
\surroundwithmdframed[
   topline=false,
   rightline=false,
   bottomline=false,
   leftmargin=\parindent,
   skipabove=8pt,
   skipbelow=8pt,
   linecolor=blue,
   innerbottommargin=10pt,
   % backgroundcolor=bg,font=\color{orange}\sffamily, fontcolor=white
]{definition}

\usepackage{empheq}
\usepackage[most]{tcolorbox}

\newtcbox{\mymath}[1][]{%
    nobeforeafter, math upper, tcbox raise base,
    enhanced, colframe=blue!30!black,
    colback=red!10, boxrule=1pt,
    #1}

\usepackage{unixode}


\DeclareMathOperator{\ord}{ord}
\DeclareMathOperator{\orb}{orb}
\DeclareMathOperator{\stab}{stab}
\DeclareMathOperator{\Stab}{stab}
\DeclareMathOperator{\ppcm}{ppcm}
\DeclareMathOperator{\conj}{Conj}
\DeclareMathOperator{\End}{End}
\DeclareMathOperator{\rot}{rot}
\DeclareMathOperator{\trs}{trace}
\DeclareMathOperator{\Ind}{Ind}
\DeclareMathOperator{\mat}{Mat}
\DeclareMathOperator{\id}{Id}
\DeclareMathOperator{\vect}{vect}
\DeclareMathOperator{\img}{img}
\DeclareMathOperator{\cov}{Cov}
\DeclareMathOperator{\dist}{dist}
\DeclareMathOperator{\irr}{Irr}
\DeclareMathOperator{\image}{Im}
\DeclareMathOperator{\pd}{\partial}
\DeclareMathOperator{\epi}{epi}
\DeclareMathOperator{\Argmin}{Argmin}
\DeclareMathOperator{\dom}{dom}
\DeclareMathOperator{\proj}{proj}
\DeclareMathOperator{\ctg}{ctg}
\DeclareMathOperator{\supp}{supp}
\DeclareMathOperator{\argmin}{argmin}
\DeclareMathOperator{\mult}{mult}
\DeclareMathOperator{\ch}{ch}
\DeclareMathOperator{\sh}{sh}
\DeclareMathOperator{\rang}{rang}
\DeclareMathOperator{\diam}{diam}
\DeclareMathOperator{\Epigraphe}{Epigraphe}




\usepackage{xcolor}
\everymath{\color{blue}}
%\everymath{\color[rgb]{0,1,1}}
%\pagecolor[rgb]{0,0,0.5}


\newcommand*{\pdtest}[3][]{\ensuremath{\frac{\partial^{#1} #2}{\partial #3}}}

\newcommand*{\deffunc}[6][]{\ensuremath{
\begin{array}{rcl}
#2 : #3 &\rightarrow& #4\\
#5 &\mapsto& #6
\end{array}
}}

\newcommand{\eqcolon}{\mathrel{\resizebox{\widthof{$\mathord{=}$}}{\height}{ $\!\!=\!\!\resizebox{1.2\width}{0.8\height}{\raisebox{0.23ex}{$\mathop{:}$}}\!\!$ }}}
\newcommand{\coloneq}{\mathrel{\resizebox{\widthof{$\mathord{=}$}}{\height}{ $\!\!\resizebox{1.2\width}{0.8\height}{\raisebox{0.23ex}{$\mathop{:}$}}\!\!=\!\!$ }}}
\newcommand{\eqcolonl}{\ensuremath{\mathrel{=\!\!\mathop{:}}}}
\newcommand{\coloneql}{\ensuremath{\mathrel{\mathop{:} \!\! =}}}
\newcommand{\vc}[1]{% inline column vector
  \left(\begin{smallmatrix}#1\end{smallmatrix}\right)%
}
\newcommand{\vr}[1]{% inline row vector
  \begin{smallmatrix}(\,#1\,)\end{smallmatrix}%
}
\makeatletter
\newcommand*{\defeq}{\ =\mathrel{\rlap{%
                     \raisebox{0.3ex}{$\m@th\cdot$}}%
                     \raisebox{-0.3ex}{$\m@th\cdot$}}%
                     }
\makeatother

\newcommand{\mathcircle}[1]{% inline row vector
 \overset{\circ}{#1}
}
\newcommand{\ulim}{% low limit
 \underline{\lim}
}
\newcommand{\ssi}{% iff
\iff
}
\newcommand{\ps}[2]{
\expval{#1 | #2}
}
\newcommand{\df}[1]{
\mqty{#1}
}
\newcommand{\n}[1]{
\norm{#1}
}
\newcommand{\sys}[1]{
\left\{\smqty{#1}\right.
}


\newcommand{\eqdef}{\ensuremath{\overset{\text{def}}=}}


\def\Circlearrowright{\ensuremath{%
  \rotatebox[origin=c]{230}{$\circlearrowright$}}}

\newcommand\ct[1]{\text{\rmfamily\upshape #1}}
\newcommand\question[1]{ {\color{red} ...!? \small #1}}
\newcommand\caz[1]{\left\{\begin{array} #1 \end{array}\right.}
\newcommand\const{\text{\rmfamily\upshape const}}
\newcommand\toP{ \overset{\pro}{\to}}
\newcommand\toPP{ \overset{\text{PP}}{\to}}
\newcommand{\oeq}{\mathrel{\text{\textcircled{$=$}}}}





\usepackage{xcolor}
% \usepackage[normalem]{ulem}
\usepackage{lipsum}
\makeatletter
% \newcommand\colorwave[1][blue]{\bgroup \markoverwith{\lower3.5\p@\hbox{\sixly \textcolor{#1}{\char58}}}\ULon}
%\font\sixly=lasy6 % does not re-load if already loaded, so no memory problem.

\newmdtheoremenv[
linewidth= 1pt,linecolor= blue,%
leftmargin=20,rightmargin=20,innertopmargin=0pt, innerrightmargin=40,%
tikzsetting = { draw=lightgray, line width = 0.3pt,dashed,%
dash pattern = on 15pt off 3pt},%
splittopskip=\topskip,skipbelow=\baselineskip,%
skipabove=\baselineskip,ntheorem,roundcorner=0pt,
% backgroundcolor=pagebg,font=\color{orange}\sffamily, fontcolor=white
]{examplebox}{Exemple}[section]



\newcommand\R{\mathbb{R}}
\newcommand\Z{\mathbb{Z}}
\newcommand\N{\mathbb{N}}
\newcommand\E{\mathbb{E}}
\newcommand\F{\mathcal{F}}
\newcommand\cH{\mathcal{H}}
\newcommand\V{\mathbb{V}}
\newcommand\dmo{ ^{-1} }
\newcommand\kapa{\kappa}
\newcommand\im{Im}
\newcommand\hs{\mathcal{H}}





\usepackage{soul}

\makeatletter
\newcommand*{\whiten}[1]{\llap{\textcolor{white}{{\the\SOUL@token}}\hspace{#1pt}}}
\DeclareRobustCommand*\myul{%
    \def\SOUL@everyspace{\underline{\space}\kern\z@}%
    \def\SOUL@everytoken{%
     \setbox0=\hbox{\the\SOUL@token}%
     \ifdim\dp0>\z@
        \raisebox{\dp0}{\underline{\phantom{\the\SOUL@token}}}%
        \whiten{1}\whiten{0}%
        \whiten{-1}\whiten{-2}%
        \llap{\the\SOUL@token}%
     \else
        \underline{\the\SOUL@token}%
     \fi}%
\SOUL@}
\makeatother

\newcommand*{\demp}{\fontfamily{lmtt}\selectfont}

\DeclareTextFontCommand{\textdemp}{\demp}

\begin{document}

\ifcomment
Multiline
comment
\fi
\ifcomment
\myul{Typesetting test}
% \color[rgb]{1,1,1}
$∑_i^n≠ 60º±∞π∆¬≈√j∫h≤≥µ$

$\CR \R\pro\ind\pro\gS\pro
\mqty[a&b\\c&d]$
$\pro\mathbb{P}$
$\dd{x}$

  \[
    \alpha(x)=\left\{
                \begin{array}{ll}
                  x\\
                  \frac{1}{1+e^{-kx}}\\
                  \frac{e^x-e^{-x}}{e^x+e^{-x}}
                \end{array}
              \right.
  \]

  $\expval{x}$
  
  $\chi_\rho(ghg\dmo)=\Tr(\rho_{ghg\dmo})=\Tr(\rho_g\circ\rho_h\circ\rho\dmo_g)=\Tr(\rho_h)\overset{\mbox{\scalebox{0.5}{$\Tr(AB)=\Tr(BA)$}}}{=}\chi_\rho(h)$
  	$\mathop{\oplus}_{\substack{x\in X}}$

$\mat(\rho_g)=(a_{ij}(g))_{\scriptsize \substack{1\leq i\leq d \\ 1\leq j\leq d}}$ et $\mat(\rho'_g)=(a'_{ij}(g))_{\scriptsize \substack{1\leq i'\leq d' \\ 1\leq j'\leq d'}}$



\[\int_a^b{\mathbb{R}^2}g(u, v)\dd{P_{XY}}(u, v)=\iint g(u,v) f_{XY}(u, v)\dd \lambda(u) \dd \lambda(v)\]
$$\lim_{x\to\infty} f(x)$$	
$$\iiiint_V \mu(t,u,v,w) \,dt\,du\,dv\,dw$$
$$\sum_{n=1}^{\infty} 2^{-n} = 1$$	
\begin{definition}
	Si $X$ et $Y$ sont 2 v.a. ou definit la \textsc{Covariance} entre $X$ et $Y$ comme
	$\cov(X,Y)\overset{\text{def}}{=}\E\left[(X-\E(X))(Y-\E(Y))\right]=\E(XY)-\E(X)\E(Y)$.
\end{definition}
\fi
\pagebreak

% \tableofcontents

% insert your code here
%\input{./algebra/main.tex}
%\input{./geometrie-differentielle/main.tex}
%\input{./probabilite/main.tex}
%\input{./analyse-fonctionnelle/main.tex}
% \input{./Analyse-convexe-et-dualite-en-optimisation/main.tex}
%\input{./tikz/main.tex}
%\input{./Theorie-du-distributions/main.tex}
%\input{./optimisation/mine.tex}
 \input{./modelisation/main.tex}

% yves.aubry@univ-tln.fr : algebra

\end{document}

%% !TEX encoding = UTF-8 Unicode
% !TEX TS-program = xelatex

\documentclass[french]{report}

%\usepackage[utf8]{inputenc}
%\usepackage[T1]{fontenc}
\usepackage{babel}


\newif\ifcomment
%\commenttrue # Show comments

\usepackage{physics}
\usepackage{amssymb}


\usepackage{amsthm}
% \usepackage{thmtools}
\usepackage{mathtools}
\usepackage{amsfonts}

\usepackage{color}

\usepackage{tikz}

\usepackage{geometry}
\geometry{a5paper, margin=0.1in, right=1cm}

\usepackage{dsfont}

\usepackage{graphicx}
\graphicspath{ {images/} }

\usepackage{faktor}

\usepackage{IEEEtrantools}
\usepackage{enumerate}   
\usepackage[PostScript=dvips]{"/Users/aware/Documents/Courses/diagrams"}


\newtheorem{theorem}{Théorème}[section]
\renewcommand{\thetheorem}{\arabic{theorem}}
\newtheorem{lemme}{Lemme}[section]
\renewcommand{\thelemme}{\arabic{lemme}}
\newtheorem{proposition}{Proposition}[section]
\renewcommand{\theproposition}{\arabic{proposition}}
\newtheorem{notations}{Notations}[section]
\newtheorem{problem}{Problème}[section]
\newtheorem{corollary}{Corollaire}[theorem]
\renewcommand{\thecorollary}{\arabic{corollary}}
\newtheorem{property}{Propriété}[section]
\newtheorem{objective}{Objectif}[section]

\theoremstyle{definition}
\newtheorem{definition}{Définition}[section]
\renewcommand{\thedefinition}{\arabic{definition}}
\newtheorem{exercise}{Exercice}[chapter]
\renewcommand{\theexercise}{\arabic{exercise}}
\newtheorem{example}{Exemple}[chapter]
\renewcommand{\theexample}{\arabic{example}}
\newtheorem*{solution}{Solution}
\newtheorem*{application}{Application}
\newtheorem*{notation}{Notation}
\newtheorem*{vocabulary}{Vocabulaire}
\newtheorem*{properties}{Propriétés}



\theoremstyle{remark}
\newtheorem*{remark}{Remarque}
\newtheorem*{rappel}{Rappel}


\usepackage{etoolbox}
\AtBeginEnvironment{exercise}{\small}
\AtBeginEnvironment{example}{\small}

\usepackage{cases}
\usepackage[red]{mypack}

\usepackage[framemethod=TikZ]{mdframed}

\definecolor{bg}{rgb}{0.4,0.25,0.95}
\definecolor{pagebg}{rgb}{0,0,0.5}
\surroundwithmdframed[
   topline=false,
   rightline=false,
   bottomline=false,
   leftmargin=\parindent,
   skipabove=8pt,
   skipbelow=8pt,
   linecolor=blue,
   innerbottommargin=10pt,
   % backgroundcolor=bg,font=\color{orange}\sffamily, fontcolor=white
]{definition}

\usepackage{empheq}
\usepackage[most]{tcolorbox}

\newtcbox{\mymath}[1][]{%
    nobeforeafter, math upper, tcbox raise base,
    enhanced, colframe=blue!30!black,
    colback=red!10, boxrule=1pt,
    #1}

\usepackage{unixode}


\DeclareMathOperator{\ord}{ord}
\DeclareMathOperator{\orb}{orb}
\DeclareMathOperator{\stab}{stab}
\DeclareMathOperator{\Stab}{stab}
\DeclareMathOperator{\ppcm}{ppcm}
\DeclareMathOperator{\conj}{Conj}
\DeclareMathOperator{\End}{End}
\DeclareMathOperator{\rot}{rot}
\DeclareMathOperator{\trs}{trace}
\DeclareMathOperator{\Ind}{Ind}
\DeclareMathOperator{\mat}{Mat}
\DeclareMathOperator{\id}{Id}
\DeclareMathOperator{\vect}{vect}
\DeclareMathOperator{\img}{img}
\DeclareMathOperator{\cov}{Cov}
\DeclareMathOperator{\dist}{dist}
\DeclareMathOperator{\irr}{Irr}
\DeclareMathOperator{\image}{Im}
\DeclareMathOperator{\pd}{\partial}
\DeclareMathOperator{\epi}{epi}
\DeclareMathOperator{\Argmin}{Argmin}
\DeclareMathOperator{\dom}{dom}
\DeclareMathOperator{\proj}{proj}
\DeclareMathOperator{\ctg}{ctg}
\DeclareMathOperator{\supp}{supp}
\DeclareMathOperator{\argmin}{argmin}
\DeclareMathOperator{\mult}{mult}
\DeclareMathOperator{\ch}{ch}
\DeclareMathOperator{\sh}{sh}
\DeclareMathOperator{\rang}{rang}
\DeclareMathOperator{\diam}{diam}
\DeclareMathOperator{\Epigraphe}{Epigraphe}




\usepackage{xcolor}
\everymath{\color{blue}}
%\everymath{\color[rgb]{0,1,1}}
%\pagecolor[rgb]{0,0,0.5}


\newcommand*{\pdtest}[3][]{\ensuremath{\frac{\partial^{#1} #2}{\partial #3}}}

\newcommand*{\deffunc}[6][]{\ensuremath{
\begin{array}{rcl}
#2 : #3 &\rightarrow& #4\\
#5 &\mapsto& #6
\end{array}
}}

\newcommand{\eqcolon}{\mathrel{\resizebox{\widthof{$\mathord{=}$}}{\height}{ $\!\!=\!\!\resizebox{1.2\width}{0.8\height}{\raisebox{0.23ex}{$\mathop{:}$}}\!\!$ }}}
\newcommand{\coloneq}{\mathrel{\resizebox{\widthof{$\mathord{=}$}}{\height}{ $\!\!\resizebox{1.2\width}{0.8\height}{\raisebox{0.23ex}{$\mathop{:}$}}\!\!=\!\!$ }}}
\newcommand{\eqcolonl}{\ensuremath{\mathrel{=\!\!\mathop{:}}}}
\newcommand{\coloneql}{\ensuremath{\mathrel{\mathop{:} \!\! =}}}
\newcommand{\vc}[1]{% inline column vector
  \left(\begin{smallmatrix}#1\end{smallmatrix}\right)%
}
\newcommand{\vr}[1]{% inline row vector
  \begin{smallmatrix}(\,#1\,)\end{smallmatrix}%
}
\makeatletter
\newcommand*{\defeq}{\ =\mathrel{\rlap{%
                     \raisebox{0.3ex}{$\m@th\cdot$}}%
                     \raisebox{-0.3ex}{$\m@th\cdot$}}%
                     }
\makeatother

\newcommand{\mathcircle}[1]{% inline row vector
 \overset{\circ}{#1}
}
\newcommand{\ulim}{% low limit
 \underline{\lim}
}
\newcommand{\ssi}{% iff
\iff
}
\newcommand{\ps}[2]{
\expval{#1 | #2}
}
\newcommand{\df}[1]{
\mqty{#1}
}
\newcommand{\n}[1]{
\norm{#1}
}
\newcommand{\sys}[1]{
\left\{\smqty{#1}\right.
}


\newcommand{\eqdef}{\ensuremath{\overset{\text{def}}=}}


\def\Circlearrowright{\ensuremath{%
  \rotatebox[origin=c]{230}{$\circlearrowright$}}}

\newcommand\ct[1]{\text{\rmfamily\upshape #1}}
\newcommand\question[1]{ {\color{red} ...!? \small #1}}
\newcommand\caz[1]{\left\{\begin{array} #1 \end{array}\right.}
\newcommand\const{\text{\rmfamily\upshape const}}
\newcommand\toP{ \overset{\pro}{\to}}
\newcommand\toPP{ \overset{\text{PP}}{\to}}
\newcommand{\oeq}{\mathrel{\text{\textcircled{$=$}}}}





\usepackage{xcolor}
% \usepackage[normalem]{ulem}
\usepackage{lipsum}
\makeatletter
% \newcommand\colorwave[1][blue]{\bgroup \markoverwith{\lower3.5\p@\hbox{\sixly \textcolor{#1}{\char58}}}\ULon}
%\font\sixly=lasy6 % does not re-load if already loaded, so no memory problem.

\newmdtheoremenv[
linewidth= 1pt,linecolor= blue,%
leftmargin=20,rightmargin=20,innertopmargin=0pt, innerrightmargin=40,%
tikzsetting = { draw=lightgray, line width = 0.3pt,dashed,%
dash pattern = on 15pt off 3pt},%
splittopskip=\topskip,skipbelow=\baselineskip,%
skipabove=\baselineskip,ntheorem,roundcorner=0pt,
% backgroundcolor=pagebg,font=\color{orange}\sffamily, fontcolor=white
]{examplebox}{Exemple}[section]



\newcommand\R{\mathbb{R}}
\newcommand\Z{\mathbb{Z}}
\newcommand\N{\mathbb{N}}
\newcommand\E{\mathbb{E}}
\newcommand\F{\mathcal{F}}
\newcommand\cH{\mathcal{H}}
\newcommand\V{\mathbb{V}}
\newcommand\dmo{ ^{-1} }
\newcommand\kapa{\kappa}
\newcommand\im{Im}
\newcommand\hs{\mathcal{H}}





\usepackage{soul}

\makeatletter
\newcommand*{\whiten}[1]{\llap{\textcolor{white}{{\the\SOUL@token}}\hspace{#1pt}}}
\DeclareRobustCommand*\myul{%
    \def\SOUL@everyspace{\underline{\space}\kern\z@}%
    \def\SOUL@everytoken{%
     \setbox0=\hbox{\the\SOUL@token}%
     \ifdim\dp0>\z@
        \raisebox{\dp0}{\underline{\phantom{\the\SOUL@token}}}%
        \whiten{1}\whiten{0}%
        \whiten{-1}\whiten{-2}%
        \llap{\the\SOUL@token}%
     \else
        \underline{\the\SOUL@token}%
     \fi}%
\SOUL@}
\makeatother

\newcommand*{\demp}{\fontfamily{lmtt}\selectfont}

\DeclareTextFontCommand{\textdemp}{\demp}

\begin{document}

\ifcomment
Multiline
comment
\fi
\ifcomment
\myul{Typesetting test}
% \color[rgb]{1,1,1}
$∑_i^n≠ 60º±∞π∆¬≈√j∫h≤≥µ$

$\CR \R\pro\ind\pro\gS\pro
\mqty[a&b\\c&d]$
$\pro\mathbb{P}$
$\dd{x}$

  \[
    \alpha(x)=\left\{
                \begin{array}{ll}
                  x\\
                  \frac{1}{1+e^{-kx}}\\
                  \frac{e^x-e^{-x}}{e^x+e^{-x}}
                \end{array}
              \right.
  \]

  $\expval{x}$
  
  $\chi_\rho(ghg\dmo)=\Tr(\rho_{ghg\dmo})=\Tr(\rho_g\circ\rho_h\circ\rho\dmo_g)=\Tr(\rho_h)\overset{\mbox{\scalebox{0.5}{$\Tr(AB)=\Tr(BA)$}}}{=}\chi_\rho(h)$
  	$\mathop{\oplus}_{\substack{x\in X}}$

$\mat(\rho_g)=(a_{ij}(g))_{\scriptsize \substack{1\leq i\leq d \\ 1\leq j\leq d}}$ et $\mat(\rho'_g)=(a'_{ij}(g))_{\scriptsize \substack{1\leq i'\leq d' \\ 1\leq j'\leq d'}}$



\[\int_a^b{\mathbb{R}^2}g(u, v)\dd{P_{XY}}(u, v)=\iint g(u,v) f_{XY}(u, v)\dd \lambda(u) \dd \lambda(v)\]
$$\lim_{x\to\infty} f(x)$$	
$$\iiiint_V \mu(t,u,v,w) \,dt\,du\,dv\,dw$$
$$\sum_{n=1}^{\infty} 2^{-n} = 1$$	
\begin{definition}
	Si $X$ et $Y$ sont 2 v.a. ou definit la \textsc{Covariance} entre $X$ et $Y$ comme
	$\cov(X,Y)\overset{\text{def}}{=}\E\left[(X-\E(X))(Y-\E(Y))\right]=\E(XY)-\E(X)\E(Y)$.
\end{definition}
\fi
\pagebreak

% \tableofcontents

% insert your code here
%\input{./algebra/main.tex}
%\input{./geometrie-differentielle/main.tex}
%\input{./probabilite/main.tex}
%\input{./analyse-fonctionnelle/main.tex}
% \input{./Analyse-convexe-et-dualite-en-optimisation/main.tex}
%\input{./tikz/main.tex}
%\input{./Theorie-du-distributions/main.tex}
%\input{./optimisation/mine.tex}
 \input{./modelisation/main.tex}

% yves.aubry@univ-tln.fr : algebra

\end{document}

%\input{./optimisation/mine.tex}
 % !TEX encoding = UTF-8 Unicode
% !TEX TS-program = xelatex

\documentclass[french]{report}

%\usepackage[utf8]{inputenc}
%\usepackage[T1]{fontenc}
\usepackage{babel}


\newif\ifcomment
%\commenttrue # Show comments

\usepackage{physics}
\usepackage{amssymb}


\usepackage{amsthm}
% \usepackage{thmtools}
\usepackage{mathtools}
\usepackage{amsfonts}

\usepackage{color}

\usepackage{tikz}

\usepackage{geometry}
\geometry{a5paper, margin=0.1in, right=1cm}

\usepackage{dsfont}

\usepackage{graphicx}
\graphicspath{ {images/} }

\usepackage{faktor}

\usepackage{IEEEtrantools}
\usepackage{enumerate}   
\usepackage[PostScript=dvips]{"/Users/aware/Documents/Courses/diagrams"}


\newtheorem{theorem}{Théorème}[section]
\renewcommand{\thetheorem}{\arabic{theorem}}
\newtheorem{lemme}{Lemme}[section]
\renewcommand{\thelemme}{\arabic{lemme}}
\newtheorem{proposition}{Proposition}[section]
\renewcommand{\theproposition}{\arabic{proposition}}
\newtheorem{notations}{Notations}[section]
\newtheorem{problem}{Problème}[section]
\newtheorem{corollary}{Corollaire}[theorem]
\renewcommand{\thecorollary}{\arabic{corollary}}
\newtheorem{property}{Propriété}[section]
\newtheorem{objective}{Objectif}[section]

\theoremstyle{definition}
\newtheorem{definition}{Définition}[section]
\renewcommand{\thedefinition}{\arabic{definition}}
\newtheorem{exercise}{Exercice}[chapter]
\renewcommand{\theexercise}{\arabic{exercise}}
\newtheorem{example}{Exemple}[chapter]
\renewcommand{\theexample}{\arabic{example}}
\newtheorem*{solution}{Solution}
\newtheorem*{application}{Application}
\newtheorem*{notation}{Notation}
\newtheorem*{vocabulary}{Vocabulaire}
\newtheorem*{properties}{Propriétés}



\theoremstyle{remark}
\newtheorem*{remark}{Remarque}
\newtheorem*{rappel}{Rappel}


\usepackage{etoolbox}
\AtBeginEnvironment{exercise}{\small}
\AtBeginEnvironment{example}{\small}

\usepackage{cases}
\usepackage[red]{mypack}

\usepackage[framemethod=TikZ]{mdframed}

\definecolor{bg}{rgb}{0.4,0.25,0.95}
\definecolor{pagebg}{rgb}{0,0,0.5}
\surroundwithmdframed[
   topline=false,
   rightline=false,
   bottomline=false,
   leftmargin=\parindent,
   skipabove=8pt,
   skipbelow=8pt,
   linecolor=blue,
   innerbottommargin=10pt,
   % backgroundcolor=bg,font=\color{orange}\sffamily, fontcolor=white
]{definition}

\usepackage{empheq}
\usepackage[most]{tcolorbox}

\newtcbox{\mymath}[1][]{%
    nobeforeafter, math upper, tcbox raise base,
    enhanced, colframe=blue!30!black,
    colback=red!10, boxrule=1pt,
    #1}

\usepackage{unixode}


\DeclareMathOperator{\ord}{ord}
\DeclareMathOperator{\orb}{orb}
\DeclareMathOperator{\stab}{stab}
\DeclareMathOperator{\Stab}{stab}
\DeclareMathOperator{\ppcm}{ppcm}
\DeclareMathOperator{\conj}{Conj}
\DeclareMathOperator{\End}{End}
\DeclareMathOperator{\rot}{rot}
\DeclareMathOperator{\trs}{trace}
\DeclareMathOperator{\Ind}{Ind}
\DeclareMathOperator{\mat}{Mat}
\DeclareMathOperator{\id}{Id}
\DeclareMathOperator{\vect}{vect}
\DeclareMathOperator{\img}{img}
\DeclareMathOperator{\cov}{Cov}
\DeclareMathOperator{\dist}{dist}
\DeclareMathOperator{\irr}{Irr}
\DeclareMathOperator{\image}{Im}
\DeclareMathOperator{\pd}{\partial}
\DeclareMathOperator{\epi}{epi}
\DeclareMathOperator{\Argmin}{Argmin}
\DeclareMathOperator{\dom}{dom}
\DeclareMathOperator{\proj}{proj}
\DeclareMathOperator{\ctg}{ctg}
\DeclareMathOperator{\supp}{supp}
\DeclareMathOperator{\argmin}{argmin}
\DeclareMathOperator{\mult}{mult}
\DeclareMathOperator{\ch}{ch}
\DeclareMathOperator{\sh}{sh}
\DeclareMathOperator{\rang}{rang}
\DeclareMathOperator{\diam}{diam}
\DeclareMathOperator{\Epigraphe}{Epigraphe}




\usepackage{xcolor}
\everymath{\color{blue}}
%\everymath{\color[rgb]{0,1,1}}
%\pagecolor[rgb]{0,0,0.5}


\newcommand*{\pdtest}[3][]{\ensuremath{\frac{\partial^{#1} #2}{\partial #3}}}

\newcommand*{\deffunc}[6][]{\ensuremath{
\begin{array}{rcl}
#2 : #3 &\rightarrow& #4\\
#5 &\mapsto& #6
\end{array}
}}

\newcommand{\eqcolon}{\mathrel{\resizebox{\widthof{$\mathord{=}$}}{\height}{ $\!\!=\!\!\resizebox{1.2\width}{0.8\height}{\raisebox{0.23ex}{$\mathop{:}$}}\!\!$ }}}
\newcommand{\coloneq}{\mathrel{\resizebox{\widthof{$\mathord{=}$}}{\height}{ $\!\!\resizebox{1.2\width}{0.8\height}{\raisebox{0.23ex}{$\mathop{:}$}}\!\!=\!\!$ }}}
\newcommand{\eqcolonl}{\ensuremath{\mathrel{=\!\!\mathop{:}}}}
\newcommand{\coloneql}{\ensuremath{\mathrel{\mathop{:} \!\! =}}}
\newcommand{\vc}[1]{% inline column vector
  \left(\begin{smallmatrix}#1\end{smallmatrix}\right)%
}
\newcommand{\vr}[1]{% inline row vector
  \begin{smallmatrix}(\,#1\,)\end{smallmatrix}%
}
\makeatletter
\newcommand*{\defeq}{\ =\mathrel{\rlap{%
                     \raisebox{0.3ex}{$\m@th\cdot$}}%
                     \raisebox{-0.3ex}{$\m@th\cdot$}}%
                     }
\makeatother

\newcommand{\mathcircle}[1]{% inline row vector
 \overset{\circ}{#1}
}
\newcommand{\ulim}{% low limit
 \underline{\lim}
}
\newcommand{\ssi}{% iff
\iff
}
\newcommand{\ps}[2]{
\expval{#1 | #2}
}
\newcommand{\df}[1]{
\mqty{#1}
}
\newcommand{\n}[1]{
\norm{#1}
}
\newcommand{\sys}[1]{
\left\{\smqty{#1}\right.
}


\newcommand{\eqdef}{\ensuremath{\overset{\text{def}}=}}


\def\Circlearrowright{\ensuremath{%
  \rotatebox[origin=c]{230}{$\circlearrowright$}}}

\newcommand\ct[1]{\text{\rmfamily\upshape #1}}
\newcommand\question[1]{ {\color{red} ...!? \small #1}}
\newcommand\caz[1]{\left\{\begin{array} #1 \end{array}\right.}
\newcommand\const{\text{\rmfamily\upshape const}}
\newcommand\toP{ \overset{\pro}{\to}}
\newcommand\toPP{ \overset{\text{PP}}{\to}}
\newcommand{\oeq}{\mathrel{\text{\textcircled{$=$}}}}





\usepackage{xcolor}
% \usepackage[normalem]{ulem}
\usepackage{lipsum}
\makeatletter
% \newcommand\colorwave[1][blue]{\bgroup \markoverwith{\lower3.5\p@\hbox{\sixly \textcolor{#1}{\char58}}}\ULon}
%\font\sixly=lasy6 % does not re-load if already loaded, so no memory problem.

\newmdtheoremenv[
linewidth= 1pt,linecolor= blue,%
leftmargin=20,rightmargin=20,innertopmargin=0pt, innerrightmargin=40,%
tikzsetting = { draw=lightgray, line width = 0.3pt,dashed,%
dash pattern = on 15pt off 3pt},%
splittopskip=\topskip,skipbelow=\baselineskip,%
skipabove=\baselineskip,ntheorem,roundcorner=0pt,
% backgroundcolor=pagebg,font=\color{orange}\sffamily, fontcolor=white
]{examplebox}{Exemple}[section]



\newcommand\R{\mathbb{R}}
\newcommand\Z{\mathbb{Z}}
\newcommand\N{\mathbb{N}}
\newcommand\E{\mathbb{E}}
\newcommand\F{\mathcal{F}}
\newcommand\cH{\mathcal{H}}
\newcommand\V{\mathbb{V}}
\newcommand\dmo{ ^{-1} }
\newcommand\kapa{\kappa}
\newcommand\im{Im}
\newcommand\hs{\mathcal{H}}





\usepackage{soul}

\makeatletter
\newcommand*{\whiten}[1]{\llap{\textcolor{white}{{\the\SOUL@token}}\hspace{#1pt}}}
\DeclareRobustCommand*\myul{%
    \def\SOUL@everyspace{\underline{\space}\kern\z@}%
    \def\SOUL@everytoken{%
     \setbox0=\hbox{\the\SOUL@token}%
     \ifdim\dp0>\z@
        \raisebox{\dp0}{\underline{\phantom{\the\SOUL@token}}}%
        \whiten{1}\whiten{0}%
        \whiten{-1}\whiten{-2}%
        \llap{\the\SOUL@token}%
     \else
        \underline{\the\SOUL@token}%
     \fi}%
\SOUL@}
\makeatother

\newcommand*{\demp}{\fontfamily{lmtt}\selectfont}

\DeclareTextFontCommand{\textdemp}{\demp}

\begin{document}

\ifcomment
Multiline
comment
\fi
\ifcomment
\myul{Typesetting test}
% \color[rgb]{1,1,1}
$∑_i^n≠ 60º±∞π∆¬≈√j∫h≤≥µ$

$\CR \R\pro\ind\pro\gS\pro
\mqty[a&b\\c&d]$
$\pro\mathbb{P}$
$\dd{x}$

  \[
    \alpha(x)=\left\{
                \begin{array}{ll}
                  x\\
                  \frac{1}{1+e^{-kx}}\\
                  \frac{e^x-e^{-x}}{e^x+e^{-x}}
                \end{array}
              \right.
  \]

  $\expval{x}$
  
  $\chi_\rho(ghg\dmo)=\Tr(\rho_{ghg\dmo})=\Tr(\rho_g\circ\rho_h\circ\rho\dmo_g)=\Tr(\rho_h)\overset{\mbox{\scalebox{0.5}{$\Tr(AB)=\Tr(BA)$}}}{=}\chi_\rho(h)$
  	$\mathop{\oplus}_{\substack{x\in X}}$

$\mat(\rho_g)=(a_{ij}(g))_{\scriptsize \substack{1\leq i\leq d \\ 1\leq j\leq d}}$ et $\mat(\rho'_g)=(a'_{ij}(g))_{\scriptsize \substack{1\leq i'\leq d' \\ 1\leq j'\leq d'}}$



\[\int_a^b{\mathbb{R}^2}g(u, v)\dd{P_{XY}}(u, v)=\iint g(u,v) f_{XY}(u, v)\dd \lambda(u) \dd \lambda(v)\]
$$\lim_{x\to\infty} f(x)$$	
$$\iiiint_V \mu(t,u,v,w) \,dt\,du\,dv\,dw$$
$$\sum_{n=1}^{\infty} 2^{-n} = 1$$	
\begin{definition}
	Si $X$ et $Y$ sont 2 v.a. ou definit la \textsc{Covariance} entre $X$ et $Y$ comme
	$\cov(X,Y)\overset{\text{def}}{=}\E\left[(X-\E(X))(Y-\E(Y))\right]=\E(XY)-\E(X)\E(Y)$.
\end{definition}
\fi
\pagebreak

% \tableofcontents

% insert your code here
%\input{./algebra/main.tex}
%\input{./geometrie-differentielle/main.tex}
%\input{./probabilite/main.tex}
%\input{./analyse-fonctionnelle/main.tex}
% \input{./Analyse-convexe-et-dualite-en-optimisation/main.tex}
%\input{./tikz/main.tex}
%\input{./Theorie-du-distributions/main.tex}
%\input{./optimisation/mine.tex}
 \input{./modelisation/main.tex}

% yves.aubry@univ-tln.fr : algebra

\end{document}


% yves.aubry@univ-tln.fr : algebra

\end{document}

%% !TEX encoding = UTF-8 Unicode
% !TEX TS-program = xelatex

\documentclass[french]{report}

%\usepackage[utf8]{inputenc}
%\usepackage[T1]{fontenc}
\usepackage{babel}


\newif\ifcomment
%\commenttrue # Show comments

\usepackage{physics}
\usepackage{amssymb}


\usepackage{amsthm}
% \usepackage{thmtools}
\usepackage{mathtools}
\usepackage{amsfonts}

\usepackage{color}

\usepackage{tikz}

\usepackage{geometry}
\geometry{a5paper, margin=0.1in, right=1cm}

\usepackage{dsfont}

\usepackage{graphicx}
\graphicspath{ {images/} }

\usepackage{faktor}

\usepackage{IEEEtrantools}
\usepackage{enumerate}   
\usepackage[PostScript=dvips]{"/Users/aware/Documents/Courses/diagrams"}


\newtheorem{theorem}{Théorème}[section]
\renewcommand{\thetheorem}{\arabic{theorem}}
\newtheorem{lemme}{Lemme}[section]
\renewcommand{\thelemme}{\arabic{lemme}}
\newtheorem{proposition}{Proposition}[section]
\renewcommand{\theproposition}{\arabic{proposition}}
\newtheorem{notations}{Notations}[section]
\newtheorem{problem}{Problème}[section]
\newtheorem{corollary}{Corollaire}[theorem]
\renewcommand{\thecorollary}{\arabic{corollary}}
\newtheorem{property}{Propriété}[section]
\newtheorem{objective}{Objectif}[section]

\theoremstyle{definition}
\newtheorem{definition}{Définition}[section]
\renewcommand{\thedefinition}{\arabic{definition}}
\newtheorem{exercise}{Exercice}[chapter]
\renewcommand{\theexercise}{\arabic{exercise}}
\newtheorem{example}{Exemple}[chapter]
\renewcommand{\theexample}{\arabic{example}}
\newtheorem*{solution}{Solution}
\newtheorem*{application}{Application}
\newtheorem*{notation}{Notation}
\newtheorem*{vocabulary}{Vocabulaire}
\newtheorem*{properties}{Propriétés}



\theoremstyle{remark}
\newtheorem*{remark}{Remarque}
\newtheorem*{rappel}{Rappel}


\usepackage{etoolbox}
\AtBeginEnvironment{exercise}{\small}
\AtBeginEnvironment{example}{\small}

\usepackage{cases}
\usepackage[red]{mypack}

\usepackage[framemethod=TikZ]{mdframed}

\definecolor{bg}{rgb}{0.4,0.25,0.95}
\definecolor{pagebg}{rgb}{0,0,0.5}
\surroundwithmdframed[
   topline=false,
   rightline=false,
   bottomline=false,
   leftmargin=\parindent,
   skipabove=8pt,
   skipbelow=8pt,
   linecolor=blue,
   innerbottommargin=10pt,
   % backgroundcolor=bg,font=\color{orange}\sffamily, fontcolor=white
]{definition}

\usepackage{empheq}
\usepackage[most]{tcolorbox}

\newtcbox{\mymath}[1][]{%
    nobeforeafter, math upper, tcbox raise base,
    enhanced, colframe=blue!30!black,
    colback=red!10, boxrule=1pt,
    #1}

\usepackage{unixode}


\DeclareMathOperator{\ord}{ord}
\DeclareMathOperator{\orb}{orb}
\DeclareMathOperator{\stab}{stab}
\DeclareMathOperator{\Stab}{stab}
\DeclareMathOperator{\ppcm}{ppcm}
\DeclareMathOperator{\conj}{Conj}
\DeclareMathOperator{\End}{End}
\DeclareMathOperator{\rot}{rot}
\DeclareMathOperator{\trs}{trace}
\DeclareMathOperator{\Ind}{Ind}
\DeclareMathOperator{\mat}{Mat}
\DeclareMathOperator{\id}{Id}
\DeclareMathOperator{\vect}{vect}
\DeclareMathOperator{\img}{img}
\DeclareMathOperator{\cov}{Cov}
\DeclareMathOperator{\dist}{dist}
\DeclareMathOperator{\irr}{Irr}
\DeclareMathOperator{\image}{Im}
\DeclareMathOperator{\pd}{\partial}
\DeclareMathOperator{\epi}{epi}
\DeclareMathOperator{\Argmin}{Argmin}
\DeclareMathOperator{\dom}{dom}
\DeclareMathOperator{\proj}{proj}
\DeclareMathOperator{\ctg}{ctg}
\DeclareMathOperator{\supp}{supp}
\DeclareMathOperator{\argmin}{argmin}
\DeclareMathOperator{\mult}{mult}
\DeclareMathOperator{\ch}{ch}
\DeclareMathOperator{\sh}{sh}
\DeclareMathOperator{\rang}{rang}
\DeclareMathOperator{\diam}{diam}
\DeclareMathOperator{\Epigraphe}{Epigraphe}




\usepackage{xcolor}
\everymath{\color{blue}}
%\everymath{\color[rgb]{0,1,1}}
%\pagecolor[rgb]{0,0,0.5}


\newcommand*{\pdtest}[3][]{\ensuremath{\frac{\partial^{#1} #2}{\partial #3}}}

\newcommand*{\deffunc}[6][]{\ensuremath{
\begin{array}{rcl}
#2 : #3 &\rightarrow& #4\\
#5 &\mapsto& #6
\end{array}
}}

\newcommand{\eqcolon}{\mathrel{\resizebox{\widthof{$\mathord{=}$}}{\height}{ $\!\!=\!\!\resizebox{1.2\width}{0.8\height}{\raisebox{0.23ex}{$\mathop{:}$}}\!\!$ }}}
\newcommand{\coloneq}{\mathrel{\resizebox{\widthof{$\mathord{=}$}}{\height}{ $\!\!\resizebox{1.2\width}{0.8\height}{\raisebox{0.23ex}{$\mathop{:}$}}\!\!=\!\!$ }}}
\newcommand{\eqcolonl}{\ensuremath{\mathrel{=\!\!\mathop{:}}}}
\newcommand{\coloneql}{\ensuremath{\mathrel{\mathop{:} \!\! =}}}
\newcommand{\vc}[1]{% inline column vector
  \left(\begin{smallmatrix}#1\end{smallmatrix}\right)%
}
\newcommand{\vr}[1]{% inline row vector
  \begin{smallmatrix}(\,#1\,)\end{smallmatrix}%
}
\makeatletter
\newcommand*{\defeq}{\ =\mathrel{\rlap{%
                     \raisebox{0.3ex}{$\m@th\cdot$}}%
                     \raisebox{-0.3ex}{$\m@th\cdot$}}%
                     }
\makeatother

\newcommand{\mathcircle}[1]{% inline row vector
 \overset{\circ}{#1}
}
\newcommand{\ulim}{% low limit
 \underline{\lim}
}
\newcommand{\ssi}{% iff
\iff
}
\newcommand{\ps}[2]{
\expval{#1 | #2}
}
\newcommand{\df}[1]{
\mqty{#1}
}
\newcommand{\n}[1]{
\norm{#1}
}
\newcommand{\sys}[1]{
\left\{\smqty{#1}\right.
}


\newcommand{\eqdef}{\ensuremath{\overset{\text{def}}=}}


\def\Circlearrowright{\ensuremath{%
  \rotatebox[origin=c]{230}{$\circlearrowright$}}}

\newcommand\ct[1]{\text{\rmfamily\upshape #1}}
\newcommand\question[1]{ {\color{red} ...!? \small #1}}
\newcommand\caz[1]{\left\{\begin{array} #1 \end{array}\right.}
\newcommand\const{\text{\rmfamily\upshape const}}
\newcommand\toP{ \overset{\pro}{\to}}
\newcommand\toPP{ \overset{\text{PP}}{\to}}
\newcommand{\oeq}{\mathrel{\text{\textcircled{$=$}}}}





\usepackage{xcolor}
% \usepackage[normalem]{ulem}
\usepackage{lipsum}
\makeatletter
% \newcommand\colorwave[1][blue]{\bgroup \markoverwith{\lower3.5\p@\hbox{\sixly \textcolor{#1}{\char58}}}\ULon}
%\font\sixly=lasy6 % does not re-load if already loaded, so no memory problem.

\newmdtheoremenv[
linewidth= 1pt,linecolor= blue,%
leftmargin=20,rightmargin=20,innertopmargin=0pt, innerrightmargin=40,%
tikzsetting = { draw=lightgray, line width = 0.3pt,dashed,%
dash pattern = on 15pt off 3pt},%
splittopskip=\topskip,skipbelow=\baselineskip,%
skipabove=\baselineskip,ntheorem,roundcorner=0pt,
% backgroundcolor=pagebg,font=\color{orange}\sffamily, fontcolor=white
]{examplebox}{Exemple}[section]



\newcommand\R{\mathbb{R}}
\newcommand\Z{\mathbb{Z}}
\newcommand\N{\mathbb{N}}
\newcommand\E{\mathbb{E}}
\newcommand\F{\mathcal{F}}
\newcommand\cH{\mathcal{H}}
\newcommand\V{\mathbb{V}}
\newcommand\dmo{ ^{-1} }
\newcommand\kapa{\kappa}
\newcommand\im{Im}
\newcommand\hs{\mathcal{H}}





\usepackage{soul}

\makeatletter
\newcommand*{\whiten}[1]{\llap{\textcolor{white}{{\the\SOUL@token}}\hspace{#1pt}}}
\DeclareRobustCommand*\myul{%
    \def\SOUL@everyspace{\underline{\space}\kern\z@}%
    \def\SOUL@everytoken{%
     \setbox0=\hbox{\the\SOUL@token}%
     \ifdim\dp0>\z@
        \raisebox{\dp0}{\underline{\phantom{\the\SOUL@token}}}%
        \whiten{1}\whiten{0}%
        \whiten{-1}\whiten{-2}%
        \llap{\the\SOUL@token}%
     \else
        \underline{\the\SOUL@token}%
     \fi}%
\SOUL@}
\makeatother

\newcommand*{\demp}{\fontfamily{lmtt}\selectfont}

\DeclareTextFontCommand{\textdemp}{\demp}

\begin{document}

\ifcomment
Multiline
comment
\fi
\ifcomment
\myul{Typesetting test}
% \color[rgb]{1,1,1}
$∑_i^n≠ 60º±∞π∆¬≈√j∫h≤≥µ$

$\CR \R\pro\ind\pro\gS\pro
\mqty[a&b\\c&d]$
$\pro\mathbb{P}$
$\dd{x}$

  \[
    \alpha(x)=\left\{
                \begin{array}{ll}
                  x\\
                  \frac{1}{1+e^{-kx}}\\
                  \frac{e^x-e^{-x}}{e^x+e^{-x}}
                \end{array}
              \right.
  \]

  $\expval{x}$
  
  $\chi_\rho(ghg\dmo)=\Tr(\rho_{ghg\dmo})=\Tr(\rho_g\circ\rho_h\circ\rho\dmo_g)=\Tr(\rho_h)\overset{\mbox{\scalebox{0.5}{$\Tr(AB)=\Tr(BA)$}}}{=}\chi_\rho(h)$
  	$\mathop{\oplus}_{\substack{x\in X}}$

$\mat(\rho_g)=(a_{ij}(g))_{\scriptsize \substack{1\leq i\leq d \\ 1\leq j\leq d}}$ et $\mat(\rho'_g)=(a'_{ij}(g))_{\scriptsize \substack{1\leq i'\leq d' \\ 1\leq j'\leq d'}}$



\[\int_a^b{\mathbb{R}^2}g(u, v)\dd{P_{XY}}(u, v)=\iint g(u,v) f_{XY}(u, v)\dd \lambda(u) \dd \lambda(v)\]
$$\lim_{x\to\infty} f(x)$$	
$$\iiiint_V \mu(t,u,v,w) \,dt\,du\,dv\,dw$$
$$\sum_{n=1}^{\infty} 2^{-n} = 1$$	
\begin{definition}
	Si $X$ et $Y$ sont 2 v.a. ou definit la \textsc{Covariance} entre $X$ et $Y$ comme
	$\cov(X,Y)\overset{\text{def}}{=}\E\left[(X-\E(X))(Y-\E(Y))\right]=\E(XY)-\E(X)\E(Y)$.
\end{definition}
\fi
\pagebreak

% \tableofcontents

% insert your code here
%% !TEX encoding = UTF-8 Unicode
% !TEX TS-program = xelatex

\documentclass[french]{report}

%\usepackage[utf8]{inputenc}
%\usepackage[T1]{fontenc}
\usepackage{babel}


\newif\ifcomment
%\commenttrue # Show comments

\usepackage{physics}
\usepackage{amssymb}


\usepackage{amsthm}
% \usepackage{thmtools}
\usepackage{mathtools}
\usepackage{amsfonts}

\usepackage{color}

\usepackage{tikz}

\usepackage{geometry}
\geometry{a5paper, margin=0.1in, right=1cm}

\usepackage{dsfont}

\usepackage{graphicx}
\graphicspath{ {images/} }

\usepackage{faktor}

\usepackage{IEEEtrantools}
\usepackage{enumerate}   
\usepackage[PostScript=dvips]{"/Users/aware/Documents/Courses/diagrams"}


\newtheorem{theorem}{Théorème}[section]
\renewcommand{\thetheorem}{\arabic{theorem}}
\newtheorem{lemme}{Lemme}[section]
\renewcommand{\thelemme}{\arabic{lemme}}
\newtheorem{proposition}{Proposition}[section]
\renewcommand{\theproposition}{\arabic{proposition}}
\newtheorem{notations}{Notations}[section]
\newtheorem{problem}{Problème}[section]
\newtheorem{corollary}{Corollaire}[theorem]
\renewcommand{\thecorollary}{\arabic{corollary}}
\newtheorem{property}{Propriété}[section]
\newtheorem{objective}{Objectif}[section]

\theoremstyle{definition}
\newtheorem{definition}{Définition}[section]
\renewcommand{\thedefinition}{\arabic{definition}}
\newtheorem{exercise}{Exercice}[chapter]
\renewcommand{\theexercise}{\arabic{exercise}}
\newtheorem{example}{Exemple}[chapter]
\renewcommand{\theexample}{\arabic{example}}
\newtheorem*{solution}{Solution}
\newtheorem*{application}{Application}
\newtheorem*{notation}{Notation}
\newtheorem*{vocabulary}{Vocabulaire}
\newtheorem*{properties}{Propriétés}



\theoremstyle{remark}
\newtheorem*{remark}{Remarque}
\newtheorem*{rappel}{Rappel}


\usepackage{etoolbox}
\AtBeginEnvironment{exercise}{\small}
\AtBeginEnvironment{example}{\small}

\usepackage{cases}
\usepackage[red]{mypack}

\usepackage[framemethod=TikZ]{mdframed}

\definecolor{bg}{rgb}{0.4,0.25,0.95}
\definecolor{pagebg}{rgb}{0,0,0.5}
\surroundwithmdframed[
   topline=false,
   rightline=false,
   bottomline=false,
   leftmargin=\parindent,
   skipabove=8pt,
   skipbelow=8pt,
   linecolor=blue,
   innerbottommargin=10pt,
   % backgroundcolor=bg,font=\color{orange}\sffamily, fontcolor=white
]{definition}

\usepackage{empheq}
\usepackage[most]{tcolorbox}

\newtcbox{\mymath}[1][]{%
    nobeforeafter, math upper, tcbox raise base,
    enhanced, colframe=blue!30!black,
    colback=red!10, boxrule=1pt,
    #1}

\usepackage{unixode}


\DeclareMathOperator{\ord}{ord}
\DeclareMathOperator{\orb}{orb}
\DeclareMathOperator{\stab}{stab}
\DeclareMathOperator{\Stab}{stab}
\DeclareMathOperator{\ppcm}{ppcm}
\DeclareMathOperator{\conj}{Conj}
\DeclareMathOperator{\End}{End}
\DeclareMathOperator{\rot}{rot}
\DeclareMathOperator{\trs}{trace}
\DeclareMathOperator{\Ind}{Ind}
\DeclareMathOperator{\mat}{Mat}
\DeclareMathOperator{\id}{Id}
\DeclareMathOperator{\vect}{vect}
\DeclareMathOperator{\img}{img}
\DeclareMathOperator{\cov}{Cov}
\DeclareMathOperator{\dist}{dist}
\DeclareMathOperator{\irr}{Irr}
\DeclareMathOperator{\image}{Im}
\DeclareMathOperator{\pd}{\partial}
\DeclareMathOperator{\epi}{epi}
\DeclareMathOperator{\Argmin}{Argmin}
\DeclareMathOperator{\dom}{dom}
\DeclareMathOperator{\proj}{proj}
\DeclareMathOperator{\ctg}{ctg}
\DeclareMathOperator{\supp}{supp}
\DeclareMathOperator{\argmin}{argmin}
\DeclareMathOperator{\mult}{mult}
\DeclareMathOperator{\ch}{ch}
\DeclareMathOperator{\sh}{sh}
\DeclareMathOperator{\rang}{rang}
\DeclareMathOperator{\diam}{diam}
\DeclareMathOperator{\Epigraphe}{Epigraphe}




\usepackage{xcolor}
\everymath{\color{blue}}
%\everymath{\color[rgb]{0,1,1}}
%\pagecolor[rgb]{0,0,0.5}


\newcommand*{\pdtest}[3][]{\ensuremath{\frac{\partial^{#1} #2}{\partial #3}}}

\newcommand*{\deffunc}[6][]{\ensuremath{
\begin{array}{rcl}
#2 : #3 &\rightarrow& #4\\
#5 &\mapsto& #6
\end{array}
}}

\newcommand{\eqcolon}{\mathrel{\resizebox{\widthof{$\mathord{=}$}}{\height}{ $\!\!=\!\!\resizebox{1.2\width}{0.8\height}{\raisebox{0.23ex}{$\mathop{:}$}}\!\!$ }}}
\newcommand{\coloneq}{\mathrel{\resizebox{\widthof{$\mathord{=}$}}{\height}{ $\!\!\resizebox{1.2\width}{0.8\height}{\raisebox{0.23ex}{$\mathop{:}$}}\!\!=\!\!$ }}}
\newcommand{\eqcolonl}{\ensuremath{\mathrel{=\!\!\mathop{:}}}}
\newcommand{\coloneql}{\ensuremath{\mathrel{\mathop{:} \!\! =}}}
\newcommand{\vc}[1]{% inline column vector
  \left(\begin{smallmatrix}#1\end{smallmatrix}\right)%
}
\newcommand{\vr}[1]{% inline row vector
  \begin{smallmatrix}(\,#1\,)\end{smallmatrix}%
}
\makeatletter
\newcommand*{\defeq}{\ =\mathrel{\rlap{%
                     \raisebox{0.3ex}{$\m@th\cdot$}}%
                     \raisebox{-0.3ex}{$\m@th\cdot$}}%
                     }
\makeatother

\newcommand{\mathcircle}[1]{% inline row vector
 \overset{\circ}{#1}
}
\newcommand{\ulim}{% low limit
 \underline{\lim}
}
\newcommand{\ssi}{% iff
\iff
}
\newcommand{\ps}[2]{
\expval{#1 | #2}
}
\newcommand{\df}[1]{
\mqty{#1}
}
\newcommand{\n}[1]{
\norm{#1}
}
\newcommand{\sys}[1]{
\left\{\smqty{#1}\right.
}


\newcommand{\eqdef}{\ensuremath{\overset{\text{def}}=}}


\def\Circlearrowright{\ensuremath{%
  \rotatebox[origin=c]{230}{$\circlearrowright$}}}

\newcommand\ct[1]{\text{\rmfamily\upshape #1}}
\newcommand\question[1]{ {\color{red} ...!? \small #1}}
\newcommand\caz[1]{\left\{\begin{array} #1 \end{array}\right.}
\newcommand\const{\text{\rmfamily\upshape const}}
\newcommand\toP{ \overset{\pro}{\to}}
\newcommand\toPP{ \overset{\text{PP}}{\to}}
\newcommand{\oeq}{\mathrel{\text{\textcircled{$=$}}}}





\usepackage{xcolor}
% \usepackage[normalem]{ulem}
\usepackage{lipsum}
\makeatletter
% \newcommand\colorwave[1][blue]{\bgroup \markoverwith{\lower3.5\p@\hbox{\sixly \textcolor{#1}{\char58}}}\ULon}
%\font\sixly=lasy6 % does not re-load if already loaded, so no memory problem.

\newmdtheoremenv[
linewidth= 1pt,linecolor= blue,%
leftmargin=20,rightmargin=20,innertopmargin=0pt, innerrightmargin=40,%
tikzsetting = { draw=lightgray, line width = 0.3pt,dashed,%
dash pattern = on 15pt off 3pt},%
splittopskip=\topskip,skipbelow=\baselineskip,%
skipabove=\baselineskip,ntheorem,roundcorner=0pt,
% backgroundcolor=pagebg,font=\color{orange}\sffamily, fontcolor=white
]{examplebox}{Exemple}[section]



\newcommand\R{\mathbb{R}}
\newcommand\Z{\mathbb{Z}}
\newcommand\N{\mathbb{N}}
\newcommand\E{\mathbb{E}}
\newcommand\F{\mathcal{F}}
\newcommand\cH{\mathcal{H}}
\newcommand\V{\mathbb{V}}
\newcommand\dmo{ ^{-1} }
\newcommand\kapa{\kappa}
\newcommand\im{Im}
\newcommand\hs{\mathcal{H}}





\usepackage{soul}

\makeatletter
\newcommand*{\whiten}[1]{\llap{\textcolor{white}{{\the\SOUL@token}}\hspace{#1pt}}}
\DeclareRobustCommand*\myul{%
    \def\SOUL@everyspace{\underline{\space}\kern\z@}%
    \def\SOUL@everytoken{%
     \setbox0=\hbox{\the\SOUL@token}%
     \ifdim\dp0>\z@
        \raisebox{\dp0}{\underline{\phantom{\the\SOUL@token}}}%
        \whiten{1}\whiten{0}%
        \whiten{-1}\whiten{-2}%
        \llap{\the\SOUL@token}%
     \else
        \underline{\the\SOUL@token}%
     \fi}%
\SOUL@}
\makeatother

\newcommand*{\demp}{\fontfamily{lmtt}\selectfont}

\DeclareTextFontCommand{\textdemp}{\demp}

\begin{document}

\ifcomment
Multiline
comment
\fi
\ifcomment
\myul{Typesetting test}
% \color[rgb]{1,1,1}
$∑_i^n≠ 60º±∞π∆¬≈√j∫h≤≥µ$

$\CR \R\pro\ind\pro\gS\pro
\mqty[a&b\\c&d]$
$\pro\mathbb{P}$
$\dd{x}$

  \[
    \alpha(x)=\left\{
                \begin{array}{ll}
                  x\\
                  \frac{1}{1+e^{-kx}}\\
                  \frac{e^x-e^{-x}}{e^x+e^{-x}}
                \end{array}
              \right.
  \]

  $\expval{x}$
  
  $\chi_\rho(ghg\dmo)=\Tr(\rho_{ghg\dmo})=\Tr(\rho_g\circ\rho_h\circ\rho\dmo_g)=\Tr(\rho_h)\overset{\mbox{\scalebox{0.5}{$\Tr(AB)=\Tr(BA)$}}}{=}\chi_\rho(h)$
  	$\mathop{\oplus}_{\substack{x\in X}}$

$\mat(\rho_g)=(a_{ij}(g))_{\scriptsize \substack{1\leq i\leq d \\ 1\leq j\leq d}}$ et $\mat(\rho'_g)=(a'_{ij}(g))_{\scriptsize \substack{1\leq i'\leq d' \\ 1\leq j'\leq d'}}$



\[\int_a^b{\mathbb{R}^2}g(u, v)\dd{P_{XY}}(u, v)=\iint g(u,v) f_{XY}(u, v)\dd \lambda(u) \dd \lambda(v)\]
$$\lim_{x\to\infty} f(x)$$	
$$\iiiint_V \mu(t,u,v,w) \,dt\,du\,dv\,dw$$
$$\sum_{n=1}^{\infty} 2^{-n} = 1$$	
\begin{definition}
	Si $X$ et $Y$ sont 2 v.a. ou definit la \textsc{Covariance} entre $X$ et $Y$ comme
	$\cov(X,Y)\overset{\text{def}}{=}\E\left[(X-\E(X))(Y-\E(Y))\right]=\E(XY)-\E(X)\E(Y)$.
\end{definition}
\fi
\pagebreak

% \tableofcontents

% insert your code here
%\input{./algebra/main.tex}
%\input{./geometrie-differentielle/main.tex}
%\input{./probabilite/main.tex}
%\input{./analyse-fonctionnelle/main.tex}
% \input{./Analyse-convexe-et-dualite-en-optimisation/main.tex}
%\input{./tikz/main.tex}
%\input{./Theorie-du-distributions/main.tex}
%\input{./optimisation/mine.tex}
 \input{./modelisation/main.tex}

% yves.aubry@univ-tln.fr : algebra

\end{document}

%% !TEX encoding = UTF-8 Unicode
% !TEX TS-program = xelatex

\documentclass[french]{report}

%\usepackage[utf8]{inputenc}
%\usepackage[T1]{fontenc}
\usepackage{babel}


\newif\ifcomment
%\commenttrue # Show comments

\usepackage{physics}
\usepackage{amssymb}


\usepackage{amsthm}
% \usepackage{thmtools}
\usepackage{mathtools}
\usepackage{amsfonts}

\usepackage{color}

\usepackage{tikz}

\usepackage{geometry}
\geometry{a5paper, margin=0.1in, right=1cm}

\usepackage{dsfont}

\usepackage{graphicx}
\graphicspath{ {images/} }

\usepackage{faktor}

\usepackage{IEEEtrantools}
\usepackage{enumerate}   
\usepackage[PostScript=dvips]{"/Users/aware/Documents/Courses/diagrams"}


\newtheorem{theorem}{Théorème}[section]
\renewcommand{\thetheorem}{\arabic{theorem}}
\newtheorem{lemme}{Lemme}[section]
\renewcommand{\thelemme}{\arabic{lemme}}
\newtheorem{proposition}{Proposition}[section]
\renewcommand{\theproposition}{\arabic{proposition}}
\newtheorem{notations}{Notations}[section]
\newtheorem{problem}{Problème}[section]
\newtheorem{corollary}{Corollaire}[theorem]
\renewcommand{\thecorollary}{\arabic{corollary}}
\newtheorem{property}{Propriété}[section]
\newtheorem{objective}{Objectif}[section]

\theoremstyle{definition}
\newtheorem{definition}{Définition}[section]
\renewcommand{\thedefinition}{\arabic{definition}}
\newtheorem{exercise}{Exercice}[chapter]
\renewcommand{\theexercise}{\arabic{exercise}}
\newtheorem{example}{Exemple}[chapter]
\renewcommand{\theexample}{\arabic{example}}
\newtheorem*{solution}{Solution}
\newtheorem*{application}{Application}
\newtheorem*{notation}{Notation}
\newtheorem*{vocabulary}{Vocabulaire}
\newtheorem*{properties}{Propriétés}



\theoremstyle{remark}
\newtheorem*{remark}{Remarque}
\newtheorem*{rappel}{Rappel}


\usepackage{etoolbox}
\AtBeginEnvironment{exercise}{\small}
\AtBeginEnvironment{example}{\small}

\usepackage{cases}
\usepackage[red]{mypack}

\usepackage[framemethod=TikZ]{mdframed}

\definecolor{bg}{rgb}{0.4,0.25,0.95}
\definecolor{pagebg}{rgb}{0,0,0.5}
\surroundwithmdframed[
   topline=false,
   rightline=false,
   bottomline=false,
   leftmargin=\parindent,
   skipabove=8pt,
   skipbelow=8pt,
   linecolor=blue,
   innerbottommargin=10pt,
   % backgroundcolor=bg,font=\color{orange}\sffamily, fontcolor=white
]{definition}

\usepackage{empheq}
\usepackage[most]{tcolorbox}

\newtcbox{\mymath}[1][]{%
    nobeforeafter, math upper, tcbox raise base,
    enhanced, colframe=blue!30!black,
    colback=red!10, boxrule=1pt,
    #1}

\usepackage{unixode}


\DeclareMathOperator{\ord}{ord}
\DeclareMathOperator{\orb}{orb}
\DeclareMathOperator{\stab}{stab}
\DeclareMathOperator{\Stab}{stab}
\DeclareMathOperator{\ppcm}{ppcm}
\DeclareMathOperator{\conj}{Conj}
\DeclareMathOperator{\End}{End}
\DeclareMathOperator{\rot}{rot}
\DeclareMathOperator{\trs}{trace}
\DeclareMathOperator{\Ind}{Ind}
\DeclareMathOperator{\mat}{Mat}
\DeclareMathOperator{\id}{Id}
\DeclareMathOperator{\vect}{vect}
\DeclareMathOperator{\img}{img}
\DeclareMathOperator{\cov}{Cov}
\DeclareMathOperator{\dist}{dist}
\DeclareMathOperator{\irr}{Irr}
\DeclareMathOperator{\image}{Im}
\DeclareMathOperator{\pd}{\partial}
\DeclareMathOperator{\epi}{epi}
\DeclareMathOperator{\Argmin}{Argmin}
\DeclareMathOperator{\dom}{dom}
\DeclareMathOperator{\proj}{proj}
\DeclareMathOperator{\ctg}{ctg}
\DeclareMathOperator{\supp}{supp}
\DeclareMathOperator{\argmin}{argmin}
\DeclareMathOperator{\mult}{mult}
\DeclareMathOperator{\ch}{ch}
\DeclareMathOperator{\sh}{sh}
\DeclareMathOperator{\rang}{rang}
\DeclareMathOperator{\diam}{diam}
\DeclareMathOperator{\Epigraphe}{Epigraphe}




\usepackage{xcolor}
\everymath{\color{blue}}
%\everymath{\color[rgb]{0,1,1}}
%\pagecolor[rgb]{0,0,0.5}


\newcommand*{\pdtest}[3][]{\ensuremath{\frac{\partial^{#1} #2}{\partial #3}}}

\newcommand*{\deffunc}[6][]{\ensuremath{
\begin{array}{rcl}
#2 : #3 &\rightarrow& #4\\
#5 &\mapsto& #6
\end{array}
}}

\newcommand{\eqcolon}{\mathrel{\resizebox{\widthof{$\mathord{=}$}}{\height}{ $\!\!=\!\!\resizebox{1.2\width}{0.8\height}{\raisebox{0.23ex}{$\mathop{:}$}}\!\!$ }}}
\newcommand{\coloneq}{\mathrel{\resizebox{\widthof{$\mathord{=}$}}{\height}{ $\!\!\resizebox{1.2\width}{0.8\height}{\raisebox{0.23ex}{$\mathop{:}$}}\!\!=\!\!$ }}}
\newcommand{\eqcolonl}{\ensuremath{\mathrel{=\!\!\mathop{:}}}}
\newcommand{\coloneql}{\ensuremath{\mathrel{\mathop{:} \!\! =}}}
\newcommand{\vc}[1]{% inline column vector
  \left(\begin{smallmatrix}#1\end{smallmatrix}\right)%
}
\newcommand{\vr}[1]{% inline row vector
  \begin{smallmatrix}(\,#1\,)\end{smallmatrix}%
}
\makeatletter
\newcommand*{\defeq}{\ =\mathrel{\rlap{%
                     \raisebox{0.3ex}{$\m@th\cdot$}}%
                     \raisebox{-0.3ex}{$\m@th\cdot$}}%
                     }
\makeatother

\newcommand{\mathcircle}[1]{% inline row vector
 \overset{\circ}{#1}
}
\newcommand{\ulim}{% low limit
 \underline{\lim}
}
\newcommand{\ssi}{% iff
\iff
}
\newcommand{\ps}[2]{
\expval{#1 | #2}
}
\newcommand{\df}[1]{
\mqty{#1}
}
\newcommand{\n}[1]{
\norm{#1}
}
\newcommand{\sys}[1]{
\left\{\smqty{#1}\right.
}


\newcommand{\eqdef}{\ensuremath{\overset{\text{def}}=}}


\def\Circlearrowright{\ensuremath{%
  \rotatebox[origin=c]{230}{$\circlearrowright$}}}

\newcommand\ct[1]{\text{\rmfamily\upshape #1}}
\newcommand\question[1]{ {\color{red} ...!? \small #1}}
\newcommand\caz[1]{\left\{\begin{array} #1 \end{array}\right.}
\newcommand\const{\text{\rmfamily\upshape const}}
\newcommand\toP{ \overset{\pro}{\to}}
\newcommand\toPP{ \overset{\text{PP}}{\to}}
\newcommand{\oeq}{\mathrel{\text{\textcircled{$=$}}}}





\usepackage{xcolor}
% \usepackage[normalem]{ulem}
\usepackage{lipsum}
\makeatletter
% \newcommand\colorwave[1][blue]{\bgroup \markoverwith{\lower3.5\p@\hbox{\sixly \textcolor{#1}{\char58}}}\ULon}
%\font\sixly=lasy6 % does not re-load if already loaded, so no memory problem.

\newmdtheoremenv[
linewidth= 1pt,linecolor= blue,%
leftmargin=20,rightmargin=20,innertopmargin=0pt, innerrightmargin=40,%
tikzsetting = { draw=lightgray, line width = 0.3pt,dashed,%
dash pattern = on 15pt off 3pt},%
splittopskip=\topskip,skipbelow=\baselineskip,%
skipabove=\baselineskip,ntheorem,roundcorner=0pt,
% backgroundcolor=pagebg,font=\color{orange}\sffamily, fontcolor=white
]{examplebox}{Exemple}[section]



\newcommand\R{\mathbb{R}}
\newcommand\Z{\mathbb{Z}}
\newcommand\N{\mathbb{N}}
\newcommand\E{\mathbb{E}}
\newcommand\F{\mathcal{F}}
\newcommand\cH{\mathcal{H}}
\newcommand\V{\mathbb{V}}
\newcommand\dmo{ ^{-1} }
\newcommand\kapa{\kappa}
\newcommand\im{Im}
\newcommand\hs{\mathcal{H}}





\usepackage{soul}

\makeatletter
\newcommand*{\whiten}[1]{\llap{\textcolor{white}{{\the\SOUL@token}}\hspace{#1pt}}}
\DeclareRobustCommand*\myul{%
    \def\SOUL@everyspace{\underline{\space}\kern\z@}%
    \def\SOUL@everytoken{%
     \setbox0=\hbox{\the\SOUL@token}%
     \ifdim\dp0>\z@
        \raisebox{\dp0}{\underline{\phantom{\the\SOUL@token}}}%
        \whiten{1}\whiten{0}%
        \whiten{-1}\whiten{-2}%
        \llap{\the\SOUL@token}%
     \else
        \underline{\the\SOUL@token}%
     \fi}%
\SOUL@}
\makeatother

\newcommand*{\demp}{\fontfamily{lmtt}\selectfont}

\DeclareTextFontCommand{\textdemp}{\demp}

\begin{document}

\ifcomment
Multiline
comment
\fi
\ifcomment
\myul{Typesetting test}
% \color[rgb]{1,1,1}
$∑_i^n≠ 60º±∞π∆¬≈√j∫h≤≥µ$

$\CR \R\pro\ind\pro\gS\pro
\mqty[a&b\\c&d]$
$\pro\mathbb{P}$
$\dd{x}$

  \[
    \alpha(x)=\left\{
                \begin{array}{ll}
                  x\\
                  \frac{1}{1+e^{-kx}}\\
                  \frac{e^x-e^{-x}}{e^x+e^{-x}}
                \end{array}
              \right.
  \]

  $\expval{x}$
  
  $\chi_\rho(ghg\dmo)=\Tr(\rho_{ghg\dmo})=\Tr(\rho_g\circ\rho_h\circ\rho\dmo_g)=\Tr(\rho_h)\overset{\mbox{\scalebox{0.5}{$\Tr(AB)=\Tr(BA)$}}}{=}\chi_\rho(h)$
  	$\mathop{\oplus}_{\substack{x\in X}}$

$\mat(\rho_g)=(a_{ij}(g))_{\scriptsize \substack{1\leq i\leq d \\ 1\leq j\leq d}}$ et $\mat(\rho'_g)=(a'_{ij}(g))_{\scriptsize \substack{1\leq i'\leq d' \\ 1\leq j'\leq d'}}$



\[\int_a^b{\mathbb{R}^2}g(u, v)\dd{P_{XY}}(u, v)=\iint g(u,v) f_{XY}(u, v)\dd \lambda(u) \dd \lambda(v)\]
$$\lim_{x\to\infty} f(x)$$	
$$\iiiint_V \mu(t,u,v,w) \,dt\,du\,dv\,dw$$
$$\sum_{n=1}^{\infty} 2^{-n} = 1$$	
\begin{definition}
	Si $X$ et $Y$ sont 2 v.a. ou definit la \textsc{Covariance} entre $X$ et $Y$ comme
	$\cov(X,Y)\overset{\text{def}}{=}\E\left[(X-\E(X))(Y-\E(Y))\right]=\E(XY)-\E(X)\E(Y)$.
\end{definition}
\fi
\pagebreak

% \tableofcontents

% insert your code here
%\input{./algebra/main.tex}
%\input{./geometrie-differentielle/main.tex}
%\input{./probabilite/main.tex}
%\input{./analyse-fonctionnelle/main.tex}
% \input{./Analyse-convexe-et-dualite-en-optimisation/main.tex}
%\input{./tikz/main.tex}
%\input{./Theorie-du-distributions/main.tex}
%\input{./optimisation/mine.tex}
 \input{./modelisation/main.tex}

% yves.aubry@univ-tln.fr : algebra

\end{document}

%% !TEX encoding = UTF-8 Unicode
% !TEX TS-program = xelatex

\documentclass[french]{report}

%\usepackage[utf8]{inputenc}
%\usepackage[T1]{fontenc}
\usepackage{babel}


\newif\ifcomment
%\commenttrue # Show comments

\usepackage{physics}
\usepackage{amssymb}


\usepackage{amsthm}
% \usepackage{thmtools}
\usepackage{mathtools}
\usepackage{amsfonts}

\usepackage{color}

\usepackage{tikz}

\usepackage{geometry}
\geometry{a5paper, margin=0.1in, right=1cm}

\usepackage{dsfont}

\usepackage{graphicx}
\graphicspath{ {images/} }

\usepackage{faktor}

\usepackage{IEEEtrantools}
\usepackage{enumerate}   
\usepackage[PostScript=dvips]{"/Users/aware/Documents/Courses/diagrams"}


\newtheorem{theorem}{Théorème}[section]
\renewcommand{\thetheorem}{\arabic{theorem}}
\newtheorem{lemme}{Lemme}[section]
\renewcommand{\thelemme}{\arabic{lemme}}
\newtheorem{proposition}{Proposition}[section]
\renewcommand{\theproposition}{\arabic{proposition}}
\newtheorem{notations}{Notations}[section]
\newtheorem{problem}{Problème}[section]
\newtheorem{corollary}{Corollaire}[theorem]
\renewcommand{\thecorollary}{\arabic{corollary}}
\newtheorem{property}{Propriété}[section]
\newtheorem{objective}{Objectif}[section]

\theoremstyle{definition}
\newtheorem{definition}{Définition}[section]
\renewcommand{\thedefinition}{\arabic{definition}}
\newtheorem{exercise}{Exercice}[chapter]
\renewcommand{\theexercise}{\arabic{exercise}}
\newtheorem{example}{Exemple}[chapter]
\renewcommand{\theexample}{\arabic{example}}
\newtheorem*{solution}{Solution}
\newtheorem*{application}{Application}
\newtheorem*{notation}{Notation}
\newtheorem*{vocabulary}{Vocabulaire}
\newtheorem*{properties}{Propriétés}



\theoremstyle{remark}
\newtheorem*{remark}{Remarque}
\newtheorem*{rappel}{Rappel}


\usepackage{etoolbox}
\AtBeginEnvironment{exercise}{\small}
\AtBeginEnvironment{example}{\small}

\usepackage{cases}
\usepackage[red]{mypack}

\usepackage[framemethod=TikZ]{mdframed}

\definecolor{bg}{rgb}{0.4,0.25,0.95}
\definecolor{pagebg}{rgb}{0,0,0.5}
\surroundwithmdframed[
   topline=false,
   rightline=false,
   bottomline=false,
   leftmargin=\parindent,
   skipabove=8pt,
   skipbelow=8pt,
   linecolor=blue,
   innerbottommargin=10pt,
   % backgroundcolor=bg,font=\color{orange}\sffamily, fontcolor=white
]{definition}

\usepackage{empheq}
\usepackage[most]{tcolorbox}

\newtcbox{\mymath}[1][]{%
    nobeforeafter, math upper, tcbox raise base,
    enhanced, colframe=blue!30!black,
    colback=red!10, boxrule=1pt,
    #1}

\usepackage{unixode}


\DeclareMathOperator{\ord}{ord}
\DeclareMathOperator{\orb}{orb}
\DeclareMathOperator{\stab}{stab}
\DeclareMathOperator{\Stab}{stab}
\DeclareMathOperator{\ppcm}{ppcm}
\DeclareMathOperator{\conj}{Conj}
\DeclareMathOperator{\End}{End}
\DeclareMathOperator{\rot}{rot}
\DeclareMathOperator{\trs}{trace}
\DeclareMathOperator{\Ind}{Ind}
\DeclareMathOperator{\mat}{Mat}
\DeclareMathOperator{\id}{Id}
\DeclareMathOperator{\vect}{vect}
\DeclareMathOperator{\img}{img}
\DeclareMathOperator{\cov}{Cov}
\DeclareMathOperator{\dist}{dist}
\DeclareMathOperator{\irr}{Irr}
\DeclareMathOperator{\image}{Im}
\DeclareMathOperator{\pd}{\partial}
\DeclareMathOperator{\epi}{epi}
\DeclareMathOperator{\Argmin}{Argmin}
\DeclareMathOperator{\dom}{dom}
\DeclareMathOperator{\proj}{proj}
\DeclareMathOperator{\ctg}{ctg}
\DeclareMathOperator{\supp}{supp}
\DeclareMathOperator{\argmin}{argmin}
\DeclareMathOperator{\mult}{mult}
\DeclareMathOperator{\ch}{ch}
\DeclareMathOperator{\sh}{sh}
\DeclareMathOperator{\rang}{rang}
\DeclareMathOperator{\diam}{diam}
\DeclareMathOperator{\Epigraphe}{Epigraphe}




\usepackage{xcolor}
\everymath{\color{blue}}
%\everymath{\color[rgb]{0,1,1}}
%\pagecolor[rgb]{0,0,0.5}


\newcommand*{\pdtest}[3][]{\ensuremath{\frac{\partial^{#1} #2}{\partial #3}}}

\newcommand*{\deffunc}[6][]{\ensuremath{
\begin{array}{rcl}
#2 : #3 &\rightarrow& #4\\
#5 &\mapsto& #6
\end{array}
}}

\newcommand{\eqcolon}{\mathrel{\resizebox{\widthof{$\mathord{=}$}}{\height}{ $\!\!=\!\!\resizebox{1.2\width}{0.8\height}{\raisebox{0.23ex}{$\mathop{:}$}}\!\!$ }}}
\newcommand{\coloneq}{\mathrel{\resizebox{\widthof{$\mathord{=}$}}{\height}{ $\!\!\resizebox{1.2\width}{0.8\height}{\raisebox{0.23ex}{$\mathop{:}$}}\!\!=\!\!$ }}}
\newcommand{\eqcolonl}{\ensuremath{\mathrel{=\!\!\mathop{:}}}}
\newcommand{\coloneql}{\ensuremath{\mathrel{\mathop{:} \!\! =}}}
\newcommand{\vc}[1]{% inline column vector
  \left(\begin{smallmatrix}#1\end{smallmatrix}\right)%
}
\newcommand{\vr}[1]{% inline row vector
  \begin{smallmatrix}(\,#1\,)\end{smallmatrix}%
}
\makeatletter
\newcommand*{\defeq}{\ =\mathrel{\rlap{%
                     \raisebox{0.3ex}{$\m@th\cdot$}}%
                     \raisebox{-0.3ex}{$\m@th\cdot$}}%
                     }
\makeatother

\newcommand{\mathcircle}[1]{% inline row vector
 \overset{\circ}{#1}
}
\newcommand{\ulim}{% low limit
 \underline{\lim}
}
\newcommand{\ssi}{% iff
\iff
}
\newcommand{\ps}[2]{
\expval{#1 | #2}
}
\newcommand{\df}[1]{
\mqty{#1}
}
\newcommand{\n}[1]{
\norm{#1}
}
\newcommand{\sys}[1]{
\left\{\smqty{#1}\right.
}


\newcommand{\eqdef}{\ensuremath{\overset{\text{def}}=}}


\def\Circlearrowright{\ensuremath{%
  \rotatebox[origin=c]{230}{$\circlearrowright$}}}

\newcommand\ct[1]{\text{\rmfamily\upshape #1}}
\newcommand\question[1]{ {\color{red} ...!? \small #1}}
\newcommand\caz[1]{\left\{\begin{array} #1 \end{array}\right.}
\newcommand\const{\text{\rmfamily\upshape const}}
\newcommand\toP{ \overset{\pro}{\to}}
\newcommand\toPP{ \overset{\text{PP}}{\to}}
\newcommand{\oeq}{\mathrel{\text{\textcircled{$=$}}}}





\usepackage{xcolor}
% \usepackage[normalem]{ulem}
\usepackage{lipsum}
\makeatletter
% \newcommand\colorwave[1][blue]{\bgroup \markoverwith{\lower3.5\p@\hbox{\sixly \textcolor{#1}{\char58}}}\ULon}
%\font\sixly=lasy6 % does not re-load if already loaded, so no memory problem.

\newmdtheoremenv[
linewidth= 1pt,linecolor= blue,%
leftmargin=20,rightmargin=20,innertopmargin=0pt, innerrightmargin=40,%
tikzsetting = { draw=lightgray, line width = 0.3pt,dashed,%
dash pattern = on 15pt off 3pt},%
splittopskip=\topskip,skipbelow=\baselineskip,%
skipabove=\baselineskip,ntheorem,roundcorner=0pt,
% backgroundcolor=pagebg,font=\color{orange}\sffamily, fontcolor=white
]{examplebox}{Exemple}[section]



\newcommand\R{\mathbb{R}}
\newcommand\Z{\mathbb{Z}}
\newcommand\N{\mathbb{N}}
\newcommand\E{\mathbb{E}}
\newcommand\F{\mathcal{F}}
\newcommand\cH{\mathcal{H}}
\newcommand\V{\mathbb{V}}
\newcommand\dmo{ ^{-1} }
\newcommand\kapa{\kappa}
\newcommand\im{Im}
\newcommand\hs{\mathcal{H}}





\usepackage{soul}

\makeatletter
\newcommand*{\whiten}[1]{\llap{\textcolor{white}{{\the\SOUL@token}}\hspace{#1pt}}}
\DeclareRobustCommand*\myul{%
    \def\SOUL@everyspace{\underline{\space}\kern\z@}%
    \def\SOUL@everytoken{%
     \setbox0=\hbox{\the\SOUL@token}%
     \ifdim\dp0>\z@
        \raisebox{\dp0}{\underline{\phantom{\the\SOUL@token}}}%
        \whiten{1}\whiten{0}%
        \whiten{-1}\whiten{-2}%
        \llap{\the\SOUL@token}%
     \else
        \underline{\the\SOUL@token}%
     \fi}%
\SOUL@}
\makeatother

\newcommand*{\demp}{\fontfamily{lmtt}\selectfont}

\DeclareTextFontCommand{\textdemp}{\demp}

\begin{document}

\ifcomment
Multiline
comment
\fi
\ifcomment
\myul{Typesetting test}
% \color[rgb]{1,1,1}
$∑_i^n≠ 60º±∞π∆¬≈√j∫h≤≥µ$

$\CR \R\pro\ind\pro\gS\pro
\mqty[a&b\\c&d]$
$\pro\mathbb{P}$
$\dd{x}$

  \[
    \alpha(x)=\left\{
                \begin{array}{ll}
                  x\\
                  \frac{1}{1+e^{-kx}}\\
                  \frac{e^x-e^{-x}}{e^x+e^{-x}}
                \end{array}
              \right.
  \]

  $\expval{x}$
  
  $\chi_\rho(ghg\dmo)=\Tr(\rho_{ghg\dmo})=\Tr(\rho_g\circ\rho_h\circ\rho\dmo_g)=\Tr(\rho_h)\overset{\mbox{\scalebox{0.5}{$\Tr(AB)=\Tr(BA)$}}}{=}\chi_\rho(h)$
  	$\mathop{\oplus}_{\substack{x\in X}}$

$\mat(\rho_g)=(a_{ij}(g))_{\scriptsize \substack{1\leq i\leq d \\ 1\leq j\leq d}}$ et $\mat(\rho'_g)=(a'_{ij}(g))_{\scriptsize \substack{1\leq i'\leq d' \\ 1\leq j'\leq d'}}$



\[\int_a^b{\mathbb{R}^2}g(u, v)\dd{P_{XY}}(u, v)=\iint g(u,v) f_{XY}(u, v)\dd \lambda(u) \dd \lambda(v)\]
$$\lim_{x\to\infty} f(x)$$	
$$\iiiint_V \mu(t,u,v,w) \,dt\,du\,dv\,dw$$
$$\sum_{n=1}^{\infty} 2^{-n} = 1$$	
\begin{definition}
	Si $X$ et $Y$ sont 2 v.a. ou definit la \textsc{Covariance} entre $X$ et $Y$ comme
	$\cov(X,Y)\overset{\text{def}}{=}\E\left[(X-\E(X))(Y-\E(Y))\right]=\E(XY)-\E(X)\E(Y)$.
\end{definition}
\fi
\pagebreak

% \tableofcontents

% insert your code here
%\input{./algebra/main.tex}
%\input{./geometrie-differentielle/main.tex}
%\input{./probabilite/main.tex}
%\input{./analyse-fonctionnelle/main.tex}
% \input{./Analyse-convexe-et-dualite-en-optimisation/main.tex}
%\input{./tikz/main.tex}
%\input{./Theorie-du-distributions/main.tex}
%\input{./optimisation/mine.tex}
 \input{./modelisation/main.tex}

% yves.aubry@univ-tln.fr : algebra

\end{document}

%% !TEX encoding = UTF-8 Unicode
% !TEX TS-program = xelatex

\documentclass[french]{report}

%\usepackage[utf8]{inputenc}
%\usepackage[T1]{fontenc}
\usepackage{babel}


\newif\ifcomment
%\commenttrue # Show comments

\usepackage{physics}
\usepackage{amssymb}


\usepackage{amsthm}
% \usepackage{thmtools}
\usepackage{mathtools}
\usepackage{amsfonts}

\usepackage{color}

\usepackage{tikz}

\usepackage{geometry}
\geometry{a5paper, margin=0.1in, right=1cm}

\usepackage{dsfont}

\usepackage{graphicx}
\graphicspath{ {images/} }

\usepackage{faktor}

\usepackage{IEEEtrantools}
\usepackage{enumerate}   
\usepackage[PostScript=dvips]{"/Users/aware/Documents/Courses/diagrams"}


\newtheorem{theorem}{Théorème}[section]
\renewcommand{\thetheorem}{\arabic{theorem}}
\newtheorem{lemme}{Lemme}[section]
\renewcommand{\thelemme}{\arabic{lemme}}
\newtheorem{proposition}{Proposition}[section]
\renewcommand{\theproposition}{\arabic{proposition}}
\newtheorem{notations}{Notations}[section]
\newtheorem{problem}{Problème}[section]
\newtheorem{corollary}{Corollaire}[theorem]
\renewcommand{\thecorollary}{\arabic{corollary}}
\newtheorem{property}{Propriété}[section]
\newtheorem{objective}{Objectif}[section]

\theoremstyle{definition}
\newtheorem{definition}{Définition}[section]
\renewcommand{\thedefinition}{\arabic{definition}}
\newtheorem{exercise}{Exercice}[chapter]
\renewcommand{\theexercise}{\arabic{exercise}}
\newtheorem{example}{Exemple}[chapter]
\renewcommand{\theexample}{\arabic{example}}
\newtheorem*{solution}{Solution}
\newtheorem*{application}{Application}
\newtheorem*{notation}{Notation}
\newtheorem*{vocabulary}{Vocabulaire}
\newtheorem*{properties}{Propriétés}



\theoremstyle{remark}
\newtheorem*{remark}{Remarque}
\newtheorem*{rappel}{Rappel}


\usepackage{etoolbox}
\AtBeginEnvironment{exercise}{\small}
\AtBeginEnvironment{example}{\small}

\usepackage{cases}
\usepackage[red]{mypack}

\usepackage[framemethod=TikZ]{mdframed}

\definecolor{bg}{rgb}{0.4,0.25,0.95}
\definecolor{pagebg}{rgb}{0,0,0.5}
\surroundwithmdframed[
   topline=false,
   rightline=false,
   bottomline=false,
   leftmargin=\parindent,
   skipabove=8pt,
   skipbelow=8pt,
   linecolor=blue,
   innerbottommargin=10pt,
   % backgroundcolor=bg,font=\color{orange}\sffamily, fontcolor=white
]{definition}

\usepackage{empheq}
\usepackage[most]{tcolorbox}

\newtcbox{\mymath}[1][]{%
    nobeforeafter, math upper, tcbox raise base,
    enhanced, colframe=blue!30!black,
    colback=red!10, boxrule=1pt,
    #1}

\usepackage{unixode}


\DeclareMathOperator{\ord}{ord}
\DeclareMathOperator{\orb}{orb}
\DeclareMathOperator{\stab}{stab}
\DeclareMathOperator{\Stab}{stab}
\DeclareMathOperator{\ppcm}{ppcm}
\DeclareMathOperator{\conj}{Conj}
\DeclareMathOperator{\End}{End}
\DeclareMathOperator{\rot}{rot}
\DeclareMathOperator{\trs}{trace}
\DeclareMathOperator{\Ind}{Ind}
\DeclareMathOperator{\mat}{Mat}
\DeclareMathOperator{\id}{Id}
\DeclareMathOperator{\vect}{vect}
\DeclareMathOperator{\img}{img}
\DeclareMathOperator{\cov}{Cov}
\DeclareMathOperator{\dist}{dist}
\DeclareMathOperator{\irr}{Irr}
\DeclareMathOperator{\image}{Im}
\DeclareMathOperator{\pd}{\partial}
\DeclareMathOperator{\epi}{epi}
\DeclareMathOperator{\Argmin}{Argmin}
\DeclareMathOperator{\dom}{dom}
\DeclareMathOperator{\proj}{proj}
\DeclareMathOperator{\ctg}{ctg}
\DeclareMathOperator{\supp}{supp}
\DeclareMathOperator{\argmin}{argmin}
\DeclareMathOperator{\mult}{mult}
\DeclareMathOperator{\ch}{ch}
\DeclareMathOperator{\sh}{sh}
\DeclareMathOperator{\rang}{rang}
\DeclareMathOperator{\diam}{diam}
\DeclareMathOperator{\Epigraphe}{Epigraphe}




\usepackage{xcolor}
\everymath{\color{blue}}
%\everymath{\color[rgb]{0,1,1}}
%\pagecolor[rgb]{0,0,0.5}


\newcommand*{\pdtest}[3][]{\ensuremath{\frac{\partial^{#1} #2}{\partial #3}}}

\newcommand*{\deffunc}[6][]{\ensuremath{
\begin{array}{rcl}
#2 : #3 &\rightarrow& #4\\
#5 &\mapsto& #6
\end{array}
}}

\newcommand{\eqcolon}{\mathrel{\resizebox{\widthof{$\mathord{=}$}}{\height}{ $\!\!=\!\!\resizebox{1.2\width}{0.8\height}{\raisebox{0.23ex}{$\mathop{:}$}}\!\!$ }}}
\newcommand{\coloneq}{\mathrel{\resizebox{\widthof{$\mathord{=}$}}{\height}{ $\!\!\resizebox{1.2\width}{0.8\height}{\raisebox{0.23ex}{$\mathop{:}$}}\!\!=\!\!$ }}}
\newcommand{\eqcolonl}{\ensuremath{\mathrel{=\!\!\mathop{:}}}}
\newcommand{\coloneql}{\ensuremath{\mathrel{\mathop{:} \!\! =}}}
\newcommand{\vc}[1]{% inline column vector
  \left(\begin{smallmatrix}#1\end{smallmatrix}\right)%
}
\newcommand{\vr}[1]{% inline row vector
  \begin{smallmatrix}(\,#1\,)\end{smallmatrix}%
}
\makeatletter
\newcommand*{\defeq}{\ =\mathrel{\rlap{%
                     \raisebox{0.3ex}{$\m@th\cdot$}}%
                     \raisebox{-0.3ex}{$\m@th\cdot$}}%
                     }
\makeatother

\newcommand{\mathcircle}[1]{% inline row vector
 \overset{\circ}{#1}
}
\newcommand{\ulim}{% low limit
 \underline{\lim}
}
\newcommand{\ssi}{% iff
\iff
}
\newcommand{\ps}[2]{
\expval{#1 | #2}
}
\newcommand{\df}[1]{
\mqty{#1}
}
\newcommand{\n}[1]{
\norm{#1}
}
\newcommand{\sys}[1]{
\left\{\smqty{#1}\right.
}


\newcommand{\eqdef}{\ensuremath{\overset{\text{def}}=}}


\def\Circlearrowright{\ensuremath{%
  \rotatebox[origin=c]{230}{$\circlearrowright$}}}

\newcommand\ct[1]{\text{\rmfamily\upshape #1}}
\newcommand\question[1]{ {\color{red} ...!? \small #1}}
\newcommand\caz[1]{\left\{\begin{array} #1 \end{array}\right.}
\newcommand\const{\text{\rmfamily\upshape const}}
\newcommand\toP{ \overset{\pro}{\to}}
\newcommand\toPP{ \overset{\text{PP}}{\to}}
\newcommand{\oeq}{\mathrel{\text{\textcircled{$=$}}}}





\usepackage{xcolor}
% \usepackage[normalem]{ulem}
\usepackage{lipsum}
\makeatletter
% \newcommand\colorwave[1][blue]{\bgroup \markoverwith{\lower3.5\p@\hbox{\sixly \textcolor{#1}{\char58}}}\ULon}
%\font\sixly=lasy6 % does not re-load if already loaded, so no memory problem.

\newmdtheoremenv[
linewidth= 1pt,linecolor= blue,%
leftmargin=20,rightmargin=20,innertopmargin=0pt, innerrightmargin=40,%
tikzsetting = { draw=lightgray, line width = 0.3pt,dashed,%
dash pattern = on 15pt off 3pt},%
splittopskip=\topskip,skipbelow=\baselineskip,%
skipabove=\baselineskip,ntheorem,roundcorner=0pt,
% backgroundcolor=pagebg,font=\color{orange}\sffamily, fontcolor=white
]{examplebox}{Exemple}[section]



\newcommand\R{\mathbb{R}}
\newcommand\Z{\mathbb{Z}}
\newcommand\N{\mathbb{N}}
\newcommand\E{\mathbb{E}}
\newcommand\F{\mathcal{F}}
\newcommand\cH{\mathcal{H}}
\newcommand\V{\mathbb{V}}
\newcommand\dmo{ ^{-1} }
\newcommand\kapa{\kappa}
\newcommand\im{Im}
\newcommand\hs{\mathcal{H}}





\usepackage{soul}

\makeatletter
\newcommand*{\whiten}[1]{\llap{\textcolor{white}{{\the\SOUL@token}}\hspace{#1pt}}}
\DeclareRobustCommand*\myul{%
    \def\SOUL@everyspace{\underline{\space}\kern\z@}%
    \def\SOUL@everytoken{%
     \setbox0=\hbox{\the\SOUL@token}%
     \ifdim\dp0>\z@
        \raisebox{\dp0}{\underline{\phantom{\the\SOUL@token}}}%
        \whiten{1}\whiten{0}%
        \whiten{-1}\whiten{-2}%
        \llap{\the\SOUL@token}%
     \else
        \underline{\the\SOUL@token}%
     \fi}%
\SOUL@}
\makeatother

\newcommand*{\demp}{\fontfamily{lmtt}\selectfont}

\DeclareTextFontCommand{\textdemp}{\demp}

\begin{document}

\ifcomment
Multiline
comment
\fi
\ifcomment
\myul{Typesetting test}
% \color[rgb]{1,1,1}
$∑_i^n≠ 60º±∞π∆¬≈√j∫h≤≥µ$

$\CR \R\pro\ind\pro\gS\pro
\mqty[a&b\\c&d]$
$\pro\mathbb{P}$
$\dd{x}$

  \[
    \alpha(x)=\left\{
                \begin{array}{ll}
                  x\\
                  \frac{1}{1+e^{-kx}}\\
                  \frac{e^x-e^{-x}}{e^x+e^{-x}}
                \end{array}
              \right.
  \]

  $\expval{x}$
  
  $\chi_\rho(ghg\dmo)=\Tr(\rho_{ghg\dmo})=\Tr(\rho_g\circ\rho_h\circ\rho\dmo_g)=\Tr(\rho_h)\overset{\mbox{\scalebox{0.5}{$\Tr(AB)=\Tr(BA)$}}}{=}\chi_\rho(h)$
  	$\mathop{\oplus}_{\substack{x\in X}}$

$\mat(\rho_g)=(a_{ij}(g))_{\scriptsize \substack{1\leq i\leq d \\ 1\leq j\leq d}}$ et $\mat(\rho'_g)=(a'_{ij}(g))_{\scriptsize \substack{1\leq i'\leq d' \\ 1\leq j'\leq d'}}$



\[\int_a^b{\mathbb{R}^2}g(u, v)\dd{P_{XY}}(u, v)=\iint g(u,v) f_{XY}(u, v)\dd \lambda(u) \dd \lambda(v)\]
$$\lim_{x\to\infty} f(x)$$	
$$\iiiint_V \mu(t,u,v,w) \,dt\,du\,dv\,dw$$
$$\sum_{n=1}^{\infty} 2^{-n} = 1$$	
\begin{definition}
	Si $X$ et $Y$ sont 2 v.a. ou definit la \textsc{Covariance} entre $X$ et $Y$ comme
	$\cov(X,Y)\overset{\text{def}}{=}\E\left[(X-\E(X))(Y-\E(Y))\right]=\E(XY)-\E(X)\E(Y)$.
\end{definition}
\fi
\pagebreak

% \tableofcontents

% insert your code here
%\input{./algebra/main.tex}
%\input{./geometrie-differentielle/main.tex}
%\input{./probabilite/main.tex}
%\input{./analyse-fonctionnelle/main.tex}
% \input{./Analyse-convexe-et-dualite-en-optimisation/main.tex}
%\input{./tikz/main.tex}
%\input{./Theorie-du-distributions/main.tex}
%\input{./optimisation/mine.tex}
 \input{./modelisation/main.tex}

% yves.aubry@univ-tln.fr : algebra

\end{document}

% % !TEX encoding = UTF-8 Unicode
% !TEX TS-program = xelatex

\documentclass[french]{report}

%\usepackage[utf8]{inputenc}
%\usepackage[T1]{fontenc}
\usepackage{babel}


\newif\ifcomment
%\commenttrue # Show comments

\usepackage{physics}
\usepackage{amssymb}


\usepackage{amsthm}
% \usepackage{thmtools}
\usepackage{mathtools}
\usepackage{amsfonts}

\usepackage{color}

\usepackage{tikz}

\usepackage{geometry}
\geometry{a5paper, margin=0.1in, right=1cm}

\usepackage{dsfont}

\usepackage{graphicx}
\graphicspath{ {images/} }

\usepackage{faktor}

\usepackage{IEEEtrantools}
\usepackage{enumerate}   
\usepackage[PostScript=dvips]{"/Users/aware/Documents/Courses/diagrams"}


\newtheorem{theorem}{Théorème}[section]
\renewcommand{\thetheorem}{\arabic{theorem}}
\newtheorem{lemme}{Lemme}[section]
\renewcommand{\thelemme}{\arabic{lemme}}
\newtheorem{proposition}{Proposition}[section]
\renewcommand{\theproposition}{\arabic{proposition}}
\newtheorem{notations}{Notations}[section]
\newtheorem{problem}{Problème}[section]
\newtheorem{corollary}{Corollaire}[theorem]
\renewcommand{\thecorollary}{\arabic{corollary}}
\newtheorem{property}{Propriété}[section]
\newtheorem{objective}{Objectif}[section]

\theoremstyle{definition}
\newtheorem{definition}{Définition}[section]
\renewcommand{\thedefinition}{\arabic{definition}}
\newtheorem{exercise}{Exercice}[chapter]
\renewcommand{\theexercise}{\arabic{exercise}}
\newtheorem{example}{Exemple}[chapter]
\renewcommand{\theexample}{\arabic{example}}
\newtheorem*{solution}{Solution}
\newtheorem*{application}{Application}
\newtheorem*{notation}{Notation}
\newtheorem*{vocabulary}{Vocabulaire}
\newtheorem*{properties}{Propriétés}



\theoremstyle{remark}
\newtheorem*{remark}{Remarque}
\newtheorem*{rappel}{Rappel}


\usepackage{etoolbox}
\AtBeginEnvironment{exercise}{\small}
\AtBeginEnvironment{example}{\small}

\usepackage{cases}
\usepackage[red]{mypack}

\usepackage[framemethod=TikZ]{mdframed}

\definecolor{bg}{rgb}{0.4,0.25,0.95}
\definecolor{pagebg}{rgb}{0,0,0.5}
\surroundwithmdframed[
   topline=false,
   rightline=false,
   bottomline=false,
   leftmargin=\parindent,
   skipabove=8pt,
   skipbelow=8pt,
   linecolor=blue,
   innerbottommargin=10pt,
   % backgroundcolor=bg,font=\color{orange}\sffamily, fontcolor=white
]{definition}

\usepackage{empheq}
\usepackage[most]{tcolorbox}

\newtcbox{\mymath}[1][]{%
    nobeforeafter, math upper, tcbox raise base,
    enhanced, colframe=blue!30!black,
    colback=red!10, boxrule=1pt,
    #1}

\usepackage{unixode}


\DeclareMathOperator{\ord}{ord}
\DeclareMathOperator{\orb}{orb}
\DeclareMathOperator{\stab}{stab}
\DeclareMathOperator{\Stab}{stab}
\DeclareMathOperator{\ppcm}{ppcm}
\DeclareMathOperator{\conj}{Conj}
\DeclareMathOperator{\End}{End}
\DeclareMathOperator{\rot}{rot}
\DeclareMathOperator{\trs}{trace}
\DeclareMathOperator{\Ind}{Ind}
\DeclareMathOperator{\mat}{Mat}
\DeclareMathOperator{\id}{Id}
\DeclareMathOperator{\vect}{vect}
\DeclareMathOperator{\img}{img}
\DeclareMathOperator{\cov}{Cov}
\DeclareMathOperator{\dist}{dist}
\DeclareMathOperator{\irr}{Irr}
\DeclareMathOperator{\image}{Im}
\DeclareMathOperator{\pd}{\partial}
\DeclareMathOperator{\epi}{epi}
\DeclareMathOperator{\Argmin}{Argmin}
\DeclareMathOperator{\dom}{dom}
\DeclareMathOperator{\proj}{proj}
\DeclareMathOperator{\ctg}{ctg}
\DeclareMathOperator{\supp}{supp}
\DeclareMathOperator{\argmin}{argmin}
\DeclareMathOperator{\mult}{mult}
\DeclareMathOperator{\ch}{ch}
\DeclareMathOperator{\sh}{sh}
\DeclareMathOperator{\rang}{rang}
\DeclareMathOperator{\diam}{diam}
\DeclareMathOperator{\Epigraphe}{Epigraphe}




\usepackage{xcolor}
\everymath{\color{blue}}
%\everymath{\color[rgb]{0,1,1}}
%\pagecolor[rgb]{0,0,0.5}


\newcommand*{\pdtest}[3][]{\ensuremath{\frac{\partial^{#1} #2}{\partial #3}}}

\newcommand*{\deffunc}[6][]{\ensuremath{
\begin{array}{rcl}
#2 : #3 &\rightarrow& #4\\
#5 &\mapsto& #6
\end{array}
}}

\newcommand{\eqcolon}{\mathrel{\resizebox{\widthof{$\mathord{=}$}}{\height}{ $\!\!=\!\!\resizebox{1.2\width}{0.8\height}{\raisebox{0.23ex}{$\mathop{:}$}}\!\!$ }}}
\newcommand{\coloneq}{\mathrel{\resizebox{\widthof{$\mathord{=}$}}{\height}{ $\!\!\resizebox{1.2\width}{0.8\height}{\raisebox{0.23ex}{$\mathop{:}$}}\!\!=\!\!$ }}}
\newcommand{\eqcolonl}{\ensuremath{\mathrel{=\!\!\mathop{:}}}}
\newcommand{\coloneql}{\ensuremath{\mathrel{\mathop{:} \!\! =}}}
\newcommand{\vc}[1]{% inline column vector
  \left(\begin{smallmatrix}#1\end{smallmatrix}\right)%
}
\newcommand{\vr}[1]{% inline row vector
  \begin{smallmatrix}(\,#1\,)\end{smallmatrix}%
}
\makeatletter
\newcommand*{\defeq}{\ =\mathrel{\rlap{%
                     \raisebox{0.3ex}{$\m@th\cdot$}}%
                     \raisebox{-0.3ex}{$\m@th\cdot$}}%
                     }
\makeatother

\newcommand{\mathcircle}[1]{% inline row vector
 \overset{\circ}{#1}
}
\newcommand{\ulim}{% low limit
 \underline{\lim}
}
\newcommand{\ssi}{% iff
\iff
}
\newcommand{\ps}[2]{
\expval{#1 | #2}
}
\newcommand{\df}[1]{
\mqty{#1}
}
\newcommand{\n}[1]{
\norm{#1}
}
\newcommand{\sys}[1]{
\left\{\smqty{#1}\right.
}


\newcommand{\eqdef}{\ensuremath{\overset{\text{def}}=}}


\def\Circlearrowright{\ensuremath{%
  \rotatebox[origin=c]{230}{$\circlearrowright$}}}

\newcommand\ct[1]{\text{\rmfamily\upshape #1}}
\newcommand\question[1]{ {\color{red} ...!? \small #1}}
\newcommand\caz[1]{\left\{\begin{array} #1 \end{array}\right.}
\newcommand\const{\text{\rmfamily\upshape const}}
\newcommand\toP{ \overset{\pro}{\to}}
\newcommand\toPP{ \overset{\text{PP}}{\to}}
\newcommand{\oeq}{\mathrel{\text{\textcircled{$=$}}}}





\usepackage{xcolor}
% \usepackage[normalem]{ulem}
\usepackage{lipsum}
\makeatletter
% \newcommand\colorwave[1][blue]{\bgroup \markoverwith{\lower3.5\p@\hbox{\sixly \textcolor{#1}{\char58}}}\ULon}
%\font\sixly=lasy6 % does not re-load if already loaded, so no memory problem.

\newmdtheoremenv[
linewidth= 1pt,linecolor= blue,%
leftmargin=20,rightmargin=20,innertopmargin=0pt, innerrightmargin=40,%
tikzsetting = { draw=lightgray, line width = 0.3pt,dashed,%
dash pattern = on 15pt off 3pt},%
splittopskip=\topskip,skipbelow=\baselineskip,%
skipabove=\baselineskip,ntheorem,roundcorner=0pt,
% backgroundcolor=pagebg,font=\color{orange}\sffamily, fontcolor=white
]{examplebox}{Exemple}[section]



\newcommand\R{\mathbb{R}}
\newcommand\Z{\mathbb{Z}}
\newcommand\N{\mathbb{N}}
\newcommand\E{\mathbb{E}}
\newcommand\F{\mathcal{F}}
\newcommand\cH{\mathcal{H}}
\newcommand\V{\mathbb{V}}
\newcommand\dmo{ ^{-1} }
\newcommand\kapa{\kappa}
\newcommand\im{Im}
\newcommand\hs{\mathcal{H}}





\usepackage{soul}

\makeatletter
\newcommand*{\whiten}[1]{\llap{\textcolor{white}{{\the\SOUL@token}}\hspace{#1pt}}}
\DeclareRobustCommand*\myul{%
    \def\SOUL@everyspace{\underline{\space}\kern\z@}%
    \def\SOUL@everytoken{%
     \setbox0=\hbox{\the\SOUL@token}%
     \ifdim\dp0>\z@
        \raisebox{\dp0}{\underline{\phantom{\the\SOUL@token}}}%
        \whiten{1}\whiten{0}%
        \whiten{-1}\whiten{-2}%
        \llap{\the\SOUL@token}%
     \else
        \underline{\the\SOUL@token}%
     \fi}%
\SOUL@}
\makeatother

\newcommand*{\demp}{\fontfamily{lmtt}\selectfont}

\DeclareTextFontCommand{\textdemp}{\demp}

\begin{document}

\ifcomment
Multiline
comment
\fi
\ifcomment
\myul{Typesetting test}
% \color[rgb]{1,1,1}
$∑_i^n≠ 60º±∞π∆¬≈√j∫h≤≥µ$

$\CR \R\pro\ind\pro\gS\pro
\mqty[a&b\\c&d]$
$\pro\mathbb{P}$
$\dd{x}$

  \[
    \alpha(x)=\left\{
                \begin{array}{ll}
                  x\\
                  \frac{1}{1+e^{-kx}}\\
                  \frac{e^x-e^{-x}}{e^x+e^{-x}}
                \end{array}
              \right.
  \]

  $\expval{x}$
  
  $\chi_\rho(ghg\dmo)=\Tr(\rho_{ghg\dmo})=\Tr(\rho_g\circ\rho_h\circ\rho\dmo_g)=\Tr(\rho_h)\overset{\mbox{\scalebox{0.5}{$\Tr(AB)=\Tr(BA)$}}}{=}\chi_\rho(h)$
  	$\mathop{\oplus}_{\substack{x\in X}}$

$\mat(\rho_g)=(a_{ij}(g))_{\scriptsize \substack{1\leq i\leq d \\ 1\leq j\leq d}}$ et $\mat(\rho'_g)=(a'_{ij}(g))_{\scriptsize \substack{1\leq i'\leq d' \\ 1\leq j'\leq d'}}$



\[\int_a^b{\mathbb{R}^2}g(u, v)\dd{P_{XY}}(u, v)=\iint g(u,v) f_{XY}(u, v)\dd \lambda(u) \dd \lambda(v)\]
$$\lim_{x\to\infty} f(x)$$	
$$\iiiint_V \mu(t,u,v,w) \,dt\,du\,dv\,dw$$
$$\sum_{n=1}^{\infty} 2^{-n} = 1$$	
\begin{definition}
	Si $X$ et $Y$ sont 2 v.a. ou definit la \textsc{Covariance} entre $X$ et $Y$ comme
	$\cov(X,Y)\overset{\text{def}}{=}\E\left[(X-\E(X))(Y-\E(Y))\right]=\E(XY)-\E(X)\E(Y)$.
\end{definition}
\fi
\pagebreak

% \tableofcontents

% insert your code here
%\input{./algebra/main.tex}
%\input{./geometrie-differentielle/main.tex}
%\input{./probabilite/main.tex}
%\input{./analyse-fonctionnelle/main.tex}
% \input{./Analyse-convexe-et-dualite-en-optimisation/main.tex}
%\input{./tikz/main.tex}
%\input{./Theorie-du-distributions/main.tex}
%\input{./optimisation/mine.tex}
 \input{./modelisation/main.tex}

% yves.aubry@univ-tln.fr : algebra

\end{document}

%% !TEX encoding = UTF-8 Unicode
% !TEX TS-program = xelatex

\documentclass[french]{report}

%\usepackage[utf8]{inputenc}
%\usepackage[T1]{fontenc}
\usepackage{babel}


\newif\ifcomment
%\commenttrue # Show comments

\usepackage{physics}
\usepackage{amssymb}


\usepackage{amsthm}
% \usepackage{thmtools}
\usepackage{mathtools}
\usepackage{amsfonts}

\usepackage{color}

\usepackage{tikz}

\usepackage{geometry}
\geometry{a5paper, margin=0.1in, right=1cm}

\usepackage{dsfont}

\usepackage{graphicx}
\graphicspath{ {images/} }

\usepackage{faktor}

\usepackage{IEEEtrantools}
\usepackage{enumerate}   
\usepackage[PostScript=dvips]{"/Users/aware/Documents/Courses/diagrams"}


\newtheorem{theorem}{Théorème}[section]
\renewcommand{\thetheorem}{\arabic{theorem}}
\newtheorem{lemme}{Lemme}[section]
\renewcommand{\thelemme}{\arabic{lemme}}
\newtheorem{proposition}{Proposition}[section]
\renewcommand{\theproposition}{\arabic{proposition}}
\newtheorem{notations}{Notations}[section]
\newtheorem{problem}{Problème}[section]
\newtheorem{corollary}{Corollaire}[theorem]
\renewcommand{\thecorollary}{\arabic{corollary}}
\newtheorem{property}{Propriété}[section]
\newtheorem{objective}{Objectif}[section]

\theoremstyle{definition}
\newtheorem{definition}{Définition}[section]
\renewcommand{\thedefinition}{\arabic{definition}}
\newtheorem{exercise}{Exercice}[chapter]
\renewcommand{\theexercise}{\arabic{exercise}}
\newtheorem{example}{Exemple}[chapter]
\renewcommand{\theexample}{\arabic{example}}
\newtheorem*{solution}{Solution}
\newtheorem*{application}{Application}
\newtheorem*{notation}{Notation}
\newtheorem*{vocabulary}{Vocabulaire}
\newtheorem*{properties}{Propriétés}



\theoremstyle{remark}
\newtheorem*{remark}{Remarque}
\newtheorem*{rappel}{Rappel}


\usepackage{etoolbox}
\AtBeginEnvironment{exercise}{\small}
\AtBeginEnvironment{example}{\small}

\usepackage{cases}
\usepackage[red]{mypack}

\usepackage[framemethod=TikZ]{mdframed}

\definecolor{bg}{rgb}{0.4,0.25,0.95}
\definecolor{pagebg}{rgb}{0,0,0.5}
\surroundwithmdframed[
   topline=false,
   rightline=false,
   bottomline=false,
   leftmargin=\parindent,
   skipabove=8pt,
   skipbelow=8pt,
   linecolor=blue,
   innerbottommargin=10pt,
   % backgroundcolor=bg,font=\color{orange}\sffamily, fontcolor=white
]{definition}

\usepackage{empheq}
\usepackage[most]{tcolorbox}

\newtcbox{\mymath}[1][]{%
    nobeforeafter, math upper, tcbox raise base,
    enhanced, colframe=blue!30!black,
    colback=red!10, boxrule=1pt,
    #1}

\usepackage{unixode}


\DeclareMathOperator{\ord}{ord}
\DeclareMathOperator{\orb}{orb}
\DeclareMathOperator{\stab}{stab}
\DeclareMathOperator{\Stab}{stab}
\DeclareMathOperator{\ppcm}{ppcm}
\DeclareMathOperator{\conj}{Conj}
\DeclareMathOperator{\End}{End}
\DeclareMathOperator{\rot}{rot}
\DeclareMathOperator{\trs}{trace}
\DeclareMathOperator{\Ind}{Ind}
\DeclareMathOperator{\mat}{Mat}
\DeclareMathOperator{\id}{Id}
\DeclareMathOperator{\vect}{vect}
\DeclareMathOperator{\img}{img}
\DeclareMathOperator{\cov}{Cov}
\DeclareMathOperator{\dist}{dist}
\DeclareMathOperator{\irr}{Irr}
\DeclareMathOperator{\image}{Im}
\DeclareMathOperator{\pd}{\partial}
\DeclareMathOperator{\epi}{epi}
\DeclareMathOperator{\Argmin}{Argmin}
\DeclareMathOperator{\dom}{dom}
\DeclareMathOperator{\proj}{proj}
\DeclareMathOperator{\ctg}{ctg}
\DeclareMathOperator{\supp}{supp}
\DeclareMathOperator{\argmin}{argmin}
\DeclareMathOperator{\mult}{mult}
\DeclareMathOperator{\ch}{ch}
\DeclareMathOperator{\sh}{sh}
\DeclareMathOperator{\rang}{rang}
\DeclareMathOperator{\diam}{diam}
\DeclareMathOperator{\Epigraphe}{Epigraphe}




\usepackage{xcolor}
\everymath{\color{blue}}
%\everymath{\color[rgb]{0,1,1}}
%\pagecolor[rgb]{0,0,0.5}


\newcommand*{\pdtest}[3][]{\ensuremath{\frac{\partial^{#1} #2}{\partial #3}}}

\newcommand*{\deffunc}[6][]{\ensuremath{
\begin{array}{rcl}
#2 : #3 &\rightarrow& #4\\
#5 &\mapsto& #6
\end{array}
}}

\newcommand{\eqcolon}{\mathrel{\resizebox{\widthof{$\mathord{=}$}}{\height}{ $\!\!=\!\!\resizebox{1.2\width}{0.8\height}{\raisebox{0.23ex}{$\mathop{:}$}}\!\!$ }}}
\newcommand{\coloneq}{\mathrel{\resizebox{\widthof{$\mathord{=}$}}{\height}{ $\!\!\resizebox{1.2\width}{0.8\height}{\raisebox{0.23ex}{$\mathop{:}$}}\!\!=\!\!$ }}}
\newcommand{\eqcolonl}{\ensuremath{\mathrel{=\!\!\mathop{:}}}}
\newcommand{\coloneql}{\ensuremath{\mathrel{\mathop{:} \!\! =}}}
\newcommand{\vc}[1]{% inline column vector
  \left(\begin{smallmatrix}#1\end{smallmatrix}\right)%
}
\newcommand{\vr}[1]{% inline row vector
  \begin{smallmatrix}(\,#1\,)\end{smallmatrix}%
}
\makeatletter
\newcommand*{\defeq}{\ =\mathrel{\rlap{%
                     \raisebox{0.3ex}{$\m@th\cdot$}}%
                     \raisebox{-0.3ex}{$\m@th\cdot$}}%
                     }
\makeatother

\newcommand{\mathcircle}[1]{% inline row vector
 \overset{\circ}{#1}
}
\newcommand{\ulim}{% low limit
 \underline{\lim}
}
\newcommand{\ssi}{% iff
\iff
}
\newcommand{\ps}[2]{
\expval{#1 | #2}
}
\newcommand{\df}[1]{
\mqty{#1}
}
\newcommand{\n}[1]{
\norm{#1}
}
\newcommand{\sys}[1]{
\left\{\smqty{#1}\right.
}


\newcommand{\eqdef}{\ensuremath{\overset{\text{def}}=}}


\def\Circlearrowright{\ensuremath{%
  \rotatebox[origin=c]{230}{$\circlearrowright$}}}

\newcommand\ct[1]{\text{\rmfamily\upshape #1}}
\newcommand\question[1]{ {\color{red} ...!? \small #1}}
\newcommand\caz[1]{\left\{\begin{array} #1 \end{array}\right.}
\newcommand\const{\text{\rmfamily\upshape const}}
\newcommand\toP{ \overset{\pro}{\to}}
\newcommand\toPP{ \overset{\text{PP}}{\to}}
\newcommand{\oeq}{\mathrel{\text{\textcircled{$=$}}}}





\usepackage{xcolor}
% \usepackage[normalem]{ulem}
\usepackage{lipsum}
\makeatletter
% \newcommand\colorwave[1][blue]{\bgroup \markoverwith{\lower3.5\p@\hbox{\sixly \textcolor{#1}{\char58}}}\ULon}
%\font\sixly=lasy6 % does not re-load if already loaded, so no memory problem.

\newmdtheoremenv[
linewidth= 1pt,linecolor= blue,%
leftmargin=20,rightmargin=20,innertopmargin=0pt, innerrightmargin=40,%
tikzsetting = { draw=lightgray, line width = 0.3pt,dashed,%
dash pattern = on 15pt off 3pt},%
splittopskip=\topskip,skipbelow=\baselineskip,%
skipabove=\baselineskip,ntheorem,roundcorner=0pt,
% backgroundcolor=pagebg,font=\color{orange}\sffamily, fontcolor=white
]{examplebox}{Exemple}[section]



\newcommand\R{\mathbb{R}}
\newcommand\Z{\mathbb{Z}}
\newcommand\N{\mathbb{N}}
\newcommand\E{\mathbb{E}}
\newcommand\F{\mathcal{F}}
\newcommand\cH{\mathcal{H}}
\newcommand\V{\mathbb{V}}
\newcommand\dmo{ ^{-1} }
\newcommand\kapa{\kappa}
\newcommand\im{Im}
\newcommand\hs{\mathcal{H}}





\usepackage{soul}

\makeatletter
\newcommand*{\whiten}[1]{\llap{\textcolor{white}{{\the\SOUL@token}}\hspace{#1pt}}}
\DeclareRobustCommand*\myul{%
    \def\SOUL@everyspace{\underline{\space}\kern\z@}%
    \def\SOUL@everytoken{%
     \setbox0=\hbox{\the\SOUL@token}%
     \ifdim\dp0>\z@
        \raisebox{\dp0}{\underline{\phantom{\the\SOUL@token}}}%
        \whiten{1}\whiten{0}%
        \whiten{-1}\whiten{-2}%
        \llap{\the\SOUL@token}%
     \else
        \underline{\the\SOUL@token}%
     \fi}%
\SOUL@}
\makeatother

\newcommand*{\demp}{\fontfamily{lmtt}\selectfont}

\DeclareTextFontCommand{\textdemp}{\demp}

\begin{document}

\ifcomment
Multiline
comment
\fi
\ifcomment
\myul{Typesetting test}
% \color[rgb]{1,1,1}
$∑_i^n≠ 60º±∞π∆¬≈√j∫h≤≥µ$

$\CR \R\pro\ind\pro\gS\pro
\mqty[a&b\\c&d]$
$\pro\mathbb{P}$
$\dd{x}$

  \[
    \alpha(x)=\left\{
                \begin{array}{ll}
                  x\\
                  \frac{1}{1+e^{-kx}}\\
                  \frac{e^x-e^{-x}}{e^x+e^{-x}}
                \end{array}
              \right.
  \]

  $\expval{x}$
  
  $\chi_\rho(ghg\dmo)=\Tr(\rho_{ghg\dmo})=\Tr(\rho_g\circ\rho_h\circ\rho\dmo_g)=\Tr(\rho_h)\overset{\mbox{\scalebox{0.5}{$\Tr(AB)=\Tr(BA)$}}}{=}\chi_\rho(h)$
  	$\mathop{\oplus}_{\substack{x\in X}}$

$\mat(\rho_g)=(a_{ij}(g))_{\scriptsize \substack{1\leq i\leq d \\ 1\leq j\leq d}}$ et $\mat(\rho'_g)=(a'_{ij}(g))_{\scriptsize \substack{1\leq i'\leq d' \\ 1\leq j'\leq d'}}$



\[\int_a^b{\mathbb{R}^2}g(u, v)\dd{P_{XY}}(u, v)=\iint g(u,v) f_{XY}(u, v)\dd \lambda(u) \dd \lambda(v)\]
$$\lim_{x\to\infty} f(x)$$	
$$\iiiint_V \mu(t,u,v,w) \,dt\,du\,dv\,dw$$
$$\sum_{n=1}^{\infty} 2^{-n} = 1$$	
\begin{definition}
	Si $X$ et $Y$ sont 2 v.a. ou definit la \textsc{Covariance} entre $X$ et $Y$ comme
	$\cov(X,Y)\overset{\text{def}}{=}\E\left[(X-\E(X))(Y-\E(Y))\right]=\E(XY)-\E(X)\E(Y)$.
\end{definition}
\fi
\pagebreak

% \tableofcontents

% insert your code here
%\input{./algebra/main.tex}
%\input{./geometrie-differentielle/main.tex}
%\input{./probabilite/main.tex}
%\input{./analyse-fonctionnelle/main.tex}
% \input{./Analyse-convexe-et-dualite-en-optimisation/main.tex}
%\input{./tikz/main.tex}
%\input{./Theorie-du-distributions/main.tex}
%\input{./optimisation/mine.tex}
 \input{./modelisation/main.tex}

% yves.aubry@univ-tln.fr : algebra

\end{document}

%% !TEX encoding = UTF-8 Unicode
% !TEX TS-program = xelatex

\documentclass[french]{report}

%\usepackage[utf8]{inputenc}
%\usepackage[T1]{fontenc}
\usepackage{babel}


\newif\ifcomment
%\commenttrue # Show comments

\usepackage{physics}
\usepackage{amssymb}


\usepackage{amsthm}
% \usepackage{thmtools}
\usepackage{mathtools}
\usepackage{amsfonts}

\usepackage{color}

\usepackage{tikz}

\usepackage{geometry}
\geometry{a5paper, margin=0.1in, right=1cm}

\usepackage{dsfont}

\usepackage{graphicx}
\graphicspath{ {images/} }

\usepackage{faktor}

\usepackage{IEEEtrantools}
\usepackage{enumerate}   
\usepackage[PostScript=dvips]{"/Users/aware/Documents/Courses/diagrams"}


\newtheorem{theorem}{Théorème}[section]
\renewcommand{\thetheorem}{\arabic{theorem}}
\newtheorem{lemme}{Lemme}[section]
\renewcommand{\thelemme}{\arabic{lemme}}
\newtheorem{proposition}{Proposition}[section]
\renewcommand{\theproposition}{\arabic{proposition}}
\newtheorem{notations}{Notations}[section]
\newtheorem{problem}{Problème}[section]
\newtheorem{corollary}{Corollaire}[theorem]
\renewcommand{\thecorollary}{\arabic{corollary}}
\newtheorem{property}{Propriété}[section]
\newtheorem{objective}{Objectif}[section]

\theoremstyle{definition}
\newtheorem{definition}{Définition}[section]
\renewcommand{\thedefinition}{\arabic{definition}}
\newtheorem{exercise}{Exercice}[chapter]
\renewcommand{\theexercise}{\arabic{exercise}}
\newtheorem{example}{Exemple}[chapter]
\renewcommand{\theexample}{\arabic{example}}
\newtheorem*{solution}{Solution}
\newtheorem*{application}{Application}
\newtheorem*{notation}{Notation}
\newtheorem*{vocabulary}{Vocabulaire}
\newtheorem*{properties}{Propriétés}



\theoremstyle{remark}
\newtheorem*{remark}{Remarque}
\newtheorem*{rappel}{Rappel}


\usepackage{etoolbox}
\AtBeginEnvironment{exercise}{\small}
\AtBeginEnvironment{example}{\small}

\usepackage{cases}
\usepackage[red]{mypack}

\usepackage[framemethod=TikZ]{mdframed}

\definecolor{bg}{rgb}{0.4,0.25,0.95}
\definecolor{pagebg}{rgb}{0,0,0.5}
\surroundwithmdframed[
   topline=false,
   rightline=false,
   bottomline=false,
   leftmargin=\parindent,
   skipabove=8pt,
   skipbelow=8pt,
   linecolor=blue,
   innerbottommargin=10pt,
   % backgroundcolor=bg,font=\color{orange}\sffamily, fontcolor=white
]{definition}

\usepackage{empheq}
\usepackage[most]{tcolorbox}

\newtcbox{\mymath}[1][]{%
    nobeforeafter, math upper, tcbox raise base,
    enhanced, colframe=blue!30!black,
    colback=red!10, boxrule=1pt,
    #1}

\usepackage{unixode}


\DeclareMathOperator{\ord}{ord}
\DeclareMathOperator{\orb}{orb}
\DeclareMathOperator{\stab}{stab}
\DeclareMathOperator{\Stab}{stab}
\DeclareMathOperator{\ppcm}{ppcm}
\DeclareMathOperator{\conj}{Conj}
\DeclareMathOperator{\End}{End}
\DeclareMathOperator{\rot}{rot}
\DeclareMathOperator{\trs}{trace}
\DeclareMathOperator{\Ind}{Ind}
\DeclareMathOperator{\mat}{Mat}
\DeclareMathOperator{\id}{Id}
\DeclareMathOperator{\vect}{vect}
\DeclareMathOperator{\img}{img}
\DeclareMathOperator{\cov}{Cov}
\DeclareMathOperator{\dist}{dist}
\DeclareMathOperator{\irr}{Irr}
\DeclareMathOperator{\image}{Im}
\DeclareMathOperator{\pd}{\partial}
\DeclareMathOperator{\epi}{epi}
\DeclareMathOperator{\Argmin}{Argmin}
\DeclareMathOperator{\dom}{dom}
\DeclareMathOperator{\proj}{proj}
\DeclareMathOperator{\ctg}{ctg}
\DeclareMathOperator{\supp}{supp}
\DeclareMathOperator{\argmin}{argmin}
\DeclareMathOperator{\mult}{mult}
\DeclareMathOperator{\ch}{ch}
\DeclareMathOperator{\sh}{sh}
\DeclareMathOperator{\rang}{rang}
\DeclareMathOperator{\diam}{diam}
\DeclareMathOperator{\Epigraphe}{Epigraphe}




\usepackage{xcolor}
\everymath{\color{blue}}
%\everymath{\color[rgb]{0,1,1}}
%\pagecolor[rgb]{0,0,0.5}


\newcommand*{\pdtest}[3][]{\ensuremath{\frac{\partial^{#1} #2}{\partial #3}}}

\newcommand*{\deffunc}[6][]{\ensuremath{
\begin{array}{rcl}
#2 : #3 &\rightarrow& #4\\
#5 &\mapsto& #6
\end{array}
}}

\newcommand{\eqcolon}{\mathrel{\resizebox{\widthof{$\mathord{=}$}}{\height}{ $\!\!=\!\!\resizebox{1.2\width}{0.8\height}{\raisebox{0.23ex}{$\mathop{:}$}}\!\!$ }}}
\newcommand{\coloneq}{\mathrel{\resizebox{\widthof{$\mathord{=}$}}{\height}{ $\!\!\resizebox{1.2\width}{0.8\height}{\raisebox{0.23ex}{$\mathop{:}$}}\!\!=\!\!$ }}}
\newcommand{\eqcolonl}{\ensuremath{\mathrel{=\!\!\mathop{:}}}}
\newcommand{\coloneql}{\ensuremath{\mathrel{\mathop{:} \!\! =}}}
\newcommand{\vc}[1]{% inline column vector
  \left(\begin{smallmatrix}#1\end{smallmatrix}\right)%
}
\newcommand{\vr}[1]{% inline row vector
  \begin{smallmatrix}(\,#1\,)\end{smallmatrix}%
}
\makeatletter
\newcommand*{\defeq}{\ =\mathrel{\rlap{%
                     \raisebox{0.3ex}{$\m@th\cdot$}}%
                     \raisebox{-0.3ex}{$\m@th\cdot$}}%
                     }
\makeatother

\newcommand{\mathcircle}[1]{% inline row vector
 \overset{\circ}{#1}
}
\newcommand{\ulim}{% low limit
 \underline{\lim}
}
\newcommand{\ssi}{% iff
\iff
}
\newcommand{\ps}[2]{
\expval{#1 | #2}
}
\newcommand{\df}[1]{
\mqty{#1}
}
\newcommand{\n}[1]{
\norm{#1}
}
\newcommand{\sys}[1]{
\left\{\smqty{#1}\right.
}


\newcommand{\eqdef}{\ensuremath{\overset{\text{def}}=}}


\def\Circlearrowright{\ensuremath{%
  \rotatebox[origin=c]{230}{$\circlearrowright$}}}

\newcommand\ct[1]{\text{\rmfamily\upshape #1}}
\newcommand\question[1]{ {\color{red} ...!? \small #1}}
\newcommand\caz[1]{\left\{\begin{array} #1 \end{array}\right.}
\newcommand\const{\text{\rmfamily\upshape const}}
\newcommand\toP{ \overset{\pro}{\to}}
\newcommand\toPP{ \overset{\text{PP}}{\to}}
\newcommand{\oeq}{\mathrel{\text{\textcircled{$=$}}}}





\usepackage{xcolor}
% \usepackage[normalem]{ulem}
\usepackage{lipsum}
\makeatletter
% \newcommand\colorwave[1][blue]{\bgroup \markoverwith{\lower3.5\p@\hbox{\sixly \textcolor{#1}{\char58}}}\ULon}
%\font\sixly=lasy6 % does not re-load if already loaded, so no memory problem.

\newmdtheoremenv[
linewidth= 1pt,linecolor= blue,%
leftmargin=20,rightmargin=20,innertopmargin=0pt, innerrightmargin=40,%
tikzsetting = { draw=lightgray, line width = 0.3pt,dashed,%
dash pattern = on 15pt off 3pt},%
splittopskip=\topskip,skipbelow=\baselineskip,%
skipabove=\baselineskip,ntheorem,roundcorner=0pt,
% backgroundcolor=pagebg,font=\color{orange}\sffamily, fontcolor=white
]{examplebox}{Exemple}[section]



\newcommand\R{\mathbb{R}}
\newcommand\Z{\mathbb{Z}}
\newcommand\N{\mathbb{N}}
\newcommand\E{\mathbb{E}}
\newcommand\F{\mathcal{F}}
\newcommand\cH{\mathcal{H}}
\newcommand\V{\mathbb{V}}
\newcommand\dmo{ ^{-1} }
\newcommand\kapa{\kappa}
\newcommand\im{Im}
\newcommand\hs{\mathcal{H}}





\usepackage{soul}

\makeatletter
\newcommand*{\whiten}[1]{\llap{\textcolor{white}{{\the\SOUL@token}}\hspace{#1pt}}}
\DeclareRobustCommand*\myul{%
    \def\SOUL@everyspace{\underline{\space}\kern\z@}%
    \def\SOUL@everytoken{%
     \setbox0=\hbox{\the\SOUL@token}%
     \ifdim\dp0>\z@
        \raisebox{\dp0}{\underline{\phantom{\the\SOUL@token}}}%
        \whiten{1}\whiten{0}%
        \whiten{-1}\whiten{-2}%
        \llap{\the\SOUL@token}%
     \else
        \underline{\the\SOUL@token}%
     \fi}%
\SOUL@}
\makeatother

\newcommand*{\demp}{\fontfamily{lmtt}\selectfont}

\DeclareTextFontCommand{\textdemp}{\demp}

\begin{document}

\ifcomment
Multiline
comment
\fi
\ifcomment
\myul{Typesetting test}
% \color[rgb]{1,1,1}
$∑_i^n≠ 60º±∞π∆¬≈√j∫h≤≥µ$

$\CR \R\pro\ind\pro\gS\pro
\mqty[a&b\\c&d]$
$\pro\mathbb{P}$
$\dd{x}$

  \[
    \alpha(x)=\left\{
                \begin{array}{ll}
                  x\\
                  \frac{1}{1+e^{-kx}}\\
                  \frac{e^x-e^{-x}}{e^x+e^{-x}}
                \end{array}
              \right.
  \]

  $\expval{x}$
  
  $\chi_\rho(ghg\dmo)=\Tr(\rho_{ghg\dmo})=\Tr(\rho_g\circ\rho_h\circ\rho\dmo_g)=\Tr(\rho_h)\overset{\mbox{\scalebox{0.5}{$\Tr(AB)=\Tr(BA)$}}}{=}\chi_\rho(h)$
  	$\mathop{\oplus}_{\substack{x\in X}}$

$\mat(\rho_g)=(a_{ij}(g))_{\scriptsize \substack{1\leq i\leq d \\ 1\leq j\leq d}}$ et $\mat(\rho'_g)=(a'_{ij}(g))_{\scriptsize \substack{1\leq i'\leq d' \\ 1\leq j'\leq d'}}$



\[\int_a^b{\mathbb{R}^2}g(u, v)\dd{P_{XY}}(u, v)=\iint g(u,v) f_{XY}(u, v)\dd \lambda(u) \dd \lambda(v)\]
$$\lim_{x\to\infty} f(x)$$	
$$\iiiint_V \mu(t,u,v,w) \,dt\,du\,dv\,dw$$
$$\sum_{n=1}^{\infty} 2^{-n} = 1$$	
\begin{definition}
	Si $X$ et $Y$ sont 2 v.a. ou definit la \textsc{Covariance} entre $X$ et $Y$ comme
	$\cov(X,Y)\overset{\text{def}}{=}\E\left[(X-\E(X))(Y-\E(Y))\right]=\E(XY)-\E(X)\E(Y)$.
\end{definition}
\fi
\pagebreak

% \tableofcontents

% insert your code here
%\input{./algebra/main.tex}
%\input{./geometrie-differentielle/main.tex}
%\input{./probabilite/main.tex}
%\input{./analyse-fonctionnelle/main.tex}
% \input{./Analyse-convexe-et-dualite-en-optimisation/main.tex}
%\input{./tikz/main.tex}
%\input{./Theorie-du-distributions/main.tex}
%\input{./optimisation/mine.tex}
 \input{./modelisation/main.tex}

% yves.aubry@univ-tln.fr : algebra

\end{document}

%\input{./optimisation/mine.tex}
 % !TEX encoding = UTF-8 Unicode
% !TEX TS-program = xelatex

\documentclass[french]{report}

%\usepackage[utf8]{inputenc}
%\usepackage[T1]{fontenc}
\usepackage{babel}


\newif\ifcomment
%\commenttrue # Show comments

\usepackage{physics}
\usepackage{amssymb}


\usepackage{amsthm}
% \usepackage{thmtools}
\usepackage{mathtools}
\usepackage{amsfonts}

\usepackage{color}

\usepackage{tikz}

\usepackage{geometry}
\geometry{a5paper, margin=0.1in, right=1cm}

\usepackage{dsfont}

\usepackage{graphicx}
\graphicspath{ {images/} }

\usepackage{faktor}

\usepackage{IEEEtrantools}
\usepackage{enumerate}   
\usepackage[PostScript=dvips]{"/Users/aware/Documents/Courses/diagrams"}


\newtheorem{theorem}{Théorème}[section]
\renewcommand{\thetheorem}{\arabic{theorem}}
\newtheorem{lemme}{Lemme}[section]
\renewcommand{\thelemme}{\arabic{lemme}}
\newtheorem{proposition}{Proposition}[section]
\renewcommand{\theproposition}{\arabic{proposition}}
\newtheorem{notations}{Notations}[section]
\newtheorem{problem}{Problème}[section]
\newtheorem{corollary}{Corollaire}[theorem]
\renewcommand{\thecorollary}{\arabic{corollary}}
\newtheorem{property}{Propriété}[section]
\newtheorem{objective}{Objectif}[section]

\theoremstyle{definition}
\newtheorem{definition}{Définition}[section]
\renewcommand{\thedefinition}{\arabic{definition}}
\newtheorem{exercise}{Exercice}[chapter]
\renewcommand{\theexercise}{\arabic{exercise}}
\newtheorem{example}{Exemple}[chapter]
\renewcommand{\theexample}{\arabic{example}}
\newtheorem*{solution}{Solution}
\newtheorem*{application}{Application}
\newtheorem*{notation}{Notation}
\newtheorem*{vocabulary}{Vocabulaire}
\newtheorem*{properties}{Propriétés}



\theoremstyle{remark}
\newtheorem*{remark}{Remarque}
\newtheorem*{rappel}{Rappel}


\usepackage{etoolbox}
\AtBeginEnvironment{exercise}{\small}
\AtBeginEnvironment{example}{\small}

\usepackage{cases}
\usepackage[red]{mypack}

\usepackage[framemethod=TikZ]{mdframed}

\definecolor{bg}{rgb}{0.4,0.25,0.95}
\definecolor{pagebg}{rgb}{0,0,0.5}
\surroundwithmdframed[
   topline=false,
   rightline=false,
   bottomline=false,
   leftmargin=\parindent,
   skipabove=8pt,
   skipbelow=8pt,
   linecolor=blue,
   innerbottommargin=10pt,
   % backgroundcolor=bg,font=\color{orange}\sffamily, fontcolor=white
]{definition}

\usepackage{empheq}
\usepackage[most]{tcolorbox}

\newtcbox{\mymath}[1][]{%
    nobeforeafter, math upper, tcbox raise base,
    enhanced, colframe=blue!30!black,
    colback=red!10, boxrule=1pt,
    #1}

\usepackage{unixode}


\DeclareMathOperator{\ord}{ord}
\DeclareMathOperator{\orb}{orb}
\DeclareMathOperator{\stab}{stab}
\DeclareMathOperator{\Stab}{stab}
\DeclareMathOperator{\ppcm}{ppcm}
\DeclareMathOperator{\conj}{Conj}
\DeclareMathOperator{\End}{End}
\DeclareMathOperator{\rot}{rot}
\DeclareMathOperator{\trs}{trace}
\DeclareMathOperator{\Ind}{Ind}
\DeclareMathOperator{\mat}{Mat}
\DeclareMathOperator{\id}{Id}
\DeclareMathOperator{\vect}{vect}
\DeclareMathOperator{\img}{img}
\DeclareMathOperator{\cov}{Cov}
\DeclareMathOperator{\dist}{dist}
\DeclareMathOperator{\irr}{Irr}
\DeclareMathOperator{\image}{Im}
\DeclareMathOperator{\pd}{\partial}
\DeclareMathOperator{\epi}{epi}
\DeclareMathOperator{\Argmin}{Argmin}
\DeclareMathOperator{\dom}{dom}
\DeclareMathOperator{\proj}{proj}
\DeclareMathOperator{\ctg}{ctg}
\DeclareMathOperator{\supp}{supp}
\DeclareMathOperator{\argmin}{argmin}
\DeclareMathOperator{\mult}{mult}
\DeclareMathOperator{\ch}{ch}
\DeclareMathOperator{\sh}{sh}
\DeclareMathOperator{\rang}{rang}
\DeclareMathOperator{\diam}{diam}
\DeclareMathOperator{\Epigraphe}{Epigraphe}




\usepackage{xcolor}
\everymath{\color{blue}}
%\everymath{\color[rgb]{0,1,1}}
%\pagecolor[rgb]{0,0,0.5}


\newcommand*{\pdtest}[3][]{\ensuremath{\frac{\partial^{#1} #2}{\partial #3}}}

\newcommand*{\deffunc}[6][]{\ensuremath{
\begin{array}{rcl}
#2 : #3 &\rightarrow& #4\\
#5 &\mapsto& #6
\end{array}
}}

\newcommand{\eqcolon}{\mathrel{\resizebox{\widthof{$\mathord{=}$}}{\height}{ $\!\!=\!\!\resizebox{1.2\width}{0.8\height}{\raisebox{0.23ex}{$\mathop{:}$}}\!\!$ }}}
\newcommand{\coloneq}{\mathrel{\resizebox{\widthof{$\mathord{=}$}}{\height}{ $\!\!\resizebox{1.2\width}{0.8\height}{\raisebox{0.23ex}{$\mathop{:}$}}\!\!=\!\!$ }}}
\newcommand{\eqcolonl}{\ensuremath{\mathrel{=\!\!\mathop{:}}}}
\newcommand{\coloneql}{\ensuremath{\mathrel{\mathop{:} \!\! =}}}
\newcommand{\vc}[1]{% inline column vector
  \left(\begin{smallmatrix}#1\end{smallmatrix}\right)%
}
\newcommand{\vr}[1]{% inline row vector
  \begin{smallmatrix}(\,#1\,)\end{smallmatrix}%
}
\makeatletter
\newcommand*{\defeq}{\ =\mathrel{\rlap{%
                     \raisebox{0.3ex}{$\m@th\cdot$}}%
                     \raisebox{-0.3ex}{$\m@th\cdot$}}%
                     }
\makeatother

\newcommand{\mathcircle}[1]{% inline row vector
 \overset{\circ}{#1}
}
\newcommand{\ulim}{% low limit
 \underline{\lim}
}
\newcommand{\ssi}{% iff
\iff
}
\newcommand{\ps}[2]{
\expval{#1 | #2}
}
\newcommand{\df}[1]{
\mqty{#1}
}
\newcommand{\n}[1]{
\norm{#1}
}
\newcommand{\sys}[1]{
\left\{\smqty{#1}\right.
}


\newcommand{\eqdef}{\ensuremath{\overset{\text{def}}=}}


\def\Circlearrowright{\ensuremath{%
  \rotatebox[origin=c]{230}{$\circlearrowright$}}}

\newcommand\ct[1]{\text{\rmfamily\upshape #1}}
\newcommand\question[1]{ {\color{red} ...!? \small #1}}
\newcommand\caz[1]{\left\{\begin{array} #1 \end{array}\right.}
\newcommand\const{\text{\rmfamily\upshape const}}
\newcommand\toP{ \overset{\pro}{\to}}
\newcommand\toPP{ \overset{\text{PP}}{\to}}
\newcommand{\oeq}{\mathrel{\text{\textcircled{$=$}}}}





\usepackage{xcolor}
% \usepackage[normalem]{ulem}
\usepackage{lipsum}
\makeatletter
% \newcommand\colorwave[1][blue]{\bgroup \markoverwith{\lower3.5\p@\hbox{\sixly \textcolor{#1}{\char58}}}\ULon}
%\font\sixly=lasy6 % does not re-load if already loaded, so no memory problem.

\newmdtheoremenv[
linewidth= 1pt,linecolor= blue,%
leftmargin=20,rightmargin=20,innertopmargin=0pt, innerrightmargin=40,%
tikzsetting = { draw=lightgray, line width = 0.3pt,dashed,%
dash pattern = on 15pt off 3pt},%
splittopskip=\topskip,skipbelow=\baselineskip,%
skipabove=\baselineskip,ntheorem,roundcorner=0pt,
% backgroundcolor=pagebg,font=\color{orange}\sffamily, fontcolor=white
]{examplebox}{Exemple}[section]



\newcommand\R{\mathbb{R}}
\newcommand\Z{\mathbb{Z}}
\newcommand\N{\mathbb{N}}
\newcommand\E{\mathbb{E}}
\newcommand\F{\mathcal{F}}
\newcommand\cH{\mathcal{H}}
\newcommand\V{\mathbb{V}}
\newcommand\dmo{ ^{-1} }
\newcommand\kapa{\kappa}
\newcommand\im{Im}
\newcommand\hs{\mathcal{H}}





\usepackage{soul}

\makeatletter
\newcommand*{\whiten}[1]{\llap{\textcolor{white}{{\the\SOUL@token}}\hspace{#1pt}}}
\DeclareRobustCommand*\myul{%
    \def\SOUL@everyspace{\underline{\space}\kern\z@}%
    \def\SOUL@everytoken{%
     \setbox0=\hbox{\the\SOUL@token}%
     \ifdim\dp0>\z@
        \raisebox{\dp0}{\underline{\phantom{\the\SOUL@token}}}%
        \whiten{1}\whiten{0}%
        \whiten{-1}\whiten{-2}%
        \llap{\the\SOUL@token}%
     \else
        \underline{\the\SOUL@token}%
     \fi}%
\SOUL@}
\makeatother

\newcommand*{\demp}{\fontfamily{lmtt}\selectfont}

\DeclareTextFontCommand{\textdemp}{\demp}

\begin{document}

\ifcomment
Multiline
comment
\fi
\ifcomment
\myul{Typesetting test}
% \color[rgb]{1,1,1}
$∑_i^n≠ 60º±∞π∆¬≈√j∫h≤≥µ$

$\CR \R\pro\ind\pro\gS\pro
\mqty[a&b\\c&d]$
$\pro\mathbb{P}$
$\dd{x}$

  \[
    \alpha(x)=\left\{
                \begin{array}{ll}
                  x\\
                  \frac{1}{1+e^{-kx}}\\
                  \frac{e^x-e^{-x}}{e^x+e^{-x}}
                \end{array}
              \right.
  \]

  $\expval{x}$
  
  $\chi_\rho(ghg\dmo)=\Tr(\rho_{ghg\dmo})=\Tr(\rho_g\circ\rho_h\circ\rho\dmo_g)=\Tr(\rho_h)\overset{\mbox{\scalebox{0.5}{$\Tr(AB)=\Tr(BA)$}}}{=}\chi_\rho(h)$
  	$\mathop{\oplus}_{\substack{x\in X}}$

$\mat(\rho_g)=(a_{ij}(g))_{\scriptsize \substack{1\leq i\leq d \\ 1\leq j\leq d}}$ et $\mat(\rho'_g)=(a'_{ij}(g))_{\scriptsize \substack{1\leq i'\leq d' \\ 1\leq j'\leq d'}}$



\[\int_a^b{\mathbb{R}^2}g(u, v)\dd{P_{XY}}(u, v)=\iint g(u,v) f_{XY}(u, v)\dd \lambda(u) \dd \lambda(v)\]
$$\lim_{x\to\infty} f(x)$$	
$$\iiiint_V \mu(t,u,v,w) \,dt\,du\,dv\,dw$$
$$\sum_{n=1}^{\infty} 2^{-n} = 1$$	
\begin{definition}
	Si $X$ et $Y$ sont 2 v.a. ou definit la \textsc{Covariance} entre $X$ et $Y$ comme
	$\cov(X,Y)\overset{\text{def}}{=}\E\left[(X-\E(X))(Y-\E(Y))\right]=\E(XY)-\E(X)\E(Y)$.
\end{definition}
\fi
\pagebreak

% \tableofcontents

% insert your code here
%\input{./algebra/main.tex}
%\input{./geometrie-differentielle/main.tex}
%\input{./probabilite/main.tex}
%\input{./analyse-fonctionnelle/main.tex}
% \input{./Analyse-convexe-et-dualite-en-optimisation/main.tex}
%\input{./tikz/main.tex}
%\input{./Theorie-du-distributions/main.tex}
%\input{./optimisation/mine.tex}
 \input{./modelisation/main.tex}

% yves.aubry@univ-tln.fr : algebra

\end{document}


% yves.aubry@univ-tln.fr : algebra

\end{document}

%\input{./optimisation/mine.tex}
 % !TEX encoding = UTF-8 Unicode
% !TEX TS-program = xelatex

\documentclass[french]{report}

%\usepackage[utf8]{inputenc}
%\usepackage[T1]{fontenc}
\usepackage{babel}


\newif\ifcomment
%\commenttrue # Show comments

\usepackage{physics}
\usepackage{amssymb}


\usepackage{amsthm}
% \usepackage{thmtools}
\usepackage{mathtools}
\usepackage{amsfonts}

\usepackage{color}

\usepackage{tikz}

\usepackage{geometry}
\geometry{a5paper, margin=0.1in, right=1cm}

\usepackage{dsfont}

\usepackage{graphicx}
\graphicspath{ {images/} }

\usepackage{faktor}

\usepackage{IEEEtrantools}
\usepackage{enumerate}   
\usepackage[PostScript=dvips]{"/Users/aware/Documents/Courses/diagrams"}


\newtheorem{theorem}{Théorème}[section]
\renewcommand{\thetheorem}{\arabic{theorem}}
\newtheorem{lemme}{Lemme}[section]
\renewcommand{\thelemme}{\arabic{lemme}}
\newtheorem{proposition}{Proposition}[section]
\renewcommand{\theproposition}{\arabic{proposition}}
\newtheorem{notations}{Notations}[section]
\newtheorem{problem}{Problème}[section]
\newtheorem{corollary}{Corollaire}[theorem]
\renewcommand{\thecorollary}{\arabic{corollary}}
\newtheorem{property}{Propriété}[section]
\newtheorem{objective}{Objectif}[section]

\theoremstyle{definition}
\newtheorem{definition}{Définition}[section]
\renewcommand{\thedefinition}{\arabic{definition}}
\newtheorem{exercise}{Exercice}[chapter]
\renewcommand{\theexercise}{\arabic{exercise}}
\newtheorem{example}{Exemple}[chapter]
\renewcommand{\theexample}{\arabic{example}}
\newtheorem*{solution}{Solution}
\newtheorem*{application}{Application}
\newtheorem*{notation}{Notation}
\newtheorem*{vocabulary}{Vocabulaire}
\newtheorem*{properties}{Propriétés}



\theoremstyle{remark}
\newtheorem*{remark}{Remarque}
\newtheorem*{rappel}{Rappel}


\usepackage{etoolbox}
\AtBeginEnvironment{exercise}{\small}
\AtBeginEnvironment{example}{\small}

\usepackage{cases}
\usepackage[red]{mypack}

\usepackage[framemethod=TikZ]{mdframed}

\definecolor{bg}{rgb}{0.4,0.25,0.95}
\definecolor{pagebg}{rgb}{0,0,0.5}
\surroundwithmdframed[
   topline=false,
   rightline=false,
   bottomline=false,
   leftmargin=\parindent,
   skipabove=8pt,
   skipbelow=8pt,
   linecolor=blue,
   innerbottommargin=10pt,
   % backgroundcolor=bg,font=\color{orange}\sffamily, fontcolor=white
]{definition}

\usepackage{empheq}
\usepackage[most]{tcolorbox}

\newtcbox{\mymath}[1][]{%
    nobeforeafter, math upper, tcbox raise base,
    enhanced, colframe=blue!30!black,
    colback=red!10, boxrule=1pt,
    #1}

\usepackage{unixode}


\DeclareMathOperator{\ord}{ord}
\DeclareMathOperator{\orb}{orb}
\DeclareMathOperator{\stab}{stab}
\DeclareMathOperator{\Stab}{stab}
\DeclareMathOperator{\ppcm}{ppcm}
\DeclareMathOperator{\conj}{Conj}
\DeclareMathOperator{\End}{End}
\DeclareMathOperator{\rot}{rot}
\DeclareMathOperator{\trs}{trace}
\DeclareMathOperator{\Ind}{Ind}
\DeclareMathOperator{\mat}{Mat}
\DeclareMathOperator{\id}{Id}
\DeclareMathOperator{\vect}{vect}
\DeclareMathOperator{\img}{img}
\DeclareMathOperator{\cov}{Cov}
\DeclareMathOperator{\dist}{dist}
\DeclareMathOperator{\irr}{Irr}
\DeclareMathOperator{\image}{Im}
\DeclareMathOperator{\pd}{\partial}
\DeclareMathOperator{\epi}{epi}
\DeclareMathOperator{\Argmin}{Argmin}
\DeclareMathOperator{\dom}{dom}
\DeclareMathOperator{\proj}{proj}
\DeclareMathOperator{\ctg}{ctg}
\DeclareMathOperator{\supp}{supp}
\DeclareMathOperator{\argmin}{argmin}
\DeclareMathOperator{\mult}{mult}
\DeclareMathOperator{\ch}{ch}
\DeclareMathOperator{\sh}{sh}
\DeclareMathOperator{\rang}{rang}
\DeclareMathOperator{\diam}{diam}
\DeclareMathOperator{\Epigraphe}{Epigraphe}




\usepackage{xcolor}
\everymath{\color{blue}}
%\everymath{\color[rgb]{0,1,1}}
%\pagecolor[rgb]{0,0,0.5}


\newcommand*{\pdtest}[3][]{\ensuremath{\frac{\partial^{#1} #2}{\partial #3}}}

\newcommand*{\deffunc}[6][]{\ensuremath{
\begin{array}{rcl}
#2 : #3 &\rightarrow& #4\\
#5 &\mapsto& #6
\end{array}
}}

\newcommand{\eqcolon}{\mathrel{\resizebox{\widthof{$\mathord{=}$}}{\height}{ $\!\!=\!\!\resizebox{1.2\width}{0.8\height}{\raisebox{0.23ex}{$\mathop{:}$}}\!\!$ }}}
\newcommand{\coloneq}{\mathrel{\resizebox{\widthof{$\mathord{=}$}}{\height}{ $\!\!\resizebox{1.2\width}{0.8\height}{\raisebox{0.23ex}{$\mathop{:}$}}\!\!=\!\!$ }}}
\newcommand{\eqcolonl}{\ensuremath{\mathrel{=\!\!\mathop{:}}}}
\newcommand{\coloneql}{\ensuremath{\mathrel{\mathop{:} \!\! =}}}
\newcommand{\vc}[1]{% inline column vector
  \left(\begin{smallmatrix}#1\end{smallmatrix}\right)%
}
\newcommand{\vr}[1]{% inline row vector
  \begin{smallmatrix}(\,#1\,)\end{smallmatrix}%
}
\makeatletter
\newcommand*{\defeq}{\ =\mathrel{\rlap{%
                     \raisebox{0.3ex}{$\m@th\cdot$}}%
                     \raisebox{-0.3ex}{$\m@th\cdot$}}%
                     }
\makeatother

\newcommand{\mathcircle}[1]{% inline row vector
 \overset{\circ}{#1}
}
\newcommand{\ulim}{% low limit
 \underline{\lim}
}
\newcommand{\ssi}{% iff
\iff
}
\newcommand{\ps}[2]{
\expval{#1 | #2}
}
\newcommand{\df}[1]{
\mqty{#1}
}
\newcommand{\n}[1]{
\norm{#1}
}
\newcommand{\sys}[1]{
\left\{\smqty{#1}\right.
}


\newcommand{\eqdef}{\ensuremath{\overset{\text{def}}=}}


\def\Circlearrowright{\ensuremath{%
  \rotatebox[origin=c]{230}{$\circlearrowright$}}}

\newcommand\ct[1]{\text{\rmfamily\upshape #1}}
\newcommand\question[1]{ {\color{red} ...!? \small #1}}
\newcommand\caz[1]{\left\{\begin{array} #1 \end{array}\right.}
\newcommand\const{\text{\rmfamily\upshape const}}
\newcommand\toP{ \overset{\pro}{\to}}
\newcommand\toPP{ \overset{\text{PP}}{\to}}
\newcommand{\oeq}{\mathrel{\text{\textcircled{$=$}}}}





\usepackage{xcolor}
% \usepackage[normalem]{ulem}
\usepackage{lipsum}
\makeatletter
% \newcommand\colorwave[1][blue]{\bgroup \markoverwith{\lower3.5\p@\hbox{\sixly \textcolor{#1}{\char58}}}\ULon}
%\font\sixly=lasy6 % does not re-load if already loaded, so no memory problem.

\newmdtheoremenv[
linewidth= 1pt,linecolor= blue,%
leftmargin=20,rightmargin=20,innertopmargin=0pt, innerrightmargin=40,%
tikzsetting = { draw=lightgray, line width = 0.3pt,dashed,%
dash pattern = on 15pt off 3pt},%
splittopskip=\topskip,skipbelow=\baselineskip,%
skipabove=\baselineskip,ntheorem,roundcorner=0pt,
% backgroundcolor=pagebg,font=\color{orange}\sffamily, fontcolor=white
]{examplebox}{Exemple}[section]



\newcommand\R{\mathbb{R}}
\newcommand\Z{\mathbb{Z}}
\newcommand\N{\mathbb{N}}
\newcommand\E{\mathbb{E}}
\newcommand\F{\mathcal{F}}
\newcommand\cH{\mathcal{H}}
\newcommand\V{\mathbb{V}}
\newcommand\dmo{ ^{-1} }
\newcommand\kapa{\kappa}
\newcommand\im{Im}
\newcommand\hs{\mathcal{H}}





\usepackage{soul}

\makeatletter
\newcommand*{\whiten}[1]{\llap{\textcolor{white}{{\the\SOUL@token}}\hspace{#1pt}}}
\DeclareRobustCommand*\myul{%
    \def\SOUL@everyspace{\underline{\space}\kern\z@}%
    \def\SOUL@everytoken{%
     \setbox0=\hbox{\the\SOUL@token}%
     \ifdim\dp0>\z@
        \raisebox{\dp0}{\underline{\phantom{\the\SOUL@token}}}%
        \whiten{1}\whiten{0}%
        \whiten{-1}\whiten{-2}%
        \llap{\the\SOUL@token}%
     \else
        \underline{\the\SOUL@token}%
     \fi}%
\SOUL@}
\makeatother

\newcommand*{\demp}{\fontfamily{lmtt}\selectfont}

\DeclareTextFontCommand{\textdemp}{\demp}

\begin{document}

\ifcomment
Multiline
comment
\fi
\ifcomment
\myul{Typesetting test}
% \color[rgb]{1,1,1}
$∑_i^n≠ 60º±∞π∆¬≈√j∫h≤≥µ$

$\CR \R\pro\ind\pro\gS\pro
\mqty[a&b\\c&d]$
$\pro\mathbb{P}$
$\dd{x}$

  \[
    \alpha(x)=\left\{
                \begin{array}{ll}
                  x\\
                  \frac{1}{1+e^{-kx}}\\
                  \frac{e^x-e^{-x}}{e^x+e^{-x}}
                \end{array}
              \right.
  \]

  $\expval{x}$
  
  $\chi_\rho(ghg\dmo)=\Tr(\rho_{ghg\dmo})=\Tr(\rho_g\circ\rho_h\circ\rho\dmo_g)=\Tr(\rho_h)\overset{\mbox{\scalebox{0.5}{$\Tr(AB)=\Tr(BA)$}}}{=}\chi_\rho(h)$
  	$\mathop{\oplus}_{\substack{x\in X}}$

$\mat(\rho_g)=(a_{ij}(g))_{\scriptsize \substack{1\leq i\leq d \\ 1\leq j\leq d}}$ et $\mat(\rho'_g)=(a'_{ij}(g))_{\scriptsize \substack{1\leq i'\leq d' \\ 1\leq j'\leq d'}}$



\[\int_a^b{\mathbb{R}^2}g(u, v)\dd{P_{XY}}(u, v)=\iint g(u,v) f_{XY}(u, v)\dd \lambda(u) \dd \lambda(v)\]
$$\lim_{x\to\infty} f(x)$$	
$$\iiiint_V \mu(t,u,v,w) \,dt\,du\,dv\,dw$$
$$\sum_{n=1}^{\infty} 2^{-n} = 1$$	
\begin{definition}
	Si $X$ et $Y$ sont 2 v.a. ou definit la \textsc{Covariance} entre $X$ et $Y$ comme
	$\cov(X,Y)\overset{\text{def}}{=}\E\left[(X-\E(X))(Y-\E(Y))\right]=\E(XY)-\E(X)\E(Y)$.
\end{definition}
\fi
\pagebreak

% \tableofcontents

% insert your code here
%% !TEX encoding = UTF-8 Unicode
% !TEX TS-program = xelatex

\documentclass[french]{report}

%\usepackage[utf8]{inputenc}
%\usepackage[T1]{fontenc}
\usepackage{babel}


\newif\ifcomment
%\commenttrue # Show comments

\usepackage{physics}
\usepackage{amssymb}


\usepackage{amsthm}
% \usepackage{thmtools}
\usepackage{mathtools}
\usepackage{amsfonts}

\usepackage{color}

\usepackage{tikz}

\usepackage{geometry}
\geometry{a5paper, margin=0.1in, right=1cm}

\usepackage{dsfont}

\usepackage{graphicx}
\graphicspath{ {images/} }

\usepackage{faktor}

\usepackage{IEEEtrantools}
\usepackage{enumerate}   
\usepackage[PostScript=dvips]{"/Users/aware/Documents/Courses/diagrams"}


\newtheorem{theorem}{Théorème}[section]
\renewcommand{\thetheorem}{\arabic{theorem}}
\newtheorem{lemme}{Lemme}[section]
\renewcommand{\thelemme}{\arabic{lemme}}
\newtheorem{proposition}{Proposition}[section]
\renewcommand{\theproposition}{\arabic{proposition}}
\newtheorem{notations}{Notations}[section]
\newtheorem{problem}{Problème}[section]
\newtheorem{corollary}{Corollaire}[theorem]
\renewcommand{\thecorollary}{\arabic{corollary}}
\newtheorem{property}{Propriété}[section]
\newtheorem{objective}{Objectif}[section]

\theoremstyle{definition}
\newtheorem{definition}{Définition}[section]
\renewcommand{\thedefinition}{\arabic{definition}}
\newtheorem{exercise}{Exercice}[chapter]
\renewcommand{\theexercise}{\arabic{exercise}}
\newtheorem{example}{Exemple}[chapter]
\renewcommand{\theexample}{\arabic{example}}
\newtheorem*{solution}{Solution}
\newtheorem*{application}{Application}
\newtheorem*{notation}{Notation}
\newtheorem*{vocabulary}{Vocabulaire}
\newtheorem*{properties}{Propriétés}



\theoremstyle{remark}
\newtheorem*{remark}{Remarque}
\newtheorem*{rappel}{Rappel}


\usepackage{etoolbox}
\AtBeginEnvironment{exercise}{\small}
\AtBeginEnvironment{example}{\small}

\usepackage{cases}
\usepackage[red]{mypack}

\usepackage[framemethod=TikZ]{mdframed}

\definecolor{bg}{rgb}{0.4,0.25,0.95}
\definecolor{pagebg}{rgb}{0,0,0.5}
\surroundwithmdframed[
   topline=false,
   rightline=false,
   bottomline=false,
   leftmargin=\parindent,
   skipabove=8pt,
   skipbelow=8pt,
   linecolor=blue,
   innerbottommargin=10pt,
   % backgroundcolor=bg,font=\color{orange}\sffamily, fontcolor=white
]{definition}

\usepackage{empheq}
\usepackage[most]{tcolorbox}

\newtcbox{\mymath}[1][]{%
    nobeforeafter, math upper, tcbox raise base,
    enhanced, colframe=blue!30!black,
    colback=red!10, boxrule=1pt,
    #1}

\usepackage{unixode}


\DeclareMathOperator{\ord}{ord}
\DeclareMathOperator{\orb}{orb}
\DeclareMathOperator{\stab}{stab}
\DeclareMathOperator{\Stab}{stab}
\DeclareMathOperator{\ppcm}{ppcm}
\DeclareMathOperator{\conj}{Conj}
\DeclareMathOperator{\End}{End}
\DeclareMathOperator{\rot}{rot}
\DeclareMathOperator{\trs}{trace}
\DeclareMathOperator{\Ind}{Ind}
\DeclareMathOperator{\mat}{Mat}
\DeclareMathOperator{\id}{Id}
\DeclareMathOperator{\vect}{vect}
\DeclareMathOperator{\img}{img}
\DeclareMathOperator{\cov}{Cov}
\DeclareMathOperator{\dist}{dist}
\DeclareMathOperator{\irr}{Irr}
\DeclareMathOperator{\image}{Im}
\DeclareMathOperator{\pd}{\partial}
\DeclareMathOperator{\epi}{epi}
\DeclareMathOperator{\Argmin}{Argmin}
\DeclareMathOperator{\dom}{dom}
\DeclareMathOperator{\proj}{proj}
\DeclareMathOperator{\ctg}{ctg}
\DeclareMathOperator{\supp}{supp}
\DeclareMathOperator{\argmin}{argmin}
\DeclareMathOperator{\mult}{mult}
\DeclareMathOperator{\ch}{ch}
\DeclareMathOperator{\sh}{sh}
\DeclareMathOperator{\rang}{rang}
\DeclareMathOperator{\diam}{diam}
\DeclareMathOperator{\Epigraphe}{Epigraphe}




\usepackage{xcolor}
\everymath{\color{blue}}
%\everymath{\color[rgb]{0,1,1}}
%\pagecolor[rgb]{0,0,0.5}


\newcommand*{\pdtest}[3][]{\ensuremath{\frac{\partial^{#1} #2}{\partial #3}}}

\newcommand*{\deffunc}[6][]{\ensuremath{
\begin{array}{rcl}
#2 : #3 &\rightarrow& #4\\
#5 &\mapsto& #6
\end{array}
}}

\newcommand{\eqcolon}{\mathrel{\resizebox{\widthof{$\mathord{=}$}}{\height}{ $\!\!=\!\!\resizebox{1.2\width}{0.8\height}{\raisebox{0.23ex}{$\mathop{:}$}}\!\!$ }}}
\newcommand{\coloneq}{\mathrel{\resizebox{\widthof{$\mathord{=}$}}{\height}{ $\!\!\resizebox{1.2\width}{0.8\height}{\raisebox{0.23ex}{$\mathop{:}$}}\!\!=\!\!$ }}}
\newcommand{\eqcolonl}{\ensuremath{\mathrel{=\!\!\mathop{:}}}}
\newcommand{\coloneql}{\ensuremath{\mathrel{\mathop{:} \!\! =}}}
\newcommand{\vc}[1]{% inline column vector
  \left(\begin{smallmatrix}#1\end{smallmatrix}\right)%
}
\newcommand{\vr}[1]{% inline row vector
  \begin{smallmatrix}(\,#1\,)\end{smallmatrix}%
}
\makeatletter
\newcommand*{\defeq}{\ =\mathrel{\rlap{%
                     \raisebox{0.3ex}{$\m@th\cdot$}}%
                     \raisebox{-0.3ex}{$\m@th\cdot$}}%
                     }
\makeatother

\newcommand{\mathcircle}[1]{% inline row vector
 \overset{\circ}{#1}
}
\newcommand{\ulim}{% low limit
 \underline{\lim}
}
\newcommand{\ssi}{% iff
\iff
}
\newcommand{\ps}[2]{
\expval{#1 | #2}
}
\newcommand{\df}[1]{
\mqty{#1}
}
\newcommand{\n}[1]{
\norm{#1}
}
\newcommand{\sys}[1]{
\left\{\smqty{#1}\right.
}


\newcommand{\eqdef}{\ensuremath{\overset{\text{def}}=}}


\def\Circlearrowright{\ensuremath{%
  \rotatebox[origin=c]{230}{$\circlearrowright$}}}

\newcommand\ct[1]{\text{\rmfamily\upshape #1}}
\newcommand\question[1]{ {\color{red} ...!? \small #1}}
\newcommand\caz[1]{\left\{\begin{array} #1 \end{array}\right.}
\newcommand\const{\text{\rmfamily\upshape const}}
\newcommand\toP{ \overset{\pro}{\to}}
\newcommand\toPP{ \overset{\text{PP}}{\to}}
\newcommand{\oeq}{\mathrel{\text{\textcircled{$=$}}}}





\usepackage{xcolor}
% \usepackage[normalem]{ulem}
\usepackage{lipsum}
\makeatletter
% \newcommand\colorwave[1][blue]{\bgroup \markoverwith{\lower3.5\p@\hbox{\sixly \textcolor{#1}{\char58}}}\ULon}
%\font\sixly=lasy6 % does not re-load if already loaded, so no memory problem.

\newmdtheoremenv[
linewidth= 1pt,linecolor= blue,%
leftmargin=20,rightmargin=20,innertopmargin=0pt, innerrightmargin=40,%
tikzsetting = { draw=lightgray, line width = 0.3pt,dashed,%
dash pattern = on 15pt off 3pt},%
splittopskip=\topskip,skipbelow=\baselineskip,%
skipabove=\baselineskip,ntheorem,roundcorner=0pt,
% backgroundcolor=pagebg,font=\color{orange}\sffamily, fontcolor=white
]{examplebox}{Exemple}[section]



\newcommand\R{\mathbb{R}}
\newcommand\Z{\mathbb{Z}}
\newcommand\N{\mathbb{N}}
\newcommand\E{\mathbb{E}}
\newcommand\F{\mathcal{F}}
\newcommand\cH{\mathcal{H}}
\newcommand\V{\mathbb{V}}
\newcommand\dmo{ ^{-1} }
\newcommand\kapa{\kappa}
\newcommand\im{Im}
\newcommand\hs{\mathcal{H}}





\usepackage{soul}

\makeatletter
\newcommand*{\whiten}[1]{\llap{\textcolor{white}{{\the\SOUL@token}}\hspace{#1pt}}}
\DeclareRobustCommand*\myul{%
    \def\SOUL@everyspace{\underline{\space}\kern\z@}%
    \def\SOUL@everytoken{%
     \setbox0=\hbox{\the\SOUL@token}%
     \ifdim\dp0>\z@
        \raisebox{\dp0}{\underline{\phantom{\the\SOUL@token}}}%
        \whiten{1}\whiten{0}%
        \whiten{-1}\whiten{-2}%
        \llap{\the\SOUL@token}%
     \else
        \underline{\the\SOUL@token}%
     \fi}%
\SOUL@}
\makeatother

\newcommand*{\demp}{\fontfamily{lmtt}\selectfont}

\DeclareTextFontCommand{\textdemp}{\demp}

\begin{document}

\ifcomment
Multiline
comment
\fi
\ifcomment
\myul{Typesetting test}
% \color[rgb]{1,1,1}
$∑_i^n≠ 60º±∞π∆¬≈√j∫h≤≥µ$

$\CR \R\pro\ind\pro\gS\pro
\mqty[a&b\\c&d]$
$\pro\mathbb{P}$
$\dd{x}$

  \[
    \alpha(x)=\left\{
                \begin{array}{ll}
                  x\\
                  \frac{1}{1+e^{-kx}}\\
                  \frac{e^x-e^{-x}}{e^x+e^{-x}}
                \end{array}
              \right.
  \]

  $\expval{x}$
  
  $\chi_\rho(ghg\dmo)=\Tr(\rho_{ghg\dmo})=\Tr(\rho_g\circ\rho_h\circ\rho\dmo_g)=\Tr(\rho_h)\overset{\mbox{\scalebox{0.5}{$\Tr(AB)=\Tr(BA)$}}}{=}\chi_\rho(h)$
  	$\mathop{\oplus}_{\substack{x\in X}}$

$\mat(\rho_g)=(a_{ij}(g))_{\scriptsize \substack{1\leq i\leq d \\ 1\leq j\leq d}}$ et $\mat(\rho'_g)=(a'_{ij}(g))_{\scriptsize \substack{1\leq i'\leq d' \\ 1\leq j'\leq d'}}$



\[\int_a^b{\mathbb{R}^2}g(u, v)\dd{P_{XY}}(u, v)=\iint g(u,v) f_{XY}(u, v)\dd \lambda(u) \dd \lambda(v)\]
$$\lim_{x\to\infty} f(x)$$	
$$\iiiint_V \mu(t,u,v,w) \,dt\,du\,dv\,dw$$
$$\sum_{n=1}^{\infty} 2^{-n} = 1$$	
\begin{definition}
	Si $X$ et $Y$ sont 2 v.a. ou definit la \textsc{Covariance} entre $X$ et $Y$ comme
	$\cov(X,Y)\overset{\text{def}}{=}\E\left[(X-\E(X))(Y-\E(Y))\right]=\E(XY)-\E(X)\E(Y)$.
\end{definition}
\fi
\pagebreak

% \tableofcontents

% insert your code here
%\input{./algebra/main.tex}
%\input{./geometrie-differentielle/main.tex}
%\input{./probabilite/main.tex}
%\input{./analyse-fonctionnelle/main.tex}
% \input{./Analyse-convexe-et-dualite-en-optimisation/main.tex}
%\input{./tikz/main.tex}
%\input{./Theorie-du-distributions/main.tex}
%\input{./optimisation/mine.tex}
 \input{./modelisation/main.tex}

% yves.aubry@univ-tln.fr : algebra

\end{document}

%% !TEX encoding = UTF-8 Unicode
% !TEX TS-program = xelatex

\documentclass[french]{report}

%\usepackage[utf8]{inputenc}
%\usepackage[T1]{fontenc}
\usepackage{babel}


\newif\ifcomment
%\commenttrue # Show comments

\usepackage{physics}
\usepackage{amssymb}


\usepackage{amsthm}
% \usepackage{thmtools}
\usepackage{mathtools}
\usepackage{amsfonts}

\usepackage{color}

\usepackage{tikz}

\usepackage{geometry}
\geometry{a5paper, margin=0.1in, right=1cm}

\usepackage{dsfont}

\usepackage{graphicx}
\graphicspath{ {images/} }

\usepackage{faktor}

\usepackage{IEEEtrantools}
\usepackage{enumerate}   
\usepackage[PostScript=dvips]{"/Users/aware/Documents/Courses/diagrams"}


\newtheorem{theorem}{Théorème}[section]
\renewcommand{\thetheorem}{\arabic{theorem}}
\newtheorem{lemme}{Lemme}[section]
\renewcommand{\thelemme}{\arabic{lemme}}
\newtheorem{proposition}{Proposition}[section]
\renewcommand{\theproposition}{\arabic{proposition}}
\newtheorem{notations}{Notations}[section]
\newtheorem{problem}{Problème}[section]
\newtheorem{corollary}{Corollaire}[theorem]
\renewcommand{\thecorollary}{\arabic{corollary}}
\newtheorem{property}{Propriété}[section]
\newtheorem{objective}{Objectif}[section]

\theoremstyle{definition}
\newtheorem{definition}{Définition}[section]
\renewcommand{\thedefinition}{\arabic{definition}}
\newtheorem{exercise}{Exercice}[chapter]
\renewcommand{\theexercise}{\arabic{exercise}}
\newtheorem{example}{Exemple}[chapter]
\renewcommand{\theexample}{\arabic{example}}
\newtheorem*{solution}{Solution}
\newtheorem*{application}{Application}
\newtheorem*{notation}{Notation}
\newtheorem*{vocabulary}{Vocabulaire}
\newtheorem*{properties}{Propriétés}



\theoremstyle{remark}
\newtheorem*{remark}{Remarque}
\newtheorem*{rappel}{Rappel}


\usepackage{etoolbox}
\AtBeginEnvironment{exercise}{\small}
\AtBeginEnvironment{example}{\small}

\usepackage{cases}
\usepackage[red]{mypack}

\usepackage[framemethod=TikZ]{mdframed}

\definecolor{bg}{rgb}{0.4,0.25,0.95}
\definecolor{pagebg}{rgb}{0,0,0.5}
\surroundwithmdframed[
   topline=false,
   rightline=false,
   bottomline=false,
   leftmargin=\parindent,
   skipabove=8pt,
   skipbelow=8pt,
   linecolor=blue,
   innerbottommargin=10pt,
   % backgroundcolor=bg,font=\color{orange}\sffamily, fontcolor=white
]{definition}

\usepackage{empheq}
\usepackage[most]{tcolorbox}

\newtcbox{\mymath}[1][]{%
    nobeforeafter, math upper, tcbox raise base,
    enhanced, colframe=blue!30!black,
    colback=red!10, boxrule=1pt,
    #1}

\usepackage{unixode}


\DeclareMathOperator{\ord}{ord}
\DeclareMathOperator{\orb}{orb}
\DeclareMathOperator{\stab}{stab}
\DeclareMathOperator{\Stab}{stab}
\DeclareMathOperator{\ppcm}{ppcm}
\DeclareMathOperator{\conj}{Conj}
\DeclareMathOperator{\End}{End}
\DeclareMathOperator{\rot}{rot}
\DeclareMathOperator{\trs}{trace}
\DeclareMathOperator{\Ind}{Ind}
\DeclareMathOperator{\mat}{Mat}
\DeclareMathOperator{\id}{Id}
\DeclareMathOperator{\vect}{vect}
\DeclareMathOperator{\img}{img}
\DeclareMathOperator{\cov}{Cov}
\DeclareMathOperator{\dist}{dist}
\DeclareMathOperator{\irr}{Irr}
\DeclareMathOperator{\image}{Im}
\DeclareMathOperator{\pd}{\partial}
\DeclareMathOperator{\epi}{epi}
\DeclareMathOperator{\Argmin}{Argmin}
\DeclareMathOperator{\dom}{dom}
\DeclareMathOperator{\proj}{proj}
\DeclareMathOperator{\ctg}{ctg}
\DeclareMathOperator{\supp}{supp}
\DeclareMathOperator{\argmin}{argmin}
\DeclareMathOperator{\mult}{mult}
\DeclareMathOperator{\ch}{ch}
\DeclareMathOperator{\sh}{sh}
\DeclareMathOperator{\rang}{rang}
\DeclareMathOperator{\diam}{diam}
\DeclareMathOperator{\Epigraphe}{Epigraphe}




\usepackage{xcolor}
\everymath{\color{blue}}
%\everymath{\color[rgb]{0,1,1}}
%\pagecolor[rgb]{0,0,0.5}


\newcommand*{\pdtest}[3][]{\ensuremath{\frac{\partial^{#1} #2}{\partial #3}}}

\newcommand*{\deffunc}[6][]{\ensuremath{
\begin{array}{rcl}
#2 : #3 &\rightarrow& #4\\
#5 &\mapsto& #6
\end{array}
}}

\newcommand{\eqcolon}{\mathrel{\resizebox{\widthof{$\mathord{=}$}}{\height}{ $\!\!=\!\!\resizebox{1.2\width}{0.8\height}{\raisebox{0.23ex}{$\mathop{:}$}}\!\!$ }}}
\newcommand{\coloneq}{\mathrel{\resizebox{\widthof{$\mathord{=}$}}{\height}{ $\!\!\resizebox{1.2\width}{0.8\height}{\raisebox{0.23ex}{$\mathop{:}$}}\!\!=\!\!$ }}}
\newcommand{\eqcolonl}{\ensuremath{\mathrel{=\!\!\mathop{:}}}}
\newcommand{\coloneql}{\ensuremath{\mathrel{\mathop{:} \!\! =}}}
\newcommand{\vc}[1]{% inline column vector
  \left(\begin{smallmatrix}#1\end{smallmatrix}\right)%
}
\newcommand{\vr}[1]{% inline row vector
  \begin{smallmatrix}(\,#1\,)\end{smallmatrix}%
}
\makeatletter
\newcommand*{\defeq}{\ =\mathrel{\rlap{%
                     \raisebox{0.3ex}{$\m@th\cdot$}}%
                     \raisebox{-0.3ex}{$\m@th\cdot$}}%
                     }
\makeatother

\newcommand{\mathcircle}[1]{% inline row vector
 \overset{\circ}{#1}
}
\newcommand{\ulim}{% low limit
 \underline{\lim}
}
\newcommand{\ssi}{% iff
\iff
}
\newcommand{\ps}[2]{
\expval{#1 | #2}
}
\newcommand{\df}[1]{
\mqty{#1}
}
\newcommand{\n}[1]{
\norm{#1}
}
\newcommand{\sys}[1]{
\left\{\smqty{#1}\right.
}


\newcommand{\eqdef}{\ensuremath{\overset{\text{def}}=}}


\def\Circlearrowright{\ensuremath{%
  \rotatebox[origin=c]{230}{$\circlearrowright$}}}

\newcommand\ct[1]{\text{\rmfamily\upshape #1}}
\newcommand\question[1]{ {\color{red} ...!? \small #1}}
\newcommand\caz[1]{\left\{\begin{array} #1 \end{array}\right.}
\newcommand\const{\text{\rmfamily\upshape const}}
\newcommand\toP{ \overset{\pro}{\to}}
\newcommand\toPP{ \overset{\text{PP}}{\to}}
\newcommand{\oeq}{\mathrel{\text{\textcircled{$=$}}}}





\usepackage{xcolor}
% \usepackage[normalem]{ulem}
\usepackage{lipsum}
\makeatletter
% \newcommand\colorwave[1][blue]{\bgroup \markoverwith{\lower3.5\p@\hbox{\sixly \textcolor{#1}{\char58}}}\ULon}
%\font\sixly=lasy6 % does not re-load if already loaded, so no memory problem.

\newmdtheoremenv[
linewidth= 1pt,linecolor= blue,%
leftmargin=20,rightmargin=20,innertopmargin=0pt, innerrightmargin=40,%
tikzsetting = { draw=lightgray, line width = 0.3pt,dashed,%
dash pattern = on 15pt off 3pt},%
splittopskip=\topskip,skipbelow=\baselineskip,%
skipabove=\baselineskip,ntheorem,roundcorner=0pt,
% backgroundcolor=pagebg,font=\color{orange}\sffamily, fontcolor=white
]{examplebox}{Exemple}[section]



\newcommand\R{\mathbb{R}}
\newcommand\Z{\mathbb{Z}}
\newcommand\N{\mathbb{N}}
\newcommand\E{\mathbb{E}}
\newcommand\F{\mathcal{F}}
\newcommand\cH{\mathcal{H}}
\newcommand\V{\mathbb{V}}
\newcommand\dmo{ ^{-1} }
\newcommand\kapa{\kappa}
\newcommand\im{Im}
\newcommand\hs{\mathcal{H}}





\usepackage{soul}

\makeatletter
\newcommand*{\whiten}[1]{\llap{\textcolor{white}{{\the\SOUL@token}}\hspace{#1pt}}}
\DeclareRobustCommand*\myul{%
    \def\SOUL@everyspace{\underline{\space}\kern\z@}%
    \def\SOUL@everytoken{%
     \setbox0=\hbox{\the\SOUL@token}%
     \ifdim\dp0>\z@
        \raisebox{\dp0}{\underline{\phantom{\the\SOUL@token}}}%
        \whiten{1}\whiten{0}%
        \whiten{-1}\whiten{-2}%
        \llap{\the\SOUL@token}%
     \else
        \underline{\the\SOUL@token}%
     \fi}%
\SOUL@}
\makeatother

\newcommand*{\demp}{\fontfamily{lmtt}\selectfont}

\DeclareTextFontCommand{\textdemp}{\demp}

\begin{document}

\ifcomment
Multiline
comment
\fi
\ifcomment
\myul{Typesetting test}
% \color[rgb]{1,1,1}
$∑_i^n≠ 60º±∞π∆¬≈√j∫h≤≥µ$

$\CR \R\pro\ind\pro\gS\pro
\mqty[a&b\\c&d]$
$\pro\mathbb{P}$
$\dd{x}$

  \[
    \alpha(x)=\left\{
                \begin{array}{ll}
                  x\\
                  \frac{1}{1+e^{-kx}}\\
                  \frac{e^x-e^{-x}}{e^x+e^{-x}}
                \end{array}
              \right.
  \]

  $\expval{x}$
  
  $\chi_\rho(ghg\dmo)=\Tr(\rho_{ghg\dmo})=\Tr(\rho_g\circ\rho_h\circ\rho\dmo_g)=\Tr(\rho_h)\overset{\mbox{\scalebox{0.5}{$\Tr(AB)=\Tr(BA)$}}}{=}\chi_\rho(h)$
  	$\mathop{\oplus}_{\substack{x\in X}}$

$\mat(\rho_g)=(a_{ij}(g))_{\scriptsize \substack{1\leq i\leq d \\ 1\leq j\leq d}}$ et $\mat(\rho'_g)=(a'_{ij}(g))_{\scriptsize \substack{1\leq i'\leq d' \\ 1\leq j'\leq d'}}$



\[\int_a^b{\mathbb{R}^2}g(u, v)\dd{P_{XY}}(u, v)=\iint g(u,v) f_{XY}(u, v)\dd \lambda(u) \dd \lambda(v)\]
$$\lim_{x\to\infty} f(x)$$	
$$\iiiint_V \mu(t,u,v,w) \,dt\,du\,dv\,dw$$
$$\sum_{n=1}^{\infty} 2^{-n} = 1$$	
\begin{definition}
	Si $X$ et $Y$ sont 2 v.a. ou definit la \textsc{Covariance} entre $X$ et $Y$ comme
	$\cov(X,Y)\overset{\text{def}}{=}\E\left[(X-\E(X))(Y-\E(Y))\right]=\E(XY)-\E(X)\E(Y)$.
\end{definition}
\fi
\pagebreak

% \tableofcontents

% insert your code here
%\input{./algebra/main.tex}
%\input{./geometrie-differentielle/main.tex}
%\input{./probabilite/main.tex}
%\input{./analyse-fonctionnelle/main.tex}
% \input{./Analyse-convexe-et-dualite-en-optimisation/main.tex}
%\input{./tikz/main.tex}
%\input{./Theorie-du-distributions/main.tex}
%\input{./optimisation/mine.tex}
 \input{./modelisation/main.tex}

% yves.aubry@univ-tln.fr : algebra

\end{document}

%% !TEX encoding = UTF-8 Unicode
% !TEX TS-program = xelatex

\documentclass[french]{report}

%\usepackage[utf8]{inputenc}
%\usepackage[T1]{fontenc}
\usepackage{babel}


\newif\ifcomment
%\commenttrue # Show comments

\usepackage{physics}
\usepackage{amssymb}


\usepackage{amsthm}
% \usepackage{thmtools}
\usepackage{mathtools}
\usepackage{amsfonts}

\usepackage{color}

\usepackage{tikz}

\usepackage{geometry}
\geometry{a5paper, margin=0.1in, right=1cm}

\usepackage{dsfont}

\usepackage{graphicx}
\graphicspath{ {images/} }

\usepackage{faktor}

\usepackage{IEEEtrantools}
\usepackage{enumerate}   
\usepackage[PostScript=dvips]{"/Users/aware/Documents/Courses/diagrams"}


\newtheorem{theorem}{Théorème}[section]
\renewcommand{\thetheorem}{\arabic{theorem}}
\newtheorem{lemme}{Lemme}[section]
\renewcommand{\thelemme}{\arabic{lemme}}
\newtheorem{proposition}{Proposition}[section]
\renewcommand{\theproposition}{\arabic{proposition}}
\newtheorem{notations}{Notations}[section]
\newtheorem{problem}{Problème}[section]
\newtheorem{corollary}{Corollaire}[theorem]
\renewcommand{\thecorollary}{\arabic{corollary}}
\newtheorem{property}{Propriété}[section]
\newtheorem{objective}{Objectif}[section]

\theoremstyle{definition}
\newtheorem{definition}{Définition}[section]
\renewcommand{\thedefinition}{\arabic{definition}}
\newtheorem{exercise}{Exercice}[chapter]
\renewcommand{\theexercise}{\arabic{exercise}}
\newtheorem{example}{Exemple}[chapter]
\renewcommand{\theexample}{\arabic{example}}
\newtheorem*{solution}{Solution}
\newtheorem*{application}{Application}
\newtheorem*{notation}{Notation}
\newtheorem*{vocabulary}{Vocabulaire}
\newtheorem*{properties}{Propriétés}



\theoremstyle{remark}
\newtheorem*{remark}{Remarque}
\newtheorem*{rappel}{Rappel}


\usepackage{etoolbox}
\AtBeginEnvironment{exercise}{\small}
\AtBeginEnvironment{example}{\small}

\usepackage{cases}
\usepackage[red]{mypack}

\usepackage[framemethod=TikZ]{mdframed}

\definecolor{bg}{rgb}{0.4,0.25,0.95}
\definecolor{pagebg}{rgb}{0,0,0.5}
\surroundwithmdframed[
   topline=false,
   rightline=false,
   bottomline=false,
   leftmargin=\parindent,
   skipabove=8pt,
   skipbelow=8pt,
   linecolor=blue,
   innerbottommargin=10pt,
   % backgroundcolor=bg,font=\color{orange}\sffamily, fontcolor=white
]{definition}

\usepackage{empheq}
\usepackage[most]{tcolorbox}

\newtcbox{\mymath}[1][]{%
    nobeforeafter, math upper, tcbox raise base,
    enhanced, colframe=blue!30!black,
    colback=red!10, boxrule=1pt,
    #1}

\usepackage{unixode}


\DeclareMathOperator{\ord}{ord}
\DeclareMathOperator{\orb}{orb}
\DeclareMathOperator{\stab}{stab}
\DeclareMathOperator{\Stab}{stab}
\DeclareMathOperator{\ppcm}{ppcm}
\DeclareMathOperator{\conj}{Conj}
\DeclareMathOperator{\End}{End}
\DeclareMathOperator{\rot}{rot}
\DeclareMathOperator{\trs}{trace}
\DeclareMathOperator{\Ind}{Ind}
\DeclareMathOperator{\mat}{Mat}
\DeclareMathOperator{\id}{Id}
\DeclareMathOperator{\vect}{vect}
\DeclareMathOperator{\img}{img}
\DeclareMathOperator{\cov}{Cov}
\DeclareMathOperator{\dist}{dist}
\DeclareMathOperator{\irr}{Irr}
\DeclareMathOperator{\image}{Im}
\DeclareMathOperator{\pd}{\partial}
\DeclareMathOperator{\epi}{epi}
\DeclareMathOperator{\Argmin}{Argmin}
\DeclareMathOperator{\dom}{dom}
\DeclareMathOperator{\proj}{proj}
\DeclareMathOperator{\ctg}{ctg}
\DeclareMathOperator{\supp}{supp}
\DeclareMathOperator{\argmin}{argmin}
\DeclareMathOperator{\mult}{mult}
\DeclareMathOperator{\ch}{ch}
\DeclareMathOperator{\sh}{sh}
\DeclareMathOperator{\rang}{rang}
\DeclareMathOperator{\diam}{diam}
\DeclareMathOperator{\Epigraphe}{Epigraphe}




\usepackage{xcolor}
\everymath{\color{blue}}
%\everymath{\color[rgb]{0,1,1}}
%\pagecolor[rgb]{0,0,0.5}


\newcommand*{\pdtest}[3][]{\ensuremath{\frac{\partial^{#1} #2}{\partial #3}}}

\newcommand*{\deffunc}[6][]{\ensuremath{
\begin{array}{rcl}
#2 : #3 &\rightarrow& #4\\
#5 &\mapsto& #6
\end{array}
}}

\newcommand{\eqcolon}{\mathrel{\resizebox{\widthof{$\mathord{=}$}}{\height}{ $\!\!=\!\!\resizebox{1.2\width}{0.8\height}{\raisebox{0.23ex}{$\mathop{:}$}}\!\!$ }}}
\newcommand{\coloneq}{\mathrel{\resizebox{\widthof{$\mathord{=}$}}{\height}{ $\!\!\resizebox{1.2\width}{0.8\height}{\raisebox{0.23ex}{$\mathop{:}$}}\!\!=\!\!$ }}}
\newcommand{\eqcolonl}{\ensuremath{\mathrel{=\!\!\mathop{:}}}}
\newcommand{\coloneql}{\ensuremath{\mathrel{\mathop{:} \!\! =}}}
\newcommand{\vc}[1]{% inline column vector
  \left(\begin{smallmatrix}#1\end{smallmatrix}\right)%
}
\newcommand{\vr}[1]{% inline row vector
  \begin{smallmatrix}(\,#1\,)\end{smallmatrix}%
}
\makeatletter
\newcommand*{\defeq}{\ =\mathrel{\rlap{%
                     \raisebox{0.3ex}{$\m@th\cdot$}}%
                     \raisebox{-0.3ex}{$\m@th\cdot$}}%
                     }
\makeatother

\newcommand{\mathcircle}[1]{% inline row vector
 \overset{\circ}{#1}
}
\newcommand{\ulim}{% low limit
 \underline{\lim}
}
\newcommand{\ssi}{% iff
\iff
}
\newcommand{\ps}[2]{
\expval{#1 | #2}
}
\newcommand{\df}[1]{
\mqty{#1}
}
\newcommand{\n}[1]{
\norm{#1}
}
\newcommand{\sys}[1]{
\left\{\smqty{#1}\right.
}


\newcommand{\eqdef}{\ensuremath{\overset{\text{def}}=}}


\def\Circlearrowright{\ensuremath{%
  \rotatebox[origin=c]{230}{$\circlearrowright$}}}

\newcommand\ct[1]{\text{\rmfamily\upshape #1}}
\newcommand\question[1]{ {\color{red} ...!? \small #1}}
\newcommand\caz[1]{\left\{\begin{array} #1 \end{array}\right.}
\newcommand\const{\text{\rmfamily\upshape const}}
\newcommand\toP{ \overset{\pro}{\to}}
\newcommand\toPP{ \overset{\text{PP}}{\to}}
\newcommand{\oeq}{\mathrel{\text{\textcircled{$=$}}}}





\usepackage{xcolor}
% \usepackage[normalem]{ulem}
\usepackage{lipsum}
\makeatletter
% \newcommand\colorwave[1][blue]{\bgroup \markoverwith{\lower3.5\p@\hbox{\sixly \textcolor{#1}{\char58}}}\ULon}
%\font\sixly=lasy6 % does not re-load if already loaded, so no memory problem.

\newmdtheoremenv[
linewidth= 1pt,linecolor= blue,%
leftmargin=20,rightmargin=20,innertopmargin=0pt, innerrightmargin=40,%
tikzsetting = { draw=lightgray, line width = 0.3pt,dashed,%
dash pattern = on 15pt off 3pt},%
splittopskip=\topskip,skipbelow=\baselineskip,%
skipabove=\baselineskip,ntheorem,roundcorner=0pt,
% backgroundcolor=pagebg,font=\color{orange}\sffamily, fontcolor=white
]{examplebox}{Exemple}[section]



\newcommand\R{\mathbb{R}}
\newcommand\Z{\mathbb{Z}}
\newcommand\N{\mathbb{N}}
\newcommand\E{\mathbb{E}}
\newcommand\F{\mathcal{F}}
\newcommand\cH{\mathcal{H}}
\newcommand\V{\mathbb{V}}
\newcommand\dmo{ ^{-1} }
\newcommand\kapa{\kappa}
\newcommand\im{Im}
\newcommand\hs{\mathcal{H}}





\usepackage{soul}

\makeatletter
\newcommand*{\whiten}[1]{\llap{\textcolor{white}{{\the\SOUL@token}}\hspace{#1pt}}}
\DeclareRobustCommand*\myul{%
    \def\SOUL@everyspace{\underline{\space}\kern\z@}%
    \def\SOUL@everytoken{%
     \setbox0=\hbox{\the\SOUL@token}%
     \ifdim\dp0>\z@
        \raisebox{\dp0}{\underline{\phantom{\the\SOUL@token}}}%
        \whiten{1}\whiten{0}%
        \whiten{-1}\whiten{-2}%
        \llap{\the\SOUL@token}%
     \else
        \underline{\the\SOUL@token}%
     \fi}%
\SOUL@}
\makeatother

\newcommand*{\demp}{\fontfamily{lmtt}\selectfont}

\DeclareTextFontCommand{\textdemp}{\demp}

\begin{document}

\ifcomment
Multiline
comment
\fi
\ifcomment
\myul{Typesetting test}
% \color[rgb]{1,1,1}
$∑_i^n≠ 60º±∞π∆¬≈√j∫h≤≥µ$

$\CR \R\pro\ind\pro\gS\pro
\mqty[a&b\\c&d]$
$\pro\mathbb{P}$
$\dd{x}$

  \[
    \alpha(x)=\left\{
                \begin{array}{ll}
                  x\\
                  \frac{1}{1+e^{-kx}}\\
                  \frac{e^x-e^{-x}}{e^x+e^{-x}}
                \end{array}
              \right.
  \]

  $\expval{x}$
  
  $\chi_\rho(ghg\dmo)=\Tr(\rho_{ghg\dmo})=\Tr(\rho_g\circ\rho_h\circ\rho\dmo_g)=\Tr(\rho_h)\overset{\mbox{\scalebox{0.5}{$\Tr(AB)=\Tr(BA)$}}}{=}\chi_\rho(h)$
  	$\mathop{\oplus}_{\substack{x\in X}}$

$\mat(\rho_g)=(a_{ij}(g))_{\scriptsize \substack{1\leq i\leq d \\ 1\leq j\leq d}}$ et $\mat(\rho'_g)=(a'_{ij}(g))_{\scriptsize \substack{1\leq i'\leq d' \\ 1\leq j'\leq d'}}$



\[\int_a^b{\mathbb{R}^2}g(u, v)\dd{P_{XY}}(u, v)=\iint g(u,v) f_{XY}(u, v)\dd \lambda(u) \dd \lambda(v)\]
$$\lim_{x\to\infty} f(x)$$	
$$\iiiint_V \mu(t,u,v,w) \,dt\,du\,dv\,dw$$
$$\sum_{n=1}^{\infty} 2^{-n} = 1$$	
\begin{definition}
	Si $X$ et $Y$ sont 2 v.a. ou definit la \textsc{Covariance} entre $X$ et $Y$ comme
	$\cov(X,Y)\overset{\text{def}}{=}\E\left[(X-\E(X))(Y-\E(Y))\right]=\E(XY)-\E(X)\E(Y)$.
\end{definition}
\fi
\pagebreak

% \tableofcontents

% insert your code here
%\input{./algebra/main.tex}
%\input{./geometrie-differentielle/main.tex}
%\input{./probabilite/main.tex}
%\input{./analyse-fonctionnelle/main.tex}
% \input{./Analyse-convexe-et-dualite-en-optimisation/main.tex}
%\input{./tikz/main.tex}
%\input{./Theorie-du-distributions/main.tex}
%\input{./optimisation/mine.tex}
 \input{./modelisation/main.tex}

% yves.aubry@univ-tln.fr : algebra

\end{document}

%% !TEX encoding = UTF-8 Unicode
% !TEX TS-program = xelatex

\documentclass[french]{report}

%\usepackage[utf8]{inputenc}
%\usepackage[T1]{fontenc}
\usepackage{babel}


\newif\ifcomment
%\commenttrue # Show comments

\usepackage{physics}
\usepackage{amssymb}


\usepackage{amsthm}
% \usepackage{thmtools}
\usepackage{mathtools}
\usepackage{amsfonts}

\usepackage{color}

\usepackage{tikz}

\usepackage{geometry}
\geometry{a5paper, margin=0.1in, right=1cm}

\usepackage{dsfont}

\usepackage{graphicx}
\graphicspath{ {images/} }

\usepackage{faktor}

\usepackage{IEEEtrantools}
\usepackage{enumerate}   
\usepackage[PostScript=dvips]{"/Users/aware/Documents/Courses/diagrams"}


\newtheorem{theorem}{Théorème}[section]
\renewcommand{\thetheorem}{\arabic{theorem}}
\newtheorem{lemme}{Lemme}[section]
\renewcommand{\thelemme}{\arabic{lemme}}
\newtheorem{proposition}{Proposition}[section]
\renewcommand{\theproposition}{\arabic{proposition}}
\newtheorem{notations}{Notations}[section]
\newtheorem{problem}{Problème}[section]
\newtheorem{corollary}{Corollaire}[theorem]
\renewcommand{\thecorollary}{\arabic{corollary}}
\newtheorem{property}{Propriété}[section]
\newtheorem{objective}{Objectif}[section]

\theoremstyle{definition}
\newtheorem{definition}{Définition}[section]
\renewcommand{\thedefinition}{\arabic{definition}}
\newtheorem{exercise}{Exercice}[chapter]
\renewcommand{\theexercise}{\arabic{exercise}}
\newtheorem{example}{Exemple}[chapter]
\renewcommand{\theexample}{\arabic{example}}
\newtheorem*{solution}{Solution}
\newtheorem*{application}{Application}
\newtheorem*{notation}{Notation}
\newtheorem*{vocabulary}{Vocabulaire}
\newtheorem*{properties}{Propriétés}



\theoremstyle{remark}
\newtheorem*{remark}{Remarque}
\newtheorem*{rappel}{Rappel}


\usepackage{etoolbox}
\AtBeginEnvironment{exercise}{\small}
\AtBeginEnvironment{example}{\small}

\usepackage{cases}
\usepackage[red]{mypack}

\usepackage[framemethod=TikZ]{mdframed}

\definecolor{bg}{rgb}{0.4,0.25,0.95}
\definecolor{pagebg}{rgb}{0,0,0.5}
\surroundwithmdframed[
   topline=false,
   rightline=false,
   bottomline=false,
   leftmargin=\parindent,
   skipabove=8pt,
   skipbelow=8pt,
   linecolor=blue,
   innerbottommargin=10pt,
   % backgroundcolor=bg,font=\color{orange}\sffamily, fontcolor=white
]{definition}

\usepackage{empheq}
\usepackage[most]{tcolorbox}

\newtcbox{\mymath}[1][]{%
    nobeforeafter, math upper, tcbox raise base,
    enhanced, colframe=blue!30!black,
    colback=red!10, boxrule=1pt,
    #1}

\usepackage{unixode}


\DeclareMathOperator{\ord}{ord}
\DeclareMathOperator{\orb}{orb}
\DeclareMathOperator{\stab}{stab}
\DeclareMathOperator{\Stab}{stab}
\DeclareMathOperator{\ppcm}{ppcm}
\DeclareMathOperator{\conj}{Conj}
\DeclareMathOperator{\End}{End}
\DeclareMathOperator{\rot}{rot}
\DeclareMathOperator{\trs}{trace}
\DeclareMathOperator{\Ind}{Ind}
\DeclareMathOperator{\mat}{Mat}
\DeclareMathOperator{\id}{Id}
\DeclareMathOperator{\vect}{vect}
\DeclareMathOperator{\img}{img}
\DeclareMathOperator{\cov}{Cov}
\DeclareMathOperator{\dist}{dist}
\DeclareMathOperator{\irr}{Irr}
\DeclareMathOperator{\image}{Im}
\DeclareMathOperator{\pd}{\partial}
\DeclareMathOperator{\epi}{epi}
\DeclareMathOperator{\Argmin}{Argmin}
\DeclareMathOperator{\dom}{dom}
\DeclareMathOperator{\proj}{proj}
\DeclareMathOperator{\ctg}{ctg}
\DeclareMathOperator{\supp}{supp}
\DeclareMathOperator{\argmin}{argmin}
\DeclareMathOperator{\mult}{mult}
\DeclareMathOperator{\ch}{ch}
\DeclareMathOperator{\sh}{sh}
\DeclareMathOperator{\rang}{rang}
\DeclareMathOperator{\diam}{diam}
\DeclareMathOperator{\Epigraphe}{Epigraphe}




\usepackage{xcolor}
\everymath{\color{blue}}
%\everymath{\color[rgb]{0,1,1}}
%\pagecolor[rgb]{0,0,0.5}


\newcommand*{\pdtest}[3][]{\ensuremath{\frac{\partial^{#1} #2}{\partial #3}}}

\newcommand*{\deffunc}[6][]{\ensuremath{
\begin{array}{rcl}
#2 : #3 &\rightarrow& #4\\
#5 &\mapsto& #6
\end{array}
}}

\newcommand{\eqcolon}{\mathrel{\resizebox{\widthof{$\mathord{=}$}}{\height}{ $\!\!=\!\!\resizebox{1.2\width}{0.8\height}{\raisebox{0.23ex}{$\mathop{:}$}}\!\!$ }}}
\newcommand{\coloneq}{\mathrel{\resizebox{\widthof{$\mathord{=}$}}{\height}{ $\!\!\resizebox{1.2\width}{0.8\height}{\raisebox{0.23ex}{$\mathop{:}$}}\!\!=\!\!$ }}}
\newcommand{\eqcolonl}{\ensuremath{\mathrel{=\!\!\mathop{:}}}}
\newcommand{\coloneql}{\ensuremath{\mathrel{\mathop{:} \!\! =}}}
\newcommand{\vc}[1]{% inline column vector
  \left(\begin{smallmatrix}#1\end{smallmatrix}\right)%
}
\newcommand{\vr}[1]{% inline row vector
  \begin{smallmatrix}(\,#1\,)\end{smallmatrix}%
}
\makeatletter
\newcommand*{\defeq}{\ =\mathrel{\rlap{%
                     \raisebox{0.3ex}{$\m@th\cdot$}}%
                     \raisebox{-0.3ex}{$\m@th\cdot$}}%
                     }
\makeatother

\newcommand{\mathcircle}[1]{% inline row vector
 \overset{\circ}{#1}
}
\newcommand{\ulim}{% low limit
 \underline{\lim}
}
\newcommand{\ssi}{% iff
\iff
}
\newcommand{\ps}[2]{
\expval{#1 | #2}
}
\newcommand{\df}[1]{
\mqty{#1}
}
\newcommand{\n}[1]{
\norm{#1}
}
\newcommand{\sys}[1]{
\left\{\smqty{#1}\right.
}


\newcommand{\eqdef}{\ensuremath{\overset{\text{def}}=}}


\def\Circlearrowright{\ensuremath{%
  \rotatebox[origin=c]{230}{$\circlearrowright$}}}

\newcommand\ct[1]{\text{\rmfamily\upshape #1}}
\newcommand\question[1]{ {\color{red} ...!? \small #1}}
\newcommand\caz[1]{\left\{\begin{array} #1 \end{array}\right.}
\newcommand\const{\text{\rmfamily\upshape const}}
\newcommand\toP{ \overset{\pro}{\to}}
\newcommand\toPP{ \overset{\text{PP}}{\to}}
\newcommand{\oeq}{\mathrel{\text{\textcircled{$=$}}}}





\usepackage{xcolor}
% \usepackage[normalem]{ulem}
\usepackage{lipsum}
\makeatletter
% \newcommand\colorwave[1][blue]{\bgroup \markoverwith{\lower3.5\p@\hbox{\sixly \textcolor{#1}{\char58}}}\ULon}
%\font\sixly=lasy6 % does not re-load if already loaded, so no memory problem.

\newmdtheoremenv[
linewidth= 1pt,linecolor= blue,%
leftmargin=20,rightmargin=20,innertopmargin=0pt, innerrightmargin=40,%
tikzsetting = { draw=lightgray, line width = 0.3pt,dashed,%
dash pattern = on 15pt off 3pt},%
splittopskip=\topskip,skipbelow=\baselineskip,%
skipabove=\baselineskip,ntheorem,roundcorner=0pt,
% backgroundcolor=pagebg,font=\color{orange}\sffamily, fontcolor=white
]{examplebox}{Exemple}[section]



\newcommand\R{\mathbb{R}}
\newcommand\Z{\mathbb{Z}}
\newcommand\N{\mathbb{N}}
\newcommand\E{\mathbb{E}}
\newcommand\F{\mathcal{F}}
\newcommand\cH{\mathcal{H}}
\newcommand\V{\mathbb{V}}
\newcommand\dmo{ ^{-1} }
\newcommand\kapa{\kappa}
\newcommand\im{Im}
\newcommand\hs{\mathcal{H}}





\usepackage{soul}

\makeatletter
\newcommand*{\whiten}[1]{\llap{\textcolor{white}{{\the\SOUL@token}}\hspace{#1pt}}}
\DeclareRobustCommand*\myul{%
    \def\SOUL@everyspace{\underline{\space}\kern\z@}%
    \def\SOUL@everytoken{%
     \setbox0=\hbox{\the\SOUL@token}%
     \ifdim\dp0>\z@
        \raisebox{\dp0}{\underline{\phantom{\the\SOUL@token}}}%
        \whiten{1}\whiten{0}%
        \whiten{-1}\whiten{-2}%
        \llap{\the\SOUL@token}%
     \else
        \underline{\the\SOUL@token}%
     \fi}%
\SOUL@}
\makeatother

\newcommand*{\demp}{\fontfamily{lmtt}\selectfont}

\DeclareTextFontCommand{\textdemp}{\demp}

\begin{document}

\ifcomment
Multiline
comment
\fi
\ifcomment
\myul{Typesetting test}
% \color[rgb]{1,1,1}
$∑_i^n≠ 60º±∞π∆¬≈√j∫h≤≥µ$

$\CR \R\pro\ind\pro\gS\pro
\mqty[a&b\\c&d]$
$\pro\mathbb{P}$
$\dd{x}$

  \[
    \alpha(x)=\left\{
                \begin{array}{ll}
                  x\\
                  \frac{1}{1+e^{-kx}}\\
                  \frac{e^x-e^{-x}}{e^x+e^{-x}}
                \end{array}
              \right.
  \]

  $\expval{x}$
  
  $\chi_\rho(ghg\dmo)=\Tr(\rho_{ghg\dmo})=\Tr(\rho_g\circ\rho_h\circ\rho\dmo_g)=\Tr(\rho_h)\overset{\mbox{\scalebox{0.5}{$\Tr(AB)=\Tr(BA)$}}}{=}\chi_\rho(h)$
  	$\mathop{\oplus}_{\substack{x\in X}}$

$\mat(\rho_g)=(a_{ij}(g))_{\scriptsize \substack{1\leq i\leq d \\ 1\leq j\leq d}}$ et $\mat(\rho'_g)=(a'_{ij}(g))_{\scriptsize \substack{1\leq i'\leq d' \\ 1\leq j'\leq d'}}$



\[\int_a^b{\mathbb{R}^2}g(u, v)\dd{P_{XY}}(u, v)=\iint g(u,v) f_{XY}(u, v)\dd \lambda(u) \dd \lambda(v)\]
$$\lim_{x\to\infty} f(x)$$	
$$\iiiint_V \mu(t,u,v,w) \,dt\,du\,dv\,dw$$
$$\sum_{n=1}^{\infty} 2^{-n} = 1$$	
\begin{definition}
	Si $X$ et $Y$ sont 2 v.a. ou definit la \textsc{Covariance} entre $X$ et $Y$ comme
	$\cov(X,Y)\overset{\text{def}}{=}\E\left[(X-\E(X))(Y-\E(Y))\right]=\E(XY)-\E(X)\E(Y)$.
\end{definition}
\fi
\pagebreak

% \tableofcontents

% insert your code here
%\input{./algebra/main.tex}
%\input{./geometrie-differentielle/main.tex}
%\input{./probabilite/main.tex}
%\input{./analyse-fonctionnelle/main.tex}
% \input{./Analyse-convexe-et-dualite-en-optimisation/main.tex}
%\input{./tikz/main.tex}
%\input{./Theorie-du-distributions/main.tex}
%\input{./optimisation/mine.tex}
 \input{./modelisation/main.tex}

% yves.aubry@univ-tln.fr : algebra

\end{document}

% % !TEX encoding = UTF-8 Unicode
% !TEX TS-program = xelatex

\documentclass[french]{report}

%\usepackage[utf8]{inputenc}
%\usepackage[T1]{fontenc}
\usepackage{babel}


\newif\ifcomment
%\commenttrue # Show comments

\usepackage{physics}
\usepackage{amssymb}


\usepackage{amsthm}
% \usepackage{thmtools}
\usepackage{mathtools}
\usepackage{amsfonts}

\usepackage{color}

\usepackage{tikz}

\usepackage{geometry}
\geometry{a5paper, margin=0.1in, right=1cm}

\usepackage{dsfont}

\usepackage{graphicx}
\graphicspath{ {images/} }

\usepackage{faktor}

\usepackage{IEEEtrantools}
\usepackage{enumerate}   
\usepackage[PostScript=dvips]{"/Users/aware/Documents/Courses/diagrams"}


\newtheorem{theorem}{Théorème}[section]
\renewcommand{\thetheorem}{\arabic{theorem}}
\newtheorem{lemme}{Lemme}[section]
\renewcommand{\thelemme}{\arabic{lemme}}
\newtheorem{proposition}{Proposition}[section]
\renewcommand{\theproposition}{\arabic{proposition}}
\newtheorem{notations}{Notations}[section]
\newtheorem{problem}{Problème}[section]
\newtheorem{corollary}{Corollaire}[theorem]
\renewcommand{\thecorollary}{\arabic{corollary}}
\newtheorem{property}{Propriété}[section]
\newtheorem{objective}{Objectif}[section]

\theoremstyle{definition}
\newtheorem{definition}{Définition}[section]
\renewcommand{\thedefinition}{\arabic{definition}}
\newtheorem{exercise}{Exercice}[chapter]
\renewcommand{\theexercise}{\arabic{exercise}}
\newtheorem{example}{Exemple}[chapter]
\renewcommand{\theexample}{\arabic{example}}
\newtheorem*{solution}{Solution}
\newtheorem*{application}{Application}
\newtheorem*{notation}{Notation}
\newtheorem*{vocabulary}{Vocabulaire}
\newtheorem*{properties}{Propriétés}



\theoremstyle{remark}
\newtheorem*{remark}{Remarque}
\newtheorem*{rappel}{Rappel}


\usepackage{etoolbox}
\AtBeginEnvironment{exercise}{\small}
\AtBeginEnvironment{example}{\small}

\usepackage{cases}
\usepackage[red]{mypack}

\usepackage[framemethod=TikZ]{mdframed}

\definecolor{bg}{rgb}{0.4,0.25,0.95}
\definecolor{pagebg}{rgb}{0,0,0.5}
\surroundwithmdframed[
   topline=false,
   rightline=false,
   bottomline=false,
   leftmargin=\parindent,
   skipabove=8pt,
   skipbelow=8pt,
   linecolor=blue,
   innerbottommargin=10pt,
   % backgroundcolor=bg,font=\color{orange}\sffamily, fontcolor=white
]{definition}

\usepackage{empheq}
\usepackage[most]{tcolorbox}

\newtcbox{\mymath}[1][]{%
    nobeforeafter, math upper, tcbox raise base,
    enhanced, colframe=blue!30!black,
    colback=red!10, boxrule=1pt,
    #1}

\usepackage{unixode}


\DeclareMathOperator{\ord}{ord}
\DeclareMathOperator{\orb}{orb}
\DeclareMathOperator{\stab}{stab}
\DeclareMathOperator{\Stab}{stab}
\DeclareMathOperator{\ppcm}{ppcm}
\DeclareMathOperator{\conj}{Conj}
\DeclareMathOperator{\End}{End}
\DeclareMathOperator{\rot}{rot}
\DeclareMathOperator{\trs}{trace}
\DeclareMathOperator{\Ind}{Ind}
\DeclareMathOperator{\mat}{Mat}
\DeclareMathOperator{\id}{Id}
\DeclareMathOperator{\vect}{vect}
\DeclareMathOperator{\img}{img}
\DeclareMathOperator{\cov}{Cov}
\DeclareMathOperator{\dist}{dist}
\DeclareMathOperator{\irr}{Irr}
\DeclareMathOperator{\image}{Im}
\DeclareMathOperator{\pd}{\partial}
\DeclareMathOperator{\epi}{epi}
\DeclareMathOperator{\Argmin}{Argmin}
\DeclareMathOperator{\dom}{dom}
\DeclareMathOperator{\proj}{proj}
\DeclareMathOperator{\ctg}{ctg}
\DeclareMathOperator{\supp}{supp}
\DeclareMathOperator{\argmin}{argmin}
\DeclareMathOperator{\mult}{mult}
\DeclareMathOperator{\ch}{ch}
\DeclareMathOperator{\sh}{sh}
\DeclareMathOperator{\rang}{rang}
\DeclareMathOperator{\diam}{diam}
\DeclareMathOperator{\Epigraphe}{Epigraphe}




\usepackage{xcolor}
\everymath{\color{blue}}
%\everymath{\color[rgb]{0,1,1}}
%\pagecolor[rgb]{0,0,0.5}


\newcommand*{\pdtest}[3][]{\ensuremath{\frac{\partial^{#1} #2}{\partial #3}}}

\newcommand*{\deffunc}[6][]{\ensuremath{
\begin{array}{rcl}
#2 : #3 &\rightarrow& #4\\
#5 &\mapsto& #6
\end{array}
}}

\newcommand{\eqcolon}{\mathrel{\resizebox{\widthof{$\mathord{=}$}}{\height}{ $\!\!=\!\!\resizebox{1.2\width}{0.8\height}{\raisebox{0.23ex}{$\mathop{:}$}}\!\!$ }}}
\newcommand{\coloneq}{\mathrel{\resizebox{\widthof{$\mathord{=}$}}{\height}{ $\!\!\resizebox{1.2\width}{0.8\height}{\raisebox{0.23ex}{$\mathop{:}$}}\!\!=\!\!$ }}}
\newcommand{\eqcolonl}{\ensuremath{\mathrel{=\!\!\mathop{:}}}}
\newcommand{\coloneql}{\ensuremath{\mathrel{\mathop{:} \!\! =}}}
\newcommand{\vc}[1]{% inline column vector
  \left(\begin{smallmatrix}#1\end{smallmatrix}\right)%
}
\newcommand{\vr}[1]{% inline row vector
  \begin{smallmatrix}(\,#1\,)\end{smallmatrix}%
}
\makeatletter
\newcommand*{\defeq}{\ =\mathrel{\rlap{%
                     \raisebox{0.3ex}{$\m@th\cdot$}}%
                     \raisebox{-0.3ex}{$\m@th\cdot$}}%
                     }
\makeatother

\newcommand{\mathcircle}[1]{% inline row vector
 \overset{\circ}{#1}
}
\newcommand{\ulim}{% low limit
 \underline{\lim}
}
\newcommand{\ssi}{% iff
\iff
}
\newcommand{\ps}[2]{
\expval{#1 | #2}
}
\newcommand{\df}[1]{
\mqty{#1}
}
\newcommand{\n}[1]{
\norm{#1}
}
\newcommand{\sys}[1]{
\left\{\smqty{#1}\right.
}


\newcommand{\eqdef}{\ensuremath{\overset{\text{def}}=}}


\def\Circlearrowright{\ensuremath{%
  \rotatebox[origin=c]{230}{$\circlearrowright$}}}

\newcommand\ct[1]{\text{\rmfamily\upshape #1}}
\newcommand\question[1]{ {\color{red} ...!? \small #1}}
\newcommand\caz[1]{\left\{\begin{array} #1 \end{array}\right.}
\newcommand\const{\text{\rmfamily\upshape const}}
\newcommand\toP{ \overset{\pro}{\to}}
\newcommand\toPP{ \overset{\text{PP}}{\to}}
\newcommand{\oeq}{\mathrel{\text{\textcircled{$=$}}}}





\usepackage{xcolor}
% \usepackage[normalem]{ulem}
\usepackage{lipsum}
\makeatletter
% \newcommand\colorwave[1][blue]{\bgroup \markoverwith{\lower3.5\p@\hbox{\sixly \textcolor{#1}{\char58}}}\ULon}
%\font\sixly=lasy6 % does not re-load if already loaded, so no memory problem.

\newmdtheoremenv[
linewidth= 1pt,linecolor= blue,%
leftmargin=20,rightmargin=20,innertopmargin=0pt, innerrightmargin=40,%
tikzsetting = { draw=lightgray, line width = 0.3pt,dashed,%
dash pattern = on 15pt off 3pt},%
splittopskip=\topskip,skipbelow=\baselineskip,%
skipabove=\baselineskip,ntheorem,roundcorner=0pt,
% backgroundcolor=pagebg,font=\color{orange}\sffamily, fontcolor=white
]{examplebox}{Exemple}[section]



\newcommand\R{\mathbb{R}}
\newcommand\Z{\mathbb{Z}}
\newcommand\N{\mathbb{N}}
\newcommand\E{\mathbb{E}}
\newcommand\F{\mathcal{F}}
\newcommand\cH{\mathcal{H}}
\newcommand\V{\mathbb{V}}
\newcommand\dmo{ ^{-1} }
\newcommand\kapa{\kappa}
\newcommand\im{Im}
\newcommand\hs{\mathcal{H}}





\usepackage{soul}

\makeatletter
\newcommand*{\whiten}[1]{\llap{\textcolor{white}{{\the\SOUL@token}}\hspace{#1pt}}}
\DeclareRobustCommand*\myul{%
    \def\SOUL@everyspace{\underline{\space}\kern\z@}%
    \def\SOUL@everytoken{%
     \setbox0=\hbox{\the\SOUL@token}%
     \ifdim\dp0>\z@
        \raisebox{\dp0}{\underline{\phantom{\the\SOUL@token}}}%
        \whiten{1}\whiten{0}%
        \whiten{-1}\whiten{-2}%
        \llap{\the\SOUL@token}%
     \else
        \underline{\the\SOUL@token}%
     \fi}%
\SOUL@}
\makeatother

\newcommand*{\demp}{\fontfamily{lmtt}\selectfont}

\DeclareTextFontCommand{\textdemp}{\demp}

\begin{document}

\ifcomment
Multiline
comment
\fi
\ifcomment
\myul{Typesetting test}
% \color[rgb]{1,1,1}
$∑_i^n≠ 60º±∞π∆¬≈√j∫h≤≥µ$

$\CR \R\pro\ind\pro\gS\pro
\mqty[a&b\\c&d]$
$\pro\mathbb{P}$
$\dd{x}$

  \[
    \alpha(x)=\left\{
                \begin{array}{ll}
                  x\\
                  \frac{1}{1+e^{-kx}}\\
                  \frac{e^x-e^{-x}}{e^x+e^{-x}}
                \end{array}
              \right.
  \]

  $\expval{x}$
  
  $\chi_\rho(ghg\dmo)=\Tr(\rho_{ghg\dmo})=\Tr(\rho_g\circ\rho_h\circ\rho\dmo_g)=\Tr(\rho_h)\overset{\mbox{\scalebox{0.5}{$\Tr(AB)=\Tr(BA)$}}}{=}\chi_\rho(h)$
  	$\mathop{\oplus}_{\substack{x\in X}}$

$\mat(\rho_g)=(a_{ij}(g))_{\scriptsize \substack{1\leq i\leq d \\ 1\leq j\leq d}}$ et $\mat(\rho'_g)=(a'_{ij}(g))_{\scriptsize \substack{1\leq i'\leq d' \\ 1\leq j'\leq d'}}$



\[\int_a^b{\mathbb{R}^2}g(u, v)\dd{P_{XY}}(u, v)=\iint g(u,v) f_{XY}(u, v)\dd \lambda(u) \dd \lambda(v)\]
$$\lim_{x\to\infty} f(x)$$	
$$\iiiint_V \mu(t,u,v,w) \,dt\,du\,dv\,dw$$
$$\sum_{n=1}^{\infty} 2^{-n} = 1$$	
\begin{definition}
	Si $X$ et $Y$ sont 2 v.a. ou definit la \textsc{Covariance} entre $X$ et $Y$ comme
	$\cov(X,Y)\overset{\text{def}}{=}\E\left[(X-\E(X))(Y-\E(Y))\right]=\E(XY)-\E(X)\E(Y)$.
\end{definition}
\fi
\pagebreak

% \tableofcontents

% insert your code here
%\input{./algebra/main.tex}
%\input{./geometrie-differentielle/main.tex}
%\input{./probabilite/main.tex}
%\input{./analyse-fonctionnelle/main.tex}
% \input{./Analyse-convexe-et-dualite-en-optimisation/main.tex}
%\input{./tikz/main.tex}
%\input{./Theorie-du-distributions/main.tex}
%\input{./optimisation/mine.tex}
 \input{./modelisation/main.tex}

% yves.aubry@univ-tln.fr : algebra

\end{document}

%% !TEX encoding = UTF-8 Unicode
% !TEX TS-program = xelatex

\documentclass[french]{report}

%\usepackage[utf8]{inputenc}
%\usepackage[T1]{fontenc}
\usepackage{babel}


\newif\ifcomment
%\commenttrue # Show comments

\usepackage{physics}
\usepackage{amssymb}


\usepackage{amsthm}
% \usepackage{thmtools}
\usepackage{mathtools}
\usepackage{amsfonts}

\usepackage{color}

\usepackage{tikz}

\usepackage{geometry}
\geometry{a5paper, margin=0.1in, right=1cm}

\usepackage{dsfont}

\usepackage{graphicx}
\graphicspath{ {images/} }

\usepackage{faktor}

\usepackage{IEEEtrantools}
\usepackage{enumerate}   
\usepackage[PostScript=dvips]{"/Users/aware/Documents/Courses/diagrams"}


\newtheorem{theorem}{Théorème}[section]
\renewcommand{\thetheorem}{\arabic{theorem}}
\newtheorem{lemme}{Lemme}[section]
\renewcommand{\thelemme}{\arabic{lemme}}
\newtheorem{proposition}{Proposition}[section]
\renewcommand{\theproposition}{\arabic{proposition}}
\newtheorem{notations}{Notations}[section]
\newtheorem{problem}{Problème}[section]
\newtheorem{corollary}{Corollaire}[theorem]
\renewcommand{\thecorollary}{\arabic{corollary}}
\newtheorem{property}{Propriété}[section]
\newtheorem{objective}{Objectif}[section]

\theoremstyle{definition}
\newtheorem{definition}{Définition}[section]
\renewcommand{\thedefinition}{\arabic{definition}}
\newtheorem{exercise}{Exercice}[chapter]
\renewcommand{\theexercise}{\arabic{exercise}}
\newtheorem{example}{Exemple}[chapter]
\renewcommand{\theexample}{\arabic{example}}
\newtheorem*{solution}{Solution}
\newtheorem*{application}{Application}
\newtheorem*{notation}{Notation}
\newtheorem*{vocabulary}{Vocabulaire}
\newtheorem*{properties}{Propriétés}



\theoremstyle{remark}
\newtheorem*{remark}{Remarque}
\newtheorem*{rappel}{Rappel}


\usepackage{etoolbox}
\AtBeginEnvironment{exercise}{\small}
\AtBeginEnvironment{example}{\small}

\usepackage{cases}
\usepackage[red]{mypack}

\usepackage[framemethod=TikZ]{mdframed}

\definecolor{bg}{rgb}{0.4,0.25,0.95}
\definecolor{pagebg}{rgb}{0,0,0.5}
\surroundwithmdframed[
   topline=false,
   rightline=false,
   bottomline=false,
   leftmargin=\parindent,
   skipabove=8pt,
   skipbelow=8pt,
   linecolor=blue,
   innerbottommargin=10pt,
   % backgroundcolor=bg,font=\color{orange}\sffamily, fontcolor=white
]{definition}

\usepackage{empheq}
\usepackage[most]{tcolorbox}

\newtcbox{\mymath}[1][]{%
    nobeforeafter, math upper, tcbox raise base,
    enhanced, colframe=blue!30!black,
    colback=red!10, boxrule=1pt,
    #1}

\usepackage{unixode}


\DeclareMathOperator{\ord}{ord}
\DeclareMathOperator{\orb}{orb}
\DeclareMathOperator{\stab}{stab}
\DeclareMathOperator{\Stab}{stab}
\DeclareMathOperator{\ppcm}{ppcm}
\DeclareMathOperator{\conj}{Conj}
\DeclareMathOperator{\End}{End}
\DeclareMathOperator{\rot}{rot}
\DeclareMathOperator{\trs}{trace}
\DeclareMathOperator{\Ind}{Ind}
\DeclareMathOperator{\mat}{Mat}
\DeclareMathOperator{\id}{Id}
\DeclareMathOperator{\vect}{vect}
\DeclareMathOperator{\img}{img}
\DeclareMathOperator{\cov}{Cov}
\DeclareMathOperator{\dist}{dist}
\DeclareMathOperator{\irr}{Irr}
\DeclareMathOperator{\image}{Im}
\DeclareMathOperator{\pd}{\partial}
\DeclareMathOperator{\epi}{epi}
\DeclareMathOperator{\Argmin}{Argmin}
\DeclareMathOperator{\dom}{dom}
\DeclareMathOperator{\proj}{proj}
\DeclareMathOperator{\ctg}{ctg}
\DeclareMathOperator{\supp}{supp}
\DeclareMathOperator{\argmin}{argmin}
\DeclareMathOperator{\mult}{mult}
\DeclareMathOperator{\ch}{ch}
\DeclareMathOperator{\sh}{sh}
\DeclareMathOperator{\rang}{rang}
\DeclareMathOperator{\diam}{diam}
\DeclareMathOperator{\Epigraphe}{Epigraphe}




\usepackage{xcolor}
\everymath{\color{blue}}
%\everymath{\color[rgb]{0,1,1}}
%\pagecolor[rgb]{0,0,0.5}


\newcommand*{\pdtest}[3][]{\ensuremath{\frac{\partial^{#1} #2}{\partial #3}}}

\newcommand*{\deffunc}[6][]{\ensuremath{
\begin{array}{rcl}
#2 : #3 &\rightarrow& #4\\
#5 &\mapsto& #6
\end{array}
}}

\newcommand{\eqcolon}{\mathrel{\resizebox{\widthof{$\mathord{=}$}}{\height}{ $\!\!=\!\!\resizebox{1.2\width}{0.8\height}{\raisebox{0.23ex}{$\mathop{:}$}}\!\!$ }}}
\newcommand{\coloneq}{\mathrel{\resizebox{\widthof{$\mathord{=}$}}{\height}{ $\!\!\resizebox{1.2\width}{0.8\height}{\raisebox{0.23ex}{$\mathop{:}$}}\!\!=\!\!$ }}}
\newcommand{\eqcolonl}{\ensuremath{\mathrel{=\!\!\mathop{:}}}}
\newcommand{\coloneql}{\ensuremath{\mathrel{\mathop{:} \!\! =}}}
\newcommand{\vc}[1]{% inline column vector
  \left(\begin{smallmatrix}#1\end{smallmatrix}\right)%
}
\newcommand{\vr}[1]{% inline row vector
  \begin{smallmatrix}(\,#1\,)\end{smallmatrix}%
}
\makeatletter
\newcommand*{\defeq}{\ =\mathrel{\rlap{%
                     \raisebox{0.3ex}{$\m@th\cdot$}}%
                     \raisebox{-0.3ex}{$\m@th\cdot$}}%
                     }
\makeatother

\newcommand{\mathcircle}[1]{% inline row vector
 \overset{\circ}{#1}
}
\newcommand{\ulim}{% low limit
 \underline{\lim}
}
\newcommand{\ssi}{% iff
\iff
}
\newcommand{\ps}[2]{
\expval{#1 | #2}
}
\newcommand{\df}[1]{
\mqty{#1}
}
\newcommand{\n}[1]{
\norm{#1}
}
\newcommand{\sys}[1]{
\left\{\smqty{#1}\right.
}


\newcommand{\eqdef}{\ensuremath{\overset{\text{def}}=}}


\def\Circlearrowright{\ensuremath{%
  \rotatebox[origin=c]{230}{$\circlearrowright$}}}

\newcommand\ct[1]{\text{\rmfamily\upshape #1}}
\newcommand\question[1]{ {\color{red} ...!? \small #1}}
\newcommand\caz[1]{\left\{\begin{array} #1 \end{array}\right.}
\newcommand\const{\text{\rmfamily\upshape const}}
\newcommand\toP{ \overset{\pro}{\to}}
\newcommand\toPP{ \overset{\text{PP}}{\to}}
\newcommand{\oeq}{\mathrel{\text{\textcircled{$=$}}}}





\usepackage{xcolor}
% \usepackage[normalem]{ulem}
\usepackage{lipsum}
\makeatletter
% \newcommand\colorwave[1][blue]{\bgroup \markoverwith{\lower3.5\p@\hbox{\sixly \textcolor{#1}{\char58}}}\ULon}
%\font\sixly=lasy6 % does not re-load if already loaded, so no memory problem.

\newmdtheoremenv[
linewidth= 1pt,linecolor= blue,%
leftmargin=20,rightmargin=20,innertopmargin=0pt, innerrightmargin=40,%
tikzsetting = { draw=lightgray, line width = 0.3pt,dashed,%
dash pattern = on 15pt off 3pt},%
splittopskip=\topskip,skipbelow=\baselineskip,%
skipabove=\baselineskip,ntheorem,roundcorner=0pt,
% backgroundcolor=pagebg,font=\color{orange}\sffamily, fontcolor=white
]{examplebox}{Exemple}[section]



\newcommand\R{\mathbb{R}}
\newcommand\Z{\mathbb{Z}}
\newcommand\N{\mathbb{N}}
\newcommand\E{\mathbb{E}}
\newcommand\F{\mathcal{F}}
\newcommand\cH{\mathcal{H}}
\newcommand\V{\mathbb{V}}
\newcommand\dmo{ ^{-1} }
\newcommand\kapa{\kappa}
\newcommand\im{Im}
\newcommand\hs{\mathcal{H}}





\usepackage{soul}

\makeatletter
\newcommand*{\whiten}[1]{\llap{\textcolor{white}{{\the\SOUL@token}}\hspace{#1pt}}}
\DeclareRobustCommand*\myul{%
    \def\SOUL@everyspace{\underline{\space}\kern\z@}%
    \def\SOUL@everytoken{%
     \setbox0=\hbox{\the\SOUL@token}%
     \ifdim\dp0>\z@
        \raisebox{\dp0}{\underline{\phantom{\the\SOUL@token}}}%
        \whiten{1}\whiten{0}%
        \whiten{-1}\whiten{-2}%
        \llap{\the\SOUL@token}%
     \else
        \underline{\the\SOUL@token}%
     \fi}%
\SOUL@}
\makeatother

\newcommand*{\demp}{\fontfamily{lmtt}\selectfont}

\DeclareTextFontCommand{\textdemp}{\demp}

\begin{document}

\ifcomment
Multiline
comment
\fi
\ifcomment
\myul{Typesetting test}
% \color[rgb]{1,1,1}
$∑_i^n≠ 60º±∞π∆¬≈√j∫h≤≥µ$

$\CR \R\pro\ind\pro\gS\pro
\mqty[a&b\\c&d]$
$\pro\mathbb{P}$
$\dd{x}$

  \[
    \alpha(x)=\left\{
                \begin{array}{ll}
                  x\\
                  \frac{1}{1+e^{-kx}}\\
                  \frac{e^x-e^{-x}}{e^x+e^{-x}}
                \end{array}
              \right.
  \]

  $\expval{x}$
  
  $\chi_\rho(ghg\dmo)=\Tr(\rho_{ghg\dmo})=\Tr(\rho_g\circ\rho_h\circ\rho\dmo_g)=\Tr(\rho_h)\overset{\mbox{\scalebox{0.5}{$\Tr(AB)=\Tr(BA)$}}}{=}\chi_\rho(h)$
  	$\mathop{\oplus}_{\substack{x\in X}}$

$\mat(\rho_g)=(a_{ij}(g))_{\scriptsize \substack{1\leq i\leq d \\ 1\leq j\leq d}}$ et $\mat(\rho'_g)=(a'_{ij}(g))_{\scriptsize \substack{1\leq i'\leq d' \\ 1\leq j'\leq d'}}$



\[\int_a^b{\mathbb{R}^2}g(u, v)\dd{P_{XY}}(u, v)=\iint g(u,v) f_{XY}(u, v)\dd \lambda(u) \dd \lambda(v)\]
$$\lim_{x\to\infty} f(x)$$	
$$\iiiint_V \mu(t,u,v,w) \,dt\,du\,dv\,dw$$
$$\sum_{n=1}^{\infty} 2^{-n} = 1$$	
\begin{definition}
	Si $X$ et $Y$ sont 2 v.a. ou definit la \textsc{Covariance} entre $X$ et $Y$ comme
	$\cov(X,Y)\overset{\text{def}}{=}\E\left[(X-\E(X))(Y-\E(Y))\right]=\E(XY)-\E(X)\E(Y)$.
\end{definition}
\fi
\pagebreak

% \tableofcontents

% insert your code here
%\input{./algebra/main.tex}
%\input{./geometrie-differentielle/main.tex}
%\input{./probabilite/main.tex}
%\input{./analyse-fonctionnelle/main.tex}
% \input{./Analyse-convexe-et-dualite-en-optimisation/main.tex}
%\input{./tikz/main.tex}
%\input{./Theorie-du-distributions/main.tex}
%\input{./optimisation/mine.tex}
 \input{./modelisation/main.tex}

% yves.aubry@univ-tln.fr : algebra

\end{document}

%% !TEX encoding = UTF-8 Unicode
% !TEX TS-program = xelatex

\documentclass[french]{report}

%\usepackage[utf8]{inputenc}
%\usepackage[T1]{fontenc}
\usepackage{babel}


\newif\ifcomment
%\commenttrue # Show comments

\usepackage{physics}
\usepackage{amssymb}


\usepackage{amsthm}
% \usepackage{thmtools}
\usepackage{mathtools}
\usepackage{amsfonts}

\usepackage{color}

\usepackage{tikz}

\usepackage{geometry}
\geometry{a5paper, margin=0.1in, right=1cm}

\usepackage{dsfont}

\usepackage{graphicx}
\graphicspath{ {images/} }

\usepackage{faktor}

\usepackage{IEEEtrantools}
\usepackage{enumerate}   
\usepackage[PostScript=dvips]{"/Users/aware/Documents/Courses/diagrams"}


\newtheorem{theorem}{Théorème}[section]
\renewcommand{\thetheorem}{\arabic{theorem}}
\newtheorem{lemme}{Lemme}[section]
\renewcommand{\thelemme}{\arabic{lemme}}
\newtheorem{proposition}{Proposition}[section]
\renewcommand{\theproposition}{\arabic{proposition}}
\newtheorem{notations}{Notations}[section]
\newtheorem{problem}{Problème}[section]
\newtheorem{corollary}{Corollaire}[theorem]
\renewcommand{\thecorollary}{\arabic{corollary}}
\newtheorem{property}{Propriété}[section]
\newtheorem{objective}{Objectif}[section]

\theoremstyle{definition}
\newtheorem{definition}{Définition}[section]
\renewcommand{\thedefinition}{\arabic{definition}}
\newtheorem{exercise}{Exercice}[chapter]
\renewcommand{\theexercise}{\arabic{exercise}}
\newtheorem{example}{Exemple}[chapter]
\renewcommand{\theexample}{\arabic{example}}
\newtheorem*{solution}{Solution}
\newtheorem*{application}{Application}
\newtheorem*{notation}{Notation}
\newtheorem*{vocabulary}{Vocabulaire}
\newtheorem*{properties}{Propriétés}



\theoremstyle{remark}
\newtheorem*{remark}{Remarque}
\newtheorem*{rappel}{Rappel}


\usepackage{etoolbox}
\AtBeginEnvironment{exercise}{\small}
\AtBeginEnvironment{example}{\small}

\usepackage{cases}
\usepackage[red]{mypack}

\usepackage[framemethod=TikZ]{mdframed}

\definecolor{bg}{rgb}{0.4,0.25,0.95}
\definecolor{pagebg}{rgb}{0,0,0.5}
\surroundwithmdframed[
   topline=false,
   rightline=false,
   bottomline=false,
   leftmargin=\parindent,
   skipabove=8pt,
   skipbelow=8pt,
   linecolor=blue,
   innerbottommargin=10pt,
   % backgroundcolor=bg,font=\color{orange}\sffamily, fontcolor=white
]{definition}

\usepackage{empheq}
\usepackage[most]{tcolorbox}

\newtcbox{\mymath}[1][]{%
    nobeforeafter, math upper, tcbox raise base,
    enhanced, colframe=blue!30!black,
    colback=red!10, boxrule=1pt,
    #1}

\usepackage{unixode}


\DeclareMathOperator{\ord}{ord}
\DeclareMathOperator{\orb}{orb}
\DeclareMathOperator{\stab}{stab}
\DeclareMathOperator{\Stab}{stab}
\DeclareMathOperator{\ppcm}{ppcm}
\DeclareMathOperator{\conj}{Conj}
\DeclareMathOperator{\End}{End}
\DeclareMathOperator{\rot}{rot}
\DeclareMathOperator{\trs}{trace}
\DeclareMathOperator{\Ind}{Ind}
\DeclareMathOperator{\mat}{Mat}
\DeclareMathOperator{\id}{Id}
\DeclareMathOperator{\vect}{vect}
\DeclareMathOperator{\img}{img}
\DeclareMathOperator{\cov}{Cov}
\DeclareMathOperator{\dist}{dist}
\DeclareMathOperator{\irr}{Irr}
\DeclareMathOperator{\image}{Im}
\DeclareMathOperator{\pd}{\partial}
\DeclareMathOperator{\epi}{epi}
\DeclareMathOperator{\Argmin}{Argmin}
\DeclareMathOperator{\dom}{dom}
\DeclareMathOperator{\proj}{proj}
\DeclareMathOperator{\ctg}{ctg}
\DeclareMathOperator{\supp}{supp}
\DeclareMathOperator{\argmin}{argmin}
\DeclareMathOperator{\mult}{mult}
\DeclareMathOperator{\ch}{ch}
\DeclareMathOperator{\sh}{sh}
\DeclareMathOperator{\rang}{rang}
\DeclareMathOperator{\diam}{diam}
\DeclareMathOperator{\Epigraphe}{Epigraphe}




\usepackage{xcolor}
\everymath{\color{blue}}
%\everymath{\color[rgb]{0,1,1}}
%\pagecolor[rgb]{0,0,0.5}


\newcommand*{\pdtest}[3][]{\ensuremath{\frac{\partial^{#1} #2}{\partial #3}}}

\newcommand*{\deffunc}[6][]{\ensuremath{
\begin{array}{rcl}
#2 : #3 &\rightarrow& #4\\
#5 &\mapsto& #6
\end{array}
}}

\newcommand{\eqcolon}{\mathrel{\resizebox{\widthof{$\mathord{=}$}}{\height}{ $\!\!=\!\!\resizebox{1.2\width}{0.8\height}{\raisebox{0.23ex}{$\mathop{:}$}}\!\!$ }}}
\newcommand{\coloneq}{\mathrel{\resizebox{\widthof{$\mathord{=}$}}{\height}{ $\!\!\resizebox{1.2\width}{0.8\height}{\raisebox{0.23ex}{$\mathop{:}$}}\!\!=\!\!$ }}}
\newcommand{\eqcolonl}{\ensuremath{\mathrel{=\!\!\mathop{:}}}}
\newcommand{\coloneql}{\ensuremath{\mathrel{\mathop{:} \!\! =}}}
\newcommand{\vc}[1]{% inline column vector
  \left(\begin{smallmatrix}#1\end{smallmatrix}\right)%
}
\newcommand{\vr}[1]{% inline row vector
  \begin{smallmatrix}(\,#1\,)\end{smallmatrix}%
}
\makeatletter
\newcommand*{\defeq}{\ =\mathrel{\rlap{%
                     \raisebox{0.3ex}{$\m@th\cdot$}}%
                     \raisebox{-0.3ex}{$\m@th\cdot$}}%
                     }
\makeatother

\newcommand{\mathcircle}[1]{% inline row vector
 \overset{\circ}{#1}
}
\newcommand{\ulim}{% low limit
 \underline{\lim}
}
\newcommand{\ssi}{% iff
\iff
}
\newcommand{\ps}[2]{
\expval{#1 | #2}
}
\newcommand{\df}[1]{
\mqty{#1}
}
\newcommand{\n}[1]{
\norm{#1}
}
\newcommand{\sys}[1]{
\left\{\smqty{#1}\right.
}


\newcommand{\eqdef}{\ensuremath{\overset{\text{def}}=}}


\def\Circlearrowright{\ensuremath{%
  \rotatebox[origin=c]{230}{$\circlearrowright$}}}

\newcommand\ct[1]{\text{\rmfamily\upshape #1}}
\newcommand\question[1]{ {\color{red} ...!? \small #1}}
\newcommand\caz[1]{\left\{\begin{array} #1 \end{array}\right.}
\newcommand\const{\text{\rmfamily\upshape const}}
\newcommand\toP{ \overset{\pro}{\to}}
\newcommand\toPP{ \overset{\text{PP}}{\to}}
\newcommand{\oeq}{\mathrel{\text{\textcircled{$=$}}}}





\usepackage{xcolor}
% \usepackage[normalem]{ulem}
\usepackage{lipsum}
\makeatletter
% \newcommand\colorwave[1][blue]{\bgroup \markoverwith{\lower3.5\p@\hbox{\sixly \textcolor{#1}{\char58}}}\ULon}
%\font\sixly=lasy6 % does not re-load if already loaded, so no memory problem.

\newmdtheoremenv[
linewidth= 1pt,linecolor= blue,%
leftmargin=20,rightmargin=20,innertopmargin=0pt, innerrightmargin=40,%
tikzsetting = { draw=lightgray, line width = 0.3pt,dashed,%
dash pattern = on 15pt off 3pt},%
splittopskip=\topskip,skipbelow=\baselineskip,%
skipabove=\baselineskip,ntheorem,roundcorner=0pt,
% backgroundcolor=pagebg,font=\color{orange}\sffamily, fontcolor=white
]{examplebox}{Exemple}[section]



\newcommand\R{\mathbb{R}}
\newcommand\Z{\mathbb{Z}}
\newcommand\N{\mathbb{N}}
\newcommand\E{\mathbb{E}}
\newcommand\F{\mathcal{F}}
\newcommand\cH{\mathcal{H}}
\newcommand\V{\mathbb{V}}
\newcommand\dmo{ ^{-1} }
\newcommand\kapa{\kappa}
\newcommand\im{Im}
\newcommand\hs{\mathcal{H}}





\usepackage{soul}

\makeatletter
\newcommand*{\whiten}[1]{\llap{\textcolor{white}{{\the\SOUL@token}}\hspace{#1pt}}}
\DeclareRobustCommand*\myul{%
    \def\SOUL@everyspace{\underline{\space}\kern\z@}%
    \def\SOUL@everytoken{%
     \setbox0=\hbox{\the\SOUL@token}%
     \ifdim\dp0>\z@
        \raisebox{\dp0}{\underline{\phantom{\the\SOUL@token}}}%
        \whiten{1}\whiten{0}%
        \whiten{-1}\whiten{-2}%
        \llap{\the\SOUL@token}%
     \else
        \underline{\the\SOUL@token}%
     \fi}%
\SOUL@}
\makeatother

\newcommand*{\demp}{\fontfamily{lmtt}\selectfont}

\DeclareTextFontCommand{\textdemp}{\demp}

\begin{document}

\ifcomment
Multiline
comment
\fi
\ifcomment
\myul{Typesetting test}
% \color[rgb]{1,1,1}
$∑_i^n≠ 60º±∞π∆¬≈√j∫h≤≥µ$

$\CR \R\pro\ind\pro\gS\pro
\mqty[a&b\\c&d]$
$\pro\mathbb{P}$
$\dd{x}$

  \[
    \alpha(x)=\left\{
                \begin{array}{ll}
                  x\\
                  \frac{1}{1+e^{-kx}}\\
                  \frac{e^x-e^{-x}}{e^x+e^{-x}}
                \end{array}
              \right.
  \]

  $\expval{x}$
  
  $\chi_\rho(ghg\dmo)=\Tr(\rho_{ghg\dmo})=\Tr(\rho_g\circ\rho_h\circ\rho\dmo_g)=\Tr(\rho_h)\overset{\mbox{\scalebox{0.5}{$\Tr(AB)=\Tr(BA)$}}}{=}\chi_\rho(h)$
  	$\mathop{\oplus}_{\substack{x\in X}}$

$\mat(\rho_g)=(a_{ij}(g))_{\scriptsize \substack{1\leq i\leq d \\ 1\leq j\leq d}}$ et $\mat(\rho'_g)=(a'_{ij}(g))_{\scriptsize \substack{1\leq i'\leq d' \\ 1\leq j'\leq d'}}$



\[\int_a^b{\mathbb{R}^2}g(u, v)\dd{P_{XY}}(u, v)=\iint g(u,v) f_{XY}(u, v)\dd \lambda(u) \dd \lambda(v)\]
$$\lim_{x\to\infty} f(x)$$	
$$\iiiint_V \mu(t,u,v,w) \,dt\,du\,dv\,dw$$
$$\sum_{n=1}^{\infty} 2^{-n} = 1$$	
\begin{definition}
	Si $X$ et $Y$ sont 2 v.a. ou definit la \textsc{Covariance} entre $X$ et $Y$ comme
	$\cov(X,Y)\overset{\text{def}}{=}\E\left[(X-\E(X))(Y-\E(Y))\right]=\E(XY)-\E(X)\E(Y)$.
\end{definition}
\fi
\pagebreak

% \tableofcontents

% insert your code here
%\input{./algebra/main.tex}
%\input{./geometrie-differentielle/main.tex}
%\input{./probabilite/main.tex}
%\input{./analyse-fonctionnelle/main.tex}
% \input{./Analyse-convexe-et-dualite-en-optimisation/main.tex}
%\input{./tikz/main.tex}
%\input{./Theorie-du-distributions/main.tex}
%\input{./optimisation/mine.tex}
 \input{./modelisation/main.tex}

% yves.aubry@univ-tln.fr : algebra

\end{document}

%\input{./optimisation/mine.tex}
 % !TEX encoding = UTF-8 Unicode
% !TEX TS-program = xelatex

\documentclass[french]{report}

%\usepackage[utf8]{inputenc}
%\usepackage[T1]{fontenc}
\usepackage{babel}


\newif\ifcomment
%\commenttrue # Show comments

\usepackage{physics}
\usepackage{amssymb}


\usepackage{amsthm}
% \usepackage{thmtools}
\usepackage{mathtools}
\usepackage{amsfonts}

\usepackage{color}

\usepackage{tikz}

\usepackage{geometry}
\geometry{a5paper, margin=0.1in, right=1cm}

\usepackage{dsfont}

\usepackage{graphicx}
\graphicspath{ {images/} }

\usepackage{faktor}

\usepackage{IEEEtrantools}
\usepackage{enumerate}   
\usepackage[PostScript=dvips]{"/Users/aware/Documents/Courses/diagrams"}


\newtheorem{theorem}{Théorème}[section]
\renewcommand{\thetheorem}{\arabic{theorem}}
\newtheorem{lemme}{Lemme}[section]
\renewcommand{\thelemme}{\arabic{lemme}}
\newtheorem{proposition}{Proposition}[section]
\renewcommand{\theproposition}{\arabic{proposition}}
\newtheorem{notations}{Notations}[section]
\newtheorem{problem}{Problème}[section]
\newtheorem{corollary}{Corollaire}[theorem]
\renewcommand{\thecorollary}{\arabic{corollary}}
\newtheorem{property}{Propriété}[section]
\newtheorem{objective}{Objectif}[section]

\theoremstyle{definition}
\newtheorem{definition}{Définition}[section]
\renewcommand{\thedefinition}{\arabic{definition}}
\newtheorem{exercise}{Exercice}[chapter]
\renewcommand{\theexercise}{\arabic{exercise}}
\newtheorem{example}{Exemple}[chapter]
\renewcommand{\theexample}{\arabic{example}}
\newtheorem*{solution}{Solution}
\newtheorem*{application}{Application}
\newtheorem*{notation}{Notation}
\newtheorem*{vocabulary}{Vocabulaire}
\newtheorem*{properties}{Propriétés}



\theoremstyle{remark}
\newtheorem*{remark}{Remarque}
\newtheorem*{rappel}{Rappel}


\usepackage{etoolbox}
\AtBeginEnvironment{exercise}{\small}
\AtBeginEnvironment{example}{\small}

\usepackage{cases}
\usepackage[red]{mypack}

\usepackage[framemethod=TikZ]{mdframed}

\definecolor{bg}{rgb}{0.4,0.25,0.95}
\definecolor{pagebg}{rgb}{0,0,0.5}
\surroundwithmdframed[
   topline=false,
   rightline=false,
   bottomline=false,
   leftmargin=\parindent,
   skipabove=8pt,
   skipbelow=8pt,
   linecolor=blue,
   innerbottommargin=10pt,
   % backgroundcolor=bg,font=\color{orange}\sffamily, fontcolor=white
]{definition}

\usepackage{empheq}
\usepackage[most]{tcolorbox}

\newtcbox{\mymath}[1][]{%
    nobeforeafter, math upper, tcbox raise base,
    enhanced, colframe=blue!30!black,
    colback=red!10, boxrule=1pt,
    #1}

\usepackage{unixode}


\DeclareMathOperator{\ord}{ord}
\DeclareMathOperator{\orb}{orb}
\DeclareMathOperator{\stab}{stab}
\DeclareMathOperator{\Stab}{stab}
\DeclareMathOperator{\ppcm}{ppcm}
\DeclareMathOperator{\conj}{Conj}
\DeclareMathOperator{\End}{End}
\DeclareMathOperator{\rot}{rot}
\DeclareMathOperator{\trs}{trace}
\DeclareMathOperator{\Ind}{Ind}
\DeclareMathOperator{\mat}{Mat}
\DeclareMathOperator{\id}{Id}
\DeclareMathOperator{\vect}{vect}
\DeclareMathOperator{\img}{img}
\DeclareMathOperator{\cov}{Cov}
\DeclareMathOperator{\dist}{dist}
\DeclareMathOperator{\irr}{Irr}
\DeclareMathOperator{\image}{Im}
\DeclareMathOperator{\pd}{\partial}
\DeclareMathOperator{\epi}{epi}
\DeclareMathOperator{\Argmin}{Argmin}
\DeclareMathOperator{\dom}{dom}
\DeclareMathOperator{\proj}{proj}
\DeclareMathOperator{\ctg}{ctg}
\DeclareMathOperator{\supp}{supp}
\DeclareMathOperator{\argmin}{argmin}
\DeclareMathOperator{\mult}{mult}
\DeclareMathOperator{\ch}{ch}
\DeclareMathOperator{\sh}{sh}
\DeclareMathOperator{\rang}{rang}
\DeclareMathOperator{\diam}{diam}
\DeclareMathOperator{\Epigraphe}{Epigraphe}




\usepackage{xcolor}
\everymath{\color{blue}}
%\everymath{\color[rgb]{0,1,1}}
%\pagecolor[rgb]{0,0,0.5}


\newcommand*{\pdtest}[3][]{\ensuremath{\frac{\partial^{#1} #2}{\partial #3}}}

\newcommand*{\deffunc}[6][]{\ensuremath{
\begin{array}{rcl}
#2 : #3 &\rightarrow& #4\\
#5 &\mapsto& #6
\end{array}
}}

\newcommand{\eqcolon}{\mathrel{\resizebox{\widthof{$\mathord{=}$}}{\height}{ $\!\!=\!\!\resizebox{1.2\width}{0.8\height}{\raisebox{0.23ex}{$\mathop{:}$}}\!\!$ }}}
\newcommand{\coloneq}{\mathrel{\resizebox{\widthof{$\mathord{=}$}}{\height}{ $\!\!\resizebox{1.2\width}{0.8\height}{\raisebox{0.23ex}{$\mathop{:}$}}\!\!=\!\!$ }}}
\newcommand{\eqcolonl}{\ensuremath{\mathrel{=\!\!\mathop{:}}}}
\newcommand{\coloneql}{\ensuremath{\mathrel{\mathop{:} \!\! =}}}
\newcommand{\vc}[1]{% inline column vector
  \left(\begin{smallmatrix}#1\end{smallmatrix}\right)%
}
\newcommand{\vr}[1]{% inline row vector
  \begin{smallmatrix}(\,#1\,)\end{smallmatrix}%
}
\makeatletter
\newcommand*{\defeq}{\ =\mathrel{\rlap{%
                     \raisebox{0.3ex}{$\m@th\cdot$}}%
                     \raisebox{-0.3ex}{$\m@th\cdot$}}%
                     }
\makeatother

\newcommand{\mathcircle}[1]{% inline row vector
 \overset{\circ}{#1}
}
\newcommand{\ulim}{% low limit
 \underline{\lim}
}
\newcommand{\ssi}{% iff
\iff
}
\newcommand{\ps}[2]{
\expval{#1 | #2}
}
\newcommand{\df}[1]{
\mqty{#1}
}
\newcommand{\n}[1]{
\norm{#1}
}
\newcommand{\sys}[1]{
\left\{\smqty{#1}\right.
}


\newcommand{\eqdef}{\ensuremath{\overset{\text{def}}=}}


\def\Circlearrowright{\ensuremath{%
  \rotatebox[origin=c]{230}{$\circlearrowright$}}}

\newcommand\ct[1]{\text{\rmfamily\upshape #1}}
\newcommand\question[1]{ {\color{red} ...!? \small #1}}
\newcommand\caz[1]{\left\{\begin{array} #1 \end{array}\right.}
\newcommand\const{\text{\rmfamily\upshape const}}
\newcommand\toP{ \overset{\pro}{\to}}
\newcommand\toPP{ \overset{\text{PP}}{\to}}
\newcommand{\oeq}{\mathrel{\text{\textcircled{$=$}}}}





\usepackage{xcolor}
% \usepackage[normalem]{ulem}
\usepackage{lipsum}
\makeatletter
% \newcommand\colorwave[1][blue]{\bgroup \markoverwith{\lower3.5\p@\hbox{\sixly \textcolor{#1}{\char58}}}\ULon}
%\font\sixly=lasy6 % does not re-load if already loaded, so no memory problem.

\newmdtheoremenv[
linewidth= 1pt,linecolor= blue,%
leftmargin=20,rightmargin=20,innertopmargin=0pt, innerrightmargin=40,%
tikzsetting = { draw=lightgray, line width = 0.3pt,dashed,%
dash pattern = on 15pt off 3pt},%
splittopskip=\topskip,skipbelow=\baselineskip,%
skipabove=\baselineskip,ntheorem,roundcorner=0pt,
% backgroundcolor=pagebg,font=\color{orange}\sffamily, fontcolor=white
]{examplebox}{Exemple}[section]



\newcommand\R{\mathbb{R}}
\newcommand\Z{\mathbb{Z}}
\newcommand\N{\mathbb{N}}
\newcommand\E{\mathbb{E}}
\newcommand\F{\mathcal{F}}
\newcommand\cH{\mathcal{H}}
\newcommand\V{\mathbb{V}}
\newcommand\dmo{ ^{-1} }
\newcommand\kapa{\kappa}
\newcommand\im{Im}
\newcommand\hs{\mathcal{H}}





\usepackage{soul}

\makeatletter
\newcommand*{\whiten}[1]{\llap{\textcolor{white}{{\the\SOUL@token}}\hspace{#1pt}}}
\DeclareRobustCommand*\myul{%
    \def\SOUL@everyspace{\underline{\space}\kern\z@}%
    \def\SOUL@everytoken{%
     \setbox0=\hbox{\the\SOUL@token}%
     \ifdim\dp0>\z@
        \raisebox{\dp0}{\underline{\phantom{\the\SOUL@token}}}%
        \whiten{1}\whiten{0}%
        \whiten{-1}\whiten{-2}%
        \llap{\the\SOUL@token}%
     \else
        \underline{\the\SOUL@token}%
     \fi}%
\SOUL@}
\makeatother

\newcommand*{\demp}{\fontfamily{lmtt}\selectfont}

\DeclareTextFontCommand{\textdemp}{\demp}

\begin{document}

\ifcomment
Multiline
comment
\fi
\ifcomment
\myul{Typesetting test}
% \color[rgb]{1,1,1}
$∑_i^n≠ 60º±∞π∆¬≈√j∫h≤≥µ$

$\CR \R\pro\ind\pro\gS\pro
\mqty[a&b\\c&d]$
$\pro\mathbb{P}$
$\dd{x}$

  \[
    \alpha(x)=\left\{
                \begin{array}{ll}
                  x\\
                  \frac{1}{1+e^{-kx}}\\
                  \frac{e^x-e^{-x}}{e^x+e^{-x}}
                \end{array}
              \right.
  \]

  $\expval{x}$
  
  $\chi_\rho(ghg\dmo)=\Tr(\rho_{ghg\dmo})=\Tr(\rho_g\circ\rho_h\circ\rho\dmo_g)=\Tr(\rho_h)\overset{\mbox{\scalebox{0.5}{$\Tr(AB)=\Tr(BA)$}}}{=}\chi_\rho(h)$
  	$\mathop{\oplus}_{\substack{x\in X}}$

$\mat(\rho_g)=(a_{ij}(g))_{\scriptsize \substack{1\leq i\leq d \\ 1\leq j\leq d}}$ et $\mat(\rho'_g)=(a'_{ij}(g))_{\scriptsize \substack{1\leq i'\leq d' \\ 1\leq j'\leq d'}}$



\[\int_a^b{\mathbb{R}^2}g(u, v)\dd{P_{XY}}(u, v)=\iint g(u,v) f_{XY}(u, v)\dd \lambda(u) \dd \lambda(v)\]
$$\lim_{x\to\infty} f(x)$$	
$$\iiiint_V \mu(t,u,v,w) \,dt\,du\,dv\,dw$$
$$\sum_{n=1}^{\infty} 2^{-n} = 1$$	
\begin{definition}
	Si $X$ et $Y$ sont 2 v.a. ou definit la \textsc{Covariance} entre $X$ et $Y$ comme
	$\cov(X,Y)\overset{\text{def}}{=}\E\left[(X-\E(X))(Y-\E(Y))\right]=\E(XY)-\E(X)\E(Y)$.
\end{definition}
\fi
\pagebreak

% \tableofcontents

% insert your code here
%\input{./algebra/main.tex}
%\input{./geometrie-differentielle/main.tex}
%\input{./probabilite/main.tex}
%\input{./analyse-fonctionnelle/main.tex}
% \input{./Analyse-convexe-et-dualite-en-optimisation/main.tex}
%\input{./tikz/main.tex}
%\input{./Theorie-du-distributions/main.tex}
%\input{./optimisation/mine.tex}
 \input{./modelisation/main.tex}

% yves.aubry@univ-tln.fr : algebra

\end{document}


% yves.aubry@univ-tln.fr : algebra

\end{document}


% yves.aubry@univ-tln.fr : algebra

\end{document}

%% !TEX encoding = UTF-8 Unicode
% !TEX TS-program = xelatex

\documentclass[french]{report}

%\usepackage[utf8]{inputenc}
%\usepackage[T1]{fontenc}
\usepackage{babel}


\newif\ifcomment
%\commenttrue # Show comments

\usepackage{physics}
\usepackage{amssymb}


\usepackage{amsthm}
% \usepackage{thmtools}
\usepackage{mathtools}
\usepackage{amsfonts}

\usepackage{color}

\usepackage{tikz}

\usepackage{geometry}
\geometry{a5paper, margin=0.1in, right=1cm}

\usepackage{dsfont}

\usepackage{graphicx}
\graphicspath{ {images/} }

\usepackage{faktor}

\usepackage{IEEEtrantools}
\usepackage{enumerate}   
\usepackage[PostScript=dvips]{"/Users/aware/Documents/Courses/diagrams"}


\newtheorem{theorem}{Théorème}[section]
\renewcommand{\thetheorem}{\arabic{theorem}}
\newtheorem{lemme}{Lemme}[section]
\renewcommand{\thelemme}{\arabic{lemme}}
\newtheorem{proposition}{Proposition}[section]
\renewcommand{\theproposition}{\arabic{proposition}}
\newtheorem{notations}{Notations}[section]
\newtheorem{problem}{Problème}[section]
\newtheorem{corollary}{Corollaire}[theorem]
\renewcommand{\thecorollary}{\arabic{corollary}}
\newtheorem{property}{Propriété}[section]
\newtheorem{objective}{Objectif}[section]

\theoremstyle{definition}
\newtheorem{definition}{Définition}[section]
\renewcommand{\thedefinition}{\arabic{definition}}
\newtheorem{exercise}{Exercice}[chapter]
\renewcommand{\theexercise}{\arabic{exercise}}
\newtheorem{example}{Exemple}[chapter]
\renewcommand{\theexample}{\arabic{example}}
\newtheorem*{solution}{Solution}
\newtheorem*{application}{Application}
\newtheorem*{notation}{Notation}
\newtheorem*{vocabulary}{Vocabulaire}
\newtheorem*{properties}{Propriétés}



\theoremstyle{remark}
\newtheorem*{remark}{Remarque}
\newtheorem*{rappel}{Rappel}


\usepackage{etoolbox}
\AtBeginEnvironment{exercise}{\small}
\AtBeginEnvironment{example}{\small}

\usepackage{cases}
\usepackage[red]{mypack}

\usepackage[framemethod=TikZ]{mdframed}

\definecolor{bg}{rgb}{0.4,0.25,0.95}
\definecolor{pagebg}{rgb}{0,0,0.5}
\surroundwithmdframed[
   topline=false,
   rightline=false,
   bottomline=false,
   leftmargin=\parindent,
   skipabove=8pt,
   skipbelow=8pt,
   linecolor=blue,
   innerbottommargin=10pt,
   % backgroundcolor=bg,font=\color{orange}\sffamily, fontcolor=white
]{definition}

\usepackage{empheq}
\usepackage[most]{tcolorbox}

\newtcbox{\mymath}[1][]{%
    nobeforeafter, math upper, tcbox raise base,
    enhanced, colframe=blue!30!black,
    colback=red!10, boxrule=1pt,
    #1}

\usepackage{unixode}


\DeclareMathOperator{\ord}{ord}
\DeclareMathOperator{\orb}{orb}
\DeclareMathOperator{\stab}{stab}
\DeclareMathOperator{\Stab}{stab}
\DeclareMathOperator{\ppcm}{ppcm}
\DeclareMathOperator{\conj}{Conj}
\DeclareMathOperator{\End}{End}
\DeclareMathOperator{\rot}{rot}
\DeclareMathOperator{\trs}{trace}
\DeclareMathOperator{\Ind}{Ind}
\DeclareMathOperator{\mat}{Mat}
\DeclareMathOperator{\id}{Id}
\DeclareMathOperator{\vect}{vect}
\DeclareMathOperator{\img}{img}
\DeclareMathOperator{\cov}{Cov}
\DeclareMathOperator{\dist}{dist}
\DeclareMathOperator{\irr}{Irr}
\DeclareMathOperator{\image}{Im}
\DeclareMathOperator{\pd}{\partial}
\DeclareMathOperator{\epi}{epi}
\DeclareMathOperator{\Argmin}{Argmin}
\DeclareMathOperator{\dom}{dom}
\DeclareMathOperator{\proj}{proj}
\DeclareMathOperator{\ctg}{ctg}
\DeclareMathOperator{\supp}{supp}
\DeclareMathOperator{\argmin}{argmin}
\DeclareMathOperator{\mult}{mult}
\DeclareMathOperator{\ch}{ch}
\DeclareMathOperator{\sh}{sh}
\DeclareMathOperator{\rang}{rang}
\DeclareMathOperator{\diam}{diam}
\DeclareMathOperator{\Epigraphe}{Epigraphe}




\usepackage{xcolor}
\everymath{\color{blue}}
%\everymath{\color[rgb]{0,1,1}}
%\pagecolor[rgb]{0,0,0.5}


\newcommand*{\pdtest}[3][]{\ensuremath{\frac{\partial^{#1} #2}{\partial #3}}}

\newcommand*{\deffunc}[6][]{\ensuremath{
\begin{array}{rcl}
#2 : #3 &\rightarrow& #4\\
#5 &\mapsto& #6
\end{array}
}}

\newcommand{\eqcolon}{\mathrel{\resizebox{\widthof{$\mathord{=}$}}{\height}{ $\!\!=\!\!\resizebox{1.2\width}{0.8\height}{\raisebox{0.23ex}{$\mathop{:}$}}\!\!$ }}}
\newcommand{\coloneq}{\mathrel{\resizebox{\widthof{$\mathord{=}$}}{\height}{ $\!\!\resizebox{1.2\width}{0.8\height}{\raisebox{0.23ex}{$\mathop{:}$}}\!\!=\!\!$ }}}
\newcommand{\eqcolonl}{\ensuremath{\mathrel{=\!\!\mathop{:}}}}
\newcommand{\coloneql}{\ensuremath{\mathrel{\mathop{:} \!\! =}}}
\newcommand{\vc}[1]{% inline column vector
  \left(\begin{smallmatrix}#1\end{smallmatrix}\right)%
}
\newcommand{\vr}[1]{% inline row vector
  \begin{smallmatrix}(\,#1\,)\end{smallmatrix}%
}
\makeatletter
\newcommand*{\defeq}{\ =\mathrel{\rlap{%
                     \raisebox{0.3ex}{$\m@th\cdot$}}%
                     \raisebox{-0.3ex}{$\m@th\cdot$}}%
                     }
\makeatother

\newcommand{\mathcircle}[1]{% inline row vector
 \overset{\circ}{#1}
}
\newcommand{\ulim}{% low limit
 \underline{\lim}
}
\newcommand{\ssi}{% iff
\iff
}
\newcommand{\ps}[2]{
\expval{#1 | #2}
}
\newcommand{\df}[1]{
\mqty{#1}
}
\newcommand{\n}[1]{
\norm{#1}
}
\newcommand{\sys}[1]{
\left\{\smqty{#1}\right.
}


\newcommand{\eqdef}{\ensuremath{\overset{\text{def}}=}}


\def\Circlearrowright{\ensuremath{%
  \rotatebox[origin=c]{230}{$\circlearrowright$}}}

\newcommand\ct[1]{\text{\rmfamily\upshape #1}}
\newcommand\question[1]{ {\color{red} ...!? \small #1}}
\newcommand\caz[1]{\left\{\begin{array} #1 \end{array}\right.}
\newcommand\const{\text{\rmfamily\upshape const}}
\newcommand\toP{ \overset{\pro}{\to}}
\newcommand\toPP{ \overset{\text{PP}}{\to}}
\newcommand{\oeq}{\mathrel{\text{\textcircled{$=$}}}}





\usepackage{xcolor}
% \usepackage[normalem]{ulem}
\usepackage{lipsum}
\makeatletter
% \newcommand\colorwave[1][blue]{\bgroup \markoverwith{\lower3.5\p@\hbox{\sixly \textcolor{#1}{\char58}}}\ULon}
%\font\sixly=lasy6 % does not re-load if already loaded, so no memory problem.

\newmdtheoremenv[
linewidth= 1pt,linecolor= blue,%
leftmargin=20,rightmargin=20,innertopmargin=0pt, innerrightmargin=40,%
tikzsetting = { draw=lightgray, line width = 0.3pt,dashed,%
dash pattern = on 15pt off 3pt},%
splittopskip=\topskip,skipbelow=\baselineskip,%
skipabove=\baselineskip,ntheorem,roundcorner=0pt,
% backgroundcolor=pagebg,font=\color{orange}\sffamily, fontcolor=white
]{examplebox}{Exemple}[section]



\newcommand\R{\mathbb{R}}
\newcommand\Z{\mathbb{Z}}
\newcommand\N{\mathbb{N}}
\newcommand\E{\mathbb{E}}
\newcommand\F{\mathcal{F}}
\newcommand\cH{\mathcal{H}}
\newcommand\V{\mathbb{V}}
\newcommand\dmo{ ^{-1} }
\newcommand\kapa{\kappa}
\newcommand\im{Im}
\newcommand\hs{\mathcal{H}}





\usepackage{soul}

\makeatletter
\newcommand*{\whiten}[1]{\llap{\textcolor{white}{{\the\SOUL@token}}\hspace{#1pt}}}
\DeclareRobustCommand*\myul{%
    \def\SOUL@everyspace{\underline{\space}\kern\z@}%
    \def\SOUL@everytoken{%
     \setbox0=\hbox{\the\SOUL@token}%
     \ifdim\dp0>\z@
        \raisebox{\dp0}{\underline{\phantom{\the\SOUL@token}}}%
        \whiten{1}\whiten{0}%
        \whiten{-1}\whiten{-2}%
        \llap{\the\SOUL@token}%
     \else
        \underline{\the\SOUL@token}%
     \fi}%
\SOUL@}
\makeatother

\newcommand*{\demp}{\fontfamily{lmtt}\selectfont}

\DeclareTextFontCommand{\textdemp}{\demp}

\begin{document}

\ifcomment
Multiline
comment
\fi
\ifcomment
\myul{Typesetting test}
% \color[rgb]{1,1,1}
$∑_i^n≠ 60º±∞π∆¬≈√j∫h≤≥µ$

$\CR \R\pro\ind\pro\gS\pro
\mqty[a&b\\c&d]$
$\pro\mathbb{P}$
$\dd{x}$

  \[
    \alpha(x)=\left\{
                \begin{array}{ll}
                  x\\
                  \frac{1}{1+e^{-kx}}\\
                  \frac{e^x-e^{-x}}{e^x+e^{-x}}
                \end{array}
              \right.
  \]

  $\expval{x}$
  
  $\chi_\rho(ghg\dmo)=\Tr(\rho_{ghg\dmo})=\Tr(\rho_g\circ\rho_h\circ\rho\dmo_g)=\Tr(\rho_h)\overset{\mbox{\scalebox{0.5}{$\Tr(AB)=\Tr(BA)$}}}{=}\chi_\rho(h)$
  	$\mathop{\oplus}_{\substack{x\in X}}$

$\mat(\rho_g)=(a_{ij}(g))_{\scriptsize \substack{1\leq i\leq d \\ 1\leq j\leq d}}$ et $\mat(\rho'_g)=(a'_{ij}(g))_{\scriptsize \substack{1\leq i'\leq d' \\ 1\leq j'\leq d'}}$



\[\int_a^b{\mathbb{R}^2}g(u, v)\dd{P_{XY}}(u, v)=\iint g(u,v) f_{XY}(u, v)\dd \lambda(u) \dd \lambda(v)\]
$$\lim_{x\to\infty} f(x)$$	
$$\iiiint_V \mu(t,u,v,w) \,dt\,du\,dv\,dw$$
$$\sum_{n=1}^{\infty} 2^{-n} = 1$$	
\begin{definition}
	Si $X$ et $Y$ sont 2 v.a. ou definit la \textsc{Covariance} entre $X$ et $Y$ comme
	$\cov(X,Y)\overset{\text{def}}{=}\E\left[(X-\E(X))(Y-\E(Y))\right]=\E(XY)-\E(X)\E(Y)$.
\end{definition}
\fi
\pagebreak

% \tableofcontents

% insert your code here
%% !TEX encoding = UTF-8 Unicode
% !TEX TS-program = xelatex

\documentclass[french]{report}

%\usepackage[utf8]{inputenc}
%\usepackage[T1]{fontenc}
\usepackage{babel}


\newif\ifcomment
%\commenttrue # Show comments

\usepackage{physics}
\usepackage{amssymb}


\usepackage{amsthm}
% \usepackage{thmtools}
\usepackage{mathtools}
\usepackage{amsfonts}

\usepackage{color}

\usepackage{tikz}

\usepackage{geometry}
\geometry{a5paper, margin=0.1in, right=1cm}

\usepackage{dsfont}

\usepackage{graphicx}
\graphicspath{ {images/} }

\usepackage{faktor}

\usepackage{IEEEtrantools}
\usepackage{enumerate}   
\usepackage[PostScript=dvips]{"/Users/aware/Documents/Courses/diagrams"}


\newtheorem{theorem}{Théorème}[section]
\renewcommand{\thetheorem}{\arabic{theorem}}
\newtheorem{lemme}{Lemme}[section]
\renewcommand{\thelemme}{\arabic{lemme}}
\newtheorem{proposition}{Proposition}[section]
\renewcommand{\theproposition}{\arabic{proposition}}
\newtheorem{notations}{Notations}[section]
\newtheorem{problem}{Problème}[section]
\newtheorem{corollary}{Corollaire}[theorem]
\renewcommand{\thecorollary}{\arabic{corollary}}
\newtheorem{property}{Propriété}[section]
\newtheorem{objective}{Objectif}[section]

\theoremstyle{definition}
\newtheorem{definition}{Définition}[section]
\renewcommand{\thedefinition}{\arabic{definition}}
\newtheorem{exercise}{Exercice}[chapter]
\renewcommand{\theexercise}{\arabic{exercise}}
\newtheorem{example}{Exemple}[chapter]
\renewcommand{\theexample}{\arabic{example}}
\newtheorem*{solution}{Solution}
\newtheorem*{application}{Application}
\newtheorem*{notation}{Notation}
\newtheorem*{vocabulary}{Vocabulaire}
\newtheorem*{properties}{Propriétés}



\theoremstyle{remark}
\newtheorem*{remark}{Remarque}
\newtheorem*{rappel}{Rappel}


\usepackage{etoolbox}
\AtBeginEnvironment{exercise}{\small}
\AtBeginEnvironment{example}{\small}

\usepackage{cases}
\usepackage[red]{mypack}

\usepackage[framemethod=TikZ]{mdframed}

\definecolor{bg}{rgb}{0.4,0.25,0.95}
\definecolor{pagebg}{rgb}{0,0,0.5}
\surroundwithmdframed[
   topline=false,
   rightline=false,
   bottomline=false,
   leftmargin=\parindent,
   skipabove=8pt,
   skipbelow=8pt,
   linecolor=blue,
   innerbottommargin=10pt,
   % backgroundcolor=bg,font=\color{orange}\sffamily, fontcolor=white
]{definition}

\usepackage{empheq}
\usepackage[most]{tcolorbox}

\newtcbox{\mymath}[1][]{%
    nobeforeafter, math upper, tcbox raise base,
    enhanced, colframe=blue!30!black,
    colback=red!10, boxrule=1pt,
    #1}

\usepackage{unixode}


\DeclareMathOperator{\ord}{ord}
\DeclareMathOperator{\orb}{orb}
\DeclareMathOperator{\stab}{stab}
\DeclareMathOperator{\Stab}{stab}
\DeclareMathOperator{\ppcm}{ppcm}
\DeclareMathOperator{\conj}{Conj}
\DeclareMathOperator{\End}{End}
\DeclareMathOperator{\rot}{rot}
\DeclareMathOperator{\trs}{trace}
\DeclareMathOperator{\Ind}{Ind}
\DeclareMathOperator{\mat}{Mat}
\DeclareMathOperator{\id}{Id}
\DeclareMathOperator{\vect}{vect}
\DeclareMathOperator{\img}{img}
\DeclareMathOperator{\cov}{Cov}
\DeclareMathOperator{\dist}{dist}
\DeclareMathOperator{\irr}{Irr}
\DeclareMathOperator{\image}{Im}
\DeclareMathOperator{\pd}{\partial}
\DeclareMathOperator{\epi}{epi}
\DeclareMathOperator{\Argmin}{Argmin}
\DeclareMathOperator{\dom}{dom}
\DeclareMathOperator{\proj}{proj}
\DeclareMathOperator{\ctg}{ctg}
\DeclareMathOperator{\supp}{supp}
\DeclareMathOperator{\argmin}{argmin}
\DeclareMathOperator{\mult}{mult}
\DeclareMathOperator{\ch}{ch}
\DeclareMathOperator{\sh}{sh}
\DeclareMathOperator{\rang}{rang}
\DeclareMathOperator{\diam}{diam}
\DeclareMathOperator{\Epigraphe}{Epigraphe}




\usepackage{xcolor}
\everymath{\color{blue}}
%\everymath{\color[rgb]{0,1,1}}
%\pagecolor[rgb]{0,0,0.5}


\newcommand*{\pdtest}[3][]{\ensuremath{\frac{\partial^{#1} #2}{\partial #3}}}

\newcommand*{\deffunc}[6][]{\ensuremath{
\begin{array}{rcl}
#2 : #3 &\rightarrow& #4\\
#5 &\mapsto& #6
\end{array}
}}

\newcommand{\eqcolon}{\mathrel{\resizebox{\widthof{$\mathord{=}$}}{\height}{ $\!\!=\!\!\resizebox{1.2\width}{0.8\height}{\raisebox{0.23ex}{$\mathop{:}$}}\!\!$ }}}
\newcommand{\coloneq}{\mathrel{\resizebox{\widthof{$\mathord{=}$}}{\height}{ $\!\!\resizebox{1.2\width}{0.8\height}{\raisebox{0.23ex}{$\mathop{:}$}}\!\!=\!\!$ }}}
\newcommand{\eqcolonl}{\ensuremath{\mathrel{=\!\!\mathop{:}}}}
\newcommand{\coloneql}{\ensuremath{\mathrel{\mathop{:} \!\! =}}}
\newcommand{\vc}[1]{% inline column vector
  \left(\begin{smallmatrix}#1\end{smallmatrix}\right)%
}
\newcommand{\vr}[1]{% inline row vector
  \begin{smallmatrix}(\,#1\,)\end{smallmatrix}%
}
\makeatletter
\newcommand*{\defeq}{\ =\mathrel{\rlap{%
                     \raisebox{0.3ex}{$\m@th\cdot$}}%
                     \raisebox{-0.3ex}{$\m@th\cdot$}}%
                     }
\makeatother

\newcommand{\mathcircle}[1]{% inline row vector
 \overset{\circ}{#1}
}
\newcommand{\ulim}{% low limit
 \underline{\lim}
}
\newcommand{\ssi}{% iff
\iff
}
\newcommand{\ps}[2]{
\expval{#1 | #2}
}
\newcommand{\df}[1]{
\mqty{#1}
}
\newcommand{\n}[1]{
\norm{#1}
}
\newcommand{\sys}[1]{
\left\{\smqty{#1}\right.
}


\newcommand{\eqdef}{\ensuremath{\overset{\text{def}}=}}


\def\Circlearrowright{\ensuremath{%
  \rotatebox[origin=c]{230}{$\circlearrowright$}}}

\newcommand\ct[1]{\text{\rmfamily\upshape #1}}
\newcommand\question[1]{ {\color{red} ...!? \small #1}}
\newcommand\caz[1]{\left\{\begin{array} #1 \end{array}\right.}
\newcommand\const{\text{\rmfamily\upshape const}}
\newcommand\toP{ \overset{\pro}{\to}}
\newcommand\toPP{ \overset{\text{PP}}{\to}}
\newcommand{\oeq}{\mathrel{\text{\textcircled{$=$}}}}





\usepackage{xcolor}
% \usepackage[normalem]{ulem}
\usepackage{lipsum}
\makeatletter
% \newcommand\colorwave[1][blue]{\bgroup \markoverwith{\lower3.5\p@\hbox{\sixly \textcolor{#1}{\char58}}}\ULon}
%\font\sixly=lasy6 % does not re-load if already loaded, so no memory problem.

\newmdtheoremenv[
linewidth= 1pt,linecolor= blue,%
leftmargin=20,rightmargin=20,innertopmargin=0pt, innerrightmargin=40,%
tikzsetting = { draw=lightgray, line width = 0.3pt,dashed,%
dash pattern = on 15pt off 3pt},%
splittopskip=\topskip,skipbelow=\baselineskip,%
skipabove=\baselineskip,ntheorem,roundcorner=0pt,
% backgroundcolor=pagebg,font=\color{orange}\sffamily, fontcolor=white
]{examplebox}{Exemple}[section]



\newcommand\R{\mathbb{R}}
\newcommand\Z{\mathbb{Z}}
\newcommand\N{\mathbb{N}}
\newcommand\E{\mathbb{E}}
\newcommand\F{\mathcal{F}}
\newcommand\cH{\mathcal{H}}
\newcommand\V{\mathbb{V}}
\newcommand\dmo{ ^{-1} }
\newcommand\kapa{\kappa}
\newcommand\im{Im}
\newcommand\hs{\mathcal{H}}





\usepackage{soul}

\makeatletter
\newcommand*{\whiten}[1]{\llap{\textcolor{white}{{\the\SOUL@token}}\hspace{#1pt}}}
\DeclareRobustCommand*\myul{%
    \def\SOUL@everyspace{\underline{\space}\kern\z@}%
    \def\SOUL@everytoken{%
     \setbox0=\hbox{\the\SOUL@token}%
     \ifdim\dp0>\z@
        \raisebox{\dp0}{\underline{\phantom{\the\SOUL@token}}}%
        \whiten{1}\whiten{0}%
        \whiten{-1}\whiten{-2}%
        \llap{\the\SOUL@token}%
     \else
        \underline{\the\SOUL@token}%
     \fi}%
\SOUL@}
\makeatother

\newcommand*{\demp}{\fontfamily{lmtt}\selectfont}

\DeclareTextFontCommand{\textdemp}{\demp}

\begin{document}

\ifcomment
Multiline
comment
\fi
\ifcomment
\myul{Typesetting test}
% \color[rgb]{1,1,1}
$∑_i^n≠ 60º±∞π∆¬≈√j∫h≤≥µ$

$\CR \R\pro\ind\pro\gS\pro
\mqty[a&b\\c&d]$
$\pro\mathbb{P}$
$\dd{x}$

  \[
    \alpha(x)=\left\{
                \begin{array}{ll}
                  x\\
                  \frac{1}{1+e^{-kx}}\\
                  \frac{e^x-e^{-x}}{e^x+e^{-x}}
                \end{array}
              \right.
  \]

  $\expval{x}$
  
  $\chi_\rho(ghg\dmo)=\Tr(\rho_{ghg\dmo})=\Tr(\rho_g\circ\rho_h\circ\rho\dmo_g)=\Tr(\rho_h)\overset{\mbox{\scalebox{0.5}{$\Tr(AB)=\Tr(BA)$}}}{=}\chi_\rho(h)$
  	$\mathop{\oplus}_{\substack{x\in X}}$

$\mat(\rho_g)=(a_{ij}(g))_{\scriptsize \substack{1\leq i\leq d \\ 1\leq j\leq d}}$ et $\mat(\rho'_g)=(a'_{ij}(g))_{\scriptsize \substack{1\leq i'\leq d' \\ 1\leq j'\leq d'}}$



\[\int_a^b{\mathbb{R}^2}g(u, v)\dd{P_{XY}}(u, v)=\iint g(u,v) f_{XY}(u, v)\dd \lambda(u) \dd \lambda(v)\]
$$\lim_{x\to\infty} f(x)$$	
$$\iiiint_V \mu(t,u,v,w) \,dt\,du\,dv\,dw$$
$$\sum_{n=1}^{\infty} 2^{-n} = 1$$	
\begin{definition}
	Si $X$ et $Y$ sont 2 v.a. ou definit la \textsc{Covariance} entre $X$ et $Y$ comme
	$\cov(X,Y)\overset{\text{def}}{=}\E\left[(X-\E(X))(Y-\E(Y))\right]=\E(XY)-\E(X)\E(Y)$.
\end{definition}
\fi
\pagebreak

% \tableofcontents

% insert your code here
%% !TEX encoding = UTF-8 Unicode
% !TEX TS-program = xelatex

\documentclass[french]{report}

%\usepackage[utf8]{inputenc}
%\usepackage[T1]{fontenc}
\usepackage{babel}


\newif\ifcomment
%\commenttrue # Show comments

\usepackage{physics}
\usepackage{amssymb}


\usepackage{amsthm}
% \usepackage{thmtools}
\usepackage{mathtools}
\usepackage{amsfonts}

\usepackage{color}

\usepackage{tikz}

\usepackage{geometry}
\geometry{a5paper, margin=0.1in, right=1cm}

\usepackage{dsfont}

\usepackage{graphicx}
\graphicspath{ {images/} }

\usepackage{faktor}

\usepackage{IEEEtrantools}
\usepackage{enumerate}   
\usepackage[PostScript=dvips]{"/Users/aware/Documents/Courses/diagrams"}


\newtheorem{theorem}{Théorème}[section]
\renewcommand{\thetheorem}{\arabic{theorem}}
\newtheorem{lemme}{Lemme}[section]
\renewcommand{\thelemme}{\arabic{lemme}}
\newtheorem{proposition}{Proposition}[section]
\renewcommand{\theproposition}{\arabic{proposition}}
\newtheorem{notations}{Notations}[section]
\newtheorem{problem}{Problème}[section]
\newtheorem{corollary}{Corollaire}[theorem]
\renewcommand{\thecorollary}{\arabic{corollary}}
\newtheorem{property}{Propriété}[section]
\newtheorem{objective}{Objectif}[section]

\theoremstyle{definition}
\newtheorem{definition}{Définition}[section]
\renewcommand{\thedefinition}{\arabic{definition}}
\newtheorem{exercise}{Exercice}[chapter]
\renewcommand{\theexercise}{\arabic{exercise}}
\newtheorem{example}{Exemple}[chapter]
\renewcommand{\theexample}{\arabic{example}}
\newtheorem*{solution}{Solution}
\newtheorem*{application}{Application}
\newtheorem*{notation}{Notation}
\newtheorem*{vocabulary}{Vocabulaire}
\newtheorem*{properties}{Propriétés}



\theoremstyle{remark}
\newtheorem*{remark}{Remarque}
\newtheorem*{rappel}{Rappel}


\usepackage{etoolbox}
\AtBeginEnvironment{exercise}{\small}
\AtBeginEnvironment{example}{\small}

\usepackage{cases}
\usepackage[red]{mypack}

\usepackage[framemethod=TikZ]{mdframed}

\definecolor{bg}{rgb}{0.4,0.25,0.95}
\definecolor{pagebg}{rgb}{0,0,0.5}
\surroundwithmdframed[
   topline=false,
   rightline=false,
   bottomline=false,
   leftmargin=\parindent,
   skipabove=8pt,
   skipbelow=8pt,
   linecolor=blue,
   innerbottommargin=10pt,
   % backgroundcolor=bg,font=\color{orange}\sffamily, fontcolor=white
]{definition}

\usepackage{empheq}
\usepackage[most]{tcolorbox}

\newtcbox{\mymath}[1][]{%
    nobeforeafter, math upper, tcbox raise base,
    enhanced, colframe=blue!30!black,
    colback=red!10, boxrule=1pt,
    #1}

\usepackage{unixode}


\DeclareMathOperator{\ord}{ord}
\DeclareMathOperator{\orb}{orb}
\DeclareMathOperator{\stab}{stab}
\DeclareMathOperator{\Stab}{stab}
\DeclareMathOperator{\ppcm}{ppcm}
\DeclareMathOperator{\conj}{Conj}
\DeclareMathOperator{\End}{End}
\DeclareMathOperator{\rot}{rot}
\DeclareMathOperator{\trs}{trace}
\DeclareMathOperator{\Ind}{Ind}
\DeclareMathOperator{\mat}{Mat}
\DeclareMathOperator{\id}{Id}
\DeclareMathOperator{\vect}{vect}
\DeclareMathOperator{\img}{img}
\DeclareMathOperator{\cov}{Cov}
\DeclareMathOperator{\dist}{dist}
\DeclareMathOperator{\irr}{Irr}
\DeclareMathOperator{\image}{Im}
\DeclareMathOperator{\pd}{\partial}
\DeclareMathOperator{\epi}{epi}
\DeclareMathOperator{\Argmin}{Argmin}
\DeclareMathOperator{\dom}{dom}
\DeclareMathOperator{\proj}{proj}
\DeclareMathOperator{\ctg}{ctg}
\DeclareMathOperator{\supp}{supp}
\DeclareMathOperator{\argmin}{argmin}
\DeclareMathOperator{\mult}{mult}
\DeclareMathOperator{\ch}{ch}
\DeclareMathOperator{\sh}{sh}
\DeclareMathOperator{\rang}{rang}
\DeclareMathOperator{\diam}{diam}
\DeclareMathOperator{\Epigraphe}{Epigraphe}




\usepackage{xcolor}
\everymath{\color{blue}}
%\everymath{\color[rgb]{0,1,1}}
%\pagecolor[rgb]{0,0,0.5}


\newcommand*{\pdtest}[3][]{\ensuremath{\frac{\partial^{#1} #2}{\partial #3}}}

\newcommand*{\deffunc}[6][]{\ensuremath{
\begin{array}{rcl}
#2 : #3 &\rightarrow& #4\\
#5 &\mapsto& #6
\end{array}
}}

\newcommand{\eqcolon}{\mathrel{\resizebox{\widthof{$\mathord{=}$}}{\height}{ $\!\!=\!\!\resizebox{1.2\width}{0.8\height}{\raisebox{0.23ex}{$\mathop{:}$}}\!\!$ }}}
\newcommand{\coloneq}{\mathrel{\resizebox{\widthof{$\mathord{=}$}}{\height}{ $\!\!\resizebox{1.2\width}{0.8\height}{\raisebox{0.23ex}{$\mathop{:}$}}\!\!=\!\!$ }}}
\newcommand{\eqcolonl}{\ensuremath{\mathrel{=\!\!\mathop{:}}}}
\newcommand{\coloneql}{\ensuremath{\mathrel{\mathop{:} \!\! =}}}
\newcommand{\vc}[1]{% inline column vector
  \left(\begin{smallmatrix}#1\end{smallmatrix}\right)%
}
\newcommand{\vr}[1]{% inline row vector
  \begin{smallmatrix}(\,#1\,)\end{smallmatrix}%
}
\makeatletter
\newcommand*{\defeq}{\ =\mathrel{\rlap{%
                     \raisebox{0.3ex}{$\m@th\cdot$}}%
                     \raisebox{-0.3ex}{$\m@th\cdot$}}%
                     }
\makeatother

\newcommand{\mathcircle}[1]{% inline row vector
 \overset{\circ}{#1}
}
\newcommand{\ulim}{% low limit
 \underline{\lim}
}
\newcommand{\ssi}{% iff
\iff
}
\newcommand{\ps}[2]{
\expval{#1 | #2}
}
\newcommand{\df}[1]{
\mqty{#1}
}
\newcommand{\n}[1]{
\norm{#1}
}
\newcommand{\sys}[1]{
\left\{\smqty{#1}\right.
}


\newcommand{\eqdef}{\ensuremath{\overset{\text{def}}=}}


\def\Circlearrowright{\ensuremath{%
  \rotatebox[origin=c]{230}{$\circlearrowright$}}}

\newcommand\ct[1]{\text{\rmfamily\upshape #1}}
\newcommand\question[1]{ {\color{red} ...!? \small #1}}
\newcommand\caz[1]{\left\{\begin{array} #1 \end{array}\right.}
\newcommand\const{\text{\rmfamily\upshape const}}
\newcommand\toP{ \overset{\pro}{\to}}
\newcommand\toPP{ \overset{\text{PP}}{\to}}
\newcommand{\oeq}{\mathrel{\text{\textcircled{$=$}}}}





\usepackage{xcolor}
% \usepackage[normalem]{ulem}
\usepackage{lipsum}
\makeatletter
% \newcommand\colorwave[1][blue]{\bgroup \markoverwith{\lower3.5\p@\hbox{\sixly \textcolor{#1}{\char58}}}\ULon}
%\font\sixly=lasy6 % does not re-load if already loaded, so no memory problem.

\newmdtheoremenv[
linewidth= 1pt,linecolor= blue,%
leftmargin=20,rightmargin=20,innertopmargin=0pt, innerrightmargin=40,%
tikzsetting = { draw=lightgray, line width = 0.3pt,dashed,%
dash pattern = on 15pt off 3pt},%
splittopskip=\topskip,skipbelow=\baselineskip,%
skipabove=\baselineskip,ntheorem,roundcorner=0pt,
% backgroundcolor=pagebg,font=\color{orange}\sffamily, fontcolor=white
]{examplebox}{Exemple}[section]



\newcommand\R{\mathbb{R}}
\newcommand\Z{\mathbb{Z}}
\newcommand\N{\mathbb{N}}
\newcommand\E{\mathbb{E}}
\newcommand\F{\mathcal{F}}
\newcommand\cH{\mathcal{H}}
\newcommand\V{\mathbb{V}}
\newcommand\dmo{ ^{-1} }
\newcommand\kapa{\kappa}
\newcommand\im{Im}
\newcommand\hs{\mathcal{H}}





\usepackage{soul}

\makeatletter
\newcommand*{\whiten}[1]{\llap{\textcolor{white}{{\the\SOUL@token}}\hspace{#1pt}}}
\DeclareRobustCommand*\myul{%
    \def\SOUL@everyspace{\underline{\space}\kern\z@}%
    \def\SOUL@everytoken{%
     \setbox0=\hbox{\the\SOUL@token}%
     \ifdim\dp0>\z@
        \raisebox{\dp0}{\underline{\phantom{\the\SOUL@token}}}%
        \whiten{1}\whiten{0}%
        \whiten{-1}\whiten{-2}%
        \llap{\the\SOUL@token}%
     \else
        \underline{\the\SOUL@token}%
     \fi}%
\SOUL@}
\makeatother

\newcommand*{\demp}{\fontfamily{lmtt}\selectfont}

\DeclareTextFontCommand{\textdemp}{\demp}

\begin{document}

\ifcomment
Multiline
comment
\fi
\ifcomment
\myul{Typesetting test}
% \color[rgb]{1,1,1}
$∑_i^n≠ 60º±∞π∆¬≈√j∫h≤≥µ$

$\CR \R\pro\ind\pro\gS\pro
\mqty[a&b\\c&d]$
$\pro\mathbb{P}$
$\dd{x}$

  \[
    \alpha(x)=\left\{
                \begin{array}{ll}
                  x\\
                  \frac{1}{1+e^{-kx}}\\
                  \frac{e^x-e^{-x}}{e^x+e^{-x}}
                \end{array}
              \right.
  \]

  $\expval{x}$
  
  $\chi_\rho(ghg\dmo)=\Tr(\rho_{ghg\dmo})=\Tr(\rho_g\circ\rho_h\circ\rho\dmo_g)=\Tr(\rho_h)\overset{\mbox{\scalebox{0.5}{$\Tr(AB)=\Tr(BA)$}}}{=}\chi_\rho(h)$
  	$\mathop{\oplus}_{\substack{x\in X}}$

$\mat(\rho_g)=(a_{ij}(g))_{\scriptsize \substack{1\leq i\leq d \\ 1\leq j\leq d}}$ et $\mat(\rho'_g)=(a'_{ij}(g))_{\scriptsize \substack{1\leq i'\leq d' \\ 1\leq j'\leq d'}}$



\[\int_a^b{\mathbb{R}^2}g(u, v)\dd{P_{XY}}(u, v)=\iint g(u,v) f_{XY}(u, v)\dd \lambda(u) \dd \lambda(v)\]
$$\lim_{x\to\infty} f(x)$$	
$$\iiiint_V \mu(t,u,v,w) \,dt\,du\,dv\,dw$$
$$\sum_{n=1}^{\infty} 2^{-n} = 1$$	
\begin{definition}
	Si $X$ et $Y$ sont 2 v.a. ou definit la \textsc{Covariance} entre $X$ et $Y$ comme
	$\cov(X,Y)\overset{\text{def}}{=}\E\left[(X-\E(X))(Y-\E(Y))\right]=\E(XY)-\E(X)\E(Y)$.
\end{definition}
\fi
\pagebreak

% \tableofcontents

% insert your code here
%\input{./algebra/main.tex}
%\input{./geometrie-differentielle/main.tex}
%\input{./probabilite/main.tex}
%\input{./analyse-fonctionnelle/main.tex}
% \input{./Analyse-convexe-et-dualite-en-optimisation/main.tex}
%\input{./tikz/main.tex}
%\input{./Theorie-du-distributions/main.tex}
%\input{./optimisation/mine.tex}
 \input{./modelisation/main.tex}

% yves.aubry@univ-tln.fr : algebra

\end{document}

%% !TEX encoding = UTF-8 Unicode
% !TEX TS-program = xelatex

\documentclass[french]{report}

%\usepackage[utf8]{inputenc}
%\usepackage[T1]{fontenc}
\usepackage{babel}


\newif\ifcomment
%\commenttrue # Show comments

\usepackage{physics}
\usepackage{amssymb}


\usepackage{amsthm}
% \usepackage{thmtools}
\usepackage{mathtools}
\usepackage{amsfonts}

\usepackage{color}

\usepackage{tikz}

\usepackage{geometry}
\geometry{a5paper, margin=0.1in, right=1cm}

\usepackage{dsfont}

\usepackage{graphicx}
\graphicspath{ {images/} }

\usepackage{faktor}

\usepackage{IEEEtrantools}
\usepackage{enumerate}   
\usepackage[PostScript=dvips]{"/Users/aware/Documents/Courses/diagrams"}


\newtheorem{theorem}{Théorème}[section]
\renewcommand{\thetheorem}{\arabic{theorem}}
\newtheorem{lemme}{Lemme}[section]
\renewcommand{\thelemme}{\arabic{lemme}}
\newtheorem{proposition}{Proposition}[section]
\renewcommand{\theproposition}{\arabic{proposition}}
\newtheorem{notations}{Notations}[section]
\newtheorem{problem}{Problème}[section]
\newtheorem{corollary}{Corollaire}[theorem]
\renewcommand{\thecorollary}{\arabic{corollary}}
\newtheorem{property}{Propriété}[section]
\newtheorem{objective}{Objectif}[section]

\theoremstyle{definition}
\newtheorem{definition}{Définition}[section]
\renewcommand{\thedefinition}{\arabic{definition}}
\newtheorem{exercise}{Exercice}[chapter]
\renewcommand{\theexercise}{\arabic{exercise}}
\newtheorem{example}{Exemple}[chapter]
\renewcommand{\theexample}{\arabic{example}}
\newtheorem*{solution}{Solution}
\newtheorem*{application}{Application}
\newtheorem*{notation}{Notation}
\newtheorem*{vocabulary}{Vocabulaire}
\newtheorem*{properties}{Propriétés}



\theoremstyle{remark}
\newtheorem*{remark}{Remarque}
\newtheorem*{rappel}{Rappel}


\usepackage{etoolbox}
\AtBeginEnvironment{exercise}{\small}
\AtBeginEnvironment{example}{\small}

\usepackage{cases}
\usepackage[red]{mypack}

\usepackage[framemethod=TikZ]{mdframed}

\definecolor{bg}{rgb}{0.4,0.25,0.95}
\definecolor{pagebg}{rgb}{0,0,0.5}
\surroundwithmdframed[
   topline=false,
   rightline=false,
   bottomline=false,
   leftmargin=\parindent,
   skipabove=8pt,
   skipbelow=8pt,
   linecolor=blue,
   innerbottommargin=10pt,
   % backgroundcolor=bg,font=\color{orange}\sffamily, fontcolor=white
]{definition}

\usepackage{empheq}
\usepackage[most]{tcolorbox}

\newtcbox{\mymath}[1][]{%
    nobeforeafter, math upper, tcbox raise base,
    enhanced, colframe=blue!30!black,
    colback=red!10, boxrule=1pt,
    #1}

\usepackage{unixode}


\DeclareMathOperator{\ord}{ord}
\DeclareMathOperator{\orb}{orb}
\DeclareMathOperator{\stab}{stab}
\DeclareMathOperator{\Stab}{stab}
\DeclareMathOperator{\ppcm}{ppcm}
\DeclareMathOperator{\conj}{Conj}
\DeclareMathOperator{\End}{End}
\DeclareMathOperator{\rot}{rot}
\DeclareMathOperator{\trs}{trace}
\DeclareMathOperator{\Ind}{Ind}
\DeclareMathOperator{\mat}{Mat}
\DeclareMathOperator{\id}{Id}
\DeclareMathOperator{\vect}{vect}
\DeclareMathOperator{\img}{img}
\DeclareMathOperator{\cov}{Cov}
\DeclareMathOperator{\dist}{dist}
\DeclareMathOperator{\irr}{Irr}
\DeclareMathOperator{\image}{Im}
\DeclareMathOperator{\pd}{\partial}
\DeclareMathOperator{\epi}{epi}
\DeclareMathOperator{\Argmin}{Argmin}
\DeclareMathOperator{\dom}{dom}
\DeclareMathOperator{\proj}{proj}
\DeclareMathOperator{\ctg}{ctg}
\DeclareMathOperator{\supp}{supp}
\DeclareMathOperator{\argmin}{argmin}
\DeclareMathOperator{\mult}{mult}
\DeclareMathOperator{\ch}{ch}
\DeclareMathOperator{\sh}{sh}
\DeclareMathOperator{\rang}{rang}
\DeclareMathOperator{\diam}{diam}
\DeclareMathOperator{\Epigraphe}{Epigraphe}




\usepackage{xcolor}
\everymath{\color{blue}}
%\everymath{\color[rgb]{0,1,1}}
%\pagecolor[rgb]{0,0,0.5}


\newcommand*{\pdtest}[3][]{\ensuremath{\frac{\partial^{#1} #2}{\partial #3}}}

\newcommand*{\deffunc}[6][]{\ensuremath{
\begin{array}{rcl}
#2 : #3 &\rightarrow& #4\\
#5 &\mapsto& #6
\end{array}
}}

\newcommand{\eqcolon}{\mathrel{\resizebox{\widthof{$\mathord{=}$}}{\height}{ $\!\!=\!\!\resizebox{1.2\width}{0.8\height}{\raisebox{0.23ex}{$\mathop{:}$}}\!\!$ }}}
\newcommand{\coloneq}{\mathrel{\resizebox{\widthof{$\mathord{=}$}}{\height}{ $\!\!\resizebox{1.2\width}{0.8\height}{\raisebox{0.23ex}{$\mathop{:}$}}\!\!=\!\!$ }}}
\newcommand{\eqcolonl}{\ensuremath{\mathrel{=\!\!\mathop{:}}}}
\newcommand{\coloneql}{\ensuremath{\mathrel{\mathop{:} \!\! =}}}
\newcommand{\vc}[1]{% inline column vector
  \left(\begin{smallmatrix}#1\end{smallmatrix}\right)%
}
\newcommand{\vr}[1]{% inline row vector
  \begin{smallmatrix}(\,#1\,)\end{smallmatrix}%
}
\makeatletter
\newcommand*{\defeq}{\ =\mathrel{\rlap{%
                     \raisebox{0.3ex}{$\m@th\cdot$}}%
                     \raisebox{-0.3ex}{$\m@th\cdot$}}%
                     }
\makeatother

\newcommand{\mathcircle}[1]{% inline row vector
 \overset{\circ}{#1}
}
\newcommand{\ulim}{% low limit
 \underline{\lim}
}
\newcommand{\ssi}{% iff
\iff
}
\newcommand{\ps}[2]{
\expval{#1 | #2}
}
\newcommand{\df}[1]{
\mqty{#1}
}
\newcommand{\n}[1]{
\norm{#1}
}
\newcommand{\sys}[1]{
\left\{\smqty{#1}\right.
}


\newcommand{\eqdef}{\ensuremath{\overset{\text{def}}=}}


\def\Circlearrowright{\ensuremath{%
  \rotatebox[origin=c]{230}{$\circlearrowright$}}}

\newcommand\ct[1]{\text{\rmfamily\upshape #1}}
\newcommand\question[1]{ {\color{red} ...!? \small #1}}
\newcommand\caz[1]{\left\{\begin{array} #1 \end{array}\right.}
\newcommand\const{\text{\rmfamily\upshape const}}
\newcommand\toP{ \overset{\pro}{\to}}
\newcommand\toPP{ \overset{\text{PP}}{\to}}
\newcommand{\oeq}{\mathrel{\text{\textcircled{$=$}}}}





\usepackage{xcolor}
% \usepackage[normalem]{ulem}
\usepackage{lipsum}
\makeatletter
% \newcommand\colorwave[1][blue]{\bgroup \markoverwith{\lower3.5\p@\hbox{\sixly \textcolor{#1}{\char58}}}\ULon}
%\font\sixly=lasy6 % does not re-load if already loaded, so no memory problem.

\newmdtheoremenv[
linewidth= 1pt,linecolor= blue,%
leftmargin=20,rightmargin=20,innertopmargin=0pt, innerrightmargin=40,%
tikzsetting = { draw=lightgray, line width = 0.3pt,dashed,%
dash pattern = on 15pt off 3pt},%
splittopskip=\topskip,skipbelow=\baselineskip,%
skipabove=\baselineskip,ntheorem,roundcorner=0pt,
% backgroundcolor=pagebg,font=\color{orange}\sffamily, fontcolor=white
]{examplebox}{Exemple}[section]



\newcommand\R{\mathbb{R}}
\newcommand\Z{\mathbb{Z}}
\newcommand\N{\mathbb{N}}
\newcommand\E{\mathbb{E}}
\newcommand\F{\mathcal{F}}
\newcommand\cH{\mathcal{H}}
\newcommand\V{\mathbb{V}}
\newcommand\dmo{ ^{-1} }
\newcommand\kapa{\kappa}
\newcommand\im{Im}
\newcommand\hs{\mathcal{H}}





\usepackage{soul}

\makeatletter
\newcommand*{\whiten}[1]{\llap{\textcolor{white}{{\the\SOUL@token}}\hspace{#1pt}}}
\DeclareRobustCommand*\myul{%
    \def\SOUL@everyspace{\underline{\space}\kern\z@}%
    \def\SOUL@everytoken{%
     \setbox0=\hbox{\the\SOUL@token}%
     \ifdim\dp0>\z@
        \raisebox{\dp0}{\underline{\phantom{\the\SOUL@token}}}%
        \whiten{1}\whiten{0}%
        \whiten{-1}\whiten{-2}%
        \llap{\the\SOUL@token}%
     \else
        \underline{\the\SOUL@token}%
     \fi}%
\SOUL@}
\makeatother

\newcommand*{\demp}{\fontfamily{lmtt}\selectfont}

\DeclareTextFontCommand{\textdemp}{\demp}

\begin{document}

\ifcomment
Multiline
comment
\fi
\ifcomment
\myul{Typesetting test}
% \color[rgb]{1,1,1}
$∑_i^n≠ 60º±∞π∆¬≈√j∫h≤≥µ$

$\CR \R\pro\ind\pro\gS\pro
\mqty[a&b\\c&d]$
$\pro\mathbb{P}$
$\dd{x}$

  \[
    \alpha(x)=\left\{
                \begin{array}{ll}
                  x\\
                  \frac{1}{1+e^{-kx}}\\
                  \frac{e^x-e^{-x}}{e^x+e^{-x}}
                \end{array}
              \right.
  \]

  $\expval{x}$
  
  $\chi_\rho(ghg\dmo)=\Tr(\rho_{ghg\dmo})=\Tr(\rho_g\circ\rho_h\circ\rho\dmo_g)=\Tr(\rho_h)\overset{\mbox{\scalebox{0.5}{$\Tr(AB)=\Tr(BA)$}}}{=}\chi_\rho(h)$
  	$\mathop{\oplus}_{\substack{x\in X}}$

$\mat(\rho_g)=(a_{ij}(g))_{\scriptsize \substack{1\leq i\leq d \\ 1\leq j\leq d}}$ et $\mat(\rho'_g)=(a'_{ij}(g))_{\scriptsize \substack{1\leq i'\leq d' \\ 1\leq j'\leq d'}}$



\[\int_a^b{\mathbb{R}^2}g(u, v)\dd{P_{XY}}(u, v)=\iint g(u,v) f_{XY}(u, v)\dd \lambda(u) \dd \lambda(v)\]
$$\lim_{x\to\infty} f(x)$$	
$$\iiiint_V \mu(t,u,v,w) \,dt\,du\,dv\,dw$$
$$\sum_{n=1}^{\infty} 2^{-n} = 1$$	
\begin{definition}
	Si $X$ et $Y$ sont 2 v.a. ou definit la \textsc{Covariance} entre $X$ et $Y$ comme
	$\cov(X,Y)\overset{\text{def}}{=}\E\left[(X-\E(X))(Y-\E(Y))\right]=\E(XY)-\E(X)\E(Y)$.
\end{definition}
\fi
\pagebreak

% \tableofcontents

% insert your code here
%\input{./algebra/main.tex}
%\input{./geometrie-differentielle/main.tex}
%\input{./probabilite/main.tex}
%\input{./analyse-fonctionnelle/main.tex}
% \input{./Analyse-convexe-et-dualite-en-optimisation/main.tex}
%\input{./tikz/main.tex}
%\input{./Theorie-du-distributions/main.tex}
%\input{./optimisation/mine.tex}
 \input{./modelisation/main.tex}

% yves.aubry@univ-tln.fr : algebra

\end{document}

%% !TEX encoding = UTF-8 Unicode
% !TEX TS-program = xelatex

\documentclass[french]{report}

%\usepackage[utf8]{inputenc}
%\usepackage[T1]{fontenc}
\usepackage{babel}


\newif\ifcomment
%\commenttrue # Show comments

\usepackage{physics}
\usepackage{amssymb}


\usepackage{amsthm}
% \usepackage{thmtools}
\usepackage{mathtools}
\usepackage{amsfonts}

\usepackage{color}

\usepackage{tikz}

\usepackage{geometry}
\geometry{a5paper, margin=0.1in, right=1cm}

\usepackage{dsfont}

\usepackage{graphicx}
\graphicspath{ {images/} }

\usepackage{faktor}

\usepackage{IEEEtrantools}
\usepackage{enumerate}   
\usepackage[PostScript=dvips]{"/Users/aware/Documents/Courses/diagrams"}


\newtheorem{theorem}{Théorème}[section]
\renewcommand{\thetheorem}{\arabic{theorem}}
\newtheorem{lemme}{Lemme}[section]
\renewcommand{\thelemme}{\arabic{lemme}}
\newtheorem{proposition}{Proposition}[section]
\renewcommand{\theproposition}{\arabic{proposition}}
\newtheorem{notations}{Notations}[section]
\newtheorem{problem}{Problème}[section]
\newtheorem{corollary}{Corollaire}[theorem]
\renewcommand{\thecorollary}{\arabic{corollary}}
\newtheorem{property}{Propriété}[section]
\newtheorem{objective}{Objectif}[section]

\theoremstyle{definition}
\newtheorem{definition}{Définition}[section]
\renewcommand{\thedefinition}{\arabic{definition}}
\newtheorem{exercise}{Exercice}[chapter]
\renewcommand{\theexercise}{\arabic{exercise}}
\newtheorem{example}{Exemple}[chapter]
\renewcommand{\theexample}{\arabic{example}}
\newtheorem*{solution}{Solution}
\newtheorem*{application}{Application}
\newtheorem*{notation}{Notation}
\newtheorem*{vocabulary}{Vocabulaire}
\newtheorem*{properties}{Propriétés}



\theoremstyle{remark}
\newtheorem*{remark}{Remarque}
\newtheorem*{rappel}{Rappel}


\usepackage{etoolbox}
\AtBeginEnvironment{exercise}{\small}
\AtBeginEnvironment{example}{\small}

\usepackage{cases}
\usepackage[red]{mypack}

\usepackage[framemethod=TikZ]{mdframed}

\definecolor{bg}{rgb}{0.4,0.25,0.95}
\definecolor{pagebg}{rgb}{0,0,0.5}
\surroundwithmdframed[
   topline=false,
   rightline=false,
   bottomline=false,
   leftmargin=\parindent,
   skipabove=8pt,
   skipbelow=8pt,
   linecolor=blue,
   innerbottommargin=10pt,
   % backgroundcolor=bg,font=\color{orange}\sffamily, fontcolor=white
]{definition}

\usepackage{empheq}
\usepackage[most]{tcolorbox}

\newtcbox{\mymath}[1][]{%
    nobeforeafter, math upper, tcbox raise base,
    enhanced, colframe=blue!30!black,
    colback=red!10, boxrule=1pt,
    #1}

\usepackage{unixode}


\DeclareMathOperator{\ord}{ord}
\DeclareMathOperator{\orb}{orb}
\DeclareMathOperator{\stab}{stab}
\DeclareMathOperator{\Stab}{stab}
\DeclareMathOperator{\ppcm}{ppcm}
\DeclareMathOperator{\conj}{Conj}
\DeclareMathOperator{\End}{End}
\DeclareMathOperator{\rot}{rot}
\DeclareMathOperator{\trs}{trace}
\DeclareMathOperator{\Ind}{Ind}
\DeclareMathOperator{\mat}{Mat}
\DeclareMathOperator{\id}{Id}
\DeclareMathOperator{\vect}{vect}
\DeclareMathOperator{\img}{img}
\DeclareMathOperator{\cov}{Cov}
\DeclareMathOperator{\dist}{dist}
\DeclareMathOperator{\irr}{Irr}
\DeclareMathOperator{\image}{Im}
\DeclareMathOperator{\pd}{\partial}
\DeclareMathOperator{\epi}{epi}
\DeclareMathOperator{\Argmin}{Argmin}
\DeclareMathOperator{\dom}{dom}
\DeclareMathOperator{\proj}{proj}
\DeclareMathOperator{\ctg}{ctg}
\DeclareMathOperator{\supp}{supp}
\DeclareMathOperator{\argmin}{argmin}
\DeclareMathOperator{\mult}{mult}
\DeclareMathOperator{\ch}{ch}
\DeclareMathOperator{\sh}{sh}
\DeclareMathOperator{\rang}{rang}
\DeclareMathOperator{\diam}{diam}
\DeclareMathOperator{\Epigraphe}{Epigraphe}




\usepackage{xcolor}
\everymath{\color{blue}}
%\everymath{\color[rgb]{0,1,1}}
%\pagecolor[rgb]{0,0,0.5}


\newcommand*{\pdtest}[3][]{\ensuremath{\frac{\partial^{#1} #2}{\partial #3}}}

\newcommand*{\deffunc}[6][]{\ensuremath{
\begin{array}{rcl}
#2 : #3 &\rightarrow& #4\\
#5 &\mapsto& #6
\end{array}
}}

\newcommand{\eqcolon}{\mathrel{\resizebox{\widthof{$\mathord{=}$}}{\height}{ $\!\!=\!\!\resizebox{1.2\width}{0.8\height}{\raisebox{0.23ex}{$\mathop{:}$}}\!\!$ }}}
\newcommand{\coloneq}{\mathrel{\resizebox{\widthof{$\mathord{=}$}}{\height}{ $\!\!\resizebox{1.2\width}{0.8\height}{\raisebox{0.23ex}{$\mathop{:}$}}\!\!=\!\!$ }}}
\newcommand{\eqcolonl}{\ensuremath{\mathrel{=\!\!\mathop{:}}}}
\newcommand{\coloneql}{\ensuremath{\mathrel{\mathop{:} \!\! =}}}
\newcommand{\vc}[1]{% inline column vector
  \left(\begin{smallmatrix}#1\end{smallmatrix}\right)%
}
\newcommand{\vr}[1]{% inline row vector
  \begin{smallmatrix}(\,#1\,)\end{smallmatrix}%
}
\makeatletter
\newcommand*{\defeq}{\ =\mathrel{\rlap{%
                     \raisebox{0.3ex}{$\m@th\cdot$}}%
                     \raisebox{-0.3ex}{$\m@th\cdot$}}%
                     }
\makeatother

\newcommand{\mathcircle}[1]{% inline row vector
 \overset{\circ}{#1}
}
\newcommand{\ulim}{% low limit
 \underline{\lim}
}
\newcommand{\ssi}{% iff
\iff
}
\newcommand{\ps}[2]{
\expval{#1 | #2}
}
\newcommand{\df}[1]{
\mqty{#1}
}
\newcommand{\n}[1]{
\norm{#1}
}
\newcommand{\sys}[1]{
\left\{\smqty{#1}\right.
}


\newcommand{\eqdef}{\ensuremath{\overset{\text{def}}=}}


\def\Circlearrowright{\ensuremath{%
  \rotatebox[origin=c]{230}{$\circlearrowright$}}}

\newcommand\ct[1]{\text{\rmfamily\upshape #1}}
\newcommand\question[1]{ {\color{red} ...!? \small #1}}
\newcommand\caz[1]{\left\{\begin{array} #1 \end{array}\right.}
\newcommand\const{\text{\rmfamily\upshape const}}
\newcommand\toP{ \overset{\pro}{\to}}
\newcommand\toPP{ \overset{\text{PP}}{\to}}
\newcommand{\oeq}{\mathrel{\text{\textcircled{$=$}}}}





\usepackage{xcolor}
% \usepackage[normalem]{ulem}
\usepackage{lipsum}
\makeatletter
% \newcommand\colorwave[1][blue]{\bgroup \markoverwith{\lower3.5\p@\hbox{\sixly \textcolor{#1}{\char58}}}\ULon}
%\font\sixly=lasy6 % does not re-load if already loaded, so no memory problem.

\newmdtheoremenv[
linewidth= 1pt,linecolor= blue,%
leftmargin=20,rightmargin=20,innertopmargin=0pt, innerrightmargin=40,%
tikzsetting = { draw=lightgray, line width = 0.3pt,dashed,%
dash pattern = on 15pt off 3pt},%
splittopskip=\topskip,skipbelow=\baselineskip,%
skipabove=\baselineskip,ntheorem,roundcorner=0pt,
% backgroundcolor=pagebg,font=\color{orange}\sffamily, fontcolor=white
]{examplebox}{Exemple}[section]



\newcommand\R{\mathbb{R}}
\newcommand\Z{\mathbb{Z}}
\newcommand\N{\mathbb{N}}
\newcommand\E{\mathbb{E}}
\newcommand\F{\mathcal{F}}
\newcommand\cH{\mathcal{H}}
\newcommand\V{\mathbb{V}}
\newcommand\dmo{ ^{-1} }
\newcommand\kapa{\kappa}
\newcommand\im{Im}
\newcommand\hs{\mathcal{H}}





\usepackage{soul}

\makeatletter
\newcommand*{\whiten}[1]{\llap{\textcolor{white}{{\the\SOUL@token}}\hspace{#1pt}}}
\DeclareRobustCommand*\myul{%
    \def\SOUL@everyspace{\underline{\space}\kern\z@}%
    \def\SOUL@everytoken{%
     \setbox0=\hbox{\the\SOUL@token}%
     \ifdim\dp0>\z@
        \raisebox{\dp0}{\underline{\phantom{\the\SOUL@token}}}%
        \whiten{1}\whiten{0}%
        \whiten{-1}\whiten{-2}%
        \llap{\the\SOUL@token}%
     \else
        \underline{\the\SOUL@token}%
     \fi}%
\SOUL@}
\makeatother

\newcommand*{\demp}{\fontfamily{lmtt}\selectfont}

\DeclareTextFontCommand{\textdemp}{\demp}

\begin{document}

\ifcomment
Multiline
comment
\fi
\ifcomment
\myul{Typesetting test}
% \color[rgb]{1,1,1}
$∑_i^n≠ 60º±∞π∆¬≈√j∫h≤≥µ$

$\CR \R\pro\ind\pro\gS\pro
\mqty[a&b\\c&d]$
$\pro\mathbb{P}$
$\dd{x}$

  \[
    \alpha(x)=\left\{
                \begin{array}{ll}
                  x\\
                  \frac{1}{1+e^{-kx}}\\
                  \frac{e^x-e^{-x}}{e^x+e^{-x}}
                \end{array}
              \right.
  \]

  $\expval{x}$
  
  $\chi_\rho(ghg\dmo)=\Tr(\rho_{ghg\dmo})=\Tr(\rho_g\circ\rho_h\circ\rho\dmo_g)=\Tr(\rho_h)\overset{\mbox{\scalebox{0.5}{$\Tr(AB)=\Tr(BA)$}}}{=}\chi_\rho(h)$
  	$\mathop{\oplus}_{\substack{x\in X}}$

$\mat(\rho_g)=(a_{ij}(g))_{\scriptsize \substack{1\leq i\leq d \\ 1\leq j\leq d}}$ et $\mat(\rho'_g)=(a'_{ij}(g))_{\scriptsize \substack{1\leq i'\leq d' \\ 1\leq j'\leq d'}}$



\[\int_a^b{\mathbb{R}^2}g(u, v)\dd{P_{XY}}(u, v)=\iint g(u,v) f_{XY}(u, v)\dd \lambda(u) \dd \lambda(v)\]
$$\lim_{x\to\infty} f(x)$$	
$$\iiiint_V \mu(t,u,v,w) \,dt\,du\,dv\,dw$$
$$\sum_{n=1}^{\infty} 2^{-n} = 1$$	
\begin{definition}
	Si $X$ et $Y$ sont 2 v.a. ou definit la \textsc{Covariance} entre $X$ et $Y$ comme
	$\cov(X,Y)\overset{\text{def}}{=}\E\left[(X-\E(X))(Y-\E(Y))\right]=\E(XY)-\E(X)\E(Y)$.
\end{definition}
\fi
\pagebreak

% \tableofcontents

% insert your code here
%\input{./algebra/main.tex}
%\input{./geometrie-differentielle/main.tex}
%\input{./probabilite/main.tex}
%\input{./analyse-fonctionnelle/main.tex}
% \input{./Analyse-convexe-et-dualite-en-optimisation/main.tex}
%\input{./tikz/main.tex}
%\input{./Theorie-du-distributions/main.tex}
%\input{./optimisation/mine.tex}
 \input{./modelisation/main.tex}

% yves.aubry@univ-tln.fr : algebra

\end{document}

%% !TEX encoding = UTF-8 Unicode
% !TEX TS-program = xelatex

\documentclass[french]{report}

%\usepackage[utf8]{inputenc}
%\usepackage[T1]{fontenc}
\usepackage{babel}


\newif\ifcomment
%\commenttrue # Show comments

\usepackage{physics}
\usepackage{amssymb}


\usepackage{amsthm}
% \usepackage{thmtools}
\usepackage{mathtools}
\usepackage{amsfonts}

\usepackage{color}

\usepackage{tikz}

\usepackage{geometry}
\geometry{a5paper, margin=0.1in, right=1cm}

\usepackage{dsfont}

\usepackage{graphicx}
\graphicspath{ {images/} }

\usepackage{faktor}

\usepackage{IEEEtrantools}
\usepackage{enumerate}   
\usepackage[PostScript=dvips]{"/Users/aware/Documents/Courses/diagrams"}


\newtheorem{theorem}{Théorème}[section]
\renewcommand{\thetheorem}{\arabic{theorem}}
\newtheorem{lemme}{Lemme}[section]
\renewcommand{\thelemme}{\arabic{lemme}}
\newtheorem{proposition}{Proposition}[section]
\renewcommand{\theproposition}{\arabic{proposition}}
\newtheorem{notations}{Notations}[section]
\newtheorem{problem}{Problème}[section]
\newtheorem{corollary}{Corollaire}[theorem]
\renewcommand{\thecorollary}{\arabic{corollary}}
\newtheorem{property}{Propriété}[section]
\newtheorem{objective}{Objectif}[section]

\theoremstyle{definition}
\newtheorem{definition}{Définition}[section]
\renewcommand{\thedefinition}{\arabic{definition}}
\newtheorem{exercise}{Exercice}[chapter]
\renewcommand{\theexercise}{\arabic{exercise}}
\newtheorem{example}{Exemple}[chapter]
\renewcommand{\theexample}{\arabic{example}}
\newtheorem*{solution}{Solution}
\newtheorem*{application}{Application}
\newtheorem*{notation}{Notation}
\newtheorem*{vocabulary}{Vocabulaire}
\newtheorem*{properties}{Propriétés}



\theoremstyle{remark}
\newtheorem*{remark}{Remarque}
\newtheorem*{rappel}{Rappel}


\usepackage{etoolbox}
\AtBeginEnvironment{exercise}{\small}
\AtBeginEnvironment{example}{\small}

\usepackage{cases}
\usepackage[red]{mypack}

\usepackage[framemethod=TikZ]{mdframed}

\definecolor{bg}{rgb}{0.4,0.25,0.95}
\definecolor{pagebg}{rgb}{0,0,0.5}
\surroundwithmdframed[
   topline=false,
   rightline=false,
   bottomline=false,
   leftmargin=\parindent,
   skipabove=8pt,
   skipbelow=8pt,
   linecolor=blue,
   innerbottommargin=10pt,
   % backgroundcolor=bg,font=\color{orange}\sffamily, fontcolor=white
]{definition}

\usepackage{empheq}
\usepackage[most]{tcolorbox}

\newtcbox{\mymath}[1][]{%
    nobeforeafter, math upper, tcbox raise base,
    enhanced, colframe=blue!30!black,
    colback=red!10, boxrule=1pt,
    #1}

\usepackage{unixode}


\DeclareMathOperator{\ord}{ord}
\DeclareMathOperator{\orb}{orb}
\DeclareMathOperator{\stab}{stab}
\DeclareMathOperator{\Stab}{stab}
\DeclareMathOperator{\ppcm}{ppcm}
\DeclareMathOperator{\conj}{Conj}
\DeclareMathOperator{\End}{End}
\DeclareMathOperator{\rot}{rot}
\DeclareMathOperator{\trs}{trace}
\DeclareMathOperator{\Ind}{Ind}
\DeclareMathOperator{\mat}{Mat}
\DeclareMathOperator{\id}{Id}
\DeclareMathOperator{\vect}{vect}
\DeclareMathOperator{\img}{img}
\DeclareMathOperator{\cov}{Cov}
\DeclareMathOperator{\dist}{dist}
\DeclareMathOperator{\irr}{Irr}
\DeclareMathOperator{\image}{Im}
\DeclareMathOperator{\pd}{\partial}
\DeclareMathOperator{\epi}{epi}
\DeclareMathOperator{\Argmin}{Argmin}
\DeclareMathOperator{\dom}{dom}
\DeclareMathOperator{\proj}{proj}
\DeclareMathOperator{\ctg}{ctg}
\DeclareMathOperator{\supp}{supp}
\DeclareMathOperator{\argmin}{argmin}
\DeclareMathOperator{\mult}{mult}
\DeclareMathOperator{\ch}{ch}
\DeclareMathOperator{\sh}{sh}
\DeclareMathOperator{\rang}{rang}
\DeclareMathOperator{\diam}{diam}
\DeclareMathOperator{\Epigraphe}{Epigraphe}




\usepackage{xcolor}
\everymath{\color{blue}}
%\everymath{\color[rgb]{0,1,1}}
%\pagecolor[rgb]{0,0,0.5}


\newcommand*{\pdtest}[3][]{\ensuremath{\frac{\partial^{#1} #2}{\partial #3}}}

\newcommand*{\deffunc}[6][]{\ensuremath{
\begin{array}{rcl}
#2 : #3 &\rightarrow& #4\\
#5 &\mapsto& #6
\end{array}
}}

\newcommand{\eqcolon}{\mathrel{\resizebox{\widthof{$\mathord{=}$}}{\height}{ $\!\!=\!\!\resizebox{1.2\width}{0.8\height}{\raisebox{0.23ex}{$\mathop{:}$}}\!\!$ }}}
\newcommand{\coloneq}{\mathrel{\resizebox{\widthof{$\mathord{=}$}}{\height}{ $\!\!\resizebox{1.2\width}{0.8\height}{\raisebox{0.23ex}{$\mathop{:}$}}\!\!=\!\!$ }}}
\newcommand{\eqcolonl}{\ensuremath{\mathrel{=\!\!\mathop{:}}}}
\newcommand{\coloneql}{\ensuremath{\mathrel{\mathop{:} \!\! =}}}
\newcommand{\vc}[1]{% inline column vector
  \left(\begin{smallmatrix}#1\end{smallmatrix}\right)%
}
\newcommand{\vr}[1]{% inline row vector
  \begin{smallmatrix}(\,#1\,)\end{smallmatrix}%
}
\makeatletter
\newcommand*{\defeq}{\ =\mathrel{\rlap{%
                     \raisebox{0.3ex}{$\m@th\cdot$}}%
                     \raisebox{-0.3ex}{$\m@th\cdot$}}%
                     }
\makeatother

\newcommand{\mathcircle}[1]{% inline row vector
 \overset{\circ}{#1}
}
\newcommand{\ulim}{% low limit
 \underline{\lim}
}
\newcommand{\ssi}{% iff
\iff
}
\newcommand{\ps}[2]{
\expval{#1 | #2}
}
\newcommand{\df}[1]{
\mqty{#1}
}
\newcommand{\n}[1]{
\norm{#1}
}
\newcommand{\sys}[1]{
\left\{\smqty{#1}\right.
}


\newcommand{\eqdef}{\ensuremath{\overset{\text{def}}=}}


\def\Circlearrowright{\ensuremath{%
  \rotatebox[origin=c]{230}{$\circlearrowright$}}}

\newcommand\ct[1]{\text{\rmfamily\upshape #1}}
\newcommand\question[1]{ {\color{red} ...!? \small #1}}
\newcommand\caz[1]{\left\{\begin{array} #1 \end{array}\right.}
\newcommand\const{\text{\rmfamily\upshape const}}
\newcommand\toP{ \overset{\pro}{\to}}
\newcommand\toPP{ \overset{\text{PP}}{\to}}
\newcommand{\oeq}{\mathrel{\text{\textcircled{$=$}}}}





\usepackage{xcolor}
% \usepackage[normalem]{ulem}
\usepackage{lipsum}
\makeatletter
% \newcommand\colorwave[1][blue]{\bgroup \markoverwith{\lower3.5\p@\hbox{\sixly \textcolor{#1}{\char58}}}\ULon}
%\font\sixly=lasy6 % does not re-load if already loaded, so no memory problem.

\newmdtheoremenv[
linewidth= 1pt,linecolor= blue,%
leftmargin=20,rightmargin=20,innertopmargin=0pt, innerrightmargin=40,%
tikzsetting = { draw=lightgray, line width = 0.3pt,dashed,%
dash pattern = on 15pt off 3pt},%
splittopskip=\topskip,skipbelow=\baselineskip,%
skipabove=\baselineskip,ntheorem,roundcorner=0pt,
% backgroundcolor=pagebg,font=\color{orange}\sffamily, fontcolor=white
]{examplebox}{Exemple}[section]



\newcommand\R{\mathbb{R}}
\newcommand\Z{\mathbb{Z}}
\newcommand\N{\mathbb{N}}
\newcommand\E{\mathbb{E}}
\newcommand\F{\mathcal{F}}
\newcommand\cH{\mathcal{H}}
\newcommand\V{\mathbb{V}}
\newcommand\dmo{ ^{-1} }
\newcommand\kapa{\kappa}
\newcommand\im{Im}
\newcommand\hs{\mathcal{H}}





\usepackage{soul}

\makeatletter
\newcommand*{\whiten}[1]{\llap{\textcolor{white}{{\the\SOUL@token}}\hspace{#1pt}}}
\DeclareRobustCommand*\myul{%
    \def\SOUL@everyspace{\underline{\space}\kern\z@}%
    \def\SOUL@everytoken{%
     \setbox0=\hbox{\the\SOUL@token}%
     \ifdim\dp0>\z@
        \raisebox{\dp0}{\underline{\phantom{\the\SOUL@token}}}%
        \whiten{1}\whiten{0}%
        \whiten{-1}\whiten{-2}%
        \llap{\the\SOUL@token}%
     \else
        \underline{\the\SOUL@token}%
     \fi}%
\SOUL@}
\makeatother

\newcommand*{\demp}{\fontfamily{lmtt}\selectfont}

\DeclareTextFontCommand{\textdemp}{\demp}

\begin{document}

\ifcomment
Multiline
comment
\fi
\ifcomment
\myul{Typesetting test}
% \color[rgb]{1,1,1}
$∑_i^n≠ 60º±∞π∆¬≈√j∫h≤≥µ$

$\CR \R\pro\ind\pro\gS\pro
\mqty[a&b\\c&d]$
$\pro\mathbb{P}$
$\dd{x}$

  \[
    \alpha(x)=\left\{
                \begin{array}{ll}
                  x\\
                  \frac{1}{1+e^{-kx}}\\
                  \frac{e^x-e^{-x}}{e^x+e^{-x}}
                \end{array}
              \right.
  \]

  $\expval{x}$
  
  $\chi_\rho(ghg\dmo)=\Tr(\rho_{ghg\dmo})=\Tr(\rho_g\circ\rho_h\circ\rho\dmo_g)=\Tr(\rho_h)\overset{\mbox{\scalebox{0.5}{$\Tr(AB)=\Tr(BA)$}}}{=}\chi_\rho(h)$
  	$\mathop{\oplus}_{\substack{x\in X}}$

$\mat(\rho_g)=(a_{ij}(g))_{\scriptsize \substack{1\leq i\leq d \\ 1\leq j\leq d}}$ et $\mat(\rho'_g)=(a'_{ij}(g))_{\scriptsize \substack{1\leq i'\leq d' \\ 1\leq j'\leq d'}}$



\[\int_a^b{\mathbb{R}^2}g(u, v)\dd{P_{XY}}(u, v)=\iint g(u,v) f_{XY}(u, v)\dd \lambda(u) \dd \lambda(v)\]
$$\lim_{x\to\infty} f(x)$$	
$$\iiiint_V \mu(t,u,v,w) \,dt\,du\,dv\,dw$$
$$\sum_{n=1}^{\infty} 2^{-n} = 1$$	
\begin{definition}
	Si $X$ et $Y$ sont 2 v.a. ou definit la \textsc{Covariance} entre $X$ et $Y$ comme
	$\cov(X,Y)\overset{\text{def}}{=}\E\left[(X-\E(X))(Y-\E(Y))\right]=\E(XY)-\E(X)\E(Y)$.
\end{definition}
\fi
\pagebreak

% \tableofcontents

% insert your code here
%\input{./algebra/main.tex}
%\input{./geometrie-differentielle/main.tex}
%\input{./probabilite/main.tex}
%\input{./analyse-fonctionnelle/main.tex}
% \input{./Analyse-convexe-et-dualite-en-optimisation/main.tex}
%\input{./tikz/main.tex}
%\input{./Theorie-du-distributions/main.tex}
%\input{./optimisation/mine.tex}
 \input{./modelisation/main.tex}

% yves.aubry@univ-tln.fr : algebra

\end{document}

% % !TEX encoding = UTF-8 Unicode
% !TEX TS-program = xelatex

\documentclass[french]{report}

%\usepackage[utf8]{inputenc}
%\usepackage[T1]{fontenc}
\usepackage{babel}


\newif\ifcomment
%\commenttrue # Show comments

\usepackage{physics}
\usepackage{amssymb}


\usepackage{amsthm}
% \usepackage{thmtools}
\usepackage{mathtools}
\usepackage{amsfonts}

\usepackage{color}

\usepackage{tikz}

\usepackage{geometry}
\geometry{a5paper, margin=0.1in, right=1cm}

\usepackage{dsfont}

\usepackage{graphicx}
\graphicspath{ {images/} }

\usepackage{faktor}

\usepackage{IEEEtrantools}
\usepackage{enumerate}   
\usepackage[PostScript=dvips]{"/Users/aware/Documents/Courses/diagrams"}


\newtheorem{theorem}{Théorème}[section]
\renewcommand{\thetheorem}{\arabic{theorem}}
\newtheorem{lemme}{Lemme}[section]
\renewcommand{\thelemme}{\arabic{lemme}}
\newtheorem{proposition}{Proposition}[section]
\renewcommand{\theproposition}{\arabic{proposition}}
\newtheorem{notations}{Notations}[section]
\newtheorem{problem}{Problème}[section]
\newtheorem{corollary}{Corollaire}[theorem]
\renewcommand{\thecorollary}{\arabic{corollary}}
\newtheorem{property}{Propriété}[section]
\newtheorem{objective}{Objectif}[section]

\theoremstyle{definition}
\newtheorem{definition}{Définition}[section]
\renewcommand{\thedefinition}{\arabic{definition}}
\newtheorem{exercise}{Exercice}[chapter]
\renewcommand{\theexercise}{\arabic{exercise}}
\newtheorem{example}{Exemple}[chapter]
\renewcommand{\theexample}{\arabic{example}}
\newtheorem*{solution}{Solution}
\newtheorem*{application}{Application}
\newtheorem*{notation}{Notation}
\newtheorem*{vocabulary}{Vocabulaire}
\newtheorem*{properties}{Propriétés}



\theoremstyle{remark}
\newtheorem*{remark}{Remarque}
\newtheorem*{rappel}{Rappel}


\usepackage{etoolbox}
\AtBeginEnvironment{exercise}{\small}
\AtBeginEnvironment{example}{\small}

\usepackage{cases}
\usepackage[red]{mypack}

\usepackage[framemethod=TikZ]{mdframed}

\definecolor{bg}{rgb}{0.4,0.25,0.95}
\definecolor{pagebg}{rgb}{0,0,0.5}
\surroundwithmdframed[
   topline=false,
   rightline=false,
   bottomline=false,
   leftmargin=\parindent,
   skipabove=8pt,
   skipbelow=8pt,
   linecolor=blue,
   innerbottommargin=10pt,
   % backgroundcolor=bg,font=\color{orange}\sffamily, fontcolor=white
]{definition}

\usepackage{empheq}
\usepackage[most]{tcolorbox}

\newtcbox{\mymath}[1][]{%
    nobeforeafter, math upper, tcbox raise base,
    enhanced, colframe=blue!30!black,
    colback=red!10, boxrule=1pt,
    #1}

\usepackage{unixode}


\DeclareMathOperator{\ord}{ord}
\DeclareMathOperator{\orb}{orb}
\DeclareMathOperator{\stab}{stab}
\DeclareMathOperator{\Stab}{stab}
\DeclareMathOperator{\ppcm}{ppcm}
\DeclareMathOperator{\conj}{Conj}
\DeclareMathOperator{\End}{End}
\DeclareMathOperator{\rot}{rot}
\DeclareMathOperator{\trs}{trace}
\DeclareMathOperator{\Ind}{Ind}
\DeclareMathOperator{\mat}{Mat}
\DeclareMathOperator{\id}{Id}
\DeclareMathOperator{\vect}{vect}
\DeclareMathOperator{\img}{img}
\DeclareMathOperator{\cov}{Cov}
\DeclareMathOperator{\dist}{dist}
\DeclareMathOperator{\irr}{Irr}
\DeclareMathOperator{\image}{Im}
\DeclareMathOperator{\pd}{\partial}
\DeclareMathOperator{\epi}{epi}
\DeclareMathOperator{\Argmin}{Argmin}
\DeclareMathOperator{\dom}{dom}
\DeclareMathOperator{\proj}{proj}
\DeclareMathOperator{\ctg}{ctg}
\DeclareMathOperator{\supp}{supp}
\DeclareMathOperator{\argmin}{argmin}
\DeclareMathOperator{\mult}{mult}
\DeclareMathOperator{\ch}{ch}
\DeclareMathOperator{\sh}{sh}
\DeclareMathOperator{\rang}{rang}
\DeclareMathOperator{\diam}{diam}
\DeclareMathOperator{\Epigraphe}{Epigraphe}




\usepackage{xcolor}
\everymath{\color{blue}}
%\everymath{\color[rgb]{0,1,1}}
%\pagecolor[rgb]{0,0,0.5}


\newcommand*{\pdtest}[3][]{\ensuremath{\frac{\partial^{#1} #2}{\partial #3}}}

\newcommand*{\deffunc}[6][]{\ensuremath{
\begin{array}{rcl}
#2 : #3 &\rightarrow& #4\\
#5 &\mapsto& #6
\end{array}
}}

\newcommand{\eqcolon}{\mathrel{\resizebox{\widthof{$\mathord{=}$}}{\height}{ $\!\!=\!\!\resizebox{1.2\width}{0.8\height}{\raisebox{0.23ex}{$\mathop{:}$}}\!\!$ }}}
\newcommand{\coloneq}{\mathrel{\resizebox{\widthof{$\mathord{=}$}}{\height}{ $\!\!\resizebox{1.2\width}{0.8\height}{\raisebox{0.23ex}{$\mathop{:}$}}\!\!=\!\!$ }}}
\newcommand{\eqcolonl}{\ensuremath{\mathrel{=\!\!\mathop{:}}}}
\newcommand{\coloneql}{\ensuremath{\mathrel{\mathop{:} \!\! =}}}
\newcommand{\vc}[1]{% inline column vector
  \left(\begin{smallmatrix}#1\end{smallmatrix}\right)%
}
\newcommand{\vr}[1]{% inline row vector
  \begin{smallmatrix}(\,#1\,)\end{smallmatrix}%
}
\makeatletter
\newcommand*{\defeq}{\ =\mathrel{\rlap{%
                     \raisebox{0.3ex}{$\m@th\cdot$}}%
                     \raisebox{-0.3ex}{$\m@th\cdot$}}%
                     }
\makeatother

\newcommand{\mathcircle}[1]{% inline row vector
 \overset{\circ}{#1}
}
\newcommand{\ulim}{% low limit
 \underline{\lim}
}
\newcommand{\ssi}{% iff
\iff
}
\newcommand{\ps}[2]{
\expval{#1 | #2}
}
\newcommand{\df}[1]{
\mqty{#1}
}
\newcommand{\n}[1]{
\norm{#1}
}
\newcommand{\sys}[1]{
\left\{\smqty{#1}\right.
}


\newcommand{\eqdef}{\ensuremath{\overset{\text{def}}=}}


\def\Circlearrowright{\ensuremath{%
  \rotatebox[origin=c]{230}{$\circlearrowright$}}}

\newcommand\ct[1]{\text{\rmfamily\upshape #1}}
\newcommand\question[1]{ {\color{red} ...!? \small #1}}
\newcommand\caz[1]{\left\{\begin{array} #1 \end{array}\right.}
\newcommand\const{\text{\rmfamily\upshape const}}
\newcommand\toP{ \overset{\pro}{\to}}
\newcommand\toPP{ \overset{\text{PP}}{\to}}
\newcommand{\oeq}{\mathrel{\text{\textcircled{$=$}}}}





\usepackage{xcolor}
% \usepackage[normalem]{ulem}
\usepackage{lipsum}
\makeatletter
% \newcommand\colorwave[1][blue]{\bgroup \markoverwith{\lower3.5\p@\hbox{\sixly \textcolor{#1}{\char58}}}\ULon}
%\font\sixly=lasy6 % does not re-load if already loaded, so no memory problem.

\newmdtheoremenv[
linewidth= 1pt,linecolor= blue,%
leftmargin=20,rightmargin=20,innertopmargin=0pt, innerrightmargin=40,%
tikzsetting = { draw=lightgray, line width = 0.3pt,dashed,%
dash pattern = on 15pt off 3pt},%
splittopskip=\topskip,skipbelow=\baselineskip,%
skipabove=\baselineskip,ntheorem,roundcorner=0pt,
% backgroundcolor=pagebg,font=\color{orange}\sffamily, fontcolor=white
]{examplebox}{Exemple}[section]



\newcommand\R{\mathbb{R}}
\newcommand\Z{\mathbb{Z}}
\newcommand\N{\mathbb{N}}
\newcommand\E{\mathbb{E}}
\newcommand\F{\mathcal{F}}
\newcommand\cH{\mathcal{H}}
\newcommand\V{\mathbb{V}}
\newcommand\dmo{ ^{-1} }
\newcommand\kapa{\kappa}
\newcommand\im{Im}
\newcommand\hs{\mathcal{H}}





\usepackage{soul}

\makeatletter
\newcommand*{\whiten}[1]{\llap{\textcolor{white}{{\the\SOUL@token}}\hspace{#1pt}}}
\DeclareRobustCommand*\myul{%
    \def\SOUL@everyspace{\underline{\space}\kern\z@}%
    \def\SOUL@everytoken{%
     \setbox0=\hbox{\the\SOUL@token}%
     \ifdim\dp0>\z@
        \raisebox{\dp0}{\underline{\phantom{\the\SOUL@token}}}%
        \whiten{1}\whiten{0}%
        \whiten{-1}\whiten{-2}%
        \llap{\the\SOUL@token}%
     \else
        \underline{\the\SOUL@token}%
     \fi}%
\SOUL@}
\makeatother

\newcommand*{\demp}{\fontfamily{lmtt}\selectfont}

\DeclareTextFontCommand{\textdemp}{\demp}

\begin{document}

\ifcomment
Multiline
comment
\fi
\ifcomment
\myul{Typesetting test}
% \color[rgb]{1,1,1}
$∑_i^n≠ 60º±∞π∆¬≈√j∫h≤≥µ$

$\CR \R\pro\ind\pro\gS\pro
\mqty[a&b\\c&d]$
$\pro\mathbb{P}$
$\dd{x}$

  \[
    \alpha(x)=\left\{
                \begin{array}{ll}
                  x\\
                  \frac{1}{1+e^{-kx}}\\
                  \frac{e^x-e^{-x}}{e^x+e^{-x}}
                \end{array}
              \right.
  \]

  $\expval{x}$
  
  $\chi_\rho(ghg\dmo)=\Tr(\rho_{ghg\dmo})=\Tr(\rho_g\circ\rho_h\circ\rho\dmo_g)=\Tr(\rho_h)\overset{\mbox{\scalebox{0.5}{$\Tr(AB)=\Tr(BA)$}}}{=}\chi_\rho(h)$
  	$\mathop{\oplus}_{\substack{x\in X}}$

$\mat(\rho_g)=(a_{ij}(g))_{\scriptsize \substack{1\leq i\leq d \\ 1\leq j\leq d}}$ et $\mat(\rho'_g)=(a'_{ij}(g))_{\scriptsize \substack{1\leq i'\leq d' \\ 1\leq j'\leq d'}}$



\[\int_a^b{\mathbb{R}^2}g(u, v)\dd{P_{XY}}(u, v)=\iint g(u,v) f_{XY}(u, v)\dd \lambda(u) \dd \lambda(v)\]
$$\lim_{x\to\infty} f(x)$$	
$$\iiiint_V \mu(t,u,v,w) \,dt\,du\,dv\,dw$$
$$\sum_{n=1}^{\infty} 2^{-n} = 1$$	
\begin{definition}
	Si $X$ et $Y$ sont 2 v.a. ou definit la \textsc{Covariance} entre $X$ et $Y$ comme
	$\cov(X,Y)\overset{\text{def}}{=}\E\left[(X-\E(X))(Y-\E(Y))\right]=\E(XY)-\E(X)\E(Y)$.
\end{definition}
\fi
\pagebreak

% \tableofcontents

% insert your code here
%\input{./algebra/main.tex}
%\input{./geometrie-differentielle/main.tex}
%\input{./probabilite/main.tex}
%\input{./analyse-fonctionnelle/main.tex}
% \input{./Analyse-convexe-et-dualite-en-optimisation/main.tex}
%\input{./tikz/main.tex}
%\input{./Theorie-du-distributions/main.tex}
%\input{./optimisation/mine.tex}
 \input{./modelisation/main.tex}

% yves.aubry@univ-tln.fr : algebra

\end{document}

%% !TEX encoding = UTF-8 Unicode
% !TEX TS-program = xelatex

\documentclass[french]{report}

%\usepackage[utf8]{inputenc}
%\usepackage[T1]{fontenc}
\usepackage{babel}


\newif\ifcomment
%\commenttrue # Show comments

\usepackage{physics}
\usepackage{amssymb}


\usepackage{amsthm}
% \usepackage{thmtools}
\usepackage{mathtools}
\usepackage{amsfonts}

\usepackage{color}

\usepackage{tikz}

\usepackage{geometry}
\geometry{a5paper, margin=0.1in, right=1cm}

\usepackage{dsfont}

\usepackage{graphicx}
\graphicspath{ {images/} }

\usepackage{faktor}

\usepackage{IEEEtrantools}
\usepackage{enumerate}   
\usepackage[PostScript=dvips]{"/Users/aware/Documents/Courses/diagrams"}


\newtheorem{theorem}{Théorème}[section]
\renewcommand{\thetheorem}{\arabic{theorem}}
\newtheorem{lemme}{Lemme}[section]
\renewcommand{\thelemme}{\arabic{lemme}}
\newtheorem{proposition}{Proposition}[section]
\renewcommand{\theproposition}{\arabic{proposition}}
\newtheorem{notations}{Notations}[section]
\newtheorem{problem}{Problème}[section]
\newtheorem{corollary}{Corollaire}[theorem]
\renewcommand{\thecorollary}{\arabic{corollary}}
\newtheorem{property}{Propriété}[section]
\newtheorem{objective}{Objectif}[section]

\theoremstyle{definition}
\newtheorem{definition}{Définition}[section]
\renewcommand{\thedefinition}{\arabic{definition}}
\newtheorem{exercise}{Exercice}[chapter]
\renewcommand{\theexercise}{\arabic{exercise}}
\newtheorem{example}{Exemple}[chapter]
\renewcommand{\theexample}{\arabic{example}}
\newtheorem*{solution}{Solution}
\newtheorem*{application}{Application}
\newtheorem*{notation}{Notation}
\newtheorem*{vocabulary}{Vocabulaire}
\newtheorem*{properties}{Propriétés}



\theoremstyle{remark}
\newtheorem*{remark}{Remarque}
\newtheorem*{rappel}{Rappel}


\usepackage{etoolbox}
\AtBeginEnvironment{exercise}{\small}
\AtBeginEnvironment{example}{\small}

\usepackage{cases}
\usepackage[red]{mypack}

\usepackage[framemethod=TikZ]{mdframed}

\definecolor{bg}{rgb}{0.4,0.25,0.95}
\definecolor{pagebg}{rgb}{0,0,0.5}
\surroundwithmdframed[
   topline=false,
   rightline=false,
   bottomline=false,
   leftmargin=\parindent,
   skipabove=8pt,
   skipbelow=8pt,
   linecolor=blue,
   innerbottommargin=10pt,
   % backgroundcolor=bg,font=\color{orange}\sffamily, fontcolor=white
]{definition}

\usepackage{empheq}
\usepackage[most]{tcolorbox}

\newtcbox{\mymath}[1][]{%
    nobeforeafter, math upper, tcbox raise base,
    enhanced, colframe=blue!30!black,
    colback=red!10, boxrule=1pt,
    #1}

\usepackage{unixode}


\DeclareMathOperator{\ord}{ord}
\DeclareMathOperator{\orb}{orb}
\DeclareMathOperator{\stab}{stab}
\DeclareMathOperator{\Stab}{stab}
\DeclareMathOperator{\ppcm}{ppcm}
\DeclareMathOperator{\conj}{Conj}
\DeclareMathOperator{\End}{End}
\DeclareMathOperator{\rot}{rot}
\DeclareMathOperator{\trs}{trace}
\DeclareMathOperator{\Ind}{Ind}
\DeclareMathOperator{\mat}{Mat}
\DeclareMathOperator{\id}{Id}
\DeclareMathOperator{\vect}{vect}
\DeclareMathOperator{\img}{img}
\DeclareMathOperator{\cov}{Cov}
\DeclareMathOperator{\dist}{dist}
\DeclareMathOperator{\irr}{Irr}
\DeclareMathOperator{\image}{Im}
\DeclareMathOperator{\pd}{\partial}
\DeclareMathOperator{\epi}{epi}
\DeclareMathOperator{\Argmin}{Argmin}
\DeclareMathOperator{\dom}{dom}
\DeclareMathOperator{\proj}{proj}
\DeclareMathOperator{\ctg}{ctg}
\DeclareMathOperator{\supp}{supp}
\DeclareMathOperator{\argmin}{argmin}
\DeclareMathOperator{\mult}{mult}
\DeclareMathOperator{\ch}{ch}
\DeclareMathOperator{\sh}{sh}
\DeclareMathOperator{\rang}{rang}
\DeclareMathOperator{\diam}{diam}
\DeclareMathOperator{\Epigraphe}{Epigraphe}




\usepackage{xcolor}
\everymath{\color{blue}}
%\everymath{\color[rgb]{0,1,1}}
%\pagecolor[rgb]{0,0,0.5}


\newcommand*{\pdtest}[3][]{\ensuremath{\frac{\partial^{#1} #2}{\partial #3}}}

\newcommand*{\deffunc}[6][]{\ensuremath{
\begin{array}{rcl}
#2 : #3 &\rightarrow& #4\\
#5 &\mapsto& #6
\end{array}
}}

\newcommand{\eqcolon}{\mathrel{\resizebox{\widthof{$\mathord{=}$}}{\height}{ $\!\!=\!\!\resizebox{1.2\width}{0.8\height}{\raisebox{0.23ex}{$\mathop{:}$}}\!\!$ }}}
\newcommand{\coloneq}{\mathrel{\resizebox{\widthof{$\mathord{=}$}}{\height}{ $\!\!\resizebox{1.2\width}{0.8\height}{\raisebox{0.23ex}{$\mathop{:}$}}\!\!=\!\!$ }}}
\newcommand{\eqcolonl}{\ensuremath{\mathrel{=\!\!\mathop{:}}}}
\newcommand{\coloneql}{\ensuremath{\mathrel{\mathop{:} \!\! =}}}
\newcommand{\vc}[1]{% inline column vector
  \left(\begin{smallmatrix}#1\end{smallmatrix}\right)%
}
\newcommand{\vr}[1]{% inline row vector
  \begin{smallmatrix}(\,#1\,)\end{smallmatrix}%
}
\makeatletter
\newcommand*{\defeq}{\ =\mathrel{\rlap{%
                     \raisebox{0.3ex}{$\m@th\cdot$}}%
                     \raisebox{-0.3ex}{$\m@th\cdot$}}%
                     }
\makeatother

\newcommand{\mathcircle}[1]{% inline row vector
 \overset{\circ}{#1}
}
\newcommand{\ulim}{% low limit
 \underline{\lim}
}
\newcommand{\ssi}{% iff
\iff
}
\newcommand{\ps}[2]{
\expval{#1 | #2}
}
\newcommand{\df}[1]{
\mqty{#1}
}
\newcommand{\n}[1]{
\norm{#1}
}
\newcommand{\sys}[1]{
\left\{\smqty{#1}\right.
}


\newcommand{\eqdef}{\ensuremath{\overset{\text{def}}=}}


\def\Circlearrowright{\ensuremath{%
  \rotatebox[origin=c]{230}{$\circlearrowright$}}}

\newcommand\ct[1]{\text{\rmfamily\upshape #1}}
\newcommand\question[1]{ {\color{red} ...!? \small #1}}
\newcommand\caz[1]{\left\{\begin{array} #1 \end{array}\right.}
\newcommand\const{\text{\rmfamily\upshape const}}
\newcommand\toP{ \overset{\pro}{\to}}
\newcommand\toPP{ \overset{\text{PP}}{\to}}
\newcommand{\oeq}{\mathrel{\text{\textcircled{$=$}}}}





\usepackage{xcolor}
% \usepackage[normalem]{ulem}
\usepackage{lipsum}
\makeatletter
% \newcommand\colorwave[1][blue]{\bgroup \markoverwith{\lower3.5\p@\hbox{\sixly \textcolor{#1}{\char58}}}\ULon}
%\font\sixly=lasy6 % does not re-load if already loaded, so no memory problem.

\newmdtheoremenv[
linewidth= 1pt,linecolor= blue,%
leftmargin=20,rightmargin=20,innertopmargin=0pt, innerrightmargin=40,%
tikzsetting = { draw=lightgray, line width = 0.3pt,dashed,%
dash pattern = on 15pt off 3pt},%
splittopskip=\topskip,skipbelow=\baselineskip,%
skipabove=\baselineskip,ntheorem,roundcorner=0pt,
% backgroundcolor=pagebg,font=\color{orange}\sffamily, fontcolor=white
]{examplebox}{Exemple}[section]



\newcommand\R{\mathbb{R}}
\newcommand\Z{\mathbb{Z}}
\newcommand\N{\mathbb{N}}
\newcommand\E{\mathbb{E}}
\newcommand\F{\mathcal{F}}
\newcommand\cH{\mathcal{H}}
\newcommand\V{\mathbb{V}}
\newcommand\dmo{ ^{-1} }
\newcommand\kapa{\kappa}
\newcommand\im{Im}
\newcommand\hs{\mathcal{H}}





\usepackage{soul}

\makeatletter
\newcommand*{\whiten}[1]{\llap{\textcolor{white}{{\the\SOUL@token}}\hspace{#1pt}}}
\DeclareRobustCommand*\myul{%
    \def\SOUL@everyspace{\underline{\space}\kern\z@}%
    \def\SOUL@everytoken{%
     \setbox0=\hbox{\the\SOUL@token}%
     \ifdim\dp0>\z@
        \raisebox{\dp0}{\underline{\phantom{\the\SOUL@token}}}%
        \whiten{1}\whiten{0}%
        \whiten{-1}\whiten{-2}%
        \llap{\the\SOUL@token}%
     \else
        \underline{\the\SOUL@token}%
     \fi}%
\SOUL@}
\makeatother

\newcommand*{\demp}{\fontfamily{lmtt}\selectfont}

\DeclareTextFontCommand{\textdemp}{\demp}

\begin{document}

\ifcomment
Multiline
comment
\fi
\ifcomment
\myul{Typesetting test}
% \color[rgb]{1,1,1}
$∑_i^n≠ 60º±∞π∆¬≈√j∫h≤≥µ$

$\CR \R\pro\ind\pro\gS\pro
\mqty[a&b\\c&d]$
$\pro\mathbb{P}$
$\dd{x}$

  \[
    \alpha(x)=\left\{
                \begin{array}{ll}
                  x\\
                  \frac{1}{1+e^{-kx}}\\
                  \frac{e^x-e^{-x}}{e^x+e^{-x}}
                \end{array}
              \right.
  \]

  $\expval{x}$
  
  $\chi_\rho(ghg\dmo)=\Tr(\rho_{ghg\dmo})=\Tr(\rho_g\circ\rho_h\circ\rho\dmo_g)=\Tr(\rho_h)\overset{\mbox{\scalebox{0.5}{$\Tr(AB)=\Tr(BA)$}}}{=}\chi_\rho(h)$
  	$\mathop{\oplus}_{\substack{x\in X}}$

$\mat(\rho_g)=(a_{ij}(g))_{\scriptsize \substack{1\leq i\leq d \\ 1\leq j\leq d}}$ et $\mat(\rho'_g)=(a'_{ij}(g))_{\scriptsize \substack{1\leq i'\leq d' \\ 1\leq j'\leq d'}}$



\[\int_a^b{\mathbb{R}^2}g(u, v)\dd{P_{XY}}(u, v)=\iint g(u,v) f_{XY}(u, v)\dd \lambda(u) \dd \lambda(v)\]
$$\lim_{x\to\infty} f(x)$$	
$$\iiiint_V \mu(t,u,v,w) \,dt\,du\,dv\,dw$$
$$\sum_{n=1}^{\infty} 2^{-n} = 1$$	
\begin{definition}
	Si $X$ et $Y$ sont 2 v.a. ou definit la \textsc{Covariance} entre $X$ et $Y$ comme
	$\cov(X,Y)\overset{\text{def}}{=}\E\left[(X-\E(X))(Y-\E(Y))\right]=\E(XY)-\E(X)\E(Y)$.
\end{definition}
\fi
\pagebreak

% \tableofcontents

% insert your code here
%\input{./algebra/main.tex}
%\input{./geometrie-differentielle/main.tex}
%\input{./probabilite/main.tex}
%\input{./analyse-fonctionnelle/main.tex}
% \input{./Analyse-convexe-et-dualite-en-optimisation/main.tex}
%\input{./tikz/main.tex}
%\input{./Theorie-du-distributions/main.tex}
%\input{./optimisation/mine.tex}
 \input{./modelisation/main.tex}

% yves.aubry@univ-tln.fr : algebra

\end{document}

%% !TEX encoding = UTF-8 Unicode
% !TEX TS-program = xelatex

\documentclass[french]{report}

%\usepackage[utf8]{inputenc}
%\usepackage[T1]{fontenc}
\usepackage{babel}


\newif\ifcomment
%\commenttrue # Show comments

\usepackage{physics}
\usepackage{amssymb}


\usepackage{amsthm}
% \usepackage{thmtools}
\usepackage{mathtools}
\usepackage{amsfonts}

\usepackage{color}

\usepackage{tikz}

\usepackage{geometry}
\geometry{a5paper, margin=0.1in, right=1cm}

\usepackage{dsfont}

\usepackage{graphicx}
\graphicspath{ {images/} }

\usepackage{faktor}

\usepackage{IEEEtrantools}
\usepackage{enumerate}   
\usepackage[PostScript=dvips]{"/Users/aware/Documents/Courses/diagrams"}


\newtheorem{theorem}{Théorème}[section]
\renewcommand{\thetheorem}{\arabic{theorem}}
\newtheorem{lemme}{Lemme}[section]
\renewcommand{\thelemme}{\arabic{lemme}}
\newtheorem{proposition}{Proposition}[section]
\renewcommand{\theproposition}{\arabic{proposition}}
\newtheorem{notations}{Notations}[section]
\newtheorem{problem}{Problème}[section]
\newtheorem{corollary}{Corollaire}[theorem]
\renewcommand{\thecorollary}{\arabic{corollary}}
\newtheorem{property}{Propriété}[section]
\newtheorem{objective}{Objectif}[section]

\theoremstyle{definition}
\newtheorem{definition}{Définition}[section]
\renewcommand{\thedefinition}{\arabic{definition}}
\newtheorem{exercise}{Exercice}[chapter]
\renewcommand{\theexercise}{\arabic{exercise}}
\newtheorem{example}{Exemple}[chapter]
\renewcommand{\theexample}{\arabic{example}}
\newtheorem*{solution}{Solution}
\newtheorem*{application}{Application}
\newtheorem*{notation}{Notation}
\newtheorem*{vocabulary}{Vocabulaire}
\newtheorem*{properties}{Propriétés}



\theoremstyle{remark}
\newtheorem*{remark}{Remarque}
\newtheorem*{rappel}{Rappel}


\usepackage{etoolbox}
\AtBeginEnvironment{exercise}{\small}
\AtBeginEnvironment{example}{\small}

\usepackage{cases}
\usepackage[red]{mypack}

\usepackage[framemethod=TikZ]{mdframed}

\definecolor{bg}{rgb}{0.4,0.25,0.95}
\definecolor{pagebg}{rgb}{0,0,0.5}
\surroundwithmdframed[
   topline=false,
   rightline=false,
   bottomline=false,
   leftmargin=\parindent,
   skipabove=8pt,
   skipbelow=8pt,
   linecolor=blue,
   innerbottommargin=10pt,
   % backgroundcolor=bg,font=\color{orange}\sffamily, fontcolor=white
]{definition}

\usepackage{empheq}
\usepackage[most]{tcolorbox}

\newtcbox{\mymath}[1][]{%
    nobeforeafter, math upper, tcbox raise base,
    enhanced, colframe=blue!30!black,
    colback=red!10, boxrule=1pt,
    #1}

\usepackage{unixode}


\DeclareMathOperator{\ord}{ord}
\DeclareMathOperator{\orb}{orb}
\DeclareMathOperator{\stab}{stab}
\DeclareMathOperator{\Stab}{stab}
\DeclareMathOperator{\ppcm}{ppcm}
\DeclareMathOperator{\conj}{Conj}
\DeclareMathOperator{\End}{End}
\DeclareMathOperator{\rot}{rot}
\DeclareMathOperator{\trs}{trace}
\DeclareMathOperator{\Ind}{Ind}
\DeclareMathOperator{\mat}{Mat}
\DeclareMathOperator{\id}{Id}
\DeclareMathOperator{\vect}{vect}
\DeclareMathOperator{\img}{img}
\DeclareMathOperator{\cov}{Cov}
\DeclareMathOperator{\dist}{dist}
\DeclareMathOperator{\irr}{Irr}
\DeclareMathOperator{\image}{Im}
\DeclareMathOperator{\pd}{\partial}
\DeclareMathOperator{\epi}{epi}
\DeclareMathOperator{\Argmin}{Argmin}
\DeclareMathOperator{\dom}{dom}
\DeclareMathOperator{\proj}{proj}
\DeclareMathOperator{\ctg}{ctg}
\DeclareMathOperator{\supp}{supp}
\DeclareMathOperator{\argmin}{argmin}
\DeclareMathOperator{\mult}{mult}
\DeclareMathOperator{\ch}{ch}
\DeclareMathOperator{\sh}{sh}
\DeclareMathOperator{\rang}{rang}
\DeclareMathOperator{\diam}{diam}
\DeclareMathOperator{\Epigraphe}{Epigraphe}




\usepackage{xcolor}
\everymath{\color{blue}}
%\everymath{\color[rgb]{0,1,1}}
%\pagecolor[rgb]{0,0,0.5}


\newcommand*{\pdtest}[3][]{\ensuremath{\frac{\partial^{#1} #2}{\partial #3}}}

\newcommand*{\deffunc}[6][]{\ensuremath{
\begin{array}{rcl}
#2 : #3 &\rightarrow& #4\\
#5 &\mapsto& #6
\end{array}
}}

\newcommand{\eqcolon}{\mathrel{\resizebox{\widthof{$\mathord{=}$}}{\height}{ $\!\!=\!\!\resizebox{1.2\width}{0.8\height}{\raisebox{0.23ex}{$\mathop{:}$}}\!\!$ }}}
\newcommand{\coloneq}{\mathrel{\resizebox{\widthof{$\mathord{=}$}}{\height}{ $\!\!\resizebox{1.2\width}{0.8\height}{\raisebox{0.23ex}{$\mathop{:}$}}\!\!=\!\!$ }}}
\newcommand{\eqcolonl}{\ensuremath{\mathrel{=\!\!\mathop{:}}}}
\newcommand{\coloneql}{\ensuremath{\mathrel{\mathop{:} \!\! =}}}
\newcommand{\vc}[1]{% inline column vector
  \left(\begin{smallmatrix}#1\end{smallmatrix}\right)%
}
\newcommand{\vr}[1]{% inline row vector
  \begin{smallmatrix}(\,#1\,)\end{smallmatrix}%
}
\makeatletter
\newcommand*{\defeq}{\ =\mathrel{\rlap{%
                     \raisebox{0.3ex}{$\m@th\cdot$}}%
                     \raisebox{-0.3ex}{$\m@th\cdot$}}%
                     }
\makeatother

\newcommand{\mathcircle}[1]{% inline row vector
 \overset{\circ}{#1}
}
\newcommand{\ulim}{% low limit
 \underline{\lim}
}
\newcommand{\ssi}{% iff
\iff
}
\newcommand{\ps}[2]{
\expval{#1 | #2}
}
\newcommand{\df}[1]{
\mqty{#1}
}
\newcommand{\n}[1]{
\norm{#1}
}
\newcommand{\sys}[1]{
\left\{\smqty{#1}\right.
}


\newcommand{\eqdef}{\ensuremath{\overset{\text{def}}=}}


\def\Circlearrowright{\ensuremath{%
  \rotatebox[origin=c]{230}{$\circlearrowright$}}}

\newcommand\ct[1]{\text{\rmfamily\upshape #1}}
\newcommand\question[1]{ {\color{red} ...!? \small #1}}
\newcommand\caz[1]{\left\{\begin{array} #1 \end{array}\right.}
\newcommand\const{\text{\rmfamily\upshape const}}
\newcommand\toP{ \overset{\pro}{\to}}
\newcommand\toPP{ \overset{\text{PP}}{\to}}
\newcommand{\oeq}{\mathrel{\text{\textcircled{$=$}}}}





\usepackage{xcolor}
% \usepackage[normalem]{ulem}
\usepackage{lipsum}
\makeatletter
% \newcommand\colorwave[1][blue]{\bgroup \markoverwith{\lower3.5\p@\hbox{\sixly \textcolor{#1}{\char58}}}\ULon}
%\font\sixly=lasy6 % does not re-load if already loaded, so no memory problem.

\newmdtheoremenv[
linewidth= 1pt,linecolor= blue,%
leftmargin=20,rightmargin=20,innertopmargin=0pt, innerrightmargin=40,%
tikzsetting = { draw=lightgray, line width = 0.3pt,dashed,%
dash pattern = on 15pt off 3pt},%
splittopskip=\topskip,skipbelow=\baselineskip,%
skipabove=\baselineskip,ntheorem,roundcorner=0pt,
% backgroundcolor=pagebg,font=\color{orange}\sffamily, fontcolor=white
]{examplebox}{Exemple}[section]



\newcommand\R{\mathbb{R}}
\newcommand\Z{\mathbb{Z}}
\newcommand\N{\mathbb{N}}
\newcommand\E{\mathbb{E}}
\newcommand\F{\mathcal{F}}
\newcommand\cH{\mathcal{H}}
\newcommand\V{\mathbb{V}}
\newcommand\dmo{ ^{-1} }
\newcommand\kapa{\kappa}
\newcommand\im{Im}
\newcommand\hs{\mathcal{H}}





\usepackage{soul}

\makeatletter
\newcommand*{\whiten}[1]{\llap{\textcolor{white}{{\the\SOUL@token}}\hspace{#1pt}}}
\DeclareRobustCommand*\myul{%
    \def\SOUL@everyspace{\underline{\space}\kern\z@}%
    \def\SOUL@everytoken{%
     \setbox0=\hbox{\the\SOUL@token}%
     \ifdim\dp0>\z@
        \raisebox{\dp0}{\underline{\phantom{\the\SOUL@token}}}%
        \whiten{1}\whiten{0}%
        \whiten{-1}\whiten{-2}%
        \llap{\the\SOUL@token}%
     \else
        \underline{\the\SOUL@token}%
     \fi}%
\SOUL@}
\makeatother

\newcommand*{\demp}{\fontfamily{lmtt}\selectfont}

\DeclareTextFontCommand{\textdemp}{\demp}

\begin{document}

\ifcomment
Multiline
comment
\fi
\ifcomment
\myul{Typesetting test}
% \color[rgb]{1,1,1}
$∑_i^n≠ 60º±∞π∆¬≈√j∫h≤≥µ$

$\CR \R\pro\ind\pro\gS\pro
\mqty[a&b\\c&d]$
$\pro\mathbb{P}$
$\dd{x}$

  \[
    \alpha(x)=\left\{
                \begin{array}{ll}
                  x\\
                  \frac{1}{1+e^{-kx}}\\
                  \frac{e^x-e^{-x}}{e^x+e^{-x}}
                \end{array}
              \right.
  \]

  $\expval{x}$
  
  $\chi_\rho(ghg\dmo)=\Tr(\rho_{ghg\dmo})=\Tr(\rho_g\circ\rho_h\circ\rho\dmo_g)=\Tr(\rho_h)\overset{\mbox{\scalebox{0.5}{$\Tr(AB)=\Tr(BA)$}}}{=}\chi_\rho(h)$
  	$\mathop{\oplus}_{\substack{x\in X}}$

$\mat(\rho_g)=(a_{ij}(g))_{\scriptsize \substack{1\leq i\leq d \\ 1\leq j\leq d}}$ et $\mat(\rho'_g)=(a'_{ij}(g))_{\scriptsize \substack{1\leq i'\leq d' \\ 1\leq j'\leq d'}}$



\[\int_a^b{\mathbb{R}^2}g(u, v)\dd{P_{XY}}(u, v)=\iint g(u,v) f_{XY}(u, v)\dd \lambda(u) \dd \lambda(v)\]
$$\lim_{x\to\infty} f(x)$$	
$$\iiiint_V \mu(t,u,v,w) \,dt\,du\,dv\,dw$$
$$\sum_{n=1}^{\infty} 2^{-n} = 1$$	
\begin{definition}
	Si $X$ et $Y$ sont 2 v.a. ou definit la \textsc{Covariance} entre $X$ et $Y$ comme
	$\cov(X,Y)\overset{\text{def}}{=}\E\left[(X-\E(X))(Y-\E(Y))\right]=\E(XY)-\E(X)\E(Y)$.
\end{definition}
\fi
\pagebreak

% \tableofcontents

% insert your code here
%\input{./algebra/main.tex}
%\input{./geometrie-differentielle/main.tex}
%\input{./probabilite/main.tex}
%\input{./analyse-fonctionnelle/main.tex}
% \input{./Analyse-convexe-et-dualite-en-optimisation/main.tex}
%\input{./tikz/main.tex}
%\input{./Theorie-du-distributions/main.tex}
%\input{./optimisation/mine.tex}
 \input{./modelisation/main.tex}

% yves.aubry@univ-tln.fr : algebra

\end{document}

%\input{./optimisation/mine.tex}
 % !TEX encoding = UTF-8 Unicode
% !TEX TS-program = xelatex

\documentclass[french]{report}

%\usepackage[utf8]{inputenc}
%\usepackage[T1]{fontenc}
\usepackage{babel}


\newif\ifcomment
%\commenttrue # Show comments

\usepackage{physics}
\usepackage{amssymb}


\usepackage{amsthm}
% \usepackage{thmtools}
\usepackage{mathtools}
\usepackage{amsfonts}

\usepackage{color}

\usepackage{tikz}

\usepackage{geometry}
\geometry{a5paper, margin=0.1in, right=1cm}

\usepackage{dsfont}

\usepackage{graphicx}
\graphicspath{ {images/} }

\usepackage{faktor}

\usepackage{IEEEtrantools}
\usepackage{enumerate}   
\usepackage[PostScript=dvips]{"/Users/aware/Documents/Courses/diagrams"}


\newtheorem{theorem}{Théorème}[section]
\renewcommand{\thetheorem}{\arabic{theorem}}
\newtheorem{lemme}{Lemme}[section]
\renewcommand{\thelemme}{\arabic{lemme}}
\newtheorem{proposition}{Proposition}[section]
\renewcommand{\theproposition}{\arabic{proposition}}
\newtheorem{notations}{Notations}[section]
\newtheorem{problem}{Problème}[section]
\newtheorem{corollary}{Corollaire}[theorem]
\renewcommand{\thecorollary}{\arabic{corollary}}
\newtheorem{property}{Propriété}[section]
\newtheorem{objective}{Objectif}[section]

\theoremstyle{definition}
\newtheorem{definition}{Définition}[section]
\renewcommand{\thedefinition}{\arabic{definition}}
\newtheorem{exercise}{Exercice}[chapter]
\renewcommand{\theexercise}{\arabic{exercise}}
\newtheorem{example}{Exemple}[chapter]
\renewcommand{\theexample}{\arabic{example}}
\newtheorem*{solution}{Solution}
\newtheorem*{application}{Application}
\newtheorem*{notation}{Notation}
\newtheorem*{vocabulary}{Vocabulaire}
\newtheorem*{properties}{Propriétés}



\theoremstyle{remark}
\newtheorem*{remark}{Remarque}
\newtheorem*{rappel}{Rappel}


\usepackage{etoolbox}
\AtBeginEnvironment{exercise}{\small}
\AtBeginEnvironment{example}{\small}

\usepackage{cases}
\usepackage[red]{mypack}

\usepackage[framemethod=TikZ]{mdframed}

\definecolor{bg}{rgb}{0.4,0.25,0.95}
\definecolor{pagebg}{rgb}{0,0,0.5}
\surroundwithmdframed[
   topline=false,
   rightline=false,
   bottomline=false,
   leftmargin=\parindent,
   skipabove=8pt,
   skipbelow=8pt,
   linecolor=blue,
   innerbottommargin=10pt,
   % backgroundcolor=bg,font=\color{orange}\sffamily, fontcolor=white
]{definition}

\usepackage{empheq}
\usepackage[most]{tcolorbox}

\newtcbox{\mymath}[1][]{%
    nobeforeafter, math upper, tcbox raise base,
    enhanced, colframe=blue!30!black,
    colback=red!10, boxrule=1pt,
    #1}

\usepackage{unixode}


\DeclareMathOperator{\ord}{ord}
\DeclareMathOperator{\orb}{orb}
\DeclareMathOperator{\stab}{stab}
\DeclareMathOperator{\Stab}{stab}
\DeclareMathOperator{\ppcm}{ppcm}
\DeclareMathOperator{\conj}{Conj}
\DeclareMathOperator{\End}{End}
\DeclareMathOperator{\rot}{rot}
\DeclareMathOperator{\trs}{trace}
\DeclareMathOperator{\Ind}{Ind}
\DeclareMathOperator{\mat}{Mat}
\DeclareMathOperator{\id}{Id}
\DeclareMathOperator{\vect}{vect}
\DeclareMathOperator{\img}{img}
\DeclareMathOperator{\cov}{Cov}
\DeclareMathOperator{\dist}{dist}
\DeclareMathOperator{\irr}{Irr}
\DeclareMathOperator{\image}{Im}
\DeclareMathOperator{\pd}{\partial}
\DeclareMathOperator{\epi}{epi}
\DeclareMathOperator{\Argmin}{Argmin}
\DeclareMathOperator{\dom}{dom}
\DeclareMathOperator{\proj}{proj}
\DeclareMathOperator{\ctg}{ctg}
\DeclareMathOperator{\supp}{supp}
\DeclareMathOperator{\argmin}{argmin}
\DeclareMathOperator{\mult}{mult}
\DeclareMathOperator{\ch}{ch}
\DeclareMathOperator{\sh}{sh}
\DeclareMathOperator{\rang}{rang}
\DeclareMathOperator{\diam}{diam}
\DeclareMathOperator{\Epigraphe}{Epigraphe}




\usepackage{xcolor}
\everymath{\color{blue}}
%\everymath{\color[rgb]{0,1,1}}
%\pagecolor[rgb]{0,0,0.5}


\newcommand*{\pdtest}[3][]{\ensuremath{\frac{\partial^{#1} #2}{\partial #3}}}

\newcommand*{\deffunc}[6][]{\ensuremath{
\begin{array}{rcl}
#2 : #3 &\rightarrow& #4\\
#5 &\mapsto& #6
\end{array}
}}

\newcommand{\eqcolon}{\mathrel{\resizebox{\widthof{$\mathord{=}$}}{\height}{ $\!\!=\!\!\resizebox{1.2\width}{0.8\height}{\raisebox{0.23ex}{$\mathop{:}$}}\!\!$ }}}
\newcommand{\coloneq}{\mathrel{\resizebox{\widthof{$\mathord{=}$}}{\height}{ $\!\!\resizebox{1.2\width}{0.8\height}{\raisebox{0.23ex}{$\mathop{:}$}}\!\!=\!\!$ }}}
\newcommand{\eqcolonl}{\ensuremath{\mathrel{=\!\!\mathop{:}}}}
\newcommand{\coloneql}{\ensuremath{\mathrel{\mathop{:} \!\! =}}}
\newcommand{\vc}[1]{% inline column vector
  \left(\begin{smallmatrix}#1\end{smallmatrix}\right)%
}
\newcommand{\vr}[1]{% inline row vector
  \begin{smallmatrix}(\,#1\,)\end{smallmatrix}%
}
\makeatletter
\newcommand*{\defeq}{\ =\mathrel{\rlap{%
                     \raisebox{0.3ex}{$\m@th\cdot$}}%
                     \raisebox{-0.3ex}{$\m@th\cdot$}}%
                     }
\makeatother

\newcommand{\mathcircle}[1]{% inline row vector
 \overset{\circ}{#1}
}
\newcommand{\ulim}{% low limit
 \underline{\lim}
}
\newcommand{\ssi}{% iff
\iff
}
\newcommand{\ps}[2]{
\expval{#1 | #2}
}
\newcommand{\df}[1]{
\mqty{#1}
}
\newcommand{\n}[1]{
\norm{#1}
}
\newcommand{\sys}[1]{
\left\{\smqty{#1}\right.
}


\newcommand{\eqdef}{\ensuremath{\overset{\text{def}}=}}


\def\Circlearrowright{\ensuremath{%
  \rotatebox[origin=c]{230}{$\circlearrowright$}}}

\newcommand\ct[1]{\text{\rmfamily\upshape #1}}
\newcommand\question[1]{ {\color{red} ...!? \small #1}}
\newcommand\caz[1]{\left\{\begin{array} #1 \end{array}\right.}
\newcommand\const{\text{\rmfamily\upshape const}}
\newcommand\toP{ \overset{\pro}{\to}}
\newcommand\toPP{ \overset{\text{PP}}{\to}}
\newcommand{\oeq}{\mathrel{\text{\textcircled{$=$}}}}





\usepackage{xcolor}
% \usepackage[normalem]{ulem}
\usepackage{lipsum}
\makeatletter
% \newcommand\colorwave[1][blue]{\bgroup \markoverwith{\lower3.5\p@\hbox{\sixly \textcolor{#1}{\char58}}}\ULon}
%\font\sixly=lasy6 % does not re-load if already loaded, so no memory problem.

\newmdtheoremenv[
linewidth= 1pt,linecolor= blue,%
leftmargin=20,rightmargin=20,innertopmargin=0pt, innerrightmargin=40,%
tikzsetting = { draw=lightgray, line width = 0.3pt,dashed,%
dash pattern = on 15pt off 3pt},%
splittopskip=\topskip,skipbelow=\baselineskip,%
skipabove=\baselineskip,ntheorem,roundcorner=0pt,
% backgroundcolor=pagebg,font=\color{orange}\sffamily, fontcolor=white
]{examplebox}{Exemple}[section]



\newcommand\R{\mathbb{R}}
\newcommand\Z{\mathbb{Z}}
\newcommand\N{\mathbb{N}}
\newcommand\E{\mathbb{E}}
\newcommand\F{\mathcal{F}}
\newcommand\cH{\mathcal{H}}
\newcommand\V{\mathbb{V}}
\newcommand\dmo{ ^{-1} }
\newcommand\kapa{\kappa}
\newcommand\im{Im}
\newcommand\hs{\mathcal{H}}





\usepackage{soul}

\makeatletter
\newcommand*{\whiten}[1]{\llap{\textcolor{white}{{\the\SOUL@token}}\hspace{#1pt}}}
\DeclareRobustCommand*\myul{%
    \def\SOUL@everyspace{\underline{\space}\kern\z@}%
    \def\SOUL@everytoken{%
     \setbox0=\hbox{\the\SOUL@token}%
     \ifdim\dp0>\z@
        \raisebox{\dp0}{\underline{\phantom{\the\SOUL@token}}}%
        \whiten{1}\whiten{0}%
        \whiten{-1}\whiten{-2}%
        \llap{\the\SOUL@token}%
     \else
        \underline{\the\SOUL@token}%
     \fi}%
\SOUL@}
\makeatother

\newcommand*{\demp}{\fontfamily{lmtt}\selectfont}

\DeclareTextFontCommand{\textdemp}{\demp}

\begin{document}

\ifcomment
Multiline
comment
\fi
\ifcomment
\myul{Typesetting test}
% \color[rgb]{1,1,1}
$∑_i^n≠ 60º±∞π∆¬≈√j∫h≤≥µ$

$\CR \R\pro\ind\pro\gS\pro
\mqty[a&b\\c&d]$
$\pro\mathbb{P}$
$\dd{x}$

  \[
    \alpha(x)=\left\{
                \begin{array}{ll}
                  x\\
                  \frac{1}{1+e^{-kx}}\\
                  \frac{e^x-e^{-x}}{e^x+e^{-x}}
                \end{array}
              \right.
  \]

  $\expval{x}$
  
  $\chi_\rho(ghg\dmo)=\Tr(\rho_{ghg\dmo})=\Tr(\rho_g\circ\rho_h\circ\rho\dmo_g)=\Tr(\rho_h)\overset{\mbox{\scalebox{0.5}{$\Tr(AB)=\Tr(BA)$}}}{=}\chi_\rho(h)$
  	$\mathop{\oplus}_{\substack{x\in X}}$

$\mat(\rho_g)=(a_{ij}(g))_{\scriptsize \substack{1\leq i\leq d \\ 1\leq j\leq d}}$ et $\mat(\rho'_g)=(a'_{ij}(g))_{\scriptsize \substack{1\leq i'\leq d' \\ 1\leq j'\leq d'}}$



\[\int_a^b{\mathbb{R}^2}g(u, v)\dd{P_{XY}}(u, v)=\iint g(u,v) f_{XY}(u, v)\dd \lambda(u) \dd \lambda(v)\]
$$\lim_{x\to\infty} f(x)$$	
$$\iiiint_V \mu(t,u,v,w) \,dt\,du\,dv\,dw$$
$$\sum_{n=1}^{\infty} 2^{-n} = 1$$	
\begin{definition}
	Si $X$ et $Y$ sont 2 v.a. ou definit la \textsc{Covariance} entre $X$ et $Y$ comme
	$\cov(X,Y)\overset{\text{def}}{=}\E\left[(X-\E(X))(Y-\E(Y))\right]=\E(XY)-\E(X)\E(Y)$.
\end{definition}
\fi
\pagebreak

% \tableofcontents

% insert your code here
%\input{./algebra/main.tex}
%\input{./geometrie-differentielle/main.tex}
%\input{./probabilite/main.tex}
%\input{./analyse-fonctionnelle/main.tex}
% \input{./Analyse-convexe-et-dualite-en-optimisation/main.tex}
%\input{./tikz/main.tex}
%\input{./Theorie-du-distributions/main.tex}
%\input{./optimisation/mine.tex}
 \input{./modelisation/main.tex}

% yves.aubry@univ-tln.fr : algebra

\end{document}


% yves.aubry@univ-tln.fr : algebra

\end{document}

%% !TEX encoding = UTF-8 Unicode
% !TEX TS-program = xelatex

\documentclass[french]{report}

%\usepackage[utf8]{inputenc}
%\usepackage[T1]{fontenc}
\usepackage{babel}


\newif\ifcomment
%\commenttrue # Show comments

\usepackage{physics}
\usepackage{amssymb}


\usepackage{amsthm}
% \usepackage{thmtools}
\usepackage{mathtools}
\usepackage{amsfonts}

\usepackage{color}

\usepackage{tikz}

\usepackage{geometry}
\geometry{a5paper, margin=0.1in, right=1cm}

\usepackage{dsfont}

\usepackage{graphicx}
\graphicspath{ {images/} }

\usepackage{faktor}

\usepackage{IEEEtrantools}
\usepackage{enumerate}   
\usepackage[PostScript=dvips]{"/Users/aware/Documents/Courses/diagrams"}


\newtheorem{theorem}{Théorème}[section]
\renewcommand{\thetheorem}{\arabic{theorem}}
\newtheorem{lemme}{Lemme}[section]
\renewcommand{\thelemme}{\arabic{lemme}}
\newtheorem{proposition}{Proposition}[section]
\renewcommand{\theproposition}{\arabic{proposition}}
\newtheorem{notations}{Notations}[section]
\newtheorem{problem}{Problème}[section]
\newtheorem{corollary}{Corollaire}[theorem]
\renewcommand{\thecorollary}{\arabic{corollary}}
\newtheorem{property}{Propriété}[section]
\newtheorem{objective}{Objectif}[section]

\theoremstyle{definition}
\newtheorem{definition}{Définition}[section]
\renewcommand{\thedefinition}{\arabic{definition}}
\newtheorem{exercise}{Exercice}[chapter]
\renewcommand{\theexercise}{\arabic{exercise}}
\newtheorem{example}{Exemple}[chapter]
\renewcommand{\theexample}{\arabic{example}}
\newtheorem*{solution}{Solution}
\newtheorem*{application}{Application}
\newtheorem*{notation}{Notation}
\newtheorem*{vocabulary}{Vocabulaire}
\newtheorem*{properties}{Propriétés}



\theoremstyle{remark}
\newtheorem*{remark}{Remarque}
\newtheorem*{rappel}{Rappel}


\usepackage{etoolbox}
\AtBeginEnvironment{exercise}{\small}
\AtBeginEnvironment{example}{\small}

\usepackage{cases}
\usepackage[red]{mypack}

\usepackage[framemethod=TikZ]{mdframed}

\definecolor{bg}{rgb}{0.4,0.25,0.95}
\definecolor{pagebg}{rgb}{0,0,0.5}
\surroundwithmdframed[
   topline=false,
   rightline=false,
   bottomline=false,
   leftmargin=\parindent,
   skipabove=8pt,
   skipbelow=8pt,
   linecolor=blue,
   innerbottommargin=10pt,
   % backgroundcolor=bg,font=\color{orange}\sffamily, fontcolor=white
]{definition}

\usepackage{empheq}
\usepackage[most]{tcolorbox}

\newtcbox{\mymath}[1][]{%
    nobeforeafter, math upper, tcbox raise base,
    enhanced, colframe=blue!30!black,
    colback=red!10, boxrule=1pt,
    #1}

\usepackage{unixode}


\DeclareMathOperator{\ord}{ord}
\DeclareMathOperator{\orb}{orb}
\DeclareMathOperator{\stab}{stab}
\DeclareMathOperator{\Stab}{stab}
\DeclareMathOperator{\ppcm}{ppcm}
\DeclareMathOperator{\conj}{Conj}
\DeclareMathOperator{\End}{End}
\DeclareMathOperator{\rot}{rot}
\DeclareMathOperator{\trs}{trace}
\DeclareMathOperator{\Ind}{Ind}
\DeclareMathOperator{\mat}{Mat}
\DeclareMathOperator{\id}{Id}
\DeclareMathOperator{\vect}{vect}
\DeclareMathOperator{\img}{img}
\DeclareMathOperator{\cov}{Cov}
\DeclareMathOperator{\dist}{dist}
\DeclareMathOperator{\irr}{Irr}
\DeclareMathOperator{\image}{Im}
\DeclareMathOperator{\pd}{\partial}
\DeclareMathOperator{\epi}{epi}
\DeclareMathOperator{\Argmin}{Argmin}
\DeclareMathOperator{\dom}{dom}
\DeclareMathOperator{\proj}{proj}
\DeclareMathOperator{\ctg}{ctg}
\DeclareMathOperator{\supp}{supp}
\DeclareMathOperator{\argmin}{argmin}
\DeclareMathOperator{\mult}{mult}
\DeclareMathOperator{\ch}{ch}
\DeclareMathOperator{\sh}{sh}
\DeclareMathOperator{\rang}{rang}
\DeclareMathOperator{\diam}{diam}
\DeclareMathOperator{\Epigraphe}{Epigraphe}




\usepackage{xcolor}
\everymath{\color{blue}}
%\everymath{\color[rgb]{0,1,1}}
%\pagecolor[rgb]{0,0,0.5}


\newcommand*{\pdtest}[3][]{\ensuremath{\frac{\partial^{#1} #2}{\partial #3}}}

\newcommand*{\deffunc}[6][]{\ensuremath{
\begin{array}{rcl}
#2 : #3 &\rightarrow& #4\\
#5 &\mapsto& #6
\end{array}
}}

\newcommand{\eqcolon}{\mathrel{\resizebox{\widthof{$\mathord{=}$}}{\height}{ $\!\!=\!\!\resizebox{1.2\width}{0.8\height}{\raisebox{0.23ex}{$\mathop{:}$}}\!\!$ }}}
\newcommand{\coloneq}{\mathrel{\resizebox{\widthof{$\mathord{=}$}}{\height}{ $\!\!\resizebox{1.2\width}{0.8\height}{\raisebox{0.23ex}{$\mathop{:}$}}\!\!=\!\!$ }}}
\newcommand{\eqcolonl}{\ensuremath{\mathrel{=\!\!\mathop{:}}}}
\newcommand{\coloneql}{\ensuremath{\mathrel{\mathop{:} \!\! =}}}
\newcommand{\vc}[1]{% inline column vector
  \left(\begin{smallmatrix}#1\end{smallmatrix}\right)%
}
\newcommand{\vr}[1]{% inline row vector
  \begin{smallmatrix}(\,#1\,)\end{smallmatrix}%
}
\makeatletter
\newcommand*{\defeq}{\ =\mathrel{\rlap{%
                     \raisebox{0.3ex}{$\m@th\cdot$}}%
                     \raisebox{-0.3ex}{$\m@th\cdot$}}%
                     }
\makeatother

\newcommand{\mathcircle}[1]{% inline row vector
 \overset{\circ}{#1}
}
\newcommand{\ulim}{% low limit
 \underline{\lim}
}
\newcommand{\ssi}{% iff
\iff
}
\newcommand{\ps}[2]{
\expval{#1 | #2}
}
\newcommand{\df}[1]{
\mqty{#1}
}
\newcommand{\n}[1]{
\norm{#1}
}
\newcommand{\sys}[1]{
\left\{\smqty{#1}\right.
}


\newcommand{\eqdef}{\ensuremath{\overset{\text{def}}=}}


\def\Circlearrowright{\ensuremath{%
  \rotatebox[origin=c]{230}{$\circlearrowright$}}}

\newcommand\ct[1]{\text{\rmfamily\upshape #1}}
\newcommand\question[1]{ {\color{red} ...!? \small #1}}
\newcommand\caz[1]{\left\{\begin{array} #1 \end{array}\right.}
\newcommand\const{\text{\rmfamily\upshape const}}
\newcommand\toP{ \overset{\pro}{\to}}
\newcommand\toPP{ \overset{\text{PP}}{\to}}
\newcommand{\oeq}{\mathrel{\text{\textcircled{$=$}}}}





\usepackage{xcolor}
% \usepackage[normalem]{ulem}
\usepackage{lipsum}
\makeatletter
% \newcommand\colorwave[1][blue]{\bgroup \markoverwith{\lower3.5\p@\hbox{\sixly \textcolor{#1}{\char58}}}\ULon}
%\font\sixly=lasy6 % does not re-load if already loaded, so no memory problem.

\newmdtheoremenv[
linewidth= 1pt,linecolor= blue,%
leftmargin=20,rightmargin=20,innertopmargin=0pt, innerrightmargin=40,%
tikzsetting = { draw=lightgray, line width = 0.3pt,dashed,%
dash pattern = on 15pt off 3pt},%
splittopskip=\topskip,skipbelow=\baselineskip,%
skipabove=\baselineskip,ntheorem,roundcorner=0pt,
% backgroundcolor=pagebg,font=\color{orange}\sffamily, fontcolor=white
]{examplebox}{Exemple}[section]



\newcommand\R{\mathbb{R}}
\newcommand\Z{\mathbb{Z}}
\newcommand\N{\mathbb{N}}
\newcommand\E{\mathbb{E}}
\newcommand\F{\mathcal{F}}
\newcommand\cH{\mathcal{H}}
\newcommand\V{\mathbb{V}}
\newcommand\dmo{ ^{-1} }
\newcommand\kapa{\kappa}
\newcommand\im{Im}
\newcommand\hs{\mathcal{H}}





\usepackage{soul}

\makeatletter
\newcommand*{\whiten}[1]{\llap{\textcolor{white}{{\the\SOUL@token}}\hspace{#1pt}}}
\DeclareRobustCommand*\myul{%
    \def\SOUL@everyspace{\underline{\space}\kern\z@}%
    \def\SOUL@everytoken{%
     \setbox0=\hbox{\the\SOUL@token}%
     \ifdim\dp0>\z@
        \raisebox{\dp0}{\underline{\phantom{\the\SOUL@token}}}%
        \whiten{1}\whiten{0}%
        \whiten{-1}\whiten{-2}%
        \llap{\the\SOUL@token}%
     \else
        \underline{\the\SOUL@token}%
     \fi}%
\SOUL@}
\makeatother

\newcommand*{\demp}{\fontfamily{lmtt}\selectfont}

\DeclareTextFontCommand{\textdemp}{\demp}

\begin{document}

\ifcomment
Multiline
comment
\fi
\ifcomment
\myul{Typesetting test}
% \color[rgb]{1,1,1}
$∑_i^n≠ 60º±∞π∆¬≈√j∫h≤≥µ$

$\CR \R\pro\ind\pro\gS\pro
\mqty[a&b\\c&d]$
$\pro\mathbb{P}$
$\dd{x}$

  \[
    \alpha(x)=\left\{
                \begin{array}{ll}
                  x\\
                  \frac{1}{1+e^{-kx}}\\
                  \frac{e^x-e^{-x}}{e^x+e^{-x}}
                \end{array}
              \right.
  \]

  $\expval{x}$
  
  $\chi_\rho(ghg\dmo)=\Tr(\rho_{ghg\dmo})=\Tr(\rho_g\circ\rho_h\circ\rho\dmo_g)=\Tr(\rho_h)\overset{\mbox{\scalebox{0.5}{$\Tr(AB)=\Tr(BA)$}}}{=}\chi_\rho(h)$
  	$\mathop{\oplus}_{\substack{x\in X}}$

$\mat(\rho_g)=(a_{ij}(g))_{\scriptsize \substack{1\leq i\leq d \\ 1\leq j\leq d}}$ et $\mat(\rho'_g)=(a'_{ij}(g))_{\scriptsize \substack{1\leq i'\leq d' \\ 1\leq j'\leq d'}}$



\[\int_a^b{\mathbb{R}^2}g(u, v)\dd{P_{XY}}(u, v)=\iint g(u,v) f_{XY}(u, v)\dd \lambda(u) \dd \lambda(v)\]
$$\lim_{x\to\infty} f(x)$$	
$$\iiiint_V \mu(t,u,v,w) \,dt\,du\,dv\,dw$$
$$\sum_{n=1}^{\infty} 2^{-n} = 1$$	
\begin{definition}
	Si $X$ et $Y$ sont 2 v.a. ou definit la \textsc{Covariance} entre $X$ et $Y$ comme
	$\cov(X,Y)\overset{\text{def}}{=}\E\left[(X-\E(X))(Y-\E(Y))\right]=\E(XY)-\E(X)\E(Y)$.
\end{definition}
\fi
\pagebreak

% \tableofcontents

% insert your code here
%% !TEX encoding = UTF-8 Unicode
% !TEX TS-program = xelatex

\documentclass[french]{report}

%\usepackage[utf8]{inputenc}
%\usepackage[T1]{fontenc}
\usepackage{babel}


\newif\ifcomment
%\commenttrue # Show comments

\usepackage{physics}
\usepackage{amssymb}


\usepackage{amsthm}
% \usepackage{thmtools}
\usepackage{mathtools}
\usepackage{amsfonts}

\usepackage{color}

\usepackage{tikz}

\usepackage{geometry}
\geometry{a5paper, margin=0.1in, right=1cm}

\usepackage{dsfont}

\usepackage{graphicx}
\graphicspath{ {images/} }

\usepackage{faktor}

\usepackage{IEEEtrantools}
\usepackage{enumerate}   
\usepackage[PostScript=dvips]{"/Users/aware/Documents/Courses/diagrams"}


\newtheorem{theorem}{Théorème}[section]
\renewcommand{\thetheorem}{\arabic{theorem}}
\newtheorem{lemme}{Lemme}[section]
\renewcommand{\thelemme}{\arabic{lemme}}
\newtheorem{proposition}{Proposition}[section]
\renewcommand{\theproposition}{\arabic{proposition}}
\newtheorem{notations}{Notations}[section]
\newtheorem{problem}{Problème}[section]
\newtheorem{corollary}{Corollaire}[theorem]
\renewcommand{\thecorollary}{\arabic{corollary}}
\newtheorem{property}{Propriété}[section]
\newtheorem{objective}{Objectif}[section]

\theoremstyle{definition}
\newtheorem{definition}{Définition}[section]
\renewcommand{\thedefinition}{\arabic{definition}}
\newtheorem{exercise}{Exercice}[chapter]
\renewcommand{\theexercise}{\arabic{exercise}}
\newtheorem{example}{Exemple}[chapter]
\renewcommand{\theexample}{\arabic{example}}
\newtheorem*{solution}{Solution}
\newtheorem*{application}{Application}
\newtheorem*{notation}{Notation}
\newtheorem*{vocabulary}{Vocabulaire}
\newtheorem*{properties}{Propriétés}



\theoremstyle{remark}
\newtheorem*{remark}{Remarque}
\newtheorem*{rappel}{Rappel}


\usepackage{etoolbox}
\AtBeginEnvironment{exercise}{\small}
\AtBeginEnvironment{example}{\small}

\usepackage{cases}
\usepackage[red]{mypack}

\usepackage[framemethod=TikZ]{mdframed}

\definecolor{bg}{rgb}{0.4,0.25,0.95}
\definecolor{pagebg}{rgb}{0,0,0.5}
\surroundwithmdframed[
   topline=false,
   rightline=false,
   bottomline=false,
   leftmargin=\parindent,
   skipabove=8pt,
   skipbelow=8pt,
   linecolor=blue,
   innerbottommargin=10pt,
   % backgroundcolor=bg,font=\color{orange}\sffamily, fontcolor=white
]{definition}

\usepackage{empheq}
\usepackage[most]{tcolorbox}

\newtcbox{\mymath}[1][]{%
    nobeforeafter, math upper, tcbox raise base,
    enhanced, colframe=blue!30!black,
    colback=red!10, boxrule=1pt,
    #1}

\usepackage{unixode}


\DeclareMathOperator{\ord}{ord}
\DeclareMathOperator{\orb}{orb}
\DeclareMathOperator{\stab}{stab}
\DeclareMathOperator{\Stab}{stab}
\DeclareMathOperator{\ppcm}{ppcm}
\DeclareMathOperator{\conj}{Conj}
\DeclareMathOperator{\End}{End}
\DeclareMathOperator{\rot}{rot}
\DeclareMathOperator{\trs}{trace}
\DeclareMathOperator{\Ind}{Ind}
\DeclareMathOperator{\mat}{Mat}
\DeclareMathOperator{\id}{Id}
\DeclareMathOperator{\vect}{vect}
\DeclareMathOperator{\img}{img}
\DeclareMathOperator{\cov}{Cov}
\DeclareMathOperator{\dist}{dist}
\DeclareMathOperator{\irr}{Irr}
\DeclareMathOperator{\image}{Im}
\DeclareMathOperator{\pd}{\partial}
\DeclareMathOperator{\epi}{epi}
\DeclareMathOperator{\Argmin}{Argmin}
\DeclareMathOperator{\dom}{dom}
\DeclareMathOperator{\proj}{proj}
\DeclareMathOperator{\ctg}{ctg}
\DeclareMathOperator{\supp}{supp}
\DeclareMathOperator{\argmin}{argmin}
\DeclareMathOperator{\mult}{mult}
\DeclareMathOperator{\ch}{ch}
\DeclareMathOperator{\sh}{sh}
\DeclareMathOperator{\rang}{rang}
\DeclareMathOperator{\diam}{diam}
\DeclareMathOperator{\Epigraphe}{Epigraphe}




\usepackage{xcolor}
\everymath{\color{blue}}
%\everymath{\color[rgb]{0,1,1}}
%\pagecolor[rgb]{0,0,0.5}


\newcommand*{\pdtest}[3][]{\ensuremath{\frac{\partial^{#1} #2}{\partial #3}}}

\newcommand*{\deffunc}[6][]{\ensuremath{
\begin{array}{rcl}
#2 : #3 &\rightarrow& #4\\
#5 &\mapsto& #6
\end{array}
}}

\newcommand{\eqcolon}{\mathrel{\resizebox{\widthof{$\mathord{=}$}}{\height}{ $\!\!=\!\!\resizebox{1.2\width}{0.8\height}{\raisebox{0.23ex}{$\mathop{:}$}}\!\!$ }}}
\newcommand{\coloneq}{\mathrel{\resizebox{\widthof{$\mathord{=}$}}{\height}{ $\!\!\resizebox{1.2\width}{0.8\height}{\raisebox{0.23ex}{$\mathop{:}$}}\!\!=\!\!$ }}}
\newcommand{\eqcolonl}{\ensuremath{\mathrel{=\!\!\mathop{:}}}}
\newcommand{\coloneql}{\ensuremath{\mathrel{\mathop{:} \!\! =}}}
\newcommand{\vc}[1]{% inline column vector
  \left(\begin{smallmatrix}#1\end{smallmatrix}\right)%
}
\newcommand{\vr}[1]{% inline row vector
  \begin{smallmatrix}(\,#1\,)\end{smallmatrix}%
}
\makeatletter
\newcommand*{\defeq}{\ =\mathrel{\rlap{%
                     \raisebox{0.3ex}{$\m@th\cdot$}}%
                     \raisebox{-0.3ex}{$\m@th\cdot$}}%
                     }
\makeatother

\newcommand{\mathcircle}[1]{% inline row vector
 \overset{\circ}{#1}
}
\newcommand{\ulim}{% low limit
 \underline{\lim}
}
\newcommand{\ssi}{% iff
\iff
}
\newcommand{\ps}[2]{
\expval{#1 | #2}
}
\newcommand{\df}[1]{
\mqty{#1}
}
\newcommand{\n}[1]{
\norm{#1}
}
\newcommand{\sys}[1]{
\left\{\smqty{#1}\right.
}


\newcommand{\eqdef}{\ensuremath{\overset{\text{def}}=}}


\def\Circlearrowright{\ensuremath{%
  \rotatebox[origin=c]{230}{$\circlearrowright$}}}

\newcommand\ct[1]{\text{\rmfamily\upshape #1}}
\newcommand\question[1]{ {\color{red} ...!? \small #1}}
\newcommand\caz[1]{\left\{\begin{array} #1 \end{array}\right.}
\newcommand\const{\text{\rmfamily\upshape const}}
\newcommand\toP{ \overset{\pro}{\to}}
\newcommand\toPP{ \overset{\text{PP}}{\to}}
\newcommand{\oeq}{\mathrel{\text{\textcircled{$=$}}}}





\usepackage{xcolor}
% \usepackage[normalem]{ulem}
\usepackage{lipsum}
\makeatletter
% \newcommand\colorwave[1][blue]{\bgroup \markoverwith{\lower3.5\p@\hbox{\sixly \textcolor{#1}{\char58}}}\ULon}
%\font\sixly=lasy6 % does not re-load if already loaded, so no memory problem.

\newmdtheoremenv[
linewidth= 1pt,linecolor= blue,%
leftmargin=20,rightmargin=20,innertopmargin=0pt, innerrightmargin=40,%
tikzsetting = { draw=lightgray, line width = 0.3pt,dashed,%
dash pattern = on 15pt off 3pt},%
splittopskip=\topskip,skipbelow=\baselineskip,%
skipabove=\baselineskip,ntheorem,roundcorner=0pt,
% backgroundcolor=pagebg,font=\color{orange}\sffamily, fontcolor=white
]{examplebox}{Exemple}[section]



\newcommand\R{\mathbb{R}}
\newcommand\Z{\mathbb{Z}}
\newcommand\N{\mathbb{N}}
\newcommand\E{\mathbb{E}}
\newcommand\F{\mathcal{F}}
\newcommand\cH{\mathcal{H}}
\newcommand\V{\mathbb{V}}
\newcommand\dmo{ ^{-1} }
\newcommand\kapa{\kappa}
\newcommand\im{Im}
\newcommand\hs{\mathcal{H}}





\usepackage{soul}

\makeatletter
\newcommand*{\whiten}[1]{\llap{\textcolor{white}{{\the\SOUL@token}}\hspace{#1pt}}}
\DeclareRobustCommand*\myul{%
    \def\SOUL@everyspace{\underline{\space}\kern\z@}%
    \def\SOUL@everytoken{%
     \setbox0=\hbox{\the\SOUL@token}%
     \ifdim\dp0>\z@
        \raisebox{\dp0}{\underline{\phantom{\the\SOUL@token}}}%
        \whiten{1}\whiten{0}%
        \whiten{-1}\whiten{-2}%
        \llap{\the\SOUL@token}%
     \else
        \underline{\the\SOUL@token}%
     \fi}%
\SOUL@}
\makeatother

\newcommand*{\demp}{\fontfamily{lmtt}\selectfont}

\DeclareTextFontCommand{\textdemp}{\demp}

\begin{document}

\ifcomment
Multiline
comment
\fi
\ifcomment
\myul{Typesetting test}
% \color[rgb]{1,1,1}
$∑_i^n≠ 60º±∞π∆¬≈√j∫h≤≥µ$

$\CR \R\pro\ind\pro\gS\pro
\mqty[a&b\\c&d]$
$\pro\mathbb{P}$
$\dd{x}$

  \[
    \alpha(x)=\left\{
                \begin{array}{ll}
                  x\\
                  \frac{1}{1+e^{-kx}}\\
                  \frac{e^x-e^{-x}}{e^x+e^{-x}}
                \end{array}
              \right.
  \]

  $\expval{x}$
  
  $\chi_\rho(ghg\dmo)=\Tr(\rho_{ghg\dmo})=\Tr(\rho_g\circ\rho_h\circ\rho\dmo_g)=\Tr(\rho_h)\overset{\mbox{\scalebox{0.5}{$\Tr(AB)=\Tr(BA)$}}}{=}\chi_\rho(h)$
  	$\mathop{\oplus}_{\substack{x\in X}}$

$\mat(\rho_g)=(a_{ij}(g))_{\scriptsize \substack{1\leq i\leq d \\ 1\leq j\leq d}}$ et $\mat(\rho'_g)=(a'_{ij}(g))_{\scriptsize \substack{1\leq i'\leq d' \\ 1\leq j'\leq d'}}$



\[\int_a^b{\mathbb{R}^2}g(u, v)\dd{P_{XY}}(u, v)=\iint g(u,v) f_{XY}(u, v)\dd \lambda(u) \dd \lambda(v)\]
$$\lim_{x\to\infty} f(x)$$	
$$\iiiint_V \mu(t,u,v,w) \,dt\,du\,dv\,dw$$
$$\sum_{n=1}^{\infty} 2^{-n} = 1$$	
\begin{definition}
	Si $X$ et $Y$ sont 2 v.a. ou definit la \textsc{Covariance} entre $X$ et $Y$ comme
	$\cov(X,Y)\overset{\text{def}}{=}\E\left[(X-\E(X))(Y-\E(Y))\right]=\E(XY)-\E(X)\E(Y)$.
\end{definition}
\fi
\pagebreak

% \tableofcontents

% insert your code here
%\input{./algebra/main.tex}
%\input{./geometrie-differentielle/main.tex}
%\input{./probabilite/main.tex}
%\input{./analyse-fonctionnelle/main.tex}
% \input{./Analyse-convexe-et-dualite-en-optimisation/main.tex}
%\input{./tikz/main.tex}
%\input{./Theorie-du-distributions/main.tex}
%\input{./optimisation/mine.tex}
 \input{./modelisation/main.tex}

% yves.aubry@univ-tln.fr : algebra

\end{document}

%% !TEX encoding = UTF-8 Unicode
% !TEX TS-program = xelatex

\documentclass[french]{report}

%\usepackage[utf8]{inputenc}
%\usepackage[T1]{fontenc}
\usepackage{babel}


\newif\ifcomment
%\commenttrue # Show comments

\usepackage{physics}
\usepackage{amssymb}


\usepackage{amsthm}
% \usepackage{thmtools}
\usepackage{mathtools}
\usepackage{amsfonts}

\usepackage{color}

\usepackage{tikz}

\usepackage{geometry}
\geometry{a5paper, margin=0.1in, right=1cm}

\usepackage{dsfont}

\usepackage{graphicx}
\graphicspath{ {images/} }

\usepackage{faktor}

\usepackage{IEEEtrantools}
\usepackage{enumerate}   
\usepackage[PostScript=dvips]{"/Users/aware/Documents/Courses/diagrams"}


\newtheorem{theorem}{Théorème}[section]
\renewcommand{\thetheorem}{\arabic{theorem}}
\newtheorem{lemme}{Lemme}[section]
\renewcommand{\thelemme}{\arabic{lemme}}
\newtheorem{proposition}{Proposition}[section]
\renewcommand{\theproposition}{\arabic{proposition}}
\newtheorem{notations}{Notations}[section]
\newtheorem{problem}{Problème}[section]
\newtheorem{corollary}{Corollaire}[theorem]
\renewcommand{\thecorollary}{\arabic{corollary}}
\newtheorem{property}{Propriété}[section]
\newtheorem{objective}{Objectif}[section]

\theoremstyle{definition}
\newtheorem{definition}{Définition}[section]
\renewcommand{\thedefinition}{\arabic{definition}}
\newtheorem{exercise}{Exercice}[chapter]
\renewcommand{\theexercise}{\arabic{exercise}}
\newtheorem{example}{Exemple}[chapter]
\renewcommand{\theexample}{\arabic{example}}
\newtheorem*{solution}{Solution}
\newtheorem*{application}{Application}
\newtheorem*{notation}{Notation}
\newtheorem*{vocabulary}{Vocabulaire}
\newtheorem*{properties}{Propriétés}



\theoremstyle{remark}
\newtheorem*{remark}{Remarque}
\newtheorem*{rappel}{Rappel}


\usepackage{etoolbox}
\AtBeginEnvironment{exercise}{\small}
\AtBeginEnvironment{example}{\small}

\usepackage{cases}
\usepackage[red]{mypack}

\usepackage[framemethod=TikZ]{mdframed}

\definecolor{bg}{rgb}{0.4,0.25,0.95}
\definecolor{pagebg}{rgb}{0,0,0.5}
\surroundwithmdframed[
   topline=false,
   rightline=false,
   bottomline=false,
   leftmargin=\parindent,
   skipabove=8pt,
   skipbelow=8pt,
   linecolor=blue,
   innerbottommargin=10pt,
   % backgroundcolor=bg,font=\color{orange}\sffamily, fontcolor=white
]{definition}

\usepackage{empheq}
\usepackage[most]{tcolorbox}

\newtcbox{\mymath}[1][]{%
    nobeforeafter, math upper, tcbox raise base,
    enhanced, colframe=blue!30!black,
    colback=red!10, boxrule=1pt,
    #1}

\usepackage{unixode}


\DeclareMathOperator{\ord}{ord}
\DeclareMathOperator{\orb}{orb}
\DeclareMathOperator{\stab}{stab}
\DeclareMathOperator{\Stab}{stab}
\DeclareMathOperator{\ppcm}{ppcm}
\DeclareMathOperator{\conj}{Conj}
\DeclareMathOperator{\End}{End}
\DeclareMathOperator{\rot}{rot}
\DeclareMathOperator{\trs}{trace}
\DeclareMathOperator{\Ind}{Ind}
\DeclareMathOperator{\mat}{Mat}
\DeclareMathOperator{\id}{Id}
\DeclareMathOperator{\vect}{vect}
\DeclareMathOperator{\img}{img}
\DeclareMathOperator{\cov}{Cov}
\DeclareMathOperator{\dist}{dist}
\DeclareMathOperator{\irr}{Irr}
\DeclareMathOperator{\image}{Im}
\DeclareMathOperator{\pd}{\partial}
\DeclareMathOperator{\epi}{epi}
\DeclareMathOperator{\Argmin}{Argmin}
\DeclareMathOperator{\dom}{dom}
\DeclareMathOperator{\proj}{proj}
\DeclareMathOperator{\ctg}{ctg}
\DeclareMathOperator{\supp}{supp}
\DeclareMathOperator{\argmin}{argmin}
\DeclareMathOperator{\mult}{mult}
\DeclareMathOperator{\ch}{ch}
\DeclareMathOperator{\sh}{sh}
\DeclareMathOperator{\rang}{rang}
\DeclareMathOperator{\diam}{diam}
\DeclareMathOperator{\Epigraphe}{Epigraphe}




\usepackage{xcolor}
\everymath{\color{blue}}
%\everymath{\color[rgb]{0,1,1}}
%\pagecolor[rgb]{0,0,0.5}


\newcommand*{\pdtest}[3][]{\ensuremath{\frac{\partial^{#1} #2}{\partial #3}}}

\newcommand*{\deffunc}[6][]{\ensuremath{
\begin{array}{rcl}
#2 : #3 &\rightarrow& #4\\
#5 &\mapsto& #6
\end{array}
}}

\newcommand{\eqcolon}{\mathrel{\resizebox{\widthof{$\mathord{=}$}}{\height}{ $\!\!=\!\!\resizebox{1.2\width}{0.8\height}{\raisebox{0.23ex}{$\mathop{:}$}}\!\!$ }}}
\newcommand{\coloneq}{\mathrel{\resizebox{\widthof{$\mathord{=}$}}{\height}{ $\!\!\resizebox{1.2\width}{0.8\height}{\raisebox{0.23ex}{$\mathop{:}$}}\!\!=\!\!$ }}}
\newcommand{\eqcolonl}{\ensuremath{\mathrel{=\!\!\mathop{:}}}}
\newcommand{\coloneql}{\ensuremath{\mathrel{\mathop{:} \!\! =}}}
\newcommand{\vc}[1]{% inline column vector
  \left(\begin{smallmatrix}#1\end{smallmatrix}\right)%
}
\newcommand{\vr}[1]{% inline row vector
  \begin{smallmatrix}(\,#1\,)\end{smallmatrix}%
}
\makeatletter
\newcommand*{\defeq}{\ =\mathrel{\rlap{%
                     \raisebox{0.3ex}{$\m@th\cdot$}}%
                     \raisebox{-0.3ex}{$\m@th\cdot$}}%
                     }
\makeatother

\newcommand{\mathcircle}[1]{% inline row vector
 \overset{\circ}{#1}
}
\newcommand{\ulim}{% low limit
 \underline{\lim}
}
\newcommand{\ssi}{% iff
\iff
}
\newcommand{\ps}[2]{
\expval{#1 | #2}
}
\newcommand{\df}[1]{
\mqty{#1}
}
\newcommand{\n}[1]{
\norm{#1}
}
\newcommand{\sys}[1]{
\left\{\smqty{#1}\right.
}


\newcommand{\eqdef}{\ensuremath{\overset{\text{def}}=}}


\def\Circlearrowright{\ensuremath{%
  \rotatebox[origin=c]{230}{$\circlearrowright$}}}

\newcommand\ct[1]{\text{\rmfamily\upshape #1}}
\newcommand\question[1]{ {\color{red} ...!? \small #1}}
\newcommand\caz[1]{\left\{\begin{array} #1 \end{array}\right.}
\newcommand\const{\text{\rmfamily\upshape const}}
\newcommand\toP{ \overset{\pro}{\to}}
\newcommand\toPP{ \overset{\text{PP}}{\to}}
\newcommand{\oeq}{\mathrel{\text{\textcircled{$=$}}}}





\usepackage{xcolor}
% \usepackage[normalem]{ulem}
\usepackage{lipsum}
\makeatletter
% \newcommand\colorwave[1][blue]{\bgroup \markoverwith{\lower3.5\p@\hbox{\sixly \textcolor{#1}{\char58}}}\ULon}
%\font\sixly=lasy6 % does not re-load if already loaded, so no memory problem.

\newmdtheoremenv[
linewidth= 1pt,linecolor= blue,%
leftmargin=20,rightmargin=20,innertopmargin=0pt, innerrightmargin=40,%
tikzsetting = { draw=lightgray, line width = 0.3pt,dashed,%
dash pattern = on 15pt off 3pt},%
splittopskip=\topskip,skipbelow=\baselineskip,%
skipabove=\baselineskip,ntheorem,roundcorner=0pt,
% backgroundcolor=pagebg,font=\color{orange}\sffamily, fontcolor=white
]{examplebox}{Exemple}[section]



\newcommand\R{\mathbb{R}}
\newcommand\Z{\mathbb{Z}}
\newcommand\N{\mathbb{N}}
\newcommand\E{\mathbb{E}}
\newcommand\F{\mathcal{F}}
\newcommand\cH{\mathcal{H}}
\newcommand\V{\mathbb{V}}
\newcommand\dmo{ ^{-1} }
\newcommand\kapa{\kappa}
\newcommand\im{Im}
\newcommand\hs{\mathcal{H}}





\usepackage{soul}

\makeatletter
\newcommand*{\whiten}[1]{\llap{\textcolor{white}{{\the\SOUL@token}}\hspace{#1pt}}}
\DeclareRobustCommand*\myul{%
    \def\SOUL@everyspace{\underline{\space}\kern\z@}%
    \def\SOUL@everytoken{%
     \setbox0=\hbox{\the\SOUL@token}%
     \ifdim\dp0>\z@
        \raisebox{\dp0}{\underline{\phantom{\the\SOUL@token}}}%
        \whiten{1}\whiten{0}%
        \whiten{-1}\whiten{-2}%
        \llap{\the\SOUL@token}%
     \else
        \underline{\the\SOUL@token}%
     \fi}%
\SOUL@}
\makeatother

\newcommand*{\demp}{\fontfamily{lmtt}\selectfont}

\DeclareTextFontCommand{\textdemp}{\demp}

\begin{document}

\ifcomment
Multiline
comment
\fi
\ifcomment
\myul{Typesetting test}
% \color[rgb]{1,1,1}
$∑_i^n≠ 60º±∞π∆¬≈√j∫h≤≥µ$

$\CR \R\pro\ind\pro\gS\pro
\mqty[a&b\\c&d]$
$\pro\mathbb{P}$
$\dd{x}$

  \[
    \alpha(x)=\left\{
                \begin{array}{ll}
                  x\\
                  \frac{1}{1+e^{-kx}}\\
                  \frac{e^x-e^{-x}}{e^x+e^{-x}}
                \end{array}
              \right.
  \]

  $\expval{x}$
  
  $\chi_\rho(ghg\dmo)=\Tr(\rho_{ghg\dmo})=\Tr(\rho_g\circ\rho_h\circ\rho\dmo_g)=\Tr(\rho_h)\overset{\mbox{\scalebox{0.5}{$\Tr(AB)=\Tr(BA)$}}}{=}\chi_\rho(h)$
  	$\mathop{\oplus}_{\substack{x\in X}}$

$\mat(\rho_g)=(a_{ij}(g))_{\scriptsize \substack{1\leq i\leq d \\ 1\leq j\leq d}}$ et $\mat(\rho'_g)=(a'_{ij}(g))_{\scriptsize \substack{1\leq i'\leq d' \\ 1\leq j'\leq d'}}$



\[\int_a^b{\mathbb{R}^2}g(u, v)\dd{P_{XY}}(u, v)=\iint g(u,v) f_{XY}(u, v)\dd \lambda(u) \dd \lambda(v)\]
$$\lim_{x\to\infty} f(x)$$	
$$\iiiint_V \mu(t,u,v,w) \,dt\,du\,dv\,dw$$
$$\sum_{n=1}^{\infty} 2^{-n} = 1$$	
\begin{definition}
	Si $X$ et $Y$ sont 2 v.a. ou definit la \textsc{Covariance} entre $X$ et $Y$ comme
	$\cov(X,Y)\overset{\text{def}}{=}\E\left[(X-\E(X))(Y-\E(Y))\right]=\E(XY)-\E(X)\E(Y)$.
\end{definition}
\fi
\pagebreak

% \tableofcontents

% insert your code here
%\input{./algebra/main.tex}
%\input{./geometrie-differentielle/main.tex}
%\input{./probabilite/main.tex}
%\input{./analyse-fonctionnelle/main.tex}
% \input{./Analyse-convexe-et-dualite-en-optimisation/main.tex}
%\input{./tikz/main.tex}
%\input{./Theorie-du-distributions/main.tex}
%\input{./optimisation/mine.tex}
 \input{./modelisation/main.tex}

% yves.aubry@univ-tln.fr : algebra

\end{document}

%% !TEX encoding = UTF-8 Unicode
% !TEX TS-program = xelatex

\documentclass[french]{report}

%\usepackage[utf8]{inputenc}
%\usepackage[T1]{fontenc}
\usepackage{babel}


\newif\ifcomment
%\commenttrue # Show comments

\usepackage{physics}
\usepackage{amssymb}


\usepackage{amsthm}
% \usepackage{thmtools}
\usepackage{mathtools}
\usepackage{amsfonts}

\usepackage{color}

\usepackage{tikz}

\usepackage{geometry}
\geometry{a5paper, margin=0.1in, right=1cm}

\usepackage{dsfont}

\usepackage{graphicx}
\graphicspath{ {images/} }

\usepackage{faktor}

\usepackage{IEEEtrantools}
\usepackage{enumerate}   
\usepackage[PostScript=dvips]{"/Users/aware/Documents/Courses/diagrams"}


\newtheorem{theorem}{Théorème}[section]
\renewcommand{\thetheorem}{\arabic{theorem}}
\newtheorem{lemme}{Lemme}[section]
\renewcommand{\thelemme}{\arabic{lemme}}
\newtheorem{proposition}{Proposition}[section]
\renewcommand{\theproposition}{\arabic{proposition}}
\newtheorem{notations}{Notations}[section]
\newtheorem{problem}{Problème}[section]
\newtheorem{corollary}{Corollaire}[theorem]
\renewcommand{\thecorollary}{\arabic{corollary}}
\newtheorem{property}{Propriété}[section]
\newtheorem{objective}{Objectif}[section]

\theoremstyle{definition}
\newtheorem{definition}{Définition}[section]
\renewcommand{\thedefinition}{\arabic{definition}}
\newtheorem{exercise}{Exercice}[chapter]
\renewcommand{\theexercise}{\arabic{exercise}}
\newtheorem{example}{Exemple}[chapter]
\renewcommand{\theexample}{\arabic{example}}
\newtheorem*{solution}{Solution}
\newtheorem*{application}{Application}
\newtheorem*{notation}{Notation}
\newtheorem*{vocabulary}{Vocabulaire}
\newtheorem*{properties}{Propriétés}



\theoremstyle{remark}
\newtheorem*{remark}{Remarque}
\newtheorem*{rappel}{Rappel}


\usepackage{etoolbox}
\AtBeginEnvironment{exercise}{\small}
\AtBeginEnvironment{example}{\small}

\usepackage{cases}
\usepackage[red]{mypack}

\usepackage[framemethod=TikZ]{mdframed}

\definecolor{bg}{rgb}{0.4,0.25,0.95}
\definecolor{pagebg}{rgb}{0,0,0.5}
\surroundwithmdframed[
   topline=false,
   rightline=false,
   bottomline=false,
   leftmargin=\parindent,
   skipabove=8pt,
   skipbelow=8pt,
   linecolor=blue,
   innerbottommargin=10pt,
   % backgroundcolor=bg,font=\color{orange}\sffamily, fontcolor=white
]{definition}

\usepackage{empheq}
\usepackage[most]{tcolorbox}

\newtcbox{\mymath}[1][]{%
    nobeforeafter, math upper, tcbox raise base,
    enhanced, colframe=blue!30!black,
    colback=red!10, boxrule=1pt,
    #1}

\usepackage{unixode}


\DeclareMathOperator{\ord}{ord}
\DeclareMathOperator{\orb}{orb}
\DeclareMathOperator{\stab}{stab}
\DeclareMathOperator{\Stab}{stab}
\DeclareMathOperator{\ppcm}{ppcm}
\DeclareMathOperator{\conj}{Conj}
\DeclareMathOperator{\End}{End}
\DeclareMathOperator{\rot}{rot}
\DeclareMathOperator{\trs}{trace}
\DeclareMathOperator{\Ind}{Ind}
\DeclareMathOperator{\mat}{Mat}
\DeclareMathOperator{\id}{Id}
\DeclareMathOperator{\vect}{vect}
\DeclareMathOperator{\img}{img}
\DeclareMathOperator{\cov}{Cov}
\DeclareMathOperator{\dist}{dist}
\DeclareMathOperator{\irr}{Irr}
\DeclareMathOperator{\image}{Im}
\DeclareMathOperator{\pd}{\partial}
\DeclareMathOperator{\epi}{epi}
\DeclareMathOperator{\Argmin}{Argmin}
\DeclareMathOperator{\dom}{dom}
\DeclareMathOperator{\proj}{proj}
\DeclareMathOperator{\ctg}{ctg}
\DeclareMathOperator{\supp}{supp}
\DeclareMathOperator{\argmin}{argmin}
\DeclareMathOperator{\mult}{mult}
\DeclareMathOperator{\ch}{ch}
\DeclareMathOperator{\sh}{sh}
\DeclareMathOperator{\rang}{rang}
\DeclareMathOperator{\diam}{diam}
\DeclareMathOperator{\Epigraphe}{Epigraphe}




\usepackage{xcolor}
\everymath{\color{blue}}
%\everymath{\color[rgb]{0,1,1}}
%\pagecolor[rgb]{0,0,0.5}


\newcommand*{\pdtest}[3][]{\ensuremath{\frac{\partial^{#1} #2}{\partial #3}}}

\newcommand*{\deffunc}[6][]{\ensuremath{
\begin{array}{rcl}
#2 : #3 &\rightarrow& #4\\
#5 &\mapsto& #6
\end{array}
}}

\newcommand{\eqcolon}{\mathrel{\resizebox{\widthof{$\mathord{=}$}}{\height}{ $\!\!=\!\!\resizebox{1.2\width}{0.8\height}{\raisebox{0.23ex}{$\mathop{:}$}}\!\!$ }}}
\newcommand{\coloneq}{\mathrel{\resizebox{\widthof{$\mathord{=}$}}{\height}{ $\!\!\resizebox{1.2\width}{0.8\height}{\raisebox{0.23ex}{$\mathop{:}$}}\!\!=\!\!$ }}}
\newcommand{\eqcolonl}{\ensuremath{\mathrel{=\!\!\mathop{:}}}}
\newcommand{\coloneql}{\ensuremath{\mathrel{\mathop{:} \!\! =}}}
\newcommand{\vc}[1]{% inline column vector
  \left(\begin{smallmatrix}#1\end{smallmatrix}\right)%
}
\newcommand{\vr}[1]{% inline row vector
  \begin{smallmatrix}(\,#1\,)\end{smallmatrix}%
}
\makeatletter
\newcommand*{\defeq}{\ =\mathrel{\rlap{%
                     \raisebox{0.3ex}{$\m@th\cdot$}}%
                     \raisebox{-0.3ex}{$\m@th\cdot$}}%
                     }
\makeatother

\newcommand{\mathcircle}[1]{% inline row vector
 \overset{\circ}{#1}
}
\newcommand{\ulim}{% low limit
 \underline{\lim}
}
\newcommand{\ssi}{% iff
\iff
}
\newcommand{\ps}[2]{
\expval{#1 | #2}
}
\newcommand{\df}[1]{
\mqty{#1}
}
\newcommand{\n}[1]{
\norm{#1}
}
\newcommand{\sys}[1]{
\left\{\smqty{#1}\right.
}


\newcommand{\eqdef}{\ensuremath{\overset{\text{def}}=}}


\def\Circlearrowright{\ensuremath{%
  \rotatebox[origin=c]{230}{$\circlearrowright$}}}

\newcommand\ct[1]{\text{\rmfamily\upshape #1}}
\newcommand\question[1]{ {\color{red} ...!? \small #1}}
\newcommand\caz[1]{\left\{\begin{array} #1 \end{array}\right.}
\newcommand\const{\text{\rmfamily\upshape const}}
\newcommand\toP{ \overset{\pro}{\to}}
\newcommand\toPP{ \overset{\text{PP}}{\to}}
\newcommand{\oeq}{\mathrel{\text{\textcircled{$=$}}}}





\usepackage{xcolor}
% \usepackage[normalem]{ulem}
\usepackage{lipsum}
\makeatletter
% \newcommand\colorwave[1][blue]{\bgroup \markoverwith{\lower3.5\p@\hbox{\sixly \textcolor{#1}{\char58}}}\ULon}
%\font\sixly=lasy6 % does not re-load if already loaded, so no memory problem.

\newmdtheoremenv[
linewidth= 1pt,linecolor= blue,%
leftmargin=20,rightmargin=20,innertopmargin=0pt, innerrightmargin=40,%
tikzsetting = { draw=lightgray, line width = 0.3pt,dashed,%
dash pattern = on 15pt off 3pt},%
splittopskip=\topskip,skipbelow=\baselineskip,%
skipabove=\baselineskip,ntheorem,roundcorner=0pt,
% backgroundcolor=pagebg,font=\color{orange}\sffamily, fontcolor=white
]{examplebox}{Exemple}[section]



\newcommand\R{\mathbb{R}}
\newcommand\Z{\mathbb{Z}}
\newcommand\N{\mathbb{N}}
\newcommand\E{\mathbb{E}}
\newcommand\F{\mathcal{F}}
\newcommand\cH{\mathcal{H}}
\newcommand\V{\mathbb{V}}
\newcommand\dmo{ ^{-1} }
\newcommand\kapa{\kappa}
\newcommand\im{Im}
\newcommand\hs{\mathcal{H}}





\usepackage{soul}

\makeatletter
\newcommand*{\whiten}[1]{\llap{\textcolor{white}{{\the\SOUL@token}}\hspace{#1pt}}}
\DeclareRobustCommand*\myul{%
    \def\SOUL@everyspace{\underline{\space}\kern\z@}%
    \def\SOUL@everytoken{%
     \setbox0=\hbox{\the\SOUL@token}%
     \ifdim\dp0>\z@
        \raisebox{\dp0}{\underline{\phantom{\the\SOUL@token}}}%
        \whiten{1}\whiten{0}%
        \whiten{-1}\whiten{-2}%
        \llap{\the\SOUL@token}%
     \else
        \underline{\the\SOUL@token}%
     \fi}%
\SOUL@}
\makeatother

\newcommand*{\demp}{\fontfamily{lmtt}\selectfont}

\DeclareTextFontCommand{\textdemp}{\demp}

\begin{document}

\ifcomment
Multiline
comment
\fi
\ifcomment
\myul{Typesetting test}
% \color[rgb]{1,1,1}
$∑_i^n≠ 60º±∞π∆¬≈√j∫h≤≥µ$

$\CR \R\pro\ind\pro\gS\pro
\mqty[a&b\\c&d]$
$\pro\mathbb{P}$
$\dd{x}$

  \[
    \alpha(x)=\left\{
                \begin{array}{ll}
                  x\\
                  \frac{1}{1+e^{-kx}}\\
                  \frac{e^x-e^{-x}}{e^x+e^{-x}}
                \end{array}
              \right.
  \]

  $\expval{x}$
  
  $\chi_\rho(ghg\dmo)=\Tr(\rho_{ghg\dmo})=\Tr(\rho_g\circ\rho_h\circ\rho\dmo_g)=\Tr(\rho_h)\overset{\mbox{\scalebox{0.5}{$\Tr(AB)=\Tr(BA)$}}}{=}\chi_\rho(h)$
  	$\mathop{\oplus}_{\substack{x\in X}}$

$\mat(\rho_g)=(a_{ij}(g))_{\scriptsize \substack{1\leq i\leq d \\ 1\leq j\leq d}}$ et $\mat(\rho'_g)=(a'_{ij}(g))_{\scriptsize \substack{1\leq i'\leq d' \\ 1\leq j'\leq d'}}$



\[\int_a^b{\mathbb{R}^2}g(u, v)\dd{P_{XY}}(u, v)=\iint g(u,v) f_{XY}(u, v)\dd \lambda(u) \dd \lambda(v)\]
$$\lim_{x\to\infty} f(x)$$	
$$\iiiint_V \mu(t,u,v,w) \,dt\,du\,dv\,dw$$
$$\sum_{n=1}^{\infty} 2^{-n} = 1$$	
\begin{definition}
	Si $X$ et $Y$ sont 2 v.a. ou definit la \textsc{Covariance} entre $X$ et $Y$ comme
	$\cov(X,Y)\overset{\text{def}}{=}\E\left[(X-\E(X))(Y-\E(Y))\right]=\E(XY)-\E(X)\E(Y)$.
\end{definition}
\fi
\pagebreak

% \tableofcontents

% insert your code here
%\input{./algebra/main.tex}
%\input{./geometrie-differentielle/main.tex}
%\input{./probabilite/main.tex}
%\input{./analyse-fonctionnelle/main.tex}
% \input{./Analyse-convexe-et-dualite-en-optimisation/main.tex}
%\input{./tikz/main.tex}
%\input{./Theorie-du-distributions/main.tex}
%\input{./optimisation/mine.tex}
 \input{./modelisation/main.tex}

% yves.aubry@univ-tln.fr : algebra

\end{document}

%% !TEX encoding = UTF-8 Unicode
% !TEX TS-program = xelatex

\documentclass[french]{report}

%\usepackage[utf8]{inputenc}
%\usepackage[T1]{fontenc}
\usepackage{babel}


\newif\ifcomment
%\commenttrue # Show comments

\usepackage{physics}
\usepackage{amssymb}


\usepackage{amsthm}
% \usepackage{thmtools}
\usepackage{mathtools}
\usepackage{amsfonts}

\usepackage{color}

\usepackage{tikz}

\usepackage{geometry}
\geometry{a5paper, margin=0.1in, right=1cm}

\usepackage{dsfont}

\usepackage{graphicx}
\graphicspath{ {images/} }

\usepackage{faktor}

\usepackage{IEEEtrantools}
\usepackage{enumerate}   
\usepackage[PostScript=dvips]{"/Users/aware/Documents/Courses/diagrams"}


\newtheorem{theorem}{Théorème}[section]
\renewcommand{\thetheorem}{\arabic{theorem}}
\newtheorem{lemme}{Lemme}[section]
\renewcommand{\thelemme}{\arabic{lemme}}
\newtheorem{proposition}{Proposition}[section]
\renewcommand{\theproposition}{\arabic{proposition}}
\newtheorem{notations}{Notations}[section]
\newtheorem{problem}{Problème}[section]
\newtheorem{corollary}{Corollaire}[theorem]
\renewcommand{\thecorollary}{\arabic{corollary}}
\newtheorem{property}{Propriété}[section]
\newtheorem{objective}{Objectif}[section]

\theoremstyle{definition}
\newtheorem{definition}{Définition}[section]
\renewcommand{\thedefinition}{\arabic{definition}}
\newtheorem{exercise}{Exercice}[chapter]
\renewcommand{\theexercise}{\arabic{exercise}}
\newtheorem{example}{Exemple}[chapter]
\renewcommand{\theexample}{\arabic{example}}
\newtheorem*{solution}{Solution}
\newtheorem*{application}{Application}
\newtheorem*{notation}{Notation}
\newtheorem*{vocabulary}{Vocabulaire}
\newtheorem*{properties}{Propriétés}



\theoremstyle{remark}
\newtheorem*{remark}{Remarque}
\newtheorem*{rappel}{Rappel}


\usepackage{etoolbox}
\AtBeginEnvironment{exercise}{\small}
\AtBeginEnvironment{example}{\small}

\usepackage{cases}
\usepackage[red]{mypack}

\usepackage[framemethod=TikZ]{mdframed}

\definecolor{bg}{rgb}{0.4,0.25,0.95}
\definecolor{pagebg}{rgb}{0,0,0.5}
\surroundwithmdframed[
   topline=false,
   rightline=false,
   bottomline=false,
   leftmargin=\parindent,
   skipabove=8pt,
   skipbelow=8pt,
   linecolor=blue,
   innerbottommargin=10pt,
   % backgroundcolor=bg,font=\color{orange}\sffamily, fontcolor=white
]{definition}

\usepackage{empheq}
\usepackage[most]{tcolorbox}

\newtcbox{\mymath}[1][]{%
    nobeforeafter, math upper, tcbox raise base,
    enhanced, colframe=blue!30!black,
    colback=red!10, boxrule=1pt,
    #1}

\usepackage{unixode}


\DeclareMathOperator{\ord}{ord}
\DeclareMathOperator{\orb}{orb}
\DeclareMathOperator{\stab}{stab}
\DeclareMathOperator{\Stab}{stab}
\DeclareMathOperator{\ppcm}{ppcm}
\DeclareMathOperator{\conj}{Conj}
\DeclareMathOperator{\End}{End}
\DeclareMathOperator{\rot}{rot}
\DeclareMathOperator{\trs}{trace}
\DeclareMathOperator{\Ind}{Ind}
\DeclareMathOperator{\mat}{Mat}
\DeclareMathOperator{\id}{Id}
\DeclareMathOperator{\vect}{vect}
\DeclareMathOperator{\img}{img}
\DeclareMathOperator{\cov}{Cov}
\DeclareMathOperator{\dist}{dist}
\DeclareMathOperator{\irr}{Irr}
\DeclareMathOperator{\image}{Im}
\DeclareMathOperator{\pd}{\partial}
\DeclareMathOperator{\epi}{epi}
\DeclareMathOperator{\Argmin}{Argmin}
\DeclareMathOperator{\dom}{dom}
\DeclareMathOperator{\proj}{proj}
\DeclareMathOperator{\ctg}{ctg}
\DeclareMathOperator{\supp}{supp}
\DeclareMathOperator{\argmin}{argmin}
\DeclareMathOperator{\mult}{mult}
\DeclareMathOperator{\ch}{ch}
\DeclareMathOperator{\sh}{sh}
\DeclareMathOperator{\rang}{rang}
\DeclareMathOperator{\diam}{diam}
\DeclareMathOperator{\Epigraphe}{Epigraphe}




\usepackage{xcolor}
\everymath{\color{blue}}
%\everymath{\color[rgb]{0,1,1}}
%\pagecolor[rgb]{0,0,0.5}


\newcommand*{\pdtest}[3][]{\ensuremath{\frac{\partial^{#1} #2}{\partial #3}}}

\newcommand*{\deffunc}[6][]{\ensuremath{
\begin{array}{rcl}
#2 : #3 &\rightarrow& #4\\
#5 &\mapsto& #6
\end{array}
}}

\newcommand{\eqcolon}{\mathrel{\resizebox{\widthof{$\mathord{=}$}}{\height}{ $\!\!=\!\!\resizebox{1.2\width}{0.8\height}{\raisebox{0.23ex}{$\mathop{:}$}}\!\!$ }}}
\newcommand{\coloneq}{\mathrel{\resizebox{\widthof{$\mathord{=}$}}{\height}{ $\!\!\resizebox{1.2\width}{0.8\height}{\raisebox{0.23ex}{$\mathop{:}$}}\!\!=\!\!$ }}}
\newcommand{\eqcolonl}{\ensuremath{\mathrel{=\!\!\mathop{:}}}}
\newcommand{\coloneql}{\ensuremath{\mathrel{\mathop{:} \!\! =}}}
\newcommand{\vc}[1]{% inline column vector
  \left(\begin{smallmatrix}#1\end{smallmatrix}\right)%
}
\newcommand{\vr}[1]{% inline row vector
  \begin{smallmatrix}(\,#1\,)\end{smallmatrix}%
}
\makeatletter
\newcommand*{\defeq}{\ =\mathrel{\rlap{%
                     \raisebox{0.3ex}{$\m@th\cdot$}}%
                     \raisebox{-0.3ex}{$\m@th\cdot$}}%
                     }
\makeatother

\newcommand{\mathcircle}[1]{% inline row vector
 \overset{\circ}{#1}
}
\newcommand{\ulim}{% low limit
 \underline{\lim}
}
\newcommand{\ssi}{% iff
\iff
}
\newcommand{\ps}[2]{
\expval{#1 | #2}
}
\newcommand{\df}[1]{
\mqty{#1}
}
\newcommand{\n}[1]{
\norm{#1}
}
\newcommand{\sys}[1]{
\left\{\smqty{#1}\right.
}


\newcommand{\eqdef}{\ensuremath{\overset{\text{def}}=}}


\def\Circlearrowright{\ensuremath{%
  \rotatebox[origin=c]{230}{$\circlearrowright$}}}

\newcommand\ct[1]{\text{\rmfamily\upshape #1}}
\newcommand\question[1]{ {\color{red} ...!? \small #1}}
\newcommand\caz[1]{\left\{\begin{array} #1 \end{array}\right.}
\newcommand\const{\text{\rmfamily\upshape const}}
\newcommand\toP{ \overset{\pro}{\to}}
\newcommand\toPP{ \overset{\text{PP}}{\to}}
\newcommand{\oeq}{\mathrel{\text{\textcircled{$=$}}}}





\usepackage{xcolor}
% \usepackage[normalem]{ulem}
\usepackage{lipsum}
\makeatletter
% \newcommand\colorwave[1][blue]{\bgroup \markoverwith{\lower3.5\p@\hbox{\sixly \textcolor{#1}{\char58}}}\ULon}
%\font\sixly=lasy6 % does not re-load if already loaded, so no memory problem.

\newmdtheoremenv[
linewidth= 1pt,linecolor= blue,%
leftmargin=20,rightmargin=20,innertopmargin=0pt, innerrightmargin=40,%
tikzsetting = { draw=lightgray, line width = 0.3pt,dashed,%
dash pattern = on 15pt off 3pt},%
splittopskip=\topskip,skipbelow=\baselineskip,%
skipabove=\baselineskip,ntheorem,roundcorner=0pt,
% backgroundcolor=pagebg,font=\color{orange}\sffamily, fontcolor=white
]{examplebox}{Exemple}[section]



\newcommand\R{\mathbb{R}}
\newcommand\Z{\mathbb{Z}}
\newcommand\N{\mathbb{N}}
\newcommand\E{\mathbb{E}}
\newcommand\F{\mathcal{F}}
\newcommand\cH{\mathcal{H}}
\newcommand\V{\mathbb{V}}
\newcommand\dmo{ ^{-1} }
\newcommand\kapa{\kappa}
\newcommand\im{Im}
\newcommand\hs{\mathcal{H}}





\usepackage{soul}

\makeatletter
\newcommand*{\whiten}[1]{\llap{\textcolor{white}{{\the\SOUL@token}}\hspace{#1pt}}}
\DeclareRobustCommand*\myul{%
    \def\SOUL@everyspace{\underline{\space}\kern\z@}%
    \def\SOUL@everytoken{%
     \setbox0=\hbox{\the\SOUL@token}%
     \ifdim\dp0>\z@
        \raisebox{\dp0}{\underline{\phantom{\the\SOUL@token}}}%
        \whiten{1}\whiten{0}%
        \whiten{-1}\whiten{-2}%
        \llap{\the\SOUL@token}%
     \else
        \underline{\the\SOUL@token}%
     \fi}%
\SOUL@}
\makeatother

\newcommand*{\demp}{\fontfamily{lmtt}\selectfont}

\DeclareTextFontCommand{\textdemp}{\demp}

\begin{document}

\ifcomment
Multiline
comment
\fi
\ifcomment
\myul{Typesetting test}
% \color[rgb]{1,1,1}
$∑_i^n≠ 60º±∞π∆¬≈√j∫h≤≥µ$

$\CR \R\pro\ind\pro\gS\pro
\mqty[a&b\\c&d]$
$\pro\mathbb{P}$
$\dd{x}$

  \[
    \alpha(x)=\left\{
                \begin{array}{ll}
                  x\\
                  \frac{1}{1+e^{-kx}}\\
                  \frac{e^x-e^{-x}}{e^x+e^{-x}}
                \end{array}
              \right.
  \]

  $\expval{x}$
  
  $\chi_\rho(ghg\dmo)=\Tr(\rho_{ghg\dmo})=\Tr(\rho_g\circ\rho_h\circ\rho\dmo_g)=\Tr(\rho_h)\overset{\mbox{\scalebox{0.5}{$\Tr(AB)=\Tr(BA)$}}}{=}\chi_\rho(h)$
  	$\mathop{\oplus}_{\substack{x\in X}}$

$\mat(\rho_g)=(a_{ij}(g))_{\scriptsize \substack{1\leq i\leq d \\ 1\leq j\leq d}}$ et $\mat(\rho'_g)=(a'_{ij}(g))_{\scriptsize \substack{1\leq i'\leq d' \\ 1\leq j'\leq d'}}$



\[\int_a^b{\mathbb{R}^2}g(u, v)\dd{P_{XY}}(u, v)=\iint g(u,v) f_{XY}(u, v)\dd \lambda(u) \dd \lambda(v)\]
$$\lim_{x\to\infty} f(x)$$	
$$\iiiint_V \mu(t,u,v,w) \,dt\,du\,dv\,dw$$
$$\sum_{n=1}^{\infty} 2^{-n} = 1$$	
\begin{definition}
	Si $X$ et $Y$ sont 2 v.a. ou definit la \textsc{Covariance} entre $X$ et $Y$ comme
	$\cov(X,Y)\overset{\text{def}}{=}\E\left[(X-\E(X))(Y-\E(Y))\right]=\E(XY)-\E(X)\E(Y)$.
\end{definition}
\fi
\pagebreak

% \tableofcontents

% insert your code here
%\input{./algebra/main.tex}
%\input{./geometrie-differentielle/main.tex}
%\input{./probabilite/main.tex}
%\input{./analyse-fonctionnelle/main.tex}
% \input{./Analyse-convexe-et-dualite-en-optimisation/main.tex}
%\input{./tikz/main.tex}
%\input{./Theorie-du-distributions/main.tex}
%\input{./optimisation/mine.tex}
 \input{./modelisation/main.tex}

% yves.aubry@univ-tln.fr : algebra

\end{document}

% % !TEX encoding = UTF-8 Unicode
% !TEX TS-program = xelatex

\documentclass[french]{report}

%\usepackage[utf8]{inputenc}
%\usepackage[T1]{fontenc}
\usepackage{babel}


\newif\ifcomment
%\commenttrue # Show comments

\usepackage{physics}
\usepackage{amssymb}


\usepackage{amsthm}
% \usepackage{thmtools}
\usepackage{mathtools}
\usepackage{amsfonts}

\usepackage{color}

\usepackage{tikz}

\usepackage{geometry}
\geometry{a5paper, margin=0.1in, right=1cm}

\usepackage{dsfont}

\usepackage{graphicx}
\graphicspath{ {images/} }

\usepackage{faktor}

\usepackage{IEEEtrantools}
\usepackage{enumerate}   
\usepackage[PostScript=dvips]{"/Users/aware/Documents/Courses/diagrams"}


\newtheorem{theorem}{Théorème}[section]
\renewcommand{\thetheorem}{\arabic{theorem}}
\newtheorem{lemme}{Lemme}[section]
\renewcommand{\thelemme}{\arabic{lemme}}
\newtheorem{proposition}{Proposition}[section]
\renewcommand{\theproposition}{\arabic{proposition}}
\newtheorem{notations}{Notations}[section]
\newtheorem{problem}{Problème}[section]
\newtheorem{corollary}{Corollaire}[theorem]
\renewcommand{\thecorollary}{\arabic{corollary}}
\newtheorem{property}{Propriété}[section]
\newtheorem{objective}{Objectif}[section]

\theoremstyle{definition}
\newtheorem{definition}{Définition}[section]
\renewcommand{\thedefinition}{\arabic{definition}}
\newtheorem{exercise}{Exercice}[chapter]
\renewcommand{\theexercise}{\arabic{exercise}}
\newtheorem{example}{Exemple}[chapter]
\renewcommand{\theexample}{\arabic{example}}
\newtheorem*{solution}{Solution}
\newtheorem*{application}{Application}
\newtheorem*{notation}{Notation}
\newtheorem*{vocabulary}{Vocabulaire}
\newtheorem*{properties}{Propriétés}



\theoremstyle{remark}
\newtheorem*{remark}{Remarque}
\newtheorem*{rappel}{Rappel}


\usepackage{etoolbox}
\AtBeginEnvironment{exercise}{\small}
\AtBeginEnvironment{example}{\small}

\usepackage{cases}
\usepackage[red]{mypack}

\usepackage[framemethod=TikZ]{mdframed}

\definecolor{bg}{rgb}{0.4,0.25,0.95}
\definecolor{pagebg}{rgb}{0,0,0.5}
\surroundwithmdframed[
   topline=false,
   rightline=false,
   bottomline=false,
   leftmargin=\parindent,
   skipabove=8pt,
   skipbelow=8pt,
   linecolor=blue,
   innerbottommargin=10pt,
   % backgroundcolor=bg,font=\color{orange}\sffamily, fontcolor=white
]{definition}

\usepackage{empheq}
\usepackage[most]{tcolorbox}

\newtcbox{\mymath}[1][]{%
    nobeforeafter, math upper, tcbox raise base,
    enhanced, colframe=blue!30!black,
    colback=red!10, boxrule=1pt,
    #1}

\usepackage{unixode}


\DeclareMathOperator{\ord}{ord}
\DeclareMathOperator{\orb}{orb}
\DeclareMathOperator{\stab}{stab}
\DeclareMathOperator{\Stab}{stab}
\DeclareMathOperator{\ppcm}{ppcm}
\DeclareMathOperator{\conj}{Conj}
\DeclareMathOperator{\End}{End}
\DeclareMathOperator{\rot}{rot}
\DeclareMathOperator{\trs}{trace}
\DeclareMathOperator{\Ind}{Ind}
\DeclareMathOperator{\mat}{Mat}
\DeclareMathOperator{\id}{Id}
\DeclareMathOperator{\vect}{vect}
\DeclareMathOperator{\img}{img}
\DeclareMathOperator{\cov}{Cov}
\DeclareMathOperator{\dist}{dist}
\DeclareMathOperator{\irr}{Irr}
\DeclareMathOperator{\image}{Im}
\DeclareMathOperator{\pd}{\partial}
\DeclareMathOperator{\epi}{epi}
\DeclareMathOperator{\Argmin}{Argmin}
\DeclareMathOperator{\dom}{dom}
\DeclareMathOperator{\proj}{proj}
\DeclareMathOperator{\ctg}{ctg}
\DeclareMathOperator{\supp}{supp}
\DeclareMathOperator{\argmin}{argmin}
\DeclareMathOperator{\mult}{mult}
\DeclareMathOperator{\ch}{ch}
\DeclareMathOperator{\sh}{sh}
\DeclareMathOperator{\rang}{rang}
\DeclareMathOperator{\diam}{diam}
\DeclareMathOperator{\Epigraphe}{Epigraphe}




\usepackage{xcolor}
\everymath{\color{blue}}
%\everymath{\color[rgb]{0,1,1}}
%\pagecolor[rgb]{0,0,0.5}


\newcommand*{\pdtest}[3][]{\ensuremath{\frac{\partial^{#1} #2}{\partial #3}}}

\newcommand*{\deffunc}[6][]{\ensuremath{
\begin{array}{rcl}
#2 : #3 &\rightarrow& #4\\
#5 &\mapsto& #6
\end{array}
}}

\newcommand{\eqcolon}{\mathrel{\resizebox{\widthof{$\mathord{=}$}}{\height}{ $\!\!=\!\!\resizebox{1.2\width}{0.8\height}{\raisebox{0.23ex}{$\mathop{:}$}}\!\!$ }}}
\newcommand{\coloneq}{\mathrel{\resizebox{\widthof{$\mathord{=}$}}{\height}{ $\!\!\resizebox{1.2\width}{0.8\height}{\raisebox{0.23ex}{$\mathop{:}$}}\!\!=\!\!$ }}}
\newcommand{\eqcolonl}{\ensuremath{\mathrel{=\!\!\mathop{:}}}}
\newcommand{\coloneql}{\ensuremath{\mathrel{\mathop{:} \!\! =}}}
\newcommand{\vc}[1]{% inline column vector
  \left(\begin{smallmatrix}#1\end{smallmatrix}\right)%
}
\newcommand{\vr}[1]{% inline row vector
  \begin{smallmatrix}(\,#1\,)\end{smallmatrix}%
}
\makeatletter
\newcommand*{\defeq}{\ =\mathrel{\rlap{%
                     \raisebox{0.3ex}{$\m@th\cdot$}}%
                     \raisebox{-0.3ex}{$\m@th\cdot$}}%
                     }
\makeatother

\newcommand{\mathcircle}[1]{% inline row vector
 \overset{\circ}{#1}
}
\newcommand{\ulim}{% low limit
 \underline{\lim}
}
\newcommand{\ssi}{% iff
\iff
}
\newcommand{\ps}[2]{
\expval{#1 | #2}
}
\newcommand{\df}[1]{
\mqty{#1}
}
\newcommand{\n}[1]{
\norm{#1}
}
\newcommand{\sys}[1]{
\left\{\smqty{#1}\right.
}


\newcommand{\eqdef}{\ensuremath{\overset{\text{def}}=}}


\def\Circlearrowright{\ensuremath{%
  \rotatebox[origin=c]{230}{$\circlearrowright$}}}

\newcommand\ct[1]{\text{\rmfamily\upshape #1}}
\newcommand\question[1]{ {\color{red} ...!? \small #1}}
\newcommand\caz[1]{\left\{\begin{array} #1 \end{array}\right.}
\newcommand\const{\text{\rmfamily\upshape const}}
\newcommand\toP{ \overset{\pro}{\to}}
\newcommand\toPP{ \overset{\text{PP}}{\to}}
\newcommand{\oeq}{\mathrel{\text{\textcircled{$=$}}}}





\usepackage{xcolor}
% \usepackage[normalem]{ulem}
\usepackage{lipsum}
\makeatletter
% \newcommand\colorwave[1][blue]{\bgroup \markoverwith{\lower3.5\p@\hbox{\sixly \textcolor{#1}{\char58}}}\ULon}
%\font\sixly=lasy6 % does not re-load if already loaded, so no memory problem.

\newmdtheoremenv[
linewidth= 1pt,linecolor= blue,%
leftmargin=20,rightmargin=20,innertopmargin=0pt, innerrightmargin=40,%
tikzsetting = { draw=lightgray, line width = 0.3pt,dashed,%
dash pattern = on 15pt off 3pt},%
splittopskip=\topskip,skipbelow=\baselineskip,%
skipabove=\baselineskip,ntheorem,roundcorner=0pt,
% backgroundcolor=pagebg,font=\color{orange}\sffamily, fontcolor=white
]{examplebox}{Exemple}[section]



\newcommand\R{\mathbb{R}}
\newcommand\Z{\mathbb{Z}}
\newcommand\N{\mathbb{N}}
\newcommand\E{\mathbb{E}}
\newcommand\F{\mathcal{F}}
\newcommand\cH{\mathcal{H}}
\newcommand\V{\mathbb{V}}
\newcommand\dmo{ ^{-1} }
\newcommand\kapa{\kappa}
\newcommand\im{Im}
\newcommand\hs{\mathcal{H}}





\usepackage{soul}

\makeatletter
\newcommand*{\whiten}[1]{\llap{\textcolor{white}{{\the\SOUL@token}}\hspace{#1pt}}}
\DeclareRobustCommand*\myul{%
    \def\SOUL@everyspace{\underline{\space}\kern\z@}%
    \def\SOUL@everytoken{%
     \setbox0=\hbox{\the\SOUL@token}%
     \ifdim\dp0>\z@
        \raisebox{\dp0}{\underline{\phantom{\the\SOUL@token}}}%
        \whiten{1}\whiten{0}%
        \whiten{-1}\whiten{-2}%
        \llap{\the\SOUL@token}%
     \else
        \underline{\the\SOUL@token}%
     \fi}%
\SOUL@}
\makeatother

\newcommand*{\demp}{\fontfamily{lmtt}\selectfont}

\DeclareTextFontCommand{\textdemp}{\demp}

\begin{document}

\ifcomment
Multiline
comment
\fi
\ifcomment
\myul{Typesetting test}
% \color[rgb]{1,1,1}
$∑_i^n≠ 60º±∞π∆¬≈√j∫h≤≥µ$

$\CR \R\pro\ind\pro\gS\pro
\mqty[a&b\\c&d]$
$\pro\mathbb{P}$
$\dd{x}$

  \[
    \alpha(x)=\left\{
                \begin{array}{ll}
                  x\\
                  \frac{1}{1+e^{-kx}}\\
                  \frac{e^x-e^{-x}}{e^x+e^{-x}}
                \end{array}
              \right.
  \]

  $\expval{x}$
  
  $\chi_\rho(ghg\dmo)=\Tr(\rho_{ghg\dmo})=\Tr(\rho_g\circ\rho_h\circ\rho\dmo_g)=\Tr(\rho_h)\overset{\mbox{\scalebox{0.5}{$\Tr(AB)=\Tr(BA)$}}}{=}\chi_\rho(h)$
  	$\mathop{\oplus}_{\substack{x\in X}}$

$\mat(\rho_g)=(a_{ij}(g))_{\scriptsize \substack{1\leq i\leq d \\ 1\leq j\leq d}}$ et $\mat(\rho'_g)=(a'_{ij}(g))_{\scriptsize \substack{1\leq i'\leq d' \\ 1\leq j'\leq d'}}$



\[\int_a^b{\mathbb{R}^2}g(u, v)\dd{P_{XY}}(u, v)=\iint g(u,v) f_{XY}(u, v)\dd \lambda(u) \dd \lambda(v)\]
$$\lim_{x\to\infty} f(x)$$	
$$\iiiint_V \mu(t,u,v,w) \,dt\,du\,dv\,dw$$
$$\sum_{n=1}^{\infty} 2^{-n} = 1$$	
\begin{definition}
	Si $X$ et $Y$ sont 2 v.a. ou definit la \textsc{Covariance} entre $X$ et $Y$ comme
	$\cov(X,Y)\overset{\text{def}}{=}\E\left[(X-\E(X))(Y-\E(Y))\right]=\E(XY)-\E(X)\E(Y)$.
\end{definition}
\fi
\pagebreak

% \tableofcontents

% insert your code here
%\input{./algebra/main.tex}
%\input{./geometrie-differentielle/main.tex}
%\input{./probabilite/main.tex}
%\input{./analyse-fonctionnelle/main.tex}
% \input{./Analyse-convexe-et-dualite-en-optimisation/main.tex}
%\input{./tikz/main.tex}
%\input{./Theorie-du-distributions/main.tex}
%\input{./optimisation/mine.tex}
 \input{./modelisation/main.tex}

% yves.aubry@univ-tln.fr : algebra

\end{document}

%% !TEX encoding = UTF-8 Unicode
% !TEX TS-program = xelatex

\documentclass[french]{report}

%\usepackage[utf8]{inputenc}
%\usepackage[T1]{fontenc}
\usepackage{babel}


\newif\ifcomment
%\commenttrue # Show comments

\usepackage{physics}
\usepackage{amssymb}


\usepackage{amsthm}
% \usepackage{thmtools}
\usepackage{mathtools}
\usepackage{amsfonts}

\usepackage{color}

\usepackage{tikz}

\usepackage{geometry}
\geometry{a5paper, margin=0.1in, right=1cm}

\usepackage{dsfont}

\usepackage{graphicx}
\graphicspath{ {images/} }

\usepackage{faktor}

\usepackage{IEEEtrantools}
\usepackage{enumerate}   
\usepackage[PostScript=dvips]{"/Users/aware/Documents/Courses/diagrams"}


\newtheorem{theorem}{Théorème}[section]
\renewcommand{\thetheorem}{\arabic{theorem}}
\newtheorem{lemme}{Lemme}[section]
\renewcommand{\thelemme}{\arabic{lemme}}
\newtheorem{proposition}{Proposition}[section]
\renewcommand{\theproposition}{\arabic{proposition}}
\newtheorem{notations}{Notations}[section]
\newtheorem{problem}{Problème}[section]
\newtheorem{corollary}{Corollaire}[theorem]
\renewcommand{\thecorollary}{\arabic{corollary}}
\newtheorem{property}{Propriété}[section]
\newtheorem{objective}{Objectif}[section]

\theoremstyle{definition}
\newtheorem{definition}{Définition}[section]
\renewcommand{\thedefinition}{\arabic{definition}}
\newtheorem{exercise}{Exercice}[chapter]
\renewcommand{\theexercise}{\arabic{exercise}}
\newtheorem{example}{Exemple}[chapter]
\renewcommand{\theexample}{\arabic{example}}
\newtheorem*{solution}{Solution}
\newtheorem*{application}{Application}
\newtheorem*{notation}{Notation}
\newtheorem*{vocabulary}{Vocabulaire}
\newtheorem*{properties}{Propriétés}



\theoremstyle{remark}
\newtheorem*{remark}{Remarque}
\newtheorem*{rappel}{Rappel}


\usepackage{etoolbox}
\AtBeginEnvironment{exercise}{\small}
\AtBeginEnvironment{example}{\small}

\usepackage{cases}
\usepackage[red]{mypack}

\usepackage[framemethod=TikZ]{mdframed}

\definecolor{bg}{rgb}{0.4,0.25,0.95}
\definecolor{pagebg}{rgb}{0,0,0.5}
\surroundwithmdframed[
   topline=false,
   rightline=false,
   bottomline=false,
   leftmargin=\parindent,
   skipabove=8pt,
   skipbelow=8pt,
   linecolor=blue,
   innerbottommargin=10pt,
   % backgroundcolor=bg,font=\color{orange}\sffamily, fontcolor=white
]{definition}

\usepackage{empheq}
\usepackage[most]{tcolorbox}

\newtcbox{\mymath}[1][]{%
    nobeforeafter, math upper, tcbox raise base,
    enhanced, colframe=blue!30!black,
    colback=red!10, boxrule=1pt,
    #1}

\usepackage{unixode}


\DeclareMathOperator{\ord}{ord}
\DeclareMathOperator{\orb}{orb}
\DeclareMathOperator{\stab}{stab}
\DeclareMathOperator{\Stab}{stab}
\DeclareMathOperator{\ppcm}{ppcm}
\DeclareMathOperator{\conj}{Conj}
\DeclareMathOperator{\End}{End}
\DeclareMathOperator{\rot}{rot}
\DeclareMathOperator{\trs}{trace}
\DeclareMathOperator{\Ind}{Ind}
\DeclareMathOperator{\mat}{Mat}
\DeclareMathOperator{\id}{Id}
\DeclareMathOperator{\vect}{vect}
\DeclareMathOperator{\img}{img}
\DeclareMathOperator{\cov}{Cov}
\DeclareMathOperator{\dist}{dist}
\DeclareMathOperator{\irr}{Irr}
\DeclareMathOperator{\image}{Im}
\DeclareMathOperator{\pd}{\partial}
\DeclareMathOperator{\epi}{epi}
\DeclareMathOperator{\Argmin}{Argmin}
\DeclareMathOperator{\dom}{dom}
\DeclareMathOperator{\proj}{proj}
\DeclareMathOperator{\ctg}{ctg}
\DeclareMathOperator{\supp}{supp}
\DeclareMathOperator{\argmin}{argmin}
\DeclareMathOperator{\mult}{mult}
\DeclareMathOperator{\ch}{ch}
\DeclareMathOperator{\sh}{sh}
\DeclareMathOperator{\rang}{rang}
\DeclareMathOperator{\diam}{diam}
\DeclareMathOperator{\Epigraphe}{Epigraphe}




\usepackage{xcolor}
\everymath{\color{blue}}
%\everymath{\color[rgb]{0,1,1}}
%\pagecolor[rgb]{0,0,0.5}


\newcommand*{\pdtest}[3][]{\ensuremath{\frac{\partial^{#1} #2}{\partial #3}}}

\newcommand*{\deffunc}[6][]{\ensuremath{
\begin{array}{rcl}
#2 : #3 &\rightarrow& #4\\
#5 &\mapsto& #6
\end{array}
}}

\newcommand{\eqcolon}{\mathrel{\resizebox{\widthof{$\mathord{=}$}}{\height}{ $\!\!=\!\!\resizebox{1.2\width}{0.8\height}{\raisebox{0.23ex}{$\mathop{:}$}}\!\!$ }}}
\newcommand{\coloneq}{\mathrel{\resizebox{\widthof{$\mathord{=}$}}{\height}{ $\!\!\resizebox{1.2\width}{0.8\height}{\raisebox{0.23ex}{$\mathop{:}$}}\!\!=\!\!$ }}}
\newcommand{\eqcolonl}{\ensuremath{\mathrel{=\!\!\mathop{:}}}}
\newcommand{\coloneql}{\ensuremath{\mathrel{\mathop{:} \!\! =}}}
\newcommand{\vc}[1]{% inline column vector
  \left(\begin{smallmatrix}#1\end{smallmatrix}\right)%
}
\newcommand{\vr}[1]{% inline row vector
  \begin{smallmatrix}(\,#1\,)\end{smallmatrix}%
}
\makeatletter
\newcommand*{\defeq}{\ =\mathrel{\rlap{%
                     \raisebox{0.3ex}{$\m@th\cdot$}}%
                     \raisebox{-0.3ex}{$\m@th\cdot$}}%
                     }
\makeatother

\newcommand{\mathcircle}[1]{% inline row vector
 \overset{\circ}{#1}
}
\newcommand{\ulim}{% low limit
 \underline{\lim}
}
\newcommand{\ssi}{% iff
\iff
}
\newcommand{\ps}[2]{
\expval{#1 | #2}
}
\newcommand{\df}[1]{
\mqty{#1}
}
\newcommand{\n}[1]{
\norm{#1}
}
\newcommand{\sys}[1]{
\left\{\smqty{#1}\right.
}


\newcommand{\eqdef}{\ensuremath{\overset{\text{def}}=}}


\def\Circlearrowright{\ensuremath{%
  \rotatebox[origin=c]{230}{$\circlearrowright$}}}

\newcommand\ct[1]{\text{\rmfamily\upshape #1}}
\newcommand\question[1]{ {\color{red} ...!? \small #1}}
\newcommand\caz[1]{\left\{\begin{array} #1 \end{array}\right.}
\newcommand\const{\text{\rmfamily\upshape const}}
\newcommand\toP{ \overset{\pro}{\to}}
\newcommand\toPP{ \overset{\text{PP}}{\to}}
\newcommand{\oeq}{\mathrel{\text{\textcircled{$=$}}}}





\usepackage{xcolor}
% \usepackage[normalem]{ulem}
\usepackage{lipsum}
\makeatletter
% \newcommand\colorwave[1][blue]{\bgroup \markoverwith{\lower3.5\p@\hbox{\sixly \textcolor{#1}{\char58}}}\ULon}
%\font\sixly=lasy6 % does not re-load if already loaded, so no memory problem.

\newmdtheoremenv[
linewidth= 1pt,linecolor= blue,%
leftmargin=20,rightmargin=20,innertopmargin=0pt, innerrightmargin=40,%
tikzsetting = { draw=lightgray, line width = 0.3pt,dashed,%
dash pattern = on 15pt off 3pt},%
splittopskip=\topskip,skipbelow=\baselineskip,%
skipabove=\baselineskip,ntheorem,roundcorner=0pt,
% backgroundcolor=pagebg,font=\color{orange}\sffamily, fontcolor=white
]{examplebox}{Exemple}[section]



\newcommand\R{\mathbb{R}}
\newcommand\Z{\mathbb{Z}}
\newcommand\N{\mathbb{N}}
\newcommand\E{\mathbb{E}}
\newcommand\F{\mathcal{F}}
\newcommand\cH{\mathcal{H}}
\newcommand\V{\mathbb{V}}
\newcommand\dmo{ ^{-1} }
\newcommand\kapa{\kappa}
\newcommand\im{Im}
\newcommand\hs{\mathcal{H}}





\usepackage{soul}

\makeatletter
\newcommand*{\whiten}[1]{\llap{\textcolor{white}{{\the\SOUL@token}}\hspace{#1pt}}}
\DeclareRobustCommand*\myul{%
    \def\SOUL@everyspace{\underline{\space}\kern\z@}%
    \def\SOUL@everytoken{%
     \setbox0=\hbox{\the\SOUL@token}%
     \ifdim\dp0>\z@
        \raisebox{\dp0}{\underline{\phantom{\the\SOUL@token}}}%
        \whiten{1}\whiten{0}%
        \whiten{-1}\whiten{-2}%
        \llap{\the\SOUL@token}%
     \else
        \underline{\the\SOUL@token}%
     \fi}%
\SOUL@}
\makeatother

\newcommand*{\demp}{\fontfamily{lmtt}\selectfont}

\DeclareTextFontCommand{\textdemp}{\demp}

\begin{document}

\ifcomment
Multiline
comment
\fi
\ifcomment
\myul{Typesetting test}
% \color[rgb]{1,1,1}
$∑_i^n≠ 60º±∞π∆¬≈√j∫h≤≥µ$

$\CR \R\pro\ind\pro\gS\pro
\mqty[a&b\\c&d]$
$\pro\mathbb{P}$
$\dd{x}$

  \[
    \alpha(x)=\left\{
                \begin{array}{ll}
                  x\\
                  \frac{1}{1+e^{-kx}}\\
                  \frac{e^x-e^{-x}}{e^x+e^{-x}}
                \end{array}
              \right.
  \]

  $\expval{x}$
  
  $\chi_\rho(ghg\dmo)=\Tr(\rho_{ghg\dmo})=\Tr(\rho_g\circ\rho_h\circ\rho\dmo_g)=\Tr(\rho_h)\overset{\mbox{\scalebox{0.5}{$\Tr(AB)=\Tr(BA)$}}}{=}\chi_\rho(h)$
  	$\mathop{\oplus}_{\substack{x\in X}}$

$\mat(\rho_g)=(a_{ij}(g))_{\scriptsize \substack{1\leq i\leq d \\ 1\leq j\leq d}}$ et $\mat(\rho'_g)=(a'_{ij}(g))_{\scriptsize \substack{1\leq i'\leq d' \\ 1\leq j'\leq d'}}$



\[\int_a^b{\mathbb{R}^2}g(u, v)\dd{P_{XY}}(u, v)=\iint g(u,v) f_{XY}(u, v)\dd \lambda(u) \dd \lambda(v)\]
$$\lim_{x\to\infty} f(x)$$	
$$\iiiint_V \mu(t,u,v,w) \,dt\,du\,dv\,dw$$
$$\sum_{n=1}^{\infty} 2^{-n} = 1$$	
\begin{definition}
	Si $X$ et $Y$ sont 2 v.a. ou definit la \textsc{Covariance} entre $X$ et $Y$ comme
	$\cov(X,Y)\overset{\text{def}}{=}\E\left[(X-\E(X))(Y-\E(Y))\right]=\E(XY)-\E(X)\E(Y)$.
\end{definition}
\fi
\pagebreak

% \tableofcontents

% insert your code here
%\input{./algebra/main.tex}
%\input{./geometrie-differentielle/main.tex}
%\input{./probabilite/main.tex}
%\input{./analyse-fonctionnelle/main.tex}
% \input{./Analyse-convexe-et-dualite-en-optimisation/main.tex}
%\input{./tikz/main.tex}
%\input{./Theorie-du-distributions/main.tex}
%\input{./optimisation/mine.tex}
 \input{./modelisation/main.tex}

% yves.aubry@univ-tln.fr : algebra

\end{document}

%% !TEX encoding = UTF-8 Unicode
% !TEX TS-program = xelatex

\documentclass[french]{report}

%\usepackage[utf8]{inputenc}
%\usepackage[T1]{fontenc}
\usepackage{babel}


\newif\ifcomment
%\commenttrue # Show comments

\usepackage{physics}
\usepackage{amssymb}


\usepackage{amsthm}
% \usepackage{thmtools}
\usepackage{mathtools}
\usepackage{amsfonts}

\usepackage{color}

\usepackage{tikz}

\usepackage{geometry}
\geometry{a5paper, margin=0.1in, right=1cm}

\usepackage{dsfont}

\usepackage{graphicx}
\graphicspath{ {images/} }

\usepackage{faktor}

\usepackage{IEEEtrantools}
\usepackage{enumerate}   
\usepackage[PostScript=dvips]{"/Users/aware/Documents/Courses/diagrams"}


\newtheorem{theorem}{Théorème}[section]
\renewcommand{\thetheorem}{\arabic{theorem}}
\newtheorem{lemme}{Lemme}[section]
\renewcommand{\thelemme}{\arabic{lemme}}
\newtheorem{proposition}{Proposition}[section]
\renewcommand{\theproposition}{\arabic{proposition}}
\newtheorem{notations}{Notations}[section]
\newtheorem{problem}{Problème}[section]
\newtheorem{corollary}{Corollaire}[theorem]
\renewcommand{\thecorollary}{\arabic{corollary}}
\newtheorem{property}{Propriété}[section]
\newtheorem{objective}{Objectif}[section]

\theoremstyle{definition}
\newtheorem{definition}{Définition}[section]
\renewcommand{\thedefinition}{\arabic{definition}}
\newtheorem{exercise}{Exercice}[chapter]
\renewcommand{\theexercise}{\arabic{exercise}}
\newtheorem{example}{Exemple}[chapter]
\renewcommand{\theexample}{\arabic{example}}
\newtheorem*{solution}{Solution}
\newtheorem*{application}{Application}
\newtheorem*{notation}{Notation}
\newtheorem*{vocabulary}{Vocabulaire}
\newtheorem*{properties}{Propriétés}



\theoremstyle{remark}
\newtheorem*{remark}{Remarque}
\newtheorem*{rappel}{Rappel}


\usepackage{etoolbox}
\AtBeginEnvironment{exercise}{\small}
\AtBeginEnvironment{example}{\small}

\usepackage{cases}
\usepackage[red]{mypack}

\usepackage[framemethod=TikZ]{mdframed}

\definecolor{bg}{rgb}{0.4,0.25,0.95}
\definecolor{pagebg}{rgb}{0,0,0.5}
\surroundwithmdframed[
   topline=false,
   rightline=false,
   bottomline=false,
   leftmargin=\parindent,
   skipabove=8pt,
   skipbelow=8pt,
   linecolor=blue,
   innerbottommargin=10pt,
   % backgroundcolor=bg,font=\color{orange}\sffamily, fontcolor=white
]{definition}

\usepackage{empheq}
\usepackage[most]{tcolorbox}

\newtcbox{\mymath}[1][]{%
    nobeforeafter, math upper, tcbox raise base,
    enhanced, colframe=blue!30!black,
    colback=red!10, boxrule=1pt,
    #1}

\usepackage{unixode}


\DeclareMathOperator{\ord}{ord}
\DeclareMathOperator{\orb}{orb}
\DeclareMathOperator{\stab}{stab}
\DeclareMathOperator{\Stab}{stab}
\DeclareMathOperator{\ppcm}{ppcm}
\DeclareMathOperator{\conj}{Conj}
\DeclareMathOperator{\End}{End}
\DeclareMathOperator{\rot}{rot}
\DeclareMathOperator{\trs}{trace}
\DeclareMathOperator{\Ind}{Ind}
\DeclareMathOperator{\mat}{Mat}
\DeclareMathOperator{\id}{Id}
\DeclareMathOperator{\vect}{vect}
\DeclareMathOperator{\img}{img}
\DeclareMathOperator{\cov}{Cov}
\DeclareMathOperator{\dist}{dist}
\DeclareMathOperator{\irr}{Irr}
\DeclareMathOperator{\image}{Im}
\DeclareMathOperator{\pd}{\partial}
\DeclareMathOperator{\epi}{epi}
\DeclareMathOperator{\Argmin}{Argmin}
\DeclareMathOperator{\dom}{dom}
\DeclareMathOperator{\proj}{proj}
\DeclareMathOperator{\ctg}{ctg}
\DeclareMathOperator{\supp}{supp}
\DeclareMathOperator{\argmin}{argmin}
\DeclareMathOperator{\mult}{mult}
\DeclareMathOperator{\ch}{ch}
\DeclareMathOperator{\sh}{sh}
\DeclareMathOperator{\rang}{rang}
\DeclareMathOperator{\diam}{diam}
\DeclareMathOperator{\Epigraphe}{Epigraphe}




\usepackage{xcolor}
\everymath{\color{blue}}
%\everymath{\color[rgb]{0,1,1}}
%\pagecolor[rgb]{0,0,0.5}


\newcommand*{\pdtest}[3][]{\ensuremath{\frac{\partial^{#1} #2}{\partial #3}}}

\newcommand*{\deffunc}[6][]{\ensuremath{
\begin{array}{rcl}
#2 : #3 &\rightarrow& #4\\
#5 &\mapsto& #6
\end{array}
}}

\newcommand{\eqcolon}{\mathrel{\resizebox{\widthof{$\mathord{=}$}}{\height}{ $\!\!=\!\!\resizebox{1.2\width}{0.8\height}{\raisebox{0.23ex}{$\mathop{:}$}}\!\!$ }}}
\newcommand{\coloneq}{\mathrel{\resizebox{\widthof{$\mathord{=}$}}{\height}{ $\!\!\resizebox{1.2\width}{0.8\height}{\raisebox{0.23ex}{$\mathop{:}$}}\!\!=\!\!$ }}}
\newcommand{\eqcolonl}{\ensuremath{\mathrel{=\!\!\mathop{:}}}}
\newcommand{\coloneql}{\ensuremath{\mathrel{\mathop{:} \!\! =}}}
\newcommand{\vc}[1]{% inline column vector
  \left(\begin{smallmatrix}#1\end{smallmatrix}\right)%
}
\newcommand{\vr}[1]{% inline row vector
  \begin{smallmatrix}(\,#1\,)\end{smallmatrix}%
}
\makeatletter
\newcommand*{\defeq}{\ =\mathrel{\rlap{%
                     \raisebox{0.3ex}{$\m@th\cdot$}}%
                     \raisebox{-0.3ex}{$\m@th\cdot$}}%
                     }
\makeatother

\newcommand{\mathcircle}[1]{% inline row vector
 \overset{\circ}{#1}
}
\newcommand{\ulim}{% low limit
 \underline{\lim}
}
\newcommand{\ssi}{% iff
\iff
}
\newcommand{\ps}[2]{
\expval{#1 | #2}
}
\newcommand{\df}[1]{
\mqty{#1}
}
\newcommand{\n}[1]{
\norm{#1}
}
\newcommand{\sys}[1]{
\left\{\smqty{#1}\right.
}


\newcommand{\eqdef}{\ensuremath{\overset{\text{def}}=}}


\def\Circlearrowright{\ensuremath{%
  \rotatebox[origin=c]{230}{$\circlearrowright$}}}

\newcommand\ct[1]{\text{\rmfamily\upshape #1}}
\newcommand\question[1]{ {\color{red} ...!? \small #1}}
\newcommand\caz[1]{\left\{\begin{array} #1 \end{array}\right.}
\newcommand\const{\text{\rmfamily\upshape const}}
\newcommand\toP{ \overset{\pro}{\to}}
\newcommand\toPP{ \overset{\text{PP}}{\to}}
\newcommand{\oeq}{\mathrel{\text{\textcircled{$=$}}}}





\usepackage{xcolor}
% \usepackage[normalem]{ulem}
\usepackage{lipsum}
\makeatletter
% \newcommand\colorwave[1][blue]{\bgroup \markoverwith{\lower3.5\p@\hbox{\sixly \textcolor{#1}{\char58}}}\ULon}
%\font\sixly=lasy6 % does not re-load if already loaded, so no memory problem.

\newmdtheoremenv[
linewidth= 1pt,linecolor= blue,%
leftmargin=20,rightmargin=20,innertopmargin=0pt, innerrightmargin=40,%
tikzsetting = { draw=lightgray, line width = 0.3pt,dashed,%
dash pattern = on 15pt off 3pt},%
splittopskip=\topskip,skipbelow=\baselineskip,%
skipabove=\baselineskip,ntheorem,roundcorner=0pt,
% backgroundcolor=pagebg,font=\color{orange}\sffamily, fontcolor=white
]{examplebox}{Exemple}[section]



\newcommand\R{\mathbb{R}}
\newcommand\Z{\mathbb{Z}}
\newcommand\N{\mathbb{N}}
\newcommand\E{\mathbb{E}}
\newcommand\F{\mathcal{F}}
\newcommand\cH{\mathcal{H}}
\newcommand\V{\mathbb{V}}
\newcommand\dmo{ ^{-1} }
\newcommand\kapa{\kappa}
\newcommand\im{Im}
\newcommand\hs{\mathcal{H}}





\usepackage{soul}

\makeatletter
\newcommand*{\whiten}[1]{\llap{\textcolor{white}{{\the\SOUL@token}}\hspace{#1pt}}}
\DeclareRobustCommand*\myul{%
    \def\SOUL@everyspace{\underline{\space}\kern\z@}%
    \def\SOUL@everytoken{%
     \setbox0=\hbox{\the\SOUL@token}%
     \ifdim\dp0>\z@
        \raisebox{\dp0}{\underline{\phantom{\the\SOUL@token}}}%
        \whiten{1}\whiten{0}%
        \whiten{-1}\whiten{-2}%
        \llap{\the\SOUL@token}%
     \else
        \underline{\the\SOUL@token}%
     \fi}%
\SOUL@}
\makeatother

\newcommand*{\demp}{\fontfamily{lmtt}\selectfont}

\DeclareTextFontCommand{\textdemp}{\demp}

\begin{document}

\ifcomment
Multiline
comment
\fi
\ifcomment
\myul{Typesetting test}
% \color[rgb]{1,1,1}
$∑_i^n≠ 60º±∞π∆¬≈√j∫h≤≥µ$

$\CR \R\pro\ind\pro\gS\pro
\mqty[a&b\\c&d]$
$\pro\mathbb{P}$
$\dd{x}$

  \[
    \alpha(x)=\left\{
                \begin{array}{ll}
                  x\\
                  \frac{1}{1+e^{-kx}}\\
                  \frac{e^x-e^{-x}}{e^x+e^{-x}}
                \end{array}
              \right.
  \]

  $\expval{x}$
  
  $\chi_\rho(ghg\dmo)=\Tr(\rho_{ghg\dmo})=\Tr(\rho_g\circ\rho_h\circ\rho\dmo_g)=\Tr(\rho_h)\overset{\mbox{\scalebox{0.5}{$\Tr(AB)=\Tr(BA)$}}}{=}\chi_\rho(h)$
  	$\mathop{\oplus}_{\substack{x\in X}}$

$\mat(\rho_g)=(a_{ij}(g))_{\scriptsize \substack{1\leq i\leq d \\ 1\leq j\leq d}}$ et $\mat(\rho'_g)=(a'_{ij}(g))_{\scriptsize \substack{1\leq i'\leq d' \\ 1\leq j'\leq d'}}$



\[\int_a^b{\mathbb{R}^2}g(u, v)\dd{P_{XY}}(u, v)=\iint g(u,v) f_{XY}(u, v)\dd \lambda(u) \dd \lambda(v)\]
$$\lim_{x\to\infty} f(x)$$	
$$\iiiint_V \mu(t,u,v,w) \,dt\,du\,dv\,dw$$
$$\sum_{n=1}^{\infty} 2^{-n} = 1$$	
\begin{definition}
	Si $X$ et $Y$ sont 2 v.a. ou definit la \textsc{Covariance} entre $X$ et $Y$ comme
	$\cov(X,Y)\overset{\text{def}}{=}\E\left[(X-\E(X))(Y-\E(Y))\right]=\E(XY)-\E(X)\E(Y)$.
\end{definition}
\fi
\pagebreak

% \tableofcontents

% insert your code here
%\input{./algebra/main.tex}
%\input{./geometrie-differentielle/main.tex}
%\input{./probabilite/main.tex}
%\input{./analyse-fonctionnelle/main.tex}
% \input{./Analyse-convexe-et-dualite-en-optimisation/main.tex}
%\input{./tikz/main.tex}
%\input{./Theorie-du-distributions/main.tex}
%\input{./optimisation/mine.tex}
 \input{./modelisation/main.tex}

% yves.aubry@univ-tln.fr : algebra

\end{document}

%\input{./optimisation/mine.tex}
 % !TEX encoding = UTF-8 Unicode
% !TEX TS-program = xelatex

\documentclass[french]{report}

%\usepackage[utf8]{inputenc}
%\usepackage[T1]{fontenc}
\usepackage{babel}


\newif\ifcomment
%\commenttrue # Show comments

\usepackage{physics}
\usepackage{amssymb}


\usepackage{amsthm}
% \usepackage{thmtools}
\usepackage{mathtools}
\usepackage{amsfonts}

\usepackage{color}

\usepackage{tikz}

\usepackage{geometry}
\geometry{a5paper, margin=0.1in, right=1cm}

\usepackage{dsfont}

\usepackage{graphicx}
\graphicspath{ {images/} }

\usepackage{faktor}

\usepackage{IEEEtrantools}
\usepackage{enumerate}   
\usepackage[PostScript=dvips]{"/Users/aware/Documents/Courses/diagrams"}


\newtheorem{theorem}{Théorème}[section]
\renewcommand{\thetheorem}{\arabic{theorem}}
\newtheorem{lemme}{Lemme}[section]
\renewcommand{\thelemme}{\arabic{lemme}}
\newtheorem{proposition}{Proposition}[section]
\renewcommand{\theproposition}{\arabic{proposition}}
\newtheorem{notations}{Notations}[section]
\newtheorem{problem}{Problème}[section]
\newtheorem{corollary}{Corollaire}[theorem]
\renewcommand{\thecorollary}{\arabic{corollary}}
\newtheorem{property}{Propriété}[section]
\newtheorem{objective}{Objectif}[section]

\theoremstyle{definition}
\newtheorem{definition}{Définition}[section]
\renewcommand{\thedefinition}{\arabic{definition}}
\newtheorem{exercise}{Exercice}[chapter]
\renewcommand{\theexercise}{\arabic{exercise}}
\newtheorem{example}{Exemple}[chapter]
\renewcommand{\theexample}{\arabic{example}}
\newtheorem*{solution}{Solution}
\newtheorem*{application}{Application}
\newtheorem*{notation}{Notation}
\newtheorem*{vocabulary}{Vocabulaire}
\newtheorem*{properties}{Propriétés}



\theoremstyle{remark}
\newtheorem*{remark}{Remarque}
\newtheorem*{rappel}{Rappel}


\usepackage{etoolbox}
\AtBeginEnvironment{exercise}{\small}
\AtBeginEnvironment{example}{\small}

\usepackage{cases}
\usepackage[red]{mypack}

\usepackage[framemethod=TikZ]{mdframed}

\definecolor{bg}{rgb}{0.4,0.25,0.95}
\definecolor{pagebg}{rgb}{0,0,0.5}
\surroundwithmdframed[
   topline=false,
   rightline=false,
   bottomline=false,
   leftmargin=\parindent,
   skipabove=8pt,
   skipbelow=8pt,
   linecolor=blue,
   innerbottommargin=10pt,
   % backgroundcolor=bg,font=\color{orange}\sffamily, fontcolor=white
]{definition}

\usepackage{empheq}
\usepackage[most]{tcolorbox}

\newtcbox{\mymath}[1][]{%
    nobeforeafter, math upper, tcbox raise base,
    enhanced, colframe=blue!30!black,
    colback=red!10, boxrule=1pt,
    #1}

\usepackage{unixode}


\DeclareMathOperator{\ord}{ord}
\DeclareMathOperator{\orb}{orb}
\DeclareMathOperator{\stab}{stab}
\DeclareMathOperator{\Stab}{stab}
\DeclareMathOperator{\ppcm}{ppcm}
\DeclareMathOperator{\conj}{Conj}
\DeclareMathOperator{\End}{End}
\DeclareMathOperator{\rot}{rot}
\DeclareMathOperator{\trs}{trace}
\DeclareMathOperator{\Ind}{Ind}
\DeclareMathOperator{\mat}{Mat}
\DeclareMathOperator{\id}{Id}
\DeclareMathOperator{\vect}{vect}
\DeclareMathOperator{\img}{img}
\DeclareMathOperator{\cov}{Cov}
\DeclareMathOperator{\dist}{dist}
\DeclareMathOperator{\irr}{Irr}
\DeclareMathOperator{\image}{Im}
\DeclareMathOperator{\pd}{\partial}
\DeclareMathOperator{\epi}{epi}
\DeclareMathOperator{\Argmin}{Argmin}
\DeclareMathOperator{\dom}{dom}
\DeclareMathOperator{\proj}{proj}
\DeclareMathOperator{\ctg}{ctg}
\DeclareMathOperator{\supp}{supp}
\DeclareMathOperator{\argmin}{argmin}
\DeclareMathOperator{\mult}{mult}
\DeclareMathOperator{\ch}{ch}
\DeclareMathOperator{\sh}{sh}
\DeclareMathOperator{\rang}{rang}
\DeclareMathOperator{\diam}{diam}
\DeclareMathOperator{\Epigraphe}{Epigraphe}




\usepackage{xcolor}
\everymath{\color{blue}}
%\everymath{\color[rgb]{0,1,1}}
%\pagecolor[rgb]{0,0,0.5}


\newcommand*{\pdtest}[3][]{\ensuremath{\frac{\partial^{#1} #2}{\partial #3}}}

\newcommand*{\deffunc}[6][]{\ensuremath{
\begin{array}{rcl}
#2 : #3 &\rightarrow& #4\\
#5 &\mapsto& #6
\end{array}
}}

\newcommand{\eqcolon}{\mathrel{\resizebox{\widthof{$\mathord{=}$}}{\height}{ $\!\!=\!\!\resizebox{1.2\width}{0.8\height}{\raisebox{0.23ex}{$\mathop{:}$}}\!\!$ }}}
\newcommand{\coloneq}{\mathrel{\resizebox{\widthof{$\mathord{=}$}}{\height}{ $\!\!\resizebox{1.2\width}{0.8\height}{\raisebox{0.23ex}{$\mathop{:}$}}\!\!=\!\!$ }}}
\newcommand{\eqcolonl}{\ensuremath{\mathrel{=\!\!\mathop{:}}}}
\newcommand{\coloneql}{\ensuremath{\mathrel{\mathop{:} \!\! =}}}
\newcommand{\vc}[1]{% inline column vector
  \left(\begin{smallmatrix}#1\end{smallmatrix}\right)%
}
\newcommand{\vr}[1]{% inline row vector
  \begin{smallmatrix}(\,#1\,)\end{smallmatrix}%
}
\makeatletter
\newcommand*{\defeq}{\ =\mathrel{\rlap{%
                     \raisebox{0.3ex}{$\m@th\cdot$}}%
                     \raisebox{-0.3ex}{$\m@th\cdot$}}%
                     }
\makeatother

\newcommand{\mathcircle}[1]{% inline row vector
 \overset{\circ}{#1}
}
\newcommand{\ulim}{% low limit
 \underline{\lim}
}
\newcommand{\ssi}{% iff
\iff
}
\newcommand{\ps}[2]{
\expval{#1 | #2}
}
\newcommand{\df}[1]{
\mqty{#1}
}
\newcommand{\n}[1]{
\norm{#1}
}
\newcommand{\sys}[1]{
\left\{\smqty{#1}\right.
}


\newcommand{\eqdef}{\ensuremath{\overset{\text{def}}=}}


\def\Circlearrowright{\ensuremath{%
  \rotatebox[origin=c]{230}{$\circlearrowright$}}}

\newcommand\ct[1]{\text{\rmfamily\upshape #1}}
\newcommand\question[1]{ {\color{red} ...!? \small #1}}
\newcommand\caz[1]{\left\{\begin{array} #1 \end{array}\right.}
\newcommand\const{\text{\rmfamily\upshape const}}
\newcommand\toP{ \overset{\pro}{\to}}
\newcommand\toPP{ \overset{\text{PP}}{\to}}
\newcommand{\oeq}{\mathrel{\text{\textcircled{$=$}}}}





\usepackage{xcolor}
% \usepackage[normalem]{ulem}
\usepackage{lipsum}
\makeatletter
% \newcommand\colorwave[1][blue]{\bgroup \markoverwith{\lower3.5\p@\hbox{\sixly \textcolor{#1}{\char58}}}\ULon}
%\font\sixly=lasy6 % does not re-load if already loaded, so no memory problem.

\newmdtheoremenv[
linewidth= 1pt,linecolor= blue,%
leftmargin=20,rightmargin=20,innertopmargin=0pt, innerrightmargin=40,%
tikzsetting = { draw=lightgray, line width = 0.3pt,dashed,%
dash pattern = on 15pt off 3pt},%
splittopskip=\topskip,skipbelow=\baselineskip,%
skipabove=\baselineskip,ntheorem,roundcorner=0pt,
% backgroundcolor=pagebg,font=\color{orange}\sffamily, fontcolor=white
]{examplebox}{Exemple}[section]



\newcommand\R{\mathbb{R}}
\newcommand\Z{\mathbb{Z}}
\newcommand\N{\mathbb{N}}
\newcommand\E{\mathbb{E}}
\newcommand\F{\mathcal{F}}
\newcommand\cH{\mathcal{H}}
\newcommand\V{\mathbb{V}}
\newcommand\dmo{ ^{-1} }
\newcommand\kapa{\kappa}
\newcommand\im{Im}
\newcommand\hs{\mathcal{H}}





\usepackage{soul}

\makeatletter
\newcommand*{\whiten}[1]{\llap{\textcolor{white}{{\the\SOUL@token}}\hspace{#1pt}}}
\DeclareRobustCommand*\myul{%
    \def\SOUL@everyspace{\underline{\space}\kern\z@}%
    \def\SOUL@everytoken{%
     \setbox0=\hbox{\the\SOUL@token}%
     \ifdim\dp0>\z@
        \raisebox{\dp0}{\underline{\phantom{\the\SOUL@token}}}%
        \whiten{1}\whiten{0}%
        \whiten{-1}\whiten{-2}%
        \llap{\the\SOUL@token}%
     \else
        \underline{\the\SOUL@token}%
     \fi}%
\SOUL@}
\makeatother

\newcommand*{\demp}{\fontfamily{lmtt}\selectfont}

\DeclareTextFontCommand{\textdemp}{\demp}

\begin{document}

\ifcomment
Multiline
comment
\fi
\ifcomment
\myul{Typesetting test}
% \color[rgb]{1,1,1}
$∑_i^n≠ 60º±∞π∆¬≈√j∫h≤≥µ$

$\CR \R\pro\ind\pro\gS\pro
\mqty[a&b\\c&d]$
$\pro\mathbb{P}$
$\dd{x}$

  \[
    \alpha(x)=\left\{
                \begin{array}{ll}
                  x\\
                  \frac{1}{1+e^{-kx}}\\
                  \frac{e^x-e^{-x}}{e^x+e^{-x}}
                \end{array}
              \right.
  \]

  $\expval{x}$
  
  $\chi_\rho(ghg\dmo)=\Tr(\rho_{ghg\dmo})=\Tr(\rho_g\circ\rho_h\circ\rho\dmo_g)=\Tr(\rho_h)\overset{\mbox{\scalebox{0.5}{$\Tr(AB)=\Tr(BA)$}}}{=}\chi_\rho(h)$
  	$\mathop{\oplus}_{\substack{x\in X}}$

$\mat(\rho_g)=(a_{ij}(g))_{\scriptsize \substack{1\leq i\leq d \\ 1\leq j\leq d}}$ et $\mat(\rho'_g)=(a'_{ij}(g))_{\scriptsize \substack{1\leq i'\leq d' \\ 1\leq j'\leq d'}}$



\[\int_a^b{\mathbb{R}^2}g(u, v)\dd{P_{XY}}(u, v)=\iint g(u,v) f_{XY}(u, v)\dd \lambda(u) \dd \lambda(v)\]
$$\lim_{x\to\infty} f(x)$$	
$$\iiiint_V \mu(t,u,v,w) \,dt\,du\,dv\,dw$$
$$\sum_{n=1}^{\infty} 2^{-n} = 1$$	
\begin{definition}
	Si $X$ et $Y$ sont 2 v.a. ou definit la \textsc{Covariance} entre $X$ et $Y$ comme
	$\cov(X,Y)\overset{\text{def}}{=}\E\left[(X-\E(X))(Y-\E(Y))\right]=\E(XY)-\E(X)\E(Y)$.
\end{definition}
\fi
\pagebreak

% \tableofcontents

% insert your code here
%\input{./algebra/main.tex}
%\input{./geometrie-differentielle/main.tex}
%\input{./probabilite/main.tex}
%\input{./analyse-fonctionnelle/main.tex}
% \input{./Analyse-convexe-et-dualite-en-optimisation/main.tex}
%\input{./tikz/main.tex}
%\input{./Theorie-du-distributions/main.tex}
%\input{./optimisation/mine.tex}
 \input{./modelisation/main.tex}

% yves.aubry@univ-tln.fr : algebra

\end{document}


% yves.aubry@univ-tln.fr : algebra

\end{document}

%% !TEX encoding = UTF-8 Unicode
% !TEX TS-program = xelatex

\documentclass[french]{report}

%\usepackage[utf8]{inputenc}
%\usepackage[T1]{fontenc}
\usepackage{babel}


\newif\ifcomment
%\commenttrue # Show comments

\usepackage{physics}
\usepackage{amssymb}


\usepackage{amsthm}
% \usepackage{thmtools}
\usepackage{mathtools}
\usepackage{amsfonts}

\usepackage{color}

\usepackage{tikz}

\usepackage{geometry}
\geometry{a5paper, margin=0.1in, right=1cm}

\usepackage{dsfont}

\usepackage{graphicx}
\graphicspath{ {images/} }

\usepackage{faktor}

\usepackage{IEEEtrantools}
\usepackage{enumerate}   
\usepackage[PostScript=dvips]{"/Users/aware/Documents/Courses/diagrams"}


\newtheorem{theorem}{Théorème}[section]
\renewcommand{\thetheorem}{\arabic{theorem}}
\newtheorem{lemme}{Lemme}[section]
\renewcommand{\thelemme}{\arabic{lemme}}
\newtheorem{proposition}{Proposition}[section]
\renewcommand{\theproposition}{\arabic{proposition}}
\newtheorem{notations}{Notations}[section]
\newtheorem{problem}{Problème}[section]
\newtheorem{corollary}{Corollaire}[theorem]
\renewcommand{\thecorollary}{\arabic{corollary}}
\newtheorem{property}{Propriété}[section]
\newtheorem{objective}{Objectif}[section]

\theoremstyle{definition}
\newtheorem{definition}{Définition}[section]
\renewcommand{\thedefinition}{\arabic{definition}}
\newtheorem{exercise}{Exercice}[chapter]
\renewcommand{\theexercise}{\arabic{exercise}}
\newtheorem{example}{Exemple}[chapter]
\renewcommand{\theexample}{\arabic{example}}
\newtheorem*{solution}{Solution}
\newtheorem*{application}{Application}
\newtheorem*{notation}{Notation}
\newtheorem*{vocabulary}{Vocabulaire}
\newtheorem*{properties}{Propriétés}



\theoremstyle{remark}
\newtheorem*{remark}{Remarque}
\newtheorem*{rappel}{Rappel}


\usepackage{etoolbox}
\AtBeginEnvironment{exercise}{\small}
\AtBeginEnvironment{example}{\small}

\usepackage{cases}
\usepackage[red]{mypack}

\usepackage[framemethod=TikZ]{mdframed}

\definecolor{bg}{rgb}{0.4,0.25,0.95}
\definecolor{pagebg}{rgb}{0,0,0.5}
\surroundwithmdframed[
   topline=false,
   rightline=false,
   bottomline=false,
   leftmargin=\parindent,
   skipabove=8pt,
   skipbelow=8pt,
   linecolor=blue,
   innerbottommargin=10pt,
   % backgroundcolor=bg,font=\color{orange}\sffamily, fontcolor=white
]{definition}

\usepackage{empheq}
\usepackage[most]{tcolorbox}

\newtcbox{\mymath}[1][]{%
    nobeforeafter, math upper, tcbox raise base,
    enhanced, colframe=blue!30!black,
    colback=red!10, boxrule=1pt,
    #1}

\usepackage{unixode}


\DeclareMathOperator{\ord}{ord}
\DeclareMathOperator{\orb}{orb}
\DeclareMathOperator{\stab}{stab}
\DeclareMathOperator{\Stab}{stab}
\DeclareMathOperator{\ppcm}{ppcm}
\DeclareMathOperator{\conj}{Conj}
\DeclareMathOperator{\End}{End}
\DeclareMathOperator{\rot}{rot}
\DeclareMathOperator{\trs}{trace}
\DeclareMathOperator{\Ind}{Ind}
\DeclareMathOperator{\mat}{Mat}
\DeclareMathOperator{\id}{Id}
\DeclareMathOperator{\vect}{vect}
\DeclareMathOperator{\img}{img}
\DeclareMathOperator{\cov}{Cov}
\DeclareMathOperator{\dist}{dist}
\DeclareMathOperator{\irr}{Irr}
\DeclareMathOperator{\image}{Im}
\DeclareMathOperator{\pd}{\partial}
\DeclareMathOperator{\epi}{epi}
\DeclareMathOperator{\Argmin}{Argmin}
\DeclareMathOperator{\dom}{dom}
\DeclareMathOperator{\proj}{proj}
\DeclareMathOperator{\ctg}{ctg}
\DeclareMathOperator{\supp}{supp}
\DeclareMathOperator{\argmin}{argmin}
\DeclareMathOperator{\mult}{mult}
\DeclareMathOperator{\ch}{ch}
\DeclareMathOperator{\sh}{sh}
\DeclareMathOperator{\rang}{rang}
\DeclareMathOperator{\diam}{diam}
\DeclareMathOperator{\Epigraphe}{Epigraphe}




\usepackage{xcolor}
\everymath{\color{blue}}
%\everymath{\color[rgb]{0,1,1}}
%\pagecolor[rgb]{0,0,0.5}


\newcommand*{\pdtest}[3][]{\ensuremath{\frac{\partial^{#1} #2}{\partial #3}}}

\newcommand*{\deffunc}[6][]{\ensuremath{
\begin{array}{rcl}
#2 : #3 &\rightarrow& #4\\
#5 &\mapsto& #6
\end{array}
}}

\newcommand{\eqcolon}{\mathrel{\resizebox{\widthof{$\mathord{=}$}}{\height}{ $\!\!=\!\!\resizebox{1.2\width}{0.8\height}{\raisebox{0.23ex}{$\mathop{:}$}}\!\!$ }}}
\newcommand{\coloneq}{\mathrel{\resizebox{\widthof{$\mathord{=}$}}{\height}{ $\!\!\resizebox{1.2\width}{0.8\height}{\raisebox{0.23ex}{$\mathop{:}$}}\!\!=\!\!$ }}}
\newcommand{\eqcolonl}{\ensuremath{\mathrel{=\!\!\mathop{:}}}}
\newcommand{\coloneql}{\ensuremath{\mathrel{\mathop{:} \!\! =}}}
\newcommand{\vc}[1]{% inline column vector
  \left(\begin{smallmatrix}#1\end{smallmatrix}\right)%
}
\newcommand{\vr}[1]{% inline row vector
  \begin{smallmatrix}(\,#1\,)\end{smallmatrix}%
}
\makeatletter
\newcommand*{\defeq}{\ =\mathrel{\rlap{%
                     \raisebox{0.3ex}{$\m@th\cdot$}}%
                     \raisebox{-0.3ex}{$\m@th\cdot$}}%
                     }
\makeatother

\newcommand{\mathcircle}[1]{% inline row vector
 \overset{\circ}{#1}
}
\newcommand{\ulim}{% low limit
 \underline{\lim}
}
\newcommand{\ssi}{% iff
\iff
}
\newcommand{\ps}[2]{
\expval{#1 | #2}
}
\newcommand{\df}[1]{
\mqty{#1}
}
\newcommand{\n}[1]{
\norm{#1}
}
\newcommand{\sys}[1]{
\left\{\smqty{#1}\right.
}


\newcommand{\eqdef}{\ensuremath{\overset{\text{def}}=}}


\def\Circlearrowright{\ensuremath{%
  \rotatebox[origin=c]{230}{$\circlearrowright$}}}

\newcommand\ct[1]{\text{\rmfamily\upshape #1}}
\newcommand\question[1]{ {\color{red} ...!? \small #1}}
\newcommand\caz[1]{\left\{\begin{array} #1 \end{array}\right.}
\newcommand\const{\text{\rmfamily\upshape const}}
\newcommand\toP{ \overset{\pro}{\to}}
\newcommand\toPP{ \overset{\text{PP}}{\to}}
\newcommand{\oeq}{\mathrel{\text{\textcircled{$=$}}}}





\usepackage{xcolor}
% \usepackage[normalem]{ulem}
\usepackage{lipsum}
\makeatletter
% \newcommand\colorwave[1][blue]{\bgroup \markoverwith{\lower3.5\p@\hbox{\sixly \textcolor{#1}{\char58}}}\ULon}
%\font\sixly=lasy6 % does not re-load if already loaded, so no memory problem.

\newmdtheoremenv[
linewidth= 1pt,linecolor= blue,%
leftmargin=20,rightmargin=20,innertopmargin=0pt, innerrightmargin=40,%
tikzsetting = { draw=lightgray, line width = 0.3pt,dashed,%
dash pattern = on 15pt off 3pt},%
splittopskip=\topskip,skipbelow=\baselineskip,%
skipabove=\baselineskip,ntheorem,roundcorner=0pt,
% backgroundcolor=pagebg,font=\color{orange}\sffamily, fontcolor=white
]{examplebox}{Exemple}[section]



\newcommand\R{\mathbb{R}}
\newcommand\Z{\mathbb{Z}}
\newcommand\N{\mathbb{N}}
\newcommand\E{\mathbb{E}}
\newcommand\F{\mathcal{F}}
\newcommand\cH{\mathcal{H}}
\newcommand\V{\mathbb{V}}
\newcommand\dmo{ ^{-1} }
\newcommand\kapa{\kappa}
\newcommand\im{Im}
\newcommand\hs{\mathcal{H}}





\usepackage{soul}

\makeatletter
\newcommand*{\whiten}[1]{\llap{\textcolor{white}{{\the\SOUL@token}}\hspace{#1pt}}}
\DeclareRobustCommand*\myul{%
    \def\SOUL@everyspace{\underline{\space}\kern\z@}%
    \def\SOUL@everytoken{%
     \setbox0=\hbox{\the\SOUL@token}%
     \ifdim\dp0>\z@
        \raisebox{\dp0}{\underline{\phantom{\the\SOUL@token}}}%
        \whiten{1}\whiten{0}%
        \whiten{-1}\whiten{-2}%
        \llap{\the\SOUL@token}%
     \else
        \underline{\the\SOUL@token}%
     \fi}%
\SOUL@}
\makeatother

\newcommand*{\demp}{\fontfamily{lmtt}\selectfont}

\DeclareTextFontCommand{\textdemp}{\demp}

\begin{document}

\ifcomment
Multiline
comment
\fi
\ifcomment
\myul{Typesetting test}
% \color[rgb]{1,1,1}
$∑_i^n≠ 60º±∞π∆¬≈√j∫h≤≥µ$

$\CR \R\pro\ind\pro\gS\pro
\mqty[a&b\\c&d]$
$\pro\mathbb{P}$
$\dd{x}$

  \[
    \alpha(x)=\left\{
                \begin{array}{ll}
                  x\\
                  \frac{1}{1+e^{-kx}}\\
                  \frac{e^x-e^{-x}}{e^x+e^{-x}}
                \end{array}
              \right.
  \]

  $\expval{x}$
  
  $\chi_\rho(ghg\dmo)=\Tr(\rho_{ghg\dmo})=\Tr(\rho_g\circ\rho_h\circ\rho\dmo_g)=\Tr(\rho_h)\overset{\mbox{\scalebox{0.5}{$\Tr(AB)=\Tr(BA)$}}}{=}\chi_\rho(h)$
  	$\mathop{\oplus}_{\substack{x\in X}}$

$\mat(\rho_g)=(a_{ij}(g))_{\scriptsize \substack{1\leq i\leq d \\ 1\leq j\leq d}}$ et $\mat(\rho'_g)=(a'_{ij}(g))_{\scriptsize \substack{1\leq i'\leq d' \\ 1\leq j'\leq d'}}$



\[\int_a^b{\mathbb{R}^2}g(u, v)\dd{P_{XY}}(u, v)=\iint g(u,v) f_{XY}(u, v)\dd \lambda(u) \dd \lambda(v)\]
$$\lim_{x\to\infty} f(x)$$	
$$\iiiint_V \mu(t,u,v,w) \,dt\,du\,dv\,dw$$
$$\sum_{n=1}^{\infty} 2^{-n} = 1$$	
\begin{definition}
	Si $X$ et $Y$ sont 2 v.a. ou definit la \textsc{Covariance} entre $X$ et $Y$ comme
	$\cov(X,Y)\overset{\text{def}}{=}\E\left[(X-\E(X))(Y-\E(Y))\right]=\E(XY)-\E(X)\E(Y)$.
\end{definition}
\fi
\pagebreak

% \tableofcontents

% insert your code here
%% !TEX encoding = UTF-8 Unicode
% !TEX TS-program = xelatex

\documentclass[french]{report}

%\usepackage[utf8]{inputenc}
%\usepackage[T1]{fontenc}
\usepackage{babel}


\newif\ifcomment
%\commenttrue # Show comments

\usepackage{physics}
\usepackage{amssymb}


\usepackage{amsthm}
% \usepackage{thmtools}
\usepackage{mathtools}
\usepackage{amsfonts}

\usepackage{color}

\usepackage{tikz}

\usepackage{geometry}
\geometry{a5paper, margin=0.1in, right=1cm}

\usepackage{dsfont}

\usepackage{graphicx}
\graphicspath{ {images/} }

\usepackage{faktor}

\usepackage{IEEEtrantools}
\usepackage{enumerate}   
\usepackage[PostScript=dvips]{"/Users/aware/Documents/Courses/diagrams"}


\newtheorem{theorem}{Théorème}[section]
\renewcommand{\thetheorem}{\arabic{theorem}}
\newtheorem{lemme}{Lemme}[section]
\renewcommand{\thelemme}{\arabic{lemme}}
\newtheorem{proposition}{Proposition}[section]
\renewcommand{\theproposition}{\arabic{proposition}}
\newtheorem{notations}{Notations}[section]
\newtheorem{problem}{Problème}[section]
\newtheorem{corollary}{Corollaire}[theorem]
\renewcommand{\thecorollary}{\arabic{corollary}}
\newtheorem{property}{Propriété}[section]
\newtheorem{objective}{Objectif}[section]

\theoremstyle{definition}
\newtheorem{definition}{Définition}[section]
\renewcommand{\thedefinition}{\arabic{definition}}
\newtheorem{exercise}{Exercice}[chapter]
\renewcommand{\theexercise}{\arabic{exercise}}
\newtheorem{example}{Exemple}[chapter]
\renewcommand{\theexample}{\arabic{example}}
\newtheorem*{solution}{Solution}
\newtheorem*{application}{Application}
\newtheorem*{notation}{Notation}
\newtheorem*{vocabulary}{Vocabulaire}
\newtheorem*{properties}{Propriétés}



\theoremstyle{remark}
\newtheorem*{remark}{Remarque}
\newtheorem*{rappel}{Rappel}


\usepackage{etoolbox}
\AtBeginEnvironment{exercise}{\small}
\AtBeginEnvironment{example}{\small}

\usepackage{cases}
\usepackage[red]{mypack}

\usepackage[framemethod=TikZ]{mdframed}

\definecolor{bg}{rgb}{0.4,0.25,0.95}
\definecolor{pagebg}{rgb}{0,0,0.5}
\surroundwithmdframed[
   topline=false,
   rightline=false,
   bottomline=false,
   leftmargin=\parindent,
   skipabove=8pt,
   skipbelow=8pt,
   linecolor=blue,
   innerbottommargin=10pt,
   % backgroundcolor=bg,font=\color{orange}\sffamily, fontcolor=white
]{definition}

\usepackage{empheq}
\usepackage[most]{tcolorbox}

\newtcbox{\mymath}[1][]{%
    nobeforeafter, math upper, tcbox raise base,
    enhanced, colframe=blue!30!black,
    colback=red!10, boxrule=1pt,
    #1}

\usepackage{unixode}


\DeclareMathOperator{\ord}{ord}
\DeclareMathOperator{\orb}{orb}
\DeclareMathOperator{\stab}{stab}
\DeclareMathOperator{\Stab}{stab}
\DeclareMathOperator{\ppcm}{ppcm}
\DeclareMathOperator{\conj}{Conj}
\DeclareMathOperator{\End}{End}
\DeclareMathOperator{\rot}{rot}
\DeclareMathOperator{\trs}{trace}
\DeclareMathOperator{\Ind}{Ind}
\DeclareMathOperator{\mat}{Mat}
\DeclareMathOperator{\id}{Id}
\DeclareMathOperator{\vect}{vect}
\DeclareMathOperator{\img}{img}
\DeclareMathOperator{\cov}{Cov}
\DeclareMathOperator{\dist}{dist}
\DeclareMathOperator{\irr}{Irr}
\DeclareMathOperator{\image}{Im}
\DeclareMathOperator{\pd}{\partial}
\DeclareMathOperator{\epi}{epi}
\DeclareMathOperator{\Argmin}{Argmin}
\DeclareMathOperator{\dom}{dom}
\DeclareMathOperator{\proj}{proj}
\DeclareMathOperator{\ctg}{ctg}
\DeclareMathOperator{\supp}{supp}
\DeclareMathOperator{\argmin}{argmin}
\DeclareMathOperator{\mult}{mult}
\DeclareMathOperator{\ch}{ch}
\DeclareMathOperator{\sh}{sh}
\DeclareMathOperator{\rang}{rang}
\DeclareMathOperator{\diam}{diam}
\DeclareMathOperator{\Epigraphe}{Epigraphe}




\usepackage{xcolor}
\everymath{\color{blue}}
%\everymath{\color[rgb]{0,1,1}}
%\pagecolor[rgb]{0,0,0.5}


\newcommand*{\pdtest}[3][]{\ensuremath{\frac{\partial^{#1} #2}{\partial #3}}}

\newcommand*{\deffunc}[6][]{\ensuremath{
\begin{array}{rcl}
#2 : #3 &\rightarrow& #4\\
#5 &\mapsto& #6
\end{array}
}}

\newcommand{\eqcolon}{\mathrel{\resizebox{\widthof{$\mathord{=}$}}{\height}{ $\!\!=\!\!\resizebox{1.2\width}{0.8\height}{\raisebox{0.23ex}{$\mathop{:}$}}\!\!$ }}}
\newcommand{\coloneq}{\mathrel{\resizebox{\widthof{$\mathord{=}$}}{\height}{ $\!\!\resizebox{1.2\width}{0.8\height}{\raisebox{0.23ex}{$\mathop{:}$}}\!\!=\!\!$ }}}
\newcommand{\eqcolonl}{\ensuremath{\mathrel{=\!\!\mathop{:}}}}
\newcommand{\coloneql}{\ensuremath{\mathrel{\mathop{:} \!\! =}}}
\newcommand{\vc}[1]{% inline column vector
  \left(\begin{smallmatrix}#1\end{smallmatrix}\right)%
}
\newcommand{\vr}[1]{% inline row vector
  \begin{smallmatrix}(\,#1\,)\end{smallmatrix}%
}
\makeatletter
\newcommand*{\defeq}{\ =\mathrel{\rlap{%
                     \raisebox{0.3ex}{$\m@th\cdot$}}%
                     \raisebox{-0.3ex}{$\m@th\cdot$}}%
                     }
\makeatother

\newcommand{\mathcircle}[1]{% inline row vector
 \overset{\circ}{#1}
}
\newcommand{\ulim}{% low limit
 \underline{\lim}
}
\newcommand{\ssi}{% iff
\iff
}
\newcommand{\ps}[2]{
\expval{#1 | #2}
}
\newcommand{\df}[1]{
\mqty{#1}
}
\newcommand{\n}[1]{
\norm{#1}
}
\newcommand{\sys}[1]{
\left\{\smqty{#1}\right.
}


\newcommand{\eqdef}{\ensuremath{\overset{\text{def}}=}}


\def\Circlearrowright{\ensuremath{%
  \rotatebox[origin=c]{230}{$\circlearrowright$}}}

\newcommand\ct[1]{\text{\rmfamily\upshape #1}}
\newcommand\question[1]{ {\color{red} ...!? \small #1}}
\newcommand\caz[1]{\left\{\begin{array} #1 \end{array}\right.}
\newcommand\const{\text{\rmfamily\upshape const}}
\newcommand\toP{ \overset{\pro}{\to}}
\newcommand\toPP{ \overset{\text{PP}}{\to}}
\newcommand{\oeq}{\mathrel{\text{\textcircled{$=$}}}}





\usepackage{xcolor}
% \usepackage[normalem]{ulem}
\usepackage{lipsum}
\makeatletter
% \newcommand\colorwave[1][blue]{\bgroup \markoverwith{\lower3.5\p@\hbox{\sixly \textcolor{#1}{\char58}}}\ULon}
%\font\sixly=lasy6 % does not re-load if already loaded, so no memory problem.

\newmdtheoremenv[
linewidth= 1pt,linecolor= blue,%
leftmargin=20,rightmargin=20,innertopmargin=0pt, innerrightmargin=40,%
tikzsetting = { draw=lightgray, line width = 0.3pt,dashed,%
dash pattern = on 15pt off 3pt},%
splittopskip=\topskip,skipbelow=\baselineskip,%
skipabove=\baselineskip,ntheorem,roundcorner=0pt,
% backgroundcolor=pagebg,font=\color{orange}\sffamily, fontcolor=white
]{examplebox}{Exemple}[section]



\newcommand\R{\mathbb{R}}
\newcommand\Z{\mathbb{Z}}
\newcommand\N{\mathbb{N}}
\newcommand\E{\mathbb{E}}
\newcommand\F{\mathcal{F}}
\newcommand\cH{\mathcal{H}}
\newcommand\V{\mathbb{V}}
\newcommand\dmo{ ^{-1} }
\newcommand\kapa{\kappa}
\newcommand\im{Im}
\newcommand\hs{\mathcal{H}}





\usepackage{soul}

\makeatletter
\newcommand*{\whiten}[1]{\llap{\textcolor{white}{{\the\SOUL@token}}\hspace{#1pt}}}
\DeclareRobustCommand*\myul{%
    \def\SOUL@everyspace{\underline{\space}\kern\z@}%
    \def\SOUL@everytoken{%
     \setbox0=\hbox{\the\SOUL@token}%
     \ifdim\dp0>\z@
        \raisebox{\dp0}{\underline{\phantom{\the\SOUL@token}}}%
        \whiten{1}\whiten{0}%
        \whiten{-1}\whiten{-2}%
        \llap{\the\SOUL@token}%
     \else
        \underline{\the\SOUL@token}%
     \fi}%
\SOUL@}
\makeatother

\newcommand*{\demp}{\fontfamily{lmtt}\selectfont}

\DeclareTextFontCommand{\textdemp}{\demp}

\begin{document}

\ifcomment
Multiline
comment
\fi
\ifcomment
\myul{Typesetting test}
% \color[rgb]{1,1,1}
$∑_i^n≠ 60º±∞π∆¬≈√j∫h≤≥µ$

$\CR \R\pro\ind\pro\gS\pro
\mqty[a&b\\c&d]$
$\pro\mathbb{P}$
$\dd{x}$

  \[
    \alpha(x)=\left\{
                \begin{array}{ll}
                  x\\
                  \frac{1}{1+e^{-kx}}\\
                  \frac{e^x-e^{-x}}{e^x+e^{-x}}
                \end{array}
              \right.
  \]

  $\expval{x}$
  
  $\chi_\rho(ghg\dmo)=\Tr(\rho_{ghg\dmo})=\Tr(\rho_g\circ\rho_h\circ\rho\dmo_g)=\Tr(\rho_h)\overset{\mbox{\scalebox{0.5}{$\Tr(AB)=\Tr(BA)$}}}{=}\chi_\rho(h)$
  	$\mathop{\oplus}_{\substack{x\in X}}$

$\mat(\rho_g)=(a_{ij}(g))_{\scriptsize \substack{1\leq i\leq d \\ 1\leq j\leq d}}$ et $\mat(\rho'_g)=(a'_{ij}(g))_{\scriptsize \substack{1\leq i'\leq d' \\ 1\leq j'\leq d'}}$



\[\int_a^b{\mathbb{R}^2}g(u, v)\dd{P_{XY}}(u, v)=\iint g(u,v) f_{XY}(u, v)\dd \lambda(u) \dd \lambda(v)\]
$$\lim_{x\to\infty} f(x)$$	
$$\iiiint_V \mu(t,u,v,w) \,dt\,du\,dv\,dw$$
$$\sum_{n=1}^{\infty} 2^{-n} = 1$$	
\begin{definition}
	Si $X$ et $Y$ sont 2 v.a. ou definit la \textsc{Covariance} entre $X$ et $Y$ comme
	$\cov(X,Y)\overset{\text{def}}{=}\E\left[(X-\E(X))(Y-\E(Y))\right]=\E(XY)-\E(X)\E(Y)$.
\end{definition}
\fi
\pagebreak

% \tableofcontents

% insert your code here
%\input{./algebra/main.tex}
%\input{./geometrie-differentielle/main.tex}
%\input{./probabilite/main.tex}
%\input{./analyse-fonctionnelle/main.tex}
% \input{./Analyse-convexe-et-dualite-en-optimisation/main.tex}
%\input{./tikz/main.tex}
%\input{./Theorie-du-distributions/main.tex}
%\input{./optimisation/mine.tex}
 \input{./modelisation/main.tex}

% yves.aubry@univ-tln.fr : algebra

\end{document}

%% !TEX encoding = UTF-8 Unicode
% !TEX TS-program = xelatex

\documentclass[french]{report}

%\usepackage[utf8]{inputenc}
%\usepackage[T1]{fontenc}
\usepackage{babel}


\newif\ifcomment
%\commenttrue # Show comments

\usepackage{physics}
\usepackage{amssymb}


\usepackage{amsthm}
% \usepackage{thmtools}
\usepackage{mathtools}
\usepackage{amsfonts}

\usepackage{color}

\usepackage{tikz}

\usepackage{geometry}
\geometry{a5paper, margin=0.1in, right=1cm}

\usepackage{dsfont}

\usepackage{graphicx}
\graphicspath{ {images/} }

\usepackage{faktor}

\usepackage{IEEEtrantools}
\usepackage{enumerate}   
\usepackage[PostScript=dvips]{"/Users/aware/Documents/Courses/diagrams"}


\newtheorem{theorem}{Théorème}[section]
\renewcommand{\thetheorem}{\arabic{theorem}}
\newtheorem{lemme}{Lemme}[section]
\renewcommand{\thelemme}{\arabic{lemme}}
\newtheorem{proposition}{Proposition}[section]
\renewcommand{\theproposition}{\arabic{proposition}}
\newtheorem{notations}{Notations}[section]
\newtheorem{problem}{Problème}[section]
\newtheorem{corollary}{Corollaire}[theorem]
\renewcommand{\thecorollary}{\arabic{corollary}}
\newtheorem{property}{Propriété}[section]
\newtheorem{objective}{Objectif}[section]

\theoremstyle{definition}
\newtheorem{definition}{Définition}[section]
\renewcommand{\thedefinition}{\arabic{definition}}
\newtheorem{exercise}{Exercice}[chapter]
\renewcommand{\theexercise}{\arabic{exercise}}
\newtheorem{example}{Exemple}[chapter]
\renewcommand{\theexample}{\arabic{example}}
\newtheorem*{solution}{Solution}
\newtheorem*{application}{Application}
\newtheorem*{notation}{Notation}
\newtheorem*{vocabulary}{Vocabulaire}
\newtheorem*{properties}{Propriétés}



\theoremstyle{remark}
\newtheorem*{remark}{Remarque}
\newtheorem*{rappel}{Rappel}


\usepackage{etoolbox}
\AtBeginEnvironment{exercise}{\small}
\AtBeginEnvironment{example}{\small}

\usepackage{cases}
\usepackage[red]{mypack}

\usepackage[framemethod=TikZ]{mdframed}

\definecolor{bg}{rgb}{0.4,0.25,0.95}
\definecolor{pagebg}{rgb}{0,0,0.5}
\surroundwithmdframed[
   topline=false,
   rightline=false,
   bottomline=false,
   leftmargin=\parindent,
   skipabove=8pt,
   skipbelow=8pt,
   linecolor=blue,
   innerbottommargin=10pt,
   % backgroundcolor=bg,font=\color{orange}\sffamily, fontcolor=white
]{definition}

\usepackage{empheq}
\usepackage[most]{tcolorbox}

\newtcbox{\mymath}[1][]{%
    nobeforeafter, math upper, tcbox raise base,
    enhanced, colframe=blue!30!black,
    colback=red!10, boxrule=1pt,
    #1}

\usepackage{unixode}


\DeclareMathOperator{\ord}{ord}
\DeclareMathOperator{\orb}{orb}
\DeclareMathOperator{\stab}{stab}
\DeclareMathOperator{\Stab}{stab}
\DeclareMathOperator{\ppcm}{ppcm}
\DeclareMathOperator{\conj}{Conj}
\DeclareMathOperator{\End}{End}
\DeclareMathOperator{\rot}{rot}
\DeclareMathOperator{\trs}{trace}
\DeclareMathOperator{\Ind}{Ind}
\DeclareMathOperator{\mat}{Mat}
\DeclareMathOperator{\id}{Id}
\DeclareMathOperator{\vect}{vect}
\DeclareMathOperator{\img}{img}
\DeclareMathOperator{\cov}{Cov}
\DeclareMathOperator{\dist}{dist}
\DeclareMathOperator{\irr}{Irr}
\DeclareMathOperator{\image}{Im}
\DeclareMathOperator{\pd}{\partial}
\DeclareMathOperator{\epi}{epi}
\DeclareMathOperator{\Argmin}{Argmin}
\DeclareMathOperator{\dom}{dom}
\DeclareMathOperator{\proj}{proj}
\DeclareMathOperator{\ctg}{ctg}
\DeclareMathOperator{\supp}{supp}
\DeclareMathOperator{\argmin}{argmin}
\DeclareMathOperator{\mult}{mult}
\DeclareMathOperator{\ch}{ch}
\DeclareMathOperator{\sh}{sh}
\DeclareMathOperator{\rang}{rang}
\DeclareMathOperator{\diam}{diam}
\DeclareMathOperator{\Epigraphe}{Epigraphe}




\usepackage{xcolor}
\everymath{\color{blue}}
%\everymath{\color[rgb]{0,1,1}}
%\pagecolor[rgb]{0,0,0.5}


\newcommand*{\pdtest}[3][]{\ensuremath{\frac{\partial^{#1} #2}{\partial #3}}}

\newcommand*{\deffunc}[6][]{\ensuremath{
\begin{array}{rcl}
#2 : #3 &\rightarrow& #4\\
#5 &\mapsto& #6
\end{array}
}}

\newcommand{\eqcolon}{\mathrel{\resizebox{\widthof{$\mathord{=}$}}{\height}{ $\!\!=\!\!\resizebox{1.2\width}{0.8\height}{\raisebox{0.23ex}{$\mathop{:}$}}\!\!$ }}}
\newcommand{\coloneq}{\mathrel{\resizebox{\widthof{$\mathord{=}$}}{\height}{ $\!\!\resizebox{1.2\width}{0.8\height}{\raisebox{0.23ex}{$\mathop{:}$}}\!\!=\!\!$ }}}
\newcommand{\eqcolonl}{\ensuremath{\mathrel{=\!\!\mathop{:}}}}
\newcommand{\coloneql}{\ensuremath{\mathrel{\mathop{:} \!\! =}}}
\newcommand{\vc}[1]{% inline column vector
  \left(\begin{smallmatrix}#1\end{smallmatrix}\right)%
}
\newcommand{\vr}[1]{% inline row vector
  \begin{smallmatrix}(\,#1\,)\end{smallmatrix}%
}
\makeatletter
\newcommand*{\defeq}{\ =\mathrel{\rlap{%
                     \raisebox{0.3ex}{$\m@th\cdot$}}%
                     \raisebox{-0.3ex}{$\m@th\cdot$}}%
                     }
\makeatother

\newcommand{\mathcircle}[1]{% inline row vector
 \overset{\circ}{#1}
}
\newcommand{\ulim}{% low limit
 \underline{\lim}
}
\newcommand{\ssi}{% iff
\iff
}
\newcommand{\ps}[2]{
\expval{#1 | #2}
}
\newcommand{\df}[1]{
\mqty{#1}
}
\newcommand{\n}[1]{
\norm{#1}
}
\newcommand{\sys}[1]{
\left\{\smqty{#1}\right.
}


\newcommand{\eqdef}{\ensuremath{\overset{\text{def}}=}}


\def\Circlearrowright{\ensuremath{%
  \rotatebox[origin=c]{230}{$\circlearrowright$}}}

\newcommand\ct[1]{\text{\rmfamily\upshape #1}}
\newcommand\question[1]{ {\color{red} ...!? \small #1}}
\newcommand\caz[1]{\left\{\begin{array} #1 \end{array}\right.}
\newcommand\const{\text{\rmfamily\upshape const}}
\newcommand\toP{ \overset{\pro}{\to}}
\newcommand\toPP{ \overset{\text{PP}}{\to}}
\newcommand{\oeq}{\mathrel{\text{\textcircled{$=$}}}}





\usepackage{xcolor}
% \usepackage[normalem]{ulem}
\usepackage{lipsum}
\makeatletter
% \newcommand\colorwave[1][blue]{\bgroup \markoverwith{\lower3.5\p@\hbox{\sixly \textcolor{#1}{\char58}}}\ULon}
%\font\sixly=lasy6 % does not re-load if already loaded, so no memory problem.

\newmdtheoremenv[
linewidth= 1pt,linecolor= blue,%
leftmargin=20,rightmargin=20,innertopmargin=0pt, innerrightmargin=40,%
tikzsetting = { draw=lightgray, line width = 0.3pt,dashed,%
dash pattern = on 15pt off 3pt},%
splittopskip=\topskip,skipbelow=\baselineskip,%
skipabove=\baselineskip,ntheorem,roundcorner=0pt,
% backgroundcolor=pagebg,font=\color{orange}\sffamily, fontcolor=white
]{examplebox}{Exemple}[section]



\newcommand\R{\mathbb{R}}
\newcommand\Z{\mathbb{Z}}
\newcommand\N{\mathbb{N}}
\newcommand\E{\mathbb{E}}
\newcommand\F{\mathcal{F}}
\newcommand\cH{\mathcal{H}}
\newcommand\V{\mathbb{V}}
\newcommand\dmo{ ^{-1} }
\newcommand\kapa{\kappa}
\newcommand\im{Im}
\newcommand\hs{\mathcal{H}}





\usepackage{soul}

\makeatletter
\newcommand*{\whiten}[1]{\llap{\textcolor{white}{{\the\SOUL@token}}\hspace{#1pt}}}
\DeclareRobustCommand*\myul{%
    \def\SOUL@everyspace{\underline{\space}\kern\z@}%
    \def\SOUL@everytoken{%
     \setbox0=\hbox{\the\SOUL@token}%
     \ifdim\dp0>\z@
        \raisebox{\dp0}{\underline{\phantom{\the\SOUL@token}}}%
        \whiten{1}\whiten{0}%
        \whiten{-1}\whiten{-2}%
        \llap{\the\SOUL@token}%
     \else
        \underline{\the\SOUL@token}%
     \fi}%
\SOUL@}
\makeatother

\newcommand*{\demp}{\fontfamily{lmtt}\selectfont}

\DeclareTextFontCommand{\textdemp}{\demp}

\begin{document}

\ifcomment
Multiline
comment
\fi
\ifcomment
\myul{Typesetting test}
% \color[rgb]{1,1,1}
$∑_i^n≠ 60º±∞π∆¬≈√j∫h≤≥µ$

$\CR \R\pro\ind\pro\gS\pro
\mqty[a&b\\c&d]$
$\pro\mathbb{P}$
$\dd{x}$

  \[
    \alpha(x)=\left\{
                \begin{array}{ll}
                  x\\
                  \frac{1}{1+e^{-kx}}\\
                  \frac{e^x-e^{-x}}{e^x+e^{-x}}
                \end{array}
              \right.
  \]

  $\expval{x}$
  
  $\chi_\rho(ghg\dmo)=\Tr(\rho_{ghg\dmo})=\Tr(\rho_g\circ\rho_h\circ\rho\dmo_g)=\Tr(\rho_h)\overset{\mbox{\scalebox{0.5}{$\Tr(AB)=\Tr(BA)$}}}{=}\chi_\rho(h)$
  	$\mathop{\oplus}_{\substack{x\in X}}$

$\mat(\rho_g)=(a_{ij}(g))_{\scriptsize \substack{1\leq i\leq d \\ 1\leq j\leq d}}$ et $\mat(\rho'_g)=(a'_{ij}(g))_{\scriptsize \substack{1\leq i'\leq d' \\ 1\leq j'\leq d'}}$



\[\int_a^b{\mathbb{R}^2}g(u, v)\dd{P_{XY}}(u, v)=\iint g(u,v) f_{XY}(u, v)\dd \lambda(u) \dd \lambda(v)\]
$$\lim_{x\to\infty} f(x)$$	
$$\iiiint_V \mu(t,u,v,w) \,dt\,du\,dv\,dw$$
$$\sum_{n=1}^{\infty} 2^{-n} = 1$$	
\begin{definition}
	Si $X$ et $Y$ sont 2 v.a. ou definit la \textsc{Covariance} entre $X$ et $Y$ comme
	$\cov(X,Y)\overset{\text{def}}{=}\E\left[(X-\E(X))(Y-\E(Y))\right]=\E(XY)-\E(X)\E(Y)$.
\end{definition}
\fi
\pagebreak

% \tableofcontents

% insert your code here
%\input{./algebra/main.tex}
%\input{./geometrie-differentielle/main.tex}
%\input{./probabilite/main.tex}
%\input{./analyse-fonctionnelle/main.tex}
% \input{./Analyse-convexe-et-dualite-en-optimisation/main.tex}
%\input{./tikz/main.tex}
%\input{./Theorie-du-distributions/main.tex}
%\input{./optimisation/mine.tex}
 \input{./modelisation/main.tex}

% yves.aubry@univ-tln.fr : algebra

\end{document}

%% !TEX encoding = UTF-8 Unicode
% !TEX TS-program = xelatex

\documentclass[french]{report}

%\usepackage[utf8]{inputenc}
%\usepackage[T1]{fontenc}
\usepackage{babel}


\newif\ifcomment
%\commenttrue # Show comments

\usepackage{physics}
\usepackage{amssymb}


\usepackage{amsthm}
% \usepackage{thmtools}
\usepackage{mathtools}
\usepackage{amsfonts}

\usepackage{color}

\usepackage{tikz}

\usepackage{geometry}
\geometry{a5paper, margin=0.1in, right=1cm}

\usepackage{dsfont}

\usepackage{graphicx}
\graphicspath{ {images/} }

\usepackage{faktor}

\usepackage{IEEEtrantools}
\usepackage{enumerate}   
\usepackage[PostScript=dvips]{"/Users/aware/Documents/Courses/diagrams"}


\newtheorem{theorem}{Théorème}[section]
\renewcommand{\thetheorem}{\arabic{theorem}}
\newtheorem{lemme}{Lemme}[section]
\renewcommand{\thelemme}{\arabic{lemme}}
\newtheorem{proposition}{Proposition}[section]
\renewcommand{\theproposition}{\arabic{proposition}}
\newtheorem{notations}{Notations}[section]
\newtheorem{problem}{Problème}[section]
\newtheorem{corollary}{Corollaire}[theorem]
\renewcommand{\thecorollary}{\arabic{corollary}}
\newtheorem{property}{Propriété}[section]
\newtheorem{objective}{Objectif}[section]

\theoremstyle{definition}
\newtheorem{definition}{Définition}[section]
\renewcommand{\thedefinition}{\arabic{definition}}
\newtheorem{exercise}{Exercice}[chapter]
\renewcommand{\theexercise}{\arabic{exercise}}
\newtheorem{example}{Exemple}[chapter]
\renewcommand{\theexample}{\arabic{example}}
\newtheorem*{solution}{Solution}
\newtheorem*{application}{Application}
\newtheorem*{notation}{Notation}
\newtheorem*{vocabulary}{Vocabulaire}
\newtheorem*{properties}{Propriétés}



\theoremstyle{remark}
\newtheorem*{remark}{Remarque}
\newtheorem*{rappel}{Rappel}


\usepackage{etoolbox}
\AtBeginEnvironment{exercise}{\small}
\AtBeginEnvironment{example}{\small}

\usepackage{cases}
\usepackage[red]{mypack}

\usepackage[framemethod=TikZ]{mdframed}

\definecolor{bg}{rgb}{0.4,0.25,0.95}
\definecolor{pagebg}{rgb}{0,0,0.5}
\surroundwithmdframed[
   topline=false,
   rightline=false,
   bottomline=false,
   leftmargin=\parindent,
   skipabove=8pt,
   skipbelow=8pt,
   linecolor=blue,
   innerbottommargin=10pt,
   % backgroundcolor=bg,font=\color{orange}\sffamily, fontcolor=white
]{definition}

\usepackage{empheq}
\usepackage[most]{tcolorbox}

\newtcbox{\mymath}[1][]{%
    nobeforeafter, math upper, tcbox raise base,
    enhanced, colframe=blue!30!black,
    colback=red!10, boxrule=1pt,
    #1}

\usepackage{unixode}


\DeclareMathOperator{\ord}{ord}
\DeclareMathOperator{\orb}{orb}
\DeclareMathOperator{\stab}{stab}
\DeclareMathOperator{\Stab}{stab}
\DeclareMathOperator{\ppcm}{ppcm}
\DeclareMathOperator{\conj}{Conj}
\DeclareMathOperator{\End}{End}
\DeclareMathOperator{\rot}{rot}
\DeclareMathOperator{\trs}{trace}
\DeclareMathOperator{\Ind}{Ind}
\DeclareMathOperator{\mat}{Mat}
\DeclareMathOperator{\id}{Id}
\DeclareMathOperator{\vect}{vect}
\DeclareMathOperator{\img}{img}
\DeclareMathOperator{\cov}{Cov}
\DeclareMathOperator{\dist}{dist}
\DeclareMathOperator{\irr}{Irr}
\DeclareMathOperator{\image}{Im}
\DeclareMathOperator{\pd}{\partial}
\DeclareMathOperator{\epi}{epi}
\DeclareMathOperator{\Argmin}{Argmin}
\DeclareMathOperator{\dom}{dom}
\DeclareMathOperator{\proj}{proj}
\DeclareMathOperator{\ctg}{ctg}
\DeclareMathOperator{\supp}{supp}
\DeclareMathOperator{\argmin}{argmin}
\DeclareMathOperator{\mult}{mult}
\DeclareMathOperator{\ch}{ch}
\DeclareMathOperator{\sh}{sh}
\DeclareMathOperator{\rang}{rang}
\DeclareMathOperator{\diam}{diam}
\DeclareMathOperator{\Epigraphe}{Epigraphe}




\usepackage{xcolor}
\everymath{\color{blue}}
%\everymath{\color[rgb]{0,1,1}}
%\pagecolor[rgb]{0,0,0.5}


\newcommand*{\pdtest}[3][]{\ensuremath{\frac{\partial^{#1} #2}{\partial #3}}}

\newcommand*{\deffunc}[6][]{\ensuremath{
\begin{array}{rcl}
#2 : #3 &\rightarrow& #4\\
#5 &\mapsto& #6
\end{array}
}}

\newcommand{\eqcolon}{\mathrel{\resizebox{\widthof{$\mathord{=}$}}{\height}{ $\!\!=\!\!\resizebox{1.2\width}{0.8\height}{\raisebox{0.23ex}{$\mathop{:}$}}\!\!$ }}}
\newcommand{\coloneq}{\mathrel{\resizebox{\widthof{$\mathord{=}$}}{\height}{ $\!\!\resizebox{1.2\width}{0.8\height}{\raisebox{0.23ex}{$\mathop{:}$}}\!\!=\!\!$ }}}
\newcommand{\eqcolonl}{\ensuremath{\mathrel{=\!\!\mathop{:}}}}
\newcommand{\coloneql}{\ensuremath{\mathrel{\mathop{:} \!\! =}}}
\newcommand{\vc}[1]{% inline column vector
  \left(\begin{smallmatrix}#1\end{smallmatrix}\right)%
}
\newcommand{\vr}[1]{% inline row vector
  \begin{smallmatrix}(\,#1\,)\end{smallmatrix}%
}
\makeatletter
\newcommand*{\defeq}{\ =\mathrel{\rlap{%
                     \raisebox{0.3ex}{$\m@th\cdot$}}%
                     \raisebox{-0.3ex}{$\m@th\cdot$}}%
                     }
\makeatother

\newcommand{\mathcircle}[1]{% inline row vector
 \overset{\circ}{#1}
}
\newcommand{\ulim}{% low limit
 \underline{\lim}
}
\newcommand{\ssi}{% iff
\iff
}
\newcommand{\ps}[2]{
\expval{#1 | #2}
}
\newcommand{\df}[1]{
\mqty{#1}
}
\newcommand{\n}[1]{
\norm{#1}
}
\newcommand{\sys}[1]{
\left\{\smqty{#1}\right.
}


\newcommand{\eqdef}{\ensuremath{\overset{\text{def}}=}}


\def\Circlearrowright{\ensuremath{%
  \rotatebox[origin=c]{230}{$\circlearrowright$}}}

\newcommand\ct[1]{\text{\rmfamily\upshape #1}}
\newcommand\question[1]{ {\color{red} ...!? \small #1}}
\newcommand\caz[1]{\left\{\begin{array} #1 \end{array}\right.}
\newcommand\const{\text{\rmfamily\upshape const}}
\newcommand\toP{ \overset{\pro}{\to}}
\newcommand\toPP{ \overset{\text{PP}}{\to}}
\newcommand{\oeq}{\mathrel{\text{\textcircled{$=$}}}}





\usepackage{xcolor}
% \usepackage[normalem]{ulem}
\usepackage{lipsum}
\makeatletter
% \newcommand\colorwave[1][blue]{\bgroup \markoverwith{\lower3.5\p@\hbox{\sixly \textcolor{#1}{\char58}}}\ULon}
%\font\sixly=lasy6 % does not re-load if already loaded, so no memory problem.

\newmdtheoremenv[
linewidth= 1pt,linecolor= blue,%
leftmargin=20,rightmargin=20,innertopmargin=0pt, innerrightmargin=40,%
tikzsetting = { draw=lightgray, line width = 0.3pt,dashed,%
dash pattern = on 15pt off 3pt},%
splittopskip=\topskip,skipbelow=\baselineskip,%
skipabove=\baselineskip,ntheorem,roundcorner=0pt,
% backgroundcolor=pagebg,font=\color{orange}\sffamily, fontcolor=white
]{examplebox}{Exemple}[section]



\newcommand\R{\mathbb{R}}
\newcommand\Z{\mathbb{Z}}
\newcommand\N{\mathbb{N}}
\newcommand\E{\mathbb{E}}
\newcommand\F{\mathcal{F}}
\newcommand\cH{\mathcal{H}}
\newcommand\V{\mathbb{V}}
\newcommand\dmo{ ^{-1} }
\newcommand\kapa{\kappa}
\newcommand\im{Im}
\newcommand\hs{\mathcal{H}}





\usepackage{soul}

\makeatletter
\newcommand*{\whiten}[1]{\llap{\textcolor{white}{{\the\SOUL@token}}\hspace{#1pt}}}
\DeclareRobustCommand*\myul{%
    \def\SOUL@everyspace{\underline{\space}\kern\z@}%
    \def\SOUL@everytoken{%
     \setbox0=\hbox{\the\SOUL@token}%
     \ifdim\dp0>\z@
        \raisebox{\dp0}{\underline{\phantom{\the\SOUL@token}}}%
        \whiten{1}\whiten{0}%
        \whiten{-1}\whiten{-2}%
        \llap{\the\SOUL@token}%
     \else
        \underline{\the\SOUL@token}%
     \fi}%
\SOUL@}
\makeatother

\newcommand*{\demp}{\fontfamily{lmtt}\selectfont}

\DeclareTextFontCommand{\textdemp}{\demp}

\begin{document}

\ifcomment
Multiline
comment
\fi
\ifcomment
\myul{Typesetting test}
% \color[rgb]{1,1,1}
$∑_i^n≠ 60º±∞π∆¬≈√j∫h≤≥µ$

$\CR \R\pro\ind\pro\gS\pro
\mqty[a&b\\c&d]$
$\pro\mathbb{P}$
$\dd{x}$

  \[
    \alpha(x)=\left\{
                \begin{array}{ll}
                  x\\
                  \frac{1}{1+e^{-kx}}\\
                  \frac{e^x-e^{-x}}{e^x+e^{-x}}
                \end{array}
              \right.
  \]

  $\expval{x}$
  
  $\chi_\rho(ghg\dmo)=\Tr(\rho_{ghg\dmo})=\Tr(\rho_g\circ\rho_h\circ\rho\dmo_g)=\Tr(\rho_h)\overset{\mbox{\scalebox{0.5}{$\Tr(AB)=\Tr(BA)$}}}{=}\chi_\rho(h)$
  	$\mathop{\oplus}_{\substack{x\in X}}$

$\mat(\rho_g)=(a_{ij}(g))_{\scriptsize \substack{1\leq i\leq d \\ 1\leq j\leq d}}$ et $\mat(\rho'_g)=(a'_{ij}(g))_{\scriptsize \substack{1\leq i'\leq d' \\ 1\leq j'\leq d'}}$



\[\int_a^b{\mathbb{R}^2}g(u, v)\dd{P_{XY}}(u, v)=\iint g(u,v) f_{XY}(u, v)\dd \lambda(u) \dd \lambda(v)\]
$$\lim_{x\to\infty} f(x)$$	
$$\iiiint_V \mu(t,u,v,w) \,dt\,du\,dv\,dw$$
$$\sum_{n=1}^{\infty} 2^{-n} = 1$$	
\begin{definition}
	Si $X$ et $Y$ sont 2 v.a. ou definit la \textsc{Covariance} entre $X$ et $Y$ comme
	$\cov(X,Y)\overset{\text{def}}{=}\E\left[(X-\E(X))(Y-\E(Y))\right]=\E(XY)-\E(X)\E(Y)$.
\end{definition}
\fi
\pagebreak

% \tableofcontents

% insert your code here
%\input{./algebra/main.tex}
%\input{./geometrie-differentielle/main.tex}
%\input{./probabilite/main.tex}
%\input{./analyse-fonctionnelle/main.tex}
% \input{./Analyse-convexe-et-dualite-en-optimisation/main.tex}
%\input{./tikz/main.tex}
%\input{./Theorie-du-distributions/main.tex}
%\input{./optimisation/mine.tex}
 \input{./modelisation/main.tex}

% yves.aubry@univ-tln.fr : algebra

\end{document}

%% !TEX encoding = UTF-8 Unicode
% !TEX TS-program = xelatex

\documentclass[french]{report}

%\usepackage[utf8]{inputenc}
%\usepackage[T1]{fontenc}
\usepackage{babel}


\newif\ifcomment
%\commenttrue # Show comments

\usepackage{physics}
\usepackage{amssymb}


\usepackage{amsthm}
% \usepackage{thmtools}
\usepackage{mathtools}
\usepackage{amsfonts}

\usepackage{color}

\usepackage{tikz}

\usepackage{geometry}
\geometry{a5paper, margin=0.1in, right=1cm}

\usepackage{dsfont}

\usepackage{graphicx}
\graphicspath{ {images/} }

\usepackage{faktor}

\usepackage{IEEEtrantools}
\usepackage{enumerate}   
\usepackage[PostScript=dvips]{"/Users/aware/Documents/Courses/diagrams"}


\newtheorem{theorem}{Théorème}[section]
\renewcommand{\thetheorem}{\arabic{theorem}}
\newtheorem{lemme}{Lemme}[section]
\renewcommand{\thelemme}{\arabic{lemme}}
\newtheorem{proposition}{Proposition}[section]
\renewcommand{\theproposition}{\arabic{proposition}}
\newtheorem{notations}{Notations}[section]
\newtheorem{problem}{Problème}[section]
\newtheorem{corollary}{Corollaire}[theorem]
\renewcommand{\thecorollary}{\arabic{corollary}}
\newtheorem{property}{Propriété}[section]
\newtheorem{objective}{Objectif}[section]

\theoremstyle{definition}
\newtheorem{definition}{Définition}[section]
\renewcommand{\thedefinition}{\arabic{definition}}
\newtheorem{exercise}{Exercice}[chapter]
\renewcommand{\theexercise}{\arabic{exercise}}
\newtheorem{example}{Exemple}[chapter]
\renewcommand{\theexample}{\arabic{example}}
\newtheorem*{solution}{Solution}
\newtheorem*{application}{Application}
\newtheorem*{notation}{Notation}
\newtheorem*{vocabulary}{Vocabulaire}
\newtheorem*{properties}{Propriétés}



\theoremstyle{remark}
\newtheorem*{remark}{Remarque}
\newtheorem*{rappel}{Rappel}


\usepackage{etoolbox}
\AtBeginEnvironment{exercise}{\small}
\AtBeginEnvironment{example}{\small}

\usepackage{cases}
\usepackage[red]{mypack}

\usepackage[framemethod=TikZ]{mdframed}

\definecolor{bg}{rgb}{0.4,0.25,0.95}
\definecolor{pagebg}{rgb}{0,0,0.5}
\surroundwithmdframed[
   topline=false,
   rightline=false,
   bottomline=false,
   leftmargin=\parindent,
   skipabove=8pt,
   skipbelow=8pt,
   linecolor=blue,
   innerbottommargin=10pt,
   % backgroundcolor=bg,font=\color{orange}\sffamily, fontcolor=white
]{definition}

\usepackage{empheq}
\usepackage[most]{tcolorbox}

\newtcbox{\mymath}[1][]{%
    nobeforeafter, math upper, tcbox raise base,
    enhanced, colframe=blue!30!black,
    colback=red!10, boxrule=1pt,
    #1}

\usepackage{unixode}


\DeclareMathOperator{\ord}{ord}
\DeclareMathOperator{\orb}{orb}
\DeclareMathOperator{\stab}{stab}
\DeclareMathOperator{\Stab}{stab}
\DeclareMathOperator{\ppcm}{ppcm}
\DeclareMathOperator{\conj}{Conj}
\DeclareMathOperator{\End}{End}
\DeclareMathOperator{\rot}{rot}
\DeclareMathOperator{\trs}{trace}
\DeclareMathOperator{\Ind}{Ind}
\DeclareMathOperator{\mat}{Mat}
\DeclareMathOperator{\id}{Id}
\DeclareMathOperator{\vect}{vect}
\DeclareMathOperator{\img}{img}
\DeclareMathOperator{\cov}{Cov}
\DeclareMathOperator{\dist}{dist}
\DeclareMathOperator{\irr}{Irr}
\DeclareMathOperator{\image}{Im}
\DeclareMathOperator{\pd}{\partial}
\DeclareMathOperator{\epi}{epi}
\DeclareMathOperator{\Argmin}{Argmin}
\DeclareMathOperator{\dom}{dom}
\DeclareMathOperator{\proj}{proj}
\DeclareMathOperator{\ctg}{ctg}
\DeclareMathOperator{\supp}{supp}
\DeclareMathOperator{\argmin}{argmin}
\DeclareMathOperator{\mult}{mult}
\DeclareMathOperator{\ch}{ch}
\DeclareMathOperator{\sh}{sh}
\DeclareMathOperator{\rang}{rang}
\DeclareMathOperator{\diam}{diam}
\DeclareMathOperator{\Epigraphe}{Epigraphe}




\usepackage{xcolor}
\everymath{\color{blue}}
%\everymath{\color[rgb]{0,1,1}}
%\pagecolor[rgb]{0,0,0.5}


\newcommand*{\pdtest}[3][]{\ensuremath{\frac{\partial^{#1} #2}{\partial #3}}}

\newcommand*{\deffunc}[6][]{\ensuremath{
\begin{array}{rcl}
#2 : #3 &\rightarrow& #4\\
#5 &\mapsto& #6
\end{array}
}}

\newcommand{\eqcolon}{\mathrel{\resizebox{\widthof{$\mathord{=}$}}{\height}{ $\!\!=\!\!\resizebox{1.2\width}{0.8\height}{\raisebox{0.23ex}{$\mathop{:}$}}\!\!$ }}}
\newcommand{\coloneq}{\mathrel{\resizebox{\widthof{$\mathord{=}$}}{\height}{ $\!\!\resizebox{1.2\width}{0.8\height}{\raisebox{0.23ex}{$\mathop{:}$}}\!\!=\!\!$ }}}
\newcommand{\eqcolonl}{\ensuremath{\mathrel{=\!\!\mathop{:}}}}
\newcommand{\coloneql}{\ensuremath{\mathrel{\mathop{:} \!\! =}}}
\newcommand{\vc}[1]{% inline column vector
  \left(\begin{smallmatrix}#1\end{smallmatrix}\right)%
}
\newcommand{\vr}[1]{% inline row vector
  \begin{smallmatrix}(\,#1\,)\end{smallmatrix}%
}
\makeatletter
\newcommand*{\defeq}{\ =\mathrel{\rlap{%
                     \raisebox{0.3ex}{$\m@th\cdot$}}%
                     \raisebox{-0.3ex}{$\m@th\cdot$}}%
                     }
\makeatother

\newcommand{\mathcircle}[1]{% inline row vector
 \overset{\circ}{#1}
}
\newcommand{\ulim}{% low limit
 \underline{\lim}
}
\newcommand{\ssi}{% iff
\iff
}
\newcommand{\ps}[2]{
\expval{#1 | #2}
}
\newcommand{\df}[1]{
\mqty{#1}
}
\newcommand{\n}[1]{
\norm{#1}
}
\newcommand{\sys}[1]{
\left\{\smqty{#1}\right.
}


\newcommand{\eqdef}{\ensuremath{\overset{\text{def}}=}}


\def\Circlearrowright{\ensuremath{%
  \rotatebox[origin=c]{230}{$\circlearrowright$}}}

\newcommand\ct[1]{\text{\rmfamily\upshape #1}}
\newcommand\question[1]{ {\color{red} ...!? \small #1}}
\newcommand\caz[1]{\left\{\begin{array} #1 \end{array}\right.}
\newcommand\const{\text{\rmfamily\upshape const}}
\newcommand\toP{ \overset{\pro}{\to}}
\newcommand\toPP{ \overset{\text{PP}}{\to}}
\newcommand{\oeq}{\mathrel{\text{\textcircled{$=$}}}}





\usepackage{xcolor}
% \usepackage[normalem]{ulem}
\usepackage{lipsum}
\makeatletter
% \newcommand\colorwave[1][blue]{\bgroup \markoverwith{\lower3.5\p@\hbox{\sixly \textcolor{#1}{\char58}}}\ULon}
%\font\sixly=lasy6 % does not re-load if already loaded, so no memory problem.

\newmdtheoremenv[
linewidth= 1pt,linecolor= blue,%
leftmargin=20,rightmargin=20,innertopmargin=0pt, innerrightmargin=40,%
tikzsetting = { draw=lightgray, line width = 0.3pt,dashed,%
dash pattern = on 15pt off 3pt},%
splittopskip=\topskip,skipbelow=\baselineskip,%
skipabove=\baselineskip,ntheorem,roundcorner=0pt,
% backgroundcolor=pagebg,font=\color{orange}\sffamily, fontcolor=white
]{examplebox}{Exemple}[section]



\newcommand\R{\mathbb{R}}
\newcommand\Z{\mathbb{Z}}
\newcommand\N{\mathbb{N}}
\newcommand\E{\mathbb{E}}
\newcommand\F{\mathcal{F}}
\newcommand\cH{\mathcal{H}}
\newcommand\V{\mathbb{V}}
\newcommand\dmo{ ^{-1} }
\newcommand\kapa{\kappa}
\newcommand\im{Im}
\newcommand\hs{\mathcal{H}}





\usepackage{soul}

\makeatletter
\newcommand*{\whiten}[1]{\llap{\textcolor{white}{{\the\SOUL@token}}\hspace{#1pt}}}
\DeclareRobustCommand*\myul{%
    \def\SOUL@everyspace{\underline{\space}\kern\z@}%
    \def\SOUL@everytoken{%
     \setbox0=\hbox{\the\SOUL@token}%
     \ifdim\dp0>\z@
        \raisebox{\dp0}{\underline{\phantom{\the\SOUL@token}}}%
        \whiten{1}\whiten{0}%
        \whiten{-1}\whiten{-2}%
        \llap{\the\SOUL@token}%
     \else
        \underline{\the\SOUL@token}%
     \fi}%
\SOUL@}
\makeatother

\newcommand*{\demp}{\fontfamily{lmtt}\selectfont}

\DeclareTextFontCommand{\textdemp}{\demp}

\begin{document}

\ifcomment
Multiline
comment
\fi
\ifcomment
\myul{Typesetting test}
% \color[rgb]{1,1,1}
$∑_i^n≠ 60º±∞π∆¬≈√j∫h≤≥µ$

$\CR \R\pro\ind\pro\gS\pro
\mqty[a&b\\c&d]$
$\pro\mathbb{P}$
$\dd{x}$

  \[
    \alpha(x)=\left\{
                \begin{array}{ll}
                  x\\
                  \frac{1}{1+e^{-kx}}\\
                  \frac{e^x-e^{-x}}{e^x+e^{-x}}
                \end{array}
              \right.
  \]

  $\expval{x}$
  
  $\chi_\rho(ghg\dmo)=\Tr(\rho_{ghg\dmo})=\Tr(\rho_g\circ\rho_h\circ\rho\dmo_g)=\Tr(\rho_h)\overset{\mbox{\scalebox{0.5}{$\Tr(AB)=\Tr(BA)$}}}{=}\chi_\rho(h)$
  	$\mathop{\oplus}_{\substack{x\in X}}$

$\mat(\rho_g)=(a_{ij}(g))_{\scriptsize \substack{1\leq i\leq d \\ 1\leq j\leq d}}$ et $\mat(\rho'_g)=(a'_{ij}(g))_{\scriptsize \substack{1\leq i'\leq d' \\ 1\leq j'\leq d'}}$



\[\int_a^b{\mathbb{R}^2}g(u, v)\dd{P_{XY}}(u, v)=\iint g(u,v) f_{XY}(u, v)\dd \lambda(u) \dd \lambda(v)\]
$$\lim_{x\to\infty} f(x)$$	
$$\iiiint_V \mu(t,u,v,w) \,dt\,du\,dv\,dw$$
$$\sum_{n=1}^{\infty} 2^{-n} = 1$$	
\begin{definition}
	Si $X$ et $Y$ sont 2 v.a. ou definit la \textsc{Covariance} entre $X$ et $Y$ comme
	$\cov(X,Y)\overset{\text{def}}{=}\E\left[(X-\E(X))(Y-\E(Y))\right]=\E(XY)-\E(X)\E(Y)$.
\end{definition}
\fi
\pagebreak

% \tableofcontents

% insert your code here
%\input{./algebra/main.tex}
%\input{./geometrie-differentielle/main.tex}
%\input{./probabilite/main.tex}
%\input{./analyse-fonctionnelle/main.tex}
% \input{./Analyse-convexe-et-dualite-en-optimisation/main.tex}
%\input{./tikz/main.tex}
%\input{./Theorie-du-distributions/main.tex}
%\input{./optimisation/mine.tex}
 \input{./modelisation/main.tex}

% yves.aubry@univ-tln.fr : algebra

\end{document}

% % !TEX encoding = UTF-8 Unicode
% !TEX TS-program = xelatex

\documentclass[french]{report}

%\usepackage[utf8]{inputenc}
%\usepackage[T1]{fontenc}
\usepackage{babel}


\newif\ifcomment
%\commenttrue # Show comments

\usepackage{physics}
\usepackage{amssymb}


\usepackage{amsthm}
% \usepackage{thmtools}
\usepackage{mathtools}
\usepackage{amsfonts}

\usepackage{color}

\usepackage{tikz}

\usepackage{geometry}
\geometry{a5paper, margin=0.1in, right=1cm}

\usepackage{dsfont}

\usepackage{graphicx}
\graphicspath{ {images/} }

\usepackage{faktor}

\usepackage{IEEEtrantools}
\usepackage{enumerate}   
\usepackage[PostScript=dvips]{"/Users/aware/Documents/Courses/diagrams"}


\newtheorem{theorem}{Théorème}[section]
\renewcommand{\thetheorem}{\arabic{theorem}}
\newtheorem{lemme}{Lemme}[section]
\renewcommand{\thelemme}{\arabic{lemme}}
\newtheorem{proposition}{Proposition}[section]
\renewcommand{\theproposition}{\arabic{proposition}}
\newtheorem{notations}{Notations}[section]
\newtheorem{problem}{Problème}[section]
\newtheorem{corollary}{Corollaire}[theorem]
\renewcommand{\thecorollary}{\arabic{corollary}}
\newtheorem{property}{Propriété}[section]
\newtheorem{objective}{Objectif}[section]

\theoremstyle{definition}
\newtheorem{definition}{Définition}[section]
\renewcommand{\thedefinition}{\arabic{definition}}
\newtheorem{exercise}{Exercice}[chapter]
\renewcommand{\theexercise}{\arabic{exercise}}
\newtheorem{example}{Exemple}[chapter]
\renewcommand{\theexample}{\arabic{example}}
\newtheorem*{solution}{Solution}
\newtheorem*{application}{Application}
\newtheorem*{notation}{Notation}
\newtheorem*{vocabulary}{Vocabulaire}
\newtheorem*{properties}{Propriétés}



\theoremstyle{remark}
\newtheorem*{remark}{Remarque}
\newtheorem*{rappel}{Rappel}


\usepackage{etoolbox}
\AtBeginEnvironment{exercise}{\small}
\AtBeginEnvironment{example}{\small}

\usepackage{cases}
\usepackage[red]{mypack}

\usepackage[framemethod=TikZ]{mdframed}

\definecolor{bg}{rgb}{0.4,0.25,0.95}
\definecolor{pagebg}{rgb}{0,0,0.5}
\surroundwithmdframed[
   topline=false,
   rightline=false,
   bottomline=false,
   leftmargin=\parindent,
   skipabove=8pt,
   skipbelow=8pt,
   linecolor=blue,
   innerbottommargin=10pt,
   % backgroundcolor=bg,font=\color{orange}\sffamily, fontcolor=white
]{definition}

\usepackage{empheq}
\usepackage[most]{tcolorbox}

\newtcbox{\mymath}[1][]{%
    nobeforeafter, math upper, tcbox raise base,
    enhanced, colframe=blue!30!black,
    colback=red!10, boxrule=1pt,
    #1}

\usepackage{unixode}


\DeclareMathOperator{\ord}{ord}
\DeclareMathOperator{\orb}{orb}
\DeclareMathOperator{\stab}{stab}
\DeclareMathOperator{\Stab}{stab}
\DeclareMathOperator{\ppcm}{ppcm}
\DeclareMathOperator{\conj}{Conj}
\DeclareMathOperator{\End}{End}
\DeclareMathOperator{\rot}{rot}
\DeclareMathOperator{\trs}{trace}
\DeclareMathOperator{\Ind}{Ind}
\DeclareMathOperator{\mat}{Mat}
\DeclareMathOperator{\id}{Id}
\DeclareMathOperator{\vect}{vect}
\DeclareMathOperator{\img}{img}
\DeclareMathOperator{\cov}{Cov}
\DeclareMathOperator{\dist}{dist}
\DeclareMathOperator{\irr}{Irr}
\DeclareMathOperator{\image}{Im}
\DeclareMathOperator{\pd}{\partial}
\DeclareMathOperator{\epi}{epi}
\DeclareMathOperator{\Argmin}{Argmin}
\DeclareMathOperator{\dom}{dom}
\DeclareMathOperator{\proj}{proj}
\DeclareMathOperator{\ctg}{ctg}
\DeclareMathOperator{\supp}{supp}
\DeclareMathOperator{\argmin}{argmin}
\DeclareMathOperator{\mult}{mult}
\DeclareMathOperator{\ch}{ch}
\DeclareMathOperator{\sh}{sh}
\DeclareMathOperator{\rang}{rang}
\DeclareMathOperator{\diam}{diam}
\DeclareMathOperator{\Epigraphe}{Epigraphe}




\usepackage{xcolor}
\everymath{\color{blue}}
%\everymath{\color[rgb]{0,1,1}}
%\pagecolor[rgb]{0,0,0.5}


\newcommand*{\pdtest}[3][]{\ensuremath{\frac{\partial^{#1} #2}{\partial #3}}}

\newcommand*{\deffunc}[6][]{\ensuremath{
\begin{array}{rcl}
#2 : #3 &\rightarrow& #4\\
#5 &\mapsto& #6
\end{array}
}}

\newcommand{\eqcolon}{\mathrel{\resizebox{\widthof{$\mathord{=}$}}{\height}{ $\!\!=\!\!\resizebox{1.2\width}{0.8\height}{\raisebox{0.23ex}{$\mathop{:}$}}\!\!$ }}}
\newcommand{\coloneq}{\mathrel{\resizebox{\widthof{$\mathord{=}$}}{\height}{ $\!\!\resizebox{1.2\width}{0.8\height}{\raisebox{0.23ex}{$\mathop{:}$}}\!\!=\!\!$ }}}
\newcommand{\eqcolonl}{\ensuremath{\mathrel{=\!\!\mathop{:}}}}
\newcommand{\coloneql}{\ensuremath{\mathrel{\mathop{:} \!\! =}}}
\newcommand{\vc}[1]{% inline column vector
  \left(\begin{smallmatrix}#1\end{smallmatrix}\right)%
}
\newcommand{\vr}[1]{% inline row vector
  \begin{smallmatrix}(\,#1\,)\end{smallmatrix}%
}
\makeatletter
\newcommand*{\defeq}{\ =\mathrel{\rlap{%
                     \raisebox{0.3ex}{$\m@th\cdot$}}%
                     \raisebox{-0.3ex}{$\m@th\cdot$}}%
                     }
\makeatother

\newcommand{\mathcircle}[1]{% inline row vector
 \overset{\circ}{#1}
}
\newcommand{\ulim}{% low limit
 \underline{\lim}
}
\newcommand{\ssi}{% iff
\iff
}
\newcommand{\ps}[2]{
\expval{#1 | #2}
}
\newcommand{\df}[1]{
\mqty{#1}
}
\newcommand{\n}[1]{
\norm{#1}
}
\newcommand{\sys}[1]{
\left\{\smqty{#1}\right.
}


\newcommand{\eqdef}{\ensuremath{\overset{\text{def}}=}}


\def\Circlearrowright{\ensuremath{%
  \rotatebox[origin=c]{230}{$\circlearrowright$}}}

\newcommand\ct[1]{\text{\rmfamily\upshape #1}}
\newcommand\question[1]{ {\color{red} ...!? \small #1}}
\newcommand\caz[1]{\left\{\begin{array} #1 \end{array}\right.}
\newcommand\const{\text{\rmfamily\upshape const}}
\newcommand\toP{ \overset{\pro}{\to}}
\newcommand\toPP{ \overset{\text{PP}}{\to}}
\newcommand{\oeq}{\mathrel{\text{\textcircled{$=$}}}}





\usepackage{xcolor}
% \usepackage[normalem]{ulem}
\usepackage{lipsum}
\makeatletter
% \newcommand\colorwave[1][blue]{\bgroup \markoverwith{\lower3.5\p@\hbox{\sixly \textcolor{#1}{\char58}}}\ULon}
%\font\sixly=lasy6 % does not re-load if already loaded, so no memory problem.

\newmdtheoremenv[
linewidth= 1pt,linecolor= blue,%
leftmargin=20,rightmargin=20,innertopmargin=0pt, innerrightmargin=40,%
tikzsetting = { draw=lightgray, line width = 0.3pt,dashed,%
dash pattern = on 15pt off 3pt},%
splittopskip=\topskip,skipbelow=\baselineskip,%
skipabove=\baselineskip,ntheorem,roundcorner=0pt,
% backgroundcolor=pagebg,font=\color{orange}\sffamily, fontcolor=white
]{examplebox}{Exemple}[section]



\newcommand\R{\mathbb{R}}
\newcommand\Z{\mathbb{Z}}
\newcommand\N{\mathbb{N}}
\newcommand\E{\mathbb{E}}
\newcommand\F{\mathcal{F}}
\newcommand\cH{\mathcal{H}}
\newcommand\V{\mathbb{V}}
\newcommand\dmo{ ^{-1} }
\newcommand\kapa{\kappa}
\newcommand\im{Im}
\newcommand\hs{\mathcal{H}}





\usepackage{soul}

\makeatletter
\newcommand*{\whiten}[1]{\llap{\textcolor{white}{{\the\SOUL@token}}\hspace{#1pt}}}
\DeclareRobustCommand*\myul{%
    \def\SOUL@everyspace{\underline{\space}\kern\z@}%
    \def\SOUL@everytoken{%
     \setbox0=\hbox{\the\SOUL@token}%
     \ifdim\dp0>\z@
        \raisebox{\dp0}{\underline{\phantom{\the\SOUL@token}}}%
        \whiten{1}\whiten{0}%
        \whiten{-1}\whiten{-2}%
        \llap{\the\SOUL@token}%
     \else
        \underline{\the\SOUL@token}%
     \fi}%
\SOUL@}
\makeatother

\newcommand*{\demp}{\fontfamily{lmtt}\selectfont}

\DeclareTextFontCommand{\textdemp}{\demp}

\begin{document}

\ifcomment
Multiline
comment
\fi
\ifcomment
\myul{Typesetting test}
% \color[rgb]{1,1,1}
$∑_i^n≠ 60º±∞π∆¬≈√j∫h≤≥µ$

$\CR \R\pro\ind\pro\gS\pro
\mqty[a&b\\c&d]$
$\pro\mathbb{P}$
$\dd{x}$

  \[
    \alpha(x)=\left\{
                \begin{array}{ll}
                  x\\
                  \frac{1}{1+e^{-kx}}\\
                  \frac{e^x-e^{-x}}{e^x+e^{-x}}
                \end{array}
              \right.
  \]

  $\expval{x}$
  
  $\chi_\rho(ghg\dmo)=\Tr(\rho_{ghg\dmo})=\Tr(\rho_g\circ\rho_h\circ\rho\dmo_g)=\Tr(\rho_h)\overset{\mbox{\scalebox{0.5}{$\Tr(AB)=\Tr(BA)$}}}{=}\chi_\rho(h)$
  	$\mathop{\oplus}_{\substack{x\in X}}$

$\mat(\rho_g)=(a_{ij}(g))_{\scriptsize \substack{1\leq i\leq d \\ 1\leq j\leq d}}$ et $\mat(\rho'_g)=(a'_{ij}(g))_{\scriptsize \substack{1\leq i'\leq d' \\ 1\leq j'\leq d'}}$



\[\int_a^b{\mathbb{R}^2}g(u, v)\dd{P_{XY}}(u, v)=\iint g(u,v) f_{XY}(u, v)\dd \lambda(u) \dd \lambda(v)\]
$$\lim_{x\to\infty} f(x)$$	
$$\iiiint_V \mu(t,u,v,w) \,dt\,du\,dv\,dw$$
$$\sum_{n=1}^{\infty} 2^{-n} = 1$$	
\begin{definition}
	Si $X$ et $Y$ sont 2 v.a. ou definit la \textsc{Covariance} entre $X$ et $Y$ comme
	$\cov(X,Y)\overset{\text{def}}{=}\E\left[(X-\E(X))(Y-\E(Y))\right]=\E(XY)-\E(X)\E(Y)$.
\end{definition}
\fi
\pagebreak

% \tableofcontents

% insert your code here
%\input{./algebra/main.tex}
%\input{./geometrie-differentielle/main.tex}
%\input{./probabilite/main.tex}
%\input{./analyse-fonctionnelle/main.tex}
% \input{./Analyse-convexe-et-dualite-en-optimisation/main.tex}
%\input{./tikz/main.tex}
%\input{./Theorie-du-distributions/main.tex}
%\input{./optimisation/mine.tex}
 \input{./modelisation/main.tex}

% yves.aubry@univ-tln.fr : algebra

\end{document}

%% !TEX encoding = UTF-8 Unicode
% !TEX TS-program = xelatex

\documentclass[french]{report}

%\usepackage[utf8]{inputenc}
%\usepackage[T1]{fontenc}
\usepackage{babel}


\newif\ifcomment
%\commenttrue # Show comments

\usepackage{physics}
\usepackage{amssymb}


\usepackage{amsthm}
% \usepackage{thmtools}
\usepackage{mathtools}
\usepackage{amsfonts}

\usepackage{color}

\usepackage{tikz}

\usepackage{geometry}
\geometry{a5paper, margin=0.1in, right=1cm}

\usepackage{dsfont}

\usepackage{graphicx}
\graphicspath{ {images/} }

\usepackage{faktor}

\usepackage{IEEEtrantools}
\usepackage{enumerate}   
\usepackage[PostScript=dvips]{"/Users/aware/Documents/Courses/diagrams"}


\newtheorem{theorem}{Théorème}[section]
\renewcommand{\thetheorem}{\arabic{theorem}}
\newtheorem{lemme}{Lemme}[section]
\renewcommand{\thelemme}{\arabic{lemme}}
\newtheorem{proposition}{Proposition}[section]
\renewcommand{\theproposition}{\arabic{proposition}}
\newtheorem{notations}{Notations}[section]
\newtheorem{problem}{Problème}[section]
\newtheorem{corollary}{Corollaire}[theorem]
\renewcommand{\thecorollary}{\arabic{corollary}}
\newtheorem{property}{Propriété}[section]
\newtheorem{objective}{Objectif}[section]

\theoremstyle{definition}
\newtheorem{definition}{Définition}[section]
\renewcommand{\thedefinition}{\arabic{definition}}
\newtheorem{exercise}{Exercice}[chapter]
\renewcommand{\theexercise}{\arabic{exercise}}
\newtheorem{example}{Exemple}[chapter]
\renewcommand{\theexample}{\arabic{example}}
\newtheorem*{solution}{Solution}
\newtheorem*{application}{Application}
\newtheorem*{notation}{Notation}
\newtheorem*{vocabulary}{Vocabulaire}
\newtheorem*{properties}{Propriétés}



\theoremstyle{remark}
\newtheorem*{remark}{Remarque}
\newtheorem*{rappel}{Rappel}


\usepackage{etoolbox}
\AtBeginEnvironment{exercise}{\small}
\AtBeginEnvironment{example}{\small}

\usepackage{cases}
\usepackage[red]{mypack}

\usepackage[framemethod=TikZ]{mdframed}

\definecolor{bg}{rgb}{0.4,0.25,0.95}
\definecolor{pagebg}{rgb}{0,0,0.5}
\surroundwithmdframed[
   topline=false,
   rightline=false,
   bottomline=false,
   leftmargin=\parindent,
   skipabove=8pt,
   skipbelow=8pt,
   linecolor=blue,
   innerbottommargin=10pt,
   % backgroundcolor=bg,font=\color{orange}\sffamily, fontcolor=white
]{definition}

\usepackage{empheq}
\usepackage[most]{tcolorbox}

\newtcbox{\mymath}[1][]{%
    nobeforeafter, math upper, tcbox raise base,
    enhanced, colframe=blue!30!black,
    colback=red!10, boxrule=1pt,
    #1}

\usepackage{unixode}


\DeclareMathOperator{\ord}{ord}
\DeclareMathOperator{\orb}{orb}
\DeclareMathOperator{\stab}{stab}
\DeclareMathOperator{\Stab}{stab}
\DeclareMathOperator{\ppcm}{ppcm}
\DeclareMathOperator{\conj}{Conj}
\DeclareMathOperator{\End}{End}
\DeclareMathOperator{\rot}{rot}
\DeclareMathOperator{\trs}{trace}
\DeclareMathOperator{\Ind}{Ind}
\DeclareMathOperator{\mat}{Mat}
\DeclareMathOperator{\id}{Id}
\DeclareMathOperator{\vect}{vect}
\DeclareMathOperator{\img}{img}
\DeclareMathOperator{\cov}{Cov}
\DeclareMathOperator{\dist}{dist}
\DeclareMathOperator{\irr}{Irr}
\DeclareMathOperator{\image}{Im}
\DeclareMathOperator{\pd}{\partial}
\DeclareMathOperator{\epi}{epi}
\DeclareMathOperator{\Argmin}{Argmin}
\DeclareMathOperator{\dom}{dom}
\DeclareMathOperator{\proj}{proj}
\DeclareMathOperator{\ctg}{ctg}
\DeclareMathOperator{\supp}{supp}
\DeclareMathOperator{\argmin}{argmin}
\DeclareMathOperator{\mult}{mult}
\DeclareMathOperator{\ch}{ch}
\DeclareMathOperator{\sh}{sh}
\DeclareMathOperator{\rang}{rang}
\DeclareMathOperator{\diam}{diam}
\DeclareMathOperator{\Epigraphe}{Epigraphe}




\usepackage{xcolor}
\everymath{\color{blue}}
%\everymath{\color[rgb]{0,1,1}}
%\pagecolor[rgb]{0,0,0.5}


\newcommand*{\pdtest}[3][]{\ensuremath{\frac{\partial^{#1} #2}{\partial #3}}}

\newcommand*{\deffunc}[6][]{\ensuremath{
\begin{array}{rcl}
#2 : #3 &\rightarrow& #4\\
#5 &\mapsto& #6
\end{array}
}}

\newcommand{\eqcolon}{\mathrel{\resizebox{\widthof{$\mathord{=}$}}{\height}{ $\!\!=\!\!\resizebox{1.2\width}{0.8\height}{\raisebox{0.23ex}{$\mathop{:}$}}\!\!$ }}}
\newcommand{\coloneq}{\mathrel{\resizebox{\widthof{$\mathord{=}$}}{\height}{ $\!\!\resizebox{1.2\width}{0.8\height}{\raisebox{0.23ex}{$\mathop{:}$}}\!\!=\!\!$ }}}
\newcommand{\eqcolonl}{\ensuremath{\mathrel{=\!\!\mathop{:}}}}
\newcommand{\coloneql}{\ensuremath{\mathrel{\mathop{:} \!\! =}}}
\newcommand{\vc}[1]{% inline column vector
  \left(\begin{smallmatrix}#1\end{smallmatrix}\right)%
}
\newcommand{\vr}[1]{% inline row vector
  \begin{smallmatrix}(\,#1\,)\end{smallmatrix}%
}
\makeatletter
\newcommand*{\defeq}{\ =\mathrel{\rlap{%
                     \raisebox{0.3ex}{$\m@th\cdot$}}%
                     \raisebox{-0.3ex}{$\m@th\cdot$}}%
                     }
\makeatother

\newcommand{\mathcircle}[1]{% inline row vector
 \overset{\circ}{#1}
}
\newcommand{\ulim}{% low limit
 \underline{\lim}
}
\newcommand{\ssi}{% iff
\iff
}
\newcommand{\ps}[2]{
\expval{#1 | #2}
}
\newcommand{\df}[1]{
\mqty{#1}
}
\newcommand{\n}[1]{
\norm{#1}
}
\newcommand{\sys}[1]{
\left\{\smqty{#1}\right.
}


\newcommand{\eqdef}{\ensuremath{\overset{\text{def}}=}}


\def\Circlearrowright{\ensuremath{%
  \rotatebox[origin=c]{230}{$\circlearrowright$}}}

\newcommand\ct[1]{\text{\rmfamily\upshape #1}}
\newcommand\question[1]{ {\color{red} ...!? \small #1}}
\newcommand\caz[1]{\left\{\begin{array} #1 \end{array}\right.}
\newcommand\const{\text{\rmfamily\upshape const}}
\newcommand\toP{ \overset{\pro}{\to}}
\newcommand\toPP{ \overset{\text{PP}}{\to}}
\newcommand{\oeq}{\mathrel{\text{\textcircled{$=$}}}}





\usepackage{xcolor}
% \usepackage[normalem]{ulem}
\usepackage{lipsum}
\makeatletter
% \newcommand\colorwave[1][blue]{\bgroup \markoverwith{\lower3.5\p@\hbox{\sixly \textcolor{#1}{\char58}}}\ULon}
%\font\sixly=lasy6 % does not re-load if already loaded, so no memory problem.

\newmdtheoremenv[
linewidth= 1pt,linecolor= blue,%
leftmargin=20,rightmargin=20,innertopmargin=0pt, innerrightmargin=40,%
tikzsetting = { draw=lightgray, line width = 0.3pt,dashed,%
dash pattern = on 15pt off 3pt},%
splittopskip=\topskip,skipbelow=\baselineskip,%
skipabove=\baselineskip,ntheorem,roundcorner=0pt,
% backgroundcolor=pagebg,font=\color{orange}\sffamily, fontcolor=white
]{examplebox}{Exemple}[section]



\newcommand\R{\mathbb{R}}
\newcommand\Z{\mathbb{Z}}
\newcommand\N{\mathbb{N}}
\newcommand\E{\mathbb{E}}
\newcommand\F{\mathcal{F}}
\newcommand\cH{\mathcal{H}}
\newcommand\V{\mathbb{V}}
\newcommand\dmo{ ^{-1} }
\newcommand\kapa{\kappa}
\newcommand\im{Im}
\newcommand\hs{\mathcal{H}}





\usepackage{soul}

\makeatletter
\newcommand*{\whiten}[1]{\llap{\textcolor{white}{{\the\SOUL@token}}\hspace{#1pt}}}
\DeclareRobustCommand*\myul{%
    \def\SOUL@everyspace{\underline{\space}\kern\z@}%
    \def\SOUL@everytoken{%
     \setbox0=\hbox{\the\SOUL@token}%
     \ifdim\dp0>\z@
        \raisebox{\dp0}{\underline{\phantom{\the\SOUL@token}}}%
        \whiten{1}\whiten{0}%
        \whiten{-1}\whiten{-2}%
        \llap{\the\SOUL@token}%
     \else
        \underline{\the\SOUL@token}%
     \fi}%
\SOUL@}
\makeatother

\newcommand*{\demp}{\fontfamily{lmtt}\selectfont}

\DeclareTextFontCommand{\textdemp}{\demp}

\begin{document}

\ifcomment
Multiline
comment
\fi
\ifcomment
\myul{Typesetting test}
% \color[rgb]{1,1,1}
$∑_i^n≠ 60º±∞π∆¬≈√j∫h≤≥µ$

$\CR \R\pro\ind\pro\gS\pro
\mqty[a&b\\c&d]$
$\pro\mathbb{P}$
$\dd{x}$

  \[
    \alpha(x)=\left\{
                \begin{array}{ll}
                  x\\
                  \frac{1}{1+e^{-kx}}\\
                  \frac{e^x-e^{-x}}{e^x+e^{-x}}
                \end{array}
              \right.
  \]

  $\expval{x}$
  
  $\chi_\rho(ghg\dmo)=\Tr(\rho_{ghg\dmo})=\Tr(\rho_g\circ\rho_h\circ\rho\dmo_g)=\Tr(\rho_h)\overset{\mbox{\scalebox{0.5}{$\Tr(AB)=\Tr(BA)$}}}{=}\chi_\rho(h)$
  	$\mathop{\oplus}_{\substack{x\in X}}$

$\mat(\rho_g)=(a_{ij}(g))_{\scriptsize \substack{1\leq i\leq d \\ 1\leq j\leq d}}$ et $\mat(\rho'_g)=(a'_{ij}(g))_{\scriptsize \substack{1\leq i'\leq d' \\ 1\leq j'\leq d'}}$



\[\int_a^b{\mathbb{R}^2}g(u, v)\dd{P_{XY}}(u, v)=\iint g(u,v) f_{XY}(u, v)\dd \lambda(u) \dd \lambda(v)\]
$$\lim_{x\to\infty} f(x)$$	
$$\iiiint_V \mu(t,u,v,w) \,dt\,du\,dv\,dw$$
$$\sum_{n=1}^{\infty} 2^{-n} = 1$$	
\begin{definition}
	Si $X$ et $Y$ sont 2 v.a. ou definit la \textsc{Covariance} entre $X$ et $Y$ comme
	$\cov(X,Y)\overset{\text{def}}{=}\E\left[(X-\E(X))(Y-\E(Y))\right]=\E(XY)-\E(X)\E(Y)$.
\end{definition}
\fi
\pagebreak

% \tableofcontents

% insert your code here
%\input{./algebra/main.tex}
%\input{./geometrie-differentielle/main.tex}
%\input{./probabilite/main.tex}
%\input{./analyse-fonctionnelle/main.tex}
% \input{./Analyse-convexe-et-dualite-en-optimisation/main.tex}
%\input{./tikz/main.tex}
%\input{./Theorie-du-distributions/main.tex}
%\input{./optimisation/mine.tex}
 \input{./modelisation/main.tex}

% yves.aubry@univ-tln.fr : algebra

\end{document}

%% !TEX encoding = UTF-8 Unicode
% !TEX TS-program = xelatex

\documentclass[french]{report}

%\usepackage[utf8]{inputenc}
%\usepackage[T1]{fontenc}
\usepackage{babel}


\newif\ifcomment
%\commenttrue # Show comments

\usepackage{physics}
\usepackage{amssymb}


\usepackage{amsthm}
% \usepackage{thmtools}
\usepackage{mathtools}
\usepackage{amsfonts}

\usepackage{color}

\usepackage{tikz}

\usepackage{geometry}
\geometry{a5paper, margin=0.1in, right=1cm}

\usepackage{dsfont}

\usepackage{graphicx}
\graphicspath{ {images/} }

\usepackage{faktor}

\usepackage{IEEEtrantools}
\usepackage{enumerate}   
\usepackage[PostScript=dvips]{"/Users/aware/Documents/Courses/diagrams"}


\newtheorem{theorem}{Théorème}[section]
\renewcommand{\thetheorem}{\arabic{theorem}}
\newtheorem{lemme}{Lemme}[section]
\renewcommand{\thelemme}{\arabic{lemme}}
\newtheorem{proposition}{Proposition}[section]
\renewcommand{\theproposition}{\arabic{proposition}}
\newtheorem{notations}{Notations}[section]
\newtheorem{problem}{Problème}[section]
\newtheorem{corollary}{Corollaire}[theorem]
\renewcommand{\thecorollary}{\arabic{corollary}}
\newtheorem{property}{Propriété}[section]
\newtheorem{objective}{Objectif}[section]

\theoremstyle{definition}
\newtheorem{definition}{Définition}[section]
\renewcommand{\thedefinition}{\arabic{definition}}
\newtheorem{exercise}{Exercice}[chapter]
\renewcommand{\theexercise}{\arabic{exercise}}
\newtheorem{example}{Exemple}[chapter]
\renewcommand{\theexample}{\arabic{example}}
\newtheorem*{solution}{Solution}
\newtheorem*{application}{Application}
\newtheorem*{notation}{Notation}
\newtheorem*{vocabulary}{Vocabulaire}
\newtheorem*{properties}{Propriétés}



\theoremstyle{remark}
\newtheorem*{remark}{Remarque}
\newtheorem*{rappel}{Rappel}


\usepackage{etoolbox}
\AtBeginEnvironment{exercise}{\small}
\AtBeginEnvironment{example}{\small}

\usepackage{cases}
\usepackage[red]{mypack}

\usepackage[framemethod=TikZ]{mdframed}

\definecolor{bg}{rgb}{0.4,0.25,0.95}
\definecolor{pagebg}{rgb}{0,0,0.5}
\surroundwithmdframed[
   topline=false,
   rightline=false,
   bottomline=false,
   leftmargin=\parindent,
   skipabove=8pt,
   skipbelow=8pt,
   linecolor=blue,
   innerbottommargin=10pt,
   % backgroundcolor=bg,font=\color{orange}\sffamily, fontcolor=white
]{definition}

\usepackage{empheq}
\usepackage[most]{tcolorbox}

\newtcbox{\mymath}[1][]{%
    nobeforeafter, math upper, tcbox raise base,
    enhanced, colframe=blue!30!black,
    colback=red!10, boxrule=1pt,
    #1}

\usepackage{unixode}


\DeclareMathOperator{\ord}{ord}
\DeclareMathOperator{\orb}{orb}
\DeclareMathOperator{\stab}{stab}
\DeclareMathOperator{\Stab}{stab}
\DeclareMathOperator{\ppcm}{ppcm}
\DeclareMathOperator{\conj}{Conj}
\DeclareMathOperator{\End}{End}
\DeclareMathOperator{\rot}{rot}
\DeclareMathOperator{\trs}{trace}
\DeclareMathOperator{\Ind}{Ind}
\DeclareMathOperator{\mat}{Mat}
\DeclareMathOperator{\id}{Id}
\DeclareMathOperator{\vect}{vect}
\DeclareMathOperator{\img}{img}
\DeclareMathOperator{\cov}{Cov}
\DeclareMathOperator{\dist}{dist}
\DeclareMathOperator{\irr}{Irr}
\DeclareMathOperator{\image}{Im}
\DeclareMathOperator{\pd}{\partial}
\DeclareMathOperator{\epi}{epi}
\DeclareMathOperator{\Argmin}{Argmin}
\DeclareMathOperator{\dom}{dom}
\DeclareMathOperator{\proj}{proj}
\DeclareMathOperator{\ctg}{ctg}
\DeclareMathOperator{\supp}{supp}
\DeclareMathOperator{\argmin}{argmin}
\DeclareMathOperator{\mult}{mult}
\DeclareMathOperator{\ch}{ch}
\DeclareMathOperator{\sh}{sh}
\DeclareMathOperator{\rang}{rang}
\DeclareMathOperator{\diam}{diam}
\DeclareMathOperator{\Epigraphe}{Epigraphe}




\usepackage{xcolor}
\everymath{\color{blue}}
%\everymath{\color[rgb]{0,1,1}}
%\pagecolor[rgb]{0,0,0.5}


\newcommand*{\pdtest}[3][]{\ensuremath{\frac{\partial^{#1} #2}{\partial #3}}}

\newcommand*{\deffunc}[6][]{\ensuremath{
\begin{array}{rcl}
#2 : #3 &\rightarrow& #4\\
#5 &\mapsto& #6
\end{array}
}}

\newcommand{\eqcolon}{\mathrel{\resizebox{\widthof{$\mathord{=}$}}{\height}{ $\!\!=\!\!\resizebox{1.2\width}{0.8\height}{\raisebox{0.23ex}{$\mathop{:}$}}\!\!$ }}}
\newcommand{\coloneq}{\mathrel{\resizebox{\widthof{$\mathord{=}$}}{\height}{ $\!\!\resizebox{1.2\width}{0.8\height}{\raisebox{0.23ex}{$\mathop{:}$}}\!\!=\!\!$ }}}
\newcommand{\eqcolonl}{\ensuremath{\mathrel{=\!\!\mathop{:}}}}
\newcommand{\coloneql}{\ensuremath{\mathrel{\mathop{:} \!\! =}}}
\newcommand{\vc}[1]{% inline column vector
  \left(\begin{smallmatrix}#1\end{smallmatrix}\right)%
}
\newcommand{\vr}[1]{% inline row vector
  \begin{smallmatrix}(\,#1\,)\end{smallmatrix}%
}
\makeatletter
\newcommand*{\defeq}{\ =\mathrel{\rlap{%
                     \raisebox{0.3ex}{$\m@th\cdot$}}%
                     \raisebox{-0.3ex}{$\m@th\cdot$}}%
                     }
\makeatother

\newcommand{\mathcircle}[1]{% inline row vector
 \overset{\circ}{#1}
}
\newcommand{\ulim}{% low limit
 \underline{\lim}
}
\newcommand{\ssi}{% iff
\iff
}
\newcommand{\ps}[2]{
\expval{#1 | #2}
}
\newcommand{\df}[1]{
\mqty{#1}
}
\newcommand{\n}[1]{
\norm{#1}
}
\newcommand{\sys}[1]{
\left\{\smqty{#1}\right.
}


\newcommand{\eqdef}{\ensuremath{\overset{\text{def}}=}}


\def\Circlearrowright{\ensuremath{%
  \rotatebox[origin=c]{230}{$\circlearrowright$}}}

\newcommand\ct[1]{\text{\rmfamily\upshape #1}}
\newcommand\question[1]{ {\color{red} ...!? \small #1}}
\newcommand\caz[1]{\left\{\begin{array} #1 \end{array}\right.}
\newcommand\const{\text{\rmfamily\upshape const}}
\newcommand\toP{ \overset{\pro}{\to}}
\newcommand\toPP{ \overset{\text{PP}}{\to}}
\newcommand{\oeq}{\mathrel{\text{\textcircled{$=$}}}}





\usepackage{xcolor}
% \usepackage[normalem]{ulem}
\usepackage{lipsum}
\makeatletter
% \newcommand\colorwave[1][blue]{\bgroup \markoverwith{\lower3.5\p@\hbox{\sixly \textcolor{#1}{\char58}}}\ULon}
%\font\sixly=lasy6 % does not re-load if already loaded, so no memory problem.

\newmdtheoremenv[
linewidth= 1pt,linecolor= blue,%
leftmargin=20,rightmargin=20,innertopmargin=0pt, innerrightmargin=40,%
tikzsetting = { draw=lightgray, line width = 0.3pt,dashed,%
dash pattern = on 15pt off 3pt},%
splittopskip=\topskip,skipbelow=\baselineskip,%
skipabove=\baselineskip,ntheorem,roundcorner=0pt,
% backgroundcolor=pagebg,font=\color{orange}\sffamily, fontcolor=white
]{examplebox}{Exemple}[section]



\newcommand\R{\mathbb{R}}
\newcommand\Z{\mathbb{Z}}
\newcommand\N{\mathbb{N}}
\newcommand\E{\mathbb{E}}
\newcommand\F{\mathcal{F}}
\newcommand\cH{\mathcal{H}}
\newcommand\V{\mathbb{V}}
\newcommand\dmo{ ^{-1} }
\newcommand\kapa{\kappa}
\newcommand\im{Im}
\newcommand\hs{\mathcal{H}}





\usepackage{soul}

\makeatletter
\newcommand*{\whiten}[1]{\llap{\textcolor{white}{{\the\SOUL@token}}\hspace{#1pt}}}
\DeclareRobustCommand*\myul{%
    \def\SOUL@everyspace{\underline{\space}\kern\z@}%
    \def\SOUL@everytoken{%
     \setbox0=\hbox{\the\SOUL@token}%
     \ifdim\dp0>\z@
        \raisebox{\dp0}{\underline{\phantom{\the\SOUL@token}}}%
        \whiten{1}\whiten{0}%
        \whiten{-1}\whiten{-2}%
        \llap{\the\SOUL@token}%
     \else
        \underline{\the\SOUL@token}%
     \fi}%
\SOUL@}
\makeatother

\newcommand*{\demp}{\fontfamily{lmtt}\selectfont}

\DeclareTextFontCommand{\textdemp}{\demp}

\begin{document}

\ifcomment
Multiline
comment
\fi
\ifcomment
\myul{Typesetting test}
% \color[rgb]{1,1,1}
$∑_i^n≠ 60º±∞π∆¬≈√j∫h≤≥µ$

$\CR \R\pro\ind\pro\gS\pro
\mqty[a&b\\c&d]$
$\pro\mathbb{P}$
$\dd{x}$

  \[
    \alpha(x)=\left\{
                \begin{array}{ll}
                  x\\
                  \frac{1}{1+e^{-kx}}\\
                  \frac{e^x-e^{-x}}{e^x+e^{-x}}
                \end{array}
              \right.
  \]

  $\expval{x}$
  
  $\chi_\rho(ghg\dmo)=\Tr(\rho_{ghg\dmo})=\Tr(\rho_g\circ\rho_h\circ\rho\dmo_g)=\Tr(\rho_h)\overset{\mbox{\scalebox{0.5}{$\Tr(AB)=\Tr(BA)$}}}{=}\chi_\rho(h)$
  	$\mathop{\oplus}_{\substack{x\in X}}$

$\mat(\rho_g)=(a_{ij}(g))_{\scriptsize \substack{1\leq i\leq d \\ 1\leq j\leq d}}$ et $\mat(\rho'_g)=(a'_{ij}(g))_{\scriptsize \substack{1\leq i'\leq d' \\ 1\leq j'\leq d'}}$



\[\int_a^b{\mathbb{R}^2}g(u, v)\dd{P_{XY}}(u, v)=\iint g(u,v) f_{XY}(u, v)\dd \lambda(u) \dd \lambda(v)\]
$$\lim_{x\to\infty} f(x)$$	
$$\iiiint_V \mu(t,u,v,w) \,dt\,du\,dv\,dw$$
$$\sum_{n=1}^{\infty} 2^{-n} = 1$$	
\begin{definition}
	Si $X$ et $Y$ sont 2 v.a. ou definit la \textsc{Covariance} entre $X$ et $Y$ comme
	$\cov(X,Y)\overset{\text{def}}{=}\E\left[(X-\E(X))(Y-\E(Y))\right]=\E(XY)-\E(X)\E(Y)$.
\end{definition}
\fi
\pagebreak

% \tableofcontents

% insert your code here
%\input{./algebra/main.tex}
%\input{./geometrie-differentielle/main.tex}
%\input{./probabilite/main.tex}
%\input{./analyse-fonctionnelle/main.tex}
% \input{./Analyse-convexe-et-dualite-en-optimisation/main.tex}
%\input{./tikz/main.tex}
%\input{./Theorie-du-distributions/main.tex}
%\input{./optimisation/mine.tex}
 \input{./modelisation/main.tex}

% yves.aubry@univ-tln.fr : algebra

\end{document}

%\input{./optimisation/mine.tex}
 % !TEX encoding = UTF-8 Unicode
% !TEX TS-program = xelatex

\documentclass[french]{report}

%\usepackage[utf8]{inputenc}
%\usepackage[T1]{fontenc}
\usepackage{babel}


\newif\ifcomment
%\commenttrue # Show comments

\usepackage{physics}
\usepackage{amssymb}


\usepackage{amsthm}
% \usepackage{thmtools}
\usepackage{mathtools}
\usepackage{amsfonts}

\usepackage{color}

\usepackage{tikz}

\usepackage{geometry}
\geometry{a5paper, margin=0.1in, right=1cm}

\usepackage{dsfont}

\usepackage{graphicx}
\graphicspath{ {images/} }

\usepackage{faktor}

\usepackage{IEEEtrantools}
\usepackage{enumerate}   
\usepackage[PostScript=dvips]{"/Users/aware/Documents/Courses/diagrams"}


\newtheorem{theorem}{Théorème}[section]
\renewcommand{\thetheorem}{\arabic{theorem}}
\newtheorem{lemme}{Lemme}[section]
\renewcommand{\thelemme}{\arabic{lemme}}
\newtheorem{proposition}{Proposition}[section]
\renewcommand{\theproposition}{\arabic{proposition}}
\newtheorem{notations}{Notations}[section]
\newtheorem{problem}{Problème}[section]
\newtheorem{corollary}{Corollaire}[theorem]
\renewcommand{\thecorollary}{\arabic{corollary}}
\newtheorem{property}{Propriété}[section]
\newtheorem{objective}{Objectif}[section]

\theoremstyle{definition}
\newtheorem{definition}{Définition}[section]
\renewcommand{\thedefinition}{\arabic{definition}}
\newtheorem{exercise}{Exercice}[chapter]
\renewcommand{\theexercise}{\arabic{exercise}}
\newtheorem{example}{Exemple}[chapter]
\renewcommand{\theexample}{\arabic{example}}
\newtheorem*{solution}{Solution}
\newtheorem*{application}{Application}
\newtheorem*{notation}{Notation}
\newtheorem*{vocabulary}{Vocabulaire}
\newtheorem*{properties}{Propriétés}



\theoremstyle{remark}
\newtheorem*{remark}{Remarque}
\newtheorem*{rappel}{Rappel}


\usepackage{etoolbox}
\AtBeginEnvironment{exercise}{\small}
\AtBeginEnvironment{example}{\small}

\usepackage{cases}
\usepackage[red]{mypack}

\usepackage[framemethod=TikZ]{mdframed}

\definecolor{bg}{rgb}{0.4,0.25,0.95}
\definecolor{pagebg}{rgb}{0,0,0.5}
\surroundwithmdframed[
   topline=false,
   rightline=false,
   bottomline=false,
   leftmargin=\parindent,
   skipabove=8pt,
   skipbelow=8pt,
   linecolor=blue,
   innerbottommargin=10pt,
   % backgroundcolor=bg,font=\color{orange}\sffamily, fontcolor=white
]{definition}

\usepackage{empheq}
\usepackage[most]{tcolorbox}

\newtcbox{\mymath}[1][]{%
    nobeforeafter, math upper, tcbox raise base,
    enhanced, colframe=blue!30!black,
    colback=red!10, boxrule=1pt,
    #1}

\usepackage{unixode}


\DeclareMathOperator{\ord}{ord}
\DeclareMathOperator{\orb}{orb}
\DeclareMathOperator{\stab}{stab}
\DeclareMathOperator{\Stab}{stab}
\DeclareMathOperator{\ppcm}{ppcm}
\DeclareMathOperator{\conj}{Conj}
\DeclareMathOperator{\End}{End}
\DeclareMathOperator{\rot}{rot}
\DeclareMathOperator{\trs}{trace}
\DeclareMathOperator{\Ind}{Ind}
\DeclareMathOperator{\mat}{Mat}
\DeclareMathOperator{\id}{Id}
\DeclareMathOperator{\vect}{vect}
\DeclareMathOperator{\img}{img}
\DeclareMathOperator{\cov}{Cov}
\DeclareMathOperator{\dist}{dist}
\DeclareMathOperator{\irr}{Irr}
\DeclareMathOperator{\image}{Im}
\DeclareMathOperator{\pd}{\partial}
\DeclareMathOperator{\epi}{epi}
\DeclareMathOperator{\Argmin}{Argmin}
\DeclareMathOperator{\dom}{dom}
\DeclareMathOperator{\proj}{proj}
\DeclareMathOperator{\ctg}{ctg}
\DeclareMathOperator{\supp}{supp}
\DeclareMathOperator{\argmin}{argmin}
\DeclareMathOperator{\mult}{mult}
\DeclareMathOperator{\ch}{ch}
\DeclareMathOperator{\sh}{sh}
\DeclareMathOperator{\rang}{rang}
\DeclareMathOperator{\diam}{diam}
\DeclareMathOperator{\Epigraphe}{Epigraphe}




\usepackage{xcolor}
\everymath{\color{blue}}
%\everymath{\color[rgb]{0,1,1}}
%\pagecolor[rgb]{0,0,0.5}


\newcommand*{\pdtest}[3][]{\ensuremath{\frac{\partial^{#1} #2}{\partial #3}}}

\newcommand*{\deffunc}[6][]{\ensuremath{
\begin{array}{rcl}
#2 : #3 &\rightarrow& #4\\
#5 &\mapsto& #6
\end{array}
}}

\newcommand{\eqcolon}{\mathrel{\resizebox{\widthof{$\mathord{=}$}}{\height}{ $\!\!=\!\!\resizebox{1.2\width}{0.8\height}{\raisebox{0.23ex}{$\mathop{:}$}}\!\!$ }}}
\newcommand{\coloneq}{\mathrel{\resizebox{\widthof{$\mathord{=}$}}{\height}{ $\!\!\resizebox{1.2\width}{0.8\height}{\raisebox{0.23ex}{$\mathop{:}$}}\!\!=\!\!$ }}}
\newcommand{\eqcolonl}{\ensuremath{\mathrel{=\!\!\mathop{:}}}}
\newcommand{\coloneql}{\ensuremath{\mathrel{\mathop{:} \!\! =}}}
\newcommand{\vc}[1]{% inline column vector
  \left(\begin{smallmatrix}#1\end{smallmatrix}\right)%
}
\newcommand{\vr}[1]{% inline row vector
  \begin{smallmatrix}(\,#1\,)\end{smallmatrix}%
}
\makeatletter
\newcommand*{\defeq}{\ =\mathrel{\rlap{%
                     \raisebox{0.3ex}{$\m@th\cdot$}}%
                     \raisebox{-0.3ex}{$\m@th\cdot$}}%
                     }
\makeatother

\newcommand{\mathcircle}[1]{% inline row vector
 \overset{\circ}{#1}
}
\newcommand{\ulim}{% low limit
 \underline{\lim}
}
\newcommand{\ssi}{% iff
\iff
}
\newcommand{\ps}[2]{
\expval{#1 | #2}
}
\newcommand{\df}[1]{
\mqty{#1}
}
\newcommand{\n}[1]{
\norm{#1}
}
\newcommand{\sys}[1]{
\left\{\smqty{#1}\right.
}


\newcommand{\eqdef}{\ensuremath{\overset{\text{def}}=}}


\def\Circlearrowright{\ensuremath{%
  \rotatebox[origin=c]{230}{$\circlearrowright$}}}

\newcommand\ct[1]{\text{\rmfamily\upshape #1}}
\newcommand\question[1]{ {\color{red} ...!? \small #1}}
\newcommand\caz[1]{\left\{\begin{array} #1 \end{array}\right.}
\newcommand\const{\text{\rmfamily\upshape const}}
\newcommand\toP{ \overset{\pro}{\to}}
\newcommand\toPP{ \overset{\text{PP}}{\to}}
\newcommand{\oeq}{\mathrel{\text{\textcircled{$=$}}}}





\usepackage{xcolor}
% \usepackage[normalem]{ulem}
\usepackage{lipsum}
\makeatletter
% \newcommand\colorwave[1][blue]{\bgroup \markoverwith{\lower3.5\p@\hbox{\sixly \textcolor{#1}{\char58}}}\ULon}
%\font\sixly=lasy6 % does not re-load if already loaded, so no memory problem.

\newmdtheoremenv[
linewidth= 1pt,linecolor= blue,%
leftmargin=20,rightmargin=20,innertopmargin=0pt, innerrightmargin=40,%
tikzsetting = { draw=lightgray, line width = 0.3pt,dashed,%
dash pattern = on 15pt off 3pt},%
splittopskip=\topskip,skipbelow=\baselineskip,%
skipabove=\baselineskip,ntheorem,roundcorner=0pt,
% backgroundcolor=pagebg,font=\color{orange}\sffamily, fontcolor=white
]{examplebox}{Exemple}[section]



\newcommand\R{\mathbb{R}}
\newcommand\Z{\mathbb{Z}}
\newcommand\N{\mathbb{N}}
\newcommand\E{\mathbb{E}}
\newcommand\F{\mathcal{F}}
\newcommand\cH{\mathcal{H}}
\newcommand\V{\mathbb{V}}
\newcommand\dmo{ ^{-1} }
\newcommand\kapa{\kappa}
\newcommand\im{Im}
\newcommand\hs{\mathcal{H}}





\usepackage{soul}

\makeatletter
\newcommand*{\whiten}[1]{\llap{\textcolor{white}{{\the\SOUL@token}}\hspace{#1pt}}}
\DeclareRobustCommand*\myul{%
    \def\SOUL@everyspace{\underline{\space}\kern\z@}%
    \def\SOUL@everytoken{%
     \setbox0=\hbox{\the\SOUL@token}%
     \ifdim\dp0>\z@
        \raisebox{\dp0}{\underline{\phantom{\the\SOUL@token}}}%
        \whiten{1}\whiten{0}%
        \whiten{-1}\whiten{-2}%
        \llap{\the\SOUL@token}%
     \else
        \underline{\the\SOUL@token}%
     \fi}%
\SOUL@}
\makeatother

\newcommand*{\demp}{\fontfamily{lmtt}\selectfont}

\DeclareTextFontCommand{\textdemp}{\demp}

\begin{document}

\ifcomment
Multiline
comment
\fi
\ifcomment
\myul{Typesetting test}
% \color[rgb]{1,1,1}
$∑_i^n≠ 60º±∞π∆¬≈√j∫h≤≥µ$

$\CR \R\pro\ind\pro\gS\pro
\mqty[a&b\\c&d]$
$\pro\mathbb{P}$
$\dd{x}$

  \[
    \alpha(x)=\left\{
                \begin{array}{ll}
                  x\\
                  \frac{1}{1+e^{-kx}}\\
                  \frac{e^x-e^{-x}}{e^x+e^{-x}}
                \end{array}
              \right.
  \]

  $\expval{x}$
  
  $\chi_\rho(ghg\dmo)=\Tr(\rho_{ghg\dmo})=\Tr(\rho_g\circ\rho_h\circ\rho\dmo_g)=\Tr(\rho_h)\overset{\mbox{\scalebox{0.5}{$\Tr(AB)=\Tr(BA)$}}}{=}\chi_\rho(h)$
  	$\mathop{\oplus}_{\substack{x\in X}}$

$\mat(\rho_g)=(a_{ij}(g))_{\scriptsize \substack{1\leq i\leq d \\ 1\leq j\leq d}}$ et $\mat(\rho'_g)=(a'_{ij}(g))_{\scriptsize \substack{1\leq i'\leq d' \\ 1\leq j'\leq d'}}$



\[\int_a^b{\mathbb{R}^2}g(u, v)\dd{P_{XY}}(u, v)=\iint g(u,v) f_{XY}(u, v)\dd \lambda(u) \dd \lambda(v)\]
$$\lim_{x\to\infty} f(x)$$	
$$\iiiint_V \mu(t,u,v,w) \,dt\,du\,dv\,dw$$
$$\sum_{n=1}^{\infty} 2^{-n} = 1$$	
\begin{definition}
	Si $X$ et $Y$ sont 2 v.a. ou definit la \textsc{Covariance} entre $X$ et $Y$ comme
	$\cov(X,Y)\overset{\text{def}}{=}\E\left[(X-\E(X))(Y-\E(Y))\right]=\E(XY)-\E(X)\E(Y)$.
\end{definition}
\fi
\pagebreak

% \tableofcontents

% insert your code here
%\input{./algebra/main.tex}
%\input{./geometrie-differentielle/main.tex}
%\input{./probabilite/main.tex}
%\input{./analyse-fonctionnelle/main.tex}
% \input{./Analyse-convexe-et-dualite-en-optimisation/main.tex}
%\input{./tikz/main.tex}
%\input{./Theorie-du-distributions/main.tex}
%\input{./optimisation/mine.tex}
 \input{./modelisation/main.tex}

% yves.aubry@univ-tln.fr : algebra

\end{document}


% yves.aubry@univ-tln.fr : algebra

\end{document}

%% !TEX encoding = UTF-8 Unicode
% !TEX TS-program = xelatex

\documentclass[french]{report}

%\usepackage[utf8]{inputenc}
%\usepackage[T1]{fontenc}
\usepackage{babel}


\newif\ifcomment
%\commenttrue # Show comments

\usepackage{physics}
\usepackage{amssymb}


\usepackage{amsthm}
% \usepackage{thmtools}
\usepackage{mathtools}
\usepackage{amsfonts}

\usepackage{color}

\usepackage{tikz}

\usepackage{geometry}
\geometry{a5paper, margin=0.1in, right=1cm}

\usepackage{dsfont}

\usepackage{graphicx}
\graphicspath{ {images/} }

\usepackage{faktor}

\usepackage{IEEEtrantools}
\usepackage{enumerate}   
\usepackage[PostScript=dvips]{"/Users/aware/Documents/Courses/diagrams"}


\newtheorem{theorem}{Théorème}[section]
\renewcommand{\thetheorem}{\arabic{theorem}}
\newtheorem{lemme}{Lemme}[section]
\renewcommand{\thelemme}{\arabic{lemme}}
\newtheorem{proposition}{Proposition}[section]
\renewcommand{\theproposition}{\arabic{proposition}}
\newtheorem{notations}{Notations}[section]
\newtheorem{problem}{Problème}[section]
\newtheorem{corollary}{Corollaire}[theorem]
\renewcommand{\thecorollary}{\arabic{corollary}}
\newtheorem{property}{Propriété}[section]
\newtheorem{objective}{Objectif}[section]

\theoremstyle{definition}
\newtheorem{definition}{Définition}[section]
\renewcommand{\thedefinition}{\arabic{definition}}
\newtheorem{exercise}{Exercice}[chapter]
\renewcommand{\theexercise}{\arabic{exercise}}
\newtheorem{example}{Exemple}[chapter]
\renewcommand{\theexample}{\arabic{example}}
\newtheorem*{solution}{Solution}
\newtheorem*{application}{Application}
\newtheorem*{notation}{Notation}
\newtheorem*{vocabulary}{Vocabulaire}
\newtheorem*{properties}{Propriétés}



\theoremstyle{remark}
\newtheorem*{remark}{Remarque}
\newtheorem*{rappel}{Rappel}


\usepackage{etoolbox}
\AtBeginEnvironment{exercise}{\small}
\AtBeginEnvironment{example}{\small}

\usepackage{cases}
\usepackage[red]{mypack}

\usepackage[framemethod=TikZ]{mdframed}

\definecolor{bg}{rgb}{0.4,0.25,0.95}
\definecolor{pagebg}{rgb}{0,0,0.5}
\surroundwithmdframed[
   topline=false,
   rightline=false,
   bottomline=false,
   leftmargin=\parindent,
   skipabove=8pt,
   skipbelow=8pt,
   linecolor=blue,
   innerbottommargin=10pt,
   % backgroundcolor=bg,font=\color{orange}\sffamily, fontcolor=white
]{definition}

\usepackage{empheq}
\usepackage[most]{tcolorbox}

\newtcbox{\mymath}[1][]{%
    nobeforeafter, math upper, tcbox raise base,
    enhanced, colframe=blue!30!black,
    colback=red!10, boxrule=1pt,
    #1}

\usepackage{unixode}


\DeclareMathOperator{\ord}{ord}
\DeclareMathOperator{\orb}{orb}
\DeclareMathOperator{\stab}{stab}
\DeclareMathOperator{\Stab}{stab}
\DeclareMathOperator{\ppcm}{ppcm}
\DeclareMathOperator{\conj}{Conj}
\DeclareMathOperator{\End}{End}
\DeclareMathOperator{\rot}{rot}
\DeclareMathOperator{\trs}{trace}
\DeclareMathOperator{\Ind}{Ind}
\DeclareMathOperator{\mat}{Mat}
\DeclareMathOperator{\id}{Id}
\DeclareMathOperator{\vect}{vect}
\DeclareMathOperator{\img}{img}
\DeclareMathOperator{\cov}{Cov}
\DeclareMathOperator{\dist}{dist}
\DeclareMathOperator{\irr}{Irr}
\DeclareMathOperator{\image}{Im}
\DeclareMathOperator{\pd}{\partial}
\DeclareMathOperator{\epi}{epi}
\DeclareMathOperator{\Argmin}{Argmin}
\DeclareMathOperator{\dom}{dom}
\DeclareMathOperator{\proj}{proj}
\DeclareMathOperator{\ctg}{ctg}
\DeclareMathOperator{\supp}{supp}
\DeclareMathOperator{\argmin}{argmin}
\DeclareMathOperator{\mult}{mult}
\DeclareMathOperator{\ch}{ch}
\DeclareMathOperator{\sh}{sh}
\DeclareMathOperator{\rang}{rang}
\DeclareMathOperator{\diam}{diam}
\DeclareMathOperator{\Epigraphe}{Epigraphe}




\usepackage{xcolor}
\everymath{\color{blue}}
%\everymath{\color[rgb]{0,1,1}}
%\pagecolor[rgb]{0,0,0.5}


\newcommand*{\pdtest}[3][]{\ensuremath{\frac{\partial^{#1} #2}{\partial #3}}}

\newcommand*{\deffunc}[6][]{\ensuremath{
\begin{array}{rcl}
#2 : #3 &\rightarrow& #4\\
#5 &\mapsto& #6
\end{array}
}}

\newcommand{\eqcolon}{\mathrel{\resizebox{\widthof{$\mathord{=}$}}{\height}{ $\!\!=\!\!\resizebox{1.2\width}{0.8\height}{\raisebox{0.23ex}{$\mathop{:}$}}\!\!$ }}}
\newcommand{\coloneq}{\mathrel{\resizebox{\widthof{$\mathord{=}$}}{\height}{ $\!\!\resizebox{1.2\width}{0.8\height}{\raisebox{0.23ex}{$\mathop{:}$}}\!\!=\!\!$ }}}
\newcommand{\eqcolonl}{\ensuremath{\mathrel{=\!\!\mathop{:}}}}
\newcommand{\coloneql}{\ensuremath{\mathrel{\mathop{:} \!\! =}}}
\newcommand{\vc}[1]{% inline column vector
  \left(\begin{smallmatrix}#1\end{smallmatrix}\right)%
}
\newcommand{\vr}[1]{% inline row vector
  \begin{smallmatrix}(\,#1\,)\end{smallmatrix}%
}
\makeatletter
\newcommand*{\defeq}{\ =\mathrel{\rlap{%
                     \raisebox{0.3ex}{$\m@th\cdot$}}%
                     \raisebox{-0.3ex}{$\m@th\cdot$}}%
                     }
\makeatother

\newcommand{\mathcircle}[1]{% inline row vector
 \overset{\circ}{#1}
}
\newcommand{\ulim}{% low limit
 \underline{\lim}
}
\newcommand{\ssi}{% iff
\iff
}
\newcommand{\ps}[2]{
\expval{#1 | #2}
}
\newcommand{\df}[1]{
\mqty{#1}
}
\newcommand{\n}[1]{
\norm{#1}
}
\newcommand{\sys}[1]{
\left\{\smqty{#1}\right.
}


\newcommand{\eqdef}{\ensuremath{\overset{\text{def}}=}}


\def\Circlearrowright{\ensuremath{%
  \rotatebox[origin=c]{230}{$\circlearrowright$}}}

\newcommand\ct[1]{\text{\rmfamily\upshape #1}}
\newcommand\question[1]{ {\color{red} ...!? \small #1}}
\newcommand\caz[1]{\left\{\begin{array} #1 \end{array}\right.}
\newcommand\const{\text{\rmfamily\upshape const}}
\newcommand\toP{ \overset{\pro}{\to}}
\newcommand\toPP{ \overset{\text{PP}}{\to}}
\newcommand{\oeq}{\mathrel{\text{\textcircled{$=$}}}}





\usepackage{xcolor}
% \usepackage[normalem]{ulem}
\usepackage{lipsum}
\makeatletter
% \newcommand\colorwave[1][blue]{\bgroup \markoverwith{\lower3.5\p@\hbox{\sixly \textcolor{#1}{\char58}}}\ULon}
%\font\sixly=lasy6 % does not re-load if already loaded, so no memory problem.

\newmdtheoremenv[
linewidth= 1pt,linecolor= blue,%
leftmargin=20,rightmargin=20,innertopmargin=0pt, innerrightmargin=40,%
tikzsetting = { draw=lightgray, line width = 0.3pt,dashed,%
dash pattern = on 15pt off 3pt},%
splittopskip=\topskip,skipbelow=\baselineskip,%
skipabove=\baselineskip,ntheorem,roundcorner=0pt,
% backgroundcolor=pagebg,font=\color{orange}\sffamily, fontcolor=white
]{examplebox}{Exemple}[section]



\newcommand\R{\mathbb{R}}
\newcommand\Z{\mathbb{Z}}
\newcommand\N{\mathbb{N}}
\newcommand\E{\mathbb{E}}
\newcommand\F{\mathcal{F}}
\newcommand\cH{\mathcal{H}}
\newcommand\V{\mathbb{V}}
\newcommand\dmo{ ^{-1} }
\newcommand\kapa{\kappa}
\newcommand\im{Im}
\newcommand\hs{\mathcal{H}}





\usepackage{soul}

\makeatletter
\newcommand*{\whiten}[1]{\llap{\textcolor{white}{{\the\SOUL@token}}\hspace{#1pt}}}
\DeclareRobustCommand*\myul{%
    \def\SOUL@everyspace{\underline{\space}\kern\z@}%
    \def\SOUL@everytoken{%
     \setbox0=\hbox{\the\SOUL@token}%
     \ifdim\dp0>\z@
        \raisebox{\dp0}{\underline{\phantom{\the\SOUL@token}}}%
        \whiten{1}\whiten{0}%
        \whiten{-1}\whiten{-2}%
        \llap{\the\SOUL@token}%
     \else
        \underline{\the\SOUL@token}%
     \fi}%
\SOUL@}
\makeatother

\newcommand*{\demp}{\fontfamily{lmtt}\selectfont}

\DeclareTextFontCommand{\textdemp}{\demp}

\begin{document}

\ifcomment
Multiline
comment
\fi
\ifcomment
\myul{Typesetting test}
% \color[rgb]{1,1,1}
$∑_i^n≠ 60º±∞π∆¬≈√j∫h≤≥µ$

$\CR \R\pro\ind\pro\gS\pro
\mqty[a&b\\c&d]$
$\pro\mathbb{P}$
$\dd{x}$

  \[
    \alpha(x)=\left\{
                \begin{array}{ll}
                  x\\
                  \frac{1}{1+e^{-kx}}\\
                  \frac{e^x-e^{-x}}{e^x+e^{-x}}
                \end{array}
              \right.
  \]

  $\expval{x}$
  
  $\chi_\rho(ghg\dmo)=\Tr(\rho_{ghg\dmo})=\Tr(\rho_g\circ\rho_h\circ\rho\dmo_g)=\Tr(\rho_h)\overset{\mbox{\scalebox{0.5}{$\Tr(AB)=\Tr(BA)$}}}{=}\chi_\rho(h)$
  	$\mathop{\oplus}_{\substack{x\in X}}$

$\mat(\rho_g)=(a_{ij}(g))_{\scriptsize \substack{1\leq i\leq d \\ 1\leq j\leq d}}$ et $\mat(\rho'_g)=(a'_{ij}(g))_{\scriptsize \substack{1\leq i'\leq d' \\ 1\leq j'\leq d'}}$



\[\int_a^b{\mathbb{R}^2}g(u, v)\dd{P_{XY}}(u, v)=\iint g(u,v) f_{XY}(u, v)\dd \lambda(u) \dd \lambda(v)\]
$$\lim_{x\to\infty} f(x)$$	
$$\iiiint_V \mu(t,u,v,w) \,dt\,du\,dv\,dw$$
$$\sum_{n=1}^{\infty} 2^{-n} = 1$$	
\begin{definition}
	Si $X$ et $Y$ sont 2 v.a. ou definit la \textsc{Covariance} entre $X$ et $Y$ comme
	$\cov(X,Y)\overset{\text{def}}{=}\E\left[(X-\E(X))(Y-\E(Y))\right]=\E(XY)-\E(X)\E(Y)$.
\end{definition}
\fi
\pagebreak

% \tableofcontents

% insert your code here
%% !TEX encoding = UTF-8 Unicode
% !TEX TS-program = xelatex

\documentclass[french]{report}

%\usepackage[utf8]{inputenc}
%\usepackage[T1]{fontenc}
\usepackage{babel}


\newif\ifcomment
%\commenttrue # Show comments

\usepackage{physics}
\usepackage{amssymb}


\usepackage{amsthm}
% \usepackage{thmtools}
\usepackage{mathtools}
\usepackage{amsfonts}

\usepackage{color}

\usepackage{tikz}

\usepackage{geometry}
\geometry{a5paper, margin=0.1in, right=1cm}

\usepackage{dsfont}

\usepackage{graphicx}
\graphicspath{ {images/} }

\usepackage{faktor}

\usepackage{IEEEtrantools}
\usepackage{enumerate}   
\usepackage[PostScript=dvips]{"/Users/aware/Documents/Courses/diagrams"}


\newtheorem{theorem}{Théorème}[section]
\renewcommand{\thetheorem}{\arabic{theorem}}
\newtheorem{lemme}{Lemme}[section]
\renewcommand{\thelemme}{\arabic{lemme}}
\newtheorem{proposition}{Proposition}[section]
\renewcommand{\theproposition}{\arabic{proposition}}
\newtheorem{notations}{Notations}[section]
\newtheorem{problem}{Problème}[section]
\newtheorem{corollary}{Corollaire}[theorem]
\renewcommand{\thecorollary}{\arabic{corollary}}
\newtheorem{property}{Propriété}[section]
\newtheorem{objective}{Objectif}[section]

\theoremstyle{definition}
\newtheorem{definition}{Définition}[section]
\renewcommand{\thedefinition}{\arabic{definition}}
\newtheorem{exercise}{Exercice}[chapter]
\renewcommand{\theexercise}{\arabic{exercise}}
\newtheorem{example}{Exemple}[chapter]
\renewcommand{\theexample}{\arabic{example}}
\newtheorem*{solution}{Solution}
\newtheorem*{application}{Application}
\newtheorem*{notation}{Notation}
\newtheorem*{vocabulary}{Vocabulaire}
\newtheorem*{properties}{Propriétés}



\theoremstyle{remark}
\newtheorem*{remark}{Remarque}
\newtheorem*{rappel}{Rappel}


\usepackage{etoolbox}
\AtBeginEnvironment{exercise}{\small}
\AtBeginEnvironment{example}{\small}

\usepackage{cases}
\usepackage[red]{mypack}

\usepackage[framemethod=TikZ]{mdframed}

\definecolor{bg}{rgb}{0.4,0.25,0.95}
\definecolor{pagebg}{rgb}{0,0,0.5}
\surroundwithmdframed[
   topline=false,
   rightline=false,
   bottomline=false,
   leftmargin=\parindent,
   skipabove=8pt,
   skipbelow=8pt,
   linecolor=blue,
   innerbottommargin=10pt,
   % backgroundcolor=bg,font=\color{orange}\sffamily, fontcolor=white
]{definition}

\usepackage{empheq}
\usepackage[most]{tcolorbox}

\newtcbox{\mymath}[1][]{%
    nobeforeafter, math upper, tcbox raise base,
    enhanced, colframe=blue!30!black,
    colback=red!10, boxrule=1pt,
    #1}

\usepackage{unixode}


\DeclareMathOperator{\ord}{ord}
\DeclareMathOperator{\orb}{orb}
\DeclareMathOperator{\stab}{stab}
\DeclareMathOperator{\Stab}{stab}
\DeclareMathOperator{\ppcm}{ppcm}
\DeclareMathOperator{\conj}{Conj}
\DeclareMathOperator{\End}{End}
\DeclareMathOperator{\rot}{rot}
\DeclareMathOperator{\trs}{trace}
\DeclareMathOperator{\Ind}{Ind}
\DeclareMathOperator{\mat}{Mat}
\DeclareMathOperator{\id}{Id}
\DeclareMathOperator{\vect}{vect}
\DeclareMathOperator{\img}{img}
\DeclareMathOperator{\cov}{Cov}
\DeclareMathOperator{\dist}{dist}
\DeclareMathOperator{\irr}{Irr}
\DeclareMathOperator{\image}{Im}
\DeclareMathOperator{\pd}{\partial}
\DeclareMathOperator{\epi}{epi}
\DeclareMathOperator{\Argmin}{Argmin}
\DeclareMathOperator{\dom}{dom}
\DeclareMathOperator{\proj}{proj}
\DeclareMathOperator{\ctg}{ctg}
\DeclareMathOperator{\supp}{supp}
\DeclareMathOperator{\argmin}{argmin}
\DeclareMathOperator{\mult}{mult}
\DeclareMathOperator{\ch}{ch}
\DeclareMathOperator{\sh}{sh}
\DeclareMathOperator{\rang}{rang}
\DeclareMathOperator{\diam}{diam}
\DeclareMathOperator{\Epigraphe}{Epigraphe}




\usepackage{xcolor}
\everymath{\color{blue}}
%\everymath{\color[rgb]{0,1,1}}
%\pagecolor[rgb]{0,0,0.5}


\newcommand*{\pdtest}[3][]{\ensuremath{\frac{\partial^{#1} #2}{\partial #3}}}

\newcommand*{\deffunc}[6][]{\ensuremath{
\begin{array}{rcl}
#2 : #3 &\rightarrow& #4\\
#5 &\mapsto& #6
\end{array}
}}

\newcommand{\eqcolon}{\mathrel{\resizebox{\widthof{$\mathord{=}$}}{\height}{ $\!\!=\!\!\resizebox{1.2\width}{0.8\height}{\raisebox{0.23ex}{$\mathop{:}$}}\!\!$ }}}
\newcommand{\coloneq}{\mathrel{\resizebox{\widthof{$\mathord{=}$}}{\height}{ $\!\!\resizebox{1.2\width}{0.8\height}{\raisebox{0.23ex}{$\mathop{:}$}}\!\!=\!\!$ }}}
\newcommand{\eqcolonl}{\ensuremath{\mathrel{=\!\!\mathop{:}}}}
\newcommand{\coloneql}{\ensuremath{\mathrel{\mathop{:} \!\! =}}}
\newcommand{\vc}[1]{% inline column vector
  \left(\begin{smallmatrix}#1\end{smallmatrix}\right)%
}
\newcommand{\vr}[1]{% inline row vector
  \begin{smallmatrix}(\,#1\,)\end{smallmatrix}%
}
\makeatletter
\newcommand*{\defeq}{\ =\mathrel{\rlap{%
                     \raisebox{0.3ex}{$\m@th\cdot$}}%
                     \raisebox{-0.3ex}{$\m@th\cdot$}}%
                     }
\makeatother

\newcommand{\mathcircle}[1]{% inline row vector
 \overset{\circ}{#1}
}
\newcommand{\ulim}{% low limit
 \underline{\lim}
}
\newcommand{\ssi}{% iff
\iff
}
\newcommand{\ps}[2]{
\expval{#1 | #2}
}
\newcommand{\df}[1]{
\mqty{#1}
}
\newcommand{\n}[1]{
\norm{#1}
}
\newcommand{\sys}[1]{
\left\{\smqty{#1}\right.
}


\newcommand{\eqdef}{\ensuremath{\overset{\text{def}}=}}


\def\Circlearrowright{\ensuremath{%
  \rotatebox[origin=c]{230}{$\circlearrowright$}}}

\newcommand\ct[1]{\text{\rmfamily\upshape #1}}
\newcommand\question[1]{ {\color{red} ...!? \small #1}}
\newcommand\caz[1]{\left\{\begin{array} #1 \end{array}\right.}
\newcommand\const{\text{\rmfamily\upshape const}}
\newcommand\toP{ \overset{\pro}{\to}}
\newcommand\toPP{ \overset{\text{PP}}{\to}}
\newcommand{\oeq}{\mathrel{\text{\textcircled{$=$}}}}





\usepackage{xcolor}
% \usepackage[normalem]{ulem}
\usepackage{lipsum}
\makeatletter
% \newcommand\colorwave[1][blue]{\bgroup \markoverwith{\lower3.5\p@\hbox{\sixly \textcolor{#1}{\char58}}}\ULon}
%\font\sixly=lasy6 % does not re-load if already loaded, so no memory problem.

\newmdtheoremenv[
linewidth= 1pt,linecolor= blue,%
leftmargin=20,rightmargin=20,innertopmargin=0pt, innerrightmargin=40,%
tikzsetting = { draw=lightgray, line width = 0.3pt,dashed,%
dash pattern = on 15pt off 3pt},%
splittopskip=\topskip,skipbelow=\baselineskip,%
skipabove=\baselineskip,ntheorem,roundcorner=0pt,
% backgroundcolor=pagebg,font=\color{orange}\sffamily, fontcolor=white
]{examplebox}{Exemple}[section]



\newcommand\R{\mathbb{R}}
\newcommand\Z{\mathbb{Z}}
\newcommand\N{\mathbb{N}}
\newcommand\E{\mathbb{E}}
\newcommand\F{\mathcal{F}}
\newcommand\cH{\mathcal{H}}
\newcommand\V{\mathbb{V}}
\newcommand\dmo{ ^{-1} }
\newcommand\kapa{\kappa}
\newcommand\im{Im}
\newcommand\hs{\mathcal{H}}





\usepackage{soul}

\makeatletter
\newcommand*{\whiten}[1]{\llap{\textcolor{white}{{\the\SOUL@token}}\hspace{#1pt}}}
\DeclareRobustCommand*\myul{%
    \def\SOUL@everyspace{\underline{\space}\kern\z@}%
    \def\SOUL@everytoken{%
     \setbox0=\hbox{\the\SOUL@token}%
     \ifdim\dp0>\z@
        \raisebox{\dp0}{\underline{\phantom{\the\SOUL@token}}}%
        \whiten{1}\whiten{0}%
        \whiten{-1}\whiten{-2}%
        \llap{\the\SOUL@token}%
     \else
        \underline{\the\SOUL@token}%
     \fi}%
\SOUL@}
\makeatother

\newcommand*{\demp}{\fontfamily{lmtt}\selectfont}

\DeclareTextFontCommand{\textdemp}{\demp}

\begin{document}

\ifcomment
Multiline
comment
\fi
\ifcomment
\myul{Typesetting test}
% \color[rgb]{1,1,1}
$∑_i^n≠ 60º±∞π∆¬≈√j∫h≤≥µ$

$\CR \R\pro\ind\pro\gS\pro
\mqty[a&b\\c&d]$
$\pro\mathbb{P}$
$\dd{x}$

  \[
    \alpha(x)=\left\{
                \begin{array}{ll}
                  x\\
                  \frac{1}{1+e^{-kx}}\\
                  \frac{e^x-e^{-x}}{e^x+e^{-x}}
                \end{array}
              \right.
  \]

  $\expval{x}$
  
  $\chi_\rho(ghg\dmo)=\Tr(\rho_{ghg\dmo})=\Tr(\rho_g\circ\rho_h\circ\rho\dmo_g)=\Tr(\rho_h)\overset{\mbox{\scalebox{0.5}{$\Tr(AB)=\Tr(BA)$}}}{=}\chi_\rho(h)$
  	$\mathop{\oplus}_{\substack{x\in X}}$

$\mat(\rho_g)=(a_{ij}(g))_{\scriptsize \substack{1\leq i\leq d \\ 1\leq j\leq d}}$ et $\mat(\rho'_g)=(a'_{ij}(g))_{\scriptsize \substack{1\leq i'\leq d' \\ 1\leq j'\leq d'}}$



\[\int_a^b{\mathbb{R}^2}g(u, v)\dd{P_{XY}}(u, v)=\iint g(u,v) f_{XY}(u, v)\dd \lambda(u) \dd \lambda(v)\]
$$\lim_{x\to\infty} f(x)$$	
$$\iiiint_V \mu(t,u,v,w) \,dt\,du\,dv\,dw$$
$$\sum_{n=1}^{\infty} 2^{-n} = 1$$	
\begin{definition}
	Si $X$ et $Y$ sont 2 v.a. ou definit la \textsc{Covariance} entre $X$ et $Y$ comme
	$\cov(X,Y)\overset{\text{def}}{=}\E\left[(X-\E(X))(Y-\E(Y))\right]=\E(XY)-\E(X)\E(Y)$.
\end{definition}
\fi
\pagebreak

% \tableofcontents

% insert your code here
%\input{./algebra/main.tex}
%\input{./geometrie-differentielle/main.tex}
%\input{./probabilite/main.tex}
%\input{./analyse-fonctionnelle/main.tex}
% \input{./Analyse-convexe-et-dualite-en-optimisation/main.tex}
%\input{./tikz/main.tex}
%\input{./Theorie-du-distributions/main.tex}
%\input{./optimisation/mine.tex}
 \input{./modelisation/main.tex}

% yves.aubry@univ-tln.fr : algebra

\end{document}

%% !TEX encoding = UTF-8 Unicode
% !TEX TS-program = xelatex

\documentclass[french]{report}

%\usepackage[utf8]{inputenc}
%\usepackage[T1]{fontenc}
\usepackage{babel}


\newif\ifcomment
%\commenttrue # Show comments

\usepackage{physics}
\usepackage{amssymb}


\usepackage{amsthm}
% \usepackage{thmtools}
\usepackage{mathtools}
\usepackage{amsfonts}

\usepackage{color}

\usepackage{tikz}

\usepackage{geometry}
\geometry{a5paper, margin=0.1in, right=1cm}

\usepackage{dsfont}

\usepackage{graphicx}
\graphicspath{ {images/} }

\usepackage{faktor}

\usepackage{IEEEtrantools}
\usepackage{enumerate}   
\usepackage[PostScript=dvips]{"/Users/aware/Documents/Courses/diagrams"}


\newtheorem{theorem}{Théorème}[section]
\renewcommand{\thetheorem}{\arabic{theorem}}
\newtheorem{lemme}{Lemme}[section]
\renewcommand{\thelemme}{\arabic{lemme}}
\newtheorem{proposition}{Proposition}[section]
\renewcommand{\theproposition}{\arabic{proposition}}
\newtheorem{notations}{Notations}[section]
\newtheorem{problem}{Problème}[section]
\newtheorem{corollary}{Corollaire}[theorem]
\renewcommand{\thecorollary}{\arabic{corollary}}
\newtheorem{property}{Propriété}[section]
\newtheorem{objective}{Objectif}[section]

\theoremstyle{definition}
\newtheorem{definition}{Définition}[section]
\renewcommand{\thedefinition}{\arabic{definition}}
\newtheorem{exercise}{Exercice}[chapter]
\renewcommand{\theexercise}{\arabic{exercise}}
\newtheorem{example}{Exemple}[chapter]
\renewcommand{\theexample}{\arabic{example}}
\newtheorem*{solution}{Solution}
\newtheorem*{application}{Application}
\newtheorem*{notation}{Notation}
\newtheorem*{vocabulary}{Vocabulaire}
\newtheorem*{properties}{Propriétés}



\theoremstyle{remark}
\newtheorem*{remark}{Remarque}
\newtheorem*{rappel}{Rappel}


\usepackage{etoolbox}
\AtBeginEnvironment{exercise}{\small}
\AtBeginEnvironment{example}{\small}

\usepackage{cases}
\usepackage[red]{mypack}

\usepackage[framemethod=TikZ]{mdframed}

\definecolor{bg}{rgb}{0.4,0.25,0.95}
\definecolor{pagebg}{rgb}{0,0,0.5}
\surroundwithmdframed[
   topline=false,
   rightline=false,
   bottomline=false,
   leftmargin=\parindent,
   skipabove=8pt,
   skipbelow=8pt,
   linecolor=blue,
   innerbottommargin=10pt,
   % backgroundcolor=bg,font=\color{orange}\sffamily, fontcolor=white
]{definition}

\usepackage{empheq}
\usepackage[most]{tcolorbox}

\newtcbox{\mymath}[1][]{%
    nobeforeafter, math upper, tcbox raise base,
    enhanced, colframe=blue!30!black,
    colback=red!10, boxrule=1pt,
    #1}

\usepackage{unixode}


\DeclareMathOperator{\ord}{ord}
\DeclareMathOperator{\orb}{orb}
\DeclareMathOperator{\stab}{stab}
\DeclareMathOperator{\Stab}{stab}
\DeclareMathOperator{\ppcm}{ppcm}
\DeclareMathOperator{\conj}{Conj}
\DeclareMathOperator{\End}{End}
\DeclareMathOperator{\rot}{rot}
\DeclareMathOperator{\trs}{trace}
\DeclareMathOperator{\Ind}{Ind}
\DeclareMathOperator{\mat}{Mat}
\DeclareMathOperator{\id}{Id}
\DeclareMathOperator{\vect}{vect}
\DeclareMathOperator{\img}{img}
\DeclareMathOperator{\cov}{Cov}
\DeclareMathOperator{\dist}{dist}
\DeclareMathOperator{\irr}{Irr}
\DeclareMathOperator{\image}{Im}
\DeclareMathOperator{\pd}{\partial}
\DeclareMathOperator{\epi}{epi}
\DeclareMathOperator{\Argmin}{Argmin}
\DeclareMathOperator{\dom}{dom}
\DeclareMathOperator{\proj}{proj}
\DeclareMathOperator{\ctg}{ctg}
\DeclareMathOperator{\supp}{supp}
\DeclareMathOperator{\argmin}{argmin}
\DeclareMathOperator{\mult}{mult}
\DeclareMathOperator{\ch}{ch}
\DeclareMathOperator{\sh}{sh}
\DeclareMathOperator{\rang}{rang}
\DeclareMathOperator{\diam}{diam}
\DeclareMathOperator{\Epigraphe}{Epigraphe}




\usepackage{xcolor}
\everymath{\color{blue}}
%\everymath{\color[rgb]{0,1,1}}
%\pagecolor[rgb]{0,0,0.5}


\newcommand*{\pdtest}[3][]{\ensuremath{\frac{\partial^{#1} #2}{\partial #3}}}

\newcommand*{\deffunc}[6][]{\ensuremath{
\begin{array}{rcl}
#2 : #3 &\rightarrow& #4\\
#5 &\mapsto& #6
\end{array}
}}

\newcommand{\eqcolon}{\mathrel{\resizebox{\widthof{$\mathord{=}$}}{\height}{ $\!\!=\!\!\resizebox{1.2\width}{0.8\height}{\raisebox{0.23ex}{$\mathop{:}$}}\!\!$ }}}
\newcommand{\coloneq}{\mathrel{\resizebox{\widthof{$\mathord{=}$}}{\height}{ $\!\!\resizebox{1.2\width}{0.8\height}{\raisebox{0.23ex}{$\mathop{:}$}}\!\!=\!\!$ }}}
\newcommand{\eqcolonl}{\ensuremath{\mathrel{=\!\!\mathop{:}}}}
\newcommand{\coloneql}{\ensuremath{\mathrel{\mathop{:} \!\! =}}}
\newcommand{\vc}[1]{% inline column vector
  \left(\begin{smallmatrix}#1\end{smallmatrix}\right)%
}
\newcommand{\vr}[1]{% inline row vector
  \begin{smallmatrix}(\,#1\,)\end{smallmatrix}%
}
\makeatletter
\newcommand*{\defeq}{\ =\mathrel{\rlap{%
                     \raisebox{0.3ex}{$\m@th\cdot$}}%
                     \raisebox{-0.3ex}{$\m@th\cdot$}}%
                     }
\makeatother

\newcommand{\mathcircle}[1]{% inline row vector
 \overset{\circ}{#1}
}
\newcommand{\ulim}{% low limit
 \underline{\lim}
}
\newcommand{\ssi}{% iff
\iff
}
\newcommand{\ps}[2]{
\expval{#1 | #2}
}
\newcommand{\df}[1]{
\mqty{#1}
}
\newcommand{\n}[1]{
\norm{#1}
}
\newcommand{\sys}[1]{
\left\{\smqty{#1}\right.
}


\newcommand{\eqdef}{\ensuremath{\overset{\text{def}}=}}


\def\Circlearrowright{\ensuremath{%
  \rotatebox[origin=c]{230}{$\circlearrowright$}}}

\newcommand\ct[1]{\text{\rmfamily\upshape #1}}
\newcommand\question[1]{ {\color{red} ...!? \small #1}}
\newcommand\caz[1]{\left\{\begin{array} #1 \end{array}\right.}
\newcommand\const{\text{\rmfamily\upshape const}}
\newcommand\toP{ \overset{\pro}{\to}}
\newcommand\toPP{ \overset{\text{PP}}{\to}}
\newcommand{\oeq}{\mathrel{\text{\textcircled{$=$}}}}





\usepackage{xcolor}
% \usepackage[normalem]{ulem}
\usepackage{lipsum}
\makeatletter
% \newcommand\colorwave[1][blue]{\bgroup \markoverwith{\lower3.5\p@\hbox{\sixly \textcolor{#1}{\char58}}}\ULon}
%\font\sixly=lasy6 % does not re-load if already loaded, so no memory problem.

\newmdtheoremenv[
linewidth= 1pt,linecolor= blue,%
leftmargin=20,rightmargin=20,innertopmargin=0pt, innerrightmargin=40,%
tikzsetting = { draw=lightgray, line width = 0.3pt,dashed,%
dash pattern = on 15pt off 3pt},%
splittopskip=\topskip,skipbelow=\baselineskip,%
skipabove=\baselineskip,ntheorem,roundcorner=0pt,
% backgroundcolor=pagebg,font=\color{orange}\sffamily, fontcolor=white
]{examplebox}{Exemple}[section]



\newcommand\R{\mathbb{R}}
\newcommand\Z{\mathbb{Z}}
\newcommand\N{\mathbb{N}}
\newcommand\E{\mathbb{E}}
\newcommand\F{\mathcal{F}}
\newcommand\cH{\mathcal{H}}
\newcommand\V{\mathbb{V}}
\newcommand\dmo{ ^{-1} }
\newcommand\kapa{\kappa}
\newcommand\im{Im}
\newcommand\hs{\mathcal{H}}





\usepackage{soul}

\makeatletter
\newcommand*{\whiten}[1]{\llap{\textcolor{white}{{\the\SOUL@token}}\hspace{#1pt}}}
\DeclareRobustCommand*\myul{%
    \def\SOUL@everyspace{\underline{\space}\kern\z@}%
    \def\SOUL@everytoken{%
     \setbox0=\hbox{\the\SOUL@token}%
     \ifdim\dp0>\z@
        \raisebox{\dp0}{\underline{\phantom{\the\SOUL@token}}}%
        \whiten{1}\whiten{0}%
        \whiten{-1}\whiten{-2}%
        \llap{\the\SOUL@token}%
     \else
        \underline{\the\SOUL@token}%
     \fi}%
\SOUL@}
\makeatother

\newcommand*{\demp}{\fontfamily{lmtt}\selectfont}

\DeclareTextFontCommand{\textdemp}{\demp}

\begin{document}

\ifcomment
Multiline
comment
\fi
\ifcomment
\myul{Typesetting test}
% \color[rgb]{1,1,1}
$∑_i^n≠ 60º±∞π∆¬≈√j∫h≤≥µ$

$\CR \R\pro\ind\pro\gS\pro
\mqty[a&b\\c&d]$
$\pro\mathbb{P}$
$\dd{x}$

  \[
    \alpha(x)=\left\{
                \begin{array}{ll}
                  x\\
                  \frac{1}{1+e^{-kx}}\\
                  \frac{e^x-e^{-x}}{e^x+e^{-x}}
                \end{array}
              \right.
  \]

  $\expval{x}$
  
  $\chi_\rho(ghg\dmo)=\Tr(\rho_{ghg\dmo})=\Tr(\rho_g\circ\rho_h\circ\rho\dmo_g)=\Tr(\rho_h)\overset{\mbox{\scalebox{0.5}{$\Tr(AB)=\Tr(BA)$}}}{=}\chi_\rho(h)$
  	$\mathop{\oplus}_{\substack{x\in X}}$

$\mat(\rho_g)=(a_{ij}(g))_{\scriptsize \substack{1\leq i\leq d \\ 1\leq j\leq d}}$ et $\mat(\rho'_g)=(a'_{ij}(g))_{\scriptsize \substack{1\leq i'\leq d' \\ 1\leq j'\leq d'}}$



\[\int_a^b{\mathbb{R}^2}g(u, v)\dd{P_{XY}}(u, v)=\iint g(u,v) f_{XY}(u, v)\dd \lambda(u) \dd \lambda(v)\]
$$\lim_{x\to\infty} f(x)$$	
$$\iiiint_V \mu(t,u,v,w) \,dt\,du\,dv\,dw$$
$$\sum_{n=1}^{\infty} 2^{-n} = 1$$	
\begin{definition}
	Si $X$ et $Y$ sont 2 v.a. ou definit la \textsc{Covariance} entre $X$ et $Y$ comme
	$\cov(X,Y)\overset{\text{def}}{=}\E\left[(X-\E(X))(Y-\E(Y))\right]=\E(XY)-\E(X)\E(Y)$.
\end{definition}
\fi
\pagebreak

% \tableofcontents

% insert your code here
%\input{./algebra/main.tex}
%\input{./geometrie-differentielle/main.tex}
%\input{./probabilite/main.tex}
%\input{./analyse-fonctionnelle/main.tex}
% \input{./Analyse-convexe-et-dualite-en-optimisation/main.tex}
%\input{./tikz/main.tex}
%\input{./Theorie-du-distributions/main.tex}
%\input{./optimisation/mine.tex}
 \input{./modelisation/main.tex}

% yves.aubry@univ-tln.fr : algebra

\end{document}

%% !TEX encoding = UTF-8 Unicode
% !TEX TS-program = xelatex

\documentclass[french]{report}

%\usepackage[utf8]{inputenc}
%\usepackage[T1]{fontenc}
\usepackage{babel}


\newif\ifcomment
%\commenttrue # Show comments

\usepackage{physics}
\usepackage{amssymb}


\usepackage{amsthm}
% \usepackage{thmtools}
\usepackage{mathtools}
\usepackage{amsfonts}

\usepackage{color}

\usepackage{tikz}

\usepackage{geometry}
\geometry{a5paper, margin=0.1in, right=1cm}

\usepackage{dsfont}

\usepackage{graphicx}
\graphicspath{ {images/} }

\usepackage{faktor}

\usepackage{IEEEtrantools}
\usepackage{enumerate}   
\usepackage[PostScript=dvips]{"/Users/aware/Documents/Courses/diagrams"}


\newtheorem{theorem}{Théorème}[section]
\renewcommand{\thetheorem}{\arabic{theorem}}
\newtheorem{lemme}{Lemme}[section]
\renewcommand{\thelemme}{\arabic{lemme}}
\newtheorem{proposition}{Proposition}[section]
\renewcommand{\theproposition}{\arabic{proposition}}
\newtheorem{notations}{Notations}[section]
\newtheorem{problem}{Problème}[section]
\newtheorem{corollary}{Corollaire}[theorem]
\renewcommand{\thecorollary}{\arabic{corollary}}
\newtheorem{property}{Propriété}[section]
\newtheorem{objective}{Objectif}[section]

\theoremstyle{definition}
\newtheorem{definition}{Définition}[section]
\renewcommand{\thedefinition}{\arabic{definition}}
\newtheorem{exercise}{Exercice}[chapter]
\renewcommand{\theexercise}{\arabic{exercise}}
\newtheorem{example}{Exemple}[chapter]
\renewcommand{\theexample}{\arabic{example}}
\newtheorem*{solution}{Solution}
\newtheorem*{application}{Application}
\newtheorem*{notation}{Notation}
\newtheorem*{vocabulary}{Vocabulaire}
\newtheorem*{properties}{Propriétés}



\theoremstyle{remark}
\newtheorem*{remark}{Remarque}
\newtheorem*{rappel}{Rappel}


\usepackage{etoolbox}
\AtBeginEnvironment{exercise}{\small}
\AtBeginEnvironment{example}{\small}

\usepackage{cases}
\usepackage[red]{mypack}

\usepackage[framemethod=TikZ]{mdframed}

\definecolor{bg}{rgb}{0.4,0.25,0.95}
\definecolor{pagebg}{rgb}{0,0,0.5}
\surroundwithmdframed[
   topline=false,
   rightline=false,
   bottomline=false,
   leftmargin=\parindent,
   skipabove=8pt,
   skipbelow=8pt,
   linecolor=blue,
   innerbottommargin=10pt,
   % backgroundcolor=bg,font=\color{orange}\sffamily, fontcolor=white
]{definition}

\usepackage{empheq}
\usepackage[most]{tcolorbox}

\newtcbox{\mymath}[1][]{%
    nobeforeafter, math upper, tcbox raise base,
    enhanced, colframe=blue!30!black,
    colback=red!10, boxrule=1pt,
    #1}

\usepackage{unixode}


\DeclareMathOperator{\ord}{ord}
\DeclareMathOperator{\orb}{orb}
\DeclareMathOperator{\stab}{stab}
\DeclareMathOperator{\Stab}{stab}
\DeclareMathOperator{\ppcm}{ppcm}
\DeclareMathOperator{\conj}{Conj}
\DeclareMathOperator{\End}{End}
\DeclareMathOperator{\rot}{rot}
\DeclareMathOperator{\trs}{trace}
\DeclareMathOperator{\Ind}{Ind}
\DeclareMathOperator{\mat}{Mat}
\DeclareMathOperator{\id}{Id}
\DeclareMathOperator{\vect}{vect}
\DeclareMathOperator{\img}{img}
\DeclareMathOperator{\cov}{Cov}
\DeclareMathOperator{\dist}{dist}
\DeclareMathOperator{\irr}{Irr}
\DeclareMathOperator{\image}{Im}
\DeclareMathOperator{\pd}{\partial}
\DeclareMathOperator{\epi}{epi}
\DeclareMathOperator{\Argmin}{Argmin}
\DeclareMathOperator{\dom}{dom}
\DeclareMathOperator{\proj}{proj}
\DeclareMathOperator{\ctg}{ctg}
\DeclareMathOperator{\supp}{supp}
\DeclareMathOperator{\argmin}{argmin}
\DeclareMathOperator{\mult}{mult}
\DeclareMathOperator{\ch}{ch}
\DeclareMathOperator{\sh}{sh}
\DeclareMathOperator{\rang}{rang}
\DeclareMathOperator{\diam}{diam}
\DeclareMathOperator{\Epigraphe}{Epigraphe}




\usepackage{xcolor}
\everymath{\color{blue}}
%\everymath{\color[rgb]{0,1,1}}
%\pagecolor[rgb]{0,0,0.5}


\newcommand*{\pdtest}[3][]{\ensuremath{\frac{\partial^{#1} #2}{\partial #3}}}

\newcommand*{\deffunc}[6][]{\ensuremath{
\begin{array}{rcl}
#2 : #3 &\rightarrow& #4\\
#5 &\mapsto& #6
\end{array}
}}

\newcommand{\eqcolon}{\mathrel{\resizebox{\widthof{$\mathord{=}$}}{\height}{ $\!\!=\!\!\resizebox{1.2\width}{0.8\height}{\raisebox{0.23ex}{$\mathop{:}$}}\!\!$ }}}
\newcommand{\coloneq}{\mathrel{\resizebox{\widthof{$\mathord{=}$}}{\height}{ $\!\!\resizebox{1.2\width}{0.8\height}{\raisebox{0.23ex}{$\mathop{:}$}}\!\!=\!\!$ }}}
\newcommand{\eqcolonl}{\ensuremath{\mathrel{=\!\!\mathop{:}}}}
\newcommand{\coloneql}{\ensuremath{\mathrel{\mathop{:} \!\! =}}}
\newcommand{\vc}[1]{% inline column vector
  \left(\begin{smallmatrix}#1\end{smallmatrix}\right)%
}
\newcommand{\vr}[1]{% inline row vector
  \begin{smallmatrix}(\,#1\,)\end{smallmatrix}%
}
\makeatletter
\newcommand*{\defeq}{\ =\mathrel{\rlap{%
                     \raisebox{0.3ex}{$\m@th\cdot$}}%
                     \raisebox{-0.3ex}{$\m@th\cdot$}}%
                     }
\makeatother

\newcommand{\mathcircle}[1]{% inline row vector
 \overset{\circ}{#1}
}
\newcommand{\ulim}{% low limit
 \underline{\lim}
}
\newcommand{\ssi}{% iff
\iff
}
\newcommand{\ps}[2]{
\expval{#1 | #2}
}
\newcommand{\df}[1]{
\mqty{#1}
}
\newcommand{\n}[1]{
\norm{#1}
}
\newcommand{\sys}[1]{
\left\{\smqty{#1}\right.
}


\newcommand{\eqdef}{\ensuremath{\overset{\text{def}}=}}


\def\Circlearrowright{\ensuremath{%
  \rotatebox[origin=c]{230}{$\circlearrowright$}}}

\newcommand\ct[1]{\text{\rmfamily\upshape #1}}
\newcommand\question[1]{ {\color{red} ...!? \small #1}}
\newcommand\caz[1]{\left\{\begin{array} #1 \end{array}\right.}
\newcommand\const{\text{\rmfamily\upshape const}}
\newcommand\toP{ \overset{\pro}{\to}}
\newcommand\toPP{ \overset{\text{PP}}{\to}}
\newcommand{\oeq}{\mathrel{\text{\textcircled{$=$}}}}





\usepackage{xcolor}
% \usepackage[normalem]{ulem}
\usepackage{lipsum}
\makeatletter
% \newcommand\colorwave[1][blue]{\bgroup \markoverwith{\lower3.5\p@\hbox{\sixly \textcolor{#1}{\char58}}}\ULon}
%\font\sixly=lasy6 % does not re-load if already loaded, so no memory problem.

\newmdtheoremenv[
linewidth= 1pt,linecolor= blue,%
leftmargin=20,rightmargin=20,innertopmargin=0pt, innerrightmargin=40,%
tikzsetting = { draw=lightgray, line width = 0.3pt,dashed,%
dash pattern = on 15pt off 3pt},%
splittopskip=\topskip,skipbelow=\baselineskip,%
skipabove=\baselineskip,ntheorem,roundcorner=0pt,
% backgroundcolor=pagebg,font=\color{orange}\sffamily, fontcolor=white
]{examplebox}{Exemple}[section]



\newcommand\R{\mathbb{R}}
\newcommand\Z{\mathbb{Z}}
\newcommand\N{\mathbb{N}}
\newcommand\E{\mathbb{E}}
\newcommand\F{\mathcal{F}}
\newcommand\cH{\mathcal{H}}
\newcommand\V{\mathbb{V}}
\newcommand\dmo{ ^{-1} }
\newcommand\kapa{\kappa}
\newcommand\im{Im}
\newcommand\hs{\mathcal{H}}





\usepackage{soul}

\makeatletter
\newcommand*{\whiten}[1]{\llap{\textcolor{white}{{\the\SOUL@token}}\hspace{#1pt}}}
\DeclareRobustCommand*\myul{%
    \def\SOUL@everyspace{\underline{\space}\kern\z@}%
    \def\SOUL@everytoken{%
     \setbox0=\hbox{\the\SOUL@token}%
     \ifdim\dp0>\z@
        \raisebox{\dp0}{\underline{\phantom{\the\SOUL@token}}}%
        \whiten{1}\whiten{0}%
        \whiten{-1}\whiten{-2}%
        \llap{\the\SOUL@token}%
     \else
        \underline{\the\SOUL@token}%
     \fi}%
\SOUL@}
\makeatother

\newcommand*{\demp}{\fontfamily{lmtt}\selectfont}

\DeclareTextFontCommand{\textdemp}{\demp}

\begin{document}

\ifcomment
Multiline
comment
\fi
\ifcomment
\myul{Typesetting test}
% \color[rgb]{1,1,1}
$∑_i^n≠ 60º±∞π∆¬≈√j∫h≤≥µ$

$\CR \R\pro\ind\pro\gS\pro
\mqty[a&b\\c&d]$
$\pro\mathbb{P}$
$\dd{x}$

  \[
    \alpha(x)=\left\{
                \begin{array}{ll}
                  x\\
                  \frac{1}{1+e^{-kx}}\\
                  \frac{e^x-e^{-x}}{e^x+e^{-x}}
                \end{array}
              \right.
  \]

  $\expval{x}$
  
  $\chi_\rho(ghg\dmo)=\Tr(\rho_{ghg\dmo})=\Tr(\rho_g\circ\rho_h\circ\rho\dmo_g)=\Tr(\rho_h)\overset{\mbox{\scalebox{0.5}{$\Tr(AB)=\Tr(BA)$}}}{=}\chi_\rho(h)$
  	$\mathop{\oplus}_{\substack{x\in X}}$

$\mat(\rho_g)=(a_{ij}(g))_{\scriptsize \substack{1\leq i\leq d \\ 1\leq j\leq d}}$ et $\mat(\rho'_g)=(a'_{ij}(g))_{\scriptsize \substack{1\leq i'\leq d' \\ 1\leq j'\leq d'}}$



\[\int_a^b{\mathbb{R}^2}g(u, v)\dd{P_{XY}}(u, v)=\iint g(u,v) f_{XY}(u, v)\dd \lambda(u) \dd \lambda(v)\]
$$\lim_{x\to\infty} f(x)$$	
$$\iiiint_V \mu(t,u,v,w) \,dt\,du\,dv\,dw$$
$$\sum_{n=1}^{\infty} 2^{-n} = 1$$	
\begin{definition}
	Si $X$ et $Y$ sont 2 v.a. ou definit la \textsc{Covariance} entre $X$ et $Y$ comme
	$\cov(X,Y)\overset{\text{def}}{=}\E\left[(X-\E(X))(Y-\E(Y))\right]=\E(XY)-\E(X)\E(Y)$.
\end{definition}
\fi
\pagebreak

% \tableofcontents

% insert your code here
%\input{./algebra/main.tex}
%\input{./geometrie-differentielle/main.tex}
%\input{./probabilite/main.tex}
%\input{./analyse-fonctionnelle/main.tex}
% \input{./Analyse-convexe-et-dualite-en-optimisation/main.tex}
%\input{./tikz/main.tex}
%\input{./Theorie-du-distributions/main.tex}
%\input{./optimisation/mine.tex}
 \input{./modelisation/main.tex}

% yves.aubry@univ-tln.fr : algebra

\end{document}

%% !TEX encoding = UTF-8 Unicode
% !TEX TS-program = xelatex

\documentclass[french]{report}

%\usepackage[utf8]{inputenc}
%\usepackage[T1]{fontenc}
\usepackage{babel}


\newif\ifcomment
%\commenttrue # Show comments

\usepackage{physics}
\usepackage{amssymb}


\usepackage{amsthm}
% \usepackage{thmtools}
\usepackage{mathtools}
\usepackage{amsfonts}

\usepackage{color}

\usepackage{tikz}

\usepackage{geometry}
\geometry{a5paper, margin=0.1in, right=1cm}

\usepackage{dsfont}

\usepackage{graphicx}
\graphicspath{ {images/} }

\usepackage{faktor}

\usepackage{IEEEtrantools}
\usepackage{enumerate}   
\usepackage[PostScript=dvips]{"/Users/aware/Documents/Courses/diagrams"}


\newtheorem{theorem}{Théorème}[section]
\renewcommand{\thetheorem}{\arabic{theorem}}
\newtheorem{lemme}{Lemme}[section]
\renewcommand{\thelemme}{\arabic{lemme}}
\newtheorem{proposition}{Proposition}[section]
\renewcommand{\theproposition}{\arabic{proposition}}
\newtheorem{notations}{Notations}[section]
\newtheorem{problem}{Problème}[section]
\newtheorem{corollary}{Corollaire}[theorem]
\renewcommand{\thecorollary}{\arabic{corollary}}
\newtheorem{property}{Propriété}[section]
\newtheorem{objective}{Objectif}[section]

\theoremstyle{definition}
\newtheorem{definition}{Définition}[section]
\renewcommand{\thedefinition}{\arabic{definition}}
\newtheorem{exercise}{Exercice}[chapter]
\renewcommand{\theexercise}{\arabic{exercise}}
\newtheorem{example}{Exemple}[chapter]
\renewcommand{\theexample}{\arabic{example}}
\newtheorem*{solution}{Solution}
\newtheorem*{application}{Application}
\newtheorem*{notation}{Notation}
\newtheorem*{vocabulary}{Vocabulaire}
\newtheorem*{properties}{Propriétés}



\theoremstyle{remark}
\newtheorem*{remark}{Remarque}
\newtheorem*{rappel}{Rappel}


\usepackage{etoolbox}
\AtBeginEnvironment{exercise}{\small}
\AtBeginEnvironment{example}{\small}

\usepackage{cases}
\usepackage[red]{mypack}

\usepackage[framemethod=TikZ]{mdframed}

\definecolor{bg}{rgb}{0.4,0.25,0.95}
\definecolor{pagebg}{rgb}{0,0,0.5}
\surroundwithmdframed[
   topline=false,
   rightline=false,
   bottomline=false,
   leftmargin=\parindent,
   skipabove=8pt,
   skipbelow=8pt,
   linecolor=blue,
   innerbottommargin=10pt,
   % backgroundcolor=bg,font=\color{orange}\sffamily, fontcolor=white
]{definition}

\usepackage{empheq}
\usepackage[most]{tcolorbox}

\newtcbox{\mymath}[1][]{%
    nobeforeafter, math upper, tcbox raise base,
    enhanced, colframe=blue!30!black,
    colback=red!10, boxrule=1pt,
    #1}

\usepackage{unixode}


\DeclareMathOperator{\ord}{ord}
\DeclareMathOperator{\orb}{orb}
\DeclareMathOperator{\stab}{stab}
\DeclareMathOperator{\Stab}{stab}
\DeclareMathOperator{\ppcm}{ppcm}
\DeclareMathOperator{\conj}{Conj}
\DeclareMathOperator{\End}{End}
\DeclareMathOperator{\rot}{rot}
\DeclareMathOperator{\trs}{trace}
\DeclareMathOperator{\Ind}{Ind}
\DeclareMathOperator{\mat}{Mat}
\DeclareMathOperator{\id}{Id}
\DeclareMathOperator{\vect}{vect}
\DeclareMathOperator{\img}{img}
\DeclareMathOperator{\cov}{Cov}
\DeclareMathOperator{\dist}{dist}
\DeclareMathOperator{\irr}{Irr}
\DeclareMathOperator{\image}{Im}
\DeclareMathOperator{\pd}{\partial}
\DeclareMathOperator{\epi}{epi}
\DeclareMathOperator{\Argmin}{Argmin}
\DeclareMathOperator{\dom}{dom}
\DeclareMathOperator{\proj}{proj}
\DeclareMathOperator{\ctg}{ctg}
\DeclareMathOperator{\supp}{supp}
\DeclareMathOperator{\argmin}{argmin}
\DeclareMathOperator{\mult}{mult}
\DeclareMathOperator{\ch}{ch}
\DeclareMathOperator{\sh}{sh}
\DeclareMathOperator{\rang}{rang}
\DeclareMathOperator{\diam}{diam}
\DeclareMathOperator{\Epigraphe}{Epigraphe}




\usepackage{xcolor}
\everymath{\color{blue}}
%\everymath{\color[rgb]{0,1,1}}
%\pagecolor[rgb]{0,0,0.5}


\newcommand*{\pdtest}[3][]{\ensuremath{\frac{\partial^{#1} #2}{\partial #3}}}

\newcommand*{\deffunc}[6][]{\ensuremath{
\begin{array}{rcl}
#2 : #3 &\rightarrow& #4\\
#5 &\mapsto& #6
\end{array}
}}

\newcommand{\eqcolon}{\mathrel{\resizebox{\widthof{$\mathord{=}$}}{\height}{ $\!\!=\!\!\resizebox{1.2\width}{0.8\height}{\raisebox{0.23ex}{$\mathop{:}$}}\!\!$ }}}
\newcommand{\coloneq}{\mathrel{\resizebox{\widthof{$\mathord{=}$}}{\height}{ $\!\!\resizebox{1.2\width}{0.8\height}{\raisebox{0.23ex}{$\mathop{:}$}}\!\!=\!\!$ }}}
\newcommand{\eqcolonl}{\ensuremath{\mathrel{=\!\!\mathop{:}}}}
\newcommand{\coloneql}{\ensuremath{\mathrel{\mathop{:} \!\! =}}}
\newcommand{\vc}[1]{% inline column vector
  \left(\begin{smallmatrix}#1\end{smallmatrix}\right)%
}
\newcommand{\vr}[1]{% inline row vector
  \begin{smallmatrix}(\,#1\,)\end{smallmatrix}%
}
\makeatletter
\newcommand*{\defeq}{\ =\mathrel{\rlap{%
                     \raisebox{0.3ex}{$\m@th\cdot$}}%
                     \raisebox{-0.3ex}{$\m@th\cdot$}}%
                     }
\makeatother

\newcommand{\mathcircle}[1]{% inline row vector
 \overset{\circ}{#1}
}
\newcommand{\ulim}{% low limit
 \underline{\lim}
}
\newcommand{\ssi}{% iff
\iff
}
\newcommand{\ps}[2]{
\expval{#1 | #2}
}
\newcommand{\df}[1]{
\mqty{#1}
}
\newcommand{\n}[1]{
\norm{#1}
}
\newcommand{\sys}[1]{
\left\{\smqty{#1}\right.
}


\newcommand{\eqdef}{\ensuremath{\overset{\text{def}}=}}


\def\Circlearrowright{\ensuremath{%
  \rotatebox[origin=c]{230}{$\circlearrowright$}}}

\newcommand\ct[1]{\text{\rmfamily\upshape #1}}
\newcommand\question[1]{ {\color{red} ...!? \small #1}}
\newcommand\caz[1]{\left\{\begin{array} #1 \end{array}\right.}
\newcommand\const{\text{\rmfamily\upshape const}}
\newcommand\toP{ \overset{\pro}{\to}}
\newcommand\toPP{ \overset{\text{PP}}{\to}}
\newcommand{\oeq}{\mathrel{\text{\textcircled{$=$}}}}





\usepackage{xcolor}
% \usepackage[normalem]{ulem}
\usepackage{lipsum}
\makeatletter
% \newcommand\colorwave[1][blue]{\bgroup \markoverwith{\lower3.5\p@\hbox{\sixly \textcolor{#1}{\char58}}}\ULon}
%\font\sixly=lasy6 % does not re-load if already loaded, so no memory problem.

\newmdtheoremenv[
linewidth= 1pt,linecolor= blue,%
leftmargin=20,rightmargin=20,innertopmargin=0pt, innerrightmargin=40,%
tikzsetting = { draw=lightgray, line width = 0.3pt,dashed,%
dash pattern = on 15pt off 3pt},%
splittopskip=\topskip,skipbelow=\baselineskip,%
skipabove=\baselineskip,ntheorem,roundcorner=0pt,
% backgroundcolor=pagebg,font=\color{orange}\sffamily, fontcolor=white
]{examplebox}{Exemple}[section]



\newcommand\R{\mathbb{R}}
\newcommand\Z{\mathbb{Z}}
\newcommand\N{\mathbb{N}}
\newcommand\E{\mathbb{E}}
\newcommand\F{\mathcal{F}}
\newcommand\cH{\mathcal{H}}
\newcommand\V{\mathbb{V}}
\newcommand\dmo{ ^{-1} }
\newcommand\kapa{\kappa}
\newcommand\im{Im}
\newcommand\hs{\mathcal{H}}





\usepackage{soul}

\makeatletter
\newcommand*{\whiten}[1]{\llap{\textcolor{white}{{\the\SOUL@token}}\hspace{#1pt}}}
\DeclareRobustCommand*\myul{%
    \def\SOUL@everyspace{\underline{\space}\kern\z@}%
    \def\SOUL@everytoken{%
     \setbox0=\hbox{\the\SOUL@token}%
     \ifdim\dp0>\z@
        \raisebox{\dp0}{\underline{\phantom{\the\SOUL@token}}}%
        \whiten{1}\whiten{0}%
        \whiten{-1}\whiten{-2}%
        \llap{\the\SOUL@token}%
     \else
        \underline{\the\SOUL@token}%
     \fi}%
\SOUL@}
\makeatother

\newcommand*{\demp}{\fontfamily{lmtt}\selectfont}

\DeclareTextFontCommand{\textdemp}{\demp}

\begin{document}

\ifcomment
Multiline
comment
\fi
\ifcomment
\myul{Typesetting test}
% \color[rgb]{1,1,1}
$∑_i^n≠ 60º±∞π∆¬≈√j∫h≤≥µ$

$\CR \R\pro\ind\pro\gS\pro
\mqty[a&b\\c&d]$
$\pro\mathbb{P}$
$\dd{x}$

  \[
    \alpha(x)=\left\{
                \begin{array}{ll}
                  x\\
                  \frac{1}{1+e^{-kx}}\\
                  \frac{e^x-e^{-x}}{e^x+e^{-x}}
                \end{array}
              \right.
  \]

  $\expval{x}$
  
  $\chi_\rho(ghg\dmo)=\Tr(\rho_{ghg\dmo})=\Tr(\rho_g\circ\rho_h\circ\rho\dmo_g)=\Tr(\rho_h)\overset{\mbox{\scalebox{0.5}{$\Tr(AB)=\Tr(BA)$}}}{=}\chi_\rho(h)$
  	$\mathop{\oplus}_{\substack{x\in X}}$

$\mat(\rho_g)=(a_{ij}(g))_{\scriptsize \substack{1\leq i\leq d \\ 1\leq j\leq d}}$ et $\mat(\rho'_g)=(a'_{ij}(g))_{\scriptsize \substack{1\leq i'\leq d' \\ 1\leq j'\leq d'}}$



\[\int_a^b{\mathbb{R}^2}g(u, v)\dd{P_{XY}}(u, v)=\iint g(u,v) f_{XY}(u, v)\dd \lambda(u) \dd \lambda(v)\]
$$\lim_{x\to\infty} f(x)$$	
$$\iiiint_V \mu(t,u,v,w) \,dt\,du\,dv\,dw$$
$$\sum_{n=1}^{\infty} 2^{-n} = 1$$	
\begin{definition}
	Si $X$ et $Y$ sont 2 v.a. ou definit la \textsc{Covariance} entre $X$ et $Y$ comme
	$\cov(X,Y)\overset{\text{def}}{=}\E\left[(X-\E(X))(Y-\E(Y))\right]=\E(XY)-\E(X)\E(Y)$.
\end{definition}
\fi
\pagebreak

% \tableofcontents

% insert your code here
%\input{./algebra/main.tex}
%\input{./geometrie-differentielle/main.tex}
%\input{./probabilite/main.tex}
%\input{./analyse-fonctionnelle/main.tex}
% \input{./Analyse-convexe-et-dualite-en-optimisation/main.tex}
%\input{./tikz/main.tex}
%\input{./Theorie-du-distributions/main.tex}
%\input{./optimisation/mine.tex}
 \input{./modelisation/main.tex}

% yves.aubry@univ-tln.fr : algebra

\end{document}

% % !TEX encoding = UTF-8 Unicode
% !TEX TS-program = xelatex

\documentclass[french]{report}

%\usepackage[utf8]{inputenc}
%\usepackage[T1]{fontenc}
\usepackage{babel}


\newif\ifcomment
%\commenttrue # Show comments

\usepackage{physics}
\usepackage{amssymb}


\usepackage{amsthm}
% \usepackage{thmtools}
\usepackage{mathtools}
\usepackage{amsfonts}

\usepackage{color}

\usepackage{tikz}

\usepackage{geometry}
\geometry{a5paper, margin=0.1in, right=1cm}

\usepackage{dsfont}

\usepackage{graphicx}
\graphicspath{ {images/} }

\usepackage{faktor}

\usepackage{IEEEtrantools}
\usepackage{enumerate}   
\usepackage[PostScript=dvips]{"/Users/aware/Documents/Courses/diagrams"}


\newtheorem{theorem}{Théorème}[section]
\renewcommand{\thetheorem}{\arabic{theorem}}
\newtheorem{lemme}{Lemme}[section]
\renewcommand{\thelemme}{\arabic{lemme}}
\newtheorem{proposition}{Proposition}[section]
\renewcommand{\theproposition}{\arabic{proposition}}
\newtheorem{notations}{Notations}[section]
\newtheorem{problem}{Problème}[section]
\newtheorem{corollary}{Corollaire}[theorem]
\renewcommand{\thecorollary}{\arabic{corollary}}
\newtheorem{property}{Propriété}[section]
\newtheorem{objective}{Objectif}[section]

\theoremstyle{definition}
\newtheorem{definition}{Définition}[section]
\renewcommand{\thedefinition}{\arabic{definition}}
\newtheorem{exercise}{Exercice}[chapter]
\renewcommand{\theexercise}{\arabic{exercise}}
\newtheorem{example}{Exemple}[chapter]
\renewcommand{\theexample}{\arabic{example}}
\newtheorem*{solution}{Solution}
\newtheorem*{application}{Application}
\newtheorem*{notation}{Notation}
\newtheorem*{vocabulary}{Vocabulaire}
\newtheorem*{properties}{Propriétés}



\theoremstyle{remark}
\newtheorem*{remark}{Remarque}
\newtheorem*{rappel}{Rappel}


\usepackage{etoolbox}
\AtBeginEnvironment{exercise}{\small}
\AtBeginEnvironment{example}{\small}

\usepackage{cases}
\usepackage[red]{mypack}

\usepackage[framemethod=TikZ]{mdframed}

\definecolor{bg}{rgb}{0.4,0.25,0.95}
\definecolor{pagebg}{rgb}{0,0,0.5}
\surroundwithmdframed[
   topline=false,
   rightline=false,
   bottomline=false,
   leftmargin=\parindent,
   skipabove=8pt,
   skipbelow=8pt,
   linecolor=blue,
   innerbottommargin=10pt,
   % backgroundcolor=bg,font=\color{orange}\sffamily, fontcolor=white
]{definition}

\usepackage{empheq}
\usepackage[most]{tcolorbox}

\newtcbox{\mymath}[1][]{%
    nobeforeafter, math upper, tcbox raise base,
    enhanced, colframe=blue!30!black,
    colback=red!10, boxrule=1pt,
    #1}

\usepackage{unixode}


\DeclareMathOperator{\ord}{ord}
\DeclareMathOperator{\orb}{orb}
\DeclareMathOperator{\stab}{stab}
\DeclareMathOperator{\Stab}{stab}
\DeclareMathOperator{\ppcm}{ppcm}
\DeclareMathOperator{\conj}{Conj}
\DeclareMathOperator{\End}{End}
\DeclareMathOperator{\rot}{rot}
\DeclareMathOperator{\trs}{trace}
\DeclareMathOperator{\Ind}{Ind}
\DeclareMathOperator{\mat}{Mat}
\DeclareMathOperator{\id}{Id}
\DeclareMathOperator{\vect}{vect}
\DeclareMathOperator{\img}{img}
\DeclareMathOperator{\cov}{Cov}
\DeclareMathOperator{\dist}{dist}
\DeclareMathOperator{\irr}{Irr}
\DeclareMathOperator{\image}{Im}
\DeclareMathOperator{\pd}{\partial}
\DeclareMathOperator{\epi}{epi}
\DeclareMathOperator{\Argmin}{Argmin}
\DeclareMathOperator{\dom}{dom}
\DeclareMathOperator{\proj}{proj}
\DeclareMathOperator{\ctg}{ctg}
\DeclareMathOperator{\supp}{supp}
\DeclareMathOperator{\argmin}{argmin}
\DeclareMathOperator{\mult}{mult}
\DeclareMathOperator{\ch}{ch}
\DeclareMathOperator{\sh}{sh}
\DeclareMathOperator{\rang}{rang}
\DeclareMathOperator{\diam}{diam}
\DeclareMathOperator{\Epigraphe}{Epigraphe}




\usepackage{xcolor}
\everymath{\color{blue}}
%\everymath{\color[rgb]{0,1,1}}
%\pagecolor[rgb]{0,0,0.5}


\newcommand*{\pdtest}[3][]{\ensuremath{\frac{\partial^{#1} #2}{\partial #3}}}

\newcommand*{\deffunc}[6][]{\ensuremath{
\begin{array}{rcl}
#2 : #3 &\rightarrow& #4\\
#5 &\mapsto& #6
\end{array}
}}

\newcommand{\eqcolon}{\mathrel{\resizebox{\widthof{$\mathord{=}$}}{\height}{ $\!\!=\!\!\resizebox{1.2\width}{0.8\height}{\raisebox{0.23ex}{$\mathop{:}$}}\!\!$ }}}
\newcommand{\coloneq}{\mathrel{\resizebox{\widthof{$\mathord{=}$}}{\height}{ $\!\!\resizebox{1.2\width}{0.8\height}{\raisebox{0.23ex}{$\mathop{:}$}}\!\!=\!\!$ }}}
\newcommand{\eqcolonl}{\ensuremath{\mathrel{=\!\!\mathop{:}}}}
\newcommand{\coloneql}{\ensuremath{\mathrel{\mathop{:} \!\! =}}}
\newcommand{\vc}[1]{% inline column vector
  \left(\begin{smallmatrix}#1\end{smallmatrix}\right)%
}
\newcommand{\vr}[1]{% inline row vector
  \begin{smallmatrix}(\,#1\,)\end{smallmatrix}%
}
\makeatletter
\newcommand*{\defeq}{\ =\mathrel{\rlap{%
                     \raisebox{0.3ex}{$\m@th\cdot$}}%
                     \raisebox{-0.3ex}{$\m@th\cdot$}}%
                     }
\makeatother

\newcommand{\mathcircle}[1]{% inline row vector
 \overset{\circ}{#1}
}
\newcommand{\ulim}{% low limit
 \underline{\lim}
}
\newcommand{\ssi}{% iff
\iff
}
\newcommand{\ps}[2]{
\expval{#1 | #2}
}
\newcommand{\df}[1]{
\mqty{#1}
}
\newcommand{\n}[1]{
\norm{#1}
}
\newcommand{\sys}[1]{
\left\{\smqty{#1}\right.
}


\newcommand{\eqdef}{\ensuremath{\overset{\text{def}}=}}


\def\Circlearrowright{\ensuremath{%
  \rotatebox[origin=c]{230}{$\circlearrowright$}}}

\newcommand\ct[1]{\text{\rmfamily\upshape #1}}
\newcommand\question[1]{ {\color{red} ...!? \small #1}}
\newcommand\caz[1]{\left\{\begin{array} #1 \end{array}\right.}
\newcommand\const{\text{\rmfamily\upshape const}}
\newcommand\toP{ \overset{\pro}{\to}}
\newcommand\toPP{ \overset{\text{PP}}{\to}}
\newcommand{\oeq}{\mathrel{\text{\textcircled{$=$}}}}





\usepackage{xcolor}
% \usepackage[normalem]{ulem}
\usepackage{lipsum}
\makeatletter
% \newcommand\colorwave[1][blue]{\bgroup \markoverwith{\lower3.5\p@\hbox{\sixly \textcolor{#1}{\char58}}}\ULon}
%\font\sixly=lasy6 % does not re-load if already loaded, so no memory problem.

\newmdtheoremenv[
linewidth= 1pt,linecolor= blue,%
leftmargin=20,rightmargin=20,innertopmargin=0pt, innerrightmargin=40,%
tikzsetting = { draw=lightgray, line width = 0.3pt,dashed,%
dash pattern = on 15pt off 3pt},%
splittopskip=\topskip,skipbelow=\baselineskip,%
skipabove=\baselineskip,ntheorem,roundcorner=0pt,
% backgroundcolor=pagebg,font=\color{orange}\sffamily, fontcolor=white
]{examplebox}{Exemple}[section]



\newcommand\R{\mathbb{R}}
\newcommand\Z{\mathbb{Z}}
\newcommand\N{\mathbb{N}}
\newcommand\E{\mathbb{E}}
\newcommand\F{\mathcal{F}}
\newcommand\cH{\mathcal{H}}
\newcommand\V{\mathbb{V}}
\newcommand\dmo{ ^{-1} }
\newcommand\kapa{\kappa}
\newcommand\im{Im}
\newcommand\hs{\mathcal{H}}





\usepackage{soul}

\makeatletter
\newcommand*{\whiten}[1]{\llap{\textcolor{white}{{\the\SOUL@token}}\hspace{#1pt}}}
\DeclareRobustCommand*\myul{%
    \def\SOUL@everyspace{\underline{\space}\kern\z@}%
    \def\SOUL@everytoken{%
     \setbox0=\hbox{\the\SOUL@token}%
     \ifdim\dp0>\z@
        \raisebox{\dp0}{\underline{\phantom{\the\SOUL@token}}}%
        \whiten{1}\whiten{0}%
        \whiten{-1}\whiten{-2}%
        \llap{\the\SOUL@token}%
     \else
        \underline{\the\SOUL@token}%
     \fi}%
\SOUL@}
\makeatother

\newcommand*{\demp}{\fontfamily{lmtt}\selectfont}

\DeclareTextFontCommand{\textdemp}{\demp}

\begin{document}

\ifcomment
Multiline
comment
\fi
\ifcomment
\myul{Typesetting test}
% \color[rgb]{1,1,1}
$∑_i^n≠ 60º±∞π∆¬≈√j∫h≤≥µ$

$\CR \R\pro\ind\pro\gS\pro
\mqty[a&b\\c&d]$
$\pro\mathbb{P}$
$\dd{x}$

  \[
    \alpha(x)=\left\{
                \begin{array}{ll}
                  x\\
                  \frac{1}{1+e^{-kx}}\\
                  \frac{e^x-e^{-x}}{e^x+e^{-x}}
                \end{array}
              \right.
  \]

  $\expval{x}$
  
  $\chi_\rho(ghg\dmo)=\Tr(\rho_{ghg\dmo})=\Tr(\rho_g\circ\rho_h\circ\rho\dmo_g)=\Tr(\rho_h)\overset{\mbox{\scalebox{0.5}{$\Tr(AB)=\Tr(BA)$}}}{=}\chi_\rho(h)$
  	$\mathop{\oplus}_{\substack{x\in X}}$

$\mat(\rho_g)=(a_{ij}(g))_{\scriptsize \substack{1\leq i\leq d \\ 1\leq j\leq d}}$ et $\mat(\rho'_g)=(a'_{ij}(g))_{\scriptsize \substack{1\leq i'\leq d' \\ 1\leq j'\leq d'}}$



\[\int_a^b{\mathbb{R}^2}g(u, v)\dd{P_{XY}}(u, v)=\iint g(u,v) f_{XY}(u, v)\dd \lambda(u) \dd \lambda(v)\]
$$\lim_{x\to\infty} f(x)$$	
$$\iiiint_V \mu(t,u,v,w) \,dt\,du\,dv\,dw$$
$$\sum_{n=1}^{\infty} 2^{-n} = 1$$	
\begin{definition}
	Si $X$ et $Y$ sont 2 v.a. ou definit la \textsc{Covariance} entre $X$ et $Y$ comme
	$\cov(X,Y)\overset{\text{def}}{=}\E\left[(X-\E(X))(Y-\E(Y))\right]=\E(XY)-\E(X)\E(Y)$.
\end{definition}
\fi
\pagebreak

% \tableofcontents

% insert your code here
%\input{./algebra/main.tex}
%\input{./geometrie-differentielle/main.tex}
%\input{./probabilite/main.tex}
%\input{./analyse-fonctionnelle/main.tex}
% \input{./Analyse-convexe-et-dualite-en-optimisation/main.tex}
%\input{./tikz/main.tex}
%\input{./Theorie-du-distributions/main.tex}
%\input{./optimisation/mine.tex}
 \input{./modelisation/main.tex}

% yves.aubry@univ-tln.fr : algebra

\end{document}

%% !TEX encoding = UTF-8 Unicode
% !TEX TS-program = xelatex

\documentclass[french]{report}

%\usepackage[utf8]{inputenc}
%\usepackage[T1]{fontenc}
\usepackage{babel}


\newif\ifcomment
%\commenttrue # Show comments

\usepackage{physics}
\usepackage{amssymb}


\usepackage{amsthm}
% \usepackage{thmtools}
\usepackage{mathtools}
\usepackage{amsfonts}

\usepackage{color}

\usepackage{tikz}

\usepackage{geometry}
\geometry{a5paper, margin=0.1in, right=1cm}

\usepackage{dsfont}

\usepackage{graphicx}
\graphicspath{ {images/} }

\usepackage{faktor}

\usepackage{IEEEtrantools}
\usepackage{enumerate}   
\usepackage[PostScript=dvips]{"/Users/aware/Documents/Courses/diagrams"}


\newtheorem{theorem}{Théorème}[section]
\renewcommand{\thetheorem}{\arabic{theorem}}
\newtheorem{lemme}{Lemme}[section]
\renewcommand{\thelemme}{\arabic{lemme}}
\newtheorem{proposition}{Proposition}[section]
\renewcommand{\theproposition}{\arabic{proposition}}
\newtheorem{notations}{Notations}[section]
\newtheorem{problem}{Problème}[section]
\newtheorem{corollary}{Corollaire}[theorem]
\renewcommand{\thecorollary}{\arabic{corollary}}
\newtheorem{property}{Propriété}[section]
\newtheorem{objective}{Objectif}[section]

\theoremstyle{definition}
\newtheorem{definition}{Définition}[section]
\renewcommand{\thedefinition}{\arabic{definition}}
\newtheorem{exercise}{Exercice}[chapter]
\renewcommand{\theexercise}{\arabic{exercise}}
\newtheorem{example}{Exemple}[chapter]
\renewcommand{\theexample}{\arabic{example}}
\newtheorem*{solution}{Solution}
\newtheorem*{application}{Application}
\newtheorem*{notation}{Notation}
\newtheorem*{vocabulary}{Vocabulaire}
\newtheorem*{properties}{Propriétés}



\theoremstyle{remark}
\newtheorem*{remark}{Remarque}
\newtheorem*{rappel}{Rappel}


\usepackage{etoolbox}
\AtBeginEnvironment{exercise}{\small}
\AtBeginEnvironment{example}{\small}

\usepackage{cases}
\usepackage[red]{mypack}

\usepackage[framemethod=TikZ]{mdframed}

\definecolor{bg}{rgb}{0.4,0.25,0.95}
\definecolor{pagebg}{rgb}{0,0,0.5}
\surroundwithmdframed[
   topline=false,
   rightline=false,
   bottomline=false,
   leftmargin=\parindent,
   skipabove=8pt,
   skipbelow=8pt,
   linecolor=blue,
   innerbottommargin=10pt,
   % backgroundcolor=bg,font=\color{orange}\sffamily, fontcolor=white
]{definition}

\usepackage{empheq}
\usepackage[most]{tcolorbox}

\newtcbox{\mymath}[1][]{%
    nobeforeafter, math upper, tcbox raise base,
    enhanced, colframe=blue!30!black,
    colback=red!10, boxrule=1pt,
    #1}

\usepackage{unixode}


\DeclareMathOperator{\ord}{ord}
\DeclareMathOperator{\orb}{orb}
\DeclareMathOperator{\stab}{stab}
\DeclareMathOperator{\Stab}{stab}
\DeclareMathOperator{\ppcm}{ppcm}
\DeclareMathOperator{\conj}{Conj}
\DeclareMathOperator{\End}{End}
\DeclareMathOperator{\rot}{rot}
\DeclareMathOperator{\trs}{trace}
\DeclareMathOperator{\Ind}{Ind}
\DeclareMathOperator{\mat}{Mat}
\DeclareMathOperator{\id}{Id}
\DeclareMathOperator{\vect}{vect}
\DeclareMathOperator{\img}{img}
\DeclareMathOperator{\cov}{Cov}
\DeclareMathOperator{\dist}{dist}
\DeclareMathOperator{\irr}{Irr}
\DeclareMathOperator{\image}{Im}
\DeclareMathOperator{\pd}{\partial}
\DeclareMathOperator{\epi}{epi}
\DeclareMathOperator{\Argmin}{Argmin}
\DeclareMathOperator{\dom}{dom}
\DeclareMathOperator{\proj}{proj}
\DeclareMathOperator{\ctg}{ctg}
\DeclareMathOperator{\supp}{supp}
\DeclareMathOperator{\argmin}{argmin}
\DeclareMathOperator{\mult}{mult}
\DeclareMathOperator{\ch}{ch}
\DeclareMathOperator{\sh}{sh}
\DeclareMathOperator{\rang}{rang}
\DeclareMathOperator{\diam}{diam}
\DeclareMathOperator{\Epigraphe}{Epigraphe}




\usepackage{xcolor}
\everymath{\color{blue}}
%\everymath{\color[rgb]{0,1,1}}
%\pagecolor[rgb]{0,0,0.5}


\newcommand*{\pdtest}[3][]{\ensuremath{\frac{\partial^{#1} #2}{\partial #3}}}

\newcommand*{\deffunc}[6][]{\ensuremath{
\begin{array}{rcl}
#2 : #3 &\rightarrow& #4\\
#5 &\mapsto& #6
\end{array}
}}

\newcommand{\eqcolon}{\mathrel{\resizebox{\widthof{$\mathord{=}$}}{\height}{ $\!\!=\!\!\resizebox{1.2\width}{0.8\height}{\raisebox{0.23ex}{$\mathop{:}$}}\!\!$ }}}
\newcommand{\coloneq}{\mathrel{\resizebox{\widthof{$\mathord{=}$}}{\height}{ $\!\!\resizebox{1.2\width}{0.8\height}{\raisebox{0.23ex}{$\mathop{:}$}}\!\!=\!\!$ }}}
\newcommand{\eqcolonl}{\ensuremath{\mathrel{=\!\!\mathop{:}}}}
\newcommand{\coloneql}{\ensuremath{\mathrel{\mathop{:} \!\! =}}}
\newcommand{\vc}[1]{% inline column vector
  \left(\begin{smallmatrix}#1\end{smallmatrix}\right)%
}
\newcommand{\vr}[1]{% inline row vector
  \begin{smallmatrix}(\,#1\,)\end{smallmatrix}%
}
\makeatletter
\newcommand*{\defeq}{\ =\mathrel{\rlap{%
                     \raisebox{0.3ex}{$\m@th\cdot$}}%
                     \raisebox{-0.3ex}{$\m@th\cdot$}}%
                     }
\makeatother

\newcommand{\mathcircle}[1]{% inline row vector
 \overset{\circ}{#1}
}
\newcommand{\ulim}{% low limit
 \underline{\lim}
}
\newcommand{\ssi}{% iff
\iff
}
\newcommand{\ps}[2]{
\expval{#1 | #2}
}
\newcommand{\df}[1]{
\mqty{#1}
}
\newcommand{\n}[1]{
\norm{#1}
}
\newcommand{\sys}[1]{
\left\{\smqty{#1}\right.
}


\newcommand{\eqdef}{\ensuremath{\overset{\text{def}}=}}


\def\Circlearrowright{\ensuremath{%
  \rotatebox[origin=c]{230}{$\circlearrowright$}}}

\newcommand\ct[1]{\text{\rmfamily\upshape #1}}
\newcommand\question[1]{ {\color{red} ...!? \small #1}}
\newcommand\caz[1]{\left\{\begin{array} #1 \end{array}\right.}
\newcommand\const{\text{\rmfamily\upshape const}}
\newcommand\toP{ \overset{\pro}{\to}}
\newcommand\toPP{ \overset{\text{PP}}{\to}}
\newcommand{\oeq}{\mathrel{\text{\textcircled{$=$}}}}





\usepackage{xcolor}
% \usepackage[normalem]{ulem}
\usepackage{lipsum}
\makeatletter
% \newcommand\colorwave[1][blue]{\bgroup \markoverwith{\lower3.5\p@\hbox{\sixly \textcolor{#1}{\char58}}}\ULon}
%\font\sixly=lasy6 % does not re-load if already loaded, so no memory problem.

\newmdtheoremenv[
linewidth= 1pt,linecolor= blue,%
leftmargin=20,rightmargin=20,innertopmargin=0pt, innerrightmargin=40,%
tikzsetting = { draw=lightgray, line width = 0.3pt,dashed,%
dash pattern = on 15pt off 3pt},%
splittopskip=\topskip,skipbelow=\baselineskip,%
skipabove=\baselineskip,ntheorem,roundcorner=0pt,
% backgroundcolor=pagebg,font=\color{orange}\sffamily, fontcolor=white
]{examplebox}{Exemple}[section]



\newcommand\R{\mathbb{R}}
\newcommand\Z{\mathbb{Z}}
\newcommand\N{\mathbb{N}}
\newcommand\E{\mathbb{E}}
\newcommand\F{\mathcal{F}}
\newcommand\cH{\mathcal{H}}
\newcommand\V{\mathbb{V}}
\newcommand\dmo{ ^{-1} }
\newcommand\kapa{\kappa}
\newcommand\im{Im}
\newcommand\hs{\mathcal{H}}





\usepackage{soul}

\makeatletter
\newcommand*{\whiten}[1]{\llap{\textcolor{white}{{\the\SOUL@token}}\hspace{#1pt}}}
\DeclareRobustCommand*\myul{%
    \def\SOUL@everyspace{\underline{\space}\kern\z@}%
    \def\SOUL@everytoken{%
     \setbox0=\hbox{\the\SOUL@token}%
     \ifdim\dp0>\z@
        \raisebox{\dp0}{\underline{\phantom{\the\SOUL@token}}}%
        \whiten{1}\whiten{0}%
        \whiten{-1}\whiten{-2}%
        \llap{\the\SOUL@token}%
     \else
        \underline{\the\SOUL@token}%
     \fi}%
\SOUL@}
\makeatother

\newcommand*{\demp}{\fontfamily{lmtt}\selectfont}

\DeclareTextFontCommand{\textdemp}{\demp}

\begin{document}

\ifcomment
Multiline
comment
\fi
\ifcomment
\myul{Typesetting test}
% \color[rgb]{1,1,1}
$∑_i^n≠ 60º±∞π∆¬≈√j∫h≤≥µ$

$\CR \R\pro\ind\pro\gS\pro
\mqty[a&b\\c&d]$
$\pro\mathbb{P}$
$\dd{x}$

  \[
    \alpha(x)=\left\{
                \begin{array}{ll}
                  x\\
                  \frac{1}{1+e^{-kx}}\\
                  \frac{e^x-e^{-x}}{e^x+e^{-x}}
                \end{array}
              \right.
  \]

  $\expval{x}$
  
  $\chi_\rho(ghg\dmo)=\Tr(\rho_{ghg\dmo})=\Tr(\rho_g\circ\rho_h\circ\rho\dmo_g)=\Tr(\rho_h)\overset{\mbox{\scalebox{0.5}{$\Tr(AB)=\Tr(BA)$}}}{=}\chi_\rho(h)$
  	$\mathop{\oplus}_{\substack{x\in X}}$

$\mat(\rho_g)=(a_{ij}(g))_{\scriptsize \substack{1\leq i\leq d \\ 1\leq j\leq d}}$ et $\mat(\rho'_g)=(a'_{ij}(g))_{\scriptsize \substack{1\leq i'\leq d' \\ 1\leq j'\leq d'}}$



\[\int_a^b{\mathbb{R}^2}g(u, v)\dd{P_{XY}}(u, v)=\iint g(u,v) f_{XY}(u, v)\dd \lambda(u) \dd \lambda(v)\]
$$\lim_{x\to\infty} f(x)$$	
$$\iiiint_V \mu(t,u,v,w) \,dt\,du\,dv\,dw$$
$$\sum_{n=1}^{\infty} 2^{-n} = 1$$	
\begin{definition}
	Si $X$ et $Y$ sont 2 v.a. ou definit la \textsc{Covariance} entre $X$ et $Y$ comme
	$\cov(X,Y)\overset{\text{def}}{=}\E\left[(X-\E(X))(Y-\E(Y))\right]=\E(XY)-\E(X)\E(Y)$.
\end{definition}
\fi
\pagebreak

% \tableofcontents

% insert your code here
%\input{./algebra/main.tex}
%\input{./geometrie-differentielle/main.tex}
%\input{./probabilite/main.tex}
%\input{./analyse-fonctionnelle/main.tex}
% \input{./Analyse-convexe-et-dualite-en-optimisation/main.tex}
%\input{./tikz/main.tex}
%\input{./Theorie-du-distributions/main.tex}
%\input{./optimisation/mine.tex}
 \input{./modelisation/main.tex}

% yves.aubry@univ-tln.fr : algebra

\end{document}

%% !TEX encoding = UTF-8 Unicode
% !TEX TS-program = xelatex

\documentclass[french]{report}

%\usepackage[utf8]{inputenc}
%\usepackage[T1]{fontenc}
\usepackage{babel}


\newif\ifcomment
%\commenttrue # Show comments

\usepackage{physics}
\usepackage{amssymb}


\usepackage{amsthm}
% \usepackage{thmtools}
\usepackage{mathtools}
\usepackage{amsfonts}

\usepackage{color}

\usepackage{tikz}

\usepackage{geometry}
\geometry{a5paper, margin=0.1in, right=1cm}

\usepackage{dsfont}

\usepackage{graphicx}
\graphicspath{ {images/} }

\usepackage{faktor}

\usepackage{IEEEtrantools}
\usepackage{enumerate}   
\usepackage[PostScript=dvips]{"/Users/aware/Documents/Courses/diagrams"}


\newtheorem{theorem}{Théorème}[section]
\renewcommand{\thetheorem}{\arabic{theorem}}
\newtheorem{lemme}{Lemme}[section]
\renewcommand{\thelemme}{\arabic{lemme}}
\newtheorem{proposition}{Proposition}[section]
\renewcommand{\theproposition}{\arabic{proposition}}
\newtheorem{notations}{Notations}[section]
\newtheorem{problem}{Problème}[section]
\newtheorem{corollary}{Corollaire}[theorem]
\renewcommand{\thecorollary}{\arabic{corollary}}
\newtheorem{property}{Propriété}[section]
\newtheorem{objective}{Objectif}[section]

\theoremstyle{definition}
\newtheorem{definition}{Définition}[section]
\renewcommand{\thedefinition}{\arabic{definition}}
\newtheorem{exercise}{Exercice}[chapter]
\renewcommand{\theexercise}{\arabic{exercise}}
\newtheorem{example}{Exemple}[chapter]
\renewcommand{\theexample}{\arabic{example}}
\newtheorem*{solution}{Solution}
\newtheorem*{application}{Application}
\newtheorem*{notation}{Notation}
\newtheorem*{vocabulary}{Vocabulaire}
\newtheorem*{properties}{Propriétés}



\theoremstyle{remark}
\newtheorem*{remark}{Remarque}
\newtheorem*{rappel}{Rappel}


\usepackage{etoolbox}
\AtBeginEnvironment{exercise}{\small}
\AtBeginEnvironment{example}{\small}

\usepackage{cases}
\usepackage[red]{mypack}

\usepackage[framemethod=TikZ]{mdframed}

\definecolor{bg}{rgb}{0.4,0.25,0.95}
\definecolor{pagebg}{rgb}{0,0,0.5}
\surroundwithmdframed[
   topline=false,
   rightline=false,
   bottomline=false,
   leftmargin=\parindent,
   skipabove=8pt,
   skipbelow=8pt,
   linecolor=blue,
   innerbottommargin=10pt,
   % backgroundcolor=bg,font=\color{orange}\sffamily, fontcolor=white
]{definition}

\usepackage{empheq}
\usepackage[most]{tcolorbox}

\newtcbox{\mymath}[1][]{%
    nobeforeafter, math upper, tcbox raise base,
    enhanced, colframe=blue!30!black,
    colback=red!10, boxrule=1pt,
    #1}

\usepackage{unixode}


\DeclareMathOperator{\ord}{ord}
\DeclareMathOperator{\orb}{orb}
\DeclareMathOperator{\stab}{stab}
\DeclareMathOperator{\Stab}{stab}
\DeclareMathOperator{\ppcm}{ppcm}
\DeclareMathOperator{\conj}{Conj}
\DeclareMathOperator{\End}{End}
\DeclareMathOperator{\rot}{rot}
\DeclareMathOperator{\trs}{trace}
\DeclareMathOperator{\Ind}{Ind}
\DeclareMathOperator{\mat}{Mat}
\DeclareMathOperator{\id}{Id}
\DeclareMathOperator{\vect}{vect}
\DeclareMathOperator{\img}{img}
\DeclareMathOperator{\cov}{Cov}
\DeclareMathOperator{\dist}{dist}
\DeclareMathOperator{\irr}{Irr}
\DeclareMathOperator{\image}{Im}
\DeclareMathOperator{\pd}{\partial}
\DeclareMathOperator{\epi}{epi}
\DeclareMathOperator{\Argmin}{Argmin}
\DeclareMathOperator{\dom}{dom}
\DeclareMathOperator{\proj}{proj}
\DeclareMathOperator{\ctg}{ctg}
\DeclareMathOperator{\supp}{supp}
\DeclareMathOperator{\argmin}{argmin}
\DeclareMathOperator{\mult}{mult}
\DeclareMathOperator{\ch}{ch}
\DeclareMathOperator{\sh}{sh}
\DeclareMathOperator{\rang}{rang}
\DeclareMathOperator{\diam}{diam}
\DeclareMathOperator{\Epigraphe}{Epigraphe}




\usepackage{xcolor}
\everymath{\color{blue}}
%\everymath{\color[rgb]{0,1,1}}
%\pagecolor[rgb]{0,0,0.5}


\newcommand*{\pdtest}[3][]{\ensuremath{\frac{\partial^{#1} #2}{\partial #3}}}

\newcommand*{\deffunc}[6][]{\ensuremath{
\begin{array}{rcl}
#2 : #3 &\rightarrow& #4\\
#5 &\mapsto& #6
\end{array}
}}

\newcommand{\eqcolon}{\mathrel{\resizebox{\widthof{$\mathord{=}$}}{\height}{ $\!\!=\!\!\resizebox{1.2\width}{0.8\height}{\raisebox{0.23ex}{$\mathop{:}$}}\!\!$ }}}
\newcommand{\coloneq}{\mathrel{\resizebox{\widthof{$\mathord{=}$}}{\height}{ $\!\!\resizebox{1.2\width}{0.8\height}{\raisebox{0.23ex}{$\mathop{:}$}}\!\!=\!\!$ }}}
\newcommand{\eqcolonl}{\ensuremath{\mathrel{=\!\!\mathop{:}}}}
\newcommand{\coloneql}{\ensuremath{\mathrel{\mathop{:} \!\! =}}}
\newcommand{\vc}[1]{% inline column vector
  \left(\begin{smallmatrix}#1\end{smallmatrix}\right)%
}
\newcommand{\vr}[1]{% inline row vector
  \begin{smallmatrix}(\,#1\,)\end{smallmatrix}%
}
\makeatletter
\newcommand*{\defeq}{\ =\mathrel{\rlap{%
                     \raisebox{0.3ex}{$\m@th\cdot$}}%
                     \raisebox{-0.3ex}{$\m@th\cdot$}}%
                     }
\makeatother

\newcommand{\mathcircle}[1]{% inline row vector
 \overset{\circ}{#1}
}
\newcommand{\ulim}{% low limit
 \underline{\lim}
}
\newcommand{\ssi}{% iff
\iff
}
\newcommand{\ps}[2]{
\expval{#1 | #2}
}
\newcommand{\df}[1]{
\mqty{#1}
}
\newcommand{\n}[1]{
\norm{#1}
}
\newcommand{\sys}[1]{
\left\{\smqty{#1}\right.
}


\newcommand{\eqdef}{\ensuremath{\overset{\text{def}}=}}


\def\Circlearrowright{\ensuremath{%
  \rotatebox[origin=c]{230}{$\circlearrowright$}}}

\newcommand\ct[1]{\text{\rmfamily\upshape #1}}
\newcommand\question[1]{ {\color{red} ...!? \small #1}}
\newcommand\caz[1]{\left\{\begin{array} #1 \end{array}\right.}
\newcommand\const{\text{\rmfamily\upshape const}}
\newcommand\toP{ \overset{\pro}{\to}}
\newcommand\toPP{ \overset{\text{PP}}{\to}}
\newcommand{\oeq}{\mathrel{\text{\textcircled{$=$}}}}





\usepackage{xcolor}
% \usepackage[normalem]{ulem}
\usepackage{lipsum}
\makeatletter
% \newcommand\colorwave[1][blue]{\bgroup \markoverwith{\lower3.5\p@\hbox{\sixly \textcolor{#1}{\char58}}}\ULon}
%\font\sixly=lasy6 % does not re-load if already loaded, so no memory problem.

\newmdtheoremenv[
linewidth= 1pt,linecolor= blue,%
leftmargin=20,rightmargin=20,innertopmargin=0pt, innerrightmargin=40,%
tikzsetting = { draw=lightgray, line width = 0.3pt,dashed,%
dash pattern = on 15pt off 3pt},%
splittopskip=\topskip,skipbelow=\baselineskip,%
skipabove=\baselineskip,ntheorem,roundcorner=0pt,
% backgroundcolor=pagebg,font=\color{orange}\sffamily, fontcolor=white
]{examplebox}{Exemple}[section]



\newcommand\R{\mathbb{R}}
\newcommand\Z{\mathbb{Z}}
\newcommand\N{\mathbb{N}}
\newcommand\E{\mathbb{E}}
\newcommand\F{\mathcal{F}}
\newcommand\cH{\mathcal{H}}
\newcommand\V{\mathbb{V}}
\newcommand\dmo{ ^{-1} }
\newcommand\kapa{\kappa}
\newcommand\im{Im}
\newcommand\hs{\mathcal{H}}





\usepackage{soul}

\makeatletter
\newcommand*{\whiten}[1]{\llap{\textcolor{white}{{\the\SOUL@token}}\hspace{#1pt}}}
\DeclareRobustCommand*\myul{%
    \def\SOUL@everyspace{\underline{\space}\kern\z@}%
    \def\SOUL@everytoken{%
     \setbox0=\hbox{\the\SOUL@token}%
     \ifdim\dp0>\z@
        \raisebox{\dp0}{\underline{\phantom{\the\SOUL@token}}}%
        \whiten{1}\whiten{0}%
        \whiten{-1}\whiten{-2}%
        \llap{\the\SOUL@token}%
     \else
        \underline{\the\SOUL@token}%
     \fi}%
\SOUL@}
\makeatother

\newcommand*{\demp}{\fontfamily{lmtt}\selectfont}

\DeclareTextFontCommand{\textdemp}{\demp}

\begin{document}

\ifcomment
Multiline
comment
\fi
\ifcomment
\myul{Typesetting test}
% \color[rgb]{1,1,1}
$∑_i^n≠ 60º±∞π∆¬≈√j∫h≤≥µ$

$\CR \R\pro\ind\pro\gS\pro
\mqty[a&b\\c&d]$
$\pro\mathbb{P}$
$\dd{x}$

  \[
    \alpha(x)=\left\{
                \begin{array}{ll}
                  x\\
                  \frac{1}{1+e^{-kx}}\\
                  \frac{e^x-e^{-x}}{e^x+e^{-x}}
                \end{array}
              \right.
  \]

  $\expval{x}$
  
  $\chi_\rho(ghg\dmo)=\Tr(\rho_{ghg\dmo})=\Tr(\rho_g\circ\rho_h\circ\rho\dmo_g)=\Tr(\rho_h)\overset{\mbox{\scalebox{0.5}{$\Tr(AB)=\Tr(BA)$}}}{=}\chi_\rho(h)$
  	$\mathop{\oplus}_{\substack{x\in X}}$

$\mat(\rho_g)=(a_{ij}(g))_{\scriptsize \substack{1\leq i\leq d \\ 1\leq j\leq d}}$ et $\mat(\rho'_g)=(a'_{ij}(g))_{\scriptsize \substack{1\leq i'\leq d' \\ 1\leq j'\leq d'}}$



\[\int_a^b{\mathbb{R}^2}g(u, v)\dd{P_{XY}}(u, v)=\iint g(u,v) f_{XY}(u, v)\dd \lambda(u) \dd \lambda(v)\]
$$\lim_{x\to\infty} f(x)$$	
$$\iiiint_V \mu(t,u,v,w) \,dt\,du\,dv\,dw$$
$$\sum_{n=1}^{\infty} 2^{-n} = 1$$	
\begin{definition}
	Si $X$ et $Y$ sont 2 v.a. ou definit la \textsc{Covariance} entre $X$ et $Y$ comme
	$\cov(X,Y)\overset{\text{def}}{=}\E\left[(X-\E(X))(Y-\E(Y))\right]=\E(XY)-\E(X)\E(Y)$.
\end{definition}
\fi
\pagebreak

% \tableofcontents

% insert your code here
%\input{./algebra/main.tex}
%\input{./geometrie-differentielle/main.tex}
%\input{./probabilite/main.tex}
%\input{./analyse-fonctionnelle/main.tex}
% \input{./Analyse-convexe-et-dualite-en-optimisation/main.tex}
%\input{./tikz/main.tex}
%\input{./Theorie-du-distributions/main.tex}
%\input{./optimisation/mine.tex}
 \input{./modelisation/main.tex}

% yves.aubry@univ-tln.fr : algebra

\end{document}

%\input{./optimisation/mine.tex}
 % !TEX encoding = UTF-8 Unicode
% !TEX TS-program = xelatex

\documentclass[french]{report}

%\usepackage[utf8]{inputenc}
%\usepackage[T1]{fontenc}
\usepackage{babel}


\newif\ifcomment
%\commenttrue # Show comments

\usepackage{physics}
\usepackage{amssymb}


\usepackage{amsthm}
% \usepackage{thmtools}
\usepackage{mathtools}
\usepackage{amsfonts}

\usepackage{color}

\usepackage{tikz}

\usepackage{geometry}
\geometry{a5paper, margin=0.1in, right=1cm}

\usepackage{dsfont}

\usepackage{graphicx}
\graphicspath{ {images/} }

\usepackage{faktor}

\usepackage{IEEEtrantools}
\usepackage{enumerate}   
\usepackage[PostScript=dvips]{"/Users/aware/Documents/Courses/diagrams"}


\newtheorem{theorem}{Théorème}[section]
\renewcommand{\thetheorem}{\arabic{theorem}}
\newtheorem{lemme}{Lemme}[section]
\renewcommand{\thelemme}{\arabic{lemme}}
\newtheorem{proposition}{Proposition}[section]
\renewcommand{\theproposition}{\arabic{proposition}}
\newtheorem{notations}{Notations}[section]
\newtheorem{problem}{Problème}[section]
\newtheorem{corollary}{Corollaire}[theorem]
\renewcommand{\thecorollary}{\arabic{corollary}}
\newtheorem{property}{Propriété}[section]
\newtheorem{objective}{Objectif}[section]

\theoremstyle{definition}
\newtheorem{definition}{Définition}[section]
\renewcommand{\thedefinition}{\arabic{definition}}
\newtheorem{exercise}{Exercice}[chapter]
\renewcommand{\theexercise}{\arabic{exercise}}
\newtheorem{example}{Exemple}[chapter]
\renewcommand{\theexample}{\arabic{example}}
\newtheorem*{solution}{Solution}
\newtheorem*{application}{Application}
\newtheorem*{notation}{Notation}
\newtheorem*{vocabulary}{Vocabulaire}
\newtheorem*{properties}{Propriétés}



\theoremstyle{remark}
\newtheorem*{remark}{Remarque}
\newtheorem*{rappel}{Rappel}


\usepackage{etoolbox}
\AtBeginEnvironment{exercise}{\small}
\AtBeginEnvironment{example}{\small}

\usepackage{cases}
\usepackage[red]{mypack}

\usepackage[framemethod=TikZ]{mdframed}

\definecolor{bg}{rgb}{0.4,0.25,0.95}
\definecolor{pagebg}{rgb}{0,0,0.5}
\surroundwithmdframed[
   topline=false,
   rightline=false,
   bottomline=false,
   leftmargin=\parindent,
   skipabove=8pt,
   skipbelow=8pt,
   linecolor=blue,
   innerbottommargin=10pt,
   % backgroundcolor=bg,font=\color{orange}\sffamily, fontcolor=white
]{definition}

\usepackage{empheq}
\usepackage[most]{tcolorbox}

\newtcbox{\mymath}[1][]{%
    nobeforeafter, math upper, tcbox raise base,
    enhanced, colframe=blue!30!black,
    colback=red!10, boxrule=1pt,
    #1}

\usepackage{unixode}


\DeclareMathOperator{\ord}{ord}
\DeclareMathOperator{\orb}{orb}
\DeclareMathOperator{\stab}{stab}
\DeclareMathOperator{\Stab}{stab}
\DeclareMathOperator{\ppcm}{ppcm}
\DeclareMathOperator{\conj}{Conj}
\DeclareMathOperator{\End}{End}
\DeclareMathOperator{\rot}{rot}
\DeclareMathOperator{\trs}{trace}
\DeclareMathOperator{\Ind}{Ind}
\DeclareMathOperator{\mat}{Mat}
\DeclareMathOperator{\id}{Id}
\DeclareMathOperator{\vect}{vect}
\DeclareMathOperator{\img}{img}
\DeclareMathOperator{\cov}{Cov}
\DeclareMathOperator{\dist}{dist}
\DeclareMathOperator{\irr}{Irr}
\DeclareMathOperator{\image}{Im}
\DeclareMathOperator{\pd}{\partial}
\DeclareMathOperator{\epi}{epi}
\DeclareMathOperator{\Argmin}{Argmin}
\DeclareMathOperator{\dom}{dom}
\DeclareMathOperator{\proj}{proj}
\DeclareMathOperator{\ctg}{ctg}
\DeclareMathOperator{\supp}{supp}
\DeclareMathOperator{\argmin}{argmin}
\DeclareMathOperator{\mult}{mult}
\DeclareMathOperator{\ch}{ch}
\DeclareMathOperator{\sh}{sh}
\DeclareMathOperator{\rang}{rang}
\DeclareMathOperator{\diam}{diam}
\DeclareMathOperator{\Epigraphe}{Epigraphe}




\usepackage{xcolor}
\everymath{\color{blue}}
%\everymath{\color[rgb]{0,1,1}}
%\pagecolor[rgb]{0,0,0.5}


\newcommand*{\pdtest}[3][]{\ensuremath{\frac{\partial^{#1} #2}{\partial #3}}}

\newcommand*{\deffunc}[6][]{\ensuremath{
\begin{array}{rcl}
#2 : #3 &\rightarrow& #4\\
#5 &\mapsto& #6
\end{array}
}}

\newcommand{\eqcolon}{\mathrel{\resizebox{\widthof{$\mathord{=}$}}{\height}{ $\!\!=\!\!\resizebox{1.2\width}{0.8\height}{\raisebox{0.23ex}{$\mathop{:}$}}\!\!$ }}}
\newcommand{\coloneq}{\mathrel{\resizebox{\widthof{$\mathord{=}$}}{\height}{ $\!\!\resizebox{1.2\width}{0.8\height}{\raisebox{0.23ex}{$\mathop{:}$}}\!\!=\!\!$ }}}
\newcommand{\eqcolonl}{\ensuremath{\mathrel{=\!\!\mathop{:}}}}
\newcommand{\coloneql}{\ensuremath{\mathrel{\mathop{:} \!\! =}}}
\newcommand{\vc}[1]{% inline column vector
  \left(\begin{smallmatrix}#1\end{smallmatrix}\right)%
}
\newcommand{\vr}[1]{% inline row vector
  \begin{smallmatrix}(\,#1\,)\end{smallmatrix}%
}
\makeatletter
\newcommand*{\defeq}{\ =\mathrel{\rlap{%
                     \raisebox{0.3ex}{$\m@th\cdot$}}%
                     \raisebox{-0.3ex}{$\m@th\cdot$}}%
                     }
\makeatother

\newcommand{\mathcircle}[1]{% inline row vector
 \overset{\circ}{#1}
}
\newcommand{\ulim}{% low limit
 \underline{\lim}
}
\newcommand{\ssi}{% iff
\iff
}
\newcommand{\ps}[2]{
\expval{#1 | #2}
}
\newcommand{\df}[1]{
\mqty{#1}
}
\newcommand{\n}[1]{
\norm{#1}
}
\newcommand{\sys}[1]{
\left\{\smqty{#1}\right.
}


\newcommand{\eqdef}{\ensuremath{\overset{\text{def}}=}}


\def\Circlearrowright{\ensuremath{%
  \rotatebox[origin=c]{230}{$\circlearrowright$}}}

\newcommand\ct[1]{\text{\rmfamily\upshape #1}}
\newcommand\question[1]{ {\color{red} ...!? \small #1}}
\newcommand\caz[1]{\left\{\begin{array} #1 \end{array}\right.}
\newcommand\const{\text{\rmfamily\upshape const}}
\newcommand\toP{ \overset{\pro}{\to}}
\newcommand\toPP{ \overset{\text{PP}}{\to}}
\newcommand{\oeq}{\mathrel{\text{\textcircled{$=$}}}}





\usepackage{xcolor}
% \usepackage[normalem]{ulem}
\usepackage{lipsum}
\makeatletter
% \newcommand\colorwave[1][blue]{\bgroup \markoverwith{\lower3.5\p@\hbox{\sixly \textcolor{#1}{\char58}}}\ULon}
%\font\sixly=lasy6 % does not re-load if already loaded, so no memory problem.

\newmdtheoremenv[
linewidth= 1pt,linecolor= blue,%
leftmargin=20,rightmargin=20,innertopmargin=0pt, innerrightmargin=40,%
tikzsetting = { draw=lightgray, line width = 0.3pt,dashed,%
dash pattern = on 15pt off 3pt},%
splittopskip=\topskip,skipbelow=\baselineskip,%
skipabove=\baselineskip,ntheorem,roundcorner=0pt,
% backgroundcolor=pagebg,font=\color{orange}\sffamily, fontcolor=white
]{examplebox}{Exemple}[section]



\newcommand\R{\mathbb{R}}
\newcommand\Z{\mathbb{Z}}
\newcommand\N{\mathbb{N}}
\newcommand\E{\mathbb{E}}
\newcommand\F{\mathcal{F}}
\newcommand\cH{\mathcal{H}}
\newcommand\V{\mathbb{V}}
\newcommand\dmo{ ^{-1} }
\newcommand\kapa{\kappa}
\newcommand\im{Im}
\newcommand\hs{\mathcal{H}}





\usepackage{soul}

\makeatletter
\newcommand*{\whiten}[1]{\llap{\textcolor{white}{{\the\SOUL@token}}\hspace{#1pt}}}
\DeclareRobustCommand*\myul{%
    \def\SOUL@everyspace{\underline{\space}\kern\z@}%
    \def\SOUL@everytoken{%
     \setbox0=\hbox{\the\SOUL@token}%
     \ifdim\dp0>\z@
        \raisebox{\dp0}{\underline{\phantom{\the\SOUL@token}}}%
        \whiten{1}\whiten{0}%
        \whiten{-1}\whiten{-2}%
        \llap{\the\SOUL@token}%
     \else
        \underline{\the\SOUL@token}%
     \fi}%
\SOUL@}
\makeatother

\newcommand*{\demp}{\fontfamily{lmtt}\selectfont}

\DeclareTextFontCommand{\textdemp}{\demp}

\begin{document}

\ifcomment
Multiline
comment
\fi
\ifcomment
\myul{Typesetting test}
% \color[rgb]{1,1,1}
$∑_i^n≠ 60º±∞π∆¬≈√j∫h≤≥µ$

$\CR \R\pro\ind\pro\gS\pro
\mqty[a&b\\c&d]$
$\pro\mathbb{P}$
$\dd{x}$

  \[
    \alpha(x)=\left\{
                \begin{array}{ll}
                  x\\
                  \frac{1}{1+e^{-kx}}\\
                  \frac{e^x-e^{-x}}{e^x+e^{-x}}
                \end{array}
              \right.
  \]

  $\expval{x}$
  
  $\chi_\rho(ghg\dmo)=\Tr(\rho_{ghg\dmo})=\Tr(\rho_g\circ\rho_h\circ\rho\dmo_g)=\Tr(\rho_h)\overset{\mbox{\scalebox{0.5}{$\Tr(AB)=\Tr(BA)$}}}{=}\chi_\rho(h)$
  	$\mathop{\oplus}_{\substack{x\in X}}$

$\mat(\rho_g)=(a_{ij}(g))_{\scriptsize \substack{1\leq i\leq d \\ 1\leq j\leq d}}$ et $\mat(\rho'_g)=(a'_{ij}(g))_{\scriptsize \substack{1\leq i'\leq d' \\ 1\leq j'\leq d'}}$



\[\int_a^b{\mathbb{R}^2}g(u, v)\dd{P_{XY}}(u, v)=\iint g(u,v) f_{XY}(u, v)\dd \lambda(u) \dd \lambda(v)\]
$$\lim_{x\to\infty} f(x)$$	
$$\iiiint_V \mu(t,u,v,w) \,dt\,du\,dv\,dw$$
$$\sum_{n=1}^{\infty} 2^{-n} = 1$$	
\begin{definition}
	Si $X$ et $Y$ sont 2 v.a. ou definit la \textsc{Covariance} entre $X$ et $Y$ comme
	$\cov(X,Y)\overset{\text{def}}{=}\E\left[(X-\E(X))(Y-\E(Y))\right]=\E(XY)-\E(X)\E(Y)$.
\end{definition}
\fi
\pagebreak

% \tableofcontents

% insert your code here
%\input{./algebra/main.tex}
%\input{./geometrie-differentielle/main.tex}
%\input{./probabilite/main.tex}
%\input{./analyse-fonctionnelle/main.tex}
% \input{./Analyse-convexe-et-dualite-en-optimisation/main.tex}
%\input{./tikz/main.tex}
%\input{./Theorie-du-distributions/main.tex}
%\input{./optimisation/mine.tex}
 \input{./modelisation/main.tex}

% yves.aubry@univ-tln.fr : algebra

\end{document}


% yves.aubry@univ-tln.fr : algebra

\end{document}

% % !TEX encoding = UTF-8 Unicode
% !TEX TS-program = xelatex

\documentclass[french]{report}

%\usepackage[utf8]{inputenc}
%\usepackage[T1]{fontenc}
\usepackage{babel}


\newif\ifcomment
%\commenttrue # Show comments

\usepackage{physics}
\usepackage{amssymb}


\usepackage{amsthm}
% \usepackage{thmtools}
\usepackage{mathtools}
\usepackage{amsfonts}

\usepackage{color}

\usepackage{tikz}

\usepackage{geometry}
\geometry{a5paper, margin=0.1in, right=1cm}

\usepackage{dsfont}

\usepackage{graphicx}
\graphicspath{ {images/} }

\usepackage{faktor}

\usepackage{IEEEtrantools}
\usepackage{enumerate}   
\usepackage[PostScript=dvips]{"/Users/aware/Documents/Courses/diagrams"}


\newtheorem{theorem}{Théorème}[section]
\renewcommand{\thetheorem}{\arabic{theorem}}
\newtheorem{lemme}{Lemme}[section]
\renewcommand{\thelemme}{\arabic{lemme}}
\newtheorem{proposition}{Proposition}[section]
\renewcommand{\theproposition}{\arabic{proposition}}
\newtheorem{notations}{Notations}[section]
\newtheorem{problem}{Problème}[section]
\newtheorem{corollary}{Corollaire}[theorem]
\renewcommand{\thecorollary}{\arabic{corollary}}
\newtheorem{property}{Propriété}[section]
\newtheorem{objective}{Objectif}[section]

\theoremstyle{definition}
\newtheorem{definition}{Définition}[section]
\renewcommand{\thedefinition}{\arabic{definition}}
\newtheorem{exercise}{Exercice}[chapter]
\renewcommand{\theexercise}{\arabic{exercise}}
\newtheorem{example}{Exemple}[chapter]
\renewcommand{\theexample}{\arabic{example}}
\newtheorem*{solution}{Solution}
\newtheorem*{application}{Application}
\newtheorem*{notation}{Notation}
\newtheorem*{vocabulary}{Vocabulaire}
\newtheorem*{properties}{Propriétés}



\theoremstyle{remark}
\newtheorem*{remark}{Remarque}
\newtheorem*{rappel}{Rappel}


\usepackage{etoolbox}
\AtBeginEnvironment{exercise}{\small}
\AtBeginEnvironment{example}{\small}

\usepackage{cases}
\usepackage[red]{mypack}

\usepackage[framemethod=TikZ]{mdframed}

\definecolor{bg}{rgb}{0.4,0.25,0.95}
\definecolor{pagebg}{rgb}{0,0,0.5}
\surroundwithmdframed[
   topline=false,
   rightline=false,
   bottomline=false,
   leftmargin=\parindent,
   skipabove=8pt,
   skipbelow=8pt,
   linecolor=blue,
   innerbottommargin=10pt,
   % backgroundcolor=bg,font=\color{orange}\sffamily, fontcolor=white
]{definition}

\usepackage{empheq}
\usepackage[most]{tcolorbox}

\newtcbox{\mymath}[1][]{%
    nobeforeafter, math upper, tcbox raise base,
    enhanced, colframe=blue!30!black,
    colback=red!10, boxrule=1pt,
    #1}

\usepackage{unixode}


\DeclareMathOperator{\ord}{ord}
\DeclareMathOperator{\orb}{orb}
\DeclareMathOperator{\stab}{stab}
\DeclareMathOperator{\Stab}{stab}
\DeclareMathOperator{\ppcm}{ppcm}
\DeclareMathOperator{\conj}{Conj}
\DeclareMathOperator{\End}{End}
\DeclareMathOperator{\rot}{rot}
\DeclareMathOperator{\trs}{trace}
\DeclareMathOperator{\Ind}{Ind}
\DeclareMathOperator{\mat}{Mat}
\DeclareMathOperator{\id}{Id}
\DeclareMathOperator{\vect}{vect}
\DeclareMathOperator{\img}{img}
\DeclareMathOperator{\cov}{Cov}
\DeclareMathOperator{\dist}{dist}
\DeclareMathOperator{\irr}{Irr}
\DeclareMathOperator{\image}{Im}
\DeclareMathOperator{\pd}{\partial}
\DeclareMathOperator{\epi}{epi}
\DeclareMathOperator{\Argmin}{Argmin}
\DeclareMathOperator{\dom}{dom}
\DeclareMathOperator{\proj}{proj}
\DeclareMathOperator{\ctg}{ctg}
\DeclareMathOperator{\supp}{supp}
\DeclareMathOperator{\argmin}{argmin}
\DeclareMathOperator{\mult}{mult}
\DeclareMathOperator{\ch}{ch}
\DeclareMathOperator{\sh}{sh}
\DeclareMathOperator{\rang}{rang}
\DeclareMathOperator{\diam}{diam}
\DeclareMathOperator{\Epigraphe}{Epigraphe}




\usepackage{xcolor}
\everymath{\color{blue}}
%\everymath{\color[rgb]{0,1,1}}
%\pagecolor[rgb]{0,0,0.5}


\newcommand*{\pdtest}[3][]{\ensuremath{\frac{\partial^{#1} #2}{\partial #3}}}

\newcommand*{\deffunc}[6][]{\ensuremath{
\begin{array}{rcl}
#2 : #3 &\rightarrow& #4\\
#5 &\mapsto& #6
\end{array}
}}

\newcommand{\eqcolon}{\mathrel{\resizebox{\widthof{$\mathord{=}$}}{\height}{ $\!\!=\!\!\resizebox{1.2\width}{0.8\height}{\raisebox{0.23ex}{$\mathop{:}$}}\!\!$ }}}
\newcommand{\coloneq}{\mathrel{\resizebox{\widthof{$\mathord{=}$}}{\height}{ $\!\!\resizebox{1.2\width}{0.8\height}{\raisebox{0.23ex}{$\mathop{:}$}}\!\!=\!\!$ }}}
\newcommand{\eqcolonl}{\ensuremath{\mathrel{=\!\!\mathop{:}}}}
\newcommand{\coloneql}{\ensuremath{\mathrel{\mathop{:} \!\! =}}}
\newcommand{\vc}[1]{% inline column vector
  \left(\begin{smallmatrix}#1\end{smallmatrix}\right)%
}
\newcommand{\vr}[1]{% inline row vector
  \begin{smallmatrix}(\,#1\,)\end{smallmatrix}%
}
\makeatletter
\newcommand*{\defeq}{\ =\mathrel{\rlap{%
                     \raisebox{0.3ex}{$\m@th\cdot$}}%
                     \raisebox{-0.3ex}{$\m@th\cdot$}}%
                     }
\makeatother

\newcommand{\mathcircle}[1]{% inline row vector
 \overset{\circ}{#1}
}
\newcommand{\ulim}{% low limit
 \underline{\lim}
}
\newcommand{\ssi}{% iff
\iff
}
\newcommand{\ps}[2]{
\expval{#1 | #2}
}
\newcommand{\df}[1]{
\mqty{#1}
}
\newcommand{\n}[1]{
\norm{#1}
}
\newcommand{\sys}[1]{
\left\{\smqty{#1}\right.
}


\newcommand{\eqdef}{\ensuremath{\overset{\text{def}}=}}


\def\Circlearrowright{\ensuremath{%
  \rotatebox[origin=c]{230}{$\circlearrowright$}}}

\newcommand\ct[1]{\text{\rmfamily\upshape #1}}
\newcommand\question[1]{ {\color{red} ...!? \small #1}}
\newcommand\caz[1]{\left\{\begin{array} #1 \end{array}\right.}
\newcommand\const{\text{\rmfamily\upshape const}}
\newcommand\toP{ \overset{\pro}{\to}}
\newcommand\toPP{ \overset{\text{PP}}{\to}}
\newcommand{\oeq}{\mathrel{\text{\textcircled{$=$}}}}





\usepackage{xcolor}
% \usepackage[normalem]{ulem}
\usepackage{lipsum}
\makeatletter
% \newcommand\colorwave[1][blue]{\bgroup \markoverwith{\lower3.5\p@\hbox{\sixly \textcolor{#1}{\char58}}}\ULon}
%\font\sixly=lasy6 % does not re-load if already loaded, so no memory problem.

\newmdtheoremenv[
linewidth= 1pt,linecolor= blue,%
leftmargin=20,rightmargin=20,innertopmargin=0pt, innerrightmargin=40,%
tikzsetting = { draw=lightgray, line width = 0.3pt,dashed,%
dash pattern = on 15pt off 3pt},%
splittopskip=\topskip,skipbelow=\baselineskip,%
skipabove=\baselineskip,ntheorem,roundcorner=0pt,
% backgroundcolor=pagebg,font=\color{orange}\sffamily, fontcolor=white
]{examplebox}{Exemple}[section]



\newcommand\R{\mathbb{R}}
\newcommand\Z{\mathbb{Z}}
\newcommand\N{\mathbb{N}}
\newcommand\E{\mathbb{E}}
\newcommand\F{\mathcal{F}}
\newcommand\cH{\mathcal{H}}
\newcommand\V{\mathbb{V}}
\newcommand\dmo{ ^{-1} }
\newcommand\kapa{\kappa}
\newcommand\im{Im}
\newcommand\hs{\mathcal{H}}





\usepackage{soul}

\makeatletter
\newcommand*{\whiten}[1]{\llap{\textcolor{white}{{\the\SOUL@token}}\hspace{#1pt}}}
\DeclareRobustCommand*\myul{%
    \def\SOUL@everyspace{\underline{\space}\kern\z@}%
    \def\SOUL@everytoken{%
     \setbox0=\hbox{\the\SOUL@token}%
     \ifdim\dp0>\z@
        \raisebox{\dp0}{\underline{\phantom{\the\SOUL@token}}}%
        \whiten{1}\whiten{0}%
        \whiten{-1}\whiten{-2}%
        \llap{\the\SOUL@token}%
     \else
        \underline{\the\SOUL@token}%
     \fi}%
\SOUL@}
\makeatother

\newcommand*{\demp}{\fontfamily{lmtt}\selectfont}

\DeclareTextFontCommand{\textdemp}{\demp}

\begin{document}

\ifcomment
Multiline
comment
\fi
\ifcomment
\myul{Typesetting test}
% \color[rgb]{1,1,1}
$∑_i^n≠ 60º±∞π∆¬≈√j∫h≤≥µ$

$\CR \R\pro\ind\pro\gS\pro
\mqty[a&b\\c&d]$
$\pro\mathbb{P}$
$\dd{x}$

  \[
    \alpha(x)=\left\{
                \begin{array}{ll}
                  x\\
                  \frac{1}{1+e^{-kx}}\\
                  \frac{e^x-e^{-x}}{e^x+e^{-x}}
                \end{array}
              \right.
  \]

  $\expval{x}$
  
  $\chi_\rho(ghg\dmo)=\Tr(\rho_{ghg\dmo})=\Tr(\rho_g\circ\rho_h\circ\rho\dmo_g)=\Tr(\rho_h)\overset{\mbox{\scalebox{0.5}{$\Tr(AB)=\Tr(BA)$}}}{=}\chi_\rho(h)$
  	$\mathop{\oplus}_{\substack{x\in X}}$

$\mat(\rho_g)=(a_{ij}(g))_{\scriptsize \substack{1\leq i\leq d \\ 1\leq j\leq d}}$ et $\mat(\rho'_g)=(a'_{ij}(g))_{\scriptsize \substack{1\leq i'\leq d' \\ 1\leq j'\leq d'}}$



\[\int_a^b{\mathbb{R}^2}g(u, v)\dd{P_{XY}}(u, v)=\iint g(u,v) f_{XY}(u, v)\dd \lambda(u) \dd \lambda(v)\]
$$\lim_{x\to\infty} f(x)$$	
$$\iiiint_V \mu(t,u,v,w) \,dt\,du\,dv\,dw$$
$$\sum_{n=1}^{\infty} 2^{-n} = 1$$	
\begin{definition}
	Si $X$ et $Y$ sont 2 v.a. ou definit la \textsc{Covariance} entre $X$ et $Y$ comme
	$\cov(X,Y)\overset{\text{def}}{=}\E\left[(X-\E(X))(Y-\E(Y))\right]=\E(XY)-\E(X)\E(Y)$.
\end{definition}
\fi
\pagebreak

% \tableofcontents

% insert your code here
%% !TEX encoding = UTF-8 Unicode
% !TEX TS-program = xelatex

\documentclass[french]{report}

%\usepackage[utf8]{inputenc}
%\usepackage[T1]{fontenc}
\usepackage{babel}


\newif\ifcomment
%\commenttrue # Show comments

\usepackage{physics}
\usepackage{amssymb}


\usepackage{amsthm}
% \usepackage{thmtools}
\usepackage{mathtools}
\usepackage{amsfonts}

\usepackage{color}

\usepackage{tikz}

\usepackage{geometry}
\geometry{a5paper, margin=0.1in, right=1cm}

\usepackage{dsfont}

\usepackage{graphicx}
\graphicspath{ {images/} }

\usepackage{faktor}

\usepackage{IEEEtrantools}
\usepackage{enumerate}   
\usepackage[PostScript=dvips]{"/Users/aware/Documents/Courses/diagrams"}


\newtheorem{theorem}{Théorème}[section]
\renewcommand{\thetheorem}{\arabic{theorem}}
\newtheorem{lemme}{Lemme}[section]
\renewcommand{\thelemme}{\arabic{lemme}}
\newtheorem{proposition}{Proposition}[section]
\renewcommand{\theproposition}{\arabic{proposition}}
\newtheorem{notations}{Notations}[section]
\newtheorem{problem}{Problème}[section]
\newtheorem{corollary}{Corollaire}[theorem]
\renewcommand{\thecorollary}{\arabic{corollary}}
\newtheorem{property}{Propriété}[section]
\newtheorem{objective}{Objectif}[section]

\theoremstyle{definition}
\newtheorem{definition}{Définition}[section]
\renewcommand{\thedefinition}{\arabic{definition}}
\newtheorem{exercise}{Exercice}[chapter]
\renewcommand{\theexercise}{\arabic{exercise}}
\newtheorem{example}{Exemple}[chapter]
\renewcommand{\theexample}{\arabic{example}}
\newtheorem*{solution}{Solution}
\newtheorem*{application}{Application}
\newtheorem*{notation}{Notation}
\newtheorem*{vocabulary}{Vocabulaire}
\newtheorem*{properties}{Propriétés}



\theoremstyle{remark}
\newtheorem*{remark}{Remarque}
\newtheorem*{rappel}{Rappel}


\usepackage{etoolbox}
\AtBeginEnvironment{exercise}{\small}
\AtBeginEnvironment{example}{\small}

\usepackage{cases}
\usepackage[red]{mypack}

\usepackage[framemethod=TikZ]{mdframed}

\definecolor{bg}{rgb}{0.4,0.25,0.95}
\definecolor{pagebg}{rgb}{0,0,0.5}
\surroundwithmdframed[
   topline=false,
   rightline=false,
   bottomline=false,
   leftmargin=\parindent,
   skipabove=8pt,
   skipbelow=8pt,
   linecolor=blue,
   innerbottommargin=10pt,
   % backgroundcolor=bg,font=\color{orange}\sffamily, fontcolor=white
]{definition}

\usepackage{empheq}
\usepackage[most]{tcolorbox}

\newtcbox{\mymath}[1][]{%
    nobeforeafter, math upper, tcbox raise base,
    enhanced, colframe=blue!30!black,
    colback=red!10, boxrule=1pt,
    #1}

\usepackage{unixode}


\DeclareMathOperator{\ord}{ord}
\DeclareMathOperator{\orb}{orb}
\DeclareMathOperator{\stab}{stab}
\DeclareMathOperator{\Stab}{stab}
\DeclareMathOperator{\ppcm}{ppcm}
\DeclareMathOperator{\conj}{Conj}
\DeclareMathOperator{\End}{End}
\DeclareMathOperator{\rot}{rot}
\DeclareMathOperator{\trs}{trace}
\DeclareMathOperator{\Ind}{Ind}
\DeclareMathOperator{\mat}{Mat}
\DeclareMathOperator{\id}{Id}
\DeclareMathOperator{\vect}{vect}
\DeclareMathOperator{\img}{img}
\DeclareMathOperator{\cov}{Cov}
\DeclareMathOperator{\dist}{dist}
\DeclareMathOperator{\irr}{Irr}
\DeclareMathOperator{\image}{Im}
\DeclareMathOperator{\pd}{\partial}
\DeclareMathOperator{\epi}{epi}
\DeclareMathOperator{\Argmin}{Argmin}
\DeclareMathOperator{\dom}{dom}
\DeclareMathOperator{\proj}{proj}
\DeclareMathOperator{\ctg}{ctg}
\DeclareMathOperator{\supp}{supp}
\DeclareMathOperator{\argmin}{argmin}
\DeclareMathOperator{\mult}{mult}
\DeclareMathOperator{\ch}{ch}
\DeclareMathOperator{\sh}{sh}
\DeclareMathOperator{\rang}{rang}
\DeclareMathOperator{\diam}{diam}
\DeclareMathOperator{\Epigraphe}{Epigraphe}




\usepackage{xcolor}
\everymath{\color{blue}}
%\everymath{\color[rgb]{0,1,1}}
%\pagecolor[rgb]{0,0,0.5}


\newcommand*{\pdtest}[3][]{\ensuremath{\frac{\partial^{#1} #2}{\partial #3}}}

\newcommand*{\deffunc}[6][]{\ensuremath{
\begin{array}{rcl}
#2 : #3 &\rightarrow& #4\\
#5 &\mapsto& #6
\end{array}
}}

\newcommand{\eqcolon}{\mathrel{\resizebox{\widthof{$\mathord{=}$}}{\height}{ $\!\!=\!\!\resizebox{1.2\width}{0.8\height}{\raisebox{0.23ex}{$\mathop{:}$}}\!\!$ }}}
\newcommand{\coloneq}{\mathrel{\resizebox{\widthof{$\mathord{=}$}}{\height}{ $\!\!\resizebox{1.2\width}{0.8\height}{\raisebox{0.23ex}{$\mathop{:}$}}\!\!=\!\!$ }}}
\newcommand{\eqcolonl}{\ensuremath{\mathrel{=\!\!\mathop{:}}}}
\newcommand{\coloneql}{\ensuremath{\mathrel{\mathop{:} \!\! =}}}
\newcommand{\vc}[1]{% inline column vector
  \left(\begin{smallmatrix}#1\end{smallmatrix}\right)%
}
\newcommand{\vr}[1]{% inline row vector
  \begin{smallmatrix}(\,#1\,)\end{smallmatrix}%
}
\makeatletter
\newcommand*{\defeq}{\ =\mathrel{\rlap{%
                     \raisebox{0.3ex}{$\m@th\cdot$}}%
                     \raisebox{-0.3ex}{$\m@th\cdot$}}%
                     }
\makeatother

\newcommand{\mathcircle}[1]{% inline row vector
 \overset{\circ}{#1}
}
\newcommand{\ulim}{% low limit
 \underline{\lim}
}
\newcommand{\ssi}{% iff
\iff
}
\newcommand{\ps}[2]{
\expval{#1 | #2}
}
\newcommand{\df}[1]{
\mqty{#1}
}
\newcommand{\n}[1]{
\norm{#1}
}
\newcommand{\sys}[1]{
\left\{\smqty{#1}\right.
}


\newcommand{\eqdef}{\ensuremath{\overset{\text{def}}=}}


\def\Circlearrowright{\ensuremath{%
  \rotatebox[origin=c]{230}{$\circlearrowright$}}}

\newcommand\ct[1]{\text{\rmfamily\upshape #1}}
\newcommand\question[1]{ {\color{red} ...!? \small #1}}
\newcommand\caz[1]{\left\{\begin{array} #1 \end{array}\right.}
\newcommand\const{\text{\rmfamily\upshape const}}
\newcommand\toP{ \overset{\pro}{\to}}
\newcommand\toPP{ \overset{\text{PP}}{\to}}
\newcommand{\oeq}{\mathrel{\text{\textcircled{$=$}}}}





\usepackage{xcolor}
% \usepackage[normalem]{ulem}
\usepackage{lipsum}
\makeatletter
% \newcommand\colorwave[1][blue]{\bgroup \markoverwith{\lower3.5\p@\hbox{\sixly \textcolor{#1}{\char58}}}\ULon}
%\font\sixly=lasy6 % does not re-load if already loaded, so no memory problem.

\newmdtheoremenv[
linewidth= 1pt,linecolor= blue,%
leftmargin=20,rightmargin=20,innertopmargin=0pt, innerrightmargin=40,%
tikzsetting = { draw=lightgray, line width = 0.3pt,dashed,%
dash pattern = on 15pt off 3pt},%
splittopskip=\topskip,skipbelow=\baselineskip,%
skipabove=\baselineskip,ntheorem,roundcorner=0pt,
% backgroundcolor=pagebg,font=\color{orange}\sffamily, fontcolor=white
]{examplebox}{Exemple}[section]



\newcommand\R{\mathbb{R}}
\newcommand\Z{\mathbb{Z}}
\newcommand\N{\mathbb{N}}
\newcommand\E{\mathbb{E}}
\newcommand\F{\mathcal{F}}
\newcommand\cH{\mathcal{H}}
\newcommand\V{\mathbb{V}}
\newcommand\dmo{ ^{-1} }
\newcommand\kapa{\kappa}
\newcommand\im{Im}
\newcommand\hs{\mathcal{H}}





\usepackage{soul}

\makeatletter
\newcommand*{\whiten}[1]{\llap{\textcolor{white}{{\the\SOUL@token}}\hspace{#1pt}}}
\DeclareRobustCommand*\myul{%
    \def\SOUL@everyspace{\underline{\space}\kern\z@}%
    \def\SOUL@everytoken{%
     \setbox0=\hbox{\the\SOUL@token}%
     \ifdim\dp0>\z@
        \raisebox{\dp0}{\underline{\phantom{\the\SOUL@token}}}%
        \whiten{1}\whiten{0}%
        \whiten{-1}\whiten{-2}%
        \llap{\the\SOUL@token}%
     \else
        \underline{\the\SOUL@token}%
     \fi}%
\SOUL@}
\makeatother

\newcommand*{\demp}{\fontfamily{lmtt}\selectfont}

\DeclareTextFontCommand{\textdemp}{\demp}

\begin{document}

\ifcomment
Multiline
comment
\fi
\ifcomment
\myul{Typesetting test}
% \color[rgb]{1,1,1}
$∑_i^n≠ 60º±∞π∆¬≈√j∫h≤≥µ$

$\CR \R\pro\ind\pro\gS\pro
\mqty[a&b\\c&d]$
$\pro\mathbb{P}$
$\dd{x}$

  \[
    \alpha(x)=\left\{
                \begin{array}{ll}
                  x\\
                  \frac{1}{1+e^{-kx}}\\
                  \frac{e^x-e^{-x}}{e^x+e^{-x}}
                \end{array}
              \right.
  \]

  $\expval{x}$
  
  $\chi_\rho(ghg\dmo)=\Tr(\rho_{ghg\dmo})=\Tr(\rho_g\circ\rho_h\circ\rho\dmo_g)=\Tr(\rho_h)\overset{\mbox{\scalebox{0.5}{$\Tr(AB)=\Tr(BA)$}}}{=}\chi_\rho(h)$
  	$\mathop{\oplus}_{\substack{x\in X}}$

$\mat(\rho_g)=(a_{ij}(g))_{\scriptsize \substack{1\leq i\leq d \\ 1\leq j\leq d}}$ et $\mat(\rho'_g)=(a'_{ij}(g))_{\scriptsize \substack{1\leq i'\leq d' \\ 1\leq j'\leq d'}}$



\[\int_a^b{\mathbb{R}^2}g(u, v)\dd{P_{XY}}(u, v)=\iint g(u,v) f_{XY}(u, v)\dd \lambda(u) \dd \lambda(v)\]
$$\lim_{x\to\infty} f(x)$$	
$$\iiiint_V \mu(t,u,v,w) \,dt\,du\,dv\,dw$$
$$\sum_{n=1}^{\infty} 2^{-n} = 1$$	
\begin{definition}
	Si $X$ et $Y$ sont 2 v.a. ou definit la \textsc{Covariance} entre $X$ et $Y$ comme
	$\cov(X,Y)\overset{\text{def}}{=}\E\left[(X-\E(X))(Y-\E(Y))\right]=\E(XY)-\E(X)\E(Y)$.
\end{definition}
\fi
\pagebreak

% \tableofcontents

% insert your code here
%\input{./algebra/main.tex}
%\input{./geometrie-differentielle/main.tex}
%\input{./probabilite/main.tex}
%\input{./analyse-fonctionnelle/main.tex}
% \input{./Analyse-convexe-et-dualite-en-optimisation/main.tex}
%\input{./tikz/main.tex}
%\input{./Theorie-du-distributions/main.tex}
%\input{./optimisation/mine.tex}
 \input{./modelisation/main.tex}

% yves.aubry@univ-tln.fr : algebra

\end{document}

%% !TEX encoding = UTF-8 Unicode
% !TEX TS-program = xelatex

\documentclass[french]{report}

%\usepackage[utf8]{inputenc}
%\usepackage[T1]{fontenc}
\usepackage{babel}


\newif\ifcomment
%\commenttrue # Show comments

\usepackage{physics}
\usepackage{amssymb}


\usepackage{amsthm}
% \usepackage{thmtools}
\usepackage{mathtools}
\usepackage{amsfonts}

\usepackage{color}

\usepackage{tikz}

\usepackage{geometry}
\geometry{a5paper, margin=0.1in, right=1cm}

\usepackage{dsfont}

\usepackage{graphicx}
\graphicspath{ {images/} }

\usepackage{faktor}

\usepackage{IEEEtrantools}
\usepackage{enumerate}   
\usepackage[PostScript=dvips]{"/Users/aware/Documents/Courses/diagrams"}


\newtheorem{theorem}{Théorème}[section]
\renewcommand{\thetheorem}{\arabic{theorem}}
\newtheorem{lemme}{Lemme}[section]
\renewcommand{\thelemme}{\arabic{lemme}}
\newtheorem{proposition}{Proposition}[section]
\renewcommand{\theproposition}{\arabic{proposition}}
\newtheorem{notations}{Notations}[section]
\newtheorem{problem}{Problème}[section]
\newtheorem{corollary}{Corollaire}[theorem]
\renewcommand{\thecorollary}{\arabic{corollary}}
\newtheorem{property}{Propriété}[section]
\newtheorem{objective}{Objectif}[section]

\theoremstyle{definition}
\newtheorem{definition}{Définition}[section]
\renewcommand{\thedefinition}{\arabic{definition}}
\newtheorem{exercise}{Exercice}[chapter]
\renewcommand{\theexercise}{\arabic{exercise}}
\newtheorem{example}{Exemple}[chapter]
\renewcommand{\theexample}{\arabic{example}}
\newtheorem*{solution}{Solution}
\newtheorem*{application}{Application}
\newtheorem*{notation}{Notation}
\newtheorem*{vocabulary}{Vocabulaire}
\newtheorem*{properties}{Propriétés}



\theoremstyle{remark}
\newtheorem*{remark}{Remarque}
\newtheorem*{rappel}{Rappel}


\usepackage{etoolbox}
\AtBeginEnvironment{exercise}{\small}
\AtBeginEnvironment{example}{\small}

\usepackage{cases}
\usepackage[red]{mypack}

\usepackage[framemethod=TikZ]{mdframed}

\definecolor{bg}{rgb}{0.4,0.25,0.95}
\definecolor{pagebg}{rgb}{0,0,0.5}
\surroundwithmdframed[
   topline=false,
   rightline=false,
   bottomline=false,
   leftmargin=\parindent,
   skipabove=8pt,
   skipbelow=8pt,
   linecolor=blue,
   innerbottommargin=10pt,
   % backgroundcolor=bg,font=\color{orange}\sffamily, fontcolor=white
]{definition}

\usepackage{empheq}
\usepackage[most]{tcolorbox}

\newtcbox{\mymath}[1][]{%
    nobeforeafter, math upper, tcbox raise base,
    enhanced, colframe=blue!30!black,
    colback=red!10, boxrule=1pt,
    #1}

\usepackage{unixode}


\DeclareMathOperator{\ord}{ord}
\DeclareMathOperator{\orb}{orb}
\DeclareMathOperator{\stab}{stab}
\DeclareMathOperator{\Stab}{stab}
\DeclareMathOperator{\ppcm}{ppcm}
\DeclareMathOperator{\conj}{Conj}
\DeclareMathOperator{\End}{End}
\DeclareMathOperator{\rot}{rot}
\DeclareMathOperator{\trs}{trace}
\DeclareMathOperator{\Ind}{Ind}
\DeclareMathOperator{\mat}{Mat}
\DeclareMathOperator{\id}{Id}
\DeclareMathOperator{\vect}{vect}
\DeclareMathOperator{\img}{img}
\DeclareMathOperator{\cov}{Cov}
\DeclareMathOperator{\dist}{dist}
\DeclareMathOperator{\irr}{Irr}
\DeclareMathOperator{\image}{Im}
\DeclareMathOperator{\pd}{\partial}
\DeclareMathOperator{\epi}{epi}
\DeclareMathOperator{\Argmin}{Argmin}
\DeclareMathOperator{\dom}{dom}
\DeclareMathOperator{\proj}{proj}
\DeclareMathOperator{\ctg}{ctg}
\DeclareMathOperator{\supp}{supp}
\DeclareMathOperator{\argmin}{argmin}
\DeclareMathOperator{\mult}{mult}
\DeclareMathOperator{\ch}{ch}
\DeclareMathOperator{\sh}{sh}
\DeclareMathOperator{\rang}{rang}
\DeclareMathOperator{\diam}{diam}
\DeclareMathOperator{\Epigraphe}{Epigraphe}




\usepackage{xcolor}
\everymath{\color{blue}}
%\everymath{\color[rgb]{0,1,1}}
%\pagecolor[rgb]{0,0,0.5}


\newcommand*{\pdtest}[3][]{\ensuremath{\frac{\partial^{#1} #2}{\partial #3}}}

\newcommand*{\deffunc}[6][]{\ensuremath{
\begin{array}{rcl}
#2 : #3 &\rightarrow& #4\\
#5 &\mapsto& #6
\end{array}
}}

\newcommand{\eqcolon}{\mathrel{\resizebox{\widthof{$\mathord{=}$}}{\height}{ $\!\!=\!\!\resizebox{1.2\width}{0.8\height}{\raisebox{0.23ex}{$\mathop{:}$}}\!\!$ }}}
\newcommand{\coloneq}{\mathrel{\resizebox{\widthof{$\mathord{=}$}}{\height}{ $\!\!\resizebox{1.2\width}{0.8\height}{\raisebox{0.23ex}{$\mathop{:}$}}\!\!=\!\!$ }}}
\newcommand{\eqcolonl}{\ensuremath{\mathrel{=\!\!\mathop{:}}}}
\newcommand{\coloneql}{\ensuremath{\mathrel{\mathop{:} \!\! =}}}
\newcommand{\vc}[1]{% inline column vector
  \left(\begin{smallmatrix}#1\end{smallmatrix}\right)%
}
\newcommand{\vr}[1]{% inline row vector
  \begin{smallmatrix}(\,#1\,)\end{smallmatrix}%
}
\makeatletter
\newcommand*{\defeq}{\ =\mathrel{\rlap{%
                     \raisebox{0.3ex}{$\m@th\cdot$}}%
                     \raisebox{-0.3ex}{$\m@th\cdot$}}%
                     }
\makeatother

\newcommand{\mathcircle}[1]{% inline row vector
 \overset{\circ}{#1}
}
\newcommand{\ulim}{% low limit
 \underline{\lim}
}
\newcommand{\ssi}{% iff
\iff
}
\newcommand{\ps}[2]{
\expval{#1 | #2}
}
\newcommand{\df}[1]{
\mqty{#1}
}
\newcommand{\n}[1]{
\norm{#1}
}
\newcommand{\sys}[1]{
\left\{\smqty{#1}\right.
}


\newcommand{\eqdef}{\ensuremath{\overset{\text{def}}=}}


\def\Circlearrowright{\ensuremath{%
  \rotatebox[origin=c]{230}{$\circlearrowright$}}}

\newcommand\ct[1]{\text{\rmfamily\upshape #1}}
\newcommand\question[1]{ {\color{red} ...!? \small #1}}
\newcommand\caz[1]{\left\{\begin{array} #1 \end{array}\right.}
\newcommand\const{\text{\rmfamily\upshape const}}
\newcommand\toP{ \overset{\pro}{\to}}
\newcommand\toPP{ \overset{\text{PP}}{\to}}
\newcommand{\oeq}{\mathrel{\text{\textcircled{$=$}}}}





\usepackage{xcolor}
% \usepackage[normalem]{ulem}
\usepackage{lipsum}
\makeatletter
% \newcommand\colorwave[1][blue]{\bgroup \markoverwith{\lower3.5\p@\hbox{\sixly \textcolor{#1}{\char58}}}\ULon}
%\font\sixly=lasy6 % does not re-load if already loaded, so no memory problem.

\newmdtheoremenv[
linewidth= 1pt,linecolor= blue,%
leftmargin=20,rightmargin=20,innertopmargin=0pt, innerrightmargin=40,%
tikzsetting = { draw=lightgray, line width = 0.3pt,dashed,%
dash pattern = on 15pt off 3pt},%
splittopskip=\topskip,skipbelow=\baselineskip,%
skipabove=\baselineskip,ntheorem,roundcorner=0pt,
% backgroundcolor=pagebg,font=\color{orange}\sffamily, fontcolor=white
]{examplebox}{Exemple}[section]



\newcommand\R{\mathbb{R}}
\newcommand\Z{\mathbb{Z}}
\newcommand\N{\mathbb{N}}
\newcommand\E{\mathbb{E}}
\newcommand\F{\mathcal{F}}
\newcommand\cH{\mathcal{H}}
\newcommand\V{\mathbb{V}}
\newcommand\dmo{ ^{-1} }
\newcommand\kapa{\kappa}
\newcommand\im{Im}
\newcommand\hs{\mathcal{H}}





\usepackage{soul}

\makeatletter
\newcommand*{\whiten}[1]{\llap{\textcolor{white}{{\the\SOUL@token}}\hspace{#1pt}}}
\DeclareRobustCommand*\myul{%
    \def\SOUL@everyspace{\underline{\space}\kern\z@}%
    \def\SOUL@everytoken{%
     \setbox0=\hbox{\the\SOUL@token}%
     \ifdim\dp0>\z@
        \raisebox{\dp0}{\underline{\phantom{\the\SOUL@token}}}%
        \whiten{1}\whiten{0}%
        \whiten{-1}\whiten{-2}%
        \llap{\the\SOUL@token}%
     \else
        \underline{\the\SOUL@token}%
     \fi}%
\SOUL@}
\makeatother

\newcommand*{\demp}{\fontfamily{lmtt}\selectfont}

\DeclareTextFontCommand{\textdemp}{\demp}

\begin{document}

\ifcomment
Multiline
comment
\fi
\ifcomment
\myul{Typesetting test}
% \color[rgb]{1,1,1}
$∑_i^n≠ 60º±∞π∆¬≈√j∫h≤≥µ$

$\CR \R\pro\ind\pro\gS\pro
\mqty[a&b\\c&d]$
$\pro\mathbb{P}$
$\dd{x}$

  \[
    \alpha(x)=\left\{
                \begin{array}{ll}
                  x\\
                  \frac{1}{1+e^{-kx}}\\
                  \frac{e^x-e^{-x}}{e^x+e^{-x}}
                \end{array}
              \right.
  \]

  $\expval{x}$
  
  $\chi_\rho(ghg\dmo)=\Tr(\rho_{ghg\dmo})=\Tr(\rho_g\circ\rho_h\circ\rho\dmo_g)=\Tr(\rho_h)\overset{\mbox{\scalebox{0.5}{$\Tr(AB)=\Tr(BA)$}}}{=}\chi_\rho(h)$
  	$\mathop{\oplus}_{\substack{x\in X}}$

$\mat(\rho_g)=(a_{ij}(g))_{\scriptsize \substack{1\leq i\leq d \\ 1\leq j\leq d}}$ et $\mat(\rho'_g)=(a'_{ij}(g))_{\scriptsize \substack{1\leq i'\leq d' \\ 1\leq j'\leq d'}}$



\[\int_a^b{\mathbb{R}^2}g(u, v)\dd{P_{XY}}(u, v)=\iint g(u,v) f_{XY}(u, v)\dd \lambda(u) \dd \lambda(v)\]
$$\lim_{x\to\infty} f(x)$$	
$$\iiiint_V \mu(t,u,v,w) \,dt\,du\,dv\,dw$$
$$\sum_{n=1}^{\infty} 2^{-n} = 1$$	
\begin{definition}
	Si $X$ et $Y$ sont 2 v.a. ou definit la \textsc{Covariance} entre $X$ et $Y$ comme
	$\cov(X,Y)\overset{\text{def}}{=}\E\left[(X-\E(X))(Y-\E(Y))\right]=\E(XY)-\E(X)\E(Y)$.
\end{definition}
\fi
\pagebreak

% \tableofcontents

% insert your code here
%\input{./algebra/main.tex}
%\input{./geometrie-differentielle/main.tex}
%\input{./probabilite/main.tex}
%\input{./analyse-fonctionnelle/main.tex}
% \input{./Analyse-convexe-et-dualite-en-optimisation/main.tex}
%\input{./tikz/main.tex}
%\input{./Theorie-du-distributions/main.tex}
%\input{./optimisation/mine.tex}
 \input{./modelisation/main.tex}

% yves.aubry@univ-tln.fr : algebra

\end{document}

%% !TEX encoding = UTF-8 Unicode
% !TEX TS-program = xelatex

\documentclass[french]{report}

%\usepackage[utf8]{inputenc}
%\usepackage[T1]{fontenc}
\usepackage{babel}


\newif\ifcomment
%\commenttrue # Show comments

\usepackage{physics}
\usepackage{amssymb}


\usepackage{amsthm}
% \usepackage{thmtools}
\usepackage{mathtools}
\usepackage{amsfonts}

\usepackage{color}

\usepackage{tikz}

\usepackage{geometry}
\geometry{a5paper, margin=0.1in, right=1cm}

\usepackage{dsfont}

\usepackage{graphicx}
\graphicspath{ {images/} }

\usepackage{faktor}

\usepackage{IEEEtrantools}
\usepackage{enumerate}   
\usepackage[PostScript=dvips]{"/Users/aware/Documents/Courses/diagrams"}


\newtheorem{theorem}{Théorème}[section]
\renewcommand{\thetheorem}{\arabic{theorem}}
\newtheorem{lemme}{Lemme}[section]
\renewcommand{\thelemme}{\arabic{lemme}}
\newtheorem{proposition}{Proposition}[section]
\renewcommand{\theproposition}{\arabic{proposition}}
\newtheorem{notations}{Notations}[section]
\newtheorem{problem}{Problème}[section]
\newtheorem{corollary}{Corollaire}[theorem]
\renewcommand{\thecorollary}{\arabic{corollary}}
\newtheorem{property}{Propriété}[section]
\newtheorem{objective}{Objectif}[section]

\theoremstyle{definition}
\newtheorem{definition}{Définition}[section]
\renewcommand{\thedefinition}{\arabic{definition}}
\newtheorem{exercise}{Exercice}[chapter]
\renewcommand{\theexercise}{\arabic{exercise}}
\newtheorem{example}{Exemple}[chapter]
\renewcommand{\theexample}{\arabic{example}}
\newtheorem*{solution}{Solution}
\newtheorem*{application}{Application}
\newtheorem*{notation}{Notation}
\newtheorem*{vocabulary}{Vocabulaire}
\newtheorem*{properties}{Propriétés}



\theoremstyle{remark}
\newtheorem*{remark}{Remarque}
\newtheorem*{rappel}{Rappel}


\usepackage{etoolbox}
\AtBeginEnvironment{exercise}{\small}
\AtBeginEnvironment{example}{\small}

\usepackage{cases}
\usepackage[red]{mypack}

\usepackage[framemethod=TikZ]{mdframed}

\definecolor{bg}{rgb}{0.4,0.25,0.95}
\definecolor{pagebg}{rgb}{0,0,0.5}
\surroundwithmdframed[
   topline=false,
   rightline=false,
   bottomline=false,
   leftmargin=\parindent,
   skipabove=8pt,
   skipbelow=8pt,
   linecolor=blue,
   innerbottommargin=10pt,
   % backgroundcolor=bg,font=\color{orange}\sffamily, fontcolor=white
]{definition}

\usepackage{empheq}
\usepackage[most]{tcolorbox}

\newtcbox{\mymath}[1][]{%
    nobeforeafter, math upper, tcbox raise base,
    enhanced, colframe=blue!30!black,
    colback=red!10, boxrule=1pt,
    #1}

\usepackage{unixode}


\DeclareMathOperator{\ord}{ord}
\DeclareMathOperator{\orb}{orb}
\DeclareMathOperator{\stab}{stab}
\DeclareMathOperator{\Stab}{stab}
\DeclareMathOperator{\ppcm}{ppcm}
\DeclareMathOperator{\conj}{Conj}
\DeclareMathOperator{\End}{End}
\DeclareMathOperator{\rot}{rot}
\DeclareMathOperator{\trs}{trace}
\DeclareMathOperator{\Ind}{Ind}
\DeclareMathOperator{\mat}{Mat}
\DeclareMathOperator{\id}{Id}
\DeclareMathOperator{\vect}{vect}
\DeclareMathOperator{\img}{img}
\DeclareMathOperator{\cov}{Cov}
\DeclareMathOperator{\dist}{dist}
\DeclareMathOperator{\irr}{Irr}
\DeclareMathOperator{\image}{Im}
\DeclareMathOperator{\pd}{\partial}
\DeclareMathOperator{\epi}{epi}
\DeclareMathOperator{\Argmin}{Argmin}
\DeclareMathOperator{\dom}{dom}
\DeclareMathOperator{\proj}{proj}
\DeclareMathOperator{\ctg}{ctg}
\DeclareMathOperator{\supp}{supp}
\DeclareMathOperator{\argmin}{argmin}
\DeclareMathOperator{\mult}{mult}
\DeclareMathOperator{\ch}{ch}
\DeclareMathOperator{\sh}{sh}
\DeclareMathOperator{\rang}{rang}
\DeclareMathOperator{\diam}{diam}
\DeclareMathOperator{\Epigraphe}{Epigraphe}




\usepackage{xcolor}
\everymath{\color{blue}}
%\everymath{\color[rgb]{0,1,1}}
%\pagecolor[rgb]{0,0,0.5}


\newcommand*{\pdtest}[3][]{\ensuremath{\frac{\partial^{#1} #2}{\partial #3}}}

\newcommand*{\deffunc}[6][]{\ensuremath{
\begin{array}{rcl}
#2 : #3 &\rightarrow& #4\\
#5 &\mapsto& #6
\end{array}
}}

\newcommand{\eqcolon}{\mathrel{\resizebox{\widthof{$\mathord{=}$}}{\height}{ $\!\!=\!\!\resizebox{1.2\width}{0.8\height}{\raisebox{0.23ex}{$\mathop{:}$}}\!\!$ }}}
\newcommand{\coloneq}{\mathrel{\resizebox{\widthof{$\mathord{=}$}}{\height}{ $\!\!\resizebox{1.2\width}{0.8\height}{\raisebox{0.23ex}{$\mathop{:}$}}\!\!=\!\!$ }}}
\newcommand{\eqcolonl}{\ensuremath{\mathrel{=\!\!\mathop{:}}}}
\newcommand{\coloneql}{\ensuremath{\mathrel{\mathop{:} \!\! =}}}
\newcommand{\vc}[1]{% inline column vector
  \left(\begin{smallmatrix}#1\end{smallmatrix}\right)%
}
\newcommand{\vr}[1]{% inline row vector
  \begin{smallmatrix}(\,#1\,)\end{smallmatrix}%
}
\makeatletter
\newcommand*{\defeq}{\ =\mathrel{\rlap{%
                     \raisebox{0.3ex}{$\m@th\cdot$}}%
                     \raisebox{-0.3ex}{$\m@th\cdot$}}%
                     }
\makeatother

\newcommand{\mathcircle}[1]{% inline row vector
 \overset{\circ}{#1}
}
\newcommand{\ulim}{% low limit
 \underline{\lim}
}
\newcommand{\ssi}{% iff
\iff
}
\newcommand{\ps}[2]{
\expval{#1 | #2}
}
\newcommand{\df}[1]{
\mqty{#1}
}
\newcommand{\n}[1]{
\norm{#1}
}
\newcommand{\sys}[1]{
\left\{\smqty{#1}\right.
}


\newcommand{\eqdef}{\ensuremath{\overset{\text{def}}=}}


\def\Circlearrowright{\ensuremath{%
  \rotatebox[origin=c]{230}{$\circlearrowright$}}}

\newcommand\ct[1]{\text{\rmfamily\upshape #1}}
\newcommand\question[1]{ {\color{red} ...!? \small #1}}
\newcommand\caz[1]{\left\{\begin{array} #1 \end{array}\right.}
\newcommand\const{\text{\rmfamily\upshape const}}
\newcommand\toP{ \overset{\pro}{\to}}
\newcommand\toPP{ \overset{\text{PP}}{\to}}
\newcommand{\oeq}{\mathrel{\text{\textcircled{$=$}}}}





\usepackage{xcolor}
% \usepackage[normalem]{ulem}
\usepackage{lipsum}
\makeatletter
% \newcommand\colorwave[1][blue]{\bgroup \markoverwith{\lower3.5\p@\hbox{\sixly \textcolor{#1}{\char58}}}\ULon}
%\font\sixly=lasy6 % does not re-load if already loaded, so no memory problem.

\newmdtheoremenv[
linewidth= 1pt,linecolor= blue,%
leftmargin=20,rightmargin=20,innertopmargin=0pt, innerrightmargin=40,%
tikzsetting = { draw=lightgray, line width = 0.3pt,dashed,%
dash pattern = on 15pt off 3pt},%
splittopskip=\topskip,skipbelow=\baselineskip,%
skipabove=\baselineskip,ntheorem,roundcorner=0pt,
% backgroundcolor=pagebg,font=\color{orange}\sffamily, fontcolor=white
]{examplebox}{Exemple}[section]



\newcommand\R{\mathbb{R}}
\newcommand\Z{\mathbb{Z}}
\newcommand\N{\mathbb{N}}
\newcommand\E{\mathbb{E}}
\newcommand\F{\mathcal{F}}
\newcommand\cH{\mathcal{H}}
\newcommand\V{\mathbb{V}}
\newcommand\dmo{ ^{-1} }
\newcommand\kapa{\kappa}
\newcommand\im{Im}
\newcommand\hs{\mathcal{H}}





\usepackage{soul}

\makeatletter
\newcommand*{\whiten}[1]{\llap{\textcolor{white}{{\the\SOUL@token}}\hspace{#1pt}}}
\DeclareRobustCommand*\myul{%
    \def\SOUL@everyspace{\underline{\space}\kern\z@}%
    \def\SOUL@everytoken{%
     \setbox0=\hbox{\the\SOUL@token}%
     \ifdim\dp0>\z@
        \raisebox{\dp0}{\underline{\phantom{\the\SOUL@token}}}%
        \whiten{1}\whiten{0}%
        \whiten{-1}\whiten{-2}%
        \llap{\the\SOUL@token}%
     \else
        \underline{\the\SOUL@token}%
     \fi}%
\SOUL@}
\makeatother

\newcommand*{\demp}{\fontfamily{lmtt}\selectfont}

\DeclareTextFontCommand{\textdemp}{\demp}

\begin{document}

\ifcomment
Multiline
comment
\fi
\ifcomment
\myul{Typesetting test}
% \color[rgb]{1,1,1}
$∑_i^n≠ 60º±∞π∆¬≈√j∫h≤≥µ$

$\CR \R\pro\ind\pro\gS\pro
\mqty[a&b\\c&d]$
$\pro\mathbb{P}$
$\dd{x}$

  \[
    \alpha(x)=\left\{
                \begin{array}{ll}
                  x\\
                  \frac{1}{1+e^{-kx}}\\
                  \frac{e^x-e^{-x}}{e^x+e^{-x}}
                \end{array}
              \right.
  \]

  $\expval{x}$
  
  $\chi_\rho(ghg\dmo)=\Tr(\rho_{ghg\dmo})=\Tr(\rho_g\circ\rho_h\circ\rho\dmo_g)=\Tr(\rho_h)\overset{\mbox{\scalebox{0.5}{$\Tr(AB)=\Tr(BA)$}}}{=}\chi_\rho(h)$
  	$\mathop{\oplus}_{\substack{x\in X}}$

$\mat(\rho_g)=(a_{ij}(g))_{\scriptsize \substack{1\leq i\leq d \\ 1\leq j\leq d}}$ et $\mat(\rho'_g)=(a'_{ij}(g))_{\scriptsize \substack{1\leq i'\leq d' \\ 1\leq j'\leq d'}}$



\[\int_a^b{\mathbb{R}^2}g(u, v)\dd{P_{XY}}(u, v)=\iint g(u,v) f_{XY}(u, v)\dd \lambda(u) \dd \lambda(v)\]
$$\lim_{x\to\infty} f(x)$$	
$$\iiiint_V \mu(t,u,v,w) \,dt\,du\,dv\,dw$$
$$\sum_{n=1}^{\infty} 2^{-n} = 1$$	
\begin{definition}
	Si $X$ et $Y$ sont 2 v.a. ou definit la \textsc{Covariance} entre $X$ et $Y$ comme
	$\cov(X,Y)\overset{\text{def}}{=}\E\left[(X-\E(X))(Y-\E(Y))\right]=\E(XY)-\E(X)\E(Y)$.
\end{definition}
\fi
\pagebreak

% \tableofcontents

% insert your code here
%\input{./algebra/main.tex}
%\input{./geometrie-differentielle/main.tex}
%\input{./probabilite/main.tex}
%\input{./analyse-fonctionnelle/main.tex}
% \input{./Analyse-convexe-et-dualite-en-optimisation/main.tex}
%\input{./tikz/main.tex}
%\input{./Theorie-du-distributions/main.tex}
%\input{./optimisation/mine.tex}
 \input{./modelisation/main.tex}

% yves.aubry@univ-tln.fr : algebra

\end{document}

%% !TEX encoding = UTF-8 Unicode
% !TEX TS-program = xelatex

\documentclass[french]{report}

%\usepackage[utf8]{inputenc}
%\usepackage[T1]{fontenc}
\usepackage{babel}


\newif\ifcomment
%\commenttrue # Show comments

\usepackage{physics}
\usepackage{amssymb}


\usepackage{amsthm}
% \usepackage{thmtools}
\usepackage{mathtools}
\usepackage{amsfonts}

\usepackage{color}

\usepackage{tikz}

\usepackage{geometry}
\geometry{a5paper, margin=0.1in, right=1cm}

\usepackage{dsfont}

\usepackage{graphicx}
\graphicspath{ {images/} }

\usepackage{faktor}

\usepackage{IEEEtrantools}
\usepackage{enumerate}   
\usepackage[PostScript=dvips]{"/Users/aware/Documents/Courses/diagrams"}


\newtheorem{theorem}{Théorème}[section]
\renewcommand{\thetheorem}{\arabic{theorem}}
\newtheorem{lemme}{Lemme}[section]
\renewcommand{\thelemme}{\arabic{lemme}}
\newtheorem{proposition}{Proposition}[section]
\renewcommand{\theproposition}{\arabic{proposition}}
\newtheorem{notations}{Notations}[section]
\newtheorem{problem}{Problème}[section]
\newtheorem{corollary}{Corollaire}[theorem]
\renewcommand{\thecorollary}{\arabic{corollary}}
\newtheorem{property}{Propriété}[section]
\newtheorem{objective}{Objectif}[section]

\theoremstyle{definition}
\newtheorem{definition}{Définition}[section]
\renewcommand{\thedefinition}{\arabic{definition}}
\newtheorem{exercise}{Exercice}[chapter]
\renewcommand{\theexercise}{\arabic{exercise}}
\newtheorem{example}{Exemple}[chapter]
\renewcommand{\theexample}{\arabic{example}}
\newtheorem*{solution}{Solution}
\newtheorem*{application}{Application}
\newtheorem*{notation}{Notation}
\newtheorem*{vocabulary}{Vocabulaire}
\newtheorem*{properties}{Propriétés}



\theoremstyle{remark}
\newtheorem*{remark}{Remarque}
\newtheorem*{rappel}{Rappel}


\usepackage{etoolbox}
\AtBeginEnvironment{exercise}{\small}
\AtBeginEnvironment{example}{\small}

\usepackage{cases}
\usepackage[red]{mypack}

\usepackage[framemethod=TikZ]{mdframed}

\definecolor{bg}{rgb}{0.4,0.25,0.95}
\definecolor{pagebg}{rgb}{0,0,0.5}
\surroundwithmdframed[
   topline=false,
   rightline=false,
   bottomline=false,
   leftmargin=\parindent,
   skipabove=8pt,
   skipbelow=8pt,
   linecolor=blue,
   innerbottommargin=10pt,
   % backgroundcolor=bg,font=\color{orange}\sffamily, fontcolor=white
]{definition}

\usepackage{empheq}
\usepackage[most]{tcolorbox}

\newtcbox{\mymath}[1][]{%
    nobeforeafter, math upper, tcbox raise base,
    enhanced, colframe=blue!30!black,
    colback=red!10, boxrule=1pt,
    #1}

\usepackage{unixode}


\DeclareMathOperator{\ord}{ord}
\DeclareMathOperator{\orb}{orb}
\DeclareMathOperator{\stab}{stab}
\DeclareMathOperator{\Stab}{stab}
\DeclareMathOperator{\ppcm}{ppcm}
\DeclareMathOperator{\conj}{Conj}
\DeclareMathOperator{\End}{End}
\DeclareMathOperator{\rot}{rot}
\DeclareMathOperator{\trs}{trace}
\DeclareMathOperator{\Ind}{Ind}
\DeclareMathOperator{\mat}{Mat}
\DeclareMathOperator{\id}{Id}
\DeclareMathOperator{\vect}{vect}
\DeclareMathOperator{\img}{img}
\DeclareMathOperator{\cov}{Cov}
\DeclareMathOperator{\dist}{dist}
\DeclareMathOperator{\irr}{Irr}
\DeclareMathOperator{\image}{Im}
\DeclareMathOperator{\pd}{\partial}
\DeclareMathOperator{\epi}{epi}
\DeclareMathOperator{\Argmin}{Argmin}
\DeclareMathOperator{\dom}{dom}
\DeclareMathOperator{\proj}{proj}
\DeclareMathOperator{\ctg}{ctg}
\DeclareMathOperator{\supp}{supp}
\DeclareMathOperator{\argmin}{argmin}
\DeclareMathOperator{\mult}{mult}
\DeclareMathOperator{\ch}{ch}
\DeclareMathOperator{\sh}{sh}
\DeclareMathOperator{\rang}{rang}
\DeclareMathOperator{\diam}{diam}
\DeclareMathOperator{\Epigraphe}{Epigraphe}




\usepackage{xcolor}
\everymath{\color{blue}}
%\everymath{\color[rgb]{0,1,1}}
%\pagecolor[rgb]{0,0,0.5}


\newcommand*{\pdtest}[3][]{\ensuremath{\frac{\partial^{#1} #2}{\partial #3}}}

\newcommand*{\deffunc}[6][]{\ensuremath{
\begin{array}{rcl}
#2 : #3 &\rightarrow& #4\\
#5 &\mapsto& #6
\end{array}
}}

\newcommand{\eqcolon}{\mathrel{\resizebox{\widthof{$\mathord{=}$}}{\height}{ $\!\!=\!\!\resizebox{1.2\width}{0.8\height}{\raisebox{0.23ex}{$\mathop{:}$}}\!\!$ }}}
\newcommand{\coloneq}{\mathrel{\resizebox{\widthof{$\mathord{=}$}}{\height}{ $\!\!\resizebox{1.2\width}{0.8\height}{\raisebox{0.23ex}{$\mathop{:}$}}\!\!=\!\!$ }}}
\newcommand{\eqcolonl}{\ensuremath{\mathrel{=\!\!\mathop{:}}}}
\newcommand{\coloneql}{\ensuremath{\mathrel{\mathop{:} \!\! =}}}
\newcommand{\vc}[1]{% inline column vector
  \left(\begin{smallmatrix}#1\end{smallmatrix}\right)%
}
\newcommand{\vr}[1]{% inline row vector
  \begin{smallmatrix}(\,#1\,)\end{smallmatrix}%
}
\makeatletter
\newcommand*{\defeq}{\ =\mathrel{\rlap{%
                     \raisebox{0.3ex}{$\m@th\cdot$}}%
                     \raisebox{-0.3ex}{$\m@th\cdot$}}%
                     }
\makeatother

\newcommand{\mathcircle}[1]{% inline row vector
 \overset{\circ}{#1}
}
\newcommand{\ulim}{% low limit
 \underline{\lim}
}
\newcommand{\ssi}{% iff
\iff
}
\newcommand{\ps}[2]{
\expval{#1 | #2}
}
\newcommand{\df}[1]{
\mqty{#1}
}
\newcommand{\n}[1]{
\norm{#1}
}
\newcommand{\sys}[1]{
\left\{\smqty{#1}\right.
}


\newcommand{\eqdef}{\ensuremath{\overset{\text{def}}=}}


\def\Circlearrowright{\ensuremath{%
  \rotatebox[origin=c]{230}{$\circlearrowright$}}}

\newcommand\ct[1]{\text{\rmfamily\upshape #1}}
\newcommand\question[1]{ {\color{red} ...!? \small #1}}
\newcommand\caz[1]{\left\{\begin{array} #1 \end{array}\right.}
\newcommand\const{\text{\rmfamily\upshape const}}
\newcommand\toP{ \overset{\pro}{\to}}
\newcommand\toPP{ \overset{\text{PP}}{\to}}
\newcommand{\oeq}{\mathrel{\text{\textcircled{$=$}}}}





\usepackage{xcolor}
% \usepackage[normalem]{ulem}
\usepackage{lipsum}
\makeatletter
% \newcommand\colorwave[1][blue]{\bgroup \markoverwith{\lower3.5\p@\hbox{\sixly \textcolor{#1}{\char58}}}\ULon}
%\font\sixly=lasy6 % does not re-load if already loaded, so no memory problem.

\newmdtheoremenv[
linewidth= 1pt,linecolor= blue,%
leftmargin=20,rightmargin=20,innertopmargin=0pt, innerrightmargin=40,%
tikzsetting = { draw=lightgray, line width = 0.3pt,dashed,%
dash pattern = on 15pt off 3pt},%
splittopskip=\topskip,skipbelow=\baselineskip,%
skipabove=\baselineskip,ntheorem,roundcorner=0pt,
% backgroundcolor=pagebg,font=\color{orange}\sffamily, fontcolor=white
]{examplebox}{Exemple}[section]



\newcommand\R{\mathbb{R}}
\newcommand\Z{\mathbb{Z}}
\newcommand\N{\mathbb{N}}
\newcommand\E{\mathbb{E}}
\newcommand\F{\mathcal{F}}
\newcommand\cH{\mathcal{H}}
\newcommand\V{\mathbb{V}}
\newcommand\dmo{ ^{-1} }
\newcommand\kapa{\kappa}
\newcommand\im{Im}
\newcommand\hs{\mathcal{H}}





\usepackage{soul}

\makeatletter
\newcommand*{\whiten}[1]{\llap{\textcolor{white}{{\the\SOUL@token}}\hspace{#1pt}}}
\DeclareRobustCommand*\myul{%
    \def\SOUL@everyspace{\underline{\space}\kern\z@}%
    \def\SOUL@everytoken{%
     \setbox0=\hbox{\the\SOUL@token}%
     \ifdim\dp0>\z@
        \raisebox{\dp0}{\underline{\phantom{\the\SOUL@token}}}%
        \whiten{1}\whiten{0}%
        \whiten{-1}\whiten{-2}%
        \llap{\the\SOUL@token}%
     \else
        \underline{\the\SOUL@token}%
     \fi}%
\SOUL@}
\makeatother

\newcommand*{\demp}{\fontfamily{lmtt}\selectfont}

\DeclareTextFontCommand{\textdemp}{\demp}

\begin{document}

\ifcomment
Multiline
comment
\fi
\ifcomment
\myul{Typesetting test}
% \color[rgb]{1,1,1}
$∑_i^n≠ 60º±∞π∆¬≈√j∫h≤≥µ$

$\CR \R\pro\ind\pro\gS\pro
\mqty[a&b\\c&d]$
$\pro\mathbb{P}$
$\dd{x}$

  \[
    \alpha(x)=\left\{
                \begin{array}{ll}
                  x\\
                  \frac{1}{1+e^{-kx}}\\
                  \frac{e^x-e^{-x}}{e^x+e^{-x}}
                \end{array}
              \right.
  \]

  $\expval{x}$
  
  $\chi_\rho(ghg\dmo)=\Tr(\rho_{ghg\dmo})=\Tr(\rho_g\circ\rho_h\circ\rho\dmo_g)=\Tr(\rho_h)\overset{\mbox{\scalebox{0.5}{$\Tr(AB)=\Tr(BA)$}}}{=}\chi_\rho(h)$
  	$\mathop{\oplus}_{\substack{x\in X}}$

$\mat(\rho_g)=(a_{ij}(g))_{\scriptsize \substack{1\leq i\leq d \\ 1\leq j\leq d}}$ et $\mat(\rho'_g)=(a'_{ij}(g))_{\scriptsize \substack{1\leq i'\leq d' \\ 1\leq j'\leq d'}}$



\[\int_a^b{\mathbb{R}^2}g(u, v)\dd{P_{XY}}(u, v)=\iint g(u,v) f_{XY}(u, v)\dd \lambda(u) \dd \lambda(v)\]
$$\lim_{x\to\infty} f(x)$$	
$$\iiiint_V \mu(t,u,v,w) \,dt\,du\,dv\,dw$$
$$\sum_{n=1}^{\infty} 2^{-n} = 1$$	
\begin{definition}
	Si $X$ et $Y$ sont 2 v.a. ou definit la \textsc{Covariance} entre $X$ et $Y$ comme
	$\cov(X,Y)\overset{\text{def}}{=}\E\left[(X-\E(X))(Y-\E(Y))\right]=\E(XY)-\E(X)\E(Y)$.
\end{definition}
\fi
\pagebreak

% \tableofcontents

% insert your code here
%\input{./algebra/main.tex}
%\input{./geometrie-differentielle/main.tex}
%\input{./probabilite/main.tex}
%\input{./analyse-fonctionnelle/main.tex}
% \input{./Analyse-convexe-et-dualite-en-optimisation/main.tex}
%\input{./tikz/main.tex}
%\input{./Theorie-du-distributions/main.tex}
%\input{./optimisation/mine.tex}
 \input{./modelisation/main.tex}

% yves.aubry@univ-tln.fr : algebra

\end{document}

% % !TEX encoding = UTF-8 Unicode
% !TEX TS-program = xelatex

\documentclass[french]{report}

%\usepackage[utf8]{inputenc}
%\usepackage[T1]{fontenc}
\usepackage{babel}


\newif\ifcomment
%\commenttrue # Show comments

\usepackage{physics}
\usepackage{amssymb}


\usepackage{amsthm}
% \usepackage{thmtools}
\usepackage{mathtools}
\usepackage{amsfonts}

\usepackage{color}

\usepackage{tikz}

\usepackage{geometry}
\geometry{a5paper, margin=0.1in, right=1cm}

\usepackage{dsfont}

\usepackage{graphicx}
\graphicspath{ {images/} }

\usepackage{faktor}

\usepackage{IEEEtrantools}
\usepackage{enumerate}   
\usepackage[PostScript=dvips]{"/Users/aware/Documents/Courses/diagrams"}


\newtheorem{theorem}{Théorème}[section]
\renewcommand{\thetheorem}{\arabic{theorem}}
\newtheorem{lemme}{Lemme}[section]
\renewcommand{\thelemme}{\arabic{lemme}}
\newtheorem{proposition}{Proposition}[section]
\renewcommand{\theproposition}{\arabic{proposition}}
\newtheorem{notations}{Notations}[section]
\newtheorem{problem}{Problème}[section]
\newtheorem{corollary}{Corollaire}[theorem]
\renewcommand{\thecorollary}{\arabic{corollary}}
\newtheorem{property}{Propriété}[section]
\newtheorem{objective}{Objectif}[section]

\theoremstyle{definition}
\newtheorem{definition}{Définition}[section]
\renewcommand{\thedefinition}{\arabic{definition}}
\newtheorem{exercise}{Exercice}[chapter]
\renewcommand{\theexercise}{\arabic{exercise}}
\newtheorem{example}{Exemple}[chapter]
\renewcommand{\theexample}{\arabic{example}}
\newtheorem*{solution}{Solution}
\newtheorem*{application}{Application}
\newtheorem*{notation}{Notation}
\newtheorem*{vocabulary}{Vocabulaire}
\newtheorem*{properties}{Propriétés}



\theoremstyle{remark}
\newtheorem*{remark}{Remarque}
\newtheorem*{rappel}{Rappel}


\usepackage{etoolbox}
\AtBeginEnvironment{exercise}{\small}
\AtBeginEnvironment{example}{\small}

\usepackage{cases}
\usepackage[red]{mypack}

\usepackage[framemethod=TikZ]{mdframed}

\definecolor{bg}{rgb}{0.4,0.25,0.95}
\definecolor{pagebg}{rgb}{0,0,0.5}
\surroundwithmdframed[
   topline=false,
   rightline=false,
   bottomline=false,
   leftmargin=\parindent,
   skipabove=8pt,
   skipbelow=8pt,
   linecolor=blue,
   innerbottommargin=10pt,
   % backgroundcolor=bg,font=\color{orange}\sffamily, fontcolor=white
]{definition}

\usepackage{empheq}
\usepackage[most]{tcolorbox}

\newtcbox{\mymath}[1][]{%
    nobeforeafter, math upper, tcbox raise base,
    enhanced, colframe=blue!30!black,
    colback=red!10, boxrule=1pt,
    #1}

\usepackage{unixode}


\DeclareMathOperator{\ord}{ord}
\DeclareMathOperator{\orb}{orb}
\DeclareMathOperator{\stab}{stab}
\DeclareMathOperator{\Stab}{stab}
\DeclareMathOperator{\ppcm}{ppcm}
\DeclareMathOperator{\conj}{Conj}
\DeclareMathOperator{\End}{End}
\DeclareMathOperator{\rot}{rot}
\DeclareMathOperator{\trs}{trace}
\DeclareMathOperator{\Ind}{Ind}
\DeclareMathOperator{\mat}{Mat}
\DeclareMathOperator{\id}{Id}
\DeclareMathOperator{\vect}{vect}
\DeclareMathOperator{\img}{img}
\DeclareMathOperator{\cov}{Cov}
\DeclareMathOperator{\dist}{dist}
\DeclareMathOperator{\irr}{Irr}
\DeclareMathOperator{\image}{Im}
\DeclareMathOperator{\pd}{\partial}
\DeclareMathOperator{\epi}{epi}
\DeclareMathOperator{\Argmin}{Argmin}
\DeclareMathOperator{\dom}{dom}
\DeclareMathOperator{\proj}{proj}
\DeclareMathOperator{\ctg}{ctg}
\DeclareMathOperator{\supp}{supp}
\DeclareMathOperator{\argmin}{argmin}
\DeclareMathOperator{\mult}{mult}
\DeclareMathOperator{\ch}{ch}
\DeclareMathOperator{\sh}{sh}
\DeclareMathOperator{\rang}{rang}
\DeclareMathOperator{\diam}{diam}
\DeclareMathOperator{\Epigraphe}{Epigraphe}




\usepackage{xcolor}
\everymath{\color{blue}}
%\everymath{\color[rgb]{0,1,1}}
%\pagecolor[rgb]{0,0,0.5}


\newcommand*{\pdtest}[3][]{\ensuremath{\frac{\partial^{#1} #2}{\partial #3}}}

\newcommand*{\deffunc}[6][]{\ensuremath{
\begin{array}{rcl}
#2 : #3 &\rightarrow& #4\\
#5 &\mapsto& #6
\end{array}
}}

\newcommand{\eqcolon}{\mathrel{\resizebox{\widthof{$\mathord{=}$}}{\height}{ $\!\!=\!\!\resizebox{1.2\width}{0.8\height}{\raisebox{0.23ex}{$\mathop{:}$}}\!\!$ }}}
\newcommand{\coloneq}{\mathrel{\resizebox{\widthof{$\mathord{=}$}}{\height}{ $\!\!\resizebox{1.2\width}{0.8\height}{\raisebox{0.23ex}{$\mathop{:}$}}\!\!=\!\!$ }}}
\newcommand{\eqcolonl}{\ensuremath{\mathrel{=\!\!\mathop{:}}}}
\newcommand{\coloneql}{\ensuremath{\mathrel{\mathop{:} \!\! =}}}
\newcommand{\vc}[1]{% inline column vector
  \left(\begin{smallmatrix}#1\end{smallmatrix}\right)%
}
\newcommand{\vr}[1]{% inline row vector
  \begin{smallmatrix}(\,#1\,)\end{smallmatrix}%
}
\makeatletter
\newcommand*{\defeq}{\ =\mathrel{\rlap{%
                     \raisebox{0.3ex}{$\m@th\cdot$}}%
                     \raisebox{-0.3ex}{$\m@th\cdot$}}%
                     }
\makeatother

\newcommand{\mathcircle}[1]{% inline row vector
 \overset{\circ}{#1}
}
\newcommand{\ulim}{% low limit
 \underline{\lim}
}
\newcommand{\ssi}{% iff
\iff
}
\newcommand{\ps}[2]{
\expval{#1 | #2}
}
\newcommand{\df}[1]{
\mqty{#1}
}
\newcommand{\n}[1]{
\norm{#1}
}
\newcommand{\sys}[1]{
\left\{\smqty{#1}\right.
}


\newcommand{\eqdef}{\ensuremath{\overset{\text{def}}=}}


\def\Circlearrowright{\ensuremath{%
  \rotatebox[origin=c]{230}{$\circlearrowright$}}}

\newcommand\ct[1]{\text{\rmfamily\upshape #1}}
\newcommand\question[1]{ {\color{red} ...!? \small #1}}
\newcommand\caz[1]{\left\{\begin{array} #1 \end{array}\right.}
\newcommand\const{\text{\rmfamily\upshape const}}
\newcommand\toP{ \overset{\pro}{\to}}
\newcommand\toPP{ \overset{\text{PP}}{\to}}
\newcommand{\oeq}{\mathrel{\text{\textcircled{$=$}}}}





\usepackage{xcolor}
% \usepackage[normalem]{ulem}
\usepackage{lipsum}
\makeatletter
% \newcommand\colorwave[1][blue]{\bgroup \markoverwith{\lower3.5\p@\hbox{\sixly \textcolor{#1}{\char58}}}\ULon}
%\font\sixly=lasy6 % does not re-load if already loaded, so no memory problem.

\newmdtheoremenv[
linewidth= 1pt,linecolor= blue,%
leftmargin=20,rightmargin=20,innertopmargin=0pt, innerrightmargin=40,%
tikzsetting = { draw=lightgray, line width = 0.3pt,dashed,%
dash pattern = on 15pt off 3pt},%
splittopskip=\topskip,skipbelow=\baselineskip,%
skipabove=\baselineskip,ntheorem,roundcorner=0pt,
% backgroundcolor=pagebg,font=\color{orange}\sffamily, fontcolor=white
]{examplebox}{Exemple}[section]



\newcommand\R{\mathbb{R}}
\newcommand\Z{\mathbb{Z}}
\newcommand\N{\mathbb{N}}
\newcommand\E{\mathbb{E}}
\newcommand\F{\mathcal{F}}
\newcommand\cH{\mathcal{H}}
\newcommand\V{\mathbb{V}}
\newcommand\dmo{ ^{-1} }
\newcommand\kapa{\kappa}
\newcommand\im{Im}
\newcommand\hs{\mathcal{H}}





\usepackage{soul}

\makeatletter
\newcommand*{\whiten}[1]{\llap{\textcolor{white}{{\the\SOUL@token}}\hspace{#1pt}}}
\DeclareRobustCommand*\myul{%
    \def\SOUL@everyspace{\underline{\space}\kern\z@}%
    \def\SOUL@everytoken{%
     \setbox0=\hbox{\the\SOUL@token}%
     \ifdim\dp0>\z@
        \raisebox{\dp0}{\underline{\phantom{\the\SOUL@token}}}%
        \whiten{1}\whiten{0}%
        \whiten{-1}\whiten{-2}%
        \llap{\the\SOUL@token}%
     \else
        \underline{\the\SOUL@token}%
     \fi}%
\SOUL@}
\makeatother

\newcommand*{\demp}{\fontfamily{lmtt}\selectfont}

\DeclareTextFontCommand{\textdemp}{\demp}

\begin{document}

\ifcomment
Multiline
comment
\fi
\ifcomment
\myul{Typesetting test}
% \color[rgb]{1,1,1}
$∑_i^n≠ 60º±∞π∆¬≈√j∫h≤≥µ$

$\CR \R\pro\ind\pro\gS\pro
\mqty[a&b\\c&d]$
$\pro\mathbb{P}$
$\dd{x}$

  \[
    \alpha(x)=\left\{
                \begin{array}{ll}
                  x\\
                  \frac{1}{1+e^{-kx}}\\
                  \frac{e^x-e^{-x}}{e^x+e^{-x}}
                \end{array}
              \right.
  \]

  $\expval{x}$
  
  $\chi_\rho(ghg\dmo)=\Tr(\rho_{ghg\dmo})=\Tr(\rho_g\circ\rho_h\circ\rho\dmo_g)=\Tr(\rho_h)\overset{\mbox{\scalebox{0.5}{$\Tr(AB)=\Tr(BA)$}}}{=}\chi_\rho(h)$
  	$\mathop{\oplus}_{\substack{x\in X}}$

$\mat(\rho_g)=(a_{ij}(g))_{\scriptsize \substack{1\leq i\leq d \\ 1\leq j\leq d}}$ et $\mat(\rho'_g)=(a'_{ij}(g))_{\scriptsize \substack{1\leq i'\leq d' \\ 1\leq j'\leq d'}}$



\[\int_a^b{\mathbb{R}^2}g(u, v)\dd{P_{XY}}(u, v)=\iint g(u,v) f_{XY}(u, v)\dd \lambda(u) \dd \lambda(v)\]
$$\lim_{x\to\infty} f(x)$$	
$$\iiiint_V \mu(t,u,v,w) \,dt\,du\,dv\,dw$$
$$\sum_{n=1}^{\infty} 2^{-n} = 1$$	
\begin{definition}
	Si $X$ et $Y$ sont 2 v.a. ou definit la \textsc{Covariance} entre $X$ et $Y$ comme
	$\cov(X,Y)\overset{\text{def}}{=}\E\left[(X-\E(X))(Y-\E(Y))\right]=\E(XY)-\E(X)\E(Y)$.
\end{definition}
\fi
\pagebreak

% \tableofcontents

% insert your code here
%\input{./algebra/main.tex}
%\input{./geometrie-differentielle/main.tex}
%\input{./probabilite/main.tex}
%\input{./analyse-fonctionnelle/main.tex}
% \input{./Analyse-convexe-et-dualite-en-optimisation/main.tex}
%\input{./tikz/main.tex}
%\input{./Theorie-du-distributions/main.tex}
%\input{./optimisation/mine.tex}
 \input{./modelisation/main.tex}

% yves.aubry@univ-tln.fr : algebra

\end{document}

%% !TEX encoding = UTF-8 Unicode
% !TEX TS-program = xelatex

\documentclass[french]{report}

%\usepackage[utf8]{inputenc}
%\usepackage[T1]{fontenc}
\usepackage{babel}


\newif\ifcomment
%\commenttrue # Show comments

\usepackage{physics}
\usepackage{amssymb}


\usepackage{amsthm}
% \usepackage{thmtools}
\usepackage{mathtools}
\usepackage{amsfonts}

\usepackage{color}

\usepackage{tikz}

\usepackage{geometry}
\geometry{a5paper, margin=0.1in, right=1cm}

\usepackage{dsfont}

\usepackage{graphicx}
\graphicspath{ {images/} }

\usepackage{faktor}

\usepackage{IEEEtrantools}
\usepackage{enumerate}   
\usepackage[PostScript=dvips]{"/Users/aware/Documents/Courses/diagrams"}


\newtheorem{theorem}{Théorème}[section]
\renewcommand{\thetheorem}{\arabic{theorem}}
\newtheorem{lemme}{Lemme}[section]
\renewcommand{\thelemme}{\arabic{lemme}}
\newtheorem{proposition}{Proposition}[section]
\renewcommand{\theproposition}{\arabic{proposition}}
\newtheorem{notations}{Notations}[section]
\newtheorem{problem}{Problème}[section]
\newtheorem{corollary}{Corollaire}[theorem]
\renewcommand{\thecorollary}{\arabic{corollary}}
\newtheorem{property}{Propriété}[section]
\newtheorem{objective}{Objectif}[section]

\theoremstyle{definition}
\newtheorem{definition}{Définition}[section]
\renewcommand{\thedefinition}{\arabic{definition}}
\newtheorem{exercise}{Exercice}[chapter]
\renewcommand{\theexercise}{\arabic{exercise}}
\newtheorem{example}{Exemple}[chapter]
\renewcommand{\theexample}{\arabic{example}}
\newtheorem*{solution}{Solution}
\newtheorem*{application}{Application}
\newtheorem*{notation}{Notation}
\newtheorem*{vocabulary}{Vocabulaire}
\newtheorem*{properties}{Propriétés}



\theoremstyle{remark}
\newtheorem*{remark}{Remarque}
\newtheorem*{rappel}{Rappel}


\usepackage{etoolbox}
\AtBeginEnvironment{exercise}{\small}
\AtBeginEnvironment{example}{\small}

\usepackage{cases}
\usepackage[red]{mypack}

\usepackage[framemethod=TikZ]{mdframed}

\definecolor{bg}{rgb}{0.4,0.25,0.95}
\definecolor{pagebg}{rgb}{0,0,0.5}
\surroundwithmdframed[
   topline=false,
   rightline=false,
   bottomline=false,
   leftmargin=\parindent,
   skipabove=8pt,
   skipbelow=8pt,
   linecolor=blue,
   innerbottommargin=10pt,
   % backgroundcolor=bg,font=\color{orange}\sffamily, fontcolor=white
]{definition}

\usepackage{empheq}
\usepackage[most]{tcolorbox}

\newtcbox{\mymath}[1][]{%
    nobeforeafter, math upper, tcbox raise base,
    enhanced, colframe=blue!30!black,
    colback=red!10, boxrule=1pt,
    #1}

\usepackage{unixode}


\DeclareMathOperator{\ord}{ord}
\DeclareMathOperator{\orb}{orb}
\DeclareMathOperator{\stab}{stab}
\DeclareMathOperator{\Stab}{stab}
\DeclareMathOperator{\ppcm}{ppcm}
\DeclareMathOperator{\conj}{Conj}
\DeclareMathOperator{\End}{End}
\DeclareMathOperator{\rot}{rot}
\DeclareMathOperator{\trs}{trace}
\DeclareMathOperator{\Ind}{Ind}
\DeclareMathOperator{\mat}{Mat}
\DeclareMathOperator{\id}{Id}
\DeclareMathOperator{\vect}{vect}
\DeclareMathOperator{\img}{img}
\DeclareMathOperator{\cov}{Cov}
\DeclareMathOperator{\dist}{dist}
\DeclareMathOperator{\irr}{Irr}
\DeclareMathOperator{\image}{Im}
\DeclareMathOperator{\pd}{\partial}
\DeclareMathOperator{\epi}{epi}
\DeclareMathOperator{\Argmin}{Argmin}
\DeclareMathOperator{\dom}{dom}
\DeclareMathOperator{\proj}{proj}
\DeclareMathOperator{\ctg}{ctg}
\DeclareMathOperator{\supp}{supp}
\DeclareMathOperator{\argmin}{argmin}
\DeclareMathOperator{\mult}{mult}
\DeclareMathOperator{\ch}{ch}
\DeclareMathOperator{\sh}{sh}
\DeclareMathOperator{\rang}{rang}
\DeclareMathOperator{\diam}{diam}
\DeclareMathOperator{\Epigraphe}{Epigraphe}




\usepackage{xcolor}
\everymath{\color{blue}}
%\everymath{\color[rgb]{0,1,1}}
%\pagecolor[rgb]{0,0,0.5}


\newcommand*{\pdtest}[3][]{\ensuremath{\frac{\partial^{#1} #2}{\partial #3}}}

\newcommand*{\deffunc}[6][]{\ensuremath{
\begin{array}{rcl}
#2 : #3 &\rightarrow& #4\\
#5 &\mapsto& #6
\end{array}
}}

\newcommand{\eqcolon}{\mathrel{\resizebox{\widthof{$\mathord{=}$}}{\height}{ $\!\!=\!\!\resizebox{1.2\width}{0.8\height}{\raisebox{0.23ex}{$\mathop{:}$}}\!\!$ }}}
\newcommand{\coloneq}{\mathrel{\resizebox{\widthof{$\mathord{=}$}}{\height}{ $\!\!\resizebox{1.2\width}{0.8\height}{\raisebox{0.23ex}{$\mathop{:}$}}\!\!=\!\!$ }}}
\newcommand{\eqcolonl}{\ensuremath{\mathrel{=\!\!\mathop{:}}}}
\newcommand{\coloneql}{\ensuremath{\mathrel{\mathop{:} \!\! =}}}
\newcommand{\vc}[1]{% inline column vector
  \left(\begin{smallmatrix}#1\end{smallmatrix}\right)%
}
\newcommand{\vr}[1]{% inline row vector
  \begin{smallmatrix}(\,#1\,)\end{smallmatrix}%
}
\makeatletter
\newcommand*{\defeq}{\ =\mathrel{\rlap{%
                     \raisebox{0.3ex}{$\m@th\cdot$}}%
                     \raisebox{-0.3ex}{$\m@th\cdot$}}%
                     }
\makeatother

\newcommand{\mathcircle}[1]{% inline row vector
 \overset{\circ}{#1}
}
\newcommand{\ulim}{% low limit
 \underline{\lim}
}
\newcommand{\ssi}{% iff
\iff
}
\newcommand{\ps}[2]{
\expval{#1 | #2}
}
\newcommand{\df}[1]{
\mqty{#1}
}
\newcommand{\n}[1]{
\norm{#1}
}
\newcommand{\sys}[1]{
\left\{\smqty{#1}\right.
}


\newcommand{\eqdef}{\ensuremath{\overset{\text{def}}=}}


\def\Circlearrowright{\ensuremath{%
  \rotatebox[origin=c]{230}{$\circlearrowright$}}}

\newcommand\ct[1]{\text{\rmfamily\upshape #1}}
\newcommand\question[1]{ {\color{red} ...!? \small #1}}
\newcommand\caz[1]{\left\{\begin{array} #1 \end{array}\right.}
\newcommand\const{\text{\rmfamily\upshape const}}
\newcommand\toP{ \overset{\pro}{\to}}
\newcommand\toPP{ \overset{\text{PP}}{\to}}
\newcommand{\oeq}{\mathrel{\text{\textcircled{$=$}}}}





\usepackage{xcolor}
% \usepackage[normalem]{ulem}
\usepackage{lipsum}
\makeatletter
% \newcommand\colorwave[1][blue]{\bgroup \markoverwith{\lower3.5\p@\hbox{\sixly \textcolor{#1}{\char58}}}\ULon}
%\font\sixly=lasy6 % does not re-load if already loaded, so no memory problem.

\newmdtheoremenv[
linewidth= 1pt,linecolor= blue,%
leftmargin=20,rightmargin=20,innertopmargin=0pt, innerrightmargin=40,%
tikzsetting = { draw=lightgray, line width = 0.3pt,dashed,%
dash pattern = on 15pt off 3pt},%
splittopskip=\topskip,skipbelow=\baselineskip,%
skipabove=\baselineskip,ntheorem,roundcorner=0pt,
% backgroundcolor=pagebg,font=\color{orange}\sffamily, fontcolor=white
]{examplebox}{Exemple}[section]



\newcommand\R{\mathbb{R}}
\newcommand\Z{\mathbb{Z}}
\newcommand\N{\mathbb{N}}
\newcommand\E{\mathbb{E}}
\newcommand\F{\mathcal{F}}
\newcommand\cH{\mathcal{H}}
\newcommand\V{\mathbb{V}}
\newcommand\dmo{ ^{-1} }
\newcommand\kapa{\kappa}
\newcommand\im{Im}
\newcommand\hs{\mathcal{H}}





\usepackage{soul}

\makeatletter
\newcommand*{\whiten}[1]{\llap{\textcolor{white}{{\the\SOUL@token}}\hspace{#1pt}}}
\DeclareRobustCommand*\myul{%
    \def\SOUL@everyspace{\underline{\space}\kern\z@}%
    \def\SOUL@everytoken{%
     \setbox0=\hbox{\the\SOUL@token}%
     \ifdim\dp0>\z@
        \raisebox{\dp0}{\underline{\phantom{\the\SOUL@token}}}%
        \whiten{1}\whiten{0}%
        \whiten{-1}\whiten{-2}%
        \llap{\the\SOUL@token}%
     \else
        \underline{\the\SOUL@token}%
     \fi}%
\SOUL@}
\makeatother

\newcommand*{\demp}{\fontfamily{lmtt}\selectfont}

\DeclareTextFontCommand{\textdemp}{\demp}

\begin{document}

\ifcomment
Multiline
comment
\fi
\ifcomment
\myul{Typesetting test}
% \color[rgb]{1,1,1}
$∑_i^n≠ 60º±∞π∆¬≈√j∫h≤≥µ$

$\CR \R\pro\ind\pro\gS\pro
\mqty[a&b\\c&d]$
$\pro\mathbb{P}$
$\dd{x}$

  \[
    \alpha(x)=\left\{
                \begin{array}{ll}
                  x\\
                  \frac{1}{1+e^{-kx}}\\
                  \frac{e^x-e^{-x}}{e^x+e^{-x}}
                \end{array}
              \right.
  \]

  $\expval{x}$
  
  $\chi_\rho(ghg\dmo)=\Tr(\rho_{ghg\dmo})=\Tr(\rho_g\circ\rho_h\circ\rho\dmo_g)=\Tr(\rho_h)\overset{\mbox{\scalebox{0.5}{$\Tr(AB)=\Tr(BA)$}}}{=}\chi_\rho(h)$
  	$\mathop{\oplus}_{\substack{x\in X}}$

$\mat(\rho_g)=(a_{ij}(g))_{\scriptsize \substack{1\leq i\leq d \\ 1\leq j\leq d}}$ et $\mat(\rho'_g)=(a'_{ij}(g))_{\scriptsize \substack{1\leq i'\leq d' \\ 1\leq j'\leq d'}}$



\[\int_a^b{\mathbb{R}^2}g(u, v)\dd{P_{XY}}(u, v)=\iint g(u,v) f_{XY}(u, v)\dd \lambda(u) \dd \lambda(v)\]
$$\lim_{x\to\infty} f(x)$$	
$$\iiiint_V \mu(t,u,v,w) \,dt\,du\,dv\,dw$$
$$\sum_{n=1}^{\infty} 2^{-n} = 1$$	
\begin{definition}
	Si $X$ et $Y$ sont 2 v.a. ou definit la \textsc{Covariance} entre $X$ et $Y$ comme
	$\cov(X,Y)\overset{\text{def}}{=}\E\left[(X-\E(X))(Y-\E(Y))\right]=\E(XY)-\E(X)\E(Y)$.
\end{definition}
\fi
\pagebreak

% \tableofcontents

% insert your code here
%\input{./algebra/main.tex}
%\input{./geometrie-differentielle/main.tex}
%\input{./probabilite/main.tex}
%\input{./analyse-fonctionnelle/main.tex}
% \input{./Analyse-convexe-et-dualite-en-optimisation/main.tex}
%\input{./tikz/main.tex}
%\input{./Theorie-du-distributions/main.tex}
%\input{./optimisation/mine.tex}
 \input{./modelisation/main.tex}

% yves.aubry@univ-tln.fr : algebra

\end{document}

%% !TEX encoding = UTF-8 Unicode
% !TEX TS-program = xelatex

\documentclass[french]{report}

%\usepackage[utf8]{inputenc}
%\usepackage[T1]{fontenc}
\usepackage{babel}


\newif\ifcomment
%\commenttrue # Show comments

\usepackage{physics}
\usepackage{amssymb}


\usepackage{amsthm}
% \usepackage{thmtools}
\usepackage{mathtools}
\usepackage{amsfonts}

\usepackage{color}

\usepackage{tikz}

\usepackage{geometry}
\geometry{a5paper, margin=0.1in, right=1cm}

\usepackage{dsfont}

\usepackage{graphicx}
\graphicspath{ {images/} }

\usepackage{faktor}

\usepackage{IEEEtrantools}
\usepackage{enumerate}   
\usepackage[PostScript=dvips]{"/Users/aware/Documents/Courses/diagrams"}


\newtheorem{theorem}{Théorème}[section]
\renewcommand{\thetheorem}{\arabic{theorem}}
\newtheorem{lemme}{Lemme}[section]
\renewcommand{\thelemme}{\arabic{lemme}}
\newtheorem{proposition}{Proposition}[section]
\renewcommand{\theproposition}{\arabic{proposition}}
\newtheorem{notations}{Notations}[section]
\newtheorem{problem}{Problème}[section]
\newtheorem{corollary}{Corollaire}[theorem]
\renewcommand{\thecorollary}{\arabic{corollary}}
\newtheorem{property}{Propriété}[section]
\newtheorem{objective}{Objectif}[section]

\theoremstyle{definition}
\newtheorem{definition}{Définition}[section]
\renewcommand{\thedefinition}{\arabic{definition}}
\newtheorem{exercise}{Exercice}[chapter]
\renewcommand{\theexercise}{\arabic{exercise}}
\newtheorem{example}{Exemple}[chapter]
\renewcommand{\theexample}{\arabic{example}}
\newtheorem*{solution}{Solution}
\newtheorem*{application}{Application}
\newtheorem*{notation}{Notation}
\newtheorem*{vocabulary}{Vocabulaire}
\newtheorem*{properties}{Propriétés}



\theoremstyle{remark}
\newtheorem*{remark}{Remarque}
\newtheorem*{rappel}{Rappel}


\usepackage{etoolbox}
\AtBeginEnvironment{exercise}{\small}
\AtBeginEnvironment{example}{\small}

\usepackage{cases}
\usepackage[red]{mypack}

\usepackage[framemethod=TikZ]{mdframed}

\definecolor{bg}{rgb}{0.4,0.25,0.95}
\definecolor{pagebg}{rgb}{0,0,0.5}
\surroundwithmdframed[
   topline=false,
   rightline=false,
   bottomline=false,
   leftmargin=\parindent,
   skipabove=8pt,
   skipbelow=8pt,
   linecolor=blue,
   innerbottommargin=10pt,
   % backgroundcolor=bg,font=\color{orange}\sffamily, fontcolor=white
]{definition}

\usepackage{empheq}
\usepackage[most]{tcolorbox}

\newtcbox{\mymath}[1][]{%
    nobeforeafter, math upper, tcbox raise base,
    enhanced, colframe=blue!30!black,
    colback=red!10, boxrule=1pt,
    #1}

\usepackage{unixode}


\DeclareMathOperator{\ord}{ord}
\DeclareMathOperator{\orb}{orb}
\DeclareMathOperator{\stab}{stab}
\DeclareMathOperator{\Stab}{stab}
\DeclareMathOperator{\ppcm}{ppcm}
\DeclareMathOperator{\conj}{Conj}
\DeclareMathOperator{\End}{End}
\DeclareMathOperator{\rot}{rot}
\DeclareMathOperator{\trs}{trace}
\DeclareMathOperator{\Ind}{Ind}
\DeclareMathOperator{\mat}{Mat}
\DeclareMathOperator{\id}{Id}
\DeclareMathOperator{\vect}{vect}
\DeclareMathOperator{\img}{img}
\DeclareMathOperator{\cov}{Cov}
\DeclareMathOperator{\dist}{dist}
\DeclareMathOperator{\irr}{Irr}
\DeclareMathOperator{\image}{Im}
\DeclareMathOperator{\pd}{\partial}
\DeclareMathOperator{\epi}{epi}
\DeclareMathOperator{\Argmin}{Argmin}
\DeclareMathOperator{\dom}{dom}
\DeclareMathOperator{\proj}{proj}
\DeclareMathOperator{\ctg}{ctg}
\DeclareMathOperator{\supp}{supp}
\DeclareMathOperator{\argmin}{argmin}
\DeclareMathOperator{\mult}{mult}
\DeclareMathOperator{\ch}{ch}
\DeclareMathOperator{\sh}{sh}
\DeclareMathOperator{\rang}{rang}
\DeclareMathOperator{\diam}{diam}
\DeclareMathOperator{\Epigraphe}{Epigraphe}




\usepackage{xcolor}
\everymath{\color{blue}}
%\everymath{\color[rgb]{0,1,1}}
%\pagecolor[rgb]{0,0,0.5}


\newcommand*{\pdtest}[3][]{\ensuremath{\frac{\partial^{#1} #2}{\partial #3}}}

\newcommand*{\deffunc}[6][]{\ensuremath{
\begin{array}{rcl}
#2 : #3 &\rightarrow& #4\\
#5 &\mapsto& #6
\end{array}
}}

\newcommand{\eqcolon}{\mathrel{\resizebox{\widthof{$\mathord{=}$}}{\height}{ $\!\!=\!\!\resizebox{1.2\width}{0.8\height}{\raisebox{0.23ex}{$\mathop{:}$}}\!\!$ }}}
\newcommand{\coloneq}{\mathrel{\resizebox{\widthof{$\mathord{=}$}}{\height}{ $\!\!\resizebox{1.2\width}{0.8\height}{\raisebox{0.23ex}{$\mathop{:}$}}\!\!=\!\!$ }}}
\newcommand{\eqcolonl}{\ensuremath{\mathrel{=\!\!\mathop{:}}}}
\newcommand{\coloneql}{\ensuremath{\mathrel{\mathop{:} \!\! =}}}
\newcommand{\vc}[1]{% inline column vector
  \left(\begin{smallmatrix}#1\end{smallmatrix}\right)%
}
\newcommand{\vr}[1]{% inline row vector
  \begin{smallmatrix}(\,#1\,)\end{smallmatrix}%
}
\makeatletter
\newcommand*{\defeq}{\ =\mathrel{\rlap{%
                     \raisebox{0.3ex}{$\m@th\cdot$}}%
                     \raisebox{-0.3ex}{$\m@th\cdot$}}%
                     }
\makeatother

\newcommand{\mathcircle}[1]{% inline row vector
 \overset{\circ}{#1}
}
\newcommand{\ulim}{% low limit
 \underline{\lim}
}
\newcommand{\ssi}{% iff
\iff
}
\newcommand{\ps}[2]{
\expval{#1 | #2}
}
\newcommand{\df}[1]{
\mqty{#1}
}
\newcommand{\n}[1]{
\norm{#1}
}
\newcommand{\sys}[1]{
\left\{\smqty{#1}\right.
}


\newcommand{\eqdef}{\ensuremath{\overset{\text{def}}=}}


\def\Circlearrowright{\ensuremath{%
  \rotatebox[origin=c]{230}{$\circlearrowright$}}}

\newcommand\ct[1]{\text{\rmfamily\upshape #1}}
\newcommand\question[1]{ {\color{red} ...!? \small #1}}
\newcommand\caz[1]{\left\{\begin{array} #1 \end{array}\right.}
\newcommand\const{\text{\rmfamily\upshape const}}
\newcommand\toP{ \overset{\pro}{\to}}
\newcommand\toPP{ \overset{\text{PP}}{\to}}
\newcommand{\oeq}{\mathrel{\text{\textcircled{$=$}}}}





\usepackage{xcolor}
% \usepackage[normalem]{ulem}
\usepackage{lipsum}
\makeatletter
% \newcommand\colorwave[1][blue]{\bgroup \markoverwith{\lower3.5\p@\hbox{\sixly \textcolor{#1}{\char58}}}\ULon}
%\font\sixly=lasy6 % does not re-load if already loaded, so no memory problem.

\newmdtheoremenv[
linewidth= 1pt,linecolor= blue,%
leftmargin=20,rightmargin=20,innertopmargin=0pt, innerrightmargin=40,%
tikzsetting = { draw=lightgray, line width = 0.3pt,dashed,%
dash pattern = on 15pt off 3pt},%
splittopskip=\topskip,skipbelow=\baselineskip,%
skipabove=\baselineskip,ntheorem,roundcorner=0pt,
% backgroundcolor=pagebg,font=\color{orange}\sffamily, fontcolor=white
]{examplebox}{Exemple}[section]



\newcommand\R{\mathbb{R}}
\newcommand\Z{\mathbb{Z}}
\newcommand\N{\mathbb{N}}
\newcommand\E{\mathbb{E}}
\newcommand\F{\mathcal{F}}
\newcommand\cH{\mathcal{H}}
\newcommand\V{\mathbb{V}}
\newcommand\dmo{ ^{-1} }
\newcommand\kapa{\kappa}
\newcommand\im{Im}
\newcommand\hs{\mathcal{H}}





\usepackage{soul}

\makeatletter
\newcommand*{\whiten}[1]{\llap{\textcolor{white}{{\the\SOUL@token}}\hspace{#1pt}}}
\DeclareRobustCommand*\myul{%
    \def\SOUL@everyspace{\underline{\space}\kern\z@}%
    \def\SOUL@everytoken{%
     \setbox0=\hbox{\the\SOUL@token}%
     \ifdim\dp0>\z@
        \raisebox{\dp0}{\underline{\phantom{\the\SOUL@token}}}%
        \whiten{1}\whiten{0}%
        \whiten{-1}\whiten{-2}%
        \llap{\the\SOUL@token}%
     \else
        \underline{\the\SOUL@token}%
     \fi}%
\SOUL@}
\makeatother

\newcommand*{\demp}{\fontfamily{lmtt}\selectfont}

\DeclareTextFontCommand{\textdemp}{\demp}

\begin{document}

\ifcomment
Multiline
comment
\fi
\ifcomment
\myul{Typesetting test}
% \color[rgb]{1,1,1}
$∑_i^n≠ 60º±∞π∆¬≈√j∫h≤≥µ$

$\CR \R\pro\ind\pro\gS\pro
\mqty[a&b\\c&d]$
$\pro\mathbb{P}$
$\dd{x}$

  \[
    \alpha(x)=\left\{
                \begin{array}{ll}
                  x\\
                  \frac{1}{1+e^{-kx}}\\
                  \frac{e^x-e^{-x}}{e^x+e^{-x}}
                \end{array}
              \right.
  \]

  $\expval{x}$
  
  $\chi_\rho(ghg\dmo)=\Tr(\rho_{ghg\dmo})=\Tr(\rho_g\circ\rho_h\circ\rho\dmo_g)=\Tr(\rho_h)\overset{\mbox{\scalebox{0.5}{$\Tr(AB)=\Tr(BA)$}}}{=}\chi_\rho(h)$
  	$\mathop{\oplus}_{\substack{x\in X}}$

$\mat(\rho_g)=(a_{ij}(g))_{\scriptsize \substack{1\leq i\leq d \\ 1\leq j\leq d}}$ et $\mat(\rho'_g)=(a'_{ij}(g))_{\scriptsize \substack{1\leq i'\leq d' \\ 1\leq j'\leq d'}}$



\[\int_a^b{\mathbb{R}^2}g(u, v)\dd{P_{XY}}(u, v)=\iint g(u,v) f_{XY}(u, v)\dd \lambda(u) \dd \lambda(v)\]
$$\lim_{x\to\infty} f(x)$$	
$$\iiiint_V \mu(t,u,v,w) \,dt\,du\,dv\,dw$$
$$\sum_{n=1}^{\infty} 2^{-n} = 1$$	
\begin{definition}
	Si $X$ et $Y$ sont 2 v.a. ou definit la \textsc{Covariance} entre $X$ et $Y$ comme
	$\cov(X,Y)\overset{\text{def}}{=}\E\left[(X-\E(X))(Y-\E(Y))\right]=\E(XY)-\E(X)\E(Y)$.
\end{definition}
\fi
\pagebreak

% \tableofcontents

% insert your code here
%\input{./algebra/main.tex}
%\input{./geometrie-differentielle/main.tex}
%\input{./probabilite/main.tex}
%\input{./analyse-fonctionnelle/main.tex}
% \input{./Analyse-convexe-et-dualite-en-optimisation/main.tex}
%\input{./tikz/main.tex}
%\input{./Theorie-du-distributions/main.tex}
%\input{./optimisation/mine.tex}
 \input{./modelisation/main.tex}

% yves.aubry@univ-tln.fr : algebra

\end{document}

%\input{./optimisation/mine.tex}
 % !TEX encoding = UTF-8 Unicode
% !TEX TS-program = xelatex

\documentclass[french]{report}

%\usepackage[utf8]{inputenc}
%\usepackage[T1]{fontenc}
\usepackage{babel}


\newif\ifcomment
%\commenttrue # Show comments

\usepackage{physics}
\usepackage{amssymb}


\usepackage{amsthm}
% \usepackage{thmtools}
\usepackage{mathtools}
\usepackage{amsfonts}

\usepackage{color}

\usepackage{tikz}

\usepackage{geometry}
\geometry{a5paper, margin=0.1in, right=1cm}

\usepackage{dsfont}

\usepackage{graphicx}
\graphicspath{ {images/} }

\usepackage{faktor}

\usepackage{IEEEtrantools}
\usepackage{enumerate}   
\usepackage[PostScript=dvips]{"/Users/aware/Documents/Courses/diagrams"}


\newtheorem{theorem}{Théorème}[section]
\renewcommand{\thetheorem}{\arabic{theorem}}
\newtheorem{lemme}{Lemme}[section]
\renewcommand{\thelemme}{\arabic{lemme}}
\newtheorem{proposition}{Proposition}[section]
\renewcommand{\theproposition}{\arabic{proposition}}
\newtheorem{notations}{Notations}[section]
\newtheorem{problem}{Problème}[section]
\newtheorem{corollary}{Corollaire}[theorem]
\renewcommand{\thecorollary}{\arabic{corollary}}
\newtheorem{property}{Propriété}[section]
\newtheorem{objective}{Objectif}[section]

\theoremstyle{definition}
\newtheorem{definition}{Définition}[section]
\renewcommand{\thedefinition}{\arabic{definition}}
\newtheorem{exercise}{Exercice}[chapter]
\renewcommand{\theexercise}{\arabic{exercise}}
\newtheorem{example}{Exemple}[chapter]
\renewcommand{\theexample}{\arabic{example}}
\newtheorem*{solution}{Solution}
\newtheorem*{application}{Application}
\newtheorem*{notation}{Notation}
\newtheorem*{vocabulary}{Vocabulaire}
\newtheorem*{properties}{Propriétés}



\theoremstyle{remark}
\newtheorem*{remark}{Remarque}
\newtheorem*{rappel}{Rappel}


\usepackage{etoolbox}
\AtBeginEnvironment{exercise}{\small}
\AtBeginEnvironment{example}{\small}

\usepackage{cases}
\usepackage[red]{mypack}

\usepackage[framemethod=TikZ]{mdframed}

\definecolor{bg}{rgb}{0.4,0.25,0.95}
\definecolor{pagebg}{rgb}{0,0,0.5}
\surroundwithmdframed[
   topline=false,
   rightline=false,
   bottomline=false,
   leftmargin=\parindent,
   skipabove=8pt,
   skipbelow=8pt,
   linecolor=blue,
   innerbottommargin=10pt,
   % backgroundcolor=bg,font=\color{orange}\sffamily, fontcolor=white
]{definition}

\usepackage{empheq}
\usepackage[most]{tcolorbox}

\newtcbox{\mymath}[1][]{%
    nobeforeafter, math upper, tcbox raise base,
    enhanced, colframe=blue!30!black,
    colback=red!10, boxrule=1pt,
    #1}

\usepackage{unixode}


\DeclareMathOperator{\ord}{ord}
\DeclareMathOperator{\orb}{orb}
\DeclareMathOperator{\stab}{stab}
\DeclareMathOperator{\Stab}{stab}
\DeclareMathOperator{\ppcm}{ppcm}
\DeclareMathOperator{\conj}{Conj}
\DeclareMathOperator{\End}{End}
\DeclareMathOperator{\rot}{rot}
\DeclareMathOperator{\trs}{trace}
\DeclareMathOperator{\Ind}{Ind}
\DeclareMathOperator{\mat}{Mat}
\DeclareMathOperator{\id}{Id}
\DeclareMathOperator{\vect}{vect}
\DeclareMathOperator{\img}{img}
\DeclareMathOperator{\cov}{Cov}
\DeclareMathOperator{\dist}{dist}
\DeclareMathOperator{\irr}{Irr}
\DeclareMathOperator{\image}{Im}
\DeclareMathOperator{\pd}{\partial}
\DeclareMathOperator{\epi}{epi}
\DeclareMathOperator{\Argmin}{Argmin}
\DeclareMathOperator{\dom}{dom}
\DeclareMathOperator{\proj}{proj}
\DeclareMathOperator{\ctg}{ctg}
\DeclareMathOperator{\supp}{supp}
\DeclareMathOperator{\argmin}{argmin}
\DeclareMathOperator{\mult}{mult}
\DeclareMathOperator{\ch}{ch}
\DeclareMathOperator{\sh}{sh}
\DeclareMathOperator{\rang}{rang}
\DeclareMathOperator{\diam}{diam}
\DeclareMathOperator{\Epigraphe}{Epigraphe}




\usepackage{xcolor}
\everymath{\color{blue}}
%\everymath{\color[rgb]{0,1,1}}
%\pagecolor[rgb]{0,0,0.5}


\newcommand*{\pdtest}[3][]{\ensuremath{\frac{\partial^{#1} #2}{\partial #3}}}

\newcommand*{\deffunc}[6][]{\ensuremath{
\begin{array}{rcl}
#2 : #3 &\rightarrow& #4\\
#5 &\mapsto& #6
\end{array}
}}

\newcommand{\eqcolon}{\mathrel{\resizebox{\widthof{$\mathord{=}$}}{\height}{ $\!\!=\!\!\resizebox{1.2\width}{0.8\height}{\raisebox{0.23ex}{$\mathop{:}$}}\!\!$ }}}
\newcommand{\coloneq}{\mathrel{\resizebox{\widthof{$\mathord{=}$}}{\height}{ $\!\!\resizebox{1.2\width}{0.8\height}{\raisebox{0.23ex}{$\mathop{:}$}}\!\!=\!\!$ }}}
\newcommand{\eqcolonl}{\ensuremath{\mathrel{=\!\!\mathop{:}}}}
\newcommand{\coloneql}{\ensuremath{\mathrel{\mathop{:} \!\! =}}}
\newcommand{\vc}[1]{% inline column vector
  \left(\begin{smallmatrix}#1\end{smallmatrix}\right)%
}
\newcommand{\vr}[1]{% inline row vector
  \begin{smallmatrix}(\,#1\,)\end{smallmatrix}%
}
\makeatletter
\newcommand*{\defeq}{\ =\mathrel{\rlap{%
                     \raisebox{0.3ex}{$\m@th\cdot$}}%
                     \raisebox{-0.3ex}{$\m@th\cdot$}}%
                     }
\makeatother

\newcommand{\mathcircle}[1]{% inline row vector
 \overset{\circ}{#1}
}
\newcommand{\ulim}{% low limit
 \underline{\lim}
}
\newcommand{\ssi}{% iff
\iff
}
\newcommand{\ps}[2]{
\expval{#1 | #2}
}
\newcommand{\df}[1]{
\mqty{#1}
}
\newcommand{\n}[1]{
\norm{#1}
}
\newcommand{\sys}[1]{
\left\{\smqty{#1}\right.
}


\newcommand{\eqdef}{\ensuremath{\overset{\text{def}}=}}


\def\Circlearrowright{\ensuremath{%
  \rotatebox[origin=c]{230}{$\circlearrowright$}}}

\newcommand\ct[1]{\text{\rmfamily\upshape #1}}
\newcommand\question[1]{ {\color{red} ...!? \small #1}}
\newcommand\caz[1]{\left\{\begin{array} #1 \end{array}\right.}
\newcommand\const{\text{\rmfamily\upshape const}}
\newcommand\toP{ \overset{\pro}{\to}}
\newcommand\toPP{ \overset{\text{PP}}{\to}}
\newcommand{\oeq}{\mathrel{\text{\textcircled{$=$}}}}





\usepackage{xcolor}
% \usepackage[normalem]{ulem}
\usepackage{lipsum}
\makeatletter
% \newcommand\colorwave[1][blue]{\bgroup \markoverwith{\lower3.5\p@\hbox{\sixly \textcolor{#1}{\char58}}}\ULon}
%\font\sixly=lasy6 % does not re-load if already loaded, so no memory problem.

\newmdtheoremenv[
linewidth= 1pt,linecolor= blue,%
leftmargin=20,rightmargin=20,innertopmargin=0pt, innerrightmargin=40,%
tikzsetting = { draw=lightgray, line width = 0.3pt,dashed,%
dash pattern = on 15pt off 3pt},%
splittopskip=\topskip,skipbelow=\baselineskip,%
skipabove=\baselineskip,ntheorem,roundcorner=0pt,
% backgroundcolor=pagebg,font=\color{orange}\sffamily, fontcolor=white
]{examplebox}{Exemple}[section]



\newcommand\R{\mathbb{R}}
\newcommand\Z{\mathbb{Z}}
\newcommand\N{\mathbb{N}}
\newcommand\E{\mathbb{E}}
\newcommand\F{\mathcal{F}}
\newcommand\cH{\mathcal{H}}
\newcommand\V{\mathbb{V}}
\newcommand\dmo{ ^{-1} }
\newcommand\kapa{\kappa}
\newcommand\im{Im}
\newcommand\hs{\mathcal{H}}





\usepackage{soul}

\makeatletter
\newcommand*{\whiten}[1]{\llap{\textcolor{white}{{\the\SOUL@token}}\hspace{#1pt}}}
\DeclareRobustCommand*\myul{%
    \def\SOUL@everyspace{\underline{\space}\kern\z@}%
    \def\SOUL@everytoken{%
     \setbox0=\hbox{\the\SOUL@token}%
     \ifdim\dp0>\z@
        \raisebox{\dp0}{\underline{\phantom{\the\SOUL@token}}}%
        \whiten{1}\whiten{0}%
        \whiten{-1}\whiten{-2}%
        \llap{\the\SOUL@token}%
     \else
        \underline{\the\SOUL@token}%
     \fi}%
\SOUL@}
\makeatother

\newcommand*{\demp}{\fontfamily{lmtt}\selectfont}

\DeclareTextFontCommand{\textdemp}{\demp}

\begin{document}

\ifcomment
Multiline
comment
\fi
\ifcomment
\myul{Typesetting test}
% \color[rgb]{1,1,1}
$∑_i^n≠ 60º±∞π∆¬≈√j∫h≤≥µ$

$\CR \R\pro\ind\pro\gS\pro
\mqty[a&b\\c&d]$
$\pro\mathbb{P}$
$\dd{x}$

  \[
    \alpha(x)=\left\{
                \begin{array}{ll}
                  x\\
                  \frac{1}{1+e^{-kx}}\\
                  \frac{e^x-e^{-x}}{e^x+e^{-x}}
                \end{array}
              \right.
  \]

  $\expval{x}$
  
  $\chi_\rho(ghg\dmo)=\Tr(\rho_{ghg\dmo})=\Tr(\rho_g\circ\rho_h\circ\rho\dmo_g)=\Tr(\rho_h)\overset{\mbox{\scalebox{0.5}{$\Tr(AB)=\Tr(BA)$}}}{=}\chi_\rho(h)$
  	$\mathop{\oplus}_{\substack{x\in X}}$

$\mat(\rho_g)=(a_{ij}(g))_{\scriptsize \substack{1\leq i\leq d \\ 1\leq j\leq d}}$ et $\mat(\rho'_g)=(a'_{ij}(g))_{\scriptsize \substack{1\leq i'\leq d' \\ 1\leq j'\leq d'}}$



\[\int_a^b{\mathbb{R}^2}g(u, v)\dd{P_{XY}}(u, v)=\iint g(u,v) f_{XY}(u, v)\dd \lambda(u) \dd \lambda(v)\]
$$\lim_{x\to\infty} f(x)$$	
$$\iiiint_V \mu(t,u,v,w) \,dt\,du\,dv\,dw$$
$$\sum_{n=1}^{\infty} 2^{-n} = 1$$	
\begin{definition}
	Si $X$ et $Y$ sont 2 v.a. ou definit la \textsc{Covariance} entre $X$ et $Y$ comme
	$\cov(X,Y)\overset{\text{def}}{=}\E\left[(X-\E(X))(Y-\E(Y))\right]=\E(XY)-\E(X)\E(Y)$.
\end{definition}
\fi
\pagebreak

% \tableofcontents

% insert your code here
%\input{./algebra/main.tex}
%\input{./geometrie-differentielle/main.tex}
%\input{./probabilite/main.tex}
%\input{./analyse-fonctionnelle/main.tex}
% \input{./Analyse-convexe-et-dualite-en-optimisation/main.tex}
%\input{./tikz/main.tex}
%\input{./Theorie-du-distributions/main.tex}
%\input{./optimisation/mine.tex}
 \input{./modelisation/main.tex}

% yves.aubry@univ-tln.fr : algebra

\end{document}


% yves.aubry@univ-tln.fr : algebra

\end{document}

%% !TEX encoding = UTF-8 Unicode
% !TEX TS-program = xelatex

\documentclass[french]{report}

%\usepackage[utf8]{inputenc}
%\usepackage[T1]{fontenc}
\usepackage{babel}


\newif\ifcomment
%\commenttrue # Show comments

\usepackage{physics}
\usepackage{amssymb}


\usepackage{amsthm}
% \usepackage{thmtools}
\usepackage{mathtools}
\usepackage{amsfonts}

\usepackage{color}

\usepackage{tikz}

\usepackage{geometry}
\geometry{a5paper, margin=0.1in, right=1cm}

\usepackage{dsfont}

\usepackage{graphicx}
\graphicspath{ {images/} }

\usepackage{faktor}

\usepackage{IEEEtrantools}
\usepackage{enumerate}   
\usepackage[PostScript=dvips]{"/Users/aware/Documents/Courses/diagrams"}


\newtheorem{theorem}{Théorème}[section]
\renewcommand{\thetheorem}{\arabic{theorem}}
\newtheorem{lemme}{Lemme}[section]
\renewcommand{\thelemme}{\arabic{lemme}}
\newtheorem{proposition}{Proposition}[section]
\renewcommand{\theproposition}{\arabic{proposition}}
\newtheorem{notations}{Notations}[section]
\newtheorem{problem}{Problème}[section]
\newtheorem{corollary}{Corollaire}[theorem]
\renewcommand{\thecorollary}{\arabic{corollary}}
\newtheorem{property}{Propriété}[section]
\newtheorem{objective}{Objectif}[section]

\theoremstyle{definition}
\newtheorem{definition}{Définition}[section]
\renewcommand{\thedefinition}{\arabic{definition}}
\newtheorem{exercise}{Exercice}[chapter]
\renewcommand{\theexercise}{\arabic{exercise}}
\newtheorem{example}{Exemple}[chapter]
\renewcommand{\theexample}{\arabic{example}}
\newtheorem*{solution}{Solution}
\newtheorem*{application}{Application}
\newtheorem*{notation}{Notation}
\newtheorem*{vocabulary}{Vocabulaire}
\newtheorem*{properties}{Propriétés}



\theoremstyle{remark}
\newtheorem*{remark}{Remarque}
\newtheorem*{rappel}{Rappel}


\usepackage{etoolbox}
\AtBeginEnvironment{exercise}{\small}
\AtBeginEnvironment{example}{\small}

\usepackage{cases}
\usepackage[red]{mypack}

\usepackage[framemethod=TikZ]{mdframed}

\definecolor{bg}{rgb}{0.4,0.25,0.95}
\definecolor{pagebg}{rgb}{0,0,0.5}
\surroundwithmdframed[
   topline=false,
   rightline=false,
   bottomline=false,
   leftmargin=\parindent,
   skipabove=8pt,
   skipbelow=8pt,
   linecolor=blue,
   innerbottommargin=10pt,
   % backgroundcolor=bg,font=\color{orange}\sffamily, fontcolor=white
]{definition}

\usepackage{empheq}
\usepackage[most]{tcolorbox}

\newtcbox{\mymath}[1][]{%
    nobeforeafter, math upper, tcbox raise base,
    enhanced, colframe=blue!30!black,
    colback=red!10, boxrule=1pt,
    #1}

\usepackage{unixode}


\DeclareMathOperator{\ord}{ord}
\DeclareMathOperator{\orb}{orb}
\DeclareMathOperator{\stab}{stab}
\DeclareMathOperator{\Stab}{stab}
\DeclareMathOperator{\ppcm}{ppcm}
\DeclareMathOperator{\conj}{Conj}
\DeclareMathOperator{\End}{End}
\DeclareMathOperator{\rot}{rot}
\DeclareMathOperator{\trs}{trace}
\DeclareMathOperator{\Ind}{Ind}
\DeclareMathOperator{\mat}{Mat}
\DeclareMathOperator{\id}{Id}
\DeclareMathOperator{\vect}{vect}
\DeclareMathOperator{\img}{img}
\DeclareMathOperator{\cov}{Cov}
\DeclareMathOperator{\dist}{dist}
\DeclareMathOperator{\irr}{Irr}
\DeclareMathOperator{\image}{Im}
\DeclareMathOperator{\pd}{\partial}
\DeclareMathOperator{\epi}{epi}
\DeclareMathOperator{\Argmin}{Argmin}
\DeclareMathOperator{\dom}{dom}
\DeclareMathOperator{\proj}{proj}
\DeclareMathOperator{\ctg}{ctg}
\DeclareMathOperator{\supp}{supp}
\DeclareMathOperator{\argmin}{argmin}
\DeclareMathOperator{\mult}{mult}
\DeclareMathOperator{\ch}{ch}
\DeclareMathOperator{\sh}{sh}
\DeclareMathOperator{\rang}{rang}
\DeclareMathOperator{\diam}{diam}
\DeclareMathOperator{\Epigraphe}{Epigraphe}




\usepackage{xcolor}
\everymath{\color{blue}}
%\everymath{\color[rgb]{0,1,1}}
%\pagecolor[rgb]{0,0,0.5}


\newcommand*{\pdtest}[3][]{\ensuremath{\frac{\partial^{#1} #2}{\partial #3}}}

\newcommand*{\deffunc}[6][]{\ensuremath{
\begin{array}{rcl}
#2 : #3 &\rightarrow& #4\\
#5 &\mapsto& #6
\end{array}
}}

\newcommand{\eqcolon}{\mathrel{\resizebox{\widthof{$\mathord{=}$}}{\height}{ $\!\!=\!\!\resizebox{1.2\width}{0.8\height}{\raisebox{0.23ex}{$\mathop{:}$}}\!\!$ }}}
\newcommand{\coloneq}{\mathrel{\resizebox{\widthof{$\mathord{=}$}}{\height}{ $\!\!\resizebox{1.2\width}{0.8\height}{\raisebox{0.23ex}{$\mathop{:}$}}\!\!=\!\!$ }}}
\newcommand{\eqcolonl}{\ensuremath{\mathrel{=\!\!\mathop{:}}}}
\newcommand{\coloneql}{\ensuremath{\mathrel{\mathop{:} \!\! =}}}
\newcommand{\vc}[1]{% inline column vector
  \left(\begin{smallmatrix}#1\end{smallmatrix}\right)%
}
\newcommand{\vr}[1]{% inline row vector
  \begin{smallmatrix}(\,#1\,)\end{smallmatrix}%
}
\makeatletter
\newcommand*{\defeq}{\ =\mathrel{\rlap{%
                     \raisebox{0.3ex}{$\m@th\cdot$}}%
                     \raisebox{-0.3ex}{$\m@th\cdot$}}%
                     }
\makeatother

\newcommand{\mathcircle}[1]{% inline row vector
 \overset{\circ}{#1}
}
\newcommand{\ulim}{% low limit
 \underline{\lim}
}
\newcommand{\ssi}{% iff
\iff
}
\newcommand{\ps}[2]{
\expval{#1 | #2}
}
\newcommand{\df}[1]{
\mqty{#1}
}
\newcommand{\n}[1]{
\norm{#1}
}
\newcommand{\sys}[1]{
\left\{\smqty{#1}\right.
}


\newcommand{\eqdef}{\ensuremath{\overset{\text{def}}=}}


\def\Circlearrowright{\ensuremath{%
  \rotatebox[origin=c]{230}{$\circlearrowright$}}}

\newcommand\ct[1]{\text{\rmfamily\upshape #1}}
\newcommand\question[1]{ {\color{red} ...!? \small #1}}
\newcommand\caz[1]{\left\{\begin{array} #1 \end{array}\right.}
\newcommand\const{\text{\rmfamily\upshape const}}
\newcommand\toP{ \overset{\pro}{\to}}
\newcommand\toPP{ \overset{\text{PP}}{\to}}
\newcommand{\oeq}{\mathrel{\text{\textcircled{$=$}}}}





\usepackage{xcolor}
% \usepackage[normalem]{ulem}
\usepackage{lipsum}
\makeatletter
% \newcommand\colorwave[1][blue]{\bgroup \markoverwith{\lower3.5\p@\hbox{\sixly \textcolor{#1}{\char58}}}\ULon}
%\font\sixly=lasy6 % does not re-load if already loaded, so no memory problem.

\newmdtheoremenv[
linewidth= 1pt,linecolor= blue,%
leftmargin=20,rightmargin=20,innertopmargin=0pt, innerrightmargin=40,%
tikzsetting = { draw=lightgray, line width = 0.3pt,dashed,%
dash pattern = on 15pt off 3pt},%
splittopskip=\topskip,skipbelow=\baselineskip,%
skipabove=\baselineskip,ntheorem,roundcorner=0pt,
% backgroundcolor=pagebg,font=\color{orange}\sffamily, fontcolor=white
]{examplebox}{Exemple}[section]



\newcommand\R{\mathbb{R}}
\newcommand\Z{\mathbb{Z}}
\newcommand\N{\mathbb{N}}
\newcommand\E{\mathbb{E}}
\newcommand\F{\mathcal{F}}
\newcommand\cH{\mathcal{H}}
\newcommand\V{\mathbb{V}}
\newcommand\dmo{ ^{-1} }
\newcommand\kapa{\kappa}
\newcommand\im{Im}
\newcommand\hs{\mathcal{H}}





\usepackage{soul}

\makeatletter
\newcommand*{\whiten}[1]{\llap{\textcolor{white}{{\the\SOUL@token}}\hspace{#1pt}}}
\DeclareRobustCommand*\myul{%
    \def\SOUL@everyspace{\underline{\space}\kern\z@}%
    \def\SOUL@everytoken{%
     \setbox0=\hbox{\the\SOUL@token}%
     \ifdim\dp0>\z@
        \raisebox{\dp0}{\underline{\phantom{\the\SOUL@token}}}%
        \whiten{1}\whiten{0}%
        \whiten{-1}\whiten{-2}%
        \llap{\the\SOUL@token}%
     \else
        \underline{\the\SOUL@token}%
     \fi}%
\SOUL@}
\makeatother

\newcommand*{\demp}{\fontfamily{lmtt}\selectfont}

\DeclareTextFontCommand{\textdemp}{\demp}

\begin{document}

\ifcomment
Multiline
comment
\fi
\ifcomment
\myul{Typesetting test}
% \color[rgb]{1,1,1}
$∑_i^n≠ 60º±∞π∆¬≈√j∫h≤≥µ$

$\CR \R\pro\ind\pro\gS\pro
\mqty[a&b\\c&d]$
$\pro\mathbb{P}$
$\dd{x}$

  \[
    \alpha(x)=\left\{
                \begin{array}{ll}
                  x\\
                  \frac{1}{1+e^{-kx}}\\
                  \frac{e^x-e^{-x}}{e^x+e^{-x}}
                \end{array}
              \right.
  \]

  $\expval{x}$
  
  $\chi_\rho(ghg\dmo)=\Tr(\rho_{ghg\dmo})=\Tr(\rho_g\circ\rho_h\circ\rho\dmo_g)=\Tr(\rho_h)\overset{\mbox{\scalebox{0.5}{$\Tr(AB)=\Tr(BA)$}}}{=}\chi_\rho(h)$
  	$\mathop{\oplus}_{\substack{x\in X}}$

$\mat(\rho_g)=(a_{ij}(g))_{\scriptsize \substack{1\leq i\leq d \\ 1\leq j\leq d}}$ et $\mat(\rho'_g)=(a'_{ij}(g))_{\scriptsize \substack{1\leq i'\leq d' \\ 1\leq j'\leq d'}}$



\[\int_a^b{\mathbb{R}^2}g(u, v)\dd{P_{XY}}(u, v)=\iint g(u,v) f_{XY}(u, v)\dd \lambda(u) \dd \lambda(v)\]
$$\lim_{x\to\infty} f(x)$$	
$$\iiiint_V \mu(t,u,v,w) \,dt\,du\,dv\,dw$$
$$\sum_{n=1}^{\infty} 2^{-n} = 1$$	
\begin{definition}
	Si $X$ et $Y$ sont 2 v.a. ou definit la \textsc{Covariance} entre $X$ et $Y$ comme
	$\cov(X,Y)\overset{\text{def}}{=}\E\left[(X-\E(X))(Y-\E(Y))\right]=\E(XY)-\E(X)\E(Y)$.
\end{definition}
\fi
\pagebreak

% \tableofcontents

% insert your code here
%% !TEX encoding = UTF-8 Unicode
% !TEX TS-program = xelatex

\documentclass[french]{report}

%\usepackage[utf8]{inputenc}
%\usepackage[T1]{fontenc}
\usepackage{babel}


\newif\ifcomment
%\commenttrue # Show comments

\usepackage{physics}
\usepackage{amssymb}


\usepackage{amsthm}
% \usepackage{thmtools}
\usepackage{mathtools}
\usepackage{amsfonts}

\usepackage{color}

\usepackage{tikz}

\usepackage{geometry}
\geometry{a5paper, margin=0.1in, right=1cm}

\usepackage{dsfont}

\usepackage{graphicx}
\graphicspath{ {images/} }

\usepackage{faktor}

\usepackage{IEEEtrantools}
\usepackage{enumerate}   
\usepackage[PostScript=dvips]{"/Users/aware/Documents/Courses/diagrams"}


\newtheorem{theorem}{Théorème}[section]
\renewcommand{\thetheorem}{\arabic{theorem}}
\newtheorem{lemme}{Lemme}[section]
\renewcommand{\thelemme}{\arabic{lemme}}
\newtheorem{proposition}{Proposition}[section]
\renewcommand{\theproposition}{\arabic{proposition}}
\newtheorem{notations}{Notations}[section]
\newtheorem{problem}{Problème}[section]
\newtheorem{corollary}{Corollaire}[theorem]
\renewcommand{\thecorollary}{\arabic{corollary}}
\newtheorem{property}{Propriété}[section]
\newtheorem{objective}{Objectif}[section]

\theoremstyle{definition}
\newtheorem{definition}{Définition}[section]
\renewcommand{\thedefinition}{\arabic{definition}}
\newtheorem{exercise}{Exercice}[chapter]
\renewcommand{\theexercise}{\arabic{exercise}}
\newtheorem{example}{Exemple}[chapter]
\renewcommand{\theexample}{\arabic{example}}
\newtheorem*{solution}{Solution}
\newtheorem*{application}{Application}
\newtheorem*{notation}{Notation}
\newtheorem*{vocabulary}{Vocabulaire}
\newtheorem*{properties}{Propriétés}



\theoremstyle{remark}
\newtheorem*{remark}{Remarque}
\newtheorem*{rappel}{Rappel}


\usepackage{etoolbox}
\AtBeginEnvironment{exercise}{\small}
\AtBeginEnvironment{example}{\small}

\usepackage{cases}
\usepackage[red]{mypack}

\usepackage[framemethod=TikZ]{mdframed}

\definecolor{bg}{rgb}{0.4,0.25,0.95}
\definecolor{pagebg}{rgb}{0,0,0.5}
\surroundwithmdframed[
   topline=false,
   rightline=false,
   bottomline=false,
   leftmargin=\parindent,
   skipabove=8pt,
   skipbelow=8pt,
   linecolor=blue,
   innerbottommargin=10pt,
   % backgroundcolor=bg,font=\color{orange}\sffamily, fontcolor=white
]{definition}

\usepackage{empheq}
\usepackage[most]{tcolorbox}

\newtcbox{\mymath}[1][]{%
    nobeforeafter, math upper, tcbox raise base,
    enhanced, colframe=blue!30!black,
    colback=red!10, boxrule=1pt,
    #1}

\usepackage{unixode}


\DeclareMathOperator{\ord}{ord}
\DeclareMathOperator{\orb}{orb}
\DeclareMathOperator{\stab}{stab}
\DeclareMathOperator{\Stab}{stab}
\DeclareMathOperator{\ppcm}{ppcm}
\DeclareMathOperator{\conj}{Conj}
\DeclareMathOperator{\End}{End}
\DeclareMathOperator{\rot}{rot}
\DeclareMathOperator{\trs}{trace}
\DeclareMathOperator{\Ind}{Ind}
\DeclareMathOperator{\mat}{Mat}
\DeclareMathOperator{\id}{Id}
\DeclareMathOperator{\vect}{vect}
\DeclareMathOperator{\img}{img}
\DeclareMathOperator{\cov}{Cov}
\DeclareMathOperator{\dist}{dist}
\DeclareMathOperator{\irr}{Irr}
\DeclareMathOperator{\image}{Im}
\DeclareMathOperator{\pd}{\partial}
\DeclareMathOperator{\epi}{epi}
\DeclareMathOperator{\Argmin}{Argmin}
\DeclareMathOperator{\dom}{dom}
\DeclareMathOperator{\proj}{proj}
\DeclareMathOperator{\ctg}{ctg}
\DeclareMathOperator{\supp}{supp}
\DeclareMathOperator{\argmin}{argmin}
\DeclareMathOperator{\mult}{mult}
\DeclareMathOperator{\ch}{ch}
\DeclareMathOperator{\sh}{sh}
\DeclareMathOperator{\rang}{rang}
\DeclareMathOperator{\diam}{diam}
\DeclareMathOperator{\Epigraphe}{Epigraphe}




\usepackage{xcolor}
\everymath{\color{blue}}
%\everymath{\color[rgb]{0,1,1}}
%\pagecolor[rgb]{0,0,0.5}


\newcommand*{\pdtest}[3][]{\ensuremath{\frac{\partial^{#1} #2}{\partial #3}}}

\newcommand*{\deffunc}[6][]{\ensuremath{
\begin{array}{rcl}
#2 : #3 &\rightarrow& #4\\
#5 &\mapsto& #6
\end{array}
}}

\newcommand{\eqcolon}{\mathrel{\resizebox{\widthof{$\mathord{=}$}}{\height}{ $\!\!=\!\!\resizebox{1.2\width}{0.8\height}{\raisebox{0.23ex}{$\mathop{:}$}}\!\!$ }}}
\newcommand{\coloneq}{\mathrel{\resizebox{\widthof{$\mathord{=}$}}{\height}{ $\!\!\resizebox{1.2\width}{0.8\height}{\raisebox{0.23ex}{$\mathop{:}$}}\!\!=\!\!$ }}}
\newcommand{\eqcolonl}{\ensuremath{\mathrel{=\!\!\mathop{:}}}}
\newcommand{\coloneql}{\ensuremath{\mathrel{\mathop{:} \!\! =}}}
\newcommand{\vc}[1]{% inline column vector
  \left(\begin{smallmatrix}#1\end{smallmatrix}\right)%
}
\newcommand{\vr}[1]{% inline row vector
  \begin{smallmatrix}(\,#1\,)\end{smallmatrix}%
}
\makeatletter
\newcommand*{\defeq}{\ =\mathrel{\rlap{%
                     \raisebox{0.3ex}{$\m@th\cdot$}}%
                     \raisebox{-0.3ex}{$\m@th\cdot$}}%
                     }
\makeatother

\newcommand{\mathcircle}[1]{% inline row vector
 \overset{\circ}{#1}
}
\newcommand{\ulim}{% low limit
 \underline{\lim}
}
\newcommand{\ssi}{% iff
\iff
}
\newcommand{\ps}[2]{
\expval{#1 | #2}
}
\newcommand{\df}[1]{
\mqty{#1}
}
\newcommand{\n}[1]{
\norm{#1}
}
\newcommand{\sys}[1]{
\left\{\smqty{#1}\right.
}


\newcommand{\eqdef}{\ensuremath{\overset{\text{def}}=}}


\def\Circlearrowright{\ensuremath{%
  \rotatebox[origin=c]{230}{$\circlearrowright$}}}

\newcommand\ct[1]{\text{\rmfamily\upshape #1}}
\newcommand\question[1]{ {\color{red} ...!? \small #1}}
\newcommand\caz[1]{\left\{\begin{array} #1 \end{array}\right.}
\newcommand\const{\text{\rmfamily\upshape const}}
\newcommand\toP{ \overset{\pro}{\to}}
\newcommand\toPP{ \overset{\text{PP}}{\to}}
\newcommand{\oeq}{\mathrel{\text{\textcircled{$=$}}}}





\usepackage{xcolor}
% \usepackage[normalem]{ulem}
\usepackage{lipsum}
\makeatletter
% \newcommand\colorwave[1][blue]{\bgroup \markoverwith{\lower3.5\p@\hbox{\sixly \textcolor{#1}{\char58}}}\ULon}
%\font\sixly=lasy6 % does not re-load if already loaded, so no memory problem.

\newmdtheoremenv[
linewidth= 1pt,linecolor= blue,%
leftmargin=20,rightmargin=20,innertopmargin=0pt, innerrightmargin=40,%
tikzsetting = { draw=lightgray, line width = 0.3pt,dashed,%
dash pattern = on 15pt off 3pt},%
splittopskip=\topskip,skipbelow=\baselineskip,%
skipabove=\baselineskip,ntheorem,roundcorner=0pt,
% backgroundcolor=pagebg,font=\color{orange}\sffamily, fontcolor=white
]{examplebox}{Exemple}[section]



\newcommand\R{\mathbb{R}}
\newcommand\Z{\mathbb{Z}}
\newcommand\N{\mathbb{N}}
\newcommand\E{\mathbb{E}}
\newcommand\F{\mathcal{F}}
\newcommand\cH{\mathcal{H}}
\newcommand\V{\mathbb{V}}
\newcommand\dmo{ ^{-1} }
\newcommand\kapa{\kappa}
\newcommand\im{Im}
\newcommand\hs{\mathcal{H}}





\usepackage{soul}

\makeatletter
\newcommand*{\whiten}[1]{\llap{\textcolor{white}{{\the\SOUL@token}}\hspace{#1pt}}}
\DeclareRobustCommand*\myul{%
    \def\SOUL@everyspace{\underline{\space}\kern\z@}%
    \def\SOUL@everytoken{%
     \setbox0=\hbox{\the\SOUL@token}%
     \ifdim\dp0>\z@
        \raisebox{\dp0}{\underline{\phantom{\the\SOUL@token}}}%
        \whiten{1}\whiten{0}%
        \whiten{-1}\whiten{-2}%
        \llap{\the\SOUL@token}%
     \else
        \underline{\the\SOUL@token}%
     \fi}%
\SOUL@}
\makeatother

\newcommand*{\demp}{\fontfamily{lmtt}\selectfont}

\DeclareTextFontCommand{\textdemp}{\demp}

\begin{document}

\ifcomment
Multiline
comment
\fi
\ifcomment
\myul{Typesetting test}
% \color[rgb]{1,1,1}
$∑_i^n≠ 60º±∞π∆¬≈√j∫h≤≥µ$

$\CR \R\pro\ind\pro\gS\pro
\mqty[a&b\\c&d]$
$\pro\mathbb{P}$
$\dd{x}$

  \[
    \alpha(x)=\left\{
                \begin{array}{ll}
                  x\\
                  \frac{1}{1+e^{-kx}}\\
                  \frac{e^x-e^{-x}}{e^x+e^{-x}}
                \end{array}
              \right.
  \]

  $\expval{x}$
  
  $\chi_\rho(ghg\dmo)=\Tr(\rho_{ghg\dmo})=\Tr(\rho_g\circ\rho_h\circ\rho\dmo_g)=\Tr(\rho_h)\overset{\mbox{\scalebox{0.5}{$\Tr(AB)=\Tr(BA)$}}}{=}\chi_\rho(h)$
  	$\mathop{\oplus}_{\substack{x\in X}}$

$\mat(\rho_g)=(a_{ij}(g))_{\scriptsize \substack{1\leq i\leq d \\ 1\leq j\leq d}}$ et $\mat(\rho'_g)=(a'_{ij}(g))_{\scriptsize \substack{1\leq i'\leq d' \\ 1\leq j'\leq d'}}$



\[\int_a^b{\mathbb{R}^2}g(u, v)\dd{P_{XY}}(u, v)=\iint g(u,v) f_{XY}(u, v)\dd \lambda(u) \dd \lambda(v)\]
$$\lim_{x\to\infty} f(x)$$	
$$\iiiint_V \mu(t,u,v,w) \,dt\,du\,dv\,dw$$
$$\sum_{n=1}^{\infty} 2^{-n} = 1$$	
\begin{definition}
	Si $X$ et $Y$ sont 2 v.a. ou definit la \textsc{Covariance} entre $X$ et $Y$ comme
	$\cov(X,Y)\overset{\text{def}}{=}\E\left[(X-\E(X))(Y-\E(Y))\right]=\E(XY)-\E(X)\E(Y)$.
\end{definition}
\fi
\pagebreak

% \tableofcontents

% insert your code here
%\input{./algebra/main.tex}
%\input{./geometrie-differentielle/main.tex}
%\input{./probabilite/main.tex}
%\input{./analyse-fonctionnelle/main.tex}
% \input{./Analyse-convexe-et-dualite-en-optimisation/main.tex}
%\input{./tikz/main.tex}
%\input{./Theorie-du-distributions/main.tex}
%\input{./optimisation/mine.tex}
 \input{./modelisation/main.tex}

% yves.aubry@univ-tln.fr : algebra

\end{document}

%% !TEX encoding = UTF-8 Unicode
% !TEX TS-program = xelatex

\documentclass[french]{report}

%\usepackage[utf8]{inputenc}
%\usepackage[T1]{fontenc}
\usepackage{babel}


\newif\ifcomment
%\commenttrue # Show comments

\usepackage{physics}
\usepackage{amssymb}


\usepackage{amsthm}
% \usepackage{thmtools}
\usepackage{mathtools}
\usepackage{amsfonts}

\usepackage{color}

\usepackage{tikz}

\usepackage{geometry}
\geometry{a5paper, margin=0.1in, right=1cm}

\usepackage{dsfont}

\usepackage{graphicx}
\graphicspath{ {images/} }

\usepackage{faktor}

\usepackage{IEEEtrantools}
\usepackage{enumerate}   
\usepackage[PostScript=dvips]{"/Users/aware/Documents/Courses/diagrams"}


\newtheorem{theorem}{Théorème}[section]
\renewcommand{\thetheorem}{\arabic{theorem}}
\newtheorem{lemme}{Lemme}[section]
\renewcommand{\thelemme}{\arabic{lemme}}
\newtheorem{proposition}{Proposition}[section]
\renewcommand{\theproposition}{\arabic{proposition}}
\newtheorem{notations}{Notations}[section]
\newtheorem{problem}{Problème}[section]
\newtheorem{corollary}{Corollaire}[theorem]
\renewcommand{\thecorollary}{\arabic{corollary}}
\newtheorem{property}{Propriété}[section]
\newtheorem{objective}{Objectif}[section]

\theoremstyle{definition}
\newtheorem{definition}{Définition}[section]
\renewcommand{\thedefinition}{\arabic{definition}}
\newtheorem{exercise}{Exercice}[chapter]
\renewcommand{\theexercise}{\arabic{exercise}}
\newtheorem{example}{Exemple}[chapter]
\renewcommand{\theexample}{\arabic{example}}
\newtheorem*{solution}{Solution}
\newtheorem*{application}{Application}
\newtheorem*{notation}{Notation}
\newtheorem*{vocabulary}{Vocabulaire}
\newtheorem*{properties}{Propriétés}



\theoremstyle{remark}
\newtheorem*{remark}{Remarque}
\newtheorem*{rappel}{Rappel}


\usepackage{etoolbox}
\AtBeginEnvironment{exercise}{\small}
\AtBeginEnvironment{example}{\small}

\usepackage{cases}
\usepackage[red]{mypack}

\usepackage[framemethod=TikZ]{mdframed}

\definecolor{bg}{rgb}{0.4,0.25,0.95}
\definecolor{pagebg}{rgb}{0,0,0.5}
\surroundwithmdframed[
   topline=false,
   rightline=false,
   bottomline=false,
   leftmargin=\parindent,
   skipabove=8pt,
   skipbelow=8pt,
   linecolor=blue,
   innerbottommargin=10pt,
   % backgroundcolor=bg,font=\color{orange}\sffamily, fontcolor=white
]{definition}

\usepackage{empheq}
\usepackage[most]{tcolorbox}

\newtcbox{\mymath}[1][]{%
    nobeforeafter, math upper, tcbox raise base,
    enhanced, colframe=blue!30!black,
    colback=red!10, boxrule=1pt,
    #1}

\usepackage{unixode}


\DeclareMathOperator{\ord}{ord}
\DeclareMathOperator{\orb}{orb}
\DeclareMathOperator{\stab}{stab}
\DeclareMathOperator{\Stab}{stab}
\DeclareMathOperator{\ppcm}{ppcm}
\DeclareMathOperator{\conj}{Conj}
\DeclareMathOperator{\End}{End}
\DeclareMathOperator{\rot}{rot}
\DeclareMathOperator{\trs}{trace}
\DeclareMathOperator{\Ind}{Ind}
\DeclareMathOperator{\mat}{Mat}
\DeclareMathOperator{\id}{Id}
\DeclareMathOperator{\vect}{vect}
\DeclareMathOperator{\img}{img}
\DeclareMathOperator{\cov}{Cov}
\DeclareMathOperator{\dist}{dist}
\DeclareMathOperator{\irr}{Irr}
\DeclareMathOperator{\image}{Im}
\DeclareMathOperator{\pd}{\partial}
\DeclareMathOperator{\epi}{epi}
\DeclareMathOperator{\Argmin}{Argmin}
\DeclareMathOperator{\dom}{dom}
\DeclareMathOperator{\proj}{proj}
\DeclareMathOperator{\ctg}{ctg}
\DeclareMathOperator{\supp}{supp}
\DeclareMathOperator{\argmin}{argmin}
\DeclareMathOperator{\mult}{mult}
\DeclareMathOperator{\ch}{ch}
\DeclareMathOperator{\sh}{sh}
\DeclareMathOperator{\rang}{rang}
\DeclareMathOperator{\diam}{diam}
\DeclareMathOperator{\Epigraphe}{Epigraphe}




\usepackage{xcolor}
\everymath{\color{blue}}
%\everymath{\color[rgb]{0,1,1}}
%\pagecolor[rgb]{0,0,0.5}


\newcommand*{\pdtest}[3][]{\ensuremath{\frac{\partial^{#1} #2}{\partial #3}}}

\newcommand*{\deffunc}[6][]{\ensuremath{
\begin{array}{rcl}
#2 : #3 &\rightarrow& #4\\
#5 &\mapsto& #6
\end{array}
}}

\newcommand{\eqcolon}{\mathrel{\resizebox{\widthof{$\mathord{=}$}}{\height}{ $\!\!=\!\!\resizebox{1.2\width}{0.8\height}{\raisebox{0.23ex}{$\mathop{:}$}}\!\!$ }}}
\newcommand{\coloneq}{\mathrel{\resizebox{\widthof{$\mathord{=}$}}{\height}{ $\!\!\resizebox{1.2\width}{0.8\height}{\raisebox{0.23ex}{$\mathop{:}$}}\!\!=\!\!$ }}}
\newcommand{\eqcolonl}{\ensuremath{\mathrel{=\!\!\mathop{:}}}}
\newcommand{\coloneql}{\ensuremath{\mathrel{\mathop{:} \!\! =}}}
\newcommand{\vc}[1]{% inline column vector
  \left(\begin{smallmatrix}#1\end{smallmatrix}\right)%
}
\newcommand{\vr}[1]{% inline row vector
  \begin{smallmatrix}(\,#1\,)\end{smallmatrix}%
}
\makeatletter
\newcommand*{\defeq}{\ =\mathrel{\rlap{%
                     \raisebox{0.3ex}{$\m@th\cdot$}}%
                     \raisebox{-0.3ex}{$\m@th\cdot$}}%
                     }
\makeatother

\newcommand{\mathcircle}[1]{% inline row vector
 \overset{\circ}{#1}
}
\newcommand{\ulim}{% low limit
 \underline{\lim}
}
\newcommand{\ssi}{% iff
\iff
}
\newcommand{\ps}[2]{
\expval{#1 | #2}
}
\newcommand{\df}[1]{
\mqty{#1}
}
\newcommand{\n}[1]{
\norm{#1}
}
\newcommand{\sys}[1]{
\left\{\smqty{#1}\right.
}


\newcommand{\eqdef}{\ensuremath{\overset{\text{def}}=}}


\def\Circlearrowright{\ensuremath{%
  \rotatebox[origin=c]{230}{$\circlearrowright$}}}

\newcommand\ct[1]{\text{\rmfamily\upshape #1}}
\newcommand\question[1]{ {\color{red} ...!? \small #1}}
\newcommand\caz[1]{\left\{\begin{array} #1 \end{array}\right.}
\newcommand\const{\text{\rmfamily\upshape const}}
\newcommand\toP{ \overset{\pro}{\to}}
\newcommand\toPP{ \overset{\text{PP}}{\to}}
\newcommand{\oeq}{\mathrel{\text{\textcircled{$=$}}}}





\usepackage{xcolor}
% \usepackage[normalem]{ulem}
\usepackage{lipsum}
\makeatletter
% \newcommand\colorwave[1][blue]{\bgroup \markoverwith{\lower3.5\p@\hbox{\sixly \textcolor{#1}{\char58}}}\ULon}
%\font\sixly=lasy6 % does not re-load if already loaded, so no memory problem.

\newmdtheoremenv[
linewidth= 1pt,linecolor= blue,%
leftmargin=20,rightmargin=20,innertopmargin=0pt, innerrightmargin=40,%
tikzsetting = { draw=lightgray, line width = 0.3pt,dashed,%
dash pattern = on 15pt off 3pt},%
splittopskip=\topskip,skipbelow=\baselineskip,%
skipabove=\baselineskip,ntheorem,roundcorner=0pt,
% backgroundcolor=pagebg,font=\color{orange}\sffamily, fontcolor=white
]{examplebox}{Exemple}[section]



\newcommand\R{\mathbb{R}}
\newcommand\Z{\mathbb{Z}}
\newcommand\N{\mathbb{N}}
\newcommand\E{\mathbb{E}}
\newcommand\F{\mathcal{F}}
\newcommand\cH{\mathcal{H}}
\newcommand\V{\mathbb{V}}
\newcommand\dmo{ ^{-1} }
\newcommand\kapa{\kappa}
\newcommand\im{Im}
\newcommand\hs{\mathcal{H}}





\usepackage{soul}

\makeatletter
\newcommand*{\whiten}[1]{\llap{\textcolor{white}{{\the\SOUL@token}}\hspace{#1pt}}}
\DeclareRobustCommand*\myul{%
    \def\SOUL@everyspace{\underline{\space}\kern\z@}%
    \def\SOUL@everytoken{%
     \setbox0=\hbox{\the\SOUL@token}%
     \ifdim\dp0>\z@
        \raisebox{\dp0}{\underline{\phantom{\the\SOUL@token}}}%
        \whiten{1}\whiten{0}%
        \whiten{-1}\whiten{-2}%
        \llap{\the\SOUL@token}%
     \else
        \underline{\the\SOUL@token}%
     \fi}%
\SOUL@}
\makeatother

\newcommand*{\demp}{\fontfamily{lmtt}\selectfont}

\DeclareTextFontCommand{\textdemp}{\demp}

\begin{document}

\ifcomment
Multiline
comment
\fi
\ifcomment
\myul{Typesetting test}
% \color[rgb]{1,1,1}
$∑_i^n≠ 60º±∞π∆¬≈√j∫h≤≥µ$

$\CR \R\pro\ind\pro\gS\pro
\mqty[a&b\\c&d]$
$\pro\mathbb{P}$
$\dd{x}$

  \[
    \alpha(x)=\left\{
                \begin{array}{ll}
                  x\\
                  \frac{1}{1+e^{-kx}}\\
                  \frac{e^x-e^{-x}}{e^x+e^{-x}}
                \end{array}
              \right.
  \]

  $\expval{x}$
  
  $\chi_\rho(ghg\dmo)=\Tr(\rho_{ghg\dmo})=\Tr(\rho_g\circ\rho_h\circ\rho\dmo_g)=\Tr(\rho_h)\overset{\mbox{\scalebox{0.5}{$\Tr(AB)=\Tr(BA)$}}}{=}\chi_\rho(h)$
  	$\mathop{\oplus}_{\substack{x\in X}}$

$\mat(\rho_g)=(a_{ij}(g))_{\scriptsize \substack{1\leq i\leq d \\ 1\leq j\leq d}}$ et $\mat(\rho'_g)=(a'_{ij}(g))_{\scriptsize \substack{1\leq i'\leq d' \\ 1\leq j'\leq d'}}$



\[\int_a^b{\mathbb{R}^2}g(u, v)\dd{P_{XY}}(u, v)=\iint g(u,v) f_{XY}(u, v)\dd \lambda(u) \dd \lambda(v)\]
$$\lim_{x\to\infty} f(x)$$	
$$\iiiint_V \mu(t,u,v,w) \,dt\,du\,dv\,dw$$
$$\sum_{n=1}^{\infty} 2^{-n} = 1$$	
\begin{definition}
	Si $X$ et $Y$ sont 2 v.a. ou definit la \textsc{Covariance} entre $X$ et $Y$ comme
	$\cov(X,Y)\overset{\text{def}}{=}\E\left[(X-\E(X))(Y-\E(Y))\right]=\E(XY)-\E(X)\E(Y)$.
\end{definition}
\fi
\pagebreak

% \tableofcontents

% insert your code here
%\input{./algebra/main.tex}
%\input{./geometrie-differentielle/main.tex}
%\input{./probabilite/main.tex}
%\input{./analyse-fonctionnelle/main.tex}
% \input{./Analyse-convexe-et-dualite-en-optimisation/main.tex}
%\input{./tikz/main.tex}
%\input{./Theorie-du-distributions/main.tex}
%\input{./optimisation/mine.tex}
 \input{./modelisation/main.tex}

% yves.aubry@univ-tln.fr : algebra

\end{document}

%% !TEX encoding = UTF-8 Unicode
% !TEX TS-program = xelatex

\documentclass[french]{report}

%\usepackage[utf8]{inputenc}
%\usepackage[T1]{fontenc}
\usepackage{babel}


\newif\ifcomment
%\commenttrue # Show comments

\usepackage{physics}
\usepackage{amssymb}


\usepackage{amsthm}
% \usepackage{thmtools}
\usepackage{mathtools}
\usepackage{amsfonts}

\usepackage{color}

\usepackage{tikz}

\usepackage{geometry}
\geometry{a5paper, margin=0.1in, right=1cm}

\usepackage{dsfont}

\usepackage{graphicx}
\graphicspath{ {images/} }

\usepackage{faktor}

\usepackage{IEEEtrantools}
\usepackage{enumerate}   
\usepackage[PostScript=dvips]{"/Users/aware/Documents/Courses/diagrams"}


\newtheorem{theorem}{Théorème}[section]
\renewcommand{\thetheorem}{\arabic{theorem}}
\newtheorem{lemme}{Lemme}[section]
\renewcommand{\thelemme}{\arabic{lemme}}
\newtheorem{proposition}{Proposition}[section]
\renewcommand{\theproposition}{\arabic{proposition}}
\newtheorem{notations}{Notations}[section]
\newtheorem{problem}{Problème}[section]
\newtheorem{corollary}{Corollaire}[theorem]
\renewcommand{\thecorollary}{\arabic{corollary}}
\newtheorem{property}{Propriété}[section]
\newtheorem{objective}{Objectif}[section]

\theoremstyle{definition}
\newtheorem{definition}{Définition}[section]
\renewcommand{\thedefinition}{\arabic{definition}}
\newtheorem{exercise}{Exercice}[chapter]
\renewcommand{\theexercise}{\arabic{exercise}}
\newtheorem{example}{Exemple}[chapter]
\renewcommand{\theexample}{\arabic{example}}
\newtheorem*{solution}{Solution}
\newtheorem*{application}{Application}
\newtheorem*{notation}{Notation}
\newtheorem*{vocabulary}{Vocabulaire}
\newtheorem*{properties}{Propriétés}



\theoremstyle{remark}
\newtheorem*{remark}{Remarque}
\newtheorem*{rappel}{Rappel}


\usepackage{etoolbox}
\AtBeginEnvironment{exercise}{\small}
\AtBeginEnvironment{example}{\small}

\usepackage{cases}
\usepackage[red]{mypack}

\usepackage[framemethod=TikZ]{mdframed}

\definecolor{bg}{rgb}{0.4,0.25,0.95}
\definecolor{pagebg}{rgb}{0,0,0.5}
\surroundwithmdframed[
   topline=false,
   rightline=false,
   bottomline=false,
   leftmargin=\parindent,
   skipabove=8pt,
   skipbelow=8pt,
   linecolor=blue,
   innerbottommargin=10pt,
   % backgroundcolor=bg,font=\color{orange}\sffamily, fontcolor=white
]{definition}

\usepackage{empheq}
\usepackage[most]{tcolorbox}

\newtcbox{\mymath}[1][]{%
    nobeforeafter, math upper, tcbox raise base,
    enhanced, colframe=blue!30!black,
    colback=red!10, boxrule=1pt,
    #1}

\usepackage{unixode}


\DeclareMathOperator{\ord}{ord}
\DeclareMathOperator{\orb}{orb}
\DeclareMathOperator{\stab}{stab}
\DeclareMathOperator{\Stab}{stab}
\DeclareMathOperator{\ppcm}{ppcm}
\DeclareMathOperator{\conj}{Conj}
\DeclareMathOperator{\End}{End}
\DeclareMathOperator{\rot}{rot}
\DeclareMathOperator{\trs}{trace}
\DeclareMathOperator{\Ind}{Ind}
\DeclareMathOperator{\mat}{Mat}
\DeclareMathOperator{\id}{Id}
\DeclareMathOperator{\vect}{vect}
\DeclareMathOperator{\img}{img}
\DeclareMathOperator{\cov}{Cov}
\DeclareMathOperator{\dist}{dist}
\DeclareMathOperator{\irr}{Irr}
\DeclareMathOperator{\image}{Im}
\DeclareMathOperator{\pd}{\partial}
\DeclareMathOperator{\epi}{epi}
\DeclareMathOperator{\Argmin}{Argmin}
\DeclareMathOperator{\dom}{dom}
\DeclareMathOperator{\proj}{proj}
\DeclareMathOperator{\ctg}{ctg}
\DeclareMathOperator{\supp}{supp}
\DeclareMathOperator{\argmin}{argmin}
\DeclareMathOperator{\mult}{mult}
\DeclareMathOperator{\ch}{ch}
\DeclareMathOperator{\sh}{sh}
\DeclareMathOperator{\rang}{rang}
\DeclareMathOperator{\diam}{diam}
\DeclareMathOperator{\Epigraphe}{Epigraphe}




\usepackage{xcolor}
\everymath{\color{blue}}
%\everymath{\color[rgb]{0,1,1}}
%\pagecolor[rgb]{0,0,0.5}


\newcommand*{\pdtest}[3][]{\ensuremath{\frac{\partial^{#1} #2}{\partial #3}}}

\newcommand*{\deffunc}[6][]{\ensuremath{
\begin{array}{rcl}
#2 : #3 &\rightarrow& #4\\
#5 &\mapsto& #6
\end{array}
}}

\newcommand{\eqcolon}{\mathrel{\resizebox{\widthof{$\mathord{=}$}}{\height}{ $\!\!=\!\!\resizebox{1.2\width}{0.8\height}{\raisebox{0.23ex}{$\mathop{:}$}}\!\!$ }}}
\newcommand{\coloneq}{\mathrel{\resizebox{\widthof{$\mathord{=}$}}{\height}{ $\!\!\resizebox{1.2\width}{0.8\height}{\raisebox{0.23ex}{$\mathop{:}$}}\!\!=\!\!$ }}}
\newcommand{\eqcolonl}{\ensuremath{\mathrel{=\!\!\mathop{:}}}}
\newcommand{\coloneql}{\ensuremath{\mathrel{\mathop{:} \!\! =}}}
\newcommand{\vc}[1]{% inline column vector
  \left(\begin{smallmatrix}#1\end{smallmatrix}\right)%
}
\newcommand{\vr}[1]{% inline row vector
  \begin{smallmatrix}(\,#1\,)\end{smallmatrix}%
}
\makeatletter
\newcommand*{\defeq}{\ =\mathrel{\rlap{%
                     \raisebox{0.3ex}{$\m@th\cdot$}}%
                     \raisebox{-0.3ex}{$\m@th\cdot$}}%
                     }
\makeatother

\newcommand{\mathcircle}[1]{% inline row vector
 \overset{\circ}{#1}
}
\newcommand{\ulim}{% low limit
 \underline{\lim}
}
\newcommand{\ssi}{% iff
\iff
}
\newcommand{\ps}[2]{
\expval{#1 | #2}
}
\newcommand{\df}[1]{
\mqty{#1}
}
\newcommand{\n}[1]{
\norm{#1}
}
\newcommand{\sys}[1]{
\left\{\smqty{#1}\right.
}


\newcommand{\eqdef}{\ensuremath{\overset{\text{def}}=}}


\def\Circlearrowright{\ensuremath{%
  \rotatebox[origin=c]{230}{$\circlearrowright$}}}

\newcommand\ct[1]{\text{\rmfamily\upshape #1}}
\newcommand\question[1]{ {\color{red} ...!? \small #1}}
\newcommand\caz[1]{\left\{\begin{array} #1 \end{array}\right.}
\newcommand\const{\text{\rmfamily\upshape const}}
\newcommand\toP{ \overset{\pro}{\to}}
\newcommand\toPP{ \overset{\text{PP}}{\to}}
\newcommand{\oeq}{\mathrel{\text{\textcircled{$=$}}}}





\usepackage{xcolor}
% \usepackage[normalem]{ulem}
\usepackage{lipsum}
\makeatletter
% \newcommand\colorwave[1][blue]{\bgroup \markoverwith{\lower3.5\p@\hbox{\sixly \textcolor{#1}{\char58}}}\ULon}
%\font\sixly=lasy6 % does not re-load if already loaded, so no memory problem.

\newmdtheoremenv[
linewidth= 1pt,linecolor= blue,%
leftmargin=20,rightmargin=20,innertopmargin=0pt, innerrightmargin=40,%
tikzsetting = { draw=lightgray, line width = 0.3pt,dashed,%
dash pattern = on 15pt off 3pt},%
splittopskip=\topskip,skipbelow=\baselineskip,%
skipabove=\baselineskip,ntheorem,roundcorner=0pt,
% backgroundcolor=pagebg,font=\color{orange}\sffamily, fontcolor=white
]{examplebox}{Exemple}[section]



\newcommand\R{\mathbb{R}}
\newcommand\Z{\mathbb{Z}}
\newcommand\N{\mathbb{N}}
\newcommand\E{\mathbb{E}}
\newcommand\F{\mathcal{F}}
\newcommand\cH{\mathcal{H}}
\newcommand\V{\mathbb{V}}
\newcommand\dmo{ ^{-1} }
\newcommand\kapa{\kappa}
\newcommand\im{Im}
\newcommand\hs{\mathcal{H}}





\usepackage{soul}

\makeatletter
\newcommand*{\whiten}[1]{\llap{\textcolor{white}{{\the\SOUL@token}}\hspace{#1pt}}}
\DeclareRobustCommand*\myul{%
    \def\SOUL@everyspace{\underline{\space}\kern\z@}%
    \def\SOUL@everytoken{%
     \setbox0=\hbox{\the\SOUL@token}%
     \ifdim\dp0>\z@
        \raisebox{\dp0}{\underline{\phantom{\the\SOUL@token}}}%
        \whiten{1}\whiten{0}%
        \whiten{-1}\whiten{-2}%
        \llap{\the\SOUL@token}%
     \else
        \underline{\the\SOUL@token}%
     \fi}%
\SOUL@}
\makeatother

\newcommand*{\demp}{\fontfamily{lmtt}\selectfont}

\DeclareTextFontCommand{\textdemp}{\demp}

\begin{document}

\ifcomment
Multiline
comment
\fi
\ifcomment
\myul{Typesetting test}
% \color[rgb]{1,1,1}
$∑_i^n≠ 60º±∞π∆¬≈√j∫h≤≥µ$

$\CR \R\pro\ind\pro\gS\pro
\mqty[a&b\\c&d]$
$\pro\mathbb{P}$
$\dd{x}$

  \[
    \alpha(x)=\left\{
                \begin{array}{ll}
                  x\\
                  \frac{1}{1+e^{-kx}}\\
                  \frac{e^x-e^{-x}}{e^x+e^{-x}}
                \end{array}
              \right.
  \]

  $\expval{x}$
  
  $\chi_\rho(ghg\dmo)=\Tr(\rho_{ghg\dmo})=\Tr(\rho_g\circ\rho_h\circ\rho\dmo_g)=\Tr(\rho_h)\overset{\mbox{\scalebox{0.5}{$\Tr(AB)=\Tr(BA)$}}}{=}\chi_\rho(h)$
  	$\mathop{\oplus}_{\substack{x\in X}}$

$\mat(\rho_g)=(a_{ij}(g))_{\scriptsize \substack{1\leq i\leq d \\ 1\leq j\leq d}}$ et $\mat(\rho'_g)=(a'_{ij}(g))_{\scriptsize \substack{1\leq i'\leq d' \\ 1\leq j'\leq d'}}$



\[\int_a^b{\mathbb{R}^2}g(u, v)\dd{P_{XY}}(u, v)=\iint g(u,v) f_{XY}(u, v)\dd \lambda(u) \dd \lambda(v)\]
$$\lim_{x\to\infty} f(x)$$	
$$\iiiint_V \mu(t,u,v,w) \,dt\,du\,dv\,dw$$
$$\sum_{n=1}^{\infty} 2^{-n} = 1$$	
\begin{definition}
	Si $X$ et $Y$ sont 2 v.a. ou definit la \textsc{Covariance} entre $X$ et $Y$ comme
	$\cov(X,Y)\overset{\text{def}}{=}\E\left[(X-\E(X))(Y-\E(Y))\right]=\E(XY)-\E(X)\E(Y)$.
\end{definition}
\fi
\pagebreak

% \tableofcontents

% insert your code here
%\input{./algebra/main.tex}
%\input{./geometrie-differentielle/main.tex}
%\input{./probabilite/main.tex}
%\input{./analyse-fonctionnelle/main.tex}
% \input{./Analyse-convexe-et-dualite-en-optimisation/main.tex}
%\input{./tikz/main.tex}
%\input{./Theorie-du-distributions/main.tex}
%\input{./optimisation/mine.tex}
 \input{./modelisation/main.tex}

% yves.aubry@univ-tln.fr : algebra

\end{document}

%% !TEX encoding = UTF-8 Unicode
% !TEX TS-program = xelatex

\documentclass[french]{report}

%\usepackage[utf8]{inputenc}
%\usepackage[T1]{fontenc}
\usepackage{babel}


\newif\ifcomment
%\commenttrue # Show comments

\usepackage{physics}
\usepackage{amssymb}


\usepackage{amsthm}
% \usepackage{thmtools}
\usepackage{mathtools}
\usepackage{amsfonts}

\usepackage{color}

\usepackage{tikz}

\usepackage{geometry}
\geometry{a5paper, margin=0.1in, right=1cm}

\usepackage{dsfont}

\usepackage{graphicx}
\graphicspath{ {images/} }

\usepackage{faktor}

\usepackage{IEEEtrantools}
\usepackage{enumerate}   
\usepackage[PostScript=dvips]{"/Users/aware/Documents/Courses/diagrams"}


\newtheorem{theorem}{Théorème}[section]
\renewcommand{\thetheorem}{\arabic{theorem}}
\newtheorem{lemme}{Lemme}[section]
\renewcommand{\thelemme}{\arabic{lemme}}
\newtheorem{proposition}{Proposition}[section]
\renewcommand{\theproposition}{\arabic{proposition}}
\newtheorem{notations}{Notations}[section]
\newtheorem{problem}{Problème}[section]
\newtheorem{corollary}{Corollaire}[theorem]
\renewcommand{\thecorollary}{\arabic{corollary}}
\newtheorem{property}{Propriété}[section]
\newtheorem{objective}{Objectif}[section]

\theoremstyle{definition}
\newtheorem{definition}{Définition}[section]
\renewcommand{\thedefinition}{\arabic{definition}}
\newtheorem{exercise}{Exercice}[chapter]
\renewcommand{\theexercise}{\arabic{exercise}}
\newtheorem{example}{Exemple}[chapter]
\renewcommand{\theexample}{\arabic{example}}
\newtheorem*{solution}{Solution}
\newtheorem*{application}{Application}
\newtheorem*{notation}{Notation}
\newtheorem*{vocabulary}{Vocabulaire}
\newtheorem*{properties}{Propriétés}



\theoremstyle{remark}
\newtheorem*{remark}{Remarque}
\newtheorem*{rappel}{Rappel}


\usepackage{etoolbox}
\AtBeginEnvironment{exercise}{\small}
\AtBeginEnvironment{example}{\small}

\usepackage{cases}
\usepackage[red]{mypack}

\usepackage[framemethod=TikZ]{mdframed}

\definecolor{bg}{rgb}{0.4,0.25,0.95}
\definecolor{pagebg}{rgb}{0,0,0.5}
\surroundwithmdframed[
   topline=false,
   rightline=false,
   bottomline=false,
   leftmargin=\parindent,
   skipabove=8pt,
   skipbelow=8pt,
   linecolor=blue,
   innerbottommargin=10pt,
   % backgroundcolor=bg,font=\color{orange}\sffamily, fontcolor=white
]{definition}

\usepackage{empheq}
\usepackage[most]{tcolorbox}

\newtcbox{\mymath}[1][]{%
    nobeforeafter, math upper, tcbox raise base,
    enhanced, colframe=blue!30!black,
    colback=red!10, boxrule=1pt,
    #1}

\usepackage{unixode}


\DeclareMathOperator{\ord}{ord}
\DeclareMathOperator{\orb}{orb}
\DeclareMathOperator{\stab}{stab}
\DeclareMathOperator{\Stab}{stab}
\DeclareMathOperator{\ppcm}{ppcm}
\DeclareMathOperator{\conj}{Conj}
\DeclareMathOperator{\End}{End}
\DeclareMathOperator{\rot}{rot}
\DeclareMathOperator{\trs}{trace}
\DeclareMathOperator{\Ind}{Ind}
\DeclareMathOperator{\mat}{Mat}
\DeclareMathOperator{\id}{Id}
\DeclareMathOperator{\vect}{vect}
\DeclareMathOperator{\img}{img}
\DeclareMathOperator{\cov}{Cov}
\DeclareMathOperator{\dist}{dist}
\DeclareMathOperator{\irr}{Irr}
\DeclareMathOperator{\image}{Im}
\DeclareMathOperator{\pd}{\partial}
\DeclareMathOperator{\epi}{epi}
\DeclareMathOperator{\Argmin}{Argmin}
\DeclareMathOperator{\dom}{dom}
\DeclareMathOperator{\proj}{proj}
\DeclareMathOperator{\ctg}{ctg}
\DeclareMathOperator{\supp}{supp}
\DeclareMathOperator{\argmin}{argmin}
\DeclareMathOperator{\mult}{mult}
\DeclareMathOperator{\ch}{ch}
\DeclareMathOperator{\sh}{sh}
\DeclareMathOperator{\rang}{rang}
\DeclareMathOperator{\diam}{diam}
\DeclareMathOperator{\Epigraphe}{Epigraphe}




\usepackage{xcolor}
\everymath{\color{blue}}
%\everymath{\color[rgb]{0,1,1}}
%\pagecolor[rgb]{0,0,0.5}


\newcommand*{\pdtest}[3][]{\ensuremath{\frac{\partial^{#1} #2}{\partial #3}}}

\newcommand*{\deffunc}[6][]{\ensuremath{
\begin{array}{rcl}
#2 : #3 &\rightarrow& #4\\
#5 &\mapsto& #6
\end{array}
}}

\newcommand{\eqcolon}{\mathrel{\resizebox{\widthof{$\mathord{=}$}}{\height}{ $\!\!=\!\!\resizebox{1.2\width}{0.8\height}{\raisebox{0.23ex}{$\mathop{:}$}}\!\!$ }}}
\newcommand{\coloneq}{\mathrel{\resizebox{\widthof{$\mathord{=}$}}{\height}{ $\!\!\resizebox{1.2\width}{0.8\height}{\raisebox{0.23ex}{$\mathop{:}$}}\!\!=\!\!$ }}}
\newcommand{\eqcolonl}{\ensuremath{\mathrel{=\!\!\mathop{:}}}}
\newcommand{\coloneql}{\ensuremath{\mathrel{\mathop{:} \!\! =}}}
\newcommand{\vc}[1]{% inline column vector
  \left(\begin{smallmatrix}#1\end{smallmatrix}\right)%
}
\newcommand{\vr}[1]{% inline row vector
  \begin{smallmatrix}(\,#1\,)\end{smallmatrix}%
}
\makeatletter
\newcommand*{\defeq}{\ =\mathrel{\rlap{%
                     \raisebox{0.3ex}{$\m@th\cdot$}}%
                     \raisebox{-0.3ex}{$\m@th\cdot$}}%
                     }
\makeatother

\newcommand{\mathcircle}[1]{% inline row vector
 \overset{\circ}{#1}
}
\newcommand{\ulim}{% low limit
 \underline{\lim}
}
\newcommand{\ssi}{% iff
\iff
}
\newcommand{\ps}[2]{
\expval{#1 | #2}
}
\newcommand{\df}[1]{
\mqty{#1}
}
\newcommand{\n}[1]{
\norm{#1}
}
\newcommand{\sys}[1]{
\left\{\smqty{#1}\right.
}


\newcommand{\eqdef}{\ensuremath{\overset{\text{def}}=}}


\def\Circlearrowright{\ensuremath{%
  \rotatebox[origin=c]{230}{$\circlearrowright$}}}

\newcommand\ct[1]{\text{\rmfamily\upshape #1}}
\newcommand\question[1]{ {\color{red} ...!? \small #1}}
\newcommand\caz[1]{\left\{\begin{array} #1 \end{array}\right.}
\newcommand\const{\text{\rmfamily\upshape const}}
\newcommand\toP{ \overset{\pro}{\to}}
\newcommand\toPP{ \overset{\text{PP}}{\to}}
\newcommand{\oeq}{\mathrel{\text{\textcircled{$=$}}}}





\usepackage{xcolor}
% \usepackage[normalem]{ulem}
\usepackage{lipsum}
\makeatletter
% \newcommand\colorwave[1][blue]{\bgroup \markoverwith{\lower3.5\p@\hbox{\sixly \textcolor{#1}{\char58}}}\ULon}
%\font\sixly=lasy6 % does not re-load if already loaded, so no memory problem.

\newmdtheoremenv[
linewidth= 1pt,linecolor= blue,%
leftmargin=20,rightmargin=20,innertopmargin=0pt, innerrightmargin=40,%
tikzsetting = { draw=lightgray, line width = 0.3pt,dashed,%
dash pattern = on 15pt off 3pt},%
splittopskip=\topskip,skipbelow=\baselineskip,%
skipabove=\baselineskip,ntheorem,roundcorner=0pt,
% backgroundcolor=pagebg,font=\color{orange}\sffamily, fontcolor=white
]{examplebox}{Exemple}[section]



\newcommand\R{\mathbb{R}}
\newcommand\Z{\mathbb{Z}}
\newcommand\N{\mathbb{N}}
\newcommand\E{\mathbb{E}}
\newcommand\F{\mathcal{F}}
\newcommand\cH{\mathcal{H}}
\newcommand\V{\mathbb{V}}
\newcommand\dmo{ ^{-1} }
\newcommand\kapa{\kappa}
\newcommand\im{Im}
\newcommand\hs{\mathcal{H}}





\usepackage{soul}

\makeatletter
\newcommand*{\whiten}[1]{\llap{\textcolor{white}{{\the\SOUL@token}}\hspace{#1pt}}}
\DeclareRobustCommand*\myul{%
    \def\SOUL@everyspace{\underline{\space}\kern\z@}%
    \def\SOUL@everytoken{%
     \setbox0=\hbox{\the\SOUL@token}%
     \ifdim\dp0>\z@
        \raisebox{\dp0}{\underline{\phantom{\the\SOUL@token}}}%
        \whiten{1}\whiten{0}%
        \whiten{-1}\whiten{-2}%
        \llap{\the\SOUL@token}%
     \else
        \underline{\the\SOUL@token}%
     \fi}%
\SOUL@}
\makeatother

\newcommand*{\demp}{\fontfamily{lmtt}\selectfont}

\DeclareTextFontCommand{\textdemp}{\demp}

\begin{document}

\ifcomment
Multiline
comment
\fi
\ifcomment
\myul{Typesetting test}
% \color[rgb]{1,1,1}
$∑_i^n≠ 60º±∞π∆¬≈√j∫h≤≥µ$

$\CR \R\pro\ind\pro\gS\pro
\mqty[a&b\\c&d]$
$\pro\mathbb{P}$
$\dd{x}$

  \[
    \alpha(x)=\left\{
                \begin{array}{ll}
                  x\\
                  \frac{1}{1+e^{-kx}}\\
                  \frac{e^x-e^{-x}}{e^x+e^{-x}}
                \end{array}
              \right.
  \]

  $\expval{x}$
  
  $\chi_\rho(ghg\dmo)=\Tr(\rho_{ghg\dmo})=\Tr(\rho_g\circ\rho_h\circ\rho\dmo_g)=\Tr(\rho_h)\overset{\mbox{\scalebox{0.5}{$\Tr(AB)=\Tr(BA)$}}}{=}\chi_\rho(h)$
  	$\mathop{\oplus}_{\substack{x\in X}}$

$\mat(\rho_g)=(a_{ij}(g))_{\scriptsize \substack{1\leq i\leq d \\ 1\leq j\leq d}}$ et $\mat(\rho'_g)=(a'_{ij}(g))_{\scriptsize \substack{1\leq i'\leq d' \\ 1\leq j'\leq d'}}$



\[\int_a^b{\mathbb{R}^2}g(u, v)\dd{P_{XY}}(u, v)=\iint g(u,v) f_{XY}(u, v)\dd \lambda(u) \dd \lambda(v)\]
$$\lim_{x\to\infty} f(x)$$	
$$\iiiint_V \mu(t,u,v,w) \,dt\,du\,dv\,dw$$
$$\sum_{n=1}^{\infty} 2^{-n} = 1$$	
\begin{definition}
	Si $X$ et $Y$ sont 2 v.a. ou definit la \textsc{Covariance} entre $X$ et $Y$ comme
	$\cov(X,Y)\overset{\text{def}}{=}\E\left[(X-\E(X))(Y-\E(Y))\right]=\E(XY)-\E(X)\E(Y)$.
\end{definition}
\fi
\pagebreak

% \tableofcontents

% insert your code here
%\input{./algebra/main.tex}
%\input{./geometrie-differentielle/main.tex}
%\input{./probabilite/main.tex}
%\input{./analyse-fonctionnelle/main.tex}
% \input{./Analyse-convexe-et-dualite-en-optimisation/main.tex}
%\input{./tikz/main.tex}
%\input{./Theorie-du-distributions/main.tex}
%\input{./optimisation/mine.tex}
 \input{./modelisation/main.tex}

% yves.aubry@univ-tln.fr : algebra

\end{document}

% % !TEX encoding = UTF-8 Unicode
% !TEX TS-program = xelatex

\documentclass[french]{report}

%\usepackage[utf8]{inputenc}
%\usepackage[T1]{fontenc}
\usepackage{babel}


\newif\ifcomment
%\commenttrue # Show comments

\usepackage{physics}
\usepackage{amssymb}


\usepackage{amsthm}
% \usepackage{thmtools}
\usepackage{mathtools}
\usepackage{amsfonts}

\usepackage{color}

\usepackage{tikz}

\usepackage{geometry}
\geometry{a5paper, margin=0.1in, right=1cm}

\usepackage{dsfont}

\usepackage{graphicx}
\graphicspath{ {images/} }

\usepackage{faktor}

\usepackage{IEEEtrantools}
\usepackage{enumerate}   
\usepackage[PostScript=dvips]{"/Users/aware/Documents/Courses/diagrams"}


\newtheorem{theorem}{Théorème}[section]
\renewcommand{\thetheorem}{\arabic{theorem}}
\newtheorem{lemme}{Lemme}[section]
\renewcommand{\thelemme}{\arabic{lemme}}
\newtheorem{proposition}{Proposition}[section]
\renewcommand{\theproposition}{\arabic{proposition}}
\newtheorem{notations}{Notations}[section]
\newtheorem{problem}{Problème}[section]
\newtheorem{corollary}{Corollaire}[theorem]
\renewcommand{\thecorollary}{\arabic{corollary}}
\newtheorem{property}{Propriété}[section]
\newtheorem{objective}{Objectif}[section]

\theoremstyle{definition}
\newtheorem{definition}{Définition}[section]
\renewcommand{\thedefinition}{\arabic{definition}}
\newtheorem{exercise}{Exercice}[chapter]
\renewcommand{\theexercise}{\arabic{exercise}}
\newtheorem{example}{Exemple}[chapter]
\renewcommand{\theexample}{\arabic{example}}
\newtheorem*{solution}{Solution}
\newtheorem*{application}{Application}
\newtheorem*{notation}{Notation}
\newtheorem*{vocabulary}{Vocabulaire}
\newtheorem*{properties}{Propriétés}



\theoremstyle{remark}
\newtheorem*{remark}{Remarque}
\newtheorem*{rappel}{Rappel}


\usepackage{etoolbox}
\AtBeginEnvironment{exercise}{\small}
\AtBeginEnvironment{example}{\small}

\usepackage{cases}
\usepackage[red]{mypack}

\usepackage[framemethod=TikZ]{mdframed}

\definecolor{bg}{rgb}{0.4,0.25,0.95}
\definecolor{pagebg}{rgb}{0,0,0.5}
\surroundwithmdframed[
   topline=false,
   rightline=false,
   bottomline=false,
   leftmargin=\parindent,
   skipabove=8pt,
   skipbelow=8pt,
   linecolor=blue,
   innerbottommargin=10pt,
   % backgroundcolor=bg,font=\color{orange}\sffamily, fontcolor=white
]{definition}

\usepackage{empheq}
\usepackage[most]{tcolorbox}

\newtcbox{\mymath}[1][]{%
    nobeforeafter, math upper, tcbox raise base,
    enhanced, colframe=blue!30!black,
    colback=red!10, boxrule=1pt,
    #1}

\usepackage{unixode}


\DeclareMathOperator{\ord}{ord}
\DeclareMathOperator{\orb}{orb}
\DeclareMathOperator{\stab}{stab}
\DeclareMathOperator{\Stab}{stab}
\DeclareMathOperator{\ppcm}{ppcm}
\DeclareMathOperator{\conj}{Conj}
\DeclareMathOperator{\End}{End}
\DeclareMathOperator{\rot}{rot}
\DeclareMathOperator{\trs}{trace}
\DeclareMathOperator{\Ind}{Ind}
\DeclareMathOperator{\mat}{Mat}
\DeclareMathOperator{\id}{Id}
\DeclareMathOperator{\vect}{vect}
\DeclareMathOperator{\img}{img}
\DeclareMathOperator{\cov}{Cov}
\DeclareMathOperator{\dist}{dist}
\DeclareMathOperator{\irr}{Irr}
\DeclareMathOperator{\image}{Im}
\DeclareMathOperator{\pd}{\partial}
\DeclareMathOperator{\epi}{epi}
\DeclareMathOperator{\Argmin}{Argmin}
\DeclareMathOperator{\dom}{dom}
\DeclareMathOperator{\proj}{proj}
\DeclareMathOperator{\ctg}{ctg}
\DeclareMathOperator{\supp}{supp}
\DeclareMathOperator{\argmin}{argmin}
\DeclareMathOperator{\mult}{mult}
\DeclareMathOperator{\ch}{ch}
\DeclareMathOperator{\sh}{sh}
\DeclareMathOperator{\rang}{rang}
\DeclareMathOperator{\diam}{diam}
\DeclareMathOperator{\Epigraphe}{Epigraphe}




\usepackage{xcolor}
\everymath{\color{blue}}
%\everymath{\color[rgb]{0,1,1}}
%\pagecolor[rgb]{0,0,0.5}


\newcommand*{\pdtest}[3][]{\ensuremath{\frac{\partial^{#1} #2}{\partial #3}}}

\newcommand*{\deffunc}[6][]{\ensuremath{
\begin{array}{rcl}
#2 : #3 &\rightarrow& #4\\
#5 &\mapsto& #6
\end{array}
}}

\newcommand{\eqcolon}{\mathrel{\resizebox{\widthof{$\mathord{=}$}}{\height}{ $\!\!=\!\!\resizebox{1.2\width}{0.8\height}{\raisebox{0.23ex}{$\mathop{:}$}}\!\!$ }}}
\newcommand{\coloneq}{\mathrel{\resizebox{\widthof{$\mathord{=}$}}{\height}{ $\!\!\resizebox{1.2\width}{0.8\height}{\raisebox{0.23ex}{$\mathop{:}$}}\!\!=\!\!$ }}}
\newcommand{\eqcolonl}{\ensuremath{\mathrel{=\!\!\mathop{:}}}}
\newcommand{\coloneql}{\ensuremath{\mathrel{\mathop{:} \!\! =}}}
\newcommand{\vc}[1]{% inline column vector
  \left(\begin{smallmatrix}#1\end{smallmatrix}\right)%
}
\newcommand{\vr}[1]{% inline row vector
  \begin{smallmatrix}(\,#1\,)\end{smallmatrix}%
}
\makeatletter
\newcommand*{\defeq}{\ =\mathrel{\rlap{%
                     \raisebox{0.3ex}{$\m@th\cdot$}}%
                     \raisebox{-0.3ex}{$\m@th\cdot$}}%
                     }
\makeatother

\newcommand{\mathcircle}[1]{% inline row vector
 \overset{\circ}{#1}
}
\newcommand{\ulim}{% low limit
 \underline{\lim}
}
\newcommand{\ssi}{% iff
\iff
}
\newcommand{\ps}[2]{
\expval{#1 | #2}
}
\newcommand{\df}[1]{
\mqty{#1}
}
\newcommand{\n}[1]{
\norm{#1}
}
\newcommand{\sys}[1]{
\left\{\smqty{#1}\right.
}


\newcommand{\eqdef}{\ensuremath{\overset{\text{def}}=}}


\def\Circlearrowright{\ensuremath{%
  \rotatebox[origin=c]{230}{$\circlearrowright$}}}

\newcommand\ct[1]{\text{\rmfamily\upshape #1}}
\newcommand\question[1]{ {\color{red} ...!? \small #1}}
\newcommand\caz[1]{\left\{\begin{array} #1 \end{array}\right.}
\newcommand\const{\text{\rmfamily\upshape const}}
\newcommand\toP{ \overset{\pro}{\to}}
\newcommand\toPP{ \overset{\text{PP}}{\to}}
\newcommand{\oeq}{\mathrel{\text{\textcircled{$=$}}}}





\usepackage{xcolor}
% \usepackage[normalem]{ulem}
\usepackage{lipsum}
\makeatletter
% \newcommand\colorwave[1][blue]{\bgroup \markoverwith{\lower3.5\p@\hbox{\sixly \textcolor{#1}{\char58}}}\ULon}
%\font\sixly=lasy6 % does not re-load if already loaded, so no memory problem.

\newmdtheoremenv[
linewidth= 1pt,linecolor= blue,%
leftmargin=20,rightmargin=20,innertopmargin=0pt, innerrightmargin=40,%
tikzsetting = { draw=lightgray, line width = 0.3pt,dashed,%
dash pattern = on 15pt off 3pt},%
splittopskip=\topskip,skipbelow=\baselineskip,%
skipabove=\baselineskip,ntheorem,roundcorner=0pt,
% backgroundcolor=pagebg,font=\color{orange}\sffamily, fontcolor=white
]{examplebox}{Exemple}[section]



\newcommand\R{\mathbb{R}}
\newcommand\Z{\mathbb{Z}}
\newcommand\N{\mathbb{N}}
\newcommand\E{\mathbb{E}}
\newcommand\F{\mathcal{F}}
\newcommand\cH{\mathcal{H}}
\newcommand\V{\mathbb{V}}
\newcommand\dmo{ ^{-1} }
\newcommand\kapa{\kappa}
\newcommand\im{Im}
\newcommand\hs{\mathcal{H}}





\usepackage{soul}

\makeatletter
\newcommand*{\whiten}[1]{\llap{\textcolor{white}{{\the\SOUL@token}}\hspace{#1pt}}}
\DeclareRobustCommand*\myul{%
    \def\SOUL@everyspace{\underline{\space}\kern\z@}%
    \def\SOUL@everytoken{%
     \setbox0=\hbox{\the\SOUL@token}%
     \ifdim\dp0>\z@
        \raisebox{\dp0}{\underline{\phantom{\the\SOUL@token}}}%
        \whiten{1}\whiten{0}%
        \whiten{-1}\whiten{-2}%
        \llap{\the\SOUL@token}%
     \else
        \underline{\the\SOUL@token}%
     \fi}%
\SOUL@}
\makeatother

\newcommand*{\demp}{\fontfamily{lmtt}\selectfont}

\DeclareTextFontCommand{\textdemp}{\demp}

\begin{document}

\ifcomment
Multiline
comment
\fi
\ifcomment
\myul{Typesetting test}
% \color[rgb]{1,1,1}
$∑_i^n≠ 60º±∞π∆¬≈√j∫h≤≥µ$

$\CR \R\pro\ind\pro\gS\pro
\mqty[a&b\\c&d]$
$\pro\mathbb{P}$
$\dd{x}$

  \[
    \alpha(x)=\left\{
                \begin{array}{ll}
                  x\\
                  \frac{1}{1+e^{-kx}}\\
                  \frac{e^x-e^{-x}}{e^x+e^{-x}}
                \end{array}
              \right.
  \]

  $\expval{x}$
  
  $\chi_\rho(ghg\dmo)=\Tr(\rho_{ghg\dmo})=\Tr(\rho_g\circ\rho_h\circ\rho\dmo_g)=\Tr(\rho_h)\overset{\mbox{\scalebox{0.5}{$\Tr(AB)=\Tr(BA)$}}}{=}\chi_\rho(h)$
  	$\mathop{\oplus}_{\substack{x\in X}}$

$\mat(\rho_g)=(a_{ij}(g))_{\scriptsize \substack{1\leq i\leq d \\ 1\leq j\leq d}}$ et $\mat(\rho'_g)=(a'_{ij}(g))_{\scriptsize \substack{1\leq i'\leq d' \\ 1\leq j'\leq d'}}$



\[\int_a^b{\mathbb{R}^2}g(u, v)\dd{P_{XY}}(u, v)=\iint g(u,v) f_{XY}(u, v)\dd \lambda(u) \dd \lambda(v)\]
$$\lim_{x\to\infty} f(x)$$	
$$\iiiint_V \mu(t,u,v,w) \,dt\,du\,dv\,dw$$
$$\sum_{n=1}^{\infty} 2^{-n} = 1$$	
\begin{definition}
	Si $X$ et $Y$ sont 2 v.a. ou definit la \textsc{Covariance} entre $X$ et $Y$ comme
	$\cov(X,Y)\overset{\text{def}}{=}\E\left[(X-\E(X))(Y-\E(Y))\right]=\E(XY)-\E(X)\E(Y)$.
\end{definition}
\fi
\pagebreak

% \tableofcontents

% insert your code here
%\input{./algebra/main.tex}
%\input{./geometrie-differentielle/main.tex}
%\input{./probabilite/main.tex}
%\input{./analyse-fonctionnelle/main.tex}
% \input{./Analyse-convexe-et-dualite-en-optimisation/main.tex}
%\input{./tikz/main.tex}
%\input{./Theorie-du-distributions/main.tex}
%\input{./optimisation/mine.tex}
 \input{./modelisation/main.tex}

% yves.aubry@univ-tln.fr : algebra

\end{document}

%% !TEX encoding = UTF-8 Unicode
% !TEX TS-program = xelatex

\documentclass[french]{report}

%\usepackage[utf8]{inputenc}
%\usepackage[T1]{fontenc}
\usepackage{babel}


\newif\ifcomment
%\commenttrue # Show comments

\usepackage{physics}
\usepackage{amssymb}


\usepackage{amsthm}
% \usepackage{thmtools}
\usepackage{mathtools}
\usepackage{amsfonts}

\usepackage{color}

\usepackage{tikz}

\usepackage{geometry}
\geometry{a5paper, margin=0.1in, right=1cm}

\usepackage{dsfont}

\usepackage{graphicx}
\graphicspath{ {images/} }

\usepackage{faktor}

\usepackage{IEEEtrantools}
\usepackage{enumerate}   
\usepackage[PostScript=dvips]{"/Users/aware/Documents/Courses/diagrams"}


\newtheorem{theorem}{Théorème}[section]
\renewcommand{\thetheorem}{\arabic{theorem}}
\newtheorem{lemme}{Lemme}[section]
\renewcommand{\thelemme}{\arabic{lemme}}
\newtheorem{proposition}{Proposition}[section]
\renewcommand{\theproposition}{\arabic{proposition}}
\newtheorem{notations}{Notations}[section]
\newtheorem{problem}{Problème}[section]
\newtheorem{corollary}{Corollaire}[theorem]
\renewcommand{\thecorollary}{\arabic{corollary}}
\newtheorem{property}{Propriété}[section]
\newtheorem{objective}{Objectif}[section]

\theoremstyle{definition}
\newtheorem{definition}{Définition}[section]
\renewcommand{\thedefinition}{\arabic{definition}}
\newtheorem{exercise}{Exercice}[chapter]
\renewcommand{\theexercise}{\arabic{exercise}}
\newtheorem{example}{Exemple}[chapter]
\renewcommand{\theexample}{\arabic{example}}
\newtheorem*{solution}{Solution}
\newtheorem*{application}{Application}
\newtheorem*{notation}{Notation}
\newtheorem*{vocabulary}{Vocabulaire}
\newtheorem*{properties}{Propriétés}



\theoremstyle{remark}
\newtheorem*{remark}{Remarque}
\newtheorem*{rappel}{Rappel}


\usepackage{etoolbox}
\AtBeginEnvironment{exercise}{\small}
\AtBeginEnvironment{example}{\small}

\usepackage{cases}
\usepackage[red]{mypack}

\usepackage[framemethod=TikZ]{mdframed}

\definecolor{bg}{rgb}{0.4,0.25,0.95}
\definecolor{pagebg}{rgb}{0,0,0.5}
\surroundwithmdframed[
   topline=false,
   rightline=false,
   bottomline=false,
   leftmargin=\parindent,
   skipabove=8pt,
   skipbelow=8pt,
   linecolor=blue,
   innerbottommargin=10pt,
   % backgroundcolor=bg,font=\color{orange}\sffamily, fontcolor=white
]{definition}

\usepackage{empheq}
\usepackage[most]{tcolorbox}

\newtcbox{\mymath}[1][]{%
    nobeforeafter, math upper, tcbox raise base,
    enhanced, colframe=blue!30!black,
    colback=red!10, boxrule=1pt,
    #1}

\usepackage{unixode}


\DeclareMathOperator{\ord}{ord}
\DeclareMathOperator{\orb}{orb}
\DeclareMathOperator{\stab}{stab}
\DeclareMathOperator{\Stab}{stab}
\DeclareMathOperator{\ppcm}{ppcm}
\DeclareMathOperator{\conj}{Conj}
\DeclareMathOperator{\End}{End}
\DeclareMathOperator{\rot}{rot}
\DeclareMathOperator{\trs}{trace}
\DeclareMathOperator{\Ind}{Ind}
\DeclareMathOperator{\mat}{Mat}
\DeclareMathOperator{\id}{Id}
\DeclareMathOperator{\vect}{vect}
\DeclareMathOperator{\img}{img}
\DeclareMathOperator{\cov}{Cov}
\DeclareMathOperator{\dist}{dist}
\DeclareMathOperator{\irr}{Irr}
\DeclareMathOperator{\image}{Im}
\DeclareMathOperator{\pd}{\partial}
\DeclareMathOperator{\epi}{epi}
\DeclareMathOperator{\Argmin}{Argmin}
\DeclareMathOperator{\dom}{dom}
\DeclareMathOperator{\proj}{proj}
\DeclareMathOperator{\ctg}{ctg}
\DeclareMathOperator{\supp}{supp}
\DeclareMathOperator{\argmin}{argmin}
\DeclareMathOperator{\mult}{mult}
\DeclareMathOperator{\ch}{ch}
\DeclareMathOperator{\sh}{sh}
\DeclareMathOperator{\rang}{rang}
\DeclareMathOperator{\diam}{diam}
\DeclareMathOperator{\Epigraphe}{Epigraphe}




\usepackage{xcolor}
\everymath{\color{blue}}
%\everymath{\color[rgb]{0,1,1}}
%\pagecolor[rgb]{0,0,0.5}


\newcommand*{\pdtest}[3][]{\ensuremath{\frac{\partial^{#1} #2}{\partial #3}}}

\newcommand*{\deffunc}[6][]{\ensuremath{
\begin{array}{rcl}
#2 : #3 &\rightarrow& #4\\
#5 &\mapsto& #6
\end{array}
}}

\newcommand{\eqcolon}{\mathrel{\resizebox{\widthof{$\mathord{=}$}}{\height}{ $\!\!=\!\!\resizebox{1.2\width}{0.8\height}{\raisebox{0.23ex}{$\mathop{:}$}}\!\!$ }}}
\newcommand{\coloneq}{\mathrel{\resizebox{\widthof{$\mathord{=}$}}{\height}{ $\!\!\resizebox{1.2\width}{0.8\height}{\raisebox{0.23ex}{$\mathop{:}$}}\!\!=\!\!$ }}}
\newcommand{\eqcolonl}{\ensuremath{\mathrel{=\!\!\mathop{:}}}}
\newcommand{\coloneql}{\ensuremath{\mathrel{\mathop{:} \!\! =}}}
\newcommand{\vc}[1]{% inline column vector
  \left(\begin{smallmatrix}#1\end{smallmatrix}\right)%
}
\newcommand{\vr}[1]{% inline row vector
  \begin{smallmatrix}(\,#1\,)\end{smallmatrix}%
}
\makeatletter
\newcommand*{\defeq}{\ =\mathrel{\rlap{%
                     \raisebox{0.3ex}{$\m@th\cdot$}}%
                     \raisebox{-0.3ex}{$\m@th\cdot$}}%
                     }
\makeatother

\newcommand{\mathcircle}[1]{% inline row vector
 \overset{\circ}{#1}
}
\newcommand{\ulim}{% low limit
 \underline{\lim}
}
\newcommand{\ssi}{% iff
\iff
}
\newcommand{\ps}[2]{
\expval{#1 | #2}
}
\newcommand{\df}[1]{
\mqty{#1}
}
\newcommand{\n}[1]{
\norm{#1}
}
\newcommand{\sys}[1]{
\left\{\smqty{#1}\right.
}


\newcommand{\eqdef}{\ensuremath{\overset{\text{def}}=}}


\def\Circlearrowright{\ensuremath{%
  \rotatebox[origin=c]{230}{$\circlearrowright$}}}

\newcommand\ct[1]{\text{\rmfamily\upshape #1}}
\newcommand\question[1]{ {\color{red} ...!? \small #1}}
\newcommand\caz[1]{\left\{\begin{array} #1 \end{array}\right.}
\newcommand\const{\text{\rmfamily\upshape const}}
\newcommand\toP{ \overset{\pro}{\to}}
\newcommand\toPP{ \overset{\text{PP}}{\to}}
\newcommand{\oeq}{\mathrel{\text{\textcircled{$=$}}}}





\usepackage{xcolor}
% \usepackage[normalem]{ulem}
\usepackage{lipsum}
\makeatletter
% \newcommand\colorwave[1][blue]{\bgroup \markoverwith{\lower3.5\p@\hbox{\sixly \textcolor{#1}{\char58}}}\ULon}
%\font\sixly=lasy6 % does not re-load if already loaded, so no memory problem.

\newmdtheoremenv[
linewidth= 1pt,linecolor= blue,%
leftmargin=20,rightmargin=20,innertopmargin=0pt, innerrightmargin=40,%
tikzsetting = { draw=lightgray, line width = 0.3pt,dashed,%
dash pattern = on 15pt off 3pt},%
splittopskip=\topskip,skipbelow=\baselineskip,%
skipabove=\baselineskip,ntheorem,roundcorner=0pt,
% backgroundcolor=pagebg,font=\color{orange}\sffamily, fontcolor=white
]{examplebox}{Exemple}[section]



\newcommand\R{\mathbb{R}}
\newcommand\Z{\mathbb{Z}}
\newcommand\N{\mathbb{N}}
\newcommand\E{\mathbb{E}}
\newcommand\F{\mathcal{F}}
\newcommand\cH{\mathcal{H}}
\newcommand\V{\mathbb{V}}
\newcommand\dmo{ ^{-1} }
\newcommand\kapa{\kappa}
\newcommand\im{Im}
\newcommand\hs{\mathcal{H}}





\usepackage{soul}

\makeatletter
\newcommand*{\whiten}[1]{\llap{\textcolor{white}{{\the\SOUL@token}}\hspace{#1pt}}}
\DeclareRobustCommand*\myul{%
    \def\SOUL@everyspace{\underline{\space}\kern\z@}%
    \def\SOUL@everytoken{%
     \setbox0=\hbox{\the\SOUL@token}%
     \ifdim\dp0>\z@
        \raisebox{\dp0}{\underline{\phantom{\the\SOUL@token}}}%
        \whiten{1}\whiten{0}%
        \whiten{-1}\whiten{-2}%
        \llap{\the\SOUL@token}%
     \else
        \underline{\the\SOUL@token}%
     \fi}%
\SOUL@}
\makeatother

\newcommand*{\demp}{\fontfamily{lmtt}\selectfont}

\DeclareTextFontCommand{\textdemp}{\demp}

\begin{document}

\ifcomment
Multiline
comment
\fi
\ifcomment
\myul{Typesetting test}
% \color[rgb]{1,1,1}
$∑_i^n≠ 60º±∞π∆¬≈√j∫h≤≥µ$

$\CR \R\pro\ind\pro\gS\pro
\mqty[a&b\\c&d]$
$\pro\mathbb{P}$
$\dd{x}$

  \[
    \alpha(x)=\left\{
                \begin{array}{ll}
                  x\\
                  \frac{1}{1+e^{-kx}}\\
                  \frac{e^x-e^{-x}}{e^x+e^{-x}}
                \end{array}
              \right.
  \]

  $\expval{x}$
  
  $\chi_\rho(ghg\dmo)=\Tr(\rho_{ghg\dmo})=\Tr(\rho_g\circ\rho_h\circ\rho\dmo_g)=\Tr(\rho_h)\overset{\mbox{\scalebox{0.5}{$\Tr(AB)=\Tr(BA)$}}}{=}\chi_\rho(h)$
  	$\mathop{\oplus}_{\substack{x\in X}}$

$\mat(\rho_g)=(a_{ij}(g))_{\scriptsize \substack{1\leq i\leq d \\ 1\leq j\leq d}}$ et $\mat(\rho'_g)=(a'_{ij}(g))_{\scriptsize \substack{1\leq i'\leq d' \\ 1\leq j'\leq d'}}$



\[\int_a^b{\mathbb{R}^2}g(u, v)\dd{P_{XY}}(u, v)=\iint g(u,v) f_{XY}(u, v)\dd \lambda(u) \dd \lambda(v)\]
$$\lim_{x\to\infty} f(x)$$	
$$\iiiint_V \mu(t,u,v,w) \,dt\,du\,dv\,dw$$
$$\sum_{n=1}^{\infty} 2^{-n} = 1$$	
\begin{definition}
	Si $X$ et $Y$ sont 2 v.a. ou definit la \textsc{Covariance} entre $X$ et $Y$ comme
	$\cov(X,Y)\overset{\text{def}}{=}\E\left[(X-\E(X))(Y-\E(Y))\right]=\E(XY)-\E(X)\E(Y)$.
\end{definition}
\fi
\pagebreak

% \tableofcontents

% insert your code here
%\input{./algebra/main.tex}
%\input{./geometrie-differentielle/main.tex}
%\input{./probabilite/main.tex}
%\input{./analyse-fonctionnelle/main.tex}
% \input{./Analyse-convexe-et-dualite-en-optimisation/main.tex}
%\input{./tikz/main.tex}
%\input{./Theorie-du-distributions/main.tex}
%\input{./optimisation/mine.tex}
 \input{./modelisation/main.tex}

% yves.aubry@univ-tln.fr : algebra

\end{document}

%% !TEX encoding = UTF-8 Unicode
% !TEX TS-program = xelatex

\documentclass[french]{report}

%\usepackage[utf8]{inputenc}
%\usepackage[T1]{fontenc}
\usepackage{babel}


\newif\ifcomment
%\commenttrue # Show comments

\usepackage{physics}
\usepackage{amssymb}


\usepackage{amsthm}
% \usepackage{thmtools}
\usepackage{mathtools}
\usepackage{amsfonts}

\usepackage{color}

\usepackage{tikz}

\usepackage{geometry}
\geometry{a5paper, margin=0.1in, right=1cm}

\usepackage{dsfont}

\usepackage{graphicx}
\graphicspath{ {images/} }

\usepackage{faktor}

\usepackage{IEEEtrantools}
\usepackage{enumerate}   
\usepackage[PostScript=dvips]{"/Users/aware/Documents/Courses/diagrams"}


\newtheorem{theorem}{Théorème}[section]
\renewcommand{\thetheorem}{\arabic{theorem}}
\newtheorem{lemme}{Lemme}[section]
\renewcommand{\thelemme}{\arabic{lemme}}
\newtheorem{proposition}{Proposition}[section]
\renewcommand{\theproposition}{\arabic{proposition}}
\newtheorem{notations}{Notations}[section]
\newtheorem{problem}{Problème}[section]
\newtheorem{corollary}{Corollaire}[theorem]
\renewcommand{\thecorollary}{\arabic{corollary}}
\newtheorem{property}{Propriété}[section]
\newtheorem{objective}{Objectif}[section]

\theoremstyle{definition}
\newtheorem{definition}{Définition}[section]
\renewcommand{\thedefinition}{\arabic{definition}}
\newtheorem{exercise}{Exercice}[chapter]
\renewcommand{\theexercise}{\arabic{exercise}}
\newtheorem{example}{Exemple}[chapter]
\renewcommand{\theexample}{\arabic{example}}
\newtheorem*{solution}{Solution}
\newtheorem*{application}{Application}
\newtheorem*{notation}{Notation}
\newtheorem*{vocabulary}{Vocabulaire}
\newtheorem*{properties}{Propriétés}



\theoremstyle{remark}
\newtheorem*{remark}{Remarque}
\newtheorem*{rappel}{Rappel}


\usepackage{etoolbox}
\AtBeginEnvironment{exercise}{\small}
\AtBeginEnvironment{example}{\small}

\usepackage{cases}
\usepackage[red]{mypack}

\usepackage[framemethod=TikZ]{mdframed}

\definecolor{bg}{rgb}{0.4,0.25,0.95}
\definecolor{pagebg}{rgb}{0,0,0.5}
\surroundwithmdframed[
   topline=false,
   rightline=false,
   bottomline=false,
   leftmargin=\parindent,
   skipabove=8pt,
   skipbelow=8pt,
   linecolor=blue,
   innerbottommargin=10pt,
   % backgroundcolor=bg,font=\color{orange}\sffamily, fontcolor=white
]{definition}

\usepackage{empheq}
\usepackage[most]{tcolorbox}

\newtcbox{\mymath}[1][]{%
    nobeforeafter, math upper, tcbox raise base,
    enhanced, colframe=blue!30!black,
    colback=red!10, boxrule=1pt,
    #1}

\usepackage{unixode}


\DeclareMathOperator{\ord}{ord}
\DeclareMathOperator{\orb}{orb}
\DeclareMathOperator{\stab}{stab}
\DeclareMathOperator{\Stab}{stab}
\DeclareMathOperator{\ppcm}{ppcm}
\DeclareMathOperator{\conj}{Conj}
\DeclareMathOperator{\End}{End}
\DeclareMathOperator{\rot}{rot}
\DeclareMathOperator{\trs}{trace}
\DeclareMathOperator{\Ind}{Ind}
\DeclareMathOperator{\mat}{Mat}
\DeclareMathOperator{\id}{Id}
\DeclareMathOperator{\vect}{vect}
\DeclareMathOperator{\img}{img}
\DeclareMathOperator{\cov}{Cov}
\DeclareMathOperator{\dist}{dist}
\DeclareMathOperator{\irr}{Irr}
\DeclareMathOperator{\image}{Im}
\DeclareMathOperator{\pd}{\partial}
\DeclareMathOperator{\epi}{epi}
\DeclareMathOperator{\Argmin}{Argmin}
\DeclareMathOperator{\dom}{dom}
\DeclareMathOperator{\proj}{proj}
\DeclareMathOperator{\ctg}{ctg}
\DeclareMathOperator{\supp}{supp}
\DeclareMathOperator{\argmin}{argmin}
\DeclareMathOperator{\mult}{mult}
\DeclareMathOperator{\ch}{ch}
\DeclareMathOperator{\sh}{sh}
\DeclareMathOperator{\rang}{rang}
\DeclareMathOperator{\diam}{diam}
\DeclareMathOperator{\Epigraphe}{Epigraphe}




\usepackage{xcolor}
\everymath{\color{blue}}
%\everymath{\color[rgb]{0,1,1}}
%\pagecolor[rgb]{0,0,0.5}


\newcommand*{\pdtest}[3][]{\ensuremath{\frac{\partial^{#1} #2}{\partial #3}}}

\newcommand*{\deffunc}[6][]{\ensuremath{
\begin{array}{rcl}
#2 : #3 &\rightarrow& #4\\
#5 &\mapsto& #6
\end{array}
}}

\newcommand{\eqcolon}{\mathrel{\resizebox{\widthof{$\mathord{=}$}}{\height}{ $\!\!=\!\!\resizebox{1.2\width}{0.8\height}{\raisebox{0.23ex}{$\mathop{:}$}}\!\!$ }}}
\newcommand{\coloneq}{\mathrel{\resizebox{\widthof{$\mathord{=}$}}{\height}{ $\!\!\resizebox{1.2\width}{0.8\height}{\raisebox{0.23ex}{$\mathop{:}$}}\!\!=\!\!$ }}}
\newcommand{\eqcolonl}{\ensuremath{\mathrel{=\!\!\mathop{:}}}}
\newcommand{\coloneql}{\ensuremath{\mathrel{\mathop{:} \!\! =}}}
\newcommand{\vc}[1]{% inline column vector
  \left(\begin{smallmatrix}#1\end{smallmatrix}\right)%
}
\newcommand{\vr}[1]{% inline row vector
  \begin{smallmatrix}(\,#1\,)\end{smallmatrix}%
}
\makeatletter
\newcommand*{\defeq}{\ =\mathrel{\rlap{%
                     \raisebox{0.3ex}{$\m@th\cdot$}}%
                     \raisebox{-0.3ex}{$\m@th\cdot$}}%
                     }
\makeatother

\newcommand{\mathcircle}[1]{% inline row vector
 \overset{\circ}{#1}
}
\newcommand{\ulim}{% low limit
 \underline{\lim}
}
\newcommand{\ssi}{% iff
\iff
}
\newcommand{\ps}[2]{
\expval{#1 | #2}
}
\newcommand{\df}[1]{
\mqty{#1}
}
\newcommand{\n}[1]{
\norm{#1}
}
\newcommand{\sys}[1]{
\left\{\smqty{#1}\right.
}


\newcommand{\eqdef}{\ensuremath{\overset{\text{def}}=}}


\def\Circlearrowright{\ensuremath{%
  \rotatebox[origin=c]{230}{$\circlearrowright$}}}

\newcommand\ct[1]{\text{\rmfamily\upshape #1}}
\newcommand\question[1]{ {\color{red} ...!? \small #1}}
\newcommand\caz[1]{\left\{\begin{array} #1 \end{array}\right.}
\newcommand\const{\text{\rmfamily\upshape const}}
\newcommand\toP{ \overset{\pro}{\to}}
\newcommand\toPP{ \overset{\text{PP}}{\to}}
\newcommand{\oeq}{\mathrel{\text{\textcircled{$=$}}}}





\usepackage{xcolor}
% \usepackage[normalem]{ulem}
\usepackage{lipsum}
\makeatletter
% \newcommand\colorwave[1][blue]{\bgroup \markoverwith{\lower3.5\p@\hbox{\sixly \textcolor{#1}{\char58}}}\ULon}
%\font\sixly=lasy6 % does not re-load if already loaded, so no memory problem.

\newmdtheoremenv[
linewidth= 1pt,linecolor= blue,%
leftmargin=20,rightmargin=20,innertopmargin=0pt, innerrightmargin=40,%
tikzsetting = { draw=lightgray, line width = 0.3pt,dashed,%
dash pattern = on 15pt off 3pt},%
splittopskip=\topskip,skipbelow=\baselineskip,%
skipabove=\baselineskip,ntheorem,roundcorner=0pt,
% backgroundcolor=pagebg,font=\color{orange}\sffamily, fontcolor=white
]{examplebox}{Exemple}[section]



\newcommand\R{\mathbb{R}}
\newcommand\Z{\mathbb{Z}}
\newcommand\N{\mathbb{N}}
\newcommand\E{\mathbb{E}}
\newcommand\F{\mathcal{F}}
\newcommand\cH{\mathcal{H}}
\newcommand\V{\mathbb{V}}
\newcommand\dmo{ ^{-1} }
\newcommand\kapa{\kappa}
\newcommand\im{Im}
\newcommand\hs{\mathcal{H}}





\usepackage{soul}

\makeatletter
\newcommand*{\whiten}[1]{\llap{\textcolor{white}{{\the\SOUL@token}}\hspace{#1pt}}}
\DeclareRobustCommand*\myul{%
    \def\SOUL@everyspace{\underline{\space}\kern\z@}%
    \def\SOUL@everytoken{%
     \setbox0=\hbox{\the\SOUL@token}%
     \ifdim\dp0>\z@
        \raisebox{\dp0}{\underline{\phantom{\the\SOUL@token}}}%
        \whiten{1}\whiten{0}%
        \whiten{-1}\whiten{-2}%
        \llap{\the\SOUL@token}%
     \else
        \underline{\the\SOUL@token}%
     \fi}%
\SOUL@}
\makeatother

\newcommand*{\demp}{\fontfamily{lmtt}\selectfont}

\DeclareTextFontCommand{\textdemp}{\demp}

\begin{document}

\ifcomment
Multiline
comment
\fi
\ifcomment
\myul{Typesetting test}
% \color[rgb]{1,1,1}
$∑_i^n≠ 60º±∞π∆¬≈√j∫h≤≥µ$

$\CR \R\pro\ind\pro\gS\pro
\mqty[a&b\\c&d]$
$\pro\mathbb{P}$
$\dd{x}$

  \[
    \alpha(x)=\left\{
                \begin{array}{ll}
                  x\\
                  \frac{1}{1+e^{-kx}}\\
                  \frac{e^x-e^{-x}}{e^x+e^{-x}}
                \end{array}
              \right.
  \]

  $\expval{x}$
  
  $\chi_\rho(ghg\dmo)=\Tr(\rho_{ghg\dmo})=\Tr(\rho_g\circ\rho_h\circ\rho\dmo_g)=\Tr(\rho_h)\overset{\mbox{\scalebox{0.5}{$\Tr(AB)=\Tr(BA)$}}}{=}\chi_\rho(h)$
  	$\mathop{\oplus}_{\substack{x\in X}}$

$\mat(\rho_g)=(a_{ij}(g))_{\scriptsize \substack{1\leq i\leq d \\ 1\leq j\leq d}}$ et $\mat(\rho'_g)=(a'_{ij}(g))_{\scriptsize \substack{1\leq i'\leq d' \\ 1\leq j'\leq d'}}$



\[\int_a^b{\mathbb{R}^2}g(u, v)\dd{P_{XY}}(u, v)=\iint g(u,v) f_{XY}(u, v)\dd \lambda(u) \dd \lambda(v)\]
$$\lim_{x\to\infty} f(x)$$	
$$\iiiint_V \mu(t,u,v,w) \,dt\,du\,dv\,dw$$
$$\sum_{n=1}^{\infty} 2^{-n} = 1$$	
\begin{definition}
	Si $X$ et $Y$ sont 2 v.a. ou definit la \textsc{Covariance} entre $X$ et $Y$ comme
	$\cov(X,Y)\overset{\text{def}}{=}\E\left[(X-\E(X))(Y-\E(Y))\right]=\E(XY)-\E(X)\E(Y)$.
\end{definition}
\fi
\pagebreak

% \tableofcontents

% insert your code here
%\input{./algebra/main.tex}
%\input{./geometrie-differentielle/main.tex}
%\input{./probabilite/main.tex}
%\input{./analyse-fonctionnelle/main.tex}
% \input{./Analyse-convexe-et-dualite-en-optimisation/main.tex}
%\input{./tikz/main.tex}
%\input{./Theorie-du-distributions/main.tex}
%\input{./optimisation/mine.tex}
 \input{./modelisation/main.tex}

% yves.aubry@univ-tln.fr : algebra

\end{document}

%\input{./optimisation/mine.tex}
 % !TEX encoding = UTF-8 Unicode
% !TEX TS-program = xelatex

\documentclass[french]{report}

%\usepackage[utf8]{inputenc}
%\usepackage[T1]{fontenc}
\usepackage{babel}


\newif\ifcomment
%\commenttrue # Show comments

\usepackage{physics}
\usepackage{amssymb}


\usepackage{amsthm}
% \usepackage{thmtools}
\usepackage{mathtools}
\usepackage{amsfonts}

\usepackage{color}

\usepackage{tikz}

\usepackage{geometry}
\geometry{a5paper, margin=0.1in, right=1cm}

\usepackage{dsfont}

\usepackage{graphicx}
\graphicspath{ {images/} }

\usepackage{faktor}

\usepackage{IEEEtrantools}
\usepackage{enumerate}   
\usepackage[PostScript=dvips]{"/Users/aware/Documents/Courses/diagrams"}


\newtheorem{theorem}{Théorème}[section]
\renewcommand{\thetheorem}{\arabic{theorem}}
\newtheorem{lemme}{Lemme}[section]
\renewcommand{\thelemme}{\arabic{lemme}}
\newtheorem{proposition}{Proposition}[section]
\renewcommand{\theproposition}{\arabic{proposition}}
\newtheorem{notations}{Notations}[section]
\newtheorem{problem}{Problème}[section]
\newtheorem{corollary}{Corollaire}[theorem]
\renewcommand{\thecorollary}{\arabic{corollary}}
\newtheorem{property}{Propriété}[section]
\newtheorem{objective}{Objectif}[section]

\theoremstyle{definition}
\newtheorem{definition}{Définition}[section]
\renewcommand{\thedefinition}{\arabic{definition}}
\newtheorem{exercise}{Exercice}[chapter]
\renewcommand{\theexercise}{\arabic{exercise}}
\newtheorem{example}{Exemple}[chapter]
\renewcommand{\theexample}{\arabic{example}}
\newtheorem*{solution}{Solution}
\newtheorem*{application}{Application}
\newtheorem*{notation}{Notation}
\newtheorem*{vocabulary}{Vocabulaire}
\newtheorem*{properties}{Propriétés}



\theoremstyle{remark}
\newtheorem*{remark}{Remarque}
\newtheorem*{rappel}{Rappel}


\usepackage{etoolbox}
\AtBeginEnvironment{exercise}{\small}
\AtBeginEnvironment{example}{\small}

\usepackage{cases}
\usepackage[red]{mypack}

\usepackage[framemethod=TikZ]{mdframed}

\definecolor{bg}{rgb}{0.4,0.25,0.95}
\definecolor{pagebg}{rgb}{0,0,0.5}
\surroundwithmdframed[
   topline=false,
   rightline=false,
   bottomline=false,
   leftmargin=\parindent,
   skipabove=8pt,
   skipbelow=8pt,
   linecolor=blue,
   innerbottommargin=10pt,
   % backgroundcolor=bg,font=\color{orange}\sffamily, fontcolor=white
]{definition}

\usepackage{empheq}
\usepackage[most]{tcolorbox}

\newtcbox{\mymath}[1][]{%
    nobeforeafter, math upper, tcbox raise base,
    enhanced, colframe=blue!30!black,
    colback=red!10, boxrule=1pt,
    #1}

\usepackage{unixode}


\DeclareMathOperator{\ord}{ord}
\DeclareMathOperator{\orb}{orb}
\DeclareMathOperator{\stab}{stab}
\DeclareMathOperator{\Stab}{stab}
\DeclareMathOperator{\ppcm}{ppcm}
\DeclareMathOperator{\conj}{Conj}
\DeclareMathOperator{\End}{End}
\DeclareMathOperator{\rot}{rot}
\DeclareMathOperator{\trs}{trace}
\DeclareMathOperator{\Ind}{Ind}
\DeclareMathOperator{\mat}{Mat}
\DeclareMathOperator{\id}{Id}
\DeclareMathOperator{\vect}{vect}
\DeclareMathOperator{\img}{img}
\DeclareMathOperator{\cov}{Cov}
\DeclareMathOperator{\dist}{dist}
\DeclareMathOperator{\irr}{Irr}
\DeclareMathOperator{\image}{Im}
\DeclareMathOperator{\pd}{\partial}
\DeclareMathOperator{\epi}{epi}
\DeclareMathOperator{\Argmin}{Argmin}
\DeclareMathOperator{\dom}{dom}
\DeclareMathOperator{\proj}{proj}
\DeclareMathOperator{\ctg}{ctg}
\DeclareMathOperator{\supp}{supp}
\DeclareMathOperator{\argmin}{argmin}
\DeclareMathOperator{\mult}{mult}
\DeclareMathOperator{\ch}{ch}
\DeclareMathOperator{\sh}{sh}
\DeclareMathOperator{\rang}{rang}
\DeclareMathOperator{\diam}{diam}
\DeclareMathOperator{\Epigraphe}{Epigraphe}




\usepackage{xcolor}
\everymath{\color{blue}}
%\everymath{\color[rgb]{0,1,1}}
%\pagecolor[rgb]{0,0,0.5}


\newcommand*{\pdtest}[3][]{\ensuremath{\frac{\partial^{#1} #2}{\partial #3}}}

\newcommand*{\deffunc}[6][]{\ensuremath{
\begin{array}{rcl}
#2 : #3 &\rightarrow& #4\\
#5 &\mapsto& #6
\end{array}
}}

\newcommand{\eqcolon}{\mathrel{\resizebox{\widthof{$\mathord{=}$}}{\height}{ $\!\!=\!\!\resizebox{1.2\width}{0.8\height}{\raisebox{0.23ex}{$\mathop{:}$}}\!\!$ }}}
\newcommand{\coloneq}{\mathrel{\resizebox{\widthof{$\mathord{=}$}}{\height}{ $\!\!\resizebox{1.2\width}{0.8\height}{\raisebox{0.23ex}{$\mathop{:}$}}\!\!=\!\!$ }}}
\newcommand{\eqcolonl}{\ensuremath{\mathrel{=\!\!\mathop{:}}}}
\newcommand{\coloneql}{\ensuremath{\mathrel{\mathop{:} \!\! =}}}
\newcommand{\vc}[1]{% inline column vector
  \left(\begin{smallmatrix}#1\end{smallmatrix}\right)%
}
\newcommand{\vr}[1]{% inline row vector
  \begin{smallmatrix}(\,#1\,)\end{smallmatrix}%
}
\makeatletter
\newcommand*{\defeq}{\ =\mathrel{\rlap{%
                     \raisebox{0.3ex}{$\m@th\cdot$}}%
                     \raisebox{-0.3ex}{$\m@th\cdot$}}%
                     }
\makeatother

\newcommand{\mathcircle}[1]{% inline row vector
 \overset{\circ}{#1}
}
\newcommand{\ulim}{% low limit
 \underline{\lim}
}
\newcommand{\ssi}{% iff
\iff
}
\newcommand{\ps}[2]{
\expval{#1 | #2}
}
\newcommand{\df}[1]{
\mqty{#1}
}
\newcommand{\n}[1]{
\norm{#1}
}
\newcommand{\sys}[1]{
\left\{\smqty{#1}\right.
}


\newcommand{\eqdef}{\ensuremath{\overset{\text{def}}=}}


\def\Circlearrowright{\ensuremath{%
  \rotatebox[origin=c]{230}{$\circlearrowright$}}}

\newcommand\ct[1]{\text{\rmfamily\upshape #1}}
\newcommand\question[1]{ {\color{red} ...!? \small #1}}
\newcommand\caz[1]{\left\{\begin{array} #1 \end{array}\right.}
\newcommand\const{\text{\rmfamily\upshape const}}
\newcommand\toP{ \overset{\pro}{\to}}
\newcommand\toPP{ \overset{\text{PP}}{\to}}
\newcommand{\oeq}{\mathrel{\text{\textcircled{$=$}}}}





\usepackage{xcolor}
% \usepackage[normalem]{ulem}
\usepackage{lipsum}
\makeatletter
% \newcommand\colorwave[1][blue]{\bgroup \markoverwith{\lower3.5\p@\hbox{\sixly \textcolor{#1}{\char58}}}\ULon}
%\font\sixly=lasy6 % does not re-load if already loaded, so no memory problem.

\newmdtheoremenv[
linewidth= 1pt,linecolor= blue,%
leftmargin=20,rightmargin=20,innertopmargin=0pt, innerrightmargin=40,%
tikzsetting = { draw=lightgray, line width = 0.3pt,dashed,%
dash pattern = on 15pt off 3pt},%
splittopskip=\topskip,skipbelow=\baselineskip,%
skipabove=\baselineskip,ntheorem,roundcorner=0pt,
% backgroundcolor=pagebg,font=\color{orange}\sffamily, fontcolor=white
]{examplebox}{Exemple}[section]



\newcommand\R{\mathbb{R}}
\newcommand\Z{\mathbb{Z}}
\newcommand\N{\mathbb{N}}
\newcommand\E{\mathbb{E}}
\newcommand\F{\mathcal{F}}
\newcommand\cH{\mathcal{H}}
\newcommand\V{\mathbb{V}}
\newcommand\dmo{ ^{-1} }
\newcommand\kapa{\kappa}
\newcommand\im{Im}
\newcommand\hs{\mathcal{H}}





\usepackage{soul}

\makeatletter
\newcommand*{\whiten}[1]{\llap{\textcolor{white}{{\the\SOUL@token}}\hspace{#1pt}}}
\DeclareRobustCommand*\myul{%
    \def\SOUL@everyspace{\underline{\space}\kern\z@}%
    \def\SOUL@everytoken{%
     \setbox0=\hbox{\the\SOUL@token}%
     \ifdim\dp0>\z@
        \raisebox{\dp0}{\underline{\phantom{\the\SOUL@token}}}%
        \whiten{1}\whiten{0}%
        \whiten{-1}\whiten{-2}%
        \llap{\the\SOUL@token}%
     \else
        \underline{\the\SOUL@token}%
     \fi}%
\SOUL@}
\makeatother

\newcommand*{\demp}{\fontfamily{lmtt}\selectfont}

\DeclareTextFontCommand{\textdemp}{\demp}

\begin{document}

\ifcomment
Multiline
comment
\fi
\ifcomment
\myul{Typesetting test}
% \color[rgb]{1,1,1}
$∑_i^n≠ 60º±∞π∆¬≈√j∫h≤≥µ$

$\CR \R\pro\ind\pro\gS\pro
\mqty[a&b\\c&d]$
$\pro\mathbb{P}$
$\dd{x}$

  \[
    \alpha(x)=\left\{
                \begin{array}{ll}
                  x\\
                  \frac{1}{1+e^{-kx}}\\
                  \frac{e^x-e^{-x}}{e^x+e^{-x}}
                \end{array}
              \right.
  \]

  $\expval{x}$
  
  $\chi_\rho(ghg\dmo)=\Tr(\rho_{ghg\dmo})=\Tr(\rho_g\circ\rho_h\circ\rho\dmo_g)=\Tr(\rho_h)\overset{\mbox{\scalebox{0.5}{$\Tr(AB)=\Tr(BA)$}}}{=}\chi_\rho(h)$
  	$\mathop{\oplus}_{\substack{x\in X}}$

$\mat(\rho_g)=(a_{ij}(g))_{\scriptsize \substack{1\leq i\leq d \\ 1\leq j\leq d}}$ et $\mat(\rho'_g)=(a'_{ij}(g))_{\scriptsize \substack{1\leq i'\leq d' \\ 1\leq j'\leq d'}}$



\[\int_a^b{\mathbb{R}^2}g(u, v)\dd{P_{XY}}(u, v)=\iint g(u,v) f_{XY}(u, v)\dd \lambda(u) \dd \lambda(v)\]
$$\lim_{x\to\infty} f(x)$$	
$$\iiiint_V \mu(t,u,v,w) \,dt\,du\,dv\,dw$$
$$\sum_{n=1}^{\infty} 2^{-n} = 1$$	
\begin{definition}
	Si $X$ et $Y$ sont 2 v.a. ou definit la \textsc{Covariance} entre $X$ et $Y$ comme
	$\cov(X,Y)\overset{\text{def}}{=}\E\left[(X-\E(X))(Y-\E(Y))\right]=\E(XY)-\E(X)\E(Y)$.
\end{definition}
\fi
\pagebreak

% \tableofcontents

% insert your code here
%\input{./algebra/main.tex}
%\input{./geometrie-differentielle/main.tex}
%\input{./probabilite/main.tex}
%\input{./analyse-fonctionnelle/main.tex}
% \input{./Analyse-convexe-et-dualite-en-optimisation/main.tex}
%\input{./tikz/main.tex}
%\input{./Theorie-du-distributions/main.tex}
%\input{./optimisation/mine.tex}
 \input{./modelisation/main.tex}

% yves.aubry@univ-tln.fr : algebra

\end{document}


% yves.aubry@univ-tln.fr : algebra

\end{document}

%% !TEX encoding = UTF-8 Unicode
% !TEX TS-program = xelatex

\documentclass[french]{report}

%\usepackage[utf8]{inputenc}
%\usepackage[T1]{fontenc}
\usepackage{babel}


\newif\ifcomment
%\commenttrue # Show comments

\usepackage{physics}
\usepackage{amssymb}


\usepackage{amsthm}
% \usepackage{thmtools}
\usepackage{mathtools}
\usepackage{amsfonts}

\usepackage{color}

\usepackage{tikz}

\usepackage{geometry}
\geometry{a5paper, margin=0.1in, right=1cm}

\usepackage{dsfont}

\usepackage{graphicx}
\graphicspath{ {images/} }

\usepackage{faktor}

\usepackage{IEEEtrantools}
\usepackage{enumerate}   
\usepackage[PostScript=dvips]{"/Users/aware/Documents/Courses/diagrams"}


\newtheorem{theorem}{Théorème}[section]
\renewcommand{\thetheorem}{\arabic{theorem}}
\newtheorem{lemme}{Lemme}[section]
\renewcommand{\thelemme}{\arabic{lemme}}
\newtheorem{proposition}{Proposition}[section]
\renewcommand{\theproposition}{\arabic{proposition}}
\newtheorem{notations}{Notations}[section]
\newtheorem{problem}{Problème}[section]
\newtheorem{corollary}{Corollaire}[theorem]
\renewcommand{\thecorollary}{\arabic{corollary}}
\newtheorem{property}{Propriété}[section]
\newtheorem{objective}{Objectif}[section]

\theoremstyle{definition}
\newtheorem{definition}{Définition}[section]
\renewcommand{\thedefinition}{\arabic{definition}}
\newtheorem{exercise}{Exercice}[chapter]
\renewcommand{\theexercise}{\arabic{exercise}}
\newtheorem{example}{Exemple}[chapter]
\renewcommand{\theexample}{\arabic{example}}
\newtheorem*{solution}{Solution}
\newtheorem*{application}{Application}
\newtheorem*{notation}{Notation}
\newtheorem*{vocabulary}{Vocabulaire}
\newtheorem*{properties}{Propriétés}



\theoremstyle{remark}
\newtheorem*{remark}{Remarque}
\newtheorem*{rappel}{Rappel}


\usepackage{etoolbox}
\AtBeginEnvironment{exercise}{\small}
\AtBeginEnvironment{example}{\small}

\usepackage{cases}
\usepackage[red]{mypack}

\usepackage[framemethod=TikZ]{mdframed}

\definecolor{bg}{rgb}{0.4,0.25,0.95}
\definecolor{pagebg}{rgb}{0,0,0.5}
\surroundwithmdframed[
   topline=false,
   rightline=false,
   bottomline=false,
   leftmargin=\parindent,
   skipabove=8pt,
   skipbelow=8pt,
   linecolor=blue,
   innerbottommargin=10pt,
   % backgroundcolor=bg,font=\color{orange}\sffamily, fontcolor=white
]{definition}

\usepackage{empheq}
\usepackage[most]{tcolorbox}

\newtcbox{\mymath}[1][]{%
    nobeforeafter, math upper, tcbox raise base,
    enhanced, colframe=blue!30!black,
    colback=red!10, boxrule=1pt,
    #1}

\usepackage{unixode}


\DeclareMathOperator{\ord}{ord}
\DeclareMathOperator{\orb}{orb}
\DeclareMathOperator{\stab}{stab}
\DeclareMathOperator{\Stab}{stab}
\DeclareMathOperator{\ppcm}{ppcm}
\DeclareMathOperator{\conj}{Conj}
\DeclareMathOperator{\End}{End}
\DeclareMathOperator{\rot}{rot}
\DeclareMathOperator{\trs}{trace}
\DeclareMathOperator{\Ind}{Ind}
\DeclareMathOperator{\mat}{Mat}
\DeclareMathOperator{\id}{Id}
\DeclareMathOperator{\vect}{vect}
\DeclareMathOperator{\img}{img}
\DeclareMathOperator{\cov}{Cov}
\DeclareMathOperator{\dist}{dist}
\DeclareMathOperator{\irr}{Irr}
\DeclareMathOperator{\image}{Im}
\DeclareMathOperator{\pd}{\partial}
\DeclareMathOperator{\epi}{epi}
\DeclareMathOperator{\Argmin}{Argmin}
\DeclareMathOperator{\dom}{dom}
\DeclareMathOperator{\proj}{proj}
\DeclareMathOperator{\ctg}{ctg}
\DeclareMathOperator{\supp}{supp}
\DeclareMathOperator{\argmin}{argmin}
\DeclareMathOperator{\mult}{mult}
\DeclareMathOperator{\ch}{ch}
\DeclareMathOperator{\sh}{sh}
\DeclareMathOperator{\rang}{rang}
\DeclareMathOperator{\diam}{diam}
\DeclareMathOperator{\Epigraphe}{Epigraphe}




\usepackage{xcolor}
\everymath{\color{blue}}
%\everymath{\color[rgb]{0,1,1}}
%\pagecolor[rgb]{0,0,0.5}


\newcommand*{\pdtest}[3][]{\ensuremath{\frac{\partial^{#1} #2}{\partial #3}}}

\newcommand*{\deffunc}[6][]{\ensuremath{
\begin{array}{rcl}
#2 : #3 &\rightarrow& #4\\
#5 &\mapsto& #6
\end{array}
}}

\newcommand{\eqcolon}{\mathrel{\resizebox{\widthof{$\mathord{=}$}}{\height}{ $\!\!=\!\!\resizebox{1.2\width}{0.8\height}{\raisebox{0.23ex}{$\mathop{:}$}}\!\!$ }}}
\newcommand{\coloneq}{\mathrel{\resizebox{\widthof{$\mathord{=}$}}{\height}{ $\!\!\resizebox{1.2\width}{0.8\height}{\raisebox{0.23ex}{$\mathop{:}$}}\!\!=\!\!$ }}}
\newcommand{\eqcolonl}{\ensuremath{\mathrel{=\!\!\mathop{:}}}}
\newcommand{\coloneql}{\ensuremath{\mathrel{\mathop{:} \!\! =}}}
\newcommand{\vc}[1]{% inline column vector
  \left(\begin{smallmatrix}#1\end{smallmatrix}\right)%
}
\newcommand{\vr}[1]{% inline row vector
  \begin{smallmatrix}(\,#1\,)\end{smallmatrix}%
}
\makeatletter
\newcommand*{\defeq}{\ =\mathrel{\rlap{%
                     \raisebox{0.3ex}{$\m@th\cdot$}}%
                     \raisebox{-0.3ex}{$\m@th\cdot$}}%
                     }
\makeatother

\newcommand{\mathcircle}[1]{% inline row vector
 \overset{\circ}{#1}
}
\newcommand{\ulim}{% low limit
 \underline{\lim}
}
\newcommand{\ssi}{% iff
\iff
}
\newcommand{\ps}[2]{
\expval{#1 | #2}
}
\newcommand{\df}[1]{
\mqty{#1}
}
\newcommand{\n}[1]{
\norm{#1}
}
\newcommand{\sys}[1]{
\left\{\smqty{#1}\right.
}


\newcommand{\eqdef}{\ensuremath{\overset{\text{def}}=}}


\def\Circlearrowright{\ensuremath{%
  \rotatebox[origin=c]{230}{$\circlearrowright$}}}

\newcommand\ct[1]{\text{\rmfamily\upshape #1}}
\newcommand\question[1]{ {\color{red} ...!? \small #1}}
\newcommand\caz[1]{\left\{\begin{array} #1 \end{array}\right.}
\newcommand\const{\text{\rmfamily\upshape const}}
\newcommand\toP{ \overset{\pro}{\to}}
\newcommand\toPP{ \overset{\text{PP}}{\to}}
\newcommand{\oeq}{\mathrel{\text{\textcircled{$=$}}}}





\usepackage{xcolor}
% \usepackage[normalem]{ulem}
\usepackage{lipsum}
\makeatletter
% \newcommand\colorwave[1][blue]{\bgroup \markoverwith{\lower3.5\p@\hbox{\sixly \textcolor{#1}{\char58}}}\ULon}
%\font\sixly=lasy6 % does not re-load if already loaded, so no memory problem.

\newmdtheoremenv[
linewidth= 1pt,linecolor= blue,%
leftmargin=20,rightmargin=20,innertopmargin=0pt, innerrightmargin=40,%
tikzsetting = { draw=lightgray, line width = 0.3pt,dashed,%
dash pattern = on 15pt off 3pt},%
splittopskip=\topskip,skipbelow=\baselineskip,%
skipabove=\baselineskip,ntheorem,roundcorner=0pt,
% backgroundcolor=pagebg,font=\color{orange}\sffamily, fontcolor=white
]{examplebox}{Exemple}[section]



\newcommand\R{\mathbb{R}}
\newcommand\Z{\mathbb{Z}}
\newcommand\N{\mathbb{N}}
\newcommand\E{\mathbb{E}}
\newcommand\F{\mathcal{F}}
\newcommand\cH{\mathcal{H}}
\newcommand\V{\mathbb{V}}
\newcommand\dmo{ ^{-1} }
\newcommand\kapa{\kappa}
\newcommand\im{Im}
\newcommand\hs{\mathcal{H}}





\usepackage{soul}

\makeatletter
\newcommand*{\whiten}[1]{\llap{\textcolor{white}{{\the\SOUL@token}}\hspace{#1pt}}}
\DeclareRobustCommand*\myul{%
    \def\SOUL@everyspace{\underline{\space}\kern\z@}%
    \def\SOUL@everytoken{%
     \setbox0=\hbox{\the\SOUL@token}%
     \ifdim\dp0>\z@
        \raisebox{\dp0}{\underline{\phantom{\the\SOUL@token}}}%
        \whiten{1}\whiten{0}%
        \whiten{-1}\whiten{-2}%
        \llap{\the\SOUL@token}%
     \else
        \underline{\the\SOUL@token}%
     \fi}%
\SOUL@}
\makeatother

\newcommand*{\demp}{\fontfamily{lmtt}\selectfont}

\DeclareTextFontCommand{\textdemp}{\demp}

\begin{document}

\ifcomment
Multiline
comment
\fi
\ifcomment
\myul{Typesetting test}
% \color[rgb]{1,1,1}
$∑_i^n≠ 60º±∞π∆¬≈√j∫h≤≥µ$

$\CR \R\pro\ind\pro\gS\pro
\mqty[a&b\\c&d]$
$\pro\mathbb{P}$
$\dd{x}$

  \[
    \alpha(x)=\left\{
                \begin{array}{ll}
                  x\\
                  \frac{1}{1+e^{-kx}}\\
                  \frac{e^x-e^{-x}}{e^x+e^{-x}}
                \end{array}
              \right.
  \]

  $\expval{x}$
  
  $\chi_\rho(ghg\dmo)=\Tr(\rho_{ghg\dmo})=\Tr(\rho_g\circ\rho_h\circ\rho\dmo_g)=\Tr(\rho_h)\overset{\mbox{\scalebox{0.5}{$\Tr(AB)=\Tr(BA)$}}}{=}\chi_\rho(h)$
  	$\mathop{\oplus}_{\substack{x\in X}}$

$\mat(\rho_g)=(a_{ij}(g))_{\scriptsize \substack{1\leq i\leq d \\ 1\leq j\leq d}}$ et $\mat(\rho'_g)=(a'_{ij}(g))_{\scriptsize \substack{1\leq i'\leq d' \\ 1\leq j'\leq d'}}$



\[\int_a^b{\mathbb{R}^2}g(u, v)\dd{P_{XY}}(u, v)=\iint g(u,v) f_{XY}(u, v)\dd \lambda(u) \dd \lambda(v)\]
$$\lim_{x\to\infty} f(x)$$	
$$\iiiint_V \mu(t,u,v,w) \,dt\,du\,dv\,dw$$
$$\sum_{n=1}^{\infty} 2^{-n} = 1$$	
\begin{definition}
	Si $X$ et $Y$ sont 2 v.a. ou definit la \textsc{Covariance} entre $X$ et $Y$ comme
	$\cov(X,Y)\overset{\text{def}}{=}\E\left[(X-\E(X))(Y-\E(Y))\right]=\E(XY)-\E(X)\E(Y)$.
\end{definition}
\fi
\pagebreak

% \tableofcontents

% insert your code here
%% !TEX encoding = UTF-8 Unicode
% !TEX TS-program = xelatex

\documentclass[french]{report}

%\usepackage[utf8]{inputenc}
%\usepackage[T1]{fontenc}
\usepackage{babel}


\newif\ifcomment
%\commenttrue # Show comments

\usepackage{physics}
\usepackage{amssymb}


\usepackage{amsthm}
% \usepackage{thmtools}
\usepackage{mathtools}
\usepackage{amsfonts}

\usepackage{color}

\usepackage{tikz}

\usepackage{geometry}
\geometry{a5paper, margin=0.1in, right=1cm}

\usepackage{dsfont}

\usepackage{graphicx}
\graphicspath{ {images/} }

\usepackage{faktor}

\usepackage{IEEEtrantools}
\usepackage{enumerate}   
\usepackage[PostScript=dvips]{"/Users/aware/Documents/Courses/diagrams"}


\newtheorem{theorem}{Théorème}[section]
\renewcommand{\thetheorem}{\arabic{theorem}}
\newtheorem{lemme}{Lemme}[section]
\renewcommand{\thelemme}{\arabic{lemme}}
\newtheorem{proposition}{Proposition}[section]
\renewcommand{\theproposition}{\arabic{proposition}}
\newtheorem{notations}{Notations}[section]
\newtheorem{problem}{Problème}[section]
\newtheorem{corollary}{Corollaire}[theorem]
\renewcommand{\thecorollary}{\arabic{corollary}}
\newtheorem{property}{Propriété}[section]
\newtheorem{objective}{Objectif}[section]

\theoremstyle{definition}
\newtheorem{definition}{Définition}[section]
\renewcommand{\thedefinition}{\arabic{definition}}
\newtheorem{exercise}{Exercice}[chapter]
\renewcommand{\theexercise}{\arabic{exercise}}
\newtheorem{example}{Exemple}[chapter]
\renewcommand{\theexample}{\arabic{example}}
\newtheorem*{solution}{Solution}
\newtheorem*{application}{Application}
\newtheorem*{notation}{Notation}
\newtheorem*{vocabulary}{Vocabulaire}
\newtheorem*{properties}{Propriétés}



\theoremstyle{remark}
\newtheorem*{remark}{Remarque}
\newtheorem*{rappel}{Rappel}


\usepackage{etoolbox}
\AtBeginEnvironment{exercise}{\small}
\AtBeginEnvironment{example}{\small}

\usepackage{cases}
\usepackage[red]{mypack}

\usepackage[framemethod=TikZ]{mdframed}

\definecolor{bg}{rgb}{0.4,0.25,0.95}
\definecolor{pagebg}{rgb}{0,0,0.5}
\surroundwithmdframed[
   topline=false,
   rightline=false,
   bottomline=false,
   leftmargin=\parindent,
   skipabove=8pt,
   skipbelow=8pt,
   linecolor=blue,
   innerbottommargin=10pt,
   % backgroundcolor=bg,font=\color{orange}\sffamily, fontcolor=white
]{definition}

\usepackage{empheq}
\usepackage[most]{tcolorbox}

\newtcbox{\mymath}[1][]{%
    nobeforeafter, math upper, tcbox raise base,
    enhanced, colframe=blue!30!black,
    colback=red!10, boxrule=1pt,
    #1}

\usepackage{unixode}


\DeclareMathOperator{\ord}{ord}
\DeclareMathOperator{\orb}{orb}
\DeclareMathOperator{\stab}{stab}
\DeclareMathOperator{\Stab}{stab}
\DeclareMathOperator{\ppcm}{ppcm}
\DeclareMathOperator{\conj}{Conj}
\DeclareMathOperator{\End}{End}
\DeclareMathOperator{\rot}{rot}
\DeclareMathOperator{\trs}{trace}
\DeclareMathOperator{\Ind}{Ind}
\DeclareMathOperator{\mat}{Mat}
\DeclareMathOperator{\id}{Id}
\DeclareMathOperator{\vect}{vect}
\DeclareMathOperator{\img}{img}
\DeclareMathOperator{\cov}{Cov}
\DeclareMathOperator{\dist}{dist}
\DeclareMathOperator{\irr}{Irr}
\DeclareMathOperator{\image}{Im}
\DeclareMathOperator{\pd}{\partial}
\DeclareMathOperator{\epi}{epi}
\DeclareMathOperator{\Argmin}{Argmin}
\DeclareMathOperator{\dom}{dom}
\DeclareMathOperator{\proj}{proj}
\DeclareMathOperator{\ctg}{ctg}
\DeclareMathOperator{\supp}{supp}
\DeclareMathOperator{\argmin}{argmin}
\DeclareMathOperator{\mult}{mult}
\DeclareMathOperator{\ch}{ch}
\DeclareMathOperator{\sh}{sh}
\DeclareMathOperator{\rang}{rang}
\DeclareMathOperator{\diam}{diam}
\DeclareMathOperator{\Epigraphe}{Epigraphe}




\usepackage{xcolor}
\everymath{\color{blue}}
%\everymath{\color[rgb]{0,1,1}}
%\pagecolor[rgb]{0,0,0.5}


\newcommand*{\pdtest}[3][]{\ensuremath{\frac{\partial^{#1} #2}{\partial #3}}}

\newcommand*{\deffunc}[6][]{\ensuremath{
\begin{array}{rcl}
#2 : #3 &\rightarrow& #4\\
#5 &\mapsto& #6
\end{array}
}}

\newcommand{\eqcolon}{\mathrel{\resizebox{\widthof{$\mathord{=}$}}{\height}{ $\!\!=\!\!\resizebox{1.2\width}{0.8\height}{\raisebox{0.23ex}{$\mathop{:}$}}\!\!$ }}}
\newcommand{\coloneq}{\mathrel{\resizebox{\widthof{$\mathord{=}$}}{\height}{ $\!\!\resizebox{1.2\width}{0.8\height}{\raisebox{0.23ex}{$\mathop{:}$}}\!\!=\!\!$ }}}
\newcommand{\eqcolonl}{\ensuremath{\mathrel{=\!\!\mathop{:}}}}
\newcommand{\coloneql}{\ensuremath{\mathrel{\mathop{:} \!\! =}}}
\newcommand{\vc}[1]{% inline column vector
  \left(\begin{smallmatrix}#1\end{smallmatrix}\right)%
}
\newcommand{\vr}[1]{% inline row vector
  \begin{smallmatrix}(\,#1\,)\end{smallmatrix}%
}
\makeatletter
\newcommand*{\defeq}{\ =\mathrel{\rlap{%
                     \raisebox{0.3ex}{$\m@th\cdot$}}%
                     \raisebox{-0.3ex}{$\m@th\cdot$}}%
                     }
\makeatother

\newcommand{\mathcircle}[1]{% inline row vector
 \overset{\circ}{#1}
}
\newcommand{\ulim}{% low limit
 \underline{\lim}
}
\newcommand{\ssi}{% iff
\iff
}
\newcommand{\ps}[2]{
\expval{#1 | #2}
}
\newcommand{\df}[1]{
\mqty{#1}
}
\newcommand{\n}[1]{
\norm{#1}
}
\newcommand{\sys}[1]{
\left\{\smqty{#1}\right.
}


\newcommand{\eqdef}{\ensuremath{\overset{\text{def}}=}}


\def\Circlearrowright{\ensuremath{%
  \rotatebox[origin=c]{230}{$\circlearrowright$}}}

\newcommand\ct[1]{\text{\rmfamily\upshape #1}}
\newcommand\question[1]{ {\color{red} ...!? \small #1}}
\newcommand\caz[1]{\left\{\begin{array} #1 \end{array}\right.}
\newcommand\const{\text{\rmfamily\upshape const}}
\newcommand\toP{ \overset{\pro}{\to}}
\newcommand\toPP{ \overset{\text{PP}}{\to}}
\newcommand{\oeq}{\mathrel{\text{\textcircled{$=$}}}}





\usepackage{xcolor}
% \usepackage[normalem]{ulem}
\usepackage{lipsum}
\makeatletter
% \newcommand\colorwave[1][blue]{\bgroup \markoverwith{\lower3.5\p@\hbox{\sixly \textcolor{#1}{\char58}}}\ULon}
%\font\sixly=lasy6 % does not re-load if already loaded, so no memory problem.

\newmdtheoremenv[
linewidth= 1pt,linecolor= blue,%
leftmargin=20,rightmargin=20,innertopmargin=0pt, innerrightmargin=40,%
tikzsetting = { draw=lightgray, line width = 0.3pt,dashed,%
dash pattern = on 15pt off 3pt},%
splittopskip=\topskip,skipbelow=\baselineskip,%
skipabove=\baselineskip,ntheorem,roundcorner=0pt,
% backgroundcolor=pagebg,font=\color{orange}\sffamily, fontcolor=white
]{examplebox}{Exemple}[section]



\newcommand\R{\mathbb{R}}
\newcommand\Z{\mathbb{Z}}
\newcommand\N{\mathbb{N}}
\newcommand\E{\mathbb{E}}
\newcommand\F{\mathcal{F}}
\newcommand\cH{\mathcal{H}}
\newcommand\V{\mathbb{V}}
\newcommand\dmo{ ^{-1} }
\newcommand\kapa{\kappa}
\newcommand\im{Im}
\newcommand\hs{\mathcal{H}}





\usepackage{soul}

\makeatletter
\newcommand*{\whiten}[1]{\llap{\textcolor{white}{{\the\SOUL@token}}\hspace{#1pt}}}
\DeclareRobustCommand*\myul{%
    \def\SOUL@everyspace{\underline{\space}\kern\z@}%
    \def\SOUL@everytoken{%
     \setbox0=\hbox{\the\SOUL@token}%
     \ifdim\dp0>\z@
        \raisebox{\dp0}{\underline{\phantom{\the\SOUL@token}}}%
        \whiten{1}\whiten{0}%
        \whiten{-1}\whiten{-2}%
        \llap{\the\SOUL@token}%
     \else
        \underline{\the\SOUL@token}%
     \fi}%
\SOUL@}
\makeatother

\newcommand*{\demp}{\fontfamily{lmtt}\selectfont}

\DeclareTextFontCommand{\textdemp}{\demp}

\begin{document}

\ifcomment
Multiline
comment
\fi
\ifcomment
\myul{Typesetting test}
% \color[rgb]{1,1,1}
$∑_i^n≠ 60º±∞π∆¬≈√j∫h≤≥µ$

$\CR \R\pro\ind\pro\gS\pro
\mqty[a&b\\c&d]$
$\pro\mathbb{P}$
$\dd{x}$

  \[
    \alpha(x)=\left\{
                \begin{array}{ll}
                  x\\
                  \frac{1}{1+e^{-kx}}\\
                  \frac{e^x-e^{-x}}{e^x+e^{-x}}
                \end{array}
              \right.
  \]

  $\expval{x}$
  
  $\chi_\rho(ghg\dmo)=\Tr(\rho_{ghg\dmo})=\Tr(\rho_g\circ\rho_h\circ\rho\dmo_g)=\Tr(\rho_h)\overset{\mbox{\scalebox{0.5}{$\Tr(AB)=\Tr(BA)$}}}{=}\chi_\rho(h)$
  	$\mathop{\oplus}_{\substack{x\in X}}$

$\mat(\rho_g)=(a_{ij}(g))_{\scriptsize \substack{1\leq i\leq d \\ 1\leq j\leq d}}$ et $\mat(\rho'_g)=(a'_{ij}(g))_{\scriptsize \substack{1\leq i'\leq d' \\ 1\leq j'\leq d'}}$



\[\int_a^b{\mathbb{R}^2}g(u, v)\dd{P_{XY}}(u, v)=\iint g(u,v) f_{XY}(u, v)\dd \lambda(u) \dd \lambda(v)\]
$$\lim_{x\to\infty} f(x)$$	
$$\iiiint_V \mu(t,u,v,w) \,dt\,du\,dv\,dw$$
$$\sum_{n=1}^{\infty} 2^{-n} = 1$$	
\begin{definition}
	Si $X$ et $Y$ sont 2 v.a. ou definit la \textsc{Covariance} entre $X$ et $Y$ comme
	$\cov(X,Y)\overset{\text{def}}{=}\E\left[(X-\E(X))(Y-\E(Y))\right]=\E(XY)-\E(X)\E(Y)$.
\end{definition}
\fi
\pagebreak

% \tableofcontents

% insert your code here
%\input{./algebra/main.tex}
%\input{./geometrie-differentielle/main.tex}
%\input{./probabilite/main.tex}
%\input{./analyse-fonctionnelle/main.tex}
% \input{./Analyse-convexe-et-dualite-en-optimisation/main.tex}
%\input{./tikz/main.tex}
%\input{./Theorie-du-distributions/main.tex}
%\input{./optimisation/mine.tex}
 \input{./modelisation/main.tex}

% yves.aubry@univ-tln.fr : algebra

\end{document}

%% !TEX encoding = UTF-8 Unicode
% !TEX TS-program = xelatex

\documentclass[french]{report}

%\usepackage[utf8]{inputenc}
%\usepackage[T1]{fontenc}
\usepackage{babel}


\newif\ifcomment
%\commenttrue # Show comments

\usepackage{physics}
\usepackage{amssymb}


\usepackage{amsthm}
% \usepackage{thmtools}
\usepackage{mathtools}
\usepackage{amsfonts}

\usepackage{color}

\usepackage{tikz}

\usepackage{geometry}
\geometry{a5paper, margin=0.1in, right=1cm}

\usepackage{dsfont}

\usepackage{graphicx}
\graphicspath{ {images/} }

\usepackage{faktor}

\usepackage{IEEEtrantools}
\usepackage{enumerate}   
\usepackage[PostScript=dvips]{"/Users/aware/Documents/Courses/diagrams"}


\newtheorem{theorem}{Théorème}[section]
\renewcommand{\thetheorem}{\arabic{theorem}}
\newtheorem{lemme}{Lemme}[section]
\renewcommand{\thelemme}{\arabic{lemme}}
\newtheorem{proposition}{Proposition}[section]
\renewcommand{\theproposition}{\arabic{proposition}}
\newtheorem{notations}{Notations}[section]
\newtheorem{problem}{Problème}[section]
\newtheorem{corollary}{Corollaire}[theorem]
\renewcommand{\thecorollary}{\arabic{corollary}}
\newtheorem{property}{Propriété}[section]
\newtheorem{objective}{Objectif}[section]

\theoremstyle{definition}
\newtheorem{definition}{Définition}[section]
\renewcommand{\thedefinition}{\arabic{definition}}
\newtheorem{exercise}{Exercice}[chapter]
\renewcommand{\theexercise}{\arabic{exercise}}
\newtheorem{example}{Exemple}[chapter]
\renewcommand{\theexample}{\arabic{example}}
\newtheorem*{solution}{Solution}
\newtheorem*{application}{Application}
\newtheorem*{notation}{Notation}
\newtheorem*{vocabulary}{Vocabulaire}
\newtheorem*{properties}{Propriétés}



\theoremstyle{remark}
\newtheorem*{remark}{Remarque}
\newtheorem*{rappel}{Rappel}


\usepackage{etoolbox}
\AtBeginEnvironment{exercise}{\small}
\AtBeginEnvironment{example}{\small}

\usepackage{cases}
\usepackage[red]{mypack}

\usepackage[framemethod=TikZ]{mdframed}

\definecolor{bg}{rgb}{0.4,0.25,0.95}
\definecolor{pagebg}{rgb}{0,0,0.5}
\surroundwithmdframed[
   topline=false,
   rightline=false,
   bottomline=false,
   leftmargin=\parindent,
   skipabove=8pt,
   skipbelow=8pt,
   linecolor=blue,
   innerbottommargin=10pt,
   % backgroundcolor=bg,font=\color{orange}\sffamily, fontcolor=white
]{definition}

\usepackage{empheq}
\usepackage[most]{tcolorbox}

\newtcbox{\mymath}[1][]{%
    nobeforeafter, math upper, tcbox raise base,
    enhanced, colframe=blue!30!black,
    colback=red!10, boxrule=1pt,
    #1}

\usepackage{unixode}


\DeclareMathOperator{\ord}{ord}
\DeclareMathOperator{\orb}{orb}
\DeclareMathOperator{\stab}{stab}
\DeclareMathOperator{\Stab}{stab}
\DeclareMathOperator{\ppcm}{ppcm}
\DeclareMathOperator{\conj}{Conj}
\DeclareMathOperator{\End}{End}
\DeclareMathOperator{\rot}{rot}
\DeclareMathOperator{\trs}{trace}
\DeclareMathOperator{\Ind}{Ind}
\DeclareMathOperator{\mat}{Mat}
\DeclareMathOperator{\id}{Id}
\DeclareMathOperator{\vect}{vect}
\DeclareMathOperator{\img}{img}
\DeclareMathOperator{\cov}{Cov}
\DeclareMathOperator{\dist}{dist}
\DeclareMathOperator{\irr}{Irr}
\DeclareMathOperator{\image}{Im}
\DeclareMathOperator{\pd}{\partial}
\DeclareMathOperator{\epi}{epi}
\DeclareMathOperator{\Argmin}{Argmin}
\DeclareMathOperator{\dom}{dom}
\DeclareMathOperator{\proj}{proj}
\DeclareMathOperator{\ctg}{ctg}
\DeclareMathOperator{\supp}{supp}
\DeclareMathOperator{\argmin}{argmin}
\DeclareMathOperator{\mult}{mult}
\DeclareMathOperator{\ch}{ch}
\DeclareMathOperator{\sh}{sh}
\DeclareMathOperator{\rang}{rang}
\DeclareMathOperator{\diam}{diam}
\DeclareMathOperator{\Epigraphe}{Epigraphe}




\usepackage{xcolor}
\everymath{\color{blue}}
%\everymath{\color[rgb]{0,1,1}}
%\pagecolor[rgb]{0,0,0.5}


\newcommand*{\pdtest}[3][]{\ensuremath{\frac{\partial^{#1} #2}{\partial #3}}}

\newcommand*{\deffunc}[6][]{\ensuremath{
\begin{array}{rcl}
#2 : #3 &\rightarrow& #4\\
#5 &\mapsto& #6
\end{array}
}}

\newcommand{\eqcolon}{\mathrel{\resizebox{\widthof{$\mathord{=}$}}{\height}{ $\!\!=\!\!\resizebox{1.2\width}{0.8\height}{\raisebox{0.23ex}{$\mathop{:}$}}\!\!$ }}}
\newcommand{\coloneq}{\mathrel{\resizebox{\widthof{$\mathord{=}$}}{\height}{ $\!\!\resizebox{1.2\width}{0.8\height}{\raisebox{0.23ex}{$\mathop{:}$}}\!\!=\!\!$ }}}
\newcommand{\eqcolonl}{\ensuremath{\mathrel{=\!\!\mathop{:}}}}
\newcommand{\coloneql}{\ensuremath{\mathrel{\mathop{:} \!\! =}}}
\newcommand{\vc}[1]{% inline column vector
  \left(\begin{smallmatrix}#1\end{smallmatrix}\right)%
}
\newcommand{\vr}[1]{% inline row vector
  \begin{smallmatrix}(\,#1\,)\end{smallmatrix}%
}
\makeatletter
\newcommand*{\defeq}{\ =\mathrel{\rlap{%
                     \raisebox{0.3ex}{$\m@th\cdot$}}%
                     \raisebox{-0.3ex}{$\m@th\cdot$}}%
                     }
\makeatother

\newcommand{\mathcircle}[1]{% inline row vector
 \overset{\circ}{#1}
}
\newcommand{\ulim}{% low limit
 \underline{\lim}
}
\newcommand{\ssi}{% iff
\iff
}
\newcommand{\ps}[2]{
\expval{#1 | #2}
}
\newcommand{\df}[1]{
\mqty{#1}
}
\newcommand{\n}[1]{
\norm{#1}
}
\newcommand{\sys}[1]{
\left\{\smqty{#1}\right.
}


\newcommand{\eqdef}{\ensuremath{\overset{\text{def}}=}}


\def\Circlearrowright{\ensuremath{%
  \rotatebox[origin=c]{230}{$\circlearrowright$}}}

\newcommand\ct[1]{\text{\rmfamily\upshape #1}}
\newcommand\question[1]{ {\color{red} ...!? \small #1}}
\newcommand\caz[1]{\left\{\begin{array} #1 \end{array}\right.}
\newcommand\const{\text{\rmfamily\upshape const}}
\newcommand\toP{ \overset{\pro}{\to}}
\newcommand\toPP{ \overset{\text{PP}}{\to}}
\newcommand{\oeq}{\mathrel{\text{\textcircled{$=$}}}}





\usepackage{xcolor}
% \usepackage[normalem]{ulem}
\usepackage{lipsum}
\makeatletter
% \newcommand\colorwave[1][blue]{\bgroup \markoverwith{\lower3.5\p@\hbox{\sixly \textcolor{#1}{\char58}}}\ULon}
%\font\sixly=lasy6 % does not re-load if already loaded, so no memory problem.

\newmdtheoremenv[
linewidth= 1pt,linecolor= blue,%
leftmargin=20,rightmargin=20,innertopmargin=0pt, innerrightmargin=40,%
tikzsetting = { draw=lightgray, line width = 0.3pt,dashed,%
dash pattern = on 15pt off 3pt},%
splittopskip=\topskip,skipbelow=\baselineskip,%
skipabove=\baselineskip,ntheorem,roundcorner=0pt,
% backgroundcolor=pagebg,font=\color{orange}\sffamily, fontcolor=white
]{examplebox}{Exemple}[section]



\newcommand\R{\mathbb{R}}
\newcommand\Z{\mathbb{Z}}
\newcommand\N{\mathbb{N}}
\newcommand\E{\mathbb{E}}
\newcommand\F{\mathcal{F}}
\newcommand\cH{\mathcal{H}}
\newcommand\V{\mathbb{V}}
\newcommand\dmo{ ^{-1} }
\newcommand\kapa{\kappa}
\newcommand\im{Im}
\newcommand\hs{\mathcal{H}}





\usepackage{soul}

\makeatletter
\newcommand*{\whiten}[1]{\llap{\textcolor{white}{{\the\SOUL@token}}\hspace{#1pt}}}
\DeclareRobustCommand*\myul{%
    \def\SOUL@everyspace{\underline{\space}\kern\z@}%
    \def\SOUL@everytoken{%
     \setbox0=\hbox{\the\SOUL@token}%
     \ifdim\dp0>\z@
        \raisebox{\dp0}{\underline{\phantom{\the\SOUL@token}}}%
        \whiten{1}\whiten{0}%
        \whiten{-1}\whiten{-2}%
        \llap{\the\SOUL@token}%
     \else
        \underline{\the\SOUL@token}%
     \fi}%
\SOUL@}
\makeatother

\newcommand*{\demp}{\fontfamily{lmtt}\selectfont}

\DeclareTextFontCommand{\textdemp}{\demp}

\begin{document}

\ifcomment
Multiline
comment
\fi
\ifcomment
\myul{Typesetting test}
% \color[rgb]{1,1,1}
$∑_i^n≠ 60º±∞π∆¬≈√j∫h≤≥µ$

$\CR \R\pro\ind\pro\gS\pro
\mqty[a&b\\c&d]$
$\pro\mathbb{P}$
$\dd{x}$

  \[
    \alpha(x)=\left\{
                \begin{array}{ll}
                  x\\
                  \frac{1}{1+e^{-kx}}\\
                  \frac{e^x-e^{-x}}{e^x+e^{-x}}
                \end{array}
              \right.
  \]

  $\expval{x}$
  
  $\chi_\rho(ghg\dmo)=\Tr(\rho_{ghg\dmo})=\Tr(\rho_g\circ\rho_h\circ\rho\dmo_g)=\Tr(\rho_h)\overset{\mbox{\scalebox{0.5}{$\Tr(AB)=\Tr(BA)$}}}{=}\chi_\rho(h)$
  	$\mathop{\oplus}_{\substack{x\in X}}$

$\mat(\rho_g)=(a_{ij}(g))_{\scriptsize \substack{1\leq i\leq d \\ 1\leq j\leq d}}$ et $\mat(\rho'_g)=(a'_{ij}(g))_{\scriptsize \substack{1\leq i'\leq d' \\ 1\leq j'\leq d'}}$



\[\int_a^b{\mathbb{R}^2}g(u, v)\dd{P_{XY}}(u, v)=\iint g(u,v) f_{XY}(u, v)\dd \lambda(u) \dd \lambda(v)\]
$$\lim_{x\to\infty} f(x)$$	
$$\iiiint_V \mu(t,u,v,w) \,dt\,du\,dv\,dw$$
$$\sum_{n=1}^{\infty} 2^{-n} = 1$$	
\begin{definition}
	Si $X$ et $Y$ sont 2 v.a. ou definit la \textsc{Covariance} entre $X$ et $Y$ comme
	$\cov(X,Y)\overset{\text{def}}{=}\E\left[(X-\E(X))(Y-\E(Y))\right]=\E(XY)-\E(X)\E(Y)$.
\end{definition}
\fi
\pagebreak

% \tableofcontents

% insert your code here
%\input{./algebra/main.tex}
%\input{./geometrie-differentielle/main.tex}
%\input{./probabilite/main.tex}
%\input{./analyse-fonctionnelle/main.tex}
% \input{./Analyse-convexe-et-dualite-en-optimisation/main.tex}
%\input{./tikz/main.tex}
%\input{./Theorie-du-distributions/main.tex}
%\input{./optimisation/mine.tex}
 \input{./modelisation/main.tex}

% yves.aubry@univ-tln.fr : algebra

\end{document}

%% !TEX encoding = UTF-8 Unicode
% !TEX TS-program = xelatex

\documentclass[french]{report}

%\usepackage[utf8]{inputenc}
%\usepackage[T1]{fontenc}
\usepackage{babel}


\newif\ifcomment
%\commenttrue # Show comments

\usepackage{physics}
\usepackage{amssymb}


\usepackage{amsthm}
% \usepackage{thmtools}
\usepackage{mathtools}
\usepackage{amsfonts}

\usepackage{color}

\usepackage{tikz}

\usepackage{geometry}
\geometry{a5paper, margin=0.1in, right=1cm}

\usepackage{dsfont}

\usepackage{graphicx}
\graphicspath{ {images/} }

\usepackage{faktor}

\usepackage{IEEEtrantools}
\usepackage{enumerate}   
\usepackage[PostScript=dvips]{"/Users/aware/Documents/Courses/diagrams"}


\newtheorem{theorem}{Théorème}[section]
\renewcommand{\thetheorem}{\arabic{theorem}}
\newtheorem{lemme}{Lemme}[section]
\renewcommand{\thelemme}{\arabic{lemme}}
\newtheorem{proposition}{Proposition}[section]
\renewcommand{\theproposition}{\arabic{proposition}}
\newtheorem{notations}{Notations}[section]
\newtheorem{problem}{Problème}[section]
\newtheorem{corollary}{Corollaire}[theorem]
\renewcommand{\thecorollary}{\arabic{corollary}}
\newtheorem{property}{Propriété}[section]
\newtheorem{objective}{Objectif}[section]

\theoremstyle{definition}
\newtheorem{definition}{Définition}[section]
\renewcommand{\thedefinition}{\arabic{definition}}
\newtheorem{exercise}{Exercice}[chapter]
\renewcommand{\theexercise}{\arabic{exercise}}
\newtheorem{example}{Exemple}[chapter]
\renewcommand{\theexample}{\arabic{example}}
\newtheorem*{solution}{Solution}
\newtheorem*{application}{Application}
\newtheorem*{notation}{Notation}
\newtheorem*{vocabulary}{Vocabulaire}
\newtheorem*{properties}{Propriétés}



\theoremstyle{remark}
\newtheorem*{remark}{Remarque}
\newtheorem*{rappel}{Rappel}


\usepackage{etoolbox}
\AtBeginEnvironment{exercise}{\small}
\AtBeginEnvironment{example}{\small}

\usepackage{cases}
\usepackage[red]{mypack}

\usepackage[framemethod=TikZ]{mdframed}

\definecolor{bg}{rgb}{0.4,0.25,0.95}
\definecolor{pagebg}{rgb}{0,0,0.5}
\surroundwithmdframed[
   topline=false,
   rightline=false,
   bottomline=false,
   leftmargin=\parindent,
   skipabove=8pt,
   skipbelow=8pt,
   linecolor=blue,
   innerbottommargin=10pt,
   % backgroundcolor=bg,font=\color{orange}\sffamily, fontcolor=white
]{definition}

\usepackage{empheq}
\usepackage[most]{tcolorbox}

\newtcbox{\mymath}[1][]{%
    nobeforeafter, math upper, tcbox raise base,
    enhanced, colframe=blue!30!black,
    colback=red!10, boxrule=1pt,
    #1}

\usepackage{unixode}


\DeclareMathOperator{\ord}{ord}
\DeclareMathOperator{\orb}{orb}
\DeclareMathOperator{\stab}{stab}
\DeclareMathOperator{\Stab}{stab}
\DeclareMathOperator{\ppcm}{ppcm}
\DeclareMathOperator{\conj}{Conj}
\DeclareMathOperator{\End}{End}
\DeclareMathOperator{\rot}{rot}
\DeclareMathOperator{\trs}{trace}
\DeclareMathOperator{\Ind}{Ind}
\DeclareMathOperator{\mat}{Mat}
\DeclareMathOperator{\id}{Id}
\DeclareMathOperator{\vect}{vect}
\DeclareMathOperator{\img}{img}
\DeclareMathOperator{\cov}{Cov}
\DeclareMathOperator{\dist}{dist}
\DeclareMathOperator{\irr}{Irr}
\DeclareMathOperator{\image}{Im}
\DeclareMathOperator{\pd}{\partial}
\DeclareMathOperator{\epi}{epi}
\DeclareMathOperator{\Argmin}{Argmin}
\DeclareMathOperator{\dom}{dom}
\DeclareMathOperator{\proj}{proj}
\DeclareMathOperator{\ctg}{ctg}
\DeclareMathOperator{\supp}{supp}
\DeclareMathOperator{\argmin}{argmin}
\DeclareMathOperator{\mult}{mult}
\DeclareMathOperator{\ch}{ch}
\DeclareMathOperator{\sh}{sh}
\DeclareMathOperator{\rang}{rang}
\DeclareMathOperator{\diam}{diam}
\DeclareMathOperator{\Epigraphe}{Epigraphe}




\usepackage{xcolor}
\everymath{\color{blue}}
%\everymath{\color[rgb]{0,1,1}}
%\pagecolor[rgb]{0,0,0.5}


\newcommand*{\pdtest}[3][]{\ensuremath{\frac{\partial^{#1} #2}{\partial #3}}}

\newcommand*{\deffunc}[6][]{\ensuremath{
\begin{array}{rcl}
#2 : #3 &\rightarrow& #4\\
#5 &\mapsto& #6
\end{array}
}}

\newcommand{\eqcolon}{\mathrel{\resizebox{\widthof{$\mathord{=}$}}{\height}{ $\!\!=\!\!\resizebox{1.2\width}{0.8\height}{\raisebox{0.23ex}{$\mathop{:}$}}\!\!$ }}}
\newcommand{\coloneq}{\mathrel{\resizebox{\widthof{$\mathord{=}$}}{\height}{ $\!\!\resizebox{1.2\width}{0.8\height}{\raisebox{0.23ex}{$\mathop{:}$}}\!\!=\!\!$ }}}
\newcommand{\eqcolonl}{\ensuremath{\mathrel{=\!\!\mathop{:}}}}
\newcommand{\coloneql}{\ensuremath{\mathrel{\mathop{:} \!\! =}}}
\newcommand{\vc}[1]{% inline column vector
  \left(\begin{smallmatrix}#1\end{smallmatrix}\right)%
}
\newcommand{\vr}[1]{% inline row vector
  \begin{smallmatrix}(\,#1\,)\end{smallmatrix}%
}
\makeatletter
\newcommand*{\defeq}{\ =\mathrel{\rlap{%
                     \raisebox{0.3ex}{$\m@th\cdot$}}%
                     \raisebox{-0.3ex}{$\m@th\cdot$}}%
                     }
\makeatother

\newcommand{\mathcircle}[1]{% inline row vector
 \overset{\circ}{#1}
}
\newcommand{\ulim}{% low limit
 \underline{\lim}
}
\newcommand{\ssi}{% iff
\iff
}
\newcommand{\ps}[2]{
\expval{#1 | #2}
}
\newcommand{\df}[1]{
\mqty{#1}
}
\newcommand{\n}[1]{
\norm{#1}
}
\newcommand{\sys}[1]{
\left\{\smqty{#1}\right.
}


\newcommand{\eqdef}{\ensuremath{\overset{\text{def}}=}}


\def\Circlearrowright{\ensuremath{%
  \rotatebox[origin=c]{230}{$\circlearrowright$}}}

\newcommand\ct[1]{\text{\rmfamily\upshape #1}}
\newcommand\question[1]{ {\color{red} ...!? \small #1}}
\newcommand\caz[1]{\left\{\begin{array} #1 \end{array}\right.}
\newcommand\const{\text{\rmfamily\upshape const}}
\newcommand\toP{ \overset{\pro}{\to}}
\newcommand\toPP{ \overset{\text{PP}}{\to}}
\newcommand{\oeq}{\mathrel{\text{\textcircled{$=$}}}}





\usepackage{xcolor}
% \usepackage[normalem]{ulem}
\usepackage{lipsum}
\makeatletter
% \newcommand\colorwave[1][blue]{\bgroup \markoverwith{\lower3.5\p@\hbox{\sixly \textcolor{#1}{\char58}}}\ULon}
%\font\sixly=lasy6 % does not re-load if already loaded, so no memory problem.

\newmdtheoremenv[
linewidth= 1pt,linecolor= blue,%
leftmargin=20,rightmargin=20,innertopmargin=0pt, innerrightmargin=40,%
tikzsetting = { draw=lightgray, line width = 0.3pt,dashed,%
dash pattern = on 15pt off 3pt},%
splittopskip=\topskip,skipbelow=\baselineskip,%
skipabove=\baselineskip,ntheorem,roundcorner=0pt,
% backgroundcolor=pagebg,font=\color{orange}\sffamily, fontcolor=white
]{examplebox}{Exemple}[section]



\newcommand\R{\mathbb{R}}
\newcommand\Z{\mathbb{Z}}
\newcommand\N{\mathbb{N}}
\newcommand\E{\mathbb{E}}
\newcommand\F{\mathcal{F}}
\newcommand\cH{\mathcal{H}}
\newcommand\V{\mathbb{V}}
\newcommand\dmo{ ^{-1} }
\newcommand\kapa{\kappa}
\newcommand\im{Im}
\newcommand\hs{\mathcal{H}}





\usepackage{soul}

\makeatletter
\newcommand*{\whiten}[1]{\llap{\textcolor{white}{{\the\SOUL@token}}\hspace{#1pt}}}
\DeclareRobustCommand*\myul{%
    \def\SOUL@everyspace{\underline{\space}\kern\z@}%
    \def\SOUL@everytoken{%
     \setbox0=\hbox{\the\SOUL@token}%
     \ifdim\dp0>\z@
        \raisebox{\dp0}{\underline{\phantom{\the\SOUL@token}}}%
        \whiten{1}\whiten{0}%
        \whiten{-1}\whiten{-2}%
        \llap{\the\SOUL@token}%
     \else
        \underline{\the\SOUL@token}%
     \fi}%
\SOUL@}
\makeatother

\newcommand*{\demp}{\fontfamily{lmtt}\selectfont}

\DeclareTextFontCommand{\textdemp}{\demp}

\begin{document}

\ifcomment
Multiline
comment
\fi
\ifcomment
\myul{Typesetting test}
% \color[rgb]{1,1,1}
$∑_i^n≠ 60º±∞π∆¬≈√j∫h≤≥µ$

$\CR \R\pro\ind\pro\gS\pro
\mqty[a&b\\c&d]$
$\pro\mathbb{P}$
$\dd{x}$

  \[
    \alpha(x)=\left\{
                \begin{array}{ll}
                  x\\
                  \frac{1}{1+e^{-kx}}\\
                  \frac{e^x-e^{-x}}{e^x+e^{-x}}
                \end{array}
              \right.
  \]

  $\expval{x}$
  
  $\chi_\rho(ghg\dmo)=\Tr(\rho_{ghg\dmo})=\Tr(\rho_g\circ\rho_h\circ\rho\dmo_g)=\Tr(\rho_h)\overset{\mbox{\scalebox{0.5}{$\Tr(AB)=\Tr(BA)$}}}{=}\chi_\rho(h)$
  	$\mathop{\oplus}_{\substack{x\in X}}$

$\mat(\rho_g)=(a_{ij}(g))_{\scriptsize \substack{1\leq i\leq d \\ 1\leq j\leq d}}$ et $\mat(\rho'_g)=(a'_{ij}(g))_{\scriptsize \substack{1\leq i'\leq d' \\ 1\leq j'\leq d'}}$



\[\int_a^b{\mathbb{R}^2}g(u, v)\dd{P_{XY}}(u, v)=\iint g(u,v) f_{XY}(u, v)\dd \lambda(u) \dd \lambda(v)\]
$$\lim_{x\to\infty} f(x)$$	
$$\iiiint_V \mu(t,u,v,w) \,dt\,du\,dv\,dw$$
$$\sum_{n=1}^{\infty} 2^{-n} = 1$$	
\begin{definition}
	Si $X$ et $Y$ sont 2 v.a. ou definit la \textsc{Covariance} entre $X$ et $Y$ comme
	$\cov(X,Y)\overset{\text{def}}{=}\E\left[(X-\E(X))(Y-\E(Y))\right]=\E(XY)-\E(X)\E(Y)$.
\end{definition}
\fi
\pagebreak

% \tableofcontents

% insert your code here
%\input{./algebra/main.tex}
%\input{./geometrie-differentielle/main.tex}
%\input{./probabilite/main.tex}
%\input{./analyse-fonctionnelle/main.tex}
% \input{./Analyse-convexe-et-dualite-en-optimisation/main.tex}
%\input{./tikz/main.tex}
%\input{./Theorie-du-distributions/main.tex}
%\input{./optimisation/mine.tex}
 \input{./modelisation/main.tex}

% yves.aubry@univ-tln.fr : algebra

\end{document}

%% !TEX encoding = UTF-8 Unicode
% !TEX TS-program = xelatex

\documentclass[french]{report}

%\usepackage[utf8]{inputenc}
%\usepackage[T1]{fontenc}
\usepackage{babel}


\newif\ifcomment
%\commenttrue # Show comments

\usepackage{physics}
\usepackage{amssymb}


\usepackage{amsthm}
% \usepackage{thmtools}
\usepackage{mathtools}
\usepackage{amsfonts}

\usepackage{color}

\usepackage{tikz}

\usepackage{geometry}
\geometry{a5paper, margin=0.1in, right=1cm}

\usepackage{dsfont}

\usepackage{graphicx}
\graphicspath{ {images/} }

\usepackage{faktor}

\usepackage{IEEEtrantools}
\usepackage{enumerate}   
\usepackage[PostScript=dvips]{"/Users/aware/Documents/Courses/diagrams"}


\newtheorem{theorem}{Théorème}[section]
\renewcommand{\thetheorem}{\arabic{theorem}}
\newtheorem{lemme}{Lemme}[section]
\renewcommand{\thelemme}{\arabic{lemme}}
\newtheorem{proposition}{Proposition}[section]
\renewcommand{\theproposition}{\arabic{proposition}}
\newtheorem{notations}{Notations}[section]
\newtheorem{problem}{Problème}[section]
\newtheorem{corollary}{Corollaire}[theorem]
\renewcommand{\thecorollary}{\arabic{corollary}}
\newtheorem{property}{Propriété}[section]
\newtheorem{objective}{Objectif}[section]

\theoremstyle{definition}
\newtheorem{definition}{Définition}[section]
\renewcommand{\thedefinition}{\arabic{definition}}
\newtheorem{exercise}{Exercice}[chapter]
\renewcommand{\theexercise}{\arabic{exercise}}
\newtheorem{example}{Exemple}[chapter]
\renewcommand{\theexample}{\arabic{example}}
\newtheorem*{solution}{Solution}
\newtheorem*{application}{Application}
\newtheorem*{notation}{Notation}
\newtheorem*{vocabulary}{Vocabulaire}
\newtheorem*{properties}{Propriétés}



\theoremstyle{remark}
\newtheorem*{remark}{Remarque}
\newtheorem*{rappel}{Rappel}


\usepackage{etoolbox}
\AtBeginEnvironment{exercise}{\small}
\AtBeginEnvironment{example}{\small}

\usepackage{cases}
\usepackage[red]{mypack}

\usepackage[framemethod=TikZ]{mdframed}

\definecolor{bg}{rgb}{0.4,0.25,0.95}
\definecolor{pagebg}{rgb}{0,0,0.5}
\surroundwithmdframed[
   topline=false,
   rightline=false,
   bottomline=false,
   leftmargin=\parindent,
   skipabove=8pt,
   skipbelow=8pt,
   linecolor=blue,
   innerbottommargin=10pt,
   % backgroundcolor=bg,font=\color{orange}\sffamily, fontcolor=white
]{definition}

\usepackage{empheq}
\usepackage[most]{tcolorbox}

\newtcbox{\mymath}[1][]{%
    nobeforeafter, math upper, tcbox raise base,
    enhanced, colframe=blue!30!black,
    colback=red!10, boxrule=1pt,
    #1}

\usepackage{unixode}


\DeclareMathOperator{\ord}{ord}
\DeclareMathOperator{\orb}{orb}
\DeclareMathOperator{\stab}{stab}
\DeclareMathOperator{\Stab}{stab}
\DeclareMathOperator{\ppcm}{ppcm}
\DeclareMathOperator{\conj}{Conj}
\DeclareMathOperator{\End}{End}
\DeclareMathOperator{\rot}{rot}
\DeclareMathOperator{\trs}{trace}
\DeclareMathOperator{\Ind}{Ind}
\DeclareMathOperator{\mat}{Mat}
\DeclareMathOperator{\id}{Id}
\DeclareMathOperator{\vect}{vect}
\DeclareMathOperator{\img}{img}
\DeclareMathOperator{\cov}{Cov}
\DeclareMathOperator{\dist}{dist}
\DeclareMathOperator{\irr}{Irr}
\DeclareMathOperator{\image}{Im}
\DeclareMathOperator{\pd}{\partial}
\DeclareMathOperator{\epi}{epi}
\DeclareMathOperator{\Argmin}{Argmin}
\DeclareMathOperator{\dom}{dom}
\DeclareMathOperator{\proj}{proj}
\DeclareMathOperator{\ctg}{ctg}
\DeclareMathOperator{\supp}{supp}
\DeclareMathOperator{\argmin}{argmin}
\DeclareMathOperator{\mult}{mult}
\DeclareMathOperator{\ch}{ch}
\DeclareMathOperator{\sh}{sh}
\DeclareMathOperator{\rang}{rang}
\DeclareMathOperator{\diam}{diam}
\DeclareMathOperator{\Epigraphe}{Epigraphe}




\usepackage{xcolor}
\everymath{\color{blue}}
%\everymath{\color[rgb]{0,1,1}}
%\pagecolor[rgb]{0,0,0.5}


\newcommand*{\pdtest}[3][]{\ensuremath{\frac{\partial^{#1} #2}{\partial #3}}}

\newcommand*{\deffunc}[6][]{\ensuremath{
\begin{array}{rcl}
#2 : #3 &\rightarrow& #4\\
#5 &\mapsto& #6
\end{array}
}}

\newcommand{\eqcolon}{\mathrel{\resizebox{\widthof{$\mathord{=}$}}{\height}{ $\!\!=\!\!\resizebox{1.2\width}{0.8\height}{\raisebox{0.23ex}{$\mathop{:}$}}\!\!$ }}}
\newcommand{\coloneq}{\mathrel{\resizebox{\widthof{$\mathord{=}$}}{\height}{ $\!\!\resizebox{1.2\width}{0.8\height}{\raisebox{0.23ex}{$\mathop{:}$}}\!\!=\!\!$ }}}
\newcommand{\eqcolonl}{\ensuremath{\mathrel{=\!\!\mathop{:}}}}
\newcommand{\coloneql}{\ensuremath{\mathrel{\mathop{:} \!\! =}}}
\newcommand{\vc}[1]{% inline column vector
  \left(\begin{smallmatrix}#1\end{smallmatrix}\right)%
}
\newcommand{\vr}[1]{% inline row vector
  \begin{smallmatrix}(\,#1\,)\end{smallmatrix}%
}
\makeatletter
\newcommand*{\defeq}{\ =\mathrel{\rlap{%
                     \raisebox{0.3ex}{$\m@th\cdot$}}%
                     \raisebox{-0.3ex}{$\m@th\cdot$}}%
                     }
\makeatother

\newcommand{\mathcircle}[1]{% inline row vector
 \overset{\circ}{#1}
}
\newcommand{\ulim}{% low limit
 \underline{\lim}
}
\newcommand{\ssi}{% iff
\iff
}
\newcommand{\ps}[2]{
\expval{#1 | #2}
}
\newcommand{\df}[1]{
\mqty{#1}
}
\newcommand{\n}[1]{
\norm{#1}
}
\newcommand{\sys}[1]{
\left\{\smqty{#1}\right.
}


\newcommand{\eqdef}{\ensuremath{\overset{\text{def}}=}}


\def\Circlearrowright{\ensuremath{%
  \rotatebox[origin=c]{230}{$\circlearrowright$}}}

\newcommand\ct[1]{\text{\rmfamily\upshape #1}}
\newcommand\question[1]{ {\color{red} ...!? \small #1}}
\newcommand\caz[1]{\left\{\begin{array} #1 \end{array}\right.}
\newcommand\const{\text{\rmfamily\upshape const}}
\newcommand\toP{ \overset{\pro}{\to}}
\newcommand\toPP{ \overset{\text{PP}}{\to}}
\newcommand{\oeq}{\mathrel{\text{\textcircled{$=$}}}}





\usepackage{xcolor}
% \usepackage[normalem]{ulem}
\usepackage{lipsum}
\makeatletter
% \newcommand\colorwave[1][blue]{\bgroup \markoverwith{\lower3.5\p@\hbox{\sixly \textcolor{#1}{\char58}}}\ULon}
%\font\sixly=lasy6 % does not re-load if already loaded, so no memory problem.

\newmdtheoremenv[
linewidth= 1pt,linecolor= blue,%
leftmargin=20,rightmargin=20,innertopmargin=0pt, innerrightmargin=40,%
tikzsetting = { draw=lightgray, line width = 0.3pt,dashed,%
dash pattern = on 15pt off 3pt},%
splittopskip=\topskip,skipbelow=\baselineskip,%
skipabove=\baselineskip,ntheorem,roundcorner=0pt,
% backgroundcolor=pagebg,font=\color{orange}\sffamily, fontcolor=white
]{examplebox}{Exemple}[section]



\newcommand\R{\mathbb{R}}
\newcommand\Z{\mathbb{Z}}
\newcommand\N{\mathbb{N}}
\newcommand\E{\mathbb{E}}
\newcommand\F{\mathcal{F}}
\newcommand\cH{\mathcal{H}}
\newcommand\V{\mathbb{V}}
\newcommand\dmo{ ^{-1} }
\newcommand\kapa{\kappa}
\newcommand\im{Im}
\newcommand\hs{\mathcal{H}}





\usepackage{soul}

\makeatletter
\newcommand*{\whiten}[1]{\llap{\textcolor{white}{{\the\SOUL@token}}\hspace{#1pt}}}
\DeclareRobustCommand*\myul{%
    \def\SOUL@everyspace{\underline{\space}\kern\z@}%
    \def\SOUL@everytoken{%
     \setbox0=\hbox{\the\SOUL@token}%
     \ifdim\dp0>\z@
        \raisebox{\dp0}{\underline{\phantom{\the\SOUL@token}}}%
        \whiten{1}\whiten{0}%
        \whiten{-1}\whiten{-2}%
        \llap{\the\SOUL@token}%
     \else
        \underline{\the\SOUL@token}%
     \fi}%
\SOUL@}
\makeatother

\newcommand*{\demp}{\fontfamily{lmtt}\selectfont}

\DeclareTextFontCommand{\textdemp}{\demp}

\begin{document}

\ifcomment
Multiline
comment
\fi
\ifcomment
\myul{Typesetting test}
% \color[rgb]{1,1,1}
$∑_i^n≠ 60º±∞π∆¬≈√j∫h≤≥µ$

$\CR \R\pro\ind\pro\gS\pro
\mqty[a&b\\c&d]$
$\pro\mathbb{P}$
$\dd{x}$

  \[
    \alpha(x)=\left\{
                \begin{array}{ll}
                  x\\
                  \frac{1}{1+e^{-kx}}\\
                  \frac{e^x-e^{-x}}{e^x+e^{-x}}
                \end{array}
              \right.
  \]

  $\expval{x}$
  
  $\chi_\rho(ghg\dmo)=\Tr(\rho_{ghg\dmo})=\Tr(\rho_g\circ\rho_h\circ\rho\dmo_g)=\Tr(\rho_h)\overset{\mbox{\scalebox{0.5}{$\Tr(AB)=\Tr(BA)$}}}{=}\chi_\rho(h)$
  	$\mathop{\oplus}_{\substack{x\in X}}$

$\mat(\rho_g)=(a_{ij}(g))_{\scriptsize \substack{1\leq i\leq d \\ 1\leq j\leq d}}$ et $\mat(\rho'_g)=(a'_{ij}(g))_{\scriptsize \substack{1\leq i'\leq d' \\ 1\leq j'\leq d'}}$



\[\int_a^b{\mathbb{R}^2}g(u, v)\dd{P_{XY}}(u, v)=\iint g(u,v) f_{XY}(u, v)\dd \lambda(u) \dd \lambda(v)\]
$$\lim_{x\to\infty} f(x)$$	
$$\iiiint_V \mu(t,u,v,w) \,dt\,du\,dv\,dw$$
$$\sum_{n=1}^{\infty} 2^{-n} = 1$$	
\begin{definition}
	Si $X$ et $Y$ sont 2 v.a. ou definit la \textsc{Covariance} entre $X$ et $Y$ comme
	$\cov(X,Y)\overset{\text{def}}{=}\E\left[(X-\E(X))(Y-\E(Y))\right]=\E(XY)-\E(X)\E(Y)$.
\end{definition}
\fi
\pagebreak

% \tableofcontents

% insert your code here
%\input{./algebra/main.tex}
%\input{./geometrie-differentielle/main.tex}
%\input{./probabilite/main.tex}
%\input{./analyse-fonctionnelle/main.tex}
% \input{./Analyse-convexe-et-dualite-en-optimisation/main.tex}
%\input{./tikz/main.tex}
%\input{./Theorie-du-distributions/main.tex}
%\input{./optimisation/mine.tex}
 \input{./modelisation/main.tex}

% yves.aubry@univ-tln.fr : algebra

\end{document}

% % !TEX encoding = UTF-8 Unicode
% !TEX TS-program = xelatex

\documentclass[french]{report}

%\usepackage[utf8]{inputenc}
%\usepackage[T1]{fontenc}
\usepackage{babel}


\newif\ifcomment
%\commenttrue # Show comments

\usepackage{physics}
\usepackage{amssymb}


\usepackage{amsthm}
% \usepackage{thmtools}
\usepackage{mathtools}
\usepackage{amsfonts}

\usepackage{color}

\usepackage{tikz}

\usepackage{geometry}
\geometry{a5paper, margin=0.1in, right=1cm}

\usepackage{dsfont}

\usepackage{graphicx}
\graphicspath{ {images/} }

\usepackage{faktor}

\usepackage{IEEEtrantools}
\usepackage{enumerate}   
\usepackage[PostScript=dvips]{"/Users/aware/Documents/Courses/diagrams"}


\newtheorem{theorem}{Théorème}[section]
\renewcommand{\thetheorem}{\arabic{theorem}}
\newtheorem{lemme}{Lemme}[section]
\renewcommand{\thelemme}{\arabic{lemme}}
\newtheorem{proposition}{Proposition}[section]
\renewcommand{\theproposition}{\arabic{proposition}}
\newtheorem{notations}{Notations}[section]
\newtheorem{problem}{Problème}[section]
\newtheorem{corollary}{Corollaire}[theorem]
\renewcommand{\thecorollary}{\arabic{corollary}}
\newtheorem{property}{Propriété}[section]
\newtheorem{objective}{Objectif}[section]

\theoremstyle{definition}
\newtheorem{definition}{Définition}[section]
\renewcommand{\thedefinition}{\arabic{definition}}
\newtheorem{exercise}{Exercice}[chapter]
\renewcommand{\theexercise}{\arabic{exercise}}
\newtheorem{example}{Exemple}[chapter]
\renewcommand{\theexample}{\arabic{example}}
\newtheorem*{solution}{Solution}
\newtheorem*{application}{Application}
\newtheorem*{notation}{Notation}
\newtheorem*{vocabulary}{Vocabulaire}
\newtheorem*{properties}{Propriétés}



\theoremstyle{remark}
\newtheorem*{remark}{Remarque}
\newtheorem*{rappel}{Rappel}


\usepackage{etoolbox}
\AtBeginEnvironment{exercise}{\small}
\AtBeginEnvironment{example}{\small}

\usepackage{cases}
\usepackage[red]{mypack}

\usepackage[framemethod=TikZ]{mdframed}

\definecolor{bg}{rgb}{0.4,0.25,0.95}
\definecolor{pagebg}{rgb}{0,0,0.5}
\surroundwithmdframed[
   topline=false,
   rightline=false,
   bottomline=false,
   leftmargin=\parindent,
   skipabove=8pt,
   skipbelow=8pt,
   linecolor=blue,
   innerbottommargin=10pt,
   % backgroundcolor=bg,font=\color{orange}\sffamily, fontcolor=white
]{definition}

\usepackage{empheq}
\usepackage[most]{tcolorbox}

\newtcbox{\mymath}[1][]{%
    nobeforeafter, math upper, tcbox raise base,
    enhanced, colframe=blue!30!black,
    colback=red!10, boxrule=1pt,
    #1}

\usepackage{unixode}


\DeclareMathOperator{\ord}{ord}
\DeclareMathOperator{\orb}{orb}
\DeclareMathOperator{\stab}{stab}
\DeclareMathOperator{\Stab}{stab}
\DeclareMathOperator{\ppcm}{ppcm}
\DeclareMathOperator{\conj}{Conj}
\DeclareMathOperator{\End}{End}
\DeclareMathOperator{\rot}{rot}
\DeclareMathOperator{\trs}{trace}
\DeclareMathOperator{\Ind}{Ind}
\DeclareMathOperator{\mat}{Mat}
\DeclareMathOperator{\id}{Id}
\DeclareMathOperator{\vect}{vect}
\DeclareMathOperator{\img}{img}
\DeclareMathOperator{\cov}{Cov}
\DeclareMathOperator{\dist}{dist}
\DeclareMathOperator{\irr}{Irr}
\DeclareMathOperator{\image}{Im}
\DeclareMathOperator{\pd}{\partial}
\DeclareMathOperator{\epi}{epi}
\DeclareMathOperator{\Argmin}{Argmin}
\DeclareMathOperator{\dom}{dom}
\DeclareMathOperator{\proj}{proj}
\DeclareMathOperator{\ctg}{ctg}
\DeclareMathOperator{\supp}{supp}
\DeclareMathOperator{\argmin}{argmin}
\DeclareMathOperator{\mult}{mult}
\DeclareMathOperator{\ch}{ch}
\DeclareMathOperator{\sh}{sh}
\DeclareMathOperator{\rang}{rang}
\DeclareMathOperator{\diam}{diam}
\DeclareMathOperator{\Epigraphe}{Epigraphe}




\usepackage{xcolor}
\everymath{\color{blue}}
%\everymath{\color[rgb]{0,1,1}}
%\pagecolor[rgb]{0,0,0.5}


\newcommand*{\pdtest}[3][]{\ensuremath{\frac{\partial^{#1} #2}{\partial #3}}}

\newcommand*{\deffunc}[6][]{\ensuremath{
\begin{array}{rcl}
#2 : #3 &\rightarrow& #4\\
#5 &\mapsto& #6
\end{array}
}}

\newcommand{\eqcolon}{\mathrel{\resizebox{\widthof{$\mathord{=}$}}{\height}{ $\!\!=\!\!\resizebox{1.2\width}{0.8\height}{\raisebox{0.23ex}{$\mathop{:}$}}\!\!$ }}}
\newcommand{\coloneq}{\mathrel{\resizebox{\widthof{$\mathord{=}$}}{\height}{ $\!\!\resizebox{1.2\width}{0.8\height}{\raisebox{0.23ex}{$\mathop{:}$}}\!\!=\!\!$ }}}
\newcommand{\eqcolonl}{\ensuremath{\mathrel{=\!\!\mathop{:}}}}
\newcommand{\coloneql}{\ensuremath{\mathrel{\mathop{:} \!\! =}}}
\newcommand{\vc}[1]{% inline column vector
  \left(\begin{smallmatrix}#1\end{smallmatrix}\right)%
}
\newcommand{\vr}[1]{% inline row vector
  \begin{smallmatrix}(\,#1\,)\end{smallmatrix}%
}
\makeatletter
\newcommand*{\defeq}{\ =\mathrel{\rlap{%
                     \raisebox{0.3ex}{$\m@th\cdot$}}%
                     \raisebox{-0.3ex}{$\m@th\cdot$}}%
                     }
\makeatother

\newcommand{\mathcircle}[1]{% inline row vector
 \overset{\circ}{#1}
}
\newcommand{\ulim}{% low limit
 \underline{\lim}
}
\newcommand{\ssi}{% iff
\iff
}
\newcommand{\ps}[2]{
\expval{#1 | #2}
}
\newcommand{\df}[1]{
\mqty{#1}
}
\newcommand{\n}[1]{
\norm{#1}
}
\newcommand{\sys}[1]{
\left\{\smqty{#1}\right.
}


\newcommand{\eqdef}{\ensuremath{\overset{\text{def}}=}}


\def\Circlearrowright{\ensuremath{%
  \rotatebox[origin=c]{230}{$\circlearrowright$}}}

\newcommand\ct[1]{\text{\rmfamily\upshape #1}}
\newcommand\question[1]{ {\color{red} ...!? \small #1}}
\newcommand\caz[1]{\left\{\begin{array} #1 \end{array}\right.}
\newcommand\const{\text{\rmfamily\upshape const}}
\newcommand\toP{ \overset{\pro}{\to}}
\newcommand\toPP{ \overset{\text{PP}}{\to}}
\newcommand{\oeq}{\mathrel{\text{\textcircled{$=$}}}}





\usepackage{xcolor}
% \usepackage[normalem]{ulem}
\usepackage{lipsum}
\makeatletter
% \newcommand\colorwave[1][blue]{\bgroup \markoverwith{\lower3.5\p@\hbox{\sixly \textcolor{#1}{\char58}}}\ULon}
%\font\sixly=lasy6 % does not re-load if already loaded, so no memory problem.

\newmdtheoremenv[
linewidth= 1pt,linecolor= blue,%
leftmargin=20,rightmargin=20,innertopmargin=0pt, innerrightmargin=40,%
tikzsetting = { draw=lightgray, line width = 0.3pt,dashed,%
dash pattern = on 15pt off 3pt},%
splittopskip=\topskip,skipbelow=\baselineskip,%
skipabove=\baselineskip,ntheorem,roundcorner=0pt,
% backgroundcolor=pagebg,font=\color{orange}\sffamily, fontcolor=white
]{examplebox}{Exemple}[section]



\newcommand\R{\mathbb{R}}
\newcommand\Z{\mathbb{Z}}
\newcommand\N{\mathbb{N}}
\newcommand\E{\mathbb{E}}
\newcommand\F{\mathcal{F}}
\newcommand\cH{\mathcal{H}}
\newcommand\V{\mathbb{V}}
\newcommand\dmo{ ^{-1} }
\newcommand\kapa{\kappa}
\newcommand\im{Im}
\newcommand\hs{\mathcal{H}}





\usepackage{soul}

\makeatletter
\newcommand*{\whiten}[1]{\llap{\textcolor{white}{{\the\SOUL@token}}\hspace{#1pt}}}
\DeclareRobustCommand*\myul{%
    \def\SOUL@everyspace{\underline{\space}\kern\z@}%
    \def\SOUL@everytoken{%
     \setbox0=\hbox{\the\SOUL@token}%
     \ifdim\dp0>\z@
        \raisebox{\dp0}{\underline{\phantom{\the\SOUL@token}}}%
        \whiten{1}\whiten{0}%
        \whiten{-1}\whiten{-2}%
        \llap{\the\SOUL@token}%
     \else
        \underline{\the\SOUL@token}%
     \fi}%
\SOUL@}
\makeatother

\newcommand*{\demp}{\fontfamily{lmtt}\selectfont}

\DeclareTextFontCommand{\textdemp}{\demp}

\begin{document}

\ifcomment
Multiline
comment
\fi
\ifcomment
\myul{Typesetting test}
% \color[rgb]{1,1,1}
$∑_i^n≠ 60º±∞π∆¬≈√j∫h≤≥µ$

$\CR \R\pro\ind\pro\gS\pro
\mqty[a&b\\c&d]$
$\pro\mathbb{P}$
$\dd{x}$

  \[
    \alpha(x)=\left\{
                \begin{array}{ll}
                  x\\
                  \frac{1}{1+e^{-kx}}\\
                  \frac{e^x-e^{-x}}{e^x+e^{-x}}
                \end{array}
              \right.
  \]

  $\expval{x}$
  
  $\chi_\rho(ghg\dmo)=\Tr(\rho_{ghg\dmo})=\Tr(\rho_g\circ\rho_h\circ\rho\dmo_g)=\Tr(\rho_h)\overset{\mbox{\scalebox{0.5}{$\Tr(AB)=\Tr(BA)$}}}{=}\chi_\rho(h)$
  	$\mathop{\oplus}_{\substack{x\in X}}$

$\mat(\rho_g)=(a_{ij}(g))_{\scriptsize \substack{1\leq i\leq d \\ 1\leq j\leq d}}$ et $\mat(\rho'_g)=(a'_{ij}(g))_{\scriptsize \substack{1\leq i'\leq d' \\ 1\leq j'\leq d'}}$



\[\int_a^b{\mathbb{R}^2}g(u, v)\dd{P_{XY}}(u, v)=\iint g(u,v) f_{XY}(u, v)\dd \lambda(u) \dd \lambda(v)\]
$$\lim_{x\to\infty} f(x)$$	
$$\iiiint_V \mu(t,u,v,w) \,dt\,du\,dv\,dw$$
$$\sum_{n=1}^{\infty} 2^{-n} = 1$$	
\begin{definition}
	Si $X$ et $Y$ sont 2 v.a. ou definit la \textsc{Covariance} entre $X$ et $Y$ comme
	$\cov(X,Y)\overset{\text{def}}{=}\E\left[(X-\E(X))(Y-\E(Y))\right]=\E(XY)-\E(X)\E(Y)$.
\end{definition}
\fi
\pagebreak

% \tableofcontents

% insert your code here
%\input{./algebra/main.tex}
%\input{./geometrie-differentielle/main.tex}
%\input{./probabilite/main.tex}
%\input{./analyse-fonctionnelle/main.tex}
% \input{./Analyse-convexe-et-dualite-en-optimisation/main.tex}
%\input{./tikz/main.tex}
%\input{./Theorie-du-distributions/main.tex}
%\input{./optimisation/mine.tex}
 \input{./modelisation/main.tex}

% yves.aubry@univ-tln.fr : algebra

\end{document}

%% !TEX encoding = UTF-8 Unicode
% !TEX TS-program = xelatex

\documentclass[french]{report}

%\usepackage[utf8]{inputenc}
%\usepackage[T1]{fontenc}
\usepackage{babel}


\newif\ifcomment
%\commenttrue # Show comments

\usepackage{physics}
\usepackage{amssymb}


\usepackage{amsthm}
% \usepackage{thmtools}
\usepackage{mathtools}
\usepackage{amsfonts}

\usepackage{color}

\usepackage{tikz}

\usepackage{geometry}
\geometry{a5paper, margin=0.1in, right=1cm}

\usepackage{dsfont}

\usepackage{graphicx}
\graphicspath{ {images/} }

\usepackage{faktor}

\usepackage{IEEEtrantools}
\usepackage{enumerate}   
\usepackage[PostScript=dvips]{"/Users/aware/Documents/Courses/diagrams"}


\newtheorem{theorem}{Théorème}[section]
\renewcommand{\thetheorem}{\arabic{theorem}}
\newtheorem{lemme}{Lemme}[section]
\renewcommand{\thelemme}{\arabic{lemme}}
\newtheorem{proposition}{Proposition}[section]
\renewcommand{\theproposition}{\arabic{proposition}}
\newtheorem{notations}{Notations}[section]
\newtheorem{problem}{Problème}[section]
\newtheorem{corollary}{Corollaire}[theorem]
\renewcommand{\thecorollary}{\arabic{corollary}}
\newtheorem{property}{Propriété}[section]
\newtheorem{objective}{Objectif}[section]

\theoremstyle{definition}
\newtheorem{definition}{Définition}[section]
\renewcommand{\thedefinition}{\arabic{definition}}
\newtheorem{exercise}{Exercice}[chapter]
\renewcommand{\theexercise}{\arabic{exercise}}
\newtheorem{example}{Exemple}[chapter]
\renewcommand{\theexample}{\arabic{example}}
\newtheorem*{solution}{Solution}
\newtheorem*{application}{Application}
\newtheorem*{notation}{Notation}
\newtheorem*{vocabulary}{Vocabulaire}
\newtheorem*{properties}{Propriétés}



\theoremstyle{remark}
\newtheorem*{remark}{Remarque}
\newtheorem*{rappel}{Rappel}


\usepackage{etoolbox}
\AtBeginEnvironment{exercise}{\small}
\AtBeginEnvironment{example}{\small}

\usepackage{cases}
\usepackage[red]{mypack}

\usepackage[framemethod=TikZ]{mdframed}

\definecolor{bg}{rgb}{0.4,0.25,0.95}
\definecolor{pagebg}{rgb}{0,0,0.5}
\surroundwithmdframed[
   topline=false,
   rightline=false,
   bottomline=false,
   leftmargin=\parindent,
   skipabove=8pt,
   skipbelow=8pt,
   linecolor=blue,
   innerbottommargin=10pt,
   % backgroundcolor=bg,font=\color{orange}\sffamily, fontcolor=white
]{definition}

\usepackage{empheq}
\usepackage[most]{tcolorbox}

\newtcbox{\mymath}[1][]{%
    nobeforeafter, math upper, tcbox raise base,
    enhanced, colframe=blue!30!black,
    colback=red!10, boxrule=1pt,
    #1}

\usepackage{unixode}


\DeclareMathOperator{\ord}{ord}
\DeclareMathOperator{\orb}{orb}
\DeclareMathOperator{\stab}{stab}
\DeclareMathOperator{\Stab}{stab}
\DeclareMathOperator{\ppcm}{ppcm}
\DeclareMathOperator{\conj}{Conj}
\DeclareMathOperator{\End}{End}
\DeclareMathOperator{\rot}{rot}
\DeclareMathOperator{\trs}{trace}
\DeclareMathOperator{\Ind}{Ind}
\DeclareMathOperator{\mat}{Mat}
\DeclareMathOperator{\id}{Id}
\DeclareMathOperator{\vect}{vect}
\DeclareMathOperator{\img}{img}
\DeclareMathOperator{\cov}{Cov}
\DeclareMathOperator{\dist}{dist}
\DeclareMathOperator{\irr}{Irr}
\DeclareMathOperator{\image}{Im}
\DeclareMathOperator{\pd}{\partial}
\DeclareMathOperator{\epi}{epi}
\DeclareMathOperator{\Argmin}{Argmin}
\DeclareMathOperator{\dom}{dom}
\DeclareMathOperator{\proj}{proj}
\DeclareMathOperator{\ctg}{ctg}
\DeclareMathOperator{\supp}{supp}
\DeclareMathOperator{\argmin}{argmin}
\DeclareMathOperator{\mult}{mult}
\DeclareMathOperator{\ch}{ch}
\DeclareMathOperator{\sh}{sh}
\DeclareMathOperator{\rang}{rang}
\DeclareMathOperator{\diam}{diam}
\DeclareMathOperator{\Epigraphe}{Epigraphe}




\usepackage{xcolor}
\everymath{\color{blue}}
%\everymath{\color[rgb]{0,1,1}}
%\pagecolor[rgb]{0,0,0.5}


\newcommand*{\pdtest}[3][]{\ensuremath{\frac{\partial^{#1} #2}{\partial #3}}}

\newcommand*{\deffunc}[6][]{\ensuremath{
\begin{array}{rcl}
#2 : #3 &\rightarrow& #4\\
#5 &\mapsto& #6
\end{array}
}}

\newcommand{\eqcolon}{\mathrel{\resizebox{\widthof{$\mathord{=}$}}{\height}{ $\!\!=\!\!\resizebox{1.2\width}{0.8\height}{\raisebox{0.23ex}{$\mathop{:}$}}\!\!$ }}}
\newcommand{\coloneq}{\mathrel{\resizebox{\widthof{$\mathord{=}$}}{\height}{ $\!\!\resizebox{1.2\width}{0.8\height}{\raisebox{0.23ex}{$\mathop{:}$}}\!\!=\!\!$ }}}
\newcommand{\eqcolonl}{\ensuremath{\mathrel{=\!\!\mathop{:}}}}
\newcommand{\coloneql}{\ensuremath{\mathrel{\mathop{:} \!\! =}}}
\newcommand{\vc}[1]{% inline column vector
  \left(\begin{smallmatrix}#1\end{smallmatrix}\right)%
}
\newcommand{\vr}[1]{% inline row vector
  \begin{smallmatrix}(\,#1\,)\end{smallmatrix}%
}
\makeatletter
\newcommand*{\defeq}{\ =\mathrel{\rlap{%
                     \raisebox{0.3ex}{$\m@th\cdot$}}%
                     \raisebox{-0.3ex}{$\m@th\cdot$}}%
                     }
\makeatother

\newcommand{\mathcircle}[1]{% inline row vector
 \overset{\circ}{#1}
}
\newcommand{\ulim}{% low limit
 \underline{\lim}
}
\newcommand{\ssi}{% iff
\iff
}
\newcommand{\ps}[2]{
\expval{#1 | #2}
}
\newcommand{\df}[1]{
\mqty{#1}
}
\newcommand{\n}[1]{
\norm{#1}
}
\newcommand{\sys}[1]{
\left\{\smqty{#1}\right.
}


\newcommand{\eqdef}{\ensuremath{\overset{\text{def}}=}}


\def\Circlearrowright{\ensuremath{%
  \rotatebox[origin=c]{230}{$\circlearrowright$}}}

\newcommand\ct[1]{\text{\rmfamily\upshape #1}}
\newcommand\question[1]{ {\color{red} ...!? \small #1}}
\newcommand\caz[1]{\left\{\begin{array} #1 \end{array}\right.}
\newcommand\const{\text{\rmfamily\upshape const}}
\newcommand\toP{ \overset{\pro}{\to}}
\newcommand\toPP{ \overset{\text{PP}}{\to}}
\newcommand{\oeq}{\mathrel{\text{\textcircled{$=$}}}}





\usepackage{xcolor}
% \usepackage[normalem]{ulem}
\usepackage{lipsum}
\makeatletter
% \newcommand\colorwave[1][blue]{\bgroup \markoverwith{\lower3.5\p@\hbox{\sixly \textcolor{#1}{\char58}}}\ULon}
%\font\sixly=lasy6 % does not re-load if already loaded, so no memory problem.

\newmdtheoremenv[
linewidth= 1pt,linecolor= blue,%
leftmargin=20,rightmargin=20,innertopmargin=0pt, innerrightmargin=40,%
tikzsetting = { draw=lightgray, line width = 0.3pt,dashed,%
dash pattern = on 15pt off 3pt},%
splittopskip=\topskip,skipbelow=\baselineskip,%
skipabove=\baselineskip,ntheorem,roundcorner=0pt,
% backgroundcolor=pagebg,font=\color{orange}\sffamily, fontcolor=white
]{examplebox}{Exemple}[section]



\newcommand\R{\mathbb{R}}
\newcommand\Z{\mathbb{Z}}
\newcommand\N{\mathbb{N}}
\newcommand\E{\mathbb{E}}
\newcommand\F{\mathcal{F}}
\newcommand\cH{\mathcal{H}}
\newcommand\V{\mathbb{V}}
\newcommand\dmo{ ^{-1} }
\newcommand\kapa{\kappa}
\newcommand\im{Im}
\newcommand\hs{\mathcal{H}}





\usepackage{soul}

\makeatletter
\newcommand*{\whiten}[1]{\llap{\textcolor{white}{{\the\SOUL@token}}\hspace{#1pt}}}
\DeclareRobustCommand*\myul{%
    \def\SOUL@everyspace{\underline{\space}\kern\z@}%
    \def\SOUL@everytoken{%
     \setbox0=\hbox{\the\SOUL@token}%
     \ifdim\dp0>\z@
        \raisebox{\dp0}{\underline{\phantom{\the\SOUL@token}}}%
        \whiten{1}\whiten{0}%
        \whiten{-1}\whiten{-2}%
        \llap{\the\SOUL@token}%
     \else
        \underline{\the\SOUL@token}%
     \fi}%
\SOUL@}
\makeatother

\newcommand*{\demp}{\fontfamily{lmtt}\selectfont}

\DeclareTextFontCommand{\textdemp}{\demp}

\begin{document}

\ifcomment
Multiline
comment
\fi
\ifcomment
\myul{Typesetting test}
% \color[rgb]{1,1,1}
$∑_i^n≠ 60º±∞π∆¬≈√j∫h≤≥µ$

$\CR \R\pro\ind\pro\gS\pro
\mqty[a&b\\c&d]$
$\pro\mathbb{P}$
$\dd{x}$

  \[
    \alpha(x)=\left\{
                \begin{array}{ll}
                  x\\
                  \frac{1}{1+e^{-kx}}\\
                  \frac{e^x-e^{-x}}{e^x+e^{-x}}
                \end{array}
              \right.
  \]

  $\expval{x}$
  
  $\chi_\rho(ghg\dmo)=\Tr(\rho_{ghg\dmo})=\Tr(\rho_g\circ\rho_h\circ\rho\dmo_g)=\Tr(\rho_h)\overset{\mbox{\scalebox{0.5}{$\Tr(AB)=\Tr(BA)$}}}{=}\chi_\rho(h)$
  	$\mathop{\oplus}_{\substack{x\in X}}$

$\mat(\rho_g)=(a_{ij}(g))_{\scriptsize \substack{1\leq i\leq d \\ 1\leq j\leq d}}$ et $\mat(\rho'_g)=(a'_{ij}(g))_{\scriptsize \substack{1\leq i'\leq d' \\ 1\leq j'\leq d'}}$



\[\int_a^b{\mathbb{R}^2}g(u, v)\dd{P_{XY}}(u, v)=\iint g(u,v) f_{XY}(u, v)\dd \lambda(u) \dd \lambda(v)\]
$$\lim_{x\to\infty} f(x)$$	
$$\iiiint_V \mu(t,u,v,w) \,dt\,du\,dv\,dw$$
$$\sum_{n=1}^{\infty} 2^{-n} = 1$$	
\begin{definition}
	Si $X$ et $Y$ sont 2 v.a. ou definit la \textsc{Covariance} entre $X$ et $Y$ comme
	$\cov(X,Y)\overset{\text{def}}{=}\E\left[(X-\E(X))(Y-\E(Y))\right]=\E(XY)-\E(X)\E(Y)$.
\end{definition}
\fi
\pagebreak

% \tableofcontents

% insert your code here
%\input{./algebra/main.tex}
%\input{./geometrie-differentielle/main.tex}
%\input{./probabilite/main.tex}
%\input{./analyse-fonctionnelle/main.tex}
% \input{./Analyse-convexe-et-dualite-en-optimisation/main.tex}
%\input{./tikz/main.tex}
%\input{./Theorie-du-distributions/main.tex}
%\input{./optimisation/mine.tex}
 \input{./modelisation/main.tex}

% yves.aubry@univ-tln.fr : algebra

\end{document}

%% !TEX encoding = UTF-8 Unicode
% !TEX TS-program = xelatex

\documentclass[french]{report}

%\usepackage[utf8]{inputenc}
%\usepackage[T1]{fontenc}
\usepackage{babel}


\newif\ifcomment
%\commenttrue # Show comments

\usepackage{physics}
\usepackage{amssymb}


\usepackage{amsthm}
% \usepackage{thmtools}
\usepackage{mathtools}
\usepackage{amsfonts}

\usepackage{color}

\usepackage{tikz}

\usepackage{geometry}
\geometry{a5paper, margin=0.1in, right=1cm}

\usepackage{dsfont}

\usepackage{graphicx}
\graphicspath{ {images/} }

\usepackage{faktor}

\usepackage{IEEEtrantools}
\usepackage{enumerate}   
\usepackage[PostScript=dvips]{"/Users/aware/Documents/Courses/diagrams"}


\newtheorem{theorem}{Théorème}[section]
\renewcommand{\thetheorem}{\arabic{theorem}}
\newtheorem{lemme}{Lemme}[section]
\renewcommand{\thelemme}{\arabic{lemme}}
\newtheorem{proposition}{Proposition}[section]
\renewcommand{\theproposition}{\arabic{proposition}}
\newtheorem{notations}{Notations}[section]
\newtheorem{problem}{Problème}[section]
\newtheorem{corollary}{Corollaire}[theorem]
\renewcommand{\thecorollary}{\arabic{corollary}}
\newtheorem{property}{Propriété}[section]
\newtheorem{objective}{Objectif}[section]

\theoremstyle{definition}
\newtheorem{definition}{Définition}[section]
\renewcommand{\thedefinition}{\arabic{definition}}
\newtheorem{exercise}{Exercice}[chapter]
\renewcommand{\theexercise}{\arabic{exercise}}
\newtheorem{example}{Exemple}[chapter]
\renewcommand{\theexample}{\arabic{example}}
\newtheorem*{solution}{Solution}
\newtheorem*{application}{Application}
\newtheorem*{notation}{Notation}
\newtheorem*{vocabulary}{Vocabulaire}
\newtheorem*{properties}{Propriétés}



\theoremstyle{remark}
\newtheorem*{remark}{Remarque}
\newtheorem*{rappel}{Rappel}


\usepackage{etoolbox}
\AtBeginEnvironment{exercise}{\small}
\AtBeginEnvironment{example}{\small}

\usepackage{cases}
\usepackage[red]{mypack}

\usepackage[framemethod=TikZ]{mdframed}

\definecolor{bg}{rgb}{0.4,0.25,0.95}
\definecolor{pagebg}{rgb}{0,0,0.5}
\surroundwithmdframed[
   topline=false,
   rightline=false,
   bottomline=false,
   leftmargin=\parindent,
   skipabove=8pt,
   skipbelow=8pt,
   linecolor=blue,
   innerbottommargin=10pt,
   % backgroundcolor=bg,font=\color{orange}\sffamily, fontcolor=white
]{definition}

\usepackage{empheq}
\usepackage[most]{tcolorbox}

\newtcbox{\mymath}[1][]{%
    nobeforeafter, math upper, tcbox raise base,
    enhanced, colframe=blue!30!black,
    colback=red!10, boxrule=1pt,
    #1}

\usepackage{unixode}


\DeclareMathOperator{\ord}{ord}
\DeclareMathOperator{\orb}{orb}
\DeclareMathOperator{\stab}{stab}
\DeclareMathOperator{\Stab}{stab}
\DeclareMathOperator{\ppcm}{ppcm}
\DeclareMathOperator{\conj}{Conj}
\DeclareMathOperator{\End}{End}
\DeclareMathOperator{\rot}{rot}
\DeclareMathOperator{\trs}{trace}
\DeclareMathOperator{\Ind}{Ind}
\DeclareMathOperator{\mat}{Mat}
\DeclareMathOperator{\id}{Id}
\DeclareMathOperator{\vect}{vect}
\DeclareMathOperator{\img}{img}
\DeclareMathOperator{\cov}{Cov}
\DeclareMathOperator{\dist}{dist}
\DeclareMathOperator{\irr}{Irr}
\DeclareMathOperator{\image}{Im}
\DeclareMathOperator{\pd}{\partial}
\DeclareMathOperator{\epi}{epi}
\DeclareMathOperator{\Argmin}{Argmin}
\DeclareMathOperator{\dom}{dom}
\DeclareMathOperator{\proj}{proj}
\DeclareMathOperator{\ctg}{ctg}
\DeclareMathOperator{\supp}{supp}
\DeclareMathOperator{\argmin}{argmin}
\DeclareMathOperator{\mult}{mult}
\DeclareMathOperator{\ch}{ch}
\DeclareMathOperator{\sh}{sh}
\DeclareMathOperator{\rang}{rang}
\DeclareMathOperator{\diam}{diam}
\DeclareMathOperator{\Epigraphe}{Epigraphe}




\usepackage{xcolor}
\everymath{\color{blue}}
%\everymath{\color[rgb]{0,1,1}}
%\pagecolor[rgb]{0,0,0.5}


\newcommand*{\pdtest}[3][]{\ensuremath{\frac{\partial^{#1} #2}{\partial #3}}}

\newcommand*{\deffunc}[6][]{\ensuremath{
\begin{array}{rcl}
#2 : #3 &\rightarrow& #4\\
#5 &\mapsto& #6
\end{array}
}}

\newcommand{\eqcolon}{\mathrel{\resizebox{\widthof{$\mathord{=}$}}{\height}{ $\!\!=\!\!\resizebox{1.2\width}{0.8\height}{\raisebox{0.23ex}{$\mathop{:}$}}\!\!$ }}}
\newcommand{\coloneq}{\mathrel{\resizebox{\widthof{$\mathord{=}$}}{\height}{ $\!\!\resizebox{1.2\width}{0.8\height}{\raisebox{0.23ex}{$\mathop{:}$}}\!\!=\!\!$ }}}
\newcommand{\eqcolonl}{\ensuremath{\mathrel{=\!\!\mathop{:}}}}
\newcommand{\coloneql}{\ensuremath{\mathrel{\mathop{:} \!\! =}}}
\newcommand{\vc}[1]{% inline column vector
  \left(\begin{smallmatrix}#1\end{smallmatrix}\right)%
}
\newcommand{\vr}[1]{% inline row vector
  \begin{smallmatrix}(\,#1\,)\end{smallmatrix}%
}
\makeatletter
\newcommand*{\defeq}{\ =\mathrel{\rlap{%
                     \raisebox{0.3ex}{$\m@th\cdot$}}%
                     \raisebox{-0.3ex}{$\m@th\cdot$}}%
                     }
\makeatother

\newcommand{\mathcircle}[1]{% inline row vector
 \overset{\circ}{#1}
}
\newcommand{\ulim}{% low limit
 \underline{\lim}
}
\newcommand{\ssi}{% iff
\iff
}
\newcommand{\ps}[2]{
\expval{#1 | #2}
}
\newcommand{\df}[1]{
\mqty{#1}
}
\newcommand{\n}[1]{
\norm{#1}
}
\newcommand{\sys}[1]{
\left\{\smqty{#1}\right.
}


\newcommand{\eqdef}{\ensuremath{\overset{\text{def}}=}}


\def\Circlearrowright{\ensuremath{%
  \rotatebox[origin=c]{230}{$\circlearrowright$}}}

\newcommand\ct[1]{\text{\rmfamily\upshape #1}}
\newcommand\question[1]{ {\color{red} ...!? \small #1}}
\newcommand\caz[1]{\left\{\begin{array} #1 \end{array}\right.}
\newcommand\const{\text{\rmfamily\upshape const}}
\newcommand\toP{ \overset{\pro}{\to}}
\newcommand\toPP{ \overset{\text{PP}}{\to}}
\newcommand{\oeq}{\mathrel{\text{\textcircled{$=$}}}}





\usepackage{xcolor}
% \usepackage[normalem]{ulem}
\usepackage{lipsum}
\makeatletter
% \newcommand\colorwave[1][blue]{\bgroup \markoverwith{\lower3.5\p@\hbox{\sixly \textcolor{#1}{\char58}}}\ULon}
%\font\sixly=lasy6 % does not re-load if already loaded, so no memory problem.

\newmdtheoremenv[
linewidth= 1pt,linecolor= blue,%
leftmargin=20,rightmargin=20,innertopmargin=0pt, innerrightmargin=40,%
tikzsetting = { draw=lightgray, line width = 0.3pt,dashed,%
dash pattern = on 15pt off 3pt},%
splittopskip=\topskip,skipbelow=\baselineskip,%
skipabove=\baselineskip,ntheorem,roundcorner=0pt,
% backgroundcolor=pagebg,font=\color{orange}\sffamily, fontcolor=white
]{examplebox}{Exemple}[section]



\newcommand\R{\mathbb{R}}
\newcommand\Z{\mathbb{Z}}
\newcommand\N{\mathbb{N}}
\newcommand\E{\mathbb{E}}
\newcommand\F{\mathcal{F}}
\newcommand\cH{\mathcal{H}}
\newcommand\V{\mathbb{V}}
\newcommand\dmo{ ^{-1} }
\newcommand\kapa{\kappa}
\newcommand\im{Im}
\newcommand\hs{\mathcal{H}}





\usepackage{soul}

\makeatletter
\newcommand*{\whiten}[1]{\llap{\textcolor{white}{{\the\SOUL@token}}\hspace{#1pt}}}
\DeclareRobustCommand*\myul{%
    \def\SOUL@everyspace{\underline{\space}\kern\z@}%
    \def\SOUL@everytoken{%
     \setbox0=\hbox{\the\SOUL@token}%
     \ifdim\dp0>\z@
        \raisebox{\dp0}{\underline{\phantom{\the\SOUL@token}}}%
        \whiten{1}\whiten{0}%
        \whiten{-1}\whiten{-2}%
        \llap{\the\SOUL@token}%
     \else
        \underline{\the\SOUL@token}%
     \fi}%
\SOUL@}
\makeatother

\newcommand*{\demp}{\fontfamily{lmtt}\selectfont}

\DeclareTextFontCommand{\textdemp}{\demp}

\begin{document}

\ifcomment
Multiline
comment
\fi
\ifcomment
\myul{Typesetting test}
% \color[rgb]{1,1,1}
$∑_i^n≠ 60º±∞π∆¬≈√j∫h≤≥µ$

$\CR \R\pro\ind\pro\gS\pro
\mqty[a&b\\c&d]$
$\pro\mathbb{P}$
$\dd{x}$

  \[
    \alpha(x)=\left\{
                \begin{array}{ll}
                  x\\
                  \frac{1}{1+e^{-kx}}\\
                  \frac{e^x-e^{-x}}{e^x+e^{-x}}
                \end{array}
              \right.
  \]

  $\expval{x}$
  
  $\chi_\rho(ghg\dmo)=\Tr(\rho_{ghg\dmo})=\Tr(\rho_g\circ\rho_h\circ\rho\dmo_g)=\Tr(\rho_h)\overset{\mbox{\scalebox{0.5}{$\Tr(AB)=\Tr(BA)$}}}{=}\chi_\rho(h)$
  	$\mathop{\oplus}_{\substack{x\in X}}$

$\mat(\rho_g)=(a_{ij}(g))_{\scriptsize \substack{1\leq i\leq d \\ 1\leq j\leq d}}$ et $\mat(\rho'_g)=(a'_{ij}(g))_{\scriptsize \substack{1\leq i'\leq d' \\ 1\leq j'\leq d'}}$



\[\int_a^b{\mathbb{R}^2}g(u, v)\dd{P_{XY}}(u, v)=\iint g(u,v) f_{XY}(u, v)\dd \lambda(u) \dd \lambda(v)\]
$$\lim_{x\to\infty} f(x)$$	
$$\iiiint_V \mu(t,u,v,w) \,dt\,du\,dv\,dw$$
$$\sum_{n=1}^{\infty} 2^{-n} = 1$$	
\begin{definition}
	Si $X$ et $Y$ sont 2 v.a. ou definit la \textsc{Covariance} entre $X$ et $Y$ comme
	$\cov(X,Y)\overset{\text{def}}{=}\E\left[(X-\E(X))(Y-\E(Y))\right]=\E(XY)-\E(X)\E(Y)$.
\end{definition}
\fi
\pagebreak

% \tableofcontents

% insert your code here
%\input{./algebra/main.tex}
%\input{./geometrie-differentielle/main.tex}
%\input{./probabilite/main.tex}
%\input{./analyse-fonctionnelle/main.tex}
% \input{./Analyse-convexe-et-dualite-en-optimisation/main.tex}
%\input{./tikz/main.tex}
%\input{./Theorie-du-distributions/main.tex}
%\input{./optimisation/mine.tex}
 \input{./modelisation/main.tex}

% yves.aubry@univ-tln.fr : algebra

\end{document}

%\input{./optimisation/mine.tex}
 % !TEX encoding = UTF-8 Unicode
% !TEX TS-program = xelatex

\documentclass[french]{report}

%\usepackage[utf8]{inputenc}
%\usepackage[T1]{fontenc}
\usepackage{babel}


\newif\ifcomment
%\commenttrue # Show comments

\usepackage{physics}
\usepackage{amssymb}


\usepackage{amsthm}
% \usepackage{thmtools}
\usepackage{mathtools}
\usepackage{amsfonts}

\usepackage{color}

\usepackage{tikz}

\usepackage{geometry}
\geometry{a5paper, margin=0.1in, right=1cm}

\usepackage{dsfont}

\usepackage{graphicx}
\graphicspath{ {images/} }

\usepackage{faktor}

\usepackage{IEEEtrantools}
\usepackage{enumerate}   
\usepackage[PostScript=dvips]{"/Users/aware/Documents/Courses/diagrams"}


\newtheorem{theorem}{Théorème}[section]
\renewcommand{\thetheorem}{\arabic{theorem}}
\newtheorem{lemme}{Lemme}[section]
\renewcommand{\thelemme}{\arabic{lemme}}
\newtheorem{proposition}{Proposition}[section]
\renewcommand{\theproposition}{\arabic{proposition}}
\newtheorem{notations}{Notations}[section]
\newtheorem{problem}{Problème}[section]
\newtheorem{corollary}{Corollaire}[theorem]
\renewcommand{\thecorollary}{\arabic{corollary}}
\newtheorem{property}{Propriété}[section]
\newtheorem{objective}{Objectif}[section]

\theoremstyle{definition}
\newtheorem{definition}{Définition}[section]
\renewcommand{\thedefinition}{\arabic{definition}}
\newtheorem{exercise}{Exercice}[chapter]
\renewcommand{\theexercise}{\arabic{exercise}}
\newtheorem{example}{Exemple}[chapter]
\renewcommand{\theexample}{\arabic{example}}
\newtheorem*{solution}{Solution}
\newtheorem*{application}{Application}
\newtheorem*{notation}{Notation}
\newtheorem*{vocabulary}{Vocabulaire}
\newtheorem*{properties}{Propriétés}



\theoremstyle{remark}
\newtheorem*{remark}{Remarque}
\newtheorem*{rappel}{Rappel}


\usepackage{etoolbox}
\AtBeginEnvironment{exercise}{\small}
\AtBeginEnvironment{example}{\small}

\usepackage{cases}
\usepackage[red]{mypack}

\usepackage[framemethod=TikZ]{mdframed}

\definecolor{bg}{rgb}{0.4,0.25,0.95}
\definecolor{pagebg}{rgb}{0,0,0.5}
\surroundwithmdframed[
   topline=false,
   rightline=false,
   bottomline=false,
   leftmargin=\parindent,
   skipabove=8pt,
   skipbelow=8pt,
   linecolor=blue,
   innerbottommargin=10pt,
   % backgroundcolor=bg,font=\color{orange}\sffamily, fontcolor=white
]{definition}

\usepackage{empheq}
\usepackage[most]{tcolorbox}

\newtcbox{\mymath}[1][]{%
    nobeforeafter, math upper, tcbox raise base,
    enhanced, colframe=blue!30!black,
    colback=red!10, boxrule=1pt,
    #1}

\usepackage{unixode}


\DeclareMathOperator{\ord}{ord}
\DeclareMathOperator{\orb}{orb}
\DeclareMathOperator{\stab}{stab}
\DeclareMathOperator{\Stab}{stab}
\DeclareMathOperator{\ppcm}{ppcm}
\DeclareMathOperator{\conj}{Conj}
\DeclareMathOperator{\End}{End}
\DeclareMathOperator{\rot}{rot}
\DeclareMathOperator{\trs}{trace}
\DeclareMathOperator{\Ind}{Ind}
\DeclareMathOperator{\mat}{Mat}
\DeclareMathOperator{\id}{Id}
\DeclareMathOperator{\vect}{vect}
\DeclareMathOperator{\img}{img}
\DeclareMathOperator{\cov}{Cov}
\DeclareMathOperator{\dist}{dist}
\DeclareMathOperator{\irr}{Irr}
\DeclareMathOperator{\image}{Im}
\DeclareMathOperator{\pd}{\partial}
\DeclareMathOperator{\epi}{epi}
\DeclareMathOperator{\Argmin}{Argmin}
\DeclareMathOperator{\dom}{dom}
\DeclareMathOperator{\proj}{proj}
\DeclareMathOperator{\ctg}{ctg}
\DeclareMathOperator{\supp}{supp}
\DeclareMathOperator{\argmin}{argmin}
\DeclareMathOperator{\mult}{mult}
\DeclareMathOperator{\ch}{ch}
\DeclareMathOperator{\sh}{sh}
\DeclareMathOperator{\rang}{rang}
\DeclareMathOperator{\diam}{diam}
\DeclareMathOperator{\Epigraphe}{Epigraphe}




\usepackage{xcolor}
\everymath{\color{blue}}
%\everymath{\color[rgb]{0,1,1}}
%\pagecolor[rgb]{0,0,0.5}


\newcommand*{\pdtest}[3][]{\ensuremath{\frac{\partial^{#1} #2}{\partial #3}}}

\newcommand*{\deffunc}[6][]{\ensuremath{
\begin{array}{rcl}
#2 : #3 &\rightarrow& #4\\
#5 &\mapsto& #6
\end{array}
}}

\newcommand{\eqcolon}{\mathrel{\resizebox{\widthof{$\mathord{=}$}}{\height}{ $\!\!=\!\!\resizebox{1.2\width}{0.8\height}{\raisebox{0.23ex}{$\mathop{:}$}}\!\!$ }}}
\newcommand{\coloneq}{\mathrel{\resizebox{\widthof{$\mathord{=}$}}{\height}{ $\!\!\resizebox{1.2\width}{0.8\height}{\raisebox{0.23ex}{$\mathop{:}$}}\!\!=\!\!$ }}}
\newcommand{\eqcolonl}{\ensuremath{\mathrel{=\!\!\mathop{:}}}}
\newcommand{\coloneql}{\ensuremath{\mathrel{\mathop{:} \!\! =}}}
\newcommand{\vc}[1]{% inline column vector
  \left(\begin{smallmatrix}#1\end{smallmatrix}\right)%
}
\newcommand{\vr}[1]{% inline row vector
  \begin{smallmatrix}(\,#1\,)\end{smallmatrix}%
}
\makeatletter
\newcommand*{\defeq}{\ =\mathrel{\rlap{%
                     \raisebox{0.3ex}{$\m@th\cdot$}}%
                     \raisebox{-0.3ex}{$\m@th\cdot$}}%
                     }
\makeatother

\newcommand{\mathcircle}[1]{% inline row vector
 \overset{\circ}{#1}
}
\newcommand{\ulim}{% low limit
 \underline{\lim}
}
\newcommand{\ssi}{% iff
\iff
}
\newcommand{\ps}[2]{
\expval{#1 | #2}
}
\newcommand{\df}[1]{
\mqty{#1}
}
\newcommand{\n}[1]{
\norm{#1}
}
\newcommand{\sys}[1]{
\left\{\smqty{#1}\right.
}


\newcommand{\eqdef}{\ensuremath{\overset{\text{def}}=}}


\def\Circlearrowright{\ensuremath{%
  \rotatebox[origin=c]{230}{$\circlearrowright$}}}

\newcommand\ct[1]{\text{\rmfamily\upshape #1}}
\newcommand\question[1]{ {\color{red} ...!? \small #1}}
\newcommand\caz[1]{\left\{\begin{array} #1 \end{array}\right.}
\newcommand\const{\text{\rmfamily\upshape const}}
\newcommand\toP{ \overset{\pro}{\to}}
\newcommand\toPP{ \overset{\text{PP}}{\to}}
\newcommand{\oeq}{\mathrel{\text{\textcircled{$=$}}}}





\usepackage{xcolor}
% \usepackage[normalem]{ulem}
\usepackage{lipsum}
\makeatletter
% \newcommand\colorwave[1][blue]{\bgroup \markoverwith{\lower3.5\p@\hbox{\sixly \textcolor{#1}{\char58}}}\ULon}
%\font\sixly=lasy6 % does not re-load if already loaded, so no memory problem.

\newmdtheoremenv[
linewidth= 1pt,linecolor= blue,%
leftmargin=20,rightmargin=20,innertopmargin=0pt, innerrightmargin=40,%
tikzsetting = { draw=lightgray, line width = 0.3pt,dashed,%
dash pattern = on 15pt off 3pt},%
splittopskip=\topskip,skipbelow=\baselineskip,%
skipabove=\baselineskip,ntheorem,roundcorner=0pt,
% backgroundcolor=pagebg,font=\color{orange}\sffamily, fontcolor=white
]{examplebox}{Exemple}[section]



\newcommand\R{\mathbb{R}}
\newcommand\Z{\mathbb{Z}}
\newcommand\N{\mathbb{N}}
\newcommand\E{\mathbb{E}}
\newcommand\F{\mathcal{F}}
\newcommand\cH{\mathcal{H}}
\newcommand\V{\mathbb{V}}
\newcommand\dmo{ ^{-1} }
\newcommand\kapa{\kappa}
\newcommand\im{Im}
\newcommand\hs{\mathcal{H}}





\usepackage{soul}

\makeatletter
\newcommand*{\whiten}[1]{\llap{\textcolor{white}{{\the\SOUL@token}}\hspace{#1pt}}}
\DeclareRobustCommand*\myul{%
    \def\SOUL@everyspace{\underline{\space}\kern\z@}%
    \def\SOUL@everytoken{%
     \setbox0=\hbox{\the\SOUL@token}%
     \ifdim\dp0>\z@
        \raisebox{\dp0}{\underline{\phantom{\the\SOUL@token}}}%
        \whiten{1}\whiten{0}%
        \whiten{-1}\whiten{-2}%
        \llap{\the\SOUL@token}%
     \else
        \underline{\the\SOUL@token}%
     \fi}%
\SOUL@}
\makeatother

\newcommand*{\demp}{\fontfamily{lmtt}\selectfont}

\DeclareTextFontCommand{\textdemp}{\demp}

\begin{document}

\ifcomment
Multiline
comment
\fi
\ifcomment
\myul{Typesetting test}
% \color[rgb]{1,1,1}
$∑_i^n≠ 60º±∞π∆¬≈√j∫h≤≥µ$

$\CR \R\pro\ind\pro\gS\pro
\mqty[a&b\\c&d]$
$\pro\mathbb{P}$
$\dd{x}$

  \[
    \alpha(x)=\left\{
                \begin{array}{ll}
                  x\\
                  \frac{1}{1+e^{-kx}}\\
                  \frac{e^x-e^{-x}}{e^x+e^{-x}}
                \end{array}
              \right.
  \]

  $\expval{x}$
  
  $\chi_\rho(ghg\dmo)=\Tr(\rho_{ghg\dmo})=\Tr(\rho_g\circ\rho_h\circ\rho\dmo_g)=\Tr(\rho_h)\overset{\mbox{\scalebox{0.5}{$\Tr(AB)=\Tr(BA)$}}}{=}\chi_\rho(h)$
  	$\mathop{\oplus}_{\substack{x\in X}}$

$\mat(\rho_g)=(a_{ij}(g))_{\scriptsize \substack{1\leq i\leq d \\ 1\leq j\leq d}}$ et $\mat(\rho'_g)=(a'_{ij}(g))_{\scriptsize \substack{1\leq i'\leq d' \\ 1\leq j'\leq d'}}$



\[\int_a^b{\mathbb{R}^2}g(u, v)\dd{P_{XY}}(u, v)=\iint g(u,v) f_{XY}(u, v)\dd \lambda(u) \dd \lambda(v)\]
$$\lim_{x\to\infty} f(x)$$	
$$\iiiint_V \mu(t,u,v,w) \,dt\,du\,dv\,dw$$
$$\sum_{n=1}^{\infty} 2^{-n} = 1$$	
\begin{definition}
	Si $X$ et $Y$ sont 2 v.a. ou definit la \textsc{Covariance} entre $X$ et $Y$ comme
	$\cov(X,Y)\overset{\text{def}}{=}\E\left[(X-\E(X))(Y-\E(Y))\right]=\E(XY)-\E(X)\E(Y)$.
\end{definition}
\fi
\pagebreak

% \tableofcontents

% insert your code here
%\input{./algebra/main.tex}
%\input{./geometrie-differentielle/main.tex}
%\input{./probabilite/main.tex}
%\input{./analyse-fonctionnelle/main.tex}
% \input{./Analyse-convexe-et-dualite-en-optimisation/main.tex}
%\input{./tikz/main.tex}
%\input{./Theorie-du-distributions/main.tex}
%\input{./optimisation/mine.tex}
 \input{./modelisation/main.tex}

% yves.aubry@univ-tln.fr : algebra

\end{document}


% yves.aubry@univ-tln.fr : algebra

\end{document}

%\input{./optimisation/mine.tex}
 % !TEX encoding = UTF-8 Unicode
% !TEX TS-program = xelatex

\documentclass[french]{report}

%\usepackage[utf8]{inputenc}
%\usepackage[T1]{fontenc}
\usepackage{babel}


\newif\ifcomment
%\commenttrue # Show comments

\usepackage{physics}
\usepackage{amssymb}


\usepackage{amsthm}
% \usepackage{thmtools}
\usepackage{mathtools}
\usepackage{amsfonts}

\usepackage{color}

\usepackage{tikz}

\usepackage{geometry}
\geometry{a5paper, margin=0.1in, right=1cm}

\usepackage{dsfont}

\usepackage{graphicx}
\graphicspath{ {images/} }

\usepackage{faktor}

\usepackage{IEEEtrantools}
\usepackage{enumerate}   
\usepackage[PostScript=dvips]{"/Users/aware/Documents/Courses/diagrams"}


\newtheorem{theorem}{Théorème}[section]
\renewcommand{\thetheorem}{\arabic{theorem}}
\newtheorem{lemme}{Lemme}[section]
\renewcommand{\thelemme}{\arabic{lemme}}
\newtheorem{proposition}{Proposition}[section]
\renewcommand{\theproposition}{\arabic{proposition}}
\newtheorem{notations}{Notations}[section]
\newtheorem{problem}{Problème}[section]
\newtheorem{corollary}{Corollaire}[theorem]
\renewcommand{\thecorollary}{\arabic{corollary}}
\newtheorem{property}{Propriété}[section]
\newtheorem{objective}{Objectif}[section]

\theoremstyle{definition}
\newtheorem{definition}{Définition}[section]
\renewcommand{\thedefinition}{\arabic{definition}}
\newtheorem{exercise}{Exercice}[chapter]
\renewcommand{\theexercise}{\arabic{exercise}}
\newtheorem{example}{Exemple}[chapter]
\renewcommand{\theexample}{\arabic{example}}
\newtheorem*{solution}{Solution}
\newtheorem*{application}{Application}
\newtheorem*{notation}{Notation}
\newtheorem*{vocabulary}{Vocabulaire}
\newtheorem*{properties}{Propriétés}



\theoremstyle{remark}
\newtheorem*{remark}{Remarque}
\newtheorem*{rappel}{Rappel}


\usepackage{etoolbox}
\AtBeginEnvironment{exercise}{\small}
\AtBeginEnvironment{example}{\small}

\usepackage{cases}
\usepackage[red]{mypack}

\usepackage[framemethod=TikZ]{mdframed}

\definecolor{bg}{rgb}{0.4,0.25,0.95}
\definecolor{pagebg}{rgb}{0,0,0.5}
\surroundwithmdframed[
   topline=false,
   rightline=false,
   bottomline=false,
   leftmargin=\parindent,
   skipabove=8pt,
   skipbelow=8pt,
   linecolor=blue,
   innerbottommargin=10pt,
   % backgroundcolor=bg,font=\color{orange}\sffamily, fontcolor=white
]{definition}

\usepackage{empheq}
\usepackage[most]{tcolorbox}

\newtcbox{\mymath}[1][]{%
    nobeforeafter, math upper, tcbox raise base,
    enhanced, colframe=blue!30!black,
    colback=red!10, boxrule=1pt,
    #1}

\usepackage{unixode}


\DeclareMathOperator{\ord}{ord}
\DeclareMathOperator{\orb}{orb}
\DeclareMathOperator{\stab}{stab}
\DeclareMathOperator{\Stab}{stab}
\DeclareMathOperator{\ppcm}{ppcm}
\DeclareMathOperator{\conj}{Conj}
\DeclareMathOperator{\End}{End}
\DeclareMathOperator{\rot}{rot}
\DeclareMathOperator{\trs}{trace}
\DeclareMathOperator{\Ind}{Ind}
\DeclareMathOperator{\mat}{Mat}
\DeclareMathOperator{\id}{Id}
\DeclareMathOperator{\vect}{vect}
\DeclareMathOperator{\img}{img}
\DeclareMathOperator{\cov}{Cov}
\DeclareMathOperator{\dist}{dist}
\DeclareMathOperator{\irr}{Irr}
\DeclareMathOperator{\image}{Im}
\DeclareMathOperator{\pd}{\partial}
\DeclareMathOperator{\epi}{epi}
\DeclareMathOperator{\Argmin}{Argmin}
\DeclareMathOperator{\dom}{dom}
\DeclareMathOperator{\proj}{proj}
\DeclareMathOperator{\ctg}{ctg}
\DeclareMathOperator{\supp}{supp}
\DeclareMathOperator{\argmin}{argmin}
\DeclareMathOperator{\mult}{mult}
\DeclareMathOperator{\ch}{ch}
\DeclareMathOperator{\sh}{sh}
\DeclareMathOperator{\rang}{rang}
\DeclareMathOperator{\diam}{diam}
\DeclareMathOperator{\Epigraphe}{Epigraphe}




\usepackage{xcolor}
\everymath{\color{blue}}
%\everymath{\color[rgb]{0,1,1}}
%\pagecolor[rgb]{0,0,0.5}


\newcommand*{\pdtest}[3][]{\ensuremath{\frac{\partial^{#1} #2}{\partial #3}}}

\newcommand*{\deffunc}[6][]{\ensuremath{
\begin{array}{rcl}
#2 : #3 &\rightarrow& #4\\
#5 &\mapsto& #6
\end{array}
}}

\newcommand{\eqcolon}{\mathrel{\resizebox{\widthof{$\mathord{=}$}}{\height}{ $\!\!=\!\!\resizebox{1.2\width}{0.8\height}{\raisebox{0.23ex}{$\mathop{:}$}}\!\!$ }}}
\newcommand{\coloneq}{\mathrel{\resizebox{\widthof{$\mathord{=}$}}{\height}{ $\!\!\resizebox{1.2\width}{0.8\height}{\raisebox{0.23ex}{$\mathop{:}$}}\!\!=\!\!$ }}}
\newcommand{\eqcolonl}{\ensuremath{\mathrel{=\!\!\mathop{:}}}}
\newcommand{\coloneql}{\ensuremath{\mathrel{\mathop{:} \!\! =}}}
\newcommand{\vc}[1]{% inline column vector
  \left(\begin{smallmatrix}#1\end{smallmatrix}\right)%
}
\newcommand{\vr}[1]{% inline row vector
  \begin{smallmatrix}(\,#1\,)\end{smallmatrix}%
}
\makeatletter
\newcommand*{\defeq}{\ =\mathrel{\rlap{%
                     \raisebox{0.3ex}{$\m@th\cdot$}}%
                     \raisebox{-0.3ex}{$\m@th\cdot$}}%
                     }
\makeatother

\newcommand{\mathcircle}[1]{% inline row vector
 \overset{\circ}{#1}
}
\newcommand{\ulim}{% low limit
 \underline{\lim}
}
\newcommand{\ssi}{% iff
\iff
}
\newcommand{\ps}[2]{
\expval{#1 | #2}
}
\newcommand{\df}[1]{
\mqty{#1}
}
\newcommand{\n}[1]{
\norm{#1}
}
\newcommand{\sys}[1]{
\left\{\smqty{#1}\right.
}


\newcommand{\eqdef}{\ensuremath{\overset{\text{def}}=}}


\def\Circlearrowright{\ensuremath{%
  \rotatebox[origin=c]{230}{$\circlearrowright$}}}

\newcommand\ct[1]{\text{\rmfamily\upshape #1}}
\newcommand\question[1]{ {\color{red} ...!? \small #1}}
\newcommand\caz[1]{\left\{\begin{array} #1 \end{array}\right.}
\newcommand\const{\text{\rmfamily\upshape const}}
\newcommand\toP{ \overset{\pro}{\to}}
\newcommand\toPP{ \overset{\text{PP}}{\to}}
\newcommand{\oeq}{\mathrel{\text{\textcircled{$=$}}}}





\usepackage{xcolor}
% \usepackage[normalem]{ulem}
\usepackage{lipsum}
\makeatletter
% \newcommand\colorwave[1][blue]{\bgroup \markoverwith{\lower3.5\p@\hbox{\sixly \textcolor{#1}{\char58}}}\ULon}
%\font\sixly=lasy6 % does not re-load if already loaded, so no memory problem.

\newmdtheoremenv[
linewidth= 1pt,linecolor= blue,%
leftmargin=20,rightmargin=20,innertopmargin=0pt, innerrightmargin=40,%
tikzsetting = { draw=lightgray, line width = 0.3pt,dashed,%
dash pattern = on 15pt off 3pt},%
splittopskip=\topskip,skipbelow=\baselineskip,%
skipabove=\baselineskip,ntheorem,roundcorner=0pt,
% backgroundcolor=pagebg,font=\color{orange}\sffamily, fontcolor=white
]{examplebox}{Exemple}[section]



\newcommand\R{\mathbb{R}}
\newcommand\Z{\mathbb{Z}}
\newcommand\N{\mathbb{N}}
\newcommand\E{\mathbb{E}}
\newcommand\F{\mathcal{F}}
\newcommand\cH{\mathcal{H}}
\newcommand\V{\mathbb{V}}
\newcommand\dmo{ ^{-1} }
\newcommand\kapa{\kappa}
\newcommand\im{Im}
\newcommand\hs{\mathcal{H}}





\usepackage{soul}

\makeatletter
\newcommand*{\whiten}[1]{\llap{\textcolor{white}{{\the\SOUL@token}}\hspace{#1pt}}}
\DeclareRobustCommand*\myul{%
    \def\SOUL@everyspace{\underline{\space}\kern\z@}%
    \def\SOUL@everytoken{%
     \setbox0=\hbox{\the\SOUL@token}%
     \ifdim\dp0>\z@
        \raisebox{\dp0}{\underline{\phantom{\the\SOUL@token}}}%
        \whiten{1}\whiten{0}%
        \whiten{-1}\whiten{-2}%
        \llap{\the\SOUL@token}%
     \else
        \underline{\the\SOUL@token}%
     \fi}%
\SOUL@}
\makeatother

\newcommand*{\demp}{\fontfamily{lmtt}\selectfont}

\DeclareTextFontCommand{\textdemp}{\demp}

\begin{document}

\ifcomment
Multiline
comment
\fi
\ifcomment
\myul{Typesetting test}
% \color[rgb]{1,1,1}
$∑_i^n≠ 60º±∞π∆¬≈√j∫h≤≥µ$

$\CR \R\pro\ind\pro\gS\pro
\mqty[a&b\\c&d]$
$\pro\mathbb{P}$
$\dd{x}$

  \[
    \alpha(x)=\left\{
                \begin{array}{ll}
                  x\\
                  \frac{1}{1+e^{-kx}}\\
                  \frac{e^x-e^{-x}}{e^x+e^{-x}}
                \end{array}
              \right.
  \]

  $\expval{x}$
  
  $\chi_\rho(ghg\dmo)=\Tr(\rho_{ghg\dmo})=\Tr(\rho_g\circ\rho_h\circ\rho\dmo_g)=\Tr(\rho_h)\overset{\mbox{\scalebox{0.5}{$\Tr(AB)=\Tr(BA)$}}}{=}\chi_\rho(h)$
  	$\mathop{\oplus}_{\substack{x\in X}}$

$\mat(\rho_g)=(a_{ij}(g))_{\scriptsize \substack{1\leq i\leq d \\ 1\leq j\leq d}}$ et $\mat(\rho'_g)=(a'_{ij}(g))_{\scriptsize \substack{1\leq i'\leq d' \\ 1\leq j'\leq d'}}$



\[\int_a^b{\mathbb{R}^2}g(u, v)\dd{P_{XY}}(u, v)=\iint g(u,v) f_{XY}(u, v)\dd \lambda(u) \dd \lambda(v)\]
$$\lim_{x\to\infty} f(x)$$	
$$\iiiint_V \mu(t,u,v,w) \,dt\,du\,dv\,dw$$
$$\sum_{n=1}^{\infty} 2^{-n} = 1$$	
\begin{definition}
	Si $X$ et $Y$ sont 2 v.a. ou definit la \textsc{Covariance} entre $X$ et $Y$ comme
	$\cov(X,Y)\overset{\text{def}}{=}\E\left[(X-\E(X))(Y-\E(Y))\right]=\E(XY)-\E(X)\E(Y)$.
\end{definition}
\fi
\pagebreak

% \tableofcontents

% insert your code here
%% !TEX encoding = UTF-8 Unicode
% !TEX TS-program = xelatex

\documentclass[french]{report}

%\usepackage[utf8]{inputenc}
%\usepackage[T1]{fontenc}
\usepackage{babel}


\newif\ifcomment
%\commenttrue # Show comments

\usepackage{physics}
\usepackage{amssymb}


\usepackage{amsthm}
% \usepackage{thmtools}
\usepackage{mathtools}
\usepackage{amsfonts}

\usepackage{color}

\usepackage{tikz}

\usepackage{geometry}
\geometry{a5paper, margin=0.1in, right=1cm}

\usepackage{dsfont}

\usepackage{graphicx}
\graphicspath{ {images/} }

\usepackage{faktor}

\usepackage{IEEEtrantools}
\usepackage{enumerate}   
\usepackage[PostScript=dvips]{"/Users/aware/Documents/Courses/diagrams"}


\newtheorem{theorem}{Théorème}[section]
\renewcommand{\thetheorem}{\arabic{theorem}}
\newtheorem{lemme}{Lemme}[section]
\renewcommand{\thelemme}{\arabic{lemme}}
\newtheorem{proposition}{Proposition}[section]
\renewcommand{\theproposition}{\arabic{proposition}}
\newtheorem{notations}{Notations}[section]
\newtheorem{problem}{Problème}[section]
\newtheorem{corollary}{Corollaire}[theorem]
\renewcommand{\thecorollary}{\arabic{corollary}}
\newtheorem{property}{Propriété}[section]
\newtheorem{objective}{Objectif}[section]

\theoremstyle{definition}
\newtheorem{definition}{Définition}[section]
\renewcommand{\thedefinition}{\arabic{definition}}
\newtheorem{exercise}{Exercice}[chapter]
\renewcommand{\theexercise}{\arabic{exercise}}
\newtheorem{example}{Exemple}[chapter]
\renewcommand{\theexample}{\arabic{example}}
\newtheorem*{solution}{Solution}
\newtheorem*{application}{Application}
\newtheorem*{notation}{Notation}
\newtheorem*{vocabulary}{Vocabulaire}
\newtheorem*{properties}{Propriétés}



\theoremstyle{remark}
\newtheorem*{remark}{Remarque}
\newtheorem*{rappel}{Rappel}


\usepackage{etoolbox}
\AtBeginEnvironment{exercise}{\small}
\AtBeginEnvironment{example}{\small}

\usepackage{cases}
\usepackage[red]{mypack}

\usepackage[framemethod=TikZ]{mdframed}

\definecolor{bg}{rgb}{0.4,0.25,0.95}
\definecolor{pagebg}{rgb}{0,0,0.5}
\surroundwithmdframed[
   topline=false,
   rightline=false,
   bottomline=false,
   leftmargin=\parindent,
   skipabove=8pt,
   skipbelow=8pt,
   linecolor=blue,
   innerbottommargin=10pt,
   % backgroundcolor=bg,font=\color{orange}\sffamily, fontcolor=white
]{definition}

\usepackage{empheq}
\usepackage[most]{tcolorbox}

\newtcbox{\mymath}[1][]{%
    nobeforeafter, math upper, tcbox raise base,
    enhanced, colframe=blue!30!black,
    colback=red!10, boxrule=1pt,
    #1}

\usepackage{unixode}


\DeclareMathOperator{\ord}{ord}
\DeclareMathOperator{\orb}{orb}
\DeclareMathOperator{\stab}{stab}
\DeclareMathOperator{\Stab}{stab}
\DeclareMathOperator{\ppcm}{ppcm}
\DeclareMathOperator{\conj}{Conj}
\DeclareMathOperator{\End}{End}
\DeclareMathOperator{\rot}{rot}
\DeclareMathOperator{\trs}{trace}
\DeclareMathOperator{\Ind}{Ind}
\DeclareMathOperator{\mat}{Mat}
\DeclareMathOperator{\id}{Id}
\DeclareMathOperator{\vect}{vect}
\DeclareMathOperator{\img}{img}
\DeclareMathOperator{\cov}{Cov}
\DeclareMathOperator{\dist}{dist}
\DeclareMathOperator{\irr}{Irr}
\DeclareMathOperator{\image}{Im}
\DeclareMathOperator{\pd}{\partial}
\DeclareMathOperator{\epi}{epi}
\DeclareMathOperator{\Argmin}{Argmin}
\DeclareMathOperator{\dom}{dom}
\DeclareMathOperator{\proj}{proj}
\DeclareMathOperator{\ctg}{ctg}
\DeclareMathOperator{\supp}{supp}
\DeclareMathOperator{\argmin}{argmin}
\DeclareMathOperator{\mult}{mult}
\DeclareMathOperator{\ch}{ch}
\DeclareMathOperator{\sh}{sh}
\DeclareMathOperator{\rang}{rang}
\DeclareMathOperator{\diam}{diam}
\DeclareMathOperator{\Epigraphe}{Epigraphe}




\usepackage{xcolor}
\everymath{\color{blue}}
%\everymath{\color[rgb]{0,1,1}}
%\pagecolor[rgb]{0,0,0.5}


\newcommand*{\pdtest}[3][]{\ensuremath{\frac{\partial^{#1} #2}{\partial #3}}}

\newcommand*{\deffunc}[6][]{\ensuremath{
\begin{array}{rcl}
#2 : #3 &\rightarrow& #4\\
#5 &\mapsto& #6
\end{array}
}}

\newcommand{\eqcolon}{\mathrel{\resizebox{\widthof{$\mathord{=}$}}{\height}{ $\!\!=\!\!\resizebox{1.2\width}{0.8\height}{\raisebox{0.23ex}{$\mathop{:}$}}\!\!$ }}}
\newcommand{\coloneq}{\mathrel{\resizebox{\widthof{$\mathord{=}$}}{\height}{ $\!\!\resizebox{1.2\width}{0.8\height}{\raisebox{0.23ex}{$\mathop{:}$}}\!\!=\!\!$ }}}
\newcommand{\eqcolonl}{\ensuremath{\mathrel{=\!\!\mathop{:}}}}
\newcommand{\coloneql}{\ensuremath{\mathrel{\mathop{:} \!\! =}}}
\newcommand{\vc}[1]{% inline column vector
  \left(\begin{smallmatrix}#1\end{smallmatrix}\right)%
}
\newcommand{\vr}[1]{% inline row vector
  \begin{smallmatrix}(\,#1\,)\end{smallmatrix}%
}
\makeatletter
\newcommand*{\defeq}{\ =\mathrel{\rlap{%
                     \raisebox{0.3ex}{$\m@th\cdot$}}%
                     \raisebox{-0.3ex}{$\m@th\cdot$}}%
                     }
\makeatother

\newcommand{\mathcircle}[1]{% inline row vector
 \overset{\circ}{#1}
}
\newcommand{\ulim}{% low limit
 \underline{\lim}
}
\newcommand{\ssi}{% iff
\iff
}
\newcommand{\ps}[2]{
\expval{#1 | #2}
}
\newcommand{\df}[1]{
\mqty{#1}
}
\newcommand{\n}[1]{
\norm{#1}
}
\newcommand{\sys}[1]{
\left\{\smqty{#1}\right.
}


\newcommand{\eqdef}{\ensuremath{\overset{\text{def}}=}}


\def\Circlearrowright{\ensuremath{%
  \rotatebox[origin=c]{230}{$\circlearrowright$}}}

\newcommand\ct[1]{\text{\rmfamily\upshape #1}}
\newcommand\question[1]{ {\color{red} ...!? \small #1}}
\newcommand\caz[1]{\left\{\begin{array} #1 \end{array}\right.}
\newcommand\const{\text{\rmfamily\upshape const}}
\newcommand\toP{ \overset{\pro}{\to}}
\newcommand\toPP{ \overset{\text{PP}}{\to}}
\newcommand{\oeq}{\mathrel{\text{\textcircled{$=$}}}}





\usepackage{xcolor}
% \usepackage[normalem]{ulem}
\usepackage{lipsum}
\makeatletter
% \newcommand\colorwave[1][blue]{\bgroup \markoverwith{\lower3.5\p@\hbox{\sixly \textcolor{#1}{\char58}}}\ULon}
%\font\sixly=lasy6 % does not re-load if already loaded, so no memory problem.

\newmdtheoremenv[
linewidth= 1pt,linecolor= blue,%
leftmargin=20,rightmargin=20,innertopmargin=0pt, innerrightmargin=40,%
tikzsetting = { draw=lightgray, line width = 0.3pt,dashed,%
dash pattern = on 15pt off 3pt},%
splittopskip=\topskip,skipbelow=\baselineskip,%
skipabove=\baselineskip,ntheorem,roundcorner=0pt,
% backgroundcolor=pagebg,font=\color{orange}\sffamily, fontcolor=white
]{examplebox}{Exemple}[section]



\newcommand\R{\mathbb{R}}
\newcommand\Z{\mathbb{Z}}
\newcommand\N{\mathbb{N}}
\newcommand\E{\mathbb{E}}
\newcommand\F{\mathcal{F}}
\newcommand\cH{\mathcal{H}}
\newcommand\V{\mathbb{V}}
\newcommand\dmo{ ^{-1} }
\newcommand\kapa{\kappa}
\newcommand\im{Im}
\newcommand\hs{\mathcal{H}}





\usepackage{soul}

\makeatletter
\newcommand*{\whiten}[1]{\llap{\textcolor{white}{{\the\SOUL@token}}\hspace{#1pt}}}
\DeclareRobustCommand*\myul{%
    \def\SOUL@everyspace{\underline{\space}\kern\z@}%
    \def\SOUL@everytoken{%
     \setbox0=\hbox{\the\SOUL@token}%
     \ifdim\dp0>\z@
        \raisebox{\dp0}{\underline{\phantom{\the\SOUL@token}}}%
        \whiten{1}\whiten{0}%
        \whiten{-1}\whiten{-2}%
        \llap{\the\SOUL@token}%
     \else
        \underline{\the\SOUL@token}%
     \fi}%
\SOUL@}
\makeatother

\newcommand*{\demp}{\fontfamily{lmtt}\selectfont}

\DeclareTextFontCommand{\textdemp}{\demp}

\begin{document}

\ifcomment
Multiline
comment
\fi
\ifcomment
\myul{Typesetting test}
% \color[rgb]{1,1,1}
$∑_i^n≠ 60º±∞π∆¬≈√j∫h≤≥µ$

$\CR \R\pro\ind\pro\gS\pro
\mqty[a&b\\c&d]$
$\pro\mathbb{P}$
$\dd{x}$

  \[
    \alpha(x)=\left\{
                \begin{array}{ll}
                  x\\
                  \frac{1}{1+e^{-kx}}\\
                  \frac{e^x-e^{-x}}{e^x+e^{-x}}
                \end{array}
              \right.
  \]

  $\expval{x}$
  
  $\chi_\rho(ghg\dmo)=\Tr(\rho_{ghg\dmo})=\Tr(\rho_g\circ\rho_h\circ\rho\dmo_g)=\Tr(\rho_h)\overset{\mbox{\scalebox{0.5}{$\Tr(AB)=\Tr(BA)$}}}{=}\chi_\rho(h)$
  	$\mathop{\oplus}_{\substack{x\in X}}$

$\mat(\rho_g)=(a_{ij}(g))_{\scriptsize \substack{1\leq i\leq d \\ 1\leq j\leq d}}$ et $\mat(\rho'_g)=(a'_{ij}(g))_{\scriptsize \substack{1\leq i'\leq d' \\ 1\leq j'\leq d'}}$



\[\int_a^b{\mathbb{R}^2}g(u, v)\dd{P_{XY}}(u, v)=\iint g(u,v) f_{XY}(u, v)\dd \lambda(u) \dd \lambda(v)\]
$$\lim_{x\to\infty} f(x)$$	
$$\iiiint_V \mu(t,u,v,w) \,dt\,du\,dv\,dw$$
$$\sum_{n=1}^{\infty} 2^{-n} = 1$$	
\begin{definition}
	Si $X$ et $Y$ sont 2 v.a. ou definit la \textsc{Covariance} entre $X$ et $Y$ comme
	$\cov(X,Y)\overset{\text{def}}{=}\E\left[(X-\E(X))(Y-\E(Y))\right]=\E(XY)-\E(X)\E(Y)$.
\end{definition}
\fi
\pagebreak

% \tableofcontents

% insert your code here
%\input{./algebra/main.tex}
%\input{./geometrie-differentielle/main.tex}
%\input{./probabilite/main.tex}
%\input{./analyse-fonctionnelle/main.tex}
% \input{./Analyse-convexe-et-dualite-en-optimisation/main.tex}
%\input{./tikz/main.tex}
%\input{./Theorie-du-distributions/main.tex}
%\input{./optimisation/mine.tex}
 \input{./modelisation/main.tex}

% yves.aubry@univ-tln.fr : algebra

\end{document}

%% !TEX encoding = UTF-8 Unicode
% !TEX TS-program = xelatex

\documentclass[french]{report}

%\usepackage[utf8]{inputenc}
%\usepackage[T1]{fontenc}
\usepackage{babel}


\newif\ifcomment
%\commenttrue # Show comments

\usepackage{physics}
\usepackage{amssymb}


\usepackage{amsthm}
% \usepackage{thmtools}
\usepackage{mathtools}
\usepackage{amsfonts}

\usepackage{color}

\usepackage{tikz}

\usepackage{geometry}
\geometry{a5paper, margin=0.1in, right=1cm}

\usepackage{dsfont}

\usepackage{graphicx}
\graphicspath{ {images/} }

\usepackage{faktor}

\usepackage{IEEEtrantools}
\usepackage{enumerate}   
\usepackage[PostScript=dvips]{"/Users/aware/Documents/Courses/diagrams"}


\newtheorem{theorem}{Théorème}[section]
\renewcommand{\thetheorem}{\arabic{theorem}}
\newtheorem{lemme}{Lemme}[section]
\renewcommand{\thelemme}{\arabic{lemme}}
\newtheorem{proposition}{Proposition}[section]
\renewcommand{\theproposition}{\arabic{proposition}}
\newtheorem{notations}{Notations}[section]
\newtheorem{problem}{Problème}[section]
\newtheorem{corollary}{Corollaire}[theorem]
\renewcommand{\thecorollary}{\arabic{corollary}}
\newtheorem{property}{Propriété}[section]
\newtheorem{objective}{Objectif}[section]

\theoremstyle{definition}
\newtheorem{definition}{Définition}[section]
\renewcommand{\thedefinition}{\arabic{definition}}
\newtheorem{exercise}{Exercice}[chapter]
\renewcommand{\theexercise}{\arabic{exercise}}
\newtheorem{example}{Exemple}[chapter]
\renewcommand{\theexample}{\arabic{example}}
\newtheorem*{solution}{Solution}
\newtheorem*{application}{Application}
\newtheorem*{notation}{Notation}
\newtheorem*{vocabulary}{Vocabulaire}
\newtheorem*{properties}{Propriétés}



\theoremstyle{remark}
\newtheorem*{remark}{Remarque}
\newtheorem*{rappel}{Rappel}


\usepackage{etoolbox}
\AtBeginEnvironment{exercise}{\small}
\AtBeginEnvironment{example}{\small}

\usepackage{cases}
\usepackage[red]{mypack}

\usepackage[framemethod=TikZ]{mdframed}

\definecolor{bg}{rgb}{0.4,0.25,0.95}
\definecolor{pagebg}{rgb}{0,0,0.5}
\surroundwithmdframed[
   topline=false,
   rightline=false,
   bottomline=false,
   leftmargin=\parindent,
   skipabove=8pt,
   skipbelow=8pt,
   linecolor=blue,
   innerbottommargin=10pt,
   % backgroundcolor=bg,font=\color{orange}\sffamily, fontcolor=white
]{definition}

\usepackage{empheq}
\usepackage[most]{tcolorbox}

\newtcbox{\mymath}[1][]{%
    nobeforeafter, math upper, tcbox raise base,
    enhanced, colframe=blue!30!black,
    colback=red!10, boxrule=1pt,
    #1}

\usepackage{unixode}


\DeclareMathOperator{\ord}{ord}
\DeclareMathOperator{\orb}{orb}
\DeclareMathOperator{\stab}{stab}
\DeclareMathOperator{\Stab}{stab}
\DeclareMathOperator{\ppcm}{ppcm}
\DeclareMathOperator{\conj}{Conj}
\DeclareMathOperator{\End}{End}
\DeclareMathOperator{\rot}{rot}
\DeclareMathOperator{\trs}{trace}
\DeclareMathOperator{\Ind}{Ind}
\DeclareMathOperator{\mat}{Mat}
\DeclareMathOperator{\id}{Id}
\DeclareMathOperator{\vect}{vect}
\DeclareMathOperator{\img}{img}
\DeclareMathOperator{\cov}{Cov}
\DeclareMathOperator{\dist}{dist}
\DeclareMathOperator{\irr}{Irr}
\DeclareMathOperator{\image}{Im}
\DeclareMathOperator{\pd}{\partial}
\DeclareMathOperator{\epi}{epi}
\DeclareMathOperator{\Argmin}{Argmin}
\DeclareMathOperator{\dom}{dom}
\DeclareMathOperator{\proj}{proj}
\DeclareMathOperator{\ctg}{ctg}
\DeclareMathOperator{\supp}{supp}
\DeclareMathOperator{\argmin}{argmin}
\DeclareMathOperator{\mult}{mult}
\DeclareMathOperator{\ch}{ch}
\DeclareMathOperator{\sh}{sh}
\DeclareMathOperator{\rang}{rang}
\DeclareMathOperator{\diam}{diam}
\DeclareMathOperator{\Epigraphe}{Epigraphe}




\usepackage{xcolor}
\everymath{\color{blue}}
%\everymath{\color[rgb]{0,1,1}}
%\pagecolor[rgb]{0,0,0.5}


\newcommand*{\pdtest}[3][]{\ensuremath{\frac{\partial^{#1} #2}{\partial #3}}}

\newcommand*{\deffunc}[6][]{\ensuremath{
\begin{array}{rcl}
#2 : #3 &\rightarrow& #4\\
#5 &\mapsto& #6
\end{array}
}}

\newcommand{\eqcolon}{\mathrel{\resizebox{\widthof{$\mathord{=}$}}{\height}{ $\!\!=\!\!\resizebox{1.2\width}{0.8\height}{\raisebox{0.23ex}{$\mathop{:}$}}\!\!$ }}}
\newcommand{\coloneq}{\mathrel{\resizebox{\widthof{$\mathord{=}$}}{\height}{ $\!\!\resizebox{1.2\width}{0.8\height}{\raisebox{0.23ex}{$\mathop{:}$}}\!\!=\!\!$ }}}
\newcommand{\eqcolonl}{\ensuremath{\mathrel{=\!\!\mathop{:}}}}
\newcommand{\coloneql}{\ensuremath{\mathrel{\mathop{:} \!\! =}}}
\newcommand{\vc}[1]{% inline column vector
  \left(\begin{smallmatrix}#1\end{smallmatrix}\right)%
}
\newcommand{\vr}[1]{% inline row vector
  \begin{smallmatrix}(\,#1\,)\end{smallmatrix}%
}
\makeatletter
\newcommand*{\defeq}{\ =\mathrel{\rlap{%
                     \raisebox{0.3ex}{$\m@th\cdot$}}%
                     \raisebox{-0.3ex}{$\m@th\cdot$}}%
                     }
\makeatother

\newcommand{\mathcircle}[1]{% inline row vector
 \overset{\circ}{#1}
}
\newcommand{\ulim}{% low limit
 \underline{\lim}
}
\newcommand{\ssi}{% iff
\iff
}
\newcommand{\ps}[2]{
\expval{#1 | #2}
}
\newcommand{\df}[1]{
\mqty{#1}
}
\newcommand{\n}[1]{
\norm{#1}
}
\newcommand{\sys}[1]{
\left\{\smqty{#1}\right.
}


\newcommand{\eqdef}{\ensuremath{\overset{\text{def}}=}}


\def\Circlearrowright{\ensuremath{%
  \rotatebox[origin=c]{230}{$\circlearrowright$}}}

\newcommand\ct[1]{\text{\rmfamily\upshape #1}}
\newcommand\question[1]{ {\color{red} ...!? \small #1}}
\newcommand\caz[1]{\left\{\begin{array} #1 \end{array}\right.}
\newcommand\const{\text{\rmfamily\upshape const}}
\newcommand\toP{ \overset{\pro}{\to}}
\newcommand\toPP{ \overset{\text{PP}}{\to}}
\newcommand{\oeq}{\mathrel{\text{\textcircled{$=$}}}}





\usepackage{xcolor}
% \usepackage[normalem]{ulem}
\usepackage{lipsum}
\makeatletter
% \newcommand\colorwave[1][blue]{\bgroup \markoverwith{\lower3.5\p@\hbox{\sixly \textcolor{#1}{\char58}}}\ULon}
%\font\sixly=lasy6 % does not re-load if already loaded, so no memory problem.

\newmdtheoremenv[
linewidth= 1pt,linecolor= blue,%
leftmargin=20,rightmargin=20,innertopmargin=0pt, innerrightmargin=40,%
tikzsetting = { draw=lightgray, line width = 0.3pt,dashed,%
dash pattern = on 15pt off 3pt},%
splittopskip=\topskip,skipbelow=\baselineskip,%
skipabove=\baselineskip,ntheorem,roundcorner=0pt,
% backgroundcolor=pagebg,font=\color{orange}\sffamily, fontcolor=white
]{examplebox}{Exemple}[section]



\newcommand\R{\mathbb{R}}
\newcommand\Z{\mathbb{Z}}
\newcommand\N{\mathbb{N}}
\newcommand\E{\mathbb{E}}
\newcommand\F{\mathcal{F}}
\newcommand\cH{\mathcal{H}}
\newcommand\V{\mathbb{V}}
\newcommand\dmo{ ^{-1} }
\newcommand\kapa{\kappa}
\newcommand\im{Im}
\newcommand\hs{\mathcal{H}}





\usepackage{soul}

\makeatletter
\newcommand*{\whiten}[1]{\llap{\textcolor{white}{{\the\SOUL@token}}\hspace{#1pt}}}
\DeclareRobustCommand*\myul{%
    \def\SOUL@everyspace{\underline{\space}\kern\z@}%
    \def\SOUL@everytoken{%
     \setbox0=\hbox{\the\SOUL@token}%
     \ifdim\dp0>\z@
        \raisebox{\dp0}{\underline{\phantom{\the\SOUL@token}}}%
        \whiten{1}\whiten{0}%
        \whiten{-1}\whiten{-2}%
        \llap{\the\SOUL@token}%
     \else
        \underline{\the\SOUL@token}%
     \fi}%
\SOUL@}
\makeatother

\newcommand*{\demp}{\fontfamily{lmtt}\selectfont}

\DeclareTextFontCommand{\textdemp}{\demp}

\begin{document}

\ifcomment
Multiline
comment
\fi
\ifcomment
\myul{Typesetting test}
% \color[rgb]{1,1,1}
$∑_i^n≠ 60º±∞π∆¬≈√j∫h≤≥µ$

$\CR \R\pro\ind\pro\gS\pro
\mqty[a&b\\c&d]$
$\pro\mathbb{P}$
$\dd{x}$

  \[
    \alpha(x)=\left\{
                \begin{array}{ll}
                  x\\
                  \frac{1}{1+e^{-kx}}\\
                  \frac{e^x-e^{-x}}{e^x+e^{-x}}
                \end{array}
              \right.
  \]

  $\expval{x}$
  
  $\chi_\rho(ghg\dmo)=\Tr(\rho_{ghg\dmo})=\Tr(\rho_g\circ\rho_h\circ\rho\dmo_g)=\Tr(\rho_h)\overset{\mbox{\scalebox{0.5}{$\Tr(AB)=\Tr(BA)$}}}{=}\chi_\rho(h)$
  	$\mathop{\oplus}_{\substack{x\in X}}$

$\mat(\rho_g)=(a_{ij}(g))_{\scriptsize \substack{1\leq i\leq d \\ 1\leq j\leq d}}$ et $\mat(\rho'_g)=(a'_{ij}(g))_{\scriptsize \substack{1\leq i'\leq d' \\ 1\leq j'\leq d'}}$



\[\int_a^b{\mathbb{R}^2}g(u, v)\dd{P_{XY}}(u, v)=\iint g(u,v) f_{XY}(u, v)\dd \lambda(u) \dd \lambda(v)\]
$$\lim_{x\to\infty} f(x)$$	
$$\iiiint_V \mu(t,u,v,w) \,dt\,du\,dv\,dw$$
$$\sum_{n=1}^{\infty} 2^{-n} = 1$$	
\begin{definition}
	Si $X$ et $Y$ sont 2 v.a. ou definit la \textsc{Covariance} entre $X$ et $Y$ comme
	$\cov(X,Y)\overset{\text{def}}{=}\E\left[(X-\E(X))(Y-\E(Y))\right]=\E(XY)-\E(X)\E(Y)$.
\end{definition}
\fi
\pagebreak

% \tableofcontents

% insert your code here
%\input{./algebra/main.tex}
%\input{./geometrie-differentielle/main.tex}
%\input{./probabilite/main.tex}
%\input{./analyse-fonctionnelle/main.tex}
% \input{./Analyse-convexe-et-dualite-en-optimisation/main.tex}
%\input{./tikz/main.tex}
%\input{./Theorie-du-distributions/main.tex}
%\input{./optimisation/mine.tex}
 \input{./modelisation/main.tex}

% yves.aubry@univ-tln.fr : algebra

\end{document}

%% !TEX encoding = UTF-8 Unicode
% !TEX TS-program = xelatex

\documentclass[french]{report}

%\usepackage[utf8]{inputenc}
%\usepackage[T1]{fontenc}
\usepackage{babel}


\newif\ifcomment
%\commenttrue # Show comments

\usepackage{physics}
\usepackage{amssymb}


\usepackage{amsthm}
% \usepackage{thmtools}
\usepackage{mathtools}
\usepackage{amsfonts}

\usepackage{color}

\usepackage{tikz}

\usepackage{geometry}
\geometry{a5paper, margin=0.1in, right=1cm}

\usepackage{dsfont}

\usepackage{graphicx}
\graphicspath{ {images/} }

\usepackage{faktor}

\usepackage{IEEEtrantools}
\usepackage{enumerate}   
\usepackage[PostScript=dvips]{"/Users/aware/Documents/Courses/diagrams"}


\newtheorem{theorem}{Théorème}[section]
\renewcommand{\thetheorem}{\arabic{theorem}}
\newtheorem{lemme}{Lemme}[section]
\renewcommand{\thelemme}{\arabic{lemme}}
\newtheorem{proposition}{Proposition}[section]
\renewcommand{\theproposition}{\arabic{proposition}}
\newtheorem{notations}{Notations}[section]
\newtheorem{problem}{Problème}[section]
\newtheorem{corollary}{Corollaire}[theorem]
\renewcommand{\thecorollary}{\arabic{corollary}}
\newtheorem{property}{Propriété}[section]
\newtheorem{objective}{Objectif}[section]

\theoremstyle{definition}
\newtheorem{definition}{Définition}[section]
\renewcommand{\thedefinition}{\arabic{definition}}
\newtheorem{exercise}{Exercice}[chapter]
\renewcommand{\theexercise}{\arabic{exercise}}
\newtheorem{example}{Exemple}[chapter]
\renewcommand{\theexample}{\arabic{example}}
\newtheorem*{solution}{Solution}
\newtheorem*{application}{Application}
\newtheorem*{notation}{Notation}
\newtheorem*{vocabulary}{Vocabulaire}
\newtheorem*{properties}{Propriétés}



\theoremstyle{remark}
\newtheorem*{remark}{Remarque}
\newtheorem*{rappel}{Rappel}


\usepackage{etoolbox}
\AtBeginEnvironment{exercise}{\small}
\AtBeginEnvironment{example}{\small}

\usepackage{cases}
\usepackage[red]{mypack}

\usepackage[framemethod=TikZ]{mdframed}

\definecolor{bg}{rgb}{0.4,0.25,0.95}
\definecolor{pagebg}{rgb}{0,0,0.5}
\surroundwithmdframed[
   topline=false,
   rightline=false,
   bottomline=false,
   leftmargin=\parindent,
   skipabove=8pt,
   skipbelow=8pt,
   linecolor=blue,
   innerbottommargin=10pt,
   % backgroundcolor=bg,font=\color{orange}\sffamily, fontcolor=white
]{definition}

\usepackage{empheq}
\usepackage[most]{tcolorbox}

\newtcbox{\mymath}[1][]{%
    nobeforeafter, math upper, tcbox raise base,
    enhanced, colframe=blue!30!black,
    colback=red!10, boxrule=1pt,
    #1}

\usepackage{unixode}


\DeclareMathOperator{\ord}{ord}
\DeclareMathOperator{\orb}{orb}
\DeclareMathOperator{\stab}{stab}
\DeclareMathOperator{\Stab}{stab}
\DeclareMathOperator{\ppcm}{ppcm}
\DeclareMathOperator{\conj}{Conj}
\DeclareMathOperator{\End}{End}
\DeclareMathOperator{\rot}{rot}
\DeclareMathOperator{\trs}{trace}
\DeclareMathOperator{\Ind}{Ind}
\DeclareMathOperator{\mat}{Mat}
\DeclareMathOperator{\id}{Id}
\DeclareMathOperator{\vect}{vect}
\DeclareMathOperator{\img}{img}
\DeclareMathOperator{\cov}{Cov}
\DeclareMathOperator{\dist}{dist}
\DeclareMathOperator{\irr}{Irr}
\DeclareMathOperator{\image}{Im}
\DeclareMathOperator{\pd}{\partial}
\DeclareMathOperator{\epi}{epi}
\DeclareMathOperator{\Argmin}{Argmin}
\DeclareMathOperator{\dom}{dom}
\DeclareMathOperator{\proj}{proj}
\DeclareMathOperator{\ctg}{ctg}
\DeclareMathOperator{\supp}{supp}
\DeclareMathOperator{\argmin}{argmin}
\DeclareMathOperator{\mult}{mult}
\DeclareMathOperator{\ch}{ch}
\DeclareMathOperator{\sh}{sh}
\DeclareMathOperator{\rang}{rang}
\DeclareMathOperator{\diam}{diam}
\DeclareMathOperator{\Epigraphe}{Epigraphe}




\usepackage{xcolor}
\everymath{\color{blue}}
%\everymath{\color[rgb]{0,1,1}}
%\pagecolor[rgb]{0,0,0.5}


\newcommand*{\pdtest}[3][]{\ensuremath{\frac{\partial^{#1} #2}{\partial #3}}}

\newcommand*{\deffunc}[6][]{\ensuremath{
\begin{array}{rcl}
#2 : #3 &\rightarrow& #4\\
#5 &\mapsto& #6
\end{array}
}}

\newcommand{\eqcolon}{\mathrel{\resizebox{\widthof{$\mathord{=}$}}{\height}{ $\!\!=\!\!\resizebox{1.2\width}{0.8\height}{\raisebox{0.23ex}{$\mathop{:}$}}\!\!$ }}}
\newcommand{\coloneq}{\mathrel{\resizebox{\widthof{$\mathord{=}$}}{\height}{ $\!\!\resizebox{1.2\width}{0.8\height}{\raisebox{0.23ex}{$\mathop{:}$}}\!\!=\!\!$ }}}
\newcommand{\eqcolonl}{\ensuremath{\mathrel{=\!\!\mathop{:}}}}
\newcommand{\coloneql}{\ensuremath{\mathrel{\mathop{:} \!\! =}}}
\newcommand{\vc}[1]{% inline column vector
  \left(\begin{smallmatrix}#1\end{smallmatrix}\right)%
}
\newcommand{\vr}[1]{% inline row vector
  \begin{smallmatrix}(\,#1\,)\end{smallmatrix}%
}
\makeatletter
\newcommand*{\defeq}{\ =\mathrel{\rlap{%
                     \raisebox{0.3ex}{$\m@th\cdot$}}%
                     \raisebox{-0.3ex}{$\m@th\cdot$}}%
                     }
\makeatother

\newcommand{\mathcircle}[1]{% inline row vector
 \overset{\circ}{#1}
}
\newcommand{\ulim}{% low limit
 \underline{\lim}
}
\newcommand{\ssi}{% iff
\iff
}
\newcommand{\ps}[2]{
\expval{#1 | #2}
}
\newcommand{\df}[1]{
\mqty{#1}
}
\newcommand{\n}[1]{
\norm{#1}
}
\newcommand{\sys}[1]{
\left\{\smqty{#1}\right.
}


\newcommand{\eqdef}{\ensuremath{\overset{\text{def}}=}}


\def\Circlearrowright{\ensuremath{%
  \rotatebox[origin=c]{230}{$\circlearrowright$}}}

\newcommand\ct[1]{\text{\rmfamily\upshape #1}}
\newcommand\question[1]{ {\color{red} ...!? \small #1}}
\newcommand\caz[1]{\left\{\begin{array} #1 \end{array}\right.}
\newcommand\const{\text{\rmfamily\upshape const}}
\newcommand\toP{ \overset{\pro}{\to}}
\newcommand\toPP{ \overset{\text{PP}}{\to}}
\newcommand{\oeq}{\mathrel{\text{\textcircled{$=$}}}}





\usepackage{xcolor}
% \usepackage[normalem]{ulem}
\usepackage{lipsum}
\makeatletter
% \newcommand\colorwave[1][blue]{\bgroup \markoverwith{\lower3.5\p@\hbox{\sixly \textcolor{#1}{\char58}}}\ULon}
%\font\sixly=lasy6 % does not re-load if already loaded, so no memory problem.

\newmdtheoremenv[
linewidth= 1pt,linecolor= blue,%
leftmargin=20,rightmargin=20,innertopmargin=0pt, innerrightmargin=40,%
tikzsetting = { draw=lightgray, line width = 0.3pt,dashed,%
dash pattern = on 15pt off 3pt},%
splittopskip=\topskip,skipbelow=\baselineskip,%
skipabove=\baselineskip,ntheorem,roundcorner=0pt,
% backgroundcolor=pagebg,font=\color{orange}\sffamily, fontcolor=white
]{examplebox}{Exemple}[section]



\newcommand\R{\mathbb{R}}
\newcommand\Z{\mathbb{Z}}
\newcommand\N{\mathbb{N}}
\newcommand\E{\mathbb{E}}
\newcommand\F{\mathcal{F}}
\newcommand\cH{\mathcal{H}}
\newcommand\V{\mathbb{V}}
\newcommand\dmo{ ^{-1} }
\newcommand\kapa{\kappa}
\newcommand\im{Im}
\newcommand\hs{\mathcal{H}}





\usepackage{soul}

\makeatletter
\newcommand*{\whiten}[1]{\llap{\textcolor{white}{{\the\SOUL@token}}\hspace{#1pt}}}
\DeclareRobustCommand*\myul{%
    \def\SOUL@everyspace{\underline{\space}\kern\z@}%
    \def\SOUL@everytoken{%
     \setbox0=\hbox{\the\SOUL@token}%
     \ifdim\dp0>\z@
        \raisebox{\dp0}{\underline{\phantom{\the\SOUL@token}}}%
        \whiten{1}\whiten{0}%
        \whiten{-1}\whiten{-2}%
        \llap{\the\SOUL@token}%
     \else
        \underline{\the\SOUL@token}%
     \fi}%
\SOUL@}
\makeatother

\newcommand*{\demp}{\fontfamily{lmtt}\selectfont}

\DeclareTextFontCommand{\textdemp}{\demp}

\begin{document}

\ifcomment
Multiline
comment
\fi
\ifcomment
\myul{Typesetting test}
% \color[rgb]{1,1,1}
$∑_i^n≠ 60º±∞π∆¬≈√j∫h≤≥µ$

$\CR \R\pro\ind\pro\gS\pro
\mqty[a&b\\c&d]$
$\pro\mathbb{P}$
$\dd{x}$

  \[
    \alpha(x)=\left\{
                \begin{array}{ll}
                  x\\
                  \frac{1}{1+e^{-kx}}\\
                  \frac{e^x-e^{-x}}{e^x+e^{-x}}
                \end{array}
              \right.
  \]

  $\expval{x}$
  
  $\chi_\rho(ghg\dmo)=\Tr(\rho_{ghg\dmo})=\Tr(\rho_g\circ\rho_h\circ\rho\dmo_g)=\Tr(\rho_h)\overset{\mbox{\scalebox{0.5}{$\Tr(AB)=\Tr(BA)$}}}{=}\chi_\rho(h)$
  	$\mathop{\oplus}_{\substack{x\in X}}$

$\mat(\rho_g)=(a_{ij}(g))_{\scriptsize \substack{1\leq i\leq d \\ 1\leq j\leq d}}$ et $\mat(\rho'_g)=(a'_{ij}(g))_{\scriptsize \substack{1\leq i'\leq d' \\ 1\leq j'\leq d'}}$



\[\int_a^b{\mathbb{R}^2}g(u, v)\dd{P_{XY}}(u, v)=\iint g(u,v) f_{XY}(u, v)\dd \lambda(u) \dd \lambda(v)\]
$$\lim_{x\to\infty} f(x)$$	
$$\iiiint_V \mu(t,u,v,w) \,dt\,du\,dv\,dw$$
$$\sum_{n=1}^{\infty} 2^{-n} = 1$$	
\begin{definition}
	Si $X$ et $Y$ sont 2 v.a. ou definit la \textsc{Covariance} entre $X$ et $Y$ comme
	$\cov(X,Y)\overset{\text{def}}{=}\E\left[(X-\E(X))(Y-\E(Y))\right]=\E(XY)-\E(X)\E(Y)$.
\end{definition}
\fi
\pagebreak

% \tableofcontents

% insert your code here
%\input{./algebra/main.tex}
%\input{./geometrie-differentielle/main.tex}
%\input{./probabilite/main.tex}
%\input{./analyse-fonctionnelle/main.tex}
% \input{./Analyse-convexe-et-dualite-en-optimisation/main.tex}
%\input{./tikz/main.tex}
%\input{./Theorie-du-distributions/main.tex}
%\input{./optimisation/mine.tex}
 \input{./modelisation/main.tex}

% yves.aubry@univ-tln.fr : algebra

\end{document}

%% !TEX encoding = UTF-8 Unicode
% !TEX TS-program = xelatex

\documentclass[french]{report}

%\usepackage[utf8]{inputenc}
%\usepackage[T1]{fontenc}
\usepackage{babel}


\newif\ifcomment
%\commenttrue # Show comments

\usepackage{physics}
\usepackage{amssymb}


\usepackage{amsthm}
% \usepackage{thmtools}
\usepackage{mathtools}
\usepackage{amsfonts}

\usepackage{color}

\usepackage{tikz}

\usepackage{geometry}
\geometry{a5paper, margin=0.1in, right=1cm}

\usepackage{dsfont}

\usepackage{graphicx}
\graphicspath{ {images/} }

\usepackage{faktor}

\usepackage{IEEEtrantools}
\usepackage{enumerate}   
\usepackage[PostScript=dvips]{"/Users/aware/Documents/Courses/diagrams"}


\newtheorem{theorem}{Théorème}[section]
\renewcommand{\thetheorem}{\arabic{theorem}}
\newtheorem{lemme}{Lemme}[section]
\renewcommand{\thelemme}{\arabic{lemme}}
\newtheorem{proposition}{Proposition}[section]
\renewcommand{\theproposition}{\arabic{proposition}}
\newtheorem{notations}{Notations}[section]
\newtheorem{problem}{Problème}[section]
\newtheorem{corollary}{Corollaire}[theorem]
\renewcommand{\thecorollary}{\arabic{corollary}}
\newtheorem{property}{Propriété}[section]
\newtheorem{objective}{Objectif}[section]

\theoremstyle{definition}
\newtheorem{definition}{Définition}[section]
\renewcommand{\thedefinition}{\arabic{definition}}
\newtheorem{exercise}{Exercice}[chapter]
\renewcommand{\theexercise}{\arabic{exercise}}
\newtheorem{example}{Exemple}[chapter]
\renewcommand{\theexample}{\arabic{example}}
\newtheorem*{solution}{Solution}
\newtheorem*{application}{Application}
\newtheorem*{notation}{Notation}
\newtheorem*{vocabulary}{Vocabulaire}
\newtheorem*{properties}{Propriétés}



\theoremstyle{remark}
\newtheorem*{remark}{Remarque}
\newtheorem*{rappel}{Rappel}


\usepackage{etoolbox}
\AtBeginEnvironment{exercise}{\small}
\AtBeginEnvironment{example}{\small}

\usepackage{cases}
\usepackage[red]{mypack}

\usepackage[framemethod=TikZ]{mdframed}

\definecolor{bg}{rgb}{0.4,0.25,0.95}
\definecolor{pagebg}{rgb}{0,0,0.5}
\surroundwithmdframed[
   topline=false,
   rightline=false,
   bottomline=false,
   leftmargin=\parindent,
   skipabove=8pt,
   skipbelow=8pt,
   linecolor=blue,
   innerbottommargin=10pt,
   % backgroundcolor=bg,font=\color{orange}\sffamily, fontcolor=white
]{definition}

\usepackage{empheq}
\usepackage[most]{tcolorbox}

\newtcbox{\mymath}[1][]{%
    nobeforeafter, math upper, tcbox raise base,
    enhanced, colframe=blue!30!black,
    colback=red!10, boxrule=1pt,
    #1}

\usepackage{unixode}


\DeclareMathOperator{\ord}{ord}
\DeclareMathOperator{\orb}{orb}
\DeclareMathOperator{\stab}{stab}
\DeclareMathOperator{\Stab}{stab}
\DeclareMathOperator{\ppcm}{ppcm}
\DeclareMathOperator{\conj}{Conj}
\DeclareMathOperator{\End}{End}
\DeclareMathOperator{\rot}{rot}
\DeclareMathOperator{\trs}{trace}
\DeclareMathOperator{\Ind}{Ind}
\DeclareMathOperator{\mat}{Mat}
\DeclareMathOperator{\id}{Id}
\DeclareMathOperator{\vect}{vect}
\DeclareMathOperator{\img}{img}
\DeclareMathOperator{\cov}{Cov}
\DeclareMathOperator{\dist}{dist}
\DeclareMathOperator{\irr}{Irr}
\DeclareMathOperator{\image}{Im}
\DeclareMathOperator{\pd}{\partial}
\DeclareMathOperator{\epi}{epi}
\DeclareMathOperator{\Argmin}{Argmin}
\DeclareMathOperator{\dom}{dom}
\DeclareMathOperator{\proj}{proj}
\DeclareMathOperator{\ctg}{ctg}
\DeclareMathOperator{\supp}{supp}
\DeclareMathOperator{\argmin}{argmin}
\DeclareMathOperator{\mult}{mult}
\DeclareMathOperator{\ch}{ch}
\DeclareMathOperator{\sh}{sh}
\DeclareMathOperator{\rang}{rang}
\DeclareMathOperator{\diam}{diam}
\DeclareMathOperator{\Epigraphe}{Epigraphe}




\usepackage{xcolor}
\everymath{\color{blue}}
%\everymath{\color[rgb]{0,1,1}}
%\pagecolor[rgb]{0,0,0.5}


\newcommand*{\pdtest}[3][]{\ensuremath{\frac{\partial^{#1} #2}{\partial #3}}}

\newcommand*{\deffunc}[6][]{\ensuremath{
\begin{array}{rcl}
#2 : #3 &\rightarrow& #4\\
#5 &\mapsto& #6
\end{array}
}}

\newcommand{\eqcolon}{\mathrel{\resizebox{\widthof{$\mathord{=}$}}{\height}{ $\!\!=\!\!\resizebox{1.2\width}{0.8\height}{\raisebox{0.23ex}{$\mathop{:}$}}\!\!$ }}}
\newcommand{\coloneq}{\mathrel{\resizebox{\widthof{$\mathord{=}$}}{\height}{ $\!\!\resizebox{1.2\width}{0.8\height}{\raisebox{0.23ex}{$\mathop{:}$}}\!\!=\!\!$ }}}
\newcommand{\eqcolonl}{\ensuremath{\mathrel{=\!\!\mathop{:}}}}
\newcommand{\coloneql}{\ensuremath{\mathrel{\mathop{:} \!\! =}}}
\newcommand{\vc}[1]{% inline column vector
  \left(\begin{smallmatrix}#1\end{smallmatrix}\right)%
}
\newcommand{\vr}[1]{% inline row vector
  \begin{smallmatrix}(\,#1\,)\end{smallmatrix}%
}
\makeatletter
\newcommand*{\defeq}{\ =\mathrel{\rlap{%
                     \raisebox{0.3ex}{$\m@th\cdot$}}%
                     \raisebox{-0.3ex}{$\m@th\cdot$}}%
                     }
\makeatother

\newcommand{\mathcircle}[1]{% inline row vector
 \overset{\circ}{#1}
}
\newcommand{\ulim}{% low limit
 \underline{\lim}
}
\newcommand{\ssi}{% iff
\iff
}
\newcommand{\ps}[2]{
\expval{#1 | #2}
}
\newcommand{\df}[1]{
\mqty{#1}
}
\newcommand{\n}[1]{
\norm{#1}
}
\newcommand{\sys}[1]{
\left\{\smqty{#1}\right.
}


\newcommand{\eqdef}{\ensuremath{\overset{\text{def}}=}}


\def\Circlearrowright{\ensuremath{%
  \rotatebox[origin=c]{230}{$\circlearrowright$}}}

\newcommand\ct[1]{\text{\rmfamily\upshape #1}}
\newcommand\question[1]{ {\color{red} ...!? \small #1}}
\newcommand\caz[1]{\left\{\begin{array} #1 \end{array}\right.}
\newcommand\const{\text{\rmfamily\upshape const}}
\newcommand\toP{ \overset{\pro}{\to}}
\newcommand\toPP{ \overset{\text{PP}}{\to}}
\newcommand{\oeq}{\mathrel{\text{\textcircled{$=$}}}}





\usepackage{xcolor}
% \usepackage[normalem]{ulem}
\usepackage{lipsum}
\makeatletter
% \newcommand\colorwave[1][blue]{\bgroup \markoverwith{\lower3.5\p@\hbox{\sixly \textcolor{#1}{\char58}}}\ULon}
%\font\sixly=lasy6 % does not re-load if already loaded, so no memory problem.

\newmdtheoremenv[
linewidth= 1pt,linecolor= blue,%
leftmargin=20,rightmargin=20,innertopmargin=0pt, innerrightmargin=40,%
tikzsetting = { draw=lightgray, line width = 0.3pt,dashed,%
dash pattern = on 15pt off 3pt},%
splittopskip=\topskip,skipbelow=\baselineskip,%
skipabove=\baselineskip,ntheorem,roundcorner=0pt,
% backgroundcolor=pagebg,font=\color{orange}\sffamily, fontcolor=white
]{examplebox}{Exemple}[section]



\newcommand\R{\mathbb{R}}
\newcommand\Z{\mathbb{Z}}
\newcommand\N{\mathbb{N}}
\newcommand\E{\mathbb{E}}
\newcommand\F{\mathcal{F}}
\newcommand\cH{\mathcal{H}}
\newcommand\V{\mathbb{V}}
\newcommand\dmo{ ^{-1} }
\newcommand\kapa{\kappa}
\newcommand\im{Im}
\newcommand\hs{\mathcal{H}}





\usepackage{soul}

\makeatletter
\newcommand*{\whiten}[1]{\llap{\textcolor{white}{{\the\SOUL@token}}\hspace{#1pt}}}
\DeclareRobustCommand*\myul{%
    \def\SOUL@everyspace{\underline{\space}\kern\z@}%
    \def\SOUL@everytoken{%
     \setbox0=\hbox{\the\SOUL@token}%
     \ifdim\dp0>\z@
        \raisebox{\dp0}{\underline{\phantom{\the\SOUL@token}}}%
        \whiten{1}\whiten{0}%
        \whiten{-1}\whiten{-2}%
        \llap{\the\SOUL@token}%
     \else
        \underline{\the\SOUL@token}%
     \fi}%
\SOUL@}
\makeatother

\newcommand*{\demp}{\fontfamily{lmtt}\selectfont}

\DeclareTextFontCommand{\textdemp}{\demp}

\begin{document}

\ifcomment
Multiline
comment
\fi
\ifcomment
\myul{Typesetting test}
% \color[rgb]{1,1,1}
$∑_i^n≠ 60º±∞π∆¬≈√j∫h≤≥µ$

$\CR \R\pro\ind\pro\gS\pro
\mqty[a&b\\c&d]$
$\pro\mathbb{P}$
$\dd{x}$

  \[
    \alpha(x)=\left\{
                \begin{array}{ll}
                  x\\
                  \frac{1}{1+e^{-kx}}\\
                  \frac{e^x-e^{-x}}{e^x+e^{-x}}
                \end{array}
              \right.
  \]

  $\expval{x}$
  
  $\chi_\rho(ghg\dmo)=\Tr(\rho_{ghg\dmo})=\Tr(\rho_g\circ\rho_h\circ\rho\dmo_g)=\Tr(\rho_h)\overset{\mbox{\scalebox{0.5}{$\Tr(AB)=\Tr(BA)$}}}{=}\chi_\rho(h)$
  	$\mathop{\oplus}_{\substack{x\in X}}$

$\mat(\rho_g)=(a_{ij}(g))_{\scriptsize \substack{1\leq i\leq d \\ 1\leq j\leq d}}$ et $\mat(\rho'_g)=(a'_{ij}(g))_{\scriptsize \substack{1\leq i'\leq d' \\ 1\leq j'\leq d'}}$



\[\int_a^b{\mathbb{R}^2}g(u, v)\dd{P_{XY}}(u, v)=\iint g(u,v) f_{XY}(u, v)\dd \lambda(u) \dd \lambda(v)\]
$$\lim_{x\to\infty} f(x)$$	
$$\iiiint_V \mu(t,u,v,w) \,dt\,du\,dv\,dw$$
$$\sum_{n=1}^{\infty} 2^{-n} = 1$$	
\begin{definition}
	Si $X$ et $Y$ sont 2 v.a. ou definit la \textsc{Covariance} entre $X$ et $Y$ comme
	$\cov(X,Y)\overset{\text{def}}{=}\E\left[(X-\E(X))(Y-\E(Y))\right]=\E(XY)-\E(X)\E(Y)$.
\end{definition}
\fi
\pagebreak

% \tableofcontents

% insert your code here
%\input{./algebra/main.tex}
%\input{./geometrie-differentielle/main.tex}
%\input{./probabilite/main.tex}
%\input{./analyse-fonctionnelle/main.tex}
% \input{./Analyse-convexe-et-dualite-en-optimisation/main.tex}
%\input{./tikz/main.tex}
%\input{./Theorie-du-distributions/main.tex}
%\input{./optimisation/mine.tex}
 \input{./modelisation/main.tex}

% yves.aubry@univ-tln.fr : algebra

\end{document}

% % !TEX encoding = UTF-8 Unicode
% !TEX TS-program = xelatex

\documentclass[french]{report}

%\usepackage[utf8]{inputenc}
%\usepackage[T1]{fontenc}
\usepackage{babel}


\newif\ifcomment
%\commenttrue # Show comments

\usepackage{physics}
\usepackage{amssymb}


\usepackage{amsthm}
% \usepackage{thmtools}
\usepackage{mathtools}
\usepackage{amsfonts}

\usepackage{color}

\usepackage{tikz}

\usepackage{geometry}
\geometry{a5paper, margin=0.1in, right=1cm}

\usepackage{dsfont}

\usepackage{graphicx}
\graphicspath{ {images/} }

\usepackage{faktor}

\usepackage{IEEEtrantools}
\usepackage{enumerate}   
\usepackage[PostScript=dvips]{"/Users/aware/Documents/Courses/diagrams"}


\newtheorem{theorem}{Théorème}[section]
\renewcommand{\thetheorem}{\arabic{theorem}}
\newtheorem{lemme}{Lemme}[section]
\renewcommand{\thelemme}{\arabic{lemme}}
\newtheorem{proposition}{Proposition}[section]
\renewcommand{\theproposition}{\arabic{proposition}}
\newtheorem{notations}{Notations}[section]
\newtheorem{problem}{Problème}[section]
\newtheorem{corollary}{Corollaire}[theorem]
\renewcommand{\thecorollary}{\arabic{corollary}}
\newtheorem{property}{Propriété}[section]
\newtheorem{objective}{Objectif}[section]

\theoremstyle{definition}
\newtheorem{definition}{Définition}[section]
\renewcommand{\thedefinition}{\arabic{definition}}
\newtheorem{exercise}{Exercice}[chapter]
\renewcommand{\theexercise}{\arabic{exercise}}
\newtheorem{example}{Exemple}[chapter]
\renewcommand{\theexample}{\arabic{example}}
\newtheorem*{solution}{Solution}
\newtheorem*{application}{Application}
\newtheorem*{notation}{Notation}
\newtheorem*{vocabulary}{Vocabulaire}
\newtheorem*{properties}{Propriétés}



\theoremstyle{remark}
\newtheorem*{remark}{Remarque}
\newtheorem*{rappel}{Rappel}


\usepackage{etoolbox}
\AtBeginEnvironment{exercise}{\small}
\AtBeginEnvironment{example}{\small}

\usepackage{cases}
\usepackage[red]{mypack}

\usepackage[framemethod=TikZ]{mdframed}

\definecolor{bg}{rgb}{0.4,0.25,0.95}
\definecolor{pagebg}{rgb}{0,0,0.5}
\surroundwithmdframed[
   topline=false,
   rightline=false,
   bottomline=false,
   leftmargin=\parindent,
   skipabove=8pt,
   skipbelow=8pt,
   linecolor=blue,
   innerbottommargin=10pt,
   % backgroundcolor=bg,font=\color{orange}\sffamily, fontcolor=white
]{definition}

\usepackage{empheq}
\usepackage[most]{tcolorbox}

\newtcbox{\mymath}[1][]{%
    nobeforeafter, math upper, tcbox raise base,
    enhanced, colframe=blue!30!black,
    colback=red!10, boxrule=1pt,
    #1}

\usepackage{unixode}


\DeclareMathOperator{\ord}{ord}
\DeclareMathOperator{\orb}{orb}
\DeclareMathOperator{\stab}{stab}
\DeclareMathOperator{\Stab}{stab}
\DeclareMathOperator{\ppcm}{ppcm}
\DeclareMathOperator{\conj}{Conj}
\DeclareMathOperator{\End}{End}
\DeclareMathOperator{\rot}{rot}
\DeclareMathOperator{\trs}{trace}
\DeclareMathOperator{\Ind}{Ind}
\DeclareMathOperator{\mat}{Mat}
\DeclareMathOperator{\id}{Id}
\DeclareMathOperator{\vect}{vect}
\DeclareMathOperator{\img}{img}
\DeclareMathOperator{\cov}{Cov}
\DeclareMathOperator{\dist}{dist}
\DeclareMathOperator{\irr}{Irr}
\DeclareMathOperator{\image}{Im}
\DeclareMathOperator{\pd}{\partial}
\DeclareMathOperator{\epi}{epi}
\DeclareMathOperator{\Argmin}{Argmin}
\DeclareMathOperator{\dom}{dom}
\DeclareMathOperator{\proj}{proj}
\DeclareMathOperator{\ctg}{ctg}
\DeclareMathOperator{\supp}{supp}
\DeclareMathOperator{\argmin}{argmin}
\DeclareMathOperator{\mult}{mult}
\DeclareMathOperator{\ch}{ch}
\DeclareMathOperator{\sh}{sh}
\DeclareMathOperator{\rang}{rang}
\DeclareMathOperator{\diam}{diam}
\DeclareMathOperator{\Epigraphe}{Epigraphe}




\usepackage{xcolor}
\everymath{\color{blue}}
%\everymath{\color[rgb]{0,1,1}}
%\pagecolor[rgb]{0,0,0.5}


\newcommand*{\pdtest}[3][]{\ensuremath{\frac{\partial^{#1} #2}{\partial #3}}}

\newcommand*{\deffunc}[6][]{\ensuremath{
\begin{array}{rcl}
#2 : #3 &\rightarrow& #4\\
#5 &\mapsto& #6
\end{array}
}}

\newcommand{\eqcolon}{\mathrel{\resizebox{\widthof{$\mathord{=}$}}{\height}{ $\!\!=\!\!\resizebox{1.2\width}{0.8\height}{\raisebox{0.23ex}{$\mathop{:}$}}\!\!$ }}}
\newcommand{\coloneq}{\mathrel{\resizebox{\widthof{$\mathord{=}$}}{\height}{ $\!\!\resizebox{1.2\width}{0.8\height}{\raisebox{0.23ex}{$\mathop{:}$}}\!\!=\!\!$ }}}
\newcommand{\eqcolonl}{\ensuremath{\mathrel{=\!\!\mathop{:}}}}
\newcommand{\coloneql}{\ensuremath{\mathrel{\mathop{:} \!\! =}}}
\newcommand{\vc}[1]{% inline column vector
  \left(\begin{smallmatrix}#1\end{smallmatrix}\right)%
}
\newcommand{\vr}[1]{% inline row vector
  \begin{smallmatrix}(\,#1\,)\end{smallmatrix}%
}
\makeatletter
\newcommand*{\defeq}{\ =\mathrel{\rlap{%
                     \raisebox{0.3ex}{$\m@th\cdot$}}%
                     \raisebox{-0.3ex}{$\m@th\cdot$}}%
                     }
\makeatother

\newcommand{\mathcircle}[1]{% inline row vector
 \overset{\circ}{#1}
}
\newcommand{\ulim}{% low limit
 \underline{\lim}
}
\newcommand{\ssi}{% iff
\iff
}
\newcommand{\ps}[2]{
\expval{#1 | #2}
}
\newcommand{\df}[1]{
\mqty{#1}
}
\newcommand{\n}[1]{
\norm{#1}
}
\newcommand{\sys}[1]{
\left\{\smqty{#1}\right.
}


\newcommand{\eqdef}{\ensuremath{\overset{\text{def}}=}}


\def\Circlearrowright{\ensuremath{%
  \rotatebox[origin=c]{230}{$\circlearrowright$}}}

\newcommand\ct[1]{\text{\rmfamily\upshape #1}}
\newcommand\question[1]{ {\color{red} ...!? \small #1}}
\newcommand\caz[1]{\left\{\begin{array} #1 \end{array}\right.}
\newcommand\const{\text{\rmfamily\upshape const}}
\newcommand\toP{ \overset{\pro}{\to}}
\newcommand\toPP{ \overset{\text{PP}}{\to}}
\newcommand{\oeq}{\mathrel{\text{\textcircled{$=$}}}}





\usepackage{xcolor}
% \usepackage[normalem]{ulem}
\usepackage{lipsum}
\makeatletter
% \newcommand\colorwave[1][blue]{\bgroup \markoverwith{\lower3.5\p@\hbox{\sixly \textcolor{#1}{\char58}}}\ULon}
%\font\sixly=lasy6 % does not re-load if already loaded, so no memory problem.

\newmdtheoremenv[
linewidth= 1pt,linecolor= blue,%
leftmargin=20,rightmargin=20,innertopmargin=0pt, innerrightmargin=40,%
tikzsetting = { draw=lightgray, line width = 0.3pt,dashed,%
dash pattern = on 15pt off 3pt},%
splittopskip=\topskip,skipbelow=\baselineskip,%
skipabove=\baselineskip,ntheorem,roundcorner=0pt,
% backgroundcolor=pagebg,font=\color{orange}\sffamily, fontcolor=white
]{examplebox}{Exemple}[section]



\newcommand\R{\mathbb{R}}
\newcommand\Z{\mathbb{Z}}
\newcommand\N{\mathbb{N}}
\newcommand\E{\mathbb{E}}
\newcommand\F{\mathcal{F}}
\newcommand\cH{\mathcal{H}}
\newcommand\V{\mathbb{V}}
\newcommand\dmo{ ^{-1} }
\newcommand\kapa{\kappa}
\newcommand\im{Im}
\newcommand\hs{\mathcal{H}}





\usepackage{soul}

\makeatletter
\newcommand*{\whiten}[1]{\llap{\textcolor{white}{{\the\SOUL@token}}\hspace{#1pt}}}
\DeclareRobustCommand*\myul{%
    \def\SOUL@everyspace{\underline{\space}\kern\z@}%
    \def\SOUL@everytoken{%
     \setbox0=\hbox{\the\SOUL@token}%
     \ifdim\dp0>\z@
        \raisebox{\dp0}{\underline{\phantom{\the\SOUL@token}}}%
        \whiten{1}\whiten{0}%
        \whiten{-1}\whiten{-2}%
        \llap{\the\SOUL@token}%
     \else
        \underline{\the\SOUL@token}%
     \fi}%
\SOUL@}
\makeatother

\newcommand*{\demp}{\fontfamily{lmtt}\selectfont}

\DeclareTextFontCommand{\textdemp}{\demp}

\begin{document}

\ifcomment
Multiline
comment
\fi
\ifcomment
\myul{Typesetting test}
% \color[rgb]{1,1,1}
$∑_i^n≠ 60º±∞π∆¬≈√j∫h≤≥µ$

$\CR \R\pro\ind\pro\gS\pro
\mqty[a&b\\c&d]$
$\pro\mathbb{P}$
$\dd{x}$

  \[
    \alpha(x)=\left\{
                \begin{array}{ll}
                  x\\
                  \frac{1}{1+e^{-kx}}\\
                  \frac{e^x-e^{-x}}{e^x+e^{-x}}
                \end{array}
              \right.
  \]

  $\expval{x}$
  
  $\chi_\rho(ghg\dmo)=\Tr(\rho_{ghg\dmo})=\Tr(\rho_g\circ\rho_h\circ\rho\dmo_g)=\Tr(\rho_h)\overset{\mbox{\scalebox{0.5}{$\Tr(AB)=\Tr(BA)$}}}{=}\chi_\rho(h)$
  	$\mathop{\oplus}_{\substack{x\in X}}$

$\mat(\rho_g)=(a_{ij}(g))_{\scriptsize \substack{1\leq i\leq d \\ 1\leq j\leq d}}$ et $\mat(\rho'_g)=(a'_{ij}(g))_{\scriptsize \substack{1\leq i'\leq d' \\ 1\leq j'\leq d'}}$



\[\int_a^b{\mathbb{R}^2}g(u, v)\dd{P_{XY}}(u, v)=\iint g(u,v) f_{XY}(u, v)\dd \lambda(u) \dd \lambda(v)\]
$$\lim_{x\to\infty} f(x)$$	
$$\iiiint_V \mu(t,u,v,w) \,dt\,du\,dv\,dw$$
$$\sum_{n=1}^{\infty} 2^{-n} = 1$$	
\begin{definition}
	Si $X$ et $Y$ sont 2 v.a. ou definit la \textsc{Covariance} entre $X$ et $Y$ comme
	$\cov(X,Y)\overset{\text{def}}{=}\E\left[(X-\E(X))(Y-\E(Y))\right]=\E(XY)-\E(X)\E(Y)$.
\end{definition}
\fi
\pagebreak

% \tableofcontents

% insert your code here
%\input{./algebra/main.tex}
%\input{./geometrie-differentielle/main.tex}
%\input{./probabilite/main.tex}
%\input{./analyse-fonctionnelle/main.tex}
% \input{./Analyse-convexe-et-dualite-en-optimisation/main.tex}
%\input{./tikz/main.tex}
%\input{./Theorie-du-distributions/main.tex}
%\input{./optimisation/mine.tex}
 \input{./modelisation/main.tex}

% yves.aubry@univ-tln.fr : algebra

\end{document}

%% !TEX encoding = UTF-8 Unicode
% !TEX TS-program = xelatex

\documentclass[french]{report}

%\usepackage[utf8]{inputenc}
%\usepackage[T1]{fontenc}
\usepackage{babel}


\newif\ifcomment
%\commenttrue # Show comments

\usepackage{physics}
\usepackage{amssymb}


\usepackage{amsthm}
% \usepackage{thmtools}
\usepackage{mathtools}
\usepackage{amsfonts}

\usepackage{color}

\usepackage{tikz}

\usepackage{geometry}
\geometry{a5paper, margin=0.1in, right=1cm}

\usepackage{dsfont}

\usepackage{graphicx}
\graphicspath{ {images/} }

\usepackage{faktor}

\usepackage{IEEEtrantools}
\usepackage{enumerate}   
\usepackage[PostScript=dvips]{"/Users/aware/Documents/Courses/diagrams"}


\newtheorem{theorem}{Théorème}[section]
\renewcommand{\thetheorem}{\arabic{theorem}}
\newtheorem{lemme}{Lemme}[section]
\renewcommand{\thelemme}{\arabic{lemme}}
\newtheorem{proposition}{Proposition}[section]
\renewcommand{\theproposition}{\arabic{proposition}}
\newtheorem{notations}{Notations}[section]
\newtheorem{problem}{Problème}[section]
\newtheorem{corollary}{Corollaire}[theorem]
\renewcommand{\thecorollary}{\arabic{corollary}}
\newtheorem{property}{Propriété}[section]
\newtheorem{objective}{Objectif}[section]

\theoremstyle{definition}
\newtheorem{definition}{Définition}[section]
\renewcommand{\thedefinition}{\arabic{definition}}
\newtheorem{exercise}{Exercice}[chapter]
\renewcommand{\theexercise}{\arabic{exercise}}
\newtheorem{example}{Exemple}[chapter]
\renewcommand{\theexample}{\arabic{example}}
\newtheorem*{solution}{Solution}
\newtheorem*{application}{Application}
\newtheorem*{notation}{Notation}
\newtheorem*{vocabulary}{Vocabulaire}
\newtheorem*{properties}{Propriétés}



\theoremstyle{remark}
\newtheorem*{remark}{Remarque}
\newtheorem*{rappel}{Rappel}


\usepackage{etoolbox}
\AtBeginEnvironment{exercise}{\small}
\AtBeginEnvironment{example}{\small}

\usepackage{cases}
\usepackage[red]{mypack}

\usepackage[framemethod=TikZ]{mdframed}

\definecolor{bg}{rgb}{0.4,0.25,0.95}
\definecolor{pagebg}{rgb}{0,0,0.5}
\surroundwithmdframed[
   topline=false,
   rightline=false,
   bottomline=false,
   leftmargin=\parindent,
   skipabove=8pt,
   skipbelow=8pt,
   linecolor=blue,
   innerbottommargin=10pt,
   % backgroundcolor=bg,font=\color{orange}\sffamily, fontcolor=white
]{definition}

\usepackage{empheq}
\usepackage[most]{tcolorbox}

\newtcbox{\mymath}[1][]{%
    nobeforeafter, math upper, tcbox raise base,
    enhanced, colframe=blue!30!black,
    colback=red!10, boxrule=1pt,
    #1}

\usepackage{unixode}


\DeclareMathOperator{\ord}{ord}
\DeclareMathOperator{\orb}{orb}
\DeclareMathOperator{\stab}{stab}
\DeclareMathOperator{\Stab}{stab}
\DeclareMathOperator{\ppcm}{ppcm}
\DeclareMathOperator{\conj}{Conj}
\DeclareMathOperator{\End}{End}
\DeclareMathOperator{\rot}{rot}
\DeclareMathOperator{\trs}{trace}
\DeclareMathOperator{\Ind}{Ind}
\DeclareMathOperator{\mat}{Mat}
\DeclareMathOperator{\id}{Id}
\DeclareMathOperator{\vect}{vect}
\DeclareMathOperator{\img}{img}
\DeclareMathOperator{\cov}{Cov}
\DeclareMathOperator{\dist}{dist}
\DeclareMathOperator{\irr}{Irr}
\DeclareMathOperator{\image}{Im}
\DeclareMathOperator{\pd}{\partial}
\DeclareMathOperator{\epi}{epi}
\DeclareMathOperator{\Argmin}{Argmin}
\DeclareMathOperator{\dom}{dom}
\DeclareMathOperator{\proj}{proj}
\DeclareMathOperator{\ctg}{ctg}
\DeclareMathOperator{\supp}{supp}
\DeclareMathOperator{\argmin}{argmin}
\DeclareMathOperator{\mult}{mult}
\DeclareMathOperator{\ch}{ch}
\DeclareMathOperator{\sh}{sh}
\DeclareMathOperator{\rang}{rang}
\DeclareMathOperator{\diam}{diam}
\DeclareMathOperator{\Epigraphe}{Epigraphe}




\usepackage{xcolor}
\everymath{\color{blue}}
%\everymath{\color[rgb]{0,1,1}}
%\pagecolor[rgb]{0,0,0.5}


\newcommand*{\pdtest}[3][]{\ensuremath{\frac{\partial^{#1} #2}{\partial #3}}}

\newcommand*{\deffunc}[6][]{\ensuremath{
\begin{array}{rcl}
#2 : #3 &\rightarrow& #4\\
#5 &\mapsto& #6
\end{array}
}}

\newcommand{\eqcolon}{\mathrel{\resizebox{\widthof{$\mathord{=}$}}{\height}{ $\!\!=\!\!\resizebox{1.2\width}{0.8\height}{\raisebox{0.23ex}{$\mathop{:}$}}\!\!$ }}}
\newcommand{\coloneq}{\mathrel{\resizebox{\widthof{$\mathord{=}$}}{\height}{ $\!\!\resizebox{1.2\width}{0.8\height}{\raisebox{0.23ex}{$\mathop{:}$}}\!\!=\!\!$ }}}
\newcommand{\eqcolonl}{\ensuremath{\mathrel{=\!\!\mathop{:}}}}
\newcommand{\coloneql}{\ensuremath{\mathrel{\mathop{:} \!\! =}}}
\newcommand{\vc}[1]{% inline column vector
  \left(\begin{smallmatrix}#1\end{smallmatrix}\right)%
}
\newcommand{\vr}[1]{% inline row vector
  \begin{smallmatrix}(\,#1\,)\end{smallmatrix}%
}
\makeatletter
\newcommand*{\defeq}{\ =\mathrel{\rlap{%
                     \raisebox{0.3ex}{$\m@th\cdot$}}%
                     \raisebox{-0.3ex}{$\m@th\cdot$}}%
                     }
\makeatother

\newcommand{\mathcircle}[1]{% inline row vector
 \overset{\circ}{#1}
}
\newcommand{\ulim}{% low limit
 \underline{\lim}
}
\newcommand{\ssi}{% iff
\iff
}
\newcommand{\ps}[2]{
\expval{#1 | #2}
}
\newcommand{\df}[1]{
\mqty{#1}
}
\newcommand{\n}[1]{
\norm{#1}
}
\newcommand{\sys}[1]{
\left\{\smqty{#1}\right.
}


\newcommand{\eqdef}{\ensuremath{\overset{\text{def}}=}}


\def\Circlearrowright{\ensuremath{%
  \rotatebox[origin=c]{230}{$\circlearrowright$}}}

\newcommand\ct[1]{\text{\rmfamily\upshape #1}}
\newcommand\question[1]{ {\color{red} ...!? \small #1}}
\newcommand\caz[1]{\left\{\begin{array} #1 \end{array}\right.}
\newcommand\const{\text{\rmfamily\upshape const}}
\newcommand\toP{ \overset{\pro}{\to}}
\newcommand\toPP{ \overset{\text{PP}}{\to}}
\newcommand{\oeq}{\mathrel{\text{\textcircled{$=$}}}}





\usepackage{xcolor}
% \usepackage[normalem]{ulem}
\usepackage{lipsum}
\makeatletter
% \newcommand\colorwave[1][blue]{\bgroup \markoverwith{\lower3.5\p@\hbox{\sixly \textcolor{#1}{\char58}}}\ULon}
%\font\sixly=lasy6 % does not re-load if already loaded, so no memory problem.

\newmdtheoremenv[
linewidth= 1pt,linecolor= blue,%
leftmargin=20,rightmargin=20,innertopmargin=0pt, innerrightmargin=40,%
tikzsetting = { draw=lightgray, line width = 0.3pt,dashed,%
dash pattern = on 15pt off 3pt},%
splittopskip=\topskip,skipbelow=\baselineskip,%
skipabove=\baselineskip,ntheorem,roundcorner=0pt,
% backgroundcolor=pagebg,font=\color{orange}\sffamily, fontcolor=white
]{examplebox}{Exemple}[section]



\newcommand\R{\mathbb{R}}
\newcommand\Z{\mathbb{Z}}
\newcommand\N{\mathbb{N}}
\newcommand\E{\mathbb{E}}
\newcommand\F{\mathcal{F}}
\newcommand\cH{\mathcal{H}}
\newcommand\V{\mathbb{V}}
\newcommand\dmo{ ^{-1} }
\newcommand\kapa{\kappa}
\newcommand\im{Im}
\newcommand\hs{\mathcal{H}}





\usepackage{soul}

\makeatletter
\newcommand*{\whiten}[1]{\llap{\textcolor{white}{{\the\SOUL@token}}\hspace{#1pt}}}
\DeclareRobustCommand*\myul{%
    \def\SOUL@everyspace{\underline{\space}\kern\z@}%
    \def\SOUL@everytoken{%
     \setbox0=\hbox{\the\SOUL@token}%
     \ifdim\dp0>\z@
        \raisebox{\dp0}{\underline{\phantom{\the\SOUL@token}}}%
        \whiten{1}\whiten{0}%
        \whiten{-1}\whiten{-2}%
        \llap{\the\SOUL@token}%
     \else
        \underline{\the\SOUL@token}%
     \fi}%
\SOUL@}
\makeatother

\newcommand*{\demp}{\fontfamily{lmtt}\selectfont}

\DeclareTextFontCommand{\textdemp}{\demp}

\begin{document}

\ifcomment
Multiline
comment
\fi
\ifcomment
\myul{Typesetting test}
% \color[rgb]{1,1,1}
$∑_i^n≠ 60º±∞π∆¬≈√j∫h≤≥µ$

$\CR \R\pro\ind\pro\gS\pro
\mqty[a&b\\c&d]$
$\pro\mathbb{P}$
$\dd{x}$

  \[
    \alpha(x)=\left\{
                \begin{array}{ll}
                  x\\
                  \frac{1}{1+e^{-kx}}\\
                  \frac{e^x-e^{-x}}{e^x+e^{-x}}
                \end{array}
              \right.
  \]

  $\expval{x}$
  
  $\chi_\rho(ghg\dmo)=\Tr(\rho_{ghg\dmo})=\Tr(\rho_g\circ\rho_h\circ\rho\dmo_g)=\Tr(\rho_h)\overset{\mbox{\scalebox{0.5}{$\Tr(AB)=\Tr(BA)$}}}{=}\chi_\rho(h)$
  	$\mathop{\oplus}_{\substack{x\in X}}$

$\mat(\rho_g)=(a_{ij}(g))_{\scriptsize \substack{1\leq i\leq d \\ 1\leq j\leq d}}$ et $\mat(\rho'_g)=(a'_{ij}(g))_{\scriptsize \substack{1\leq i'\leq d' \\ 1\leq j'\leq d'}}$



\[\int_a^b{\mathbb{R}^2}g(u, v)\dd{P_{XY}}(u, v)=\iint g(u,v) f_{XY}(u, v)\dd \lambda(u) \dd \lambda(v)\]
$$\lim_{x\to\infty} f(x)$$	
$$\iiiint_V \mu(t,u,v,w) \,dt\,du\,dv\,dw$$
$$\sum_{n=1}^{\infty} 2^{-n} = 1$$	
\begin{definition}
	Si $X$ et $Y$ sont 2 v.a. ou definit la \textsc{Covariance} entre $X$ et $Y$ comme
	$\cov(X,Y)\overset{\text{def}}{=}\E\left[(X-\E(X))(Y-\E(Y))\right]=\E(XY)-\E(X)\E(Y)$.
\end{definition}
\fi
\pagebreak

% \tableofcontents

% insert your code here
%\input{./algebra/main.tex}
%\input{./geometrie-differentielle/main.tex}
%\input{./probabilite/main.tex}
%\input{./analyse-fonctionnelle/main.tex}
% \input{./Analyse-convexe-et-dualite-en-optimisation/main.tex}
%\input{./tikz/main.tex}
%\input{./Theorie-du-distributions/main.tex}
%\input{./optimisation/mine.tex}
 \input{./modelisation/main.tex}

% yves.aubry@univ-tln.fr : algebra

\end{document}

%% !TEX encoding = UTF-8 Unicode
% !TEX TS-program = xelatex

\documentclass[french]{report}

%\usepackage[utf8]{inputenc}
%\usepackage[T1]{fontenc}
\usepackage{babel}


\newif\ifcomment
%\commenttrue # Show comments

\usepackage{physics}
\usepackage{amssymb}


\usepackage{amsthm}
% \usepackage{thmtools}
\usepackage{mathtools}
\usepackage{amsfonts}

\usepackage{color}

\usepackage{tikz}

\usepackage{geometry}
\geometry{a5paper, margin=0.1in, right=1cm}

\usepackage{dsfont}

\usepackage{graphicx}
\graphicspath{ {images/} }

\usepackage{faktor}

\usepackage{IEEEtrantools}
\usepackage{enumerate}   
\usepackage[PostScript=dvips]{"/Users/aware/Documents/Courses/diagrams"}


\newtheorem{theorem}{Théorème}[section]
\renewcommand{\thetheorem}{\arabic{theorem}}
\newtheorem{lemme}{Lemme}[section]
\renewcommand{\thelemme}{\arabic{lemme}}
\newtheorem{proposition}{Proposition}[section]
\renewcommand{\theproposition}{\arabic{proposition}}
\newtheorem{notations}{Notations}[section]
\newtheorem{problem}{Problème}[section]
\newtheorem{corollary}{Corollaire}[theorem]
\renewcommand{\thecorollary}{\arabic{corollary}}
\newtheorem{property}{Propriété}[section]
\newtheorem{objective}{Objectif}[section]

\theoremstyle{definition}
\newtheorem{definition}{Définition}[section]
\renewcommand{\thedefinition}{\arabic{definition}}
\newtheorem{exercise}{Exercice}[chapter]
\renewcommand{\theexercise}{\arabic{exercise}}
\newtheorem{example}{Exemple}[chapter]
\renewcommand{\theexample}{\arabic{example}}
\newtheorem*{solution}{Solution}
\newtheorem*{application}{Application}
\newtheorem*{notation}{Notation}
\newtheorem*{vocabulary}{Vocabulaire}
\newtheorem*{properties}{Propriétés}



\theoremstyle{remark}
\newtheorem*{remark}{Remarque}
\newtheorem*{rappel}{Rappel}


\usepackage{etoolbox}
\AtBeginEnvironment{exercise}{\small}
\AtBeginEnvironment{example}{\small}

\usepackage{cases}
\usepackage[red]{mypack}

\usepackage[framemethod=TikZ]{mdframed}

\definecolor{bg}{rgb}{0.4,0.25,0.95}
\definecolor{pagebg}{rgb}{0,0,0.5}
\surroundwithmdframed[
   topline=false,
   rightline=false,
   bottomline=false,
   leftmargin=\parindent,
   skipabove=8pt,
   skipbelow=8pt,
   linecolor=blue,
   innerbottommargin=10pt,
   % backgroundcolor=bg,font=\color{orange}\sffamily, fontcolor=white
]{definition}

\usepackage{empheq}
\usepackage[most]{tcolorbox}

\newtcbox{\mymath}[1][]{%
    nobeforeafter, math upper, tcbox raise base,
    enhanced, colframe=blue!30!black,
    colback=red!10, boxrule=1pt,
    #1}

\usepackage{unixode}


\DeclareMathOperator{\ord}{ord}
\DeclareMathOperator{\orb}{orb}
\DeclareMathOperator{\stab}{stab}
\DeclareMathOperator{\Stab}{stab}
\DeclareMathOperator{\ppcm}{ppcm}
\DeclareMathOperator{\conj}{Conj}
\DeclareMathOperator{\End}{End}
\DeclareMathOperator{\rot}{rot}
\DeclareMathOperator{\trs}{trace}
\DeclareMathOperator{\Ind}{Ind}
\DeclareMathOperator{\mat}{Mat}
\DeclareMathOperator{\id}{Id}
\DeclareMathOperator{\vect}{vect}
\DeclareMathOperator{\img}{img}
\DeclareMathOperator{\cov}{Cov}
\DeclareMathOperator{\dist}{dist}
\DeclareMathOperator{\irr}{Irr}
\DeclareMathOperator{\image}{Im}
\DeclareMathOperator{\pd}{\partial}
\DeclareMathOperator{\epi}{epi}
\DeclareMathOperator{\Argmin}{Argmin}
\DeclareMathOperator{\dom}{dom}
\DeclareMathOperator{\proj}{proj}
\DeclareMathOperator{\ctg}{ctg}
\DeclareMathOperator{\supp}{supp}
\DeclareMathOperator{\argmin}{argmin}
\DeclareMathOperator{\mult}{mult}
\DeclareMathOperator{\ch}{ch}
\DeclareMathOperator{\sh}{sh}
\DeclareMathOperator{\rang}{rang}
\DeclareMathOperator{\diam}{diam}
\DeclareMathOperator{\Epigraphe}{Epigraphe}




\usepackage{xcolor}
\everymath{\color{blue}}
%\everymath{\color[rgb]{0,1,1}}
%\pagecolor[rgb]{0,0,0.5}


\newcommand*{\pdtest}[3][]{\ensuremath{\frac{\partial^{#1} #2}{\partial #3}}}

\newcommand*{\deffunc}[6][]{\ensuremath{
\begin{array}{rcl}
#2 : #3 &\rightarrow& #4\\
#5 &\mapsto& #6
\end{array}
}}

\newcommand{\eqcolon}{\mathrel{\resizebox{\widthof{$\mathord{=}$}}{\height}{ $\!\!=\!\!\resizebox{1.2\width}{0.8\height}{\raisebox{0.23ex}{$\mathop{:}$}}\!\!$ }}}
\newcommand{\coloneq}{\mathrel{\resizebox{\widthof{$\mathord{=}$}}{\height}{ $\!\!\resizebox{1.2\width}{0.8\height}{\raisebox{0.23ex}{$\mathop{:}$}}\!\!=\!\!$ }}}
\newcommand{\eqcolonl}{\ensuremath{\mathrel{=\!\!\mathop{:}}}}
\newcommand{\coloneql}{\ensuremath{\mathrel{\mathop{:} \!\! =}}}
\newcommand{\vc}[1]{% inline column vector
  \left(\begin{smallmatrix}#1\end{smallmatrix}\right)%
}
\newcommand{\vr}[1]{% inline row vector
  \begin{smallmatrix}(\,#1\,)\end{smallmatrix}%
}
\makeatletter
\newcommand*{\defeq}{\ =\mathrel{\rlap{%
                     \raisebox{0.3ex}{$\m@th\cdot$}}%
                     \raisebox{-0.3ex}{$\m@th\cdot$}}%
                     }
\makeatother

\newcommand{\mathcircle}[1]{% inline row vector
 \overset{\circ}{#1}
}
\newcommand{\ulim}{% low limit
 \underline{\lim}
}
\newcommand{\ssi}{% iff
\iff
}
\newcommand{\ps}[2]{
\expval{#1 | #2}
}
\newcommand{\df}[1]{
\mqty{#1}
}
\newcommand{\n}[1]{
\norm{#1}
}
\newcommand{\sys}[1]{
\left\{\smqty{#1}\right.
}


\newcommand{\eqdef}{\ensuremath{\overset{\text{def}}=}}


\def\Circlearrowright{\ensuremath{%
  \rotatebox[origin=c]{230}{$\circlearrowright$}}}

\newcommand\ct[1]{\text{\rmfamily\upshape #1}}
\newcommand\question[1]{ {\color{red} ...!? \small #1}}
\newcommand\caz[1]{\left\{\begin{array} #1 \end{array}\right.}
\newcommand\const{\text{\rmfamily\upshape const}}
\newcommand\toP{ \overset{\pro}{\to}}
\newcommand\toPP{ \overset{\text{PP}}{\to}}
\newcommand{\oeq}{\mathrel{\text{\textcircled{$=$}}}}





\usepackage{xcolor}
% \usepackage[normalem]{ulem}
\usepackage{lipsum}
\makeatletter
% \newcommand\colorwave[1][blue]{\bgroup \markoverwith{\lower3.5\p@\hbox{\sixly \textcolor{#1}{\char58}}}\ULon}
%\font\sixly=lasy6 % does not re-load if already loaded, so no memory problem.

\newmdtheoremenv[
linewidth= 1pt,linecolor= blue,%
leftmargin=20,rightmargin=20,innertopmargin=0pt, innerrightmargin=40,%
tikzsetting = { draw=lightgray, line width = 0.3pt,dashed,%
dash pattern = on 15pt off 3pt},%
splittopskip=\topskip,skipbelow=\baselineskip,%
skipabove=\baselineskip,ntheorem,roundcorner=0pt,
% backgroundcolor=pagebg,font=\color{orange}\sffamily, fontcolor=white
]{examplebox}{Exemple}[section]



\newcommand\R{\mathbb{R}}
\newcommand\Z{\mathbb{Z}}
\newcommand\N{\mathbb{N}}
\newcommand\E{\mathbb{E}}
\newcommand\F{\mathcal{F}}
\newcommand\cH{\mathcal{H}}
\newcommand\V{\mathbb{V}}
\newcommand\dmo{ ^{-1} }
\newcommand\kapa{\kappa}
\newcommand\im{Im}
\newcommand\hs{\mathcal{H}}





\usepackage{soul}

\makeatletter
\newcommand*{\whiten}[1]{\llap{\textcolor{white}{{\the\SOUL@token}}\hspace{#1pt}}}
\DeclareRobustCommand*\myul{%
    \def\SOUL@everyspace{\underline{\space}\kern\z@}%
    \def\SOUL@everytoken{%
     \setbox0=\hbox{\the\SOUL@token}%
     \ifdim\dp0>\z@
        \raisebox{\dp0}{\underline{\phantom{\the\SOUL@token}}}%
        \whiten{1}\whiten{0}%
        \whiten{-1}\whiten{-2}%
        \llap{\the\SOUL@token}%
     \else
        \underline{\the\SOUL@token}%
     \fi}%
\SOUL@}
\makeatother

\newcommand*{\demp}{\fontfamily{lmtt}\selectfont}

\DeclareTextFontCommand{\textdemp}{\demp}

\begin{document}

\ifcomment
Multiline
comment
\fi
\ifcomment
\myul{Typesetting test}
% \color[rgb]{1,1,1}
$∑_i^n≠ 60º±∞π∆¬≈√j∫h≤≥µ$

$\CR \R\pro\ind\pro\gS\pro
\mqty[a&b\\c&d]$
$\pro\mathbb{P}$
$\dd{x}$

  \[
    \alpha(x)=\left\{
                \begin{array}{ll}
                  x\\
                  \frac{1}{1+e^{-kx}}\\
                  \frac{e^x-e^{-x}}{e^x+e^{-x}}
                \end{array}
              \right.
  \]

  $\expval{x}$
  
  $\chi_\rho(ghg\dmo)=\Tr(\rho_{ghg\dmo})=\Tr(\rho_g\circ\rho_h\circ\rho\dmo_g)=\Tr(\rho_h)\overset{\mbox{\scalebox{0.5}{$\Tr(AB)=\Tr(BA)$}}}{=}\chi_\rho(h)$
  	$\mathop{\oplus}_{\substack{x\in X}}$

$\mat(\rho_g)=(a_{ij}(g))_{\scriptsize \substack{1\leq i\leq d \\ 1\leq j\leq d}}$ et $\mat(\rho'_g)=(a'_{ij}(g))_{\scriptsize \substack{1\leq i'\leq d' \\ 1\leq j'\leq d'}}$



\[\int_a^b{\mathbb{R}^2}g(u, v)\dd{P_{XY}}(u, v)=\iint g(u,v) f_{XY}(u, v)\dd \lambda(u) \dd \lambda(v)\]
$$\lim_{x\to\infty} f(x)$$	
$$\iiiint_V \mu(t,u,v,w) \,dt\,du\,dv\,dw$$
$$\sum_{n=1}^{\infty} 2^{-n} = 1$$	
\begin{definition}
	Si $X$ et $Y$ sont 2 v.a. ou definit la \textsc{Covariance} entre $X$ et $Y$ comme
	$\cov(X,Y)\overset{\text{def}}{=}\E\left[(X-\E(X))(Y-\E(Y))\right]=\E(XY)-\E(X)\E(Y)$.
\end{definition}
\fi
\pagebreak

% \tableofcontents

% insert your code here
%\input{./algebra/main.tex}
%\input{./geometrie-differentielle/main.tex}
%\input{./probabilite/main.tex}
%\input{./analyse-fonctionnelle/main.tex}
% \input{./Analyse-convexe-et-dualite-en-optimisation/main.tex}
%\input{./tikz/main.tex}
%\input{./Theorie-du-distributions/main.tex}
%\input{./optimisation/mine.tex}
 \input{./modelisation/main.tex}

% yves.aubry@univ-tln.fr : algebra

\end{document}

%\input{./optimisation/mine.tex}
 % !TEX encoding = UTF-8 Unicode
% !TEX TS-program = xelatex

\documentclass[french]{report}

%\usepackage[utf8]{inputenc}
%\usepackage[T1]{fontenc}
\usepackage{babel}


\newif\ifcomment
%\commenttrue # Show comments

\usepackage{physics}
\usepackage{amssymb}


\usepackage{amsthm}
% \usepackage{thmtools}
\usepackage{mathtools}
\usepackage{amsfonts}

\usepackage{color}

\usepackage{tikz}

\usepackage{geometry}
\geometry{a5paper, margin=0.1in, right=1cm}

\usepackage{dsfont}

\usepackage{graphicx}
\graphicspath{ {images/} }

\usepackage{faktor}

\usepackage{IEEEtrantools}
\usepackage{enumerate}   
\usepackage[PostScript=dvips]{"/Users/aware/Documents/Courses/diagrams"}


\newtheorem{theorem}{Théorème}[section]
\renewcommand{\thetheorem}{\arabic{theorem}}
\newtheorem{lemme}{Lemme}[section]
\renewcommand{\thelemme}{\arabic{lemme}}
\newtheorem{proposition}{Proposition}[section]
\renewcommand{\theproposition}{\arabic{proposition}}
\newtheorem{notations}{Notations}[section]
\newtheorem{problem}{Problème}[section]
\newtheorem{corollary}{Corollaire}[theorem]
\renewcommand{\thecorollary}{\arabic{corollary}}
\newtheorem{property}{Propriété}[section]
\newtheorem{objective}{Objectif}[section]

\theoremstyle{definition}
\newtheorem{definition}{Définition}[section]
\renewcommand{\thedefinition}{\arabic{definition}}
\newtheorem{exercise}{Exercice}[chapter]
\renewcommand{\theexercise}{\arabic{exercise}}
\newtheorem{example}{Exemple}[chapter]
\renewcommand{\theexample}{\arabic{example}}
\newtheorem*{solution}{Solution}
\newtheorem*{application}{Application}
\newtheorem*{notation}{Notation}
\newtheorem*{vocabulary}{Vocabulaire}
\newtheorem*{properties}{Propriétés}



\theoremstyle{remark}
\newtheorem*{remark}{Remarque}
\newtheorem*{rappel}{Rappel}


\usepackage{etoolbox}
\AtBeginEnvironment{exercise}{\small}
\AtBeginEnvironment{example}{\small}

\usepackage{cases}
\usepackage[red]{mypack}

\usepackage[framemethod=TikZ]{mdframed}

\definecolor{bg}{rgb}{0.4,0.25,0.95}
\definecolor{pagebg}{rgb}{0,0,0.5}
\surroundwithmdframed[
   topline=false,
   rightline=false,
   bottomline=false,
   leftmargin=\parindent,
   skipabove=8pt,
   skipbelow=8pt,
   linecolor=blue,
   innerbottommargin=10pt,
   % backgroundcolor=bg,font=\color{orange}\sffamily, fontcolor=white
]{definition}

\usepackage{empheq}
\usepackage[most]{tcolorbox}

\newtcbox{\mymath}[1][]{%
    nobeforeafter, math upper, tcbox raise base,
    enhanced, colframe=blue!30!black,
    colback=red!10, boxrule=1pt,
    #1}

\usepackage{unixode}


\DeclareMathOperator{\ord}{ord}
\DeclareMathOperator{\orb}{orb}
\DeclareMathOperator{\stab}{stab}
\DeclareMathOperator{\Stab}{stab}
\DeclareMathOperator{\ppcm}{ppcm}
\DeclareMathOperator{\conj}{Conj}
\DeclareMathOperator{\End}{End}
\DeclareMathOperator{\rot}{rot}
\DeclareMathOperator{\trs}{trace}
\DeclareMathOperator{\Ind}{Ind}
\DeclareMathOperator{\mat}{Mat}
\DeclareMathOperator{\id}{Id}
\DeclareMathOperator{\vect}{vect}
\DeclareMathOperator{\img}{img}
\DeclareMathOperator{\cov}{Cov}
\DeclareMathOperator{\dist}{dist}
\DeclareMathOperator{\irr}{Irr}
\DeclareMathOperator{\image}{Im}
\DeclareMathOperator{\pd}{\partial}
\DeclareMathOperator{\epi}{epi}
\DeclareMathOperator{\Argmin}{Argmin}
\DeclareMathOperator{\dom}{dom}
\DeclareMathOperator{\proj}{proj}
\DeclareMathOperator{\ctg}{ctg}
\DeclareMathOperator{\supp}{supp}
\DeclareMathOperator{\argmin}{argmin}
\DeclareMathOperator{\mult}{mult}
\DeclareMathOperator{\ch}{ch}
\DeclareMathOperator{\sh}{sh}
\DeclareMathOperator{\rang}{rang}
\DeclareMathOperator{\diam}{diam}
\DeclareMathOperator{\Epigraphe}{Epigraphe}




\usepackage{xcolor}
\everymath{\color{blue}}
%\everymath{\color[rgb]{0,1,1}}
%\pagecolor[rgb]{0,0,0.5}


\newcommand*{\pdtest}[3][]{\ensuremath{\frac{\partial^{#1} #2}{\partial #3}}}

\newcommand*{\deffunc}[6][]{\ensuremath{
\begin{array}{rcl}
#2 : #3 &\rightarrow& #4\\
#5 &\mapsto& #6
\end{array}
}}

\newcommand{\eqcolon}{\mathrel{\resizebox{\widthof{$\mathord{=}$}}{\height}{ $\!\!=\!\!\resizebox{1.2\width}{0.8\height}{\raisebox{0.23ex}{$\mathop{:}$}}\!\!$ }}}
\newcommand{\coloneq}{\mathrel{\resizebox{\widthof{$\mathord{=}$}}{\height}{ $\!\!\resizebox{1.2\width}{0.8\height}{\raisebox{0.23ex}{$\mathop{:}$}}\!\!=\!\!$ }}}
\newcommand{\eqcolonl}{\ensuremath{\mathrel{=\!\!\mathop{:}}}}
\newcommand{\coloneql}{\ensuremath{\mathrel{\mathop{:} \!\! =}}}
\newcommand{\vc}[1]{% inline column vector
  \left(\begin{smallmatrix}#1\end{smallmatrix}\right)%
}
\newcommand{\vr}[1]{% inline row vector
  \begin{smallmatrix}(\,#1\,)\end{smallmatrix}%
}
\makeatletter
\newcommand*{\defeq}{\ =\mathrel{\rlap{%
                     \raisebox{0.3ex}{$\m@th\cdot$}}%
                     \raisebox{-0.3ex}{$\m@th\cdot$}}%
                     }
\makeatother

\newcommand{\mathcircle}[1]{% inline row vector
 \overset{\circ}{#1}
}
\newcommand{\ulim}{% low limit
 \underline{\lim}
}
\newcommand{\ssi}{% iff
\iff
}
\newcommand{\ps}[2]{
\expval{#1 | #2}
}
\newcommand{\df}[1]{
\mqty{#1}
}
\newcommand{\n}[1]{
\norm{#1}
}
\newcommand{\sys}[1]{
\left\{\smqty{#1}\right.
}


\newcommand{\eqdef}{\ensuremath{\overset{\text{def}}=}}


\def\Circlearrowright{\ensuremath{%
  \rotatebox[origin=c]{230}{$\circlearrowright$}}}

\newcommand\ct[1]{\text{\rmfamily\upshape #1}}
\newcommand\question[1]{ {\color{red} ...!? \small #1}}
\newcommand\caz[1]{\left\{\begin{array} #1 \end{array}\right.}
\newcommand\const{\text{\rmfamily\upshape const}}
\newcommand\toP{ \overset{\pro}{\to}}
\newcommand\toPP{ \overset{\text{PP}}{\to}}
\newcommand{\oeq}{\mathrel{\text{\textcircled{$=$}}}}





\usepackage{xcolor}
% \usepackage[normalem]{ulem}
\usepackage{lipsum}
\makeatletter
% \newcommand\colorwave[1][blue]{\bgroup \markoverwith{\lower3.5\p@\hbox{\sixly \textcolor{#1}{\char58}}}\ULon}
%\font\sixly=lasy6 % does not re-load if already loaded, so no memory problem.

\newmdtheoremenv[
linewidth= 1pt,linecolor= blue,%
leftmargin=20,rightmargin=20,innertopmargin=0pt, innerrightmargin=40,%
tikzsetting = { draw=lightgray, line width = 0.3pt,dashed,%
dash pattern = on 15pt off 3pt},%
splittopskip=\topskip,skipbelow=\baselineskip,%
skipabove=\baselineskip,ntheorem,roundcorner=0pt,
% backgroundcolor=pagebg,font=\color{orange}\sffamily, fontcolor=white
]{examplebox}{Exemple}[section]



\newcommand\R{\mathbb{R}}
\newcommand\Z{\mathbb{Z}}
\newcommand\N{\mathbb{N}}
\newcommand\E{\mathbb{E}}
\newcommand\F{\mathcal{F}}
\newcommand\cH{\mathcal{H}}
\newcommand\V{\mathbb{V}}
\newcommand\dmo{ ^{-1} }
\newcommand\kapa{\kappa}
\newcommand\im{Im}
\newcommand\hs{\mathcal{H}}





\usepackage{soul}

\makeatletter
\newcommand*{\whiten}[1]{\llap{\textcolor{white}{{\the\SOUL@token}}\hspace{#1pt}}}
\DeclareRobustCommand*\myul{%
    \def\SOUL@everyspace{\underline{\space}\kern\z@}%
    \def\SOUL@everytoken{%
     \setbox0=\hbox{\the\SOUL@token}%
     \ifdim\dp0>\z@
        \raisebox{\dp0}{\underline{\phantom{\the\SOUL@token}}}%
        \whiten{1}\whiten{0}%
        \whiten{-1}\whiten{-2}%
        \llap{\the\SOUL@token}%
     \else
        \underline{\the\SOUL@token}%
     \fi}%
\SOUL@}
\makeatother

\newcommand*{\demp}{\fontfamily{lmtt}\selectfont}

\DeclareTextFontCommand{\textdemp}{\demp}

\begin{document}

\ifcomment
Multiline
comment
\fi
\ifcomment
\myul{Typesetting test}
% \color[rgb]{1,1,1}
$∑_i^n≠ 60º±∞π∆¬≈√j∫h≤≥µ$

$\CR \R\pro\ind\pro\gS\pro
\mqty[a&b\\c&d]$
$\pro\mathbb{P}$
$\dd{x}$

  \[
    \alpha(x)=\left\{
                \begin{array}{ll}
                  x\\
                  \frac{1}{1+e^{-kx}}\\
                  \frac{e^x-e^{-x}}{e^x+e^{-x}}
                \end{array}
              \right.
  \]

  $\expval{x}$
  
  $\chi_\rho(ghg\dmo)=\Tr(\rho_{ghg\dmo})=\Tr(\rho_g\circ\rho_h\circ\rho\dmo_g)=\Tr(\rho_h)\overset{\mbox{\scalebox{0.5}{$\Tr(AB)=\Tr(BA)$}}}{=}\chi_\rho(h)$
  	$\mathop{\oplus}_{\substack{x\in X}}$

$\mat(\rho_g)=(a_{ij}(g))_{\scriptsize \substack{1\leq i\leq d \\ 1\leq j\leq d}}$ et $\mat(\rho'_g)=(a'_{ij}(g))_{\scriptsize \substack{1\leq i'\leq d' \\ 1\leq j'\leq d'}}$



\[\int_a^b{\mathbb{R}^2}g(u, v)\dd{P_{XY}}(u, v)=\iint g(u,v) f_{XY}(u, v)\dd \lambda(u) \dd \lambda(v)\]
$$\lim_{x\to\infty} f(x)$$	
$$\iiiint_V \mu(t,u,v,w) \,dt\,du\,dv\,dw$$
$$\sum_{n=1}^{\infty} 2^{-n} = 1$$	
\begin{definition}
	Si $X$ et $Y$ sont 2 v.a. ou definit la \textsc{Covariance} entre $X$ et $Y$ comme
	$\cov(X,Y)\overset{\text{def}}{=}\E\left[(X-\E(X))(Y-\E(Y))\right]=\E(XY)-\E(X)\E(Y)$.
\end{definition}
\fi
\pagebreak

% \tableofcontents

% insert your code here
%\input{./algebra/main.tex}
%\input{./geometrie-differentielle/main.tex}
%\input{./probabilite/main.tex}
%\input{./analyse-fonctionnelle/main.tex}
% \input{./Analyse-convexe-et-dualite-en-optimisation/main.tex}
%\input{./tikz/main.tex}
%\input{./Theorie-du-distributions/main.tex}
%\input{./optimisation/mine.tex}
 \input{./modelisation/main.tex}

% yves.aubry@univ-tln.fr : algebra

\end{document}


% yves.aubry@univ-tln.fr : algebra

\end{document}


% yves.aubry@univ-tln.fr : algebra

\end{document}

%\input{./optimisation/mine.tex}
 % !TEX encoding = UTF-8 Unicode
% !TEX TS-program = xelatex

\documentclass[french]{report}

%\usepackage[utf8]{inputenc}
%\usepackage[T1]{fontenc}
\usepackage{babel}


\newif\ifcomment
%\commenttrue # Show comments

\usepackage{physics}
\usepackage{amssymb}


\usepackage{amsthm}
% \usepackage{thmtools}
\usepackage{mathtools}
\usepackage{amsfonts}

\usepackage{color}

\usepackage{tikz}

\usepackage{geometry}
\geometry{a5paper, margin=0.1in, right=1cm}

\usepackage{dsfont}

\usepackage{graphicx}
\graphicspath{ {images/} }

\usepackage{faktor}

\usepackage{IEEEtrantools}
\usepackage{enumerate}   
\usepackage[PostScript=dvips]{"/Users/aware/Documents/Courses/diagrams"}


\newtheorem{theorem}{Théorème}[section]
\renewcommand{\thetheorem}{\arabic{theorem}}
\newtheorem{lemme}{Lemme}[section]
\renewcommand{\thelemme}{\arabic{lemme}}
\newtheorem{proposition}{Proposition}[section]
\renewcommand{\theproposition}{\arabic{proposition}}
\newtheorem{notations}{Notations}[section]
\newtheorem{problem}{Problème}[section]
\newtheorem{corollary}{Corollaire}[theorem]
\renewcommand{\thecorollary}{\arabic{corollary}}
\newtheorem{property}{Propriété}[section]
\newtheorem{objective}{Objectif}[section]

\theoremstyle{definition}
\newtheorem{definition}{Définition}[section]
\renewcommand{\thedefinition}{\arabic{definition}}
\newtheorem{exercise}{Exercice}[chapter]
\renewcommand{\theexercise}{\arabic{exercise}}
\newtheorem{example}{Exemple}[chapter]
\renewcommand{\theexample}{\arabic{example}}
\newtheorem*{solution}{Solution}
\newtheorem*{application}{Application}
\newtheorem*{notation}{Notation}
\newtheorem*{vocabulary}{Vocabulaire}
\newtheorem*{properties}{Propriétés}



\theoremstyle{remark}
\newtheorem*{remark}{Remarque}
\newtheorem*{rappel}{Rappel}


\usepackage{etoolbox}
\AtBeginEnvironment{exercise}{\small}
\AtBeginEnvironment{example}{\small}

\usepackage{cases}
\usepackage[red]{mypack}

\usepackage[framemethod=TikZ]{mdframed}

\definecolor{bg}{rgb}{0.4,0.25,0.95}
\definecolor{pagebg}{rgb}{0,0,0.5}
\surroundwithmdframed[
   topline=false,
   rightline=false,
   bottomline=false,
   leftmargin=\parindent,
   skipabove=8pt,
   skipbelow=8pt,
   linecolor=blue,
   innerbottommargin=10pt,
   % backgroundcolor=bg,font=\color{orange}\sffamily, fontcolor=white
]{definition}

\usepackage{empheq}
\usepackage[most]{tcolorbox}

\newtcbox{\mymath}[1][]{%
    nobeforeafter, math upper, tcbox raise base,
    enhanced, colframe=blue!30!black,
    colback=red!10, boxrule=1pt,
    #1}

\usepackage{unixode}


\DeclareMathOperator{\ord}{ord}
\DeclareMathOperator{\orb}{orb}
\DeclareMathOperator{\stab}{stab}
\DeclareMathOperator{\Stab}{stab}
\DeclareMathOperator{\ppcm}{ppcm}
\DeclareMathOperator{\conj}{Conj}
\DeclareMathOperator{\End}{End}
\DeclareMathOperator{\rot}{rot}
\DeclareMathOperator{\trs}{trace}
\DeclareMathOperator{\Ind}{Ind}
\DeclareMathOperator{\mat}{Mat}
\DeclareMathOperator{\id}{Id}
\DeclareMathOperator{\vect}{vect}
\DeclareMathOperator{\img}{img}
\DeclareMathOperator{\cov}{Cov}
\DeclareMathOperator{\dist}{dist}
\DeclareMathOperator{\irr}{Irr}
\DeclareMathOperator{\image}{Im}
\DeclareMathOperator{\pd}{\partial}
\DeclareMathOperator{\epi}{epi}
\DeclareMathOperator{\Argmin}{Argmin}
\DeclareMathOperator{\dom}{dom}
\DeclareMathOperator{\proj}{proj}
\DeclareMathOperator{\ctg}{ctg}
\DeclareMathOperator{\supp}{supp}
\DeclareMathOperator{\argmin}{argmin}
\DeclareMathOperator{\mult}{mult}
\DeclareMathOperator{\ch}{ch}
\DeclareMathOperator{\sh}{sh}
\DeclareMathOperator{\rang}{rang}
\DeclareMathOperator{\diam}{diam}
\DeclareMathOperator{\Epigraphe}{Epigraphe}




\usepackage{xcolor}
\everymath{\color{blue}}
%\everymath{\color[rgb]{0,1,1}}
%\pagecolor[rgb]{0,0,0.5}


\newcommand*{\pdtest}[3][]{\ensuremath{\frac{\partial^{#1} #2}{\partial #3}}}

\newcommand*{\deffunc}[6][]{\ensuremath{
\begin{array}{rcl}
#2 : #3 &\rightarrow& #4\\
#5 &\mapsto& #6
\end{array}
}}

\newcommand{\eqcolon}{\mathrel{\resizebox{\widthof{$\mathord{=}$}}{\height}{ $\!\!=\!\!\resizebox{1.2\width}{0.8\height}{\raisebox{0.23ex}{$\mathop{:}$}}\!\!$ }}}
\newcommand{\coloneq}{\mathrel{\resizebox{\widthof{$\mathord{=}$}}{\height}{ $\!\!\resizebox{1.2\width}{0.8\height}{\raisebox{0.23ex}{$\mathop{:}$}}\!\!=\!\!$ }}}
\newcommand{\eqcolonl}{\ensuremath{\mathrel{=\!\!\mathop{:}}}}
\newcommand{\coloneql}{\ensuremath{\mathrel{\mathop{:} \!\! =}}}
\newcommand{\vc}[1]{% inline column vector
  \left(\begin{smallmatrix}#1\end{smallmatrix}\right)%
}
\newcommand{\vr}[1]{% inline row vector
  \begin{smallmatrix}(\,#1\,)\end{smallmatrix}%
}
\makeatletter
\newcommand*{\defeq}{\ =\mathrel{\rlap{%
                     \raisebox{0.3ex}{$\m@th\cdot$}}%
                     \raisebox{-0.3ex}{$\m@th\cdot$}}%
                     }
\makeatother

\newcommand{\mathcircle}[1]{% inline row vector
 \overset{\circ}{#1}
}
\newcommand{\ulim}{% low limit
 \underline{\lim}
}
\newcommand{\ssi}{% iff
\iff
}
\newcommand{\ps}[2]{
\expval{#1 | #2}
}
\newcommand{\df}[1]{
\mqty{#1}
}
\newcommand{\n}[1]{
\norm{#1}
}
\newcommand{\sys}[1]{
\left\{\smqty{#1}\right.
}


\newcommand{\eqdef}{\ensuremath{\overset{\text{def}}=}}


\def\Circlearrowright{\ensuremath{%
  \rotatebox[origin=c]{230}{$\circlearrowright$}}}

\newcommand\ct[1]{\text{\rmfamily\upshape #1}}
\newcommand\question[1]{ {\color{red} ...!? \small #1}}
\newcommand\caz[1]{\left\{\begin{array} #1 \end{array}\right.}
\newcommand\const{\text{\rmfamily\upshape const}}
\newcommand\toP{ \overset{\pro}{\to}}
\newcommand\toPP{ \overset{\text{PP}}{\to}}
\newcommand{\oeq}{\mathrel{\text{\textcircled{$=$}}}}





\usepackage{xcolor}
% \usepackage[normalem]{ulem}
\usepackage{lipsum}
\makeatletter
% \newcommand\colorwave[1][blue]{\bgroup \markoverwith{\lower3.5\p@\hbox{\sixly \textcolor{#1}{\char58}}}\ULon}
%\font\sixly=lasy6 % does not re-load if already loaded, so no memory problem.

\newmdtheoremenv[
linewidth= 1pt,linecolor= blue,%
leftmargin=20,rightmargin=20,innertopmargin=0pt, innerrightmargin=40,%
tikzsetting = { draw=lightgray, line width = 0.3pt,dashed,%
dash pattern = on 15pt off 3pt},%
splittopskip=\topskip,skipbelow=\baselineskip,%
skipabove=\baselineskip,ntheorem,roundcorner=0pt,
% backgroundcolor=pagebg,font=\color{orange}\sffamily, fontcolor=white
]{examplebox}{Exemple}[section]



\newcommand\R{\mathbb{R}}
\newcommand\Z{\mathbb{Z}}
\newcommand\N{\mathbb{N}}
\newcommand\E{\mathbb{E}}
\newcommand\F{\mathcal{F}}
\newcommand\cH{\mathcal{H}}
\newcommand\V{\mathbb{V}}
\newcommand\dmo{ ^{-1} }
\newcommand\kapa{\kappa}
\newcommand\im{Im}
\newcommand\hs{\mathcal{H}}





\usepackage{soul}

\makeatletter
\newcommand*{\whiten}[1]{\llap{\textcolor{white}{{\the\SOUL@token}}\hspace{#1pt}}}
\DeclareRobustCommand*\myul{%
    \def\SOUL@everyspace{\underline{\space}\kern\z@}%
    \def\SOUL@everytoken{%
     \setbox0=\hbox{\the\SOUL@token}%
     \ifdim\dp0>\z@
        \raisebox{\dp0}{\underline{\phantom{\the\SOUL@token}}}%
        \whiten{1}\whiten{0}%
        \whiten{-1}\whiten{-2}%
        \llap{\the\SOUL@token}%
     \else
        \underline{\the\SOUL@token}%
     \fi}%
\SOUL@}
\makeatother

\newcommand*{\demp}{\fontfamily{lmtt}\selectfont}

\DeclareTextFontCommand{\textdemp}{\demp}

\begin{document}

\ifcomment
Multiline
comment
\fi
\ifcomment
\myul{Typesetting test}
% \color[rgb]{1,1,1}
$∑_i^n≠ 60º±∞π∆¬≈√j∫h≤≥µ$

$\CR \R\pro\ind\pro\gS\pro
\mqty[a&b\\c&d]$
$\pro\mathbb{P}$
$\dd{x}$

  \[
    \alpha(x)=\left\{
                \begin{array}{ll}
                  x\\
                  \frac{1}{1+e^{-kx}}\\
                  \frac{e^x-e^{-x}}{e^x+e^{-x}}
                \end{array}
              \right.
  \]

  $\expval{x}$
  
  $\chi_\rho(ghg\dmo)=\Tr(\rho_{ghg\dmo})=\Tr(\rho_g\circ\rho_h\circ\rho\dmo_g)=\Tr(\rho_h)\overset{\mbox{\scalebox{0.5}{$\Tr(AB)=\Tr(BA)$}}}{=}\chi_\rho(h)$
  	$\mathop{\oplus}_{\substack{x\in X}}$

$\mat(\rho_g)=(a_{ij}(g))_{\scriptsize \substack{1\leq i\leq d \\ 1\leq j\leq d}}$ et $\mat(\rho'_g)=(a'_{ij}(g))_{\scriptsize \substack{1\leq i'\leq d' \\ 1\leq j'\leq d'}}$



\[\int_a^b{\mathbb{R}^2}g(u, v)\dd{P_{XY}}(u, v)=\iint g(u,v) f_{XY}(u, v)\dd \lambda(u) \dd \lambda(v)\]
$$\lim_{x\to\infty} f(x)$$	
$$\iiiint_V \mu(t,u,v,w) \,dt\,du\,dv\,dw$$
$$\sum_{n=1}^{\infty} 2^{-n} = 1$$	
\begin{definition}
	Si $X$ et $Y$ sont 2 v.a. ou definit la \textsc{Covariance} entre $X$ et $Y$ comme
	$\cov(X,Y)\overset{\text{def}}{=}\E\left[(X-\E(X))(Y-\E(Y))\right]=\E(XY)-\E(X)\E(Y)$.
\end{definition}
\fi
\pagebreak

% \tableofcontents

% insert your code here
%% !TEX encoding = UTF-8 Unicode
% !TEX TS-program = xelatex

\documentclass[french]{report}

%\usepackage[utf8]{inputenc}
%\usepackage[T1]{fontenc}
\usepackage{babel}


\newif\ifcomment
%\commenttrue # Show comments

\usepackage{physics}
\usepackage{amssymb}


\usepackage{amsthm}
% \usepackage{thmtools}
\usepackage{mathtools}
\usepackage{amsfonts}

\usepackage{color}

\usepackage{tikz}

\usepackage{geometry}
\geometry{a5paper, margin=0.1in, right=1cm}

\usepackage{dsfont}

\usepackage{graphicx}
\graphicspath{ {images/} }

\usepackage{faktor}

\usepackage{IEEEtrantools}
\usepackage{enumerate}   
\usepackage[PostScript=dvips]{"/Users/aware/Documents/Courses/diagrams"}


\newtheorem{theorem}{Théorème}[section]
\renewcommand{\thetheorem}{\arabic{theorem}}
\newtheorem{lemme}{Lemme}[section]
\renewcommand{\thelemme}{\arabic{lemme}}
\newtheorem{proposition}{Proposition}[section]
\renewcommand{\theproposition}{\arabic{proposition}}
\newtheorem{notations}{Notations}[section]
\newtheorem{problem}{Problème}[section]
\newtheorem{corollary}{Corollaire}[theorem]
\renewcommand{\thecorollary}{\arabic{corollary}}
\newtheorem{property}{Propriété}[section]
\newtheorem{objective}{Objectif}[section]

\theoremstyle{definition}
\newtheorem{definition}{Définition}[section]
\renewcommand{\thedefinition}{\arabic{definition}}
\newtheorem{exercise}{Exercice}[chapter]
\renewcommand{\theexercise}{\arabic{exercise}}
\newtheorem{example}{Exemple}[chapter]
\renewcommand{\theexample}{\arabic{example}}
\newtheorem*{solution}{Solution}
\newtheorem*{application}{Application}
\newtheorem*{notation}{Notation}
\newtheorem*{vocabulary}{Vocabulaire}
\newtheorem*{properties}{Propriétés}



\theoremstyle{remark}
\newtheorem*{remark}{Remarque}
\newtheorem*{rappel}{Rappel}


\usepackage{etoolbox}
\AtBeginEnvironment{exercise}{\small}
\AtBeginEnvironment{example}{\small}

\usepackage{cases}
\usepackage[red]{mypack}

\usepackage[framemethod=TikZ]{mdframed}

\definecolor{bg}{rgb}{0.4,0.25,0.95}
\definecolor{pagebg}{rgb}{0,0,0.5}
\surroundwithmdframed[
   topline=false,
   rightline=false,
   bottomline=false,
   leftmargin=\parindent,
   skipabove=8pt,
   skipbelow=8pt,
   linecolor=blue,
   innerbottommargin=10pt,
   % backgroundcolor=bg,font=\color{orange}\sffamily, fontcolor=white
]{definition}

\usepackage{empheq}
\usepackage[most]{tcolorbox}

\newtcbox{\mymath}[1][]{%
    nobeforeafter, math upper, tcbox raise base,
    enhanced, colframe=blue!30!black,
    colback=red!10, boxrule=1pt,
    #1}

\usepackage{unixode}


\DeclareMathOperator{\ord}{ord}
\DeclareMathOperator{\orb}{orb}
\DeclareMathOperator{\stab}{stab}
\DeclareMathOperator{\Stab}{stab}
\DeclareMathOperator{\ppcm}{ppcm}
\DeclareMathOperator{\conj}{Conj}
\DeclareMathOperator{\End}{End}
\DeclareMathOperator{\rot}{rot}
\DeclareMathOperator{\trs}{trace}
\DeclareMathOperator{\Ind}{Ind}
\DeclareMathOperator{\mat}{Mat}
\DeclareMathOperator{\id}{Id}
\DeclareMathOperator{\vect}{vect}
\DeclareMathOperator{\img}{img}
\DeclareMathOperator{\cov}{Cov}
\DeclareMathOperator{\dist}{dist}
\DeclareMathOperator{\irr}{Irr}
\DeclareMathOperator{\image}{Im}
\DeclareMathOperator{\pd}{\partial}
\DeclareMathOperator{\epi}{epi}
\DeclareMathOperator{\Argmin}{Argmin}
\DeclareMathOperator{\dom}{dom}
\DeclareMathOperator{\proj}{proj}
\DeclareMathOperator{\ctg}{ctg}
\DeclareMathOperator{\supp}{supp}
\DeclareMathOperator{\argmin}{argmin}
\DeclareMathOperator{\mult}{mult}
\DeclareMathOperator{\ch}{ch}
\DeclareMathOperator{\sh}{sh}
\DeclareMathOperator{\rang}{rang}
\DeclareMathOperator{\diam}{diam}
\DeclareMathOperator{\Epigraphe}{Epigraphe}




\usepackage{xcolor}
\everymath{\color{blue}}
%\everymath{\color[rgb]{0,1,1}}
%\pagecolor[rgb]{0,0,0.5}


\newcommand*{\pdtest}[3][]{\ensuremath{\frac{\partial^{#1} #2}{\partial #3}}}

\newcommand*{\deffunc}[6][]{\ensuremath{
\begin{array}{rcl}
#2 : #3 &\rightarrow& #4\\
#5 &\mapsto& #6
\end{array}
}}

\newcommand{\eqcolon}{\mathrel{\resizebox{\widthof{$\mathord{=}$}}{\height}{ $\!\!=\!\!\resizebox{1.2\width}{0.8\height}{\raisebox{0.23ex}{$\mathop{:}$}}\!\!$ }}}
\newcommand{\coloneq}{\mathrel{\resizebox{\widthof{$\mathord{=}$}}{\height}{ $\!\!\resizebox{1.2\width}{0.8\height}{\raisebox{0.23ex}{$\mathop{:}$}}\!\!=\!\!$ }}}
\newcommand{\eqcolonl}{\ensuremath{\mathrel{=\!\!\mathop{:}}}}
\newcommand{\coloneql}{\ensuremath{\mathrel{\mathop{:} \!\! =}}}
\newcommand{\vc}[1]{% inline column vector
  \left(\begin{smallmatrix}#1\end{smallmatrix}\right)%
}
\newcommand{\vr}[1]{% inline row vector
  \begin{smallmatrix}(\,#1\,)\end{smallmatrix}%
}
\makeatletter
\newcommand*{\defeq}{\ =\mathrel{\rlap{%
                     \raisebox{0.3ex}{$\m@th\cdot$}}%
                     \raisebox{-0.3ex}{$\m@th\cdot$}}%
                     }
\makeatother

\newcommand{\mathcircle}[1]{% inline row vector
 \overset{\circ}{#1}
}
\newcommand{\ulim}{% low limit
 \underline{\lim}
}
\newcommand{\ssi}{% iff
\iff
}
\newcommand{\ps}[2]{
\expval{#1 | #2}
}
\newcommand{\df}[1]{
\mqty{#1}
}
\newcommand{\n}[1]{
\norm{#1}
}
\newcommand{\sys}[1]{
\left\{\smqty{#1}\right.
}


\newcommand{\eqdef}{\ensuremath{\overset{\text{def}}=}}


\def\Circlearrowright{\ensuremath{%
  \rotatebox[origin=c]{230}{$\circlearrowright$}}}

\newcommand\ct[1]{\text{\rmfamily\upshape #1}}
\newcommand\question[1]{ {\color{red} ...!? \small #1}}
\newcommand\caz[1]{\left\{\begin{array} #1 \end{array}\right.}
\newcommand\const{\text{\rmfamily\upshape const}}
\newcommand\toP{ \overset{\pro}{\to}}
\newcommand\toPP{ \overset{\text{PP}}{\to}}
\newcommand{\oeq}{\mathrel{\text{\textcircled{$=$}}}}





\usepackage{xcolor}
% \usepackage[normalem]{ulem}
\usepackage{lipsum}
\makeatletter
% \newcommand\colorwave[1][blue]{\bgroup \markoverwith{\lower3.5\p@\hbox{\sixly \textcolor{#1}{\char58}}}\ULon}
%\font\sixly=lasy6 % does not re-load if already loaded, so no memory problem.

\newmdtheoremenv[
linewidth= 1pt,linecolor= blue,%
leftmargin=20,rightmargin=20,innertopmargin=0pt, innerrightmargin=40,%
tikzsetting = { draw=lightgray, line width = 0.3pt,dashed,%
dash pattern = on 15pt off 3pt},%
splittopskip=\topskip,skipbelow=\baselineskip,%
skipabove=\baselineskip,ntheorem,roundcorner=0pt,
% backgroundcolor=pagebg,font=\color{orange}\sffamily, fontcolor=white
]{examplebox}{Exemple}[section]



\newcommand\R{\mathbb{R}}
\newcommand\Z{\mathbb{Z}}
\newcommand\N{\mathbb{N}}
\newcommand\E{\mathbb{E}}
\newcommand\F{\mathcal{F}}
\newcommand\cH{\mathcal{H}}
\newcommand\V{\mathbb{V}}
\newcommand\dmo{ ^{-1} }
\newcommand\kapa{\kappa}
\newcommand\im{Im}
\newcommand\hs{\mathcal{H}}





\usepackage{soul}

\makeatletter
\newcommand*{\whiten}[1]{\llap{\textcolor{white}{{\the\SOUL@token}}\hspace{#1pt}}}
\DeclareRobustCommand*\myul{%
    \def\SOUL@everyspace{\underline{\space}\kern\z@}%
    \def\SOUL@everytoken{%
     \setbox0=\hbox{\the\SOUL@token}%
     \ifdim\dp0>\z@
        \raisebox{\dp0}{\underline{\phantom{\the\SOUL@token}}}%
        \whiten{1}\whiten{0}%
        \whiten{-1}\whiten{-2}%
        \llap{\the\SOUL@token}%
     \else
        \underline{\the\SOUL@token}%
     \fi}%
\SOUL@}
\makeatother

\newcommand*{\demp}{\fontfamily{lmtt}\selectfont}

\DeclareTextFontCommand{\textdemp}{\demp}

\begin{document}

\ifcomment
Multiline
comment
\fi
\ifcomment
\myul{Typesetting test}
% \color[rgb]{1,1,1}
$∑_i^n≠ 60º±∞π∆¬≈√j∫h≤≥µ$

$\CR \R\pro\ind\pro\gS\pro
\mqty[a&b\\c&d]$
$\pro\mathbb{P}$
$\dd{x}$

  \[
    \alpha(x)=\left\{
                \begin{array}{ll}
                  x\\
                  \frac{1}{1+e^{-kx}}\\
                  \frac{e^x-e^{-x}}{e^x+e^{-x}}
                \end{array}
              \right.
  \]

  $\expval{x}$
  
  $\chi_\rho(ghg\dmo)=\Tr(\rho_{ghg\dmo})=\Tr(\rho_g\circ\rho_h\circ\rho\dmo_g)=\Tr(\rho_h)\overset{\mbox{\scalebox{0.5}{$\Tr(AB)=\Tr(BA)$}}}{=}\chi_\rho(h)$
  	$\mathop{\oplus}_{\substack{x\in X}}$

$\mat(\rho_g)=(a_{ij}(g))_{\scriptsize \substack{1\leq i\leq d \\ 1\leq j\leq d}}$ et $\mat(\rho'_g)=(a'_{ij}(g))_{\scriptsize \substack{1\leq i'\leq d' \\ 1\leq j'\leq d'}}$



\[\int_a^b{\mathbb{R}^2}g(u, v)\dd{P_{XY}}(u, v)=\iint g(u,v) f_{XY}(u, v)\dd \lambda(u) \dd \lambda(v)\]
$$\lim_{x\to\infty} f(x)$$	
$$\iiiint_V \mu(t,u,v,w) \,dt\,du\,dv\,dw$$
$$\sum_{n=1}^{\infty} 2^{-n} = 1$$	
\begin{definition}
	Si $X$ et $Y$ sont 2 v.a. ou definit la \textsc{Covariance} entre $X$ et $Y$ comme
	$\cov(X,Y)\overset{\text{def}}{=}\E\left[(X-\E(X))(Y-\E(Y))\right]=\E(XY)-\E(X)\E(Y)$.
\end{definition}
\fi
\pagebreak

% \tableofcontents

% insert your code here
%% !TEX encoding = UTF-8 Unicode
% !TEX TS-program = xelatex

\documentclass[french]{report}

%\usepackage[utf8]{inputenc}
%\usepackage[T1]{fontenc}
\usepackage{babel}


\newif\ifcomment
%\commenttrue # Show comments

\usepackage{physics}
\usepackage{amssymb}


\usepackage{amsthm}
% \usepackage{thmtools}
\usepackage{mathtools}
\usepackage{amsfonts}

\usepackage{color}

\usepackage{tikz}

\usepackage{geometry}
\geometry{a5paper, margin=0.1in, right=1cm}

\usepackage{dsfont}

\usepackage{graphicx}
\graphicspath{ {images/} }

\usepackage{faktor}

\usepackage{IEEEtrantools}
\usepackage{enumerate}   
\usepackage[PostScript=dvips]{"/Users/aware/Documents/Courses/diagrams"}


\newtheorem{theorem}{Théorème}[section]
\renewcommand{\thetheorem}{\arabic{theorem}}
\newtheorem{lemme}{Lemme}[section]
\renewcommand{\thelemme}{\arabic{lemme}}
\newtheorem{proposition}{Proposition}[section]
\renewcommand{\theproposition}{\arabic{proposition}}
\newtheorem{notations}{Notations}[section]
\newtheorem{problem}{Problème}[section]
\newtheorem{corollary}{Corollaire}[theorem]
\renewcommand{\thecorollary}{\arabic{corollary}}
\newtheorem{property}{Propriété}[section]
\newtheorem{objective}{Objectif}[section]

\theoremstyle{definition}
\newtheorem{definition}{Définition}[section]
\renewcommand{\thedefinition}{\arabic{definition}}
\newtheorem{exercise}{Exercice}[chapter]
\renewcommand{\theexercise}{\arabic{exercise}}
\newtheorem{example}{Exemple}[chapter]
\renewcommand{\theexample}{\arabic{example}}
\newtheorem*{solution}{Solution}
\newtheorem*{application}{Application}
\newtheorem*{notation}{Notation}
\newtheorem*{vocabulary}{Vocabulaire}
\newtheorem*{properties}{Propriétés}



\theoremstyle{remark}
\newtheorem*{remark}{Remarque}
\newtheorem*{rappel}{Rappel}


\usepackage{etoolbox}
\AtBeginEnvironment{exercise}{\small}
\AtBeginEnvironment{example}{\small}

\usepackage{cases}
\usepackage[red]{mypack}

\usepackage[framemethod=TikZ]{mdframed}

\definecolor{bg}{rgb}{0.4,0.25,0.95}
\definecolor{pagebg}{rgb}{0,0,0.5}
\surroundwithmdframed[
   topline=false,
   rightline=false,
   bottomline=false,
   leftmargin=\parindent,
   skipabove=8pt,
   skipbelow=8pt,
   linecolor=blue,
   innerbottommargin=10pt,
   % backgroundcolor=bg,font=\color{orange}\sffamily, fontcolor=white
]{definition}

\usepackage{empheq}
\usepackage[most]{tcolorbox}

\newtcbox{\mymath}[1][]{%
    nobeforeafter, math upper, tcbox raise base,
    enhanced, colframe=blue!30!black,
    colback=red!10, boxrule=1pt,
    #1}

\usepackage{unixode}


\DeclareMathOperator{\ord}{ord}
\DeclareMathOperator{\orb}{orb}
\DeclareMathOperator{\stab}{stab}
\DeclareMathOperator{\Stab}{stab}
\DeclareMathOperator{\ppcm}{ppcm}
\DeclareMathOperator{\conj}{Conj}
\DeclareMathOperator{\End}{End}
\DeclareMathOperator{\rot}{rot}
\DeclareMathOperator{\trs}{trace}
\DeclareMathOperator{\Ind}{Ind}
\DeclareMathOperator{\mat}{Mat}
\DeclareMathOperator{\id}{Id}
\DeclareMathOperator{\vect}{vect}
\DeclareMathOperator{\img}{img}
\DeclareMathOperator{\cov}{Cov}
\DeclareMathOperator{\dist}{dist}
\DeclareMathOperator{\irr}{Irr}
\DeclareMathOperator{\image}{Im}
\DeclareMathOperator{\pd}{\partial}
\DeclareMathOperator{\epi}{epi}
\DeclareMathOperator{\Argmin}{Argmin}
\DeclareMathOperator{\dom}{dom}
\DeclareMathOperator{\proj}{proj}
\DeclareMathOperator{\ctg}{ctg}
\DeclareMathOperator{\supp}{supp}
\DeclareMathOperator{\argmin}{argmin}
\DeclareMathOperator{\mult}{mult}
\DeclareMathOperator{\ch}{ch}
\DeclareMathOperator{\sh}{sh}
\DeclareMathOperator{\rang}{rang}
\DeclareMathOperator{\diam}{diam}
\DeclareMathOperator{\Epigraphe}{Epigraphe}




\usepackage{xcolor}
\everymath{\color{blue}}
%\everymath{\color[rgb]{0,1,1}}
%\pagecolor[rgb]{0,0,0.5}


\newcommand*{\pdtest}[3][]{\ensuremath{\frac{\partial^{#1} #2}{\partial #3}}}

\newcommand*{\deffunc}[6][]{\ensuremath{
\begin{array}{rcl}
#2 : #3 &\rightarrow& #4\\
#5 &\mapsto& #6
\end{array}
}}

\newcommand{\eqcolon}{\mathrel{\resizebox{\widthof{$\mathord{=}$}}{\height}{ $\!\!=\!\!\resizebox{1.2\width}{0.8\height}{\raisebox{0.23ex}{$\mathop{:}$}}\!\!$ }}}
\newcommand{\coloneq}{\mathrel{\resizebox{\widthof{$\mathord{=}$}}{\height}{ $\!\!\resizebox{1.2\width}{0.8\height}{\raisebox{0.23ex}{$\mathop{:}$}}\!\!=\!\!$ }}}
\newcommand{\eqcolonl}{\ensuremath{\mathrel{=\!\!\mathop{:}}}}
\newcommand{\coloneql}{\ensuremath{\mathrel{\mathop{:} \!\! =}}}
\newcommand{\vc}[1]{% inline column vector
  \left(\begin{smallmatrix}#1\end{smallmatrix}\right)%
}
\newcommand{\vr}[1]{% inline row vector
  \begin{smallmatrix}(\,#1\,)\end{smallmatrix}%
}
\makeatletter
\newcommand*{\defeq}{\ =\mathrel{\rlap{%
                     \raisebox{0.3ex}{$\m@th\cdot$}}%
                     \raisebox{-0.3ex}{$\m@th\cdot$}}%
                     }
\makeatother

\newcommand{\mathcircle}[1]{% inline row vector
 \overset{\circ}{#1}
}
\newcommand{\ulim}{% low limit
 \underline{\lim}
}
\newcommand{\ssi}{% iff
\iff
}
\newcommand{\ps}[2]{
\expval{#1 | #2}
}
\newcommand{\df}[1]{
\mqty{#1}
}
\newcommand{\n}[1]{
\norm{#1}
}
\newcommand{\sys}[1]{
\left\{\smqty{#1}\right.
}


\newcommand{\eqdef}{\ensuremath{\overset{\text{def}}=}}


\def\Circlearrowright{\ensuremath{%
  \rotatebox[origin=c]{230}{$\circlearrowright$}}}

\newcommand\ct[1]{\text{\rmfamily\upshape #1}}
\newcommand\question[1]{ {\color{red} ...!? \small #1}}
\newcommand\caz[1]{\left\{\begin{array} #1 \end{array}\right.}
\newcommand\const{\text{\rmfamily\upshape const}}
\newcommand\toP{ \overset{\pro}{\to}}
\newcommand\toPP{ \overset{\text{PP}}{\to}}
\newcommand{\oeq}{\mathrel{\text{\textcircled{$=$}}}}





\usepackage{xcolor}
% \usepackage[normalem]{ulem}
\usepackage{lipsum}
\makeatletter
% \newcommand\colorwave[1][blue]{\bgroup \markoverwith{\lower3.5\p@\hbox{\sixly \textcolor{#1}{\char58}}}\ULon}
%\font\sixly=lasy6 % does not re-load if already loaded, so no memory problem.

\newmdtheoremenv[
linewidth= 1pt,linecolor= blue,%
leftmargin=20,rightmargin=20,innertopmargin=0pt, innerrightmargin=40,%
tikzsetting = { draw=lightgray, line width = 0.3pt,dashed,%
dash pattern = on 15pt off 3pt},%
splittopskip=\topskip,skipbelow=\baselineskip,%
skipabove=\baselineskip,ntheorem,roundcorner=0pt,
% backgroundcolor=pagebg,font=\color{orange}\sffamily, fontcolor=white
]{examplebox}{Exemple}[section]



\newcommand\R{\mathbb{R}}
\newcommand\Z{\mathbb{Z}}
\newcommand\N{\mathbb{N}}
\newcommand\E{\mathbb{E}}
\newcommand\F{\mathcal{F}}
\newcommand\cH{\mathcal{H}}
\newcommand\V{\mathbb{V}}
\newcommand\dmo{ ^{-1} }
\newcommand\kapa{\kappa}
\newcommand\im{Im}
\newcommand\hs{\mathcal{H}}





\usepackage{soul}

\makeatletter
\newcommand*{\whiten}[1]{\llap{\textcolor{white}{{\the\SOUL@token}}\hspace{#1pt}}}
\DeclareRobustCommand*\myul{%
    \def\SOUL@everyspace{\underline{\space}\kern\z@}%
    \def\SOUL@everytoken{%
     \setbox0=\hbox{\the\SOUL@token}%
     \ifdim\dp0>\z@
        \raisebox{\dp0}{\underline{\phantom{\the\SOUL@token}}}%
        \whiten{1}\whiten{0}%
        \whiten{-1}\whiten{-2}%
        \llap{\the\SOUL@token}%
     \else
        \underline{\the\SOUL@token}%
     \fi}%
\SOUL@}
\makeatother

\newcommand*{\demp}{\fontfamily{lmtt}\selectfont}

\DeclareTextFontCommand{\textdemp}{\demp}

\begin{document}

\ifcomment
Multiline
comment
\fi
\ifcomment
\myul{Typesetting test}
% \color[rgb]{1,1,1}
$∑_i^n≠ 60º±∞π∆¬≈√j∫h≤≥µ$

$\CR \R\pro\ind\pro\gS\pro
\mqty[a&b\\c&d]$
$\pro\mathbb{P}$
$\dd{x}$

  \[
    \alpha(x)=\left\{
                \begin{array}{ll}
                  x\\
                  \frac{1}{1+e^{-kx}}\\
                  \frac{e^x-e^{-x}}{e^x+e^{-x}}
                \end{array}
              \right.
  \]

  $\expval{x}$
  
  $\chi_\rho(ghg\dmo)=\Tr(\rho_{ghg\dmo})=\Tr(\rho_g\circ\rho_h\circ\rho\dmo_g)=\Tr(\rho_h)\overset{\mbox{\scalebox{0.5}{$\Tr(AB)=\Tr(BA)$}}}{=}\chi_\rho(h)$
  	$\mathop{\oplus}_{\substack{x\in X}}$

$\mat(\rho_g)=(a_{ij}(g))_{\scriptsize \substack{1\leq i\leq d \\ 1\leq j\leq d}}$ et $\mat(\rho'_g)=(a'_{ij}(g))_{\scriptsize \substack{1\leq i'\leq d' \\ 1\leq j'\leq d'}}$



\[\int_a^b{\mathbb{R}^2}g(u, v)\dd{P_{XY}}(u, v)=\iint g(u,v) f_{XY}(u, v)\dd \lambda(u) \dd \lambda(v)\]
$$\lim_{x\to\infty} f(x)$$	
$$\iiiint_V \mu(t,u,v,w) \,dt\,du\,dv\,dw$$
$$\sum_{n=1}^{\infty} 2^{-n} = 1$$	
\begin{definition}
	Si $X$ et $Y$ sont 2 v.a. ou definit la \textsc{Covariance} entre $X$ et $Y$ comme
	$\cov(X,Y)\overset{\text{def}}{=}\E\left[(X-\E(X))(Y-\E(Y))\right]=\E(XY)-\E(X)\E(Y)$.
\end{definition}
\fi
\pagebreak

% \tableofcontents

% insert your code here
%\input{./algebra/main.tex}
%\input{./geometrie-differentielle/main.tex}
%\input{./probabilite/main.tex}
%\input{./analyse-fonctionnelle/main.tex}
% \input{./Analyse-convexe-et-dualite-en-optimisation/main.tex}
%\input{./tikz/main.tex}
%\input{./Theorie-du-distributions/main.tex}
%\input{./optimisation/mine.tex}
 \input{./modelisation/main.tex}

% yves.aubry@univ-tln.fr : algebra

\end{document}

%% !TEX encoding = UTF-8 Unicode
% !TEX TS-program = xelatex

\documentclass[french]{report}

%\usepackage[utf8]{inputenc}
%\usepackage[T1]{fontenc}
\usepackage{babel}


\newif\ifcomment
%\commenttrue # Show comments

\usepackage{physics}
\usepackage{amssymb}


\usepackage{amsthm}
% \usepackage{thmtools}
\usepackage{mathtools}
\usepackage{amsfonts}

\usepackage{color}

\usepackage{tikz}

\usepackage{geometry}
\geometry{a5paper, margin=0.1in, right=1cm}

\usepackage{dsfont}

\usepackage{graphicx}
\graphicspath{ {images/} }

\usepackage{faktor}

\usepackage{IEEEtrantools}
\usepackage{enumerate}   
\usepackage[PostScript=dvips]{"/Users/aware/Documents/Courses/diagrams"}


\newtheorem{theorem}{Théorème}[section]
\renewcommand{\thetheorem}{\arabic{theorem}}
\newtheorem{lemme}{Lemme}[section]
\renewcommand{\thelemme}{\arabic{lemme}}
\newtheorem{proposition}{Proposition}[section]
\renewcommand{\theproposition}{\arabic{proposition}}
\newtheorem{notations}{Notations}[section]
\newtheorem{problem}{Problème}[section]
\newtheorem{corollary}{Corollaire}[theorem]
\renewcommand{\thecorollary}{\arabic{corollary}}
\newtheorem{property}{Propriété}[section]
\newtheorem{objective}{Objectif}[section]

\theoremstyle{definition}
\newtheorem{definition}{Définition}[section]
\renewcommand{\thedefinition}{\arabic{definition}}
\newtheorem{exercise}{Exercice}[chapter]
\renewcommand{\theexercise}{\arabic{exercise}}
\newtheorem{example}{Exemple}[chapter]
\renewcommand{\theexample}{\arabic{example}}
\newtheorem*{solution}{Solution}
\newtheorem*{application}{Application}
\newtheorem*{notation}{Notation}
\newtheorem*{vocabulary}{Vocabulaire}
\newtheorem*{properties}{Propriétés}



\theoremstyle{remark}
\newtheorem*{remark}{Remarque}
\newtheorem*{rappel}{Rappel}


\usepackage{etoolbox}
\AtBeginEnvironment{exercise}{\small}
\AtBeginEnvironment{example}{\small}

\usepackage{cases}
\usepackage[red]{mypack}

\usepackage[framemethod=TikZ]{mdframed}

\definecolor{bg}{rgb}{0.4,0.25,0.95}
\definecolor{pagebg}{rgb}{0,0,0.5}
\surroundwithmdframed[
   topline=false,
   rightline=false,
   bottomline=false,
   leftmargin=\parindent,
   skipabove=8pt,
   skipbelow=8pt,
   linecolor=blue,
   innerbottommargin=10pt,
   % backgroundcolor=bg,font=\color{orange}\sffamily, fontcolor=white
]{definition}

\usepackage{empheq}
\usepackage[most]{tcolorbox}

\newtcbox{\mymath}[1][]{%
    nobeforeafter, math upper, tcbox raise base,
    enhanced, colframe=blue!30!black,
    colback=red!10, boxrule=1pt,
    #1}

\usepackage{unixode}


\DeclareMathOperator{\ord}{ord}
\DeclareMathOperator{\orb}{orb}
\DeclareMathOperator{\stab}{stab}
\DeclareMathOperator{\Stab}{stab}
\DeclareMathOperator{\ppcm}{ppcm}
\DeclareMathOperator{\conj}{Conj}
\DeclareMathOperator{\End}{End}
\DeclareMathOperator{\rot}{rot}
\DeclareMathOperator{\trs}{trace}
\DeclareMathOperator{\Ind}{Ind}
\DeclareMathOperator{\mat}{Mat}
\DeclareMathOperator{\id}{Id}
\DeclareMathOperator{\vect}{vect}
\DeclareMathOperator{\img}{img}
\DeclareMathOperator{\cov}{Cov}
\DeclareMathOperator{\dist}{dist}
\DeclareMathOperator{\irr}{Irr}
\DeclareMathOperator{\image}{Im}
\DeclareMathOperator{\pd}{\partial}
\DeclareMathOperator{\epi}{epi}
\DeclareMathOperator{\Argmin}{Argmin}
\DeclareMathOperator{\dom}{dom}
\DeclareMathOperator{\proj}{proj}
\DeclareMathOperator{\ctg}{ctg}
\DeclareMathOperator{\supp}{supp}
\DeclareMathOperator{\argmin}{argmin}
\DeclareMathOperator{\mult}{mult}
\DeclareMathOperator{\ch}{ch}
\DeclareMathOperator{\sh}{sh}
\DeclareMathOperator{\rang}{rang}
\DeclareMathOperator{\diam}{diam}
\DeclareMathOperator{\Epigraphe}{Epigraphe}




\usepackage{xcolor}
\everymath{\color{blue}}
%\everymath{\color[rgb]{0,1,1}}
%\pagecolor[rgb]{0,0,0.5}


\newcommand*{\pdtest}[3][]{\ensuremath{\frac{\partial^{#1} #2}{\partial #3}}}

\newcommand*{\deffunc}[6][]{\ensuremath{
\begin{array}{rcl}
#2 : #3 &\rightarrow& #4\\
#5 &\mapsto& #6
\end{array}
}}

\newcommand{\eqcolon}{\mathrel{\resizebox{\widthof{$\mathord{=}$}}{\height}{ $\!\!=\!\!\resizebox{1.2\width}{0.8\height}{\raisebox{0.23ex}{$\mathop{:}$}}\!\!$ }}}
\newcommand{\coloneq}{\mathrel{\resizebox{\widthof{$\mathord{=}$}}{\height}{ $\!\!\resizebox{1.2\width}{0.8\height}{\raisebox{0.23ex}{$\mathop{:}$}}\!\!=\!\!$ }}}
\newcommand{\eqcolonl}{\ensuremath{\mathrel{=\!\!\mathop{:}}}}
\newcommand{\coloneql}{\ensuremath{\mathrel{\mathop{:} \!\! =}}}
\newcommand{\vc}[1]{% inline column vector
  \left(\begin{smallmatrix}#1\end{smallmatrix}\right)%
}
\newcommand{\vr}[1]{% inline row vector
  \begin{smallmatrix}(\,#1\,)\end{smallmatrix}%
}
\makeatletter
\newcommand*{\defeq}{\ =\mathrel{\rlap{%
                     \raisebox{0.3ex}{$\m@th\cdot$}}%
                     \raisebox{-0.3ex}{$\m@th\cdot$}}%
                     }
\makeatother

\newcommand{\mathcircle}[1]{% inline row vector
 \overset{\circ}{#1}
}
\newcommand{\ulim}{% low limit
 \underline{\lim}
}
\newcommand{\ssi}{% iff
\iff
}
\newcommand{\ps}[2]{
\expval{#1 | #2}
}
\newcommand{\df}[1]{
\mqty{#1}
}
\newcommand{\n}[1]{
\norm{#1}
}
\newcommand{\sys}[1]{
\left\{\smqty{#1}\right.
}


\newcommand{\eqdef}{\ensuremath{\overset{\text{def}}=}}


\def\Circlearrowright{\ensuremath{%
  \rotatebox[origin=c]{230}{$\circlearrowright$}}}

\newcommand\ct[1]{\text{\rmfamily\upshape #1}}
\newcommand\question[1]{ {\color{red} ...!? \small #1}}
\newcommand\caz[1]{\left\{\begin{array} #1 \end{array}\right.}
\newcommand\const{\text{\rmfamily\upshape const}}
\newcommand\toP{ \overset{\pro}{\to}}
\newcommand\toPP{ \overset{\text{PP}}{\to}}
\newcommand{\oeq}{\mathrel{\text{\textcircled{$=$}}}}





\usepackage{xcolor}
% \usepackage[normalem]{ulem}
\usepackage{lipsum}
\makeatletter
% \newcommand\colorwave[1][blue]{\bgroup \markoverwith{\lower3.5\p@\hbox{\sixly \textcolor{#1}{\char58}}}\ULon}
%\font\sixly=lasy6 % does not re-load if already loaded, so no memory problem.

\newmdtheoremenv[
linewidth= 1pt,linecolor= blue,%
leftmargin=20,rightmargin=20,innertopmargin=0pt, innerrightmargin=40,%
tikzsetting = { draw=lightgray, line width = 0.3pt,dashed,%
dash pattern = on 15pt off 3pt},%
splittopskip=\topskip,skipbelow=\baselineskip,%
skipabove=\baselineskip,ntheorem,roundcorner=0pt,
% backgroundcolor=pagebg,font=\color{orange}\sffamily, fontcolor=white
]{examplebox}{Exemple}[section]



\newcommand\R{\mathbb{R}}
\newcommand\Z{\mathbb{Z}}
\newcommand\N{\mathbb{N}}
\newcommand\E{\mathbb{E}}
\newcommand\F{\mathcal{F}}
\newcommand\cH{\mathcal{H}}
\newcommand\V{\mathbb{V}}
\newcommand\dmo{ ^{-1} }
\newcommand\kapa{\kappa}
\newcommand\im{Im}
\newcommand\hs{\mathcal{H}}





\usepackage{soul}

\makeatletter
\newcommand*{\whiten}[1]{\llap{\textcolor{white}{{\the\SOUL@token}}\hspace{#1pt}}}
\DeclareRobustCommand*\myul{%
    \def\SOUL@everyspace{\underline{\space}\kern\z@}%
    \def\SOUL@everytoken{%
     \setbox0=\hbox{\the\SOUL@token}%
     \ifdim\dp0>\z@
        \raisebox{\dp0}{\underline{\phantom{\the\SOUL@token}}}%
        \whiten{1}\whiten{0}%
        \whiten{-1}\whiten{-2}%
        \llap{\the\SOUL@token}%
     \else
        \underline{\the\SOUL@token}%
     \fi}%
\SOUL@}
\makeatother

\newcommand*{\demp}{\fontfamily{lmtt}\selectfont}

\DeclareTextFontCommand{\textdemp}{\demp}

\begin{document}

\ifcomment
Multiline
comment
\fi
\ifcomment
\myul{Typesetting test}
% \color[rgb]{1,1,1}
$∑_i^n≠ 60º±∞π∆¬≈√j∫h≤≥µ$

$\CR \R\pro\ind\pro\gS\pro
\mqty[a&b\\c&d]$
$\pro\mathbb{P}$
$\dd{x}$

  \[
    \alpha(x)=\left\{
                \begin{array}{ll}
                  x\\
                  \frac{1}{1+e^{-kx}}\\
                  \frac{e^x-e^{-x}}{e^x+e^{-x}}
                \end{array}
              \right.
  \]

  $\expval{x}$
  
  $\chi_\rho(ghg\dmo)=\Tr(\rho_{ghg\dmo})=\Tr(\rho_g\circ\rho_h\circ\rho\dmo_g)=\Tr(\rho_h)\overset{\mbox{\scalebox{0.5}{$\Tr(AB)=\Tr(BA)$}}}{=}\chi_\rho(h)$
  	$\mathop{\oplus}_{\substack{x\in X}}$

$\mat(\rho_g)=(a_{ij}(g))_{\scriptsize \substack{1\leq i\leq d \\ 1\leq j\leq d}}$ et $\mat(\rho'_g)=(a'_{ij}(g))_{\scriptsize \substack{1\leq i'\leq d' \\ 1\leq j'\leq d'}}$



\[\int_a^b{\mathbb{R}^2}g(u, v)\dd{P_{XY}}(u, v)=\iint g(u,v) f_{XY}(u, v)\dd \lambda(u) \dd \lambda(v)\]
$$\lim_{x\to\infty} f(x)$$	
$$\iiiint_V \mu(t,u,v,w) \,dt\,du\,dv\,dw$$
$$\sum_{n=1}^{\infty} 2^{-n} = 1$$	
\begin{definition}
	Si $X$ et $Y$ sont 2 v.a. ou definit la \textsc{Covariance} entre $X$ et $Y$ comme
	$\cov(X,Y)\overset{\text{def}}{=}\E\left[(X-\E(X))(Y-\E(Y))\right]=\E(XY)-\E(X)\E(Y)$.
\end{definition}
\fi
\pagebreak

% \tableofcontents

% insert your code here
%\input{./algebra/main.tex}
%\input{./geometrie-differentielle/main.tex}
%\input{./probabilite/main.tex}
%\input{./analyse-fonctionnelle/main.tex}
% \input{./Analyse-convexe-et-dualite-en-optimisation/main.tex}
%\input{./tikz/main.tex}
%\input{./Theorie-du-distributions/main.tex}
%\input{./optimisation/mine.tex}
 \input{./modelisation/main.tex}

% yves.aubry@univ-tln.fr : algebra

\end{document}

%% !TEX encoding = UTF-8 Unicode
% !TEX TS-program = xelatex

\documentclass[french]{report}

%\usepackage[utf8]{inputenc}
%\usepackage[T1]{fontenc}
\usepackage{babel}


\newif\ifcomment
%\commenttrue # Show comments

\usepackage{physics}
\usepackage{amssymb}


\usepackage{amsthm}
% \usepackage{thmtools}
\usepackage{mathtools}
\usepackage{amsfonts}

\usepackage{color}

\usepackage{tikz}

\usepackage{geometry}
\geometry{a5paper, margin=0.1in, right=1cm}

\usepackage{dsfont}

\usepackage{graphicx}
\graphicspath{ {images/} }

\usepackage{faktor}

\usepackage{IEEEtrantools}
\usepackage{enumerate}   
\usepackage[PostScript=dvips]{"/Users/aware/Documents/Courses/diagrams"}


\newtheorem{theorem}{Théorème}[section]
\renewcommand{\thetheorem}{\arabic{theorem}}
\newtheorem{lemme}{Lemme}[section]
\renewcommand{\thelemme}{\arabic{lemme}}
\newtheorem{proposition}{Proposition}[section]
\renewcommand{\theproposition}{\arabic{proposition}}
\newtheorem{notations}{Notations}[section]
\newtheorem{problem}{Problème}[section]
\newtheorem{corollary}{Corollaire}[theorem]
\renewcommand{\thecorollary}{\arabic{corollary}}
\newtheorem{property}{Propriété}[section]
\newtheorem{objective}{Objectif}[section]

\theoremstyle{definition}
\newtheorem{definition}{Définition}[section]
\renewcommand{\thedefinition}{\arabic{definition}}
\newtheorem{exercise}{Exercice}[chapter]
\renewcommand{\theexercise}{\arabic{exercise}}
\newtheorem{example}{Exemple}[chapter]
\renewcommand{\theexample}{\arabic{example}}
\newtheorem*{solution}{Solution}
\newtheorem*{application}{Application}
\newtheorem*{notation}{Notation}
\newtheorem*{vocabulary}{Vocabulaire}
\newtheorem*{properties}{Propriétés}



\theoremstyle{remark}
\newtheorem*{remark}{Remarque}
\newtheorem*{rappel}{Rappel}


\usepackage{etoolbox}
\AtBeginEnvironment{exercise}{\small}
\AtBeginEnvironment{example}{\small}

\usepackage{cases}
\usepackage[red]{mypack}

\usepackage[framemethod=TikZ]{mdframed}

\definecolor{bg}{rgb}{0.4,0.25,0.95}
\definecolor{pagebg}{rgb}{0,0,0.5}
\surroundwithmdframed[
   topline=false,
   rightline=false,
   bottomline=false,
   leftmargin=\parindent,
   skipabove=8pt,
   skipbelow=8pt,
   linecolor=blue,
   innerbottommargin=10pt,
   % backgroundcolor=bg,font=\color{orange}\sffamily, fontcolor=white
]{definition}

\usepackage{empheq}
\usepackage[most]{tcolorbox}

\newtcbox{\mymath}[1][]{%
    nobeforeafter, math upper, tcbox raise base,
    enhanced, colframe=blue!30!black,
    colback=red!10, boxrule=1pt,
    #1}

\usepackage{unixode}


\DeclareMathOperator{\ord}{ord}
\DeclareMathOperator{\orb}{orb}
\DeclareMathOperator{\stab}{stab}
\DeclareMathOperator{\Stab}{stab}
\DeclareMathOperator{\ppcm}{ppcm}
\DeclareMathOperator{\conj}{Conj}
\DeclareMathOperator{\End}{End}
\DeclareMathOperator{\rot}{rot}
\DeclareMathOperator{\trs}{trace}
\DeclareMathOperator{\Ind}{Ind}
\DeclareMathOperator{\mat}{Mat}
\DeclareMathOperator{\id}{Id}
\DeclareMathOperator{\vect}{vect}
\DeclareMathOperator{\img}{img}
\DeclareMathOperator{\cov}{Cov}
\DeclareMathOperator{\dist}{dist}
\DeclareMathOperator{\irr}{Irr}
\DeclareMathOperator{\image}{Im}
\DeclareMathOperator{\pd}{\partial}
\DeclareMathOperator{\epi}{epi}
\DeclareMathOperator{\Argmin}{Argmin}
\DeclareMathOperator{\dom}{dom}
\DeclareMathOperator{\proj}{proj}
\DeclareMathOperator{\ctg}{ctg}
\DeclareMathOperator{\supp}{supp}
\DeclareMathOperator{\argmin}{argmin}
\DeclareMathOperator{\mult}{mult}
\DeclareMathOperator{\ch}{ch}
\DeclareMathOperator{\sh}{sh}
\DeclareMathOperator{\rang}{rang}
\DeclareMathOperator{\diam}{diam}
\DeclareMathOperator{\Epigraphe}{Epigraphe}




\usepackage{xcolor}
\everymath{\color{blue}}
%\everymath{\color[rgb]{0,1,1}}
%\pagecolor[rgb]{0,0,0.5}


\newcommand*{\pdtest}[3][]{\ensuremath{\frac{\partial^{#1} #2}{\partial #3}}}

\newcommand*{\deffunc}[6][]{\ensuremath{
\begin{array}{rcl}
#2 : #3 &\rightarrow& #4\\
#5 &\mapsto& #6
\end{array}
}}

\newcommand{\eqcolon}{\mathrel{\resizebox{\widthof{$\mathord{=}$}}{\height}{ $\!\!=\!\!\resizebox{1.2\width}{0.8\height}{\raisebox{0.23ex}{$\mathop{:}$}}\!\!$ }}}
\newcommand{\coloneq}{\mathrel{\resizebox{\widthof{$\mathord{=}$}}{\height}{ $\!\!\resizebox{1.2\width}{0.8\height}{\raisebox{0.23ex}{$\mathop{:}$}}\!\!=\!\!$ }}}
\newcommand{\eqcolonl}{\ensuremath{\mathrel{=\!\!\mathop{:}}}}
\newcommand{\coloneql}{\ensuremath{\mathrel{\mathop{:} \!\! =}}}
\newcommand{\vc}[1]{% inline column vector
  \left(\begin{smallmatrix}#1\end{smallmatrix}\right)%
}
\newcommand{\vr}[1]{% inline row vector
  \begin{smallmatrix}(\,#1\,)\end{smallmatrix}%
}
\makeatletter
\newcommand*{\defeq}{\ =\mathrel{\rlap{%
                     \raisebox{0.3ex}{$\m@th\cdot$}}%
                     \raisebox{-0.3ex}{$\m@th\cdot$}}%
                     }
\makeatother

\newcommand{\mathcircle}[1]{% inline row vector
 \overset{\circ}{#1}
}
\newcommand{\ulim}{% low limit
 \underline{\lim}
}
\newcommand{\ssi}{% iff
\iff
}
\newcommand{\ps}[2]{
\expval{#1 | #2}
}
\newcommand{\df}[1]{
\mqty{#1}
}
\newcommand{\n}[1]{
\norm{#1}
}
\newcommand{\sys}[1]{
\left\{\smqty{#1}\right.
}


\newcommand{\eqdef}{\ensuremath{\overset{\text{def}}=}}


\def\Circlearrowright{\ensuremath{%
  \rotatebox[origin=c]{230}{$\circlearrowright$}}}

\newcommand\ct[1]{\text{\rmfamily\upshape #1}}
\newcommand\question[1]{ {\color{red} ...!? \small #1}}
\newcommand\caz[1]{\left\{\begin{array} #1 \end{array}\right.}
\newcommand\const{\text{\rmfamily\upshape const}}
\newcommand\toP{ \overset{\pro}{\to}}
\newcommand\toPP{ \overset{\text{PP}}{\to}}
\newcommand{\oeq}{\mathrel{\text{\textcircled{$=$}}}}





\usepackage{xcolor}
% \usepackage[normalem]{ulem}
\usepackage{lipsum}
\makeatletter
% \newcommand\colorwave[1][blue]{\bgroup \markoverwith{\lower3.5\p@\hbox{\sixly \textcolor{#1}{\char58}}}\ULon}
%\font\sixly=lasy6 % does not re-load if already loaded, so no memory problem.

\newmdtheoremenv[
linewidth= 1pt,linecolor= blue,%
leftmargin=20,rightmargin=20,innertopmargin=0pt, innerrightmargin=40,%
tikzsetting = { draw=lightgray, line width = 0.3pt,dashed,%
dash pattern = on 15pt off 3pt},%
splittopskip=\topskip,skipbelow=\baselineskip,%
skipabove=\baselineskip,ntheorem,roundcorner=0pt,
% backgroundcolor=pagebg,font=\color{orange}\sffamily, fontcolor=white
]{examplebox}{Exemple}[section]



\newcommand\R{\mathbb{R}}
\newcommand\Z{\mathbb{Z}}
\newcommand\N{\mathbb{N}}
\newcommand\E{\mathbb{E}}
\newcommand\F{\mathcal{F}}
\newcommand\cH{\mathcal{H}}
\newcommand\V{\mathbb{V}}
\newcommand\dmo{ ^{-1} }
\newcommand\kapa{\kappa}
\newcommand\im{Im}
\newcommand\hs{\mathcal{H}}





\usepackage{soul}

\makeatletter
\newcommand*{\whiten}[1]{\llap{\textcolor{white}{{\the\SOUL@token}}\hspace{#1pt}}}
\DeclareRobustCommand*\myul{%
    \def\SOUL@everyspace{\underline{\space}\kern\z@}%
    \def\SOUL@everytoken{%
     \setbox0=\hbox{\the\SOUL@token}%
     \ifdim\dp0>\z@
        \raisebox{\dp0}{\underline{\phantom{\the\SOUL@token}}}%
        \whiten{1}\whiten{0}%
        \whiten{-1}\whiten{-2}%
        \llap{\the\SOUL@token}%
     \else
        \underline{\the\SOUL@token}%
     \fi}%
\SOUL@}
\makeatother

\newcommand*{\demp}{\fontfamily{lmtt}\selectfont}

\DeclareTextFontCommand{\textdemp}{\demp}

\begin{document}

\ifcomment
Multiline
comment
\fi
\ifcomment
\myul{Typesetting test}
% \color[rgb]{1,1,1}
$∑_i^n≠ 60º±∞π∆¬≈√j∫h≤≥µ$

$\CR \R\pro\ind\pro\gS\pro
\mqty[a&b\\c&d]$
$\pro\mathbb{P}$
$\dd{x}$

  \[
    \alpha(x)=\left\{
                \begin{array}{ll}
                  x\\
                  \frac{1}{1+e^{-kx}}\\
                  \frac{e^x-e^{-x}}{e^x+e^{-x}}
                \end{array}
              \right.
  \]

  $\expval{x}$
  
  $\chi_\rho(ghg\dmo)=\Tr(\rho_{ghg\dmo})=\Tr(\rho_g\circ\rho_h\circ\rho\dmo_g)=\Tr(\rho_h)\overset{\mbox{\scalebox{0.5}{$\Tr(AB)=\Tr(BA)$}}}{=}\chi_\rho(h)$
  	$\mathop{\oplus}_{\substack{x\in X}}$

$\mat(\rho_g)=(a_{ij}(g))_{\scriptsize \substack{1\leq i\leq d \\ 1\leq j\leq d}}$ et $\mat(\rho'_g)=(a'_{ij}(g))_{\scriptsize \substack{1\leq i'\leq d' \\ 1\leq j'\leq d'}}$



\[\int_a^b{\mathbb{R}^2}g(u, v)\dd{P_{XY}}(u, v)=\iint g(u,v) f_{XY}(u, v)\dd \lambda(u) \dd \lambda(v)\]
$$\lim_{x\to\infty} f(x)$$	
$$\iiiint_V \mu(t,u,v,w) \,dt\,du\,dv\,dw$$
$$\sum_{n=1}^{\infty} 2^{-n} = 1$$	
\begin{definition}
	Si $X$ et $Y$ sont 2 v.a. ou definit la \textsc{Covariance} entre $X$ et $Y$ comme
	$\cov(X,Y)\overset{\text{def}}{=}\E\left[(X-\E(X))(Y-\E(Y))\right]=\E(XY)-\E(X)\E(Y)$.
\end{definition}
\fi
\pagebreak

% \tableofcontents

% insert your code here
%\input{./algebra/main.tex}
%\input{./geometrie-differentielle/main.tex}
%\input{./probabilite/main.tex}
%\input{./analyse-fonctionnelle/main.tex}
% \input{./Analyse-convexe-et-dualite-en-optimisation/main.tex}
%\input{./tikz/main.tex}
%\input{./Theorie-du-distributions/main.tex}
%\input{./optimisation/mine.tex}
 \input{./modelisation/main.tex}

% yves.aubry@univ-tln.fr : algebra

\end{document}

%% !TEX encoding = UTF-8 Unicode
% !TEX TS-program = xelatex

\documentclass[french]{report}

%\usepackage[utf8]{inputenc}
%\usepackage[T1]{fontenc}
\usepackage{babel}


\newif\ifcomment
%\commenttrue # Show comments

\usepackage{physics}
\usepackage{amssymb}


\usepackage{amsthm}
% \usepackage{thmtools}
\usepackage{mathtools}
\usepackage{amsfonts}

\usepackage{color}

\usepackage{tikz}

\usepackage{geometry}
\geometry{a5paper, margin=0.1in, right=1cm}

\usepackage{dsfont}

\usepackage{graphicx}
\graphicspath{ {images/} }

\usepackage{faktor}

\usepackage{IEEEtrantools}
\usepackage{enumerate}   
\usepackage[PostScript=dvips]{"/Users/aware/Documents/Courses/diagrams"}


\newtheorem{theorem}{Théorème}[section]
\renewcommand{\thetheorem}{\arabic{theorem}}
\newtheorem{lemme}{Lemme}[section]
\renewcommand{\thelemme}{\arabic{lemme}}
\newtheorem{proposition}{Proposition}[section]
\renewcommand{\theproposition}{\arabic{proposition}}
\newtheorem{notations}{Notations}[section]
\newtheorem{problem}{Problème}[section]
\newtheorem{corollary}{Corollaire}[theorem]
\renewcommand{\thecorollary}{\arabic{corollary}}
\newtheorem{property}{Propriété}[section]
\newtheorem{objective}{Objectif}[section]

\theoremstyle{definition}
\newtheorem{definition}{Définition}[section]
\renewcommand{\thedefinition}{\arabic{definition}}
\newtheorem{exercise}{Exercice}[chapter]
\renewcommand{\theexercise}{\arabic{exercise}}
\newtheorem{example}{Exemple}[chapter]
\renewcommand{\theexample}{\arabic{example}}
\newtheorem*{solution}{Solution}
\newtheorem*{application}{Application}
\newtheorem*{notation}{Notation}
\newtheorem*{vocabulary}{Vocabulaire}
\newtheorem*{properties}{Propriétés}



\theoremstyle{remark}
\newtheorem*{remark}{Remarque}
\newtheorem*{rappel}{Rappel}


\usepackage{etoolbox}
\AtBeginEnvironment{exercise}{\small}
\AtBeginEnvironment{example}{\small}

\usepackage{cases}
\usepackage[red]{mypack}

\usepackage[framemethod=TikZ]{mdframed}

\definecolor{bg}{rgb}{0.4,0.25,0.95}
\definecolor{pagebg}{rgb}{0,0,0.5}
\surroundwithmdframed[
   topline=false,
   rightline=false,
   bottomline=false,
   leftmargin=\parindent,
   skipabove=8pt,
   skipbelow=8pt,
   linecolor=blue,
   innerbottommargin=10pt,
   % backgroundcolor=bg,font=\color{orange}\sffamily, fontcolor=white
]{definition}

\usepackage{empheq}
\usepackage[most]{tcolorbox}

\newtcbox{\mymath}[1][]{%
    nobeforeafter, math upper, tcbox raise base,
    enhanced, colframe=blue!30!black,
    colback=red!10, boxrule=1pt,
    #1}

\usepackage{unixode}


\DeclareMathOperator{\ord}{ord}
\DeclareMathOperator{\orb}{orb}
\DeclareMathOperator{\stab}{stab}
\DeclareMathOperator{\Stab}{stab}
\DeclareMathOperator{\ppcm}{ppcm}
\DeclareMathOperator{\conj}{Conj}
\DeclareMathOperator{\End}{End}
\DeclareMathOperator{\rot}{rot}
\DeclareMathOperator{\trs}{trace}
\DeclareMathOperator{\Ind}{Ind}
\DeclareMathOperator{\mat}{Mat}
\DeclareMathOperator{\id}{Id}
\DeclareMathOperator{\vect}{vect}
\DeclareMathOperator{\img}{img}
\DeclareMathOperator{\cov}{Cov}
\DeclareMathOperator{\dist}{dist}
\DeclareMathOperator{\irr}{Irr}
\DeclareMathOperator{\image}{Im}
\DeclareMathOperator{\pd}{\partial}
\DeclareMathOperator{\epi}{epi}
\DeclareMathOperator{\Argmin}{Argmin}
\DeclareMathOperator{\dom}{dom}
\DeclareMathOperator{\proj}{proj}
\DeclareMathOperator{\ctg}{ctg}
\DeclareMathOperator{\supp}{supp}
\DeclareMathOperator{\argmin}{argmin}
\DeclareMathOperator{\mult}{mult}
\DeclareMathOperator{\ch}{ch}
\DeclareMathOperator{\sh}{sh}
\DeclareMathOperator{\rang}{rang}
\DeclareMathOperator{\diam}{diam}
\DeclareMathOperator{\Epigraphe}{Epigraphe}




\usepackage{xcolor}
\everymath{\color{blue}}
%\everymath{\color[rgb]{0,1,1}}
%\pagecolor[rgb]{0,0,0.5}


\newcommand*{\pdtest}[3][]{\ensuremath{\frac{\partial^{#1} #2}{\partial #3}}}

\newcommand*{\deffunc}[6][]{\ensuremath{
\begin{array}{rcl}
#2 : #3 &\rightarrow& #4\\
#5 &\mapsto& #6
\end{array}
}}

\newcommand{\eqcolon}{\mathrel{\resizebox{\widthof{$\mathord{=}$}}{\height}{ $\!\!=\!\!\resizebox{1.2\width}{0.8\height}{\raisebox{0.23ex}{$\mathop{:}$}}\!\!$ }}}
\newcommand{\coloneq}{\mathrel{\resizebox{\widthof{$\mathord{=}$}}{\height}{ $\!\!\resizebox{1.2\width}{0.8\height}{\raisebox{0.23ex}{$\mathop{:}$}}\!\!=\!\!$ }}}
\newcommand{\eqcolonl}{\ensuremath{\mathrel{=\!\!\mathop{:}}}}
\newcommand{\coloneql}{\ensuremath{\mathrel{\mathop{:} \!\! =}}}
\newcommand{\vc}[1]{% inline column vector
  \left(\begin{smallmatrix}#1\end{smallmatrix}\right)%
}
\newcommand{\vr}[1]{% inline row vector
  \begin{smallmatrix}(\,#1\,)\end{smallmatrix}%
}
\makeatletter
\newcommand*{\defeq}{\ =\mathrel{\rlap{%
                     \raisebox{0.3ex}{$\m@th\cdot$}}%
                     \raisebox{-0.3ex}{$\m@th\cdot$}}%
                     }
\makeatother

\newcommand{\mathcircle}[1]{% inline row vector
 \overset{\circ}{#1}
}
\newcommand{\ulim}{% low limit
 \underline{\lim}
}
\newcommand{\ssi}{% iff
\iff
}
\newcommand{\ps}[2]{
\expval{#1 | #2}
}
\newcommand{\df}[1]{
\mqty{#1}
}
\newcommand{\n}[1]{
\norm{#1}
}
\newcommand{\sys}[1]{
\left\{\smqty{#1}\right.
}


\newcommand{\eqdef}{\ensuremath{\overset{\text{def}}=}}


\def\Circlearrowright{\ensuremath{%
  \rotatebox[origin=c]{230}{$\circlearrowright$}}}

\newcommand\ct[1]{\text{\rmfamily\upshape #1}}
\newcommand\question[1]{ {\color{red} ...!? \small #1}}
\newcommand\caz[1]{\left\{\begin{array} #1 \end{array}\right.}
\newcommand\const{\text{\rmfamily\upshape const}}
\newcommand\toP{ \overset{\pro}{\to}}
\newcommand\toPP{ \overset{\text{PP}}{\to}}
\newcommand{\oeq}{\mathrel{\text{\textcircled{$=$}}}}





\usepackage{xcolor}
% \usepackage[normalem]{ulem}
\usepackage{lipsum}
\makeatletter
% \newcommand\colorwave[1][blue]{\bgroup \markoverwith{\lower3.5\p@\hbox{\sixly \textcolor{#1}{\char58}}}\ULon}
%\font\sixly=lasy6 % does not re-load if already loaded, so no memory problem.

\newmdtheoremenv[
linewidth= 1pt,linecolor= blue,%
leftmargin=20,rightmargin=20,innertopmargin=0pt, innerrightmargin=40,%
tikzsetting = { draw=lightgray, line width = 0.3pt,dashed,%
dash pattern = on 15pt off 3pt},%
splittopskip=\topskip,skipbelow=\baselineskip,%
skipabove=\baselineskip,ntheorem,roundcorner=0pt,
% backgroundcolor=pagebg,font=\color{orange}\sffamily, fontcolor=white
]{examplebox}{Exemple}[section]



\newcommand\R{\mathbb{R}}
\newcommand\Z{\mathbb{Z}}
\newcommand\N{\mathbb{N}}
\newcommand\E{\mathbb{E}}
\newcommand\F{\mathcal{F}}
\newcommand\cH{\mathcal{H}}
\newcommand\V{\mathbb{V}}
\newcommand\dmo{ ^{-1} }
\newcommand\kapa{\kappa}
\newcommand\im{Im}
\newcommand\hs{\mathcal{H}}





\usepackage{soul}

\makeatletter
\newcommand*{\whiten}[1]{\llap{\textcolor{white}{{\the\SOUL@token}}\hspace{#1pt}}}
\DeclareRobustCommand*\myul{%
    \def\SOUL@everyspace{\underline{\space}\kern\z@}%
    \def\SOUL@everytoken{%
     \setbox0=\hbox{\the\SOUL@token}%
     \ifdim\dp0>\z@
        \raisebox{\dp0}{\underline{\phantom{\the\SOUL@token}}}%
        \whiten{1}\whiten{0}%
        \whiten{-1}\whiten{-2}%
        \llap{\the\SOUL@token}%
     \else
        \underline{\the\SOUL@token}%
     \fi}%
\SOUL@}
\makeatother

\newcommand*{\demp}{\fontfamily{lmtt}\selectfont}

\DeclareTextFontCommand{\textdemp}{\demp}

\begin{document}

\ifcomment
Multiline
comment
\fi
\ifcomment
\myul{Typesetting test}
% \color[rgb]{1,1,1}
$∑_i^n≠ 60º±∞π∆¬≈√j∫h≤≥µ$

$\CR \R\pro\ind\pro\gS\pro
\mqty[a&b\\c&d]$
$\pro\mathbb{P}$
$\dd{x}$

  \[
    \alpha(x)=\left\{
                \begin{array}{ll}
                  x\\
                  \frac{1}{1+e^{-kx}}\\
                  \frac{e^x-e^{-x}}{e^x+e^{-x}}
                \end{array}
              \right.
  \]

  $\expval{x}$
  
  $\chi_\rho(ghg\dmo)=\Tr(\rho_{ghg\dmo})=\Tr(\rho_g\circ\rho_h\circ\rho\dmo_g)=\Tr(\rho_h)\overset{\mbox{\scalebox{0.5}{$\Tr(AB)=\Tr(BA)$}}}{=}\chi_\rho(h)$
  	$\mathop{\oplus}_{\substack{x\in X}}$

$\mat(\rho_g)=(a_{ij}(g))_{\scriptsize \substack{1\leq i\leq d \\ 1\leq j\leq d}}$ et $\mat(\rho'_g)=(a'_{ij}(g))_{\scriptsize \substack{1\leq i'\leq d' \\ 1\leq j'\leq d'}}$



\[\int_a^b{\mathbb{R}^2}g(u, v)\dd{P_{XY}}(u, v)=\iint g(u,v) f_{XY}(u, v)\dd \lambda(u) \dd \lambda(v)\]
$$\lim_{x\to\infty} f(x)$$	
$$\iiiint_V \mu(t,u,v,w) \,dt\,du\,dv\,dw$$
$$\sum_{n=1}^{\infty} 2^{-n} = 1$$	
\begin{definition}
	Si $X$ et $Y$ sont 2 v.a. ou definit la \textsc{Covariance} entre $X$ et $Y$ comme
	$\cov(X,Y)\overset{\text{def}}{=}\E\left[(X-\E(X))(Y-\E(Y))\right]=\E(XY)-\E(X)\E(Y)$.
\end{definition}
\fi
\pagebreak

% \tableofcontents

% insert your code here
%\input{./algebra/main.tex}
%\input{./geometrie-differentielle/main.tex}
%\input{./probabilite/main.tex}
%\input{./analyse-fonctionnelle/main.tex}
% \input{./Analyse-convexe-et-dualite-en-optimisation/main.tex}
%\input{./tikz/main.tex}
%\input{./Theorie-du-distributions/main.tex}
%\input{./optimisation/mine.tex}
 \input{./modelisation/main.tex}

% yves.aubry@univ-tln.fr : algebra

\end{document}

% % !TEX encoding = UTF-8 Unicode
% !TEX TS-program = xelatex

\documentclass[french]{report}

%\usepackage[utf8]{inputenc}
%\usepackage[T1]{fontenc}
\usepackage{babel}


\newif\ifcomment
%\commenttrue # Show comments

\usepackage{physics}
\usepackage{amssymb}


\usepackage{amsthm}
% \usepackage{thmtools}
\usepackage{mathtools}
\usepackage{amsfonts}

\usepackage{color}

\usepackage{tikz}

\usepackage{geometry}
\geometry{a5paper, margin=0.1in, right=1cm}

\usepackage{dsfont}

\usepackage{graphicx}
\graphicspath{ {images/} }

\usepackage{faktor}

\usepackage{IEEEtrantools}
\usepackage{enumerate}   
\usepackage[PostScript=dvips]{"/Users/aware/Documents/Courses/diagrams"}


\newtheorem{theorem}{Théorème}[section]
\renewcommand{\thetheorem}{\arabic{theorem}}
\newtheorem{lemme}{Lemme}[section]
\renewcommand{\thelemme}{\arabic{lemme}}
\newtheorem{proposition}{Proposition}[section]
\renewcommand{\theproposition}{\arabic{proposition}}
\newtheorem{notations}{Notations}[section]
\newtheorem{problem}{Problème}[section]
\newtheorem{corollary}{Corollaire}[theorem]
\renewcommand{\thecorollary}{\arabic{corollary}}
\newtheorem{property}{Propriété}[section]
\newtheorem{objective}{Objectif}[section]

\theoremstyle{definition}
\newtheorem{definition}{Définition}[section]
\renewcommand{\thedefinition}{\arabic{definition}}
\newtheorem{exercise}{Exercice}[chapter]
\renewcommand{\theexercise}{\arabic{exercise}}
\newtheorem{example}{Exemple}[chapter]
\renewcommand{\theexample}{\arabic{example}}
\newtheorem*{solution}{Solution}
\newtheorem*{application}{Application}
\newtheorem*{notation}{Notation}
\newtheorem*{vocabulary}{Vocabulaire}
\newtheorem*{properties}{Propriétés}



\theoremstyle{remark}
\newtheorem*{remark}{Remarque}
\newtheorem*{rappel}{Rappel}


\usepackage{etoolbox}
\AtBeginEnvironment{exercise}{\small}
\AtBeginEnvironment{example}{\small}

\usepackage{cases}
\usepackage[red]{mypack}

\usepackage[framemethod=TikZ]{mdframed}

\definecolor{bg}{rgb}{0.4,0.25,0.95}
\definecolor{pagebg}{rgb}{0,0,0.5}
\surroundwithmdframed[
   topline=false,
   rightline=false,
   bottomline=false,
   leftmargin=\parindent,
   skipabove=8pt,
   skipbelow=8pt,
   linecolor=blue,
   innerbottommargin=10pt,
   % backgroundcolor=bg,font=\color{orange}\sffamily, fontcolor=white
]{definition}

\usepackage{empheq}
\usepackage[most]{tcolorbox}

\newtcbox{\mymath}[1][]{%
    nobeforeafter, math upper, tcbox raise base,
    enhanced, colframe=blue!30!black,
    colback=red!10, boxrule=1pt,
    #1}

\usepackage{unixode}


\DeclareMathOperator{\ord}{ord}
\DeclareMathOperator{\orb}{orb}
\DeclareMathOperator{\stab}{stab}
\DeclareMathOperator{\Stab}{stab}
\DeclareMathOperator{\ppcm}{ppcm}
\DeclareMathOperator{\conj}{Conj}
\DeclareMathOperator{\End}{End}
\DeclareMathOperator{\rot}{rot}
\DeclareMathOperator{\trs}{trace}
\DeclareMathOperator{\Ind}{Ind}
\DeclareMathOperator{\mat}{Mat}
\DeclareMathOperator{\id}{Id}
\DeclareMathOperator{\vect}{vect}
\DeclareMathOperator{\img}{img}
\DeclareMathOperator{\cov}{Cov}
\DeclareMathOperator{\dist}{dist}
\DeclareMathOperator{\irr}{Irr}
\DeclareMathOperator{\image}{Im}
\DeclareMathOperator{\pd}{\partial}
\DeclareMathOperator{\epi}{epi}
\DeclareMathOperator{\Argmin}{Argmin}
\DeclareMathOperator{\dom}{dom}
\DeclareMathOperator{\proj}{proj}
\DeclareMathOperator{\ctg}{ctg}
\DeclareMathOperator{\supp}{supp}
\DeclareMathOperator{\argmin}{argmin}
\DeclareMathOperator{\mult}{mult}
\DeclareMathOperator{\ch}{ch}
\DeclareMathOperator{\sh}{sh}
\DeclareMathOperator{\rang}{rang}
\DeclareMathOperator{\diam}{diam}
\DeclareMathOperator{\Epigraphe}{Epigraphe}




\usepackage{xcolor}
\everymath{\color{blue}}
%\everymath{\color[rgb]{0,1,1}}
%\pagecolor[rgb]{0,0,0.5}


\newcommand*{\pdtest}[3][]{\ensuremath{\frac{\partial^{#1} #2}{\partial #3}}}

\newcommand*{\deffunc}[6][]{\ensuremath{
\begin{array}{rcl}
#2 : #3 &\rightarrow& #4\\
#5 &\mapsto& #6
\end{array}
}}

\newcommand{\eqcolon}{\mathrel{\resizebox{\widthof{$\mathord{=}$}}{\height}{ $\!\!=\!\!\resizebox{1.2\width}{0.8\height}{\raisebox{0.23ex}{$\mathop{:}$}}\!\!$ }}}
\newcommand{\coloneq}{\mathrel{\resizebox{\widthof{$\mathord{=}$}}{\height}{ $\!\!\resizebox{1.2\width}{0.8\height}{\raisebox{0.23ex}{$\mathop{:}$}}\!\!=\!\!$ }}}
\newcommand{\eqcolonl}{\ensuremath{\mathrel{=\!\!\mathop{:}}}}
\newcommand{\coloneql}{\ensuremath{\mathrel{\mathop{:} \!\! =}}}
\newcommand{\vc}[1]{% inline column vector
  \left(\begin{smallmatrix}#1\end{smallmatrix}\right)%
}
\newcommand{\vr}[1]{% inline row vector
  \begin{smallmatrix}(\,#1\,)\end{smallmatrix}%
}
\makeatletter
\newcommand*{\defeq}{\ =\mathrel{\rlap{%
                     \raisebox{0.3ex}{$\m@th\cdot$}}%
                     \raisebox{-0.3ex}{$\m@th\cdot$}}%
                     }
\makeatother

\newcommand{\mathcircle}[1]{% inline row vector
 \overset{\circ}{#1}
}
\newcommand{\ulim}{% low limit
 \underline{\lim}
}
\newcommand{\ssi}{% iff
\iff
}
\newcommand{\ps}[2]{
\expval{#1 | #2}
}
\newcommand{\df}[1]{
\mqty{#1}
}
\newcommand{\n}[1]{
\norm{#1}
}
\newcommand{\sys}[1]{
\left\{\smqty{#1}\right.
}


\newcommand{\eqdef}{\ensuremath{\overset{\text{def}}=}}


\def\Circlearrowright{\ensuremath{%
  \rotatebox[origin=c]{230}{$\circlearrowright$}}}

\newcommand\ct[1]{\text{\rmfamily\upshape #1}}
\newcommand\question[1]{ {\color{red} ...!? \small #1}}
\newcommand\caz[1]{\left\{\begin{array} #1 \end{array}\right.}
\newcommand\const{\text{\rmfamily\upshape const}}
\newcommand\toP{ \overset{\pro}{\to}}
\newcommand\toPP{ \overset{\text{PP}}{\to}}
\newcommand{\oeq}{\mathrel{\text{\textcircled{$=$}}}}





\usepackage{xcolor}
% \usepackage[normalem]{ulem}
\usepackage{lipsum}
\makeatletter
% \newcommand\colorwave[1][blue]{\bgroup \markoverwith{\lower3.5\p@\hbox{\sixly \textcolor{#1}{\char58}}}\ULon}
%\font\sixly=lasy6 % does not re-load if already loaded, so no memory problem.

\newmdtheoremenv[
linewidth= 1pt,linecolor= blue,%
leftmargin=20,rightmargin=20,innertopmargin=0pt, innerrightmargin=40,%
tikzsetting = { draw=lightgray, line width = 0.3pt,dashed,%
dash pattern = on 15pt off 3pt},%
splittopskip=\topskip,skipbelow=\baselineskip,%
skipabove=\baselineskip,ntheorem,roundcorner=0pt,
% backgroundcolor=pagebg,font=\color{orange}\sffamily, fontcolor=white
]{examplebox}{Exemple}[section]



\newcommand\R{\mathbb{R}}
\newcommand\Z{\mathbb{Z}}
\newcommand\N{\mathbb{N}}
\newcommand\E{\mathbb{E}}
\newcommand\F{\mathcal{F}}
\newcommand\cH{\mathcal{H}}
\newcommand\V{\mathbb{V}}
\newcommand\dmo{ ^{-1} }
\newcommand\kapa{\kappa}
\newcommand\im{Im}
\newcommand\hs{\mathcal{H}}





\usepackage{soul}

\makeatletter
\newcommand*{\whiten}[1]{\llap{\textcolor{white}{{\the\SOUL@token}}\hspace{#1pt}}}
\DeclareRobustCommand*\myul{%
    \def\SOUL@everyspace{\underline{\space}\kern\z@}%
    \def\SOUL@everytoken{%
     \setbox0=\hbox{\the\SOUL@token}%
     \ifdim\dp0>\z@
        \raisebox{\dp0}{\underline{\phantom{\the\SOUL@token}}}%
        \whiten{1}\whiten{0}%
        \whiten{-1}\whiten{-2}%
        \llap{\the\SOUL@token}%
     \else
        \underline{\the\SOUL@token}%
     \fi}%
\SOUL@}
\makeatother

\newcommand*{\demp}{\fontfamily{lmtt}\selectfont}

\DeclareTextFontCommand{\textdemp}{\demp}

\begin{document}

\ifcomment
Multiline
comment
\fi
\ifcomment
\myul{Typesetting test}
% \color[rgb]{1,1,1}
$∑_i^n≠ 60º±∞π∆¬≈√j∫h≤≥µ$

$\CR \R\pro\ind\pro\gS\pro
\mqty[a&b\\c&d]$
$\pro\mathbb{P}$
$\dd{x}$

  \[
    \alpha(x)=\left\{
                \begin{array}{ll}
                  x\\
                  \frac{1}{1+e^{-kx}}\\
                  \frac{e^x-e^{-x}}{e^x+e^{-x}}
                \end{array}
              \right.
  \]

  $\expval{x}$
  
  $\chi_\rho(ghg\dmo)=\Tr(\rho_{ghg\dmo})=\Tr(\rho_g\circ\rho_h\circ\rho\dmo_g)=\Tr(\rho_h)\overset{\mbox{\scalebox{0.5}{$\Tr(AB)=\Tr(BA)$}}}{=}\chi_\rho(h)$
  	$\mathop{\oplus}_{\substack{x\in X}}$

$\mat(\rho_g)=(a_{ij}(g))_{\scriptsize \substack{1\leq i\leq d \\ 1\leq j\leq d}}$ et $\mat(\rho'_g)=(a'_{ij}(g))_{\scriptsize \substack{1\leq i'\leq d' \\ 1\leq j'\leq d'}}$



\[\int_a^b{\mathbb{R}^2}g(u, v)\dd{P_{XY}}(u, v)=\iint g(u,v) f_{XY}(u, v)\dd \lambda(u) \dd \lambda(v)\]
$$\lim_{x\to\infty} f(x)$$	
$$\iiiint_V \mu(t,u,v,w) \,dt\,du\,dv\,dw$$
$$\sum_{n=1}^{\infty} 2^{-n} = 1$$	
\begin{definition}
	Si $X$ et $Y$ sont 2 v.a. ou definit la \textsc{Covariance} entre $X$ et $Y$ comme
	$\cov(X,Y)\overset{\text{def}}{=}\E\left[(X-\E(X))(Y-\E(Y))\right]=\E(XY)-\E(X)\E(Y)$.
\end{definition}
\fi
\pagebreak

% \tableofcontents

% insert your code here
%\input{./algebra/main.tex}
%\input{./geometrie-differentielle/main.tex}
%\input{./probabilite/main.tex}
%\input{./analyse-fonctionnelle/main.tex}
% \input{./Analyse-convexe-et-dualite-en-optimisation/main.tex}
%\input{./tikz/main.tex}
%\input{./Theorie-du-distributions/main.tex}
%\input{./optimisation/mine.tex}
 \input{./modelisation/main.tex}

% yves.aubry@univ-tln.fr : algebra

\end{document}

%% !TEX encoding = UTF-8 Unicode
% !TEX TS-program = xelatex

\documentclass[french]{report}

%\usepackage[utf8]{inputenc}
%\usepackage[T1]{fontenc}
\usepackage{babel}


\newif\ifcomment
%\commenttrue # Show comments

\usepackage{physics}
\usepackage{amssymb}


\usepackage{amsthm}
% \usepackage{thmtools}
\usepackage{mathtools}
\usepackage{amsfonts}

\usepackage{color}

\usepackage{tikz}

\usepackage{geometry}
\geometry{a5paper, margin=0.1in, right=1cm}

\usepackage{dsfont}

\usepackage{graphicx}
\graphicspath{ {images/} }

\usepackage{faktor}

\usepackage{IEEEtrantools}
\usepackage{enumerate}   
\usepackage[PostScript=dvips]{"/Users/aware/Documents/Courses/diagrams"}


\newtheorem{theorem}{Théorème}[section]
\renewcommand{\thetheorem}{\arabic{theorem}}
\newtheorem{lemme}{Lemme}[section]
\renewcommand{\thelemme}{\arabic{lemme}}
\newtheorem{proposition}{Proposition}[section]
\renewcommand{\theproposition}{\arabic{proposition}}
\newtheorem{notations}{Notations}[section]
\newtheorem{problem}{Problème}[section]
\newtheorem{corollary}{Corollaire}[theorem]
\renewcommand{\thecorollary}{\arabic{corollary}}
\newtheorem{property}{Propriété}[section]
\newtheorem{objective}{Objectif}[section]

\theoremstyle{definition}
\newtheorem{definition}{Définition}[section]
\renewcommand{\thedefinition}{\arabic{definition}}
\newtheorem{exercise}{Exercice}[chapter]
\renewcommand{\theexercise}{\arabic{exercise}}
\newtheorem{example}{Exemple}[chapter]
\renewcommand{\theexample}{\arabic{example}}
\newtheorem*{solution}{Solution}
\newtheorem*{application}{Application}
\newtheorem*{notation}{Notation}
\newtheorem*{vocabulary}{Vocabulaire}
\newtheorem*{properties}{Propriétés}



\theoremstyle{remark}
\newtheorem*{remark}{Remarque}
\newtheorem*{rappel}{Rappel}


\usepackage{etoolbox}
\AtBeginEnvironment{exercise}{\small}
\AtBeginEnvironment{example}{\small}

\usepackage{cases}
\usepackage[red]{mypack}

\usepackage[framemethod=TikZ]{mdframed}

\definecolor{bg}{rgb}{0.4,0.25,0.95}
\definecolor{pagebg}{rgb}{0,0,0.5}
\surroundwithmdframed[
   topline=false,
   rightline=false,
   bottomline=false,
   leftmargin=\parindent,
   skipabove=8pt,
   skipbelow=8pt,
   linecolor=blue,
   innerbottommargin=10pt,
   % backgroundcolor=bg,font=\color{orange}\sffamily, fontcolor=white
]{definition}

\usepackage{empheq}
\usepackage[most]{tcolorbox}

\newtcbox{\mymath}[1][]{%
    nobeforeafter, math upper, tcbox raise base,
    enhanced, colframe=blue!30!black,
    colback=red!10, boxrule=1pt,
    #1}

\usepackage{unixode}


\DeclareMathOperator{\ord}{ord}
\DeclareMathOperator{\orb}{orb}
\DeclareMathOperator{\stab}{stab}
\DeclareMathOperator{\Stab}{stab}
\DeclareMathOperator{\ppcm}{ppcm}
\DeclareMathOperator{\conj}{Conj}
\DeclareMathOperator{\End}{End}
\DeclareMathOperator{\rot}{rot}
\DeclareMathOperator{\trs}{trace}
\DeclareMathOperator{\Ind}{Ind}
\DeclareMathOperator{\mat}{Mat}
\DeclareMathOperator{\id}{Id}
\DeclareMathOperator{\vect}{vect}
\DeclareMathOperator{\img}{img}
\DeclareMathOperator{\cov}{Cov}
\DeclareMathOperator{\dist}{dist}
\DeclareMathOperator{\irr}{Irr}
\DeclareMathOperator{\image}{Im}
\DeclareMathOperator{\pd}{\partial}
\DeclareMathOperator{\epi}{epi}
\DeclareMathOperator{\Argmin}{Argmin}
\DeclareMathOperator{\dom}{dom}
\DeclareMathOperator{\proj}{proj}
\DeclareMathOperator{\ctg}{ctg}
\DeclareMathOperator{\supp}{supp}
\DeclareMathOperator{\argmin}{argmin}
\DeclareMathOperator{\mult}{mult}
\DeclareMathOperator{\ch}{ch}
\DeclareMathOperator{\sh}{sh}
\DeclareMathOperator{\rang}{rang}
\DeclareMathOperator{\diam}{diam}
\DeclareMathOperator{\Epigraphe}{Epigraphe}




\usepackage{xcolor}
\everymath{\color{blue}}
%\everymath{\color[rgb]{0,1,1}}
%\pagecolor[rgb]{0,0,0.5}


\newcommand*{\pdtest}[3][]{\ensuremath{\frac{\partial^{#1} #2}{\partial #3}}}

\newcommand*{\deffunc}[6][]{\ensuremath{
\begin{array}{rcl}
#2 : #3 &\rightarrow& #4\\
#5 &\mapsto& #6
\end{array}
}}

\newcommand{\eqcolon}{\mathrel{\resizebox{\widthof{$\mathord{=}$}}{\height}{ $\!\!=\!\!\resizebox{1.2\width}{0.8\height}{\raisebox{0.23ex}{$\mathop{:}$}}\!\!$ }}}
\newcommand{\coloneq}{\mathrel{\resizebox{\widthof{$\mathord{=}$}}{\height}{ $\!\!\resizebox{1.2\width}{0.8\height}{\raisebox{0.23ex}{$\mathop{:}$}}\!\!=\!\!$ }}}
\newcommand{\eqcolonl}{\ensuremath{\mathrel{=\!\!\mathop{:}}}}
\newcommand{\coloneql}{\ensuremath{\mathrel{\mathop{:} \!\! =}}}
\newcommand{\vc}[1]{% inline column vector
  \left(\begin{smallmatrix}#1\end{smallmatrix}\right)%
}
\newcommand{\vr}[1]{% inline row vector
  \begin{smallmatrix}(\,#1\,)\end{smallmatrix}%
}
\makeatletter
\newcommand*{\defeq}{\ =\mathrel{\rlap{%
                     \raisebox{0.3ex}{$\m@th\cdot$}}%
                     \raisebox{-0.3ex}{$\m@th\cdot$}}%
                     }
\makeatother

\newcommand{\mathcircle}[1]{% inline row vector
 \overset{\circ}{#1}
}
\newcommand{\ulim}{% low limit
 \underline{\lim}
}
\newcommand{\ssi}{% iff
\iff
}
\newcommand{\ps}[2]{
\expval{#1 | #2}
}
\newcommand{\df}[1]{
\mqty{#1}
}
\newcommand{\n}[1]{
\norm{#1}
}
\newcommand{\sys}[1]{
\left\{\smqty{#1}\right.
}


\newcommand{\eqdef}{\ensuremath{\overset{\text{def}}=}}


\def\Circlearrowright{\ensuremath{%
  \rotatebox[origin=c]{230}{$\circlearrowright$}}}

\newcommand\ct[1]{\text{\rmfamily\upshape #1}}
\newcommand\question[1]{ {\color{red} ...!? \small #1}}
\newcommand\caz[1]{\left\{\begin{array} #1 \end{array}\right.}
\newcommand\const{\text{\rmfamily\upshape const}}
\newcommand\toP{ \overset{\pro}{\to}}
\newcommand\toPP{ \overset{\text{PP}}{\to}}
\newcommand{\oeq}{\mathrel{\text{\textcircled{$=$}}}}





\usepackage{xcolor}
% \usepackage[normalem]{ulem}
\usepackage{lipsum}
\makeatletter
% \newcommand\colorwave[1][blue]{\bgroup \markoverwith{\lower3.5\p@\hbox{\sixly \textcolor{#1}{\char58}}}\ULon}
%\font\sixly=lasy6 % does not re-load if already loaded, so no memory problem.

\newmdtheoremenv[
linewidth= 1pt,linecolor= blue,%
leftmargin=20,rightmargin=20,innertopmargin=0pt, innerrightmargin=40,%
tikzsetting = { draw=lightgray, line width = 0.3pt,dashed,%
dash pattern = on 15pt off 3pt},%
splittopskip=\topskip,skipbelow=\baselineskip,%
skipabove=\baselineskip,ntheorem,roundcorner=0pt,
% backgroundcolor=pagebg,font=\color{orange}\sffamily, fontcolor=white
]{examplebox}{Exemple}[section]



\newcommand\R{\mathbb{R}}
\newcommand\Z{\mathbb{Z}}
\newcommand\N{\mathbb{N}}
\newcommand\E{\mathbb{E}}
\newcommand\F{\mathcal{F}}
\newcommand\cH{\mathcal{H}}
\newcommand\V{\mathbb{V}}
\newcommand\dmo{ ^{-1} }
\newcommand\kapa{\kappa}
\newcommand\im{Im}
\newcommand\hs{\mathcal{H}}





\usepackage{soul}

\makeatletter
\newcommand*{\whiten}[1]{\llap{\textcolor{white}{{\the\SOUL@token}}\hspace{#1pt}}}
\DeclareRobustCommand*\myul{%
    \def\SOUL@everyspace{\underline{\space}\kern\z@}%
    \def\SOUL@everytoken{%
     \setbox0=\hbox{\the\SOUL@token}%
     \ifdim\dp0>\z@
        \raisebox{\dp0}{\underline{\phantom{\the\SOUL@token}}}%
        \whiten{1}\whiten{0}%
        \whiten{-1}\whiten{-2}%
        \llap{\the\SOUL@token}%
     \else
        \underline{\the\SOUL@token}%
     \fi}%
\SOUL@}
\makeatother

\newcommand*{\demp}{\fontfamily{lmtt}\selectfont}

\DeclareTextFontCommand{\textdemp}{\demp}

\begin{document}

\ifcomment
Multiline
comment
\fi
\ifcomment
\myul{Typesetting test}
% \color[rgb]{1,1,1}
$∑_i^n≠ 60º±∞π∆¬≈√j∫h≤≥µ$

$\CR \R\pro\ind\pro\gS\pro
\mqty[a&b\\c&d]$
$\pro\mathbb{P}$
$\dd{x}$

  \[
    \alpha(x)=\left\{
                \begin{array}{ll}
                  x\\
                  \frac{1}{1+e^{-kx}}\\
                  \frac{e^x-e^{-x}}{e^x+e^{-x}}
                \end{array}
              \right.
  \]

  $\expval{x}$
  
  $\chi_\rho(ghg\dmo)=\Tr(\rho_{ghg\dmo})=\Tr(\rho_g\circ\rho_h\circ\rho\dmo_g)=\Tr(\rho_h)\overset{\mbox{\scalebox{0.5}{$\Tr(AB)=\Tr(BA)$}}}{=}\chi_\rho(h)$
  	$\mathop{\oplus}_{\substack{x\in X}}$

$\mat(\rho_g)=(a_{ij}(g))_{\scriptsize \substack{1\leq i\leq d \\ 1\leq j\leq d}}$ et $\mat(\rho'_g)=(a'_{ij}(g))_{\scriptsize \substack{1\leq i'\leq d' \\ 1\leq j'\leq d'}}$



\[\int_a^b{\mathbb{R}^2}g(u, v)\dd{P_{XY}}(u, v)=\iint g(u,v) f_{XY}(u, v)\dd \lambda(u) \dd \lambda(v)\]
$$\lim_{x\to\infty} f(x)$$	
$$\iiiint_V \mu(t,u,v,w) \,dt\,du\,dv\,dw$$
$$\sum_{n=1}^{\infty} 2^{-n} = 1$$	
\begin{definition}
	Si $X$ et $Y$ sont 2 v.a. ou definit la \textsc{Covariance} entre $X$ et $Y$ comme
	$\cov(X,Y)\overset{\text{def}}{=}\E\left[(X-\E(X))(Y-\E(Y))\right]=\E(XY)-\E(X)\E(Y)$.
\end{definition}
\fi
\pagebreak

% \tableofcontents

% insert your code here
%\input{./algebra/main.tex}
%\input{./geometrie-differentielle/main.tex}
%\input{./probabilite/main.tex}
%\input{./analyse-fonctionnelle/main.tex}
% \input{./Analyse-convexe-et-dualite-en-optimisation/main.tex}
%\input{./tikz/main.tex}
%\input{./Theorie-du-distributions/main.tex}
%\input{./optimisation/mine.tex}
 \input{./modelisation/main.tex}

% yves.aubry@univ-tln.fr : algebra

\end{document}

%% !TEX encoding = UTF-8 Unicode
% !TEX TS-program = xelatex

\documentclass[french]{report}

%\usepackage[utf8]{inputenc}
%\usepackage[T1]{fontenc}
\usepackage{babel}


\newif\ifcomment
%\commenttrue # Show comments

\usepackage{physics}
\usepackage{amssymb}


\usepackage{amsthm}
% \usepackage{thmtools}
\usepackage{mathtools}
\usepackage{amsfonts}

\usepackage{color}

\usepackage{tikz}

\usepackage{geometry}
\geometry{a5paper, margin=0.1in, right=1cm}

\usepackage{dsfont}

\usepackage{graphicx}
\graphicspath{ {images/} }

\usepackage{faktor}

\usepackage{IEEEtrantools}
\usepackage{enumerate}   
\usepackage[PostScript=dvips]{"/Users/aware/Documents/Courses/diagrams"}


\newtheorem{theorem}{Théorème}[section]
\renewcommand{\thetheorem}{\arabic{theorem}}
\newtheorem{lemme}{Lemme}[section]
\renewcommand{\thelemme}{\arabic{lemme}}
\newtheorem{proposition}{Proposition}[section]
\renewcommand{\theproposition}{\arabic{proposition}}
\newtheorem{notations}{Notations}[section]
\newtheorem{problem}{Problème}[section]
\newtheorem{corollary}{Corollaire}[theorem]
\renewcommand{\thecorollary}{\arabic{corollary}}
\newtheorem{property}{Propriété}[section]
\newtheorem{objective}{Objectif}[section]

\theoremstyle{definition}
\newtheorem{definition}{Définition}[section]
\renewcommand{\thedefinition}{\arabic{definition}}
\newtheorem{exercise}{Exercice}[chapter]
\renewcommand{\theexercise}{\arabic{exercise}}
\newtheorem{example}{Exemple}[chapter]
\renewcommand{\theexample}{\arabic{example}}
\newtheorem*{solution}{Solution}
\newtheorem*{application}{Application}
\newtheorem*{notation}{Notation}
\newtheorem*{vocabulary}{Vocabulaire}
\newtheorem*{properties}{Propriétés}



\theoremstyle{remark}
\newtheorem*{remark}{Remarque}
\newtheorem*{rappel}{Rappel}


\usepackage{etoolbox}
\AtBeginEnvironment{exercise}{\small}
\AtBeginEnvironment{example}{\small}

\usepackage{cases}
\usepackage[red]{mypack}

\usepackage[framemethod=TikZ]{mdframed}

\definecolor{bg}{rgb}{0.4,0.25,0.95}
\definecolor{pagebg}{rgb}{0,0,0.5}
\surroundwithmdframed[
   topline=false,
   rightline=false,
   bottomline=false,
   leftmargin=\parindent,
   skipabove=8pt,
   skipbelow=8pt,
   linecolor=blue,
   innerbottommargin=10pt,
   % backgroundcolor=bg,font=\color{orange}\sffamily, fontcolor=white
]{definition}

\usepackage{empheq}
\usepackage[most]{tcolorbox}

\newtcbox{\mymath}[1][]{%
    nobeforeafter, math upper, tcbox raise base,
    enhanced, colframe=blue!30!black,
    colback=red!10, boxrule=1pt,
    #1}

\usepackage{unixode}


\DeclareMathOperator{\ord}{ord}
\DeclareMathOperator{\orb}{orb}
\DeclareMathOperator{\stab}{stab}
\DeclareMathOperator{\Stab}{stab}
\DeclareMathOperator{\ppcm}{ppcm}
\DeclareMathOperator{\conj}{Conj}
\DeclareMathOperator{\End}{End}
\DeclareMathOperator{\rot}{rot}
\DeclareMathOperator{\trs}{trace}
\DeclareMathOperator{\Ind}{Ind}
\DeclareMathOperator{\mat}{Mat}
\DeclareMathOperator{\id}{Id}
\DeclareMathOperator{\vect}{vect}
\DeclareMathOperator{\img}{img}
\DeclareMathOperator{\cov}{Cov}
\DeclareMathOperator{\dist}{dist}
\DeclareMathOperator{\irr}{Irr}
\DeclareMathOperator{\image}{Im}
\DeclareMathOperator{\pd}{\partial}
\DeclareMathOperator{\epi}{epi}
\DeclareMathOperator{\Argmin}{Argmin}
\DeclareMathOperator{\dom}{dom}
\DeclareMathOperator{\proj}{proj}
\DeclareMathOperator{\ctg}{ctg}
\DeclareMathOperator{\supp}{supp}
\DeclareMathOperator{\argmin}{argmin}
\DeclareMathOperator{\mult}{mult}
\DeclareMathOperator{\ch}{ch}
\DeclareMathOperator{\sh}{sh}
\DeclareMathOperator{\rang}{rang}
\DeclareMathOperator{\diam}{diam}
\DeclareMathOperator{\Epigraphe}{Epigraphe}




\usepackage{xcolor}
\everymath{\color{blue}}
%\everymath{\color[rgb]{0,1,1}}
%\pagecolor[rgb]{0,0,0.5}


\newcommand*{\pdtest}[3][]{\ensuremath{\frac{\partial^{#1} #2}{\partial #3}}}

\newcommand*{\deffunc}[6][]{\ensuremath{
\begin{array}{rcl}
#2 : #3 &\rightarrow& #4\\
#5 &\mapsto& #6
\end{array}
}}

\newcommand{\eqcolon}{\mathrel{\resizebox{\widthof{$\mathord{=}$}}{\height}{ $\!\!=\!\!\resizebox{1.2\width}{0.8\height}{\raisebox{0.23ex}{$\mathop{:}$}}\!\!$ }}}
\newcommand{\coloneq}{\mathrel{\resizebox{\widthof{$\mathord{=}$}}{\height}{ $\!\!\resizebox{1.2\width}{0.8\height}{\raisebox{0.23ex}{$\mathop{:}$}}\!\!=\!\!$ }}}
\newcommand{\eqcolonl}{\ensuremath{\mathrel{=\!\!\mathop{:}}}}
\newcommand{\coloneql}{\ensuremath{\mathrel{\mathop{:} \!\! =}}}
\newcommand{\vc}[1]{% inline column vector
  \left(\begin{smallmatrix}#1\end{smallmatrix}\right)%
}
\newcommand{\vr}[1]{% inline row vector
  \begin{smallmatrix}(\,#1\,)\end{smallmatrix}%
}
\makeatletter
\newcommand*{\defeq}{\ =\mathrel{\rlap{%
                     \raisebox{0.3ex}{$\m@th\cdot$}}%
                     \raisebox{-0.3ex}{$\m@th\cdot$}}%
                     }
\makeatother

\newcommand{\mathcircle}[1]{% inline row vector
 \overset{\circ}{#1}
}
\newcommand{\ulim}{% low limit
 \underline{\lim}
}
\newcommand{\ssi}{% iff
\iff
}
\newcommand{\ps}[2]{
\expval{#1 | #2}
}
\newcommand{\df}[1]{
\mqty{#1}
}
\newcommand{\n}[1]{
\norm{#1}
}
\newcommand{\sys}[1]{
\left\{\smqty{#1}\right.
}


\newcommand{\eqdef}{\ensuremath{\overset{\text{def}}=}}


\def\Circlearrowright{\ensuremath{%
  \rotatebox[origin=c]{230}{$\circlearrowright$}}}

\newcommand\ct[1]{\text{\rmfamily\upshape #1}}
\newcommand\question[1]{ {\color{red} ...!? \small #1}}
\newcommand\caz[1]{\left\{\begin{array} #1 \end{array}\right.}
\newcommand\const{\text{\rmfamily\upshape const}}
\newcommand\toP{ \overset{\pro}{\to}}
\newcommand\toPP{ \overset{\text{PP}}{\to}}
\newcommand{\oeq}{\mathrel{\text{\textcircled{$=$}}}}





\usepackage{xcolor}
% \usepackage[normalem]{ulem}
\usepackage{lipsum}
\makeatletter
% \newcommand\colorwave[1][blue]{\bgroup \markoverwith{\lower3.5\p@\hbox{\sixly \textcolor{#1}{\char58}}}\ULon}
%\font\sixly=lasy6 % does not re-load if already loaded, so no memory problem.

\newmdtheoremenv[
linewidth= 1pt,linecolor= blue,%
leftmargin=20,rightmargin=20,innertopmargin=0pt, innerrightmargin=40,%
tikzsetting = { draw=lightgray, line width = 0.3pt,dashed,%
dash pattern = on 15pt off 3pt},%
splittopskip=\topskip,skipbelow=\baselineskip,%
skipabove=\baselineskip,ntheorem,roundcorner=0pt,
% backgroundcolor=pagebg,font=\color{orange}\sffamily, fontcolor=white
]{examplebox}{Exemple}[section]



\newcommand\R{\mathbb{R}}
\newcommand\Z{\mathbb{Z}}
\newcommand\N{\mathbb{N}}
\newcommand\E{\mathbb{E}}
\newcommand\F{\mathcal{F}}
\newcommand\cH{\mathcal{H}}
\newcommand\V{\mathbb{V}}
\newcommand\dmo{ ^{-1} }
\newcommand\kapa{\kappa}
\newcommand\im{Im}
\newcommand\hs{\mathcal{H}}





\usepackage{soul}

\makeatletter
\newcommand*{\whiten}[1]{\llap{\textcolor{white}{{\the\SOUL@token}}\hspace{#1pt}}}
\DeclareRobustCommand*\myul{%
    \def\SOUL@everyspace{\underline{\space}\kern\z@}%
    \def\SOUL@everytoken{%
     \setbox0=\hbox{\the\SOUL@token}%
     \ifdim\dp0>\z@
        \raisebox{\dp0}{\underline{\phantom{\the\SOUL@token}}}%
        \whiten{1}\whiten{0}%
        \whiten{-1}\whiten{-2}%
        \llap{\the\SOUL@token}%
     \else
        \underline{\the\SOUL@token}%
     \fi}%
\SOUL@}
\makeatother

\newcommand*{\demp}{\fontfamily{lmtt}\selectfont}

\DeclareTextFontCommand{\textdemp}{\demp}

\begin{document}

\ifcomment
Multiline
comment
\fi
\ifcomment
\myul{Typesetting test}
% \color[rgb]{1,1,1}
$∑_i^n≠ 60º±∞π∆¬≈√j∫h≤≥µ$

$\CR \R\pro\ind\pro\gS\pro
\mqty[a&b\\c&d]$
$\pro\mathbb{P}$
$\dd{x}$

  \[
    \alpha(x)=\left\{
                \begin{array}{ll}
                  x\\
                  \frac{1}{1+e^{-kx}}\\
                  \frac{e^x-e^{-x}}{e^x+e^{-x}}
                \end{array}
              \right.
  \]

  $\expval{x}$
  
  $\chi_\rho(ghg\dmo)=\Tr(\rho_{ghg\dmo})=\Tr(\rho_g\circ\rho_h\circ\rho\dmo_g)=\Tr(\rho_h)\overset{\mbox{\scalebox{0.5}{$\Tr(AB)=\Tr(BA)$}}}{=}\chi_\rho(h)$
  	$\mathop{\oplus}_{\substack{x\in X}}$

$\mat(\rho_g)=(a_{ij}(g))_{\scriptsize \substack{1\leq i\leq d \\ 1\leq j\leq d}}$ et $\mat(\rho'_g)=(a'_{ij}(g))_{\scriptsize \substack{1\leq i'\leq d' \\ 1\leq j'\leq d'}}$



\[\int_a^b{\mathbb{R}^2}g(u, v)\dd{P_{XY}}(u, v)=\iint g(u,v) f_{XY}(u, v)\dd \lambda(u) \dd \lambda(v)\]
$$\lim_{x\to\infty} f(x)$$	
$$\iiiint_V \mu(t,u,v,w) \,dt\,du\,dv\,dw$$
$$\sum_{n=1}^{\infty} 2^{-n} = 1$$	
\begin{definition}
	Si $X$ et $Y$ sont 2 v.a. ou definit la \textsc{Covariance} entre $X$ et $Y$ comme
	$\cov(X,Y)\overset{\text{def}}{=}\E\left[(X-\E(X))(Y-\E(Y))\right]=\E(XY)-\E(X)\E(Y)$.
\end{definition}
\fi
\pagebreak

% \tableofcontents

% insert your code here
%\input{./algebra/main.tex}
%\input{./geometrie-differentielle/main.tex}
%\input{./probabilite/main.tex}
%\input{./analyse-fonctionnelle/main.tex}
% \input{./Analyse-convexe-et-dualite-en-optimisation/main.tex}
%\input{./tikz/main.tex}
%\input{./Theorie-du-distributions/main.tex}
%\input{./optimisation/mine.tex}
 \input{./modelisation/main.tex}

% yves.aubry@univ-tln.fr : algebra

\end{document}

%\input{./optimisation/mine.tex}
 % !TEX encoding = UTF-8 Unicode
% !TEX TS-program = xelatex

\documentclass[french]{report}

%\usepackage[utf8]{inputenc}
%\usepackage[T1]{fontenc}
\usepackage{babel}


\newif\ifcomment
%\commenttrue # Show comments

\usepackage{physics}
\usepackage{amssymb}


\usepackage{amsthm}
% \usepackage{thmtools}
\usepackage{mathtools}
\usepackage{amsfonts}

\usepackage{color}

\usepackage{tikz}

\usepackage{geometry}
\geometry{a5paper, margin=0.1in, right=1cm}

\usepackage{dsfont}

\usepackage{graphicx}
\graphicspath{ {images/} }

\usepackage{faktor}

\usepackage{IEEEtrantools}
\usepackage{enumerate}   
\usepackage[PostScript=dvips]{"/Users/aware/Documents/Courses/diagrams"}


\newtheorem{theorem}{Théorème}[section]
\renewcommand{\thetheorem}{\arabic{theorem}}
\newtheorem{lemme}{Lemme}[section]
\renewcommand{\thelemme}{\arabic{lemme}}
\newtheorem{proposition}{Proposition}[section]
\renewcommand{\theproposition}{\arabic{proposition}}
\newtheorem{notations}{Notations}[section]
\newtheorem{problem}{Problème}[section]
\newtheorem{corollary}{Corollaire}[theorem]
\renewcommand{\thecorollary}{\arabic{corollary}}
\newtheorem{property}{Propriété}[section]
\newtheorem{objective}{Objectif}[section]

\theoremstyle{definition}
\newtheorem{definition}{Définition}[section]
\renewcommand{\thedefinition}{\arabic{definition}}
\newtheorem{exercise}{Exercice}[chapter]
\renewcommand{\theexercise}{\arabic{exercise}}
\newtheorem{example}{Exemple}[chapter]
\renewcommand{\theexample}{\arabic{example}}
\newtheorem*{solution}{Solution}
\newtheorem*{application}{Application}
\newtheorem*{notation}{Notation}
\newtheorem*{vocabulary}{Vocabulaire}
\newtheorem*{properties}{Propriétés}



\theoremstyle{remark}
\newtheorem*{remark}{Remarque}
\newtheorem*{rappel}{Rappel}


\usepackage{etoolbox}
\AtBeginEnvironment{exercise}{\small}
\AtBeginEnvironment{example}{\small}

\usepackage{cases}
\usepackage[red]{mypack}

\usepackage[framemethod=TikZ]{mdframed}

\definecolor{bg}{rgb}{0.4,0.25,0.95}
\definecolor{pagebg}{rgb}{0,0,0.5}
\surroundwithmdframed[
   topline=false,
   rightline=false,
   bottomline=false,
   leftmargin=\parindent,
   skipabove=8pt,
   skipbelow=8pt,
   linecolor=blue,
   innerbottommargin=10pt,
   % backgroundcolor=bg,font=\color{orange}\sffamily, fontcolor=white
]{definition}

\usepackage{empheq}
\usepackage[most]{tcolorbox}

\newtcbox{\mymath}[1][]{%
    nobeforeafter, math upper, tcbox raise base,
    enhanced, colframe=blue!30!black,
    colback=red!10, boxrule=1pt,
    #1}

\usepackage{unixode}


\DeclareMathOperator{\ord}{ord}
\DeclareMathOperator{\orb}{orb}
\DeclareMathOperator{\stab}{stab}
\DeclareMathOperator{\Stab}{stab}
\DeclareMathOperator{\ppcm}{ppcm}
\DeclareMathOperator{\conj}{Conj}
\DeclareMathOperator{\End}{End}
\DeclareMathOperator{\rot}{rot}
\DeclareMathOperator{\trs}{trace}
\DeclareMathOperator{\Ind}{Ind}
\DeclareMathOperator{\mat}{Mat}
\DeclareMathOperator{\id}{Id}
\DeclareMathOperator{\vect}{vect}
\DeclareMathOperator{\img}{img}
\DeclareMathOperator{\cov}{Cov}
\DeclareMathOperator{\dist}{dist}
\DeclareMathOperator{\irr}{Irr}
\DeclareMathOperator{\image}{Im}
\DeclareMathOperator{\pd}{\partial}
\DeclareMathOperator{\epi}{epi}
\DeclareMathOperator{\Argmin}{Argmin}
\DeclareMathOperator{\dom}{dom}
\DeclareMathOperator{\proj}{proj}
\DeclareMathOperator{\ctg}{ctg}
\DeclareMathOperator{\supp}{supp}
\DeclareMathOperator{\argmin}{argmin}
\DeclareMathOperator{\mult}{mult}
\DeclareMathOperator{\ch}{ch}
\DeclareMathOperator{\sh}{sh}
\DeclareMathOperator{\rang}{rang}
\DeclareMathOperator{\diam}{diam}
\DeclareMathOperator{\Epigraphe}{Epigraphe}




\usepackage{xcolor}
\everymath{\color{blue}}
%\everymath{\color[rgb]{0,1,1}}
%\pagecolor[rgb]{0,0,0.5}


\newcommand*{\pdtest}[3][]{\ensuremath{\frac{\partial^{#1} #2}{\partial #3}}}

\newcommand*{\deffunc}[6][]{\ensuremath{
\begin{array}{rcl}
#2 : #3 &\rightarrow& #4\\
#5 &\mapsto& #6
\end{array}
}}

\newcommand{\eqcolon}{\mathrel{\resizebox{\widthof{$\mathord{=}$}}{\height}{ $\!\!=\!\!\resizebox{1.2\width}{0.8\height}{\raisebox{0.23ex}{$\mathop{:}$}}\!\!$ }}}
\newcommand{\coloneq}{\mathrel{\resizebox{\widthof{$\mathord{=}$}}{\height}{ $\!\!\resizebox{1.2\width}{0.8\height}{\raisebox{0.23ex}{$\mathop{:}$}}\!\!=\!\!$ }}}
\newcommand{\eqcolonl}{\ensuremath{\mathrel{=\!\!\mathop{:}}}}
\newcommand{\coloneql}{\ensuremath{\mathrel{\mathop{:} \!\! =}}}
\newcommand{\vc}[1]{% inline column vector
  \left(\begin{smallmatrix}#1\end{smallmatrix}\right)%
}
\newcommand{\vr}[1]{% inline row vector
  \begin{smallmatrix}(\,#1\,)\end{smallmatrix}%
}
\makeatletter
\newcommand*{\defeq}{\ =\mathrel{\rlap{%
                     \raisebox{0.3ex}{$\m@th\cdot$}}%
                     \raisebox{-0.3ex}{$\m@th\cdot$}}%
                     }
\makeatother

\newcommand{\mathcircle}[1]{% inline row vector
 \overset{\circ}{#1}
}
\newcommand{\ulim}{% low limit
 \underline{\lim}
}
\newcommand{\ssi}{% iff
\iff
}
\newcommand{\ps}[2]{
\expval{#1 | #2}
}
\newcommand{\df}[1]{
\mqty{#1}
}
\newcommand{\n}[1]{
\norm{#1}
}
\newcommand{\sys}[1]{
\left\{\smqty{#1}\right.
}


\newcommand{\eqdef}{\ensuremath{\overset{\text{def}}=}}


\def\Circlearrowright{\ensuremath{%
  \rotatebox[origin=c]{230}{$\circlearrowright$}}}

\newcommand\ct[1]{\text{\rmfamily\upshape #1}}
\newcommand\question[1]{ {\color{red} ...!? \small #1}}
\newcommand\caz[1]{\left\{\begin{array} #1 \end{array}\right.}
\newcommand\const{\text{\rmfamily\upshape const}}
\newcommand\toP{ \overset{\pro}{\to}}
\newcommand\toPP{ \overset{\text{PP}}{\to}}
\newcommand{\oeq}{\mathrel{\text{\textcircled{$=$}}}}





\usepackage{xcolor}
% \usepackage[normalem]{ulem}
\usepackage{lipsum}
\makeatletter
% \newcommand\colorwave[1][blue]{\bgroup \markoverwith{\lower3.5\p@\hbox{\sixly \textcolor{#1}{\char58}}}\ULon}
%\font\sixly=lasy6 % does not re-load if already loaded, so no memory problem.

\newmdtheoremenv[
linewidth= 1pt,linecolor= blue,%
leftmargin=20,rightmargin=20,innertopmargin=0pt, innerrightmargin=40,%
tikzsetting = { draw=lightgray, line width = 0.3pt,dashed,%
dash pattern = on 15pt off 3pt},%
splittopskip=\topskip,skipbelow=\baselineskip,%
skipabove=\baselineskip,ntheorem,roundcorner=0pt,
% backgroundcolor=pagebg,font=\color{orange}\sffamily, fontcolor=white
]{examplebox}{Exemple}[section]



\newcommand\R{\mathbb{R}}
\newcommand\Z{\mathbb{Z}}
\newcommand\N{\mathbb{N}}
\newcommand\E{\mathbb{E}}
\newcommand\F{\mathcal{F}}
\newcommand\cH{\mathcal{H}}
\newcommand\V{\mathbb{V}}
\newcommand\dmo{ ^{-1} }
\newcommand\kapa{\kappa}
\newcommand\im{Im}
\newcommand\hs{\mathcal{H}}





\usepackage{soul}

\makeatletter
\newcommand*{\whiten}[1]{\llap{\textcolor{white}{{\the\SOUL@token}}\hspace{#1pt}}}
\DeclareRobustCommand*\myul{%
    \def\SOUL@everyspace{\underline{\space}\kern\z@}%
    \def\SOUL@everytoken{%
     \setbox0=\hbox{\the\SOUL@token}%
     \ifdim\dp0>\z@
        \raisebox{\dp0}{\underline{\phantom{\the\SOUL@token}}}%
        \whiten{1}\whiten{0}%
        \whiten{-1}\whiten{-2}%
        \llap{\the\SOUL@token}%
     \else
        \underline{\the\SOUL@token}%
     \fi}%
\SOUL@}
\makeatother

\newcommand*{\demp}{\fontfamily{lmtt}\selectfont}

\DeclareTextFontCommand{\textdemp}{\demp}

\begin{document}

\ifcomment
Multiline
comment
\fi
\ifcomment
\myul{Typesetting test}
% \color[rgb]{1,1,1}
$∑_i^n≠ 60º±∞π∆¬≈√j∫h≤≥µ$

$\CR \R\pro\ind\pro\gS\pro
\mqty[a&b\\c&d]$
$\pro\mathbb{P}$
$\dd{x}$

  \[
    \alpha(x)=\left\{
                \begin{array}{ll}
                  x\\
                  \frac{1}{1+e^{-kx}}\\
                  \frac{e^x-e^{-x}}{e^x+e^{-x}}
                \end{array}
              \right.
  \]

  $\expval{x}$
  
  $\chi_\rho(ghg\dmo)=\Tr(\rho_{ghg\dmo})=\Tr(\rho_g\circ\rho_h\circ\rho\dmo_g)=\Tr(\rho_h)\overset{\mbox{\scalebox{0.5}{$\Tr(AB)=\Tr(BA)$}}}{=}\chi_\rho(h)$
  	$\mathop{\oplus}_{\substack{x\in X}}$

$\mat(\rho_g)=(a_{ij}(g))_{\scriptsize \substack{1\leq i\leq d \\ 1\leq j\leq d}}$ et $\mat(\rho'_g)=(a'_{ij}(g))_{\scriptsize \substack{1\leq i'\leq d' \\ 1\leq j'\leq d'}}$



\[\int_a^b{\mathbb{R}^2}g(u, v)\dd{P_{XY}}(u, v)=\iint g(u,v) f_{XY}(u, v)\dd \lambda(u) \dd \lambda(v)\]
$$\lim_{x\to\infty} f(x)$$	
$$\iiiint_V \mu(t,u,v,w) \,dt\,du\,dv\,dw$$
$$\sum_{n=1}^{\infty} 2^{-n} = 1$$	
\begin{definition}
	Si $X$ et $Y$ sont 2 v.a. ou definit la \textsc{Covariance} entre $X$ et $Y$ comme
	$\cov(X,Y)\overset{\text{def}}{=}\E\left[(X-\E(X))(Y-\E(Y))\right]=\E(XY)-\E(X)\E(Y)$.
\end{definition}
\fi
\pagebreak

% \tableofcontents

% insert your code here
%\input{./algebra/main.tex}
%\input{./geometrie-differentielle/main.tex}
%\input{./probabilite/main.tex}
%\input{./analyse-fonctionnelle/main.tex}
% \input{./Analyse-convexe-et-dualite-en-optimisation/main.tex}
%\input{./tikz/main.tex}
%\input{./Theorie-du-distributions/main.tex}
%\input{./optimisation/mine.tex}
 \input{./modelisation/main.tex}

% yves.aubry@univ-tln.fr : algebra

\end{document}


% yves.aubry@univ-tln.fr : algebra

\end{document}

%% !TEX encoding = UTF-8 Unicode
% !TEX TS-program = xelatex

\documentclass[french]{report}

%\usepackage[utf8]{inputenc}
%\usepackage[T1]{fontenc}
\usepackage{babel}


\newif\ifcomment
%\commenttrue # Show comments

\usepackage{physics}
\usepackage{amssymb}


\usepackage{amsthm}
% \usepackage{thmtools}
\usepackage{mathtools}
\usepackage{amsfonts}

\usepackage{color}

\usepackage{tikz}

\usepackage{geometry}
\geometry{a5paper, margin=0.1in, right=1cm}

\usepackage{dsfont}

\usepackage{graphicx}
\graphicspath{ {images/} }

\usepackage{faktor}

\usepackage{IEEEtrantools}
\usepackage{enumerate}   
\usepackage[PostScript=dvips]{"/Users/aware/Documents/Courses/diagrams"}


\newtheorem{theorem}{Théorème}[section]
\renewcommand{\thetheorem}{\arabic{theorem}}
\newtheorem{lemme}{Lemme}[section]
\renewcommand{\thelemme}{\arabic{lemme}}
\newtheorem{proposition}{Proposition}[section]
\renewcommand{\theproposition}{\arabic{proposition}}
\newtheorem{notations}{Notations}[section]
\newtheorem{problem}{Problème}[section]
\newtheorem{corollary}{Corollaire}[theorem]
\renewcommand{\thecorollary}{\arabic{corollary}}
\newtheorem{property}{Propriété}[section]
\newtheorem{objective}{Objectif}[section]

\theoremstyle{definition}
\newtheorem{definition}{Définition}[section]
\renewcommand{\thedefinition}{\arabic{definition}}
\newtheorem{exercise}{Exercice}[chapter]
\renewcommand{\theexercise}{\arabic{exercise}}
\newtheorem{example}{Exemple}[chapter]
\renewcommand{\theexample}{\arabic{example}}
\newtheorem*{solution}{Solution}
\newtheorem*{application}{Application}
\newtheorem*{notation}{Notation}
\newtheorem*{vocabulary}{Vocabulaire}
\newtheorem*{properties}{Propriétés}



\theoremstyle{remark}
\newtheorem*{remark}{Remarque}
\newtheorem*{rappel}{Rappel}


\usepackage{etoolbox}
\AtBeginEnvironment{exercise}{\small}
\AtBeginEnvironment{example}{\small}

\usepackage{cases}
\usepackage[red]{mypack}

\usepackage[framemethod=TikZ]{mdframed}

\definecolor{bg}{rgb}{0.4,0.25,0.95}
\definecolor{pagebg}{rgb}{0,0,0.5}
\surroundwithmdframed[
   topline=false,
   rightline=false,
   bottomline=false,
   leftmargin=\parindent,
   skipabove=8pt,
   skipbelow=8pt,
   linecolor=blue,
   innerbottommargin=10pt,
   % backgroundcolor=bg,font=\color{orange}\sffamily, fontcolor=white
]{definition}

\usepackage{empheq}
\usepackage[most]{tcolorbox}

\newtcbox{\mymath}[1][]{%
    nobeforeafter, math upper, tcbox raise base,
    enhanced, colframe=blue!30!black,
    colback=red!10, boxrule=1pt,
    #1}

\usepackage{unixode}


\DeclareMathOperator{\ord}{ord}
\DeclareMathOperator{\orb}{orb}
\DeclareMathOperator{\stab}{stab}
\DeclareMathOperator{\Stab}{stab}
\DeclareMathOperator{\ppcm}{ppcm}
\DeclareMathOperator{\conj}{Conj}
\DeclareMathOperator{\End}{End}
\DeclareMathOperator{\rot}{rot}
\DeclareMathOperator{\trs}{trace}
\DeclareMathOperator{\Ind}{Ind}
\DeclareMathOperator{\mat}{Mat}
\DeclareMathOperator{\id}{Id}
\DeclareMathOperator{\vect}{vect}
\DeclareMathOperator{\img}{img}
\DeclareMathOperator{\cov}{Cov}
\DeclareMathOperator{\dist}{dist}
\DeclareMathOperator{\irr}{Irr}
\DeclareMathOperator{\image}{Im}
\DeclareMathOperator{\pd}{\partial}
\DeclareMathOperator{\epi}{epi}
\DeclareMathOperator{\Argmin}{Argmin}
\DeclareMathOperator{\dom}{dom}
\DeclareMathOperator{\proj}{proj}
\DeclareMathOperator{\ctg}{ctg}
\DeclareMathOperator{\supp}{supp}
\DeclareMathOperator{\argmin}{argmin}
\DeclareMathOperator{\mult}{mult}
\DeclareMathOperator{\ch}{ch}
\DeclareMathOperator{\sh}{sh}
\DeclareMathOperator{\rang}{rang}
\DeclareMathOperator{\diam}{diam}
\DeclareMathOperator{\Epigraphe}{Epigraphe}




\usepackage{xcolor}
\everymath{\color{blue}}
%\everymath{\color[rgb]{0,1,1}}
%\pagecolor[rgb]{0,0,0.5}


\newcommand*{\pdtest}[3][]{\ensuremath{\frac{\partial^{#1} #2}{\partial #3}}}

\newcommand*{\deffunc}[6][]{\ensuremath{
\begin{array}{rcl}
#2 : #3 &\rightarrow& #4\\
#5 &\mapsto& #6
\end{array}
}}

\newcommand{\eqcolon}{\mathrel{\resizebox{\widthof{$\mathord{=}$}}{\height}{ $\!\!=\!\!\resizebox{1.2\width}{0.8\height}{\raisebox{0.23ex}{$\mathop{:}$}}\!\!$ }}}
\newcommand{\coloneq}{\mathrel{\resizebox{\widthof{$\mathord{=}$}}{\height}{ $\!\!\resizebox{1.2\width}{0.8\height}{\raisebox{0.23ex}{$\mathop{:}$}}\!\!=\!\!$ }}}
\newcommand{\eqcolonl}{\ensuremath{\mathrel{=\!\!\mathop{:}}}}
\newcommand{\coloneql}{\ensuremath{\mathrel{\mathop{:} \!\! =}}}
\newcommand{\vc}[1]{% inline column vector
  \left(\begin{smallmatrix}#1\end{smallmatrix}\right)%
}
\newcommand{\vr}[1]{% inline row vector
  \begin{smallmatrix}(\,#1\,)\end{smallmatrix}%
}
\makeatletter
\newcommand*{\defeq}{\ =\mathrel{\rlap{%
                     \raisebox{0.3ex}{$\m@th\cdot$}}%
                     \raisebox{-0.3ex}{$\m@th\cdot$}}%
                     }
\makeatother

\newcommand{\mathcircle}[1]{% inline row vector
 \overset{\circ}{#1}
}
\newcommand{\ulim}{% low limit
 \underline{\lim}
}
\newcommand{\ssi}{% iff
\iff
}
\newcommand{\ps}[2]{
\expval{#1 | #2}
}
\newcommand{\df}[1]{
\mqty{#1}
}
\newcommand{\n}[1]{
\norm{#1}
}
\newcommand{\sys}[1]{
\left\{\smqty{#1}\right.
}


\newcommand{\eqdef}{\ensuremath{\overset{\text{def}}=}}


\def\Circlearrowright{\ensuremath{%
  \rotatebox[origin=c]{230}{$\circlearrowright$}}}

\newcommand\ct[1]{\text{\rmfamily\upshape #1}}
\newcommand\question[1]{ {\color{red} ...!? \small #1}}
\newcommand\caz[1]{\left\{\begin{array} #1 \end{array}\right.}
\newcommand\const{\text{\rmfamily\upshape const}}
\newcommand\toP{ \overset{\pro}{\to}}
\newcommand\toPP{ \overset{\text{PP}}{\to}}
\newcommand{\oeq}{\mathrel{\text{\textcircled{$=$}}}}





\usepackage{xcolor}
% \usepackage[normalem]{ulem}
\usepackage{lipsum}
\makeatletter
% \newcommand\colorwave[1][blue]{\bgroup \markoverwith{\lower3.5\p@\hbox{\sixly \textcolor{#1}{\char58}}}\ULon}
%\font\sixly=lasy6 % does not re-load if already loaded, so no memory problem.

\newmdtheoremenv[
linewidth= 1pt,linecolor= blue,%
leftmargin=20,rightmargin=20,innertopmargin=0pt, innerrightmargin=40,%
tikzsetting = { draw=lightgray, line width = 0.3pt,dashed,%
dash pattern = on 15pt off 3pt},%
splittopskip=\topskip,skipbelow=\baselineskip,%
skipabove=\baselineskip,ntheorem,roundcorner=0pt,
% backgroundcolor=pagebg,font=\color{orange}\sffamily, fontcolor=white
]{examplebox}{Exemple}[section]



\newcommand\R{\mathbb{R}}
\newcommand\Z{\mathbb{Z}}
\newcommand\N{\mathbb{N}}
\newcommand\E{\mathbb{E}}
\newcommand\F{\mathcal{F}}
\newcommand\cH{\mathcal{H}}
\newcommand\V{\mathbb{V}}
\newcommand\dmo{ ^{-1} }
\newcommand\kapa{\kappa}
\newcommand\im{Im}
\newcommand\hs{\mathcal{H}}





\usepackage{soul}

\makeatletter
\newcommand*{\whiten}[1]{\llap{\textcolor{white}{{\the\SOUL@token}}\hspace{#1pt}}}
\DeclareRobustCommand*\myul{%
    \def\SOUL@everyspace{\underline{\space}\kern\z@}%
    \def\SOUL@everytoken{%
     \setbox0=\hbox{\the\SOUL@token}%
     \ifdim\dp0>\z@
        \raisebox{\dp0}{\underline{\phantom{\the\SOUL@token}}}%
        \whiten{1}\whiten{0}%
        \whiten{-1}\whiten{-2}%
        \llap{\the\SOUL@token}%
     \else
        \underline{\the\SOUL@token}%
     \fi}%
\SOUL@}
\makeatother

\newcommand*{\demp}{\fontfamily{lmtt}\selectfont}

\DeclareTextFontCommand{\textdemp}{\demp}

\begin{document}

\ifcomment
Multiline
comment
\fi
\ifcomment
\myul{Typesetting test}
% \color[rgb]{1,1,1}
$∑_i^n≠ 60º±∞π∆¬≈√j∫h≤≥µ$

$\CR \R\pro\ind\pro\gS\pro
\mqty[a&b\\c&d]$
$\pro\mathbb{P}$
$\dd{x}$

  \[
    \alpha(x)=\left\{
                \begin{array}{ll}
                  x\\
                  \frac{1}{1+e^{-kx}}\\
                  \frac{e^x-e^{-x}}{e^x+e^{-x}}
                \end{array}
              \right.
  \]

  $\expval{x}$
  
  $\chi_\rho(ghg\dmo)=\Tr(\rho_{ghg\dmo})=\Tr(\rho_g\circ\rho_h\circ\rho\dmo_g)=\Tr(\rho_h)\overset{\mbox{\scalebox{0.5}{$\Tr(AB)=\Tr(BA)$}}}{=}\chi_\rho(h)$
  	$\mathop{\oplus}_{\substack{x\in X}}$

$\mat(\rho_g)=(a_{ij}(g))_{\scriptsize \substack{1\leq i\leq d \\ 1\leq j\leq d}}$ et $\mat(\rho'_g)=(a'_{ij}(g))_{\scriptsize \substack{1\leq i'\leq d' \\ 1\leq j'\leq d'}}$



\[\int_a^b{\mathbb{R}^2}g(u, v)\dd{P_{XY}}(u, v)=\iint g(u,v) f_{XY}(u, v)\dd \lambda(u) \dd \lambda(v)\]
$$\lim_{x\to\infty} f(x)$$	
$$\iiiint_V \mu(t,u,v,w) \,dt\,du\,dv\,dw$$
$$\sum_{n=1}^{\infty} 2^{-n} = 1$$	
\begin{definition}
	Si $X$ et $Y$ sont 2 v.a. ou definit la \textsc{Covariance} entre $X$ et $Y$ comme
	$\cov(X,Y)\overset{\text{def}}{=}\E\left[(X-\E(X))(Y-\E(Y))\right]=\E(XY)-\E(X)\E(Y)$.
\end{definition}
\fi
\pagebreak

% \tableofcontents

% insert your code here
%% !TEX encoding = UTF-8 Unicode
% !TEX TS-program = xelatex

\documentclass[french]{report}

%\usepackage[utf8]{inputenc}
%\usepackage[T1]{fontenc}
\usepackage{babel}


\newif\ifcomment
%\commenttrue # Show comments

\usepackage{physics}
\usepackage{amssymb}


\usepackage{amsthm}
% \usepackage{thmtools}
\usepackage{mathtools}
\usepackage{amsfonts}

\usepackage{color}

\usepackage{tikz}

\usepackage{geometry}
\geometry{a5paper, margin=0.1in, right=1cm}

\usepackage{dsfont}

\usepackage{graphicx}
\graphicspath{ {images/} }

\usepackage{faktor}

\usepackage{IEEEtrantools}
\usepackage{enumerate}   
\usepackage[PostScript=dvips]{"/Users/aware/Documents/Courses/diagrams"}


\newtheorem{theorem}{Théorème}[section]
\renewcommand{\thetheorem}{\arabic{theorem}}
\newtheorem{lemme}{Lemme}[section]
\renewcommand{\thelemme}{\arabic{lemme}}
\newtheorem{proposition}{Proposition}[section]
\renewcommand{\theproposition}{\arabic{proposition}}
\newtheorem{notations}{Notations}[section]
\newtheorem{problem}{Problème}[section]
\newtheorem{corollary}{Corollaire}[theorem]
\renewcommand{\thecorollary}{\arabic{corollary}}
\newtheorem{property}{Propriété}[section]
\newtheorem{objective}{Objectif}[section]

\theoremstyle{definition}
\newtheorem{definition}{Définition}[section]
\renewcommand{\thedefinition}{\arabic{definition}}
\newtheorem{exercise}{Exercice}[chapter]
\renewcommand{\theexercise}{\arabic{exercise}}
\newtheorem{example}{Exemple}[chapter]
\renewcommand{\theexample}{\arabic{example}}
\newtheorem*{solution}{Solution}
\newtheorem*{application}{Application}
\newtheorem*{notation}{Notation}
\newtheorem*{vocabulary}{Vocabulaire}
\newtheorem*{properties}{Propriétés}



\theoremstyle{remark}
\newtheorem*{remark}{Remarque}
\newtheorem*{rappel}{Rappel}


\usepackage{etoolbox}
\AtBeginEnvironment{exercise}{\small}
\AtBeginEnvironment{example}{\small}

\usepackage{cases}
\usepackage[red]{mypack}

\usepackage[framemethod=TikZ]{mdframed}

\definecolor{bg}{rgb}{0.4,0.25,0.95}
\definecolor{pagebg}{rgb}{0,0,0.5}
\surroundwithmdframed[
   topline=false,
   rightline=false,
   bottomline=false,
   leftmargin=\parindent,
   skipabove=8pt,
   skipbelow=8pt,
   linecolor=blue,
   innerbottommargin=10pt,
   % backgroundcolor=bg,font=\color{orange}\sffamily, fontcolor=white
]{definition}

\usepackage{empheq}
\usepackage[most]{tcolorbox}

\newtcbox{\mymath}[1][]{%
    nobeforeafter, math upper, tcbox raise base,
    enhanced, colframe=blue!30!black,
    colback=red!10, boxrule=1pt,
    #1}

\usepackage{unixode}


\DeclareMathOperator{\ord}{ord}
\DeclareMathOperator{\orb}{orb}
\DeclareMathOperator{\stab}{stab}
\DeclareMathOperator{\Stab}{stab}
\DeclareMathOperator{\ppcm}{ppcm}
\DeclareMathOperator{\conj}{Conj}
\DeclareMathOperator{\End}{End}
\DeclareMathOperator{\rot}{rot}
\DeclareMathOperator{\trs}{trace}
\DeclareMathOperator{\Ind}{Ind}
\DeclareMathOperator{\mat}{Mat}
\DeclareMathOperator{\id}{Id}
\DeclareMathOperator{\vect}{vect}
\DeclareMathOperator{\img}{img}
\DeclareMathOperator{\cov}{Cov}
\DeclareMathOperator{\dist}{dist}
\DeclareMathOperator{\irr}{Irr}
\DeclareMathOperator{\image}{Im}
\DeclareMathOperator{\pd}{\partial}
\DeclareMathOperator{\epi}{epi}
\DeclareMathOperator{\Argmin}{Argmin}
\DeclareMathOperator{\dom}{dom}
\DeclareMathOperator{\proj}{proj}
\DeclareMathOperator{\ctg}{ctg}
\DeclareMathOperator{\supp}{supp}
\DeclareMathOperator{\argmin}{argmin}
\DeclareMathOperator{\mult}{mult}
\DeclareMathOperator{\ch}{ch}
\DeclareMathOperator{\sh}{sh}
\DeclareMathOperator{\rang}{rang}
\DeclareMathOperator{\diam}{diam}
\DeclareMathOperator{\Epigraphe}{Epigraphe}




\usepackage{xcolor}
\everymath{\color{blue}}
%\everymath{\color[rgb]{0,1,1}}
%\pagecolor[rgb]{0,0,0.5}


\newcommand*{\pdtest}[3][]{\ensuremath{\frac{\partial^{#1} #2}{\partial #3}}}

\newcommand*{\deffunc}[6][]{\ensuremath{
\begin{array}{rcl}
#2 : #3 &\rightarrow& #4\\
#5 &\mapsto& #6
\end{array}
}}

\newcommand{\eqcolon}{\mathrel{\resizebox{\widthof{$\mathord{=}$}}{\height}{ $\!\!=\!\!\resizebox{1.2\width}{0.8\height}{\raisebox{0.23ex}{$\mathop{:}$}}\!\!$ }}}
\newcommand{\coloneq}{\mathrel{\resizebox{\widthof{$\mathord{=}$}}{\height}{ $\!\!\resizebox{1.2\width}{0.8\height}{\raisebox{0.23ex}{$\mathop{:}$}}\!\!=\!\!$ }}}
\newcommand{\eqcolonl}{\ensuremath{\mathrel{=\!\!\mathop{:}}}}
\newcommand{\coloneql}{\ensuremath{\mathrel{\mathop{:} \!\! =}}}
\newcommand{\vc}[1]{% inline column vector
  \left(\begin{smallmatrix}#1\end{smallmatrix}\right)%
}
\newcommand{\vr}[1]{% inline row vector
  \begin{smallmatrix}(\,#1\,)\end{smallmatrix}%
}
\makeatletter
\newcommand*{\defeq}{\ =\mathrel{\rlap{%
                     \raisebox{0.3ex}{$\m@th\cdot$}}%
                     \raisebox{-0.3ex}{$\m@th\cdot$}}%
                     }
\makeatother

\newcommand{\mathcircle}[1]{% inline row vector
 \overset{\circ}{#1}
}
\newcommand{\ulim}{% low limit
 \underline{\lim}
}
\newcommand{\ssi}{% iff
\iff
}
\newcommand{\ps}[2]{
\expval{#1 | #2}
}
\newcommand{\df}[1]{
\mqty{#1}
}
\newcommand{\n}[1]{
\norm{#1}
}
\newcommand{\sys}[1]{
\left\{\smqty{#1}\right.
}


\newcommand{\eqdef}{\ensuremath{\overset{\text{def}}=}}


\def\Circlearrowright{\ensuremath{%
  \rotatebox[origin=c]{230}{$\circlearrowright$}}}

\newcommand\ct[1]{\text{\rmfamily\upshape #1}}
\newcommand\question[1]{ {\color{red} ...!? \small #1}}
\newcommand\caz[1]{\left\{\begin{array} #1 \end{array}\right.}
\newcommand\const{\text{\rmfamily\upshape const}}
\newcommand\toP{ \overset{\pro}{\to}}
\newcommand\toPP{ \overset{\text{PP}}{\to}}
\newcommand{\oeq}{\mathrel{\text{\textcircled{$=$}}}}





\usepackage{xcolor}
% \usepackage[normalem]{ulem}
\usepackage{lipsum}
\makeatletter
% \newcommand\colorwave[1][blue]{\bgroup \markoverwith{\lower3.5\p@\hbox{\sixly \textcolor{#1}{\char58}}}\ULon}
%\font\sixly=lasy6 % does not re-load if already loaded, so no memory problem.

\newmdtheoremenv[
linewidth= 1pt,linecolor= blue,%
leftmargin=20,rightmargin=20,innertopmargin=0pt, innerrightmargin=40,%
tikzsetting = { draw=lightgray, line width = 0.3pt,dashed,%
dash pattern = on 15pt off 3pt},%
splittopskip=\topskip,skipbelow=\baselineskip,%
skipabove=\baselineskip,ntheorem,roundcorner=0pt,
% backgroundcolor=pagebg,font=\color{orange}\sffamily, fontcolor=white
]{examplebox}{Exemple}[section]



\newcommand\R{\mathbb{R}}
\newcommand\Z{\mathbb{Z}}
\newcommand\N{\mathbb{N}}
\newcommand\E{\mathbb{E}}
\newcommand\F{\mathcal{F}}
\newcommand\cH{\mathcal{H}}
\newcommand\V{\mathbb{V}}
\newcommand\dmo{ ^{-1} }
\newcommand\kapa{\kappa}
\newcommand\im{Im}
\newcommand\hs{\mathcal{H}}





\usepackage{soul}

\makeatletter
\newcommand*{\whiten}[1]{\llap{\textcolor{white}{{\the\SOUL@token}}\hspace{#1pt}}}
\DeclareRobustCommand*\myul{%
    \def\SOUL@everyspace{\underline{\space}\kern\z@}%
    \def\SOUL@everytoken{%
     \setbox0=\hbox{\the\SOUL@token}%
     \ifdim\dp0>\z@
        \raisebox{\dp0}{\underline{\phantom{\the\SOUL@token}}}%
        \whiten{1}\whiten{0}%
        \whiten{-1}\whiten{-2}%
        \llap{\the\SOUL@token}%
     \else
        \underline{\the\SOUL@token}%
     \fi}%
\SOUL@}
\makeatother

\newcommand*{\demp}{\fontfamily{lmtt}\selectfont}

\DeclareTextFontCommand{\textdemp}{\demp}

\begin{document}

\ifcomment
Multiline
comment
\fi
\ifcomment
\myul{Typesetting test}
% \color[rgb]{1,1,1}
$∑_i^n≠ 60º±∞π∆¬≈√j∫h≤≥µ$

$\CR \R\pro\ind\pro\gS\pro
\mqty[a&b\\c&d]$
$\pro\mathbb{P}$
$\dd{x}$

  \[
    \alpha(x)=\left\{
                \begin{array}{ll}
                  x\\
                  \frac{1}{1+e^{-kx}}\\
                  \frac{e^x-e^{-x}}{e^x+e^{-x}}
                \end{array}
              \right.
  \]

  $\expval{x}$
  
  $\chi_\rho(ghg\dmo)=\Tr(\rho_{ghg\dmo})=\Tr(\rho_g\circ\rho_h\circ\rho\dmo_g)=\Tr(\rho_h)\overset{\mbox{\scalebox{0.5}{$\Tr(AB)=\Tr(BA)$}}}{=}\chi_\rho(h)$
  	$\mathop{\oplus}_{\substack{x\in X}}$

$\mat(\rho_g)=(a_{ij}(g))_{\scriptsize \substack{1\leq i\leq d \\ 1\leq j\leq d}}$ et $\mat(\rho'_g)=(a'_{ij}(g))_{\scriptsize \substack{1\leq i'\leq d' \\ 1\leq j'\leq d'}}$



\[\int_a^b{\mathbb{R}^2}g(u, v)\dd{P_{XY}}(u, v)=\iint g(u,v) f_{XY}(u, v)\dd \lambda(u) \dd \lambda(v)\]
$$\lim_{x\to\infty} f(x)$$	
$$\iiiint_V \mu(t,u,v,w) \,dt\,du\,dv\,dw$$
$$\sum_{n=1}^{\infty} 2^{-n} = 1$$	
\begin{definition}
	Si $X$ et $Y$ sont 2 v.a. ou definit la \textsc{Covariance} entre $X$ et $Y$ comme
	$\cov(X,Y)\overset{\text{def}}{=}\E\left[(X-\E(X))(Y-\E(Y))\right]=\E(XY)-\E(X)\E(Y)$.
\end{definition}
\fi
\pagebreak

% \tableofcontents

% insert your code here
%\input{./algebra/main.tex}
%\input{./geometrie-differentielle/main.tex}
%\input{./probabilite/main.tex}
%\input{./analyse-fonctionnelle/main.tex}
% \input{./Analyse-convexe-et-dualite-en-optimisation/main.tex}
%\input{./tikz/main.tex}
%\input{./Theorie-du-distributions/main.tex}
%\input{./optimisation/mine.tex}
 \input{./modelisation/main.tex}

% yves.aubry@univ-tln.fr : algebra

\end{document}

%% !TEX encoding = UTF-8 Unicode
% !TEX TS-program = xelatex

\documentclass[french]{report}

%\usepackage[utf8]{inputenc}
%\usepackage[T1]{fontenc}
\usepackage{babel}


\newif\ifcomment
%\commenttrue # Show comments

\usepackage{physics}
\usepackage{amssymb}


\usepackage{amsthm}
% \usepackage{thmtools}
\usepackage{mathtools}
\usepackage{amsfonts}

\usepackage{color}

\usepackage{tikz}

\usepackage{geometry}
\geometry{a5paper, margin=0.1in, right=1cm}

\usepackage{dsfont}

\usepackage{graphicx}
\graphicspath{ {images/} }

\usepackage{faktor}

\usepackage{IEEEtrantools}
\usepackage{enumerate}   
\usepackage[PostScript=dvips]{"/Users/aware/Documents/Courses/diagrams"}


\newtheorem{theorem}{Théorème}[section]
\renewcommand{\thetheorem}{\arabic{theorem}}
\newtheorem{lemme}{Lemme}[section]
\renewcommand{\thelemme}{\arabic{lemme}}
\newtheorem{proposition}{Proposition}[section]
\renewcommand{\theproposition}{\arabic{proposition}}
\newtheorem{notations}{Notations}[section]
\newtheorem{problem}{Problème}[section]
\newtheorem{corollary}{Corollaire}[theorem]
\renewcommand{\thecorollary}{\arabic{corollary}}
\newtheorem{property}{Propriété}[section]
\newtheorem{objective}{Objectif}[section]

\theoremstyle{definition}
\newtheorem{definition}{Définition}[section]
\renewcommand{\thedefinition}{\arabic{definition}}
\newtheorem{exercise}{Exercice}[chapter]
\renewcommand{\theexercise}{\arabic{exercise}}
\newtheorem{example}{Exemple}[chapter]
\renewcommand{\theexample}{\arabic{example}}
\newtheorem*{solution}{Solution}
\newtheorem*{application}{Application}
\newtheorem*{notation}{Notation}
\newtheorem*{vocabulary}{Vocabulaire}
\newtheorem*{properties}{Propriétés}



\theoremstyle{remark}
\newtheorem*{remark}{Remarque}
\newtheorem*{rappel}{Rappel}


\usepackage{etoolbox}
\AtBeginEnvironment{exercise}{\small}
\AtBeginEnvironment{example}{\small}

\usepackage{cases}
\usepackage[red]{mypack}

\usepackage[framemethod=TikZ]{mdframed}

\definecolor{bg}{rgb}{0.4,0.25,0.95}
\definecolor{pagebg}{rgb}{0,0,0.5}
\surroundwithmdframed[
   topline=false,
   rightline=false,
   bottomline=false,
   leftmargin=\parindent,
   skipabove=8pt,
   skipbelow=8pt,
   linecolor=blue,
   innerbottommargin=10pt,
   % backgroundcolor=bg,font=\color{orange}\sffamily, fontcolor=white
]{definition}

\usepackage{empheq}
\usepackage[most]{tcolorbox}

\newtcbox{\mymath}[1][]{%
    nobeforeafter, math upper, tcbox raise base,
    enhanced, colframe=blue!30!black,
    colback=red!10, boxrule=1pt,
    #1}

\usepackage{unixode}


\DeclareMathOperator{\ord}{ord}
\DeclareMathOperator{\orb}{orb}
\DeclareMathOperator{\stab}{stab}
\DeclareMathOperator{\Stab}{stab}
\DeclareMathOperator{\ppcm}{ppcm}
\DeclareMathOperator{\conj}{Conj}
\DeclareMathOperator{\End}{End}
\DeclareMathOperator{\rot}{rot}
\DeclareMathOperator{\trs}{trace}
\DeclareMathOperator{\Ind}{Ind}
\DeclareMathOperator{\mat}{Mat}
\DeclareMathOperator{\id}{Id}
\DeclareMathOperator{\vect}{vect}
\DeclareMathOperator{\img}{img}
\DeclareMathOperator{\cov}{Cov}
\DeclareMathOperator{\dist}{dist}
\DeclareMathOperator{\irr}{Irr}
\DeclareMathOperator{\image}{Im}
\DeclareMathOperator{\pd}{\partial}
\DeclareMathOperator{\epi}{epi}
\DeclareMathOperator{\Argmin}{Argmin}
\DeclareMathOperator{\dom}{dom}
\DeclareMathOperator{\proj}{proj}
\DeclareMathOperator{\ctg}{ctg}
\DeclareMathOperator{\supp}{supp}
\DeclareMathOperator{\argmin}{argmin}
\DeclareMathOperator{\mult}{mult}
\DeclareMathOperator{\ch}{ch}
\DeclareMathOperator{\sh}{sh}
\DeclareMathOperator{\rang}{rang}
\DeclareMathOperator{\diam}{diam}
\DeclareMathOperator{\Epigraphe}{Epigraphe}




\usepackage{xcolor}
\everymath{\color{blue}}
%\everymath{\color[rgb]{0,1,1}}
%\pagecolor[rgb]{0,0,0.5}


\newcommand*{\pdtest}[3][]{\ensuremath{\frac{\partial^{#1} #2}{\partial #3}}}

\newcommand*{\deffunc}[6][]{\ensuremath{
\begin{array}{rcl}
#2 : #3 &\rightarrow& #4\\
#5 &\mapsto& #6
\end{array}
}}

\newcommand{\eqcolon}{\mathrel{\resizebox{\widthof{$\mathord{=}$}}{\height}{ $\!\!=\!\!\resizebox{1.2\width}{0.8\height}{\raisebox{0.23ex}{$\mathop{:}$}}\!\!$ }}}
\newcommand{\coloneq}{\mathrel{\resizebox{\widthof{$\mathord{=}$}}{\height}{ $\!\!\resizebox{1.2\width}{0.8\height}{\raisebox{0.23ex}{$\mathop{:}$}}\!\!=\!\!$ }}}
\newcommand{\eqcolonl}{\ensuremath{\mathrel{=\!\!\mathop{:}}}}
\newcommand{\coloneql}{\ensuremath{\mathrel{\mathop{:} \!\! =}}}
\newcommand{\vc}[1]{% inline column vector
  \left(\begin{smallmatrix}#1\end{smallmatrix}\right)%
}
\newcommand{\vr}[1]{% inline row vector
  \begin{smallmatrix}(\,#1\,)\end{smallmatrix}%
}
\makeatletter
\newcommand*{\defeq}{\ =\mathrel{\rlap{%
                     \raisebox{0.3ex}{$\m@th\cdot$}}%
                     \raisebox{-0.3ex}{$\m@th\cdot$}}%
                     }
\makeatother

\newcommand{\mathcircle}[1]{% inline row vector
 \overset{\circ}{#1}
}
\newcommand{\ulim}{% low limit
 \underline{\lim}
}
\newcommand{\ssi}{% iff
\iff
}
\newcommand{\ps}[2]{
\expval{#1 | #2}
}
\newcommand{\df}[1]{
\mqty{#1}
}
\newcommand{\n}[1]{
\norm{#1}
}
\newcommand{\sys}[1]{
\left\{\smqty{#1}\right.
}


\newcommand{\eqdef}{\ensuremath{\overset{\text{def}}=}}


\def\Circlearrowright{\ensuremath{%
  \rotatebox[origin=c]{230}{$\circlearrowright$}}}

\newcommand\ct[1]{\text{\rmfamily\upshape #1}}
\newcommand\question[1]{ {\color{red} ...!? \small #1}}
\newcommand\caz[1]{\left\{\begin{array} #1 \end{array}\right.}
\newcommand\const{\text{\rmfamily\upshape const}}
\newcommand\toP{ \overset{\pro}{\to}}
\newcommand\toPP{ \overset{\text{PP}}{\to}}
\newcommand{\oeq}{\mathrel{\text{\textcircled{$=$}}}}





\usepackage{xcolor}
% \usepackage[normalem]{ulem}
\usepackage{lipsum}
\makeatletter
% \newcommand\colorwave[1][blue]{\bgroup \markoverwith{\lower3.5\p@\hbox{\sixly \textcolor{#1}{\char58}}}\ULon}
%\font\sixly=lasy6 % does not re-load if already loaded, so no memory problem.

\newmdtheoremenv[
linewidth= 1pt,linecolor= blue,%
leftmargin=20,rightmargin=20,innertopmargin=0pt, innerrightmargin=40,%
tikzsetting = { draw=lightgray, line width = 0.3pt,dashed,%
dash pattern = on 15pt off 3pt},%
splittopskip=\topskip,skipbelow=\baselineskip,%
skipabove=\baselineskip,ntheorem,roundcorner=0pt,
% backgroundcolor=pagebg,font=\color{orange}\sffamily, fontcolor=white
]{examplebox}{Exemple}[section]



\newcommand\R{\mathbb{R}}
\newcommand\Z{\mathbb{Z}}
\newcommand\N{\mathbb{N}}
\newcommand\E{\mathbb{E}}
\newcommand\F{\mathcal{F}}
\newcommand\cH{\mathcal{H}}
\newcommand\V{\mathbb{V}}
\newcommand\dmo{ ^{-1} }
\newcommand\kapa{\kappa}
\newcommand\im{Im}
\newcommand\hs{\mathcal{H}}





\usepackage{soul}

\makeatletter
\newcommand*{\whiten}[1]{\llap{\textcolor{white}{{\the\SOUL@token}}\hspace{#1pt}}}
\DeclareRobustCommand*\myul{%
    \def\SOUL@everyspace{\underline{\space}\kern\z@}%
    \def\SOUL@everytoken{%
     \setbox0=\hbox{\the\SOUL@token}%
     \ifdim\dp0>\z@
        \raisebox{\dp0}{\underline{\phantom{\the\SOUL@token}}}%
        \whiten{1}\whiten{0}%
        \whiten{-1}\whiten{-2}%
        \llap{\the\SOUL@token}%
     \else
        \underline{\the\SOUL@token}%
     \fi}%
\SOUL@}
\makeatother

\newcommand*{\demp}{\fontfamily{lmtt}\selectfont}

\DeclareTextFontCommand{\textdemp}{\demp}

\begin{document}

\ifcomment
Multiline
comment
\fi
\ifcomment
\myul{Typesetting test}
% \color[rgb]{1,1,1}
$∑_i^n≠ 60º±∞π∆¬≈√j∫h≤≥µ$

$\CR \R\pro\ind\pro\gS\pro
\mqty[a&b\\c&d]$
$\pro\mathbb{P}$
$\dd{x}$

  \[
    \alpha(x)=\left\{
                \begin{array}{ll}
                  x\\
                  \frac{1}{1+e^{-kx}}\\
                  \frac{e^x-e^{-x}}{e^x+e^{-x}}
                \end{array}
              \right.
  \]

  $\expval{x}$
  
  $\chi_\rho(ghg\dmo)=\Tr(\rho_{ghg\dmo})=\Tr(\rho_g\circ\rho_h\circ\rho\dmo_g)=\Tr(\rho_h)\overset{\mbox{\scalebox{0.5}{$\Tr(AB)=\Tr(BA)$}}}{=}\chi_\rho(h)$
  	$\mathop{\oplus}_{\substack{x\in X}}$

$\mat(\rho_g)=(a_{ij}(g))_{\scriptsize \substack{1\leq i\leq d \\ 1\leq j\leq d}}$ et $\mat(\rho'_g)=(a'_{ij}(g))_{\scriptsize \substack{1\leq i'\leq d' \\ 1\leq j'\leq d'}}$



\[\int_a^b{\mathbb{R}^2}g(u, v)\dd{P_{XY}}(u, v)=\iint g(u,v) f_{XY}(u, v)\dd \lambda(u) \dd \lambda(v)\]
$$\lim_{x\to\infty} f(x)$$	
$$\iiiint_V \mu(t,u,v,w) \,dt\,du\,dv\,dw$$
$$\sum_{n=1}^{\infty} 2^{-n} = 1$$	
\begin{definition}
	Si $X$ et $Y$ sont 2 v.a. ou definit la \textsc{Covariance} entre $X$ et $Y$ comme
	$\cov(X,Y)\overset{\text{def}}{=}\E\left[(X-\E(X))(Y-\E(Y))\right]=\E(XY)-\E(X)\E(Y)$.
\end{definition}
\fi
\pagebreak

% \tableofcontents

% insert your code here
%\input{./algebra/main.tex}
%\input{./geometrie-differentielle/main.tex}
%\input{./probabilite/main.tex}
%\input{./analyse-fonctionnelle/main.tex}
% \input{./Analyse-convexe-et-dualite-en-optimisation/main.tex}
%\input{./tikz/main.tex}
%\input{./Theorie-du-distributions/main.tex}
%\input{./optimisation/mine.tex}
 \input{./modelisation/main.tex}

% yves.aubry@univ-tln.fr : algebra

\end{document}

%% !TEX encoding = UTF-8 Unicode
% !TEX TS-program = xelatex

\documentclass[french]{report}

%\usepackage[utf8]{inputenc}
%\usepackage[T1]{fontenc}
\usepackage{babel}


\newif\ifcomment
%\commenttrue # Show comments

\usepackage{physics}
\usepackage{amssymb}


\usepackage{amsthm}
% \usepackage{thmtools}
\usepackage{mathtools}
\usepackage{amsfonts}

\usepackage{color}

\usepackage{tikz}

\usepackage{geometry}
\geometry{a5paper, margin=0.1in, right=1cm}

\usepackage{dsfont}

\usepackage{graphicx}
\graphicspath{ {images/} }

\usepackage{faktor}

\usepackage{IEEEtrantools}
\usepackage{enumerate}   
\usepackage[PostScript=dvips]{"/Users/aware/Documents/Courses/diagrams"}


\newtheorem{theorem}{Théorème}[section]
\renewcommand{\thetheorem}{\arabic{theorem}}
\newtheorem{lemme}{Lemme}[section]
\renewcommand{\thelemme}{\arabic{lemme}}
\newtheorem{proposition}{Proposition}[section]
\renewcommand{\theproposition}{\arabic{proposition}}
\newtheorem{notations}{Notations}[section]
\newtheorem{problem}{Problème}[section]
\newtheorem{corollary}{Corollaire}[theorem]
\renewcommand{\thecorollary}{\arabic{corollary}}
\newtheorem{property}{Propriété}[section]
\newtheorem{objective}{Objectif}[section]

\theoremstyle{definition}
\newtheorem{definition}{Définition}[section]
\renewcommand{\thedefinition}{\arabic{definition}}
\newtheorem{exercise}{Exercice}[chapter]
\renewcommand{\theexercise}{\arabic{exercise}}
\newtheorem{example}{Exemple}[chapter]
\renewcommand{\theexample}{\arabic{example}}
\newtheorem*{solution}{Solution}
\newtheorem*{application}{Application}
\newtheorem*{notation}{Notation}
\newtheorem*{vocabulary}{Vocabulaire}
\newtheorem*{properties}{Propriétés}



\theoremstyle{remark}
\newtheorem*{remark}{Remarque}
\newtheorem*{rappel}{Rappel}


\usepackage{etoolbox}
\AtBeginEnvironment{exercise}{\small}
\AtBeginEnvironment{example}{\small}

\usepackage{cases}
\usepackage[red]{mypack}

\usepackage[framemethod=TikZ]{mdframed}

\definecolor{bg}{rgb}{0.4,0.25,0.95}
\definecolor{pagebg}{rgb}{0,0,0.5}
\surroundwithmdframed[
   topline=false,
   rightline=false,
   bottomline=false,
   leftmargin=\parindent,
   skipabove=8pt,
   skipbelow=8pt,
   linecolor=blue,
   innerbottommargin=10pt,
   % backgroundcolor=bg,font=\color{orange}\sffamily, fontcolor=white
]{definition}

\usepackage{empheq}
\usepackage[most]{tcolorbox}

\newtcbox{\mymath}[1][]{%
    nobeforeafter, math upper, tcbox raise base,
    enhanced, colframe=blue!30!black,
    colback=red!10, boxrule=1pt,
    #1}

\usepackage{unixode}


\DeclareMathOperator{\ord}{ord}
\DeclareMathOperator{\orb}{orb}
\DeclareMathOperator{\stab}{stab}
\DeclareMathOperator{\Stab}{stab}
\DeclareMathOperator{\ppcm}{ppcm}
\DeclareMathOperator{\conj}{Conj}
\DeclareMathOperator{\End}{End}
\DeclareMathOperator{\rot}{rot}
\DeclareMathOperator{\trs}{trace}
\DeclareMathOperator{\Ind}{Ind}
\DeclareMathOperator{\mat}{Mat}
\DeclareMathOperator{\id}{Id}
\DeclareMathOperator{\vect}{vect}
\DeclareMathOperator{\img}{img}
\DeclareMathOperator{\cov}{Cov}
\DeclareMathOperator{\dist}{dist}
\DeclareMathOperator{\irr}{Irr}
\DeclareMathOperator{\image}{Im}
\DeclareMathOperator{\pd}{\partial}
\DeclareMathOperator{\epi}{epi}
\DeclareMathOperator{\Argmin}{Argmin}
\DeclareMathOperator{\dom}{dom}
\DeclareMathOperator{\proj}{proj}
\DeclareMathOperator{\ctg}{ctg}
\DeclareMathOperator{\supp}{supp}
\DeclareMathOperator{\argmin}{argmin}
\DeclareMathOperator{\mult}{mult}
\DeclareMathOperator{\ch}{ch}
\DeclareMathOperator{\sh}{sh}
\DeclareMathOperator{\rang}{rang}
\DeclareMathOperator{\diam}{diam}
\DeclareMathOperator{\Epigraphe}{Epigraphe}




\usepackage{xcolor}
\everymath{\color{blue}}
%\everymath{\color[rgb]{0,1,1}}
%\pagecolor[rgb]{0,0,0.5}


\newcommand*{\pdtest}[3][]{\ensuremath{\frac{\partial^{#1} #2}{\partial #3}}}

\newcommand*{\deffunc}[6][]{\ensuremath{
\begin{array}{rcl}
#2 : #3 &\rightarrow& #4\\
#5 &\mapsto& #6
\end{array}
}}

\newcommand{\eqcolon}{\mathrel{\resizebox{\widthof{$\mathord{=}$}}{\height}{ $\!\!=\!\!\resizebox{1.2\width}{0.8\height}{\raisebox{0.23ex}{$\mathop{:}$}}\!\!$ }}}
\newcommand{\coloneq}{\mathrel{\resizebox{\widthof{$\mathord{=}$}}{\height}{ $\!\!\resizebox{1.2\width}{0.8\height}{\raisebox{0.23ex}{$\mathop{:}$}}\!\!=\!\!$ }}}
\newcommand{\eqcolonl}{\ensuremath{\mathrel{=\!\!\mathop{:}}}}
\newcommand{\coloneql}{\ensuremath{\mathrel{\mathop{:} \!\! =}}}
\newcommand{\vc}[1]{% inline column vector
  \left(\begin{smallmatrix}#1\end{smallmatrix}\right)%
}
\newcommand{\vr}[1]{% inline row vector
  \begin{smallmatrix}(\,#1\,)\end{smallmatrix}%
}
\makeatletter
\newcommand*{\defeq}{\ =\mathrel{\rlap{%
                     \raisebox{0.3ex}{$\m@th\cdot$}}%
                     \raisebox{-0.3ex}{$\m@th\cdot$}}%
                     }
\makeatother

\newcommand{\mathcircle}[1]{% inline row vector
 \overset{\circ}{#1}
}
\newcommand{\ulim}{% low limit
 \underline{\lim}
}
\newcommand{\ssi}{% iff
\iff
}
\newcommand{\ps}[2]{
\expval{#1 | #2}
}
\newcommand{\df}[1]{
\mqty{#1}
}
\newcommand{\n}[1]{
\norm{#1}
}
\newcommand{\sys}[1]{
\left\{\smqty{#1}\right.
}


\newcommand{\eqdef}{\ensuremath{\overset{\text{def}}=}}


\def\Circlearrowright{\ensuremath{%
  \rotatebox[origin=c]{230}{$\circlearrowright$}}}

\newcommand\ct[1]{\text{\rmfamily\upshape #1}}
\newcommand\question[1]{ {\color{red} ...!? \small #1}}
\newcommand\caz[1]{\left\{\begin{array} #1 \end{array}\right.}
\newcommand\const{\text{\rmfamily\upshape const}}
\newcommand\toP{ \overset{\pro}{\to}}
\newcommand\toPP{ \overset{\text{PP}}{\to}}
\newcommand{\oeq}{\mathrel{\text{\textcircled{$=$}}}}





\usepackage{xcolor}
% \usepackage[normalem]{ulem}
\usepackage{lipsum}
\makeatletter
% \newcommand\colorwave[1][blue]{\bgroup \markoverwith{\lower3.5\p@\hbox{\sixly \textcolor{#1}{\char58}}}\ULon}
%\font\sixly=lasy6 % does not re-load if already loaded, so no memory problem.

\newmdtheoremenv[
linewidth= 1pt,linecolor= blue,%
leftmargin=20,rightmargin=20,innertopmargin=0pt, innerrightmargin=40,%
tikzsetting = { draw=lightgray, line width = 0.3pt,dashed,%
dash pattern = on 15pt off 3pt},%
splittopskip=\topskip,skipbelow=\baselineskip,%
skipabove=\baselineskip,ntheorem,roundcorner=0pt,
% backgroundcolor=pagebg,font=\color{orange}\sffamily, fontcolor=white
]{examplebox}{Exemple}[section]



\newcommand\R{\mathbb{R}}
\newcommand\Z{\mathbb{Z}}
\newcommand\N{\mathbb{N}}
\newcommand\E{\mathbb{E}}
\newcommand\F{\mathcal{F}}
\newcommand\cH{\mathcal{H}}
\newcommand\V{\mathbb{V}}
\newcommand\dmo{ ^{-1} }
\newcommand\kapa{\kappa}
\newcommand\im{Im}
\newcommand\hs{\mathcal{H}}





\usepackage{soul}

\makeatletter
\newcommand*{\whiten}[1]{\llap{\textcolor{white}{{\the\SOUL@token}}\hspace{#1pt}}}
\DeclareRobustCommand*\myul{%
    \def\SOUL@everyspace{\underline{\space}\kern\z@}%
    \def\SOUL@everytoken{%
     \setbox0=\hbox{\the\SOUL@token}%
     \ifdim\dp0>\z@
        \raisebox{\dp0}{\underline{\phantom{\the\SOUL@token}}}%
        \whiten{1}\whiten{0}%
        \whiten{-1}\whiten{-2}%
        \llap{\the\SOUL@token}%
     \else
        \underline{\the\SOUL@token}%
     \fi}%
\SOUL@}
\makeatother

\newcommand*{\demp}{\fontfamily{lmtt}\selectfont}

\DeclareTextFontCommand{\textdemp}{\demp}

\begin{document}

\ifcomment
Multiline
comment
\fi
\ifcomment
\myul{Typesetting test}
% \color[rgb]{1,1,1}
$∑_i^n≠ 60º±∞π∆¬≈√j∫h≤≥µ$

$\CR \R\pro\ind\pro\gS\pro
\mqty[a&b\\c&d]$
$\pro\mathbb{P}$
$\dd{x}$

  \[
    \alpha(x)=\left\{
                \begin{array}{ll}
                  x\\
                  \frac{1}{1+e^{-kx}}\\
                  \frac{e^x-e^{-x}}{e^x+e^{-x}}
                \end{array}
              \right.
  \]

  $\expval{x}$
  
  $\chi_\rho(ghg\dmo)=\Tr(\rho_{ghg\dmo})=\Tr(\rho_g\circ\rho_h\circ\rho\dmo_g)=\Tr(\rho_h)\overset{\mbox{\scalebox{0.5}{$\Tr(AB)=\Tr(BA)$}}}{=}\chi_\rho(h)$
  	$\mathop{\oplus}_{\substack{x\in X}}$

$\mat(\rho_g)=(a_{ij}(g))_{\scriptsize \substack{1\leq i\leq d \\ 1\leq j\leq d}}$ et $\mat(\rho'_g)=(a'_{ij}(g))_{\scriptsize \substack{1\leq i'\leq d' \\ 1\leq j'\leq d'}}$



\[\int_a^b{\mathbb{R}^2}g(u, v)\dd{P_{XY}}(u, v)=\iint g(u,v) f_{XY}(u, v)\dd \lambda(u) \dd \lambda(v)\]
$$\lim_{x\to\infty} f(x)$$	
$$\iiiint_V \mu(t,u,v,w) \,dt\,du\,dv\,dw$$
$$\sum_{n=1}^{\infty} 2^{-n} = 1$$	
\begin{definition}
	Si $X$ et $Y$ sont 2 v.a. ou definit la \textsc{Covariance} entre $X$ et $Y$ comme
	$\cov(X,Y)\overset{\text{def}}{=}\E\left[(X-\E(X))(Y-\E(Y))\right]=\E(XY)-\E(X)\E(Y)$.
\end{definition}
\fi
\pagebreak

% \tableofcontents

% insert your code here
%\input{./algebra/main.tex}
%\input{./geometrie-differentielle/main.tex}
%\input{./probabilite/main.tex}
%\input{./analyse-fonctionnelle/main.tex}
% \input{./Analyse-convexe-et-dualite-en-optimisation/main.tex}
%\input{./tikz/main.tex}
%\input{./Theorie-du-distributions/main.tex}
%\input{./optimisation/mine.tex}
 \input{./modelisation/main.tex}

% yves.aubry@univ-tln.fr : algebra

\end{document}

%% !TEX encoding = UTF-8 Unicode
% !TEX TS-program = xelatex

\documentclass[french]{report}

%\usepackage[utf8]{inputenc}
%\usepackage[T1]{fontenc}
\usepackage{babel}


\newif\ifcomment
%\commenttrue # Show comments

\usepackage{physics}
\usepackage{amssymb}


\usepackage{amsthm}
% \usepackage{thmtools}
\usepackage{mathtools}
\usepackage{amsfonts}

\usepackage{color}

\usepackage{tikz}

\usepackage{geometry}
\geometry{a5paper, margin=0.1in, right=1cm}

\usepackage{dsfont}

\usepackage{graphicx}
\graphicspath{ {images/} }

\usepackage{faktor}

\usepackage{IEEEtrantools}
\usepackage{enumerate}   
\usepackage[PostScript=dvips]{"/Users/aware/Documents/Courses/diagrams"}


\newtheorem{theorem}{Théorème}[section]
\renewcommand{\thetheorem}{\arabic{theorem}}
\newtheorem{lemme}{Lemme}[section]
\renewcommand{\thelemme}{\arabic{lemme}}
\newtheorem{proposition}{Proposition}[section]
\renewcommand{\theproposition}{\arabic{proposition}}
\newtheorem{notations}{Notations}[section]
\newtheorem{problem}{Problème}[section]
\newtheorem{corollary}{Corollaire}[theorem]
\renewcommand{\thecorollary}{\arabic{corollary}}
\newtheorem{property}{Propriété}[section]
\newtheorem{objective}{Objectif}[section]

\theoremstyle{definition}
\newtheorem{definition}{Définition}[section]
\renewcommand{\thedefinition}{\arabic{definition}}
\newtheorem{exercise}{Exercice}[chapter]
\renewcommand{\theexercise}{\arabic{exercise}}
\newtheorem{example}{Exemple}[chapter]
\renewcommand{\theexample}{\arabic{example}}
\newtheorem*{solution}{Solution}
\newtheorem*{application}{Application}
\newtheorem*{notation}{Notation}
\newtheorem*{vocabulary}{Vocabulaire}
\newtheorem*{properties}{Propriétés}



\theoremstyle{remark}
\newtheorem*{remark}{Remarque}
\newtheorem*{rappel}{Rappel}


\usepackage{etoolbox}
\AtBeginEnvironment{exercise}{\small}
\AtBeginEnvironment{example}{\small}

\usepackage{cases}
\usepackage[red]{mypack}

\usepackage[framemethod=TikZ]{mdframed}

\definecolor{bg}{rgb}{0.4,0.25,0.95}
\definecolor{pagebg}{rgb}{0,0,0.5}
\surroundwithmdframed[
   topline=false,
   rightline=false,
   bottomline=false,
   leftmargin=\parindent,
   skipabove=8pt,
   skipbelow=8pt,
   linecolor=blue,
   innerbottommargin=10pt,
   % backgroundcolor=bg,font=\color{orange}\sffamily, fontcolor=white
]{definition}

\usepackage{empheq}
\usepackage[most]{tcolorbox}

\newtcbox{\mymath}[1][]{%
    nobeforeafter, math upper, tcbox raise base,
    enhanced, colframe=blue!30!black,
    colback=red!10, boxrule=1pt,
    #1}

\usepackage{unixode}


\DeclareMathOperator{\ord}{ord}
\DeclareMathOperator{\orb}{orb}
\DeclareMathOperator{\stab}{stab}
\DeclareMathOperator{\Stab}{stab}
\DeclareMathOperator{\ppcm}{ppcm}
\DeclareMathOperator{\conj}{Conj}
\DeclareMathOperator{\End}{End}
\DeclareMathOperator{\rot}{rot}
\DeclareMathOperator{\trs}{trace}
\DeclareMathOperator{\Ind}{Ind}
\DeclareMathOperator{\mat}{Mat}
\DeclareMathOperator{\id}{Id}
\DeclareMathOperator{\vect}{vect}
\DeclareMathOperator{\img}{img}
\DeclareMathOperator{\cov}{Cov}
\DeclareMathOperator{\dist}{dist}
\DeclareMathOperator{\irr}{Irr}
\DeclareMathOperator{\image}{Im}
\DeclareMathOperator{\pd}{\partial}
\DeclareMathOperator{\epi}{epi}
\DeclareMathOperator{\Argmin}{Argmin}
\DeclareMathOperator{\dom}{dom}
\DeclareMathOperator{\proj}{proj}
\DeclareMathOperator{\ctg}{ctg}
\DeclareMathOperator{\supp}{supp}
\DeclareMathOperator{\argmin}{argmin}
\DeclareMathOperator{\mult}{mult}
\DeclareMathOperator{\ch}{ch}
\DeclareMathOperator{\sh}{sh}
\DeclareMathOperator{\rang}{rang}
\DeclareMathOperator{\diam}{diam}
\DeclareMathOperator{\Epigraphe}{Epigraphe}




\usepackage{xcolor}
\everymath{\color{blue}}
%\everymath{\color[rgb]{0,1,1}}
%\pagecolor[rgb]{0,0,0.5}


\newcommand*{\pdtest}[3][]{\ensuremath{\frac{\partial^{#1} #2}{\partial #3}}}

\newcommand*{\deffunc}[6][]{\ensuremath{
\begin{array}{rcl}
#2 : #3 &\rightarrow& #4\\
#5 &\mapsto& #6
\end{array}
}}

\newcommand{\eqcolon}{\mathrel{\resizebox{\widthof{$\mathord{=}$}}{\height}{ $\!\!=\!\!\resizebox{1.2\width}{0.8\height}{\raisebox{0.23ex}{$\mathop{:}$}}\!\!$ }}}
\newcommand{\coloneq}{\mathrel{\resizebox{\widthof{$\mathord{=}$}}{\height}{ $\!\!\resizebox{1.2\width}{0.8\height}{\raisebox{0.23ex}{$\mathop{:}$}}\!\!=\!\!$ }}}
\newcommand{\eqcolonl}{\ensuremath{\mathrel{=\!\!\mathop{:}}}}
\newcommand{\coloneql}{\ensuremath{\mathrel{\mathop{:} \!\! =}}}
\newcommand{\vc}[1]{% inline column vector
  \left(\begin{smallmatrix}#1\end{smallmatrix}\right)%
}
\newcommand{\vr}[1]{% inline row vector
  \begin{smallmatrix}(\,#1\,)\end{smallmatrix}%
}
\makeatletter
\newcommand*{\defeq}{\ =\mathrel{\rlap{%
                     \raisebox{0.3ex}{$\m@th\cdot$}}%
                     \raisebox{-0.3ex}{$\m@th\cdot$}}%
                     }
\makeatother

\newcommand{\mathcircle}[1]{% inline row vector
 \overset{\circ}{#1}
}
\newcommand{\ulim}{% low limit
 \underline{\lim}
}
\newcommand{\ssi}{% iff
\iff
}
\newcommand{\ps}[2]{
\expval{#1 | #2}
}
\newcommand{\df}[1]{
\mqty{#1}
}
\newcommand{\n}[1]{
\norm{#1}
}
\newcommand{\sys}[1]{
\left\{\smqty{#1}\right.
}


\newcommand{\eqdef}{\ensuremath{\overset{\text{def}}=}}


\def\Circlearrowright{\ensuremath{%
  \rotatebox[origin=c]{230}{$\circlearrowright$}}}

\newcommand\ct[1]{\text{\rmfamily\upshape #1}}
\newcommand\question[1]{ {\color{red} ...!? \small #1}}
\newcommand\caz[1]{\left\{\begin{array} #1 \end{array}\right.}
\newcommand\const{\text{\rmfamily\upshape const}}
\newcommand\toP{ \overset{\pro}{\to}}
\newcommand\toPP{ \overset{\text{PP}}{\to}}
\newcommand{\oeq}{\mathrel{\text{\textcircled{$=$}}}}





\usepackage{xcolor}
% \usepackage[normalem]{ulem}
\usepackage{lipsum}
\makeatletter
% \newcommand\colorwave[1][blue]{\bgroup \markoverwith{\lower3.5\p@\hbox{\sixly \textcolor{#1}{\char58}}}\ULon}
%\font\sixly=lasy6 % does not re-load if already loaded, so no memory problem.

\newmdtheoremenv[
linewidth= 1pt,linecolor= blue,%
leftmargin=20,rightmargin=20,innertopmargin=0pt, innerrightmargin=40,%
tikzsetting = { draw=lightgray, line width = 0.3pt,dashed,%
dash pattern = on 15pt off 3pt},%
splittopskip=\topskip,skipbelow=\baselineskip,%
skipabove=\baselineskip,ntheorem,roundcorner=0pt,
% backgroundcolor=pagebg,font=\color{orange}\sffamily, fontcolor=white
]{examplebox}{Exemple}[section]



\newcommand\R{\mathbb{R}}
\newcommand\Z{\mathbb{Z}}
\newcommand\N{\mathbb{N}}
\newcommand\E{\mathbb{E}}
\newcommand\F{\mathcal{F}}
\newcommand\cH{\mathcal{H}}
\newcommand\V{\mathbb{V}}
\newcommand\dmo{ ^{-1} }
\newcommand\kapa{\kappa}
\newcommand\im{Im}
\newcommand\hs{\mathcal{H}}





\usepackage{soul}

\makeatletter
\newcommand*{\whiten}[1]{\llap{\textcolor{white}{{\the\SOUL@token}}\hspace{#1pt}}}
\DeclareRobustCommand*\myul{%
    \def\SOUL@everyspace{\underline{\space}\kern\z@}%
    \def\SOUL@everytoken{%
     \setbox0=\hbox{\the\SOUL@token}%
     \ifdim\dp0>\z@
        \raisebox{\dp0}{\underline{\phantom{\the\SOUL@token}}}%
        \whiten{1}\whiten{0}%
        \whiten{-1}\whiten{-2}%
        \llap{\the\SOUL@token}%
     \else
        \underline{\the\SOUL@token}%
     \fi}%
\SOUL@}
\makeatother

\newcommand*{\demp}{\fontfamily{lmtt}\selectfont}

\DeclareTextFontCommand{\textdemp}{\demp}

\begin{document}

\ifcomment
Multiline
comment
\fi
\ifcomment
\myul{Typesetting test}
% \color[rgb]{1,1,1}
$∑_i^n≠ 60º±∞π∆¬≈√j∫h≤≥µ$

$\CR \R\pro\ind\pro\gS\pro
\mqty[a&b\\c&d]$
$\pro\mathbb{P}$
$\dd{x}$

  \[
    \alpha(x)=\left\{
                \begin{array}{ll}
                  x\\
                  \frac{1}{1+e^{-kx}}\\
                  \frac{e^x-e^{-x}}{e^x+e^{-x}}
                \end{array}
              \right.
  \]

  $\expval{x}$
  
  $\chi_\rho(ghg\dmo)=\Tr(\rho_{ghg\dmo})=\Tr(\rho_g\circ\rho_h\circ\rho\dmo_g)=\Tr(\rho_h)\overset{\mbox{\scalebox{0.5}{$\Tr(AB)=\Tr(BA)$}}}{=}\chi_\rho(h)$
  	$\mathop{\oplus}_{\substack{x\in X}}$

$\mat(\rho_g)=(a_{ij}(g))_{\scriptsize \substack{1\leq i\leq d \\ 1\leq j\leq d}}$ et $\mat(\rho'_g)=(a'_{ij}(g))_{\scriptsize \substack{1\leq i'\leq d' \\ 1\leq j'\leq d'}}$



\[\int_a^b{\mathbb{R}^2}g(u, v)\dd{P_{XY}}(u, v)=\iint g(u,v) f_{XY}(u, v)\dd \lambda(u) \dd \lambda(v)\]
$$\lim_{x\to\infty} f(x)$$	
$$\iiiint_V \mu(t,u,v,w) \,dt\,du\,dv\,dw$$
$$\sum_{n=1}^{\infty} 2^{-n} = 1$$	
\begin{definition}
	Si $X$ et $Y$ sont 2 v.a. ou definit la \textsc{Covariance} entre $X$ et $Y$ comme
	$\cov(X,Y)\overset{\text{def}}{=}\E\left[(X-\E(X))(Y-\E(Y))\right]=\E(XY)-\E(X)\E(Y)$.
\end{definition}
\fi
\pagebreak

% \tableofcontents

% insert your code here
%\input{./algebra/main.tex}
%\input{./geometrie-differentielle/main.tex}
%\input{./probabilite/main.tex}
%\input{./analyse-fonctionnelle/main.tex}
% \input{./Analyse-convexe-et-dualite-en-optimisation/main.tex}
%\input{./tikz/main.tex}
%\input{./Theorie-du-distributions/main.tex}
%\input{./optimisation/mine.tex}
 \input{./modelisation/main.tex}

% yves.aubry@univ-tln.fr : algebra

\end{document}

% % !TEX encoding = UTF-8 Unicode
% !TEX TS-program = xelatex

\documentclass[french]{report}

%\usepackage[utf8]{inputenc}
%\usepackage[T1]{fontenc}
\usepackage{babel}


\newif\ifcomment
%\commenttrue # Show comments

\usepackage{physics}
\usepackage{amssymb}


\usepackage{amsthm}
% \usepackage{thmtools}
\usepackage{mathtools}
\usepackage{amsfonts}

\usepackage{color}

\usepackage{tikz}

\usepackage{geometry}
\geometry{a5paper, margin=0.1in, right=1cm}

\usepackage{dsfont}

\usepackage{graphicx}
\graphicspath{ {images/} }

\usepackage{faktor}

\usepackage{IEEEtrantools}
\usepackage{enumerate}   
\usepackage[PostScript=dvips]{"/Users/aware/Documents/Courses/diagrams"}


\newtheorem{theorem}{Théorème}[section]
\renewcommand{\thetheorem}{\arabic{theorem}}
\newtheorem{lemme}{Lemme}[section]
\renewcommand{\thelemme}{\arabic{lemme}}
\newtheorem{proposition}{Proposition}[section]
\renewcommand{\theproposition}{\arabic{proposition}}
\newtheorem{notations}{Notations}[section]
\newtheorem{problem}{Problème}[section]
\newtheorem{corollary}{Corollaire}[theorem]
\renewcommand{\thecorollary}{\arabic{corollary}}
\newtheorem{property}{Propriété}[section]
\newtheorem{objective}{Objectif}[section]

\theoremstyle{definition}
\newtheorem{definition}{Définition}[section]
\renewcommand{\thedefinition}{\arabic{definition}}
\newtheorem{exercise}{Exercice}[chapter]
\renewcommand{\theexercise}{\arabic{exercise}}
\newtheorem{example}{Exemple}[chapter]
\renewcommand{\theexample}{\arabic{example}}
\newtheorem*{solution}{Solution}
\newtheorem*{application}{Application}
\newtheorem*{notation}{Notation}
\newtheorem*{vocabulary}{Vocabulaire}
\newtheorem*{properties}{Propriétés}



\theoremstyle{remark}
\newtheorem*{remark}{Remarque}
\newtheorem*{rappel}{Rappel}


\usepackage{etoolbox}
\AtBeginEnvironment{exercise}{\small}
\AtBeginEnvironment{example}{\small}

\usepackage{cases}
\usepackage[red]{mypack}

\usepackage[framemethod=TikZ]{mdframed}

\definecolor{bg}{rgb}{0.4,0.25,0.95}
\definecolor{pagebg}{rgb}{0,0,0.5}
\surroundwithmdframed[
   topline=false,
   rightline=false,
   bottomline=false,
   leftmargin=\parindent,
   skipabove=8pt,
   skipbelow=8pt,
   linecolor=blue,
   innerbottommargin=10pt,
   % backgroundcolor=bg,font=\color{orange}\sffamily, fontcolor=white
]{definition}

\usepackage{empheq}
\usepackage[most]{tcolorbox}

\newtcbox{\mymath}[1][]{%
    nobeforeafter, math upper, tcbox raise base,
    enhanced, colframe=blue!30!black,
    colback=red!10, boxrule=1pt,
    #1}

\usepackage{unixode}


\DeclareMathOperator{\ord}{ord}
\DeclareMathOperator{\orb}{orb}
\DeclareMathOperator{\stab}{stab}
\DeclareMathOperator{\Stab}{stab}
\DeclareMathOperator{\ppcm}{ppcm}
\DeclareMathOperator{\conj}{Conj}
\DeclareMathOperator{\End}{End}
\DeclareMathOperator{\rot}{rot}
\DeclareMathOperator{\trs}{trace}
\DeclareMathOperator{\Ind}{Ind}
\DeclareMathOperator{\mat}{Mat}
\DeclareMathOperator{\id}{Id}
\DeclareMathOperator{\vect}{vect}
\DeclareMathOperator{\img}{img}
\DeclareMathOperator{\cov}{Cov}
\DeclareMathOperator{\dist}{dist}
\DeclareMathOperator{\irr}{Irr}
\DeclareMathOperator{\image}{Im}
\DeclareMathOperator{\pd}{\partial}
\DeclareMathOperator{\epi}{epi}
\DeclareMathOperator{\Argmin}{Argmin}
\DeclareMathOperator{\dom}{dom}
\DeclareMathOperator{\proj}{proj}
\DeclareMathOperator{\ctg}{ctg}
\DeclareMathOperator{\supp}{supp}
\DeclareMathOperator{\argmin}{argmin}
\DeclareMathOperator{\mult}{mult}
\DeclareMathOperator{\ch}{ch}
\DeclareMathOperator{\sh}{sh}
\DeclareMathOperator{\rang}{rang}
\DeclareMathOperator{\diam}{diam}
\DeclareMathOperator{\Epigraphe}{Epigraphe}




\usepackage{xcolor}
\everymath{\color{blue}}
%\everymath{\color[rgb]{0,1,1}}
%\pagecolor[rgb]{0,0,0.5}


\newcommand*{\pdtest}[3][]{\ensuremath{\frac{\partial^{#1} #2}{\partial #3}}}

\newcommand*{\deffunc}[6][]{\ensuremath{
\begin{array}{rcl}
#2 : #3 &\rightarrow& #4\\
#5 &\mapsto& #6
\end{array}
}}

\newcommand{\eqcolon}{\mathrel{\resizebox{\widthof{$\mathord{=}$}}{\height}{ $\!\!=\!\!\resizebox{1.2\width}{0.8\height}{\raisebox{0.23ex}{$\mathop{:}$}}\!\!$ }}}
\newcommand{\coloneq}{\mathrel{\resizebox{\widthof{$\mathord{=}$}}{\height}{ $\!\!\resizebox{1.2\width}{0.8\height}{\raisebox{0.23ex}{$\mathop{:}$}}\!\!=\!\!$ }}}
\newcommand{\eqcolonl}{\ensuremath{\mathrel{=\!\!\mathop{:}}}}
\newcommand{\coloneql}{\ensuremath{\mathrel{\mathop{:} \!\! =}}}
\newcommand{\vc}[1]{% inline column vector
  \left(\begin{smallmatrix}#1\end{smallmatrix}\right)%
}
\newcommand{\vr}[1]{% inline row vector
  \begin{smallmatrix}(\,#1\,)\end{smallmatrix}%
}
\makeatletter
\newcommand*{\defeq}{\ =\mathrel{\rlap{%
                     \raisebox{0.3ex}{$\m@th\cdot$}}%
                     \raisebox{-0.3ex}{$\m@th\cdot$}}%
                     }
\makeatother

\newcommand{\mathcircle}[1]{% inline row vector
 \overset{\circ}{#1}
}
\newcommand{\ulim}{% low limit
 \underline{\lim}
}
\newcommand{\ssi}{% iff
\iff
}
\newcommand{\ps}[2]{
\expval{#1 | #2}
}
\newcommand{\df}[1]{
\mqty{#1}
}
\newcommand{\n}[1]{
\norm{#1}
}
\newcommand{\sys}[1]{
\left\{\smqty{#1}\right.
}


\newcommand{\eqdef}{\ensuremath{\overset{\text{def}}=}}


\def\Circlearrowright{\ensuremath{%
  \rotatebox[origin=c]{230}{$\circlearrowright$}}}

\newcommand\ct[1]{\text{\rmfamily\upshape #1}}
\newcommand\question[1]{ {\color{red} ...!? \small #1}}
\newcommand\caz[1]{\left\{\begin{array} #1 \end{array}\right.}
\newcommand\const{\text{\rmfamily\upshape const}}
\newcommand\toP{ \overset{\pro}{\to}}
\newcommand\toPP{ \overset{\text{PP}}{\to}}
\newcommand{\oeq}{\mathrel{\text{\textcircled{$=$}}}}





\usepackage{xcolor}
% \usepackage[normalem]{ulem}
\usepackage{lipsum}
\makeatletter
% \newcommand\colorwave[1][blue]{\bgroup \markoverwith{\lower3.5\p@\hbox{\sixly \textcolor{#1}{\char58}}}\ULon}
%\font\sixly=lasy6 % does not re-load if already loaded, so no memory problem.

\newmdtheoremenv[
linewidth= 1pt,linecolor= blue,%
leftmargin=20,rightmargin=20,innertopmargin=0pt, innerrightmargin=40,%
tikzsetting = { draw=lightgray, line width = 0.3pt,dashed,%
dash pattern = on 15pt off 3pt},%
splittopskip=\topskip,skipbelow=\baselineskip,%
skipabove=\baselineskip,ntheorem,roundcorner=0pt,
% backgroundcolor=pagebg,font=\color{orange}\sffamily, fontcolor=white
]{examplebox}{Exemple}[section]



\newcommand\R{\mathbb{R}}
\newcommand\Z{\mathbb{Z}}
\newcommand\N{\mathbb{N}}
\newcommand\E{\mathbb{E}}
\newcommand\F{\mathcal{F}}
\newcommand\cH{\mathcal{H}}
\newcommand\V{\mathbb{V}}
\newcommand\dmo{ ^{-1} }
\newcommand\kapa{\kappa}
\newcommand\im{Im}
\newcommand\hs{\mathcal{H}}





\usepackage{soul}

\makeatletter
\newcommand*{\whiten}[1]{\llap{\textcolor{white}{{\the\SOUL@token}}\hspace{#1pt}}}
\DeclareRobustCommand*\myul{%
    \def\SOUL@everyspace{\underline{\space}\kern\z@}%
    \def\SOUL@everytoken{%
     \setbox0=\hbox{\the\SOUL@token}%
     \ifdim\dp0>\z@
        \raisebox{\dp0}{\underline{\phantom{\the\SOUL@token}}}%
        \whiten{1}\whiten{0}%
        \whiten{-1}\whiten{-2}%
        \llap{\the\SOUL@token}%
     \else
        \underline{\the\SOUL@token}%
     \fi}%
\SOUL@}
\makeatother

\newcommand*{\demp}{\fontfamily{lmtt}\selectfont}

\DeclareTextFontCommand{\textdemp}{\demp}

\begin{document}

\ifcomment
Multiline
comment
\fi
\ifcomment
\myul{Typesetting test}
% \color[rgb]{1,1,1}
$∑_i^n≠ 60º±∞π∆¬≈√j∫h≤≥µ$

$\CR \R\pro\ind\pro\gS\pro
\mqty[a&b\\c&d]$
$\pro\mathbb{P}$
$\dd{x}$

  \[
    \alpha(x)=\left\{
                \begin{array}{ll}
                  x\\
                  \frac{1}{1+e^{-kx}}\\
                  \frac{e^x-e^{-x}}{e^x+e^{-x}}
                \end{array}
              \right.
  \]

  $\expval{x}$
  
  $\chi_\rho(ghg\dmo)=\Tr(\rho_{ghg\dmo})=\Tr(\rho_g\circ\rho_h\circ\rho\dmo_g)=\Tr(\rho_h)\overset{\mbox{\scalebox{0.5}{$\Tr(AB)=\Tr(BA)$}}}{=}\chi_\rho(h)$
  	$\mathop{\oplus}_{\substack{x\in X}}$

$\mat(\rho_g)=(a_{ij}(g))_{\scriptsize \substack{1\leq i\leq d \\ 1\leq j\leq d}}$ et $\mat(\rho'_g)=(a'_{ij}(g))_{\scriptsize \substack{1\leq i'\leq d' \\ 1\leq j'\leq d'}}$



\[\int_a^b{\mathbb{R}^2}g(u, v)\dd{P_{XY}}(u, v)=\iint g(u,v) f_{XY}(u, v)\dd \lambda(u) \dd \lambda(v)\]
$$\lim_{x\to\infty} f(x)$$	
$$\iiiint_V \mu(t,u,v,w) \,dt\,du\,dv\,dw$$
$$\sum_{n=1}^{\infty} 2^{-n} = 1$$	
\begin{definition}
	Si $X$ et $Y$ sont 2 v.a. ou definit la \textsc{Covariance} entre $X$ et $Y$ comme
	$\cov(X,Y)\overset{\text{def}}{=}\E\left[(X-\E(X))(Y-\E(Y))\right]=\E(XY)-\E(X)\E(Y)$.
\end{definition}
\fi
\pagebreak

% \tableofcontents

% insert your code here
%\input{./algebra/main.tex}
%\input{./geometrie-differentielle/main.tex}
%\input{./probabilite/main.tex}
%\input{./analyse-fonctionnelle/main.tex}
% \input{./Analyse-convexe-et-dualite-en-optimisation/main.tex}
%\input{./tikz/main.tex}
%\input{./Theorie-du-distributions/main.tex}
%\input{./optimisation/mine.tex}
 \input{./modelisation/main.tex}

% yves.aubry@univ-tln.fr : algebra

\end{document}

%% !TEX encoding = UTF-8 Unicode
% !TEX TS-program = xelatex

\documentclass[french]{report}

%\usepackage[utf8]{inputenc}
%\usepackage[T1]{fontenc}
\usepackage{babel}


\newif\ifcomment
%\commenttrue # Show comments

\usepackage{physics}
\usepackage{amssymb}


\usepackage{amsthm}
% \usepackage{thmtools}
\usepackage{mathtools}
\usepackage{amsfonts}

\usepackage{color}

\usepackage{tikz}

\usepackage{geometry}
\geometry{a5paper, margin=0.1in, right=1cm}

\usepackage{dsfont}

\usepackage{graphicx}
\graphicspath{ {images/} }

\usepackage{faktor}

\usepackage{IEEEtrantools}
\usepackage{enumerate}   
\usepackage[PostScript=dvips]{"/Users/aware/Documents/Courses/diagrams"}


\newtheorem{theorem}{Théorème}[section]
\renewcommand{\thetheorem}{\arabic{theorem}}
\newtheorem{lemme}{Lemme}[section]
\renewcommand{\thelemme}{\arabic{lemme}}
\newtheorem{proposition}{Proposition}[section]
\renewcommand{\theproposition}{\arabic{proposition}}
\newtheorem{notations}{Notations}[section]
\newtheorem{problem}{Problème}[section]
\newtheorem{corollary}{Corollaire}[theorem]
\renewcommand{\thecorollary}{\arabic{corollary}}
\newtheorem{property}{Propriété}[section]
\newtheorem{objective}{Objectif}[section]

\theoremstyle{definition}
\newtheorem{definition}{Définition}[section]
\renewcommand{\thedefinition}{\arabic{definition}}
\newtheorem{exercise}{Exercice}[chapter]
\renewcommand{\theexercise}{\arabic{exercise}}
\newtheorem{example}{Exemple}[chapter]
\renewcommand{\theexample}{\arabic{example}}
\newtheorem*{solution}{Solution}
\newtheorem*{application}{Application}
\newtheorem*{notation}{Notation}
\newtheorem*{vocabulary}{Vocabulaire}
\newtheorem*{properties}{Propriétés}



\theoremstyle{remark}
\newtheorem*{remark}{Remarque}
\newtheorem*{rappel}{Rappel}


\usepackage{etoolbox}
\AtBeginEnvironment{exercise}{\small}
\AtBeginEnvironment{example}{\small}

\usepackage{cases}
\usepackage[red]{mypack}

\usepackage[framemethod=TikZ]{mdframed}

\definecolor{bg}{rgb}{0.4,0.25,0.95}
\definecolor{pagebg}{rgb}{0,0,0.5}
\surroundwithmdframed[
   topline=false,
   rightline=false,
   bottomline=false,
   leftmargin=\parindent,
   skipabove=8pt,
   skipbelow=8pt,
   linecolor=blue,
   innerbottommargin=10pt,
   % backgroundcolor=bg,font=\color{orange}\sffamily, fontcolor=white
]{definition}

\usepackage{empheq}
\usepackage[most]{tcolorbox}

\newtcbox{\mymath}[1][]{%
    nobeforeafter, math upper, tcbox raise base,
    enhanced, colframe=blue!30!black,
    colback=red!10, boxrule=1pt,
    #1}

\usepackage{unixode}


\DeclareMathOperator{\ord}{ord}
\DeclareMathOperator{\orb}{orb}
\DeclareMathOperator{\stab}{stab}
\DeclareMathOperator{\Stab}{stab}
\DeclareMathOperator{\ppcm}{ppcm}
\DeclareMathOperator{\conj}{Conj}
\DeclareMathOperator{\End}{End}
\DeclareMathOperator{\rot}{rot}
\DeclareMathOperator{\trs}{trace}
\DeclareMathOperator{\Ind}{Ind}
\DeclareMathOperator{\mat}{Mat}
\DeclareMathOperator{\id}{Id}
\DeclareMathOperator{\vect}{vect}
\DeclareMathOperator{\img}{img}
\DeclareMathOperator{\cov}{Cov}
\DeclareMathOperator{\dist}{dist}
\DeclareMathOperator{\irr}{Irr}
\DeclareMathOperator{\image}{Im}
\DeclareMathOperator{\pd}{\partial}
\DeclareMathOperator{\epi}{epi}
\DeclareMathOperator{\Argmin}{Argmin}
\DeclareMathOperator{\dom}{dom}
\DeclareMathOperator{\proj}{proj}
\DeclareMathOperator{\ctg}{ctg}
\DeclareMathOperator{\supp}{supp}
\DeclareMathOperator{\argmin}{argmin}
\DeclareMathOperator{\mult}{mult}
\DeclareMathOperator{\ch}{ch}
\DeclareMathOperator{\sh}{sh}
\DeclareMathOperator{\rang}{rang}
\DeclareMathOperator{\diam}{diam}
\DeclareMathOperator{\Epigraphe}{Epigraphe}




\usepackage{xcolor}
\everymath{\color{blue}}
%\everymath{\color[rgb]{0,1,1}}
%\pagecolor[rgb]{0,0,0.5}


\newcommand*{\pdtest}[3][]{\ensuremath{\frac{\partial^{#1} #2}{\partial #3}}}

\newcommand*{\deffunc}[6][]{\ensuremath{
\begin{array}{rcl}
#2 : #3 &\rightarrow& #4\\
#5 &\mapsto& #6
\end{array}
}}

\newcommand{\eqcolon}{\mathrel{\resizebox{\widthof{$\mathord{=}$}}{\height}{ $\!\!=\!\!\resizebox{1.2\width}{0.8\height}{\raisebox{0.23ex}{$\mathop{:}$}}\!\!$ }}}
\newcommand{\coloneq}{\mathrel{\resizebox{\widthof{$\mathord{=}$}}{\height}{ $\!\!\resizebox{1.2\width}{0.8\height}{\raisebox{0.23ex}{$\mathop{:}$}}\!\!=\!\!$ }}}
\newcommand{\eqcolonl}{\ensuremath{\mathrel{=\!\!\mathop{:}}}}
\newcommand{\coloneql}{\ensuremath{\mathrel{\mathop{:} \!\! =}}}
\newcommand{\vc}[1]{% inline column vector
  \left(\begin{smallmatrix}#1\end{smallmatrix}\right)%
}
\newcommand{\vr}[1]{% inline row vector
  \begin{smallmatrix}(\,#1\,)\end{smallmatrix}%
}
\makeatletter
\newcommand*{\defeq}{\ =\mathrel{\rlap{%
                     \raisebox{0.3ex}{$\m@th\cdot$}}%
                     \raisebox{-0.3ex}{$\m@th\cdot$}}%
                     }
\makeatother

\newcommand{\mathcircle}[1]{% inline row vector
 \overset{\circ}{#1}
}
\newcommand{\ulim}{% low limit
 \underline{\lim}
}
\newcommand{\ssi}{% iff
\iff
}
\newcommand{\ps}[2]{
\expval{#1 | #2}
}
\newcommand{\df}[1]{
\mqty{#1}
}
\newcommand{\n}[1]{
\norm{#1}
}
\newcommand{\sys}[1]{
\left\{\smqty{#1}\right.
}


\newcommand{\eqdef}{\ensuremath{\overset{\text{def}}=}}


\def\Circlearrowright{\ensuremath{%
  \rotatebox[origin=c]{230}{$\circlearrowright$}}}

\newcommand\ct[1]{\text{\rmfamily\upshape #1}}
\newcommand\question[1]{ {\color{red} ...!? \small #1}}
\newcommand\caz[1]{\left\{\begin{array} #1 \end{array}\right.}
\newcommand\const{\text{\rmfamily\upshape const}}
\newcommand\toP{ \overset{\pro}{\to}}
\newcommand\toPP{ \overset{\text{PP}}{\to}}
\newcommand{\oeq}{\mathrel{\text{\textcircled{$=$}}}}





\usepackage{xcolor}
% \usepackage[normalem]{ulem}
\usepackage{lipsum}
\makeatletter
% \newcommand\colorwave[1][blue]{\bgroup \markoverwith{\lower3.5\p@\hbox{\sixly \textcolor{#1}{\char58}}}\ULon}
%\font\sixly=lasy6 % does not re-load if already loaded, so no memory problem.

\newmdtheoremenv[
linewidth= 1pt,linecolor= blue,%
leftmargin=20,rightmargin=20,innertopmargin=0pt, innerrightmargin=40,%
tikzsetting = { draw=lightgray, line width = 0.3pt,dashed,%
dash pattern = on 15pt off 3pt},%
splittopskip=\topskip,skipbelow=\baselineskip,%
skipabove=\baselineskip,ntheorem,roundcorner=0pt,
% backgroundcolor=pagebg,font=\color{orange}\sffamily, fontcolor=white
]{examplebox}{Exemple}[section]



\newcommand\R{\mathbb{R}}
\newcommand\Z{\mathbb{Z}}
\newcommand\N{\mathbb{N}}
\newcommand\E{\mathbb{E}}
\newcommand\F{\mathcal{F}}
\newcommand\cH{\mathcal{H}}
\newcommand\V{\mathbb{V}}
\newcommand\dmo{ ^{-1} }
\newcommand\kapa{\kappa}
\newcommand\im{Im}
\newcommand\hs{\mathcal{H}}





\usepackage{soul}

\makeatletter
\newcommand*{\whiten}[1]{\llap{\textcolor{white}{{\the\SOUL@token}}\hspace{#1pt}}}
\DeclareRobustCommand*\myul{%
    \def\SOUL@everyspace{\underline{\space}\kern\z@}%
    \def\SOUL@everytoken{%
     \setbox0=\hbox{\the\SOUL@token}%
     \ifdim\dp0>\z@
        \raisebox{\dp0}{\underline{\phantom{\the\SOUL@token}}}%
        \whiten{1}\whiten{0}%
        \whiten{-1}\whiten{-2}%
        \llap{\the\SOUL@token}%
     \else
        \underline{\the\SOUL@token}%
     \fi}%
\SOUL@}
\makeatother

\newcommand*{\demp}{\fontfamily{lmtt}\selectfont}

\DeclareTextFontCommand{\textdemp}{\demp}

\begin{document}

\ifcomment
Multiline
comment
\fi
\ifcomment
\myul{Typesetting test}
% \color[rgb]{1,1,1}
$∑_i^n≠ 60º±∞π∆¬≈√j∫h≤≥µ$

$\CR \R\pro\ind\pro\gS\pro
\mqty[a&b\\c&d]$
$\pro\mathbb{P}$
$\dd{x}$

  \[
    \alpha(x)=\left\{
                \begin{array}{ll}
                  x\\
                  \frac{1}{1+e^{-kx}}\\
                  \frac{e^x-e^{-x}}{e^x+e^{-x}}
                \end{array}
              \right.
  \]

  $\expval{x}$
  
  $\chi_\rho(ghg\dmo)=\Tr(\rho_{ghg\dmo})=\Tr(\rho_g\circ\rho_h\circ\rho\dmo_g)=\Tr(\rho_h)\overset{\mbox{\scalebox{0.5}{$\Tr(AB)=\Tr(BA)$}}}{=}\chi_\rho(h)$
  	$\mathop{\oplus}_{\substack{x\in X}}$

$\mat(\rho_g)=(a_{ij}(g))_{\scriptsize \substack{1\leq i\leq d \\ 1\leq j\leq d}}$ et $\mat(\rho'_g)=(a'_{ij}(g))_{\scriptsize \substack{1\leq i'\leq d' \\ 1\leq j'\leq d'}}$



\[\int_a^b{\mathbb{R}^2}g(u, v)\dd{P_{XY}}(u, v)=\iint g(u,v) f_{XY}(u, v)\dd \lambda(u) \dd \lambda(v)\]
$$\lim_{x\to\infty} f(x)$$	
$$\iiiint_V \mu(t,u,v,w) \,dt\,du\,dv\,dw$$
$$\sum_{n=1}^{\infty} 2^{-n} = 1$$	
\begin{definition}
	Si $X$ et $Y$ sont 2 v.a. ou definit la \textsc{Covariance} entre $X$ et $Y$ comme
	$\cov(X,Y)\overset{\text{def}}{=}\E\left[(X-\E(X))(Y-\E(Y))\right]=\E(XY)-\E(X)\E(Y)$.
\end{definition}
\fi
\pagebreak

% \tableofcontents

% insert your code here
%\input{./algebra/main.tex}
%\input{./geometrie-differentielle/main.tex}
%\input{./probabilite/main.tex}
%\input{./analyse-fonctionnelle/main.tex}
% \input{./Analyse-convexe-et-dualite-en-optimisation/main.tex}
%\input{./tikz/main.tex}
%\input{./Theorie-du-distributions/main.tex}
%\input{./optimisation/mine.tex}
 \input{./modelisation/main.tex}

% yves.aubry@univ-tln.fr : algebra

\end{document}

%% !TEX encoding = UTF-8 Unicode
% !TEX TS-program = xelatex

\documentclass[french]{report}

%\usepackage[utf8]{inputenc}
%\usepackage[T1]{fontenc}
\usepackage{babel}


\newif\ifcomment
%\commenttrue # Show comments

\usepackage{physics}
\usepackage{amssymb}


\usepackage{amsthm}
% \usepackage{thmtools}
\usepackage{mathtools}
\usepackage{amsfonts}

\usepackage{color}

\usepackage{tikz}

\usepackage{geometry}
\geometry{a5paper, margin=0.1in, right=1cm}

\usepackage{dsfont}

\usepackage{graphicx}
\graphicspath{ {images/} }

\usepackage{faktor}

\usepackage{IEEEtrantools}
\usepackage{enumerate}   
\usepackage[PostScript=dvips]{"/Users/aware/Documents/Courses/diagrams"}


\newtheorem{theorem}{Théorème}[section]
\renewcommand{\thetheorem}{\arabic{theorem}}
\newtheorem{lemme}{Lemme}[section]
\renewcommand{\thelemme}{\arabic{lemme}}
\newtheorem{proposition}{Proposition}[section]
\renewcommand{\theproposition}{\arabic{proposition}}
\newtheorem{notations}{Notations}[section]
\newtheorem{problem}{Problème}[section]
\newtheorem{corollary}{Corollaire}[theorem]
\renewcommand{\thecorollary}{\arabic{corollary}}
\newtheorem{property}{Propriété}[section]
\newtheorem{objective}{Objectif}[section]

\theoremstyle{definition}
\newtheorem{definition}{Définition}[section]
\renewcommand{\thedefinition}{\arabic{definition}}
\newtheorem{exercise}{Exercice}[chapter]
\renewcommand{\theexercise}{\arabic{exercise}}
\newtheorem{example}{Exemple}[chapter]
\renewcommand{\theexample}{\arabic{example}}
\newtheorem*{solution}{Solution}
\newtheorem*{application}{Application}
\newtheorem*{notation}{Notation}
\newtheorem*{vocabulary}{Vocabulaire}
\newtheorem*{properties}{Propriétés}



\theoremstyle{remark}
\newtheorem*{remark}{Remarque}
\newtheorem*{rappel}{Rappel}


\usepackage{etoolbox}
\AtBeginEnvironment{exercise}{\small}
\AtBeginEnvironment{example}{\small}

\usepackage{cases}
\usepackage[red]{mypack}

\usepackage[framemethod=TikZ]{mdframed}

\definecolor{bg}{rgb}{0.4,0.25,0.95}
\definecolor{pagebg}{rgb}{0,0,0.5}
\surroundwithmdframed[
   topline=false,
   rightline=false,
   bottomline=false,
   leftmargin=\parindent,
   skipabove=8pt,
   skipbelow=8pt,
   linecolor=blue,
   innerbottommargin=10pt,
   % backgroundcolor=bg,font=\color{orange}\sffamily, fontcolor=white
]{definition}

\usepackage{empheq}
\usepackage[most]{tcolorbox}

\newtcbox{\mymath}[1][]{%
    nobeforeafter, math upper, tcbox raise base,
    enhanced, colframe=blue!30!black,
    colback=red!10, boxrule=1pt,
    #1}

\usepackage{unixode}


\DeclareMathOperator{\ord}{ord}
\DeclareMathOperator{\orb}{orb}
\DeclareMathOperator{\stab}{stab}
\DeclareMathOperator{\Stab}{stab}
\DeclareMathOperator{\ppcm}{ppcm}
\DeclareMathOperator{\conj}{Conj}
\DeclareMathOperator{\End}{End}
\DeclareMathOperator{\rot}{rot}
\DeclareMathOperator{\trs}{trace}
\DeclareMathOperator{\Ind}{Ind}
\DeclareMathOperator{\mat}{Mat}
\DeclareMathOperator{\id}{Id}
\DeclareMathOperator{\vect}{vect}
\DeclareMathOperator{\img}{img}
\DeclareMathOperator{\cov}{Cov}
\DeclareMathOperator{\dist}{dist}
\DeclareMathOperator{\irr}{Irr}
\DeclareMathOperator{\image}{Im}
\DeclareMathOperator{\pd}{\partial}
\DeclareMathOperator{\epi}{epi}
\DeclareMathOperator{\Argmin}{Argmin}
\DeclareMathOperator{\dom}{dom}
\DeclareMathOperator{\proj}{proj}
\DeclareMathOperator{\ctg}{ctg}
\DeclareMathOperator{\supp}{supp}
\DeclareMathOperator{\argmin}{argmin}
\DeclareMathOperator{\mult}{mult}
\DeclareMathOperator{\ch}{ch}
\DeclareMathOperator{\sh}{sh}
\DeclareMathOperator{\rang}{rang}
\DeclareMathOperator{\diam}{diam}
\DeclareMathOperator{\Epigraphe}{Epigraphe}




\usepackage{xcolor}
\everymath{\color{blue}}
%\everymath{\color[rgb]{0,1,1}}
%\pagecolor[rgb]{0,0,0.5}


\newcommand*{\pdtest}[3][]{\ensuremath{\frac{\partial^{#1} #2}{\partial #3}}}

\newcommand*{\deffunc}[6][]{\ensuremath{
\begin{array}{rcl}
#2 : #3 &\rightarrow& #4\\
#5 &\mapsto& #6
\end{array}
}}

\newcommand{\eqcolon}{\mathrel{\resizebox{\widthof{$\mathord{=}$}}{\height}{ $\!\!=\!\!\resizebox{1.2\width}{0.8\height}{\raisebox{0.23ex}{$\mathop{:}$}}\!\!$ }}}
\newcommand{\coloneq}{\mathrel{\resizebox{\widthof{$\mathord{=}$}}{\height}{ $\!\!\resizebox{1.2\width}{0.8\height}{\raisebox{0.23ex}{$\mathop{:}$}}\!\!=\!\!$ }}}
\newcommand{\eqcolonl}{\ensuremath{\mathrel{=\!\!\mathop{:}}}}
\newcommand{\coloneql}{\ensuremath{\mathrel{\mathop{:} \!\! =}}}
\newcommand{\vc}[1]{% inline column vector
  \left(\begin{smallmatrix}#1\end{smallmatrix}\right)%
}
\newcommand{\vr}[1]{% inline row vector
  \begin{smallmatrix}(\,#1\,)\end{smallmatrix}%
}
\makeatletter
\newcommand*{\defeq}{\ =\mathrel{\rlap{%
                     \raisebox{0.3ex}{$\m@th\cdot$}}%
                     \raisebox{-0.3ex}{$\m@th\cdot$}}%
                     }
\makeatother

\newcommand{\mathcircle}[1]{% inline row vector
 \overset{\circ}{#1}
}
\newcommand{\ulim}{% low limit
 \underline{\lim}
}
\newcommand{\ssi}{% iff
\iff
}
\newcommand{\ps}[2]{
\expval{#1 | #2}
}
\newcommand{\df}[1]{
\mqty{#1}
}
\newcommand{\n}[1]{
\norm{#1}
}
\newcommand{\sys}[1]{
\left\{\smqty{#1}\right.
}


\newcommand{\eqdef}{\ensuremath{\overset{\text{def}}=}}


\def\Circlearrowright{\ensuremath{%
  \rotatebox[origin=c]{230}{$\circlearrowright$}}}

\newcommand\ct[1]{\text{\rmfamily\upshape #1}}
\newcommand\question[1]{ {\color{red} ...!? \small #1}}
\newcommand\caz[1]{\left\{\begin{array} #1 \end{array}\right.}
\newcommand\const{\text{\rmfamily\upshape const}}
\newcommand\toP{ \overset{\pro}{\to}}
\newcommand\toPP{ \overset{\text{PP}}{\to}}
\newcommand{\oeq}{\mathrel{\text{\textcircled{$=$}}}}





\usepackage{xcolor}
% \usepackage[normalem]{ulem}
\usepackage{lipsum}
\makeatletter
% \newcommand\colorwave[1][blue]{\bgroup \markoverwith{\lower3.5\p@\hbox{\sixly \textcolor{#1}{\char58}}}\ULon}
%\font\sixly=lasy6 % does not re-load if already loaded, so no memory problem.

\newmdtheoremenv[
linewidth= 1pt,linecolor= blue,%
leftmargin=20,rightmargin=20,innertopmargin=0pt, innerrightmargin=40,%
tikzsetting = { draw=lightgray, line width = 0.3pt,dashed,%
dash pattern = on 15pt off 3pt},%
splittopskip=\topskip,skipbelow=\baselineskip,%
skipabove=\baselineskip,ntheorem,roundcorner=0pt,
% backgroundcolor=pagebg,font=\color{orange}\sffamily, fontcolor=white
]{examplebox}{Exemple}[section]



\newcommand\R{\mathbb{R}}
\newcommand\Z{\mathbb{Z}}
\newcommand\N{\mathbb{N}}
\newcommand\E{\mathbb{E}}
\newcommand\F{\mathcal{F}}
\newcommand\cH{\mathcal{H}}
\newcommand\V{\mathbb{V}}
\newcommand\dmo{ ^{-1} }
\newcommand\kapa{\kappa}
\newcommand\im{Im}
\newcommand\hs{\mathcal{H}}





\usepackage{soul}

\makeatletter
\newcommand*{\whiten}[1]{\llap{\textcolor{white}{{\the\SOUL@token}}\hspace{#1pt}}}
\DeclareRobustCommand*\myul{%
    \def\SOUL@everyspace{\underline{\space}\kern\z@}%
    \def\SOUL@everytoken{%
     \setbox0=\hbox{\the\SOUL@token}%
     \ifdim\dp0>\z@
        \raisebox{\dp0}{\underline{\phantom{\the\SOUL@token}}}%
        \whiten{1}\whiten{0}%
        \whiten{-1}\whiten{-2}%
        \llap{\the\SOUL@token}%
     \else
        \underline{\the\SOUL@token}%
     \fi}%
\SOUL@}
\makeatother

\newcommand*{\demp}{\fontfamily{lmtt}\selectfont}

\DeclareTextFontCommand{\textdemp}{\demp}

\begin{document}

\ifcomment
Multiline
comment
\fi
\ifcomment
\myul{Typesetting test}
% \color[rgb]{1,1,1}
$∑_i^n≠ 60º±∞π∆¬≈√j∫h≤≥µ$

$\CR \R\pro\ind\pro\gS\pro
\mqty[a&b\\c&d]$
$\pro\mathbb{P}$
$\dd{x}$

  \[
    \alpha(x)=\left\{
                \begin{array}{ll}
                  x\\
                  \frac{1}{1+e^{-kx}}\\
                  \frac{e^x-e^{-x}}{e^x+e^{-x}}
                \end{array}
              \right.
  \]

  $\expval{x}$
  
  $\chi_\rho(ghg\dmo)=\Tr(\rho_{ghg\dmo})=\Tr(\rho_g\circ\rho_h\circ\rho\dmo_g)=\Tr(\rho_h)\overset{\mbox{\scalebox{0.5}{$\Tr(AB)=\Tr(BA)$}}}{=}\chi_\rho(h)$
  	$\mathop{\oplus}_{\substack{x\in X}}$

$\mat(\rho_g)=(a_{ij}(g))_{\scriptsize \substack{1\leq i\leq d \\ 1\leq j\leq d}}$ et $\mat(\rho'_g)=(a'_{ij}(g))_{\scriptsize \substack{1\leq i'\leq d' \\ 1\leq j'\leq d'}}$



\[\int_a^b{\mathbb{R}^2}g(u, v)\dd{P_{XY}}(u, v)=\iint g(u,v) f_{XY}(u, v)\dd \lambda(u) \dd \lambda(v)\]
$$\lim_{x\to\infty} f(x)$$	
$$\iiiint_V \mu(t,u,v,w) \,dt\,du\,dv\,dw$$
$$\sum_{n=1}^{\infty} 2^{-n} = 1$$	
\begin{definition}
	Si $X$ et $Y$ sont 2 v.a. ou definit la \textsc{Covariance} entre $X$ et $Y$ comme
	$\cov(X,Y)\overset{\text{def}}{=}\E\left[(X-\E(X))(Y-\E(Y))\right]=\E(XY)-\E(X)\E(Y)$.
\end{definition}
\fi
\pagebreak

% \tableofcontents

% insert your code here
%\input{./algebra/main.tex}
%\input{./geometrie-differentielle/main.tex}
%\input{./probabilite/main.tex}
%\input{./analyse-fonctionnelle/main.tex}
% \input{./Analyse-convexe-et-dualite-en-optimisation/main.tex}
%\input{./tikz/main.tex}
%\input{./Theorie-du-distributions/main.tex}
%\input{./optimisation/mine.tex}
 \input{./modelisation/main.tex}

% yves.aubry@univ-tln.fr : algebra

\end{document}

%\input{./optimisation/mine.tex}
 % !TEX encoding = UTF-8 Unicode
% !TEX TS-program = xelatex

\documentclass[french]{report}

%\usepackage[utf8]{inputenc}
%\usepackage[T1]{fontenc}
\usepackage{babel}


\newif\ifcomment
%\commenttrue # Show comments

\usepackage{physics}
\usepackage{amssymb}


\usepackage{amsthm}
% \usepackage{thmtools}
\usepackage{mathtools}
\usepackage{amsfonts}

\usepackage{color}

\usepackage{tikz}

\usepackage{geometry}
\geometry{a5paper, margin=0.1in, right=1cm}

\usepackage{dsfont}

\usepackage{graphicx}
\graphicspath{ {images/} }

\usepackage{faktor}

\usepackage{IEEEtrantools}
\usepackage{enumerate}   
\usepackage[PostScript=dvips]{"/Users/aware/Documents/Courses/diagrams"}


\newtheorem{theorem}{Théorème}[section]
\renewcommand{\thetheorem}{\arabic{theorem}}
\newtheorem{lemme}{Lemme}[section]
\renewcommand{\thelemme}{\arabic{lemme}}
\newtheorem{proposition}{Proposition}[section]
\renewcommand{\theproposition}{\arabic{proposition}}
\newtheorem{notations}{Notations}[section]
\newtheorem{problem}{Problème}[section]
\newtheorem{corollary}{Corollaire}[theorem]
\renewcommand{\thecorollary}{\arabic{corollary}}
\newtheorem{property}{Propriété}[section]
\newtheorem{objective}{Objectif}[section]

\theoremstyle{definition}
\newtheorem{definition}{Définition}[section]
\renewcommand{\thedefinition}{\arabic{definition}}
\newtheorem{exercise}{Exercice}[chapter]
\renewcommand{\theexercise}{\arabic{exercise}}
\newtheorem{example}{Exemple}[chapter]
\renewcommand{\theexample}{\arabic{example}}
\newtheorem*{solution}{Solution}
\newtheorem*{application}{Application}
\newtheorem*{notation}{Notation}
\newtheorem*{vocabulary}{Vocabulaire}
\newtheorem*{properties}{Propriétés}



\theoremstyle{remark}
\newtheorem*{remark}{Remarque}
\newtheorem*{rappel}{Rappel}


\usepackage{etoolbox}
\AtBeginEnvironment{exercise}{\small}
\AtBeginEnvironment{example}{\small}

\usepackage{cases}
\usepackage[red]{mypack}

\usepackage[framemethod=TikZ]{mdframed}

\definecolor{bg}{rgb}{0.4,0.25,0.95}
\definecolor{pagebg}{rgb}{0,0,0.5}
\surroundwithmdframed[
   topline=false,
   rightline=false,
   bottomline=false,
   leftmargin=\parindent,
   skipabove=8pt,
   skipbelow=8pt,
   linecolor=blue,
   innerbottommargin=10pt,
   % backgroundcolor=bg,font=\color{orange}\sffamily, fontcolor=white
]{definition}

\usepackage{empheq}
\usepackage[most]{tcolorbox}

\newtcbox{\mymath}[1][]{%
    nobeforeafter, math upper, tcbox raise base,
    enhanced, colframe=blue!30!black,
    colback=red!10, boxrule=1pt,
    #1}

\usepackage{unixode}


\DeclareMathOperator{\ord}{ord}
\DeclareMathOperator{\orb}{orb}
\DeclareMathOperator{\stab}{stab}
\DeclareMathOperator{\Stab}{stab}
\DeclareMathOperator{\ppcm}{ppcm}
\DeclareMathOperator{\conj}{Conj}
\DeclareMathOperator{\End}{End}
\DeclareMathOperator{\rot}{rot}
\DeclareMathOperator{\trs}{trace}
\DeclareMathOperator{\Ind}{Ind}
\DeclareMathOperator{\mat}{Mat}
\DeclareMathOperator{\id}{Id}
\DeclareMathOperator{\vect}{vect}
\DeclareMathOperator{\img}{img}
\DeclareMathOperator{\cov}{Cov}
\DeclareMathOperator{\dist}{dist}
\DeclareMathOperator{\irr}{Irr}
\DeclareMathOperator{\image}{Im}
\DeclareMathOperator{\pd}{\partial}
\DeclareMathOperator{\epi}{epi}
\DeclareMathOperator{\Argmin}{Argmin}
\DeclareMathOperator{\dom}{dom}
\DeclareMathOperator{\proj}{proj}
\DeclareMathOperator{\ctg}{ctg}
\DeclareMathOperator{\supp}{supp}
\DeclareMathOperator{\argmin}{argmin}
\DeclareMathOperator{\mult}{mult}
\DeclareMathOperator{\ch}{ch}
\DeclareMathOperator{\sh}{sh}
\DeclareMathOperator{\rang}{rang}
\DeclareMathOperator{\diam}{diam}
\DeclareMathOperator{\Epigraphe}{Epigraphe}




\usepackage{xcolor}
\everymath{\color{blue}}
%\everymath{\color[rgb]{0,1,1}}
%\pagecolor[rgb]{0,0,0.5}


\newcommand*{\pdtest}[3][]{\ensuremath{\frac{\partial^{#1} #2}{\partial #3}}}

\newcommand*{\deffunc}[6][]{\ensuremath{
\begin{array}{rcl}
#2 : #3 &\rightarrow& #4\\
#5 &\mapsto& #6
\end{array}
}}

\newcommand{\eqcolon}{\mathrel{\resizebox{\widthof{$\mathord{=}$}}{\height}{ $\!\!=\!\!\resizebox{1.2\width}{0.8\height}{\raisebox{0.23ex}{$\mathop{:}$}}\!\!$ }}}
\newcommand{\coloneq}{\mathrel{\resizebox{\widthof{$\mathord{=}$}}{\height}{ $\!\!\resizebox{1.2\width}{0.8\height}{\raisebox{0.23ex}{$\mathop{:}$}}\!\!=\!\!$ }}}
\newcommand{\eqcolonl}{\ensuremath{\mathrel{=\!\!\mathop{:}}}}
\newcommand{\coloneql}{\ensuremath{\mathrel{\mathop{:} \!\! =}}}
\newcommand{\vc}[1]{% inline column vector
  \left(\begin{smallmatrix}#1\end{smallmatrix}\right)%
}
\newcommand{\vr}[1]{% inline row vector
  \begin{smallmatrix}(\,#1\,)\end{smallmatrix}%
}
\makeatletter
\newcommand*{\defeq}{\ =\mathrel{\rlap{%
                     \raisebox{0.3ex}{$\m@th\cdot$}}%
                     \raisebox{-0.3ex}{$\m@th\cdot$}}%
                     }
\makeatother

\newcommand{\mathcircle}[1]{% inline row vector
 \overset{\circ}{#1}
}
\newcommand{\ulim}{% low limit
 \underline{\lim}
}
\newcommand{\ssi}{% iff
\iff
}
\newcommand{\ps}[2]{
\expval{#1 | #2}
}
\newcommand{\df}[1]{
\mqty{#1}
}
\newcommand{\n}[1]{
\norm{#1}
}
\newcommand{\sys}[1]{
\left\{\smqty{#1}\right.
}


\newcommand{\eqdef}{\ensuremath{\overset{\text{def}}=}}


\def\Circlearrowright{\ensuremath{%
  \rotatebox[origin=c]{230}{$\circlearrowright$}}}

\newcommand\ct[1]{\text{\rmfamily\upshape #1}}
\newcommand\question[1]{ {\color{red} ...!? \small #1}}
\newcommand\caz[1]{\left\{\begin{array} #1 \end{array}\right.}
\newcommand\const{\text{\rmfamily\upshape const}}
\newcommand\toP{ \overset{\pro}{\to}}
\newcommand\toPP{ \overset{\text{PP}}{\to}}
\newcommand{\oeq}{\mathrel{\text{\textcircled{$=$}}}}





\usepackage{xcolor}
% \usepackage[normalem]{ulem}
\usepackage{lipsum}
\makeatletter
% \newcommand\colorwave[1][blue]{\bgroup \markoverwith{\lower3.5\p@\hbox{\sixly \textcolor{#1}{\char58}}}\ULon}
%\font\sixly=lasy6 % does not re-load if already loaded, so no memory problem.

\newmdtheoremenv[
linewidth= 1pt,linecolor= blue,%
leftmargin=20,rightmargin=20,innertopmargin=0pt, innerrightmargin=40,%
tikzsetting = { draw=lightgray, line width = 0.3pt,dashed,%
dash pattern = on 15pt off 3pt},%
splittopskip=\topskip,skipbelow=\baselineskip,%
skipabove=\baselineskip,ntheorem,roundcorner=0pt,
% backgroundcolor=pagebg,font=\color{orange}\sffamily, fontcolor=white
]{examplebox}{Exemple}[section]



\newcommand\R{\mathbb{R}}
\newcommand\Z{\mathbb{Z}}
\newcommand\N{\mathbb{N}}
\newcommand\E{\mathbb{E}}
\newcommand\F{\mathcal{F}}
\newcommand\cH{\mathcal{H}}
\newcommand\V{\mathbb{V}}
\newcommand\dmo{ ^{-1} }
\newcommand\kapa{\kappa}
\newcommand\im{Im}
\newcommand\hs{\mathcal{H}}





\usepackage{soul}

\makeatletter
\newcommand*{\whiten}[1]{\llap{\textcolor{white}{{\the\SOUL@token}}\hspace{#1pt}}}
\DeclareRobustCommand*\myul{%
    \def\SOUL@everyspace{\underline{\space}\kern\z@}%
    \def\SOUL@everytoken{%
     \setbox0=\hbox{\the\SOUL@token}%
     \ifdim\dp0>\z@
        \raisebox{\dp0}{\underline{\phantom{\the\SOUL@token}}}%
        \whiten{1}\whiten{0}%
        \whiten{-1}\whiten{-2}%
        \llap{\the\SOUL@token}%
     \else
        \underline{\the\SOUL@token}%
     \fi}%
\SOUL@}
\makeatother

\newcommand*{\demp}{\fontfamily{lmtt}\selectfont}

\DeclareTextFontCommand{\textdemp}{\demp}

\begin{document}

\ifcomment
Multiline
comment
\fi
\ifcomment
\myul{Typesetting test}
% \color[rgb]{1,1,1}
$∑_i^n≠ 60º±∞π∆¬≈√j∫h≤≥µ$

$\CR \R\pro\ind\pro\gS\pro
\mqty[a&b\\c&d]$
$\pro\mathbb{P}$
$\dd{x}$

  \[
    \alpha(x)=\left\{
                \begin{array}{ll}
                  x\\
                  \frac{1}{1+e^{-kx}}\\
                  \frac{e^x-e^{-x}}{e^x+e^{-x}}
                \end{array}
              \right.
  \]

  $\expval{x}$
  
  $\chi_\rho(ghg\dmo)=\Tr(\rho_{ghg\dmo})=\Tr(\rho_g\circ\rho_h\circ\rho\dmo_g)=\Tr(\rho_h)\overset{\mbox{\scalebox{0.5}{$\Tr(AB)=\Tr(BA)$}}}{=}\chi_\rho(h)$
  	$\mathop{\oplus}_{\substack{x\in X}}$

$\mat(\rho_g)=(a_{ij}(g))_{\scriptsize \substack{1\leq i\leq d \\ 1\leq j\leq d}}$ et $\mat(\rho'_g)=(a'_{ij}(g))_{\scriptsize \substack{1\leq i'\leq d' \\ 1\leq j'\leq d'}}$



\[\int_a^b{\mathbb{R}^2}g(u, v)\dd{P_{XY}}(u, v)=\iint g(u,v) f_{XY}(u, v)\dd \lambda(u) \dd \lambda(v)\]
$$\lim_{x\to\infty} f(x)$$	
$$\iiiint_V \mu(t,u,v,w) \,dt\,du\,dv\,dw$$
$$\sum_{n=1}^{\infty} 2^{-n} = 1$$	
\begin{definition}
	Si $X$ et $Y$ sont 2 v.a. ou definit la \textsc{Covariance} entre $X$ et $Y$ comme
	$\cov(X,Y)\overset{\text{def}}{=}\E\left[(X-\E(X))(Y-\E(Y))\right]=\E(XY)-\E(X)\E(Y)$.
\end{definition}
\fi
\pagebreak

% \tableofcontents

% insert your code here
%\input{./algebra/main.tex}
%\input{./geometrie-differentielle/main.tex}
%\input{./probabilite/main.tex}
%\input{./analyse-fonctionnelle/main.tex}
% \input{./Analyse-convexe-et-dualite-en-optimisation/main.tex}
%\input{./tikz/main.tex}
%\input{./Theorie-du-distributions/main.tex}
%\input{./optimisation/mine.tex}
 \input{./modelisation/main.tex}

% yves.aubry@univ-tln.fr : algebra

\end{document}


% yves.aubry@univ-tln.fr : algebra

\end{document}

%% !TEX encoding = UTF-8 Unicode
% !TEX TS-program = xelatex

\documentclass[french]{report}

%\usepackage[utf8]{inputenc}
%\usepackage[T1]{fontenc}
\usepackage{babel}


\newif\ifcomment
%\commenttrue # Show comments

\usepackage{physics}
\usepackage{amssymb}


\usepackage{amsthm}
% \usepackage{thmtools}
\usepackage{mathtools}
\usepackage{amsfonts}

\usepackage{color}

\usepackage{tikz}

\usepackage{geometry}
\geometry{a5paper, margin=0.1in, right=1cm}

\usepackage{dsfont}

\usepackage{graphicx}
\graphicspath{ {images/} }

\usepackage{faktor}

\usepackage{IEEEtrantools}
\usepackage{enumerate}   
\usepackage[PostScript=dvips]{"/Users/aware/Documents/Courses/diagrams"}


\newtheorem{theorem}{Théorème}[section]
\renewcommand{\thetheorem}{\arabic{theorem}}
\newtheorem{lemme}{Lemme}[section]
\renewcommand{\thelemme}{\arabic{lemme}}
\newtheorem{proposition}{Proposition}[section]
\renewcommand{\theproposition}{\arabic{proposition}}
\newtheorem{notations}{Notations}[section]
\newtheorem{problem}{Problème}[section]
\newtheorem{corollary}{Corollaire}[theorem]
\renewcommand{\thecorollary}{\arabic{corollary}}
\newtheorem{property}{Propriété}[section]
\newtheorem{objective}{Objectif}[section]

\theoremstyle{definition}
\newtheorem{definition}{Définition}[section]
\renewcommand{\thedefinition}{\arabic{definition}}
\newtheorem{exercise}{Exercice}[chapter]
\renewcommand{\theexercise}{\arabic{exercise}}
\newtheorem{example}{Exemple}[chapter]
\renewcommand{\theexample}{\arabic{example}}
\newtheorem*{solution}{Solution}
\newtheorem*{application}{Application}
\newtheorem*{notation}{Notation}
\newtheorem*{vocabulary}{Vocabulaire}
\newtheorem*{properties}{Propriétés}



\theoremstyle{remark}
\newtheorem*{remark}{Remarque}
\newtheorem*{rappel}{Rappel}


\usepackage{etoolbox}
\AtBeginEnvironment{exercise}{\small}
\AtBeginEnvironment{example}{\small}

\usepackage{cases}
\usepackage[red]{mypack}

\usepackage[framemethod=TikZ]{mdframed}

\definecolor{bg}{rgb}{0.4,0.25,0.95}
\definecolor{pagebg}{rgb}{0,0,0.5}
\surroundwithmdframed[
   topline=false,
   rightline=false,
   bottomline=false,
   leftmargin=\parindent,
   skipabove=8pt,
   skipbelow=8pt,
   linecolor=blue,
   innerbottommargin=10pt,
   % backgroundcolor=bg,font=\color{orange}\sffamily, fontcolor=white
]{definition}

\usepackage{empheq}
\usepackage[most]{tcolorbox}

\newtcbox{\mymath}[1][]{%
    nobeforeafter, math upper, tcbox raise base,
    enhanced, colframe=blue!30!black,
    colback=red!10, boxrule=1pt,
    #1}

\usepackage{unixode}


\DeclareMathOperator{\ord}{ord}
\DeclareMathOperator{\orb}{orb}
\DeclareMathOperator{\stab}{stab}
\DeclareMathOperator{\Stab}{stab}
\DeclareMathOperator{\ppcm}{ppcm}
\DeclareMathOperator{\conj}{Conj}
\DeclareMathOperator{\End}{End}
\DeclareMathOperator{\rot}{rot}
\DeclareMathOperator{\trs}{trace}
\DeclareMathOperator{\Ind}{Ind}
\DeclareMathOperator{\mat}{Mat}
\DeclareMathOperator{\id}{Id}
\DeclareMathOperator{\vect}{vect}
\DeclareMathOperator{\img}{img}
\DeclareMathOperator{\cov}{Cov}
\DeclareMathOperator{\dist}{dist}
\DeclareMathOperator{\irr}{Irr}
\DeclareMathOperator{\image}{Im}
\DeclareMathOperator{\pd}{\partial}
\DeclareMathOperator{\epi}{epi}
\DeclareMathOperator{\Argmin}{Argmin}
\DeclareMathOperator{\dom}{dom}
\DeclareMathOperator{\proj}{proj}
\DeclareMathOperator{\ctg}{ctg}
\DeclareMathOperator{\supp}{supp}
\DeclareMathOperator{\argmin}{argmin}
\DeclareMathOperator{\mult}{mult}
\DeclareMathOperator{\ch}{ch}
\DeclareMathOperator{\sh}{sh}
\DeclareMathOperator{\rang}{rang}
\DeclareMathOperator{\diam}{diam}
\DeclareMathOperator{\Epigraphe}{Epigraphe}




\usepackage{xcolor}
\everymath{\color{blue}}
%\everymath{\color[rgb]{0,1,1}}
%\pagecolor[rgb]{0,0,0.5}


\newcommand*{\pdtest}[3][]{\ensuremath{\frac{\partial^{#1} #2}{\partial #3}}}

\newcommand*{\deffunc}[6][]{\ensuremath{
\begin{array}{rcl}
#2 : #3 &\rightarrow& #4\\
#5 &\mapsto& #6
\end{array}
}}

\newcommand{\eqcolon}{\mathrel{\resizebox{\widthof{$\mathord{=}$}}{\height}{ $\!\!=\!\!\resizebox{1.2\width}{0.8\height}{\raisebox{0.23ex}{$\mathop{:}$}}\!\!$ }}}
\newcommand{\coloneq}{\mathrel{\resizebox{\widthof{$\mathord{=}$}}{\height}{ $\!\!\resizebox{1.2\width}{0.8\height}{\raisebox{0.23ex}{$\mathop{:}$}}\!\!=\!\!$ }}}
\newcommand{\eqcolonl}{\ensuremath{\mathrel{=\!\!\mathop{:}}}}
\newcommand{\coloneql}{\ensuremath{\mathrel{\mathop{:} \!\! =}}}
\newcommand{\vc}[1]{% inline column vector
  \left(\begin{smallmatrix}#1\end{smallmatrix}\right)%
}
\newcommand{\vr}[1]{% inline row vector
  \begin{smallmatrix}(\,#1\,)\end{smallmatrix}%
}
\makeatletter
\newcommand*{\defeq}{\ =\mathrel{\rlap{%
                     \raisebox{0.3ex}{$\m@th\cdot$}}%
                     \raisebox{-0.3ex}{$\m@th\cdot$}}%
                     }
\makeatother

\newcommand{\mathcircle}[1]{% inline row vector
 \overset{\circ}{#1}
}
\newcommand{\ulim}{% low limit
 \underline{\lim}
}
\newcommand{\ssi}{% iff
\iff
}
\newcommand{\ps}[2]{
\expval{#1 | #2}
}
\newcommand{\df}[1]{
\mqty{#1}
}
\newcommand{\n}[1]{
\norm{#1}
}
\newcommand{\sys}[1]{
\left\{\smqty{#1}\right.
}


\newcommand{\eqdef}{\ensuremath{\overset{\text{def}}=}}


\def\Circlearrowright{\ensuremath{%
  \rotatebox[origin=c]{230}{$\circlearrowright$}}}

\newcommand\ct[1]{\text{\rmfamily\upshape #1}}
\newcommand\question[1]{ {\color{red} ...!? \small #1}}
\newcommand\caz[1]{\left\{\begin{array} #1 \end{array}\right.}
\newcommand\const{\text{\rmfamily\upshape const}}
\newcommand\toP{ \overset{\pro}{\to}}
\newcommand\toPP{ \overset{\text{PP}}{\to}}
\newcommand{\oeq}{\mathrel{\text{\textcircled{$=$}}}}





\usepackage{xcolor}
% \usepackage[normalem]{ulem}
\usepackage{lipsum}
\makeatletter
% \newcommand\colorwave[1][blue]{\bgroup \markoverwith{\lower3.5\p@\hbox{\sixly \textcolor{#1}{\char58}}}\ULon}
%\font\sixly=lasy6 % does not re-load if already loaded, so no memory problem.

\newmdtheoremenv[
linewidth= 1pt,linecolor= blue,%
leftmargin=20,rightmargin=20,innertopmargin=0pt, innerrightmargin=40,%
tikzsetting = { draw=lightgray, line width = 0.3pt,dashed,%
dash pattern = on 15pt off 3pt},%
splittopskip=\topskip,skipbelow=\baselineskip,%
skipabove=\baselineskip,ntheorem,roundcorner=0pt,
% backgroundcolor=pagebg,font=\color{orange}\sffamily, fontcolor=white
]{examplebox}{Exemple}[section]



\newcommand\R{\mathbb{R}}
\newcommand\Z{\mathbb{Z}}
\newcommand\N{\mathbb{N}}
\newcommand\E{\mathbb{E}}
\newcommand\F{\mathcal{F}}
\newcommand\cH{\mathcal{H}}
\newcommand\V{\mathbb{V}}
\newcommand\dmo{ ^{-1} }
\newcommand\kapa{\kappa}
\newcommand\im{Im}
\newcommand\hs{\mathcal{H}}





\usepackage{soul}

\makeatletter
\newcommand*{\whiten}[1]{\llap{\textcolor{white}{{\the\SOUL@token}}\hspace{#1pt}}}
\DeclareRobustCommand*\myul{%
    \def\SOUL@everyspace{\underline{\space}\kern\z@}%
    \def\SOUL@everytoken{%
     \setbox0=\hbox{\the\SOUL@token}%
     \ifdim\dp0>\z@
        \raisebox{\dp0}{\underline{\phantom{\the\SOUL@token}}}%
        \whiten{1}\whiten{0}%
        \whiten{-1}\whiten{-2}%
        \llap{\the\SOUL@token}%
     \else
        \underline{\the\SOUL@token}%
     \fi}%
\SOUL@}
\makeatother

\newcommand*{\demp}{\fontfamily{lmtt}\selectfont}

\DeclareTextFontCommand{\textdemp}{\demp}

\begin{document}

\ifcomment
Multiline
comment
\fi
\ifcomment
\myul{Typesetting test}
% \color[rgb]{1,1,1}
$∑_i^n≠ 60º±∞π∆¬≈√j∫h≤≥µ$

$\CR \R\pro\ind\pro\gS\pro
\mqty[a&b\\c&d]$
$\pro\mathbb{P}$
$\dd{x}$

  \[
    \alpha(x)=\left\{
                \begin{array}{ll}
                  x\\
                  \frac{1}{1+e^{-kx}}\\
                  \frac{e^x-e^{-x}}{e^x+e^{-x}}
                \end{array}
              \right.
  \]

  $\expval{x}$
  
  $\chi_\rho(ghg\dmo)=\Tr(\rho_{ghg\dmo})=\Tr(\rho_g\circ\rho_h\circ\rho\dmo_g)=\Tr(\rho_h)\overset{\mbox{\scalebox{0.5}{$\Tr(AB)=\Tr(BA)$}}}{=}\chi_\rho(h)$
  	$\mathop{\oplus}_{\substack{x\in X}}$

$\mat(\rho_g)=(a_{ij}(g))_{\scriptsize \substack{1\leq i\leq d \\ 1\leq j\leq d}}$ et $\mat(\rho'_g)=(a'_{ij}(g))_{\scriptsize \substack{1\leq i'\leq d' \\ 1\leq j'\leq d'}}$



\[\int_a^b{\mathbb{R}^2}g(u, v)\dd{P_{XY}}(u, v)=\iint g(u,v) f_{XY}(u, v)\dd \lambda(u) \dd \lambda(v)\]
$$\lim_{x\to\infty} f(x)$$	
$$\iiiint_V \mu(t,u,v,w) \,dt\,du\,dv\,dw$$
$$\sum_{n=1}^{\infty} 2^{-n} = 1$$	
\begin{definition}
	Si $X$ et $Y$ sont 2 v.a. ou definit la \textsc{Covariance} entre $X$ et $Y$ comme
	$\cov(X,Y)\overset{\text{def}}{=}\E\left[(X-\E(X))(Y-\E(Y))\right]=\E(XY)-\E(X)\E(Y)$.
\end{definition}
\fi
\pagebreak

% \tableofcontents

% insert your code here
%% !TEX encoding = UTF-8 Unicode
% !TEX TS-program = xelatex

\documentclass[french]{report}

%\usepackage[utf8]{inputenc}
%\usepackage[T1]{fontenc}
\usepackage{babel}


\newif\ifcomment
%\commenttrue # Show comments

\usepackage{physics}
\usepackage{amssymb}


\usepackage{amsthm}
% \usepackage{thmtools}
\usepackage{mathtools}
\usepackage{amsfonts}

\usepackage{color}

\usepackage{tikz}

\usepackage{geometry}
\geometry{a5paper, margin=0.1in, right=1cm}

\usepackage{dsfont}

\usepackage{graphicx}
\graphicspath{ {images/} }

\usepackage{faktor}

\usepackage{IEEEtrantools}
\usepackage{enumerate}   
\usepackage[PostScript=dvips]{"/Users/aware/Documents/Courses/diagrams"}


\newtheorem{theorem}{Théorème}[section]
\renewcommand{\thetheorem}{\arabic{theorem}}
\newtheorem{lemme}{Lemme}[section]
\renewcommand{\thelemme}{\arabic{lemme}}
\newtheorem{proposition}{Proposition}[section]
\renewcommand{\theproposition}{\arabic{proposition}}
\newtheorem{notations}{Notations}[section]
\newtheorem{problem}{Problème}[section]
\newtheorem{corollary}{Corollaire}[theorem]
\renewcommand{\thecorollary}{\arabic{corollary}}
\newtheorem{property}{Propriété}[section]
\newtheorem{objective}{Objectif}[section]

\theoremstyle{definition}
\newtheorem{definition}{Définition}[section]
\renewcommand{\thedefinition}{\arabic{definition}}
\newtheorem{exercise}{Exercice}[chapter]
\renewcommand{\theexercise}{\arabic{exercise}}
\newtheorem{example}{Exemple}[chapter]
\renewcommand{\theexample}{\arabic{example}}
\newtheorem*{solution}{Solution}
\newtheorem*{application}{Application}
\newtheorem*{notation}{Notation}
\newtheorem*{vocabulary}{Vocabulaire}
\newtheorem*{properties}{Propriétés}



\theoremstyle{remark}
\newtheorem*{remark}{Remarque}
\newtheorem*{rappel}{Rappel}


\usepackage{etoolbox}
\AtBeginEnvironment{exercise}{\small}
\AtBeginEnvironment{example}{\small}

\usepackage{cases}
\usepackage[red]{mypack}

\usepackage[framemethod=TikZ]{mdframed}

\definecolor{bg}{rgb}{0.4,0.25,0.95}
\definecolor{pagebg}{rgb}{0,0,0.5}
\surroundwithmdframed[
   topline=false,
   rightline=false,
   bottomline=false,
   leftmargin=\parindent,
   skipabove=8pt,
   skipbelow=8pt,
   linecolor=blue,
   innerbottommargin=10pt,
   % backgroundcolor=bg,font=\color{orange}\sffamily, fontcolor=white
]{definition}

\usepackage{empheq}
\usepackage[most]{tcolorbox}

\newtcbox{\mymath}[1][]{%
    nobeforeafter, math upper, tcbox raise base,
    enhanced, colframe=blue!30!black,
    colback=red!10, boxrule=1pt,
    #1}

\usepackage{unixode}


\DeclareMathOperator{\ord}{ord}
\DeclareMathOperator{\orb}{orb}
\DeclareMathOperator{\stab}{stab}
\DeclareMathOperator{\Stab}{stab}
\DeclareMathOperator{\ppcm}{ppcm}
\DeclareMathOperator{\conj}{Conj}
\DeclareMathOperator{\End}{End}
\DeclareMathOperator{\rot}{rot}
\DeclareMathOperator{\trs}{trace}
\DeclareMathOperator{\Ind}{Ind}
\DeclareMathOperator{\mat}{Mat}
\DeclareMathOperator{\id}{Id}
\DeclareMathOperator{\vect}{vect}
\DeclareMathOperator{\img}{img}
\DeclareMathOperator{\cov}{Cov}
\DeclareMathOperator{\dist}{dist}
\DeclareMathOperator{\irr}{Irr}
\DeclareMathOperator{\image}{Im}
\DeclareMathOperator{\pd}{\partial}
\DeclareMathOperator{\epi}{epi}
\DeclareMathOperator{\Argmin}{Argmin}
\DeclareMathOperator{\dom}{dom}
\DeclareMathOperator{\proj}{proj}
\DeclareMathOperator{\ctg}{ctg}
\DeclareMathOperator{\supp}{supp}
\DeclareMathOperator{\argmin}{argmin}
\DeclareMathOperator{\mult}{mult}
\DeclareMathOperator{\ch}{ch}
\DeclareMathOperator{\sh}{sh}
\DeclareMathOperator{\rang}{rang}
\DeclareMathOperator{\diam}{diam}
\DeclareMathOperator{\Epigraphe}{Epigraphe}




\usepackage{xcolor}
\everymath{\color{blue}}
%\everymath{\color[rgb]{0,1,1}}
%\pagecolor[rgb]{0,0,0.5}


\newcommand*{\pdtest}[3][]{\ensuremath{\frac{\partial^{#1} #2}{\partial #3}}}

\newcommand*{\deffunc}[6][]{\ensuremath{
\begin{array}{rcl}
#2 : #3 &\rightarrow& #4\\
#5 &\mapsto& #6
\end{array}
}}

\newcommand{\eqcolon}{\mathrel{\resizebox{\widthof{$\mathord{=}$}}{\height}{ $\!\!=\!\!\resizebox{1.2\width}{0.8\height}{\raisebox{0.23ex}{$\mathop{:}$}}\!\!$ }}}
\newcommand{\coloneq}{\mathrel{\resizebox{\widthof{$\mathord{=}$}}{\height}{ $\!\!\resizebox{1.2\width}{0.8\height}{\raisebox{0.23ex}{$\mathop{:}$}}\!\!=\!\!$ }}}
\newcommand{\eqcolonl}{\ensuremath{\mathrel{=\!\!\mathop{:}}}}
\newcommand{\coloneql}{\ensuremath{\mathrel{\mathop{:} \!\! =}}}
\newcommand{\vc}[1]{% inline column vector
  \left(\begin{smallmatrix}#1\end{smallmatrix}\right)%
}
\newcommand{\vr}[1]{% inline row vector
  \begin{smallmatrix}(\,#1\,)\end{smallmatrix}%
}
\makeatletter
\newcommand*{\defeq}{\ =\mathrel{\rlap{%
                     \raisebox{0.3ex}{$\m@th\cdot$}}%
                     \raisebox{-0.3ex}{$\m@th\cdot$}}%
                     }
\makeatother

\newcommand{\mathcircle}[1]{% inline row vector
 \overset{\circ}{#1}
}
\newcommand{\ulim}{% low limit
 \underline{\lim}
}
\newcommand{\ssi}{% iff
\iff
}
\newcommand{\ps}[2]{
\expval{#1 | #2}
}
\newcommand{\df}[1]{
\mqty{#1}
}
\newcommand{\n}[1]{
\norm{#1}
}
\newcommand{\sys}[1]{
\left\{\smqty{#1}\right.
}


\newcommand{\eqdef}{\ensuremath{\overset{\text{def}}=}}


\def\Circlearrowright{\ensuremath{%
  \rotatebox[origin=c]{230}{$\circlearrowright$}}}

\newcommand\ct[1]{\text{\rmfamily\upshape #1}}
\newcommand\question[1]{ {\color{red} ...!? \small #1}}
\newcommand\caz[1]{\left\{\begin{array} #1 \end{array}\right.}
\newcommand\const{\text{\rmfamily\upshape const}}
\newcommand\toP{ \overset{\pro}{\to}}
\newcommand\toPP{ \overset{\text{PP}}{\to}}
\newcommand{\oeq}{\mathrel{\text{\textcircled{$=$}}}}





\usepackage{xcolor}
% \usepackage[normalem]{ulem}
\usepackage{lipsum}
\makeatletter
% \newcommand\colorwave[1][blue]{\bgroup \markoverwith{\lower3.5\p@\hbox{\sixly \textcolor{#1}{\char58}}}\ULon}
%\font\sixly=lasy6 % does not re-load if already loaded, so no memory problem.

\newmdtheoremenv[
linewidth= 1pt,linecolor= blue,%
leftmargin=20,rightmargin=20,innertopmargin=0pt, innerrightmargin=40,%
tikzsetting = { draw=lightgray, line width = 0.3pt,dashed,%
dash pattern = on 15pt off 3pt},%
splittopskip=\topskip,skipbelow=\baselineskip,%
skipabove=\baselineskip,ntheorem,roundcorner=0pt,
% backgroundcolor=pagebg,font=\color{orange}\sffamily, fontcolor=white
]{examplebox}{Exemple}[section]



\newcommand\R{\mathbb{R}}
\newcommand\Z{\mathbb{Z}}
\newcommand\N{\mathbb{N}}
\newcommand\E{\mathbb{E}}
\newcommand\F{\mathcal{F}}
\newcommand\cH{\mathcal{H}}
\newcommand\V{\mathbb{V}}
\newcommand\dmo{ ^{-1} }
\newcommand\kapa{\kappa}
\newcommand\im{Im}
\newcommand\hs{\mathcal{H}}





\usepackage{soul}

\makeatletter
\newcommand*{\whiten}[1]{\llap{\textcolor{white}{{\the\SOUL@token}}\hspace{#1pt}}}
\DeclareRobustCommand*\myul{%
    \def\SOUL@everyspace{\underline{\space}\kern\z@}%
    \def\SOUL@everytoken{%
     \setbox0=\hbox{\the\SOUL@token}%
     \ifdim\dp0>\z@
        \raisebox{\dp0}{\underline{\phantom{\the\SOUL@token}}}%
        \whiten{1}\whiten{0}%
        \whiten{-1}\whiten{-2}%
        \llap{\the\SOUL@token}%
     \else
        \underline{\the\SOUL@token}%
     \fi}%
\SOUL@}
\makeatother

\newcommand*{\demp}{\fontfamily{lmtt}\selectfont}

\DeclareTextFontCommand{\textdemp}{\demp}

\begin{document}

\ifcomment
Multiline
comment
\fi
\ifcomment
\myul{Typesetting test}
% \color[rgb]{1,1,1}
$∑_i^n≠ 60º±∞π∆¬≈√j∫h≤≥µ$

$\CR \R\pro\ind\pro\gS\pro
\mqty[a&b\\c&d]$
$\pro\mathbb{P}$
$\dd{x}$

  \[
    \alpha(x)=\left\{
                \begin{array}{ll}
                  x\\
                  \frac{1}{1+e^{-kx}}\\
                  \frac{e^x-e^{-x}}{e^x+e^{-x}}
                \end{array}
              \right.
  \]

  $\expval{x}$
  
  $\chi_\rho(ghg\dmo)=\Tr(\rho_{ghg\dmo})=\Tr(\rho_g\circ\rho_h\circ\rho\dmo_g)=\Tr(\rho_h)\overset{\mbox{\scalebox{0.5}{$\Tr(AB)=\Tr(BA)$}}}{=}\chi_\rho(h)$
  	$\mathop{\oplus}_{\substack{x\in X}}$

$\mat(\rho_g)=(a_{ij}(g))_{\scriptsize \substack{1\leq i\leq d \\ 1\leq j\leq d}}$ et $\mat(\rho'_g)=(a'_{ij}(g))_{\scriptsize \substack{1\leq i'\leq d' \\ 1\leq j'\leq d'}}$



\[\int_a^b{\mathbb{R}^2}g(u, v)\dd{P_{XY}}(u, v)=\iint g(u,v) f_{XY}(u, v)\dd \lambda(u) \dd \lambda(v)\]
$$\lim_{x\to\infty} f(x)$$	
$$\iiiint_V \mu(t,u,v,w) \,dt\,du\,dv\,dw$$
$$\sum_{n=1}^{\infty} 2^{-n} = 1$$	
\begin{definition}
	Si $X$ et $Y$ sont 2 v.a. ou definit la \textsc{Covariance} entre $X$ et $Y$ comme
	$\cov(X,Y)\overset{\text{def}}{=}\E\left[(X-\E(X))(Y-\E(Y))\right]=\E(XY)-\E(X)\E(Y)$.
\end{definition}
\fi
\pagebreak

% \tableofcontents

% insert your code here
%\input{./algebra/main.tex}
%\input{./geometrie-differentielle/main.tex}
%\input{./probabilite/main.tex}
%\input{./analyse-fonctionnelle/main.tex}
% \input{./Analyse-convexe-et-dualite-en-optimisation/main.tex}
%\input{./tikz/main.tex}
%\input{./Theorie-du-distributions/main.tex}
%\input{./optimisation/mine.tex}
 \input{./modelisation/main.tex}

% yves.aubry@univ-tln.fr : algebra

\end{document}

%% !TEX encoding = UTF-8 Unicode
% !TEX TS-program = xelatex

\documentclass[french]{report}

%\usepackage[utf8]{inputenc}
%\usepackage[T1]{fontenc}
\usepackage{babel}


\newif\ifcomment
%\commenttrue # Show comments

\usepackage{physics}
\usepackage{amssymb}


\usepackage{amsthm}
% \usepackage{thmtools}
\usepackage{mathtools}
\usepackage{amsfonts}

\usepackage{color}

\usepackage{tikz}

\usepackage{geometry}
\geometry{a5paper, margin=0.1in, right=1cm}

\usepackage{dsfont}

\usepackage{graphicx}
\graphicspath{ {images/} }

\usepackage{faktor}

\usepackage{IEEEtrantools}
\usepackage{enumerate}   
\usepackage[PostScript=dvips]{"/Users/aware/Documents/Courses/diagrams"}


\newtheorem{theorem}{Théorème}[section]
\renewcommand{\thetheorem}{\arabic{theorem}}
\newtheorem{lemme}{Lemme}[section]
\renewcommand{\thelemme}{\arabic{lemme}}
\newtheorem{proposition}{Proposition}[section]
\renewcommand{\theproposition}{\arabic{proposition}}
\newtheorem{notations}{Notations}[section]
\newtheorem{problem}{Problème}[section]
\newtheorem{corollary}{Corollaire}[theorem]
\renewcommand{\thecorollary}{\arabic{corollary}}
\newtheorem{property}{Propriété}[section]
\newtheorem{objective}{Objectif}[section]

\theoremstyle{definition}
\newtheorem{definition}{Définition}[section]
\renewcommand{\thedefinition}{\arabic{definition}}
\newtheorem{exercise}{Exercice}[chapter]
\renewcommand{\theexercise}{\arabic{exercise}}
\newtheorem{example}{Exemple}[chapter]
\renewcommand{\theexample}{\arabic{example}}
\newtheorem*{solution}{Solution}
\newtheorem*{application}{Application}
\newtheorem*{notation}{Notation}
\newtheorem*{vocabulary}{Vocabulaire}
\newtheorem*{properties}{Propriétés}



\theoremstyle{remark}
\newtheorem*{remark}{Remarque}
\newtheorem*{rappel}{Rappel}


\usepackage{etoolbox}
\AtBeginEnvironment{exercise}{\small}
\AtBeginEnvironment{example}{\small}

\usepackage{cases}
\usepackage[red]{mypack}

\usepackage[framemethod=TikZ]{mdframed}

\definecolor{bg}{rgb}{0.4,0.25,0.95}
\definecolor{pagebg}{rgb}{0,0,0.5}
\surroundwithmdframed[
   topline=false,
   rightline=false,
   bottomline=false,
   leftmargin=\parindent,
   skipabove=8pt,
   skipbelow=8pt,
   linecolor=blue,
   innerbottommargin=10pt,
   % backgroundcolor=bg,font=\color{orange}\sffamily, fontcolor=white
]{definition}

\usepackage{empheq}
\usepackage[most]{tcolorbox}

\newtcbox{\mymath}[1][]{%
    nobeforeafter, math upper, tcbox raise base,
    enhanced, colframe=blue!30!black,
    colback=red!10, boxrule=1pt,
    #1}

\usepackage{unixode}


\DeclareMathOperator{\ord}{ord}
\DeclareMathOperator{\orb}{orb}
\DeclareMathOperator{\stab}{stab}
\DeclareMathOperator{\Stab}{stab}
\DeclareMathOperator{\ppcm}{ppcm}
\DeclareMathOperator{\conj}{Conj}
\DeclareMathOperator{\End}{End}
\DeclareMathOperator{\rot}{rot}
\DeclareMathOperator{\trs}{trace}
\DeclareMathOperator{\Ind}{Ind}
\DeclareMathOperator{\mat}{Mat}
\DeclareMathOperator{\id}{Id}
\DeclareMathOperator{\vect}{vect}
\DeclareMathOperator{\img}{img}
\DeclareMathOperator{\cov}{Cov}
\DeclareMathOperator{\dist}{dist}
\DeclareMathOperator{\irr}{Irr}
\DeclareMathOperator{\image}{Im}
\DeclareMathOperator{\pd}{\partial}
\DeclareMathOperator{\epi}{epi}
\DeclareMathOperator{\Argmin}{Argmin}
\DeclareMathOperator{\dom}{dom}
\DeclareMathOperator{\proj}{proj}
\DeclareMathOperator{\ctg}{ctg}
\DeclareMathOperator{\supp}{supp}
\DeclareMathOperator{\argmin}{argmin}
\DeclareMathOperator{\mult}{mult}
\DeclareMathOperator{\ch}{ch}
\DeclareMathOperator{\sh}{sh}
\DeclareMathOperator{\rang}{rang}
\DeclareMathOperator{\diam}{diam}
\DeclareMathOperator{\Epigraphe}{Epigraphe}




\usepackage{xcolor}
\everymath{\color{blue}}
%\everymath{\color[rgb]{0,1,1}}
%\pagecolor[rgb]{0,0,0.5}


\newcommand*{\pdtest}[3][]{\ensuremath{\frac{\partial^{#1} #2}{\partial #3}}}

\newcommand*{\deffunc}[6][]{\ensuremath{
\begin{array}{rcl}
#2 : #3 &\rightarrow& #4\\
#5 &\mapsto& #6
\end{array}
}}

\newcommand{\eqcolon}{\mathrel{\resizebox{\widthof{$\mathord{=}$}}{\height}{ $\!\!=\!\!\resizebox{1.2\width}{0.8\height}{\raisebox{0.23ex}{$\mathop{:}$}}\!\!$ }}}
\newcommand{\coloneq}{\mathrel{\resizebox{\widthof{$\mathord{=}$}}{\height}{ $\!\!\resizebox{1.2\width}{0.8\height}{\raisebox{0.23ex}{$\mathop{:}$}}\!\!=\!\!$ }}}
\newcommand{\eqcolonl}{\ensuremath{\mathrel{=\!\!\mathop{:}}}}
\newcommand{\coloneql}{\ensuremath{\mathrel{\mathop{:} \!\! =}}}
\newcommand{\vc}[1]{% inline column vector
  \left(\begin{smallmatrix}#1\end{smallmatrix}\right)%
}
\newcommand{\vr}[1]{% inline row vector
  \begin{smallmatrix}(\,#1\,)\end{smallmatrix}%
}
\makeatletter
\newcommand*{\defeq}{\ =\mathrel{\rlap{%
                     \raisebox{0.3ex}{$\m@th\cdot$}}%
                     \raisebox{-0.3ex}{$\m@th\cdot$}}%
                     }
\makeatother

\newcommand{\mathcircle}[1]{% inline row vector
 \overset{\circ}{#1}
}
\newcommand{\ulim}{% low limit
 \underline{\lim}
}
\newcommand{\ssi}{% iff
\iff
}
\newcommand{\ps}[2]{
\expval{#1 | #2}
}
\newcommand{\df}[1]{
\mqty{#1}
}
\newcommand{\n}[1]{
\norm{#1}
}
\newcommand{\sys}[1]{
\left\{\smqty{#1}\right.
}


\newcommand{\eqdef}{\ensuremath{\overset{\text{def}}=}}


\def\Circlearrowright{\ensuremath{%
  \rotatebox[origin=c]{230}{$\circlearrowright$}}}

\newcommand\ct[1]{\text{\rmfamily\upshape #1}}
\newcommand\question[1]{ {\color{red} ...!? \small #1}}
\newcommand\caz[1]{\left\{\begin{array} #1 \end{array}\right.}
\newcommand\const{\text{\rmfamily\upshape const}}
\newcommand\toP{ \overset{\pro}{\to}}
\newcommand\toPP{ \overset{\text{PP}}{\to}}
\newcommand{\oeq}{\mathrel{\text{\textcircled{$=$}}}}





\usepackage{xcolor}
% \usepackage[normalem]{ulem}
\usepackage{lipsum}
\makeatletter
% \newcommand\colorwave[1][blue]{\bgroup \markoverwith{\lower3.5\p@\hbox{\sixly \textcolor{#1}{\char58}}}\ULon}
%\font\sixly=lasy6 % does not re-load if already loaded, so no memory problem.

\newmdtheoremenv[
linewidth= 1pt,linecolor= blue,%
leftmargin=20,rightmargin=20,innertopmargin=0pt, innerrightmargin=40,%
tikzsetting = { draw=lightgray, line width = 0.3pt,dashed,%
dash pattern = on 15pt off 3pt},%
splittopskip=\topskip,skipbelow=\baselineskip,%
skipabove=\baselineskip,ntheorem,roundcorner=0pt,
% backgroundcolor=pagebg,font=\color{orange}\sffamily, fontcolor=white
]{examplebox}{Exemple}[section]



\newcommand\R{\mathbb{R}}
\newcommand\Z{\mathbb{Z}}
\newcommand\N{\mathbb{N}}
\newcommand\E{\mathbb{E}}
\newcommand\F{\mathcal{F}}
\newcommand\cH{\mathcal{H}}
\newcommand\V{\mathbb{V}}
\newcommand\dmo{ ^{-1} }
\newcommand\kapa{\kappa}
\newcommand\im{Im}
\newcommand\hs{\mathcal{H}}





\usepackage{soul}

\makeatletter
\newcommand*{\whiten}[1]{\llap{\textcolor{white}{{\the\SOUL@token}}\hspace{#1pt}}}
\DeclareRobustCommand*\myul{%
    \def\SOUL@everyspace{\underline{\space}\kern\z@}%
    \def\SOUL@everytoken{%
     \setbox0=\hbox{\the\SOUL@token}%
     \ifdim\dp0>\z@
        \raisebox{\dp0}{\underline{\phantom{\the\SOUL@token}}}%
        \whiten{1}\whiten{0}%
        \whiten{-1}\whiten{-2}%
        \llap{\the\SOUL@token}%
     \else
        \underline{\the\SOUL@token}%
     \fi}%
\SOUL@}
\makeatother

\newcommand*{\demp}{\fontfamily{lmtt}\selectfont}

\DeclareTextFontCommand{\textdemp}{\demp}

\begin{document}

\ifcomment
Multiline
comment
\fi
\ifcomment
\myul{Typesetting test}
% \color[rgb]{1,1,1}
$∑_i^n≠ 60º±∞π∆¬≈√j∫h≤≥µ$

$\CR \R\pro\ind\pro\gS\pro
\mqty[a&b\\c&d]$
$\pro\mathbb{P}$
$\dd{x}$

  \[
    \alpha(x)=\left\{
                \begin{array}{ll}
                  x\\
                  \frac{1}{1+e^{-kx}}\\
                  \frac{e^x-e^{-x}}{e^x+e^{-x}}
                \end{array}
              \right.
  \]

  $\expval{x}$
  
  $\chi_\rho(ghg\dmo)=\Tr(\rho_{ghg\dmo})=\Tr(\rho_g\circ\rho_h\circ\rho\dmo_g)=\Tr(\rho_h)\overset{\mbox{\scalebox{0.5}{$\Tr(AB)=\Tr(BA)$}}}{=}\chi_\rho(h)$
  	$\mathop{\oplus}_{\substack{x\in X}}$

$\mat(\rho_g)=(a_{ij}(g))_{\scriptsize \substack{1\leq i\leq d \\ 1\leq j\leq d}}$ et $\mat(\rho'_g)=(a'_{ij}(g))_{\scriptsize \substack{1\leq i'\leq d' \\ 1\leq j'\leq d'}}$



\[\int_a^b{\mathbb{R}^2}g(u, v)\dd{P_{XY}}(u, v)=\iint g(u,v) f_{XY}(u, v)\dd \lambda(u) \dd \lambda(v)\]
$$\lim_{x\to\infty} f(x)$$	
$$\iiiint_V \mu(t,u,v,w) \,dt\,du\,dv\,dw$$
$$\sum_{n=1}^{\infty} 2^{-n} = 1$$	
\begin{definition}
	Si $X$ et $Y$ sont 2 v.a. ou definit la \textsc{Covariance} entre $X$ et $Y$ comme
	$\cov(X,Y)\overset{\text{def}}{=}\E\left[(X-\E(X))(Y-\E(Y))\right]=\E(XY)-\E(X)\E(Y)$.
\end{definition}
\fi
\pagebreak

% \tableofcontents

% insert your code here
%\input{./algebra/main.tex}
%\input{./geometrie-differentielle/main.tex}
%\input{./probabilite/main.tex}
%\input{./analyse-fonctionnelle/main.tex}
% \input{./Analyse-convexe-et-dualite-en-optimisation/main.tex}
%\input{./tikz/main.tex}
%\input{./Theorie-du-distributions/main.tex}
%\input{./optimisation/mine.tex}
 \input{./modelisation/main.tex}

% yves.aubry@univ-tln.fr : algebra

\end{document}

%% !TEX encoding = UTF-8 Unicode
% !TEX TS-program = xelatex

\documentclass[french]{report}

%\usepackage[utf8]{inputenc}
%\usepackage[T1]{fontenc}
\usepackage{babel}


\newif\ifcomment
%\commenttrue # Show comments

\usepackage{physics}
\usepackage{amssymb}


\usepackage{amsthm}
% \usepackage{thmtools}
\usepackage{mathtools}
\usepackage{amsfonts}

\usepackage{color}

\usepackage{tikz}

\usepackage{geometry}
\geometry{a5paper, margin=0.1in, right=1cm}

\usepackage{dsfont}

\usepackage{graphicx}
\graphicspath{ {images/} }

\usepackage{faktor}

\usepackage{IEEEtrantools}
\usepackage{enumerate}   
\usepackage[PostScript=dvips]{"/Users/aware/Documents/Courses/diagrams"}


\newtheorem{theorem}{Théorème}[section]
\renewcommand{\thetheorem}{\arabic{theorem}}
\newtheorem{lemme}{Lemme}[section]
\renewcommand{\thelemme}{\arabic{lemme}}
\newtheorem{proposition}{Proposition}[section]
\renewcommand{\theproposition}{\arabic{proposition}}
\newtheorem{notations}{Notations}[section]
\newtheorem{problem}{Problème}[section]
\newtheorem{corollary}{Corollaire}[theorem]
\renewcommand{\thecorollary}{\arabic{corollary}}
\newtheorem{property}{Propriété}[section]
\newtheorem{objective}{Objectif}[section]

\theoremstyle{definition}
\newtheorem{definition}{Définition}[section]
\renewcommand{\thedefinition}{\arabic{definition}}
\newtheorem{exercise}{Exercice}[chapter]
\renewcommand{\theexercise}{\arabic{exercise}}
\newtheorem{example}{Exemple}[chapter]
\renewcommand{\theexample}{\arabic{example}}
\newtheorem*{solution}{Solution}
\newtheorem*{application}{Application}
\newtheorem*{notation}{Notation}
\newtheorem*{vocabulary}{Vocabulaire}
\newtheorem*{properties}{Propriétés}



\theoremstyle{remark}
\newtheorem*{remark}{Remarque}
\newtheorem*{rappel}{Rappel}


\usepackage{etoolbox}
\AtBeginEnvironment{exercise}{\small}
\AtBeginEnvironment{example}{\small}

\usepackage{cases}
\usepackage[red]{mypack}

\usepackage[framemethod=TikZ]{mdframed}

\definecolor{bg}{rgb}{0.4,0.25,0.95}
\definecolor{pagebg}{rgb}{0,0,0.5}
\surroundwithmdframed[
   topline=false,
   rightline=false,
   bottomline=false,
   leftmargin=\parindent,
   skipabove=8pt,
   skipbelow=8pt,
   linecolor=blue,
   innerbottommargin=10pt,
   % backgroundcolor=bg,font=\color{orange}\sffamily, fontcolor=white
]{definition}

\usepackage{empheq}
\usepackage[most]{tcolorbox}

\newtcbox{\mymath}[1][]{%
    nobeforeafter, math upper, tcbox raise base,
    enhanced, colframe=blue!30!black,
    colback=red!10, boxrule=1pt,
    #1}

\usepackage{unixode}


\DeclareMathOperator{\ord}{ord}
\DeclareMathOperator{\orb}{orb}
\DeclareMathOperator{\stab}{stab}
\DeclareMathOperator{\Stab}{stab}
\DeclareMathOperator{\ppcm}{ppcm}
\DeclareMathOperator{\conj}{Conj}
\DeclareMathOperator{\End}{End}
\DeclareMathOperator{\rot}{rot}
\DeclareMathOperator{\trs}{trace}
\DeclareMathOperator{\Ind}{Ind}
\DeclareMathOperator{\mat}{Mat}
\DeclareMathOperator{\id}{Id}
\DeclareMathOperator{\vect}{vect}
\DeclareMathOperator{\img}{img}
\DeclareMathOperator{\cov}{Cov}
\DeclareMathOperator{\dist}{dist}
\DeclareMathOperator{\irr}{Irr}
\DeclareMathOperator{\image}{Im}
\DeclareMathOperator{\pd}{\partial}
\DeclareMathOperator{\epi}{epi}
\DeclareMathOperator{\Argmin}{Argmin}
\DeclareMathOperator{\dom}{dom}
\DeclareMathOperator{\proj}{proj}
\DeclareMathOperator{\ctg}{ctg}
\DeclareMathOperator{\supp}{supp}
\DeclareMathOperator{\argmin}{argmin}
\DeclareMathOperator{\mult}{mult}
\DeclareMathOperator{\ch}{ch}
\DeclareMathOperator{\sh}{sh}
\DeclareMathOperator{\rang}{rang}
\DeclareMathOperator{\diam}{diam}
\DeclareMathOperator{\Epigraphe}{Epigraphe}




\usepackage{xcolor}
\everymath{\color{blue}}
%\everymath{\color[rgb]{0,1,1}}
%\pagecolor[rgb]{0,0,0.5}


\newcommand*{\pdtest}[3][]{\ensuremath{\frac{\partial^{#1} #2}{\partial #3}}}

\newcommand*{\deffunc}[6][]{\ensuremath{
\begin{array}{rcl}
#2 : #3 &\rightarrow& #4\\
#5 &\mapsto& #6
\end{array}
}}

\newcommand{\eqcolon}{\mathrel{\resizebox{\widthof{$\mathord{=}$}}{\height}{ $\!\!=\!\!\resizebox{1.2\width}{0.8\height}{\raisebox{0.23ex}{$\mathop{:}$}}\!\!$ }}}
\newcommand{\coloneq}{\mathrel{\resizebox{\widthof{$\mathord{=}$}}{\height}{ $\!\!\resizebox{1.2\width}{0.8\height}{\raisebox{0.23ex}{$\mathop{:}$}}\!\!=\!\!$ }}}
\newcommand{\eqcolonl}{\ensuremath{\mathrel{=\!\!\mathop{:}}}}
\newcommand{\coloneql}{\ensuremath{\mathrel{\mathop{:} \!\! =}}}
\newcommand{\vc}[1]{% inline column vector
  \left(\begin{smallmatrix}#1\end{smallmatrix}\right)%
}
\newcommand{\vr}[1]{% inline row vector
  \begin{smallmatrix}(\,#1\,)\end{smallmatrix}%
}
\makeatletter
\newcommand*{\defeq}{\ =\mathrel{\rlap{%
                     \raisebox{0.3ex}{$\m@th\cdot$}}%
                     \raisebox{-0.3ex}{$\m@th\cdot$}}%
                     }
\makeatother

\newcommand{\mathcircle}[1]{% inline row vector
 \overset{\circ}{#1}
}
\newcommand{\ulim}{% low limit
 \underline{\lim}
}
\newcommand{\ssi}{% iff
\iff
}
\newcommand{\ps}[2]{
\expval{#1 | #2}
}
\newcommand{\df}[1]{
\mqty{#1}
}
\newcommand{\n}[1]{
\norm{#1}
}
\newcommand{\sys}[1]{
\left\{\smqty{#1}\right.
}


\newcommand{\eqdef}{\ensuremath{\overset{\text{def}}=}}


\def\Circlearrowright{\ensuremath{%
  \rotatebox[origin=c]{230}{$\circlearrowright$}}}

\newcommand\ct[1]{\text{\rmfamily\upshape #1}}
\newcommand\question[1]{ {\color{red} ...!? \small #1}}
\newcommand\caz[1]{\left\{\begin{array} #1 \end{array}\right.}
\newcommand\const{\text{\rmfamily\upshape const}}
\newcommand\toP{ \overset{\pro}{\to}}
\newcommand\toPP{ \overset{\text{PP}}{\to}}
\newcommand{\oeq}{\mathrel{\text{\textcircled{$=$}}}}





\usepackage{xcolor}
% \usepackage[normalem]{ulem}
\usepackage{lipsum}
\makeatletter
% \newcommand\colorwave[1][blue]{\bgroup \markoverwith{\lower3.5\p@\hbox{\sixly \textcolor{#1}{\char58}}}\ULon}
%\font\sixly=lasy6 % does not re-load if already loaded, so no memory problem.

\newmdtheoremenv[
linewidth= 1pt,linecolor= blue,%
leftmargin=20,rightmargin=20,innertopmargin=0pt, innerrightmargin=40,%
tikzsetting = { draw=lightgray, line width = 0.3pt,dashed,%
dash pattern = on 15pt off 3pt},%
splittopskip=\topskip,skipbelow=\baselineskip,%
skipabove=\baselineskip,ntheorem,roundcorner=0pt,
% backgroundcolor=pagebg,font=\color{orange}\sffamily, fontcolor=white
]{examplebox}{Exemple}[section]



\newcommand\R{\mathbb{R}}
\newcommand\Z{\mathbb{Z}}
\newcommand\N{\mathbb{N}}
\newcommand\E{\mathbb{E}}
\newcommand\F{\mathcal{F}}
\newcommand\cH{\mathcal{H}}
\newcommand\V{\mathbb{V}}
\newcommand\dmo{ ^{-1} }
\newcommand\kapa{\kappa}
\newcommand\im{Im}
\newcommand\hs{\mathcal{H}}





\usepackage{soul}

\makeatletter
\newcommand*{\whiten}[1]{\llap{\textcolor{white}{{\the\SOUL@token}}\hspace{#1pt}}}
\DeclareRobustCommand*\myul{%
    \def\SOUL@everyspace{\underline{\space}\kern\z@}%
    \def\SOUL@everytoken{%
     \setbox0=\hbox{\the\SOUL@token}%
     \ifdim\dp0>\z@
        \raisebox{\dp0}{\underline{\phantom{\the\SOUL@token}}}%
        \whiten{1}\whiten{0}%
        \whiten{-1}\whiten{-2}%
        \llap{\the\SOUL@token}%
     \else
        \underline{\the\SOUL@token}%
     \fi}%
\SOUL@}
\makeatother

\newcommand*{\demp}{\fontfamily{lmtt}\selectfont}

\DeclareTextFontCommand{\textdemp}{\demp}

\begin{document}

\ifcomment
Multiline
comment
\fi
\ifcomment
\myul{Typesetting test}
% \color[rgb]{1,1,1}
$∑_i^n≠ 60º±∞π∆¬≈√j∫h≤≥µ$

$\CR \R\pro\ind\pro\gS\pro
\mqty[a&b\\c&d]$
$\pro\mathbb{P}$
$\dd{x}$

  \[
    \alpha(x)=\left\{
                \begin{array}{ll}
                  x\\
                  \frac{1}{1+e^{-kx}}\\
                  \frac{e^x-e^{-x}}{e^x+e^{-x}}
                \end{array}
              \right.
  \]

  $\expval{x}$
  
  $\chi_\rho(ghg\dmo)=\Tr(\rho_{ghg\dmo})=\Tr(\rho_g\circ\rho_h\circ\rho\dmo_g)=\Tr(\rho_h)\overset{\mbox{\scalebox{0.5}{$\Tr(AB)=\Tr(BA)$}}}{=}\chi_\rho(h)$
  	$\mathop{\oplus}_{\substack{x\in X}}$

$\mat(\rho_g)=(a_{ij}(g))_{\scriptsize \substack{1\leq i\leq d \\ 1\leq j\leq d}}$ et $\mat(\rho'_g)=(a'_{ij}(g))_{\scriptsize \substack{1\leq i'\leq d' \\ 1\leq j'\leq d'}}$



\[\int_a^b{\mathbb{R}^2}g(u, v)\dd{P_{XY}}(u, v)=\iint g(u,v) f_{XY}(u, v)\dd \lambda(u) \dd \lambda(v)\]
$$\lim_{x\to\infty} f(x)$$	
$$\iiiint_V \mu(t,u,v,w) \,dt\,du\,dv\,dw$$
$$\sum_{n=1}^{\infty} 2^{-n} = 1$$	
\begin{definition}
	Si $X$ et $Y$ sont 2 v.a. ou definit la \textsc{Covariance} entre $X$ et $Y$ comme
	$\cov(X,Y)\overset{\text{def}}{=}\E\left[(X-\E(X))(Y-\E(Y))\right]=\E(XY)-\E(X)\E(Y)$.
\end{definition}
\fi
\pagebreak

% \tableofcontents

% insert your code here
%\input{./algebra/main.tex}
%\input{./geometrie-differentielle/main.tex}
%\input{./probabilite/main.tex}
%\input{./analyse-fonctionnelle/main.tex}
% \input{./Analyse-convexe-et-dualite-en-optimisation/main.tex}
%\input{./tikz/main.tex}
%\input{./Theorie-du-distributions/main.tex}
%\input{./optimisation/mine.tex}
 \input{./modelisation/main.tex}

% yves.aubry@univ-tln.fr : algebra

\end{document}

%% !TEX encoding = UTF-8 Unicode
% !TEX TS-program = xelatex

\documentclass[french]{report}

%\usepackage[utf8]{inputenc}
%\usepackage[T1]{fontenc}
\usepackage{babel}


\newif\ifcomment
%\commenttrue # Show comments

\usepackage{physics}
\usepackage{amssymb}


\usepackage{amsthm}
% \usepackage{thmtools}
\usepackage{mathtools}
\usepackage{amsfonts}

\usepackage{color}

\usepackage{tikz}

\usepackage{geometry}
\geometry{a5paper, margin=0.1in, right=1cm}

\usepackage{dsfont}

\usepackage{graphicx}
\graphicspath{ {images/} }

\usepackage{faktor}

\usepackage{IEEEtrantools}
\usepackage{enumerate}   
\usepackage[PostScript=dvips]{"/Users/aware/Documents/Courses/diagrams"}


\newtheorem{theorem}{Théorème}[section]
\renewcommand{\thetheorem}{\arabic{theorem}}
\newtheorem{lemme}{Lemme}[section]
\renewcommand{\thelemme}{\arabic{lemme}}
\newtheorem{proposition}{Proposition}[section]
\renewcommand{\theproposition}{\arabic{proposition}}
\newtheorem{notations}{Notations}[section]
\newtheorem{problem}{Problème}[section]
\newtheorem{corollary}{Corollaire}[theorem]
\renewcommand{\thecorollary}{\arabic{corollary}}
\newtheorem{property}{Propriété}[section]
\newtheorem{objective}{Objectif}[section]

\theoremstyle{definition}
\newtheorem{definition}{Définition}[section]
\renewcommand{\thedefinition}{\arabic{definition}}
\newtheorem{exercise}{Exercice}[chapter]
\renewcommand{\theexercise}{\arabic{exercise}}
\newtheorem{example}{Exemple}[chapter]
\renewcommand{\theexample}{\arabic{example}}
\newtheorem*{solution}{Solution}
\newtheorem*{application}{Application}
\newtheorem*{notation}{Notation}
\newtheorem*{vocabulary}{Vocabulaire}
\newtheorem*{properties}{Propriétés}



\theoremstyle{remark}
\newtheorem*{remark}{Remarque}
\newtheorem*{rappel}{Rappel}


\usepackage{etoolbox}
\AtBeginEnvironment{exercise}{\small}
\AtBeginEnvironment{example}{\small}

\usepackage{cases}
\usepackage[red]{mypack}

\usepackage[framemethod=TikZ]{mdframed}

\definecolor{bg}{rgb}{0.4,0.25,0.95}
\definecolor{pagebg}{rgb}{0,0,0.5}
\surroundwithmdframed[
   topline=false,
   rightline=false,
   bottomline=false,
   leftmargin=\parindent,
   skipabove=8pt,
   skipbelow=8pt,
   linecolor=blue,
   innerbottommargin=10pt,
   % backgroundcolor=bg,font=\color{orange}\sffamily, fontcolor=white
]{definition}

\usepackage{empheq}
\usepackage[most]{tcolorbox}

\newtcbox{\mymath}[1][]{%
    nobeforeafter, math upper, tcbox raise base,
    enhanced, colframe=blue!30!black,
    colback=red!10, boxrule=1pt,
    #1}

\usepackage{unixode}


\DeclareMathOperator{\ord}{ord}
\DeclareMathOperator{\orb}{orb}
\DeclareMathOperator{\stab}{stab}
\DeclareMathOperator{\Stab}{stab}
\DeclareMathOperator{\ppcm}{ppcm}
\DeclareMathOperator{\conj}{Conj}
\DeclareMathOperator{\End}{End}
\DeclareMathOperator{\rot}{rot}
\DeclareMathOperator{\trs}{trace}
\DeclareMathOperator{\Ind}{Ind}
\DeclareMathOperator{\mat}{Mat}
\DeclareMathOperator{\id}{Id}
\DeclareMathOperator{\vect}{vect}
\DeclareMathOperator{\img}{img}
\DeclareMathOperator{\cov}{Cov}
\DeclareMathOperator{\dist}{dist}
\DeclareMathOperator{\irr}{Irr}
\DeclareMathOperator{\image}{Im}
\DeclareMathOperator{\pd}{\partial}
\DeclareMathOperator{\epi}{epi}
\DeclareMathOperator{\Argmin}{Argmin}
\DeclareMathOperator{\dom}{dom}
\DeclareMathOperator{\proj}{proj}
\DeclareMathOperator{\ctg}{ctg}
\DeclareMathOperator{\supp}{supp}
\DeclareMathOperator{\argmin}{argmin}
\DeclareMathOperator{\mult}{mult}
\DeclareMathOperator{\ch}{ch}
\DeclareMathOperator{\sh}{sh}
\DeclareMathOperator{\rang}{rang}
\DeclareMathOperator{\diam}{diam}
\DeclareMathOperator{\Epigraphe}{Epigraphe}




\usepackage{xcolor}
\everymath{\color{blue}}
%\everymath{\color[rgb]{0,1,1}}
%\pagecolor[rgb]{0,0,0.5}


\newcommand*{\pdtest}[3][]{\ensuremath{\frac{\partial^{#1} #2}{\partial #3}}}

\newcommand*{\deffunc}[6][]{\ensuremath{
\begin{array}{rcl}
#2 : #3 &\rightarrow& #4\\
#5 &\mapsto& #6
\end{array}
}}

\newcommand{\eqcolon}{\mathrel{\resizebox{\widthof{$\mathord{=}$}}{\height}{ $\!\!=\!\!\resizebox{1.2\width}{0.8\height}{\raisebox{0.23ex}{$\mathop{:}$}}\!\!$ }}}
\newcommand{\coloneq}{\mathrel{\resizebox{\widthof{$\mathord{=}$}}{\height}{ $\!\!\resizebox{1.2\width}{0.8\height}{\raisebox{0.23ex}{$\mathop{:}$}}\!\!=\!\!$ }}}
\newcommand{\eqcolonl}{\ensuremath{\mathrel{=\!\!\mathop{:}}}}
\newcommand{\coloneql}{\ensuremath{\mathrel{\mathop{:} \!\! =}}}
\newcommand{\vc}[1]{% inline column vector
  \left(\begin{smallmatrix}#1\end{smallmatrix}\right)%
}
\newcommand{\vr}[1]{% inline row vector
  \begin{smallmatrix}(\,#1\,)\end{smallmatrix}%
}
\makeatletter
\newcommand*{\defeq}{\ =\mathrel{\rlap{%
                     \raisebox{0.3ex}{$\m@th\cdot$}}%
                     \raisebox{-0.3ex}{$\m@th\cdot$}}%
                     }
\makeatother

\newcommand{\mathcircle}[1]{% inline row vector
 \overset{\circ}{#1}
}
\newcommand{\ulim}{% low limit
 \underline{\lim}
}
\newcommand{\ssi}{% iff
\iff
}
\newcommand{\ps}[2]{
\expval{#1 | #2}
}
\newcommand{\df}[1]{
\mqty{#1}
}
\newcommand{\n}[1]{
\norm{#1}
}
\newcommand{\sys}[1]{
\left\{\smqty{#1}\right.
}


\newcommand{\eqdef}{\ensuremath{\overset{\text{def}}=}}


\def\Circlearrowright{\ensuremath{%
  \rotatebox[origin=c]{230}{$\circlearrowright$}}}

\newcommand\ct[1]{\text{\rmfamily\upshape #1}}
\newcommand\question[1]{ {\color{red} ...!? \small #1}}
\newcommand\caz[1]{\left\{\begin{array} #1 \end{array}\right.}
\newcommand\const{\text{\rmfamily\upshape const}}
\newcommand\toP{ \overset{\pro}{\to}}
\newcommand\toPP{ \overset{\text{PP}}{\to}}
\newcommand{\oeq}{\mathrel{\text{\textcircled{$=$}}}}





\usepackage{xcolor}
% \usepackage[normalem]{ulem}
\usepackage{lipsum}
\makeatletter
% \newcommand\colorwave[1][blue]{\bgroup \markoverwith{\lower3.5\p@\hbox{\sixly \textcolor{#1}{\char58}}}\ULon}
%\font\sixly=lasy6 % does not re-load if already loaded, so no memory problem.

\newmdtheoremenv[
linewidth= 1pt,linecolor= blue,%
leftmargin=20,rightmargin=20,innertopmargin=0pt, innerrightmargin=40,%
tikzsetting = { draw=lightgray, line width = 0.3pt,dashed,%
dash pattern = on 15pt off 3pt},%
splittopskip=\topskip,skipbelow=\baselineskip,%
skipabove=\baselineskip,ntheorem,roundcorner=0pt,
% backgroundcolor=pagebg,font=\color{orange}\sffamily, fontcolor=white
]{examplebox}{Exemple}[section]



\newcommand\R{\mathbb{R}}
\newcommand\Z{\mathbb{Z}}
\newcommand\N{\mathbb{N}}
\newcommand\E{\mathbb{E}}
\newcommand\F{\mathcal{F}}
\newcommand\cH{\mathcal{H}}
\newcommand\V{\mathbb{V}}
\newcommand\dmo{ ^{-1} }
\newcommand\kapa{\kappa}
\newcommand\im{Im}
\newcommand\hs{\mathcal{H}}





\usepackage{soul}

\makeatletter
\newcommand*{\whiten}[1]{\llap{\textcolor{white}{{\the\SOUL@token}}\hspace{#1pt}}}
\DeclareRobustCommand*\myul{%
    \def\SOUL@everyspace{\underline{\space}\kern\z@}%
    \def\SOUL@everytoken{%
     \setbox0=\hbox{\the\SOUL@token}%
     \ifdim\dp0>\z@
        \raisebox{\dp0}{\underline{\phantom{\the\SOUL@token}}}%
        \whiten{1}\whiten{0}%
        \whiten{-1}\whiten{-2}%
        \llap{\the\SOUL@token}%
     \else
        \underline{\the\SOUL@token}%
     \fi}%
\SOUL@}
\makeatother

\newcommand*{\demp}{\fontfamily{lmtt}\selectfont}

\DeclareTextFontCommand{\textdemp}{\demp}

\begin{document}

\ifcomment
Multiline
comment
\fi
\ifcomment
\myul{Typesetting test}
% \color[rgb]{1,1,1}
$∑_i^n≠ 60º±∞π∆¬≈√j∫h≤≥µ$

$\CR \R\pro\ind\pro\gS\pro
\mqty[a&b\\c&d]$
$\pro\mathbb{P}$
$\dd{x}$

  \[
    \alpha(x)=\left\{
                \begin{array}{ll}
                  x\\
                  \frac{1}{1+e^{-kx}}\\
                  \frac{e^x-e^{-x}}{e^x+e^{-x}}
                \end{array}
              \right.
  \]

  $\expval{x}$
  
  $\chi_\rho(ghg\dmo)=\Tr(\rho_{ghg\dmo})=\Tr(\rho_g\circ\rho_h\circ\rho\dmo_g)=\Tr(\rho_h)\overset{\mbox{\scalebox{0.5}{$\Tr(AB)=\Tr(BA)$}}}{=}\chi_\rho(h)$
  	$\mathop{\oplus}_{\substack{x\in X}}$

$\mat(\rho_g)=(a_{ij}(g))_{\scriptsize \substack{1\leq i\leq d \\ 1\leq j\leq d}}$ et $\mat(\rho'_g)=(a'_{ij}(g))_{\scriptsize \substack{1\leq i'\leq d' \\ 1\leq j'\leq d'}}$



\[\int_a^b{\mathbb{R}^2}g(u, v)\dd{P_{XY}}(u, v)=\iint g(u,v) f_{XY}(u, v)\dd \lambda(u) \dd \lambda(v)\]
$$\lim_{x\to\infty} f(x)$$	
$$\iiiint_V \mu(t,u,v,w) \,dt\,du\,dv\,dw$$
$$\sum_{n=1}^{\infty} 2^{-n} = 1$$	
\begin{definition}
	Si $X$ et $Y$ sont 2 v.a. ou definit la \textsc{Covariance} entre $X$ et $Y$ comme
	$\cov(X,Y)\overset{\text{def}}{=}\E\left[(X-\E(X))(Y-\E(Y))\right]=\E(XY)-\E(X)\E(Y)$.
\end{definition}
\fi
\pagebreak

% \tableofcontents

% insert your code here
%\input{./algebra/main.tex}
%\input{./geometrie-differentielle/main.tex}
%\input{./probabilite/main.tex}
%\input{./analyse-fonctionnelle/main.tex}
% \input{./Analyse-convexe-et-dualite-en-optimisation/main.tex}
%\input{./tikz/main.tex}
%\input{./Theorie-du-distributions/main.tex}
%\input{./optimisation/mine.tex}
 \input{./modelisation/main.tex}

% yves.aubry@univ-tln.fr : algebra

\end{document}

% % !TEX encoding = UTF-8 Unicode
% !TEX TS-program = xelatex

\documentclass[french]{report}

%\usepackage[utf8]{inputenc}
%\usepackage[T1]{fontenc}
\usepackage{babel}


\newif\ifcomment
%\commenttrue # Show comments

\usepackage{physics}
\usepackage{amssymb}


\usepackage{amsthm}
% \usepackage{thmtools}
\usepackage{mathtools}
\usepackage{amsfonts}

\usepackage{color}

\usepackage{tikz}

\usepackage{geometry}
\geometry{a5paper, margin=0.1in, right=1cm}

\usepackage{dsfont}

\usepackage{graphicx}
\graphicspath{ {images/} }

\usepackage{faktor}

\usepackage{IEEEtrantools}
\usepackage{enumerate}   
\usepackage[PostScript=dvips]{"/Users/aware/Documents/Courses/diagrams"}


\newtheorem{theorem}{Théorème}[section]
\renewcommand{\thetheorem}{\arabic{theorem}}
\newtheorem{lemme}{Lemme}[section]
\renewcommand{\thelemme}{\arabic{lemme}}
\newtheorem{proposition}{Proposition}[section]
\renewcommand{\theproposition}{\arabic{proposition}}
\newtheorem{notations}{Notations}[section]
\newtheorem{problem}{Problème}[section]
\newtheorem{corollary}{Corollaire}[theorem]
\renewcommand{\thecorollary}{\arabic{corollary}}
\newtheorem{property}{Propriété}[section]
\newtheorem{objective}{Objectif}[section]

\theoremstyle{definition}
\newtheorem{definition}{Définition}[section]
\renewcommand{\thedefinition}{\arabic{definition}}
\newtheorem{exercise}{Exercice}[chapter]
\renewcommand{\theexercise}{\arabic{exercise}}
\newtheorem{example}{Exemple}[chapter]
\renewcommand{\theexample}{\arabic{example}}
\newtheorem*{solution}{Solution}
\newtheorem*{application}{Application}
\newtheorem*{notation}{Notation}
\newtheorem*{vocabulary}{Vocabulaire}
\newtheorem*{properties}{Propriétés}



\theoremstyle{remark}
\newtheorem*{remark}{Remarque}
\newtheorem*{rappel}{Rappel}


\usepackage{etoolbox}
\AtBeginEnvironment{exercise}{\small}
\AtBeginEnvironment{example}{\small}

\usepackage{cases}
\usepackage[red]{mypack}

\usepackage[framemethod=TikZ]{mdframed}

\definecolor{bg}{rgb}{0.4,0.25,0.95}
\definecolor{pagebg}{rgb}{0,0,0.5}
\surroundwithmdframed[
   topline=false,
   rightline=false,
   bottomline=false,
   leftmargin=\parindent,
   skipabove=8pt,
   skipbelow=8pt,
   linecolor=blue,
   innerbottommargin=10pt,
   % backgroundcolor=bg,font=\color{orange}\sffamily, fontcolor=white
]{definition}

\usepackage{empheq}
\usepackage[most]{tcolorbox}

\newtcbox{\mymath}[1][]{%
    nobeforeafter, math upper, tcbox raise base,
    enhanced, colframe=blue!30!black,
    colback=red!10, boxrule=1pt,
    #1}

\usepackage{unixode}


\DeclareMathOperator{\ord}{ord}
\DeclareMathOperator{\orb}{orb}
\DeclareMathOperator{\stab}{stab}
\DeclareMathOperator{\Stab}{stab}
\DeclareMathOperator{\ppcm}{ppcm}
\DeclareMathOperator{\conj}{Conj}
\DeclareMathOperator{\End}{End}
\DeclareMathOperator{\rot}{rot}
\DeclareMathOperator{\trs}{trace}
\DeclareMathOperator{\Ind}{Ind}
\DeclareMathOperator{\mat}{Mat}
\DeclareMathOperator{\id}{Id}
\DeclareMathOperator{\vect}{vect}
\DeclareMathOperator{\img}{img}
\DeclareMathOperator{\cov}{Cov}
\DeclareMathOperator{\dist}{dist}
\DeclareMathOperator{\irr}{Irr}
\DeclareMathOperator{\image}{Im}
\DeclareMathOperator{\pd}{\partial}
\DeclareMathOperator{\epi}{epi}
\DeclareMathOperator{\Argmin}{Argmin}
\DeclareMathOperator{\dom}{dom}
\DeclareMathOperator{\proj}{proj}
\DeclareMathOperator{\ctg}{ctg}
\DeclareMathOperator{\supp}{supp}
\DeclareMathOperator{\argmin}{argmin}
\DeclareMathOperator{\mult}{mult}
\DeclareMathOperator{\ch}{ch}
\DeclareMathOperator{\sh}{sh}
\DeclareMathOperator{\rang}{rang}
\DeclareMathOperator{\diam}{diam}
\DeclareMathOperator{\Epigraphe}{Epigraphe}




\usepackage{xcolor}
\everymath{\color{blue}}
%\everymath{\color[rgb]{0,1,1}}
%\pagecolor[rgb]{0,0,0.5}


\newcommand*{\pdtest}[3][]{\ensuremath{\frac{\partial^{#1} #2}{\partial #3}}}

\newcommand*{\deffunc}[6][]{\ensuremath{
\begin{array}{rcl}
#2 : #3 &\rightarrow& #4\\
#5 &\mapsto& #6
\end{array}
}}

\newcommand{\eqcolon}{\mathrel{\resizebox{\widthof{$\mathord{=}$}}{\height}{ $\!\!=\!\!\resizebox{1.2\width}{0.8\height}{\raisebox{0.23ex}{$\mathop{:}$}}\!\!$ }}}
\newcommand{\coloneq}{\mathrel{\resizebox{\widthof{$\mathord{=}$}}{\height}{ $\!\!\resizebox{1.2\width}{0.8\height}{\raisebox{0.23ex}{$\mathop{:}$}}\!\!=\!\!$ }}}
\newcommand{\eqcolonl}{\ensuremath{\mathrel{=\!\!\mathop{:}}}}
\newcommand{\coloneql}{\ensuremath{\mathrel{\mathop{:} \!\! =}}}
\newcommand{\vc}[1]{% inline column vector
  \left(\begin{smallmatrix}#1\end{smallmatrix}\right)%
}
\newcommand{\vr}[1]{% inline row vector
  \begin{smallmatrix}(\,#1\,)\end{smallmatrix}%
}
\makeatletter
\newcommand*{\defeq}{\ =\mathrel{\rlap{%
                     \raisebox{0.3ex}{$\m@th\cdot$}}%
                     \raisebox{-0.3ex}{$\m@th\cdot$}}%
                     }
\makeatother

\newcommand{\mathcircle}[1]{% inline row vector
 \overset{\circ}{#1}
}
\newcommand{\ulim}{% low limit
 \underline{\lim}
}
\newcommand{\ssi}{% iff
\iff
}
\newcommand{\ps}[2]{
\expval{#1 | #2}
}
\newcommand{\df}[1]{
\mqty{#1}
}
\newcommand{\n}[1]{
\norm{#1}
}
\newcommand{\sys}[1]{
\left\{\smqty{#1}\right.
}


\newcommand{\eqdef}{\ensuremath{\overset{\text{def}}=}}


\def\Circlearrowright{\ensuremath{%
  \rotatebox[origin=c]{230}{$\circlearrowright$}}}

\newcommand\ct[1]{\text{\rmfamily\upshape #1}}
\newcommand\question[1]{ {\color{red} ...!? \small #1}}
\newcommand\caz[1]{\left\{\begin{array} #1 \end{array}\right.}
\newcommand\const{\text{\rmfamily\upshape const}}
\newcommand\toP{ \overset{\pro}{\to}}
\newcommand\toPP{ \overset{\text{PP}}{\to}}
\newcommand{\oeq}{\mathrel{\text{\textcircled{$=$}}}}





\usepackage{xcolor}
% \usepackage[normalem]{ulem}
\usepackage{lipsum}
\makeatletter
% \newcommand\colorwave[1][blue]{\bgroup \markoverwith{\lower3.5\p@\hbox{\sixly \textcolor{#1}{\char58}}}\ULon}
%\font\sixly=lasy6 % does not re-load if already loaded, so no memory problem.

\newmdtheoremenv[
linewidth= 1pt,linecolor= blue,%
leftmargin=20,rightmargin=20,innertopmargin=0pt, innerrightmargin=40,%
tikzsetting = { draw=lightgray, line width = 0.3pt,dashed,%
dash pattern = on 15pt off 3pt},%
splittopskip=\topskip,skipbelow=\baselineskip,%
skipabove=\baselineskip,ntheorem,roundcorner=0pt,
% backgroundcolor=pagebg,font=\color{orange}\sffamily, fontcolor=white
]{examplebox}{Exemple}[section]



\newcommand\R{\mathbb{R}}
\newcommand\Z{\mathbb{Z}}
\newcommand\N{\mathbb{N}}
\newcommand\E{\mathbb{E}}
\newcommand\F{\mathcal{F}}
\newcommand\cH{\mathcal{H}}
\newcommand\V{\mathbb{V}}
\newcommand\dmo{ ^{-1} }
\newcommand\kapa{\kappa}
\newcommand\im{Im}
\newcommand\hs{\mathcal{H}}





\usepackage{soul}

\makeatletter
\newcommand*{\whiten}[1]{\llap{\textcolor{white}{{\the\SOUL@token}}\hspace{#1pt}}}
\DeclareRobustCommand*\myul{%
    \def\SOUL@everyspace{\underline{\space}\kern\z@}%
    \def\SOUL@everytoken{%
     \setbox0=\hbox{\the\SOUL@token}%
     \ifdim\dp0>\z@
        \raisebox{\dp0}{\underline{\phantom{\the\SOUL@token}}}%
        \whiten{1}\whiten{0}%
        \whiten{-1}\whiten{-2}%
        \llap{\the\SOUL@token}%
     \else
        \underline{\the\SOUL@token}%
     \fi}%
\SOUL@}
\makeatother

\newcommand*{\demp}{\fontfamily{lmtt}\selectfont}

\DeclareTextFontCommand{\textdemp}{\demp}

\begin{document}

\ifcomment
Multiline
comment
\fi
\ifcomment
\myul{Typesetting test}
% \color[rgb]{1,1,1}
$∑_i^n≠ 60º±∞π∆¬≈√j∫h≤≥µ$

$\CR \R\pro\ind\pro\gS\pro
\mqty[a&b\\c&d]$
$\pro\mathbb{P}$
$\dd{x}$

  \[
    \alpha(x)=\left\{
                \begin{array}{ll}
                  x\\
                  \frac{1}{1+e^{-kx}}\\
                  \frac{e^x-e^{-x}}{e^x+e^{-x}}
                \end{array}
              \right.
  \]

  $\expval{x}$
  
  $\chi_\rho(ghg\dmo)=\Tr(\rho_{ghg\dmo})=\Tr(\rho_g\circ\rho_h\circ\rho\dmo_g)=\Tr(\rho_h)\overset{\mbox{\scalebox{0.5}{$\Tr(AB)=\Tr(BA)$}}}{=}\chi_\rho(h)$
  	$\mathop{\oplus}_{\substack{x\in X}}$

$\mat(\rho_g)=(a_{ij}(g))_{\scriptsize \substack{1\leq i\leq d \\ 1\leq j\leq d}}$ et $\mat(\rho'_g)=(a'_{ij}(g))_{\scriptsize \substack{1\leq i'\leq d' \\ 1\leq j'\leq d'}}$



\[\int_a^b{\mathbb{R}^2}g(u, v)\dd{P_{XY}}(u, v)=\iint g(u,v) f_{XY}(u, v)\dd \lambda(u) \dd \lambda(v)\]
$$\lim_{x\to\infty} f(x)$$	
$$\iiiint_V \mu(t,u,v,w) \,dt\,du\,dv\,dw$$
$$\sum_{n=1}^{\infty} 2^{-n} = 1$$	
\begin{definition}
	Si $X$ et $Y$ sont 2 v.a. ou definit la \textsc{Covariance} entre $X$ et $Y$ comme
	$\cov(X,Y)\overset{\text{def}}{=}\E\left[(X-\E(X))(Y-\E(Y))\right]=\E(XY)-\E(X)\E(Y)$.
\end{definition}
\fi
\pagebreak

% \tableofcontents

% insert your code here
%\input{./algebra/main.tex}
%\input{./geometrie-differentielle/main.tex}
%\input{./probabilite/main.tex}
%\input{./analyse-fonctionnelle/main.tex}
% \input{./Analyse-convexe-et-dualite-en-optimisation/main.tex}
%\input{./tikz/main.tex}
%\input{./Theorie-du-distributions/main.tex}
%\input{./optimisation/mine.tex}
 \input{./modelisation/main.tex}

% yves.aubry@univ-tln.fr : algebra

\end{document}

%% !TEX encoding = UTF-8 Unicode
% !TEX TS-program = xelatex

\documentclass[french]{report}

%\usepackage[utf8]{inputenc}
%\usepackage[T1]{fontenc}
\usepackage{babel}


\newif\ifcomment
%\commenttrue # Show comments

\usepackage{physics}
\usepackage{amssymb}


\usepackage{amsthm}
% \usepackage{thmtools}
\usepackage{mathtools}
\usepackage{amsfonts}

\usepackage{color}

\usepackage{tikz}

\usepackage{geometry}
\geometry{a5paper, margin=0.1in, right=1cm}

\usepackage{dsfont}

\usepackage{graphicx}
\graphicspath{ {images/} }

\usepackage{faktor}

\usepackage{IEEEtrantools}
\usepackage{enumerate}   
\usepackage[PostScript=dvips]{"/Users/aware/Documents/Courses/diagrams"}


\newtheorem{theorem}{Théorème}[section]
\renewcommand{\thetheorem}{\arabic{theorem}}
\newtheorem{lemme}{Lemme}[section]
\renewcommand{\thelemme}{\arabic{lemme}}
\newtheorem{proposition}{Proposition}[section]
\renewcommand{\theproposition}{\arabic{proposition}}
\newtheorem{notations}{Notations}[section]
\newtheorem{problem}{Problème}[section]
\newtheorem{corollary}{Corollaire}[theorem]
\renewcommand{\thecorollary}{\arabic{corollary}}
\newtheorem{property}{Propriété}[section]
\newtheorem{objective}{Objectif}[section]

\theoremstyle{definition}
\newtheorem{definition}{Définition}[section]
\renewcommand{\thedefinition}{\arabic{definition}}
\newtheorem{exercise}{Exercice}[chapter]
\renewcommand{\theexercise}{\arabic{exercise}}
\newtheorem{example}{Exemple}[chapter]
\renewcommand{\theexample}{\arabic{example}}
\newtheorem*{solution}{Solution}
\newtheorem*{application}{Application}
\newtheorem*{notation}{Notation}
\newtheorem*{vocabulary}{Vocabulaire}
\newtheorem*{properties}{Propriétés}



\theoremstyle{remark}
\newtheorem*{remark}{Remarque}
\newtheorem*{rappel}{Rappel}


\usepackage{etoolbox}
\AtBeginEnvironment{exercise}{\small}
\AtBeginEnvironment{example}{\small}

\usepackage{cases}
\usepackage[red]{mypack}

\usepackage[framemethod=TikZ]{mdframed}

\definecolor{bg}{rgb}{0.4,0.25,0.95}
\definecolor{pagebg}{rgb}{0,0,0.5}
\surroundwithmdframed[
   topline=false,
   rightline=false,
   bottomline=false,
   leftmargin=\parindent,
   skipabove=8pt,
   skipbelow=8pt,
   linecolor=blue,
   innerbottommargin=10pt,
   % backgroundcolor=bg,font=\color{orange}\sffamily, fontcolor=white
]{definition}

\usepackage{empheq}
\usepackage[most]{tcolorbox}

\newtcbox{\mymath}[1][]{%
    nobeforeafter, math upper, tcbox raise base,
    enhanced, colframe=blue!30!black,
    colback=red!10, boxrule=1pt,
    #1}

\usepackage{unixode}


\DeclareMathOperator{\ord}{ord}
\DeclareMathOperator{\orb}{orb}
\DeclareMathOperator{\stab}{stab}
\DeclareMathOperator{\Stab}{stab}
\DeclareMathOperator{\ppcm}{ppcm}
\DeclareMathOperator{\conj}{Conj}
\DeclareMathOperator{\End}{End}
\DeclareMathOperator{\rot}{rot}
\DeclareMathOperator{\trs}{trace}
\DeclareMathOperator{\Ind}{Ind}
\DeclareMathOperator{\mat}{Mat}
\DeclareMathOperator{\id}{Id}
\DeclareMathOperator{\vect}{vect}
\DeclareMathOperator{\img}{img}
\DeclareMathOperator{\cov}{Cov}
\DeclareMathOperator{\dist}{dist}
\DeclareMathOperator{\irr}{Irr}
\DeclareMathOperator{\image}{Im}
\DeclareMathOperator{\pd}{\partial}
\DeclareMathOperator{\epi}{epi}
\DeclareMathOperator{\Argmin}{Argmin}
\DeclareMathOperator{\dom}{dom}
\DeclareMathOperator{\proj}{proj}
\DeclareMathOperator{\ctg}{ctg}
\DeclareMathOperator{\supp}{supp}
\DeclareMathOperator{\argmin}{argmin}
\DeclareMathOperator{\mult}{mult}
\DeclareMathOperator{\ch}{ch}
\DeclareMathOperator{\sh}{sh}
\DeclareMathOperator{\rang}{rang}
\DeclareMathOperator{\diam}{diam}
\DeclareMathOperator{\Epigraphe}{Epigraphe}




\usepackage{xcolor}
\everymath{\color{blue}}
%\everymath{\color[rgb]{0,1,1}}
%\pagecolor[rgb]{0,0,0.5}


\newcommand*{\pdtest}[3][]{\ensuremath{\frac{\partial^{#1} #2}{\partial #3}}}

\newcommand*{\deffunc}[6][]{\ensuremath{
\begin{array}{rcl}
#2 : #3 &\rightarrow& #4\\
#5 &\mapsto& #6
\end{array}
}}

\newcommand{\eqcolon}{\mathrel{\resizebox{\widthof{$\mathord{=}$}}{\height}{ $\!\!=\!\!\resizebox{1.2\width}{0.8\height}{\raisebox{0.23ex}{$\mathop{:}$}}\!\!$ }}}
\newcommand{\coloneq}{\mathrel{\resizebox{\widthof{$\mathord{=}$}}{\height}{ $\!\!\resizebox{1.2\width}{0.8\height}{\raisebox{0.23ex}{$\mathop{:}$}}\!\!=\!\!$ }}}
\newcommand{\eqcolonl}{\ensuremath{\mathrel{=\!\!\mathop{:}}}}
\newcommand{\coloneql}{\ensuremath{\mathrel{\mathop{:} \!\! =}}}
\newcommand{\vc}[1]{% inline column vector
  \left(\begin{smallmatrix}#1\end{smallmatrix}\right)%
}
\newcommand{\vr}[1]{% inline row vector
  \begin{smallmatrix}(\,#1\,)\end{smallmatrix}%
}
\makeatletter
\newcommand*{\defeq}{\ =\mathrel{\rlap{%
                     \raisebox{0.3ex}{$\m@th\cdot$}}%
                     \raisebox{-0.3ex}{$\m@th\cdot$}}%
                     }
\makeatother

\newcommand{\mathcircle}[1]{% inline row vector
 \overset{\circ}{#1}
}
\newcommand{\ulim}{% low limit
 \underline{\lim}
}
\newcommand{\ssi}{% iff
\iff
}
\newcommand{\ps}[2]{
\expval{#1 | #2}
}
\newcommand{\df}[1]{
\mqty{#1}
}
\newcommand{\n}[1]{
\norm{#1}
}
\newcommand{\sys}[1]{
\left\{\smqty{#1}\right.
}


\newcommand{\eqdef}{\ensuremath{\overset{\text{def}}=}}


\def\Circlearrowright{\ensuremath{%
  \rotatebox[origin=c]{230}{$\circlearrowright$}}}

\newcommand\ct[1]{\text{\rmfamily\upshape #1}}
\newcommand\question[1]{ {\color{red} ...!? \small #1}}
\newcommand\caz[1]{\left\{\begin{array} #1 \end{array}\right.}
\newcommand\const{\text{\rmfamily\upshape const}}
\newcommand\toP{ \overset{\pro}{\to}}
\newcommand\toPP{ \overset{\text{PP}}{\to}}
\newcommand{\oeq}{\mathrel{\text{\textcircled{$=$}}}}





\usepackage{xcolor}
% \usepackage[normalem]{ulem}
\usepackage{lipsum}
\makeatletter
% \newcommand\colorwave[1][blue]{\bgroup \markoverwith{\lower3.5\p@\hbox{\sixly \textcolor{#1}{\char58}}}\ULon}
%\font\sixly=lasy6 % does not re-load if already loaded, so no memory problem.

\newmdtheoremenv[
linewidth= 1pt,linecolor= blue,%
leftmargin=20,rightmargin=20,innertopmargin=0pt, innerrightmargin=40,%
tikzsetting = { draw=lightgray, line width = 0.3pt,dashed,%
dash pattern = on 15pt off 3pt},%
splittopskip=\topskip,skipbelow=\baselineskip,%
skipabove=\baselineskip,ntheorem,roundcorner=0pt,
% backgroundcolor=pagebg,font=\color{orange}\sffamily, fontcolor=white
]{examplebox}{Exemple}[section]



\newcommand\R{\mathbb{R}}
\newcommand\Z{\mathbb{Z}}
\newcommand\N{\mathbb{N}}
\newcommand\E{\mathbb{E}}
\newcommand\F{\mathcal{F}}
\newcommand\cH{\mathcal{H}}
\newcommand\V{\mathbb{V}}
\newcommand\dmo{ ^{-1} }
\newcommand\kapa{\kappa}
\newcommand\im{Im}
\newcommand\hs{\mathcal{H}}





\usepackage{soul}

\makeatletter
\newcommand*{\whiten}[1]{\llap{\textcolor{white}{{\the\SOUL@token}}\hspace{#1pt}}}
\DeclareRobustCommand*\myul{%
    \def\SOUL@everyspace{\underline{\space}\kern\z@}%
    \def\SOUL@everytoken{%
     \setbox0=\hbox{\the\SOUL@token}%
     \ifdim\dp0>\z@
        \raisebox{\dp0}{\underline{\phantom{\the\SOUL@token}}}%
        \whiten{1}\whiten{0}%
        \whiten{-1}\whiten{-2}%
        \llap{\the\SOUL@token}%
     \else
        \underline{\the\SOUL@token}%
     \fi}%
\SOUL@}
\makeatother

\newcommand*{\demp}{\fontfamily{lmtt}\selectfont}

\DeclareTextFontCommand{\textdemp}{\demp}

\begin{document}

\ifcomment
Multiline
comment
\fi
\ifcomment
\myul{Typesetting test}
% \color[rgb]{1,1,1}
$∑_i^n≠ 60º±∞π∆¬≈√j∫h≤≥µ$

$\CR \R\pro\ind\pro\gS\pro
\mqty[a&b\\c&d]$
$\pro\mathbb{P}$
$\dd{x}$

  \[
    \alpha(x)=\left\{
                \begin{array}{ll}
                  x\\
                  \frac{1}{1+e^{-kx}}\\
                  \frac{e^x-e^{-x}}{e^x+e^{-x}}
                \end{array}
              \right.
  \]

  $\expval{x}$
  
  $\chi_\rho(ghg\dmo)=\Tr(\rho_{ghg\dmo})=\Tr(\rho_g\circ\rho_h\circ\rho\dmo_g)=\Tr(\rho_h)\overset{\mbox{\scalebox{0.5}{$\Tr(AB)=\Tr(BA)$}}}{=}\chi_\rho(h)$
  	$\mathop{\oplus}_{\substack{x\in X}}$

$\mat(\rho_g)=(a_{ij}(g))_{\scriptsize \substack{1\leq i\leq d \\ 1\leq j\leq d}}$ et $\mat(\rho'_g)=(a'_{ij}(g))_{\scriptsize \substack{1\leq i'\leq d' \\ 1\leq j'\leq d'}}$



\[\int_a^b{\mathbb{R}^2}g(u, v)\dd{P_{XY}}(u, v)=\iint g(u,v) f_{XY}(u, v)\dd \lambda(u) \dd \lambda(v)\]
$$\lim_{x\to\infty} f(x)$$	
$$\iiiint_V \mu(t,u,v,w) \,dt\,du\,dv\,dw$$
$$\sum_{n=1}^{\infty} 2^{-n} = 1$$	
\begin{definition}
	Si $X$ et $Y$ sont 2 v.a. ou definit la \textsc{Covariance} entre $X$ et $Y$ comme
	$\cov(X,Y)\overset{\text{def}}{=}\E\left[(X-\E(X))(Y-\E(Y))\right]=\E(XY)-\E(X)\E(Y)$.
\end{definition}
\fi
\pagebreak

% \tableofcontents

% insert your code here
%\input{./algebra/main.tex}
%\input{./geometrie-differentielle/main.tex}
%\input{./probabilite/main.tex}
%\input{./analyse-fonctionnelle/main.tex}
% \input{./Analyse-convexe-et-dualite-en-optimisation/main.tex}
%\input{./tikz/main.tex}
%\input{./Theorie-du-distributions/main.tex}
%\input{./optimisation/mine.tex}
 \input{./modelisation/main.tex}

% yves.aubry@univ-tln.fr : algebra

\end{document}

%% !TEX encoding = UTF-8 Unicode
% !TEX TS-program = xelatex

\documentclass[french]{report}

%\usepackage[utf8]{inputenc}
%\usepackage[T1]{fontenc}
\usepackage{babel}


\newif\ifcomment
%\commenttrue # Show comments

\usepackage{physics}
\usepackage{amssymb}


\usepackage{amsthm}
% \usepackage{thmtools}
\usepackage{mathtools}
\usepackage{amsfonts}

\usepackage{color}

\usepackage{tikz}

\usepackage{geometry}
\geometry{a5paper, margin=0.1in, right=1cm}

\usepackage{dsfont}

\usepackage{graphicx}
\graphicspath{ {images/} }

\usepackage{faktor}

\usepackage{IEEEtrantools}
\usepackage{enumerate}   
\usepackage[PostScript=dvips]{"/Users/aware/Documents/Courses/diagrams"}


\newtheorem{theorem}{Théorème}[section]
\renewcommand{\thetheorem}{\arabic{theorem}}
\newtheorem{lemme}{Lemme}[section]
\renewcommand{\thelemme}{\arabic{lemme}}
\newtheorem{proposition}{Proposition}[section]
\renewcommand{\theproposition}{\arabic{proposition}}
\newtheorem{notations}{Notations}[section]
\newtheorem{problem}{Problème}[section]
\newtheorem{corollary}{Corollaire}[theorem]
\renewcommand{\thecorollary}{\arabic{corollary}}
\newtheorem{property}{Propriété}[section]
\newtheorem{objective}{Objectif}[section]

\theoremstyle{definition}
\newtheorem{definition}{Définition}[section]
\renewcommand{\thedefinition}{\arabic{definition}}
\newtheorem{exercise}{Exercice}[chapter]
\renewcommand{\theexercise}{\arabic{exercise}}
\newtheorem{example}{Exemple}[chapter]
\renewcommand{\theexample}{\arabic{example}}
\newtheorem*{solution}{Solution}
\newtheorem*{application}{Application}
\newtheorem*{notation}{Notation}
\newtheorem*{vocabulary}{Vocabulaire}
\newtheorem*{properties}{Propriétés}



\theoremstyle{remark}
\newtheorem*{remark}{Remarque}
\newtheorem*{rappel}{Rappel}


\usepackage{etoolbox}
\AtBeginEnvironment{exercise}{\small}
\AtBeginEnvironment{example}{\small}

\usepackage{cases}
\usepackage[red]{mypack}

\usepackage[framemethod=TikZ]{mdframed}

\definecolor{bg}{rgb}{0.4,0.25,0.95}
\definecolor{pagebg}{rgb}{0,0,0.5}
\surroundwithmdframed[
   topline=false,
   rightline=false,
   bottomline=false,
   leftmargin=\parindent,
   skipabove=8pt,
   skipbelow=8pt,
   linecolor=blue,
   innerbottommargin=10pt,
   % backgroundcolor=bg,font=\color{orange}\sffamily, fontcolor=white
]{definition}

\usepackage{empheq}
\usepackage[most]{tcolorbox}

\newtcbox{\mymath}[1][]{%
    nobeforeafter, math upper, tcbox raise base,
    enhanced, colframe=blue!30!black,
    colback=red!10, boxrule=1pt,
    #1}

\usepackage{unixode}


\DeclareMathOperator{\ord}{ord}
\DeclareMathOperator{\orb}{orb}
\DeclareMathOperator{\stab}{stab}
\DeclareMathOperator{\Stab}{stab}
\DeclareMathOperator{\ppcm}{ppcm}
\DeclareMathOperator{\conj}{Conj}
\DeclareMathOperator{\End}{End}
\DeclareMathOperator{\rot}{rot}
\DeclareMathOperator{\trs}{trace}
\DeclareMathOperator{\Ind}{Ind}
\DeclareMathOperator{\mat}{Mat}
\DeclareMathOperator{\id}{Id}
\DeclareMathOperator{\vect}{vect}
\DeclareMathOperator{\img}{img}
\DeclareMathOperator{\cov}{Cov}
\DeclareMathOperator{\dist}{dist}
\DeclareMathOperator{\irr}{Irr}
\DeclareMathOperator{\image}{Im}
\DeclareMathOperator{\pd}{\partial}
\DeclareMathOperator{\epi}{epi}
\DeclareMathOperator{\Argmin}{Argmin}
\DeclareMathOperator{\dom}{dom}
\DeclareMathOperator{\proj}{proj}
\DeclareMathOperator{\ctg}{ctg}
\DeclareMathOperator{\supp}{supp}
\DeclareMathOperator{\argmin}{argmin}
\DeclareMathOperator{\mult}{mult}
\DeclareMathOperator{\ch}{ch}
\DeclareMathOperator{\sh}{sh}
\DeclareMathOperator{\rang}{rang}
\DeclareMathOperator{\diam}{diam}
\DeclareMathOperator{\Epigraphe}{Epigraphe}




\usepackage{xcolor}
\everymath{\color{blue}}
%\everymath{\color[rgb]{0,1,1}}
%\pagecolor[rgb]{0,0,0.5}


\newcommand*{\pdtest}[3][]{\ensuremath{\frac{\partial^{#1} #2}{\partial #3}}}

\newcommand*{\deffunc}[6][]{\ensuremath{
\begin{array}{rcl}
#2 : #3 &\rightarrow& #4\\
#5 &\mapsto& #6
\end{array}
}}

\newcommand{\eqcolon}{\mathrel{\resizebox{\widthof{$\mathord{=}$}}{\height}{ $\!\!=\!\!\resizebox{1.2\width}{0.8\height}{\raisebox{0.23ex}{$\mathop{:}$}}\!\!$ }}}
\newcommand{\coloneq}{\mathrel{\resizebox{\widthof{$\mathord{=}$}}{\height}{ $\!\!\resizebox{1.2\width}{0.8\height}{\raisebox{0.23ex}{$\mathop{:}$}}\!\!=\!\!$ }}}
\newcommand{\eqcolonl}{\ensuremath{\mathrel{=\!\!\mathop{:}}}}
\newcommand{\coloneql}{\ensuremath{\mathrel{\mathop{:} \!\! =}}}
\newcommand{\vc}[1]{% inline column vector
  \left(\begin{smallmatrix}#1\end{smallmatrix}\right)%
}
\newcommand{\vr}[1]{% inline row vector
  \begin{smallmatrix}(\,#1\,)\end{smallmatrix}%
}
\makeatletter
\newcommand*{\defeq}{\ =\mathrel{\rlap{%
                     \raisebox{0.3ex}{$\m@th\cdot$}}%
                     \raisebox{-0.3ex}{$\m@th\cdot$}}%
                     }
\makeatother

\newcommand{\mathcircle}[1]{% inline row vector
 \overset{\circ}{#1}
}
\newcommand{\ulim}{% low limit
 \underline{\lim}
}
\newcommand{\ssi}{% iff
\iff
}
\newcommand{\ps}[2]{
\expval{#1 | #2}
}
\newcommand{\df}[1]{
\mqty{#1}
}
\newcommand{\n}[1]{
\norm{#1}
}
\newcommand{\sys}[1]{
\left\{\smqty{#1}\right.
}


\newcommand{\eqdef}{\ensuremath{\overset{\text{def}}=}}


\def\Circlearrowright{\ensuremath{%
  \rotatebox[origin=c]{230}{$\circlearrowright$}}}

\newcommand\ct[1]{\text{\rmfamily\upshape #1}}
\newcommand\question[1]{ {\color{red} ...!? \small #1}}
\newcommand\caz[1]{\left\{\begin{array} #1 \end{array}\right.}
\newcommand\const{\text{\rmfamily\upshape const}}
\newcommand\toP{ \overset{\pro}{\to}}
\newcommand\toPP{ \overset{\text{PP}}{\to}}
\newcommand{\oeq}{\mathrel{\text{\textcircled{$=$}}}}





\usepackage{xcolor}
% \usepackage[normalem]{ulem}
\usepackage{lipsum}
\makeatletter
% \newcommand\colorwave[1][blue]{\bgroup \markoverwith{\lower3.5\p@\hbox{\sixly \textcolor{#1}{\char58}}}\ULon}
%\font\sixly=lasy6 % does not re-load if already loaded, so no memory problem.

\newmdtheoremenv[
linewidth= 1pt,linecolor= blue,%
leftmargin=20,rightmargin=20,innertopmargin=0pt, innerrightmargin=40,%
tikzsetting = { draw=lightgray, line width = 0.3pt,dashed,%
dash pattern = on 15pt off 3pt},%
splittopskip=\topskip,skipbelow=\baselineskip,%
skipabove=\baselineskip,ntheorem,roundcorner=0pt,
% backgroundcolor=pagebg,font=\color{orange}\sffamily, fontcolor=white
]{examplebox}{Exemple}[section]



\newcommand\R{\mathbb{R}}
\newcommand\Z{\mathbb{Z}}
\newcommand\N{\mathbb{N}}
\newcommand\E{\mathbb{E}}
\newcommand\F{\mathcal{F}}
\newcommand\cH{\mathcal{H}}
\newcommand\V{\mathbb{V}}
\newcommand\dmo{ ^{-1} }
\newcommand\kapa{\kappa}
\newcommand\im{Im}
\newcommand\hs{\mathcal{H}}





\usepackage{soul}

\makeatletter
\newcommand*{\whiten}[1]{\llap{\textcolor{white}{{\the\SOUL@token}}\hspace{#1pt}}}
\DeclareRobustCommand*\myul{%
    \def\SOUL@everyspace{\underline{\space}\kern\z@}%
    \def\SOUL@everytoken{%
     \setbox0=\hbox{\the\SOUL@token}%
     \ifdim\dp0>\z@
        \raisebox{\dp0}{\underline{\phantom{\the\SOUL@token}}}%
        \whiten{1}\whiten{0}%
        \whiten{-1}\whiten{-2}%
        \llap{\the\SOUL@token}%
     \else
        \underline{\the\SOUL@token}%
     \fi}%
\SOUL@}
\makeatother

\newcommand*{\demp}{\fontfamily{lmtt}\selectfont}

\DeclareTextFontCommand{\textdemp}{\demp}

\begin{document}

\ifcomment
Multiline
comment
\fi
\ifcomment
\myul{Typesetting test}
% \color[rgb]{1,1,1}
$∑_i^n≠ 60º±∞π∆¬≈√j∫h≤≥µ$

$\CR \R\pro\ind\pro\gS\pro
\mqty[a&b\\c&d]$
$\pro\mathbb{P}$
$\dd{x}$

  \[
    \alpha(x)=\left\{
                \begin{array}{ll}
                  x\\
                  \frac{1}{1+e^{-kx}}\\
                  \frac{e^x-e^{-x}}{e^x+e^{-x}}
                \end{array}
              \right.
  \]

  $\expval{x}$
  
  $\chi_\rho(ghg\dmo)=\Tr(\rho_{ghg\dmo})=\Tr(\rho_g\circ\rho_h\circ\rho\dmo_g)=\Tr(\rho_h)\overset{\mbox{\scalebox{0.5}{$\Tr(AB)=\Tr(BA)$}}}{=}\chi_\rho(h)$
  	$\mathop{\oplus}_{\substack{x\in X}}$

$\mat(\rho_g)=(a_{ij}(g))_{\scriptsize \substack{1\leq i\leq d \\ 1\leq j\leq d}}$ et $\mat(\rho'_g)=(a'_{ij}(g))_{\scriptsize \substack{1\leq i'\leq d' \\ 1\leq j'\leq d'}}$



\[\int_a^b{\mathbb{R}^2}g(u, v)\dd{P_{XY}}(u, v)=\iint g(u,v) f_{XY}(u, v)\dd \lambda(u) \dd \lambda(v)\]
$$\lim_{x\to\infty} f(x)$$	
$$\iiiint_V \mu(t,u,v,w) \,dt\,du\,dv\,dw$$
$$\sum_{n=1}^{\infty} 2^{-n} = 1$$	
\begin{definition}
	Si $X$ et $Y$ sont 2 v.a. ou definit la \textsc{Covariance} entre $X$ et $Y$ comme
	$\cov(X,Y)\overset{\text{def}}{=}\E\left[(X-\E(X))(Y-\E(Y))\right]=\E(XY)-\E(X)\E(Y)$.
\end{definition}
\fi
\pagebreak

% \tableofcontents

% insert your code here
%\input{./algebra/main.tex}
%\input{./geometrie-differentielle/main.tex}
%\input{./probabilite/main.tex}
%\input{./analyse-fonctionnelle/main.tex}
% \input{./Analyse-convexe-et-dualite-en-optimisation/main.tex}
%\input{./tikz/main.tex}
%\input{./Theorie-du-distributions/main.tex}
%\input{./optimisation/mine.tex}
 \input{./modelisation/main.tex}

% yves.aubry@univ-tln.fr : algebra

\end{document}

%\input{./optimisation/mine.tex}
 % !TEX encoding = UTF-8 Unicode
% !TEX TS-program = xelatex

\documentclass[french]{report}

%\usepackage[utf8]{inputenc}
%\usepackage[T1]{fontenc}
\usepackage{babel}


\newif\ifcomment
%\commenttrue # Show comments

\usepackage{physics}
\usepackage{amssymb}


\usepackage{amsthm}
% \usepackage{thmtools}
\usepackage{mathtools}
\usepackage{amsfonts}

\usepackage{color}

\usepackage{tikz}

\usepackage{geometry}
\geometry{a5paper, margin=0.1in, right=1cm}

\usepackage{dsfont}

\usepackage{graphicx}
\graphicspath{ {images/} }

\usepackage{faktor}

\usepackage{IEEEtrantools}
\usepackage{enumerate}   
\usepackage[PostScript=dvips]{"/Users/aware/Documents/Courses/diagrams"}


\newtheorem{theorem}{Théorème}[section]
\renewcommand{\thetheorem}{\arabic{theorem}}
\newtheorem{lemme}{Lemme}[section]
\renewcommand{\thelemme}{\arabic{lemme}}
\newtheorem{proposition}{Proposition}[section]
\renewcommand{\theproposition}{\arabic{proposition}}
\newtheorem{notations}{Notations}[section]
\newtheorem{problem}{Problème}[section]
\newtheorem{corollary}{Corollaire}[theorem]
\renewcommand{\thecorollary}{\arabic{corollary}}
\newtheorem{property}{Propriété}[section]
\newtheorem{objective}{Objectif}[section]

\theoremstyle{definition}
\newtheorem{definition}{Définition}[section]
\renewcommand{\thedefinition}{\arabic{definition}}
\newtheorem{exercise}{Exercice}[chapter]
\renewcommand{\theexercise}{\arabic{exercise}}
\newtheorem{example}{Exemple}[chapter]
\renewcommand{\theexample}{\arabic{example}}
\newtheorem*{solution}{Solution}
\newtheorem*{application}{Application}
\newtheorem*{notation}{Notation}
\newtheorem*{vocabulary}{Vocabulaire}
\newtheorem*{properties}{Propriétés}



\theoremstyle{remark}
\newtheorem*{remark}{Remarque}
\newtheorem*{rappel}{Rappel}


\usepackage{etoolbox}
\AtBeginEnvironment{exercise}{\small}
\AtBeginEnvironment{example}{\small}

\usepackage{cases}
\usepackage[red]{mypack}

\usepackage[framemethod=TikZ]{mdframed}

\definecolor{bg}{rgb}{0.4,0.25,0.95}
\definecolor{pagebg}{rgb}{0,0,0.5}
\surroundwithmdframed[
   topline=false,
   rightline=false,
   bottomline=false,
   leftmargin=\parindent,
   skipabove=8pt,
   skipbelow=8pt,
   linecolor=blue,
   innerbottommargin=10pt,
   % backgroundcolor=bg,font=\color{orange}\sffamily, fontcolor=white
]{definition}

\usepackage{empheq}
\usepackage[most]{tcolorbox}

\newtcbox{\mymath}[1][]{%
    nobeforeafter, math upper, tcbox raise base,
    enhanced, colframe=blue!30!black,
    colback=red!10, boxrule=1pt,
    #1}

\usepackage{unixode}


\DeclareMathOperator{\ord}{ord}
\DeclareMathOperator{\orb}{orb}
\DeclareMathOperator{\stab}{stab}
\DeclareMathOperator{\Stab}{stab}
\DeclareMathOperator{\ppcm}{ppcm}
\DeclareMathOperator{\conj}{Conj}
\DeclareMathOperator{\End}{End}
\DeclareMathOperator{\rot}{rot}
\DeclareMathOperator{\trs}{trace}
\DeclareMathOperator{\Ind}{Ind}
\DeclareMathOperator{\mat}{Mat}
\DeclareMathOperator{\id}{Id}
\DeclareMathOperator{\vect}{vect}
\DeclareMathOperator{\img}{img}
\DeclareMathOperator{\cov}{Cov}
\DeclareMathOperator{\dist}{dist}
\DeclareMathOperator{\irr}{Irr}
\DeclareMathOperator{\image}{Im}
\DeclareMathOperator{\pd}{\partial}
\DeclareMathOperator{\epi}{epi}
\DeclareMathOperator{\Argmin}{Argmin}
\DeclareMathOperator{\dom}{dom}
\DeclareMathOperator{\proj}{proj}
\DeclareMathOperator{\ctg}{ctg}
\DeclareMathOperator{\supp}{supp}
\DeclareMathOperator{\argmin}{argmin}
\DeclareMathOperator{\mult}{mult}
\DeclareMathOperator{\ch}{ch}
\DeclareMathOperator{\sh}{sh}
\DeclareMathOperator{\rang}{rang}
\DeclareMathOperator{\diam}{diam}
\DeclareMathOperator{\Epigraphe}{Epigraphe}




\usepackage{xcolor}
\everymath{\color{blue}}
%\everymath{\color[rgb]{0,1,1}}
%\pagecolor[rgb]{0,0,0.5}


\newcommand*{\pdtest}[3][]{\ensuremath{\frac{\partial^{#1} #2}{\partial #3}}}

\newcommand*{\deffunc}[6][]{\ensuremath{
\begin{array}{rcl}
#2 : #3 &\rightarrow& #4\\
#5 &\mapsto& #6
\end{array}
}}

\newcommand{\eqcolon}{\mathrel{\resizebox{\widthof{$\mathord{=}$}}{\height}{ $\!\!=\!\!\resizebox{1.2\width}{0.8\height}{\raisebox{0.23ex}{$\mathop{:}$}}\!\!$ }}}
\newcommand{\coloneq}{\mathrel{\resizebox{\widthof{$\mathord{=}$}}{\height}{ $\!\!\resizebox{1.2\width}{0.8\height}{\raisebox{0.23ex}{$\mathop{:}$}}\!\!=\!\!$ }}}
\newcommand{\eqcolonl}{\ensuremath{\mathrel{=\!\!\mathop{:}}}}
\newcommand{\coloneql}{\ensuremath{\mathrel{\mathop{:} \!\! =}}}
\newcommand{\vc}[1]{% inline column vector
  \left(\begin{smallmatrix}#1\end{smallmatrix}\right)%
}
\newcommand{\vr}[1]{% inline row vector
  \begin{smallmatrix}(\,#1\,)\end{smallmatrix}%
}
\makeatletter
\newcommand*{\defeq}{\ =\mathrel{\rlap{%
                     \raisebox{0.3ex}{$\m@th\cdot$}}%
                     \raisebox{-0.3ex}{$\m@th\cdot$}}%
                     }
\makeatother

\newcommand{\mathcircle}[1]{% inline row vector
 \overset{\circ}{#1}
}
\newcommand{\ulim}{% low limit
 \underline{\lim}
}
\newcommand{\ssi}{% iff
\iff
}
\newcommand{\ps}[2]{
\expval{#1 | #2}
}
\newcommand{\df}[1]{
\mqty{#1}
}
\newcommand{\n}[1]{
\norm{#1}
}
\newcommand{\sys}[1]{
\left\{\smqty{#1}\right.
}


\newcommand{\eqdef}{\ensuremath{\overset{\text{def}}=}}


\def\Circlearrowright{\ensuremath{%
  \rotatebox[origin=c]{230}{$\circlearrowright$}}}

\newcommand\ct[1]{\text{\rmfamily\upshape #1}}
\newcommand\question[1]{ {\color{red} ...!? \small #1}}
\newcommand\caz[1]{\left\{\begin{array} #1 \end{array}\right.}
\newcommand\const{\text{\rmfamily\upshape const}}
\newcommand\toP{ \overset{\pro}{\to}}
\newcommand\toPP{ \overset{\text{PP}}{\to}}
\newcommand{\oeq}{\mathrel{\text{\textcircled{$=$}}}}





\usepackage{xcolor}
% \usepackage[normalem]{ulem}
\usepackage{lipsum}
\makeatletter
% \newcommand\colorwave[1][blue]{\bgroup \markoverwith{\lower3.5\p@\hbox{\sixly \textcolor{#1}{\char58}}}\ULon}
%\font\sixly=lasy6 % does not re-load if already loaded, so no memory problem.

\newmdtheoremenv[
linewidth= 1pt,linecolor= blue,%
leftmargin=20,rightmargin=20,innertopmargin=0pt, innerrightmargin=40,%
tikzsetting = { draw=lightgray, line width = 0.3pt,dashed,%
dash pattern = on 15pt off 3pt},%
splittopskip=\topskip,skipbelow=\baselineskip,%
skipabove=\baselineskip,ntheorem,roundcorner=0pt,
% backgroundcolor=pagebg,font=\color{orange}\sffamily, fontcolor=white
]{examplebox}{Exemple}[section]



\newcommand\R{\mathbb{R}}
\newcommand\Z{\mathbb{Z}}
\newcommand\N{\mathbb{N}}
\newcommand\E{\mathbb{E}}
\newcommand\F{\mathcal{F}}
\newcommand\cH{\mathcal{H}}
\newcommand\V{\mathbb{V}}
\newcommand\dmo{ ^{-1} }
\newcommand\kapa{\kappa}
\newcommand\im{Im}
\newcommand\hs{\mathcal{H}}





\usepackage{soul}

\makeatletter
\newcommand*{\whiten}[1]{\llap{\textcolor{white}{{\the\SOUL@token}}\hspace{#1pt}}}
\DeclareRobustCommand*\myul{%
    \def\SOUL@everyspace{\underline{\space}\kern\z@}%
    \def\SOUL@everytoken{%
     \setbox0=\hbox{\the\SOUL@token}%
     \ifdim\dp0>\z@
        \raisebox{\dp0}{\underline{\phantom{\the\SOUL@token}}}%
        \whiten{1}\whiten{0}%
        \whiten{-1}\whiten{-2}%
        \llap{\the\SOUL@token}%
     \else
        \underline{\the\SOUL@token}%
     \fi}%
\SOUL@}
\makeatother

\newcommand*{\demp}{\fontfamily{lmtt}\selectfont}

\DeclareTextFontCommand{\textdemp}{\demp}

\begin{document}

\ifcomment
Multiline
comment
\fi
\ifcomment
\myul{Typesetting test}
% \color[rgb]{1,1,1}
$∑_i^n≠ 60º±∞π∆¬≈√j∫h≤≥µ$

$\CR \R\pro\ind\pro\gS\pro
\mqty[a&b\\c&d]$
$\pro\mathbb{P}$
$\dd{x}$

  \[
    \alpha(x)=\left\{
                \begin{array}{ll}
                  x\\
                  \frac{1}{1+e^{-kx}}\\
                  \frac{e^x-e^{-x}}{e^x+e^{-x}}
                \end{array}
              \right.
  \]

  $\expval{x}$
  
  $\chi_\rho(ghg\dmo)=\Tr(\rho_{ghg\dmo})=\Tr(\rho_g\circ\rho_h\circ\rho\dmo_g)=\Tr(\rho_h)\overset{\mbox{\scalebox{0.5}{$\Tr(AB)=\Tr(BA)$}}}{=}\chi_\rho(h)$
  	$\mathop{\oplus}_{\substack{x\in X}}$

$\mat(\rho_g)=(a_{ij}(g))_{\scriptsize \substack{1\leq i\leq d \\ 1\leq j\leq d}}$ et $\mat(\rho'_g)=(a'_{ij}(g))_{\scriptsize \substack{1\leq i'\leq d' \\ 1\leq j'\leq d'}}$



\[\int_a^b{\mathbb{R}^2}g(u, v)\dd{P_{XY}}(u, v)=\iint g(u,v) f_{XY}(u, v)\dd \lambda(u) \dd \lambda(v)\]
$$\lim_{x\to\infty} f(x)$$	
$$\iiiint_V \mu(t,u,v,w) \,dt\,du\,dv\,dw$$
$$\sum_{n=1}^{\infty} 2^{-n} = 1$$	
\begin{definition}
	Si $X$ et $Y$ sont 2 v.a. ou definit la \textsc{Covariance} entre $X$ et $Y$ comme
	$\cov(X,Y)\overset{\text{def}}{=}\E\left[(X-\E(X))(Y-\E(Y))\right]=\E(XY)-\E(X)\E(Y)$.
\end{definition}
\fi
\pagebreak

% \tableofcontents

% insert your code here
%\input{./algebra/main.tex}
%\input{./geometrie-differentielle/main.tex}
%\input{./probabilite/main.tex}
%\input{./analyse-fonctionnelle/main.tex}
% \input{./Analyse-convexe-et-dualite-en-optimisation/main.tex}
%\input{./tikz/main.tex}
%\input{./Theorie-du-distributions/main.tex}
%\input{./optimisation/mine.tex}
 \input{./modelisation/main.tex}

% yves.aubry@univ-tln.fr : algebra

\end{document}


% yves.aubry@univ-tln.fr : algebra

\end{document}

%% !TEX encoding = UTF-8 Unicode
% !TEX TS-program = xelatex

\documentclass[french]{report}

%\usepackage[utf8]{inputenc}
%\usepackage[T1]{fontenc}
\usepackage{babel}


\newif\ifcomment
%\commenttrue # Show comments

\usepackage{physics}
\usepackage{amssymb}


\usepackage{amsthm}
% \usepackage{thmtools}
\usepackage{mathtools}
\usepackage{amsfonts}

\usepackage{color}

\usepackage{tikz}

\usepackage{geometry}
\geometry{a5paper, margin=0.1in, right=1cm}

\usepackage{dsfont}

\usepackage{graphicx}
\graphicspath{ {images/} }

\usepackage{faktor}

\usepackage{IEEEtrantools}
\usepackage{enumerate}   
\usepackage[PostScript=dvips]{"/Users/aware/Documents/Courses/diagrams"}


\newtheorem{theorem}{Théorème}[section]
\renewcommand{\thetheorem}{\arabic{theorem}}
\newtheorem{lemme}{Lemme}[section]
\renewcommand{\thelemme}{\arabic{lemme}}
\newtheorem{proposition}{Proposition}[section]
\renewcommand{\theproposition}{\arabic{proposition}}
\newtheorem{notations}{Notations}[section]
\newtheorem{problem}{Problème}[section]
\newtheorem{corollary}{Corollaire}[theorem]
\renewcommand{\thecorollary}{\arabic{corollary}}
\newtheorem{property}{Propriété}[section]
\newtheorem{objective}{Objectif}[section]

\theoremstyle{definition}
\newtheorem{definition}{Définition}[section]
\renewcommand{\thedefinition}{\arabic{definition}}
\newtheorem{exercise}{Exercice}[chapter]
\renewcommand{\theexercise}{\arabic{exercise}}
\newtheorem{example}{Exemple}[chapter]
\renewcommand{\theexample}{\arabic{example}}
\newtheorem*{solution}{Solution}
\newtheorem*{application}{Application}
\newtheorem*{notation}{Notation}
\newtheorem*{vocabulary}{Vocabulaire}
\newtheorem*{properties}{Propriétés}



\theoremstyle{remark}
\newtheorem*{remark}{Remarque}
\newtheorem*{rappel}{Rappel}


\usepackage{etoolbox}
\AtBeginEnvironment{exercise}{\small}
\AtBeginEnvironment{example}{\small}

\usepackage{cases}
\usepackage[red]{mypack}

\usepackage[framemethod=TikZ]{mdframed}

\definecolor{bg}{rgb}{0.4,0.25,0.95}
\definecolor{pagebg}{rgb}{0,0,0.5}
\surroundwithmdframed[
   topline=false,
   rightline=false,
   bottomline=false,
   leftmargin=\parindent,
   skipabove=8pt,
   skipbelow=8pt,
   linecolor=blue,
   innerbottommargin=10pt,
   % backgroundcolor=bg,font=\color{orange}\sffamily, fontcolor=white
]{definition}

\usepackage{empheq}
\usepackage[most]{tcolorbox}

\newtcbox{\mymath}[1][]{%
    nobeforeafter, math upper, tcbox raise base,
    enhanced, colframe=blue!30!black,
    colback=red!10, boxrule=1pt,
    #1}

\usepackage{unixode}


\DeclareMathOperator{\ord}{ord}
\DeclareMathOperator{\orb}{orb}
\DeclareMathOperator{\stab}{stab}
\DeclareMathOperator{\Stab}{stab}
\DeclareMathOperator{\ppcm}{ppcm}
\DeclareMathOperator{\conj}{Conj}
\DeclareMathOperator{\End}{End}
\DeclareMathOperator{\rot}{rot}
\DeclareMathOperator{\trs}{trace}
\DeclareMathOperator{\Ind}{Ind}
\DeclareMathOperator{\mat}{Mat}
\DeclareMathOperator{\id}{Id}
\DeclareMathOperator{\vect}{vect}
\DeclareMathOperator{\img}{img}
\DeclareMathOperator{\cov}{Cov}
\DeclareMathOperator{\dist}{dist}
\DeclareMathOperator{\irr}{Irr}
\DeclareMathOperator{\image}{Im}
\DeclareMathOperator{\pd}{\partial}
\DeclareMathOperator{\epi}{epi}
\DeclareMathOperator{\Argmin}{Argmin}
\DeclareMathOperator{\dom}{dom}
\DeclareMathOperator{\proj}{proj}
\DeclareMathOperator{\ctg}{ctg}
\DeclareMathOperator{\supp}{supp}
\DeclareMathOperator{\argmin}{argmin}
\DeclareMathOperator{\mult}{mult}
\DeclareMathOperator{\ch}{ch}
\DeclareMathOperator{\sh}{sh}
\DeclareMathOperator{\rang}{rang}
\DeclareMathOperator{\diam}{diam}
\DeclareMathOperator{\Epigraphe}{Epigraphe}




\usepackage{xcolor}
\everymath{\color{blue}}
%\everymath{\color[rgb]{0,1,1}}
%\pagecolor[rgb]{0,0,0.5}


\newcommand*{\pdtest}[3][]{\ensuremath{\frac{\partial^{#1} #2}{\partial #3}}}

\newcommand*{\deffunc}[6][]{\ensuremath{
\begin{array}{rcl}
#2 : #3 &\rightarrow& #4\\
#5 &\mapsto& #6
\end{array}
}}

\newcommand{\eqcolon}{\mathrel{\resizebox{\widthof{$\mathord{=}$}}{\height}{ $\!\!=\!\!\resizebox{1.2\width}{0.8\height}{\raisebox{0.23ex}{$\mathop{:}$}}\!\!$ }}}
\newcommand{\coloneq}{\mathrel{\resizebox{\widthof{$\mathord{=}$}}{\height}{ $\!\!\resizebox{1.2\width}{0.8\height}{\raisebox{0.23ex}{$\mathop{:}$}}\!\!=\!\!$ }}}
\newcommand{\eqcolonl}{\ensuremath{\mathrel{=\!\!\mathop{:}}}}
\newcommand{\coloneql}{\ensuremath{\mathrel{\mathop{:} \!\! =}}}
\newcommand{\vc}[1]{% inline column vector
  \left(\begin{smallmatrix}#1\end{smallmatrix}\right)%
}
\newcommand{\vr}[1]{% inline row vector
  \begin{smallmatrix}(\,#1\,)\end{smallmatrix}%
}
\makeatletter
\newcommand*{\defeq}{\ =\mathrel{\rlap{%
                     \raisebox{0.3ex}{$\m@th\cdot$}}%
                     \raisebox{-0.3ex}{$\m@th\cdot$}}%
                     }
\makeatother

\newcommand{\mathcircle}[1]{% inline row vector
 \overset{\circ}{#1}
}
\newcommand{\ulim}{% low limit
 \underline{\lim}
}
\newcommand{\ssi}{% iff
\iff
}
\newcommand{\ps}[2]{
\expval{#1 | #2}
}
\newcommand{\df}[1]{
\mqty{#1}
}
\newcommand{\n}[1]{
\norm{#1}
}
\newcommand{\sys}[1]{
\left\{\smqty{#1}\right.
}


\newcommand{\eqdef}{\ensuremath{\overset{\text{def}}=}}


\def\Circlearrowright{\ensuremath{%
  \rotatebox[origin=c]{230}{$\circlearrowright$}}}

\newcommand\ct[1]{\text{\rmfamily\upshape #1}}
\newcommand\question[1]{ {\color{red} ...!? \small #1}}
\newcommand\caz[1]{\left\{\begin{array} #1 \end{array}\right.}
\newcommand\const{\text{\rmfamily\upshape const}}
\newcommand\toP{ \overset{\pro}{\to}}
\newcommand\toPP{ \overset{\text{PP}}{\to}}
\newcommand{\oeq}{\mathrel{\text{\textcircled{$=$}}}}





\usepackage{xcolor}
% \usepackage[normalem]{ulem}
\usepackage{lipsum}
\makeatletter
% \newcommand\colorwave[1][blue]{\bgroup \markoverwith{\lower3.5\p@\hbox{\sixly \textcolor{#1}{\char58}}}\ULon}
%\font\sixly=lasy6 % does not re-load if already loaded, so no memory problem.

\newmdtheoremenv[
linewidth= 1pt,linecolor= blue,%
leftmargin=20,rightmargin=20,innertopmargin=0pt, innerrightmargin=40,%
tikzsetting = { draw=lightgray, line width = 0.3pt,dashed,%
dash pattern = on 15pt off 3pt},%
splittopskip=\topskip,skipbelow=\baselineskip,%
skipabove=\baselineskip,ntheorem,roundcorner=0pt,
% backgroundcolor=pagebg,font=\color{orange}\sffamily, fontcolor=white
]{examplebox}{Exemple}[section]



\newcommand\R{\mathbb{R}}
\newcommand\Z{\mathbb{Z}}
\newcommand\N{\mathbb{N}}
\newcommand\E{\mathbb{E}}
\newcommand\F{\mathcal{F}}
\newcommand\cH{\mathcal{H}}
\newcommand\V{\mathbb{V}}
\newcommand\dmo{ ^{-1} }
\newcommand\kapa{\kappa}
\newcommand\im{Im}
\newcommand\hs{\mathcal{H}}





\usepackage{soul}

\makeatletter
\newcommand*{\whiten}[1]{\llap{\textcolor{white}{{\the\SOUL@token}}\hspace{#1pt}}}
\DeclareRobustCommand*\myul{%
    \def\SOUL@everyspace{\underline{\space}\kern\z@}%
    \def\SOUL@everytoken{%
     \setbox0=\hbox{\the\SOUL@token}%
     \ifdim\dp0>\z@
        \raisebox{\dp0}{\underline{\phantom{\the\SOUL@token}}}%
        \whiten{1}\whiten{0}%
        \whiten{-1}\whiten{-2}%
        \llap{\the\SOUL@token}%
     \else
        \underline{\the\SOUL@token}%
     \fi}%
\SOUL@}
\makeatother

\newcommand*{\demp}{\fontfamily{lmtt}\selectfont}

\DeclareTextFontCommand{\textdemp}{\demp}

\begin{document}

\ifcomment
Multiline
comment
\fi
\ifcomment
\myul{Typesetting test}
% \color[rgb]{1,1,1}
$∑_i^n≠ 60º±∞π∆¬≈√j∫h≤≥µ$

$\CR \R\pro\ind\pro\gS\pro
\mqty[a&b\\c&d]$
$\pro\mathbb{P}$
$\dd{x}$

  \[
    \alpha(x)=\left\{
                \begin{array}{ll}
                  x\\
                  \frac{1}{1+e^{-kx}}\\
                  \frac{e^x-e^{-x}}{e^x+e^{-x}}
                \end{array}
              \right.
  \]

  $\expval{x}$
  
  $\chi_\rho(ghg\dmo)=\Tr(\rho_{ghg\dmo})=\Tr(\rho_g\circ\rho_h\circ\rho\dmo_g)=\Tr(\rho_h)\overset{\mbox{\scalebox{0.5}{$\Tr(AB)=\Tr(BA)$}}}{=}\chi_\rho(h)$
  	$\mathop{\oplus}_{\substack{x\in X}}$

$\mat(\rho_g)=(a_{ij}(g))_{\scriptsize \substack{1\leq i\leq d \\ 1\leq j\leq d}}$ et $\mat(\rho'_g)=(a'_{ij}(g))_{\scriptsize \substack{1\leq i'\leq d' \\ 1\leq j'\leq d'}}$



\[\int_a^b{\mathbb{R}^2}g(u, v)\dd{P_{XY}}(u, v)=\iint g(u,v) f_{XY}(u, v)\dd \lambda(u) \dd \lambda(v)\]
$$\lim_{x\to\infty} f(x)$$	
$$\iiiint_V \mu(t,u,v,w) \,dt\,du\,dv\,dw$$
$$\sum_{n=1}^{\infty} 2^{-n} = 1$$	
\begin{definition}
	Si $X$ et $Y$ sont 2 v.a. ou definit la \textsc{Covariance} entre $X$ et $Y$ comme
	$\cov(X,Y)\overset{\text{def}}{=}\E\left[(X-\E(X))(Y-\E(Y))\right]=\E(XY)-\E(X)\E(Y)$.
\end{definition}
\fi
\pagebreak

% \tableofcontents

% insert your code here
%% !TEX encoding = UTF-8 Unicode
% !TEX TS-program = xelatex

\documentclass[french]{report}

%\usepackage[utf8]{inputenc}
%\usepackage[T1]{fontenc}
\usepackage{babel}


\newif\ifcomment
%\commenttrue # Show comments

\usepackage{physics}
\usepackage{amssymb}


\usepackage{amsthm}
% \usepackage{thmtools}
\usepackage{mathtools}
\usepackage{amsfonts}

\usepackage{color}

\usepackage{tikz}

\usepackage{geometry}
\geometry{a5paper, margin=0.1in, right=1cm}

\usepackage{dsfont}

\usepackage{graphicx}
\graphicspath{ {images/} }

\usepackage{faktor}

\usepackage{IEEEtrantools}
\usepackage{enumerate}   
\usepackage[PostScript=dvips]{"/Users/aware/Documents/Courses/diagrams"}


\newtheorem{theorem}{Théorème}[section]
\renewcommand{\thetheorem}{\arabic{theorem}}
\newtheorem{lemme}{Lemme}[section]
\renewcommand{\thelemme}{\arabic{lemme}}
\newtheorem{proposition}{Proposition}[section]
\renewcommand{\theproposition}{\arabic{proposition}}
\newtheorem{notations}{Notations}[section]
\newtheorem{problem}{Problème}[section]
\newtheorem{corollary}{Corollaire}[theorem]
\renewcommand{\thecorollary}{\arabic{corollary}}
\newtheorem{property}{Propriété}[section]
\newtheorem{objective}{Objectif}[section]

\theoremstyle{definition}
\newtheorem{definition}{Définition}[section]
\renewcommand{\thedefinition}{\arabic{definition}}
\newtheorem{exercise}{Exercice}[chapter]
\renewcommand{\theexercise}{\arabic{exercise}}
\newtheorem{example}{Exemple}[chapter]
\renewcommand{\theexample}{\arabic{example}}
\newtheorem*{solution}{Solution}
\newtheorem*{application}{Application}
\newtheorem*{notation}{Notation}
\newtheorem*{vocabulary}{Vocabulaire}
\newtheorem*{properties}{Propriétés}



\theoremstyle{remark}
\newtheorem*{remark}{Remarque}
\newtheorem*{rappel}{Rappel}


\usepackage{etoolbox}
\AtBeginEnvironment{exercise}{\small}
\AtBeginEnvironment{example}{\small}

\usepackage{cases}
\usepackage[red]{mypack}

\usepackage[framemethod=TikZ]{mdframed}

\definecolor{bg}{rgb}{0.4,0.25,0.95}
\definecolor{pagebg}{rgb}{0,0,0.5}
\surroundwithmdframed[
   topline=false,
   rightline=false,
   bottomline=false,
   leftmargin=\parindent,
   skipabove=8pt,
   skipbelow=8pt,
   linecolor=blue,
   innerbottommargin=10pt,
   % backgroundcolor=bg,font=\color{orange}\sffamily, fontcolor=white
]{definition}

\usepackage{empheq}
\usepackage[most]{tcolorbox}

\newtcbox{\mymath}[1][]{%
    nobeforeafter, math upper, tcbox raise base,
    enhanced, colframe=blue!30!black,
    colback=red!10, boxrule=1pt,
    #1}

\usepackage{unixode}


\DeclareMathOperator{\ord}{ord}
\DeclareMathOperator{\orb}{orb}
\DeclareMathOperator{\stab}{stab}
\DeclareMathOperator{\Stab}{stab}
\DeclareMathOperator{\ppcm}{ppcm}
\DeclareMathOperator{\conj}{Conj}
\DeclareMathOperator{\End}{End}
\DeclareMathOperator{\rot}{rot}
\DeclareMathOperator{\trs}{trace}
\DeclareMathOperator{\Ind}{Ind}
\DeclareMathOperator{\mat}{Mat}
\DeclareMathOperator{\id}{Id}
\DeclareMathOperator{\vect}{vect}
\DeclareMathOperator{\img}{img}
\DeclareMathOperator{\cov}{Cov}
\DeclareMathOperator{\dist}{dist}
\DeclareMathOperator{\irr}{Irr}
\DeclareMathOperator{\image}{Im}
\DeclareMathOperator{\pd}{\partial}
\DeclareMathOperator{\epi}{epi}
\DeclareMathOperator{\Argmin}{Argmin}
\DeclareMathOperator{\dom}{dom}
\DeclareMathOperator{\proj}{proj}
\DeclareMathOperator{\ctg}{ctg}
\DeclareMathOperator{\supp}{supp}
\DeclareMathOperator{\argmin}{argmin}
\DeclareMathOperator{\mult}{mult}
\DeclareMathOperator{\ch}{ch}
\DeclareMathOperator{\sh}{sh}
\DeclareMathOperator{\rang}{rang}
\DeclareMathOperator{\diam}{diam}
\DeclareMathOperator{\Epigraphe}{Epigraphe}




\usepackage{xcolor}
\everymath{\color{blue}}
%\everymath{\color[rgb]{0,1,1}}
%\pagecolor[rgb]{0,0,0.5}


\newcommand*{\pdtest}[3][]{\ensuremath{\frac{\partial^{#1} #2}{\partial #3}}}

\newcommand*{\deffunc}[6][]{\ensuremath{
\begin{array}{rcl}
#2 : #3 &\rightarrow& #4\\
#5 &\mapsto& #6
\end{array}
}}

\newcommand{\eqcolon}{\mathrel{\resizebox{\widthof{$\mathord{=}$}}{\height}{ $\!\!=\!\!\resizebox{1.2\width}{0.8\height}{\raisebox{0.23ex}{$\mathop{:}$}}\!\!$ }}}
\newcommand{\coloneq}{\mathrel{\resizebox{\widthof{$\mathord{=}$}}{\height}{ $\!\!\resizebox{1.2\width}{0.8\height}{\raisebox{0.23ex}{$\mathop{:}$}}\!\!=\!\!$ }}}
\newcommand{\eqcolonl}{\ensuremath{\mathrel{=\!\!\mathop{:}}}}
\newcommand{\coloneql}{\ensuremath{\mathrel{\mathop{:} \!\! =}}}
\newcommand{\vc}[1]{% inline column vector
  \left(\begin{smallmatrix}#1\end{smallmatrix}\right)%
}
\newcommand{\vr}[1]{% inline row vector
  \begin{smallmatrix}(\,#1\,)\end{smallmatrix}%
}
\makeatletter
\newcommand*{\defeq}{\ =\mathrel{\rlap{%
                     \raisebox{0.3ex}{$\m@th\cdot$}}%
                     \raisebox{-0.3ex}{$\m@th\cdot$}}%
                     }
\makeatother

\newcommand{\mathcircle}[1]{% inline row vector
 \overset{\circ}{#1}
}
\newcommand{\ulim}{% low limit
 \underline{\lim}
}
\newcommand{\ssi}{% iff
\iff
}
\newcommand{\ps}[2]{
\expval{#1 | #2}
}
\newcommand{\df}[1]{
\mqty{#1}
}
\newcommand{\n}[1]{
\norm{#1}
}
\newcommand{\sys}[1]{
\left\{\smqty{#1}\right.
}


\newcommand{\eqdef}{\ensuremath{\overset{\text{def}}=}}


\def\Circlearrowright{\ensuremath{%
  \rotatebox[origin=c]{230}{$\circlearrowright$}}}

\newcommand\ct[1]{\text{\rmfamily\upshape #1}}
\newcommand\question[1]{ {\color{red} ...!? \small #1}}
\newcommand\caz[1]{\left\{\begin{array} #1 \end{array}\right.}
\newcommand\const{\text{\rmfamily\upshape const}}
\newcommand\toP{ \overset{\pro}{\to}}
\newcommand\toPP{ \overset{\text{PP}}{\to}}
\newcommand{\oeq}{\mathrel{\text{\textcircled{$=$}}}}





\usepackage{xcolor}
% \usepackage[normalem]{ulem}
\usepackage{lipsum}
\makeatletter
% \newcommand\colorwave[1][blue]{\bgroup \markoverwith{\lower3.5\p@\hbox{\sixly \textcolor{#1}{\char58}}}\ULon}
%\font\sixly=lasy6 % does not re-load if already loaded, so no memory problem.

\newmdtheoremenv[
linewidth= 1pt,linecolor= blue,%
leftmargin=20,rightmargin=20,innertopmargin=0pt, innerrightmargin=40,%
tikzsetting = { draw=lightgray, line width = 0.3pt,dashed,%
dash pattern = on 15pt off 3pt},%
splittopskip=\topskip,skipbelow=\baselineskip,%
skipabove=\baselineskip,ntheorem,roundcorner=0pt,
% backgroundcolor=pagebg,font=\color{orange}\sffamily, fontcolor=white
]{examplebox}{Exemple}[section]



\newcommand\R{\mathbb{R}}
\newcommand\Z{\mathbb{Z}}
\newcommand\N{\mathbb{N}}
\newcommand\E{\mathbb{E}}
\newcommand\F{\mathcal{F}}
\newcommand\cH{\mathcal{H}}
\newcommand\V{\mathbb{V}}
\newcommand\dmo{ ^{-1} }
\newcommand\kapa{\kappa}
\newcommand\im{Im}
\newcommand\hs{\mathcal{H}}





\usepackage{soul}

\makeatletter
\newcommand*{\whiten}[1]{\llap{\textcolor{white}{{\the\SOUL@token}}\hspace{#1pt}}}
\DeclareRobustCommand*\myul{%
    \def\SOUL@everyspace{\underline{\space}\kern\z@}%
    \def\SOUL@everytoken{%
     \setbox0=\hbox{\the\SOUL@token}%
     \ifdim\dp0>\z@
        \raisebox{\dp0}{\underline{\phantom{\the\SOUL@token}}}%
        \whiten{1}\whiten{0}%
        \whiten{-1}\whiten{-2}%
        \llap{\the\SOUL@token}%
     \else
        \underline{\the\SOUL@token}%
     \fi}%
\SOUL@}
\makeatother

\newcommand*{\demp}{\fontfamily{lmtt}\selectfont}

\DeclareTextFontCommand{\textdemp}{\demp}

\begin{document}

\ifcomment
Multiline
comment
\fi
\ifcomment
\myul{Typesetting test}
% \color[rgb]{1,1,1}
$∑_i^n≠ 60º±∞π∆¬≈√j∫h≤≥µ$

$\CR \R\pro\ind\pro\gS\pro
\mqty[a&b\\c&d]$
$\pro\mathbb{P}$
$\dd{x}$

  \[
    \alpha(x)=\left\{
                \begin{array}{ll}
                  x\\
                  \frac{1}{1+e^{-kx}}\\
                  \frac{e^x-e^{-x}}{e^x+e^{-x}}
                \end{array}
              \right.
  \]

  $\expval{x}$
  
  $\chi_\rho(ghg\dmo)=\Tr(\rho_{ghg\dmo})=\Tr(\rho_g\circ\rho_h\circ\rho\dmo_g)=\Tr(\rho_h)\overset{\mbox{\scalebox{0.5}{$\Tr(AB)=\Tr(BA)$}}}{=}\chi_\rho(h)$
  	$\mathop{\oplus}_{\substack{x\in X}}$

$\mat(\rho_g)=(a_{ij}(g))_{\scriptsize \substack{1\leq i\leq d \\ 1\leq j\leq d}}$ et $\mat(\rho'_g)=(a'_{ij}(g))_{\scriptsize \substack{1\leq i'\leq d' \\ 1\leq j'\leq d'}}$



\[\int_a^b{\mathbb{R}^2}g(u, v)\dd{P_{XY}}(u, v)=\iint g(u,v) f_{XY}(u, v)\dd \lambda(u) \dd \lambda(v)\]
$$\lim_{x\to\infty} f(x)$$	
$$\iiiint_V \mu(t,u,v,w) \,dt\,du\,dv\,dw$$
$$\sum_{n=1}^{\infty} 2^{-n} = 1$$	
\begin{definition}
	Si $X$ et $Y$ sont 2 v.a. ou definit la \textsc{Covariance} entre $X$ et $Y$ comme
	$\cov(X,Y)\overset{\text{def}}{=}\E\left[(X-\E(X))(Y-\E(Y))\right]=\E(XY)-\E(X)\E(Y)$.
\end{definition}
\fi
\pagebreak

% \tableofcontents

% insert your code here
%\input{./algebra/main.tex}
%\input{./geometrie-differentielle/main.tex}
%\input{./probabilite/main.tex}
%\input{./analyse-fonctionnelle/main.tex}
% \input{./Analyse-convexe-et-dualite-en-optimisation/main.tex}
%\input{./tikz/main.tex}
%\input{./Theorie-du-distributions/main.tex}
%\input{./optimisation/mine.tex}
 \input{./modelisation/main.tex}

% yves.aubry@univ-tln.fr : algebra

\end{document}

%% !TEX encoding = UTF-8 Unicode
% !TEX TS-program = xelatex

\documentclass[french]{report}

%\usepackage[utf8]{inputenc}
%\usepackage[T1]{fontenc}
\usepackage{babel}


\newif\ifcomment
%\commenttrue # Show comments

\usepackage{physics}
\usepackage{amssymb}


\usepackage{amsthm}
% \usepackage{thmtools}
\usepackage{mathtools}
\usepackage{amsfonts}

\usepackage{color}

\usepackage{tikz}

\usepackage{geometry}
\geometry{a5paper, margin=0.1in, right=1cm}

\usepackage{dsfont}

\usepackage{graphicx}
\graphicspath{ {images/} }

\usepackage{faktor}

\usepackage{IEEEtrantools}
\usepackage{enumerate}   
\usepackage[PostScript=dvips]{"/Users/aware/Documents/Courses/diagrams"}


\newtheorem{theorem}{Théorème}[section]
\renewcommand{\thetheorem}{\arabic{theorem}}
\newtheorem{lemme}{Lemme}[section]
\renewcommand{\thelemme}{\arabic{lemme}}
\newtheorem{proposition}{Proposition}[section]
\renewcommand{\theproposition}{\arabic{proposition}}
\newtheorem{notations}{Notations}[section]
\newtheorem{problem}{Problème}[section]
\newtheorem{corollary}{Corollaire}[theorem]
\renewcommand{\thecorollary}{\arabic{corollary}}
\newtheorem{property}{Propriété}[section]
\newtheorem{objective}{Objectif}[section]

\theoremstyle{definition}
\newtheorem{definition}{Définition}[section]
\renewcommand{\thedefinition}{\arabic{definition}}
\newtheorem{exercise}{Exercice}[chapter]
\renewcommand{\theexercise}{\arabic{exercise}}
\newtheorem{example}{Exemple}[chapter]
\renewcommand{\theexample}{\arabic{example}}
\newtheorem*{solution}{Solution}
\newtheorem*{application}{Application}
\newtheorem*{notation}{Notation}
\newtheorem*{vocabulary}{Vocabulaire}
\newtheorem*{properties}{Propriétés}



\theoremstyle{remark}
\newtheorem*{remark}{Remarque}
\newtheorem*{rappel}{Rappel}


\usepackage{etoolbox}
\AtBeginEnvironment{exercise}{\small}
\AtBeginEnvironment{example}{\small}

\usepackage{cases}
\usepackage[red]{mypack}

\usepackage[framemethod=TikZ]{mdframed}

\definecolor{bg}{rgb}{0.4,0.25,0.95}
\definecolor{pagebg}{rgb}{0,0,0.5}
\surroundwithmdframed[
   topline=false,
   rightline=false,
   bottomline=false,
   leftmargin=\parindent,
   skipabove=8pt,
   skipbelow=8pt,
   linecolor=blue,
   innerbottommargin=10pt,
   % backgroundcolor=bg,font=\color{orange}\sffamily, fontcolor=white
]{definition}

\usepackage{empheq}
\usepackage[most]{tcolorbox}

\newtcbox{\mymath}[1][]{%
    nobeforeafter, math upper, tcbox raise base,
    enhanced, colframe=blue!30!black,
    colback=red!10, boxrule=1pt,
    #1}

\usepackage{unixode}


\DeclareMathOperator{\ord}{ord}
\DeclareMathOperator{\orb}{orb}
\DeclareMathOperator{\stab}{stab}
\DeclareMathOperator{\Stab}{stab}
\DeclareMathOperator{\ppcm}{ppcm}
\DeclareMathOperator{\conj}{Conj}
\DeclareMathOperator{\End}{End}
\DeclareMathOperator{\rot}{rot}
\DeclareMathOperator{\trs}{trace}
\DeclareMathOperator{\Ind}{Ind}
\DeclareMathOperator{\mat}{Mat}
\DeclareMathOperator{\id}{Id}
\DeclareMathOperator{\vect}{vect}
\DeclareMathOperator{\img}{img}
\DeclareMathOperator{\cov}{Cov}
\DeclareMathOperator{\dist}{dist}
\DeclareMathOperator{\irr}{Irr}
\DeclareMathOperator{\image}{Im}
\DeclareMathOperator{\pd}{\partial}
\DeclareMathOperator{\epi}{epi}
\DeclareMathOperator{\Argmin}{Argmin}
\DeclareMathOperator{\dom}{dom}
\DeclareMathOperator{\proj}{proj}
\DeclareMathOperator{\ctg}{ctg}
\DeclareMathOperator{\supp}{supp}
\DeclareMathOperator{\argmin}{argmin}
\DeclareMathOperator{\mult}{mult}
\DeclareMathOperator{\ch}{ch}
\DeclareMathOperator{\sh}{sh}
\DeclareMathOperator{\rang}{rang}
\DeclareMathOperator{\diam}{diam}
\DeclareMathOperator{\Epigraphe}{Epigraphe}




\usepackage{xcolor}
\everymath{\color{blue}}
%\everymath{\color[rgb]{0,1,1}}
%\pagecolor[rgb]{0,0,0.5}


\newcommand*{\pdtest}[3][]{\ensuremath{\frac{\partial^{#1} #2}{\partial #3}}}

\newcommand*{\deffunc}[6][]{\ensuremath{
\begin{array}{rcl}
#2 : #3 &\rightarrow& #4\\
#5 &\mapsto& #6
\end{array}
}}

\newcommand{\eqcolon}{\mathrel{\resizebox{\widthof{$\mathord{=}$}}{\height}{ $\!\!=\!\!\resizebox{1.2\width}{0.8\height}{\raisebox{0.23ex}{$\mathop{:}$}}\!\!$ }}}
\newcommand{\coloneq}{\mathrel{\resizebox{\widthof{$\mathord{=}$}}{\height}{ $\!\!\resizebox{1.2\width}{0.8\height}{\raisebox{0.23ex}{$\mathop{:}$}}\!\!=\!\!$ }}}
\newcommand{\eqcolonl}{\ensuremath{\mathrel{=\!\!\mathop{:}}}}
\newcommand{\coloneql}{\ensuremath{\mathrel{\mathop{:} \!\! =}}}
\newcommand{\vc}[1]{% inline column vector
  \left(\begin{smallmatrix}#1\end{smallmatrix}\right)%
}
\newcommand{\vr}[1]{% inline row vector
  \begin{smallmatrix}(\,#1\,)\end{smallmatrix}%
}
\makeatletter
\newcommand*{\defeq}{\ =\mathrel{\rlap{%
                     \raisebox{0.3ex}{$\m@th\cdot$}}%
                     \raisebox{-0.3ex}{$\m@th\cdot$}}%
                     }
\makeatother

\newcommand{\mathcircle}[1]{% inline row vector
 \overset{\circ}{#1}
}
\newcommand{\ulim}{% low limit
 \underline{\lim}
}
\newcommand{\ssi}{% iff
\iff
}
\newcommand{\ps}[2]{
\expval{#1 | #2}
}
\newcommand{\df}[1]{
\mqty{#1}
}
\newcommand{\n}[1]{
\norm{#1}
}
\newcommand{\sys}[1]{
\left\{\smqty{#1}\right.
}


\newcommand{\eqdef}{\ensuremath{\overset{\text{def}}=}}


\def\Circlearrowright{\ensuremath{%
  \rotatebox[origin=c]{230}{$\circlearrowright$}}}

\newcommand\ct[1]{\text{\rmfamily\upshape #1}}
\newcommand\question[1]{ {\color{red} ...!? \small #1}}
\newcommand\caz[1]{\left\{\begin{array} #1 \end{array}\right.}
\newcommand\const{\text{\rmfamily\upshape const}}
\newcommand\toP{ \overset{\pro}{\to}}
\newcommand\toPP{ \overset{\text{PP}}{\to}}
\newcommand{\oeq}{\mathrel{\text{\textcircled{$=$}}}}





\usepackage{xcolor}
% \usepackage[normalem]{ulem}
\usepackage{lipsum}
\makeatletter
% \newcommand\colorwave[1][blue]{\bgroup \markoverwith{\lower3.5\p@\hbox{\sixly \textcolor{#1}{\char58}}}\ULon}
%\font\sixly=lasy6 % does not re-load if already loaded, so no memory problem.

\newmdtheoremenv[
linewidth= 1pt,linecolor= blue,%
leftmargin=20,rightmargin=20,innertopmargin=0pt, innerrightmargin=40,%
tikzsetting = { draw=lightgray, line width = 0.3pt,dashed,%
dash pattern = on 15pt off 3pt},%
splittopskip=\topskip,skipbelow=\baselineskip,%
skipabove=\baselineskip,ntheorem,roundcorner=0pt,
% backgroundcolor=pagebg,font=\color{orange}\sffamily, fontcolor=white
]{examplebox}{Exemple}[section]



\newcommand\R{\mathbb{R}}
\newcommand\Z{\mathbb{Z}}
\newcommand\N{\mathbb{N}}
\newcommand\E{\mathbb{E}}
\newcommand\F{\mathcal{F}}
\newcommand\cH{\mathcal{H}}
\newcommand\V{\mathbb{V}}
\newcommand\dmo{ ^{-1} }
\newcommand\kapa{\kappa}
\newcommand\im{Im}
\newcommand\hs{\mathcal{H}}





\usepackage{soul}

\makeatletter
\newcommand*{\whiten}[1]{\llap{\textcolor{white}{{\the\SOUL@token}}\hspace{#1pt}}}
\DeclareRobustCommand*\myul{%
    \def\SOUL@everyspace{\underline{\space}\kern\z@}%
    \def\SOUL@everytoken{%
     \setbox0=\hbox{\the\SOUL@token}%
     \ifdim\dp0>\z@
        \raisebox{\dp0}{\underline{\phantom{\the\SOUL@token}}}%
        \whiten{1}\whiten{0}%
        \whiten{-1}\whiten{-2}%
        \llap{\the\SOUL@token}%
     \else
        \underline{\the\SOUL@token}%
     \fi}%
\SOUL@}
\makeatother

\newcommand*{\demp}{\fontfamily{lmtt}\selectfont}

\DeclareTextFontCommand{\textdemp}{\demp}

\begin{document}

\ifcomment
Multiline
comment
\fi
\ifcomment
\myul{Typesetting test}
% \color[rgb]{1,1,1}
$∑_i^n≠ 60º±∞π∆¬≈√j∫h≤≥µ$

$\CR \R\pro\ind\pro\gS\pro
\mqty[a&b\\c&d]$
$\pro\mathbb{P}$
$\dd{x}$

  \[
    \alpha(x)=\left\{
                \begin{array}{ll}
                  x\\
                  \frac{1}{1+e^{-kx}}\\
                  \frac{e^x-e^{-x}}{e^x+e^{-x}}
                \end{array}
              \right.
  \]

  $\expval{x}$
  
  $\chi_\rho(ghg\dmo)=\Tr(\rho_{ghg\dmo})=\Tr(\rho_g\circ\rho_h\circ\rho\dmo_g)=\Tr(\rho_h)\overset{\mbox{\scalebox{0.5}{$\Tr(AB)=\Tr(BA)$}}}{=}\chi_\rho(h)$
  	$\mathop{\oplus}_{\substack{x\in X}}$

$\mat(\rho_g)=(a_{ij}(g))_{\scriptsize \substack{1\leq i\leq d \\ 1\leq j\leq d}}$ et $\mat(\rho'_g)=(a'_{ij}(g))_{\scriptsize \substack{1\leq i'\leq d' \\ 1\leq j'\leq d'}}$



\[\int_a^b{\mathbb{R}^2}g(u, v)\dd{P_{XY}}(u, v)=\iint g(u,v) f_{XY}(u, v)\dd \lambda(u) \dd \lambda(v)\]
$$\lim_{x\to\infty} f(x)$$	
$$\iiiint_V \mu(t,u,v,w) \,dt\,du\,dv\,dw$$
$$\sum_{n=1}^{\infty} 2^{-n} = 1$$	
\begin{definition}
	Si $X$ et $Y$ sont 2 v.a. ou definit la \textsc{Covariance} entre $X$ et $Y$ comme
	$\cov(X,Y)\overset{\text{def}}{=}\E\left[(X-\E(X))(Y-\E(Y))\right]=\E(XY)-\E(X)\E(Y)$.
\end{definition}
\fi
\pagebreak

% \tableofcontents

% insert your code here
%\input{./algebra/main.tex}
%\input{./geometrie-differentielle/main.tex}
%\input{./probabilite/main.tex}
%\input{./analyse-fonctionnelle/main.tex}
% \input{./Analyse-convexe-et-dualite-en-optimisation/main.tex}
%\input{./tikz/main.tex}
%\input{./Theorie-du-distributions/main.tex}
%\input{./optimisation/mine.tex}
 \input{./modelisation/main.tex}

% yves.aubry@univ-tln.fr : algebra

\end{document}

%% !TEX encoding = UTF-8 Unicode
% !TEX TS-program = xelatex

\documentclass[french]{report}

%\usepackage[utf8]{inputenc}
%\usepackage[T1]{fontenc}
\usepackage{babel}


\newif\ifcomment
%\commenttrue # Show comments

\usepackage{physics}
\usepackage{amssymb}


\usepackage{amsthm}
% \usepackage{thmtools}
\usepackage{mathtools}
\usepackage{amsfonts}

\usepackage{color}

\usepackage{tikz}

\usepackage{geometry}
\geometry{a5paper, margin=0.1in, right=1cm}

\usepackage{dsfont}

\usepackage{graphicx}
\graphicspath{ {images/} }

\usepackage{faktor}

\usepackage{IEEEtrantools}
\usepackage{enumerate}   
\usepackage[PostScript=dvips]{"/Users/aware/Documents/Courses/diagrams"}


\newtheorem{theorem}{Théorème}[section]
\renewcommand{\thetheorem}{\arabic{theorem}}
\newtheorem{lemme}{Lemme}[section]
\renewcommand{\thelemme}{\arabic{lemme}}
\newtheorem{proposition}{Proposition}[section]
\renewcommand{\theproposition}{\arabic{proposition}}
\newtheorem{notations}{Notations}[section]
\newtheorem{problem}{Problème}[section]
\newtheorem{corollary}{Corollaire}[theorem]
\renewcommand{\thecorollary}{\arabic{corollary}}
\newtheorem{property}{Propriété}[section]
\newtheorem{objective}{Objectif}[section]

\theoremstyle{definition}
\newtheorem{definition}{Définition}[section]
\renewcommand{\thedefinition}{\arabic{definition}}
\newtheorem{exercise}{Exercice}[chapter]
\renewcommand{\theexercise}{\arabic{exercise}}
\newtheorem{example}{Exemple}[chapter]
\renewcommand{\theexample}{\arabic{example}}
\newtheorem*{solution}{Solution}
\newtheorem*{application}{Application}
\newtheorem*{notation}{Notation}
\newtheorem*{vocabulary}{Vocabulaire}
\newtheorem*{properties}{Propriétés}



\theoremstyle{remark}
\newtheorem*{remark}{Remarque}
\newtheorem*{rappel}{Rappel}


\usepackage{etoolbox}
\AtBeginEnvironment{exercise}{\small}
\AtBeginEnvironment{example}{\small}

\usepackage{cases}
\usepackage[red]{mypack}

\usepackage[framemethod=TikZ]{mdframed}

\definecolor{bg}{rgb}{0.4,0.25,0.95}
\definecolor{pagebg}{rgb}{0,0,0.5}
\surroundwithmdframed[
   topline=false,
   rightline=false,
   bottomline=false,
   leftmargin=\parindent,
   skipabove=8pt,
   skipbelow=8pt,
   linecolor=blue,
   innerbottommargin=10pt,
   % backgroundcolor=bg,font=\color{orange}\sffamily, fontcolor=white
]{definition}

\usepackage{empheq}
\usepackage[most]{tcolorbox}

\newtcbox{\mymath}[1][]{%
    nobeforeafter, math upper, tcbox raise base,
    enhanced, colframe=blue!30!black,
    colback=red!10, boxrule=1pt,
    #1}

\usepackage{unixode}


\DeclareMathOperator{\ord}{ord}
\DeclareMathOperator{\orb}{orb}
\DeclareMathOperator{\stab}{stab}
\DeclareMathOperator{\Stab}{stab}
\DeclareMathOperator{\ppcm}{ppcm}
\DeclareMathOperator{\conj}{Conj}
\DeclareMathOperator{\End}{End}
\DeclareMathOperator{\rot}{rot}
\DeclareMathOperator{\trs}{trace}
\DeclareMathOperator{\Ind}{Ind}
\DeclareMathOperator{\mat}{Mat}
\DeclareMathOperator{\id}{Id}
\DeclareMathOperator{\vect}{vect}
\DeclareMathOperator{\img}{img}
\DeclareMathOperator{\cov}{Cov}
\DeclareMathOperator{\dist}{dist}
\DeclareMathOperator{\irr}{Irr}
\DeclareMathOperator{\image}{Im}
\DeclareMathOperator{\pd}{\partial}
\DeclareMathOperator{\epi}{epi}
\DeclareMathOperator{\Argmin}{Argmin}
\DeclareMathOperator{\dom}{dom}
\DeclareMathOperator{\proj}{proj}
\DeclareMathOperator{\ctg}{ctg}
\DeclareMathOperator{\supp}{supp}
\DeclareMathOperator{\argmin}{argmin}
\DeclareMathOperator{\mult}{mult}
\DeclareMathOperator{\ch}{ch}
\DeclareMathOperator{\sh}{sh}
\DeclareMathOperator{\rang}{rang}
\DeclareMathOperator{\diam}{diam}
\DeclareMathOperator{\Epigraphe}{Epigraphe}




\usepackage{xcolor}
\everymath{\color{blue}}
%\everymath{\color[rgb]{0,1,1}}
%\pagecolor[rgb]{0,0,0.5}


\newcommand*{\pdtest}[3][]{\ensuremath{\frac{\partial^{#1} #2}{\partial #3}}}

\newcommand*{\deffunc}[6][]{\ensuremath{
\begin{array}{rcl}
#2 : #3 &\rightarrow& #4\\
#5 &\mapsto& #6
\end{array}
}}

\newcommand{\eqcolon}{\mathrel{\resizebox{\widthof{$\mathord{=}$}}{\height}{ $\!\!=\!\!\resizebox{1.2\width}{0.8\height}{\raisebox{0.23ex}{$\mathop{:}$}}\!\!$ }}}
\newcommand{\coloneq}{\mathrel{\resizebox{\widthof{$\mathord{=}$}}{\height}{ $\!\!\resizebox{1.2\width}{0.8\height}{\raisebox{0.23ex}{$\mathop{:}$}}\!\!=\!\!$ }}}
\newcommand{\eqcolonl}{\ensuremath{\mathrel{=\!\!\mathop{:}}}}
\newcommand{\coloneql}{\ensuremath{\mathrel{\mathop{:} \!\! =}}}
\newcommand{\vc}[1]{% inline column vector
  \left(\begin{smallmatrix}#1\end{smallmatrix}\right)%
}
\newcommand{\vr}[1]{% inline row vector
  \begin{smallmatrix}(\,#1\,)\end{smallmatrix}%
}
\makeatletter
\newcommand*{\defeq}{\ =\mathrel{\rlap{%
                     \raisebox{0.3ex}{$\m@th\cdot$}}%
                     \raisebox{-0.3ex}{$\m@th\cdot$}}%
                     }
\makeatother

\newcommand{\mathcircle}[1]{% inline row vector
 \overset{\circ}{#1}
}
\newcommand{\ulim}{% low limit
 \underline{\lim}
}
\newcommand{\ssi}{% iff
\iff
}
\newcommand{\ps}[2]{
\expval{#1 | #2}
}
\newcommand{\df}[1]{
\mqty{#1}
}
\newcommand{\n}[1]{
\norm{#1}
}
\newcommand{\sys}[1]{
\left\{\smqty{#1}\right.
}


\newcommand{\eqdef}{\ensuremath{\overset{\text{def}}=}}


\def\Circlearrowright{\ensuremath{%
  \rotatebox[origin=c]{230}{$\circlearrowright$}}}

\newcommand\ct[1]{\text{\rmfamily\upshape #1}}
\newcommand\question[1]{ {\color{red} ...!? \small #1}}
\newcommand\caz[1]{\left\{\begin{array} #1 \end{array}\right.}
\newcommand\const{\text{\rmfamily\upshape const}}
\newcommand\toP{ \overset{\pro}{\to}}
\newcommand\toPP{ \overset{\text{PP}}{\to}}
\newcommand{\oeq}{\mathrel{\text{\textcircled{$=$}}}}





\usepackage{xcolor}
% \usepackage[normalem]{ulem}
\usepackage{lipsum}
\makeatletter
% \newcommand\colorwave[1][blue]{\bgroup \markoverwith{\lower3.5\p@\hbox{\sixly \textcolor{#1}{\char58}}}\ULon}
%\font\sixly=lasy6 % does not re-load if already loaded, so no memory problem.

\newmdtheoremenv[
linewidth= 1pt,linecolor= blue,%
leftmargin=20,rightmargin=20,innertopmargin=0pt, innerrightmargin=40,%
tikzsetting = { draw=lightgray, line width = 0.3pt,dashed,%
dash pattern = on 15pt off 3pt},%
splittopskip=\topskip,skipbelow=\baselineskip,%
skipabove=\baselineskip,ntheorem,roundcorner=0pt,
% backgroundcolor=pagebg,font=\color{orange}\sffamily, fontcolor=white
]{examplebox}{Exemple}[section]



\newcommand\R{\mathbb{R}}
\newcommand\Z{\mathbb{Z}}
\newcommand\N{\mathbb{N}}
\newcommand\E{\mathbb{E}}
\newcommand\F{\mathcal{F}}
\newcommand\cH{\mathcal{H}}
\newcommand\V{\mathbb{V}}
\newcommand\dmo{ ^{-1} }
\newcommand\kapa{\kappa}
\newcommand\im{Im}
\newcommand\hs{\mathcal{H}}





\usepackage{soul}

\makeatletter
\newcommand*{\whiten}[1]{\llap{\textcolor{white}{{\the\SOUL@token}}\hspace{#1pt}}}
\DeclareRobustCommand*\myul{%
    \def\SOUL@everyspace{\underline{\space}\kern\z@}%
    \def\SOUL@everytoken{%
     \setbox0=\hbox{\the\SOUL@token}%
     \ifdim\dp0>\z@
        \raisebox{\dp0}{\underline{\phantom{\the\SOUL@token}}}%
        \whiten{1}\whiten{0}%
        \whiten{-1}\whiten{-2}%
        \llap{\the\SOUL@token}%
     \else
        \underline{\the\SOUL@token}%
     \fi}%
\SOUL@}
\makeatother

\newcommand*{\demp}{\fontfamily{lmtt}\selectfont}

\DeclareTextFontCommand{\textdemp}{\demp}

\begin{document}

\ifcomment
Multiline
comment
\fi
\ifcomment
\myul{Typesetting test}
% \color[rgb]{1,1,1}
$∑_i^n≠ 60º±∞π∆¬≈√j∫h≤≥µ$

$\CR \R\pro\ind\pro\gS\pro
\mqty[a&b\\c&d]$
$\pro\mathbb{P}$
$\dd{x}$

  \[
    \alpha(x)=\left\{
                \begin{array}{ll}
                  x\\
                  \frac{1}{1+e^{-kx}}\\
                  \frac{e^x-e^{-x}}{e^x+e^{-x}}
                \end{array}
              \right.
  \]

  $\expval{x}$
  
  $\chi_\rho(ghg\dmo)=\Tr(\rho_{ghg\dmo})=\Tr(\rho_g\circ\rho_h\circ\rho\dmo_g)=\Tr(\rho_h)\overset{\mbox{\scalebox{0.5}{$\Tr(AB)=\Tr(BA)$}}}{=}\chi_\rho(h)$
  	$\mathop{\oplus}_{\substack{x\in X}}$

$\mat(\rho_g)=(a_{ij}(g))_{\scriptsize \substack{1\leq i\leq d \\ 1\leq j\leq d}}$ et $\mat(\rho'_g)=(a'_{ij}(g))_{\scriptsize \substack{1\leq i'\leq d' \\ 1\leq j'\leq d'}}$



\[\int_a^b{\mathbb{R}^2}g(u, v)\dd{P_{XY}}(u, v)=\iint g(u,v) f_{XY}(u, v)\dd \lambda(u) \dd \lambda(v)\]
$$\lim_{x\to\infty} f(x)$$	
$$\iiiint_V \mu(t,u,v,w) \,dt\,du\,dv\,dw$$
$$\sum_{n=1}^{\infty} 2^{-n} = 1$$	
\begin{definition}
	Si $X$ et $Y$ sont 2 v.a. ou definit la \textsc{Covariance} entre $X$ et $Y$ comme
	$\cov(X,Y)\overset{\text{def}}{=}\E\left[(X-\E(X))(Y-\E(Y))\right]=\E(XY)-\E(X)\E(Y)$.
\end{definition}
\fi
\pagebreak

% \tableofcontents

% insert your code here
%\input{./algebra/main.tex}
%\input{./geometrie-differentielle/main.tex}
%\input{./probabilite/main.tex}
%\input{./analyse-fonctionnelle/main.tex}
% \input{./Analyse-convexe-et-dualite-en-optimisation/main.tex}
%\input{./tikz/main.tex}
%\input{./Theorie-du-distributions/main.tex}
%\input{./optimisation/mine.tex}
 \input{./modelisation/main.tex}

% yves.aubry@univ-tln.fr : algebra

\end{document}

%% !TEX encoding = UTF-8 Unicode
% !TEX TS-program = xelatex

\documentclass[french]{report}

%\usepackage[utf8]{inputenc}
%\usepackage[T1]{fontenc}
\usepackage{babel}


\newif\ifcomment
%\commenttrue # Show comments

\usepackage{physics}
\usepackage{amssymb}


\usepackage{amsthm}
% \usepackage{thmtools}
\usepackage{mathtools}
\usepackage{amsfonts}

\usepackage{color}

\usepackage{tikz}

\usepackage{geometry}
\geometry{a5paper, margin=0.1in, right=1cm}

\usepackage{dsfont}

\usepackage{graphicx}
\graphicspath{ {images/} }

\usepackage{faktor}

\usepackage{IEEEtrantools}
\usepackage{enumerate}   
\usepackage[PostScript=dvips]{"/Users/aware/Documents/Courses/diagrams"}


\newtheorem{theorem}{Théorème}[section]
\renewcommand{\thetheorem}{\arabic{theorem}}
\newtheorem{lemme}{Lemme}[section]
\renewcommand{\thelemme}{\arabic{lemme}}
\newtheorem{proposition}{Proposition}[section]
\renewcommand{\theproposition}{\arabic{proposition}}
\newtheorem{notations}{Notations}[section]
\newtheorem{problem}{Problème}[section]
\newtheorem{corollary}{Corollaire}[theorem]
\renewcommand{\thecorollary}{\arabic{corollary}}
\newtheorem{property}{Propriété}[section]
\newtheorem{objective}{Objectif}[section]

\theoremstyle{definition}
\newtheorem{definition}{Définition}[section]
\renewcommand{\thedefinition}{\arabic{definition}}
\newtheorem{exercise}{Exercice}[chapter]
\renewcommand{\theexercise}{\arabic{exercise}}
\newtheorem{example}{Exemple}[chapter]
\renewcommand{\theexample}{\arabic{example}}
\newtheorem*{solution}{Solution}
\newtheorem*{application}{Application}
\newtheorem*{notation}{Notation}
\newtheorem*{vocabulary}{Vocabulaire}
\newtheorem*{properties}{Propriétés}



\theoremstyle{remark}
\newtheorem*{remark}{Remarque}
\newtheorem*{rappel}{Rappel}


\usepackage{etoolbox}
\AtBeginEnvironment{exercise}{\small}
\AtBeginEnvironment{example}{\small}

\usepackage{cases}
\usepackage[red]{mypack}

\usepackage[framemethod=TikZ]{mdframed}

\definecolor{bg}{rgb}{0.4,0.25,0.95}
\definecolor{pagebg}{rgb}{0,0,0.5}
\surroundwithmdframed[
   topline=false,
   rightline=false,
   bottomline=false,
   leftmargin=\parindent,
   skipabove=8pt,
   skipbelow=8pt,
   linecolor=blue,
   innerbottommargin=10pt,
   % backgroundcolor=bg,font=\color{orange}\sffamily, fontcolor=white
]{definition}

\usepackage{empheq}
\usepackage[most]{tcolorbox}

\newtcbox{\mymath}[1][]{%
    nobeforeafter, math upper, tcbox raise base,
    enhanced, colframe=blue!30!black,
    colback=red!10, boxrule=1pt,
    #1}

\usepackage{unixode}


\DeclareMathOperator{\ord}{ord}
\DeclareMathOperator{\orb}{orb}
\DeclareMathOperator{\stab}{stab}
\DeclareMathOperator{\Stab}{stab}
\DeclareMathOperator{\ppcm}{ppcm}
\DeclareMathOperator{\conj}{Conj}
\DeclareMathOperator{\End}{End}
\DeclareMathOperator{\rot}{rot}
\DeclareMathOperator{\trs}{trace}
\DeclareMathOperator{\Ind}{Ind}
\DeclareMathOperator{\mat}{Mat}
\DeclareMathOperator{\id}{Id}
\DeclareMathOperator{\vect}{vect}
\DeclareMathOperator{\img}{img}
\DeclareMathOperator{\cov}{Cov}
\DeclareMathOperator{\dist}{dist}
\DeclareMathOperator{\irr}{Irr}
\DeclareMathOperator{\image}{Im}
\DeclareMathOperator{\pd}{\partial}
\DeclareMathOperator{\epi}{epi}
\DeclareMathOperator{\Argmin}{Argmin}
\DeclareMathOperator{\dom}{dom}
\DeclareMathOperator{\proj}{proj}
\DeclareMathOperator{\ctg}{ctg}
\DeclareMathOperator{\supp}{supp}
\DeclareMathOperator{\argmin}{argmin}
\DeclareMathOperator{\mult}{mult}
\DeclareMathOperator{\ch}{ch}
\DeclareMathOperator{\sh}{sh}
\DeclareMathOperator{\rang}{rang}
\DeclareMathOperator{\diam}{diam}
\DeclareMathOperator{\Epigraphe}{Epigraphe}




\usepackage{xcolor}
\everymath{\color{blue}}
%\everymath{\color[rgb]{0,1,1}}
%\pagecolor[rgb]{0,0,0.5}


\newcommand*{\pdtest}[3][]{\ensuremath{\frac{\partial^{#1} #2}{\partial #3}}}

\newcommand*{\deffunc}[6][]{\ensuremath{
\begin{array}{rcl}
#2 : #3 &\rightarrow& #4\\
#5 &\mapsto& #6
\end{array}
}}

\newcommand{\eqcolon}{\mathrel{\resizebox{\widthof{$\mathord{=}$}}{\height}{ $\!\!=\!\!\resizebox{1.2\width}{0.8\height}{\raisebox{0.23ex}{$\mathop{:}$}}\!\!$ }}}
\newcommand{\coloneq}{\mathrel{\resizebox{\widthof{$\mathord{=}$}}{\height}{ $\!\!\resizebox{1.2\width}{0.8\height}{\raisebox{0.23ex}{$\mathop{:}$}}\!\!=\!\!$ }}}
\newcommand{\eqcolonl}{\ensuremath{\mathrel{=\!\!\mathop{:}}}}
\newcommand{\coloneql}{\ensuremath{\mathrel{\mathop{:} \!\! =}}}
\newcommand{\vc}[1]{% inline column vector
  \left(\begin{smallmatrix}#1\end{smallmatrix}\right)%
}
\newcommand{\vr}[1]{% inline row vector
  \begin{smallmatrix}(\,#1\,)\end{smallmatrix}%
}
\makeatletter
\newcommand*{\defeq}{\ =\mathrel{\rlap{%
                     \raisebox{0.3ex}{$\m@th\cdot$}}%
                     \raisebox{-0.3ex}{$\m@th\cdot$}}%
                     }
\makeatother

\newcommand{\mathcircle}[1]{% inline row vector
 \overset{\circ}{#1}
}
\newcommand{\ulim}{% low limit
 \underline{\lim}
}
\newcommand{\ssi}{% iff
\iff
}
\newcommand{\ps}[2]{
\expval{#1 | #2}
}
\newcommand{\df}[1]{
\mqty{#1}
}
\newcommand{\n}[1]{
\norm{#1}
}
\newcommand{\sys}[1]{
\left\{\smqty{#1}\right.
}


\newcommand{\eqdef}{\ensuremath{\overset{\text{def}}=}}


\def\Circlearrowright{\ensuremath{%
  \rotatebox[origin=c]{230}{$\circlearrowright$}}}

\newcommand\ct[1]{\text{\rmfamily\upshape #1}}
\newcommand\question[1]{ {\color{red} ...!? \small #1}}
\newcommand\caz[1]{\left\{\begin{array} #1 \end{array}\right.}
\newcommand\const{\text{\rmfamily\upshape const}}
\newcommand\toP{ \overset{\pro}{\to}}
\newcommand\toPP{ \overset{\text{PP}}{\to}}
\newcommand{\oeq}{\mathrel{\text{\textcircled{$=$}}}}





\usepackage{xcolor}
% \usepackage[normalem]{ulem}
\usepackage{lipsum}
\makeatletter
% \newcommand\colorwave[1][blue]{\bgroup \markoverwith{\lower3.5\p@\hbox{\sixly \textcolor{#1}{\char58}}}\ULon}
%\font\sixly=lasy6 % does not re-load if already loaded, so no memory problem.

\newmdtheoremenv[
linewidth= 1pt,linecolor= blue,%
leftmargin=20,rightmargin=20,innertopmargin=0pt, innerrightmargin=40,%
tikzsetting = { draw=lightgray, line width = 0.3pt,dashed,%
dash pattern = on 15pt off 3pt},%
splittopskip=\topskip,skipbelow=\baselineskip,%
skipabove=\baselineskip,ntheorem,roundcorner=0pt,
% backgroundcolor=pagebg,font=\color{orange}\sffamily, fontcolor=white
]{examplebox}{Exemple}[section]



\newcommand\R{\mathbb{R}}
\newcommand\Z{\mathbb{Z}}
\newcommand\N{\mathbb{N}}
\newcommand\E{\mathbb{E}}
\newcommand\F{\mathcal{F}}
\newcommand\cH{\mathcal{H}}
\newcommand\V{\mathbb{V}}
\newcommand\dmo{ ^{-1} }
\newcommand\kapa{\kappa}
\newcommand\im{Im}
\newcommand\hs{\mathcal{H}}





\usepackage{soul}

\makeatletter
\newcommand*{\whiten}[1]{\llap{\textcolor{white}{{\the\SOUL@token}}\hspace{#1pt}}}
\DeclareRobustCommand*\myul{%
    \def\SOUL@everyspace{\underline{\space}\kern\z@}%
    \def\SOUL@everytoken{%
     \setbox0=\hbox{\the\SOUL@token}%
     \ifdim\dp0>\z@
        \raisebox{\dp0}{\underline{\phantom{\the\SOUL@token}}}%
        \whiten{1}\whiten{0}%
        \whiten{-1}\whiten{-2}%
        \llap{\the\SOUL@token}%
     \else
        \underline{\the\SOUL@token}%
     \fi}%
\SOUL@}
\makeatother

\newcommand*{\demp}{\fontfamily{lmtt}\selectfont}

\DeclareTextFontCommand{\textdemp}{\demp}

\begin{document}

\ifcomment
Multiline
comment
\fi
\ifcomment
\myul{Typesetting test}
% \color[rgb]{1,1,1}
$∑_i^n≠ 60º±∞π∆¬≈√j∫h≤≥µ$

$\CR \R\pro\ind\pro\gS\pro
\mqty[a&b\\c&d]$
$\pro\mathbb{P}$
$\dd{x}$

  \[
    \alpha(x)=\left\{
                \begin{array}{ll}
                  x\\
                  \frac{1}{1+e^{-kx}}\\
                  \frac{e^x-e^{-x}}{e^x+e^{-x}}
                \end{array}
              \right.
  \]

  $\expval{x}$
  
  $\chi_\rho(ghg\dmo)=\Tr(\rho_{ghg\dmo})=\Tr(\rho_g\circ\rho_h\circ\rho\dmo_g)=\Tr(\rho_h)\overset{\mbox{\scalebox{0.5}{$\Tr(AB)=\Tr(BA)$}}}{=}\chi_\rho(h)$
  	$\mathop{\oplus}_{\substack{x\in X}}$

$\mat(\rho_g)=(a_{ij}(g))_{\scriptsize \substack{1\leq i\leq d \\ 1\leq j\leq d}}$ et $\mat(\rho'_g)=(a'_{ij}(g))_{\scriptsize \substack{1\leq i'\leq d' \\ 1\leq j'\leq d'}}$



\[\int_a^b{\mathbb{R}^2}g(u, v)\dd{P_{XY}}(u, v)=\iint g(u,v) f_{XY}(u, v)\dd \lambda(u) \dd \lambda(v)\]
$$\lim_{x\to\infty} f(x)$$	
$$\iiiint_V \mu(t,u,v,w) \,dt\,du\,dv\,dw$$
$$\sum_{n=1}^{\infty} 2^{-n} = 1$$	
\begin{definition}
	Si $X$ et $Y$ sont 2 v.a. ou definit la \textsc{Covariance} entre $X$ et $Y$ comme
	$\cov(X,Y)\overset{\text{def}}{=}\E\left[(X-\E(X))(Y-\E(Y))\right]=\E(XY)-\E(X)\E(Y)$.
\end{definition}
\fi
\pagebreak

% \tableofcontents

% insert your code here
%\input{./algebra/main.tex}
%\input{./geometrie-differentielle/main.tex}
%\input{./probabilite/main.tex}
%\input{./analyse-fonctionnelle/main.tex}
% \input{./Analyse-convexe-et-dualite-en-optimisation/main.tex}
%\input{./tikz/main.tex}
%\input{./Theorie-du-distributions/main.tex}
%\input{./optimisation/mine.tex}
 \input{./modelisation/main.tex}

% yves.aubry@univ-tln.fr : algebra

\end{document}

% % !TEX encoding = UTF-8 Unicode
% !TEX TS-program = xelatex

\documentclass[french]{report}

%\usepackage[utf8]{inputenc}
%\usepackage[T1]{fontenc}
\usepackage{babel}


\newif\ifcomment
%\commenttrue # Show comments

\usepackage{physics}
\usepackage{amssymb}


\usepackage{amsthm}
% \usepackage{thmtools}
\usepackage{mathtools}
\usepackage{amsfonts}

\usepackage{color}

\usepackage{tikz}

\usepackage{geometry}
\geometry{a5paper, margin=0.1in, right=1cm}

\usepackage{dsfont}

\usepackage{graphicx}
\graphicspath{ {images/} }

\usepackage{faktor}

\usepackage{IEEEtrantools}
\usepackage{enumerate}   
\usepackage[PostScript=dvips]{"/Users/aware/Documents/Courses/diagrams"}


\newtheorem{theorem}{Théorème}[section]
\renewcommand{\thetheorem}{\arabic{theorem}}
\newtheorem{lemme}{Lemme}[section]
\renewcommand{\thelemme}{\arabic{lemme}}
\newtheorem{proposition}{Proposition}[section]
\renewcommand{\theproposition}{\arabic{proposition}}
\newtheorem{notations}{Notations}[section]
\newtheorem{problem}{Problème}[section]
\newtheorem{corollary}{Corollaire}[theorem]
\renewcommand{\thecorollary}{\arabic{corollary}}
\newtheorem{property}{Propriété}[section]
\newtheorem{objective}{Objectif}[section]

\theoremstyle{definition}
\newtheorem{definition}{Définition}[section]
\renewcommand{\thedefinition}{\arabic{definition}}
\newtheorem{exercise}{Exercice}[chapter]
\renewcommand{\theexercise}{\arabic{exercise}}
\newtheorem{example}{Exemple}[chapter]
\renewcommand{\theexample}{\arabic{example}}
\newtheorem*{solution}{Solution}
\newtheorem*{application}{Application}
\newtheorem*{notation}{Notation}
\newtheorem*{vocabulary}{Vocabulaire}
\newtheorem*{properties}{Propriétés}



\theoremstyle{remark}
\newtheorem*{remark}{Remarque}
\newtheorem*{rappel}{Rappel}


\usepackage{etoolbox}
\AtBeginEnvironment{exercise}{\small}
\AtBeginEnvironment{example}{\small}

\usepackage{cases}
\usepackage[red]{mypack}

\usepackage[framemethod=TikZ]{mdframed}

\definecolor{bg}{rgb}{0.4,0.25,0.95}
\definecolor{pagebg}{rgb}{0,0,0.5}
\surroundwithmdframed[
   topline=false,
   rightline=false,
   bottomline=false,
   leftmargin=\parindent,
   skipabove=8pt,
   skipbelow=8pt,
   linecolor=blue,
   innerbottommargin=10pt,
   % backgroundcolor=bg,font=\color{orange}\sffamily, fontcolor=white
]{definition}

\usepackage{empheq}
\usepackage[most]{tcolorbox}

\newtcbox{\mymath}[1][]{%
    nobeforeafter, math upper, tcbox raise base,
    enhanced, colframe=blue!30!black,
    colback=red!10, boxrule=1pt,
    #1}

\usepackage{unixode}


\DeclareMathOperator{\ord}{ord}
\DeclareMathOperator{\orb}{orb}
\DeclareMathOperator{\stab}{stab}
\DeclareMathOperator{\Stab}{stab}
\DeclareMathOperator{\ppcm}{ppcm}
\DeclareMathOperator{\conj}{Conj}
\DeclareMathOperator{\End}{End}
\DeclareMathOperator{\rot}{rot}
\DeclareMathOperator{\trs}{trace}
\DeclareMathOperator{\Ind}{Ind}
\DeclareMathOperator{\mat}{Mat}
\DeclareMathOperator{\id}{Id}
\DeclareMathOperator{\vect}{vect}
\DeclareMathOperator{\img}{img}
\DeclareMathOperator{\cov}{Cov}
\DeclareMathOperator{\dist}{dist}
\DeclareMathOperator{\irr}{Irr}
\DeclareMathOperator{\image}{Im}
\DeclareMathOperator{\pd}{\partial}
\DeclareMathOperator{\epi}{epi}
\DeclareMathOperator{\Argmin}{Argmin}
\DeclareMathOperator{\dom}{dom}
\DeclareMathOperator{\proj}{proj}
\DeclareMathOperator{\ctg}{ctg}
\DeclareMathOperator{\supp}{supp}
\DeclareMathOperator{\argmin}{argmin}
\DeclareMathOperator{\mult}{mult}
\DeclareMathOperator{\ch}{ch}
\DeclareMathOperator{\sh}{sh}
\DeclareMathOperator{\rang}{rang}
\DeclareMathOperator{\diam}{diam}
\DeclareMathOperator{\Epigraphe}{Epigraphe}




\usepackage{xcolor}
\everymath{\color{blue}}
%\everymath{\color[rgb]{0,1,1}}
%\pagecolor[rgb]{0,0,0.5}


\newcommand*{\pdtest}[3][]{\ensuremath{\frac{\partial^{#1} #2}{\partial #3}}}

\newcommand*{\deffunc}[6][]{\ensuremath{
\begin{array}{rcl}
#2 : #3 &\rightarrow& #4\\
#5 &\mapsto& #6
\end{array}
}}

\newcommand{\eqcolon}{\mathrel{\resizebox{\widthof{$\mathord{=}$}}{\height}{ $\!\!=\!\!\resizebox{1.2\width}{0.8\height}{\raisebox{0.23ex}{$\mathop{:}$}}\!\!$ }}}
\newcommand{\coloneq}{\mathrel{\resizebox{\widthof{$\mathord{=}$}}{\height}{ $\!\!\resizebox{1.2\width}{0.8\height}{\raisebox{0.23ex}{$\mathop{:}$}}\!\!=\!\!$ }}}
\newcommand{\eqcolonl}{\ensuremath{\mathrel{=\!\!\mathop{:}}}}
\newcommand{\coloneql}{\ensuremath{\mathrel{\mathop{:} \!\! =}}}
\newcommand{\vc}[1]{% inline column vector
  \left(\begin{smallmatrix}#1\end{smallmatrix}\right)%
}
\newcommand{\vr}[1]{% inline row vector
  \begin{smallmatrix}(\,#1\,)\end{smallmatrix}%
}
\makeatletter
\newcommand*{\defeq}{\ =\mathrel{\rlap{%
                     \raisebox{0.3ex}{$\m@th\cdot$}}%
                     \raisebox{-0.3ex}{$\m@th\cdot$}}%
                     }
\makeatother

\newcommand{\mathcircle}[1]{% inline row vector
 \overset{\circ}{#1}
}
\newcommand{\ulim}{% low limit
 \underline{\lim}
}
\newcommand{\ssi}{% iff
\iff
}
\newcommand{\ps}[2]{
\expval{#1 | #2}
}
\newcommand{\df}[1]{
\mqty{#1}
}
\newcommand{\n}[1]{
\norm{#1}
}
\newcommand{\sys}[1]{
\left\{\smqty{#1}\right.
}


\newcommand{\eqdef}{\ensuremath{\overset{\text{def}}=}}


\def\Circlearrowright{\ensuremath{%
  \rotatebox[origin=c]{230}{$\circlearrowright$}}}

\newcommand\ct[1]{\text{\rmfamily\upshape #1}}
\newcommand\question[1]{ {\color{red} ...!? \small #1}}
\newcommand\caz[1]{\left\{\begin{array} #1 \end{array}\right.}
\newcommand\const{\text{\rmfamily\upshape const}}
\newcommand\toP{ \overset{\pro}{\to}}
\newcommand\toPP{ \overset{\text{PP}}{\to}}
\newcommand{\oeq}{\mathrel{\text{\textcircled{$=$}}}}





\usepackage{xcolor}
% \usepackage[normalem]{ulem}
\usepackage{lipsum}
\makeatletter
% \newcommand\colorwave[1][blue]{\bgroup \markoverwith{\lower3.5\p@\hbox{\sixly \textcolor{#1}{\char58}}}\ULon}
%\font\sixly=lasy6 % does not re-load if already loaded, so no memory problem.

\newmdtheoremenv[
linewidth= 1pt,linecolor= blue,%
leftmargin=20,rightmargin=20,innertopmargin=0pt, innerrightmargin=40,%
tikzsetting = { draw=lightgray, line width = 0.3pt,dashed,%
dash pattern = on 15pt off 3pt},%
splittopskip=\topskip,skipbelow=\baselineskip,%
skipabove=\baselineskip,ntheorem,roundcorner=0pt,
% backgroundcolor=pagebg,font=\color{orange}\sffamily, fontcolor=white
]{examplebox}{Exemple}[section]



\newcommand\R{\mathbb{R}}
\newcommand\Z{\mathbb{Z}}
\newcommand\N{\mathbb{N}}
\newcommand\E{\mathbb{E}}
\newcommand\F{\mathcal{F}}
\newcommand\cH{\mathcal{H}}
\newcommand\V{\mathbb{V}}
\newcommand\dmo{ ^{-1} }
\newcommand\kapa{\kappa}
\newcommand\im{Im}
\newcommand\hs{\mathcal{H}}





\usepackage{soul}

\makeatletter
\newcommand*{\whiten}[1]{\llap{\textcolor{white}{{\the\SOUL@token}}\hspace{#1pt}}}
\DeclareRobustCommand*\myul{%
    \def\SOUL@everyspace{\underline{\space}\kern\z@}%
    \def\SOUL@everytoken{%
     \setbox0=\hbox{\the\SOUL@token}%
     \ifdim\dp0>\z@
        \raisebox{\dp0}{\underline{\phantom{\the\SOUL@token}}}%
        \whiten{1}\whiten{0}%
        \whiten{-1}\whiten{-2}%
        \llap{\the\SOUL@token}%
     \else
        \underline{\the\SOUL@token}%
     \fi}%
\SOUL@}
\makeatother

\newcommand*{\demp}{\fontfamily{lmtt}\selectfont}

\DeclareTextFontCommand{\textdemp}{\demp}

\begin{document}

\ifcomment
Multiline
comment
\fi
\ifcomment
\myul{Typesetting test}
% \color[rgb]{1,1,1}
$∑_i^n≠ 60º±∞π∆¬≈√j∫h≤≥µ$

$\CR \R\pro\ind\pro\gS\pro
\mqty[a&b\\c&d]$
$\pro\mathbb{P}$
$\dd{x}$

  \[
    \alpha(x)=\left\{
                \begin{array}{ll}
                  x\\
                  \frac{1}{1+e^{-kx}}\\
                  \frac{e^x-e^{-x}}{e^x+e^{-x}}
                \end{array}
              \right.
  \]

  $\expval{x}$
  
  $\chi_\rho(ghg\dmo)=\Tr(\rho_{ghg\dmo})=\Tr(\rho_g\circ\rho_h\circ\rho\dmo_g)=\Tr(\rho_h)\overset{\mbox{\scalebox{0.5}{$\Tr(AB)=\Tr(BA)$}}}{=}\chi_\rho(h)$
  	$\mathop{\oplus}_{\substack{x\in X}}$

$\mat(\rho_g)=(a_{ij}(g))_{\scriptsize \substack{1\leq i\leq d \\ 1\leq j\leq d}}$ et $\mat(\rho'_g)=(a'_{ij}(g))_{\scriptsize \substack{1\leq i'\leq d' \\ 1\leq j'\leq d'}}$



\[\int_a^b{\mathbb{R}^2}g(u, v)\dd{P_{XY}}(u, v)=\iint g(u,v) f_{XY}(u, v)\dd \lambda(u) \dd \lambda(v)\]
$$\lim_{x\to\infty} f(x)$$	
$$\iiiint_V \mu(t,u,v,w) \,dt\,du\,dv\,dw$$
$$\sum_{n=1}^{\infty} 2^{-n} = 1$$	
\begin{definition}
	Si $X$ et $Y$ sont 2 v.a. ou definit la \textsc{Covariance} entre $X$ et $Y$ comme
	$\cov(X,Y)\overset{\text{def}}{=}\E\left[(X-\E(X))(Y-\E(Y))\right]=\E(XY)-\E(X)\E(Y)$.
\end{definition}
\fi
\pagebreak

% \tableofcontents

% insert your code here
%\input{./algebra/main.tex}
%\input{./geometrie-differentielle/main.tex}
%\input{./probabilite/main.tex}
%\input{./analyse-fonctionnelle/main.tex}
% \input{./Analyse-convexe-et-dualite-en-optimisation/main.tex}
%\input{./tikz/main.tex}
%\input{./Theorie-du-distributions/main.tex}
%\input{./optimisation/mine.tex}
 \input{./modelisation/main.tex}

% yves.aubry@univ-tln.fr : algebra

\end{document}

%% !TEX encoding = UTF-8 Unicode
% !TEX TS-program = xelatex

\documentclass[french]{report}

%\usepackage[utf8]{inputenc}
%\usepackage[T1]{fontenc}
\usepackage{babel}


\newif\ifcomment
%\commenttrue # Show comments

\usepackage{physics}
\usepackage{amssymb}


\usepackage{amsthm}
% \usepackage{thmtools}
\usepackage{mathtools}
\usepackage{amsfonts}

\usepackage{color}

\usepackage{tikz}

\usepackage{geometry}
\geometry{a5paper, margin=0.1in, right=1cm}

\usepackage{dsfont}

\usepackage{graphicx}
\graphicspath{ {images/} }

\usepackage{faktor}

\usepackage{IEEEtrantools}
\usepackage{enumerate}   
\usepackage[PostScript=dvips]{"/Users/aware/Documents/Courses/diagrams"}


\newtheorem{theorem}{Théorème}[section]
\renewcommand{\thetheorem}{\arabic{theorem}}
\newtheorem{lemme}{Lemme}[section]
\renewcommand{\thelemme}{\arabic{lemme}}
\newtheorem{proposition}{Proposition}[section]
\renewcommand{\theproposition}{\arabic{proposition}}
\newtheorem{notations}{Notations}[section]
\newtheorem{problem}{Problème}[section]
\newtheorem{corollary}{Corollaire}[theorem]
\renewcommand{\thecorollary}{\arabic{corollary}}
\newtheorem{property}{Propriété}[section]
\newtheorem{objective}{Objectif}[section]

\theoremstyle{definition}
\newtheorem{definition}{Définition}[section]
\renewcommand{\thedefinition}{\arabic{definition}}
\newtheorem{exercise}{Exercice}[chapter]
\renewcommand{\theexercise}{\arabic{exercise}}
\newtheorem{example}{Exemple}[chapter]
\renewcommand{\theexample}{\arabic{example}}
\newtheorem*{solution}{Solution}
\newtheorem*{application}{Application}
\newtheorem*{notation}{Notation}
\newtheorem*{vocabulary}{Vocabulaire}
\newtheorem*{properties}{Propriétés}



\theoremstyle{remark}
\newtheorem*{remark}{Remarque}
\newtheorem*{rappel}{Rappel}


\usepackage{etoolbox}
\AtBeginEnvironment{exercise}{\small}
\AtBeginEnvironment{example}{\small}

\usepackage{cases}
\usepackage[red]{mypack}

\usepackage[framemethod=TikZ]{mdframed}

\definecolor{bg}{rgb}{0.4,0.25,0.95}
\definecolor{pagebg}{rgb}{0,0,0.5}
\surroundwithmdframed[
   topline=false,
   rightline=false,
   bottomline=false,
   leftmargin=\parindent,
   skipabove=8pt,
   skipbelow=8pt,
   linecolor=blue,
   innerbottommargin=10pt,
   % backgroundcolor=bg,font=\color{orange}\sffamily, fontcolor=white
]{definition}

\usepackage{empheq}
\usepackage[most]{tcolorbox}

\newtcbox{\mymath}[1][]{%
    nobeforeafter, math upper, tcbox raise base,
    enhanced, colframe=blue!30!black,
    colback=red!10, boxrule=1pt,
    #1}

\usepackage{unixode}


\DeclareMathOperator{\ord}{ord}
\DeclareMathOperator{\orb}{orb}
\DeclareMathOperator{\stab}{stab}
\DeclareMathOperator{\Stab}{stab}
\DeclareMathOperator{\ppcm}{ppcm}
\DeclareMathOperator{\conj}{Conj}
\DeclareMathOperator{\End}{End}
\DeclareMathOperator{\rot}{rot}
\DeclareMathOperator{\trs}{trace}
\DeclareMathOperator{\Ind}{Ind}
\DeclareMathOperator{\mat}{Mat}
\DeclareMathOperator{\id}{Id}
\DeclareMathOperator{\vect}{vect}
\DeclareMathOperator{\img}{img}
\DeclareMathOperator{\cov}{Cov}
\DeclareMathOperator{\dist}{dist}
\DeclareMathOperator{\irr}{Irr}
\DeclareMathOperator{\image}{Im}
\DeclareMathOperator{\pd}{\partial}
\DeclareMathOperator{\epi}{epi}
\DeclareMathOperator{\Argmin}{Argmin}
\DeclareMathOperator{\dom}{dom}
\DeclareMathOperator{\proj}{proj}
\DeclareMathOperator{\ctg}{ctg}
\DeclareMathOperator{\supp}{supp}
\DeclareMathOperator{\argmin}{argmin}
\DeclareMathOperator{\mult}{mult}
\DeclareMathOperator{\ch}{ch}
\DeclareMathOperator{\sh}{sh}
\DeclareMathOperator{\rang}{rang}
\DeclareMathOperator{\diam}{diam}
\DeclareMathOperator{\Epigraphe}{Epigraphe}




\usepackage{xcolor}
\everymath{\color{blue}}
%\everymath{\color[rgb]{0,1,1}}
%\pagecolor[rgb]{0,0,0.5}


\newcommand*{\pdtest}[3][]{\ensuremath{\frac{\partial^{#1} #2}{\partial #3}}}

\newcommand*{\deffunc}[6][]{\ensuremath{
\begin{array}{rcl}
#2 : #3 &\rightarrow& #4\\
#5 &\mapsto& #6
\end{array}
}}

\newcommand{\eqcolon}{\mathrel{\resizebox{\widthof{$\mathord{=}$}}{\height}{ $\!\!=\!\!\resizebox{1.2\width}{0.8\height}{\raisebox{0.23ex}{$\mathop{:}$}}\!\!$ }}}
\newcommand{\coloneq}{\mathrel{\resizebox{\widthof{$\mathord{=}$}}{\height}{ $\!\!\resizebox{1.2\width}{0.8\height}{\raisebox{0.23ex}{$\mathop{:}$}}\!\!=\!\!$ }}}
\newcommand{\eqcolonl}{\ensuremath{\mathrel{=\!\!\mathop{:}}}}
\newcommand{\coloneql}{\ensuremath{\mathrel{\mathop{:} \!\! =}}}
\newcommand{\vc}[1]{% inline column vector
  \left(\begin{smallmatrix}#1\end{smallmatrix}\right)%
}
\newcommand{\vr}[1]{% inline row vector
  \begin{smallmatrix}(\,#1\,)\end{smallmatrix}%
}
\makeatletter
\newcommand*{\defeq}{\ =\mathrel{\rlap{%
                     \raisebox{0.3ex}{$\m@th\cdot$}}%
                     \raisebox{-0.3ex}{$\m@th\cdot$}}%
                     }
\makeatother

\newcommand{\mathcircle}[1]{% inline row vector
 \overset{\circ}{#1}
}
\newcommand{\ulim}{% low limit
 \underline{\lim}
}
\newcommand{\ssi}{% iff
\iff
}
\newcommand{\ps}[2]{
\expval{#1 | #2}
}
\newcommand{\df}[1]{
\mqty{#1}
}
\newcommand{\n}[1]{
\norm{#1}
}
\newcommand{\sys}[1]{
\left\{\smqty{#1}\right.
}


\newcommand{\eqdef}{\ensuremath{\overset{\text{def}}=}}


\def\Circlearrowright{\ensuremath{%
  \rotatebox[origin=c]{230}{$\circlearrowright$}}}

\newcommand\ct[1]{\text{\rmfamily\upshape #1}}
\newcommand\question[1]{ {\color{red} ...!? \small #1}}
\newcommand\caz[1]{\left\{\begin{array} #1 \end{array}\right.}
\newcommand\const{\text{\rmfamily\upshape const}}
\newcommand\toP{ \overset{\pro}{\to}}
\newcommand\toPP{ \overset{\text{PP}}{\to}}
\newcommand{\oeq}{\mathrel{\text{\textcircled{$=$}}}}





\usepackage{xcolor}
% \usepackage[normalem]{ulem}
\usepackage{lipsum}
\makeatletter
% \newcommand\colorwave[1][blue]{\bgroup \markoverwith{\lower3.5\p@\hbox{\sixly \textcolor{#1}{\char58}}}\ULon}
%\font\sixly=lasy6 % does not re-load if already loaded, so no memory problem.

\newmdtheoremenv[
linewidth= 1pt,linecolor= blue,%
leftmargin=20,rightmargin=20,innertopmargin=0pt, innerrightmargin=40,%
tikzsetting = { draw=lightgray, line width = 0.3pt,dashed,%
dash pattern = on 15pt off 3pt},%
splittopskip=\topskip,skipbelow=\baselineskip,%
skipabove=\baselineskip,ntheorem,roundcorner=0pt,
% backgroundcolor=pagebg,font=\color{orange}\sffamily, fontcolor=white
]{examplebox}{Exemple}[section]



\newcommand\R{\mathbb{R}}
\newcommand\Z{\mathbb{Z}}
\newcommand\N{\mathbb{N}}
\newcommand\E{\mathbb{E}}
\newcommand\F{\mathcal{F}}
\newcommand\cH{\mathcal{H}}
\newcommand\V{\mathbb{V}}
\newcommand\dmo{ ^{-1} }
\newcommand\kapa{\kappa}
\newcommand\im{Im}
\newcommand\hs{\mathcal{H}}





\usepackage{soul}

\makeatletter
\newcommand*{\whiten}[1]{\llap{\textcolor{white}{{\the\SOUL@token}}\hspace{#1pt}}}
\DeclareRobustCommand*\myul{%
    \def\SOUL@everyspace{\underline{\space}\kern\z@}%
    \def\SOUL@everytoken{%
     \setbox0=\hbox{\the\SOUL@token}%
     \ifdim\dp0>\z@
        \raisebox{\dp0}{\underline{\phantom{\the\SOUL@token}}}%
        \whiten{1}\whiten{0}%
        \whiten{-1}\whiten{-2}%
        \llap{\the\SOUL@token}%
     \else
        \underline{\the\SOUL@token}%
     \fi}%
\SOUL@}
\makeatother

\newcommand*{\demp}{\fontfamily{lmtt}\selectfont}

\DeclareTextFontCommand{\textdemp}{\demp}

\begin{document}

\ifcomment
Multiline
comment
\fi
\ifcomment
\myul{Typesetting test}
% \color[rgb]{1,1,1}
$∑_i^n≠ 60º±∞π∆¬≈√j∫h≤≥µ$

$\CR \R\pro\ind\pro\gS\pro
\mqty[a&b\\c&d]$
$\pro\mathbb{P}$
$\dd{x}$

  \[
    \alpha(x)=\left\{
                \begin{array}{ll}
                  x\\
                  \frac{1}{1+e^{-kx}}\\
                  \frac{e^x-e^{-x}}{e^x+e^{-x}}
                \end{array}
              \right.
  \]

  $\expval{x}$
  
  $\chi_\rho(ghg\dmo)=\Tr(\rho_{ghg\dmo})=\Tr(\rho_g\circ\rho_h\circ\rho\dmo_g)=\Tr(\rho_h)\overset{\mbox{\scalebox{0.5}{$\Tr(AB)=\Tr(BA)$}}}{=}\chi_\rho(h)$
  	$\mathop{\oplus}_{\substack{x\in X}}$

$\mat(\rho_g)=(a_{ij}(g))_{\scriptsize \substack{1\leq i\leq d \\ 1\leq j\leq d}}$ et $\mat(\rho'_g)=(a'_{ij}(g))_{\scriptsize \substack{1\leq i'\leq d' \\ 1\leq j'\leq d'}}$



\[\int_a^b{\mathbb{R}^2}g(u, v)\dd{P_{XY}}(u, v)=\iint g(u,v) f_{XY}(u, v)\dd \lambda(u) \dd \lambda(v)\]
$$\lim_{x\to\infty} f(x)$$	
$$\iiiint_V \mu(t,u,v,w) \,dt\,du\,dv\,dw$$
$$\sum_{n=1}^{\infty} 2^{-n} = 1$$	
\begin{definition}
	Si $X$ et $Y$ sont 2 v.a. ou definit la \textsc{Covariance} entre $X$ et $Y$ comme
	$\cov(X,Y)\overset{\text{def}}{=}\E\left[(X-\E(X))(Y-\E(Y))\right]=\E(XY)-\E(X)\E(Y)$.
\end{definition}
\fi
\pagebreak

% \tableofcontents

% insert your code here
%\input{./algebra/main.tex}
%\input{./geometrie-differentielle/main.tex}
%\input{./probabilite/main.tex}
%\input{./analyse-fonctionnelle/main.tex}
% \input{./Analyse-convexe-et-dualite-en-optimisation/main.tex}
%\input{./tikz/main.tex}
%\input{./Theorie-du-distributions/main.tex}
%\input{./optimisation/mine.tex}
 \input{./modelisation/main.tex}

% yves.aubry@univ-tln.fr : algebra

\end{document}

%% !TEX encoding = UTF-8 Unicode
% !TEX TS-program = xelatex

\documentclass[french]{report}

%\usepackage[utf8]{inputenc}
%\usepackage[T1]{fontenc}
\usepackage{babel}


\newif\ifcomment
%\commenttrue # Show comments

\usepackage{physics}
\usepackage{amssymb}


\usepackage{amsthm}
% \usepackage{thmtools}
\usepackage{mathtools}
\usepackage{amsfonts}

\usepackage{color}

\usepackage{tikz}

\usepackage{geometry}
\geometry{a5paper, margin=0.1in, right=1cm}

\usepackage{dsfont}

\usepackage{graphicx}
\graphicspath{ {images/} }

\usepackage{faktor}

\usepackage{IEEEtrantools}
\usepackage{enumerate}   
\usepackage[PostScript=dvips]{"/Users/aware/Documents/Courses/diagrams"}


\newtheorem{theorem}{Théorème}[section]
\renewcommand{\thetheorem}{\arabic{theorem}}
\newtheorem{lemme}{Lemme}[section]
\renewcommand{\thelemme}{\arabic{lemme}}
\newtheorem{proposition}{Proposition}[section]
\renewcommand{\theproposition}{\arabic{proposition}}
\newtheorem{notations}{Notations}[section]
\newtheorem{problem}{Problème}[section]
\newtheorem{corollary}{Corollaire}[theorem]
\renewcommand{\thecorollary}{\arabic{corollary}}
\newtheorem{property}{Propriété}[section]
\newtheorem{objective}{Objectif}[section]

\theoremstyle{definition}
\newtheorem{definition}{Définition}[section]
\renewcommand{\thedefinition}{\arabic{definition}}
\newtheorem{exercise}{Exercice}[chapter]
\renewcommand{\theexercise}{\arabic{exercise}}
\newtheorem{example}{Exemple}[chapter]
\renewcommand{\theexample}{\arabic{example}}
\newtheorem*{solution}{Solution}
\newtheorem*{application}{Application}
\newtheorem*{notation}{Notation}
\newtheorem*{vocabulary}{Vocabulaire}
\newtheorem*{properties}{Propriétés}



\theoremstyle{remark}
\newtheorem*{remark}{Remarque}
\newtheorem*{rappel}{Rappel}


\usepackage{etoolbox}
\AtBeginEnvironment{exercise}{\small}
\AtBeginEnvironment{example}{\small}

\usepackage{cases}
\usepackage[red]{mypack}

\usepackage[framemethod=TikZ]{mdframed}

\definecolor{bg}{rgb}{0.4,0.25,0.95}
\definecolor{pagebg}{rgb}{0,0,0.5}
\surroundwithmdframed[
   topline=false,
   rightline=false,
   bottomline=false,
   leftmargin=\parindent,
   skipabove=8pt,
   skipbelow=8pt,
   linecolor=blue,
   innerbottommargin=10pt,
   % backgroundcolor=bg,font=\color{orange}\sffamily, fontcolor=white
]{definition}

\usepackage{empheq}
\usepackage[most]{tcolorbox}

\newtcbox{\mymath}[1][]{%
    nobeforeafter, math upper, tcbox raise base,
    enhanced, colframe=blue!30!black,
    colback=red!10, boxrule=1pt,
    #1}

\usepackage{unixode}


\DeclareMathOperator{\ord}{ord}
\DeclareMathOperator{\orb}{orb}
\DeclareMathOperator{\stab}{stab}
\DeclareMathOperator{\Stab}{stab}
\DeclareMathOperator{\ppcm}{ppcm}
\DeclareMathOperator{\conj}{Conj}
\DeclareMathOperator{\End}{End}
\DeclareMathOperator{\rot}{rot}
\DeclareMathOperator{\trs}{trace}
\DeclareMathOperator{\Ind}{Ind}
\DeclareMathOperator{\mat}{Mat}
\DeclareMathOperator{\id}{Id}
\DeclareMathOperator{\vect}{vect}
\DeclareMathOperator{\img}{img}
\DeclareMathOperator{\cov}{Cov}
\DeclareMathOperator{\dist}{dist}
\DeclareMathOperator{\irr}{Irr}
\DeclareMathOperator{\image}{Im}
\DeclareMathOperator{\pd}{\partial}
\DeclareMathOperator{\epi}{epi}
\DeclareMathOperator{\Argmin}{Argmin}
\DeclareMathOperator{\dom}{dom}
\DeclareMathOperator{\proj}{proj}
\DeclareMathOperator{\ctg}{ctg}
\DeclareMathOperator{\supp}{supp}
\DeclareMathOperator{\argmin}{argmin}
\DeclareMathOperator{\mult}{mult}
\DeclareMathOperator{\ch}{ch}
\DeclareMathOperator{\sh}{sh}
\DeclareMathOperator{\rang}{rang}
\DeclareMathOperator{\diam}{diam}
\DeclareMathOperator{\Epigraphe}{Epigraphe}




\usepackage{xcolor}
\everymath{\color{blue}}
%\everymath{\color[rgb]{0,1,1}}
%\pagecolor[rgb]{0,0,0.5}


\newcommand*{\pdtest}[3][]{\ensuremath{\frac{\partial^{#1} #2}{\partial #3}}}

\newcommand*{\deffunc}[6][]{\ensuremath{
\begin{array}{rcl}
#2 : #3 &\rightarrow& #4\\
#5 &\mapsto& #6
\end{array}
}}

\newcommand{\eqcolon}{\mathrel{\resizebox{\widthof{$\mathord{=}$}}{\height}{ $\!\!=\!\!\resizebox{1.2\width}{0.8\height}{\raisebox{0.23ex}{$\mathop{:}$}}\!\!$ }}}
\newcommand{\coloneq}{\mathrel{\resizebox{\widthof{$\mathord{=}$}}{\height}{ $\!\!\resizebox{1.2\width}{0.8\height}{\raisebox{0.23ex}{$\mathop{:}$}}\!\!=\!\!$ }}}
\newcommand{\eqcolonl}{\ensuremath{\mathrel{=\!\!\mathop{:}}}}
\newcommand{\coloneql}{\ensuremath{\mathrel{\mathop{:} \!\! =}}}
\newcommand{\vc}[1]{% inline column vector
  \left(\begin{smallmatrix}#1\end{smallmatrix}\right)%
}
\newcommand{\vr}[1]{% inline row vector
  \begin{smallmatrix}(\,#1\,)\end{smallmatrix}%
}
\makeatletter
\newcommand*{\defeq}{\ =\mathrel{\rlap{%
                     \raisebox{0.3ex}{$\m@th\cdot$}}%
                     \raisebox{-0.3ex}{$\m@th\cdot$}}%
                     }
\makeatother

\newcommand{\mathcircle}[1]{% inline row vector
 \overset{\circ}{#1}
}
\newcommand{\ulim}{% low limit
 \underline{\lim}
}
\newcommand{\ssi}{% iff
\iff
}
\newcommand{\ps}[2]{
\expval{#1 | #2}
}
\newcommand{\df}[1]{
\mqty{#1}
}
\newcommand{\n}[1]{
\norm{#1}
}
\newcommand{\sys}[1]{
\left\{\smqty{#1}\right.
}


\newcommand{\eqdef}{\ensuremath{\overset{\text{def}}=}}


\def\Circlearrowright{\ensuremath{%
  \rotatebox[origin=c]{230}{$\circlearrowright$}}}

\newcommand\ct[1]{\text{\rmfamily\upshape #1}}
\newcommand\question[1]{ {\color{red} ...!? \small #1}}
\newcommand\caz[1]{\left\{\begin{array} #1 \end{array}\right.}
\newcommand\const{\text{\rmfamily\upshape const}}
\newcommand\toP{ \overset{\pro}{\to}}
\newcommand\toPP{ \overset{\text{PP}}{\to}}
\newcommand{\oeq}{\mathrel{\text{\textcircled{$=$}}}}





\usepackage{xcolor}
% \usepackage[normalem]{ulem}
\usepackage{lipsum}
\makeatletter
% \newcommand\colorwave[1][blue]{\bgroup \markoverwith{\lower3.5\p@\hbox{\sixly \textcolor{#1}{\char58}}}\ULon}
%\font\sixly=lasy6 % does not re-load if already loaded, so no memory problem.

\newmdtheoremenv[
linewidth= 1pt,linecolor= blue,%
leftmargin=20,rightmargin=20,innertopmargin=0pt, innerrightmargin=40,%
tikzsetting = { draw=lightgray, line width = 0.3pt,dashed,%
dash pattern = on 15pt off 3pt},%
splittopskip=\topskip,skipbelow=\baselineskip,%
skipabove=\baselineskip,ntheorem,roundcorner=0pt,
% backgroundcolor=pagebg,font=\color{orange}\sffamily, fontcolor=white
]{examplebox}{Exemple}[section]



\newcommand\R{\mathbb{R}}
\newcommand\Z{\mathbb{Z}}
\newcommand\N{\mathbb{N}}
\newcommand\E{\mathbb{E}}
\newcommand\F{\mathcal{F}}
\newcommand\cH{\mathcal{H}}
\newcommand\V{\mathbb{V}}
\newcommand\dmo{ ^{-1} }
\newcommand\kapa{\kappa}
\newcommand\im{Im}
\newcommand\hs{\mathcal{H}}





\usepackage{soul}

\makeatletter
\newcommand*{\whiten}[1]{\llap{\textcolor{white}{{\the\SOUL@token}}\hspace{#1pt}}}
\DeclareRobustCommand*\myul{%
    \def\SOUL@everyspace{\underline{\space}\kern\z@}%
    \def\SOUL@everytoken{%
     \setbox0=\hbox{\the\SOUL@token}%
     \ifdim\dp0>\z@
        \raisebox{\dp0}{\underline{\phantom{\the\SOUL@token}}}%
        \whiten{1}\whiten{0}%
        \whiten{-1}\whiten{-2}%
        \llap{\the\SOUL@token}%
     \else
        \underline{\the\SOUL@token}%
     \fi}%
\SOUL@}
\makeatother

\newcommand*{\demp}{\fontfamily{lmtt}\selectfont}

\DeclareTextFontCommand{\textdemp}{\demp}

\begin{document}

\ifcomment
Multiline
comment
\fi
\ifcomment
\myul{Typesetting test}
% \color[rgb]{1,1,1}
$∑_i^n≠ 60º±∞π∆¬≈√j∫h≤≥µ$

$\CR \R\pro\ind\pro\gS\pro
\mqty[a&b\\c&d]$
$\pro\mathbb{P}$
$\dd{x}$

  \[
    \alpha(x)=\left\{
                \begin{array}{ll}
                  x\\
                  \frac{1}{1+e^{-kx}}\\
                  \frac{e^x-e^{-x}}{e^x+e^{-x}}
                \end{array}
              \right.
  \]

  $\expval{x}$
  
  $\chi_\rho(ghg\dmo)=\Tr(\rho_{ghg\dmo})=\Tr(\rho_g\circ\rho_h\circ\rho\dmo_g)=\Tr(\rho_h)\overset{\mbox{\scalebox{0.5}{$\Tr(AB)=\Tr(BA)$}}}{=}\chi_\rho(h)$
  	$\mathop{\oplus}_{\substack{x\in X}}$

$\mat(\rho_g)=(a_{ij}(g))_{\scriptsize \substack{1\leq i\leq d \\ 1\leq j\leq d}}$ et $\mat(\rho'_g)=(a'_{ij}(g))_{\scriptsize \substack{1\leq i'\leq d' \\ 1\leq j'\leq d'}}$



\[\int_a^b{\mathbb{R}^2}g(u, v)\dd{P_{XY}}(u, v)=\iint g(u,v) f_{XY}(u, v)\dd \lambda(u) \dd \lambda(v)\]
$$\lim_{x\to\infty} f(x)$$	
$$\iiiint_V \mu(t,u,v,w) \,dt\,du\,dv\,dw$$
$$\sum_{n=1}^{\infty} 2^{-n} = 1$$	
\begin{definition}
	Si $X$ et $Y$ sont 2 v.a. ou definit la \textsc{Covariance} entre $X$ et $Y$ comme
	$\cov(X,Y)\overset{\text{def}}{=}\E\left[(X-\E(X))(Y-\E(Y))\right]=\E(XY)-\E(X)\E(Y)$.
\end{definition}
\fi
\pagebreak

% \tableofcontents

% insert your code here
%\input{./algebra/main.tex}
%\input{./geometrie-differentielle/main.tex}
%\input{./probabilite/main.tex}
%\input{./analyse-fonctionnelle/main.tex}
% \input{./Analyse-convexe-et-dualite-en-optimisation/main.tex}
%\input{./tikz/main.tex}
%\input{./Theorie-du-distributions/main.tex}
%\input{./optimisation/mine.tex}
 \input{./modelisation/main.tex}

% yves.aubry@univ-tln.fr : algebra

\end{document}

%\input{./optimisation/mine.tex}
 % !TEX encoding = UTF-8 Unicode
% !TEX TS-program = xelatex

\documentclass[french]{report}

%\usepackage[utf8]{inputenc}
%\usepackage[T1]{fontenc}
\usepackage{babel}


\newif\ifcomment
%\commenttrue # Show comments

\usepackage{physics}
\usepackage{amssymb}


\usepackage{amsthm}
% \usepackage{thmtools}
\usepackage{mathtools}
\usepackage{amsfonts}

\usepackage{color}

\usepackage{tikz}

\usepackage{geometry}
\geometry{a5paper, margin=0.1in, right=1cm}

\usepackage{dsfont}

\usepackage{graphicx}
\graphicspath{ {images/} }

\usepackage{faktor}

\usepackage{IEEEtrantools}
\usepackage{enumerate}   
\usepackage[PostScript=dvips]{"/Users/aware/Documents/Courses/diagrams"}


\newtheorem{theorem}{Théorème}[section]
\renewcommand{\thetheorem}{\arabic{theorem}}
\newtheorem{lemme}{Lemme}[section]
\renewcommand{\thelemme}{\arabic{lemme}}
\newtheorem{proposition}{Proposition}[section]
\renewcommand{\theproposition}{\arabic{proposition}}
\newtheorem{notations}{Notations}[section]
\newtheorem{problem}{Problème}[section]
\newtheorem{corollary}{Corollaire}[theorem]
\renewcommand{\thecorollary}{\arabic{corollary}}
\newtheorem{property}{Propriété}[section]
\newtheorem{objective}{Objectif}[section]

\theoremstyle{definition}
\newtheorem{definition}{Définition}[section]
\renewcommand{\thedefinition}{\arabic{definition}}
\newtheorem{exercise}{Exercice}[chapter]
\renewcommand{\theexercise}{\arabic{exercise}}
\newtheorem{example}{Exemple}[chapter]
\renewcommand{\theexample}{\arabic{example}}
\newtheorem*{solution}{Solution}
\newtheorem*{application}{Application}
\newtheorem*{notation}{Notation}
\newtheorem*{vocabulary}{Vocabulaire}
\newtheorem*{properties}{Propriétés}



\theoremstyle{remark}
\newtheorem*{remark}{Remarque}
\newtheorem*{rappel}{Rappel}


\usepackage{etoolbox}
\AtBeginEnvironment{exercise}{\small}
\AtBeginEnvironment{example}{\small}

\usepackage{cases}
\usepackage[red]{mypack}

\usepackage[framemethod=TikZ]{mdframed}

\definecolor{bg}{rgb}{0.4,0.25,0.95}
\definecolor{pagebg}{rgb}{0,0,0.5}
\surroundwithmdframed[
   topline=false,
   rightline=false,
   bottomline=false,
   leftmargin=\parindent,
   skipabove=8pt,
   skipbelow=8pt,
   linecolor=blue,
   innerbottommargin=10pt,
   % backgroundcolor=bg,font=\color{orange}\sffamily, fontcolor=white
]{definition}

\usepackage{empheq}
\usepackage[most]{tcolorbox}

\newtcbox{\mymath}[1][]{%
    nobeforeafter, math upper, tcbox raise base,
    enhanced, colframe=blue!30!black,
    colback=red!10, boxrule=1pt,
    #1}

\usepackage{unixode}


\DeclareMathOperator{\ord}{ord}
\DeclareMathOperator{\orb}{orb}
\DeclareMathOperator{\stab}{stab}
\DeclareMathOperator{\Stab}{stab}
\DeclareMathOperator{\ppcm}{ppcm}
\DeclareMathOperator{\conj}{Conj}
\DeclareMathOperator{\End}{End}
\DeclareMathOperator{\rot}{rot}
\DeclareMathOperator{\trs}{trace}
\DeclareMathOperator{\Ind}{Ind}
\DeclareMathOperator{\mat}{Mat}
\DeclareMathOperator{\id}{Id}
\DeclareMathOperator{\vect}{vect}
\DeclareMathOperator{\img}{img}
\DeclareMathOperator{\cov}{Cov}
\DeclareMathOperator{\dist}{dist}
\DeclareMathOperator{\irr}{Irr}
\DeclareMathOperator{\image}{Im}
\DeclareMathOperator{\pd}{\partial}
\DeclareMathOperator{\epi}{epi}
\DeclareMathOperator{\Argmin}{Argmin}
\DeclareMathOperator{\dom}{dom}
\DeclareMathOperator{\proj}{proj}
\DeclareMathOperator{\ctg}{ctg}
\DeclareMathOperator{\supp}{supp}
\DeclareMathOperator{\argmin}{argmin}
\DeclareMathOperator{\mult}{mult}
\DeclareMathOperator{\ch}{ch}
\DeclareMathOperator{\sh}{sh}
\DeclareMathOperator{\rang}{rang}
\DeclareMathOperator{\diam}{diam}
\DeclareMathOperator{\Epigraphe}{Epigraphe}




\usepackage{xcolor}
\everymath{\color{blue}}
%\everymath{\color[rgb]{0,1,1}}
%\pagecolor[rgb]{0,0,0.5}


\newcommand*{\pdtest}[3][]{\ensuremath{\frac{\partial^{#1} #2}{\partial #3}}}

\newcommand*{\deffunc}[6][]{\ensuremath{
\begin{array}{rcl}
#2 : #3 &\rightarrow& #4\\
#5 &\mapsto& #6
\end{array}
}}

\newcommand{\eqcolon}{\mathrel{\resizebox{\widthof{$\mathord{=}$}}{\height}{ $\!\!=\!\!\resizebox{1.2\width}{0.8\height}{\raisebox{0.23ex}{$\mathop{:}$}}\!\!$ }}}
\newcommand{\coloneq}{\mathrel{\resizebox{\widthof{$\mathord{=}$}}{\height}{ $\!\!\resizebox{1.2\width}{0.8\height}{\raisebox{0.23ex}{$\mathop{:}$}}\!\!=\!\!$ }}}
\newcommand{\eqcolonl}{\ensuremath{\mathrel{=\!\!\mathop{:}}}}
\newcommand{\coloneql}{\ensuremath{\mathrel{\mathop{:} \!\! =}}}
\newcommand{\vc}[1]{% inline column vector
  \left(\begin{smallmatrix}#1\end{smallmatrix}\right)%
}
\newcommand{\vr}[1]{% inline row vector
  \begin{smallmatrix}(\,#1\,)\end{smallmatrix}%
}
\makeatletter
\newcommand*{\defeq}{\ =\mathrel{\rlap{%
                     \raisebox{0.3ex}{$\m@th\cdot$}}%
                     \raisebox{-0.3ex}{$\m@th\cdot$}}%
                     }
\makeatother

\newcommand{\mathcircle}[1]{% inline row vector
 \overset{\circ}{#1}
}
\newcommand{\ulim}{% low limit
 \underline{\lim}
}
\newcommand{\ssi}{% iff
\iff
}
\newcommand{\ps}[2]{
\expval{#1 | #2}
}
\newcommand{\df}[1]{
\mqty{#1}
}
\newcommand{\n}[1]{
\norm{#1}
}
\newcommand{\sys}[1]{
\left\{\smqty{#1}\right.
}


\newcommand{\eqdef}{\ensuremath{\overset{\text{def}}=}}


\def\Circlearrowright{\ensuremath{%
  \rotatebox[origin=c]{230}{$\circlearrowright$}}}

\newcommand\ct[1]{\text{\rmfamily\upshape #1}}
\newcommand\question[1]{ {\color{red} ...!? \small #1}}
\newcommand\caz[1]{\left\{\begin{array} #1 \end{array}\right.}
\newcommand\const{\text{\rmfamily\upshape const}}
\newcommand\toP{ \overset{\pro}{\to}}
\newcommand\toPP{ \overset{\text{PP}}{\to}}
\newcommand{\oeq}{\mathrel{\text{\textcircled{$=$}}}}





\usepackage{xcolor}
% \usepackage[normalem]{ulem}
\usepackage{lipsum}
\makeatletter
% \newcommand\colorwave[1][blue]{\bgroup \markoverwith{\lower3.5\p@\hbox{\sixly \textcolor{#1}{\char58}}}\ULon}
%\font\sixly=lasy6 % does not re-load if already loaded, so no memory problem.

\newmdtheoremenv[
linewidth= 1pt,linecolor= blue,%
leftmargin=20,rightmargin=20,innertopmargin=0pt, innerrightmargin=40,%
tikzsetting = { draw=lightgray, line width = 0.3pt,dashed,%
dash pattern = on 15pt off 3pt},%
splittopskip=\topskip,skipbelow=\baselineskip,%
skipabove=\baselineskip,ntheorem,roundcorner=0pt,
% backgroundcolor=pagebg,font=\color{orange}\sffamily, fontcolor=white
]{examplebox}{Exemple}[section]



\newcommand\R{\mathbb{R}}
\newcommand\Z{\mathbb{Z}}
\newcommand\N{\mathbb{N}}
\newcommand\E{\mathbb{E}}
\newcommand\F{\mathcal{F}}
\newcommand\cH{\mathcal{H}}
\newcommand\V{\mathbb{V}}
\newcommand\dmo{ ^{-1} }
\newcommand\kapa{\kappa}
\newcommand\im{Im}
\newcommand\hs{\mathcal{H}}





\usepackage{soul}

\makeatletter
\newcommand*{\whiten}[1]{\llap{\textcolor{white}{{\the\SOUL@token}}\hspace{#1pt}}}
\DeclareRobustCommand*\myul{%
    \def\SOUL@everyspace{\underline{\space}\kern\z@}%
    \def\SOUL@everytoken{%
     \setbox0=\hbox{\the\SOUL@token}%
     \ifdim\dp0>\z@
        \raisebox{\dp0}{\underline{\phantom{\the\SOUL@token}}}%
        \whiten{1}\whiten{0}%
        \whiten{-1}\whiten{-2}%
        \llap{\the\SOUL@token}%
     \else
        \underline{\the\SOUL@token}%
     \fi}%
\SOUL@}
\makeatother

\newcommand*{\demp}{\fontfamily{lmtt}\selectfont}

\DeclareTextFontCommand{\textdemp}{\demp}

\begin{document}

\ifcomment
Multiline
comment
\fi
\ifcomment
\myul{Typesetting test}
% \color[rgb]{1,1,1}
$∑_i^n≠ 60º±∞π∆¬≈√j∫h≤≥µ$

$\CR \R\pro\ind\pro\gS\pro
\mqty[a&b\\c&d]$
$\pro\mathbb{P}$
$\dd{x}$

  \[
    \alpha(x)=\left\{
                \begin{array}{ll}
                  x\\
                  \frac{1}{1+e^{-kx}}\\
                  \frac{e^x-e^{-x}}{e^x+e^{-x}}
                \end{array}
              \right.
  \]

  $\expval{x}$
  
  $\chi_\rho(ghg\dmo)=\Tr(\rho_{ghg\dmo})=\Tr(\rho_g\circ\rho_h\circ\rho\dmo_g)=\Tr(\rho_h)\overset{\mbox{\scalebox{0.5}{$\Tr(AB)=\Tr(BA)$}}}{=}\chi_\rho(h)$
  	$\mathop{\oplus}_{\substack{x\in X}}$

$\mat(\rho_g)=(a_{ij}(g))_{\scriptsize \substack{1\leq i\leq d \\ 1\leq j\leq d}}$ et $\mat(\rho'_g)=(a'_{ij}(g))_{\scriptsize \substack{1\leq i'\leq d' \\ 1\leq j'\leq d'}}$



\[\int_a^b{\mathbb{R}^2}g(u, v)\dd{P_{XY}}(u, v)=\iint g(u,v) f_{XY}(u, v)\dd \lambda(u) \dd \lambda(v)\]
$$\lim_{x\to\infty} f(x)$$	
$$\iiiint_V \mu(t,u,v,w) \,dt\,du\,dv\,dw$$
$$\sum_{n=1}^{\infty} 2^{-n} = 1$$	
\begin{definition}
	Si $X$ et $Y$ sont 2 v.a. ou definit la \textsc{Covariance} entre $X$ et $Y$ comme
	$\cov(X,Y)\overset{\text{def}}{=}\E\left[(X-\E(X))(Y-\E(Y))\right]=\E(XY)-\E(X)\E(Y)$.
\end{definition}
\fi
\pagebreak

% \tableofcontents

% insert your code here
%\input{./algebra/main.tex}
%\input{./geometrie-differentielle/main.tex}
%\input{./probabilite/main.tex}
%\input{./analyse-fonctionnelle/main.tex}
% \input{./Analyse-convexe-et-dualite-en-optimisation/main.tex}
%\input{./tikz/main.tex}
%\input{./Theorie-du-distributions/main.tex}
%\input{./optimisation/mine.tex}
 \input{./modelisation/main.tex}

% yves.aubry@univ-tln.fr : algebra

\end{document}


% yves.aubry@univ-tln.fr : algebra

\end{document}

% % !TEX encoding = UTF-8 Unicode
% !TEX TS-program = xelatex

\documentclass[french]{report}

%\usepackage[utf8]{inputenc}
%\usepackage[T1]{fontenc}
\usepackage{babel}


\newif\ifcomment
%\commenttrue # Show comments

\usepackage{physics}
\usepackage{amssymb}


\usepackage{amsthm}
% \usepackage{thmtools}
\usepackage{mathtools}
\usepackage{amsfonts}

\usepackage{color}

\usepackage{tikz}

\usepackage{geometry}
\geometry{a5paper, margin=0.1in, right=1cm}

\usepackage{dsfont}

\usepackage{graphicx}
\graphicspath{ {images/} }

\usepackage{faktor}

\usepackage{IEEEtrantools}
\usepackage{enumerate}   
\usepackage[PostScript=dvips]{"/Users/aware/Documents/Courses/diagrams"}


\newtheorem{theorem}{Théorème}[section]
\renewcommand{\thetheorem}{\arabic{theorem}}
\newtheorem{lemme}{Lemme}[section]
\renewcommand{\thelemme}{\arabic{lemme}}
\newtheorem{proposition}{Proposition}[section]
\renewcommand{\theproposition}{\arabic{proposition}}
\newtheorem{notations}{Notations}[section]
\newtheorem{problem}{Problème}[section]
\newtheorem{corollary}{Corollaire}[theorem]
\renewcommand{\thecorollary}{\arabic{corollary}}
\newtheorem{property}{Propriété}[section]
\newtheorem{objective}{Objectif}[section]

\theoremstyle{definition}
\newtheorem{definition}{Définition}[section]
\renewcommand{\thedefinition}{\arabic{definition}}
\newtheorem{exercise}{Exercice}[chapter]
\renewcommand{\theexercise}{\arabic{exercise}}
\newtheorem{example}{Exemple}[chapter]
\renewcommand{\theexample}{\arabic{example}}
\newtheorem*{solution}{Solution}
\newtheorem*{application}{Application}
\newtheorem*{notation}{Notation}
\newtheorem*{vocabulary}{Vocabulaire}
\newtheorem*{properties}{Propriétés}



\theoremstyle{remark}
\newtheorem*{remark}{Remarque}
\newtheorem*{rappel}{Rappel}


\usepackage{etoolbox}
\AtBeginEnvironment{exercise}{\small}
\AtBeginEnvironment{example}{\small}

\usepackage{cases}
\usepackage[red]{mypack}

\usepackage[framemethod=TikZ]{mdframed}

\definecolor{bg}{rgb}{0.4,0.25,0.95}
\definecolor{pagebg}{rgb}{0,0,0.5}
\surroundwithmdframed[
   topline=false,
   rightline=false,
   bottomline=false,
   leftmargin=\parindent,
   skipabove=8pt,
   skipbelow=8pt,
   linecolor=blue,
   innerbottommargin=10pt,
   % backgroundcolor=bg,font=\color{orange}\sffamily, fontcolor=white
]{definition}

\usepackage{empheq}
\usepackage[most]{tcolorbox}

\newtcbox{\mymath}[1][]{%
    nobeforeafter, math upper, tcbox raise base,
    enhanced, colframe=blue!30!black,
    colback=red!10, boxrule=1pt,
    #1}

\usepackage{unixode}


\DeclareMathOperator{\ord}{ord}
\DeclareMathOperator{\orb}{orb}
\DeclareMathOperator{\stab}{stab}
\DeclareMathOperator{\Stab}{stab}
\DeclareMathOperator{\ppcm}{ppcm}
\DeclareMathOperator{\conj}{Conj}
\DeclareMathOperator{\End}{End}
\DeclareMathOperator{\rot}{rot}
\DeclareMathOperator{\trs}{trace}
\DeclareMathOperator{\Ind}{Ind}
\DeclareMathOperator{\mat}{Mat}
\DeclareMathOperator{\id}{Id}
\DeclareMathOperator{\vect}{vect}
\DeclareMathOperator{\img}{img}
\DeclareMathOperator{\cov}{Cov}
\DeclareMathOperator{\dist}{dist}
\DeclareMathOperator{\irr}{Irr}
\DeclareMathOperator{\image}{Im}
\DeclareMathOperator{\pd}{\partial}
\DeclareMathOperator{\epi}{epi}
\DeclareMathOperator{\Argmin}{Argmin}
\DeclareMathOperator{\dom}{dom}
\DeclareMathOperator{\proj}{proj}
\DeclareMathOperator{\ctg}{ctg}
\DeclareMathOperator{\supp}{supp}
\DeclareMathOperator{\argmin}{argmin}
\DeclareMathOperator{\mult}{mult}
\DeclareMathOperator{\ch}{ch}
\DeclareMathOperator{\sh}{sh}
\DeclareMathOperator{\rang}{rang}
\DeclareMathOperator{\diam}{diam}
\DeclareMathOperator{\Epigraphe}{Epigraphe}




\usepackage{xcolor}
\everymath{\color{blue}}
%\everymath{\color[rgb]{0,1,1}}
%\pagecolor[rgb]{0,0,0.5}


\newcommand*{\pdtest}[3][]{\ensuremath{\frac{\partial^{#1} #2}{\partial #3}}}

\newcommand*{\deffunc}[6][]{\ensuremath{
\begin{array}{rcl}
#2 : #3 &\rightarrow& #4\\
#5 &\mapsto& #6
\end{array}
}}

\newcommand{\eqcolon}{\mathrel{\resizebox{\widthof{$\mathord{=}$}}{\height}{ $\!\!=\!\!\resizebox{1.2\width}{0.8\height}{\raisebox{0.23ex}{$\mathop{:}$}}\!\!$ }}}
\newcommand{\coloneq}{\mathrel{\resizebox{\widthof{$\mathord{=}$}}{\height}{ $\!\!\resizebox{1.2\width}{0.8\height}{\raisebox{0.23ex}{$\mathop{:}$}}\!\!=\!\!$ }}}
\newcommand{\eqcolonl}{\ensuremath{\mathrel{=\!\!\mathop{:}}}}
\newcommand{\coloneql}{\ensuremath{\mathrel{\mathop{:} \!\! =}}}
\newcommand{\vc}[1]{% inline column vector
  \left(\begin{smallmatrix}#1\end{smallmatrix}\right)%
}
\newcommand{\vr}[1]{% inline row vector
  \begin{smallmatrix}(\,#1\,)\end{smallmatrix}%
}
\makeatletter
\newcommand*{\defeq}{\ =\mathrel{\rlap{%
                     \raisebox{0.3ex}{$\m@th\cdot$}}%
                     \raisebox{-0.3ex}{$\m@th\cdot$}}%
                     }
\makeatother

\newcommand{\mathcircle}[1]{% inline row vector
 \overset{\circ}{#1}
}
\newcommand{\ulim}{% low limit
 \underline{\lim}
}
\newcommand{\ssi}{% iff
\iff
}
\newcommand{\ps}[2]{
\expval{#1 | #2}
}
\newcommand{\df}[1]{
\mqty{#1}
}
\newcommand{\n}[1]{
\norm{#1}
}
\newcommand{\sys}[1]{
\left\{\smqty{#1}\right.
}


\newcommand{\eqdef}{\ensuremath{\overset{\text{def}}=}}


\def\Circlearrowright{\ensuremath{%
  \rotatebox[origin=c]{230}{$\circlearrowright$}}}

\newcommand\ct[1]{\text{\rmfamily\upshape #1}}
\newcommand\question[1]{ {\color{red} ...!? \small #1}}
\newcommand\caz[1]{\left\{\begin{array} #1 \end{array}\right.}
\newcommand\const{\text{\rmfamily\upshape const}}
\newcommand\toP{ \overset{\pro}{\to}}
\newcommand\toPP{ \overset{\text{PP}}{\to}}
\newcommand{\oeq}{\mathrel{\text{\textcircled{$=$}}}}





\usepackage{xcolor}
% \usepackage[normalem]{ulem}
\usepackage{lipsum}
\makeatletter
% \newcommand\colorwave[1][blue]{\bgroup \markoverwith{\lower3.5\p@\hbox{\sixly \textcolor{#1}{\char58}}}\ULon}
%\font\sixly=lasy6 % does not re-load if already loaded, so no memory problem.

\newmdtheoremenv[
linewidth= 1pt,linecolor= blue,%
leftmargin=20,rightmargin=20,innertopmargin=0pt, innerrightmargin=40,%
tikzsetting = { draw=lightgray, line width = 0.3pt,dashed,%
dash pattern = on 15pt off 3pt},%
splittopskip=\topskip,skipbelow=\baselineskip,%
skipabove=\baselineskip,ntheorem,roundcorner=0pt,
% backgroundcolor=pagebg,font=\color{orange}\sffamily, fontcolor=white
]{examplebox}{Exemple}[section]



\newcommand\R{\mathbb{R}}
\newcommand\Z{\mathbb{Z}}
\newcommand\N{\mathbb{N}}
\newcommand\E{\mathbb{E}}
\newcommand\F{\mathcal{F}}
\newcommand\cH{\mathcal{H}}
\newcommand\V{\mathbb{V}}
\newcommand\dmo{ ^{-1} }
\newcommand\kapa{\kappa}
\newcommand\im{Im}
\newcommand\hs{\mathcal{H}}





\usepackage{soul}

\makeatletter
\newcommand*{\whiten}[1]{\llap{\textcolor{white}{{\the\SOUL@token}}\hspace{#1pt}}}
\DeclareRobustCommand*\myul{%
    \def\SOUL@everyspace{\underline{\space}\kern\z@}%
    \def\SOUL@everytoken{%
     \setbox0=\hbox{\the\SOUL@token}%
     \ifdim\dp0>\z@
        \raisebox{\dp0}{\underline{\phantom{\the\SOUL@token}}}%
        \whiten{1}\whiten{0}%
        \whiten{-1}\whiten{-2}%
        \llap{\the\SOUL@token}%
     \else
        \underline{\the\SOUL@token}%
     \fi}%
\SOUL@}
\makeatother

\newcommand*{\demp}{\fontfamily{lmtt}\selectfont}

\DeclareTextFontCommand{\textdemp}{\demp}

\begin{document}

\ifcomment
Multiline
comment
\fi
\ifcomment
\myul{Typesetting test}
% \color[rgb]{1,1,1}
$∑_i^n≠ 60º±∞π∆¬≈√j∫h≤≥µ$

$\CR \R\pro\ind\pro\gS\pro
\mqty[a&b\\c&d]$
$\pro\mathbb{P}$
$\dd{x}$

  \[
    \alpha(x)=\left\{
                \begin{array}{ll}
                  x\\
                  \frac{1}{1+e^{-kx}}\\
                  \frac{e^x-e^{-x}}{e^x+e^{-x}}
                \end{array}
              \right.
  \]

  $\expval{x}$
  
  $\chi_\rho(ghg\dmo)=\Tr(\rho_{ghg\dmo})=\Tr(\rho_g\circ\rho_h\circ\rho\dmo_g)=\Tr(\rho_h)\overset{\mbox{\scalebox{0.5}{$\Tr(AB)=\Tr(BA)$}}}{=}\chi_\rho(h)$
  	$\mathop{\oplus}_{\substack{x\in X}}$

$\mat(\rho_g)=(a_{ij}(g))_{\scriptsize \substack{1\leq i\leq d \\ 1\leq j\leq d}}$ et $\mat(\rho'_g)=(a'_{ij}(g))_{\scriptsize \substack{1\leq i'\leq d' \\ 1\leq j'\leq d'}}$



\[\int_a^b{\mathbb{R}^2}g(u, v)\dd{P_{XY}}(u, v)=\iint g(u,v) f_{XY}(u, v)\dd \lambda(u) \dd \lambda(v)\]
$$\lim_{x\to\infty} f(x)$$	
$$\iiiint_V \mu(t,u,v,w) \,dt\,du\,dv\,dw$$
$$\sum_{n=1}^{\infty} 2^{-n} = 1$$	
\begin{definition}
	Si $X$ et $Y$ sont 2 v.a. ou definit la \textsc{Covariance} entre $X$ et $Y$ comme
	$\cov(X,Y)\overset{\text{def}}{=}\E\left[(X-\E(X))(Y-\E(Y))\right]=\E(XY)-\E(X)\E(Y)$.
\end{definition}
\fi
\pagebreak

% \tableofcontents

% insert your code here
%% !TEX encoding = UTF-8 Unicode
% !TEX TS-program = xelatex

\documentclass[french]{report}

%\usepackage[utf8]{inputenc}
%\usepackage[T1]{fontenc}
\usepackage{babel}


\newif\ifcomment
%\commenttrue # Show comments

\usepackage{physics}
\usepackage{amssymb}


\usepackage{amsthm}
% \usepackage{thmtools}
\usepackage{mathtools}
\usepackage{amsfonts}

\usepackage{color}

\usepackage{tikz}

\usepackage{geometry}
\geometry{a5paper, margin=0.1in, right=1cm}

\usepackage{dsfont}

\usepackage{graphicx}
\graphicspath{ {images/} }

\usepackage{faktor}

\usepackage{IEEEtrantools}
\usepackage{enumerate}   
\usepackage[PostScript=dvips]{"/Users/aware/Documents/Courses/diagrams"}


\newtheorem{theorem}{Théorème}[section]
\renewcommand{\thetheorem}{\arabic{theorem}}
\newtheorem{lemme}{Lemme}[section]
\renewcommand{\thelemme}{\arabic{lemme}}
\newtheorem{proposition}{Proposition}[section]
\renewcommand{\theproposition}{\arabic{proposition}}
\newtheorem{notations}{Notations}[section]
\newtheorem{problem}{Problème}[section]
\newtheorem{corollary}{Corollaire}[theorem]
\renewcommand{\thecorollary}{\arabic{corollary}}
\newtheorem{property}{Propriété}[section]
\newtheorem{objective}{Objectif}[section]

\theoremstyle{definition}
\newtheorem{definition}{Définition}[section]
\renewcommand{\thedefinition}{\arabic{definition}}
\newtheorem{exercise}{Exercice}[chapter]
\renewcommand{\theexercise}{\arabic{exercise}}
\newtheorem{example}{Exemple}[chapter]
\renewcommand{\theexample}{\arabic{example}}
\newtheorem*{solution}{Solution}
\newtheorem*{application}{Application}
\newtheorem*{notation}{Notation}
\newtheorem*{vocabulary}{Vocabulaire}
\newtheorem*{properties}{Propriétés}



\theoremstyle{remark}
\newtheorem*{remark}{Remarque}
\newtheorem*{rappel}{Rappel}


\usepackage{etoolbox}
\AtBeginEnvironment{exercise}{\small}
\AtBeginEnvironment{example}{\small}

\usepackage{cases}
\usepackage[red]{mypack}

\usepackage[framemethod=TikZ]{mdframed}

\definecolor{bg}{rgb}{0.4,0.25,0.95}
\definecolor{pagebg}{rgb}{0,0,0.5}
\surroundwithmdframed[
   topline=false,
   rightline=false,
   bottomline=false,
   leftmargin=\parindent,
   skipabove=8pt,
   skipbelow=8pt,
   linecolor=blue,
   innerbottommargin=10pt,
   % backgroundcolor=bg,font=\color{orange}\sffamily, fontcolor=white
]{definition}

\usepackage{empheq}
\usepackage[most]{tcolorbox}

\newtcbox{\mymath}[1][]{%
    nobeforeafter, math upper, tcbox raise base,
    enhanced, colframe=blue!30!black,
    colback=red!10, boxrule=1pt,
    #1}

\usepackage{unixode}


\DeclareMathOperator{\ord}{ord}
\DeclareMathOperator{\orb}{orb}
\DeclareMathOperator{\stab}{stab}
\DeclareMathOperator{\Stab}{stab}
\DeclareMathOperator{\ppcm}{ppcm}
\DeclareMathOperator{\conj}{Conj}
\DeclareMathOperator{\End}{End}
\DeclareMathOperator{\rot}{rot}
\DeclareMathOperator{\trs}{trace}
\DeclareMathOperator{\Ind}{Ind}
\DeclareMathOperator{\mat}{Mat}
\DeclareMathOperator{\id}{Id}
\DeclareMathOperator{\vect}{vect}
\DeclareMathOperator{\img}{img}
\DeclareMathOperator{\cov}{Cov}
\DeclareMathOperator{\dist}{dist}
\DeclareMathOperator{\irr}{Irr}
\DeclareMathOperator{\image}{Im}
\DeclareMathOperator{\pd}{\partial}
\DeclareMathOperator{\epi}{epi}
\DeclareMathOperator{\Argmin}{Argmin}
\DeclareMathOperator{\dom}{dom}
\DeclareMathOperator{\proj}{proj}
\DeclareMathOperator{\ctg}{ctg}
\DeclareMathOperator{\supp}{supp}
\DeclareMathOperator{\argmin}{argmin}
\DeclareMathOperator{\mult}{mult}
\DeclareMathOperator{\ch}{ch}
\DeclareMathOperator{\sh}{sh}
\DeclareMathOperator{\rang}{rang}
\DeclareMathOperator{\diam}{diam}
\DeclareMathOperator{\Epigraphe}{Epigraphe}




\usepackage{xcolor}
\everymath{\color{blue}}
%\everymath{\color[rgb]{0,1,1}}
%\pagecolor[rgb]{0,0,0.5}


\newcommand*{\pdtest}[3][]{\ensuremath{\frac{\partial^{#1} #2}{\partial #3}}}

\newcommand*{\deffunc}[6][]{\ensuremath{
\begin{array}{rcl}
#2 : #3 &\rightarrow& #4\\
#5 &\mapsto& #6
\end{array}
}}

\newcommand{\eqcolon}{\mathrel{\resizebox{\widthof{$\mathord{=}$}}{\height}{ $\!\!=\!\!\resizebox{1.2\width}{0.8\height}{\raisebox{0.23ex}{$\mathop{:}$}}\!\!$ }}}
\newcommand{\coloneq}{\mathrel{\resizebox{\widthof{$\mathord{=}$}}{\height}{ $\!\!\resizebox{1.2\width}{0.8\height}{\raisebox{0.23ex}{$\mathop{:}$}}\!\!=\!\!$ }}}
\newcommand{\eqcolonl}{\ensuremath{\mathrel{=\!\!\mathop{:}}}}
\newcommand{\coloneql}{\ensuremath{\mathrel{\mathop{:} \!\! =}}}
\newcommand{\vc}[1]{% inline column vector
  \left(\begin{smallmatrix}#1\end{smallmatrix}\right)%
}
\newcommand{\vr}[1]{% inline row vector
  \begin{smallmatrix}(\,#1\,)\end{smallmatrix}%
}
\makeatletter
\newcommand*{\defeq}{\ =\mathrel{\rlap{%
                     \raisebox{0.3ex}{$\m@th\cdot$}}%
                     \raisebox{-0.3ex}{$\m@th\cdot$}}%
                     }
\makeatother

\newcommand{\mathcircle}[1]{% inline row vector
 \overset{\circ}{#1}
}
\newcommand{\ulim}{% low limit
 \underline{\lim}
}
\newcommand{\ssi}{% iff
\iff
}
\newcommand{\ps}[2]{
\expval{#1 | #2}
}
\newcommand{\df}[1]{
\mqty{#1}
}
\newcommand{\n}[1]{
\norm{#1}
}
\newcommand{\sys}[1]{
\left\{\smqty{#1}\right.
}


\newcommand{\eqdef}{\ensuremath{\overset{\text{def}}=}}


\def\Circlearrowright{\ensuremath{%
  \rotatebox[origin=c]{230}{$\circlearrowright$}}}

\newcommand\ct[1]{\text{\rmfamily\upshape #1}}
\newcommand\question[1]{ {\color{red} ...!? \small #1}}
\newcommand\caz[1]{\left\{\begin{array} #1 \end{array}\right.}
\newcommand\const{\text{\rmfamily\upshape const}}
\newcommand\toP{ \overset{\pro}{\to}}
\newcommand\toPP{ \overset{\text{PP}}{\to}}
\newcommand{\oeq}{\mathrel{\text{\textcircled{$=$}}}}





\usepackage{xcolor}
% \usepackage[normalem]{ulem}
\usepackage{lipsum}
\makeatletter
% \newcommand\colorwave[1][blue]{\bgroup \markoverwith{\lower3.5\p@\hbox{\sixly \textcolor{#1}{\char58}}}\ULon}
%\font\sixly=lasy6 % does not re-load if already loaded, so no memory problem.

\newmdtheoremenv[
linewidth= 1pt,linecolor= blue,%
leftmargin=20,rightmargin=20,innertopmargin=0pt, innerrightmargin=40,%
tikzsetting = { draw=lightgray, line width = 0.3pt,dashed,%
dash pattern = on 15pt off 3pt},%
splittopskip=\topskip,skipbelow=\baselineskip,%
skipabove=\baselineskip,ntheorem,roundcorner=0pt,
% backgroundcolor=pagebg,font=\color{orange}\sffamily, fontcolor=white
]{examplebox}{Exemple}[section]



\newcommand\R{\mathbb{R}}
\newcommand\Z{\mathbb{Z}}
\newcommand\N{\mathbb{N}}
\newcommand\E{\mathbb{E}}
\newcommand\F{\mathcal{F}}
\newcommand\cH{\mathcal{H}}
\newcommand\V{\mathbb{V}}
\newcommand\dmo{ ^{-1} }
\newcommand\kapa{\kappa}
\newcommand\im{Im}
\newcommand\hs{\mathcal{H}}





\usepackage{soul}

\makeatletter
\newcommand*{\whiten}[1]{\llap{\textcolor{white}{{\the\SOUL@token}}\hspace{#1pt}}}
\DeclareRobustCommand*\myul{%
    \def\SOUL@everyspace{\underline{\space}\kern\z@}%
    \def\SOUL@everytoken{%
     \setbox0=\hbox{\the\SOUL@token}%
     \ifdim\dp0>\z@
        \raisebox{\dp0}{\underline{\phantom{\the\SOUL@token}}}%
        \whiten{1}\whiten{0}%
        \whiten{-1}\whiten{-2}%
        \llap{\the\SOUL@token}%
     \else
        \underline{\the\SOUL@token}%
     \fi}%
\SOUL@}
\makeatother

\newcommand*{\demp}{\fontfamily{lmtt}\selectfont}

\DeclareTextFontCommand{\textdemp}{\demp}

\begin{document}

\ifcomment
Multiline
comment
\fi
\ifcomment
\myul{Typesetting test}
% \color[rgb]{1,1,1}
$∑_i^n≠ 60º±∞π∆¬≈√j∫h≤≥µ$

$\CR \R\pro\ind\pro\gS\pro
\mqty[a&b\\c&d]$
$\pro\mathbb{P}$
$\dd{x}$

  \[
    \alpha(x)=\left\{
                \begin{array}{ll}
                  x\\
                  \frac{1}{1+e^{-kx}}\\
                  \frac{e^x-e^{-x}}{e^x+e^{-x}}
                \end{array}
              \right.
  \]

  $\expval{x}$
  
  $\chi_\rho(ghg\dmo)=\Tr(\rho_{ghg\dmo})=\Tr(\rho_g\circ\rho_h\circ\rho\dmo_g)=\Tr(\rho_h)\overset{\mbox{\scalebox{0.5}{$\Tr(AB)=\Tr(BA)$}}}{=}\chi_\rho(h)$
  	$\mathop{\oplus}_{\substack{x\in X}}$

$\mat(\rho_g)=(a_{ij}(g))_{\scriptsize \substack{1\leq i\leq d \\ 1\leq j\leq d}}$ et $\mat(\rho'_g)=(a'_{ij}(g))_{\scriptsize \substack{1\leq i'\leq d' \\ 1\leq j'\leq d'}}$



\[\int_a^b{\mathbb{R}^2}g(u, v)\dd{P_{XY}}(u, v)=\iint g(u,v) f_{XY}(u, v)\dd \lambda(u) \dd \lambda(v)\]
$$\lim_{x\to\infty} f(x)$$	
$$\iiiint_V \mu(t,u,v,w) \,dt\,du\,dv\,dw$$
$$\sum_{n=1}^{\infty} 2^{-n} = 1$$	
\begin{definition}
	Si $X$ et $Y$ sont 2 v.a. ou definit la \textsc{Covariance} entre $X$ et $Y$ comme
	$\cov(X,Y)\overset{\text{def}}{=}\E\left[(X-\E(X))(Y-\E(Y))\right]=\E(XY)-\E(X)\E(Y)$.
\end{definition}
\fi
\pagebreak

% \tableofcontents

% insert your code here
%\input{./algebra/main.tex}
%\input{./geometrie-differentielle/main.tex}
%\input{./probabilite/main.tex}
%\input{./analyse-fonctionnelle/main.tex}
% \input{./Analyse-convexe-et-dualite-en-optimisation/main.tex}
%\input{./tikz/main.tex}
%\input{./Theorie-du-distributions/main.tex}
%\input{./optimisation/mine.tex}
 \input{./modelisation/main.tex}

% yves.aubry@univ-tln.fr : algebra

\end{document}

%% !TEX encoding = UTF-8 Unicode
% !TEX TS-program = xelatex

\documentclass[french]{report}

%\usepackage[utf8]{inputenc}
%\usepackage[T1]{fontenc}
\usepackage{babel}


\newif\ifcomment
%\commenttrue # Show comments

\usepackage{physics}
\usepackage{amssymb}


\usepackage{amsthm}
% \usepackage{thmtools}
\usepackage{mathtools}
\usepackage{amsfonts}

\usepackage{color}

\usepackage{tikz}

\usepackage{geometry}
\geometry{a5paper, margin=0.1in, right=1cm}

\usepackage{dsfont}

\usepackage{graphicx}
\graphicspath{ {images/} }

\usepackage{faktor}

\usepackage{IEEEtrantools}
\usepackage{enumerate}   
\usepackage[PostScript=dvips]{"/Users/aware/Documents/Courses/diagrams"}


\newtheorem{theorem}{Théorème}[section]
\renewcommand{\thetheorem}{\arabic{theorem}}
\newtheorem{lemme}{Lemme}[section]
\renewcommand{\thelemme}{\arabic{lemme}}
\newtheorem{proposition}{Proposition}[section]
\renewcommand{\theproposition}{\arabic{proposition}}
\newtheorem{notations}{Notations}[section]
\newtheorem{problem}{Problème}[section]
\newtheorem{corollary}{Corollaire}[theorem]
\renewcommand{\thecorollary}{\arabic{corollary}}
\newtheorem{property}{Propriété}[section]
\newtheorem{objective}{Objectif}[section]

\theoremstyle{definition}
\newtheorem{definition}{Définition}[section]
\renewcommand{\thedefinition}{\arabic{definition}}
\newtheorem{exercise}{Exercice}[chapter]
\renewcommand{\theexercise}{\arabic{exercise}}
\newtheorem{example}{Exemple}[chapter]
\renewcommand{\theexample}{\arabic{example}}
\newtheorem*{solution}{Solution}
\newtheorem*{application}{Application}
\newtheorem*{notation}{Notation}
\newtheorem*{vocabulary}{Vocabulaire}
\newtheorem*{properties}{Propriétés}



\theoremstyle{remark}
\newtheorem*{remark}{Remarque}
\newtheorem*{rappel}{Rappel}


\usepackage{etoolbox}
\AtBeginEnvironment{exercise}{\small}
\AtBeginEnvironment{example}{\small}

\usepackage{cases}
\usepackage[red]{mypack}

\usepackage[framemethod=TikZ]{mdframed}

\definecolor{bg}{rgb}{0.4,0.25,0.95}
\definecolor{pagebg}{rgb}{0,0,0.5}
\surroundwithmdframed[
   topline=false,
   rightline=false,
   bottomline=false,
   leftmargin=\parindent,
   skipabove=8pt,
   skipbelow=8pt,
   linecolor=blue,
   innerbottommargin=10pt,
   % backgroundcolor=bg,font=\color{orange}\sffamily, fontcolor=white
]{definition}

\usepackage{empheq}
\usepackage[most]{tcolorbox}

\newtcbox{\mymath}[1][]{%
    nobeforeafter, math upper, tcbox raise base,
    enhanced, colframe=blue!30!black,
    colback=red!10, boxrule=1pt,
    #1}

\usepackage{unixode}


\DeclareMathOperator{\ord}{ord}
\DeclareMathOperator{\orb}{orb}
\DeclareMathOperator{\stab}{stab}
\DeclareMathOperator{\Stab}{stab}
\DeclareMathOperator{\ppcm}{ppcm}
\DeclareMathOperator{\conj}{Conj}
\DeclareMathOperator{\End}{End}
\DeclareMathOperator{\rot}{rot}
\DeclareMathOperator{\trs}{trace}
\DeclareMathOperator{\Ind}{Ind}
\DeclareMathOperator{\mat}{Mat}
\DeclareMathOperator{\id}{Id}
\DeclareMathOperator{\vect}{vect}
\DeclareMathOperator{\img}{img}
\DeclareMathOperator{\cov}{Cov}
\DeclareMathOperator{\dist}{dist}
\DeclareMathOperator{\irr}{Irr}
\DeclareMathOperator{\image}{Im}
\DeclareMathOperator{\pd}{\partial}
\DeclareMathOperator{\epi}{epi}
\DeclareMathOperator{\Argmin}{Argmin}
\DeclareMathOperator{\dom}{dom}
\DeclareMathOperator{\proj}{proj}
\DeclareMathOperator{\ctg}{ctg}
\DeclareMathOperator{\supp}{supp}
\DeclareMathOperator{\argmin}{argmin}
\DeclareMathOperator{\mult}{mult}
\DeclareMathOperator{\ch}{ch}
\DeclareMathOperator{\sh}{sh}
\DeclareMathOperator{\rang}{rang}
\DeclareMathOperator{\diam}{diam}
\DeclareMathOperator{\Epigraphe}{Epigraphe}




\usepackage{xcolor}
\everymath{\color{blue}}
%\everymath{\color[rgb]{0,1,1}}
%\pagecolor[rgb]{0,0,0.5}


\newcommand*{\pdtest}[3][]{\ensuremath{\frac{\partial^{#1} #2}{\partial #3}}}

\newcommand*{\deffunc}[6][]{\ensuremath{
\begin{array}{rcl}
#2 : #3 &\rightarrow& #4\\
#5 &\mapsto& #6
\end{array}
}}

\newcommand{\eqcolon}{\mathrel{\resizebox{\widthof{$\mathord{=}$}}{\height}{ $\!\!=\!\!\resizebox{1.2\width}{0.8\height}{\raisebox{0.23ex}{$\mathop{:}$}}\!\!$ }}}
\newcommand{\coloneq}{\mathrel{\resizebox{\widthof{$\mathord{=}$}}{\height}{ $\!\!\resizebox{1.2\width}{0.8\height}{\raisebox{0.23ex}{$\mathop{:}$}}\!\!=\!\!$ }}}
\newcommand{\eqcolonl}{\ensuremath{\mathrel{=\!\!\mathop{:}}}}
\newcommand{\coloneql}{\ensuremath{\mathrel{\mathop{:} \!\! =}}}
\newcommand{\vc}[1]{% inline column vector
  \left(\begin{smallmatrix}#1\end{smallmatrix}\right)%
}
\newcommand{\vr}[1]{% inline row vector
  \begin{smallmatrix}(\,#1\,)\end{smallmatrix}%
}
\makeatletter
\newcommand*{\defeq}{\ =\mathrel{\rlap{%
                     \raisebox{0.3ex}{$\m@th\cdot$}}%
                     \raisebox{-0.3ex}{$\m@th\cdot$}}%
                     }
\makeatother

\newcommand{\mathcircle}[1]{% inline row vector
 \overset{\circ}{#1}
}
\newcommand{\ulim}{% low limit
 \underline{\lim}
}
\newcommand{\ssi}{% iff
\iff
}
\newcommand{\ps}[2]{
\expval{#1 | #2}
}
\newcommand{\df}[1]{
\mqty{#1}
}
\newcommand{\n}[1]{
\norm{#1}
}
\newcommand{\sys}[1]{
\left\{\smqty{#1}\right.
}


\newcommand{\eqdef}{\ensuremath{\overset{\text{def}}=}}


\def\Circlearrowright{\ensuremath{%
  \rotatebox[origin=c]{230}{$\circlearrowright$}}}

\newcommand\ct[1]{\text{\rmfamily\upshape #1}}
\newcommand\question[1]{ {\color{red} ...!? \small #1}}
\newcommand\caz[1]{\left\{\begin{array} #1 \end{array}\right.}
\newcommand\const{\text{\rmfamily\upshape const}}
\newcommand\toP{ \overset{\pro}{\to}}
\newcommand\toPP{ \overset{\text{PP}}{\to}}
\newcommand{\oeq}{\mathrel{\text{\textcircled{$=$}}}}





\usepackage{xcolor}
% \usepackage[normalem]{ulem}
\usepackage{lipsum}
\makeatletter
% \newcommand\colorwave[1][blue]{\bgroup \markoverwith{\lower3.5\p@\hbox{\sixly \textcolor{#1}{\char58}}}\ULon}
%\font\sixly=lasy6 % does not re-load if already loaded, so no memory problem.

\newmdtheoremenv[
linewidth= 1pt,linecolor= blue,%
leftmargin=20,rightmargin=20,innertopmargin=0pt, innerrightmargin=40,%
tikzsetting = { draw=lightgray, line width = 0.3pt,dashed,%
dash pattern = on 15pt off 3pt},%
splittopskip=\topskip,skipbelow=\baselineskip,%
skipabove=\baselineskip,ntheorem,roundcorner=0pt,
% backgroundcolor=pagebg,font=\color{orange}\sffamily, fontcolor=white
]{examplebox}{Exemple}[section]



\newcommand\R{\mathbb{R}}
\newcommand\Z{\mathbb{Z}}
\newcommand\N{\mathbb{N}}
\newcommand\E{\mathbb{E}}
\newcommand\F{\mathcal{F}}
\newcommand\cH{\mathcal{H}}
\newcommand\V{\mathbb{V}}
\newcommand\dmo{ ^{-1} }
\newcommand\kapa{\kappa}
\newcommand\im{Im}
\newcommand\hs{\mathcal{H}}





\usepackage{soul}

\makeatletter
\newcommand*{\whiten}[1]{\llap{\textcolor{white}{{\the\SOUL@token}}\hspace{#1pt}}}
\DeclareRobustCommand*\myul{%
    \def\SOUL@everyspace{\underline{\space}\kern\z@}%
    \def\SOUL@everytoken{%
     \setbox0=\hbox{\the\SOUL@token}%
     \ifdim\dp0>\z@
        \raisebox{\dp0}{\underline{\phantom{\the\SOUL@token}}}%
        \whiten{1}\whiten{0}%
        \whiten{-1}\whiten{-2}%
        \llap{\the\SOUL@token}%
     \else
        \underline{\the\SOUL@token}%
     \fi}%
\SOUL@}
\makeatother

\newcommand*{\demp}{\fontfamily{lmtt}\selectfont}

\DeclareTextFontCommand{\textdemp}{\demp}

\begin{document}

\ifcomment
Multiline
comment
\fi
\ifcomment
\myul{Typesetting test}
% \color[rgb]{1,1,1}
$∑_i^n≠ 60º±∞π∆¬≈√j∫h≤≥µ$

$\CR \R\pro\ind\pro\gS\pro
\mqty[a&b\\c&d]$
$\pro\mathbb{P}$
$\dd{x}$

  \[
    \alpha(x)=\left\{
                \begin{array}{ll}
                  x\\
                  \frac{1}{1+e^{-kx}}\\
                  \frac{e^x-e^{-x}}{e^x+e^{-x}}
                \end{array}
              \right.
  \]

  $\expval{x}$
  
  $\chi_\rho(ghg\dmo)=\Tr(\rho_{ghg\dmo})=\Tr(\rho_g\circ\rho_h\circ\rho\dmo_g)=\Tr(\rho_h)\overset{\mbox{\scalebox{0.5}{$\Tr(AB)=\Tr(BA)$}}}{=}\chi_\rho(h)$
  	$\mathop{\oplus}_{\substack{x\in X}}$

$\mat(\rho_g)=(a_{ij}(g))_{\scriptsize \substack{1\leq i\leq d \\ 1\leq j\leq d}}$ et $\mat(\rho'_g)=(a'_{ij}(g))_{\scriptsize \substack{1\leq i'\leq d' \\ 1\leq j'\leq d'}}$



\[\int_a^b{\mathbb{R}^2}g(u, v)\dd{P_{XY}}(u, v)=\iint g(u,v) f_{XY}(u, v)\dd \lambda(u) \dd \lambda(v)\]
$$\lim_{x\to\infty} f(x)$$	
$$\iiiint_V \mu(t,u,v,w) \,dt\,du\,dv\,dw$$
$$\sum_{n=1}^{\infty} 2^{-n} = 1$$	
\begin{definition}
	Si $X$ et $Y$ sont 2 v.a. ou definit la \textsc{Covariance} entre $X$ et $Y$ comme
	$\cov(X,Y)\overset{\text{def}}{=}\E\left[(X-\E(X))(Y-\E(Y))\right]=\E(XY)-\E(X)\E(Y)$.
\end{definition}
\fi
\pagebreak

% \tableofcontents

% insert your code here
%\input{./algebra/main.tex}
%\input{./geometrie-differentielle/main.tex}
%\input{./probabilite/main.tex}
%\input{./analyse-fonctionnelle/main.tex}
% \input{./Analyse-convexe-et-dualite-en-optimisation/main.tex}
%\input{./tikz/main.tex}
%\input{./Theorie-du-distributions/main.tex}
%\input{./optimisation/mine.tex}
 \input{./modelisation/main.tex}

% yves.aubry@univ-tln.fr : algebra

\end{document}

%% !TEX encoding = UTF-8 Unicode
% !TEX TS-program = xelatex

\documentclass[french]{report}

%\usepackage[utf8]{inputenc}
%\usepackage[T1]{fontenc}
\usepackage{babel}


\newif\ifcomment
%\commenttrue # Show comments

\usepackage{physics}
\usepackage{amssymb}


\usepackage{amsthm}
% \usepackage{thmtools}
\usepackage{mathtools}
\usepackage{amsfonts}

\usepackage{color}

\usepackage{tikz}

\usepackage{geometry}
\geometry{a5paper, margin=0.1in, right=1cm}

\usepackage{dsfont}

\usepackage{graphicx}
\graphicspath{ {images/} }

\usepackage{faktor}

\usepackage{IEEEtrantools}
\usepackage{enumerate}   
\usepackage[PostScript=dvips]{"/Users/aware/Documents/Courses/diagrams"}


\newtheorem{theorem}{Théorème}[section]
\renewcommand{\thetheorem}{\arabic{theorem}}
\newtheorem{lemme}{Lemme}[section]
\renewcommand{\thelemme}{\arabic{lemme}}
\newtheorem{proposition}{Proposition}[section]
\renewcommand{\theproposition}{\arabic{proposition}}
\newtheorem{notations}{Notations}[section]
\newtheorem{problem}{Problème}[section]
\newtheorem{corollary}{Corollaire}[theorem]
\renewcommand{\thecorollary}{\arabic{corollary}}
\newtheorem{property}{Propriété}[section]
\newtheorem{objective}{Objectif}[section]

\theoremstyle{definition}
\newtheorem{definition}{Définition}[section]
\renewcommand{\thedefinition}{\arabic{definition}}
\newtheorem{exercise}{Exercice}[chapter]
\renewcommand{\theexercise}{\arabic{exercise}}
\newtheorem{example}{Exemple}[chapter]
\renewcommand{\theexample}{\arabic{example}}
\newtheorem*{solution}{Solution}
\newtheorem*{application}{Application}
\newtheorem*{notation}{Notation}
\newtheorem*{vocabulary}{Vocabulaire}
\newtheorem*{properties}{Propriétés}



\theoremstyle{remark}
\newtheorem*{remark}{Remarque}
\newtheorem*{rappel}{Rappel}


\usepackage{etoolbox}
\AtBeginEnvironment{exercise}{\small}
\AtBeginEnvironment{example}{\small}

\usepackage{cases}
\usepackage[red]{mypack}

\usepackage[framemethod=TikZ]{mdframed}

\definecolor{bg}{rgb}{0.4,0.25,0.95}
\definecolor{pagebg}{rgb}{0,0,0.5}
\surroundwithmdframed[
   topline=false,
   rightline=false,
   bottomline=false,
   leftmargin=\parindent,
   skipabove=8pt,
   skipbelow=8pt,
   linecolor=blue,
   innerbottommargin=10pt,
   % backgroundcolor=bg,font=\color{orange}\sffamily, fontcolor=white
]{definition}

\usepackage{empheq}
\usepackage[most]{tcolorbox}

\newtcbox{\mymath}[1][]{%
    nobeforeafter, math upper, tcbox raise base,
    enhanced, colframe=blue!30!black,
    colback=red!10, boxrule=1pt,
    #1}

\usepackage{unixode}


\DeclareMathOperator{\ord}{ord}
\DeclareMathOperator{\orb}{orb}
\DeclareMathOperator{\stab}{stab}
\DeclareMathOperator{\Stab}{stab}
\DeclareMathOperator{\ppcm}{ppcm}
\DeclareMathOperator{\conj}{Conj}
\DeclareMathOperator{\End}{End}
\DeclareMathOperator{\rot}{rot}
\DeclareMathOperator{\trs}{trace}
\DeclareMathOperator{\Ind}{Ind}
\DeclareMathOperator{\mat}{Mat}
\DeclareMathOperator{\id}{Id}
\DeclareMathOperator{\vect}{vect}
\DeclareMathOperator{\img}{img}
\DeclareMathOperator{\cov}{Cov}
\DeclareMathOperator{\dist}{dist}
\DeclareMathOperator{\irr}{Irr}
\DeclareMathOperator{\image}{Im}
\DeclareMathOperator{\pd}{\partial}
\DeclareMathOperator{\epi}{epi}
\DeclareMathOperator{\Argmin}{Argmin}
\DeclareMathOperator{\dom}{dom}
\DeclareMathOperator{\proj}{proj}
\DeclareMathOperator{\ctg}{ctg}
\DeclareMathOperator{\supp}{supp}
\DeclareMathOperator{\argmin}{argmin}
\DeclareMathOperator{\mult}{mult}
\DeclareMathOperator{\ch}{ch}
\DeclareMathOperator{\sh}{sh}
\DeclareMathOperator{\rang}{rang}
\DeclareMathOperator{\diam}{diam}
\DeclareMathOperator{\Epigraphe}{Epigraphe}




\usepackage{xcolor}
\everymath{\color{blue}}
%\everymath{\color[rgb]{0,1,1}}
%\pagecolor[rgb]{0,0,0.5}


\newcommand*{\pdtest}[3][]{\ensuremath{\frac{\partial^{#1} #2}{\partial #3}}}

\newcommand*{\deffunc}[6][]{\ensuremath{
\begin{array}{rcl}
#2 : #3 &\rightarrow& #4\\
#5 &\mapsto& #6
\end{array}
}}

\newcommand{\eqcolon}{\mathrel{\resizebox{\widthof{$\mathord{=}$}}{\height}{ $\!\!=\!\!\resizebox{1.2\width}{0.8\height}{\raisebox{0.23ex}{$\mathop{:}$}}\!\!$ }}}
\newcommand{\coloneq}{\mathrel{\resizebox{\widthof{$\mathord{=}$}}{\height}{ $\!\!\resizebox{1.2\width}{0.8\height}{\raisebox{0.23ex}{$\mathop{:}$}}\!\!=\!\!$ }}}
\newcommand{\eqcolonl}{\ensuremath{\mathrel{=\!\!\mathop{:}}}}
\newcommand{\coloneql}{\ensuremath{\mathrel{\mathop{:} \!\! =}}}
\newcommand{\vc}[1]{% inline column vector
  \left(\begin{smallmatrix}#1\end{smallmatrix}\right)%
}
\newcommand{\vr}[1]{% inline row vector
  \begin{smallmatrix}(\,#1\,)\end{smallmatrix}%
}
\makeatletter
\newcommand*{\defeq}{\ =\mathrel{\rlap{%
                     \raisebox{0.3ex}{$\m@th\cdot$}}%
                     \raisebox{-0.3ex}{$\m@th\cdot$}}%
                     }
\makeatother

\newcommand{\mathcircle}[1]{% inline row vector
 \overset{\circ}{#1}
}
\newcommand{\ulim}{% low limit
 \underline{\lim}
}
\newcommand{\ssi}{% iff
\iff
}
\newcommand{\ps}[2]{
\expval{#1 | #2}
}
\newcommand{\df}[1]{
\mqty{#1}
}
\newcommand{\n}[1]{
\norm{#1}
}
\newcommand{\sys}[1]{
\left\{\smqty{#1}\right.
}


\newcommand{\eqdef}{\ensuremath{\overset{\text{def}}=}}


\def\Circlearrowright{\ensuremath{%
  \rotatebox[origin=c]{230}{$\circlearrowright$}}}

\newcommand\ct[1]{\text{\rmfamily\upshape #1}}
\newcommand\question[1]{ {\color{red} ...!? \small #1}}
\newcommand\caz[1]{\left\{\begin{array} #1 \end{array}\right.}
\newcommand\const{\text{\rmfamily\upshape const}}
\newcommand\toP{ \overset{\pro}{\to}}
\newcommand\toPP{ \overset{\text{PP}}{\to}}
\newcommand{\oeq}{\mathrel{\text{\textcircled{$=$}}}}





\usepackage{xcolor}
% \usepackage[normalem]{ulem}
\usepackage{lipsum}
\makeatletter
% \newcommand\colorwave[1][blue]{\bgroup \markoverwith{\lower3.5\p@\hbox{\sixly \textcolor{#1}{\char58}}}\ULon}
%\font\sixly=lasy6 % does not re-load if already loaded, so no memory problem.

\newmdtheoremenv[
linewidth= 1pt,linecolor= blue,%
leftmargin=20,rightmargin=20,innertopmargin=0pt, innerrightmargin=40,%
tikzsetting = { draw=lightgray, line width = 0.3pt,dashed,%
dash pattern = on 15pt off 3pt},%
splittopskip=\topskip,skipbelow=\baselineskip,%
skipabove=\baselineskip,ntheorem,roundcorner=0pt,
% backgroundcolor=pagebg,font=\color{orange}\sffamily, fontcolor=white
]{examplebox}{Exemple}[section]



\newcommand\R{\mathbb{R}}
\newcommand\Z{\mathbb{Z}}
\newcommand\N{\mathbb{N}}
\newcommand\E{\mathbb{E}}
\newcommand\F{\mathcal{F}}
\newcommand\cH{\mathcal{H}}
\newcommand\V{\mathbb{V}}
\newcommand\dmo{ ^{-1} }
\newcommand\kapa{\kappa}
\newcommand\im{Im}
\newcommand\hs{\mathcal{H}}





\usepackage{soul}

\makeatletter
\newcommand*{\whiten}[1]{\llap{\textcolor{white}{{\the\SOUL@token}}\hspace{#1pt}}}
\DeclareRobustCommand*\myul{%
    \def\SOUL@everyspace{\underline{\space}\kern\z@}%
    \def\SOUL@everytoken{%
     \setbox0=\hbox{\the\SOUL@token}%
     \ifdim\dp0>\z@
        \raisebox{\dp0}{\underline{\phantom{\the\SOUL@token}}}%
        \whiten{1}\whiten{0}%
        \whiten{-1}\whiten{-2}%
        \llap{\the\SOUL@token}%
     \else
        \underline{\the\SOUL@token}%
     \fi}%
\SOUL@}
\makeatother

\newcommand*{\demp}{\fontfamily{lmtt}\selectfont}

\DeclareTextFontCommand{\textdemp}{\demp}

\begin{document}

\ifcomment
Multiline
comment
\fi
\ifcomment
\myul{Typesetting test}
% \color[rgb]{1,1,1}
$∑_i^n≠ 60º±∞π∆¬≈√j∫h≤≥µ$

$\CR \R\pro\ind\pro\gS\pro
\mqty[a&b\\c&d]$
$\pro\mathbb{P}$
$\dd{x}$

  \[
    \alpha(x)=\left\{
                \begin{array}{ll}
                  x\\
                  \frac{1}{1+e^{-kx}}\\
                  \frac{e^x-e^{-x}}{e^x+e^{-x}}
                \end{array}
              \right.
  \]

  $\expval{x}$
  
  $\chi_\rho(ghg\dmo)=\Tr(\rho_{ghg\dmo})=\Tr(\rho_g\circ\rho_h\circ\rho\dmo_g)=\Tr(\rho_h)\overset{\mbox{\scalebox{0.5}{$\Tr(AB)=\Tr(BA)$}}}{=}\chi_\rho(h)$
  	$\mathop{\oplus}_{\substack{x\in X}}$

$\mat(\rho_g)=(a_{ij}(g))_{\scriptsize \substack{1\leq i\leq d \\ 1\leq j\leq d}}$ et $\mat(\rho'_g)=(a'_{ij}(g))_{\scriptsize \substack{1\leq i'\leq d' \\ 1\leq j'\leq d'}}$



\[\int_a^b{\mathbb{R}^2}g(u, v)\dd{P_{XY}}(u, v)=\iint g(u,v) f_{XY}(u, v)\dd \lambda(u) \dd \lambda(v)\]
$$\lim_{x\to\infty} f(x)$$	
$$\iiiint_V \mu(t,u,v,w) \,dt\,du\,dv\,dw$$
$$\sum_{n=1}^{\infty} 2^{-n} = 1$$	
\begin{definition}
	Si $X$ et $Y$ sont 2 v.a. ou definit la \textsc{Covariance} entre $X$ et $Y$ comme
	$\cov(X,Y)\overset{\text{def}}{=}\E\left[(X-\E(X))(Y-\E(Y))\right]=\E(XY)-\E(X)\E(Y)$.
\end{definition}
\fi
\pagebreak

% \tableofcontents

% insert your code here
%\input{./algebra/main.tex}
%\input{./geometrie-differentielle/main.tex}
%\input{./probabilite/main.tex}
%\input{./analyse-fonctionnelle/main.tex}
% \input{./Analyse-convexe-et-dualite-en-optimisation/main.tex}
%\input{./tikz/main.tex}
%\input{./Theorie-du-distributions/main.tex}
%\input{./optimisation/mine.tex}
 \input{./modelisation/main.tex}

% yves.aubry@univ-tln.fr : algebra

\end{document}

%% !TEX encoding = UTF-8 Unicode
% !TEX TS-program = xelatex

\documentclass[french]{report}

%\usepackage[utf8]{inputenc}
%\usepackage[T1]{fontenc}
\usepackage{babel}


\newif\ifcomment
%\commenttrue # Show comments

\usepackage{physics}
\usepackage{amssymb}


\usepackage{amsthm}
% \usepackage{thmtools}
\usepackage{mathtools}
\usepackage{amsfonts}

\usepackage{color}

\usepackage{tikz}

\usepackage{geometry}
\geometry{a5paper, margin=0.1in, right=1cm}

\usepackage{dsfont}

\usepackage{graphicx}
\graphicspath{ {images/} }

\usepackage{faktor}

\usepackage{IEEEtrantools}
\usepackage{enumerate}   
\usepackage[PostScript=dvips]{"/Users/aware/Documents/Courses/diagrams"}


\newtheorem{theorem}{Théorème}[section]
\renewcommand{\thetheorem}{\arabic{theorem}}
\newtheorem{lemme}{Lemme}[section]
\renewcommand{\thelemme}{\arabic{lemme}}
\newtheorem{proposition}{Proposition}[section]
\renewcommand{\theproposition}{\arabic{proposition}}
\newtheorem{notations}{Notations}[section]
\newtheorem{problem}{Problème}[section]
\newtheorem{corollary}{Corollaire}[theorem]
\renewcommand{\thecorollary}{\arabic{corollary}}
\newtheorem{property}{Propriété}[section]
\newtheorem{objective}{Objectif}[section]

\theoremstyle{definition}
\newtheorem{definition}{Définition}[section]
\renewcommand{\thedefinition}{\arabic{definition}}
\newtheorem{exercise}{Exercice}[chapter]
\renewcommand{\theexercise}{\arabic{exercise}}
\newtheorem{example}{Exemple}[chapter]
\renewcommand{\theexample}{\arabic{example}}
\newtheorem*{solution}{Solution}
\newtheorem*{application}{Application}
\newtheorem*{notation}{Notation}
\newtheorem*{vocabulary}{Vocabulaire}
\newtheorem*{properties}{Propriétés}



\theoremstyle{remark}
\newtheorem*{remark}{Remarque}
\newtheorem*{rappel}{Rappel}


\usepackage{etoolbox}
\AtBeginEnvironment{exercise}{\small}
\AtBeginEnvironment{example}{\small}

\usepackage{cases}
\usepackage[red]{mypack}

\usepackage[framemethod=TikZ]{mdframed}

\definecolor{bg}{rgb}{0.4,0.25,0.95}
\definecolor{pagebg}{rgb}{0,0,0.5}
\surroundwithmdframed[
   topline=false,
   rightline=false,
   bottomline=false,
   leftmargin=\parindent,
   skipabove=8pt,
   skipbelow=8pt,
   linecolor=blue,
   innerbottommargin=10pt,
   % backgroundcolor=bg,font=\color{orange}\sffamily, fontcolor=white
]{definition}

\usepackage{empheq}
\usepackage[most]{tcolorbox}

\newtcbox{\mymath}[1][]{%
    nobeforeafter, math upper, tcbox raise base,
    enhanced, colframe=blue!30!black,
    colback=red!10, boxrule=1pt,
    #1}

\usepackage{unixode}


\DeclareMathOperator{\ord}{ord}
\DeclareMathOperator{\orb}{orb}
\DeclareMathOperator{\stab}{stab}
\DeclareMathOperator{\Stab}{stab}
\DeclareMathOperator{\ppcm}{ppcm}
\DeclareMathOperator{\conj}{Conj}
\DeclareMathOperator{\End}{End}
\DeclareMathOperator{\rot}{rot}
\DeclareMathOperator{\trs}{trace}
\DeclareMathOperator{\Ind}{Ind}
\DeclareMathOperator{\mat}{Mat}
\DeclareMathOperator{\id}{Id}
\DeclareMathOperator{\vect}{vect}
\DeclareMathOperator{\img}{img}
\DeclareMathOperator{\cov}{Cov}
\DeclareMathOperator{\dist}{dist}
\DeclareMathOperator{\irr}{Irr}
\DeclareMathOperator{\image}{Im}
\DeclareMathOperator{\pd}{\partial}
\DeclareMathOperator{\epi}{epi}
\DeclareMathOperator{\Argmin}{Argmin}
\DeclareMathOperator{\dom}{dom}
\DeclareMathOperator{\proj}{proj}
\DeclareMathOperator{\ctg}{ctg}
\DeclareMathOperator{\supp}{supp}
\DeclareMathOperator{\argmin}{argmin}
\DeclareMathOperator{\mult}{mult}
\DeclareMathOperator{\ch}{ch}
\DeclareMathOperator{\sh}{sh}
\DeclareMathOperator{\rang}{rang}
\DeclareMathOperator{\diam}{diam}
\DeclareMathOperator{\Epigraphe}{Epigraphe}




\usepackage{xcolor}
\everymath{\color{blue}}
%\everymath{\color[rgb]{0,1,1}}
%\pagecolor[rgb]{0,0,0.5}


\newcommand*{\pdtest}[3][]{\ensuremath{\frac{\partial^{#1} #2}{\partial #3}}}

\newcommand*{\deffunc}[6][]{\ensuremath{
\begin{array}{rcl}
#2 : #3 &\rightarrow& #4\\
#5 &\mapsto& #6
\end{array}
}}

\newcommand{\eqcolon}{\mathrel{\resizebox{\widthof{$\mathord{=}$}}{\height}{ $\!\!=\!\!\resizebox{1.2\width}{0.8\height}{\raisebox{0.23ex}{$\mathop{:}$}}\!\!$ }}}
\newcommand{\coloneq}{\mathrel{\resizebox{\widthof{$\mathord{=}$}}{\height}{ $\!\!\resizebox{1.2\width}{0.8\height}{\raisebox{0.23ex}{$\mathop{:}$}}\!\!=\!\!$ }}}
\newcommand{\eqcolonl}{\ensuremath{\mathrel{=\!\!\mathop{:}}}}
\newcommand{\coloneql}{\ensuremath{\mathrel{\mathop{:} \!\! =}}}
\newcommand{\vc}[1]{% inline column vector
  \left(\begin{smallmatrix}#1\end{smallmatrix}\right)%
}
\newcommand{\vr}[1]{% inline row vector
  \begin{smallmatrix}(\,#1\,)\end{smallmatrix}%
}
\makeatletter
\newcommand*{\defeq}{\ =\mathrel{\rlap{%
                     \raisebox{0.3ex}{$\m@th\cdot$}}%
                     \raisebox{-0.3ex}{$\m@th\cdot$}}%
                     }
\makeatother

\newcommand{\mathcircle}[1]{% inline row vector
 \overset{\circ}{#1}
}
\newcommand{\ulim}{% low limit
 \underline{\lim}
}
\newcommand{\ssi}{% iff
\iff
}
\newcommand{\ps}[2]{
\expval{#1 | #2}
}
\newcommand{\df}[1]{
\mqty{#1}
}
\newcommand{\n}[1]{
\norm{#1}
}
\newcommand{\sys}[1]{
\left\{\smqty{#1}\right.
}


\newcommand{\eqdef}{\ensuremath{\overset{\text{def}}=}}


\def\Circlearrowright{\ensuremath{%
  \rotatebox[origin=c]{230}{$\circlearrowright$}}}

\newcommand\ct[1]{\text{\rmfamily\upshape #1}}
\newcommand\question[1]{ {\color{red} ...!? \small #1}}
\newcommand\caz[1]{\left\{\begin{array} #1 \end{array}\right.}
\newcommand\const{\text{\rmfamily\upshape const}}
\newcommand\toP{ \overset{\pro}{\to}}
\newcommand\toPP{ \overset{\text{PP}}{\to}}
\newcommand{\oeq}{\mathrel{\text{\textcircled{$=$}}}}





\usepackage{xcolor}
% \usepackage[normalem]{ulem}
\usepackage{lipsum}
\makeatletter
% \newcommand\colorwave[1][blue]{\bgroup \markoverwith{\lower3.5\p@\hbox{\sixly \textcolor{#1}{\char58}}}\ULon}
%\font\sixly=lasy6 % does not re-load if already loaded, so no memory problem.

\newmdtheoremenv[
linewidth= 1pt,linecolor= blue,%
leftmargin=20,rightmargin=20,innertopmargin=0pt, innerrightmargin=40,%
tikzsetting = { draw=lightgray, line width = 0.3pt,dashed,%
dash pattern = on 15pt off 3pt},%
splittopskip=\topskip,skipbelow=\baselineskip,%
skipabove=\baselineskip,ntheorem,roundcorner=0pt,
% backgroundcolor=pagebg,font=\color{orange}\sffamily, fontcolor=white
]{examplebox}{Exemple}[section]



\newcommand\R{\mathbb{R}}
\newcommand\Z{\mathbb{Z}}
\newcommand\N{\mathbb{N}}
\newcommand\E{\mathbb{E}}
\newcommand\F{\mathcal{F}}
\newcommand\cH{\mathcal{H}}
\newcommand\V{\mathbb{V}}
\newcommand\dmo{ ^{-1} }
\newcommand\kapa{\kappa}
\newcommand\im{Im}
\newcommand\hs{\mathcal{H}}





\usepackage{soul}

\makeatletter
\newcommand*{\whiten}[1]{\llap{\textcolor{white}{{\the\SOUL@token}}\hspace{#1pt}}}
\DeclareRobustCommand*\myul{%
    \def\SOUL@everyspace{\underline{\space}\kern\z@}%
    \def\SOUL@everytoken{%
     \setbox0=\hbox{\the\SOUL@token}%
     \ifdim\dp0>\z@
        \raisebox{\dp0}{\underline{\phantom{\the\SOUL@token}}}%
        \whiten{1}\whiten{0}%
        \whiten{-1}\whiten{-2}%
        \llap{\the\SOUL@token}%
     \else
        \underline{\the\SOUL@token}%
     \fi}%
\SOUL@}
\makeatother

\newcommand*{\demp}{\fontfamily{lmtt}\selectfont}

\DeclareTextFontCommand{\textdemp}{\demp}

\begin{document}

\ifcomment
Multiline
comment
\fi
\ifcomment
\myul{Typesetting test}
% \color[rgb]{1,1,1}
$∑_i^n≠ 60º±∞π∆¬≈√j∫h≤≥µ$

$\CR \R\pro\ind\pro\gS\pro
\mqty[a&b\\c&d]$
$\pro\mathbb{P}$
$\dd{x}$

  \[
    \alpha(x)=\left\{
                \begin{array}{ll}
                  x\\
                  \frac{1}{1+e^{-kx}}\\
                  \frac{e^x-e^{-x}}{e^x+e^{-x}}
                \end{array}
              \right.
  \]

  $\expval{x}$
  
  $\chi_\rho(ghg\dmo)=\Tr(\rho_{ghg\dmo})=\Tr(\rho_g\circ\rho_h\circ\rho\dmo_g)=\Tr(\rho_h)\overset{\mbox{\scalebox{0.5}{$\Tr(AB)=\Tr(BA)$}}}{=}\chi_\rho(h)$
  	$\mathop{\oplus}_{\substack{x\in X}}$

$\mat(\rho_g)=(a_{ij}(g))_{\scriptsize \substack{1\leq i\leq d \\ 1\leq j\leq d}}$ et $\mat(\rho'_g)=(a'_{ij}(g))_{\scriptsize \substack{1\leq i'\leq d' \\ 1\leq j'\leq d'}}$



\[\int_a^b{\mathbb{R}^2}g(u, v)\dd{P_{XY}}(u, v)=\iint g(u,v) f_{XY}(u, v)\dd \lambda(u) \dd \lambda(v)\]
$$\lim_{x\to\infty} f(x)$$	
$$\iiiint_V \mu(t,u,v,w) \,dt\,du\,dv\,dw$$
$$\sum_{n=1}^{\infty} 2^{-n} = 1$$	
\begin{definition}
	Si $X$ et $Y$ sont 2 v.a. ou definit la \textsc{Covariance} entre $X$ et $Y$ comme
	$\cov(X,Y)\overset{\text{def}}{=}\E\left[(X-\E(X))(Y-\E(Y))\right]=\E(XY)-\E(X)\E(Y)$.
\end{definition}
\fi
\pagebreak

% \tableofcontents

% insert your code here
%\input{./algebra/main.tex}
%\input{./geometrie-differentielle/main.tex}
%\input{./probabilite/main.tex}
%\input{./analyse-fonctionnelle/main.tex}
% \input{./Analyse-convexe-et-dualite-en-optimisation/main.tex}
%\input{./tikz/main.tex}
%\input{./Theorie-du-distributions/main.tex}
%\input{./optimisation/mine.tex}
 \input{./modelisation/main.tex}

% yves.aubry@univ-tln.fr : algebra

\end{document}

% % !TEX encoding = UTF-8 Unicode
% !TEX TS-program = xelatex

\documentclass[french]{report}

%\usepackage[utf8]{inputenc}
%\usepackage[T1]{fontenc}
\usepackage{babel}


\newif\ifcomment
%\commenttrue # Show comments

\usepackage{physics}
\usepackage{amssymb}


\usepackage{amsthm}
% \usepackage{thmtools}
\usepackage{mathtools}
\usepackage{amsfonts}

\usepackage{color}

\usepackage{tikz}

\usepackage{geometry}
\geometry{a5paper, margin=0.1in, right=1cm}

\usepackage{dsfont}

\usepackage{graphicx}
\graphicspath{ {images/} }

\usepackage{faktor}

\usepackage{IEEEtrantools}
\usepackage{enumerate}   
\usepackage[PostScript=dvips]{"/Users/aware/Documents/Courses/diagrams"}


\newtheorem{theorem}{Théorème}[section]
\renewcommand{\thetheorem}{\arabic{theorem}}
\newtheorem{lemme}{Lemme}[section]
\renewcommand{\thelemme}{\arabic{lemme}}
\newtheorem{proposition}{Proposition}[section]
\renewcommand{\theproposition}{\arabic{proposition}}
\newtheorem{notations}{Notations}[section]
\newtheorem{problem}{Problème}[section]
\newtheorem{corollary}{Corollaire}[theorem]
\renewcommand{\thecorollary}{\arabic{corollary}}
\newtheorem{property}{Propriété}[section]
\newtheorem{objective}{Objectif}[section]

\theoremstyle{definition}
\newtheorem{definition}{Définition}[section]
\renewcommand{\thedefinition}{\arabic{definition}}
\newtheorem{exercise}{Exercice}[chapter]
\renewcommand{\theexercise}{\arabic{exercise}}
\newtheorem{example}{Exemple}[chapter]
\renewcommand{\theexample}{\arabic{example}}
\newtheorem*{solution}{Solution}
\newtheorem*{application}{Application}
\newtheorem*{notation}{Notation}
\newtheorem*{vocabulary}{Vocabulaire}
\newtheorem*{properties}{Propriétés}



\theoremstyle{remark}
\newtheorem*{remark}{Remarque}
\newtheorem*{rappel}{Rappel}


\usepackage{etoolbox}
\AtBeginEnvironment{exercise}{\small}
\AtBeginEnvironment{example}{\small}

\usepackage{cases}
\usepackage[red]{mypack}

\usepackage[framemethod=TikZ]{mdframed}

\definecolor{bg}{rgb}{0.4,0.25,0.95}
\definecolor{pagebg}{rgb}{0,0,0.5}
\surroundwithmdframed[
   topline=false,
   rightline=false,
   bottomline=false,
   leftmargin=\parindent,
   skipabove=8pt,
   skipbelow=8pt,
   linecolor=blue,
   innerbottommargin=10pt,
   % backgroundcolor=bg,font=\color{orange}\sffamily, fontcolor=white
]{definition}

\usepackage{empheq}
\usepackage[most]{tcolorbox}

\newtcbox{\mymath}[1][]{%
    nobeforeafter, math upper, tcbox raise base,
    enhanced, colframe=blue!30!black,
    colback=red!10, boxrule=1pt,
    #1}

\usepackage{unixode}


\DeclareMathOperator{\ord}{ord}
\DeclareMathOperator{\orb}{orb}
\DeclareMathOperator{\stab}{stab}
\DeclareMathOperator{\Stab}{stab}
\DeclareMathOperator{\ppcm}{ppcm}
\DeclareMathOperator{\conj}{Conj}
\DeclareMathOperator{\End}{End}
\DeclareMathOperator{\rot}{rot}
\DeclareMathOperator{\trs}{trace}
\DeclareMathOperator{\Ind}{Ind}
\DeclareMathOperator{\mat}{Mat}
\DeclareMathOperator{\id}{Id}
\DeclareMathOperator{\vect}{vect}
\DeclareMathOperator{\img}{img}
\DeclareMathOperator{\cov}{Cov}
\DeclareMathOperator{\dist}{dist}
\DeclareMathOperator{\irr}{Irr}
\DeclareMathOperator{\image}{Im}
\DeclareMathOperator{\pd}{\partial}
\DeclareMathOperator{\epi}{epi}
\DeclareMathOperator{\Argmin}{Argmin}
\DeclareMathOperator{\dom}{dom}
\DeclareMathOperator{\proj}{proj}
\DeclareMathOperator{\ctg}{ctg}
\DeclareMathOperator{\supp}{supp}
\DeclareMathOperator{\argmin}{argmin}
\DeclareMathOperator{\mult}{mult}
\DeclareMathOperator{\ch}{ch}
\DeclareMathOperator{\sh}{sh}
\DeclareMathOperator{\rang}{rang}
\DeclareMathOperator{\diam}{diam}
\DeclareMathOperator{\Epigraphe}{Epigraphe}




\usepackage{xcolor}
\everymath{\color{blue}}
%\everymath{\color[rgb]{0,1,1}}
%\pagecolor[rgb]{0,0,0.5}


\newcommand*{\pdtest}[3][]{\ensuremath{\frac{\partial^{#1} #2}{\partial #3}}}

\newcommand*{\deffunc}[6][]{\ensuremath{
\begin{array}{rcl}
#2 : #3 &\rightarrow& #4\\
#5 &\mapsto& #6
\end{array}
}}

\newcommand{\eqcolon}{\mathrel{\resizebox{\widthof{$\mathord{=}$}}{\height}{ $\!\!=\!\!\resizebox{1.2\width}{0.8\height}{\raisebox{0.23ex}{$\mathop{:}$}}\!\!$ }}}
\newcommand{\coloneq}{\mathrel{\resizebox{\widthof{$\mathord{=}$}}{\height}{ $\!\!\resizebox{1.2\width}{0.8\height}{\raisebox{0.23ex}{$\mathop{:}$}}\!\!=\!\!$ }}}
\newcommand{\eqcolonl}{\ensuremath{\mathrel{=\!\!\mathop{:}}}}
\newcommand{\coloneql}{\ensuremath{\mathrel{\mathop{:} \!\! =}}}
\newcommand{\vc}[1]{% inline column vector
  \left(\begin{smallmatrix}#1\end{smallmatrix}\right)%
}
\newcommand{\vr}[1]{% inline row vector
  \begin{smallmatrix}(\,#1\,)\end{smallmatrix}%
}
\makeatletter
\newcommand*{\defeq}{\ =\mathrel{\rlap{%
                     \raisebox{0.3ex}{$\m@th\cdot$}}%
                     \raisebox{-0.3ex}{$\m@th\cdot$}}%
                     }
\makeatother

\newcommand{\mathcircle}[1]{% inline row vector
 \overset{\circ}{#1}
}
\newcommand{\ulim}{% low limit
 \underline{\lim}
}
\newcommand{\ssi}{% iff
\iff
}
\newcommand{\ps}[2]{
\expval{#1 | #2}
}
\newcommand{\df}[1]{
\mqty{#1}
}
\newcommand{\n}[1]{
\norm{#1}
}
\newcommand{\sys}[1]{
\left\{\smqty{#1}\right.
}


\newcommand{\eqdef}{\ensuremath{\overset{\text{def}}=}}


\def\Circlearrowright{\ensuremath{%
  \rotatebox[origin=c]{230}{$\circlearrowright$}}}

\newcommand\ct[1]{\text{\rmfamily\upshape #1}}
\newcommand\question[1]{ {\color{red} ...!? \small #1}}
\newcommand\caz[1]{\left\{\begin{array} #1 \end{array}\right.}
\newcommand\const{\text{\rmfamily\upshape const}}
\newcommand\toP{ \overset{\pro}{\to}}
\newcommand\toPP{ \overset{\text{PP}}{\to}}
\newcommand{\oeq}{\mathrel{\text{\textcircled{$=$}}}}





\usepackage{xcolor}
% \usepackage[normalem]{ulem}
\usepackage{lipsum}
\makeatletter
% \newcommand\colorwave[1][blue]{\bgroup \markoverwith{\lower3.5\p@\hbox{\sixly \textcolor{#1}{\char58}}}\ULon}
%\font\sixly=lasy6 % does not re-load if already loaded, so no memory problem.

\newmdtheoremenv[
linewidth= 1pt,linecolor= blue,%
leftmargin=20,rightmargin=20,innertopmargin=0pt, innerrightmargin=40,%
tikzsetting = { draw=lightgray, line width = 0.3pt,dashed,%
dash pattern = on 15pt off 3pt},%
splittopskip=\topskip,skipbelow=\baselineskip,%
skipabove=\baselineskip,ntheorem,roundcorner=0pt,
% backgroundcolor=pagebg,font=\color{orange}\sffamily, fontcolor=white
]{examplebox}{Exemple}[section]



\newcommand\R{\mathbb{R}}
\newcommand\Z{\mathbb{Z}}
\newcommand\N{\mathbb{N}}
\newcommand\E{\mathbb{E}}
\newcommand\F{\mathcal{F}}
\newcommand\cH{\mathcal{H}}
\newcommand\V{\mathbb{V}}
\newcommand\dmo{ ^{-1} }
\newcommand\kapa{\kappa}
\newcommand\im{Im}
\newcommand\hs{\mathcal{H}}





\usepackage{soul}

\makeatletter
\newcommand*{\whiten}[1]{\llap{\textcolor{white}{{\the\SOUL@token}}\hspace{#1pt}}}
\DeclareRobustCommand*\myul{%
    \def\SOUL@everyspace{\underline{\space}\kern\z@}%
    \def\SOUL@everytoken{%
     \setbox0=\hbox{\the\SOUL@token}%
     \ifdim\dp0>\z@
        \raisebox{\dp0}{\underline{\phantom{\the\SOUL@token}}}%
        \whiten{1}\whiten{0}%
        \whiten{-1}\whiten{-2}%
        \llap{\the\SOUL@token}%
     \else
        \underline{\the\SOUL@token}%
     \fi}%
\SOUL@}
\makeatother

\newcommand*{\demp}{\fontfamily{lmtt}\selectfont}

\DeclareTextFontCommand{\textdemp}{\demp}

\begin{document}

\ifcomment
Multiline
comment
\fi
\ifcomment
\myul{Typesetting test}
% \color[rgb]{1,1,1}
$∑_i^n≠ 60º±∞π∆¬≈√j∫h≤≥µ$

$\CR \R\pro\ind\pro\gS\pro
\mqty[a&b\\c&d]$
$\pro\mathbb{P}$
$\dd{x}$

  \[
    \alpha(x)=\left\{
                \begin{array}{ll}
                  x\\
                  \frac{1}{1+e^{-kx}}\\
                  \frac{e^x-e^{-x}}{e^x+e^{-x}}
                \end{array}
              \right.
  \]

  $\expval{x}$
  
  $\chi_\rho(ghg\dmo)=\Tr(\rho_{ghg\dmo})=\Tr(\rho_g\circ\rho_h\circ\rho\dmo_g)=\Tr(\rho_h)\overset{\mbox{\scalebox{0.5}{$\Tr(AB)=\Tr(BA)$}}}{=}\chi_\rho(h)$
  	$\mathop{\oplus}_{\substack{x\in X}}$

$\mat(\rho_g)=(a_{ij}(g))_{\scriptsize \substack{1\leq i\leq d \\ 1\leq j\leq d}}$ et $\mat(\rho'_g)=(a'_{ij}(g))_{\scriptsize \substack{1\leq i'\leq d' \\ 1\leq j'\leq d'}}$



\[\int_a^b{\mathbb{R}^2}g(u, v)\dd{P_{XY}}(u, v)=\iint g(u,v) f_{XY}(u, v)\dd \lambda(u) \dd \lambda(v)\]
$$\lim_{x\to\infty} f(x)$$	
$$\iiiint_V \mu(t,u,v,w) \,dt\,du\,dv\,dw$$
$$\sum_{n=1}^{\infty} 2^{-n} = 1$$	
\begin{definition}
	Si $X$ et $Y$ sont 2 v.a. ou definit la \textsc{Covariance} entre $X$ et $Y$ comme
	$\cov(X,Y)\overset{\text{def}}{=}\E\left[(X-\E(X))(Y-\E(Y))\right]=\E(XY)-\E(X)\E(Y)$.
\end{definition}
\fi
\pagebreak

% \tableofcontents

% insert your code here
%\input{./algebra/main.tex}
%\input{./geometrie-differentielle/main.tex}
%\input{./probabilite/main.tex}
%\input{./analyse-fonctionnelle/main.tex}
% \input{./Analyse-convexe-et-dualite-en-optimisation/main.tex}
%\input{./tikz/main.tex}
%\input{./Theorie-du-distributions/main.tex}
%\input{./optimisation/mine.tex}
 \input{./modelisation/main.tex}

% yves.aubry@univ-tln.fr : algebra

\end{document}

%% !TEX encoding = UTF-8 Unicode
% !TEX TS-program = xelatex

\documentclass[french]{report}

%\usepackage[utf8]{inputenc}
%\usepackage[T1]{fontenc}
\usepackage{babel}


\newif\ifcomment
%\commenttrue # Show comments

\usepackage{physics}
\usepackage{amssymb}


\usepackage{amsthm}
% \usepackage{thmtools}
\usepackage{mathtools}
\usepackage{amsfonts}

\usepackage{color}

\usepackage{tikz}

\usepackage{geometry}
\geometry{a5paper, margin=0.1in, right=1cm}

\usepackage{dsfont}

\usepackage{graphicx}
\graphicspath{ {images/} }

\usepackage{faktor}

\usepackage{IEEEtrantools}
\usepackage{enumerate}   
\usepackage[PostScript=dvips]{"/Users/aware/Documents/Courses/diagrams"}


\newtheorem{theorem}{Théorème}[section]
\renewcommand{\thetheorem}{\arabic{theorem}}
\newtheorem{lemme}{Lemme}[section]
\renewcommand{\thelemme}{\arabic{lemme}}
\newtheorem{proposition}{Proposition}[section]
\renewcommand{\theproposition}{\arabic{proposition}}
\newtheorem{notations}{Notations}[section]
\newtheorem{problem}{Problème}[section]
\newtheorem{corollary}{Corollaire}[theorem]
\renewcommand{\thecorollary}{\arabic{corollary}}
\newtheorem{property}{Propriété}[section]
\newtheorem{objective}{Objectif}[section]

\theoremstyle{definition}
\newtheorem{definition}{Définition}[section]
\renewcommand{\thedefinition}{\arabic{definition}}
\newtheorem{exercise}{Exercice}[chapter]
\renewcommand{\theexercise}{\arabic{exercise}}
\newtheorem{example}{Exemple}[chapter]
\renewcommand{\theexample}{\arabic{example}}
\newtheorem*{solution}{Solution}
\newtheorem*{application}{Application}
\newtheorem*{notation}{Notation}
\newtheorem*{vocabulary}{Vocabulaire}
\newtheorem*{properties}{Propriétés}



\theoremstyle{remark}
\newtheorem*{remark}{Remarque}
\newtheorem*{rappel}{Rappel}


\usepackage{etoolbox}
\AtBeginEnvironment{exercise}{\small}
\AtBeginEnvironment{example}{\small}

\usepackage{cases}
\usepackage[red]{mypack}

\usepackage[framemethod=TikZ]{mdframed}

\definecolor{bg}{rgb}{0.4,0.25,0.95}
\definecolor{pagebg}{rgb}{0,0,0.5}
\surroundwithmdframed[
   topline=false,
   rightline=false,
   bottomline=false,
   leftmargin=\parindent,
   skipabove=8pt,
   skipbelow=8pt,
   linecolor=blue,
   innerbottommargin=10pt,
   % backgroundcolor=bg,font=\color{orange}\sffamily, fontcolor=white
]{definition}

\usepackage{empheq}
\usepackage[most]{tcolorbox}

\newtcbox{\mymath}[1][]{%
    nobeforeafter, math upper, tcbox raise base,
    enhanced, colframe=blue!30!black,
    colback=red!10, boxrule=1pt,
    #1}

\usepackage{unixode}


\DeclareMathOperator{\ord}{ord}
\DeclareMathOperator{\orb}{orb}
\DeclareMathOperator{\stab}{stab}
\DeclareMathOperator{\Stab}{stab}
\DeclareMathOperator{\ppcm}{ppcm}
\DeclareMathOperator{\conj}{Conj}
\DeclareMathOperator{\End}{End}
\DeclareMathOperator{\rot}{rot}
\DeclareMathOperator{\trs}{trace}
\DeclareMathOperator{\Ind}{Ind}
\DeclareMathOperator{\mat}{Mat}
\DeclareMathOperator{\id}{Id}
\DeclareMathOperator{\vect}{vect}
\DeclareMathOperator{\img}{img}
\DeclareMathOperator{\cov}{Cov}
\DeclareMathOperator{\dist}{dist}
\DeclareMathOperator{\irr}{Irr}
\DeclareMathOperator{\image}{Im}
\DeclareMathOperator{\pd}{\partial}
\DeclareMathOperator{\epi}{epi}
\DeclareMathOperator{\Argmin}{Argmin}
\DeclareMathOperator{\dom}{dom}
\DeclareMathOperator{\proj}{proj}
\DeclareMathOperator{\ctg}{ctg}
\DeclareMathOperator{\supp}{supp}
\DeclareMathOperator{\argmin}{argmin}
\DeclareMathOperator{\mult}{mult}
\DeclareMathOperator{\ch}{ch}
\DeclareMathOperator{\sh}{sh}
\DeclareMathOperator{\rang}{rang}
\DeclareMathOperator{\diam}{diam}
\DeclareMathOperator{\Epigraphe}{Epigraphe}




\usepackage{xcolor}
\everymath{\color{blue}}
%\everymath{\color[rgb]{0,1,1}}
%\pagecolor[rgb]{0,0,0.5}


\newcommand*{\pdtest}[3][]{\ensuremath{\frac{\partial^{#1} #2}{\partial #3}}}

\newcommand*{\deffunc}[6][]{\ensuremath{
\begin{array}{rcl}
#2 : #3 &\rightarrow& #4\\
#5 &\mapsto& #6
\end{array}
}}

\newcommand{\eqcolon}{\mathrel{\resizebox{\widthof{$\mathord{=}$}}{\height}{ $\!\!=\!\!\resizebox{1.2\width}{0.8\height}{\raisebox{0.23ex}{$\mathop{:}$}}\!\!$ }}}
\newcommand{\coloneq}{\mathrel{\resizebox{\widthof{$\mathord{=}$}}{\height}{ $\!\!\resizebox{1.2\width}{0.8\height}{\raisebox{0.23ex}{$\mathop{:}$}}\!\!=\!\!$ }}}
\newcommand{\eqcolonl}{\ensuremath{\mathrel{=\!\!\mathop{:}}}}
\newcommand{\coloneql}{\ensuremath{\mathrel{\mathop{:} \!\! =}}}
\newcommand{\vc}[1]{% inline column vector
  \left(\begin{smallmatrix}#1\end{smallmatrix}\right)%
}
\newcommand{\vr}[1]{% inline row vector
  \begin{smallmatrix}(\,#1\,)\end{smallmatrix}%
}
\makeatletter
\newcommand*{\defeq}{\ =\mathrel{\rlap{%
                     \raisebox{0.3ex}{$\m@th\cdot$}}%
                     \raisebox{-0.3ex}{$\m@th\cdot$}}%
                     }
\makeatother

\newcommand{\mathcircle}[1]{% inline row vector
 \overset{\circ}{#1}
}
\newcommand{\ulim}{% low limit
 \underline{\lim}
}
\newcommand{\ssi}{% iff
\iff
}
\newcommand{\ps}[2]{
\expval{#1 | #2}
}
\newcommand{\df}[1]{
\mqty{#1}
}
\newcommand{\n}[1]{
\norm{#1}
}
\newcommand{\sys}[1]{
\left\{\smqty{#1}\right.
}


\newcommand{\eqdef}{\ensuremath{\overset{\text{def}}=}}


\def\Circlearrowright{\ensuremath{%
  \rotatebox[origin=c]{230}{$\circlearrowright$}}}

\newcommand\ct[1]{\text{\rmfamily\upshape #1}}
\newcommand\question[1]{ {\color{red} ...!? \small #1}}
\newcommand\caz[1]{\left\{\begin{array} #1 \end{array}\right.}
\newcommand\const{\text{\rmfamily\upshape const}}
\newcommand\toP{ \overset{\pro}{\to}}
\newcommand\toPP{ \overset{\text{PP}}{\to}}
\newcommand{\oeq}{\mathrel{\text{\textcircled{$=$}}}}





\usepackage{xcolor}
% \usepackage[normalem]{ulem}
\usepackage{lipsum}
\makeatletter
% \newcommand\colorwave[1][blue]{\bgroup \markoverwith{\lower3.5\p@\hbox{\sixly \textcolor{#1}{\char58}}}\ULon}
%\font\sixly=lasy6 % does not re-load if already loaded, so no memory problem.

\newmdtheoremenv[
linewidth= 1pt,linecolor= blue,%
leftmargin=20,rightmargin=20,innertopmargin=0pt, innerrightmargin=40,%
tikzsetting = { draw=lightgray, line width = 0.3pt,dashed,%
dash pattern = on 15pt off 3pt},%
splittopskip=\topskip,skipbelow=\baselineskip,%
skipabove=\baselineskip,ntheorem,roundcorner=0pt,
% backgroundcolor=pagebg,font=\color{orange}\sffamily, fontcolor=white
]{examplebox}{Exemple}[section]



\newcommand\R{\mathbb{R}}
\newcommand\Z{\mathbb{Z}}
\newcommand\N{\mathbb{N}}
\newcommand\E{\mathbb{E}}
\newcommand\F{\mathcal{F}}
\newcommand\cH{\mathcal{H}}
\newcommand\V{\mathbb{V}}
\newcommand\dmo{ ^{-1} }
\newcommand\kapa{\kappa}
\newcommand\im{Im}
\newcommand\hs{\mathcal{H}}





\usepackage{soul}

\makeatletter
\newcommand*{\whiten}[1]{\llap{\textcolor{white}{{\the\SOUL@token}}\hspace{#1pt}}}
\DeclareRobustCommand*\myul{%
    \def\SOUL@everyspace{\underline{\space}\kern\z@}%
    \def\SOUL@everytoken{%
     \setbox0=\hbox{\the\SOUL@token}%
     \ifdim\dp0>\z@
        \raisebox{\dp0}{\underline{\phantom{\the\SOUL@token}}}%
        \whiten{1}\whiten{0}%
        \whiten{-1}\whiten{-2}%
        \llap{\the\SOUL@token}%
     \else
        \underline{\the\SOUL@token}%
     \fi}%
\SOUL@}
\makeatother

\newcommand*{\demp}{\fontfamily{lmtt}\selectfont}

\DeclareTextFontCommand{\textdemp}{\demp}

\begin{document}

\ifcomment
Multiline
comment
\fi
\ifcomment
\myul{Typesetting test}
% \color[rgb]{1,1,1}
$∑_i^n≠ 60º±∞π∆¬≈√j∫h≤≥µ$

$\CR \R\pro\ind\pro\gS\pro
\mqty[a&b\\c&d]$
$\pro\mathbb{P}$
$\dd{x}$

  \[
    \alpha(x)=\left\{
                \begin{array}{ll}
                  x\\
                  \frac{1}{1+e^{-kx}}\\
                  \frac{e^x-e^{-x}}{e^x+e^{-x}}
                \end{array}
              \right.
  \]

  $\expval{x}$
  
  $\chi_\rho(ghg\dmo)=\Tr(\rho_{ghg\dmo})=\Tr(\rho_g\circ\rho_h\circ\rho\dmo_g)=\Tr(\rho_h)\overset{\mbox{\scalebox{0.5}{$\Tr(AB)=\Tr(BA)$}}}{=}\chi_\rho(h)$
  	$\mathop{\oplus}_{\substack{x\in X}}$

$\mat(\rho_g)=(a_{ij}(g))_{\scriptsize \substack{1\leq i\leq d \\ 1\leq j\leq d}}$ et $\mat(\rho'_g)=(a'_{ij}(g))_{\scriptsize \substack{1\leq i'\leq d' \\ 1\leq j'\leq d'}}$



\[\int_a^b{\mathbb{R}^2}g(u, v)\dd{P_{XY}}(u, v)=\iint g(u,v) f_{XY}(u, v)\dd \lambda(u) \dd \lambda(v)\]
$$\lim_{x\to\infty} f(x)$$	
$$\iiiint_V \mu(t,u,v,w) \,dt\,du\,dv\,dw$$
$$\sum_{n=1}^{\infty} 2^{-n} = 1$$	
\begin{definition}
	Si $X$ et $Y$ sont 2 v.a. ou definit la \textsc{Covariance} entre $X$ et $Y$ comme
	$\cov(X,Y)\overset{\text{def}}{=}\E\left[(X-\E(X))(Y-\E(Y))\right]=\E(XY)-\E(X)\E(Y)$.
\end{definition}
\fi
\pagebreak

% \tableofcontents

% insert your code here
%\input{./algebra/main.tex}
%\input{./geometrie-differentielle/main.tex}
%\input{./probabilite/main.tex}
%\input{./analyse-fonctionnelle/main.tex}
% \input{./Analyse-convexe-et-dualite-en-optimisation/main.tex}
%\input{./tikz/main.tex}
%\input{./Theorie-du-distributions/main.tex}
%\input{./optimisation/mine.tex}
 \input{./modelisation/main.tex}

% yves.aubry@univ-tln.fr : algebra

\end{document}

%% !TEX encoding = UTF-8 Unicode
% !TEX TS-program = xelatex

\documentclass[french]{report}

%\usepackage[utf8]{inputenc}
%\usepackage[T1]{fontenc}
\usepackage{babel}


\newif\ifcomment
%\commenttrue # Show comments

\usepackage{physics}
\usepackage{amssymb}


\usepackage{amsthm}
% \usepackage{thmtools}
\usepackage{mathtools}
\usepackage{amsfonts}

\usepackage{color}

\usepackage{tikz}

\usepackage{geometry}
\geometry{a5paper, margin=0.1in, right=1cm}

\usepackage{dsfont}

\usepackage{graphicx}
\graphicspath{ {images/} }

\usepackage{faktor}

\usepackage{IEEEtrantools}
\usepackage{enumerate}   
\usepackage[PostScript=dvips]{"/Users/aware/Documents/Courses/diagrams"}


\newtheorem{theorem}{Théorème}[section]
\renewcommand{\thetheorem}{\arabic{theorem}}
\newtheorem{lemme}{Lemme}[section]
\renewcommand{\thelemme}{\arabic{lemme}}
\newtheorem{proposition}{Proposition}[section]
\renewcommand{\theproposition}{\arabic{proposition}}
\newtheorem{notations}{Notations}[section]
\newtheorem{problem}{Problème}[section]
\newtheorem{corollary}{Corollaire}[theorem]
\renewcommand{\thecorollary}{\arabic{corollary}}
\newtheorem{property}{Propriété}[section]
\newtheorem{objective}{Objectif}[section]

\theoremstyle{definition}
\newtheorem{definition}{Définition}[section]
\renewcommand{\thedefinition}{\arabic{definition}}
\newtheorem{exercise}{Exercice}[chapter]
\renewcommand{\theexercise}{\arabic{exercise}}
\newtheorem{example}{Exemple}[chapter]
\renewcommand{\theexample}{\arabic{example}}
\newtheorem*{solution}{Solution}
\newtheorem*{application}{Application}
\newtheorem*{notation}{Notation}
\newtheorem*{vocabulary}{Vocabulaire}
\newtheorem*{properties}{Propriétés}



\theoremstyle{remark}
\newtheorem*{remark}{Remarque}
\newtheorem*{rappel}{Rappel}


\usepackage{etoolbox}
\AtBeginEnvironment{exercise}{\small}
\AtBeginEnvironment{example}{\small}

\usepackage{cases}
\usepackage[red]{mypack}

\usepackage[framemethod=TikZ]{mdframed}

\definecolor{bg}{rgb}{0.4,0.25,0.95}
\definecolor{pagebg}{rgb}{0,0,0.5}
\surroundwithmdframed[
   topline=false,
   rightline=false,
   bottomline=false,
   leftmargin=\parindent,
   skipabove=8pt,
   skipbelow=8pt,
   linecolor=blue,
   innerbottommargin=10pt,
   % backgroundcolor=bg,font=\color{orange}\sffamily, fontcolor=white
]{definition}

\usepackage{empheq}
\usepackage[most]{tcolorbox}

\newtcbox{\mymath}[1][]{%
    nobeforeafter, math upper, tcbox raise base,
    enhanced, colframe=blue!30!black,
    colback=red!10, boxrule=1pt,
    #1}

\usepackage{unixode}


\DeclareMathOperator{\ord}{ord}
\DeclareMathOperator{\orb}{orb}
\DeclareMathOperator{\stab}{stab}
\DeclareMathOperator{\Stab}{stab}
\DeclareMathOperator{\ppcm}{ppcm}
\DeclareMathOperator{\conj}{Conj}
\DeclareMathOperator{\End}{End}
\DeclareMathOperator{\rot}{rot}
\DeclareMathOperator{\trs}{trace}
\DeclareMathOperator{\Ind}{Ind}
\DeclareMathOperator{\mat}{Mat}
\DeclareMathOperator{\id}{Id}
\DeclareMathOperator{\vect}{vect}
\DeclareMathOperator{\img}{img}
\DeclareMathOperator{\cov}{Cov}
\DeclareMathOperator{\dist}{dist}
\DeclareMathOperator{\irr}{Irr}
\DeclareMathOperator{\image}{Im}
\DeclareMathOperator{\pd}{\partial}
\DeclareMathOperator{\epi}{epi}
\DeclareMathOperator{\Argmin}{Argmin}
\DeclareMathOperator{\dom}{dom}
\DeclareMathOperator{\proj}{proj}
\DeclareMathOperator{\ctg}{ctg}
\DeclareMathOperator{\supp}{supp}
\DeclareMathOperator{\argmin}{argmin}
\DeclareMathOperator{\mult}{mult}
\DeclareMathOperator{\ch}{ch}
\DeclareMathOperator{\sh}{sh}
\DeclareMathOperator{\rang}{rang}
\DeclareMathOperator{\diam}{diam}
\DeclareMathOperator{\Epigraphe}{Epigraphe}




\usepackage{xcolor}
\everymath{\color{blue}}
%\everymath{\color[rgb]{0,1,1}}
%\pagecolor[rgb]{0,0,0.5}


\newcommand*{\pdtest}[3][]{\ensuremath{\frac{\partial^{#1} #2}{\partial #3}}}

\newcommand*{\deffunc}[6][]{\ensuremath{
\begin{array}{rcl}
#2 : #3 &\rightarrow& #4\\
#5 &\mapsto& #6
\end{array}
}}

\newcommand{\eqcolon}{\mathrel{\resizebox{\widthof{$\mathord{=}$}}{\height}{ $\!\!=\!\!\resizebox{1.2\width}{0.8\height}{\raisebox{0.23ex}{$\mathop{:}$}}\!\!$ }}}
\newcommand{\coloneq}{\mathrel{\resizebox{\widthof{$\mathord{=}$}}{\height}{ $\!\!\resizebox{1.2\width}{0.8\height}{\raisebox{0.23ex}{$\mathop{:}$}}\!\!=\!\!$ }}}
\newcommand{\eqcolonl}{\ensuremath{\mathrel{=\!\!\mathop{:}}}}
\newcommand{\coloneql}{\ensuremath{\mathrel{\mathop{:} \!\! =}}}
\newcommand{\vc}[1]{% inline column vector
  \left(\begin{smallmatrix}#1\end{smallmatrix}\right)%
}
\newcommand{\vr}[1]{% inline row vector
  \begin{smallmatrix}(\,#1\,)\end{smallmatrix}%
}
\makeatletter
\newcommand*{\defeq}{\ =\mathrel{\rlap{%
                     \raisebox{0.3ex}{$\m@th\cdot$}}%
                     \raisebox{-0.3ex}{$\m@th\cdot$}}%
                     }
\makeatother

\newcommand{\mathcircle}[1]{% inline row vector
 \overset{\circ}{#1}
}
\newcommand{\ulim}{% low limit
 \underline{\lim}
}
\newcommand{\ssi}{% iff
\iff
}
\newcommand{\ps}[2]{
\expval{#1 | #2}
}
\newcommand{\df}[1]{
\mqty{#1}
}
\newcommand{\n}[1]{
\norm{#1}
}
\newcommand{\sys}[1]{
\left\{\smqty{#1}\right.
}


\newcommand{\eqdef}{\ensuremath{\overset{\text{def}}=}}


\def\Circlearrowright{\ensuremath{%
  \rotatebox[origin=c]{230}{$\circlearrowright$}}}

\newcommand\ct[1]{\text{\rmfamily\upshape #1}}
\newcommand\question[1]{ {\color{red} ...!? \small #1}}
\newcommand\caz[1]{\left\{\begin{array} #1 \end{array}\right.}
\newcommand\const{\text{\rmfamily\upshape const}}
\newcommand\toP{ \overset{\pro}{\to}}
\newcommand\toPP{ \overset{\text{PP}}{\to}}
\newcommand{\oeq}{\mathrel{\text{\textcircled{$=$}}}}





\usepackage{xcolor}
% \usepackage[normalem]{ulem}
\usepackage{lipsum}
\makeatletter
% \newcommand\colorwave[1][blue]{\bgroup \markoverwith{\lower3.5\p@\hbox{\sixly \textcolor{#1}{\char58}}}\ULon}
%\font\sixly=lasy6 % does not re-load if already loaded, so no memory problem.

\newmdtheoremenv[
linewidth= 1pt,linecolor= blue,%
leftmargin=20,rightmargin=20,innertopmargin=0pt, innerrightmargin=40,%
tikzsetting = { draw=lightgray, line width = 0.3pt,dashed,%
dash pattern = on 15pt off 3pt},%
splittopskip=\topskip,skipbelow=\baselineskip,%
skipabove=\baselineskip,ntheorem,roundcorner=0pt,
% backgroundcolor=pagebg,font=\color{orange}\sffamily, fontcolor=white
]{examplebox}{Exemple}[section]



\newcommand\R{\mathbb{R}}
\newcommand\Z{\mathbb{Z}}
\newcommand\N{\mathbb{N}}
\newcommand\E{\mathbb{E}}
\newcommand\F{\mathcal{F}}
\newcommand\cH{\mathcal{H}}
\newcommand\V{\mathbb{V}}
\newcommand\dmo{ ^{-1} }
\newcommand\kapa{\kappa}
\newcommand\im{Im}
\newcommand\hs{\mathcal{H}}





\usepackage{soul}

\makeatletter
\newcommand*{\whiten}[1]{\llap{\textcolor{white}{{\the\SOUL@token}}\hspace{#1pt}}}
\DeclareRobustCommand*\myul{%
    \def\SOUL@everyspace{\underline{\space}\kern\z@}%
    \def\SOUL@everytoken{%
     \setbox0=\hbox{\the\SOUL@token}%
     \ifdim\dp0>\z@
        \raisebox{\dp0}{\underline{\phantom{\the\SOUL@token}}}%
        \whiten{1}\whiten{0}%
        \whiten{-1}\whiten{-2}%
        \llap{\the\SOUL@token}%
     \else
        \underline{\the\SOUL@token}%
     \fi}%
\SOUL@}
\makeatother

\newcommand*{\demp}{\fontfamily{lmtt}\selectfont}

\DeclareTextFontCommand{\textdemp}{\demp}

\begin{document}

\ifcomment
Multiline
comment
\fi
\ifcomment
\myul{Typesetting test}
% \color[rgb]{1,1,1}
$∑_i^n≠ 60º±∞π∆¬≈√j∫h≤≥µ$

$\CR \R\pro\ind\pro\gS\pro
\mqty[a&b\\c&d]$
$\pro\mathbb{P}$
$\dd{x}$

  \[
    \alpha(x)=\left\{
                \begin{array}{ll}
                  x\\
                  \frac{1}{1+e^{-kx}}\\
                  \frac{e^x-e^{-x}}{e^x+e^{-x}}
                \end{array}
              \right.
  \]

  $\expval{x}$
  
  $\chi_\rho(ghg\dmo)=\Tr(\rho_{ghg\dmo})=\Tr(\rho_g\circ\rho_h\circ\rho\dmo_g)=\Tr(\rho_h)\overset{\mbox{\scalebox{0.5}{$\Tr(AB)=\Tr(BA)$}}}{=}\chi_\rho(h)$
  	$\mathop{\oplus}_{\substack{x\in X}}$

$\mat(\rho_g)=(a_{ij}(g))_{\scriptsize \substack{1\leq i\leq d \\ 1\leq j\leq d}}$ et $\mat(\rho'_g)=(a'_{ij}(g))_{\scriptsize \substack{1\leq i'\leq d' \\ 1\leq j'\leq d'}}$



\[\int_a^b{\mathbb{R}^2}g(u, v)\dd{P_{XY}}(u, v)=\iint g(u,v) f_{XY}(u, v)\dd \lambda(u) \dd \lambda(v)\]
$$\lim_{x\to\infty} f(x)$$	
$$\iiiint_V \mu(t,u,v,w) \,dt\,du\,dv\,dw$$
$$\sum_{n=1}^{\infty} 2^{-n} = 1$$	
\begin{definition}
	Si $X$ et $Y$ sont 2 v.a. ou definit la \textsc{Covariance} entre $X$ et $Y$ comme
	$\cov(X,Y)\overset{\text{def}}{=}\E\left[(X-\E(X))(Y-\E(Y))\right]=\E(XY)-\E(X)\E(Y)$.
\end{definition}
\fi
\pagebreak

% \tableofcontents

% insert your code here
%\input{./algebra/main.tex}
%\input{./geometrie-differentielle/main.tex}
%\input{./probabilite/main.tex}
%\input{./analyse-fonctionnelle/main.tex}
% \input{./Analyse-convexe-et-dualite-en-optimisation/main.tex}
%\input{./tikz/main.tex}
%\input{./Theorie-du-distributions/main.tex}
%\input{./optimisation/mine.tex}
 \input{./modelisation/main.tex}

% yves.aubry@univ-tln.fr : algebra

\end{document}

%\input{./optimisation/mine.tex}
 % !TEX encoding = UTF-8 Unicode
% !TEX TS-program = xelatex

\documentclass[french]{report}

%\usepackage[utf8]{inputenc}
%\usepackage[T1]{fontenc}
\usepackage{babel}


\newif\ifcomment
%\commenttrue # Show comments

\usepackage{physics}
\usepackage{amssymb}


\usepackage{amsthm}
% \usepackage{thmtools}
\usepackage{mathtools}
\usepackage{amsfonts}

\usepackage{color}

\usepackage{tikz}

\usepackage{geometry}
\geometry{a5paper, margin=0.1in, right=1cm}

\usepackage{dsfont}

\usepackage{graphicx}
\graphicspath{ {images/} }

\usepackage{faktor}

\usepackage{IEEEtrantools}
\usepackage{enumerate}   
\usepackage[PostScript=dvips]{"/Users/aware/Documents/Courses/diagrams"}


\newtheorem{theorem}{Théorème}[section]
\renewcommand{\thetheorem}{\arabic{theorem}}
\newtheorem{lemme}{Lemme}[section]
\renewcommand{\thelemme}{\arabic{lemme}}
\newtheorem{proposition}{Proposition}[section]
\renewcommand{\theproposition}{\arabic{proposition}}
\newtheorem{notations}{Notations}[section]
\newtheorem{problem}{Problème}[section]
\newtheorem{corollary}{Corollaire}[theorem]
\renewcommand{\thecorollary}{\arabic{corollary}}
\newtheorem{property}{Propriété}[section]
\newtheorem{objective}{Objectif}[section]

\theoremstyle{definition}
\newtheorem{definition}{Définition}[section]
\renewcommand{\thedefinition}{\arabic{definition}}
\newtheorem{exercise}{Exercice}[chapter]
\renewcommand{\theexercise}{\arabic{exercise}}
\newtheorem{example}{Exemple}[chapter]
\renewcommand{\theexample}{\arabic{example}}
\newtheorem*{solution}{Solution}
\newtheorem*{application}{Application}
\newtheorem*{notation}{Notation}
\newtheorem*{vocabulary}{Vocabulaire}
\newtheorem*{properties}{Propriétés}



\theoremstyle{remark}
\newtheorem*{remark}{Remarque}
\newtheorem*{rappel}{Rappel}


\usepackage{etoolbox}
\AtBeginEnvironment{exercise}{\small}
\AtBeginEnvironment{example}{\small}

\usepackage{cases}
\usepackage[red]{mypack}

\usepackage[framemethod=TikZ]{mdframed}

\definecolor{bg}{rgb}{0.4,0.25,0.95}
\definecolor{pagebg}{rgb}{0,0,0.5}
\surroundwithmdframed[
   topline=false,
   rightline=false,
   bottomline=false,
   leftmargin=\parindent,
   skipabove=8pt,
   skipbelow=8pt,
   linecolor=blue,
   innerbottommargin=10pt,
   % backgroundcolor=bg,font=\color{orange}\sffamily, fontcolor=white
]{definition}

\usepackage{empheq}
\usepackage[most]{tcolorbox}

\newtcbox{\mymath}[1][]{%
    nobeforeafter, math upper, tcbox raise base,
    enhanced, colframe=blue!30!black,
    colback=red!10, boxrule=1pt,
    #1}

\usepackage{unixode}


\DeclareMathOperator{\ord}{ord}
\DeclareMathOperator{\orb}{orb}
\DeclareMathOperator{\stab}{stab}
\DeclareMathOperator{\Stab}{stab}
\DeclareMathOperator{\ppcm}{ppcm}
\DeclareMathOperator{\conj}{Conj}
\DeclareMathOperator{\End}{End}
\DeclareMathOperator{\rot}{rot}
\DeclareMathOperator{\trs}{trace}
\DeclareMathOperator{\Ind}{Ind}
\DeclareMathOperator{\mat}{Mat}
\DeclareMathOperator{\id}{Id}
\DeclareMathOperator{\vect}{vect}
\DeclareMathOperator{\img}{img}
\DeclareMathOperator{\cov}{Cov}
\DeclareMathOperator{\dist}{dist}
\DeclareMathOperator{\irr}{Irr}
\DeclareMathOperator{\image}{Im}
\DeclareMathOperator{\pd}{\partial}
\DeclareMathOperator{\epi}{epi}
\DeclareMathOperator{\Argmin}{Argmin}
\DeclareMathOperator{\dom}{dom}
\DeclareMathOperator{\proj}{proj}
\DeclareMathOperator{\ctg}{ctg}
\DeclareMathOperator{\supp}{supp}
\DeclareMathOperator{\argmin}{argmin}
\DeclareMathOperator{\mult}{mult}
\DeclareMathOperator{\ch}{ch}
\DeclareMathOperator{\sh}{sh}
\DeclareMathOperator{\rang}{rang}
\DeclareMathOperator{\diam}{diam}
\DeclareMathOperator{\Epigraphe}{Epigraphe}




\usepackage{xcolor}
\everymath{\color{blue}}
%\everymath{\color[rgb]{0,1,1}}
%\pagecolor[rgb]{0,0,0.5}


\newcommand*{\pdtest}[3][]{\ensuremath{\frac{\partial^{#1} #2}{\partial #3}}}

\newcommand*{\deffunc}[6][]{\ensuremath{
\begin{array}{rcl}
#2 : #3 &\rightarrow& #4\\
#5 &\mapsto& #6
\end{array}
}}

\newcommand{\eqcolon}{\mathrel{\resizebox{\widthof{$\mathord{=}$}}{\height}{ $\!\!=\!\!\resizebox{1.2\width}{0.8\height}{\raisebox{0.23ex}{$\mathop{:}$}}\!\!$ }}}
\newcommand{\coloneq}{\mathrel{\resizebox{\widthof{$\mathord{=}$}}{\height}{ $\!\!\resizebox{1.2\width}{0.8\height}{\raisebox{0.23ex}{$\mathop{:}$}}\!\!=\!\!$ }}}
\newcommand{\eqcolonl}{\ensuremath{\mathrel{=\!\!\mathop{:}}}}
\newcommand{\coloneql}{\ensuremath{\mathrel{\mathop{:} \!\! =}}}
\newcommand{\vc}[1]{% inline column vector
  \left(\begin{smallmatrix}#1\end{smallmatrix}\right)%
}
\newcommand{\vr}[1]{% inline row vector
  \begin{smallmatrix}(\,#1\,)\end{smallmatrix}%
}
\makeatletter
\newcommand*{\defeq}{\ =\mathrel{\rlap{%
                     \raisebox{0.3ex}{$\m@th\cdot$}}%
                     \raisebox{-0.3ex}{$\m@th\cdot$}}%
                     }
\makeatother

\newcommand{\mathcircle}[1]{% inline row vector
 \overset{\circ}{#1}
}
\newcommand{\ulim}{% low limit
 \underline{\lim}
}
\newcommand{\ssi}{% iff
\iff
}
\newcommand{\ps}[2]{
\expval{#1 | #2}
}
\newcommand{\df}[1]{
\mqty{#1}
}
\newcommand{\n}[1]{
\norm{#1}
}
\newcommand{\sys}[1]{
\left\{\smqty{#1}\right.
}


\newcommand{\eqdef}{\ensuremath{\overset{\text{def}}=}}


\def\Circlearrowright{\ensuremath{%
  \rotatebox[origin=c]{230}{$\circlearrowright$}}}

\newcommand\ct[1]{\text{\rmfamily\upshape #1}}
\newcommand\question[1]{ {\color{red} ...!? \small #1}}
\newcommand\caz[1]{\left\{\begin{array} #1 \end{array}\right.}
\newcommand\const{\text{\rmfamily\upshape const}}
\newcommand\toP{ \overset{\pro}{\to}}
\newcommand\toPP{ \overset{\text{PP}}{\to}}
\newcommand{\oeq}{\mathrel{\text{\textcircled{$=$}}}}





\usepackage{xcolor}
% \usepackage[normalem]{ulem}
\usepackage{lipsum}
\makeatletter
% \newcommand\colorwave[1][blue]{\bgroup \markoverwith{\lower3.5\p@\hbox{\sixly \textcolor{#1}{\char58}}}\ULon}
%\font\sixly=lasy6 % does not re-load if already loaded, so no memory problem.

\newmdtheoremenv[
linewidth= 1pt,linecolor= blue,%
leftmargin=20,rightmargin=20,innertopmargin=0pt, innerrightmargin=40,%
tikzsetting = { draw=lightgray, line width = 0.3pt,dashed,%
dash pattern = on 15pt off 3pt},%
splittopskip=\topskip,skipbelow=\baselineskip,%
skipabove=\baselineskip,ntheorem,roundcorner=0pt,
% backgroundcolor=pagebg,font=\color{orange}\sffamily, fontcolor=white
]{examplebox}{Exemple}[section]



\newcommand\R{\mathbb{R}}
\newcommand\Z{\mathbb{Z}}
\newcommand\N{\mathbb{N}}
\newcommand\E{\mathbb{E}}
\newcommand\F{\mathcal{F}}
\newcommand\cH{\mathcal{H}}
\newcommand\V{\mathbb{V}}
\newcommand\dmo{ ^{-1} }
\newcommand\kapa{\kappa}
\newcommand\im{Im}
\newcommand\hs{\mathcal{H}}





\usepackage{soul}

\makeatletter
\newcommand*{\whiten}[1]{\llap{\textcolor{white}{{\the\SOUL@token}}\hspace{#1pt}}}
\DeclareRobustCommand*\myul{%
    \def\SOUL@everyspace{\underline{\space}\kern\z@}%
    \def\SOUL@everytoken{%
     \setbox0=\hbox{\the\SOUL@token}%
     \ifdim\dp0>\z@
        \raisebox{\dp0}{\underline{\phantom{\the\SOUL@token}}}%
        \whiten{1}\whiten{0}%
        \whiten{-1}\whiten{-2}%
        \llap{\the\SOUL@token}%
     \else
        \underline{\the\SOUL@token}%
     \fi}%
\SOUL@}
\makeatother

\newcommand*{\demp}{\fontfamily{lmtt}\selectfont}

\DeclareTextFontCommand{\textdemp}{\demp}

\begin{document}

\ifcomment
Multiline
comment
\fi
\ifcomment
\myul{Typesetting test}
% \color[rgb]{1,1,1}
$∑_i^n≠ 60º±∞π∆¬≈√j∫h≤≥µ$

$\CR \R\pro\ind\pro\gS\pro
\mqty[a&b\\c&d]$
$\pro\mathbb{P}$
$\dd{x}$

  \[
    \alpha(x)=\left\{
                \begin{array}{ll}
                  x\\
                  \frac{1}{1+e^{-kx}}\\
                  \frac{e^x-e^{-x}}{e^x+e^{-x}}
                \end{array}
              \right.
  \]

  $\expval{x}$
  
  $\chi_\rho(ghg\dmo)=\Tr(\rho_{ghg\dmo})=\Tr(\rho_g\circ\rho_h\circ\rho\dmo_g)=\Tr(\rho_h)\overset{\mbox{\scalebox{0.5}{$\Tr(AB)=\Tr(BA)$}}}{=}\chi_\rho(h)$
  	$\mathop{\oplus}_{\substack{x\in X}}$

$\mat(\rho_g)=(a_{ij}(g))_{\scriptsize \substack{1\leq i\leq d \\ 1\leq j\leq d}}$ et $\mat(\rho'_g)=(a'_{ij}(g))_{\scriptsize \substack{1\leq i'\leq d' \\ 1\leq j'\leq d'}}$



\[\int_a^b{\mathbb{R}^2}g(u, v)\dd{P_{XY}}(u, v)=\iint g(u,v) f_{XY}(u, v)\dd \lambda(u) \dd \lambda(v)\]
$$\lim_{x\to\infty} f(x)$$	
$$\iiiint_V \mu(t,u,v,w) \,dt\,du\,dv\,dw$$
$$\sum_{n=1}^{\infty} 2^{-n} = 1$$	
\begin{definition}
	Si $X$ et $Y$ sont 2 v.a. ou definit la \textsc{Covariance} entre $X$ et $Y$ comme
	$\cov(X,Y)\overset{\text{def}}{=}\E\left[(X-\E(X))(Y-\E(Y))\right]=\E(XY)-\E(X)\E(Y)$.
\end{definition}
\fi
\pagebreak

% \tableofcontents

% insert your code here
%\input{./algebra/main.tex}
%\input{./geometrie-differentielle/main.tex}
%\input{./probabilite/main.tex}
%\input{./analyse-fonctionnelle/main.tex}
% \input{./Analyse-convexe-et-dualite-en-optimisation/main.tex}
%\input{./tikz/main.tex}
%\input{./Theorie-du-distributions/main.tex}
%\input{./optimisation/mine.tex}
 \input{./modelisation/main.tex}

% yves.aubry@univ-tln.fr : algebra

\end{document}


% yves.aubry@univ-tln.fr : algebra

\end{document}

%% !TEX encoding = UTF-8 Unicode
% !TEX TS-program = xelatex

\documentclass[french]{report}

%\usepackage[utf8]{inputenc}
%\usepackage[T1]{fontenc}
\usepackage{babel}


\newif\ifcomment
%\commenttrue # Show comments

\usepackage{physics}
\usepackage{amssymb}


\usepackage{amsthm}
% \usepackage{thmtools}
\usepackage{mathtools}
\usepackage{amsfonts}

\usepackage{color}

\usepackage{tikz}

\usepackage{geometry}
\geometry{a5paper, margin=0.1in, right=1cm}

\usepackage{dsfont}

\usepackage{graphicx}
\graphicspath{ {images/} }

\usepackage{faktor}

\usepackage{IEEEtrantools}
\usepackage{enumerate}   
\usepackage[PostScript=dvips]{"/Users/aware/Documents/Courses/diagrams"}


\newtheorem{theorem}{Théorème}[section]
\renewcommand{\thetheorem}{\arabic{theorem}}
\newtheorem{lemme}{Lemme}[section]
\renewcommand{\thelemme}{\arabic{lemme}}
\newtheorem{proposition}{Proposition}[section]
\renewcommand{\theproposition}{\arabic{proposition}}
\newtheorem{notations}{Notations}[section]
\newtheorem{problem}{Problème}[section]
\newtheorem{corollary}{Corollaire}[theorem]
\renewcommand{\thecorollary}{\arabic{corollary}}
\newtheorem{property}{Propriété}[section]
\newtheorem{objective}{Objectif}[section]

\theoremstyle{definition}
\newtheorem{definition}{Définition}[section]
\renewcommand{\thedefinition}{\arabic{definition}}
\newtheorem{exercise}{Exercice}[chapter]
\renewcommand{\theexercise}{\arabic{exercise}}
\newtheorem{example}{Exemple}[chapter]
\renewcommand{\theexample}{\arabic{example}}
\newtheorem*{solution}{Solution}
\newtheorem*{application}{Application}
\newtheorem*{notation}{Notation}
\newtheorem*{vocabulary}{Vocabulaire}
\newtheorem*{properties}{Propriétés}



\theoremstyle{remark}
\newtheorem*{remark}{Remarque}
\newtheorem*{rappel}{Rappel}


\usepackage{etoolbox}
\AtBeginEnvironment{exercise}{\small}
\AtBeginEnvironment{example}{\small}

\usepackage{cases}
\usepackage[red]{mypack}

\usepackage[framemethod=TikZ]{mdframed}

\definecolor{bg}{rgb}{0.4,0.25,0.95}
\definecolor{pagebg}{rgb}{0,0,0.5}
\surroundwithmdframed[
   topline=false,
   rightline=false,
   bottomline=false,
   leftmargin=\parindent,
   skipabove=8pt,
   skipbelow=8pt,
   linecolor=blue,
   innerbottommargin=10pt,
   % backgroundcolor=bg,font=\color{orange}\sffamily, fontcolor=white
]{definition}

\usepackage{empheq}
\usepackage[most]{tcolorbox}

\newtcbox{\mymath}[1][]{%
    nobeforeafter, math upper, tcbox raise base,
    enhanced, colframe=blue!30!black,
    colback=red!10, boxrule=1pt,
    #1}

\usepackage{unixode}


\DeclareMathOperator{\ord}{ord}
\DeclareMathOperator{\orb}{orb}
\DeclareMathOperator{\stab}{stab}
\DeclareMathOperator{\Stab}{stab}
\DeclareMathOperator{\ppcm}{ppcm}
\DeclareMathOperator{\conj}{Conj}
\DeclareMathOperator{\End}{End}
\DeclareMathOperator{\rot}{rot}
\DeclareMathOperator{\trs}{trace}
\DeclareMathOperator{\Ind}{Ind}
\DeclareMathOperator{\mat}{Mat}
\DeclareMathOperator{\id}{Id}
\DeclareMathOperator{\vect}{vect}
\DeclareMathOperator{\img}{img}
\DeclareMathOperator{\cov}{Cov}
\DeclareMathOperator{\dist}{dist}
\DeclareMathOperator{\irr}{Irr}
\DeclareMathOperator{\image}{Im}
\DeclareMathOperator{\pd}{\partial}
\DeclareMathOperator{\epi}{epi}
\DeclareMathOperator{\Argmin}{Argmin}
\DeclareMathOperator{\dom}{dom}
\DeclareMathOperator{\proj}{proj}
\DeclareMathOperator{\ctg}{ctg}
\DeclareMathOperator{\supp}{supp}
\DeclareMathOperator{\argmin}{argmin}
\DeclareMathOperator{\mult}{mult}
\DeclareMathOperator{\ch}{ch}
\DeclareMathOperator{\sh}{sh}
\DeclareMathOperator{\rang}{rang}
\DeclareMathOperator{\diam}{diam}
\DeclareMathOperator{\Epigraphe}{Epigraphe}




\usepackage{xcolor}
\everymath{\color{blue}}
%\everymath{\color[rgb]{0,1,1}}
%\pagecolor[rgb]{0,0,0.5}


\newcommand*{\pdtest}[3][]{\ensuremath{\frac{\partial^{#1} #2}{\partial #3}}}

\newcommand*{\deffunc}[6][]{\ensuremath{
\begin{array}{rcl}
#2 : #3 &\rightarrow& #4\\
#5 &\mapsto& #6
\end{array}
}}

\newcommand{\eqcolon}{\mathrel{\resizebox{\widthof{$\mathord{=}$}}{\height}{ $\!\!=\!\!\resizebox{1.2\width}{0.8\height}{\raisebox{0.23ex}{$\mathop{:}$}}\!\!$ }}}
\newcommand{\coloneq}{\mathrel{\resizebox{\widthof{$\mathord{=}$}}{\height}{ $\!\!\resizebox{1.2\width}{0.8\height}{\raisebox{0.23ex}{$\mathop{:}$}}\!\!=\!\!$ }}}
\newcommand{\eqcolonl}{\ensuremath{\mathrel{=\!\!\mathop{:}}}}
\newcommand{\coloneql}{\ensuremath{\mathrel{\mathop{:} \!\! =}}}
\newcommand{\vc}[1]{% inline column vector
  \left(\begin{smallmatrix}#1\end{smallmatrix}\right)%
}
\newcommand{\vr}[1]{% inline row vector
  \begin{smallmatrix}(\,#1\,)\end{smallmatrix}%
}
\makeatletter
\newcommand*{\defeq}{\ =\mathrel{\rlap{%
                     \raisebox{0.3ex}{$\m@th\cdot$}}%
                     \raisebox{-0.3ex}{$\m@th\cdot$}}%
                     }
\makeatother

\newcommand{\mathcircle}[1]{% inline row vector
 \overset{\circ}{#1}
}
\newcommand{\ulim}{% low limit
 \underline{\lim}
}
\newcommand{\ssi}{% iff
\iff
}
\newcommand{\ps}[2]{
\expval{#1 | #2}
}
\newcommand{\df}[1]{
\mqty{#1}
}
\newcommand{\n}[1]{
\norm{#1}
}
\newcommand{\sys}[1]{
\left\{\smqty{#1}\right.
}


\newcommand{\eqdef}{\ensuremath{\overset{\text{def}}=}}


\def\Circlearrowright{\ensuremath{%
  \rotatebox[origin=c]{230}{$\circlearrowright$}}}

\newcommand\ct[1]{\text{\rmfamily\upshape #1}}
\newcommand\question[1]{ {\color{red} ...!? \small #1}}
\newcommand\caz[1]{\left\{\begin{array} #1 \end{array}\right.}
\newcommand\const{\text{\rmfamily\upshape const}}
\newcommand\toP{ \overset{\pro}{\to}}
\newcommand\toPP{ \overset{\text{PP}}{\to}}
\newcommand{\oeq}{\mathrel{\text{\textcircled{$=$}}}}





\usepackage{xcolor}
% \usepackage[normalem]{ulem}
\usepackage{lipsum}
\makeatletter
% \newcommand\colorwave[1][blue]{\bgroup \markoverwith{\lower3.5\p@\hbox{\sixly \textcolor{#1}{\char58}}}\ULon}
%\font\sixly=lasy6 % does not re-load if already loaded, so no memory problem.

\newmdtheoremenv[
linewidth= 1pt,linecolor= blue,%
leftmargin=20,rightmargin=20,innertopmargin=0pt, innerrightmargin=40,%
tikzsetting = { draw=lightgray, line width = 0.3pt,dashed,%
dash pattern = on 15pt off 3pt},%
splittopskip=\topskip,skipbelow=\baselineskip,%
skipabove=\baselineskip,ntheorem,roundcorner=0pt,
% backgroundcolor=pagebg,font=\color{orange}\sffamily, fontcolor=white
]{examplebox}{Exemple}[section]



\newcommand\R{\mathbb{R}}
\newcommand\Z{\mathbb{Z}}
\newcommand\N{\mathbb{N}}
\newcommand\E{\mathbb{E}}
\newcommand\F{\mathcal{F}}
\newcommand\cH{\mathcal{H}}
\newcommand\V{\mathbb{V}}
\newcommand\dmo{ ^{-1} }
\newcommand\kapa{\kappa}
\newcommand\im{Im}
\newcommand\hs{\mathcal{H}}





\usepackage{soul}

\makeatletter
\newcommand*{\whiten}[1]{\llap{\textcolor{white}{{\the\SOUL@token}}\hspace{#1pt}}}
\DeclareRobustCommand*\myul{%
    \def\SOUL@everyspace{\underline{\space}\kern\z@}%
    \def\SOUL@everytoken{%
     \setbox0=\hbox{\the\SOUL@token}%
     \ifdim\dp0>\z@
        \raisebox{\dp0}{\underline{\phantom{\the\SOUL@token}}}%
        \whiten{1}\whiten{0}%
        \whiten{-1}\whiten{-2}%
        \llap{\the\SOUL@token}%
     \else
        \underline{\the\SOUL@token}%
     \fi}%
\SOUL@}
\makeatother

\newcommand*{\demp}{\fontfamily{lmtt}\selectfont}

\DeclareTextFontCommand{\textdemp}{\demp}

\begin{document}

\ifcomment
Multiline
comment
\fi
\ifcomment
\myul{Typesetting test}
% \color[rgb]{1,1,1}
$∑_i^n≠ 60º±∞π∆¬≈√j∫h≤≥µ$

$\CR \R\pro\ind\pro\gS\pro
\mqty[a&b\\c&d]$
$\pro\mathbb{P}$
$\dd{x}$

  \[
    \alpha(x)=\left\{
                \begin{array}{ll}
                  x\\
                  \frac{1}{1+e^{-kx}}\\
                  \frac{e^x-e^{-x}}{e^x+e^{-x}}
                \end{array}
              \right.
  \]

  $\expval{x}$
  
  $\chi_\rho(ghg\dmo)=\Tr(\rho_{ghg\dmo})=\Tr(\rho_g\circ\rho_h\circ\rho\dmo_g)=\Tr(\rho_h)\overset{\mbox{\scalebox{0.5}{$\Tr(AB)=\Tr(BA)$}}}{=}\chi_\rho(h)$
  	$\mathop{\oplus}_{\substack{x\in X}}$

$\mat(\rho_g)=(a_{ij}(g))_{\scriptsize \substack{1\leq i\leq d \\ 1\leq j\leq d}}$ et $\mat(\rho'_g)=(a'_{ij}(g))_{\scriptsize \substack{1\leq i'\leq d' \\ 1\leq j'\leq d'}}$



\[\int_a^b{\mathbb{R}^2}g(u, v)\dd{P_{XY}}(u, v)=\iint g(u,v) f_{XY}(u, v)\dd \lambda(u) \dd \lambda(v)\]
$$\lim_{x\to\infty} f(x)$$	
$$\iiiint_V \mu(t,u,v,w) \,dt\,du\,dv\,dw$$
$$\sum_{n=1}^{\infty} 2^{-n} = 1$$	
\begin{definition}
	Si $X$ et $Y$ sont 2 v.a. ou definit la \textsc{Covariance} entre $X$ et $Y$ comme
	$\cov(X,Y)\overset{\text{def}}{=}\E\left[(X-\E(X))(Y-\E(Y))\right]=\E(XY)-\E(X)\E(Y)$.
\end{definition}
\fi
\pagebreak

% \tableofcontents

% insert your code here
%% !TEX encoding = UTF-8 Unicode
% !TEX TS-program = xelatex

\documentclass[french]{report}

%\usepackage[utf8]{inputenc}
%\usepackage[T1]{fontenc}
\usepackage{babel}


\newif\ifcomment
%\commenttrue # Show comments

\usepackage{physics}
\usepackage{amssymb}


\usepackage{amsthm}
% \usepackage{thmtools}
\usepackage{mathtools}
\usepackage{amsfonts}

\usepackage{color}

\usepackage{tikz}

\usepackage{geometry}
\geometry{a5paper, margin=0.1in, right=1cm}

\usepackage{dsfont}

\usepackage{graphicx}
\graphicspath{ {images/} }

\usepackage{faktor}

\usepackage{IEEEtrantools}
\usepackage{enumerate}   
\usepackage[PostScript=dvips]{"/Users/aware/Documents/Courses/diagrams"}


\newtheorem{theorem}{Théorème}[section]
\renewcommand{\thetheorem}{\arabic{theorem}}
\newtheorem{lemme}{Lemme}[section]
\renewcommand{\thelemme}{\arabic{lemme}}
\newtheorem{proposition}{Proposition}[section]
\renewcommand{\theproposition}{\arabic{proposition}}
\newtheorem{notations}{Notations}[section]
\newtheorem{problem}{Problème}[section]
\newtheorem{corollary}{Corollaire}[theorem]
\renewcommand{\thecorollary}{\arabic{corollary}}
\newtheorem{property}{Propriété}[section]
\newtheorem{objective}{Objectif}[section]

\theoremstyle{definition}
\newtheorem{definition}{Définition}[section]
\renewcommand{\thedefinition}{\arabic{definition}}
\newtheorem{exercise}{Exercice}[chapter]
\renewcommand{\theexercise}{\arabic{exercise}}
\newtheorem{example}{Exemple}[chapter]
\renewcommand{\theexample}{\arabic{example}}
\newtheorem*{solution}{Solution}
\newtheorem*{application}{Application}
\newtheorem*{notation}{Notation}
\newtheorem*{vocabulary}{Vocabulaire}
\newtheorem*{properties}{Propriétés}



\theoremstyle{remark}
\newtheorem*{remark}{Remarque}
\newtheorem*{rappel}{Rappel}


\usepackage{etoolbox}
\AtBeginEnvironment{exercise}{\small}
\AtBeginEnvironment{example}{\small}

\usepackage{cases}
\usepackage[red]{mypack}

\usepackage[framemethod=TikZ]{mdframed}

\definecolor{bg}{rgb}{0.4,0.25,0.95}
\definecolor{pagebg}{rgb}{0,0,0.5}
\surroundwithmdframed[
   topline=false,
   rightline=false,
   bottomline=false,
   leftmargin=\parindent,
   skipabove=8pt,
   skipbelow=8pt,
   linecolor=blue,
   innerbottommargin=10pt,
   % backgroundcolor=bg,font=\color{orange}\sffamily, fontcolor=white
]{definition}

\usepackage{empheq}
\usepackage[most]{tcolorbox}

\newtcbox{\mymath}[1][]{%
    nobeforeafter, math upper, tcbox raise base,
    enhanced, colframe=blue!30!black,
    colback=red!10, boxrule=1pt,
    #1}

\usepackage{unixode}


\DeclareMathOperator{\ord}{ord}
\DeclareMathOperator{\orb}{orb}
\DeclareMathOperator{\stab}{stab}
\DeclareMathOperator{\Stab}{stab}
\DeclareMathOperator{\ppcm}{ppcm}
\DeclareMathOperator{\conj}{Conj}
\DeclareMathOperator{\End}{End}
\DeclareMathOperator{\rot}{rot}
\DeclareMathOperator{\trs}{trace}
\DeclareMathOperator{\Ind}{Ind}
\DeclareMathOperator{\mat}{Mat}
\DeclareMathOperator{\id}{Id}
\DeclareMathOperator{\vect}{vect}
\DeclareMathOperator{\img}{img}
\DeclareMathOperator{\cov}{Cov}
\DeclareMathOperator{\dist}{dist}
\DeclareMathOperator{\irr}{Irr}
\DeclareMathOperator{\image}{Im}
\DeclareMathOperator{\pd}{\partial}
\DeclareMathOperator{\epi}{epi}
\DeclareMathOperator{\Argmin}{Argmin}
\DeclareMathOperator{\dom}{dom}
\DeclareMathOperator{\proj}{proj}
\DeclareMathOperator{\ctg}{ctg}
\DeclareMathOperator{\supp}{supp}
\DeclareMathOperator{\argmin}{argmin}
\DeclareMathOperator{\mult}{mult}
\DeclareMathOperator{\ch}{ch}
\DeclareMathOperator{\sh}{sh}
\DeclareMathOperator{\rang}{rang}
\DeclareMathOperator{\diam}{diam}
\DeclareMathOperator{\Epigraphe}{Epigraphe}




\usepackage{xcolor}
\everymath{\color{blue}}
%\everymath{\color[rgb]{0,1,1}}
%\pagecolor[rgb]{0,0,0.5}


\newcommand*{\pdtest}[3][]{\ensuremath{\frac{\partial^{#1} #2}{\partial #3}}}

\newcommand*{\deffunc}[6][]{\ensuremath{
\begin{array}{rcl}
#2 : #3 &\rightarrow& #4\\
#5 &\mapsto& #6
\end{array}
}}

\newcommand{\eqcolon}{\mathrel{\resizebox{\widthof{$\mathord{=}$}}{\height}{ $\!\!=\!\!\resizebox{1.2\width}{0.8\height}{\raisebox{0.23ex}{$\mathop{:}$}}\!\!$ }}}
\newcommand{\coloneq}{\mathrel{\resizebox{\widthof{$\mathord{=}$}}{\height}{ $\!\!\resizebox{1.2\width}{0.8\height}{\raisebox{0.23ex}{$\mathop{:}$}}\!\!=\!\!$ }}}
\newcommand{\eqcolonl}{\ensuremath{\mathrel{=\!\!\mathop{:}}}}
\newcommand{\coloneql}{\ensuremath{\mathrel{\mathop{:} \!\! =}}}
\newcommand{\vc}[1]{% inline column vector
  \left(\begin{smallmatrix}#1\end{smallmatrix}\right)%
}
\newcommand{\vr}[1]{% inline row vector
  \begin{smallmatrix}(\,#1\,)\end{smallmatrix}%
}
\makeatletter
\newcommand*{\defeq}{\ =\mathrel{\rlap{%
                     \raisebox{0.3ex}{$\m@th\cdot$}}%
                     \raisebox{-0.3ex}{$\m@th\cdot$}}%
                     }
\makeatother

\newcommand{\mathcircle}[1]{% inline row vector
 \overset{\circ}{#1}
}
\newcommand{\ulim}{% low limit
 \underline{\lim}
}
\newcommand{\ssi}{% iff
\iff
}
\newcommand{\ps}[2]{
\expval{#1 | #2}
}
\newcommand{\df}[1]{
\mqty{#1}
}
\newcommand{\n}[1]{
\norm{#1}
}
\newcommand{\sys}[1]{
\left\{\smqty{#1}\right.
}


\newcommand{\eqdef}{\ensuremath{\overset{\text{def}}=}}


\def\Circlearrowright{\ensuremath{%
  \rotatebox[origin=c]{230}{$\circlearrowright$}}}

\newcommand\ct[1]{\text{\rmfamily\upshape #1}}
\newcommand\question[1]{ {\color{red} ...!? \small #1}}
\newcommand\caz[1]{\left\{\begin{array} #1 \end{array}\right.}
\newcommand\const{\text{\rmfamily\upshape const}}
\newcommand\toP{ \overset{\pro}{\to}}
\newcommand\toPP{ \overset{\text{PP}}{\to}}
\newcommand{\oeq}{\mathrel{\text{\textcircled{$=$}}}}





\usepackage{xcolor}
% \usepackage[normalem]{ulem}
\usepackage{lipsum}
\makeatletter
% \newcommand\colorwave[1][blue]{\bgroup \markoverwith{\lower3.5\p@\hbox{\sixly \textcolor{#1}{\char58}}}\ULon}
%\font\sixly=lasy6 % does not re-load if already loaded, so no memory problem.

\newmdtheoremenv[
linewidth= 1pt,linecolor= blue,%
leftmargin=20,rightmargin=20,innertopmargin=0pt, innerrightmargin=40,%
tikzsetting = { draw=lightgray, line width = 0.3pt,dashed,%
dash pattern = on 15pt off 3pt},%
splittopskip=\topskip,skipbelow=\baselineskip,%
skipabove=\baselineskip,ntheorem,roundcorner=0pt,
% backgroundcolor=pagebg,font=\color{orange}\sffamily, fontcolor=white
]{examplebox}{Exemple}[section]



\newcommand\R{\mathbb{R}}
\newcommand\Z{\mathbb{Z}}
\newcommand\N{\mathbb{N}}
\newcommand\E{\mathbb{E}}
\newcommand\F{\mathcal{F}}
\newcommand\cH{\mathcal{H}}
\newcommand\V{\mathbb{V}}
\newcommand\dmo{ ^{-1} }
\newcommand\kapa{\kappa}
\newcommand\im{Im}
\newcommand\hs{\mathcal{H}}





\usepackage{soul}

\makeatletter
\newcommand*{\whiten}[1]{\llap{\textcolor{white}{{\the\SOUL@token}}\hspace{#1pt}}}
\DeclareRobustCommand*\myul{%
    \def\SOUL@everyspace{\underline{\space}\kern\z@}%
    \def\SOUL@everytoken{%
     \setbox0=\hbox{\the\SOUL@token}%
     \ifdim\dp0>\z@
        \raisebox{\dp0}{\underline{\phantom{\the\SOUL@token}}}%
        \whiten{1}\whiten{0}%
        \whiten{-1}\whiten{-2}%
        \llap{\the\SOUL@token}%
     \else
        \underline{\the\SOUL@token}%
     \fi}%
\SOUL@}
\makeatother

\newcommand*{\demp}{\fontfamily{lmtt}\selectfont}

\DeclareTextFontCommand{\textdemp}{\demp}

\begin{document}

\ifcomment
Multiline
comment
\fi
\ifcomment
\myul{Typesetting test}
% \color[rgb]{1,1,1}
$∑_i^n≠ 60º±∞π∆¬≈√j∫h≤≥µ$

$\CR \R\pro\ind\pro\gS\pro
\mqty[a&b\\c&d]$
$\pro\mathbb{P}$
$\dd{x}$

  \[
    \alpha(x)=\left\{
                \begin{array}{ll}
                  x\\
                  \frac{1}{1+e^{-kx}}\\
                  \frac{e^x-e^{-x}}{e^x+e^{-x}}
                \end{array}
              \right.
  \]

  $\expval{x}$
  
  $\chi_\rho(ghg\dmo)=\Tr(\rho_{ghg\dmo})=\Tr(\rho_g\circ\rho_h\circ\rho\dmo_g)=\Tr(\rho_h)\overset{\mbox{\scalebox{0.5}{$\Tr(AB)=\Tr(BA)$}}}{=}\chi_\rho(h)$
  	$\mathop{\oplus}_{\substack{x\in X}}$

$\mat(\rho_g)=(a_{ij}(g))_{\scriptsize \substack{1\leq i\leq d \\ 1\leq j\leq d}}$ et $\mat(\rho'_g)=(a'_{ij}(g))_{\scriptsize \substack{1\leq i'\leq d' \\ 1\leq j'\leq d'}}$



\[\int_a^b{\mathbb{R}^2}g(u, v)\dd{P_{XY}}(u, v)=\iint g(u,v) f_{XY}(u, v)\dd \lambda(u) \dd \lambda(v)\]
$$\lim_{x\to\infty} f(x)$$	
$$\iiiint_V \mu(t,u,v,w) \,dt\,du\,dv\,dw$$
$$\sum_{n=1}^{\infty} 2^{-n} = 1$$	
\begin{definition}
	Si $X$ et $Y$ sont 2 v.a. ou definit la \textsc{Covariance} entre $X$ et $Y$ comme
	$\cov(X,Y)\overset{\text{def}}{=}\E\left[(X-\E(X))(Y-\E(Y))\right]=\E(XY)-\E(X)\E(Y)$.
\end{definition}
\fi
\pagebreak

% \tableofcontents

% insert your code here
%\input{./algebra/main.tex}
%\input{./geometrie-differentielle/main.tex}
%\input{./probabilite/main.tex}
%\input{./analyse-fonctionnelle/main.tex}
% \input{./Analyse-convexe-et-dualite-en-optimisation/main.tex}
%\input{./tikz/main.tex}
%\input{./Theorie-du-distributions/main.tex}
%\input{./optimisation/mine.tex}
 \input{./modelisation/main.tex}

% yves.aubry@univ-tln.fr : algebra

\end{document}

%% !TEX encoding = UTF-8 Unicode
% !TEX TS-program = xelatex

\documentclass[french]{report}

%\usepackage[utf8]{inputenc}
%\usepackage[T1]{fontenc}
\usepackage{babel}


\newif\ifcomment
%\commenttrue # Show comments

\usepackage{physics}
\usepackage{amssymb}


\usepackage{amsthm}
% \usepackage{thmtools}
\usepackage{mathtools}
\usepackage{amsfonts}

\usepackage{color}

\usepackage{tikz}

\usepackage{geometry}
\geometry{a5paper, margin=0.1in, right=1cm}

\usepackage{dsfont}

\usepackage{graphicx}
\graphicspath{ {images/} }

\usepackage{faktor}

\usepackage{IEEEtrantools}
\usepackage{enumerate}   
\usepackage[PostScript=dvips]{"/Users/aware/Documents/Courses/diagrams"}


\newtheorem{theorem}{Théorème}[section]
\renewcommand{\thetheorem}{\arabic{theorem}}
\newtheorem{lemme}{Lemme}[section]
\renewcommand{\thelemme}{\arabic{lemme}}
\newtheorem{proposition}{Proposition}[section]
\renewcommand{\theproposition}{\arabic{proposition}}
\newtheorem{notations}{Notations}[section]
\newtheorem{problem}{Problème}[section]
\newtheorem{corollary}{Corollaire}[theorem]
\renewcommand{\thecorollary}{\arabic{corollary}}
\newtheorem{property}{Propriété}[section]
\newtheorem{objective}{Objectif}[section]

\theoremstyle{definition}
\newtheorem{definition}{Définition}[section]
\renewcommand{\thedefinition}{\arabic{definition}}
\newtheorem{exercise}{Exercice}[chapter]
\renewcommand{\theexercise}{\arabic{exercise}}
\newtheorem{example}{Exemple}[chapter]
\renewcommand{\theexample}{\arabic{example}}
\newtheorem*{solution}{Solution}
\newtheorem*{application}{Application}
\newtheorem*{notation}{Notation}
\newtheorem*{vocabulary}{Vocabulaire}
\newtheorem*{properties}{Propriétés}



\theoremstyle{remark}
\newtheorem*{remark}{Remarque}
\newtheorem*{rappel}{Rappel}


\usepackage{etoolbox}
\AtBeginEnvironment{exercise}{\small}
\AtBeginEnvironment{example}{\small}

\usepackage{cases}
\usepackage[red]{mypack}

\usepackage[framemethod=TikZ]{mdframed}

\definecolor{bg}{rgb}{0.4,0.25,0.95}
\definecolor{pagebg}{rgb}{0,0,0.5}
\surroundwithmdframed[
   topline=false,
   rightline=false,
   bottomline=false,
   leftmargin=\parindent,
   skipabove=8pt,
   skipbelow=8pt,
   linecolor=blue,
   innerbottommargin=10pt,
   % backgroundcolor=bg,font=\color{orange}\sffamily, fontcolor=white
]{definition}

\usepackage{empheq}
\usepackage[most]{tcolorbox}

\newtcbox{\mymath}[1][]{%
    nobeforeafter, math upper, tcbox raise base,
    enhanced, colframe=blue!30!black,
    colback=red!10, boxrule=1pt,
    #1}

\usepackage{unixode}


\DeclareMathOperator{\ord}{ord}
\DeclareMathOperator{\orb}{orb}
\DeclareMathOperator{\stab}{stab}
\DeclareMathOperator{\Stab}{stab}
\DeclareMathOperator{\ppcm}{ppcm}
\DeclareMathOperator{\conj}{Conj}
\DeclareMathOperator{\End}{End}
\DeclareMathOperator{\rot}{rot}
\DeclareMathOperator{\trs}{trace}
\DeclareMathOperator{\Ind}{Ind}
\DeclareMathOperator{\mat}{Mat}
\DeclareMathOperator{\id}{Id}
\DeclareMathOperator{\vect}{vect}
\DeclareMathOperator{\img}{img}
\DeclareMathOperator{\cov}{Cov}
\DeclareMathOperator{\dist}{dist}
\DeclareMathOperator{\irr}{Irr}
\DeclareMathOperator{\image}{Im}
\DeclareMathOperator{\pd}{\partial}
\DeclareMathOperator{\epi}{epi}
\DeclareMathOperator{\Argmin}{Argmin}
\DeclareMathOperator{\dom}{dom}
\DeclareMathOperator{\proj}{proj}
\DeclareMathOperator{\ctg}{ctg}
\DeclareMathOperator{\supp}{supp}
\DeclareMathOperator{\argmin}{argmin}
\DeclareMathOperator{\mult}{mult}
\DeclareMathOperator{\ch}{ch}
\DeclareMathOperator{\sh}{sh}
\DeclareMathOperator{\rang}{rang}
\DeclareMathOperator{\diam}{diam}
\DeclareMathOperator{\Epigraphe}{Epigraphe}




\usepackage{xcolor}
\everymath{\color{blue}}
%\everymath{\color[rgb]{0,1,1}}
%\pagecolor[rgb]{0,0,0.5}


\newcommand*{\pdtest}[3][]{\ensuremath{\frac{\partial^{#1} #2}{\partial #3}}}

\newcommand*{\deffunc}[6][]{\ensuremath{
\begin{array}{rcl}
#2 : #3 &\rightarrow& #4\\
#5 &\mapsto& #6
\end{array}
}}

\newcommand{\eqcolon}{\mathrel{\resizebox{\widthof{$\mathord{=}$}}{\height}{ $\!\!=\!\!\resizebox{1.2\width}{0.8\height}{\raisebox{0.23ex}{$\mathop{:}$}}\!\!$ }}}
\newcommand{\coloneq}{\mathrel{\resizebox{\widthof{$\mathord{=}$}}{\height}{ $\!\!\resizebox{1.2\width}{0.8\height}{\raisebox{0.23ex}{$\mathop{:}$}}\!\!=\!\!$ }}}
\newcommand{\eqcolonl}{\ensuremath{\mathrel{=\!\!\mathop{:}}}}
\newcommand{\coloneql}{\ensuremath{\mathrel{\mathop{:} \!\! =}}}
\newcommand{\vc}[1]{% inline column vector
  \left(\begin{smallmatrix}#1\end{smallmatrix}\right)%
}
\newcommand{\vr}[1]{% inline row vector
  \begin{smallmatrix}(\,#1\,)\end{smallmatrix}%
}
\makeatletter
\newcommand*{\defeq}{\ =\mathrel{\rlap{%
                     \raisebox{0.3ex}{$\m@th\cdot$}}%
                     \raisebox{-0.3ex}{$\m@th\cdot$}}%
                     }
\makeatother

\newcommand{\mathcircle}[1]{% inline row vector
 \overset{\circ}{#1}
}
\newcommand{\ulim}{% low limit
 \underline{\lim}
}
\newcommand{\ssi}{% iff
\iff
}
\newcommand{\ps}[2]{
\expval{#1 | #2}
}
\newcommand{\df}[1]{
\mqty{#1}
}
\newcommand{\n}[1]{
\norm{#1}
}
\newcommand{\sys}[1]{
\left\{\smqty{#1}\right.
}


\newcommand{\eqdef}{\ensuremath{\overset{\text{def}}=}}


\def\Circlearrowright{\ensuremath{%
  \rotatebox[origin=c]{230}{$\circlearrowright$}}}

\newcommand\ct[1]{\text{\rmfamily\upshape #1}}
\newcommand\question[1]{ {\color{red} ...!? \small #1}}
\newcommand\caz[1]{\left\{\begin{array} #1 \end{array}\right.}
\newcommand\const{\text{\rmfamily\upshape const}}
\newcommand\toP{ \overset{\pro}{\to}}
\newcommand\toPP{ \overset{\text{PP}}{\to}}
\newcommand{\oeq}{\mathrel{\text{\textcircled{$=$}}}}





\usepackage{xcolor}
% \usepackage[normalem]{ulem}
\usepackage{lipsum}
\makeatletter
% \newcommand\colorwave[1][blue]{\bgroup \markoverwith{\lower3.5\p@\hbox{\sixly \textcolor{#1}{\char58}}}\ULon}
%\font\sixly=lasy6 % does not re-load if already loaded, so no memory problem.

\newmdtheoremenv[
linewidth= 1pt,linecolor= blue,%
leftmargin=20,rightmargin=20,innertopmargin=0pt, innerrightmargin=40,%
tikzsetting = { draw=lightgray, line width = 0.3pt,dashed,%
dash pattern = on 15pt off 3pt},%
splittopskip=\topskip,skipbelow=\baselineskip,%
skipabove=\baselineskip,ntheorem,roundcorner=0pt,
% backgroundcolor=pagebg,font=\color{orange}\sffamily, fontcolor=white
]{examplebox}{Exemple}[section]



\newcommand\R{\mathbb{R}}
\newcommand\Z{\mathbb{Z}}
\newcommand\N{\mathbb{N}}
\newcommand\E{\mathbb{E}}
\newcommand\F{\mathcal{F}}
\newcommand\cH{\mathcal{H}}
\newcommand\V{\mathbb{V}}
\newcommand\dmo{ ^{-1} }
\newcommand\kapa{\kappa}
\newcommand\im{Im}
\newcommand\hs{\mathcal{H}}





\usepackage{soul}

\makeatletter
\newcommand*{\whiten}[1]{\llap{\textcolor{white}{{\the\SOUL@token}}\hspace{#1pt}}}
\DeclareRobustCommand*\myul{%
    \def\SOUL@everyspace{\underline{\space}\kern\z@}%
    \def\SOUL@everytoken{%
     \setbox0=\hbox{\the\SOUL@token}%
     \ifdim\dp0>\z@
        \raisebox{\dp0}{\underline{\phantom{\the\SOUL@token}}}%
        \whiten{1}\whiten{0}%
        \whiten{-1}\whiten{-2}%
        \llap{\the\SOUL@token}%
     \else
        \underline{\the\SOUL@token}%
     \fi}%
\SOUL@}
\makeatother

\newcommand*{\demp}{\fontfamily{lmtt}\selectfont}

\DeclareTextFontCommand{\textdemp}{\demp}

\begin{document}

\ifcomment
Multiline
comment
\fi
\ifcomment
\myul{Typesetting test}
% \color[rgb]{1,1,1}
$∑_i^n≠ 60º±∞π∆¬≈√j∫h≤≥µ$

$\CR \R\pro\ind\pro\gS\pro
\mqty[a&b\\c&d]$
$\pro\mathbb{P}$
$\dd{x}$

  \[
    \alpha(x)=\left\{
                \begin{array}{ll}
                  x\\
                  \frac{1}{1+e^{-kx}}\\
                  \frac{e^x-e^{-x}}{e^x+e^{-x}}
                \end{array}
              \right.
  \]

  $\expval{x}$
  
  $\chi_\rho(ghg\dmo)=\Tr(\rho_{ghg\dmo})=\Tr(\rho_g\circ\rho_h\circ\rho\dmo_g)=\Tr(\rho_h)\overset{\mbox{\scalebox{0.5}{$\Tr(AB)=\Tr(BA)$}}}{=}\chi_\rho(h)$
  	$\mathop{\oplus}_{\substack{x\in X}}$

$\mat(\rho_g)=(a_{ij}(g))_{\scriptsize \substack{1\leq i\leq d \\ 1\leq j\leq d}}$ et $\mat(\rho'_g)=(a'_{ij}(g))_{\scriptsize \substack{1\leq i'\leq d' \\ 1\leq j'\leq d'}}$



\[\int_a^b{\mathbb{R}^2}g(u, v)\dd{P_{XY}}(u, v)=\iint g(u,v) f_{XY}(u, v)\dd \lambda(u) \dd \lambda(v)\]
$$\lim_{x\to\infty} f(x)$$	
$$\iiiint_V \mu(t,u,v,w) \,dt\,du\,dv\,dw$$
$$\sum_{n=1}^{\infty} 2^{-n} = 1$$	
\begin{definition}
	Si $X$ et $Y$ sont 2 v.a. ou definit la \textsc{Covariance} entre $X$ et $Y$ comme
	$\cov(X,Y)\overset{\text{def}}{=}\E\left[(X-\E(X))(Y-\E(Y))\right]=\E(XY)-\E(X)\E(Y)$.
\end{definition}
\fi
\pagebreak

% \tableofcontents

% insert your code here
%\input{./algebra/main.tex}
%\input{./geometrie-differentielle/main.tex}
%\input{./probabilite/main.tex}
%\input{./analyse-fonctionnelle/main.tex}
% \input{./Analyse-convexe-et-dualite-en-optimisation/main.tex}
%\input{./tikz/main.tex}
%\input{./Theorie-du-distributions/main.tex}
%\input{./optimisation/mine.tex}
 \input{./modelisation/main.tex}

% yves.aubry@univ-tln.fr : algebra

\end{document}

%% !TEX encoding = UTF-8 Unicode
% !TEX TS-program = xelatex

\documentclass[french]{report}

%\usepackage[utf8]{inputenc}
%\usepackage[T1]{fontenc}
\usepackage{babel}


\newif\ifcomment
%\commenttrue # Show comments

\usepackage{physics}
\usepackage{amssymb}


\usepackage{amsthm}
% \usepackage{thmtools}
\usepackage{mathtools}
\usepackage{amsfonts}

\usepackage{color}

\usepackage{tikz}

\usepackage{geometry}
\geometry{a5paper, margin=0.1in, right=1cm}

\usepackage{dsfont}

\usepackage{graphicx}
\graphicspath{ {images/} }

\usepackage{faktor}

\usepackage{IEEEtrantools}
\usepackage{enumerate}   
\usepackage[PostScript=dvips]{"/Users/aware/Documents/Courses/diagrams"}


\newtheorem{theorem}{Théorème}[section]
\renewcommand{\thetheorem}{\arabic{theorem}}
\newtheorem{lemme}{Lemme}[section]
\renewcommand{\thelemme}{\arabic{lemme}}
\newtheorem{proposition}{Proposition}[section]
\renewcommand{\theproposition}{\arabic{proposition}}
\newtheorem{notations}{Notations}[section]
\newtheorem{problem}{Problème}[section]
\newtheorem{corollary}{Corollaire}[theorem]
\renewcommand{\thecorollary}{\arabic{corollary}}
\newtheorem{property}{Propriété}[section]
\newtheorem{objective}{Objectif}[section]

\theoremstyle{definition}
\newtheorem{definition}{Définition}[section]
\renewcommand{\thedefinition}{\arabic{definition}}
\newtheorem{exercise}{Exercice}[chapter]
\renewcommand{\theexercise}{\arabic{exercise}}
\newtheorem{example}{Exemple}[chapter]
\renewcommand{\theexample}{\arabic{example}}
\newtheorem*{solution}{Solution}
\newtheorem*{application}{Application}
\newtheorem*{notation}{Notation}
\newtheorem*{vocabulary}{Vocabulaire}
\newtheorem*{properties}{Propriétés}



\theoremstyle{remark}
\newtheorem*{remark}{Remarque}
\newtheorem*{rappel}{Rappel}


\usepackage{etoolbox}
\AtBeginEnvironment{exercise}{\small}
\AtBeginEnvironment{example}{\small}

\usepackage{cases}
\usepackage[red]{mypack}

\usepackage[framemethod=TikZ]{mdframed}

\definecolor{bg}{rgb}{0.4,0.25,0.95}
\definecolor{pagebg}{rgb}{0,0,0.5}
\surroundwithmdframed[
   topline=false,
   rightline=false,
   bottomline=false,
   leftmargin=\parindent,
   skipabove=8pt,
   skipbelow=8pt,
   linecolor=blue,
   innerbottommargin=10pt,
   % backgroundcolor=bg,font=\color{orange}\sffamily, fontcolor=white
]{definition}

\usepackage{empheq}
\usepackage[most]{tcolorbox}

\newtcbox{\mymath}[1][]{%
    nobeforeafter, math upper, tcbox raise base,
    enhanced, colframe=blue!30!black,
    colback=red!10, boxrule=1pt,
    #1}

\usepackage{unixode}


\DeclareMathOperator{\ord}{ord}
\DeclareMathOperator{\orb}{orb}
\DeclareMathOperator{\stab}{stab}
\DeclareMathOperator{\Stab}{stab}
\DeclareMathOperator{\ppcm}{ppcm}
\DeclareMathOperator{\conj}{Conj}
\DeclareMathOperator{\End}{End}
\DeclareMathOperator{\rot}{rot}
\DeclareMathOperator{\trs}{trace}
\DeclareMathOperator{\Ind}{Ind}
\DeclareMathOperator{\mat}{Mat}
\DeclareMathOperator{\id}{Id}
\DeclareMathOperator{\vect}{vect}
\DeclareMathOperator{\img}{img}
\DeclareMathOperator{\cov}{Cov}
\DeclareMathOperator{\dist}{dist}
\DeclareMathOperator{\irr}{Irr}
\DeclareMathOperator{\image}{Im}
\DeclareMathOperator{\pd}{\partial}
\DeclareMathOperator{\epi}{epi}
\DeclareMathOperator{\Argmin}{Argmin}
\DeclareMathOperator{\dom}{dom}
\DeclareMathOperator{\proj}{proj}
\DeclareMathOperator{\ctg}{ctg}
\DeclareMathOperator{\supp}{supp}
\DeclareMathOperator{\argmin}{argmin}
\DeclareMathOperator{\mult}{mult}
\DeclareMathOperator{\ch}{ch}
\DeclareMathOperator{\sh}{sh}
\DeclareMathOperator{\rang}{rang}
\DeclareMathOperator{\diam}{diam}
\DeclareMathOperator{\Epigraphe}{Epigraphe}




\usepackage{xcolor}
\everymath{\color{blue}}
%\everymath{\color[rgb]{0,1,1}}
%\pagecolor[rgb]{0,0,0.5}


\newcommand*{\pdtest}[3][]{\ensuremath{\frac{\partial^{#1} #2}{\partial #3}}}

\newcommand*{\deffunc}[6][]{\ensuremath{
\begin{array}{rcl}
#2 : #3 &\rightarrow& #4\\
#5 &\mapsto& #6
\end{array}
}}

\newcommand{\eqcolon}{\mathrel{\resizebox{\widthof{$\mathord{=}$}}{\height}{ $\!\!=\!\!\resizebox{1.2\width}{0.8\height}{\raisebox{0.23ex}{$\mathop{:}$}}\!\!$ }}}
\newcommand{\coloneq}{\mathrel{\resizebox{\widthof{$\mathord{=}$}}{\height}{ $\!\!\resizebox{1.2\width}{0.8\height}{\raisebox{0.23ex}{$\mathop{:}$}}\!\!=\!\!$ }}}
\newcommand{\eqcolonl}{\ensuremath{\mathrel{=\!\!\mathop{:}}}}
\newcommand{\coloneql}{\ensuremath{\mathrel{\mathop{:} \!\! =}}}
\newcommand{\vc}[1]{% inline column vector
  \left(\begin{smallmatrix}#1\end{smallmatrix}\right)%
}
\newcommand{\vr}[1]{% inline row vector
  \begin{smallmatrix}(\,#1\,)\end{smallmatrix}%
}
\makeatletter
\newcommand*{\defeq}{\ =\mathrel{\rlap{%
                     \raisebox{0.3ex}{$\m@th\cdot$}}%
                     \raisebox{-0.3ex}{$\m@th\cdot$}}%
                     }
\makeatother

\newcommand{\mathcircle}[1]{% inline row vector
 \overset{\circ}{#1}
}
\newcommand{\ulim}{% low limit
 \underline{\lim}
}
\newcommand{\ssi}{% iff
\iff
}
\newcommand{\ps}[2]{
\expval{#1 | #2}
}
\newcommand{\df}[1]{
\mqty{#1}
}
\newcommand{\n}[1]{
\norm{#1}
}
\newcommand{\sys}[1]{
\left\{\smqty{#1}\right.
}


\newcommand{\eqdef}{\ensuremath{\overset{\text{def}}=}}


\def\Circlearrowright{\ensuremath{%
  \rotatebox[origin=c]{230}{$\circlearrowright$}}}

\newcommand\ct[1]{\text{\rmfamily\upshape #1}}
\newcommand\question[1]{ {\color{red} ...!? \small #1}}
\newcommand\caz[1]{\left\{\begin{array} #1 \end{array}\right.}
\newcommand\const{\text{\rmfamily\upshape const}}
\newcommand\toP{ \overset{\pro}{\to}}
\newcommand\toPP{ \overset{\text{PP}}{\to}}
\newcommand{\oeq}{\mathrel{\text{\textcircled{$=$}}}}





\usepackage{xcolor}
% \usepackage[normalem]{ulem}
\usepackage{lipsum}
\makeatletter
% \newcommand\colorwave[1][blue]{\bgroup \markoverwith{\lower3.5\p@\hbox{\sixly \textcolor{#1}{\char58}}}\ULon}
%\font\sixly=lasy6 % does not re-load if already loaded, so no memory problem.

\newmdtheoremenv[
linewidth= 1pt,linecolor= blue,%
leftmargin=20,rightmargin=20,innertopmargin=0pt, innerrightmargin=40,%
tikzsetting = { draw=lightgray, line width = 0.3pt,dashed,%
dash pattern = on 15pt off 3pt},%
splittopskip=\topskip,skipbelow=\baselineskip,%
skipabove=\baselineskip,ntheorem,roundcorner=0pt,
% backgroundcolor=pagebg,font=\color{orange}\sffamily, fontcolor=white
]{examplebox}{Exemple}[section]



\newcommand\R{\mathbb{R}}
\newcommand\Z{\mathbb{Z}}
\newcommand\N{\mathbb{N}}
\newcommand\E{\mathbb{E}}
\newcommand\F{\mathcal{F}}
\newcommand\cH{\mathcal{H}}
\newcommand\V{\mathbb{V}}
\newcommand\dmo{ ^{-1} }
\newcommand\kapa{\kappa}
\newcommand\im{Im}
\newcommand\hs{\mathcal{H}}





\usepackage{soul}

\makeatletter
\newcommand*{\whiten}[1]{\llap{\textcolor{white}{{\the\SOUL@token}}\hspace{#1pt}}}
\DeclareRobustCommand*\myul{%
    \def\SOUL@everyspace{\underline{\space}\kern\z@}%
    \def\SOUL@everytoken{%
     \setbox0=\hbox{\the\SOUL@token}%
     \ifdim\dp0>\z@
        \raisebox{\dp0}{\underline{\phantom{\the\SOUL@token}}}%
        \whiten{1}\whiten{0}%
        \whiten{-1}\whiten{-2}%
        \llap{\the\SOUL@token}%
     \else
        \underline{\the\SOUL@token}%
     \fi}%
\SOUL@}
\makeatother

\newcommand*{\demp}{\fontfamily{lmtt}\selectfont}

\DeclareTextFontCommand{\textdemp}{\demp}

\begin{document}

\ifcomment
Multiline
comment
\fi
\ifcomment
\myul{Typesetting test}
% \color[rgb]{1,1,1}
$∑_i^n≠ 60º±∞π∆¬≈√j∫h≤≥µ$

$\CR \R\pro\ind\pro\gS\pro
\mqty[a&b\\c&d]$
$\pro\mathbb{P}$
$\dd{x}$

  \[
    \alpha(x)=\left\{
                \begin{array}{ll}
                  x\\
                  \frac{1}{1+e^{-kx}}\\
                  \frac{e^x-e^{-x}}{e^x+e^{-x}}
                \end{array}
              \right.
  \]

  $\expval{x}$
  
  $\chi_\rho(ghg\dmo)=\Tr(\rho_{ghg\dmo})=\Tr(\rho_g\circ\rho_h\circ\rho\dmo_g)=\Tr(\rho_h)\overset{\mbox{\scalebox{0.5}{$\Tr(AB)=\Tr(BA)$}}}{=}\chi_\rho(h)$
  	$\mathop{\oplus}_{\substack{x\in X}}$

$\mat(\rho_g)=(a_{ij}(g))_{\scriptsize \substack{1\leq i\leq d \\ 1\leq j\leq d}}$ et $\mat(\rho'_g)=(a'_{ij}(g))_{\scriptsize \substack{1\leq i'\leq d' \\ 1\leq j'\leq d'}}$



\[\int_a^b{\mathbb{R}^2}g(u, v)\dd{P_{XY}}(u, v)=\iint g(u,v) f_{XY}(u, v)\dd \lambda(u) \dd \lambda(v)\]
$$\lim_{x\to\infty} f(x)$$	
$$\iiiint_V \mu(t,u,v,w) \,dt\,du\,dv\,dw$$
$$\sum_{n=1}^{\infty} 2^{-n} = 1$$	
\begin{definition}
	Si $X$ et $Y$ sont 2 v.a. ou definit la \textsc{Covariance} entre $X$ et $Y$ comme
	$\cov(X,Y)\overset{\text{def}}{=}\E\left[(X-\E(X))(Y-\E(Y))\right]=\E(XY)-\E(X)\E(Y)$.
\end{definition}
\fi
\pagebreak

% \tableofcontents

% insert your code here
%\input{./algebra/main.tex}
%\input{./geometrie-differentielle/main.tex}
%\input{./probabilite/main.tex}
%\input{./analyse-fonctionnelle/main.tex}
% \input{./Analyse-convexe-et-dualite-en-optimisation/main.tex}
%\input{./tikz/main.tex}
%\input{./Theorie-du-distributions/main.tex}
%\input{./optimisation/mine.tex}
 \input{./modelisation/main.tex}

% yves.aubry@univ-tln.fr : algebra

\end{document}

%% !TEX encoding = UTF-8 Unicode
% !TEX TS-program = xelatex

\documentclass[french]{report}

%\usepackage[utf8]{inputenc}
%\usepackage[T1]{fontenc}
\usepackage{babel}


\newif\ifcomment
%\commenttrue # Show comments

\usepackage{physics}
\usepackage{amssymb}


\usepackage{amsthm}
% \usepackage{thmtools}
\usepackage{mathtools}
\usepackage{amsfonts}

\usepackage{color}

\usepackage{tikz}

\usepackage{geometry}
\geometry{a5paper, margin=0.1in, right=1cm}

\usepackage{dsfont}

\usepackage{graphicx}
\graphicspath{ {images/} }

\usepackage{faktor}

\usepackage{IEEEtrantools}
\usepackage{enumerate}   
\usepackage[PostScript=dvips]{"/Users/aware/Documents/Courses/diagrams"}


\newtheorem{theorem}{Théorème}[section]
\renewcommand{\thetheorem}{\arabic{theorem}}
\newtheorem{lemme}{Lemme}[section]
\renewcommand{\thelemme}{\arabic{lemme}}
\newtheorem{proposition}{Proposition}[section]
\renewcommand{\theproposition}{\arabic{proposition}}
\newtheorem{notations}{Notations}[section]
\newtheorem{problem}{Problème}[section]
\newtheorem{corollary}{Corollaire}[theorem]
\renewcommand{\thecorollary}{\arabic{corollary}}
\newtheorem{property}{Propriété}[section]
\newtheorem{objective}{Objectif}[section]

\theoremstyle{definition}
\newtheorem{definition}{Définition}[section]
\renewcommand{\thedefinition}{\arabic{definition}}
\newtheorem{exercise}{Exercice}[chapter]
\renewcommand{\theexercise}{\arabic{exercise}}
\newtheorem{example}{Exemple}[chapter]
\renewcommand{\theexample}{\arabic{example}}
\newtheorem*{solution}{Solution}
\newtheorem*{application}{Application}
\newtheorem*{notation}{Notation}
\newtheorem*{vocabulary}{Vocabulaire}
\newtheorem*{properties}{Propriétés}



\theoremstyle{remark}
\newtheorem*{remark}{Remarque}
\newtheorem*{rappel}{Rappel}


\usepackage{etoolbox}
\AtBeginEnvironment{exercise}{\small}
\AtBeginEnvironment{example}{\small}

\usepackage{cases}
\usepackage[red]{mypack}

\usepackage[framemethod=TikZ]{mdframed}

\definecolor{bg}{rgb}{0.4,0.25,0.95}
\definecolor{pagebg}{rgb}{0,0,0.5}
\surroundwithmdframed[
   topline=false,
   rightline=false,
   bottomline=false,
   leftmargin=\parindent,
   skipabove=8pt,
   skipbelow=8pt,
   linecolor=blue,
   innerbottommargin=10pt,
   % backgroundcolor=bg,font=\color{orange}\sffamily, fontcolor=white
]{definition}

\usepackage{empheq}
\usepackage[most]{tcolorbox}

\newtcbox{\mymath}[1][]{%
    nobeforeafter, math upper, tcbox raise base,
    enhanced, colframe=blue!30!black,
    colback=red!10, boxrule=1pt,
    #1}

\usepackage{unixode}


\DeclareMathOperator{\ord}{ord}
\DeclareMathOperator{\orb}{orb}
\DeclareMathOperator{\stab}{stab}
\DeclareMathOperator{\Stab}{stab}
\DeclareMathOperator{\ppcm}{ppcm}
\DeclareMathOperator{\conj}{Conj}
\DeclareMathOperator{\End}{End}
\DeclareMathOperator{\rot}{rot}
\DeclareMathOperator{\trs}{trace}
\DeclareMathOperator{\Ind}{Ind}
\DeclareMathOperator{\mat}{Mat}
\DeclareMathOperator{\id}{Id}
\DeclareMathOperator{\vect}{vect}
\DeclareMathOperator{\img}{img}
\DeclareMathOperator{\cov}{Cov}
\DeclareMathOperator{\dist}{dist}
\DeclareMathOperator{\irr}{Irr}
\DeclareMathOperator{\image}{Im}
\DeclareMathOperator{\pd}{\partial}
\DeclareMathOperator{\epi}{epi}
\DeclareMathOperator{\Argmin}{Argmin}
\DeclareMathOperator{\dom}{dom}
\DeclareMathOperator{\proj}{proj}
\DeclareMathOperator{\ctg}{ctg}
\DeclareMathOperator{\supp}{supp}
\DeclareMathOperator{\argmin}{argmin}
\DeclareMathOperator{\mult}{mult}
\DeclareMathOperator{\ch}{ch}
\DeclareMathOperator{\sh}{sh}
\DeclareMathOperator{\rang}{rang}
\DeclareMathOperator{\diam}{diam}
\DeclareMathOperator{\Epigraphe}{Epigraphe}




\usepackage{xcolor}
\everymath{\color{blue}}
%\everymath{\color[rgb]{0,1,1}}
%\pagecolor[rgb]{0,0,0.5}


\newcommand*{\pdtest}[3][]{\ensuremath{\frac{\partial^{#1} #2}{\partial #3}}}

\newcommand*{\deffunc}[6][]{\ensuremath{
\begin{array}{rcl}
#2 : #3 &\rightarrow& #4\\
#5 &\mapsto& #6
\end{array}
}}

\newcommand{\eqcolon}{\mathrel{\resizebox{\widthof{$\mathord{=}$}}{\height}{ $\!\!=\!\!\resizebox{1.2\width}{0.8\height}{\raisebox{0.23ex}{$\mathop{:}$}}\!\!$ }}}
\newcommand{\coloneq}{\mathrel{\resizebox{\widthof{$\mathord{=}$}}{\height}{ $\!\!\resizebox{1.2\width}{0.8\height}{\raisebox{0.23ex}{$\mathop{:}$}}\!\!=\!\!$ }}}
\newcommand{\eqcolonl}{\ensuremath{\mathrel{=\!\!\mathop{:}}}}
\newcommand{\coloneql}{\ensuremath{\mathrel{\mathop{:} \!\! =}}}
\newcommand{\vc}[1]{% inline column vector
  \left(\begin{smallmatrix}#1\end{smallmatrix}\right)%
}
\newcommand{\vr}[1]{% inline row vector
  \begin{smallmatrix}(\,#1\,)\end{smallmatrix}%
}
\makeatletter
\newcommand*{\defeq}{\ =\mathrel{\rlap{%
                     \raisebox{0.3ex}{$\m@th\cdot$}}%
                     \raisebox{-0.3ex}{$\m@th\cdot$}}%
                     }
\makeatother

\newcommand{\mathcircle}[1]{% inline row vector
 \overset{\circ}{#1}
}
\newcommand{\ulim}{% low limit
 \underline{\lim}
}
\newcommand{\ssi}{% iff
\iff
}
\newcommand{\ps}[2]{
\expval{#1 | #2}
}
\newcommand{\df}[1]{
\mqty{#1}
}
\newcommand{\n}[1]{
\norm{#1}
}
\newcommand{\sys}[1]{
\left\{\smqty{#1}\right.
}


\newcommand{\eqdef}{\ensuremath{\overset{\text{def}}=}}


\def\Circlearrowright{\ensuremath{%
  \rotatebox[origin=c]{230}{$\circlearrowright$}}}

\newcommand\ct[1]{\text{\rmfamily\upshape #1}}
\newcommand\question[1]{ {\color{red} ...!? \small #1}}
\newcommand\caz[1]{\left\{\begin{array} #1 \end{array}\right.}
\newcommand\const{\text{\rmfamily\upshape const}}
\newcommand\toP{ \overset{\pro}{\to}}
\newcommand\toPP{ \overset{\text{PP}}{\to}}
\newcommand{\oeq}{\mathrel{\text{\textcircled{$=$}}}}





\usepackage{xcolor}
% \usepackage[normalem]{ulem}
\usepackage{lipsum}
\makeatletter
% \newcommand\colorwave[1][blue]{\bgroup \markoverwith{\lower3.5\p@\hbox{\sixly \textcolor{#1}{\char58}}}\ULon}
%\font\sixly=lasy6 % does not re-load if already loaded, so no memory problem.

\newmdtheoremenv[
linewidth= 1pt,linecolor= blue,%
leftmargin=20,rightmargin=20,innertopmargin=0pt, innerrightmargin=40,%
tikzsetting = { draw=lightgray, line width = 0.3pt,dashed,%
dash pattern = on 15pt off 3pt},%
splittopskip=\topskip,skipbelow=\baselineskip,%
skipabove=\baselineskip,ntheorem,roundcorner=0pt,
% backgroundcolor=pagebg,font=\color{orange}\sffamily, fontcolor=white
]{examplebox}{Exemple}[section]



\newcommand\R{\mathbb{R}}
\newcommand\Z{\mathbb{Z}}
\newcommand\N{\mathbb{N}}
\newcommand\E{\mathbb{E}}
\newcommand\F{\mathcal{F}}
\newcommand\cH{\mathcal{H}}
\newcommand\V{\mathbb{V}}
\newcommand\dmo{ ^{-1} }
\newcommand\kapa{\kappa}
\newcommand\im{Im}
\newcommand\hs{\mathcal{H}}





\usepackage{soul}

\makeatletter
\newcommand*{\whiten}[1]{\llap{\textcolor{white}{{\the\SOUL@token}}\hspace{#1pt}}}
\DeclareRobustCommand*\myul{%
    \def\SOUL@everyspace{\underline{\space}\kern\z@}%
    \def\SOUL@everytoken{%
     \setbox0=\hbox{\the\SOUL@token}%
     \ifdim\dp0>\z@
        \raisebox{\dp0}{\underline{\phantom{\the\SOUL@token}}}%
        \whiten{1}\whiten{0}%
        \whiten{-1}\whiten{-2}%
        \llap{\the\SOUL@token}%
     \else
        \underline{\the\SOUL@token}%
     \fi}%
\SOUL@}
\makeatother

\newcommand*{\demp}{\fontfamily{lmtt}\selectfont}

\DeclareTextFontCommand{\textdemp}{\demp}

\begin{document}

\ifcomment
Multiline
comment
\fi
\ifcomment
\myul{Typesetting test}
% \color[rgb]{1,1,1}
$∑_i^n≠ 60º±∞π∆¬≈√j∫h≤≥µ$

$\CR \R\pro\ind\pro\gS\pro
\mqty[a&b\\c&d]$
$\pro\mathbb{P}$
$\dd{x}$

  \[
    \alpha(x)=\left\{
                \begin{array}{ll}
                  x\\
                  \frac{1}{1+e^{-kx}}\\
                  \frac{e^x-e^{-x}}{e^x+e^{-x}}
                \end{array}
              \right.
  \]

  $\expval{x}$
  
  $\chi_\rho(ghg\dmo)=\Tr(\rho_{ghg\dmo})=\Tr(\rho_g\circ\rho_h\circ\rho\dmo_g)=\Tr(\rho_h)\overset{\mbox{\scalebox{0.5}{$\Tr(AB)=\Tr(BA)$}}}{=}\chi_\rho(h)$
  	$\mathop{\oplus}_{\substack{x\in X}}$

$\mat(\rho_g)=(a_{ij}(g))_{\scriptsize \substack{1\leq i\leq d \\ 1\leq j\leq d}}$ et $\mat(\rho'_g)=(a'_{ij}(g))_{\scriptsize \substack{1\leq i'\leq d' \\ 1\leq j'\leq d'}}$



\[\int_a^b{\mathbb{R}^2}g(u, v)\dd{P_{XY}}(u, v)=\iint g(u,v) f_{XY}(u, v)\dd \lambda(u) \dd \lambda(v)\]
$$\lim_{x\to\infty} f(x)$$	
$$\iiiint_V \mu(t,u,v,w) \,dt\,du\,dv\,dw$$
$$\sum_{n=1}^{\infty} 2^{-n} = 1$$	
\begin{definition}
	Si $X$ et $Y$ sont 2 v.a. ou definit la \textsc{Covariance} entre $X$ et $Y$ comme
	$\cov(X,Y)\overset{\text{def}}{=}\E\left[(X-\E(X))(Y-\E(Y))\right]=\E(XY)-\E(X)\E(Y)$.
\end{definition}
\fi
\pagebreak

% \tableofcontents

% insert your code here
%\input{./algebra/main.tex}
%\input{./geometrie-differentielle/main.tex}
%\input{./probabilite/main.tex}
%\input{./analyse-fonctionnelle/main.tex}
% \input{./Analyse-convexe-et-dualite-en-optimisation/main.tex}
%\input{./tikz/main.tex}
%\input{./Theorie-du-distributions/main.tex}
%\input{./optimisation/mine.tex}
 \input{./modelisation/main.tex}

% yves.aubry@univ-tln.fr : algebra

\end{document}

% % !TEX encoding = UTF-8 Unicode
% !TEX TS-program = xelatex

\documentclass[french]{report}

%\usepackage[utf8]{inputenc}
%\usepackage[T1]{fontenc}
\usepackage{babel}


\newif\ifcomment
%\commenttrue # Show comments

\usepackage{physics}
\usepackage{amssymb}


\usepackage{amsthm}
% \usepackage{thmtools}
\usepackage{mathtools}
\usepackage{amsfonts}

\usepackage{color}

\usepackage{tikz}

\usepackage{geometry}
\geometry{a5paper, margin=0.1in, right=1cm}

\usepackage{dsfont}

\usepackage{graphicx}
\graphicspath{ {images/} }

\usepackage{faktor}

\usepackage{IEEEtrantools}
\usepackage{enumerate}   
\usepackage[PostScript=dvips]{"/Users/aware/Documents/Courses/diagrams"}


\newtheorem{theorem}{Théorème}[section]
\renewcommand{\thetheorem}{\arabic{theorem}}
\newtheorem{lemme}{Lemme}[section]
\renewcommand{\thelemme}{\arabic{lemme}}
\newtheorem{proposition}{Proposition}[section]
\renewcommand{\theproposition}{\arabic{proposition}}
\newtheorem{notations}{Notations}[section]
\newtheorem{problem}{Problème}[section]
\newtheorem{corollary}{Corollaire}[theorem]
\renewcommand{\thecorollary}{\arabic{corollary}}
\newtheorem{property}{Propriété}[section]
\newtheorem{objective}{Objectif}[section]

\theoremstyle{definition}
\newtheorem{definition}{Définition}[section]
\renewcommand{\thedefinition}{\arabic{definition}}
\newtheorem{exercise}{Exercice}[chapter]
\renewcommand{\theexercise}{\arabic{exercise}}
\newtheorem{example}{Exemple}[chapter]
\renewcommand{\theexample}{\arabic{example}}
\newtheorem*{solution}{Solution}
\newtheorem*{application}{Application}
\newtheorem*{notation}{Notation}
\newtheorem*{vocabulary}{Vocabulaire}
\newtheorem*{properties}{Propriétés}



\theoremstyle{remark}
\newtheorem*{remark}{Remarque}
\newtheorem*{rappel}{Rappel}


\usepackage{etoolbox}
\AtBeginEnvironment{exercise}{\small}
\AtBeginEnvironment{example}{\small}

\usepackage{cases}
\usepackage[red]{mypack}

\usepackage[framemethod=TikZ]{mdframed}

\definecolor{bg}{rgb}{0.4,0.25,0.95}
\definecolor{pagebg}{rgb}{0,0,0.5}
\surroundwithmdframed[
   topline=false,
   rightline=false,
   bottomline=false,
   leftmargin=\parindent,
   skipabove=8pt,
   skipbelow=8pt,
   linecolor=blue,
   innerbottommargin=10pt,
   % backgroundcolor=bg,font=\color{orange}\sffamily, fontcolor=white
]{definition}

\usepackage{empheq}
\usepackage[most]{tcolorbox}

\newtcbox{\mymath}[1][]{%
    nobeforeafter, math upper, tcbox raise base,
    enhanced, colframe=blue!30!black,
    colback=red!10, boxrule=1pt,
    #1}

\usepackage{unixode}


\DeclareMathOperator{\ord}{ord}
\DeclareMathOperator{\orb}{orb}
\DeclareMathOperator{\stab}{stab}
\DeclareMathOperator{\Stab}{stab}
\DeclareMathOperator{\ppcm}{ppcm}
\DeclareMathOperator{\conj}{Conj}
\DeclareMathOperator{\End}{End}
\DeclareMathOperator{\rot}{rot}
\DeclareMathOperator{\trs}{trace}
\DeclareMathOperator{\Ind}{Ind}
\DeclareMathOperator{\mat}{Mat}
\DeclareMathOperator{\id}{Id}
\DeclareMathOperator{\vect}{vect}
\DeclareMathOperator{\img}{img}
\DeclareMathOperator{\cov}{Cov}
\DeclareMathOperator{\dist}{dist}
\DeclareMathOperator{\irr}{Irr}
\DeclareMathOperator{\image}{Im}
\DeclareMathOperator{\pd}{\partial}
\DeclareMathOperator{\epi}{epi}
\DeclareMathOperator{\Argmin}{Argmin}
\DeclareMathOperator{\dom}{dom}
\DeclareMathOperator{\proj}{proj}
\DeclareMathOperator{\ctg}{ctg}
\DeclareMathOperator{\supp}{supp}
\DeclareMathOperator{\argmin}{argmin}
\DeclareMathOperator{\mult}{mult}
\DeclareMathOperator{\ch}{ch}
\DeclareMathOperator{\sh}{sh}
\DeclareMathOperator{\rang}{rang}
\DeclareMathOperator{\diam}{diam}
\DeclareMathOperator{\Epigraphe}{Epigraphe}




\usepackage{xcolor}
\everymath{\color{blue}}
%\everymath{\color[rgb]{0,1,1}}
%\pagecolor[rgb]{0,0,0.5}


\newcommand*{\pdtest}[3][]{\ensuremath{\frac{\partial^{#1} #2}{\partial #3}}}

\newcommand*{\deffunc}[6][]{\ensuremath{
\begin{array}{rcl}
#2 : #3 &\rightarrow& #4\\
#5 &\mapsto& #6
\end{array}
}}

\newcommand{\eqcolon}{\mathrel{\resizebox{\widthof{$\mathord{=}$}}{\height}{ $\!\!=\!\!\resizebox{1.2\width}{0.8\height}{\raisebox{0.23ex}{$\mathop{:}$}}\!\!$ }}}
\newcommand{\coloneq}{\mathrel{\resizebox{\widthof{$\mathord{=}$}}{\height}{ $\!\!\resizebox{1.2\width}{0.8\height}{\raisebox{0.23ex}{$\mathop{:}$}}\!\!=\!\!$ }}}
\newcommand{\eqcolonl}{\ensuremath{\mathrel{=\!\!\mathop{:}}}}
\newcommand{\coloneql}{\ensuremath{\mathrel{\mathop{:} \!\! =}}}
\newcommand{\vc}[1]{% inline column vector
  \left(\begin{smallmatrix}#1\end{smallmatrix}\right)%
}
\newcommand{\vr}[1]{% inline row vector
  \begin{smallmatrix}(\,#1\,)\end{smallmatrix}%
}
\makeatletter
\newcommand*{\defeq}{\ =\mathrel{\rlap{%
                     \raisebox{0.3ex}{$\m@th\cdot$}}%
                     \raisebox{-0.3ex}{$\m@th\cdot$}}%
                     }
\makeatother

\newcommand{\mathcircle}[1]{% inline row vector
 \overset{\circ}{#1}
}
\newcommand{\ulim}{% low limit
 \underline{\lim}
}
\newcommand{\ssi}{% iff
\iff
}
\newcommand{\ps}[2]{
\expval{#1 | #2}
}
\newcommand{\df}[1]{
\mqty{#1}
}
\newcommand{\n}[1]{
\norm{#1}
}
\newcommand{\sys}[1]{
\left\{\smqty{#1}\right.
}


\newcommand{\eqdef}{\ensuremath{\overset{\text{def}}=}}


\def\Circlearrowright{\ensuremath{%
  \rotatebox[origin=c]{230}{$\circlearrowright$}}}

\newcommand\ct[1]{\text{\rmfamily\upshape #1}}
\newcommand\question[1]{ {\color{red} ...!? \small #1}}
\newcommand\caz[1]{\left\{\begin{array} #1 \end{array}\right.}
\newcommand\const{\text{\rmfamily\upshape const}}
\newcommand\toP{ \overset{\pro}{\to}}
\newcommand\toPP{ \overset{\text{PP}}{\to}}
\newcommand{\oeq}{\mathrel{\text{\textcircled{$=$}}}}





\usepackage{xcolor}
% \usepackage[normalem]{ulem}
\usepackage{lipsum}
\makeatletter
% \newcommand\colorwave[1][blue]{\bgroup \markoverwith{\lower3.5\p@\hbox{\sixly \textcolor{#1}{\char58}}}\ULon}
%\font\sixly=lasy6 % does not re-load if already loaded, so no memory problem.

\newmdtheoremenv[
linewidth= 1pt,linecolor= blue,%
leftmargin=20,rightmargin=20,innertopmargin=0pt, innerrightmargin=40,%
tikzsetting = { draw=lightgray, line width = 0.3pt,dashed,%
dash pattern = on 15pt off 3pt},%
splittopskip=\topskip,skipbelow=\baselineskip,%
skipabove=\baselineskip,ntheorem,roundcorner=0pt,
% backgroundcolor=pagebg,font=\color{orange}\sffamily, fontcolor=white
]{examplebox}{Exemple}[section]



\newcommand\R{\mathbb{R}}
\newcommand\Z{\mathbb{Z}}
\newcommand\N{\mathbb{N}}
\newcommand\E{\mathbb{E}}
\newcommand\F{\mathcal{F}}
\newcommand\cH{\mathcal{H}}
\newcommand\V{\mathbb{V}}
\newcommand\dmo{ ^{-1} }
\newcommand\kapa{\kappa}
\newcommand\im{Im}
\newcommand\hs{\mathcal{H}}





\usepackage{soul}

\makeatletter
\newcommand*{\whiten}[1]{\llap{\textcolor{white}{{\the\SOUL@token}}\hspace{#1pt}}}
\DeclareRobustCommand*\myul{%
    \def\SOUL@everyspace{\underline{\space}\kern\z@}%
    \def\SOUL@everytoken{%
     \setbox0=\hbox{\the\SOUL@token}%
     \ifdim\dp0>\z@
        \raisebox{\dp0}{\underline{\phantom{\the\SOUL@token}}}%
        \whiten{1}\whiten{0}%
        \whiten{-1}\whiten{-2}%
        \llap{\the\SOUL@token}%
     \else
        \underline{\the\SOUL@token}%
     \fi}%
\SOUL@}
\makeatother

\newcommand*{\demp}{\fontfamily{lmtt}\selectfont}

\DeclareTextFontCommand{\textdemp}{\demp}

\begin{document}

\ifcomment
Multiline
comment
\fi
\ifcomment
\myul{Typesetting test}
% \color[rgb]{1,1,1}
$∑_i^n≠ 60º±∞π∆¬≈√j∫h≤≥µ$

$\CR \R\pro\ind\pro\gS\pro
\mqty[a&b\\c&d]$
$\pro\mathbb{P}$
$\dd{x}$

  \[
    \alpha(x)=\left\{
                \begin{array}{ll}
                  x\\
                  \frac{1}{1+e^{-kx}}\\
                  \frac{e^x-e^{-x}}{e^x+e^{-x}}
                \end{array}
              \right.
  \]

  $\expval{x}$
  
  $\chi_\rho(ghg\dmo)=\Tr(\rho_{ghg\dmo})=\Tr(\rho_g\circ\rho_h\circ\rho\dmo_g)=\Tr(\rho_h)\overset{\mbox{\scalebox{0.5}{$\Tr(AB)=\Tr(BA)$}}}{=}\chi_\rho(h)$
  	$\mathop{\oplus}_{\substack{x\in X}}$

$\mat(\rho_g)=(a_{ij}(g))_{\scriptsize \substack{1\leq i\leq d \\ 1\leq j\leq d}}$ et $\mat(\rho'_g)=(a'_{ij}(g))_{\scriptsize \substack{1\leq i'\leq d' \\ 1\leq j'\leq d'}}$



\[\int_a^b{\mathbb{R}^2}g(u, v)\dd{P_{XY}}(u, v)=\iint g(u,v) f_{XY}(u, v)\dd \lambda(u) \dd \lambda(v)\]
$$\lim_{x\to\infty} f(x)$$	
$$\iiiint_V \mu(t,u,v,w) \,dt\,du\,dv\,dw$$
$$\sum_{n=1}^{\infty} 2^{-n} = 1$$	
\begin{definition}
	Si $X$ et $Y$ sont 2 v.a. ou definit la \textsc{Covariance} entre $X$ et $Y$ comme
	$\cov(X,Y)\overset{\text{def}}{=}\E\left[(X-\E(X))(Y-\E(Y))\right]=\E(XY)-\E(X)\E(Y)$.
\end{definition}
\fi
\pagebreak

% \tableofcontents

% insert your code here
%\input{./algebra/main.tex}
%\input{./geometrie-differentielle/main.tex}
%\input{./probabilite/main.tex}
%\input{./analyse-fonctionnelle/main.tex}
% \input{./Analyse-convexe-et-dualite-en-optimisation/main.tex}
%\input{./tikz/main.tex}
%\input{./Theorie-du-distributions/main.tex}
%\input{./optimisation/mine.tex}
 \input{./modelisation/main.tex}

% yves.aubry@univ-tln.fr : algebra

\end{document}

%% !TEX encoding = UTF-8 Unicode
% !TEX TS-program = xelatex

\documentclass[french]{report}

%\usepackage[utf8]{inputenc}
%\usepackage[T1]{fontenc}
\usepackage{babel}


\newif\ifcomment
%\commenttrue # Show comments

\usepackage{physics}
\usepackage{amssymb}


\usepackage{amsthm}
% \usepackage{thmtools}
\usepackage{mathtools}
\usepackage{amsfonts}

\usepackage{color}

\usepackage{tikz}

\usepackage{geometry}
\geometry{a5paper, margin=0.1in, right=1cm}

\usepackage{dsfont}

\usepackage{graphicx}
\graphicspath{ {images/} }

\usepackage{faktor}

\usepackage{IEEEtrantools}
\usepackage{enumerate}   
\usepackage[PostScript=dvips]{"/Users/aware/Documents/Courses/diagrams"}


\newtheorem{theorem}{Théorème}[section]
\renewcommand{\thetheorem}{\arabic{theorem}}
\newtheorem{lemme}{Lemme}[section]
\renewcommand{\thelemme}{\arabic{lemme}}
\newtheorem{proposition}{Proposition}[section]
\renewcommand{\theproposition}{\arabic{proposition}}
\newtheorem{notations}{Notations}[section]
\newtheorem{problem}{Problème}[section]
\newtheorem{corollary}{Corollaire}[theorem]
\renewcommand{\thecorollary}{\arabic{corollary}}
\newtheorem{property}{Propriété}[section]
\newtheorem{objective}{Objectif}[section]

\theoremstyle{definition}
\newtheorem{definition}{Définition}[section]
\renewcommand{\thedefinition}{\arabic{definition}}
\newtheorem{exercise}{Exercice}[chapter]
\renewcommand{\theexercise}{\arabic{exercise}}
\newtheorem{example}{Exemple}[chapter]
\renewcommand{\theexample}{\arabic{example}}
\newtheorem*{solution}{Solution}
\newtheorem*{application}{Application}
\newtheorem*{notation}{Notation}
\newtheorem*{vocabulary}{Vocabulaire}
\newtheorem*{properties}{Propriétés}



\theoremstyle{remark}
\newtheorem*{remark}{Remarque}
\newtheorem*{rappel}{Rappel}


\usepackage{etoolbox}
\AtBeginEnvironment{exercise}{\small}
\AtBeginEnvironment{example}{\small}

\usepackage{cases}
\usepackage[red]{mypack}

\usepackage[framemethod=TikZ]{mdframed}

\definecolor{bg}{rgb}{0.4,0.25,0.95}
\definecolor{pagebg}{rgb}{0,0,0.5}
\surroundwithmdframed[
   topline=false,
   rightline=false,
   bottomline=false,
   leftmargin=\parindent,
   skipabove=8pt,
   skipbelow=8pt,
   linecolor=blue,
   innerbottommargin=10pt,
   % backgroundcolor=bg,font=\color{orange}\sffamily, fontcolor=white
]{definition}

\usepackage{empheq}
\usepackage[most]{tcolorbox}

\newtcbox{\mymath}[1][]{%
    nobeforeafter, math upper, tcbox raise base,
    enhanced, colframe=blue!30!black,
    colback=red!10, boxrule=1pt,
    #1}

\usepackage{unixode}


\DeclareMathOperator{\ord}{ord}
\DeclareMathOperator{\orb}{orb}
\DeclareMathOperator{\stab}{stab}
\DeclareMathOperator{\Stab}{stab}
\DeclareMathOperator{\ppcm}{ppcm}
\DeclareMathOperator{\conj}{Conj}
\DeclareMathOperator{\End}{End}
\DeclareMathOperator{\rot}{rot}
\DeclareMathOperator{\trs}{trace}
\DeclareMathOperator{\Ind}{Ind}
\DeclareMathOperator{\mat}{Mat}
\DeclareMathOperator{\id}{Id}
\DeclareMathOperator{\vect}{vect}
\DeclareMathOperator{\img}{img}
\DeclareMathOperator{\cov}{Cov}
\DeclareMathOperator{\dist}{dist}
\DeclareMathOperator{\irr}{Irr}
\DeclareMathOperator{\image}{Im}
\DeclareMathOperator{\pd}{\partial}
\DeclareMathOperator{\epi}{epi}
\DeclareMathOperator{\Argmin}{Argmin}
\DeclareMathOperator{\dom}{dom}
\DeclareMathOperator{\proj}{proj}
\DeclareMathOperator{\ctg}{ctg}
\DeclareMathOperator{\supp}{supp}
\DeclareMathOperator{\argmin}{argmin}
\DeclareMathOperator{\mult}{mult}
\DeclareMathOperator{\ch}{ch}
\DeclareMathOperator{\sh}{sh}
\DeclareMathOperator{\rang}{rang}
\DeclareMathOperator{\diam}{diam}
\DeclareMathOperator{\Epigraphe}{Epigraphe}




\usepackage{xcolor}
\everymath{\color{blue}}
%\everymath{\color[rgb]{0,1,1}}
%\pagecolor[rgb]{0,0,0.5}


\newcommand*{\pdtest}[3][]{\ensuremath{\frac{\partial^{#1} #2}{\partial #3}}}

\newcommand*{\deffunc}[6][]{\ensuremath{
\begin{array}{rcl}
#2 : #3 &\rightarrow& #4\\
#5 &\mapsto& #6
\end{array}
}}

\newcommand{\eqcolon}{\mathrel{\resizebox{\widthof{$\mathord{=}$}}{\height}{ $\!\!=\!\!\resizebox{1.2\width}{0.8\height}{\raisebox{0.23ex}{$\mathop{:}$}}\!\!$ }}}
\newcommand{\coloneq}{\mathrel{\resizebox{\widthof{$\mathord{=}$}}{\height}{ $\!\!\resizebox{1.2\width}{0.8\height}{\raisebox{0.23ex}{$\mathop{:}$}}\!\!=\!\!$ }}}
\newcommand{\eqcolonl}{\ensuremath{\mathrel{=\!\!\mathop{:}}}}
\newcommand{\coloneql}{\ensuremath{\mathrel{\mathop{:} \!\! =}}}
\newcommand{\vc}[1]{% inline column vector
  \left(\begin{smallmatrix}#1\end{smallmatrix}\right)%
}
\newcommand{\vr}[1]{% inline row vector
  \begin{smallmatrix}(\,#1\,)\end{smallmatrix}%
}
\makeatletter
\newcommand*{\defeq}{\ =\mathrel{\rlap{%
                     \raisebox{0.3ex}{$\m@th\cdot$}}%
                     \raisebox{-0.3ex}{$\m@th\cdot$}}%
                     }
\makeatother

\newcommand{\mathcircle}[1]{% inline row vector
 \overset{\circ}{#1}
}
\newcommand{\ulim}{% low limit
 \underline{\lim}
}
\newcommand{\ssi}{% iff
\iff
}
\newcommand{\ps}[2]{
\expval{#1 | #2}
}
\newcommand{\df}[1]{
\mqty{#1}
}
\newcommand{\n}[1]{
\norm{#1}
}
\newcommand{\sys}[1]{
\left\{\smqty{#1}\right.
}


\newcommand{\eqdef}{\ensuremath{\overset{\text{def}}=}}


\def\Circlearrowright{\ensuremath{%
  \rotatebox[origin=c]{230}{$\circlearrowright$}}}

\newcommand\ct[1]{\text{\rmfamily\upshape #1}}
\newcommand\question[1]{ {\color{red} ...!? \small #1}}
\newcommand\caz[1]{\left\{\begin{array} #1 \end{array}\right.}
\newcommand\const{\text{\rmfamily\upshape const}}
\newcommand\toP{ \overset{\pro}{\to}}
\newcommand\toPP{ \overset{\text{PP}}{\to}}
\newcommand{\oeq}{\mathrel{\text{\textcircled{$=$}}}}





\usepackage{xcolor}
% \usepackage[normalem]{ulem}
\usepackage{lipsum}
\makeatletter
% \newcommand\colorwave[1][blue]{\bgroup \markoverwith{\lower3.5\p@\hbox{\sixly \textcolor{#1}{\char58}}}\ULon}
%\font\sixly=lasy6 % does not re-load if already loaded, so no memory problem.

\newmdtheoremenv[
linewidth= 1pt,linecolor= blue,%
leftmargin=20,rightmargin=20,innertopmargin=0pt, innerrightmargin=40,%
tikzsetting = { draw=lightgray, line width = 0.3pt,dashed,%
dash pattern = on 15pt off 3pt},%
splittopskip=\topskip,skipbelow=\baselineskip,%
skipabove=\baselineskip,ntheorem,roundcorner=0pt,
% backgroundcolor=pagebg,font=\color{orange}\sffamily, fontcolor=white
]{examplebox}{Exemple}[section]



\newcommand\R{\mathbb{R}}
\newcommand\Z{\mathbb{Z}}
\newcommand\N{\mathbb{N}}
\newcommand\E{\mathbb{E}}
\newcommand\F{\mathcal{F}}
\newcommand\cH{\mathcal{H}}
\newcommand\V{\mathbb{V}}
\newcommand\dmo{ ^{-1} }
\newcommand\kapa{\kappa}
\newcommand\im{Im}
\newcommand\hs{\mathcal{H}}





\usepackage{soul}

\makeatletter
\newcommand*{\whiten}[1]{\llap{\textcolor{white}{{\the\SOUL@token}}\hspace{#1pt}}}
\DeclareRobustCommand*\myul{%
    \def\SOUL@everyspace{\underline{\space}\kern\z@}%
    \def\SOUL@everytoken{%
     \setbox0=\hbox{\the\SOUL@token}%
     \ifdim\dp0>\z@
        \raisebox{\dp0}{\underline{\phantom{\the\SOUL@token}}}%
        \whiten{1}\whiten{0}%
        \whiten{-1}\whiten{-2}%
        \llap{\the\SOUL@token}%
     \else
        \underline{\the\SOUL@token}%
     \fi}%
\SOUL@}
\makeatother

\newcommand*{\demp}{\fontfamily{lmtt}\selectfont}

\DeclareTextFontCommand{\textdemp}{\demp}

\begin{document}

\ifcomment
Multiline
comment
\fi
\ifcomment
\myul{Typesetting test}
% \color[rgb]{1,1,1}
$∑_i^n≠ 60º±∞π∆¬≈√j∫h≤≥µ$

$\CR \R\pro\ind\pro\gS\pro
\mqty[a&b\\c&d]$
$\pro\mathbb{P}$
$\dd{x}$

  \[
    \alpha(x)=\left\{
                \begin{array}{ll}
                  x\\
                  \frac{1}{1+e^{-kx}}\\
                  \frac{e^x-e^{-x}}{e^x+e^{-x}}
                \end{array}
              \right.
  \]

  $\expval{x}$
  
  $\chi_\rho(ghg\dmo)=\Tr(\rho_{ghg\dmo})=\Tr(\rho_g\circ\rho_h\circ\rho\dmo_g)=\Tr(\rho_h)\overset{\mbox{\scalebox{0.5}{$\Tr(AB)=\Tr(BA)$}}}{=}\chi_\rho(h)$
  	$\mathop{\oplus}_{\substack{x\in X}}$

$\mat(\rho_g)=(a_{ij}(g))_{\scriptsize \substack{1\leq i\leq d \\ 1\leq j\leq d}}$ et $\mat(\rho'_g)=(a'_{ij}(g))_{\scriptsize \substack{1\leq i'\leq d' \\ 1\leq j'\leq d'}}$



\[\int_a^b{\mathbb{R}^2}g(u, v)\dd{P_{XY}}(u, v)=\iint g(u,v) f_{XY}(u, v)\dd \lambda(u) \dd \lambda(v)\]
$$\lim_{x\to\infty} f(x)$$	
$$\iiiint_V \mu(t,u,v,w) \,dt\,du\,dv\,dw$$
$$\sum_{n=1}^{\infty} 2^{-n} = 1$$	
\begin{definition}
	Si $X$ et $Y$ sont 2 v.a. ou definit la \textsc{Covariance} entre $X$ et $Y$ comme
	$\cov(X,Y)\overset{\text{def}}{=}\E\left[(X-\E(X))(Y-\E(Y))\right]=\E(XY)-\E(X)\E(Y)$.
\end{definition}
\fi
\pagebreak

% \tableofcontents

% insert your code here
%\input{./algebra/main.tex}
%\input{./geometrie-differentielle/main.tex}
%\input{./probabilite/main.tex}
%\input{./analyse-fonctionnelle/main.tex}
% \input{./Analyse-convexe-et-dualite-en-optimisation/main.tex}
%\input{./tikz/main.tex}
%\input{./Theorie-du-distributions/main.tex}
%\input{./optimisation/mine.tex}
 \input{./modelisation/main.tex}

% yves.aubry@univ-tln.fr : algebra

\end{document}

%% !TEX encoding = UTF-8 Unicode
% !TEX TS-program = xelatex

\documentclass[french]{report}

%\usepackage[utf8]{inputenc}
%\usepackage[T1]{fontenc}
\usepackage{babel}


\newif\ifcomment
%\commenttrue # Show comments

\usepackage{physics}
\usepackage{amssymb}


\usepackage{amsthm}
% \usepackage{thmtools}
\usepackage{mathtools}
\usepackage{amsfonts}

\usepackage{color}

\usepackage{tikz}

\usepackage{geometry}
\geometry{a5paper, margin=0.1in, right=1cm}

\usepackage{dsfont}

\usepackage{graphicx}
\graphicspath{ {images/} }

\usepackage{faktor}

\usepackage{IEEEtrantools}
\usepackage{enumerate}   
\usepackage[PostScript=dvips]{"/Users/aware/Documents/Courses/diagrams"}


\newtheorem{theorem}{Théorème}[section]
\renewcommand{\thetheorem}{\arabic{theorem}}
\newtheorem{lemme}{Lemme}[section]
\renewcommand{\thelemme}{\arabic{lemme}}
\newtheorem{proposition}{Proposition}[section]
\renewcommand{\theproposition}{\arabic{proposition}}
\newtheorem{notations}{Notations}[section]
\newtheorem{problem}{Problème}[section]
\newtheorem{corollary}{Corollaire}[theorem]
\renewcommand{\thecorollary}{\arabic{corollary}}
\newtheorem{property}{Propriété}[section]
\newtheorem{objective}{Objectif}[section]

\theoremstyle{definition}
\newtheorem{definition}{Définition}[section]
\renewcommand{\thedefinition}{\arabic{definition}}
\newtheorem{exercise}{Exercice}[chapter]
\renewcommand{\theexercise}{\arabic{exercise}}
\newtheorem{example}{Exemple}[chapter]
\renewcommand{\theexample}{\arabic{example}}
\newtheorem*{solution}{Solution}
\newtheorem*{application}{Application}
\newtheorem*{notation}{Notation}
\newtheorem*{vocabulary}{Vocabulaire}
\newtheorem*{properties}{Propriétés}



\theoremstyle{remark}
\newtheorem*{remark}{Remarque}
\newtheorem*{rappel}{Rappel}


\usepackage{etoolbox}
\AtBeginEnvironment{exercise}{\small}
\AtBeginEnvironment{example}{\small}

\usepackage{cases}
\usepackage[red]{mypack}

\usepackage[framemethod=TikZ]{mdframed}

\definecolor{bg}{rgb}{0.4,0.25,0.95}
\definecolor{pagebg}{rgb}{0,0,0.5}
\surroundwithmdframed[
   topline=false,
   rightline=false,
   bottomline=false,
   leftmargin=\parindent,
   skipabove=8pt,
   skipbelow=8pt,
   linecolor=blue,
   innerbottommargin=10pt,
   % backgroundcolor=bg,font=\color{orange}\sffamily, fontcolor=white
]{definition}

\usepackage{empheq}
\usepackage[most]{tcolorbox}

\newtcbox{\mymath}[1][]{%
    nobeforeafter, math upper, tcbox raise base,
    enhanced, colframe=blue!30!black,
    colback=red!10, boxrule=1pt,
    #1}

\usepackage{unixode}


\DeclareMathOperator{\ord}{ord}
\DeclareMathOperator{\orb}{orb}
\DeclareMathOperator{\stab}{stab}
\DeclareMathOperator{\Stab}{stab}
\DeclareMathOperator{\ppcm}{ppcm}
\DeclareMathOperator{\conj}{Conj}
\DeclareMathOperator{\End}{End}
\DeclareMathOperator{\rot}{rot}
\DeclareMathOperator{\trs}{trace}
\DeclareMathOperator{\Ind}{Ind}
\DeclareMathOperator{\mat}{Mat}
\DeclareMathOperator{\id}{Id}
\DeclareMathOperator{\vect}{vect}
\DeclareMathOperator{\img}{img}
\DeclareMathOperator{\cov}{Cov}
\DeclareMathOperator{\dist}{dist}
\DeclareMathOperator{\irr}{Irr}
\DeclareMathOperator{\image}{Im}
\DeclareMathOperator{\pd}{\partial}
\DeclareMathOperator{\epi}{epi}
\DeclareMathOperator{\Argmin}{Argmin}
\DeclareMathOperator{\dom}{dom}
\DeclareMathOperator{\proj}{proj}
\DeclareMathOperator{\ctg}{ctg}
\DeclareMathOperator{\supp}{supp}
\DeclareMathOperator{\argmin}{argmin}
\DeclareMathOperator{\mult}{mult}
\DeclareMathOperator{\ch}{ch}
\DeclareMathOperator{\sh}{sh}
\DeclareMathOperator{\rang}{rang}
\DeclareMathOperator{\diam}{diam}
\DeclareMathOperator{\Epigraphe}{Epigraphe}




\usepackage{xcolor}
\everymath{\color{blue}}
%\everymath{\color[rgb]{0,1,1}}
%\pagecolor[rgb]{0,0,0.5}


\newcommand*{\pdtest}[3][]{\ensuremath{\frac{\partial^{#1} #2}{\partial #3}}}

\newcommand*{\deffunc}[6][]{\ensuremath{
\begin{array}{rcl}
#2 : #3 &\rightarrow& #4\\
#5 &\mapsto& #6
\end{array}
}}

\newcommand{\eqcolon}{\mathrel{\resizebox{\widthof{$\mathord{=}$}}{\height}{ $\!\!=\!\!\resizebox{1.2\width}{0.8\height}{\raisebox{0.23ex}{$\mathop{:}$}}\!\!$ }}}
\newcommand{\coloneq}{\mathrel{\resizebox{\widthof{$\mathord{=}$}}{\height}{ $\!\!\resizebox{1.2\width}{0.8\height}{\raisebox{0.23ex}{$\mathop{:}$}}\!\!=\!\!$ }}}
\newcommand{\eqcolonl}{\ensuremath{\mathrel{=\!\!\mathop{:}}}}
\newcommand{\coloneql}{\ensuremath{\mathrel{\mathop{:} \!\! =}}}
\newcommand{\vc}[1]{% inline column vector
  \left(\begin{smallmatrix}#1\end{smallmatrix}\right)%
}
\newcommand{\vr}[1]{% inline row vector
  \begin{smallmatrix}(\,#1\,)\end{smallmatrix}%
}
\makeatletter
\newcommand*{\defeq}{\ =\mathrel{\rlap{%
                     \raisebox{0.3ex}{$\m@th\cdot$}}%
                     \raisebox{-0.3ex}{$\m@th\cdot$}}%
                     }
\makeatother

\newcommand{\mathcircle}[1]{% inline row vector
 \overset{\circ}{#1}
}
\newcommand{\ulim}{% low limit
 \underline{\lim}
}
\newcommand{\ssi}{% iff
\iff
}
\newcommand{\ps}[2]{
\expval{#1 | #2}
}
\newcommand{\df}[1]{
\mqty{#1}
}
\newcommand{\n}[1]{
\norm{#1}
}
\newcommand{\sys}[1]{
\left\{\smqty{#1}\right.
}


\newcommand{\eqdef}{\ensuremath{\overset{\text{def}}=}}


\def\Circlearrowright{\ensuremath{%
  \rotatebox[origin=c]{230}{$\circlearrowright$}}}

\newcommand\ct[1]{\text{\rmfamily\upshape #1}}
\newcommand\question[1]{ {\color{red} ...!? \small #1}}
\newcommand\caz[1]{\left\{\begin{array} #1 \end{array}\right.}
\newcommand\const{\text{\rmfamily\upshape const}}
\newcommand\toP{ \overset{\pro}{\to}}
\newcommand\toPP{ \overset{\text{PP}}{\to}}
\newcommand{\oeq}{\mathrel{\text{\textcircled{$=$}}}}





\usepackage{xcolor}
% \usepackage[normalem]{ulem}
\usepackage{lipsum}
\makeatletter
% \newcommand\colorwave[1][blue]{\bgroup \markoverwith{\lower3.5\p@\hbox{\sixly \textcolor{#1}{\char58}}}\ULon}
%\font\sixly=lasy6 % does not re-load if already loaded, so no memory problem.

\newmdtheoremenv[
linewidth= 1pt,linecolor= blue,%
leftmargin=20,rightmargin=20,innertopmargin=0pt, innerrightmargin=40,%
tikzsetting = { draw=lightgray, line width = 0.3pt,dashed,%
dash pattern = on 15pt off 3pt},%
splittopskip=\topskip,skipbelow=\baselineskip,%
skipabove=\baselineskip,ntheorem,roundcorner=0pt,
% backgroundcolor=pagebg,font=\color{orange}\sffamily, fontcolor=white
]{examplebox}{Exemple}[section]



\newcommand\R{\mathbb{R}}
\newcommand\Z{\mathbb{Z}}
\newcommand\N{\mathbb{N}}
\newcommand\E{\mathbb{E}}
\newcommand\F{\mathcal{F}}
\newcommand\cH{\mathcal{H}}
\newcommand\V{\mathbb{V}}
\newcommand\dmo{ ^{-1} }
\newcommand\kapa{\kappa}
\newcommand\im{Im}
\newcommand\hs{\mathcal{H}}





\usepackage{soul}

\makeatletter
\newcommand*{\whiten}[1]{\llap{\textcolor{white}{{\the\SOUL@token}}\hspace{#1pt}}}
\DeclareRobustCommand*\myul{%
    \def\SOUL@everyspace{\underline{\space}\kern\z@}%
    \def\SOUL@everytoken{%
     \setbox0=\hbox{\the\SOUL@token}%
     \ifdim\dp0>\z@
        \raisebox{\dp0}{\underline{\phantom{\the\SOUL@token}}}%
        \whiten{1}\whiten{0}%
        \whiten{-1}\whiten{-2}%
        \llap{\the\SOUL@token}%
     \else
        \underline{\the\SOUL@token}%
     \fi}%
\SOUL@}
\makeatother

\newcommand*{\demp}{\fontfamily{lmtt}\selectfont}

\DeclareTextFontCommand{\textdemp}{\demp}

\begin{document}

\ifcomment
Multiline
comment
\fi
\ifcomment
\myul{Typesetting test}
% \color[rgb]{1,1,1}
$∑_i^n≠ 60º±∞π∆¬≈√j∫h≤≥µ$

$\CR \R\pro\ind\pro\gS\pro
\mqty[a&b\\c&d]$
$\pro\mathbb{P}$
$\dd{x}$

  \[
    \alpha(x)=\left\{
                \begin{array}{ll}
                  x\\
                  \frac{1}{1+e^{-kx}}\\
                  \frac{e^x-e^{-x}}{e^x+e^{-x}}
                \end{array}
              \right.
  \]

  $\expval{x}$
  
  $\chi_\rho(ghg\dmo)=\Tr(\rho_{ghg\dmo})=\Tr(\rho_g\circ\rho_h\circ\rho\dmo_g)=\Tr(\rho_h)\overset{\mbox{\scalebox{0.5}{$\Tr(AB)=\Tr(BA)$}}}{=}\chi_\rho(h)$
  	$\mathop{\oplus}_{\substack{x\in X}}$

$\mat(\rho_g)=(a_{ij}(g))_{\scriptsize \substack{1\leq i\leq d \\ 1\leq j\leq d}}$ et $\mat(\rho'_g)=(a'_{ij}(g))_{\scriptsize \substack{1\leq i'\leq d' \\ 1\leq j'\leq d'}}$



\[\int_a^b{\mathbb{R}^2}g(u, v)\dd{P_{XY}}(u, v)=\iint g(u,v) f_{XY}(u, v)\dd \lambda(u) \dd \lambda(v)\]
$$\lim_{x\to\infty} f(x)$$	
$$\iiiint_V \mu(t,u,v,w) \,dt\,du\,dv\,dw$$
$$\sum_{n=1}^{\infty} 2^{-n} = 1$$	
\begin{definition}
	Si $X$ et $Y$ sont 2 v.a. ou definit la \textsc{Covariance} entre $X$ et $Y$ comme
	$\cov(X,Y)\overset{\text{def}}{=}\E\left[(X-\E(X))(Y-\E(Y))\right]=\E(XY)-\E(X)\E(Y)$.
\end{definition}
\fi
\pagebreak

% \tableofcontents

% insert your code here
%\input{./algebra/main.tex}
%\input{./geometrie-differentielle/main.tex}
%\input{./probabilite/main.tex}
%\input{./analyse-fonctionnelle/main.tex}
% \input{./Analyse-convexe-et-dualite-en-optimisation/main.tex}
%\input{./tikz/main.tex}
%\input{./Theorie-du-distributions/main.tex}
%\input{./optimisation/mine.tex}
 \input{./modelisation/main.tex}

% yves.aubry@univ-tln.fr : algebra

\end{document}

%\input{./optimisation/mine.tex}
 % !TEX encoding = UTF-8 Unicode
% !TEX TS-program = xelatex

\documentclass[french]{report}

%\usepackage[utf8]{inputenc}
%\usepackage[T1]{fontenc}
\usepackage{babel}


\newif\ifcomment
%\commenttrue # Show comments

\usepackage{physics}
\usepackage{amssymb}


\usepackage{amsthm}
% \usepackage{thmtools}
\usepackage{mathtools}
\usepackage{amsfonts}

\usepackage{color}

\usepackage{tikz}

\usepackage{geometry}
\geometry{a5paper, margin=0.1in, right=1cm}

\usepackage{dsfont}

\usepackage{graphicx}
\graphicspath{ {images/} }

\usepackage{faktor}

\usepackage{IEEEtrantools}
\usepackage{enumerate}   
\usepackage[PostScript=dvips]{"/Users/aware/Documents/Courses/diagrams"}


\newtheorem{theorem}{Théorème}[section]
\renewcommand{\thetheorem}{\arabic{theorem}}
\newtheorem{lemme}{Lemme}[section]
\renewcommand{\thelemme}{\arabic{lemme}}
\newtheorem{proposition}{Proposition}[section]
\renewcommand{\theproposition}{\arabic{proposition}}
\newtheorem{notations}{Notations}[section]
\newtheorem{problem}{Problème}[section]
\newtheorem{corollary}{Corollaire}[theorem]
\renewcommand{\thecorollary}{\arabic{corollary}}
\newtheorem{property}{Propriété}[section]
\newtheorem{objective}{Objectif}[section]

\theoremstyle{definition}
\newtheorem{definition}{Définition}[section]
\renewcommand{\thedefinition}{\arabic{definition}}
\newtheorem{exercise}{Exercice}[chapter]
\renewcommand{\theexercise}{\arabic{exercise}}
\newtheorem{example}{Exemple}[chapter]
\renewcommand{\theexample}{\arabic{example}}
\newtheorem*{solution}{Solution}
\newtheorem*{application}{Application}
\newtheorem*{notation}{Notation}
\newtheorem*{vocabulary}{Vocabulaire}
\newtheorem*{properties}{Propriétés}



\theoremstyle{remark}
\newtheorem*{remark}{Remarque}
\newtheorem*{rappel}{Rappel}


\usepackage{etoolbox}
\AtBeginEnvironment{exercise}{\small}
\AtBeginEnvironment{example}{\small}

\usepackage{cases}
\usepackage[red]{mypack}

\usepackage[framemethod=TikZ]{mdframed}

\definecolor{bg}{rgb}{0.4,0.25,0.95}
\definecolor{pagebg}{rgb}{0,0,0.5}
\surroundwithmdframed[
   topline=false,
   rightline=false,
   bottomline=false,
   leftmargin=\parindent,
   skipabove=8pt,
   skipbelow=8pt,
   linecolor=blue,
   innerbottommargin=10pt,
   % backgroundcolor=bg,font=\color{orange}\sffamily, fontcolor=white
]{definition}

\usepackage{empheq}
\usepackage[most]{tcolorbox}

\newtcbox{\mymath}[1][]{%
    nobeforeafter, math upper, tcbox raise base,
    enhanced, colframe=blue!30!black,
    colback=red!10, boxrule=1pt,
    #1}

\usepackage{unixode}


\DeclareMathOperator{\ord}{ord}
\DeclareMathOperator{\orb}{orb}
\DeclareMathOperator{\stab}{stab}
\DeclareMathOperator{\Stab}{stab}
\DeclareMathOperator{\ppcm}{ppcm}
\DeclareMathOperator{\conj}{Conj}
\DeclareMathOperator{\End}{End}
\DeclareMathOperator{\rot}{rot}
\DeclareMathOperator{\trs}{trace}
\DeclareMathOperator{\Ind}{Ind}
\DeclareMathOperator{\mat}{Mat}
\DeclareMathOperator{\id}{Id}
\DeclareMathOperator{\vect}{vect}
\DeclareMathOperator{\img}{img}
\DeclareMathOperator{\cov}{Cov}
\DeclareMathOperator{\dist}{dist}
\DeclareMathOperator{\irr}{Irr}
\DeclareMathOperator{\image}{Im}
\DeclareMathOperator{\pd}{\partial}
\DeclareMathOperator{\epi}{epi}
\DeclareMathOperator{\Argmin}{Argmin}
\DeclareMathOperator{\dom}{dom}
\DeclareMathOperator{\proj}{proj}
\DeclareMathOperator{\ctg}{ctg}
\DeclareMathOperator{\supp}{supp}
\DeclareMathOperator{\argmin}{argmin}
\DeclareMathOperator{\mult}{mult}
\DeclareMathOperator{\ch}{ch}
\DeclareMathOperator{\sh}{sh}
\DeclareMathOperator{\rang}{rang}
\DeclareMathOperator{\diam}{diam}
\DeclareMathOperator{\Epigraphe}{Epigraphe}




\usepackage{xcolor}
\everymath{\color{blue}}
%\everymath{\color[rgb]{0,1,1}}
%\pagecolor[rgb]{0,0,0.5}


\newcommand*{\pdtest}[3][]{\ensuremath{\frac{\partial^{#1} #2}{\partial #3}}}

\newcommand*{\deffunc}[6][]{\ensuremath{
\begin{array}{rcl}
#2 : #3 &\rightarrow& #4\\
#5 &\mapsto& #6
\end{array}
}}

\newcommand{\eqcolon}{\mathrel{\resizebox{\widthof{$\mathord{=}$}}{\height}{ $\!\!=\!\!\resizebox{1.2\width}{0.8\height}{\raisebox{0.23ex}{$\mathop{:}$}}\!\!$ }}}
\newcommand{\coloneq}{\mathrel{\resizebox{\widthof{$\mathord{=}$}}{\height}{ $\!\!\resizebox{1.2\width}{0.8\height}{\raisebox{0.23ex}{$\mathop{:}$}}\!\!=\!\!$ }}}
\newcommand{\eqcolonl}{\ensuremath{\mathrel{=\!\!\mathop{:}}}}
\newcommand{\coloneql}{\ensuremath{\mathrel{\mathop{:} \!\! =}}}
\newcommand{\vc}[1]{% inline column vector
  \left(\begin{smallmatrix}#1\end{smallmatrix}\right)%
}
\newcommand{\vr}[1]{% inline row vector
  \begin{smallmatrix}(\,#1\,)\end{smallmatrix}%
}
\makeatletter
\newcommand*{\defeq}{\ =\mathrel{\rlap{%
                     \raisebox{0.3ex}{$\m@th\cdot$}}%
                     \raisebox{-0.3ex}{$\m@th\cdot$}}%
                     }
\makeatother

\newcommand{\mathcircle}[1]{% inline row vector
 \overset{\circ}{#1}
}
\newcommand{\ulim}{% low limit
 \underline{\lim}
}
\newcommand{\ssi}{% iff
\iff
}
\newcommand{\ps}[2]{
\expval{#1 | #2}
}
\newcommand{\df}[1]{
\mqty{#1}
}
\newcommand{\n}[1]{
\norm{#1}
}
\newcommand{\sys}[1]{
\left\{\smqty{#1}\right.
}


\newcommand{\eqdef}{\ensuremath{\overset{\text{def}}=}}


\def\Circlearrowright{\ensuremath{%
  \rotatebox[origin=c]{230}{$\circlearrowright$}}}

\newcommand\ct[1]{\text{\rmfamily\upshape #1}}
\newcommand\question[1]{ {\color{red} ...!? \small #1}}
\newcommand\caz[1]{\left\{\begin{array} #1 \end{array}\right.}
\newcommand\const{\text{\rmfamily\upshape const}}
\newcommand\toP{ \overset{\pro}{\to}}
\newcommand\toPP{ \overset{\text{PP}}{\to}}
\newcommand{\oeq}{\mathrel{\text{\textcircled{$=$}}}}





\usepackage{xcolor}
% \usepackage[normalem]{ulem}
\usepackage{lipsum}
\makeatletter
% \newcommand\colorwave[1][blue]{\bgroup \markoverwith{\lower3.5\p@\hbox{\sixly \textcolor{#1}{\char58}}}\ULon}
%\font\sixly=lasy6 % does not re-load if already loaded, so no memory problem.

\newmdtheoremenv[
linewidth= 1pt,linecolor= blue,%
leftmargin=20,rightmargin=20,innertopmargin=0pt, innerrightmargin=40,%
tikzsetting = { draw=lightgray, line width = 0.3pt,dashed,%
dash pattern = on 15pt off 3pt},%
splittopskip=\topskip,skipbelow=\baselineskip,%
skipabove=\baselineskip,ntheorem,roundcorner=0pt,
% backgroundcolor=pagebg,font=\color{orange}\sffamily, fontcolor=white
]{examplebox}{Exemple}[section]



\newcommand\R{\mathbb{R}}
\newcommand\Z{\mathbb{Z}}
\newcommand\N{\mathbb{N}}
\newcommand\E{\mathbb{E}}
\newcommand\F{\mathcal{F}}
\newcommand\cH{\mathcal{H}}
\newcommand\V{\mathbb{V}}
\newcommand\dmo{ ^{-1} }
\newcommand\kapa{\kappa}
\newcommand\im{Im}
\newcommand\hs{\mathcal{H}}





\usepackage{soul}

\makeatletter
\newcommand*{\whiten}[1]{\llap{\textcolor{white}{{\the\SOUL@token}}\hspace{#1pt}}}
\DeclareRobustCommand*\myul{%
    \def\SOUL@everyspace{\underline{\space}\kern\z@}%
    \def\SOUL@everytoken{%
     \setbox0=\hbox{\the\SOUL@token}%
     \ifdim\dp0>\z@
        \raisebox{\dp0}{\underline{\phantom{\the\SOUL@token}}}%
        \whiten{1}\whiten{0}%
        \whiten{-1}\whiten{-2}%
        \llap{\the\SOUL@token}%
     \else
        \underline{\the\SOUL@token}%
     \fi}%
\SOUL@}
\makeatother

\newcommand*{\demp}{\fontfamily{lmtt}\selectfont}

\DeclareTextFontCommand{\textdemp}{\demp}

\begin{document}

\ifcomment
Multiline
comment
\fi
\ifcomment
\myul{Typesetting test}
% \color[rgb]{1,1,1}
$∑_i^n≠ 60º±∞π∆¬≈√j∫h≤≥µ$

$\CR \R\pro\ind\pro\gS\pro
\mqty[a&b\\c&d]$
$\pro\mathbb{P}$
$\dd{x}$

  \[
    \alpha(x)=\left\{
                \begin{array}{ll}
                  x\\
                  \frac{1}{1+e^{-kx}}\\
                  \frac{e^x-e^{-x}}{e^x+e^{-x}}
                \end{array}
              \right.
  \]

  $\expval{x}$
  
  $\chi_\rho(ghg\dmo)=\Tr(\rho_{ghg\dmo})=\Tr(\rho_g\circ\rho_h\circ\rho\dmo_g)=\Tr(\rho_h)\overset{\mbox{\scalebox{0.5}{$\Tr(AB)=\Tr(BA)$}}}{=}\chi_\rho(h)$
  	$\mathop{\oplus}_{\substack{x\in X}}$

$\mat(\rho_g)=(a_{ij}(g))_{\scriptsize \substack{1\leq i\leq d \\ 1\leq j\leq d}}$ et $\mat(\rho'_g)=(a'_{ij}(g))_{\scriptsize \substack{1\leq i'\leq d' \\ 1\leq j'\leq d'}}$



\[\int_a^b{\mathbb{R}^2}g(u, v)\dd{P_{XY}}(u, v)=\iint g(u,v) f_{XY}(u, v)\dd \lambda(u) \dd \lambda(v)\]
$$\lim_{x\to\infty} f(x)$$	
$$\iiiint_V \mu(t,u,v,w) \,dt\,du\,dv\,dw$$
$$\sum_{n=1}^{\infty} 2^{-n} = 1$$	
\begin{definition}
	Si $X$ et $Y$ sont 2 v.a. ou definit la \textsc{Covariance} entre $X$ et $Y$ comme
	$\cov(X,Y)\overset{\text{def}}{=}\E\left[(X-\E(X))(Y-\E(Y))\right]=\E(XY)-\E(X)\E(Y)$.
\end{definition}
\fi
\pagebreak

% \tableofcontents

% insert your code here
%\input{./algebra/main.tex}
%\input{./geometrie-differentielle/main.tex}
%\input{./probabilite/main.tex}
%\input{./analyse-fonctionnelle/main.tex}
% \input{./Analyse-convexe-et-dualite-en-optimisation/main.tex}
%\input{./tikz/main.tex}
%\input{./Theorie-du-distributions/main.tex}
%\input{./optimisation/mine.tex}
 \input{./modelisation/main.tex}

% yves.aubry@univ-tln.fr : algebra

\end{document}


% yves.aubry@univ-tln.fr : algebra

\end{document}

%% !TEX encoding = UTF-8 Unicode
% !TEX TS-program = xelatex

\documentclass[french]{report}

%\usepackage[utf8]{inputenc}
%\usepackage[T1]{fontenc}
\usepackage{babel}


\newif\ifcomment
%\commenttrue # Show comments

\usepackage{physics}
\usepackage{amssymb}


\usepackage{amsthm}
% \usepackage{thmtools}
\usepackage{mathtools}
\usepackage{amsfonts}

\usepackage{color}

\usepackage{tikz}

\usepackage{geometry}
\geometry{a5paper, margin=0.1in, right=1cm}

\usepackage{dsfont}

\usepackage{graphicx}
\graphicspath{ {images/} }

\usepackage{faktor}

\usepackage{IEEEtrantools}
\usepackage{enumerate}   
\usepackage[PostScript=dvips]{"/Users/aware/Documents/Courses/diagrams"}


\newtheorem{theorem}{Théorème}[section]
\renewcommand{\thetheorem}{\arabic{theorem}}
\newtheorem{lemme}{Lemme}[section]
\renewcommand{\thelemme}{\arabic{lemme}}
\newtheorem{proposition}{Proposition}[section]
\renewcommand{\theproposition}{\arabic{proposition}}
\newtheorem{notations}{Notations}[section]
\newtheorem{problem}{Problème}[section]
\newtheorem{corollary}{Corollaire}[theorem]
\renewcommand{\thecorollary}{\arabic{corollary}}
\newtheorem{property}{Propriété}[section]
\newtheorem{objective}{Objectif}[section]

\theoremstyle{definition}
\newtheorem{definition}{Définition}[section]
\renewcommand{\thedefinition}{\arabic{definition}}
\newtheorem{exercise}{Exercice}[chapter]
\renewcommand{\theexercise}{\arabic{exercise}}
\newtheorem{example}{Exemple}[chapter]
\renewcommand{\theexample}{\arabic{example}}
\newtheorem*{solution}{Solution}
\newtheorem*{application}{Application}
\newtheorem*{notation}{Notation}
\newtheorem*{vocabulary}{Vocabulaire}
\newtheorem*{properties}{Propriétés}



\theoremstyle{remark}
\newtheorem*{remark}{Remarque}
\newtheorem*{rappel}{Rappel}


\usepackage{etoolbox}
\AtBeginEnvironment{exercise}{\small}
\AtBeginEnvironment{example}{\small}

\usepackage{cases}
\usepackage[red]{mypack}

\usepackage[framemethod=TikZ]{mdframed}

\definecolor{bg}{rgb}{0.4,0.25,0.95}
\definecolor{pagebg}{rgb}{0,0,0.5}
\surroundwithmdframed[
   topline=false,
   rightline=false,
   bottomline=false,
   leftmargin=\parindent,
   skipabove=8pt,
   skipbelow=8pt,
   linecolor=blue,
   innerbottommargin=10pt,
   % backgroundcolor=bg,font=\color{orange}\sffamily, fontcolor=white
]{definition}

\usepackage{empheq}
\usepackage[most]{tcolorbox}

\newtcbox{\mymath}[1][]{%
    nobeforeafter, math upper, tcbox raise base,
    enhanced, colframe=blue!30!black,
    colback=red!10, boxrule=1pt,
    #1}

\usepackage{unixode}


\DeclareMathOperator{\ord}{ord}
\DeclareMathOperator{\orb}{orb}
\DeclareMathOperator{\stab}{stab}
\DeclareMathOperator{\Stab}{stab}
\DeclareMathOperator{\ppcm}{ppcm}
\DeclareMathOperator{\conj}{Conj}
\DeclareMathOperator{\End}{End}
\DeclareMathOperator{\rot}{rot}
\DeclareMathOperator{\trs}{trace}
\DeclareMathOperator{\Ind}{Ind}
\DeclareMathOperator{\mat}{Mat}
\DeclareMathOperator{\id}{Id}
\DeclareMathOperator{\vect}{vect}
\DeclareMathOperator{\img}{img}
\DeclareMathOperator{\cov}{Cov}
\DeclareMathOperator{\dist}{dist}
\DeclareMathOperator{\irr}{Irr}
\DeclareMathOperator{\image}{Im}
\DeclareMathOperator{\pd}{\partial}
\DeclareMathOperator{\epi}{epi}
\DeclareMathOperator{\Argmin}{Argmin}
\DeclareMathOperator{\dom}{dom}
\DeclareMathOperator{\proj}{proj}
\DeclareMathOperator{\ctg}{ctg}
\DeclareMathOperator{\supp}{supp}
\DeclareMathOperator{\argmin}{argmin}
\DeclareMathOperator{\mult}{mult}
\DeclareMathOperator{\ch}{ch}
\DeclareMathOperator{\sh}{sh}
\DeclareMathOperator{\rang}{rang}
\DeclareMathOperator{\diam}{diam}
\DeclareMathOperator{\Epigraphe}{Epigraphe}




\usepackage{xcolor}
\everymath{\color{blue}}
%\everymath{\color[rgb]{0,1,1}}
%\pagecolor[rgb]{0,0,0.5}


\newcommand*{\pdtest}[3][]{\ensuremath{\frac{\partial^{#1} #2}{\partial #3}}}

\newcommand*{\deffunc}[6][]{\ensuremath{
\begin{array}{rcl}
#2 : #3 &\rightarrow& #4\\
#5 &\mapsto& #6
\end{array}
}}

\newcommand{\eqcolon}{\mathrel{\resizebox{\widthof{$\mathord{=}$}}{\height}{ $\!\!=\!\!\resizebox{1.2\width}{0.8\height}{\raisebox{0.23ex}{$\mathop{:}$}}\!\!$ }}}
\newcommand{\coloneq}{\mathrel{\resizebox{\widthof{$\mathord{=}$}}{\height}{ $\!\!\resizebox{1.2\width}{0.8\height}{\raisebox{0.23ex}{$\mathop{:}$}}\!\!=\!\!$ }}}
\newcommand{\eqcolonl}{\ensuremath{\mathrel{=\!\!\mathop{:}}}}
\newcommand{\coloneql}{\ensuremath{\mathrel{\mathop{:} \!\! =}}}
\newcommand{\vc}[1]{% inline column vector
  \left(\begin{smallmatrix}#1\end{smallmatrix}\right)%
}
\newcommand{\vr}[1]{% inline row vector
  \begin{smallmatrix}(\,#1\,)\end{smallmatrix}%
}
\makeatletter
\newcommand*{\defeq}{\ =\mathrel{\rlap{%
                     \raisebox{0.3ex}{$\m@th\cdot$}}%
                     \raisebox{-0.3ex}{$\m@th\cdot$}}%
                     }
\makeatother

\newcommand{\mathcircle}[1]{% inline row vector
 \overset{\circ}{#1}
}
\newcommand{\ulim}{% low limit
 \underline{\lim}
}
\newcommand{\ssi}{% iff
\iff
}
\newcommand{\ps}[2]{
\expval{#1 | #2}
}
\newcommand{\df}[1]{
\mqty{#1}
}
\newcommand{\n}[1]{
\norm{#1}
}
\newcommand{\sys}[1]{
\left\{\smqty{#1}\right.
}


\newcommand{\eqdef}{\ensuremath{\overset{\text{def}}=}}


\def\Circlearrowright{\ensuremath{%
  \rotatebox[origin=c]{230}{$\circlearrowright$}}}

\newcommand\ct[1]{\text{\rmfamily\upshape #1}}
\newcommand\question[1]{ {\color{red} ...!? \small #1}}
\newcommand\caz[1]{\left\{\begin{array} #1 \end{array}\right.}
\newcommand\const{\text{\rmfamily\upshape const}}
\newcommand\toP{ \overset{\pro}{\to}}
\newcommand\toPP{ \overset{\text{PP}}{\to}}
\newcommand{\oeq}{\mathrel{\text{\textcircled{$=$}}}}





\usepackage{xcolor}
% \usepackage[normalem]{ulem}
\usepackage{lipsum}
\makeatletter
% \newcommand\colorwave[1][blue]{\bgroup \markoverwith{\lower3.5\p@\hbox{\sixly \textcolor{#1}{\char58}}}\ULon}
%\font\sixly=lasy6 % does not re-load if already loaded, so no memory problem.

\newmdtheoremenv[
linewidth= 1pt,linecolor= blue,%
leftmargin=20,rightmargin=20,innertopmargin=0pt, innerrightmargin=40,%
tikzsetting = { draw=lightgray, line width = 0.3pt,dashed,%
dash pattern = on 15pt off 3pt},%
splittopskip=\topskip,skipbelow=\baselineskip,%
skipabove=\baselineskip,ntheorem,roundcorner=0pt,
% backgroundcolor=pagebg,font=\color{orange}\sffamily, fontcolor=white
]{examplebox}{Exemple}[section]



\newcommand\R{\mathbb{R}}
\newcommand\Z{\mathbb{Z}}
\newcommand\N{\mathbb{N}}
\newcommand\E{\mathbb{E}}
\newcommand\F{\mathcal{F}}
\newcommand\cH{\mathcal{H}}
\newcommand\V{\mathbb{V}}
\newcommand\dmo{ ^{-1} }
\newcommand\kapa{\kappa}
\newcommand\im{Im}
\newcommand\hs{\mathcal{H}}





\usepackage{soul}

\makeatletter
\newcommand*{\whiten}[1]{\llap{\textcolor{white}{{\the\SOUL@token}}\hspace{#1pt}}}
\DeclareRobustCommand*\myul{%
    \def\SOUL@everyspace{\underline{\space}\kern\z@}%
    \def\SOUL@everytoken{%
     \setbox0=\hbox{\the\SOUL@token}%
     \ifdim\dp0>\z@
        \raisebox{\dp0}{\underline{\phantom{\the\SOUL@token}}}%
        \whiten{1}\whiten{0}%
        \whiten{-1}\whiten{-2}%
        \llap{\the\SOUL@token}%
     \else
        \underline{\the\SOUL@token}%
     \fi}%
\SOUL@}
\makeatother

\newcommand*{\demp}{\fontfamily{lmtt}\selectfont}

\DeclareTextFontCommand{\textdemp}{\demp}

\begin{document}

\ifcomment
Multiline
comment
\fi
\ifcomment
\myul{Typesetting test}
% \color[rgb]{1,1,1}
$∑_i^n≠ 60º±∞π∆¬≈√j∫h≤≥µ$

$\CR \R\pro\ind\pro\gS\pro
\mqty[a&b\\c&d]$
$\pro\mathbb{P}$
$\dd{x}$

  \[
    \alpha(x)=\left\{
                \begin{array}{ll}
                  x\\
                  \frac{1}{1+e^{-kx}}\\
                  \frac{e^x-e^{-x}}{e^x+e^{-x}}
                \end{array}
              \right.
  \]

  $\expval{x}$
  
  $\chi_\rho(ghg\dmo)=\Tr(\rho_{ghg\dmo})=\Tr(\rho_g\circ\rho_h\circ\rho\dmo_g)=\Tr(\rho_h)\overset{\mbox{\scalebox{0.5}{$\Tr(AB)=\Tr(BA)$}}}{=}\chi_\rho(h)$
  	$\mathop{\oplus}_{\substack{x\in X}}$

$\mat(\rho_g)=(a_{ij}(g))_{\scriptsize \substack{1\leq i\leq d \\ 1\leq j\leq d}}$ et $\mat(\rho'_g)=(a'_{ij}(g))_{\scriptsize \substack{1\leq i'\leq d' \\ 1\leq j'\leq d'}}$



\[\int_a^b{\mathbb{R}^2}g(u, v)\dd{P_{XY}}(u, v)=\iint g(u,v) f_{XY}(u, v)\dd \lambda(u) \dd \lambda(v)\]
$$\lim_{x\to\infty} f(x)$$	
$$\iiiint_V \mu(t,u,v,w) \,dt\,du\,dv\,dw$$
$$\sum_{n=1}^{\infty} 2^{-n} = 1$$	
\begin{definition}
	Si $X$ et $Y$ sont 2 v.a. ou definit la \textsc{Covariance} entre $X$ et $Y$ comme
	$\cov(X,Y)\overset{\text{def}}{=}\E\left[(X-\E(X))(Y-\E(Y))\right]=\E(XY)-\E(X)\E(Y)$.
\end{definition}
\fi
\pagebreak

% \tableofcontents

% insert your code here
%% !TEX encoding = UTF-8 Unicode
% !TEX TS-program = xelatex

\documentclass[french]{report}

%\usepackage[utf8]{inputenc}
%\usepackage[T1]{fontenc}
\usepackage{babel}


\newif\ifcomment
%\commenttrue # Show comments

\usepackage{physics}
\usepackage{amssymb}


\usepackage{amsthm}
% \usepackage{thmtools}
\usepackage{mathtools}
\usepackage{amsfonts}

\usepackage{color}

\usepackage{tikz}

\usepackage{geometry}
\geometry{a5paper, margin=0.1in, right=1cm}

\usepackage{dsfont}

\usepackage{graphicx}
\graphicspath{ {images/} }

\usepackage{faktor}

\usepackage{IEEEtrantools}
\usepackage{enumerate}   
\usepackage[PostScript=dvips]{"/Users/aware/Documents/Courses/diagrams"}


\newtheorem{theorem}{Théorème}[section]
\renewcommand{\thetheorem}{\arabic{theorem}}
\newtheorem{lemme}{Lemme}[section]
\renewcommand{\thelemme}{\arabic{lemme}}
\newtheorem{proposition}{Proposition}[section]
\renewcommand{\theproposition}{\arabic{proposition}}
\newtheorem{notations}{Notations}[section]
\newtheorem{problem}{Problème}[section]
\newtheorem{corollary}{Corollaire}[theorem]
\renewcommand{\thecorollary}{\arabic{corollary}}
\newtheorem{property}{Propriété}[section]
\newtheorem{objective}{Objectif}[section]

\theoremstyle{definition}
\newtheorem{definition}{Définition}[section]
\renewcommand{\thedefinition}{\arabic{definition}}
\newtheorem{exercise}{Exercice}[chapter]
\renewcommand{\theexercise}{\arabic{exercise}}
\newtheorem{example}{Exemple}[chapter]
\renewcommand{\theexample}{\arabic{example}}
\newtheorem*{solution}{Solution}
\newtheorem*{application}{Application}
\newtheorem*{notation}{Notation}
\newtheorem*{vocabulary}{Vocabulaire}
\newtheorem*{properties}{Propriétés}



\theoremstyle{remark}
\newtheorem*{remark}{Remarque}
\newtheorem*{rappel}{Rappel}


\usepackage{etoolbox}
\AtBeginEnvironment{exercise}{\small}
\AtBeginEnvironment{example}{\small}

\usepackage{cases}
\usepackage[red]{mypack}

\usepackage[framemethod=TikZ]{mdframed}

\definecolor{bg}{rgb}{0.4,0.25,0.95}
\definecolor{pagebg}{rgb}{0,0,0.5}
\surroundwithmdframed[
   topline=false,
   rightline=false,
   bottomline=false,
   leftmargin=\parindent,
   skipabove=8pt,
   skipbelow=8pt,
   linecolor=blue,
   innerbottommargin=10pt,
   % backgroundcolor=bg,font=\color{orange}\sffamily, fontcolor=white
]{definition}

\usepackage{empheq}
\usepackage[most]{tcolorbox}

\newtcbox{\mymath}[1][]{%
    nobeforeafter, math upper, tcbox raise base,
    enhanced, colframe=blue!30!black,
    colback=red!10, boxrule=1pt,
    #1}

\usepackage{unixode}


\DeclareMathOperator{\ord}{ord}
\DeclareMathOperator{\orb}{orb}
\DeclareMathOperator{\stab}{stab}
\DeclareMathOperator{\Stab}{stab}
\DeclareMathOperator{\ppcm}{ppcm}
\DeclareMathOperator{\conj}{Conj}
\DeclareMathOperator{\End}{End}
\DeclareMathOperator{\rot}{rot}
\DeclareMathOperator{\trs}{trace}
\DeclareMathOperator{\Ind}{Ind}
\DeclareMathOperator{\mat}{Mat}
\DeclareMathOperator{\id}{Id}
\DeclareMathOperator{\vect}{vect}
\DeclareMathOperator{\img}{img}
\DeclareMathOperator{\cov}{Cov}
\DeclareMathOperator{\dist}{dist}
\DeclareMathOperator{\irr}{Irr}
\DeclareMathOperator{\image}{Im}
\DeclareMathOperator{\pd}{\partial}
\DeclareMathOperator{\epi}{epi}
\DeclareMathOperator{\Argmin}{Argmin}
\DeclareMathOperator{\dom}{dom}
\DeclareMathOperator{\proj}{proj}
\DeclareMathOperator{\ctg}{ctg}
\DeclareMathOperator{\supp}{supp}
\DeclareMathOperator{\argmin}{argmin}
\DeclareMathOperator{\mult}{mult}
\DeclareMathOperator{\ch}{ch}
\DeclareMathOperator{\sh}{sh}
\DeclareMathOperator{\rang}{rang}
\DeclareMathOperator{\diam}{diam}
\DeclareMathOperator{\Epigraphe}{Epigraphe}




\usepackage{xcolor}
\everymath{\color{blue}}
%\everymath{\color[rgb]{0,1,1}}
%\pagecolor[rgb]{0,0,0.5}


\newcommand*{\pdtest}[3][]{\ensuremath{\frac{\partial^{#1} #2}{\partial #3}}}

\newcommand*{\deffunc}[6][]{\ensuremath{
\begin{array}{rcl}
#2 : #3 &\rightarrow& #4\\
#5 &\mapsto& #6
\end{array}
}}

\newcommand{\eqcolon}{\mathrel{\resizebox{\widthof{$\mathord{=}$}}{\height}{ $\!\!=\!\!\resizebox{1.2\width}{0.8\height}{\raisebox{0.23ex}{$\mathop{:}$}}\!\!$ }}}
\newcommand{\coloneq}{\mathrel{\resizebox{\widthof{$\mathord{=}$}}{\height}{ $\!\!\resizebox{1.2\width}{0.8\height}{\raisebox{0.23ex}{$\mathop{:}$}}\!\!=\!\!$ }}}
\newcommand{\eqcolonl}{\ensuremath{\mathrel{=\!\!\mathop{:}}}}
\newcommand{\coloneql}{\ensuremath{\mathrel{\mathop{:} \!\! =}}}
\newcommand{\vc}[1]{% inline column vector
  \left(\begin{smallmatrix}#1\end{smallmatrix}\right)%
}
\newcommand{\vr}[1]{% inline row vector
  \begin{smallmatrix}(\,#1\,)\end{smallmatrix}%
}
\makeatletter
\newcommand*{\defeq}{\ =\mathrel{\rlap{%
                     \raisebox{0.3ex}{$\m@th\cdot$}}%
                     \raisebox{-0.3ex}{$\m@th\cdot$}}%
                     }
\makeatother

\newcommand{\mathcircle}[1]{% inline row vector
 \overset{\circ}{#1}
}
\newcommand{\ulim}{% low limit
 \underline{\lim}
}
\newcommand{\ssi}{% iff
\iff
}
\newcommand{\ps}[2]{
\expval{#1 | #2}
}
\newcommand{\df}[1]{
\mqty{#1}
}
\newcommand{\n}[1]{
\norm{#1}
}
\newcommand{\sys}[1]{
\left\{\smqty{#1}\right.
}


\newcommand{\eqdef}{\ensuremath{\overset{\text{def}}=}}


\def\Circlearrowright{\ensuremath{%
  \rotatebox[origin=c]{230}{$\circlearrowright$}}}

\newcommand\ct[1]{\text{\rmfamily\upshape #1}}
\newcommand\question[1]{ {\color{red} ...!? \small #1}}
\newcommand\caz[1]{\left\{\begin{array} #1 \end{array}\right.}
\newcommand\const{\text{\rmfamily\upshape const}}
\newcommand\toP{ \overset{\pro}{\to}}
\newcommand\toPP{ \overset{\text{PP}}{\to}}
\newcommand{\oeq}{\mathrel{\text{\textcircled{$=$}}}}





\usepackage{xcolor}
% \usepackage[normalem]{ulem}
\usepackage{lipsum}
\makeatletter
% \newcommand\colorwave[1][blue]{\bgroup \markoverwith{\lower3.5\p@\hbox{\sixly \textcolor{#1}{\char58}}}\ULon}
%\font\sixly=lasy6 % does not re-load if already loaded, so no memory problem.

\newmdtheoremenv[
linewidth= 1pt,linecolor= blue,%
leftmargin=20,rightmargin=20,innertopmargin=0pt, innerrightmargin=40,%
tikzsetting = { draw=lightgray, line width = 0.3pt,dashed,%
dash pattern = on 15pt off 3pt},%
splittopskip=\topskip,skipbelow=\baselineskip,%
skipabove=\baselineskip,ntheorem,roundcorner=0pt,
% backgroundcolor=pagebg,font=\color{orange}\sffamily, fontcolor=white
]{examplebox}{Exemple}[section]



\newcommand\R{\mathbb{R}}
\newcommand\Z{\mathbb{Z}}
\newcommand\N{\mathbb{N}}
\newcommand\E{\mathbb{E}}
\newcommand\F{\mathcal{F}}
\newcommand\cH{\mathcal{H}}
\newcommand\V{\mathbb{V}}
\newcommand\dmo{ ^{-1} }
\newcommand\kapa{\kappa}
\newcommand\im{Im}
\newcommand\hs{\mathcal{H}}





\usepackage{soul}

\makeatletter
\newcommand*{\whiten}[1]{\llap{\textcolor{white}{{\the\SOUL@token}}\hspace{#1pt}}}
\DeclareRobustCommand*\myul{%
    \def\SOUL@everyspace{\underline{\space}\kern\z@}%
    \def\SOUL@everytoken{%
     \setbox0=\hbox{\the\SOUL@token}%
     \ifdim\dp0>\z@
        \raisebox{\dp0}{\underline{\phantom{\the\SOUL@token}}}%
        \whiten{1}\whiten{0}%
        \whiten{-1}\whiten{-2}%
        \llap{\the\SOUL@token}%
     \else
        \underline{\the\SOUL@token}%
     \fi}%
\SOUL@}
\makeatother

\newcommand*{\demp}{\fontfamily{lmtt}\selectfont}

\DeclareTextFontCommand{\textdemp}{\demp}

\begin{document}

\ifcomment
Multiline
comment
\fi
\ifcomment
\myul{Typesetting test}
% \color[rgb]{1,1,1}
$∑_i^n≠ 60º±∞π∆¬≈√j∫h≤≥µ$

$\CR \R\pro\ind\pro\gS\pro
\mqty[a&b\\c&d]$
$\pro\mathbb{P}$
$\dd{x}$

  \[
    \alpha(x)=\left\{
                \begin{array}{ll}
                  x\\
                  \frac{1}{1+e^{-kx}}\\
                  \frac{e^x-e^{-x}}{e^x+e^{-x}}
                \end{array}
              \right.
  \]

  $\expval{x}$
  
  $\chi_\rho(ghg\dmo)=\Tr(\rho_{ghg\dmo})=\Tr(\rho_g\circ\rho_h\circ\rho\dmo_g)=\Tr(\rho_h)\overset{\mbox{\scalebox{0.5}{$\Tr(AB)=\Tr(BA)$}}}{=}\chi_\rho(h)$
  	$\mathop{\oplus}_{\substack{x\in X}}$

$\mat(\rho_g)=(a_{ij}(g))_{\scriptsize \substack{1\leq i\leq d \\ 1\leq j\leq d}}$ et $\mat(\rho'_g)=(a'_{ij}(g))_{\scriptsize \substack{1\leq i'\leq d' \\ 1\leq j'\leq d'}}$



\[\int_a^b{\mathbb{R}^2}g(u, v)\dd{P_{XY}}(u, v)=\iint g(u,v) f_{XY}(u, v)\dd \lambda(u) \dd \lambda(v)\]
$$\lim_{x\to\infty} f(x)$$	
$$\iiiint_V \mu(t,u,v,w) \,dt\,du\,dv\,dw$$
$$\sum_{n=1}^{\infty} 2^{-n} = 1$$	
\begin{definition}
	Si $X$ et $Y$ sont 2 v.a. ou definit la \textsc{Covariance} entre $X$ et $Y$ comme
	$\cov(X,Y)\overset{\text{def}}{=}\E\left[(X-\E(X))(Y-\E(Y))\right]=\E(XY)-\E(X)\E(Y)$.
\end{definition}
\fi
\pagebreak

% \tableofcontents

% insert your code here
%\input{./algebra/main.tex}
%\input{./geometrie-differentielle/main.tex}
%\input{./probabilite/main.tex}
%\input{./analyse-fonctionnelle/main.tex}
% \input{./Analyse-convexe-et-dualite-en-optimisation/main.tex}
%\input{./tikz/main.tex}
%\input{./Theorie-du-distributions/main.tex}
%\input{./optimisation/mine.tex}
 \input{./modelisation/main.tex}

% yves.aubry@univ-tln.fr : algebra

\end{document}

%% !TEX encoding = UTF-8 Unicode
% !TEX TS-program = xelatex

\documentclass[french]{report}

%\usepackage[utf8]{inputenc}
%\usepackage[T1]{fontenc}
\usepackage{babel}


\newif\ifcomment
%\commenttrue # Show comments

\usepackage{physics}
\usepackage{amssymb}


\usepackage{amsthm}
% \usepackage{thmtools}
\usepackage{mathtools}
\usepackage{amsfonts}

\usepackage{color}

\usepackage{tikz}

\usepackage{geometry}
\geometry{a5paper, margin=0.1in, right=1cm}

\usepackage{dsfont}

\usepackage{graphicx}
\graphicspath{ {images/} }

\usepackage{faktor}

\usepackage{IEEEtrantools}
\usepackage{enumerate}   
\usepackage[PostScript=dvips]{"/Users/aware/Documents/Courses/diagrams"}


\newtheorem{theorem}{Théorème}[section]
\renewcommand{\thetheorem}{\arabic{theorem}}
\newtheorem{lemme}{Lemme}[section]
\renewcommand{\thelemme}{\arabic{lemme}}
\newtheorem{proposition}{Proposition}[section]
\renewcommand{\theproposition}{\arabic{proposition}}
\newtheorem{notations}{Notations}[section]
\newtheorem{problem}{Problème}[section]
\newtheorem{corollary}{Corollaire}[theorem]
\renewcommand{\thecorollary}{\arabic{corollary}}
\newtheorem{property}{Propriété}[section]
\newtheorem{objective}{Objectif}[section]

\theoremstyle{definition}
\newtheorem{definition}{Définition}[section]
\renewcommand{\thedefinition}{\arabic{definition}}
\newtheorem{exercise}{Exercice}[chapter]
\renewcommand{\theexercise}{\arabic{exercise}}
\newtheorem{example}{Exemple}[chapter]
\renewcommand{\theexample}{\arabic{example}}
\newtheorem*{solution}{Solution}
\newtheorem*{application}{Application}
\newtheorem*{notation}{Notation}
\newtheorem*{vocabulary}{Vocabulaire}
\newtheorem*{properties}{Propriétés}



\theoremstyle{remark}
\newtheorem*{remark}{Remarque}
\newtheorem*{rappel}{Rappel}


\usepackage{etoolbox}
\AtBeginEnvironment{exercise}{\small}
\AtBeginEnvironment{example}{\small}

\usepackage{cases}
\usepackage[red]{mypack}

\usepackage[framemethod=TikZ]{mdframed}

\definecolor{bg}{rgb}{0.4,0.25,0.95}
\definecolor{pagebg}{rgb}{0,0,0.5}
\surroundwithmdframed[
   topline=false,
   rightline=false,
   bottomline=false,
   leftmargin=\parindent,
   skipabove=8pt,
   skipbelow=8pt,
   linecolor=blue,
   innerbottommargin=10pt,
   % backgroundcolor=bg,font=\color{orange}\sffamily, fontcolor=white
]{definition}

\usepackage{empheq}
\usepackage[most]{tcolorbox}

\newtcbox{\mymath}[1][]{%
    nobeforeafter, math upper, tcbox raise base,
    enhanced, colframe=blue!30!black,
    colback=red!10, boxrule=1pt,
    #1}

\usepackage{unixode}


\DeclareMathOperator{\ord}{ord}
\DeclareMathOperator{\orb}{orb}
\DeclareMathOperator{\stab}{stab}
\DeclareMathOperator{\Stab}{stab}
\DeclareMathOperator{\ppcm}{ppcm}
\DeclareMathOperator{\conj}{Conj}
\DeclareMathOperator{\End}{End}
\DeclareMathOperator{\rot}{rot}
\DeclareMathOperator{\trs}{trace}
\DeclareMathOperator{\Ind}{Ind}
\DeclareMathOperator{\mat}{Mat}
\DeclareMathOperator{\id}{Id}
\DeclareMathOperator{\vect}{vect}
\DeclareMathOperator{\img}{img}
\DeclareMathOperator{\cov}{Cov}
\DeclareMathOperator{\dist}{dist}
\DeclareMathOperator{\irr}{Irr}
\DeclareMathOperator{\image}{Im}
\DeclareMathOperator{\pd}{\partial}
\DeclareMathOperator{\epi}{epi}
\DeclareMathOperator{\Argmin}{Argmin}
\DeclareMathOperator{\dom}{dom}
\DeclareMathOperator{\proj}{proj}
\DeclareMathOperator{\ctg}{ctg}
\DeclareMathOperator{\supp}{supp}
\DeclareMathOperator{\argmin}{argmin}
\DeclareMathOperator{\mult}{mult}
\DeclareMathOperator{\ch}{ch}
\DeclareMathOperator{\sh}{sh}
\DeclareMathOperator{\rang}{rang}
\DeclareMathOperator{\diam}{diam}
\DeclareMathOperator{\Epigraphe}{Epigraphe}




\usepackage{xcolor}
\everymath{\color{blue}}
%\everymath{\color[rgb]{0,1,1}}
%\pagecolor[rgb]{0,0,0.5}


\newcommand*{\pdtest}[3][]{\ensuremath{\frac{\partial^{#1} #2}{\partial #3}}}

\newcommand*{\deffunc}[6][]{\ensuremath{
\begin{array}{rcl}
#2 : #3 &\rightarrow& #4\\
#5 &\mapsto& #6
\end{array}
}}

\newcommand{\eqcolon}{\mathrel{\resizebox{\widthof{$\mathord{=}$}}{\height}{ $\!\!=\!\!\resizebox{1.2\width}{0.8\height}{\raisebox{0.23ex}{$\mathop{:}$}}\!\!$ }}}
\newcommand{\coloneq}{\mathrel{\resizebox{\widthof{$\mathord{=}$}}{\height}{ $\!\!\resizebox{1.2\width}{0.8\height}{\raisebox{0.23ex}{$\mathop{:}$}}\!\!=\!\!$ }}}
\newcommand{\eqcolonl}{\ensuremath{\mathrel{=\!\!\mathop{:}}}}
\newcommand{\coloneql}{\ensuremath{\mathrel{\mathop{:} \!\! =}}}
\newcommand{\vc}[1]{% inline column vector
  \left(\begin{smallmatrix}#1\end{smallmatrix}\right)%
}
\newcommand{\vr}[1]{% inline row vector
  \begin{smallmatrix}(\,#1\,)\end{smallmatrix}%
}
\makeatletter
\newcommand*{\defeq}{\ =\mathrel{\rlap{%
                     \raisebox{0.3ex}{$\m@th\cdot$}}%
                     \raisebox{-0.3ex}{$\m@th\cdot$}}%
                     }
\makeatother

\newcommand{\mathcircle}[1]{% inline row vector
 \overset{\circ}{#1}
}
\newcommand{\ulim}{% low limit
 \underline{\lim}
}
\newcommand{\ssi}{% iff
\iff
}
\newcommand{\ps}[2]{
\expval{#1 | #2}
}
\newcommand{\df}[1]{
\mqty{#1}
}
\newcommand{\n}[1]{
\norm{#1}
}
\newcommand{\sys}[1]{
\left\{\smqty{#1}\right.
}


\newcommand{\eqdef}{\ensuremath{\overset{\text{def}}=}}


\def\Circlearrowright{\ensuremath{%
  \rotatebox[origin=c]{230}{$\circlearrowright$}}}

\newcommand\ct[1]{\text{\rmfamily\upshape #1}}
\newcommand\question[1]{ {\color{red} ...!? \small #1}}
\newcommand\caz[1]{\left\{\begin{array} #1 \end{array}\right.}
\newcommand\const{\text{\rmfamily\upshape const}}
\newcommand\toP{ \overset{\pro}{\to}}
\newcommand\toPP{ \overset{\text{PP}}{\to}}
\newcommand{\oeq}{\mathrel{\text{\textcircled{$=$}}}}





\usepackage{xcolor}
% \usepackage[normalem]{ulem}
\usepackage{lipsum}
\makeatletter
% \newcommand\colorwave[1][blue]{\bgroup \markoverwith{\lower3.5\p@\hbox{\sixly \textcolor{#1}{\char58}}}\ULon}
%\font\sixly=lasy6 % does not re-load if already loaded, so no memory problem.

\newmdtheoremenv[
linewidth= 1pt,linecolor= blue,%
leftmargin=20,rightmargin=20,innertopmargin=0pt, innerrightmargin=40,%
tikzsetting = { draw=lightgray, line width = 0.3pt,dashed,%
dash pattern = on 15pt off 3pt},%
splittopskip=\topskip,skipbelow=\baselineskip,%
skipabove=\baselineskip,ntheorem,roundcorner=0pt,
% backgroundcolor=pagebg,font=\color{orange}\sffamily, fontcolor=white
]{examplebox}{Exemple}[section]



\newcommand\R{\mathbb{R}}
\newcommand\Z{\mathbb{Z}}
\newcommand\N{\mathbb{N}}
\newcommand\E{\mathbb{E}}
\newcommand\F{\mathcal{F}}
\newcommand\cH{\mathcal{H}}
\newcommand\V{\mathbb{V}}
\newcommand\dmo{ ^{-1} }
\newcommand\kapa{\kappa}
\newcommand\im{Im}
\newcommand\hs{\mathcal{H}}





\usepackage{soul}

\makeatletter
\newcommand*{\whiten}[1]{\llap{\textcolor{white}{{\the\SOUL@token}}\hspace{#1pt}}}
\DeclareRobustCommand*\myul{%
    \def\SOUL@everyspace{\underline{\space}\kern\z@}%
    \def\SOUL@everytoken{%
     \setbox0=\hbox{\the\SOUL@token}%
     \ifdim\dp0>\z@
        \raisebox{\dp0}{\underline{\phantom{\the\SOUL@token}}}%
        \whiten{1}\whiten{0}%
        \whiten{-1}\whiten{-2}%
        \llap{\the\SOUL@token}%
     \else
        \underline{\the\SOUL@token}%
     \fi}%
\SOUL@}
\makeatother

\newcommand*{\demp}{\fontfamily{lmtt}\selectfont}

\DeclareTextFontCommand{\textdemp}{\demp}

\begin{document}

\ifcomment
Multiline
comment
\fi
\ifcomment
\myul{Typesetting test}
% \color[rgb]{1,1,1}
$∑_i^n≠ 60º±∞π∆¬≈√j∫h≤≥µ$

$\CR \R\pro\ind\pro\gS\pro
\mqty[a&b\\c&d]$
$\pro\mathbb{P}$
$\dd{x}$

  \[
    \alpha(x)=\left\{
                \begin{array}{ll}
                  x\\
                  \frac{1}{1+e^{-kx}}\\
                  \frac{e^x-e^{-x}}{e^x+e^{-x}}
                \end{array}
              \right.
  \]

  $\expval{x}$
  
  $\chi_\rho(ghg\dmo)=\Tr(\rho_{ghg\dmo})=\Tr(\rho_g\circ\rho_h\circ\rho\dmo_g)=\Tr(\rho_h)\overset{\mbox{\scalebox{0.5}{$\Tr(AB)=\Tr(BA)$}}}{=}\chi_\rho(h)$
  	$\mathop{\oplus}_{\substack{x\in X}}$

$\mat(\rho_g)=(a_{ij}(g))_{\scriptsize \substack{1\leq i\leq d \\ 1\leq j\leq d}}$ et $\mat(\rho'_g)=(a'_{ij}(g))_{\scriptsize \substack{1\leq i'\leq d' \\ 1\leq j'\leq d'}}$



\[\int_a^b{\mathbb{R}^2}g(u, v)\dd{P_{XY}}(u, v)=\iint g(u,v) f_{XY}(u, v)\dd \lambda(u) \dd \lambda(v)\]
$$\lim_{x\to\infty} f(x)$$	
$$\iiiint_V \mu(t,u,v,w) \,dt\,du\,dv\,dw$$
$$\sum_{n=1}^{\infty} 2^{-n} = 1$$	
\begin{definition}
	Si $X$ et $Y$ sont 2 v.a. ou definit la \textsc{Covariance} entre $X$ et $Y$ comme
	$\cov(X,Y)\overset{\text{def}}{=}\E\left[(X-\E(X))(Y-\E(Y))\right]=\E(XY)-\E(X)\E(Y)$.
\end{definition}
\fi
\pagebreak

% \tableofcontents

% insert your code here
%\input{./algebra/main.tex}
%\input{./geometrie-differentielle/main.tex}
%\input{./probabilite/main.tex}
%\input{./analyse-fonctionnelle/main.tex}
% \input{./Analyse-convexe-et-dualite-en-optimisation/main.tex}
%\input{./tikz/main.tex}
%\input{./Theorie-du-distributions/main.tex}
%\input{./optimisation/mine.tex}
 \input{./modelisation/main.tex}

% yves.aubry@univ-tln.fr : algebra

\end{document}

%% !TEX encoding = UTF-8 Unicode
% !TEX TS-program = xelatex

\documentclass[french]{report}

%\usepackage[utf8]{inputenc}
%\usepackage[T1]{fontenc}
\usepackage{babel}


\newif\ifcomment
%\commenttrue # Show comments

\usepackage{physics}
\usepackage{amssymb}


\usepackage{amsthm}
% \usepackage{thmtools}
\usepackage{mathtools}
\usepackage{amsfonts}

\usepackage{color}

\usepackage{tikz}

\usepackage{geometry}
\geometry{a5paper, margin=0.1in, right=1cm}

\usepackage{dsfont}

\usepackage{graphicx}
\graphicspath{ {images/} }

\usepackage{faktor}

\usepackage{IEEEtrantools}
\usepackage{enumerate}   
\usepackage[PostScript=dvips]{"/Users/aware/Documents/Courses/diagrams"}


\newtheorem{theorem}{Théorème}[section]
\renewcommand{\thetheorem}{\arabic{theorem}}
\newtheorem{lemme}{Lemme}[section]
\renewcommand{\thelemme}{\arabic{lemme}}
\newtheorem{proposition}{Proposition}[section]
\renewcommand{\theproposition}{\arabic{proposition}}
\newtheorem{notations}{Notations}[section]
\newtheorem{problem}{Problème}[section]
\newtheorem{corollary}{Corollaire}[theorem]
\renewcommand{\thecorollary}{\arabic{corollary}}
\newtheorem{property}{Propriété}[section]
\newtheorem{objective}{Objectif}[section]

\theoremstyle{definition}
\newtheorem{definition}{Définition}[section]
\renewcommand{\thedefinition}{\arabic{definition}}
\newtheorem{exercise}{Exercice}[chapter]
\renewcommand{\theexercise}{\arabic{exercise}}
\newtheorem{example}{Exemple}[chapter]
\renewcommand{\theexample}{\arabic{example}}
\newtheorem*{solution}{Solution}
\newtheorem*{application}{Application}
\newtheorem*{notation}{Notation}
\newtheorem*{vocabulary}{Vocabulaire}
\newtheorem*{properties}{Propriétés}



\theoremstyle{remark}
\newtheorem*{remark}{Remarque}
\newtheorem*{rappel}{Rappel}


\usepackage{etoolbox}
\AtBeginEnvironment{exercise}{\small}
\AtBeginEnvironment{example}{\small}

\usepackage{cases}
\usepackage[red]{mypack}

\usepackage[framemethod=TikZ]{mdframed}

\definecolor{bg}{rgb}{0.4,0.25,0.95}
\definecolor{pagebg}{rgb}{0,0,0.5}
\surroundwithmdframed[
   topline=false,
   rightline=false,
   bottomline=false,
   leftmargin=\parindent,
   skipabove=8pt,
   skipbelow=8pt,
   linecolor=blue,
   innerbottommargin=10pt,
   % backgroundcolor=bg,font=\color{orange}\sffamily, fontcolor=white
]{definition}

\usepackage{empheq}
\usepackage[most]{tcolorbox}

\newtcbox{\mymath}[1][]{%
    nobeforeafter, math upper, tcbox raise base,
    enhanced, colframe=blue!30!black,
    colback=red!10, boxrule=1pt,
    #1}

\usepackage{unixode}


\DeclareMathOperator{\ord}{ord}
\DeclareMathOperator{\orb}{orb}
\DeclareMathOperator{\stab}{stab}
\DeclareMathOperator{\Stab}{stab}
\DeclareMathOperator{\ppcm}{ppcm}
\DeclareMathOperator{\conj}{Conj}
\DeclareMathOperator{\End}{End}
\DeclareMathOperator{\rot}{rot}
\DeclareMathOperator{\trs}{trace}
\DeclareMathOperator{\Ind}{Ind}
\DeclareMathOperator{\mat}{Mat}
\DeclareMathOperator{\id}{Id}
\DeclareMathOperator{\vect}{vect}
\DeclareMathOperator{\img}{img}
\DeclareMathOperator{\cov}{Cov}
\DeclareMathOperator{\dist}{dist}
\DeclareMathOperator{\irr}{Irr}
\DeclareMathOperator{\image}{Im}
\DeclareMathOperator{\pd}{\partial}
\DeclareMathOperator{\epi}{epi}
\DeclareMathOperator{\Argmin}{Argmin}
\DeclareMathOperator{\dom}{dom}
\DeclareMathOperator{\proj}{proj}
\DeclareMathOperator{\ctg}{ctg}
\DeclareMathOperator{\supp}{supp}
\DeclareMathOperator{\argmin}{argmin}
\DeclareMathOperator{\mult}{mult}
\DeclareMathOperator{\ch}{ch}
\DeclareMathOperator{\sh}{sh}
\DeclareMathOperator{\rang}{rang}
\DeclareMathOperator{\diam}{diam}
\DeclareMathOperator{\Epigraphe}{Epigraphe}




\usepackage{xcolor}
\everymath{\color{blue}}
%\everymath{\color[rgb]{0,1,1}}
%\pagecolor[rgb]{0,0,0.5}


\newcommand*{\pdtest}[3][]{\ensuremath{\frac{\partial^{#1} #2}{\partial #3}}}

\newcommand*{\deffunc}[6][]{\ensuremath{
\begin{array}{rcl}
#2 : #3 &\rightarrow& #4\\
#5 &\mapsto& #6
\end{array}
}}

\newcommand{\eqcolon}{\mathrel{\resizebox{\widthof{$\mathord{=}$}}{\height}{ $\!\!=\!\!\resizebox{1.2\width}{0.8\height}{\raisebox{0.23ex}{$\mathop{:}$}}\!\!$ }}}
\newcommand{\coloneq}{\mathrel{\resizebox{\widthof{$\mathord{=}$}}{\height}{ $\!\!\resizebox{1.2\width}{0.8\height}{\raisebox{0.23ex}{$\mathop{:}$}}\!\!=\!\!$ }}}
\newcommand{\eqcolonl}{\ensuremath{\mathrel{=\!\!\mathop{:}}}}
\newcommand{\coloneql}{\ensuremath{\mathrel{\mathop{:} \!\! =}}}
\newcommand{\vc}[1]{% inline column vector
  \left(\begin{smallmatrix}#1\end{smallmatrix}\right)%
}
\newcommand{\vr}[1]{% inline row vector
  \begin{smallmatrix}(\,#1\,)\end{smallmatrix}%
}
\makeatletter
\newcommand*{\defeq}{\ =\mathrel{\rlap{%
                     \raisebox{0.3ex}{$\m@th\cdot$}}%
                     \raisebox{-0.3ex}{$\m@th\cdot$}}%
                     }
\makeatother

\newcommand{\mathcircle}[1]{% inline row vector
 \overset{\circ}{#1}
}
\newcommand{\ulim}{% low limit
 \underline{\lim}
}
\newcommand{\ssi}{% iff
\iff
}
\newcommand{\ps}[2]{
\expval{#1 | #2}
}
\newcommand{\df}[1]{
\mqty{#1}
}
\newcommand{\n}[1]{
\norm{#1}
}
\newcommand{\sys}[1]{
\left\{\smqty{#1}\right.
}


\newcommand{\eqdef}{\ensuremath{\overset{\text{def}}=}}


\def\Circlearrowright{\ensuremath{%
  \rotatebox[origin=c]{230}{$\circlearrowright$}}}

\newcommand\ct[1]{\text{\rmfamily\upshape #1}}
\newcommand\question[1]{ {\color{red} ...!? \small #1}}
\newcommand\caz[1]{\left\{\begin{array} #1 \end{array}\right.}
\newcommand\const{\text{\rmfamily\upshape const}}
\newcommand\toP{ \overset{\pro}{\to}}
\newcommand\toPP{ \overset{\text{PP}}{\to}}
\newcommand{\oeq}{\mathrel{\text{\textcircled{$=$}}}}





\usepackage{xcolor}
% \usepackage[normalem]{ulem}
\usepackage{lipsum}
\makeatletter
% \newcommand\colorwave[1][blue]{\bgroup \markoverwith{\lower3.5\p@\hbox{\sixly \textcolor{#1}{\char58}}}\ULon}
%\font\sixly=lasy6 % does not re-load if already loaded, so no memory problem.

\newmdtheoremenv[
linewidth= 1pt,linecolor= blue,%
leftmargin=20,rightmargin=20,innertopmargin=0pt, innerrightmargin=40,%
tikzsetting = { draw=lightgray, line width = 0.3pt,dashed,%
dash pattern = on 15pt off 3pt},%
splittopskip=\topskip,skipbelow=\baselineskip,%
skipabove=\baselineskip,ntheorem,roundcorner=0pt,
% backgroundcolor=pagebg,font=\color{orange}\sffamily, fontcolor=white
]{examplebox}{Exemple}[section]



\newcommand\R{\mathbb{R}}
\newcommand\Z{\mathbb{Z}}
\newcommand\N{\mathbb{N}}
\newcommand\E{\mathbb{E}}
\newcommand\F{\mathcal{F}}
\newcommand\cH{\mathcal{H}}
\newcommand\V{\mathbb{V}}
\newcommand\dmo{ ^{-1} }
\newcommand\kapa{\kappa}
\newcommand\im{Im}
\newcommand\hs{\mathcal{H}}





\usepackage{soul}

\makeatletter
\newcommand*{\whiten}[1]{\llap{\textcolor{white}{{\the\SOUL@token}}\hspace{#1pt}}}
\DeclareRobustCommand*\myul{%
    \def\SOUL@everyspace{\underline{\space}\kern\z@}%
    \def\SOUL@everytoken{%
     \setbox0=\hbox{\the\SOUL@token}%
     \ifdim\dp0>\z@
        \raisebox{\dp0}{\underline{\phantom{\the\SOUL@token}}}%
        \whiten{1}\whiten{0}%
        \whiten{-1}\whiten{-2}%
        \llap{\the\SOUL@token}%
     \else
        \underline{\the\SOUL@token}%
     \fi}%
\SOUL@}
\makeatother

\newcommand*{\demp}{\fontfamily{lmtt}\selectfont}

\DeclareTextFontCommand{\textdemp}{\demp}

\begin{document}

\ifcomment
Multiline
comment
\fi
\ifcomment
\myul{Typesetting test}
% \color[rgb]{1,1,1}
$∑_i^n≠ 60º±∞π∆¬≈√j∫h≤≥µ$

$\CR \R\pro\ind\pro\gS\pro
\mqty[a&b\\c&d]$
$\pro\mathbb{P}$
$\dd{x}$

  \[
    \alpha(x)=\left\{
                \begin{array}{ll}
                  x\\
                  \frac{1}{1+e^{-kx}}\\
                  \frac{e^x-e^{-x}}{e^x+e^{-x}}
                \end{array}
              \right.
  \]

  $\expval{x}$
  
  $\chi_\rho(ghg\dmo)=\Tr(\rho_{ghg\dmo})=\Tr(\rho_g\circ\rho_h\circ\rho\dmo_g)=\Tr(\rho_h)\overset{\mbox{\scalebox{0.5}{$\Tr(AB)=\Tr(BA)$}}}{=}\chi_\rho(h)$
  	$\mathop{\oplus}_{\substack{x\in X}}$

$\mat(\rho_g)=(a_{ij}(g))_{\scriptsize \substack{1\leq i\leq d \\ 1\leq j\leq d}}$ et $\mat(\rho'_g)=(a'_{ij}(g))_{\scriptsize \substack{1\leq i'\leq d' \\ 1\leq j'\leq d'}}$



\[\int_a^b{\mathbb{R}^2}g(u, v)\dd{P_{XY}}(u, v)=\iint g(u,v) f_{XY}(u, v)\dd \lambda(u) \dd \lambda(v)\]
$$\lim_{x\to\infty} f(x)$$	
$$\iiiint_V \mu(t,u,v,w) \,dt\,du\,dv\,dw$$
$$\sum_{n=1}^{\infty} 2^{-n} = 1$$	
\begin{definition}
	Si $X$ et $Y$ sont 2 v.a. ou definit la \textsc{Covariance} entre $X$ et $Y$ comme
	$\cov(X,Y)\overset{\text{def}}{=}\E\left[(X-\E(X))(Y-\E(Y))\right]=\E(XY)-\E(X)\E(Y)$.
\end{definition}
\fi
\pagebreak

% \tableofcontents

% insert your code here
%\input{./algebra/main.tex}
%\input{./geometrie-differentielle/main.tex}
%\input{./probabilite/main.tex}
%\input{./analyse-fonctionnelle/main.tex}
% \input{./Analyse-convexe-et-dualite-en-optimisation/main.tex}
%\input{./tikz/main.tex}
%\input{./Theorie-du-distributions/main.tex}
%\input{./optimisation/mine.tex}
 \input{./modelisation/main.tex}

% yves.aubry@univ-tln.fr : algebra

\end{document}

%% !TEX encoding = UTF-8 Unicode
% !TEX TS-program = xelatex

\documentclass[french]{report}

%\usepackage[utf8]{inputenc}
%\usepackage[T1]{fontenc}
\usepackage{babel}


\newif\ifcomment
%\commenttrue # Show comments

\usepackage{physics}
\usepackage{amssymb}


\usepackage{amsthm}
% \usepackage{thmtools}
\usepackage{mathtools}
\usepackage{amsfonts}

\usepackage{color}

\usepackage{tikz}

\usepackage{geometry}
\geometry{a5paper, margin=0.1in, right=1cm}

\usepackage{dsfont}

\usepackage{graphicx}
\graphicspath{ {images/} }

\usepackage{faktor}

\usepackage{IEEEtrantools}
\usepackage{enumerate}   
\usepackage[PostScript=dvips]{"/Users/aware/Documents/Courses/diagrams"}


\newtheorem{theorem}{Théorème}[section]
\renewcommand{\thetheorem}{\arabic{theorem}}
\newtheorem{lemme}{Lemme}[section]
\renewcommand{\thelemme}{\arabic{lemme}}
\newtheorem{proposition}{Proposition}[section]
\renewcommand{\theproposition}{\arabic{proposition}}
\newtheorem{notations}{Notations}[section]
\newtheorem{problem}{Problème}[section]
\newtheorem{corollary}{Corollaire}[theorem]
\renewcommand{\thecorollary}{\arabic{corollary}}
\newtheorem{property}{Propriété}[section]
\newtheorem{objective}{Objectif}[section]

\theoremstyle{definition}
\newtheorem{definition}{Définition}[section]
\renewcommand{\thedefinition}{\arabic{definition}}
\newtheorem{exercise}{Exercice}[chapter]
\renewcommand{\theexercise}{\arabic{exercise}}
\newtheorem{example}{Exemple}[chapter]
\renewcommand{\theexample}{\arabic{example}}
\newtheorem*{solution}{Solution}
\newtheorem*{application}{Application}
\newtheorem*{notation}{Notation}
\newtheorem*{vocabulary}{Vocabulaire}
\newtheorem*{properties}{Propriétés}



\theoremstyle{remark}
\newtheorem*{remark}{Remarque}
\newtheorem*{rappel}{Rappel}


\usepackage{etoolbox}
\AtBeginEnvironment{exercise}{\small}
\AtBeginEnvironment{example}{\small}

\usepackage{cases}
\usepackage[red]{mypack}

\usepackage[framemethod=TikZ]{mdframed}

\definecolor{bg}{rgb}{0.4,0.25,0.95}
\definecolor{pagebg}{rgb}{0,0,0.5}
\surroundwithmdframed[
   topline=false,
   rightline=false,
   bottomline=false,
   leftmargin=\parindent,
   skipabove=8pt,
   skipbelow=8pt,
   linecolor=blue,
   innerbottommargin=10pt,
   % backgroundcolor=bg,font=\color{orange}\sffamily, fontcolor=white
]{definition}

\usepackage{empheq}
\usepackage[most]{tcolorbox}

\newtcbox{\mymath}[1][]{%
    nobeforeafter, math upper, tcbox raise base,
    enhanced, colframe=blue!30!black,
    colback=red!10, boxrule=1pt,
    #1}

\usepackage{unixode}


\DeclareMathOperator{\ord}{ord}
\DeclareMathOperator{\orb}{orb}
\DeclareMathOperator{\stab}{stab}
\DeclareMathOperator{\Stab}{stab}
\DeclareMathOperator{\ppcm}{ppcm}
\DeclareMathOperator{\conj}{Conj}
\DeclareMathOperator{\End}{End}
\DeclareMathOperator{\rot}{rot}
\DeclareMathOperator{\trs}{trace}
\DeclareMathOperator{\Ind}{Ind}
\DeclareMathOperator{\mat}{Mat}
\DeclareMathOperator{\id}{Id}
\DeclareMathOperator{\vect}{vect}
\DeclareMathOperator{\img}{img}
\DeclareMathOperator{\cov}{Cov}
\DeclareMathOperator{\dist}{dist}
\DeclareMathOperator{\irr}{Irr}
\DeclareMathOperator{\image}{Im}
\DeclareMathOperator{\pd}{\partial}
\DeclareMathOperator{\epi}{epi}
\DeclareMathOperator{\Argmin}{Argmin}
\DeclareMathOperator{\dom}{dom}
\DeclareMathOperator{\proj}{proj}
\DeclareMathOperator{\ctg}{ctg}
\DeclareMathOperator{\supp}{supp}
\DeclareMathOperator{\argmin}{argmin}
\DeclareMathOperator{\mult}{mult}
\DeclareMathOperator{\ch}{ch}
\DeclareMathOperator{\sh}{sh}
\DeclareMathOperator{\rang}{rang}
\DeclareMathOperator{\diam}{diam}
\DeclareMathOperator{\Epigraphe}{Epigraphe}




\usepackage{xcolor}
\everymath{\color{blue}}
%\everymath{\color[rgb]{0,1,1}}
%\pagecolor[rgb]{0,0,0.5}


\newcommand*{\pdtest}[3][]{\ensuremath{\frac{\partial^{#1} #2}{\partial #3}}}

\newcommand*{\deffunc}[6][]{\ensuremath{
\begin{array}{rcl}
#2 : #3 &\rightarrow& #4\\
#5 &\mapsto& #6
\end{array}
}}

\newcommand{\eqcolon}{\mathrel{\resizebox{\widthof{$\mathord{=}$}}{\height}{ $\!\!=\!\!\resizebox{1.2\width}{0.8\height}{\raisebox{0.23ex}{$\mathop{:}$}}\!\!$ }}}
\newcommand{\coloneq}{\mathrel{\resizebox{\widthof{$\mathord{=}$}}{\height}{ $\!\!\resizebox{1.2\width}{0.8\height}{\raisebox{0.23ex}{$\mathop{:}$}}\!\!=\!\!$ }}}
\newcommand{\eqcolonl}{\ensuremath{\mathrel{=\!\!\mathop{:}}}}
\newcommand{\coloneql}{\ensuremath{\mathrel{\mathop{:} \!\! =}}}
\newcommand{\vc}[1]{% inline column vector
  \left(\begin{smallmatrix}#1\end{smallmatrix}\right)%
}
\newcommand{\vr}[1]{% inline row vector
  \begin{smallmatrix}(\,#1\,)\end{smallmatrix}%
}
\makeatletter
\newcommand*{\defeq}{\ =\mathrel{\rlap{%
                     \raisebox{0.3ex}{$\m@th\cdot$}}%
                     \raisebox{-0.3ex}{$\m@th\cdot$}}%
                     }
\makeatother

\newcommand{\mathcircle}[1]{% inline row vector
 \overset{\circ}{#1}
}
\newcommand{\ulim}{% low limit
 \underline{\lim}
}
\newcommand{\ssi}{% iff
\iff
}
\newcommand{\ps}[2]{
\expval{#1 | #2}
}
\newcommand{\df}[1]{
\mqty{#1}
}
\newcommand{\n}[1]{
\norm{#1}
}
\newcommand{\sys}[1]{
\left\{\smqty{#1}\right.
}


\newcommand{\eqdef}{\ensuremath{\overset{\text{def}}=}}


\def\Circlearrowright{\ensuremath{%
  \rotatebox[origin=c]{230}{$\circlearrowright$}}}

\newcommand\ct[1]{\text{\rmfamily\upshape #1}}
\newcommand\question[1]{ {\color{red} ...!? \small #1}}
\newcommand\caz[1]{\left\{\begin{array} #1 \end{array}\right.}
\newcommand\const{\text{\rmfamily\upshape const}}
\newcommand\toP{ \overset{\pro}{\to}}
\newcommand\toPP{ \overset{\text{PP}}{\to}}
\newcommand{\oeq}{\mathrel{\text{\textcircled{$=$}}}}





\usepackage{xcolor}
% \usepackage[normalem]{ulem}
\usepackage{lipsum}
\makeatletter
% \newcommand\colorwave[1][blue]{\bgroup \markoverwith{\lower3.5\p@\hbox{\sixly \textcolor{#1}{\char58}}}\ULon}
%\font\sixly=lasy6 % does not re-load if already loaded, so no memory problem.

\newmdtheoremenv[
linewidth= 1pt,linecolor= blue,%
leftmargin=20,rightmargin=20,innertopmargin=0pt, innerrightmargin=40,%
tikzsetting = { draw=lightgray, line width = 0.3pt,dashed,%
dash pattern = on 15pt off 3pt},%
splittopskip=\topskip,skipbelow=\baselineskip,%
skipabove=\baselineskip,ntheorem,roundcorner=0pt,
% backgroundcolor=pagebg,font=\color{orange}\sffamily, fontcolor=white
]{examplebox}{Exemple}[section]



\newcommand\R{\mathbb{R}}
\newcommand\Z{\mathbb{Z}}
\newcommand\N{\mathbb{N}}
\newcommand\E{\mathbb{E}}
\newcommand\F{\mathcal{F}}
\newcommand\cH{\mathcal{H}}
\newcommand\V{\mathbb{V}}
\newcommand\dmo{ ^{-1} }
\newcommand\kapa{\kappa}
\newcommand\im{Im}
\newcommand\hs{\mathcal{H}}





\usepackage{soul}

\makeatletter
\newcommand*{\whiten}[1]{\llap{\textcolor{white}{{\the\SOUL@token}}\hspace{#1pt}}}
\DeclareRobustCommand*\myul{%
    \def\SOUL@everyspace{\underline{\space}\kern\z@}%
    \def\SOUL@everytoken{%
     \setbox0=\hbox{\the\SOUL@token}%
     \ifdim\dp0>\z@
        \raisebox{\dp0}{\underline{\phantom{\the\SOUL@token}}}%
        \whiten{1}\whiten{0}%
        \whiten{-1}\whiten{-2}%
        \llap{\the\SOUL@token}%
     \else
        \underline{\the\SOUL@token}%
     \fi}%
\SOUL@}
\makeatother

\newcommand*{\demp}{\fontfamily{lmtt}\selectfont}

\DeclareTextFontCommand{\textdemp}{\demp}

\begin{document}

\ifcomment
Multiline
comment
\fi
\ifcomment
\myul{Typesetting test}
% \color[rgb]{1,1,1}
$∑_i^n≠ 60º±∞π∆¬≈√j∫h≤≥µ$

$\CR \R\pro\ind\pro\gS\pro
\mqty[a&b\\c&d]$
$\pro\mathbb{P}$
$\dd{x}$

  \[
    \alpha(x)=\left\{
                \begin{array}{ll}
                  x\\
                  \frac{1}{1+e^{-kx}}\\
                  \frac{e^x-e^{-x}}{e^x+e^{-x}}
                \end{array}
              \right.
  \]

  $\expval{x}$
  
  $\chi_\rho(ghg\dmo)=\Tr(\rho_{ghg\dmo})=\Tr(\rho_g\circ\rho_h\circ\rho\dmo_g)=\Tr(\rho_h)\overset{\mbox{\scalebox{0.5}{$\Tr(AB)=\Tr(BA)$}}}{=}\chi_\rho(h)$
  	$\mathop{\oplus}_{\substack{x\in X}}$

$\mat(\rho_g)=(a_{ij}(g))_{\scriptsize \substack{1\leq i\leq d \\ 1\leq j\leq d}}$ et $\mat(\rho'_g)=(a'_{ij}(g))_{\scriptsize \substack{1\leq i'\leq d' \\ 1\leq j'\leq d'}}$



\[\int_a^b{\mathbb{R}^2}g(u, v)\dd{P_{XY}}(u, v)=\iint g(u,v) f_{XY}(u, v)\dd \lambda(u) \dd \lambda(v)\]
$$\lim_{x\to\infty} f(x)$$	
$$\iiiint_V \mu(t,u,v,w) \,dt\,du\,dv\,dw$$
$$\sum_{n=1}^{\infty} 2^{-n} = 1$$	
\begin{definition}
	Si $X$ et $Y$ sont 2 v.a. ou definit la \textsc{Covariance} entre $X$ et $Y$ comme
	$\cov(X,Y)\overset{\text{def}}{=}\E\left[(X-\E(X))(Y-\E(Y))\right]=\E(XY)-\E(X)\E(Y)$.
\end{definition}
\fi
\pagebreak

% \tableofcontents

% insert your code here
%\input{./algebra/main.tex}
%\input{./geometrie-differentielle/main.tex}
%\input{./probabilite/main.tex}
%\input{./analyse-fonctionnelle/main.tex}
% \input{./Analyse-convexe-et-dualite-en-optimisation/main.tex}
%\input{./tikz/main.tex}
%\input{./Theorie-du-distributions/main.tex}
%\input{./optimisation/mine.tex}
 \input{./modelisation/main.tex}

% yves.aubry@univ-tln.fr : algebra

\end{document}

% % !TEX encoding = UTF-8 Unicode
% !TEX TS-program = xelatex

\documentclass[french]{report}

%\usepackage[utf8]{inputenc}
%\usepackage[T1]{fontenc}
\usepackage{babel}


\newif\ifcomment
%\commenttrue # Show comments

\usepackage{physics}
\usepackage{amssymb}


\usepackage{amsthm}
% \usepackage{thmtools}
\usepackage{mathtools}
\usepackage{amsfonts}

\usepackage{color}

\usepackage{tikz}

\usepackage{geometry}
\geometry{a5paper, margin=0.1in, right=1cm}

\usepackage{dsfont}

\usepackage{graphicx}
\graphicspath{ {images/} }

\usepackage{faktor}

\usepackage{IEEEtrantools}
\usepackage{enumerate}   
\usepackage[PostScript=dvips]{"/Users/aware/Documents/Courses/diagrams"}


\newtheorem{theorem}{Théorème}[section]
\renewcommand{\thetheorem}{\arabic{theorem}}
\newtheorem{lemme}{Lemme}[section]
\renewcommand{\thelemme}{\arabic{lemme}}
\newtheorem{proposition}{Proposition}[section]
\renewcommand{\theproposition}{\arabic{proposition}}
\newtheorem{notations}{Notations}[section]
\newtheorem{problem}{Problème}[section]
\newtheorem{corollary}{Corollaire}[theorem]
\renewcommand{\thecorollary}{\arabic{corollary}}
\newtheorem{property}{Propriété}[section]
\newtheorem{objective}{Objectif}[section]

\theoremstyle{definition}
\newtheorem{definition}{Définition}[section]
\renewcommand{\thedefinition}{\arabic{definition}}
\newtheorem{exercise}{Exercice}[chapter]
\renewcommand{\theexercise}{\arabic{exercise}}
\newtheorem{example}{Exemple}[chapter]
\renewcommand{\theexample}{\arabic{example}}
\newtheorem*{solution}{Solution}
\newtheorem*{application}{Application}
\newtheorem*{notation}{Notation}
\newtheorem*{vocabulary}{Vocabulaire}
\newtheorem*{properties}{Propriétés}



\theoremstyle{remark}
\newtheorem*{remark}{Remarque}
\newtheorem*{rappel}{Rappel}


\usepackage{etoolbox}
\AtBeginEnvironment{exercise}{\small}
\AtBeginEnvironment{example}{\small}

\usepackage{cases}
\usepackage[red]{mypack}

\usepackage[framemethod=TikZ]{mdframed}

\definecolor{bg}{rgb}{0.4,0.25,0.95}
\definecolor{pagebg}{rgb}{0,0,0.5}
\surroundwithmdframed[
   topline=false,
   rightline=false,
   bottomline=false,
   leftmargin=\parindent,
   skipabove=8pt,
   skipbelow=8pt,
   linecolor=blue,
   innerbottommargin=10pt,
   % backgroundcolor=bg,font=\color{orange}\sffamily, fontcolor=white
]{definition}

\usepackage{empheq}
\usepackage[most]{tcolorbox}

\newtcbox{\mymath}[1][]{%
    nobeforeafter, math upper, tcbox raise base,
    enhanced, colframe=blue!30!black,
    colback=red!10, boxrule=1pt,
    #1}

\usepackage{unixode}


\DeclareMathOperator{\ord}{ord}
\DeclareMathOperator{\orb}{orb}
\DeclareMathOperator{\stab}{stab}
\DeclareMathOperator{\Stab}{stab}
\DeclareMathOperator{\ppcm}{ppcm}
\DeclareMathOperator{\conj}{Conj}
\DeclareMathOperator{\End}{End}
\DeclareMathOperator{\rot}{rot}
\DeclareMathOperator{\trs}{trace}
\DeclareMathOperator{\Ind}{Ind}
\DeclareMathOperator{\mat}{Mat}
\DeclareMathOperator{\id}{Id}
\DeclareMathOperator{\vect}{vect}
\DeclareMathOperator{\img}{img}
\DeclareMathOperator{\cov}{Cov}
\DeclareMathOperator{\dist}{dist}
\DeclareMathOperator{\irr}{Irr}
\DeclareMathOperator{\image}{Im}
\DeclareMathOperator{\pd}{\partial}
\DeclareMathOperator{\epi}{epi}
\DeclareMathOperator{\Argmin}{Argmin}
\DeclareMathOperator{\dom}{dom}
\DeclareMathOperator{\proj}{proj}
\DeclareMathOperator{\ctg}{ctg}
\DeclareMathOperator{\supp}{supp}
\DeclareMathOperator{\argmin}{argmin}
\DeclareMathOperator{\mult}{mult}
\DeclareMathOperator{\ch}{ch}
\DeclareMathOperator{\sh}{sh}
\DeclareMathOperator{\rang}{rang}
\DeclareMathOperator{\diam}{diam}
\DeclareMathOperator{\Epigraphe}{Epigraphe}




\usepackage{xcolor}
\everymath{\color{blue}}
%\everymath{\color[rgb]{0,1,1}}
%\pagecolor[rgb]{0,0,0.5}


\newcommand*{\pdtest}[3][]{\ensuremath{\frac{\partial^{#1} #2}{\partial #3}}}

\newcommand*{\deffunc}[6][]{\ensuremath{
\begin{array}{rcl}
#2 : #3 &\rightarrow& #4\\
#5 &\mapsto& #6
\end{array}
}}

\newcommand{\eqcolon}{\mathrel{\resizebox{\widthof{$\mathord{=}$}}{\height}{ $\!\!=\!\!\resizebox{1.2\width}{0.8\height}{\raisebox{0.23ex}{$\mathop{:}$}}\!\!$ }}}
\newcommand{\coloneq}{\mathrel{\resizebox{\widthof{$\mathord{=}$}}{\height}{ $\!\!\resizebox{1.2\width}{0.8\height}{\raisebox{0.23ex}{$\mathop{:}$}}\!\!=\!\!$ }}}
\newcommand{\eqcolonl}{\ensuremath{\mathrel{=\!\!\mathop{:}}}}
\newcommand{\coloneql}{\ensuremath{\mathrel{\mathop{:} \!\! =}}}
\newcommand{\vc}[1]{% inline column vector
  \left(\begin{smallmatrix}#1\end{smallmatrix}\right)%
}
\newcommand{\vr}[1]{% inline row vector
  \begin{smallmatrix}(\,#1\,)\end{smallmatrix}%
}
\makeatletter
\newcommand*{\defeq}{\ =\mathrel{\rlap{%
                     \raisebox{0.3ex}{$\m@th\cdot$}}%
                     \raisebox{-0.3ex}{$\m@th\cdot$}}%
                     }
\makeatother

\newcommand{\mathcircle}[1]{% inline row vector
 \overset{\circ}{#1}
}
\newcommand{\ulim}{% low limit
 \underline{\lim}
}
\newcommand{\ssi}{% iff
\iff
}
\newcommand{\ps}[2]{
\expval{#1 | #2}
}
\newcommand{\df}[1]{
\mqty{#1}
}
\newcommand{\n}[1]{
\norm{#1}
}
\newcommand{\sys}[1]{
\left\{\smqty{#1}\right.
}


\newcommand{\eqdef}{\ensuremath{\overset{\text{def}}=}}


\def\Circlearrowright{\ensuremath{%
  \rotatebox[origin=c]{230}{$\circlearrowright$}}}

\newcommand\ct[1]{\text{\rmfamily\upshape #1}}
\newcommand\question[1]{ {\color{red} ...!? \small #1}}
\newcommand\caz[1]{\left\{\begin{array} #1 \end{array}\right.}
\newcommand\const{\text{\rmfamily\upshape const}}
\newcommand\toP{ \overset{\pro}{\to}}
\newcommand\toPP{ \overset{\text{PP}}{\to}}
\newcommand{\oeq}{\mathrel{\text{\textcircled{$=$}}}}





\usepackage{xcolor}
% \usepackage[normalem]{ulem}
\usepackage{lipsum}
\makeatletter
% \newcommand\colorwave[1][blue]{\bgroup \markoverwith{\lower3.5\p@\hbox{\sixly \textcolor{#1}{\char58}}}\ULon}
%\font\sixly=lasy6 % does not re-load if already loaded, so no memory problem.

\newmdtheoremenv[
linewidth= 1pt,linecolor= blue,%
leftmargin=20,rightmargin=20,innertopmargin=0pt, innerrightmargin=40,%
tikzsetting = { draw=lightgray, line width = 0.3pt,dashed,%
dash pattern = on 15pt off 3pt},%
splittopskip=\topskip,skipbelow=\baselineskip,%
skipabove=\baselineskip,ntheorem,roundcorner=0pt,
% backgroundcolor=pagebg,font=\color{orange}\sffamily, fontcolor=white
]{examplebox}{Exemple}[section]



\newcommand\R{\mathbb{R}}
\newcommand\Z{\mathbb{Z}}
\newcommand\N{\mathbb{N}}
\newcommand\E{\mathbb{E}}
\newcommand\F{\mathcal{F}}
\newcommand\cH{\mathcal{H}}
\newcommand\V{\mathbb{V}}
\newcommand\dmo{ ^{-1} }
\newcommand\kapa{\kappa}
\newcommand\im{Im}
\newcommand\hs{\mathcal{H}}





\usepackage{soul}

\makeatletter
\newcommand*{\whiten}[1]{\llap{\textcolor{white}{{\the\SOUL@token}}\hspace{#1pt}}}
\DeclareRobustCommand*\myul{%
    \def\SOUL@everyspace{\underline{\space}\kern\z@}%
    \def\SOUL@everytoken{%
     \setbox0=\hbox{\the\SOUL@token}%
     \ifdim\dp0>\z@
        \raisebox{\dp0}{\underline{\phantom{\the\SOUL@token}}}%
        \whiten{1}\whiten{0}%
        \whiten{-1}\whiten{-2}%
        \llap{\the\SOUL@token}%
     \else
        \underline{\the\SOUL@token}%
     \fi}%
\SOUL@}
\makeatother

\newcommand*{\demp}{\fontfamily{lmtt}\selectfont}

\DeclareTextFontCommand{\textdemp}{\demp}

\begin{document}

\ifcomment
Multiline
comment
\fi
\ifcomment
\myul{Typesetting test}
% \color[rgb]{1,1,1}
$∑_i^n≠ 60º±∞π∆¬≈√j∫h≤≥µ$

$\CR \R\pro\ind\pro\gS\pro
\mqty[a&b\\c&d]$
$\pro\mathbb{P}$
$\dd{x}$

  \[
    \alpha(x)=\left\{
                \begin{array}{ll}
                  x\\
                  \frac{1}{1+e^{-kx}}\\
                  \frac{e^x-e^{-x}}{e^x+e^{-x}}
                \end{array}
              \right.
  \]

  $\expval{x}$
  
  $\chi_\rho(ghg\dmo)=\Tr(\rho_{ghg\dmo})=\Tr(\rho_g\circ\rho_h\circ\rho\dmo_g)=\Tr(\rho_h)\overset{\mbox{\scalebox{0.5}{$\Tr(AB)=\Tr(BA)$}}}{=}\chi_\rho(h)$
  	$\mathop{\oplus}_{\substack{x\in X}}$

$\mat(\rho_g)=(a_{ij}(g))_{\scriptsize \substack{1\leq i\leq d \\ 1\leq j\leq d}}$ et $\mat(\rho'_g)=(a'_{ij}(g))_{\scriptsize \substack{1\leq i'\leq d' \\ 1\leq j'\leq d'}}$



\[\int_a^b{\mathbb{R}^2}g(u, v)\dd{P_{XY}}(u, v)=\iint g(u,v) f_{XY}(u, v)\dd \lambda(u) \dd \lambda(v)\]
$$\lim_{x\to\infty} f(x)$$	
$$\iiiint_V \mu(t,u,v,w) \,dt\,du\,dv\,dw$$
$$\sum_{n=1}^{\infty} 2^{-n} = 1$$	
\begin{definition}
	Si $X$ et $Y$ sont 2 v.a. ou definit la \textsc{Covariance} entre $X$ et $Y$ comme
	$\cov(X,Y)\overset{\text{def}}{=}\E\left[(X-\E(X))(Y-\E(Y))\right]=\E(XY)-\E(X)\E(Y)$.
\end{definition}
\fi
\pagebreak

% \tableofcontents

% insert your code here
%\input{./algebra/main.tex}
%\input{./geometrie-differentielle/main.tex}
%\input{./probabilite/main.tex}
%\input{./analyse-fonctionnelle/main.tex}
% \input{./Analyse-convexe-et-dualite-en-optimisation/main.tex}
%\input{./tikz/main.tex}
%\input{./Theorie-du-distributions/main.tex}
%\input{./optimisation/mine.tex}
 \input{./modelisation/main.tex}

% yves.aubry@univ-tln.fr : algebra

\end{document}

%% !TEX encoding = UTF-8 Unicode
% !TEX TS-program = xelatex

\documentclass[french]{report}

%\usepackage[utf8]{inputenc}
%\usepackage[T1]{fontenc}
\usepackage{babel}


\newif\ifcomment
%\commenttrue # Show comments

\usepackage{physics}
\usepackage{amssymb}


\usepackage{amsthm}
% \usepackage{thmtools}
\usepackage{mathtools}
\usepackage{amsfonts}

\usepackage{color}

\usepackage{tikz}

\usepackage{geometry}
\geometry{a5paper, margin=0.1in, right=1cm}

\usepackage{dsfont}

\usepackage{graphicx}
\graphicspath{ {images/} }

\usepackage{faktor}

\usepackage{IEEEtrantools}
\usepackage{enumerate}   
\usepackage[PostScript=dvips]{"/Users/aware/Documents/Courses/diagrams"}


\newtheorem{theorem}{Théorème}[section]
\renewcommand{\thetheorem}{\arabic{theorem}}
\newtheorem{lemme}{Lemme}[section]
\renewcommand{\thelemme}{\arabic{lemme}}
\newtheorem{proposition}{Proposition}[section]
\renewcommand{\theproposition}{\arabic{proposition}}
\newtheorem{notations}{Notations}[section]
\newtheorem{problem}{Problème}[section]
\newtheorem{corollary}{Corollaire}[theorem]
\renewcommand{\thecorollary}{\arabic{corollary}}
\newtheorem{property}{Propriété}[section]
\newtheorem{objective}{Objectif}[section]

\theoremstyle{definition}
\newtheorem{definition}{Définition}[section]
\renewcommand{\thedefinition}{\arabic{definition}}
\newtheorem{exercise}{Exercice}[chapter]
\renewcommand{\theexercise}{\arabic{exercise}}
\newtheorem{example}{Exemple}[chapter]
\renewcommand{\theexample}{\arabic{example}}
\newtheorem*{solution}{Solution}
\newtheorem*{application}{Application}
\newtheorem*{notation}{Notation}
\newtheorem*{vocabulary}{Vocabulaire}
\newtheorem*{properties}{Propriétés}



\theoremstyle{remark}
\newtheorem*{remark}{Remarque}
\newtheorem*{rappel}{Rappel}


\usepackage{etoolbox}
\AtBeginEnvironment{exercise}{\small}
\AtBeginEnvironment{example}{\small}

\usepackage{cases}
\usepackage[red]{mypack}

\usepackage[framemethod=TikZ]{mdframed}

\definecolor{bg}{rgb}{0.4,0.25,0.95}
\definecolor{pagebg}{rgb}{0,0,0.5}
\surroundwithmdframed[
   topline=false,
   rightline=false,
   bottomline=false,
   leftmargin=\parindent,
   skipabove=8pt,
   skipbelow=8pt,
   linecolor=blue,
   innerbottommargin=10pt,
   % backgroundcolor=bg,font=\color{orange}\sffamily, fontcolor=white
]{definition}

\usepackage{empheq}
\usepackage[most]{tcolorbox}

\newtcbox{\mymath}[1][]{%
    nobeforeafter, math upper, tcbox raise base,
    enhanced, colframe=blue!30!black,
    colback=red!10, boxrule=1pt,
    #1}

\usepackage{unixode}


\DeclareMathOperator{\ord}{ord}
\DeclareMathOperator{\orb}{orb}
\DeclareMathOperator{\stab}{stab}
\DeclareMathOperator{\Stab}{stab}
\DeclareMathOperator{\ppcm}{ppcm}
\DeclareMathOperator{\conj}{Conj}
\DeclareMathOperator{\End}{End}
\DeclareMathOperator{\rot}{rot}
\DeclareMathOperator{\trs}{trace}
\DeclareMathOperator{\Ind}{Ind}
\DeclareMathOperator{\mat}{Mat}
\DeclareMathOperator{\id}{Id}
\DeclareMathOperator{\vect}{vect}
\DeclareMathOperator{\img}{img}
\DeclareMathOperator{\cov}{Cov}
\DeclareMathOperator{\dist}{dist}
\DeclareMathOperator{\irr}{Irr}
\DeclareMathOperator{\image}{Im}
\DeclareMathOperator{\pd}{\partial}
\DeclareMathOperator{\epi}{epi}
\DeclareMathOperator{\Argmin}{Argmin}
\DeclareMathOperator{\dom}{dom}
\DeclareMathOperator{\proj}{proj}
\DeclareMathOperator{\ctg}{ctg}
\DeclareMathOperator{\supp}{supp}
\DeclareMathOperator{\argmin}{argmin}
\DeclareMathOperator{\mult}{mult}
\DeclareMathOperator{\ch}{ch}
\DeclareMathOperator{\sh}{sh}
\DeclareMathOperator{\rang}{rang}
\DeclareMathOperator{\diam}{diam}
\DeclareMathOperator{\Epigraphe}{Epigraphe}




\usepackage{xcolor}
\everymath{\color{blue}}
%\everymath{\color[rgb]{0,1,1}}
%\pagecolor[rgb]{0,0,0.5}


\newcommand*{\pdtest}[3][]{\ensuremath{\frac{\partial^{#1} #2}{\partial #3}}}

\newcommand*{\deffunc}[6][]{\ensuremath{
\begin{array}{rcl}
#2 : #3 &\rightarrow& #4\\
#5 &\mapsto& #6
\end{array}
}}

\newcommand{\eqcolon}{\mathrel{\resizebox{\widthof{$\mathord{=}$}}{\height}{ $\!\!=\!\!\resizebox{1.2\width}{0.8\height}{\raisebox{0.23ex}{$\mathop{:}$}}\!\!$ }}}
\newcommand{\coloneq}{\mathrel{\resizebox{\widthof{$\mathord{=}$}}{\height}{ $\!\!\resizebox{1.2\width}{0.8\height}{\raisebox{0.23ex}{$\mathop{:}$}}\!\!=\!\!$ }}}
\newcommand{\eqcolonl}{\ensuremath{\mathrel{=\!\!\mathop{:}}}}
\newcommand{\coloneql}{\ensuremath{\mathrel{\mathop{:} \!\! =}}}
\newcommand{\vc}[1]{% inline column vector
  \left(\begin{smallmatrix}#1\end{smallmatrix}\right)%
}
\newcommand{\vr}[1]{% inline row vector
  \begin{smallmatrix}(\,#1\,)\end{smallmatrix}%
}
\makeatletter
\newcommand*{\defeq}{\ =\mathrel{\rlap{%
                     \raisebox{0.3ex}{$\m@th\cdot$}}%
                     \raisebox{-0.3ex}{$\m@th\cdot$}}%
                     }
\makeatother

\newcommand{\mathcircle}[1]{% inline row vector
 \overset{\circ}{#1}
}
\newcommand{\ulim}{% low limit
 \underline{\lim}
}
\newcommand{\ssi}{% iff
\iff
}
\newcommand{\ps}[2]{
\expval{#1 | #2}
}
\newcommand{\df}[1]{
\mqty{#1}
}
\newcommand{\n}[1]{
\norm{#1}
}
\newcommand{\sys}[1]{
\left\{\smqty{#1}\right.
}


\newcommand{\eqdef}{\ensuremath{\overset{\text{def}}=}}


\def\Circlearrowright{\ensuremath{%
  \rotatebox[origin=c]{230}{$\circlearrowright$}}}

\newcommand\ct[1]{\text{\rmfamily\upshape #1}}
\newcommand\question[1]{ {\color{red} ...!? \small #1}}
\newcommand\caz[1]{\left\{\begin{array} #1 \end{array}\right.}
\newcommand\const{\text{\rmfamily\upshape const}}
\newcommand\toP{ \overset{\pro}{\to}}
\newcommand\toPP{ \overset{\text{PP}}{\to}}
\newcommand{\oeq}{\mathrel{\text{\textcircled{$=$}}}}





\usepackage{xcolor}
% \usepackage[normalem]{ulem}
\usepackage{lipsum}
\makeatletter
% \newcommand\colorwave[1][blue]{\bgroup \markoverwith{\lower3.5\p@\hbox{\sixly \textcolor{#1}{\char58}}}\ULon}
%\font\sixly=lasy6 % does not re-load if already loaded, so no memory problem.

\newmdtheoremenv[
linewidth= 1pt,linecolor= blue,%
leftmargin=20,rightmargin=20,innertopmargin=0pt, innerrightmargin=40,%
tikzsetting = { draw=lightgray, line width = 0.3pt,dashed,%
dash pattern = on 15pt off 3pt},%
splittopskip=\topskip,skipbelow=\baselineskip,%
skipabove=\baselineskip,ntheorem,roundcorner=0pt,
% backgroundcolor=pagebg,font=\color{orange}\sffamily, fontcolor=white
]{examplebox}{Exemple}[section]



\newcommand\R{\mathbb{R}}
\newcommand\Z{\mathbb{Z}}
\newcommand\N{\mathbb{N}}
\newcommand\E{\mathbb{E}}
\newcommand\F{\mathcal{F}}
\newcommand\cH{\mathcal{H}}
\newcommand\V{\mathbb{V}}
\newcommand\dmo{ ^{-1} }
\newcommand\kapa{\kappa}
\newcommand\im{Im}
\newcommand\hs{\mathcal{H}}





\usepackage{soul}

\makeatletter
\newcommand*{\whiten}[1]{\llap{\textcolor{white}{{\the\SOUL@token}}\hspace{#1pt}}}
\DeclareRobustCommand*\myul{%
    \def\SOUL@everyspace{\underline{\space}\kern\z@}%
    \def\SOUL@everytoken{%
     \setbox0=\hbox{\the\SOUL@token}%
     \ifdim\dp0>\z@
        \raisebox{\dp0}{\underline{\phantom{\the\SOUL@token}}}%
        \whiten{1}\whiten{0}%
        \whiten{-1}\whiten{-2}%
        \llap{\the\SOUL@token}%
     \else
        \underline{\the\SOUL@token}%
     \fi}%
\SOUL@}
\makeatother

\newcommand*{\demp}{\fontfamily{lmtt}\selectfont}

\DeclareTextFontCommand{\textdemp}{\demp}

\begin{document}

\ifcomment
Multiline
comment
\fi
\ifcomment
\myul{Typesetting test}
% \color[rgb]{1,1,1}
$∑_i^n≠ 60º±∞π∆¬≈√j∫h≤≥µ$

$\CR \R\pro\ind\pro\gS\pro
\mqty[a&b\\c&d]$
$\pro\mathbb{P}$
$\dd{x}$

  \[
    \alpha(x)=\left\{
                \begin{array}{ll}
                  x\\
                  \frac{1}{1+e^{-kx}}\\
                  \frac{e^x-e^{-x}}{e^x+e^{-x}}
                \end{array}
              \right.
  \]

  $\expval{x}$
  
  $\chi_\rho(ghg\dmo)=\Tr(\rho_{ghg\dmo})=\Tr(\rho_g\circ\rho_h\circ\rho\dmo_g)=\Tr(\rho_h)\overset{\mbox{\scalebox{0.5}{$\Tr(AB)=\Tr(BA)$}}}{=}\chi_\rho(h)$
  	$\mathop{\oplus}_{\substack{x\in X}}$

$\mat(\rho_g)=(a_{ij}(g))_{\scriptsize \substack{1\leq i\leq d \\ 1\leq j\leq d}}$ et $\mat(\rho'_g)=(a'_{ij}(g))_{\scriptsize \substack{1\leq i'\leq d' \\ 1\leq j'\leq d'}}$



\[\int_a^b{\mathbb{R}^2}g(u, v)\dd{P_{XY}}(u, v)=\iint g(u,v) f_{XY}(u, v)\dd \lambda(u) \dd \lambda(v)\]
$$\lim_{x\to\infty} f(x)$$	
$$\iiiint_V \mu(t,u,v,w) \,dt\,du\,dv\,dw$$
$$\sum_{n=1}^{\infty} 2^{-n} = 1$$	
\begin{definition}
	Si $X$ et $Y$ sont 2 v.a. ou definit la \textsc{Covariance} entre $X$ et $Y$ comme
	$\cov(X,Y)\overset{\text{def}}{=}\E\left[(X-\E(X))(Y-\E(Y))\right]=\E(XY)-\E(X)\E(Y)$.
\end{definition}
\fi
\pagebreak

% \tableofcontents

% insert your code here
%\input{./algebra/main.tex}
%\input{./geometrie-differentielle/main.tex}
%\input{./probabilite/main.tex}
%\input{./analyse-fonctionnelle/main.tex}
% \input{./Analyse-convexe-et-dualite-en-optimisation/main.tex}
%\input{./tikz/main.tex}
%\input{./Theorie-du-distributions/main.tex}
%\input{./optimisation/mine.tex}
 \input{./modelisation/main.tex}

% yves.aubry@univ-tln.fr : algebra

\end{document}

%% !TEX encoding = UTF-8 Unicode
% !TEX TS-program = xelatex

\documentclass[french]{report}

%\usepackage[utf8]{inputenc}
%\usepackage[T1]{fontenc}
\usepackage{babel}


\newif\ifcomment
%\commenttrue # Show comments

\usepackage{physics}
\usepackage{amssymb}


\usepackage{amsthm}
% \usepackage{thmtools}
\usepackage{mathtools}
\usepackage{amsfonts}

\usepackage{color}

\usepackage{tikz}

\usepackage{geometry}
\geometry{a5paper, margin=0.1in, right=1cm}

\usepackage{dsfont}

\usepackage{graphicx}
\graphicspath{ {images/} }

\usepackage{faktor}

\usepackage{IEEEtrantools}
\usepackage{enumerate}   
\usepackage[PostScript=dvips]{"/Users/aware/Documents/Courses/diagrams"}


\newtheorem{theorem}{Théorème}[section]
\renewcommand{\thetheorem}{\arabic{theorem}}
\newtheorem{lemme}{Lemme}[section]
\renewcommand{\thelemme}{\arabic{lemme}}
\newtheorem{proposition}{Proposition}[section]
\renewcommand{\theproposition}{\arabic{proposition}}
\newtheorem{notations}{Notations}[section]
\newtheorem{problem}{Problème}[section]
\newtheorem{corollary}{Corollaire}[theorem]
\renewcommand{\thecorollary}{\arabic{corollary}}
\newtheorem{property}{Propriété}[section]
\newtheorem{objective}{Objectif}[section]

\theoremstyle{definition}
\newtheorem{definition}{Définition}[section]
\renewcommand{\thedefinition}{\arabic{definition}}
\newtheorem{exercise}{Exercice}[chapter]
\renewcommand{\theexercise}{\arabic{exercise}}
\newtheorem{example}{Exemple}[chapter]
\renewcommand{\theexample}{\arabic{example}}
\newtheorem*{solution}{Solution}
\newtheorem*{application}{Application}
\newtheorem*{notation}{Notation}
\newtheorem*{vocabulary}{Vocabulaire}
\newtheorem*{properties}{Propriétés}



\theoremstyle{remark}
\newtheorem*{remark}{Remarque}
\newtheorem*{rappel}{Rappel}


\usepackage{etoolbox}
\AtBeginEnvironment{exercise}{\small}
\AtBeginEnvironment{example}{\small}

\usepackage{cases}
\usepackage[red]{mypack}

\usepackage[framemethod=TikZ]{mdframed}

\definecolor{bg}{rgb}{0.4,0.25,0.95}
\definecolor{pagebg}{rgb}{0,0,0.5}
\surroundwithmdframed[
   topline=false,
   rightline=false,
   bottomline=false,
   leftmargin=\parindent,
   skipabove=8pt,
   skipbelow=8pt,
   linecolor=blue,
   innerbottommargin=10pt,
   % backgroundcolor=bg,font=\color{orange}\sffamily, fontcolor=white
]{definition}

\usepackage{empheq}
\usepackage[most]{tcolorbox}

\newtcbox{\mymath}[1][]{%
    nobeforeafter, math upper, tcbox raise base,
    enhanced, colframe=blue!30!black,
    colback=red!10, boxrule=1pt,
    #1}

\usepackage{unixode}


\DeclareMathOperator{\ord}{ord}
\DeclareMathOperator{\orb}{orb}
\DeclareMathOperator{\stab}{stab}
\DeclareMathOperator{\Stab}{stab}
\DeclareMathOperator{\ppcm}{ppcm}
\DeclareMathOperator{\conj}{Conj}
\DeclareMathOperator{\End}{End}
\DeclareMathOperator{\rot}{rot}
\DeclareMathOperator{\trs}{trace}
\DeclareMathOperator{\Ind}{Ind}
\DeclareMathOperator{\mat}{Mat}
\DeclareMathOperator{\id}{Id}
\DeclareMathOperator{\vect}{vect}
\DeclareMathOperator{\img}{img}
\DeclareMathOperator{\cov}{Cov}
\DeclareMathOperator{\dist}{dist}
\DeclareMathOperator{\irr}{Irr}
\DeclareMathOperator{\image}{Im}
\DeclareMathOperator{\pd}{\partial}
\DeclareMathOperator{\epi}{epi}
\DeclareMathOperator{\Argmin}{Argmin}
\DeclareMathOperator{\dom}{dom}
\DeclareMathOperator{\proj}{proj}
\DeclareMathOperator{\ctg}{ctg}
\DeclareMathOperator{\supp}{supp}
\DeclareMathOperator{\argmin}{argmin}
\DeclareMathOperator{\mult}{mult}
\DeclareMathOperator{\ch}{ch}
\DeclareMathOperator{\sh}{sh}
\DeclareMathOperator{\rang}{rang}
\DeclareMathOperator{\diam}{diam}
\DeclareMathOperator{\Epigraphe}{Epigraphe}




\usepackage{xcolor}
\everymath{\color{blue}}
%\everymath{\color[rgb]{0,1,1}}
%\pagecolor[rgb]{0,0,0.5}


\newcommand*{\pdtest}[3][]{\ensuremath{\frac{\partial^{#1} #2}{\partial #3}}}

\newcommand*{\deffunc}[6][]{\ensuremath{
\begin{array}{rcl}
#2 : #3 &\rightarrow& #4\\
#5 &\mapsto& #6
\end{array}
}}

\newcommand{\eqcolon}{\mathrel{\resizebox{\widthof{$\mathord{=}$}}{\height}{ $\!\!=\!\!\resizebox{1.2\width}{0.8\height}{\raisebox{0.23ex}{$\mathop{:}$}}\!\!$ }}}
\newcommand{\coloneq}{\mathrel{\resizebox{\widthof{$\mathord{=}$}}{\height}{ $\!\!\resizebox{1.2\width}{0.8\height}{\raisebox{0.23ex}{$\mathop{:}$}}\!\!=\!\!$ }}}
\newcommand{\eqcolonl}{\ensuremath{\mathrel{=\!\!\mathop{:}}}}
\newcommand{\coloneql}{\ensuremath{\mathrel{\mathop{:} \!\! =}}}
\newcommand{\vc}[1]{% inline column vector
  \left(\begin{smallmatrix}#1\end{smallmatrix}\right)%
}
\newcommand{\vr}[1]{% inline row vector
  \begin{smallmatrix}(\,#1\,)\end{smallmatrix}%
}
\makeatletter
\newcommand*{\defeq}{\ =\mathrel{\rlap{%
                     \raisebox{0.3ex}{$\m@th\cdot$}}%
                     \raisebox{-0.3ex}{$\m@th\cdot$}}%
                     }
\makeatother

\newcommand{\mathcircle}[1]{% inline row vector
 \overset{\circ}{#1}
}
\newcommand{\ulim}{% low limit
 \underline{\lim}
}
\newcommand{\ssi}{% iff
\iff
}
\newcommand{\ps}[2]{
\expval{#1 | #2}
}
\newcommand{\df}[1]{
\mqty{#1}
}
\newcommand{\n}[1]{
\norm{#1}
}
\newcommand{\sys}[1]{
\left\{\smqty{#1}\right.
}


\newcommand{\eqdef}{\ensuremath{\overset{\text{def}}=}}


\def\Circlearrowright{\ensuremath{%
  \rotatebox[origin=c]{230}{$\circlearrowright$}}}

\newcommand\ct[1]{\text{\rmfamily\upshape #1}}
\newcommand\question[1]{ {\color{red} ...!? \small #1}}
\newcommand\caz[1]{\left\{\begin{array} #1 \end{array}\right.}
\newcommand\const{\text{\rmfamily\upshape const}}
\newcommand\toP{ \overset{\pro}{\to}}
\newcommand\toPP{ \overset{\text{PP}}{\to}}
\newcommand{\oeq}{\mathrel{\text{\textcircled{$=$}}}}





\usepackage{xcolor}
% \usepackage[normalem]{ulem}
\usepackage{lipsum}
\makeatletter
% \newcommand\colorwave[1][blue]{\bgroup \markoverwith{\lower3.5\p@\hbox{\sixly \textcolor{#1}{\char58}}}\ULon}
%\font\sixly=lasy6 % does not re-load if already loaded, so no memory problem.

\newmdtheoremenv[
linewidth= 1pt,linecolor= blue,%
leftmargin=20,rightmargin=20,innertopmargin=0pt, innerrightmargin=40,%
tikzsetting = { draw=lightgray, line width = 0.3pt,dashed,%
dash pattern = on 15pt off 3pt},%
splittopskip=\topskip,skipbelow=\baselineskip,%
skipabove=\baselineskip,ntheorem,roundcorner=0pt,
% backgroundcolor=pagebg,font=\color{orange}\sffamily, fontcolor=white
]{examplebox}{Exemple}[section]



\newcommand\R{\mathbb{R}}
\newcommand\Z{\mathbb{Z}}
\newcommand\N{\mathbb{N}}
\newcommand\E{\mathbb{E}}
\newcommand\F{\mathcal{F}}
\newcommand\cH{\mathcal{H}}
\newcommand\V{\mathbb{V}}
\newcommand\dmo{ ^{-1} }
\newcommand\kapa{\kappa}
\newcommand\im{Im}
\newcommand\hs{\mathcal{H}}





\usepackage{soul}

\makeatletter
\newcommand*{\whiten}[1]{\llap{\textcolor{white}{{\the\SOUL@token}}\hspace{#1pt}}}
\DeclareRobustCommand*\myul{%
    \def\SOUL@everyspace{\underline{\space}\kern\z@}%
    \def\SOUL@everytoken{%
     \setbox0=\hbox{\the\SOUL@token}%
     \ifdim\dp0>\z@
        \raisebox{\dp0}{\underline{\phantom{\the\SOUL@token}}}%
        \whiten{1}\whiten{0}%
        \whiten{-1}\whiten{-2}%
        \llap{\the\SOUL@token}%
     \else
        \underline{\the\SOUL@token}%
     \fi}%
\SOUL@}
\makeatother

\newcommand*{\demp}{\fontfamily{lmtt}\selectfont}

\DeclareTextFontCommand{\textdemp}{\demp}

\begin{document}

\ifcomment
Multiline
comment
\fi
\ifcomment
\myul{Typesetting test}
% \color[rgb]{1,1,1}
$∑_i^n≠ 60º±∞π∆¬≈√j∫h≤≥µ$

$\CR \R\pro\ind\pro\gS\pro
\mqty[a&b\\c&d]$
$\pro\mathbb{P}$
$\dd{x}$

  \[
    \alpha(x)=\left\{
                \begin{array}{ll}
                  x\\
                  \frac{1}{1+e^{-kx}}\\
                  \frac{e^x-e^{-x}}{e^x+e^{-x}}
                \end{array}
              \right.
  \]

  $\expval{x}$
  
  $\chi_\rho(ghg\dmo)=\Tr(\rho_{ghg\dmo})=\Tr(\rho_g\circ\rho_h\circ\rho\dmo_g)=\Tr(\rho_h)\overset{\mbox{\scalebox{0.5}{$\Tr(AB)=\Tr(BA)$}}}{=}\chi_\rho(h)$
  	$\mathop{\oplus}_{\substack{x\in X}}$

$\mat(\rho_g)=(a_{ij}(g))_{\scriptsize \substack{1\leq i\leq d \\ 1\leq j\leq d}}$ et $\mat(\rho'_g)=(a'_{ij}(g))_{\scriptsize \substack{1\leq i'\leq d' \\ 1\leq j'\leq d'}}$



\[\int_a^b{\mathbb{R}^2}g(u, v)\dd{P_{XY}}(u, v)=\iint g(u,v) f_{XY}(u, v)\dd \lambda(u) \dd \lambda(v)\]
$$\lim_{x\to\infty} f(x)$$	
$$\iiiint_V \mu(t,u,v,w) \,dt\,du\,dv\,dw$$
$$\sum_{n=1}^{\infty} 2^{-n} = 1$$	
\begin{definition}
	Si $X$ et $Y$ sont 2 v.a. ou definit la \textsc{Covariance} entre $X$ et $Y$ comme
	$\cov(X,Y)\overset{\text{def}}{=}\E\left[(X-\E(X))(Y-\E(Y))\right]=\E(XY)-\E(X)\E(Y)$.
\end{definition}
\fi
\pagebreak

% \tableofcontents

% insert your code here
%\input{./algebra/main.tex}
%\input{./geometrie-differentielle/main.tex}
%\input{./probabilite/main.tex}
%\input{./analyse-fonctionnelle/main.tex}
% \input{./Analyse-convexe-et-dualite-en-optimisation/main.tex}
%\input{./tikz/main.tex}
%\input{./Theorie-du-distributions/main.tex}
%\input{./optimisation/mine.tex}
 \input{./modelisation/main.tex}

% yves.aubry@univ-tln.fr : algebra

\end{document}

%\input{./optimisation/mine.tex}
 % !TEX encoding = UTF-8 Unicode
% !TEX TS-program = xelatex

\documentclass[french]{report}

%\usepackage[utf8]{inputenc}
%\usepackage[T1]{fontenc}
\usepackage{babel}


\newif\ifcomment
%\commenttrue # Show comments

\usepackage{physics}
\usepackage{amssymb}


\usepackage{amsthm}
% \usepackage{thmtools}
\usepackage{mathtools}
\usepackage{amsfonts}

\usepackage{color}

\usepackage{tikz}

\usepackage{geometry}
\geometry{a5paper, margin=0.1in, right=1cm}

\usepackage{dsfont}

\usepackage{graphicx}
\graphicspath{ {images/} }

\usepackage{faktor}

\usepackage{IEEEtrantools}
\usepackage{enumerate}   
\usepackage[PostScript=dvips]{"/Users/aware/Documents/Courses/diagrams"}


\newtheorem{theorem}{Théorème}[section]
\renewcommand{\thetheorem}{\arabic{theorem}}
\newtheorem{lemme}{Lemme}[section]
\renewcommand{\thelemme}{\arabic{lemme}}
\newtheorem{proposition}{Proposition}[section]
\renewcommand{\theproposition}{\arabic{proposition}}
\newtheorem{notations}{Notations}[section]
\newtheorem{problem}{Problème}[section]
\newtheorem{corollary}{Corollaire}[theorem]
\renewcommand{\thecorollary}{\arabic{corollary}}
\newtheorem{property}{Propriété}[section]
\newtheorem{objective}{Objectif}[section]

\theoremstyle{definition}
\newtheorem{definition}{Définition}[section]
\renewcommand{\thedefinition}{\arabic{definition}}
\newtheorem{exercise}{Exercice}[chapter]
\renewcommand{\theexercise}{\arabic{exercise}}
\newtheorem{example}{Exemple}[chapter]
\renewcommand{\theexample}{\arabic{example}}
\newtheorem*{solution}{Solution}
\newtheorem*{application}{Application}
\newtheorem*{notation}{Notation}
\newtheorem*{vocabulary}{Vocabulaire}
\newtheorem*{properties}{Propriétés}



\theoremstyle{remark}
\newtheorem*{remark}{Remarque}
\newtheorem*{rappel}{Rappel}


\usepackage{etoolbox}
\AtBeginEnvironment{exercise}{\small}
\AtBeginEnvironment{example}{\small}

\usepackage{cases}
\usepackage[red]{mypack}

\usepackage[framemethod=TikZ]{mdframed}

\definecolor{bg}{rgb}{0.4,0.25,0.95}
\definecolor{pagebg}{rgb}{0,0,0.5}
\surroundwithmdframed[
   topline=false,
   rightline=false,
   bottomline=false,
   leftmargin=\parindent,
   skipabove=8pt,
   skipbelow=8pt,
   linecolor=blue,
   innerbottommargin=10pt,
   % backgroundcolor=bg,font=\color{orange}\sffamily, fontcolor=white
]{definition}

\usepackage{empheq}
\usepackage[most]{tcolorbox}

\newtcbox{\mymath}[1][]{%
    nobeforeafter, math upper, tcbox raise base,
    enhanced, colframe=blue!30!black,
    colback=red!10, boxrule=1pt,
    #1}

\usepackage{unixode}


\DeclareMathOperator{\ord}{ord}
\DeclareMathOperator{\orb}{orb}
\DeclareMathOperator{\stab}{stab}
\DeclareMathOperator{\Stab}{stab}
\DeclareMathOperator{\ppcm}{ppcm}
\DeclareMathOperator{\conj}{Conj}
\DeclareMathOperator{\End}{End}
\DeclareMathOperator{\rot}{rot}
\DeclareMathOperator{\trs}{trace}
\DeclareMathOperator{\Ind}{Ind}
\DeclareMathOperator{\mat}{Mat}
\DeclareMathOperator{\id}{Id}
\DeclareMathOperator{\vect}{vect}
\DeclareMathOperator{\img}{img}
\DeclareMathOperator{\cov}{Cov}
\DeclareMathOperator{\dist}{dist}
\DeclareMathOperator{\irr}{Irr}
\DeclareMathOperator{\image}{Im}
\DeclareMathOperator{\pd}{\partial}
\DeclareMathOperator{\epi}{epi}
\DeclareMathOperator{\Argmin}{Argmin}
\DeclareMathOperator{\dom}{dom}
\DeclareMathOperator{\proj}{proj}
\DeclareMathOperator{\ctg}{ctg}
\DeclareMathOperator{\supp}{supp}
\DeclareMathOperator{\argmin}{argmin}
\DeclareMathOperator{\mult}{mult}
\DeclareMathOperator{\ch}{ch}
\DeclareMathOperator{\sh}{sh}
\DeclareMathOperator{\rang}{rang}
\DeclareMathOperator{\diam}{diam}
\DeclareMathOperator{\Epigraphe}{Epigraphe}




\usepackage{xcolor}
\everymath{\color{blue}}
%\everymath{\color[rgb]{0,1,1}}
%\pagecolor[rgb]{0,0,0.5}


\newcommand*{\pdtest}[3][]{\ensuremath{\frac{\partial^{#1} #2}{\partial #3}}}

\newcommand*{\deffunc}[6][]{\ensuremath{
\begin{array}{rcl}
#2 : #3 &\rightarrow& #4\\
#5 &\mapsto& #6
\end{array}
}}

\newcommand{\eqcolon}{\mathrel{\resizebox{\widthof{$\mathord{=}$}}{\height}{ $\!\!=\!\!\resizebox{1.2\width}{0.8\height}{\raisebox{0.23ex}{$\mathop{:}$}}\!\!$ }}}
\newcommand{\coloneq}{\mathrel{\resizebox{\widthof{$\mathord{=}$}}{\height}{ $\!\!\resizebox{1.2\width}{0.8\height}{\raisebox{0.23ex}{$\mathop{:}$}}\!\!=\!\!$ }}}
\newcommand{\eqcolonl}{\ensuremath{\mathrel{=\!\!\mathop{:}}}}
\newcommand{\coloneql}{\ensuremath{\mathrel{\mathop{:} \!\! =}}}
\newcommand{\vc}[1]{% inline column vector
  \left(\begin{smallmatrix}#1\end{smallmatrix}\right)%
}
\newcommand{\vr}[1]{% inline row vector
  \begin{smallmatrix}(\,#1\,)\end{smallmatrix}%
}
\makeatletter
\newcommand*{\defeq}{\ =\mathrel{\rlap{%
                     \raisebox{0.3ex}{$\m@th\cdot$}}%
                     \raisebox{-0.3ex}{$\m@th\cdot$}}%
                     }
\makeatother

\newcommand{\mathcircle}[1]{% inline row vector
 \overset{\circ}{#1}
}
\newcommand{\ulim}{% low limit
 \underline{\lim}
}
\newcommand{\ssi}{% iff
\iff
}
\newcommand{\ps}[2]{
\expval{#1 | #2}
}
\newcommand{\df}[1]{
\mqty{#1}
}
\newcommand{\n}[1]{
\norm{#1}
}
\newcommand{\sys}[1]{
\left\{\smqty{#1}\right.
}


\newcommand{\eqdef}{\ensuremath{\overset{\text{def}}=}}


\def\Circlearrowright{\ensuremath{%
  \rotatebox[origin=c]{230}{$\circlearrowright$}}}

\newcommand\ct[1]{\text{\rmfamily\upshape #1}}
\newcommand\question[1]{ {\color{red} ...!? \small #1}}
\newcommand\caz[1]{\left\{\begin{array} #1 \end{array}\right.}
\newcommand\const{\text{\rmfamily\upshape const}}
\newcommand\toP{ \overset{\pro}{\to}}
\newcommand\toPP{ \overset{\text{PP}}{\to}}
\newcommand{\oeq}{\mathrel{\text{\textcircled{$=$}}}}





\usepackage{xcolor}
% \usepackage[normalem]{ulem}
\usepackage{lipsum}
\makeatletter
% \newcommand\colorwave[1][blue]{\bgroup \markoverwith{\lower3.5\p@\hbox{\sixly \textcolor{#1}{\char58}}}\ULon}
%\font\sixly=lasy6 % does not re-load if already loaded, so no memory problem.

\newmdtheoremenv[
linewidth= 1pt,linecolor= blue,%
leftmargin=20,rightmargin=20,innertopmargin=0pt, innerrightmargin=40,%
tikzsetting = { draw=lightgray, line width = 0.3pt,dashed,%
dash pattern = on 15pt off 3pt},%
splittopskip=\topskip,skipbelow=\baselineskip,%
skipabove=\baselineskip,ntheorem,roundcorner=0pt,
% backgroundcolor=pagebg,font=\color{orange}\sffamily, fontcolor=white
]{examplebox}{Exemple}[section]



\newcommand\R{\mathbb{R}}
\newcommand\Z{\mathbb{Z}}
\newcommand\N{\mathbb{N}}
\newcommand\E{\mathbb{E}}
\newcommand\F{\mathcal{F}}
\newcommand\cH{\mathcal{H}}
\newcommand\V{\mathbb{V}}
\newcommand\dmo{ ^{-1} }
\newcommand\kapa{\kappa}
\newcommand\im{Im}
\newcommand\hs{\mathcal{H}}





\usepackage{soul}

\makeatletter
\newcommand*{\whiten}[1]{\llap{\textcolor{white}{{\the\SOUL@token}}\hspace{#1pt}}}
\DeclareRobustCommand*\myul{%
    \def\SOUL@everyspace{\underline{\space}\kern\z@}%
    \def\SOUL@everytoken{%
     \setbox0=\hbox{\the\SOUL@token}%
     \ifdim\dp0>\z@
        \raisebox{\dp0}{\underline{\phantom{\the\SOUL@token}}}%
        \whiten{1}\whiten{0}%
        \whiten{-1}\whiten{-2}%
        \llap{\the\SOUL@token}%
     \else
        \underline{\the\SOUL@token}%
     \fi}%
\SOUL@}
\makeatother

\newcommand*{\demp}{\fontfamily{lmtt}\selectfont}

\DeclareTextFontCommand{\textdemp}{\demp}

\begin{document}

\ifcomment
Multiline
comment
\fi
\ifcomment
\myul{Typesetting test}
% \color[rgb]{1,1,1}
$∑_i^n≠ 60º±∞π∆¬≈√j∫h≤≥µ$

$\CR \R\pro\ind\pro\gS\pro
\mqty[a&b\\c&d]$
$\pro\mathbb{P}$
$\dd{x}$

  \[
    \alpha(x)=\left\{
                \begin{array}{ll}
                  x\\
                  \frac{1}{1+e^{-kx}}\\
                  \frac{e^x-e^{-x}}{e^x+e^{-x}}
                \end{array}
              \right.
  \]

  $\expval{x}$
  
  $\chi_\rho(ghg\dmo)=\Tr(\rho_{ghg\dmo})=\Tr(\rho_g\circ\rho_h\circ\rho\dmo_g)=\Tr(\rho_h)\overset{\mbox{\scalebox{0.5}{$\Tr(AB)=\Tr(BA)$}}}{=}\chi_\rho(h)$
  	$\mathop{\oplus}_{\substack{x\in X}}$

$\mat(\rho_g)=(a_{ij}(g))_{\scriptsize \substack{1\leq i\leq d \\ 1\leq j\leq d}}$ et $\mat(\rho'_g)=(a'_{ij}(g))_{\scriptsize \substack{1\leq i'\leq d' \\ 1\leq j'\leq d'}}$



\[\int_a^b{\mathbb{R}^2}g(u, v)\dd{P_{XY}}(u, v)=\iint g(u,v) f_{XY}(u, v)\dd \lambda(u) \dd \lambda(v)\]
$$\lim_{x\to\infty} f(x)$$	
$$\iiiint_V \mu(t,u,v,w) \,dt\,du\,dv\,dw$$
$$\sum_{n=1}^{\infty} 2^{-n} = 1$$	
\begin{definition}
	Si $X$ et $Y$ sont 2 v.a. ou definit la \textsc{Covariance} entre $X$ et $Y$ comme
	$\cov(X,Y)\overset{\text{def}}{=}\E\left[(X-\E(X))(Y-\E(Y))\right]=\E(XY)-\E(X)\E(Y)$.
\end{definition}
\fi
\pagebreak

% \tableofcontents

% insert your code here
%\input{./algebra/main.tex}
%\input{./geometrie-differentielle/main.tex}
%\input{./probabilite/main.tex}
%\input{./analyse-fonctionnelle/main.tex}
% \input{./Analyse-convexe-et-dualite-en-optimisation/main.tex}
%\input{./tikz/main.tex}
%\input{./Theorie-du-distributions/main.tex}
%\input{./optimisation/mine.tex}
 \input{./modelisation/main.tex}

% yves.aubry@univ-tln.fr : algebra

\end{document}


% yves.aubry@univ-tln.fr : algebra

\end{document}

%\input{./optimisation/mine.tex}
 % !TEX encoding = UTF-8 Unicode
% !TEX TS-program = xelatex

\documentclass[french]{report}

%\usepackage[utf8]{inputenc}
%\usepackage[T1]{fontenc}
\usepackage{babel}


\newif\ifcomment
%\commenttrue # Show comments

\usepackage{physics}
\usepackage{amssymb}


\usepackage{amsthm}
% \usepackage{thmtools}
\usepackage{mathtools}
\usepackage{amsfonts}

\usepackage{color}

\usepackage{tikz}

\usepackage{geometry}
\geometry{a5paper, margin=0.1in, right=1cm}

\usepackage{dsfont}

\usepackage{graphicx}
\graphicspath{ {images/} }

\usepackage{faktor}

\usepackage{IEEEtrantools}
\usepackage{enumerate}   
\usepackage[PostScript=dvips]{"/Users/aware/Documents/Courses/diagrams"}


\newtheorem{theorem}{Théorème}[section]
\renewcommand{\thetheorem}{\arabic{theorem}}
\newtheorem{lemme}{Lemme}[section]
\renewcommand{\thelemme}{\arabic{lemme}}
\newtheorem{proposition}{Proposition}[section]
\renewcommand{\theproposition}{\arabic{proposition}}
\newtheorem{notations}{Notations}[section]
\newtheorem{problem}{Problème}[section]
\newtheorem{corollary}{Corollaire}[theorem]
\renewcommand{\thecorollary}{\arabic{corollary}}
\newtheorem{property}{Propriété}[section]
\newtheorem{objective}{Objectif}[section]

\theoremstyle{definition}
\newtheorem{definition}{Définition}[section]
\renewcommand{\thedefinition}{\arabic{definition}}
\newtheorem{exercise}{Exercice}[chapter]
\renewcommand{\theexercise}{\arabic{exercise}}
\newtheorem{example}{Exemple}[chapter]
\renewcommand{\theexample}{\arabic{example}}
\newtheorem*{solution}{Solution}
\newtheorem*{application}{Application}
\newtheorem*{notation}{Notation}
\newtheorem*{vocabulary}{Vocabulaire}
\newtheorem*{properties}{Propriétés}



\theoremstyle{remark}
\newtheorem*{remark}{Remarque}
\newtheorem*{rappel}{Rappel}


\usepackage{etoolbox}
\AtBeginEnvironment{exercise}{\small}
\AtBeginEnvironment{example}{\small}

\usepackage{cases}
\usepackage[red]{mypack}

\usepackage[framemethod=TikZ]{mdframed}

\definecolor{bg}{rgb}{0.4,0.25,0.95}
\definecolor{pagebg}{rgb}{0,0,0.5}
\surroundwithmdframed[
   topline=false,
   rightline=false,
   bottomline=false,
   leftmargin=\parindent,
   skipabove=8pt,
   skipbelow=8pt,
   linecolor=blue,
   innerbottommargin=10pt,
   % backgroundcolor=bg,font=\color{orange}\sffamily, fontcolor=white
]{definition}

\usepackage{empheq}
\usepackage[most]{tcolorbox}

\newtcbox{\mymath}[1][]{%
    nobeforeafter, math upper, tcbox raise base,
    enhanced, colframe=blue!30!black,
    colback=red!10, boxrule=1pt,
    #1}

\usepackage{unixode}


\DeclareMathOperator{\ord}{ord}
\DeclareMathOperator{\orb}{orb}
\DeclareMathOperator{\stab}{stab}
\DeclareMathOperator{\Stab}{stab}
\DeclareMathOperator{\ppcm}{ppcm}
\DeclareMathOperator{\conj}{Conj}
\DeclareMathOperator{\End}{End}
\DeclareMathOperator{\rot}{rot}
\DeclareMathOperator{\trs}{trace}
\DeclareMathOperator{\Ind}{Ind}
\DeclareMathOperator{\mat}{Mat}
\DeclareMathOperator{\id}{Id}
\DeclareMathOperator{\vect}{vect}
\DeclareMathOperator{\img}{img}
\DeclareMathOperator{\cov}{Cov}
\DeclareMathOperator{\dist}{dist}
\DeclareMathOperator{\irr}{Irr}
\DeclareMathOperator{\image}{Im}
\DeclareMathOperator{\pd}{\partial}
\DeclareMathOperator{\epi}{epi}
\DeclareMathOperator{\Argmin}{Argmin}
\DeclareMathOperator{\dom}{dom}
\DeclareMathOperator{\proj}{proj}
\DeclareMathOperator{\ctg}{ctg}
\DeclareMathOperator{\supp}{supp}
\DeclareMathOperator{\argmin}{argmin}
\DeclareMathOperator{\mult}{mult}
\DeclareMathOperator{\ch}{ch}
\DeclareMathOperator{\sh}{sh}
\DeclareMathOperator{\rang}{rang}
\DeclareMathOperator{\diam}{diam}
\DeclareMathOperator{\Epigraphe}{Epigraphe}




\usepackage{xcolor}
\everymath{\color{blue}}
%\everymath{\color[rgb]{0,1,1}}
%\pagecolor[rgb]{0,0,0.5}


\newcommand*{\pdtest}[3][]{\ensuremath{\frac{\partial^{#1} #2}{\partial #3}}}

\newcommand*{\deffunc}[6][]{\ensuremath{
\begin{array}{rcl}
#2 : #3 &\rightarrow& #4\\
#5 &\mapsto& #6
\end{array}
}}

\newcommand{\eqcolon}{\mathrel{\resizebox{\widthof{$\mathord{=}$}}{\height}{ $\!\!=\!\!\resizebox{1.2\width}{0.8\height}{\raisebox{0.23ex}{$\mathop{:}$}}\!\!$ }}}
\newcommand{\coloneq}{\mathrel{\resizebox{\widthof{$\mathord{=}$}}{\height}{ $\!\!\resizebox{1.2\width}{0.8\height}{\raisebox{0.23ex}{$\mathop{:}$}}\!\!=\!\!$ }}}
\newcommand{\eqcolonl}{\ensuremath{\mathrel{=\!\!\mathop{:}}}}
\newcommand{\coloneql}{\ensuremath{\mathrel{\mathop{:} \!\! =}}}
\newcommand{\vc}[1]{% inline column vector
  \left(\begin{smallmatrix}#1\end{smallmatrix}\right)%
}
\newcommand{\vr}[1]{% inline row vector
  \begin{smallmatrix}(\,#1\,)\end{smallmatrix}%
}
\makeatletter
\newcommand*{\defeq}{\ =\mathrel{\rlap{%
                     \raisebox{0.3ex}{$\m@th\cdot$}}%
                     \raisebox{-0.3ex}{$\m@th\cdot$}}%
                     }
\makeatother

\newcommand{\mathcircle}[1]{% inline row vector
 \overset{\circ}{#1}
}
\newcommand{\ulim}{% low limit
 \underline{\lim}
}
\newcommand{\ssi}{% iff
\iff
}
\newcommand{\ps}[2]{
\expval{#1 | #2}
}
\newcommand{\df}[1]{
\mqty{#1}
}
\newcommand{\n}[1]{
\norm{#1}
}
\newcommand{\sys}[1]{
\left\{\smqty{#1}\right.
}


\newcommand{\eqdef}{\ensuremath{\overset{\text{def}}=}}


\def\Circlearrowright{\ensuremath{%
  \rotatebox[origin=c]{230}{$\circlearrowright$}}}

\newcommand\ct[1]{\text{\rmfamily\upshape #1}}
\newcommand\question[1]{ {\color{red} ...!? \small #1}}
\newcommand\caz[1]{\left\{\begin{array} #1 \end{array}\right.}
\newcommand\const{\text{\rmfamily\upshape const}}
\newcommand\toP{ \overset{\pro}{\to}}
\newcommand\toPP{ \overset{\text{PP}}{\to}}
\newcommand{\oeq}{\mathrel{\text{\textcircled{$=$}}}}





\usepackage{xcolor}
% \usepackage[normalem]{ulem}
\usepackage{lipsum}
\makeatletter
% \newcommand\colorwave[1][blue]{\bgroup \markoverwith{\lower3.5\p@\hbox{\sixly \textcolor{#1}{\char58}}}\ULon}
%\font\sixly=lasy6 % does not re-load if already loaded, so no memory problem.

\newmdtheoremenv[
linewidth= 1pt,linecolor= blue,%
leftmargin=20,rightmargin=20,innertopmargin=0pt, innerrightmargin=40,%
tikzsetting = { draw=lightgray, line width = 0.3pt,dashed,%
dash pattern = on 15pt off 3pt},%
splittopskip=\topskip,skipbelow=\baselineskip,%
skipabove=\baselineskip,ntheorem,roundcorner=0pt,
% backgroundcolor=pagebg,font=\color{orange}\sffamily, fontcolor=white
]{examplebox}{Exemple}[section]



\newcommand\R{\mathbb{R}}
\newcommand\Z{\mathbb{Z}}
\newcommand\N{\mathbb{N}}
\newcommand\E{\mathbb{E}}
\newcommand\F{\mathcal{F}}
\newcommand\cH{\mathcal{H}}
\newcommand\V{\mathbb{V}}
\newcommand\dmo{ ^{-1} }
\newcommand\kapa{\kappa}
\newcommand\im{Im}
\newcommand\hs{\mathcal{H}}





\usepackage{soul}

\makeatletter
\newcommand*{\whiten}[1]{\llap{\textcolor{white}{{\the\SOUL@token}}\hspace{#1pt}}}
\DeclareRobustCommand*\myul{%
    \def\SOUL@everyspace{\underline{\space}\kern\z@}%
    \def\SOUL@everytoken{%
     \setbox0=\hbox{\the\SOUL@token}%
     \ifdim\dp0>\z@
        \raisebox{\dp0}{\underline{\phantom{\the\SOUL@token}}}%
        \whiten{1}\whiten{0}%
        \whiten{-1}\whiten{-2}%
        \llap{\the\SOUL@token}%
     \else
        \underline{\the\SOUL@token}%
     \fi}%
\SOUL@}
\makeatother

\newcommand*{\demp}{\fontfamily{lmtt}\selectfont}

\DeclareTextFontCommand{\textdemp}{\demp}

\begin{document}

\ifcomment
Multiline
comment
\fi
\ifcomment
\myul{Typesetting test}
% \color[rgb]{1,1,1}
$∑_i^n≠ 60º±∞π∆¬≈√j∫h≤≥µ$

$\CR \R\pro\ind\pro\gS\pro
\mqty[a&b\\c&d]$
$\pro\mathbb{P}$
$\dd{x}$

  \[
    \alpha(x)=\left\{
                \begin{array}{ll}
                  x\\
                  \frac{1}{1+e^{-kx}}\\
                  \frac{e^x-e^{-x}}{e^x+e^{-x}}
                \end{array}
              \right.
  \]

  $\expval{x}$
  
  $\chi_\rho(ghg\dmo)=\Tr(\rho_{ghg\dmo})=\Tr(\rho_g\circ\rho_h\circ\rho\dmo_g)=\Tr(\rho_h)\overset{\mbox{\scalebox{0.5}{$\Tr(AB)=\Tr(BA)$}}}{=}\chi_\rho(h)$
  	$\mathop{\oplus}_{\substack{x\in X}}$

$\mat(\rho_g)=(a_{ij}(g))_{\scriptsize \substack{1\leq i\leq d \\ 1\leq j\leq d}}$ et $\mat(\rho'_g)=(a'_{ij}(g))_{\scriptsize \substack{1\leq i'\leq d' \\ 1\leq j'\leq d'}}$



\[\int_a^b{\mathbb{R}^2}g(u, v)\dd{P_{XY}}(u, v)=\iint g(u,v) f_{XY}(u, v)\dd \lambda(u) \dd \lambda(v)\]
$$\lim_{x\to\infty} f(x)$$	
$$\iiiint_V \mu(t,u,v,w) \,dt\,du\,dv\,dw$$
$$\sum_{n=1}^{\infty} 2^{-n} = 1$$	
\begin{definition}
	Si $X$ et $Y$ sont 2 v.a. ou definit la \textsc{Covariance} entre $X$ et $Y$ comme
	$\cov(X,Y)\overset{\text{def}}{=}\E\left[(X-\E(X))(Y-\E(Y))\right]=\E(XY)-\E(X)\E(Y)$.
\end{definition}
\fi
\pagebreak

% \tableofcontents

% insert your code here
%% !TEX encoding = UTF-8 Unicode
% !TEX TS-program = xelatex

\documentclass[french]{report}

%\usepackage[utf8]{inputenc}
%\usepackage[T1]{fontenc}
\usepackage{babel}


\newif\ifcomment
%\commenttrue # Show comments

\usepackage{physics}
\usepackage{amssymb}


\usepackage{amsthm}
% \usepackage{thmtools}
\usepackage{mathtools}
\usepackage{amsfonts}

\usepackage{color}

\usepackage{tikz}

\usepackage{geometry}
\geometry{a5paper, margin=0.1in, right=1cm}

\usepackage{dsfont}

\usepackage{graphicx}
\graphicspath{ {images/} }

\usepackage{faktor}

\usepackage{IEEEtrantools}
\usepackage{enumerate}   
\usepackage[PostScript=dvips]{"/Users/aware/Documents/Courses/diagrams"}


\newtheorem{theorem}{Théorème}[section]
\renewcommand{\thetheorem}{\arabic{theorem}}
\newtheorem{lemme}{Lemme}[section]
\renewcommand{\thelemme}{\arabic{lemme}}
\newtheorem{proposition}{Proposition}[section]
\renewcommand{\theproposition}{\arabic{proposition}}
\newtheorem{notations}{Notations}[section]
\newtheorem{problem}{Problème}[section]
\newtheorem{corollary}{Corollaire}[theorem]
\renewcommand{\thecorollary}{\arabic{corollary}}
\newtheorem{property}{Propriété}[section]
\newtheorem{objective}{Objectif}[section]

\theoremstyle{definition}
\newtheorem{definition}{Définition}[section]
\renewcommand{\thedefinition}{\arabic{definition}}
\newtheorem{exercise}{Exercice}[chapter]
\renewcommand{\theexercise}{\arabic{exercise}}
\newtheorem{example}{Exemple}[chapter]
\renewcommand{\theexample}{\arabic{example}}
\newtheorem*{solution}{Solution}
\newtheorem*{application}{Application}
\newtheorem*{notation}{Notation}
\newtheorem*{vocabulary}{Vocabulaire}
\newtheorem*{properties}{Propriétés}



\theoremstyle{remark}
\newtheorem*{remark}{Remarque}
\newtheorem*{rappel}{Rappel}


\usepackage{etoolbox}
\AtBeginEnvironment{exercise}{\small}
\AtBeginEnvironment{example}{\small}

\usepackage{cases}
\usepackage[red]{mypack}

\usepackage[framemethod=TikZ]{mdframed}

\definecolor{bg}{rgb}{0.4,0.25,0.95}
\definecolor{pagebg}{rgb}{0,0,0.5}
\surroundwithmdframed[
   topline=false,
   rightline=false,
   bottomline=false,
   leftmargin=\parindent,
   skipabove=8pt,
   skipbelow=8pt,
   linecolor=blue,
   innerbottommargin=10pt,
   % backgroundcolor=bg,font=\color{orange}\sffamily, fontcolor=white
]{definition}

\usepackage{empheq}
\usepackage[most]{tcolorbox}

\newtcbox{\mymath}[1][]{%
    nobeforeafter, math upper, tcbox raise base,
    enhanced, colframe=blue!30!black,
    colback=red!10, boxrule=1pt,
    #1}

\usepackage{unixode}


\DeclareMathOperator{\ord}{ord}
\DeclareMathOperator{\orb}{orb}
\DeclareMathOperator{\stab}{stab}
\DeclareMathOperator{\Stab}{stab}
\DeclareMathOperator{\ppcm}{ppcm}
\DeclareMathOperator{\conj}{Conj}
\DeclareMathOperator{\End}{End}
\DeclareMathOperator{\rot}{rot}
\DeclareMathOperator{\trs}{trace}
\DeclareMathOperator{\Ind}{Ind}
\DeclareMathOperator{\mat}{Mat}
\DeclareMathOperator{\id}{Id}
\DeclareMathOperator{\vect}{vect}
\DeclareMathOperator{\img}{img}
\DeclareMathOperator{\cov}{Cov}
\DeclareMathOperator{\dist}{dist}
\DeclareMathOperator{\irr}{Irr}
\DeclareMathOperator{\image}{Im}
\DeclareMathOperator{\pd}{\partial}
\DeclareMathOperator{\epi}{epi}
\DeclareMathOperator{\Argmin}{Argmin}
\DeclareMathOperator{\dom}{dom}
\DeclareMathOperator{\proj}{proj}
\DeclareMathOperator{\ctg}{ctg}
\DeclareMathOperator{\supp}{supp}
\DeclareMathOperator{\argmin}{argmin}
\DeclareMathOperator{\mult}{mult}
\DeclareMathOperator{\ch}{ch}
\DeclareMathOperator{\sh}{sh}
\DeclareMathOperator{\rang}{rang}
\DeclareMathOperator{\diam}{diam}
\DeclareMathOperator{\Epigraphe}{Epigraphe}




\usepackage{xcolor}
\everymath{\color{blue}}
%\everymath{\color[rgb]{0,1,1}}
%\pagecolor[rgb]{0,0,0.5}


\newcommand*{\pdtest}[3][]{\ensuremath{\frac{\partial^{#1} #2}{\partial #3}}}

\newcommand*{\deffunc}[6][]{\ensuremath{
\begin{array}{rcl}
#2 : #3 &\rightarrow& #4\\
#5 &\mapsto& #6
\end{array}
}}

\newcommand{\eqcolon}{\mathrel{\resizebox{\widthof{$\mathord{=}$}}{\height}{ $\!\!=\!\!\resizebox{1.2\width}{0.8\height}{\raisebox{0.23ex}{$\mathop{:}$}}\!\!$ }}}
\newcommand{\coloneq}{\mathrel{\resizebox{\widthof{$\mathord{=}$}}{\height}{ $\!\!\resizebox{1.2\width}{0.8\height}{\raisebox{0.23ex}{$\mathop{:}$}}\!\!=\!\!$ }}}
\newcommand{\eqcolonl}{\ensuremath{\mathrel{=\!\!\mathop{:}}}}
\newcommand{\coloneql}{\ensuremath{\mathrel{\mathop{:} \!\! =}}}
\newcommand{\vc}[1]{% inline column vector
  \left(\begin{smallmatrix}#1\end{smallmatrix}\right)%
}
\newcommand{\vr}[1]{% inline row vector
  \begin{smallmatrix}(\,#1\,)\end{smallmatrix}%
}
\makeatletter
\newcommand*{\defeq}{\ =\mathrel{\rlap{%
                     \raisebox{0.3ex}{$\m@th\cdot$}}%
                     \raisebox{-0.3ex}{$\m@th\cdot$}}%
                     }
\makeatother

\newcommand{\mathcircle}[1]{% inline row vector
 \overset{\circ}{#1}
}
\newcommand{\ulim}{% low limit
 \underline{\lim}
}
\newcommand{\ssi}{% iff
\iff
}
\newcommand{\ps}[2]{
\expval{#1 | #2}
}
\newcommand{\df}[1]{
\mqty{#1}
}
\newcommand{\n}[1]{
\norm{#1}
}
\newcommand{\sys}[1]{
\left\{\smqty{#1}\right.
}


\newcommand{\eqdef}{\ensuremath{\overset{\text{def}}=}}


\def\Circlearrowright{\ensuremath{%
  \rotatebox[origin=c]{230}{$\circlearrowright$}}}

\newcommand\ct[1]{\text{\rmfamily\upshape #1}}
\newcommand\question[1]{ {\color{red} ...!? \small #1}}
\newcommand\caz[1]{\left\{\begin{array} #1 \end{array}\right.}
\newcommand\const{\text{\rmfamily\upshape const}}
\newcommand\toP{ \overset{\pro}{\to}}
\newcommand\toPP{ \overset{\text{PP}}{\to}}
\newcommand{\oeq}{\mathrel{\text{\textcircled{$=$}}}}





\usepackage{xcolor}
% \usepackage[normalem]{ulem}
\usepackage{lipsum}
\makeatletter
% \newcommand\colorwave[1][blue]{\bgroup \markoverwith{\lower3.5\p@\hbox{\sixly \textcolor{#1}{\char58}}}\ULon}
%\font\sixly=lasy6 % does not re-load if already loaded, so no memory problem.

\newmdtheoremenv[
linewidth= 1pt,linecolor= blue,%
leftmargin=20,rightmargin=20,innertopmargin=0pt, innerrightmargin=40,%
tikzsetting = { draw=lightgray, line width = 0.3pt,dashed,%
dash pattern = on 15pt off 3pt},%
splittopskip=\topskip,skipbelow=\baselineskip,%
skipabove=\baselineskip,ntheorem,roundcorner=0pt,
% backgroundcolor=pagebg,font=\color{orange}\sffamily, fontcolor=white
]{examplebox}{Exemple}[section]



\newcommand\R{\mathbb{R}}
\newcommand\Z{\mathbb{Z}}
\newcommand\N{\mathbb{N}}
\newcommand\E{\mathbb{E}}
\newcommand\F{\mathcal{F}}
\newcommand\cH{\mathcal{H}}
\newcommand\V{\mathbb{V}}
\newcommand\dmo{ ^{-1} }
\newcommand\kapa{\kappa}
\newcommand\im{Im}
\newcommand\hs{\mathcal{H}}





\usepackage{soul}

\makeatletter
\newcommand*{\whiten}[1]{\llap{\textcolor{white}{{\the\SOUL@token}}\hspace{#1pt}}}
\DeclareRobustCommand*\myul{%
    \def\SOUL@everyspace{\underline{\space}\kern\z@}%
    \def\SOUL@everytoken{%
     \setbox0=\hbox{\the\SOUL@token}%
     \ifdim\dp0>\z@
        \raisebox{\dp0}{\underline{\phantom{\the\SOUL@token}}}%
        \whiten{1}\whiten{0}%
        \whiten{-1}\whiten{-2}%
        \llap{\the\SOUL@token}%
     \else
        \underline{\the\SOUL@token}%
     \fi}%
\SOUL@}
\makeatother

\newcommand*{\demp}{\fontfamily{lmtt}\selectfont}

\DeclareTextFontCommand{\textdemp}{\demp}

\begin{document}

\ifcomment
Multiline
comment
\fi
\ifcomment
\myul{Typesetting test}
% \color[rgb]{1,1,1}
$∑_i^n≠ 60º±∞π∆¬≈√j∫h≤≥µ$

$\CR \R\pro\ind\pro\gS\pro
\mqty[a&b\\c&d]$
$\pro\mathbb{P}$
$\dd{x}$

  \[
    \alpha(x)=\left\{
                \begin{array}{ll}
                  x\\
                  \frac{1}{1+e^{-kx}}\\
                  \frac{e^x-e^{-x}}{e^x+e^{-x}}
                \end{array}
              \right.
  \]

  $\expval{x}$
  
  $\chi_\rho(ghg\dmo)=\Tr(\rho_{ghg\dmo})=\Tr(\rho_g\circ\rho_h\circ\rho\dmo_g)=\Tr(\rho_h)\overset{\mbox{\scalebox{0.5}{$\Tr(AB)=\Tr(BA)$}}}{=}\chi_\rho(h)$
  	$\mathop{\oplus}_{\substack{x\in X}}$

$\mat(\rho_g)=(a_{ij}(g))_{\scriptsize \substack{1\leq i\leq d \\ 1\leq j\leq d}}$ et $\mat(\rho'_g)=(a'_{ij}(g))_{\scriptsize \substack{1\leq i'\leq d' \\ 1\leq j'\leq d'}}$



\[\int_a^b{\mathbb{R}^2}g(u, v)\dd{P_{XY}}(u, v)=\iint g(u,v) f_{XY}(u, v)\dd \lambda(u) \dd \lambda(v)\]
$$\lim_{x\to\infty} f(x)$$	
$$\iiiint_V \mu(t,u,v,w) \,dt\,du\,dv\,dw$$
$$\sum_{n=1}^{\infty} 2^{-n} = 1$$	
\begin{definition}
	Si $X$ et $Y$ sont 2 v.a. ou definit la \textsc{Covariance} entre $X$ et $Y$ comme
	$\cov(X,Y)\overset{\text{def}}{=}\E\left[(X-\E(X))(Y-\E(Y))\right]=\E(XY)-\E(X)\E(Y)$.
\end{definition}
\fi
\pagebreak

% \tableofcontents

% insert your code here
%\input{./algebra/main.tex}
%\input{./geometrie-differentielle/main.tex}
%\input{./probabilite/main.tex}
%\input{./analyse-fonctionnelle/main.tex}
% \input{./Analyse-convexe-et-dualite-en-optimisation/main.tex}
%\input{./tikz/main.tex}
%\input{./Theorie-du-distributions/main.tex}
%\input{./optimisation/mine.tex}
 \input{./modelisation/main.tex}

% yves.aubry@univ-tln.fr : algebra

\end{document}

%% !TEX encoding = UTF-8 Unicode
% !TEX TS-program = xelatex

\documentclass[french]{report}

%\usepackage[utf8]{inputenc}
%\usepackage[T1]{fontenc}
\usepackage{babel}


\newif\ifcomment
%\commenttrue # Show comments

\usepackage{physics}
\usepackage{amssymb}


\usepackage{amsthm}
% \usepackage{thmtools}
\usepackage{mathtools}
\usepackage{amsfonts}

\usepackage{color}

\usepackage{tikz}

\usepackage{geometry}
\geometry{a5paper, margin=0.1in, right=1cm}

\usepackage{dsfont}

\usepackage{graphicx}
\graphicspath{ {images/} }

\usepackage{faktor}

\usepackage{IEEEtrantools}
\usepackage{enumerate}   
\usepackage[PostScript=dvips]{"/Users/aware/Documents/Courses/diagrams"}


\newtheorem{theorem}{Théorème}[section]
\renewcommand{\thetheorem}{\arabic{theorem}}
\newtheorem{lemme}{Lemme}[section]
\renewcommand{\thelemme}{\arabic{lemme}}
\newtheorem{proposition}{Proposition}[section]
\renewcommand{\theproposition}{\arabic{proposition}}
\newtheorem{notations}{Notations}[section]
\newtheorem{problem}{Problème}[section]
\newtheorem{corollary}{Corollaire}[theorem]
\renewcommand{\thecorollary}{\arabic{corollary}}
\newtheorem{property}{Propriété}[section]
\newtheorem{objective}{Objectif}[section]

\theoremstyle{definition}
\newtheorem{definition}{Définition}[section]
\renewcommand{\thedefinition}{\arabic{definition}}
\newtheorem{exercise}{Exercice}[chapter]
\renewcommand{\theexercise}{\arabic{exercise}}
\newtheorem{example}{Exemple}[chapter]
\renewcommand{\theexample}{\arabic{example}}
\newtheorem*{solution}{Solution}
\newtheorem*{application}{Application}
\newtheorem*{notation}{Notation}
\newtheorem*{vocabulary}{Vocabulaire}
\newtheorem*{properties}{Propriétés}



\theoremstyle{remark}
\newtheorem*{remark}{Remarque}
\newtheorem*{rappel}{Rappel}


\usepackage{etoolbox}
\AtBeginEnvironment{exercise}{\small}
\AtBeginEnvironment{example}{\small}

\usepackage{cases}
\usepackage[red]{mypack}

\usepackage[framemethod=TikZ]{mdframed}

\definecolor{bg}{rgb}{0.4,0.25,0.95}
\definecolor{pagebg}{rgb}{0,0,0.5}
\surroundwithmdframed[
   topline=false,
   rightline=false,
   bottomline=false,
   leftmargin=\parindent,
   skipabove=8pt,
   skipbelow=8pt,
   linecolor=blue,
   innerbottommargin=10pt,
   % backgroundcolor=bg,font=\color{orange}\sffamily, fontcolor=white
]{definition}

\usepackage{empheq}
\usepackage[most]{tcolorbox}

\newtcbox{\mymath}[1][]{%
    nobeforeafter, math upper, tcbox raise base,
    enhanced, colframe=blue!30!black,
    colback=red!10, boxrule=1pt,
    #1}

\usepackage{unixode}


\DeclareMathOperator{\ord}{ord}
\DeclareMathOperator{\orb}{orb}
\DeclareMathOperator{\stab}{stab}
\DeclareMathOperator{\Stab}{stab}
\DeclareMathOperator{\ppcm}{ppcm}
\DeclareMathOperator{\conj}{Conj}
\DeclareMathOperator{\End}{End}
\DeclareMathOperator{\rot}{rot}
\DeclareMathOperator{\trs}{trace}
\DeclareMathOperator{\Ind}{Ind}
\DeclareMathOperator{\mat}{Mat}
\DeclareMathOperator{\id}{Id}
\DeclareMathOperator{\vect}{vect}
\DeclareMathOperator{\img}{img}
\DeclareMathOperator{\cov}{Cov}
\DeclareMathOperator{\dist}{dist}
\DeclareMathOperator{\irr}{Irr}
\DeclareMathOperator{\image}{Im}
\DeclareMathOperator{\pd}{\partial}
\DeclareMathOperator{\epi}{epi}
\DeclareMathOperator{\Argmin}{Argmin}
\DeclareMathOperator{\dom}{dom}
\DeclareMathOperator{\proj}{proj}
\DeclareMathOperator{\ctg}{ctg}
\DeclareMathOperator{\supp}{supp}
\DeclareMathOperator{\argmin}{argmin}
\DeclareMathOperator{\mult}{mult}
\DeclareMathOperator{\ch}{ch}
\DeclareMathOperator{\sh}{sh}
\DeclareMathOperator{\rang}{rang}
\DeclareMathOperator{\diam}{diam}
\DeclareMathOperator{\Epigraphe}{Epigraphe}




\usepackage{xcolor}
\everymath{\color{blue}}
%\everymath{\color[rgb]{0,1,1}}
%\pagecolor[rgb]{0,0,0.5}


\newcommand*{\pdtest}[3][]{\ensuremath{\frac{\partial^{#1} #2}{\partial #3}}}

\newcommand*{\deffunc}[6][]{\ensuremath{
\begin{array}{rcl}
#2 : #3 &\rightarrow& #4\\
#5 &\mapsto& #6
\end{array}
}}

\newcommand{\eqcolon}{\mathrel{\resizebox{\widthof{$\mathord{=}$}}{\height}{ $\!\!=\!\!\resizebox{1.2\width}{0.8\height}{\raisebox{0.23ex}{$\mathop{:}$}}\!\!$ }}}
\newcommand{\coloneq}{\mathrel{\resizebox{\widthof{$\mathord{=}$}}{\height}{ $\!\!\resizebox{1.2\width}{0.8\height}{\raisebox{0.23ex}{$\mathop{:}$}}\!\!=\!\!$ }}}
\newcommand{\eqcolonl}{\ensuremath{\mathrel{=\!\!\mathop{:}}}}
\newcommand{\coloneql}{\ensuremath{\mathrel{\mathop{:} \!\! =}}}
\newcommand{\vc}[1]{% inline column vector
  \left(\begin{smallmatrix}#1\end{smallmatrix}\right)%
}
\newcommand{\vr}[1]{% inline row vector
  \begin{smallmatrix}(\,#1\,)\end{smallmatrix}%
}
\makeatletter
\newcommand*{\defeq}{\ =\mathrel{\rlap{%
                     \raisebox{0.3ex}{$\m@th\cdot$}}%
                     \raisebox{-0.3ex}{$\m@th\cdot$}}%
                     }
\makeatother

\newcommand{\mathcircle}[1]{% inline row vector
 \overset{\circ}{#1}
}
\newcommand{\ulim}{% low limit
 \underline{\lim}
}
\newcommand{\ssi}{% iff
\iff
}
\newcommand{\ps}[2]{
\expval{#1 | #2}
}
\newcommand{\df}[1]{
\mqty{#1}
}
\newcommand{\n}[1]{
\norm{#1}
}
\newcommand{\sys}[1]{
\left\{\smqty{#1}\right.
}


\newcommand{\eqdef}{\ensuremath{\overset{\text{def}}=}}


\def\Circlearrowright{\ensuremath{%
  \rotatebox[origin=c]{230}{$\circlearrowright$}}}

\newcommand\ct[1]{\text{\rmfamily\upshape #1}}
\newcommand\question[1]{ {\color{red} ...!? \small #1}}
\newcommand\caz[1]{\left\{\begin{array} #1 \end{array}\right.}
\newcommand\const{\text{\rmfamily\upshape const}}
\newcommand\toP{ \overset{\pro}{\to}}
\newcommand\toPP{ \overset{\text{PP}}{\to}}
\newcommand{\oeq}{\mathrel{\text{\textcircled{$=$}}}}





\usepackage{xcolor}
% \usepackage[normalem]{ulem}
\usepackage{lipsum}
\makeatletter
% \newcommand\colorwave[1][blue]{\bgroup \markoverwith{\lower3.5\p@\hbox{\sixly \textcolor{#1}{\char58}}}\ULon}
%\font\sixly=lasy6 % does not re-load if already loaded, so no memory problem.

\newmdtheoremenv[
linewidth= 1pt,linecolor= blue,%
leftmargin=20,rightmargin=20,innertopmargin=0pt, innerrightmargin=40,%
tikzsetting = { draw=lightgray, line width = 0.3pt,dashed,%
dash pattern = on 15pt off 3pt},%
splittopskip=\topskip,skipbelow=\baselineskip,%
skipabove=\baselineskip,ntheorem,roundcorner=0pt,
% backgroundcolor=pagebg,font=\color{orange}\sffamily, fontcolor=white
]{examplebox}{Exemple}[section]



\newcommand\R{\mathbb{R}}
\newcommand\Z{\mathbb{Z}}
\newcommand\N{\mathbb{N}}
\newcommand\E{\mathbb{E}}
\newcommand\F{\mathcal{F}}
\newcommand\cH{\mathcal{H}}
\newcommand\V{\mathbb{V}}
\newcommand\dmo{ ^{-1} }
\newcommand\kapa{\kappa}
\newcommand\im{Im}
\newcommand\hs{\mathcal{H}}





\usepackage{soul}

\makeatletter
\newcommand*{\whiten}[1]{\llap{\textcolor{white}{{\the\SOUL@token}}\hspace{#1pt}}}
\DeclareRobustCommand*\myul{%
    \def\SOUL@everyspace{\underline{\space}\kern\z@}%
    \def\SOUL@everytoken{%
     \setbox0=\hbox{\the\SOUL@token}%
     \ifdim\dp0>\z@
        \raisebox{\dp0}{\underline{\phantom{\the\SOUL@token}}}%
        \whiten{1}\whiten{0}%
        \whiten{-1}\whiten{-2}%
        \llap{\the\SOUL@token}%
     \else
        \underline{\the\SOUL@token}%
     \fi}%
\SOUL@}
\makeatother

\newcommand*{\demp}{\fontfamily{lmtt}\selectfont}

\DeclareTextFontCommand{\textdemp}{\demp}

\begin{document}

\ifcomment
Multiline
comment
\fi
\ifcomment
\myul{Typesetting test}
% \color[rgb]{1,1,1}
$∑_i^n≠ 60º±∞π∆¬≈√j∫h≤≥µ$

$\CR \R\pro\ind\pro\gS\pro
\mqty[a&b\\c&d]$
$\pro\mathbb{P}$
$\dd{x}$

  \[
    \alpha(x)=\left\{
                \begin{array}{ll}
                  x\\
                  \frac{1}{1+e^{-kx}}\\
                  \frac{e^x-e^{-x}}{e^x+e^{-x}}
                \end{array}
              \right.
  \]

  $\expval{x}$
  
  $\chi_\rho(ghg\dmo)=\Tr(\rho_{ghg\dmo})=\Tr(\rho_g\circ\rho_h\circ\rho\dmo_g)=\Tr(\rho_h)\overset{\mbox{\scalebox{0.5}{$\Tr(AB)=\Tr(BA)$}}}{=}\chi_\rho(h)$
  	$\mathop{\oplus}_{\substack{x\in X}}$

$\mat(\rho_g)=(a_{ij}(g))_{\scriptsize \substack{1\leq i\leq d \\ 1\leq j\leq d}}$ et $\mat(\rho'_g)=(a'_{ij}(g))_{\scriptsize \substack{1\leq i'\leq d' \\ 1\leq j'\leq d'}}$



\[\int_a^b{\mathbb{R}^2}g(u, v)\dd{P_{XY}}(u, v)=\iint g(u,v) f_{XY}(u, v)\dd \lambda(u) \dd \lambda(v)\]
$$\lim_{x\to\infty} f(x)$$	
$$\iiiint_V \mu(t,u,v,w) \,dt\,du\,dv\,dw$$
$$\sum_{n=1}^{\infty} 2^{-n} = 1$$	
\begin{definition}
	Si $X$ et $Y$ sont 2 v.a. ou definit la \textsc{Covariance} entre $X$ et $Y$ comme
	$\cov(X,Y)\overset{\text{def}}{=}\E\left[(X-\E(X))(Y-\E(Y))\right]=\E(XY)-\E(X)\E(Y)$.
\end{definition}
\fi
\pagebreak

% \tableofcontents

% insert your code here
%\input{./algebra/main.tex}
%\input{./geometrie-differentielle/main.tex}
%\input{./probabilite/main.tex}
%\input{./analyse-fonctionnelle/main.tex}
% \input{./Analyse-convexe-et-dualite-en-optimisation/main.tex}
%\input{./tikz/main.tex}
%\input{./Theorie-du-distributions/main.tex}
%\input{./optimisation/mine.tex}
 \input{./modelisation/main.tex}

% yves.aubry@univ-tln.fr : algebra

\end{document}

%% !TEX encoding = UTF-8 Unicode
% !TEX TS-program = xelatex

\documentclass[french]{report}

%\usepackage[utf8]{inputenc}
%\usepackage[T1]{fontenc}
\usepackage{babel}


\newif\ifcomment
%\commenttrue # Show comments

\usepackage{physics}
\usepackage{amssymb}


\usepackage{amsthm}
% \usepackage{thmtools}
\usepackage{mathtools}
\usepackage{amsfonts}

\usepackage{color}

\usepackage{tikz}

\usepackage{geometry}
\geometry{a5paper, margin=0.1in, right=1cm}

\usepackage{dsfont}

\usepackage{graphicx}
\graphicspath{ {images/} }

\usepackage{faktor}

\usepackage{IEEEtrantools}
\usepackage{enumerate}   
\usepackage[PostScript=dvips]{"/Users/aware/Documents/Courses/diagrams"}


\newtheorem{theorem}{Théorème}[section]
\renewcommand{\thetheorem}{\arabic{theorem}}
\newtheorem{lemme}{Lemme}[section]
\renewcommand{\thelemme}{\arabic{lemme}}
\newtheorem{proposition}{Proposition}[section]
\renewcommand{\theproposition}{\arabic{proposition}}
\newtheorem{notations}{Notations}[section]
\newtheorem{problem}{Problème}[section]
\newtheorem{corollary}{Corollaire}[theorem]
\renewcommand{\thecorollary}{\arabic{corollary}}
\newtheorem{property}{Propriété}[section]
\newtheorem{objective}{Objectif}[section]

\theoremstyle{definition}
\newtheorem{definition}{Définition}[section]
\renewcommand{\thedefinition}{\arabic{definition}}
\newtheorem{exercise}{Exercice}[chapter]
\renewcommand{\theexercise}{\arabic{exercise}}
\newtheorem{example}{Exemple}[chapter]
\renewcommand{\theexample}{\arabic{example}}
\newtheorem*{solution}{Solution}
\newtheorem*{application}{Application}
\newtheorem*{notation}{Notation}
\newtheorem*{vocabulary}{Vocabulaire}
\newtheorem*{properties}{Propriétés}



\theoremstyle{remark}
\newtheorem*{remark}{Remarque}
\newtheorem*{rappel}{Rappel}


\usepackage{etoolbox}
\AtBeginEnvironment{exercise}{\small}
\AtBeginEnvironment{example}{\small}

\usepackage{cases}
\usepackage[red]{mypack}

\usepackage[framemethod=TikZ]{mdframed}

\definecolor{bg}{rgb}{0.4,0.25,0.95}
\definecolor{pagebg}{rgb}{0,0,0.5}
\surroundwithmdframed[
   topline=false,
   rightline=false,
   bottomline=false,
   leftmargin=\parindent,
   skipabove=8pt,
   skipbelow=8pt,
   linecolor=blue,
   innerbottommargin=10pt,
   % backgroundcolor=bg,font=\color{orange}\sffamily, fontcolor=white
]{definition}

\usepackage{empheq}
\usepackage[most]{tcolorbox}

\newtcbox{\mymath}[1][]{%
    nobeforeafter, math upper, tcbox raise base,
    enhanced, colframe=blue!30!black,
    colback=red!10, boxrule=1pt,
    #1}

\usepackage{unixode}


\DeclareMathOperator{\ord}{ord}
\DeclareMathOperator{\orb}{orb}
\DeclareMathOperator{\stab}{stab}
\DeclareMathOperator{\Stab}{stab}
\DeclareMathOperator{\ppcm}{ppcm}
\DeclareMathOperator{\conj}{Conj}
\DeclareMathOperator{\End}{End}
\DeclareMathOperator{\rot}{rot}
\DeclareMathOperator{\trs}{trace}
\DeclareMathOperator{\Ind}{Ind}
\DeclareMathOperator{\mat}{Mat}
\DeclareMathOperator{\id}{Id}
\DeclareMathOperator{\vect}{vect}
\DeclareMathOperator{\img}{img}
\DeclareMathOperator{\cov}{Cov}
\DeclareMathOperator{\dist}{dist}
\DeclareMathOperator{\irr}{Irr}
\DeclareMathOperator{\image}{Im}
\DeclareMathOperator{\pd}{\partial}
\DeclareMathOperator{\epi}{epi}
\DeclareMathOperator{\Argmin}{Argmin}
\DeclareMathOperator{\dom}{dom}
\DeclareMathOperator{\proj}{proj}
\DeclareMathOperator{\ctg}{ctg}
\DeclareMathOperator{\supp}{supp}
\DeclareMathOperator{\argmin}{argmin}
\DeclareMathOperator{\mult}{mult}
\DeclareMathOperator{\ch}{ch}
\DeclareMathOperator{\sh}{sh}
\DeclareMathOperator{\rang}{rang}
\DeclareMathOperator{\diam}{diam}
\DeclareMathOperator{\Epigraphe}{Epigraphe}




\usepackage{xcolor}
\everymath{\color{blue}}
%\everymath{\color[rgb]{0,1,1}}
%\pagecolor[rgb]{0,0,0.5}


\newcommand*{\pdtest}[3][]{\ensuremath{\frac{\partial^{#1} #2}{\partial #3}}}

\newcommand*{\deffunc}[6][]{\ensuremath{
\begin{array}{rcl}
#2 : #3 &\rightarrow& #4\\
#5 &\mapsto& #6
\end{array}
}}

\newcommand{\eqcolon}{\mathrel{\resizebox{\widthof{$\mathord{=}$}}{\height}{ $\!\!=\!\!\resizebox{1.2\width}{0.8\height}{\raisebox{0.23ex}{$\mathop{:}$}}\!\!$ }}}
\newcommand{\coloneq}{\mathrel{\resizebox{\widthof{$\mathord{=}$}}{\height}{ $\!\!\resizebox{1.2\width}{0.8\height}{\raisebox{0.23ex}{$\mathop{:}$}}\!\!=\!\!$ }}}
\newcommand{\eqcolonl}{\ensuremath{\mathrel{=\!\!\mathop{:}}}}
\newcommand{\coloneql}{\ensuremath{\mathrel{\mathop{:} \!\! =}}}
\newcommand{\vc}[1]{% inline column vector
  \left(\begin{smallmatrix}#1\end{smallmatrix}\right)%
}
\newcommand{\vr}[1]{% inline row vector
  \begin{smallmatrix}(\,#1\,)\end{smallmatrix}%
}
\makeatletter
\newcommand*{\defeq}{\ =\mathrel{\rlap{%
                     \raisebox{0.3ex}{$\m@th\cdot$}}%
                     \raisebox{-0.3ex}{$\m@th\cdot$}}%
                     }
\makeatother

\newcommand{\mathcircle}[1]{% inline row vector
 \overset{\circ}{#1}
}
\newcommand{\ulim}{% low limit
 \underline{\lim}
}
\newcommand{\ssi}{% iff
\iff
}
\newcommand{\ps}[2]{
\expval{#1 | #2}
}
\newcommand{\df}[1]{
\mqty{#1}
}
\newcommand{\n}[1]{
\norm{#1}
}
\newcommand{\sys}[1]{
\left\{\smqty{#1}\right.
}


\newcommand{\eqdef}{\ensuremath{\overset{\text{def}}=}}


\def\Circlearrowright{\ensuremath{%
  \rotatebox[origin=c]{230}{$\circlearrowright$}}}

\newcommand\ct[1]{\text{\rmfamily\upshape #1}}
\newcommand\question[1]{ {\color{red} ...!? \small #1}}
\newcommand\caz[1]{\left\{\begin{array} #1 \end{array}\right.}
\newcommand\const{\text{\rmfamily\upshape const}}
\newcommand\toP{ \overset{\pro}{\to}}
\newcommand\toPP{ \overset{\text{PP}}{\to}}
\newcommand{\oeq}{\mathrel{\text{\textcircled{$=$}}}}





\usepackage{xcolor}
% \usepackage[normalem]{ulem}
\usepackage{lipsum}
\makeatletter
% \newcommand\colorwave[1][blue]{\bgroup \markoverwith{\lower3.5\p@\hbox{\sixly \textcolor{#1}{\char58}}}\ULon}
%\font\sixly=lasy6 % does not re-load if already loaded, so no memory problem.

\newmdtheoremenv[
linewidth= 1pt,linecolor= blue,%
leftmargin=20,rightmargin=20,innertopmargin=0pt, innerrightmargin=40,%
tikzsetting = { draw=lightgray, line width = 0.3pt,dashed,%
dash pattern = on 15pt off 3pt},%
splittopskip=\topskip,skipbelow=\baselineskip,%
skipabove=\baselineskip,ntheorem,roundcorner=0pt,
% backgroundcolor=pagebg,font=\color{orange}\sffamily, fontcolor=white
]{examplebox}{Exemple}[section]



\newcommand\R{\mathbb{R}}
\newcommand\Z{\mathbb{Z}}
\newcommand\N{\mathbb{N}}
\newcommand\E{\mathbb{E}}
\newcommand\F{\mathcal{F}}
\newcommand\cH{\mathcal{H}}
\newcommand\V{\mathbb{V}}
\newcommand\dmo{ ^{-1} }
\newcommand\kapa{\kappa}
\newcommand\im{Im}
\newcommand\hs{\mathcal{H}}





\usepackage{soul}

\makeatletter
\newcommand*{\whiten}[1]{\llap{\textcolor{white}{{\the\SOUL@token}}\hspace{#1pt}}}
\DeclareRobustCommand*\myul{%
    \def\SOUL@everyspace{\underline{\space}\kern\z@}%
    \def\SOUL@everytoken{%
     \setbox0=\hbox{\the\SOUL@token}%
     \ifdim\dp0>\z@
        \raisebox{\dp0}{\underline{\phantom{\the\SOUL@token}}}%
        \whiten{1}\whiten{0}%
        \whiten{-1}\whiten{-2}%
        \llap{\the\SOUL@token}%
     \else
        \underline{\the\SOUL@token}%
     \fi}%
\SOUL@}
\makeatother

\newcommand*{\demp}{\fontfamily{lmtt}\selectfont}

\DeclareTextFontCommand{\textdemp}{\demp}

\begin{document}

\ifcomment
Multiline
comment
\fi
\ifcomment
\myul{Typesetting test}
% \color[rgb]{1,1,1}
$∑_i^n≠ 60º±∞π∆¬≈√j∫h≤≥µ$

$\CR \R\pro\ind\pro\gS\pro
\mqty[a&b\\c&d]$
$\pro\mathbb{P}$
$\dd{x}$

  \[
    \alpha(x)=\left\{
                \begin{array}{ll}
                  x\\
                  \frac{1}{1+e^{-kx}}\\
                  \frac{e^x-e^{-x}}{e^x+e^{-x}}
                \end{array}
              \right.
  \]

  $\expval{x}$
  
  $\chi_\rho(ghg\dmo)=\Tr(\rho_{ghg\dmo})=\Tr(\rho_g\circ\rho_h\circ\rho\dmo_g)=\Tr(\rho_h)\overset{\mbox{\scalebox{0.5}{$\Tr(AB)=\Tr(BA)$}}}{=}\chi_\rho(h)$
  	$\mathop{\oplus}_{\substack{x\in X}}$

$\mat(\rho_g)=(a_{ij}(g))_{\scriptsize \substack{1\leq i\leq d \\ 1\leq j\leq d}}$ et $\mat(\rho'_g)=(a'_{ij}(g))_{\scriptsize \substack{1\leq i'\leq d' \\ 1\leq j'\leq d'}}$



\[\int_a^b{\mathbb{R}^2}g(u, v)\dd{P_{XY}}(u, v)=\iint g(u,v) f_{XY}(u, v)\dd \lambda(u) \dd \lambda(v)\]
$$\lim_{x\to\infty} f(x)$$	
$$\iiiint_V \mu(t,u,v,w) \,dt\,du\,dv\,dw$$
$$\sum_{n=1}^{\infty} 2^{-n} = 1$$	
\begin{definition}
	Si $X$ et $Y$ sont 2 v.a. ou definit la \textsc{Covariance} entre $X$ et $Y$ comme
	$\cov(X,Y)\overset{\text{def}}{=}\E\left[(X-\E(X))(Y-\E(Y))\right]=\E(XY)-\E(X)\E(Y)$.
\end{definition}
\fi
\pagebreak

% \tableofcontents

% insert your code here
%\input{./algebra/main.tex}
%\input{./geometrie-differentielle/main.tex}
%\input{./probabilite/main.tex}
%\input{./analyse-fonctionnelle/main.tex}
% \input{./Analyse-convexe-et-dualite-en-optimisation/main.tex}
%\input{./tikz/main.tex}
%\input{./Theorie-du-distributions/main.tex}
%\input{./optimisation/mine.tex}
 \input{./modelisation/main.tex}

% yves.aubry@univ-tln.fr : algebra

\end{document}

%% !TEX encoding = UTF-8 Unicode
% !TEX TS-program = xelatex

\documentclass[french]{report}

%\usepackage[utf8]{inputenc}
%\usepackage[T1]{fontenc}
\usepackage{babel}


\newif\ifcomment
%\commenttrue # Show comments

\usepackage{physics}
\usepackage{amssymb}


\usepackage{amsthm}
% \usepackage{thmtools}
\usepackage{mathtools}
\usepackage{amsfonts}

\usepackage{color}

\usepackage{tikz}

\usepackage{geometry}
\geometry{a5paper, margin=0.1in, right=1cm}

\usepackage{dsfont}

\usepackage{graphicx}
\graphicspath{ {images/} }

\usepackage{faktor}

\usepackage{IEEEtrantools}
\usepackage{enumerate}   
\usepackage[PostScript=dvips]{"/Users/aware/Documents/Courses/diagrams"}


\newtheorem{theorem}{Théorème}[section]
\renewcommand{\thetheorem}{\arabic{theorem}}
\newtheorem{lemme}{Lemme}[section]
\renewcommand{\thelemme}{\arabic{lemme}}
\newtheorem{proposition}{Proposition}[section]
\renewcommand{\theproposition}{\arabic{proposition}}
\newtheorem{notations}{Notations}[section]
\newtheorem{problem}{Problème}[section]
\newtheorem{corollary}{Corollaire}[theorem]
\renewcommand{\thecorollary}{\arabic{corollary}}
\newtheorem{property}{Propriété}[section]
\newtheorem{objective}{Objectif}[section]

\theoremstyle{definition}
\newtheorem{definition}{Définition}[section]
\renewcommand{\thedefinition}{\arabic{definition}}
\newtheorem{exercise}{Exercice}[chapter]
\renewcommand{\theexercise}{\arabic{exercise}}
\newtheorem{example}{Exemple}[chapter]
\renewcommand{\theexample}{\arabic{example}}
\newtheorem*{solution}{Solution}
\newtheorem*{application}{Application}
\newtheorem*{notation}{Notation}
\newtheorem*{vocabulary}{Vocabulaire}
\newtheorem*{properties}{Propriétés}



\theoremstyle{remark}
\newtheorem*{remark}{Remarque}
\newtheorem*{rappel}{Rappel}


\usepackage{etoolbox}
\AtBeginEnvironment{exercise}{\small}
\AtBeginEnvironment{example}{\small}

\usepackage{cases}
\usepackage[red]{mypack}

\usepackage[framemethod=TikZ]{mdframed}

\definecolor{bg}{rgb}{0.4,0.25,0.95}
\definecolor{pagebg}{rgb}{0,0,0.5}
\surroundwithmdframed[
   topline=false,
   rightline=false,
   bottomline=false,
   leftmargin=\parindent,
   skipabove=8pt,
   skipbelow=8pt,
   linecolor=blue,
   innerbottommargin=10pt,
   % backgroundcolor=bg,font=\color{orange}\sffamily, fontcolor=white
]{definition}

\usepackage{empheq}
\usepackage[most]{tcolorbox}

\newtcbox{\mymath}[1][]{%
    nobeforeafter, math upper, tcbox raise base,
    enhanced, colframe=blue!30!black,
    colback=red!10, boxrule=1pt,
    #1}

\usepackage{unixode}


\DeclareMathOperator{\ord}{ord}
\DeclareMathOperator{\orb}{orb}
\DeclareMathOperator{\stab}{stab}
\DeclareMathOperator{\Stab}{stab}
\DeclareMathOperator{\ppcm}{ppcm}
\DeclareMathOperator{\conj}{Conj}
\DeclareMathOperator{\End}{End}
\DeclareMathOperator{\rot}{rot}
\DeclareMathOperator{\trs}{trace}
\DeclareMathOperator{\Ind}{Ind}
\DeclareMathOperator{\mat}{Mat}
\DeclareMathOperator{\id}{Id}
\DeclareMathOperator{\vect}{vect}
\DeclareMathOperator{\img}{img}
\DeclareMathOperator{\cov}{Cov}
\DeclareMathOperator{\dist}{dist}
\DeclareMathOperator{\irr}{Irr}
\DeclareMathOperator{\image}{Im}
\DeclareMathOperator{\pd}{\partial}
\DeclareMathOperator{\epi}{epi}
\DeclareMathOperator{\Argmin}{Argmin}
\DeclareMathOperator{\dom}{dom}
\DeclareMathOperator{\proj}{proj}
\DeclareMathOperator{\ctg}{ctg}
\DeclareMathOperator{\supp}{supp}
\DeclareMathOperator{\argmin}{argmin}
\DeclareMathOperator{\mult}{mult}
\DeclareMathOperator{\ch}{ch}
\DeclareMathOperator{\sh}{sh}
\DeclareMathOperator{\rang}{rang}
\DeclareMathOperator{\diam}{diam}
\DeclareMathOperator{\Epigraphe}{Epigraphe}




\usepackage{xcolor}
\everymath{\color{blue}}
%\everymath{\color[rgb]{0,1,1}}
%\pagecolor[rgb]{0,0,0.5}


\newcommand*{\pdtest}[3][]{\ensuremath{\frac{\partial^{#1} #2}{\partial #3}}}

\newcommand*{\deffunc}[6][]{\ensuremath{
\begin{array}{rcl}
#2 : #3 &\rightarrow& #4\\
#5 &\mapsto& #6
\end{array}
}}

\newcommand{\eqcolon}{\mathrel{\resizebox{\widthof{$\mathord{=}$}}{\height}{ $\!\!=\!\!\resizebox{1.2\width}{0.8\height}{\raisebox{0.23ex}{$\mathop{:}$}}\!\!$ }}}
\newcommand{\coloneq}{\mathrel{\resizebox{\widthof{$\mathord{=}$}}{\height}{ $\!\!\resizebox{1.2\width}{0.8\height}{\raisebox{0.23ex}{$\mathop{:}$}}\!\!=\!\!$ }}}
\newcommand{\eqcolonl}{\ensuremath{\mathrel{=\!\!\mathop{:}}}}
\newcommand{\coloneql}{\ensuremath{\mathrel{\mathop{:} \!\! =}}}
\newcommand{\vc}[1]{% inline column vector
  \left(\begin{smallmatrix}#1\end{smallmatrix}\right)%
}
\newcommand{\vr}[1]{% inline row vector
  \begin{smallmatrix}(\,#1\,)\end{smallmatrix}%
}
\makeatletter
\newcommand*{\defeq}{\ =\mathrel{\rlap{%
                     \raisebox{0.3ex}{$\m@th\cdot$}}%
                     \raisebox{-0.3ex}{$\m@th\cdot$}}%
                     }
\makeatother

\newcommand{\mathcircle}[1]{% inline row vector
 \overset{\circ}{#1}
}
\newcommand{\ulim}{% low limit
 \underline{\lim}
}
\newcommand{\ssi}{% iff
\iff
}
\newcommand{\ps}[2]{
\expval{#1 | #2}
}
\newcommand{\df}[1]{
\mqty{#1}
}
\newcommand{\n}[1]{
\norm{#1}
}
\newcommand{\sys}[1]{
\left\{\smqty{#1}\right.
}


\newcommand{\eqdef}{\ensuremath{\overset{\text{def}}=}}


\def\Circlearrowright{\ensuremath{%
  \rotatebox[origin=c]{230}{$\circlearrowright$}}}

\newcommand\ct[1]{\text{\rmfamily\upshape #1}}
\newcommand\question[1]{ {\color{red} ...!? \small #1}}
\newcommand\caz[1]{\left\{\begin{array} #1 \end{array}\right.}
\newcommand\const{\text{\rmfamily\upshape const}}
\newcommand\toP{ \overset{\pro}{\to}}
\newcommand\toPP{ \overset{\text{PP}}{\to}}
\newcommand{\oeq}{\mathrel{\text{\textcircled{$=$}}}}





\usepackage{xcolor}
% \usepackage[normalem]{ulem}
\usepackage{lipsum}
\makeatletter
% \newcommand\colorwave[1][blue]{\bgroup \markoverwith{\lower3.5\p@\hbox{\sixly \textcolor{#1}{\char58}}}\ULon}
%\font\sixly=lasy6 % does not re-load if already loaded, so no memory problem.

\newmdtheoremenv[
linewidth= 1pt,linecolor= blue,%
leftmargin=20,rightmargin=20,innertopmargin=0pt, innerrightmargin=40,%
tikzsetting = { draw=lightgray, line width = 0.3pt,dashed,%
dash pattern = on 15pt off 3pt},%
splittopskip=\topskip,skipbelow=\baselineskip,%
skipabove=\baselineskip,ntheorem,roundcorner=0pt,
% backgroundcolor=pagebg,font=\color{orange}\sffamily, fontcolor=white
]{examplebox}{Exemple}[section]



\newcommand\R{\mathbb{R}}
\newcommand\Z{\mathbb{Z}}
\newcommand\N{\mathbb{N}}
\newcommand\E{\mathbb{E}}
\newcommand\F{\mathcal{F}}
\newcommand\cH{\mathcal{H}}
\newcommand\V{\mathbb{V}}
\newcommand\dmo{ ^{-1} }
\newcommand\kapa{\kappa}
\newcommand\im{Im}
\newcommand\hs{\mathcal{H}}





\usepackage{soul}

\makeatletter
\newcommand*{\whiten}[1]{\llap{\textcolor{white}{{\the\SOUL@token}}\hspace{#1pt}}}
\DeclareRobustCommand*\myul{%
    \def\SOUL@everyspace{\underline{\space}\kern\z@}%
    \def\SOUL@everytoken{%
     \setbox0=\hbox{\the\SOUL@token}%
     \ifdim\dp0>\z@
        \raisebox{\dp0}{\underline{\phantom{\the\SOUL@token}}}%
        \whiten{1}\whiten{0}%
        \whiten{-1}\whiten{-2}%
        \llap{\the\SOUL@token}%
     \else
        \underline{\the\SOUL@token}%
     \fi}%
\SOUL@}
\makeatother

\newcommand*{\demp}{\fontfamily{lmtt}\selectfont}

\DeclareTextFontCommand{\textdemp}{\demp}

\begin{document}

\ifcomment
Multiline
comment
\fi
\ifcomment
\myul{Typesetting test}
% \color[rgb]{1,1,1}
$∑_i^n≠ 60º±∞π∆¬≈√j∫h≤≥µ$

$\CR \R\pro\ind\pro\gS\pro
\mqty[a&b\\c&d]$
$\pro\mathbb{P}$
$\dd{x}$

  \[
    \alpha(x)=\left\{
                \begin{array}{ll}
                  x\\
                  \frac{1}{1+e^{-kx}}\\
                  \frac{e^x-e^{-x}}{e^x+e^{-x}}
                \end{array}
              \right.
  \]

  $\expval{x}$
  
  $\chi_\rho(ghg\dmo)=\Tr(\rho_{ghg\dmo})=\Tr(\rho_g\circ\rho_h\circ\rho\dmo_g)=\Tr(\rho_h)\overset{\mbox{\scalebox{0.5}{$\Tr(AB)=\Tr(BA)$}}}{=}\chi_\rho(h)$
  	$\mathop{\oplus}_{\substack{x\in X}}$

$\mat(\rho_g)=(a_{ij}(g))_{\scriptsize \substack{1\leq i\leq d \\ 1\leq j\leq d}}$ et $\mat(\rho'_g)=(a'_{ij}(g))_{\scriptsize \substack{1\leq i'\leq d' \\ 1\leq j'\leq d'}}$



\[\int_a^b{\mathbb{R}^2}g(u, v)\dd{P_{XY}}(u, v)=\iint g(u,v) f_{XY}(u, v)\dd \lambda(u) \dd \lambda(v)\]
$$\lim_{x\to\infty} f(x)$$	
$$\iiiint_V \mu(t,u,v,w) \,dt\,du\,dv\,dw$$
$$\sum_{n=1}^{\infty} 2^{-n} = 1$$	
\begin{definition}
	Si $X$ et $Y$ sont 2 v.a. ou definit la \textsc{Covariance} entre $X$ et $Y$ comme
	$\cov(X,Y)\overset{\text{def}}{=}\E\left[(X-\E(X))(Y-\E(Y))\right]=\E(XY)-\E(X)\E(Y)$.
\end{definition}
\fi
\pagebreak

% \tableofcontents

% insert your code here
%\input{./algebra/main.tex}
%\input{./geometrie-differentielle/main.tex}
%\input{./probabilite/main.tex}
%\input{./analyse-fonctionnelle/main.tex}
% \input{./Analyse-convexe-et-dualite-en-optimisation/main.tex}
%\input{./tikz/main.tex}
%\input{./Theorie-du-distributions/main.tex}
%\input{./optimisation/mine.tex}
 \input{./modelisation/main.tex}

% yves.aubry@univ-tln.fr : algebra

\end{document}

% % !TEX encoding = UTF-8 Unicode
% !TEX TS-program = xelatex

\documentclass[french]{report}

%\usepackage[utf8]{inputenc}
%\usepackage[T1]{fontenc}
\usepackage{babel}


\newif\ifcomment
%\commenttrue # Show comments

\usepackage{physics}
\usepackage{amssymb}


\usepackage{amsthm}
% \usepackage{thmtools}
\usepackage{mathtools}
\usepackage{amsfonts}

\usepackage{color}

\usepackage{tikz}

\usepackage{geometry}
\geometry{a5paper, margin=0.1in, right=1cm}

\usepackage{dsfont}

\usepackage{graphicx}
\graphicspath{ {images/} }

\usepackage{faktor}

\usepackage{IEEEtrantools}
\usepackage{enumerate}   
\usepackage[PostScript=dvips]{"/Users/aware/Documents/Courses/diagrams"}


\newtheorem{theorem}{Théorème}[section]
\renewcommand{\thetheorem}{\arabic{theorem}}
\newtheorem{lemme}{Lemme}[section]
\renewcommand{\thelemme}{\arabic{lemme}}
\newtheorem{proposition}{Proposition}[section]
\renewcommand{\theproposition}{\arabic{proposition}}
\newtheorem{notations}{Notations}[section]
\newtheorem{problem}{Problème}[section]
\newtheorem{corollary}{Corollaire}[theorem]
\renewcommand{\thecorollary}{\arabic{corollary}}
\newtheorem{property}{Propriété}[section]
\newtheorem{objective}{Objectif}[section]

\theoremstyle{definition}
\newtheorem{definition}{Définition}[section]
\renewcommand{\thedefinition}{\arabic{definition}}
\newtheorem{exercise}{Exercice}[chapter]
\renewcommand{\theexercise}{\arabic{exercise}}
\newtheorem{example}{Exemple}[chapter]
\renewcommand{\theexample}{\arabic{example}}
\newtheorem*{solution}{Solution}
\newtheorem*{application}{Application}
\newtheorem*{notation}{Notation}
\newtheorem*{vocabulary}{Vocabulaire}
\newtheorem*{properties}{Propriétés}



\theoremstyle{remark}
\newtheorem*{remark}{Remarque}
\newtheorem*{rappel}{Rappel}


\usepackage{etoolbox}
\AtBeginEnvironment{exercise}{\small}
\AtBeginEnvironment{example}{\small}

\usepackage{cases}
\usepackage[red]{mypack}

\usepackage[framemethod=TikZ]{mdframed}

\definecolor{bg}{rgb}{0.4,0.25,0.95}
\definecolor{pagebg}{rgb}{0,0,0.5}
\surroundwithmdframed[
   topline=false,
   rightline=false,
   bottomline=false,
   leftmargin=\parindent,
   skipabove=8pt,
   skipbelow=8pt,
   linecolor=blue,
   innerbottommargin=10pt,
   % backgroundcolor=bg,font=\color{orange}\sffamily, fontcolor=white
]{definition}

\usepackage{empheq}
\usepackage[most]{tcolorbox}

\newtcbox{\mymath}[1][]{%
    nobeforeafter, math upper, tcbox raise base,
    enhanced, colframe=blue!30!black,
    colback=red!10, boxrule=1pt,
    #1}

\usepackage{unixode}


\DeclareMathOperator{\ord}{ord}
\DeclareMathOperator{\orb}{orb}
\DeclareMathOperator{\stab}{stab}
\DeclareMathOperator{\Stab}{stab}
\DeclareMathOperator{\ppcm}{ppcm}
\DeclareMathOperator{\conj}{Conj}
\DeclareMathOperator{\End}{End}
\DeclareMathOperator{\rot}{rot}
\DeclareMathOperator{\trs}{trace}
\DeclareMathOperator{\Ind}{Ind}
\DeclareMathOperator{\mat}{Mat}
\DeclareMathOperator{\id}{Id}
\DeclareMathOperator{\vect}{vect}
\DeclareMathOperator{\img}{img}
\DeclareMathOperator{\cov}{Cov}
\DeclareMathOperator{\dist}{dist}
\DeclareMathOperator{\irr}{Irr}
\DeclareMathOperator{\image}{Im}
\DeclareMathOperator{\pd}{\partial}
\DeclareMathOperator{\epi}{epi}
\DeclareMathOperator{\Argmin}{Argmin}
\DeclareMathOperator{\dom}{dom}
\DeclareMathOperator{\proj}{proj}
\DeclareMathOperator{\ctg}{ctg}
\DeclareMathOperator{\supp}{supp}
\DeclareMathOperator{\argmin}{argmin}
\DeclareMathOperator{\mult}{mult}
\DeclareMathOperator{\ch}{ch}
\DeclareMathOperator{\sh}{sh}
\DeclareMathOperator{\rang}{rang}
\DeclareMathOperator{\diam}{diam}
\DeclareMathOperator{\Epigraphe}{Epigraphe}




\usepackage{xcolor}
\everymath{\color{blue}}
%\everymath{\color[rgb]{0,1,1}}
%\pagecolor[rgb]{0,0,0.5}


\newcommand*{\pdtest}[3][]{\ensuremath{\frac{\partial^{#1} #2}{\partial #3}}}

\newcommand*{\deffunc}[6][]{\ensuremath{
\begin{array}{rcl}
#2 : #3 &\rightarrow& #4\\
#5 &\mapsto& #6
\end{array}
}}

\newcommand{\eqcolon}{\mathrel{\resizebox{\widthof{$\mathord{=}$}}{\height}{ $\!\!=\!\!\resizebox{1.2\width}{0.8\height}{\raisebox{0.23ex}{$\mathop{:}$}}\!\!$ }}}
\newcommand{\coloneq}{\mathrel{\resizebox{\widthof{$\mathord{=}$}}{\height}{ $\!\!\resizebox{1.2\width}{0.8\height}{\raisebox{0.23ex}{$\mathop{:}$}}\!\!=\!\!$ }}}
\newcommand{\eqcolonl}{\ensuremath{\mathrel{=\!\!\mathop{:}}}}
\newcommand{\coloneql}{\ensuremath{\mathrel{\mathop{:} \!\! =}}}
\newcommand{\vc}[1]{% inline column vector
  \left(\begin{smallmatrix}#1\end{smallmatrix}\right)%
}
\newcommand{\vr}[1]{% inline row vector
  \begin{smallmatrix}(\,#1\,)\end{smallmatrix}%
}
\makeatletter
\newcommand*{\defeq}{\ =\mathrel{\rlap{%
                     \raisebox{0.3ex}{$\m@th\cdot$}}%
                     \raisebox{-0.3ex}{$\m@th\cdot$}}%
                     }
\makeatother

\newcommand{\mathcircle}[1]{% inline row vector
 \overset{\circ}{#1}
}
\newcommand{\ulim}{% low limit
 \underline{\lim}
}
\newcommand{\ssi}{% iff
\iff
}
\newcommand{\ps}[2]{
\expval{#1 | #2}
}
\newcommand{\df}[1]{
\mqty{#1}
}
\newcommand{\n}[1]{
\norm{#1}
}
\newcommand{\sys}[1]{
\left\{\smqty{#1}\right.
}


\newcommand{\eqdef}{\ensuremath{\overset{\text{def}}=}}


\def\Circlearrowright{\ensuremath{%
  \rotatebox[origin=c]{230}{$\circlearrowright$}}}

\newcommand\ct[1]{\text{\rmfamily\upshape #1}}
\newcommand\question[1]{ {\color{red} ...!? \small #1}}
\newcommand\caz[1]{\left\{\begin{array} #1 \end{array}\right.}
\newcommand\const{\text{\rmfamily\upshape const}}
\newcommand\toP{ \overset{\pro}{\to}}
\newcommand\toPP{ \overset{\text{PP}}{\to}}
\newcommand{\oeq}{\mathrel{\text{\textcircled{$=$}}}}





\usepackage{xcolor}
% \usepackage[normalem]{ulem}
\usepackage{lipsum}
\makeatletter
% \newcommand\colorwave[1][blue]{\bgroup \markoverwith{\lower3.5\p@\hbox{\sixly \textcolor{#1}{\char58}}}\ULon}
%\font\sixly=lasy6 % does not re-load if already loaded, so no memory problem.

\newmdtheoremenv[
linewidth= 1pt,linecolor= blue,%
leftmargin=20,rightmargin=20,innertopmargin=0pt, innerrightmargin=40,%
tikzsetting = { draw=lightgray, line width = 0.3pt,dashed,%
dash pattern = on 15pt off 3pt},%
splittopskip=\topskip,skipbelow=\baselineskip,%
skipabove=\baselineskip,ntheorem,roundcorner=0pt,
% backgroundcolor=pagebg,font=\color{orange}\sffamily, fontcolor=white
]{examplebox}{Exemple}[section]



\newcommand\R{\mathbb{R}}
\newcommand\Z{\mathbb{Z}}
\newcommand\N{\mathbb{N}}
\newcommand\E{\mathbb{E}}
\newcommand\F{\mathcal{F}}
\newcommand\cH{\mathcal{H}}
\newcommand\V{\mathbb{V}}
\newcommand\dmo{ ^{-1} }
\newcommand\kapa{\kappa}
\newcommand\im{Im}
\newcommand\hs{\mathcal{H}}





\usepackage{soul}

\makeatletter
\newcommand*{\whiten}[1]{\llap{\textcolor{white}{{\the\SOUL@token}}\hspace{#1pt}}}
\DeclareRobustCommand*\myul{%
    \def\SOUL@everyspace{\underline{\space}\kern\z@}%
    \def\SOUL@everytoken{%
     \setbox0=\hbox{\the\SOUL@token}%
     \ifdim\dp0>\z@
        \raisebox{\dp0}{\underline{\phantom{\the\SOUL@token}}}%
        \whiten{1}\whiten{0}%
        \whiten{-1}\whiten{-2}%
        \llap{\the\SOUL@token}%
     \else
        \underline{\the\SOUL@token}%
     \fi}%
\SOUL@}
\makeatother

\newcommand*{\demp}{\fontfamily{lmtt}\selectfont}

\DeclareTextFontCommand{\textdemp}{\demp}

\begin{document}

\ifcomment
Multiline
comment
\fi
\ifcomment
\myul{Typesetting test}
% \color[rgb]{1,1,1}
$∑_i^n≠ 60º±∞π∆¬≈√j∫h≤≥µ$

$\CR \R\pro\ind\pro\gS\pro
\mqty[a&b\\c&d]$
$\pro\mathbb{P}$
$\dd{x}$

  \[
    \alpha(x)=\left\{
                \begin{array}{ll}
                  x\\
                  \frac{1}{1+e^{-kx}}\\
                  \frac{e^x-e^{-x}}{e^x+e^{-x}}
                \end{array}
              \right.
  \]

  $\expval{x}$
  
  $\chi_\rho(ghg\dmo)=\Tr(\rho_{ghg\dmo})=\Tr(\rho_g\circ\rho_h\circ\rho\dmo_g)=\Tr(\rho_h)\overset{\mbox{\scalebox{0.5}{$\Tr(AB)=\Tr(BA)$}}}{=}\chi_\rho(h)$
  	$\mathop{\oplus}_{\substack{x\in X}}$

$\mat(\rho_g)=(a_{ij}(g))_{\scriptsize \substack{1\leq i\leq d \\ 1\leq j\leq d}}$ et $\mat(\rho'_g)=(a'_{ij}(g))_{\scriptsize \substack{1\leq i'\leq d' \\ 1\leq j'\leq d'}}$



\[\int_a^b{\mathbb{R}^2}g(u, v)\dd{P_{XY}}(u, v)=\iint g(u,v) f_{XY}(u, v)\dd \lambda(u) \dd \lambda(v)\]
$$\lim_{x\to\infty} f(x)$$	
$$\iiiint_V \mu(t,u,v,w) \,dt\,du\,dv\,dw$$
$$\sum_{n=1}^{\infty} 2^{-n} = 1$$	
\begin{definition}
	Si $X$ et $Y$ sont 2 v.a. ou definit la \textsc{Covariance} entre $X$ et $Y$ comme
	$\cov(X,Y)\overset{\text{def}}{=}\E\left[(X-\E(X))(Y-\E(Y))\right]=\E(XY)-\E(X)\E(Y)$.
\end{definition}
\fi
\pagebreak

% \tableofcontents

% insert your code here
%\input{./algebra/main.tex}
%\input{./geometrie-differentielle/main.tex}
%\input{./probabilite/main.tex}
%\input{./analyse-fonctionnelle/main.tex}
% \input{./Analyse-convexe-et-dualite-en-optimisation/main.tex}
%\input{./tikz/main.tex}
%\input{./Theorie-du-distributions/main.tex}
%\input{./optimisation/mine.tex}
 \input{./modelisation/main.tex}

% yves.aubry@univ-tln.fr : algebra

\end{document}

%% !TEX encoding = UTF-8 Unicode
% !TEX TS-program = xelatex

\documentclass[french]{report}

%\usepackage[utf8]{inputenc}
%\usepackage[T1]{fontenc}
\usepackage{babel}


\newif\ifcomment
%\commenttrue # Show comments

\usepackage{physics}
\usepackage{amssymb}


\usepackage{amsthm}
% \usepackage{thmtools}
\usepackage{mathtools}
\usepackage{amsfonts}

\usepackage{color}

\usepackage{tikz}

\usepackage{geometry}
\geometry{a5paper, margin=0.1in, right=1cm}

\usepackage{dsfont}

\usepackage{graphicx}
\graphicspath{ {images/} }

\usepackage{faktor}

\usepackage{IEEEtrantools}
\usepackage{enumerate}   
\usepackage[PostScript=dvips]{"/Users/aware/Documents/Courses/diagrams"}


\newtheorem{theorem}{Théorème}[section]
\renewcommand{\thetheorem}{\arabic{theorem}}
\newtheorem{lemme}{Lemme}[section]
\renewcommand{\thelemme}{\arabic{lemme}}
\newtheorem{proposition}{Proposition}[section]
\renewcommand{\theproposition}{\arabic{proposition}}
\newtheorem{notations}{Notations}[section]
\newtheorem{problem}{Problème}[section]
\newtheorem{corollary}{Corollaire}[theorem]
\renewcommand{\thecorollary}{\arabic{corollary}}
\newtheorem{property}{Propriété}[section]
\newtheorem{objective}{Objectif}[section]

\theoremstyle{definition}
\newtheorem{definition}{Définition}[section]
\renewcommand{\thedefinition}{\arabic{definition}}
\newtheorem{exercise}{Exercice}[chapter]
\renewcommand{\theexercise}{\arabic{exercise}}
\newtheorem{example}{Exemple}[chapter]
\renewcommand{\theexample}{\arabic{example}}
\newtheorem*{solution}{Solution}
\newtheorem*{application}{Application}
\newtheorem*{notation}{Notation}
\newtheorem*{vocabulary}{Vocabulaire}
\newtheorem*{properties}{Propriétés}



\theoremstyle{remark}
\newtheorem*{remark}{Remarque}
\newtheorem*{rappel}{Rappel}


\usepackage{etoolbox}
\AtBeginEnvironment{exercise}{\small}
\AtBeginEnvironment{example}{\small}

\usepackage{cases}
\usepackage[red]{mypack}

\usepackage[framemethod=TikZ]{mdframed}

\definecolor{bg}{rgb}{0.4,0.25,0.95}
\definecolor{pagebg}{rgb}{0,0,0.5}
\surroundwithmdframed[
   topline=false,
   rightline=false,
   bottomline=false,
   leftmargin=\parindent,
   skipabove=8pt,
   skipbelow=8pt,
   linecolor=blue,
   innerbottommargin=10pt,
   % backgroundcolor=bg,font=\color{orange}\sffamily, fontcolor=white
]{definition}

\usepackage{empheq}
\usepackage[most]{tcolorbox}

\newtcbox{\mymath}[1][]{%
    nobeforeafter, math upper, tcbox raise base,
    enhanced, colframe=blue!30!black,
    colback=red!10, boxrule=1pt,
    #1}

\usepackage{unixode}


\DeclareMathOperator{\ord}{ord}
\DeclareMathOperator{\orb}{orb}
\DeclareMathOperator{\stab}{stab}
\DeclareMathOperator{\Stab}{stab}
\DeclareMathOperator{\ppcm}{ppcm}
\DeclareMathOperator{\conj}{Conj}
\DeclareMathOperator{\End}{End}
\DeclareMathOperator{\rot}{rot}
\DeclareMathOperator{\trs}{trace}
\DeclareMathOperator{\Ind}{Ind}
\DeclareMathOperator{\mat}{Mat}
\DeclareMathOperator{\id}{Id}
\DeclareMathOperator{\vect}{vect}
\DeclareMathOperator{\img}{img}
\DeclareMathOperator{\cov}{Cov}
\DeclareMathOperator{\dist}{dist}
\DeclareMathOperator{\irr}{Irr}
\DeclareMathOperator{\image}{Im}
\DeclareMathOperator{\pd}{\partial}
\DeclareMathOperator{\epi}{epi}
\DeclareMathOperator{\Argmin}{Argmin}
\DeclareMathOperator{\dom}{dom}
\DeclareMathOperator{\proj}{proj}
\DeclareMathOperator{\ctg}{ctg}
\DeclareMathOperator{\supp}{supp}
\DeclareMathOperator{\argmin}{argmin}
\DeclareMathOperator{\mult}{mult}
\DeclareMathOperator{\ch}{ch}
\DeclareMathOperator{\sh}{sh}
\DeclareMathOperator{\rang}{rang}
\DeclareMathOperator{\diam}{diam}
\DeclareMathOperator{\Epigraphe}{Epigraphe}




\usepackage{xcolor}
\everymath{\color{blue}}
%\everymath{\color[rgb]{0,1,1}}
%\pagecolor[rgb]{0,0,0.5}


\newcommand*{\pdtest}[3][]{\ensuremath{\frac{\partial^{#1} #2}{\partial #3}}}

\newcommand*{\deffunc}[6][]{\ensuremath{
\begin{array}{rcl}
#2 : #3 &\rightarrow& #4\\
#5 &\mapsto& #6
\end{array}
}}

\newcommand{\eqcolon}{\mathrel{\resizebox{\widthof{$\mathord{=}$}}{\height}{ $\!\!=\!\!\resizebox{1.2\width}{0.8\height}{\raisebox{0.23ex}{$\mathop{:}$}}\!\!$ }}}
\newcommand{\coloneq}{\mathrel{\resizebox{\widthof{$\mathord{=}$}}{\height}{ $\!\!\resizebox{1.2\width}{0.8\height}{\raisebox{0.23ex}{$\mathop{:}$}}\!\!=\!\!$ }}}
\newcommand{\eqcolonl}{\ensuremath{\mathrel{=\!\!\mathop{:}}}}
\newcommand{\coloneql}{\ensuremath{\mathrel{\mathop{:} \!\! =}}}
\newcommand{\vc}[1]{% inline column vector
  \left(\begin{smallmatrix}#1\end{smallmatrix}\right)%
}
\newcommand{\vr}[1]{% inline row vector
  \begin{smallmatrix}(\,#1\,)\end{smallmatrix}%
}
\makeatletter
\newcommand*{\defeq}{\ =\mathrel{\rlap{%
                     \raisebox{0.3ex}{$\m@th\cdot$}}%
                     \raisebox{-0.3ex}{$\m@th\cdot$}}%
                     }
\makeatother

\newcommand{\mathcircle}[1]{% inline row vector
 \overset{\circ}{#1}
}
\newcommand{\ulim}{% low limit
 \underline{\lim}
}
\newcommand{\ssi}{% iff
\iff
}
\newcommand{\ps}[2]{
\expval{#1 | #2}
}
\newcommand{\df}[1]{
\mqty{#1}
}
\newcommand{\n}[1]{
\norm{#1}
}
\newcommand{\sys}[1]{
\left\{\smqty{#1}\right.
}


\newcommand{\eqdef}{\ensuremath{\overset{\text{def}}=}}


\def\Circlearrowright{\ensuremath{%
  \rotatebox[origin=c]{230}{$\circlearrowright$}}}

\newcommand\ct[1]{\text{\rmfamily\upshape #1}}
\newcommand\question[1]{ {\color{red} ...!? \small #1}}
\newcommand\caz[1]{\left\{\begin{array} #1 \end{array}\right.}
\newcommand\const{\text{\rmfamily\upshape const}}
\newcommand\toP{ \overset{\pro}{\to}}
\newcommand\toPP{ \overset{\text{PP}}{\to}}
\newcommand{\oeq}{\mathrel{\text{\textcircled{$=$}}}}





\usepackage{xcolor}
% \usepackage[normalem]{ulem}
\usepackage{lipsum}
\makeatletter
% \newcommand\colorwave[1][blue]{\bgroup \markoverwith{\lower3.5\p@\hbox{\sixly \textcolor{#1}{\char58}}}\ULon}
%\font\sixly=lasy6 % does not re-load if already loaded, so no memory problem.

\newmdtheoremenv[
linewidth= 1pt,linecolor= blue,%
leftmargin=20,rightmargin=20,innertopmargin=0pt, innerrightmargin=40,%
tikzsetting = { draw=lightgray, line width = 0.3pt,dashed,%
dash pattern = on 15pt off 3pt},%
splittopskip=\topskip,skipbelow=\baselineskip,%
skipabove=\baselineskip,ntheorem,roundcorner=0pt,
% backgroundcolor=pagebg,font=\color{orange}\sffamily, fontcolor=white
]{examplebox}{Exemple}[section]



\newcommand\R{\mathbb{R}}
\newcommand\Z{\mathbb{Z}}
\newcommand\N{\mathbb{N}}
\newcommand\E{\mathbb{E}}
\newcommand\F{\mathcal{F}}
\newcommand\cH{\mathcal{H}}
\newcommand\V{\mathbb{V}}
\newcommand\dmo{ ^{-1} }
\newcommand\kapa{\kappa}
\newcommand\im{Im}
\newcommand\hs{\mathcal{H}}





\usepackage{soul}

\makeatletter
\newcommand*{\whiten}[1]{\llap{\textcolor{white}{{\the\SOUL@token}}\hspace{#1pt}}}
\DeclareRobustCommand*\myul{%
    \def\SOUL@everyspace{\underline{\space}\kern\z@}%
    \def\SOUL@everytoken{%
     \setbox0=\hbox{\the\SOUL@token}%
     \ifdim\dp0>\z@
        \raisebox{\dp0}{\underline{\phantom{\the\SOUL@token}}}%
        \whiten{1}\whiten{0}%
        \whiten{-1}\whiten{-2}%
        \llap{\the\SOUL@token}%
     \else
        \underline{\the\SOUL@token}%
     \fi}%
\SOUL@}
\makeatother

\newcommand*{\demp}{\fontfamily{lmtt}\selectfont}

\DeclareTextFontCommand{\textdemp}{\demp}

\begin{document}

\ifcomment
Multiline
comment
\fi
\ifcomment
\myul{Typesetting test}
% \color[rgb]{1,1,1}
$∑_i^n≠ 60º±∞π∆¬≈√j∫h≤≥µ$

$\CR \R\pro\ind\pro\gS\pro
\mqty[a&b\\c&d]$
$\pro\mathbb{P}$
$\dd{x}$

  \[
    \alpha(x)=\left\{
                \begin{array}{ll}
                  x\\
                  \frac{1}{1+e^{-kx}}\\
                  \frac{e^x-e^{-x}}{e^x+e^{-x}}
                \end{array}
              \right.
  \]

  $\expval{x}$
  
  $\chi_\rho(ghg\dmo)=\Tr(\rho_{ghg\dmo})=\Tr(\rho_g\circ\rho_h\circ\rho\dmo_g)=\Tr(\rho_h)\overset{\mbox{\scalebox{0.5}{$\Tr(AB)=\Tr(BA)$}}}{=}\chi_\rho(h)$
  	$\mathop{\oplus}_{\substack{x\in X}}$

$\mat(\rho_g)=(a_{ij}(g))_{\scriptsize \substack{1\leq i\leq d \\ 1\leq j\leq d}}$ et $\mat(\rho'_g)=(a'_{ij}(g))_{\scriptsize \substack{1\leq i'\leq d' \\ 1\leq j'\leq d'}}$



\[\int_a^b{\mathbb{R}^2}g(u, v)\dd{P_{XY}}(u, v)=\iint g(u,v) f_{XY}(u, v)\dd \lambda(u) \dd \lambda(v)\]
$$\lim_{x\to\infty} f(x)$$	
$$\iiiint_V \mu(t,u,v,w) \,dt\,du\,dv\,dw$$
$$\sum_{n=1}^{\infty} 2^{-n} = 1$$	
\begin{definition}
	Si $X$ et $Y$ sont 2 v.a. ou definit la \textsc{Covariance} entre $X$ et $Y$ comme
	$\cov(X,Y)\overset{\text{def}}{=}\E\left[(X-\E(X))(Y-\E(Y))\right]=\E(XY)-\E(X)\E(Y)$.
\end{definition}
\fi
\pagebreak

% \tableofcontents

% insert your code here
%\input{./algebra/main.tex}
%\input{./geometrie-differentielle/main.tex}
%\input{./probabilite/main.tex}
%\input{./analyse-fonctionnelle/main.tex}
% \input{./Analyse-convexe-et-dualite-en-optimisation/main.tex}
%\input{./tikz/main.tex}
%\input{./Theorie-du-distributions/main.tex}
%\input{./optimisation/mine.tex}
 \input{./modelisation/main.tex}

% yves.aubry@univ-tln.fr : algebra

\end{document}

%% !TEX encoding = UTF-8 Unicode
% !TEX TS-program = xelatex

\documentclass[french]{report}

%\usepackage[utf8]{inputenc}
%\usepackage[T1]{fontenc}
\usepackage{babel}


\newif\ifcomment
%\commenttrue # Show comments

\usepackage{physics}
\usepackage{amssymb}


\usepackage{amsthm}
% \usepackage{thmtools}
\usepackage{mathtools}
\usepackage{amsfonts}

\usepackage{color}

\usepackage{tikz}

\usepackage{geometry}
\geometry{a5paper, margin=0.1in, right=1cm}

\usepackage{dsfont}

\usepackage{graphicx}
\graphicspath{ {images/} }

\usepackage{faktor}

\usepackage{IEEEtrantools}
\usepackage{enumerate}   
\usepackage[PostScript=dvips]{"/Users/aware/Documents/Courses/diagrams"}


\newtheorem{theorem}{Théorème}[section]
\renewcommand{\thetheorem}{\arabic{theorem}}
\newtheorem{lemme}{Lemme}[section]
\renewcommand{\thelemme}{\arabic{lemme}}
\newtheorem{proposition}{Proposition}[section]
\renewcommand{\theproposition}{\arabic{proposition}}
\newtheorem{notations}{Notations}[section]
\newtheorem{problem}{Problème}[section]
\newtheorem{corollary}{Corollaire}[theorem]
\renewcommand{\thecorollary}{\arabic{corollary}}
\newtheorem{property}{Propriété}[section]
\newtheorem{objective}{Objectif}[section]

\theoremstyle{definition}
\newtheorem{definition}{Définition}[section]
\renewcommand{\thedefinition}{\arabic{definition}}
\newtheorem{exercise}{Exercice}[chapter]
\renewcommand{\theexercise}{\arabic{exercise}}
\newtheorem{example}{Exemple}[chapter]
\renewcommand{\theexample}{\arabic{example}}
\newtheorem*{solution}{Solution}
\newtheorem*{application}{Application}
\newtheorem*{notation}{Notation}
\newtheorem*{vocabulary}{Vocabulaire}
\newtheorem*{properties}{Propriétés}



\theoremstyle{remark}
\newtheorem*{remark}{Remarque}
\newtheorem*{rappel}{Rappel}


\usepackage{etoolbox}
\AtBeginEnvironment{exercise}{\small}
\AtBeginEnvironment{example}{\small}

\usepackage{cases}
\usepackage[red]{mypack}

\usepackage[framemethod=TikZ]{mdframed}

\definecolor{bg}{rgb}{0.4,0.25,0.95}
\definecolor{pagebg}{rgb}{0,0,0.5}
\surroundwithmdframed[
   topline=false,
   rightline=false,
   bottomline=false,
   leftmargin=\parindent,
   skipabove=8pt,
   skipbelow=8pt,
   linecolor=blue,
   innerbottommargin=10pt,
   % backgroundcolor=bg,font=\color{orange}\sffamily, fontcolor=white
]{definition}

\usepackage{empheq}
\usepackage[most]{tcolorbox}

\newtcbox{\mymath}[1][]{%
    nobeforeafter, math upper, tcbox raise base,
    enhanced, colframe=blue!30!black,
    colback=red!10, boxrule=1pt,
    #1}

\usepackage{unixode}


\DeclareMathOperator{\ord}{ord}
\DeclareMathOperator{\orb}{orb}
\DeclareMathOperator{\stab}{stab}
\DeclareMathOperator{\Stab}{stab}
\DeclareMathOperator{\ppcm}{ppcm}
\DeclareMathOperator{\conj}{Conj}
\DeclareMathOperator{\End}{End}
\DeclareMathOperator{\rot}{rot}
\DeclareMathOperator{\trs}{trace}
\DeclareMathOperator{\Ind}{Ind}
\DeclareMathOperator{\mat}{Mat}
\DeclareMathOperator{\id}{Id}
\DeclareMathOperator{\vect}{vect}
\DeclareMathOperator{\img}{img}
\DeclareMathOperator{\cov}{Cov}
\DeclareMathOperator{\dist}{dist}
\DeclareMathOperator{\irr}{Irr}
\DeclareMathOperator{\image}{Im}
\DeclareMathOperator{\pd}{\partial}
\DeclareMathOperator{\epi}{epi}
\DeclareMathOperator{\Argmin}{Argmin}
\DeclareMathOperator{\dom}{dom}
\DeclareMathOperator{\proj}{proj}
\DeclareMathOperator{\ctg}{ctg}
\DeclareMathOperator{\supp}{supp}
\DeclareMathOperator{\argmin}{argmin}
\DeclareMathOperator{\mult}{mult}
\DeclareMathOperator{\ch}{ch}
\DeclareMathOperator{\sh}{sh}
\DeclareMathOperator{\rang}{rang}
\DeclareMathOperator{\diam}{diam}
\DeclareMathOperator{\Epigraphe}{Epigraphe}




\usepackage{xcolor}
\everymath{\color{blue}}
%\everymath{\color[rgb]{0,1,1}}
%\pagecolor[rgb]{0,0,0.5}


\newcommand*{\pdtest}[3][]{\ensuremath{\frac{\partial^{#1} #2}{\partial #3}}}

\newcommand*{\deffunc}[6][]{\ensuremath{
\begin{array}{rcl}
#2 : #3 &\rightarrow& #4\\
#5 &\mapsto& #6
\end{array}
}}

\newcommand{\eqcolon}{\mathrel{\resizebox{\widthof{$\mathord{=}$}}{\height}{ $\!\!=\!\!\resizebox{1.2\width}{0.8\height}{\raisebox{0.23ex}{$\mathop{:}$}}\!\!$ }}}
\newcommand{\coloneq}{\mathrel{\resizebox{\widthof{$\mathord{=}$}}{\height}{ $\!\!\resizebox{1.2\width}{0.8\height}{\raisebox{0.23ex}{$\mathop{:}$}}\!\!=\!\!$ }}}
\newcommand{\eqcolonl}{\ensuremath{\mathrel{=\!\!\mathop{:}}}}
\newcommand{\coloneql}{\ensuremath{\mathrel{\mathop{:} \!\! =}}}
\newcommand{\vc}[1]{% inline column vector
  \left(\begin{smallmatrix}#1\end{smallmatrix}\right)%
}
\newcommand{\vr}[1]{% inline row vector
  \begin{smallmatrix}(\,#1\,)\end{smallmatrix}%
}
\makeatletter
\newcommand*{\defeq}{\ =\mathrel{\rlap{%
                     \raisebox{0.3ex}{$\m@th\cdot$}}%
                     \raisebox{-0.3ex}{$\m@th\cdot$}}%
                     }
\makeatother

\newcommand{\mathcircle}[1]{% inline row vector
 \overset{\circ}{#1}
}
\newcommand{\ulim}{% low limit
 \underline{\lim}
}
\newcommand{\ssi}{% iff
\iff
}
\newcommand{\ps}[2]{
\expval{#1 | #2}
}
\newcommand{\df}[1]{
\mqty{#1}
}
\newcommand{\n}[1]{
\norm{#1}
}
\newcommand{\sys}[1]{
\left\{\smqty{#1}\right.
}


\newcommand{\eqdef}{\ensuremath{\overset{\text{def}}=}}


\def\Circlearrowright{\ensuremath{%
  \rotatebox[origin=c]{230}{$\circlearrowright$}}}

\newcommand\ct[1]{\text{\rmfamily\upshape #1}}
\newcommand\question[1]{ {\color{red} ...!? \small #1}}
\newcommand\caz[1]{\left\{\begin{array} #1 \end{array}\right.}
\newcommand\const{\text{\rmfamily\upshape const}}
\newcommand\toP{ \overset{\pro}{\to}}
\newcommand\toPP{ \overset{\text{PP}}{\to}}
\newcommand{\oeq}{\mathrel{\text{\textcircled{$=$}}}}





\usepackage{xcolor}
% \usepackage[normalem]{ulem}
\usepackage{lipsum}
\makeatletter
% \newcommand\colorwave[1][blue]{\bgroup \markoverwith{\lower3.5\p@\hbox{\sixly \textcolor{#1}{\char58}}}\ULon}
%\font\sixly=lasy6 % does not re-load if already loaded, so no memory problem.

\newmdtheoremenv[
linewidth= 1pt,linecolor= blue,%
leftmargin=20,rightmargin=20,innertopmargin=0pt, innerrightmargin=40,%
tikzsetting = { draw=lightgray, line width = 0.3pt,dashed,%
dash pattern = on 15pt off 3pt},%
splittopskip=\topskip,skipbelow=\baselineskip,%
skipabove=\baselineskip,ntheorem,roundcorner=0pt,
% backgroundcolor=pagebg,font=\color{orange}\sffamily, fontcolor=white
]{examplebox}{Exemple}[section]



\newcommand\R{\mathbb{R}}
\newcommand\Z{\mathbb{Z}}
\newcommand\N{\mathbb{N}}
\newcommand\E{\mathbb{E}}
\newcommand\F{\mathcal{F}}
\newcommand\cH{\mathcal{H}}
\newcommand\V{\mathbb{V}}
\newcommand\dmo{ ^{-1} }
\newcommand\kapa{\kappa}
\newcommand\im{Im}
\newcommand\hs{\mathcal{H}}





\usepackage{soul}

\makeatletter
\newcommand*{\whiten}[1]{\llap{\textcolor{white}{{\the\SOUL@token}}\hspace{#1pt}}}
\DeclareRobustCommand*\myul{%
    \def\SOUL@everyspace{\underline{\space}\kern\z@}%
    \def\SOUL@everytoken{%
     \setbox0=\hbox{\the\SOUL@token}%
     \ifdim\dp0>\z@
        \raisebox{\dp0}{\underline{\phantom{\the\SOUL@token}}}%
        \whiten{1}\whiten{0}%
        \whiten{-1}\whiten{-2}%
        \llap{\the\SOUL@token}%
     \else
        \underline{\the\SOUL@token}%
     \fi}%
\SOUL@}
\makeatother

\newcommand*{\demp}{\fontfamily{lmtt}\selectfont}

\DeclareTextFontCommand{\textdemp}{\demp}

\begin{document}

\ifcomment
Multiline
comment
\fi
\ifcomment
\myul{Typesetting test}
% \color[rgb]{1,1,1}
$∑_i^n≠ 60º±∞π∆¬≈√j∫h≤≥µ$

$\CR \R\pro\ind\pro\gS\pro
\mqty[a&b\\c&d]$
$\pro\mathbb{P}$
$\dd{x}$

  \[
    \alpha(x)=\left\{
                \begin{array}{ll}
                  x\\
                  \frac{1}{1+e^{-kx}}\\
                  \frac{e^x-e^{-x}}{e^x+e^{-x}}
                \end{array}
              \right.
  \]

  $\expval{x}$
  
  $\chi_\rho(ghg\dmo)=\Tr(\rho_{ghg\dmo})=\Tr(\rho_g\circ\rho_h\circ\rho\dmo_g)=\Tr(\rho_h)\overset{\mbox{\scalebox{0.5}{$\Tr(AB)=\Tr(BA)$}}}{=}\chi_\rho(h)$
  	$\mathop{\oplus}_{\substack{x\in X}}$

$\mat(\rho_g)=(a_{ij}(g))_{\scriptsize \substack{1\leq i\leq d \\ 1\leq j\leq d}}$ et $\mat(\rho'_g)=(a'_{ij}(g))_{\scriptsize \substack{1\leq i'\leq d' \\ 1\leq j'\leq d'}}$



\[\int_a^b{\mathbb{R}^2}g(u, v)\dd{P_{XY}}(u, v)=\iint g(u,v) f_{XY}(u, v)\dd \lambda(u) \dd \lambda(v)\]
$$\lim_{x\to\infty} f(x)$$	
$$\iiiint_V \mu(t,u,v,w) \,dt\,du\,dv\,dw$$
$$\sum_{n=1}^{\infty} 2^{-n} = 1$$	
\begin{definition}
	Si $X$ et $Y$ sont 2 v.a. ou definit la \textsc{Covariance} entre $X$ et $Y$ comme
	$\cov(X,Y)\overset{\text{def}}{=}\E\left[(X-\E(X))(Y-\E(Y))\right]=\E(XY)-\E(X)\E(Y)$.
\end{definition}
\fi
\pagebreak

% \tableofcontents

% insert your code here
%\input{./algebra/main.tex}
%\input{./geometrie-differentielle/main.tex}
%\input{./probabilite/main.tex}
%\input{./analyse-fonctionnelle/main.tex}
% \input{./Analyse-convexe-et-dualite-en-optimisation/main.tex}
%\input{./tikz/main.tex}
%\input{./Theorie-du-distributions/main.tex}
%\input{./optimisation/mine.tex}
 \input{./modelisation/main.tex}

% yves.aubry@univ-tln.fr : algebra

\end{document}

%\input{./optimisation/mine.tex}
 % !TEX encoding = UTF-8 Unicode
% !TEX TS-program = xelatex

\documentclass[french]{report}

%\usepackage[utf8]{inputenc}
%\usepackage[T1]{fontenc}
\usepackage{babel}


\newif\ifcomment
%\commenttrue # Show comments

\usepackage{physics}
\usepackage{amssymb}


\usepackage{amsthm}
% \usepackage{thmtools}
\usepackage{mathtools}
\usepackage{amsfonts}

\usepackage{color}

\usepackage{tikz}

\usepackage{geometry}
\geometry{a5paper, margin=0.1in, right=1cm}

\usepackage{dsfont}

\usepackage{graphicx}
\graphicspath{ {images/} }

\usepackage{faktor}

\usepackage{IEEEtrantools}
\usepackage{enumerate}   
\usepackage[PostScript=dvips]{"/Users/aware/Documents/Courses/diagrams"}


\newtheorem{theorem}{Théorème}[section]
\renewcommand{\thetheorem}{\arabic{theorem}}
\newtheorem{lemme}{Lemme}[section]
\renewcommand{\thelemme}{\arabic{lemme}}
\newtheorem{proposition}{Proposition}[section]
\renewcommand{\theproposition}{\arabic{proposition}}
\newtheorem{notations}{Notations}[section]
\newtheorem{problem}{Problème}[section]
\newtheorem{corollary}{Corollaire}[theorem]
\renewcommand{\thecorollary}{\arabic{corollary}}
\newtheorem{property}{Propriété}[section]
\newtheorem{objective}{Objectif}[section]

\theoremstyle{definition}
\newtheorem{definition}{Définition}[section]
\renewcommand{\thedefinition}{\arabic{definition}}
\newtheorem{exercise}{Exercice}[chapter]
\renewcommand{\theexercise}{\arabic{exercise}}
\newtheorem{example}{Exemple}[chapter]
\renewcommand{\theexample}{\arabic{example}}
\newtheorem*{solution}{Solution}
\newtheorem*{application}{Application}
\newtheorem*{notation}{Notation}
\newtheorem*{vocabulary}{Vocabulaire}
\newtheorem*{properties}{Propriétés}



\theoremstyle{remark}
\newtheorem*{remark}{Remarque}
\newtheorem*{rappel}{Rappel}


\usepackage{etoolbox}
\AtBeginEnvironment{exercise}{\small}
\AtBeginEnvironment{example}{\small}

\usepackage{cases}
\usepackage[red]{mypack}

\usepackage[framemethod=TikZ]{mdframed}

\definecolor{bg}{rgb}{0.4,0.25,0.95}
\definecolor{pagebg}{rgb}{0,0,0.5}
\surroundwithmdframed[
   topline=false,
   rightline=false,
   bottomline=false,
   leftmargin=\parindent,
   skipabove=8pt,
   skipbelow=8pt,
   linecolor=blue,
   innerbottommargin=10pt,
   % backgroundcolor=bg,font=\color{orange}\sffamily, fontcolor=white
]{definition}

\usepackage{empheq}
\usepackage[most]{tcolorbox}

\newtcbox{\mymath}[1][]{%
    nobeforeafter, math upper, tcbox raise base,
    enhanced, colframe=blue!30!black,
    colback=red!10, boxrule=1pt,
    #1}

\usepackage{unixode}


\DeclareMathOperator{\ord}{ord}
\DeclareMathOperator{\orb}{orb}
\DeclareMathOperator{\stab}{stab}
\DeclareMathOperator{\Stab}{stab}
\DeclareMathOperator{\ppcm}{ppcm}
\DeclareMathOperator{\conj}{Conj}
\DeclareMathOperator{\End}{End}
\DeclareMathOperator{\rot}{rot}
\DeclareMathOperator{\trs}{trace}
\DeclareMathOperator{\Ind}{Ind}
\DeclareMathOperator{\mat}{Mat}
\DeclareMathOperator{\id}{Id}
\DeclareMathOperator{\vect}{vect}
\DeclareMathOperator{\img}{img}
\DeclareMathOperator{\cov}{Cov}
\DeclareMathOperator{\dist}{dist}
\DeclareMathOperator{\irr}{Irr}
\DeclareMathOperator{\image}{Im}
\DeclareMathOperator{\pd}{\partial}
\DeclareMathOperator{\epi}{epi}
\DeclareMathOperator{\Argmin}{Argmin}
\DeclareMathOperator{\dom}{dom}
\DeclareMathOperator{\proj}{proj}
\DeclareMathOperator{\ctg}{ctg}
\DeclareMathOperator{\supp}{supp}
\DeclareMathOperator{\argmin}{argmin}
\DeclareMathOperator{\mult}{mult}
\DeclareMathOperator{\ch}{ch}
\DeclareMathOperator{\sh}{sh}
\DeclareMathOperator{\rang}{rang}
\DeclareMathOperator{\diam}{diam}
\DeclareMathOperator{\Epigraphe}{Epigraphe}




\usepackage{xcolor}
\everymath{\color{blue}}
%\everymath{\color[rgb]{0,1,1}}
%\pagecolor[rgb]{0,0,0.5}


\newcommand*{\pdtest}[3][]{\ensuremath{\frac{\partial^{#1} #2}{\partial #3}}}

\newcommand*{\deffunc}[6][]{\ensuremath{
\begin{array}{rcl}
#2 : #3 &\rightarrow& #4\\
#5 &\mapsto& #6
\end{array}
}}

\newcommand{\eqcolon}{\mathrel{\resizebox{\widthof{$\mathord{=}$}}{\height}{ $\!\!=\!\!\resizebox{1.2\width}{0.8\height}{\raisebox{0.23ex}{$\mathop{:}$}}\!\!$ }}}
\newcommand{\coloneq}{\mathrel{\resizebox{\widthof{$\mathord{=}$}}{\height}{ $\!\!\resizebox{1.2\width}{0.8\height}{\raisebox{0.23ex}{$\mathop{:}$}}\!\!=\!\!$ }}}
\newcommand{\eqcolonl}{\ensuremath{\mathrel{=\!\!\mathop{:}}}}
\newcommand{\coloneql}{\ensuremath{\mathrel{\mathop{:} \!\! =}}}
\newcommand{\vc}[1]{% inline column vector
  \left(\begin{smallmatrix}#1\end{smallmatrix}\right)%
}
\newcommand{\vr}[1]{% inline row vector
  \begin{smallmatrix}(\,#1\,)\end{smallmatrix}%
}
\makeatletter
\newcommand*{\defeq}{\ =\mathrel{\rlap{%
                     \raisebox{0.3ex}{$\m@th\cdot$}}%
                     \raisebox{-0.3ex}{$\m@th\cdot$}}%
                     }
\makeatother

\newcommand{\mathcircle}[1]{% inline row vector
 \overset{\circ}{#1}
}
\newcommand{\ulim}{% low limit
 \underline{\lim}
}
\newcommand{\ssi}{% iff
\iff
}
\newcommand{\ps}[2]{
\expval{#1 | #2}
}
\newcommand{\df}[1]{
\mqty{#1}
}
\newcommand{\n}[1]{
\norm{#1}
}
\newcommand{\sys}[1]{
\left\{\smqty{#1}\right.
}


\newcommand{\eqdef}{\ensuremath{\overset{\text{def}}=}}


\def\Circlearrowright{\ensuremath{%
  \rotatebox[origin=c]{230}{$\circlearrowright$}}}

\newcommand\ct[1]{\text{\rmfamily\upshape #1}}
\newcommand\question[1]{ {\color{red} ...!? \small #1}}
\newcommand\caz[1]{\left\{\begin{array} #1 \end{array}\right.}
\newcommand\const{\text{\rmfamily\upshape const}}
\newcommand\toP{ \overset{\pro}{\to}}
\newcommand\toPP{ \overset{\text{PP}}{\to}}
\newcommand{\oeq}{\mathrel{\text{\textcircled{$=$}}}}





\usepackage{xcolor}
% \usepackage[normalem]{ulem}
\usepackage{lipsum}
\makeatletter
% \newcommand\colorwave[1][blue]{\bgroup \markoverwith{\lower3.5\p@\hbox{\sixly \textcolor{#1}{\char58}}}\ULon}
%\font\sixly=lasy6 % does not re-load if already loaded, so no memory problem.

\newmdtheoremenv[
linewidth= 1pt,linecolor= blue,%
leftmargin=20,rightmargin=20,innertopmargin=0pt, innerrightmargin=40,%
tikzsetting = { draw=lightgray, line width = 0.3pt,dashed,%
dash pattern = on 15pt off 3pt},%
splittopskip=\topskip,skipbelow=\baselineskip,%
skipabove=\baselineskip,ntheorem,roundcorner=0pt,
% backgroundcolor=pagebg,font=\color{orange}\sffamily, fontcolor=white
]{examplebox}{Exemple}[section]



\newcommand\R{\mathbb{R}}
\newcommand\Z{\mathbb{Z}}
\newcommand\N{\mathbb{N}}
\newcommand\E{\mathbb{E}}
\newcommand\F{\mathcal{F}}
\newcommand\cH{\mathcal{H}}
\newcommand\V{\mathbb{V}}
\newcommand\dmo{ ^{-1} }
\newcommand\kapa{\kappa}
\newcommand\im{Im}
\newcommand\hs{\mathcal{H}}





\usepackage{soul}

\makeatletter
\newcommand*{\whiten}[1]{\llap{\textcolor{white}{{\the\SOUL@token}}\hspace{#1pt}}}
\DeclareRobustCommand*\myul{%
    \def\SOUL@everyspace{\underline{\space}\kern\z@}%
    \def\SOUL@everytoken{%
     \setbox0=\hbox{\the\SOUL@token}%
     \ifdim\dp0>\z@
        \raisebox{\dp0}{\underline{\phantom{\the\SOUL@token}}}%
        \whiten{1}\whiten{0}%
        \whiten{-1}\whiten{-2}%
        \llap{\the\SOUL@token}%
     \else
        \underline{\the\SOUL@token}%
     \fi}%
\SOUL@}
\makeatother

\newcommand*{\demp}{\fontfamily{lmtt}\selectfont}

\DeclareTextFontCommand{\textdemp}{\demp}

\begin{document}

\ifcomment
Multiline
comment
\fi
\ifcomment
\myul{Typesetting test}
% \color[rgb]{1,1,1}
$∑_i^n≠ 60º±∞π∆¬≈√j∫h≤≥µ$

$\CR \R\pro\ind\pro\gS\pro
\mqty[a&b\\c&d]$
$\pro\mathbb{P}$
$\dd{x}$

  \[
    \alpha(x)=\left\{
                \begin{array}{ll}
                  x\\
                  \frac{1}{1+e^{-kx}}\\
                  \frac{e^x-e^{-x}}{e^x+e^{-x}}
                \end{array}
              \right.
  \]

  $\expval{x}$
  
  $\chi_\rho(ghg\dmo)=\Tr(\rho_{ghg\dmo})=\Tr(\rho_g\circ\rho_h\circ\rho\dmo_g)=\Tr(\rho_h)\overset{\mbox{\scalebox{0.5}{$\Tr(AB)=\Tr(BA)$}}}{=}\chi_\rho(h)$
  	$\mathop{\oplus}_{\substack{x\in X}}$

$\mat(\rho_g)=(a_{ij}(g))_{\scriptsize \substack{1\leq i\leq d \\ 1\leq j\leq d}}$ et $\mat(\rho'_g)=(a'_{ij}(g))_{\scriptsize \substack{1\leq i'\leq d' \\ 1\leq j'\leq d'}}$



\[\int_a^b{\mathbb{R}^2}g(u, v)\dd{P_{XY}}(u, v)=\iint g(u,v) f_{XY}(u, v)\dd \lambda(u) \dd \lambda(v)\]
$$\lim_{x\to\infty} f(x)$$	
$$\iiiint_V \mu(t,u,v,w) \,dt\,du\,dv\,dw$$
$$\sum_{n=1}^{\infty} 2^{-n} = 1$$	
\begin{definition}
	Si $X$ et $Y$ sont 2 v.a. ou definit la \textsc{Covariance} entre $X$ et $Y$ comme
	$\cov(X,Y)\overset{\text{def}}{=}\E\left[(X-\E(X))(Y-\E(Y))\right]=\E(XY)-\E(X)\E(Y)$.
\end{definition}
\fi
\pagebreak

% \tableofcontents

% insert your code here
%\input{./algebra/main.tex}
%\input{./geometrie-differentielle/main.tex}
%\input{./probabilite/main.tex}
%\input{./analyse-fonctionnelle/main.tex}
% \input{./Analyse-convexe-et-dualite-en-optimisation/main.tex}
%\input{./tikz/main.tex}
%\input{./Theorie-du-distributions/main.tex}
%\input{./optimisation/mine.tex}
 \input{./modelisation/main.tex}

% yves.aubry@univ-tln.fr : algebra

\end{document}


% yves.aubry@univ-tln.fr : algebra

\end{document}


% yves.aubry@univ-tln.fr : algebra

\end{document}


% yves.aubry@univ-tln.fr : algebra

\end{document}
